\chapter{Possible Reading List}
This chapter contains a list of additional works (in a variety of bibliographic formats) that are potentially relevant, accumulated during the generation of this literature review. They are included here both for my own future reference, and to provide an indication of the variety of additional work available in this area. For some items, a note has been placed in \emph{italics} after the entry indicating the relevance of the item.

\section{Reading lists}

\url{http://sustain.cs.washington.edu/index.php/Related_Research}


\section{Social Norms}
Ackermann, M 2002, Cool comfort: America's romance with air-conditioning, Smithsonian Institution Press, Washington [USA].

Shove, E 2003, Comfort, cleanliness and convenience: the social organisation of normality, Berg Publishers, Oxford [UK].

Southerton, D, Warde, A \& Hand, M 2004, 'The limited autonomy of the consumer: implications for sustainable consumption', in D Southerton, B Van Vliet \& H Chappells (eds), Sustainable consumption: the implications of changing infrastructures of provision, Edward Elgar, Cheltenham [UK], pp. 32-48.

Goldstein, N \& Cialdini, R 2007, 'Using social norms as a lever of social influence', in AR Pratkanis (ed.), The science of social influence: advances and future progress, Psychology Press, New York [US], pp. 167-91

Cialdini, R 2003, 'Crafting normative messages to protect the environment', Current Directions in Psychological Science, vol. 12, no. 4, pp. 105-9.

Ferraro, P.J. Common Pool Resource Management without Prices:  A large-scale randomized experiment to test the effects of information, moral suasion and social norms. Presented at the Allied Social Sciences Associaiton Meetings, January 2009.











\section{Related Systems}

\url{https://www.openeco.org/} \emph{Sun Microsystems sponsored site to allow users to calculate their carbon footprint, compare with others, and collaborate on how to reduce the footprint. Seems focused on buildings and organizations, but mentions individuals too}

\url{http://www.google.co.uk/carbonfootprint/index.html} \emph{Google's carbon footprint calculator widget for the UK. Suggests actions for reduction, and can map results on local and country-wide maps}

\url{http://carbongoggles.org/} \emph{Visualizes carbon emissions in Second Life using a virtual heads-up display}

Causes, FaceBook application \url{http://apps.facebook.com/causes/} \emph{Uses social networking and multilevel marketing techniques to recruit members and money for causes}

\url{http://co2stats.com/} \emph{Records stats about a website's usage, converts that to a carbon footprint, and offers way to reduce or offset that carbon.}

\url{http://www.sei-us.org/aviationcalculations.html} Paper about calculating \COtwo emissions from air travel.

\url{http://www.zerofootprint.net/products/personal-carbon-manager} Brandable carbon calculator, with tracking over time, social networking aspects, what-if.


\section{Electricity Conservation}

Harris, C. and Cahill, V. An Empirical Study of the Potential for Context-Aware Power Management Ubicomp 2007, Springer, Innsbruck, Austria, 2007. \url{http://www.springerlink.com.eres.library.manoa.hawaii.edu/content/j0724t44782r1511/?p=12be8a774af84bfea59f5ff3951f3229&pi=13}

Holmes, T.G. Eco-visualization: combining art and technology to reduce energy consumption. Creativity and Cognition 2007, ACM, Washington, DC, USA, 2007.

Sidler O and Waide P (1999) Metering matters! Appliance efficiency 3 (4) 1999

IEA-DSM (2005) Smaller customer energy saving by end-use monitoring and feedback. International Energy Agency Demand-side Management Programme Task XI, Subtask 1. From Richard Formby, EA Technology, Chester 

Stein, Lynn F., 2004. California Information Display Pilot Technology Assessment, Final Report, prepared for Southern California Edison, Primen, Inc., Boulder, CO, 21 December 2004. \emph{detailed comparison of whole home energy meters?}

Hasbrouck, J.; Woodruff, A. "Green Homeowners as Lead Adopters: Sustainable Living and Green Computing." Intel Technology Journal. \url{http://www.intel.com/technology/itj/2008/v12i1/4-homeowners/1-abstract.htm} (February 2008). ISSN 1535-864X DOI 10.1535/itj.1201.04

Mankoff, J., et al.  Environmental Sustainability and Interaction.  CHI 2007 SIG, in CHI 2007 Extended Abstracts, 2121-2124.

Bång, M., et al. The PowerHouse: A Persuasive Computer Game Designed to Raise Awareness of Domestic Energy Consumption.  Lecture Notes in Computer Science, vol. 3962:123-32, 2006.

Gustafsson, A. and Gyllenswärd, M. The Power-Aware Cord: Energy Awareness through Ambient Information Display. CHI 2005 Extended Abstracts, 1423-1426.

Stringer, M., et al.  Kuckuck – Exploring Ways of Sensing and Displaying Energy Use in the Home.  Proc. UbiComp 2007 Workshops, 606-609.

Christopher Laughman et. al ``Advanced Nonintrusive Monitoring of Electric Loads'', IEEE Power and Energy Magazine, March/April 2003, 56 – 63. \url{http://buildings.lbl.gov/CEC/pubs/E5P22T3d.pdf} \emph{Fancy electrical monitoring that can distinguish between different devices in the home}

Mccalley, L.T. and Midden, C.J.H. Energy conservation through product-integrated feedback: the roles of goal-setting and social orientation. Journal of Economic Psychology, 23 (2002), 589-603.

Seligman, C., Becker, L.S. and Darley, J.M. Encouraging residential energy conservation through feedback. Advances in Environmental Psychology, 3 (1981), 93-113.

Wood, G. and Newborough, M. Dynamic energy-consumption indicators for domestic appliances: environment, behaviour and design. Energy and Buildings, 35, 8 (2003), 821-841.

Roberts S, Humphries H and Hyldon V (2004) Consumer preferences for improving energy consumption feedback. Report for Ofgem. Centre for Sustainable Energy, Bristol, UK \emph{customers suspicious of accuracy of comparison groups, but like seeing their own historical data}

Mountain D (2006) the impact of real-time feedback on residential electricity consumption: the Hydro One pilot. Mountain Economic Consulting and Associates Inc., Ontario

Katzev, R. D., \& Johnson, T. R. (1987). Promoting energy conservation: An analysisof behavioral research. Boulder, CO: Westview Press. \emph{problems with behaviorist methods for promoting energy conservation}

De Young, R. (1993). Changing behavior and making it stick: The conceptualization and management of conservation behavior. Environment and Behavior, 25, 485–505. \emph{problems with extrinsic motivators}

Poortinga, W., Steg, L. and Vlek, C. (2004) Values, environmental concern, and environmental behavior: A study into household energy use. Environment and Behavior 36(1):70-93.  

van Houwelingen, J. H. and W. F. van Raajj (1989). The Effect of Goal-Setting and Daily Electronic Feedback on In-Home Energy Use. Journal of Consumer Research, 16(1):98-105.

Siero, F.W., A. B. Bakker, G. B. Dekker and M. T. C. van Denburg (1996) Changing organizational energy consumption behavior through comparative feedback. Journal of Environmental Psychology, 16(3):235-246.

Archer, D., Pettigrew, T., Costanzo, M., Iritani, B., Walker, I. & White, L. 1987. Energy conservation and public policy: The mediation of individual behavior. Energy Efficiency: Perspectives on Individual Behavior, 69-92.

R. K. Harle and A. Hopper. The potential for location-aware power management. In UbiComp ’08: Proceedings of the 10th international conference on Ubiquitous computing, pages 302–311, New York, NY, USA, 2008. ACM.

Pierce, James. Energy Aware Campus Dwelling: Eco-Visualization and the IU Energy Challenge. http://www.jamesjpierce.com/documentation/energy_aware_campus_dwelling.pdf

Odom, W., Pierce, J., & Roedl, D. (2008). Social Incentive & Eco-Visualization Displays: Toward Persuading Greater Change in Dormitory Communities. In workshop proceedings of Public and Situated Displays to Support Communities. OZCHI '08. http://www.jamesjpierce.com/publications/social_incentive_and_ecovisualization_displays.pdf

Pierce, J. and Roedl, D. 2008. COVER STORY
Changing energy use through design. interactions 15, 4 (Jul. 2008), 6-12. DOI= http://doi.acm.org/10.1145/1374489.1374491

Pierce, J., Odom, W., & Blevis, E. (2008). Energy Aware Dwelling: A Critical Survey of Interaction Design for Eco-Visualizations. In Proceedings of OZCHI Conference on Human Factors in Computer Systems, Cairns, Australia. OZCHI '08. ACM Press, New York, NY. http://www.jamesjpierce.com/publications/energy_aware_dwelling.pdf

W Odom, J Pierce, D Roedl. Design of a Dormitory Consumption Visualization & New Conceptual Directions. http://www.willodom.com/projects/energychallenge/roedl_odom_pierce_EnergyChallengeProject.pdf

Schipper, L. (1989), “Linking lifestyle and energy used: a matter of time?”, Annual Review of Energy, Vol. 14, pp. 273-320. \emph{Source for choices resulting in up to 50\% of residential energy usage}

Brandon, G. and Lewis, A. (1999), ``Reducing household energy consumption: a qualitative and quantitative field study'', Journal of Environmental Psychology, Vol. 19, pp. 75-85.

Roberts, S. and Baker, W. (2003), “Towards effective energy information: improving consumer feedback on energy consumption”, report, Centre for Sustainable Energy, Bristol. 


\section{Sensors and Smart Environments}


Mozer, M. The Neural Network House: An Environment That Adapts to Its Inhabitants. Coen, M. ed. American Association for Artificial Intelligence Spring Symposium on Intelligent Environments, AAAI Press, Menlo Park, CA, 1998, 110-114.

Spiekermann, S. and Pallas, F. Technology Paternalism: Wider Implications of Ubiquitous Computing. Poiesis \& praxis, 4 (2006), 6-18. \emph{potentially interesting concept}

Haines, V., Mitchell, V., Cooper, C. and Maguire, M. Probing user values in the home environment within a technology driven Smart Home project. Personal and Ubiquitous Computing, 11, 5 (2006), 349-359. \emph{energy saving was found to be the value least related to technology}

Bonanni, L., Arroyo, E., Lee, C. and Selker, T. Smart sinks: real-world opportunities for context-aware interaction CHI 2005, ACM, Portland, OR, USA, 2005.

Arroyo, E., Bonanni, L. and Selker, T. Waterbot: exploring feedback and persuasive techniques at the sink CHI 2005, ACM, Portland, Oregon, USA, 2005.

Beckmann, C., Consolvo, S. and LaMarca, A. Some Assembly Required: Supporting End-User Sensor Installation In Domestic Ubiquitous Computing Environments Ubicomp 2004, Springer-Verlag, 2004. \emph{installing a home energy tutor system?}

Nakajima T., Lehdonvirta V., Tokunaga E., Kimura H. "Reflecting Human Behavior to Motivate Desirable Lifestyle", Proceedings of DIS 2008, February 25-27, 2008, Cape Town, South Africa. \emph{encouraging children to brush teeth using ambient display and game-like techniques}

Liao L., Patterson D.J., Fox D., and Kautz H.. Learning and Inferring Transportation Routines. In Proceedings of the National Conference on Artificial Intelligence (2004), AAAI Press: 348-353.

L. Liao, D. Fox, and H. Kautz. Learning and Inferring Transportation Routines. in Proceedings of AAAI-04 , 2004. Outstanding Paper Award. \url{http://www.liaolin.com/Research/aaai2004.pdf}

Liao, L., Patterson, D. J., Fox, D., and Kautz, H. 2007. Learning and inferring transportation routines. Artif. Intell. 171, 5-6 (Apr. 2007), 311-331. DOI= \url{http://dx.doi.org/10.1016/j.artint.2007.01.006}

Liao, Lin, Fox, Dieter, Kautz, Henry. Extracting Places and Activities from GPS Traces Using Hierarchical Conditional Random Fields. The International Journal of Robotics Research 2007 26: 119-134. doi=10.1177/0278364907073775

L. Liao. Location-based Activity Recognition. University of Washington, 2006. \url{http://www.liaolin.com/Research/liao-thesis.pdf}

Driving Plug-In Hybrid Electric Vehicles: Reports from U.S. Drivers of HEVs  
converted to PHEVs, circa 2006-07 Dr. Kenneth S. Kurani, Associate Researcher 
Dr. Reid R. Heffner, Senior Analyst Dr. Thomas S. Turrentine Director, Plug-in Hybrid Electric Vehicle Research Center. Institute of Transportation Studies 
University of California, Davis \url{http://pubs.its.ucdavis.edu/download_pdf.php?id=1193} \emph{reference for Prius drivers using milage display feedback to improve efficiency}

IEEE Pervasive special issue on Environmental Sustainability (Jan-Mar 2009 issue)

HICSS-42 is having some relevant tracks this year (technically next year), pointed out by Eric Paulos. The mini-tracks are: Information Systems for Sustainable Development, Urban Computing: The City as a Living Lab, and Pervasive Computing Technology and Applications

Blevis, E.  Sustainable Interaction Design: Invention \& Disposal, Renewal and Reuse.  Proc. CHI 2007, 503-12.

Jain, R. and Wullert, J. Challenges: Environmental Design for Pervasive Computing Systems.  Proc. MobiCom 2002, 263-270.

D. Cuff, M. Hansen, J. Kang. "Urban sensing: out of the woods." Communications of the ACM 51(3):24-33, 2008.




\section{Environmental statistics}
Energy Information Administration. US Household Electricity Report. US Dept of Energy,. 2005. \url{www.eia.doe.gov/emeu/reps/enduse/er01_us.html}. \emph{breakdown on home energy usage by category}

Article about how green consumers are major carbon emitters. \url{http://www.guardian.co.uk/environment/2008/sep/24/ethicalliving.recycling}. \emph{Need to figure out which papers by Stewart Barr (mentioned in article) contain the data to support this conclusion.}

Wackernagel, M. and Rees, W.E. (1996). Our Ecological Footprint - Reducing Human Impact on the Earth. New Society Publishers Gabriola Island, B.C., Canada.

\url{http://www.footprintstandards.org/}

Wackernagel, M., Monfreda, C., Moran, D., Wermer, P., Goldfinger, S., Deumling, D. and Murray, M. (2005). National Footprint and Biocapacity Accounts 2005: The underlying calculation method.

Wiedmann, T., Minx, J., Barrett, J. and Wackernagel, M. (2006). "Allocating ecological footprints to final consumption categories with input-output analysis". Ecological Economics 56(1): 28-48. \url{http://dx.doi.org/10.1016/j.ecolecon.2005.05.012}.

Heijungs, R. and Suh, S. (2006). "Reformulation of matrix-based LCI: from product balance to process balance". Journal of Cleaner Production 14(1): 47-51. 
\url{http://dx.doi.org/10.1016/j.jclepro.2005.05.022}.

Hammond, G. (2007). "Time to give due weight to the 'carbon footprint' issue". Nature 445(7125): 256. \url{http://dx.doi.org/10.1038/445256b}.

Lenzen, M. (2007). "Double-counting in Life-Cycle Assessment". Journal of Industrial Ecology: submitted.

Boardman B, Darby S, Killip G, Hinnells M, Jardine CN, Palmer J and Sinden G (2005) 40\% House. Environmental Change Institute, University of Oxford, UK

Keesee M (2005) Setting a new standard – the zero energy home experience in California. Sacramento Municipal Utility District

Murphy, P. (2006) Post-Peak - The change starts with us. Energy Bulletin. Published Thursday, March 23, 2006. Available at \url{http://www.energybulletin.net/print.php?id=14143} \emph{ref for half of a American's consumption being under individual's personal control}

Target Atmospheric CO2: Where Should Humanity Aim? pp.217-231 (15) Authors: James Hansen, Makiko Sato, Pushker Kharecha, David Beerling, Robert Berner, Valerie Masson-Delmotte, Mark Pagani, Maureen Raymo, Dana L. Royer, James C. Zachos \url{http://www.bentham-open.org/pages/content.php?TOASCJ/2008/00000002/00000001/217TOASCJ.SGM} \emph{claims that 350 ppm is the \COtwo value that we should shoot for}

Population and Energy Consumption. \url{http://www.worldpopulationbalance.org/pop/energy/}

Monbiot, G., and Matthew, P.  Heat: How to Stop the Planet from Burning.  Doubleday, Canada, 2006. \emph{``small changes made by large numbers of people will not be sufficient to mediate the current environmental crisis''}

Bad Deal for the Planet: Why Carbon Offsets Aren't Working...and How to Create a Fair Global Climate Accord <http://www.internationalrivers.org/node/2826> all the problems with carbon credits under the Kyoto Protocol.




\section{Motivation \& Psychology}
Zajonc, R. B. Social facilitation. Science, pp. 149, 269-274. 1965 \emph{Social facilitation is the phenomenon where a person performs better at a task when someone else, e.g. a colleague or a supervisor, is watching}

Asch, S. E. Opinions and social pressure. Scientific American, pp. 31-35. 1955 \emph{Conforming behaviour is the desire not to act against group consensus}

International Personality Item Pool: A Scientific Collaboratory for the Development of Advanced Measures of Personality Traits and Other Individual Differences. \url{http://ipip.ori.org/}. Internet Web Site. \emph{Public domain questionnaires for assessing personality traits}

Kaufman, T. (1999). A Study of the Motivations Behind Heritage Site Selection in the United States, PhD Dissertation, Virginia Tech. \url{http://scholar.lib.vt.edu/theses/available/etd-041999-163619/} \emph{Appendices have example surveys on environmental attitudes}

Harmon-Jones, E. (2002). Cognitive Dissonance Theory. Perspective on Persuasion. The Persuasion Handbook:  Developments in Theory and Practice. Sage Publications. p. 99-116.

Dwyer WO, Leeming FC, Cobern MK, Porter BE and Jackson JM (1993) Critical review of behavioural interventions to preserve the environment: research since 1980. Environment and behaviour 25 (3) 275-321.

LBNL (2001) California load a web hit. ECEEE mailing. See also \url{http://www.energycrisis.lbl.gov/}

Schwartz, S. H. (1970). Moral decision making and behavior. In J. Macauley \& L. Berkowitz (Eds.), Altruism and helping behavior (pp. 127–141). New York: Academic Press. \emph{Schwartz moral norm activation model, standard model for altruism motivation}

Allen, J.B., \& Ferrand, J.L. (1999). Environmental locus of control, sympathy, and proenvironmental behavior: A test of Geller’s actively caring hypothesis. Behavior and Environment, 31, 338–353. \emph{alternative framework for altruism motivation}

Sieber, S. D. (1981). Fatal remedies: The ironies of social intervention. New York: Plenum Press. \emph{altruism may be a fatal remedy?? Available at Hamilton, call number: HN57 .S52}

De Young, R. (1996). Some psychological aspects of a reduced consumption lifestyle: The role of intrinsic satisfaction and competence. Environment and Behavior, 28, 358–409.

Morris, A. (2000) Reflections on Social Movement Theory: Criticisms and Proposals. Contemporary Sociology, 29(3):445-454. 2000.

Passy, F. and Giugni, M. (2000) Life-Spheres, Networks and Sustained Participation in Social Movements: A Phenomenological Approach to Political Commitment. Sociological Forum, 15(1):117-144, 2000. \emph{how to sustain participation in a social movement}

Passy, F. and Giugni, M. (2000) Social Networks and Individual Perceptions: Explaining Differential Participation in Social Movements. Sociological Forum, 16(1):123-153, 2000. \emph{how social networks affect involvement in movement}

ANDREASEN, A.R., Social Marketing in the 21st Century, Sage Publications Inc, 2006.

CIALDINI, R.B., "Basic Social Influence Is Underestimated," Psychological Inquiry,  vol. 16, 2005, pp. 158-161.

Fogg, B.J.  Persuasive Technology: Using Computers to Change What We Think and Do.  Morgan Kaufmann Publishers, Boston, 2003.

Midden, C., F. G. Kaiser, L. T. McCalley. (2007) "Technology’s Four Roles in Understanding Individuals’ Conservation of Natural Resources". J. of Social Issues 63(1): 155-174.

Mayer, S.F. and Frantz, C.M. (2004), “The connectedness to nature scale: a measure of individuals’ feeling in community with nature”, Journal of Environmental Psychology, Vol. 24, pp. 503-15. 



\section{Economics}
Weber, C.L., Matthew, H.S, “Quantifying the global and distributional 
aspects of American household carbon footprint,” Ecological Economics 
(2007), doi:10.1016/j.ecolecon.2007.09.021. 

Gallego, B. and M. Lenzen (2005). "A consistent input-output formulation of shared consumer and producer responsibility." Economic Systems Research 17(4): 
365-391.

Lenzen, M., J. Murray, et al. (2007). ``Shared producer and consumer responsibility - theory and practice.'' Ecological Economics 61(1): 27-42.

Khazzoom, J. Daniel (1980). "Economic Implications of Mandated Efficiency Standards for Household Appliances." The Energy Journal 4(1): 21-40.

Khazzoom, J. Daniel (1987). "Energy Saying Resulting from the Adoption of More Efficient Appliances.' The Energy Journal 8(4): 85-89.

Khazzoom, J. Daniel (1989). "Energy Savings from More Efficient Appliances: A Rejoinder.' The Energy Journal 10(1): 157-166.

Lovins, Amory B. (1988) "Energy Saving Resulting from the Adoption of More Efficient Appliances: Another View.' The Energy Journal 9(2): 155-162.


\section{Product Labeling}

\url{http://www.carbon-label.com/} \emph{Carbon Trust in the UK working with vendors to create carbon labels for products}

\url{http://ecolabelling.org/} \emph{Directory of ``ecolabels'' that convey that a product has met some sort of environmental standard}

\section{Methodology}
Dawes, John (2008), "Do Data Characteristics Change According to the number of scale points used? An experiment using 5-point, 7-point and 10-point scales," International Journal of Market Research, 50 (1), 61-77.