\begin{abstract}

A recent focus of interest in software engineering research is on low-level software processes, which define how software developers or development teams should carry on development activities in short phases that last from several minutes to a few hours. Anecdotal evidence exists for the positive impact on quality and productivity of certain low-level software processes such as test-driven development and continuous integration. However, empirical research on low-level software processes often yields conflicting results. A significant threat to the validity of the empirical studies on low-level software processes is that they lack the ability to rigorously assess process conformance. That is to say, the degree to which developers follow the low-level software processes can not be evaluated. 

In order to improve the quality of empirical research on low-level software processes, I developed a technique called Software Development Stream Analysis (SDSA) that can infer development behaviors using automatically collected in-process software metrics. The collection of development activities is supported by Hackystat, a framework for automated software process and product metrics collection and analysis. SDSA abstracts the collected software metrics into a software development stream, a time-series data structure containing time-stamped development events. It then partitions the development stream into episodes, and then uses a rule-based system to infer low-level development behaviors exhibited in episodes. 

With the capabilities provided by Hackystat and SDSA, I developed the Zorro software system to study a specific low-level software process called Test-Driven Development (TDD). Experience reports have shown that TDD can greatly improve software quality with increased developer productivity, but empirical research findings on TDD are often mixed. An inability to rigorously assess process conformance is a possible explanation. Zorro can rigorously assess process conformance to a specific operational definition for TDD, and thus enable more controlled, comparable empirical studies.

\begin{comment}
I have conducted a series of empirical case studies to evaluate Zorro: a pilot study, a classroom case study, and an industrial case study. I recruited seven experienced Java developers to implement the stack data structure using TDD in the pilot study in Fall 2005. I video recorded the development process as a means of validating Zorro, and found that Zorro inferred TDD development behaviors with 88.4\% accuracy. The classroom case study conducted in Fall 2006 is an extended Zorro validation study. It concluded that Zorro inferred TDD development behaviors with 70.1\% accuracy and TDD conformance with 89.1\% accuracy. In addition to video recording, participants in this study provided subjective assessment of Zorro's inference as an additional data source for validation. The post-test interview suggested that TDD is hard to do for beginners, and that a CASE (Computer-Aided Software Engineering) tool such as Zorro can be very helpful. 
%In the industrial case study, I deployed Zorro in a Norwegian software company that undertook a comparative study between TDD and non-TDD. Zorro was used in this study to provide development behaviors and compliance information of TDD. It was hard to conduct this case study because of low participation. Only 10 out of 20 developers actively participated in data collection. 
The industrial case study was a tryout of Zorro. With the help of Zorro Dr. Hanssen, a researcher from SINTEF ICT of Norway, conducted a comparison study between TDD and an existing TL-practice  in an European software company. As a technique supporter of this study, I helped him deploy Zorro to the site, assisted the installation of the Hackystat sensor, and aided Dr. Hanssen in data analysis. The phone interview with Dr. Hanssen indicated that Zorro is helpful for empirical research of TDD. Moreover, Dr. Hanssen plans to extend Zorro's use in his future research.
\end{comment}

My research has demonstrated that Zorro can recognize the low-level software development behaviors that characterize TDD. Both the pilot and classroom case studies support this conclusion. The industrial case study shows that the automated data collection and development behavior inference has the potential to be useful for researchers. 

\begin{comment}
Low-level software processes have become more and more important in the software industry. Watts Humphrey claims that the Team Software Process (TSP) can help to yield higher software quality than the CMM-5, the highest level of CMM certifications. Extreme Programming, one of the agile processes, has been quickly adopted by the software industry. However, the research findings on low-level software processes are often mixed. SDSA lays a foundation that researchers can rely on to study low-level software processes without interrupting software development. The higher level of process conformance assessment with SDSA and Zorro can enable future research in which claims such as ``TDD yields improved software quality'' can be assessed more rigorously.
\end{comment}

\end{abstract}