%%%%%%%%%%%%%%%%%%%%%%%%%%%%% -*- Mode: Latex -*- %%%%%%%%%%%%%%%%%%%%%%%%%%%%
%% cv-abbrev-2008.tex -- 
%% RCS:            : $Id: nsf93-bio.tex,v 1.4 93/10/06 16:50:04 johnson Exp $
%% Author          : Philip Johnson
%% Created On      : Wed Aug 11 14:09:06 1993
%% Last Modified By: Philip Johnson
%% Last Modified On: Sat Nov 22 19:38:30 2008
%% Status          : Unknown
%%%%%%%%%%%%%%%%%%%%%%%%%%%%%%%%%%%%%%%%%%%%%%%%%%%%%%%%%%%%%%%%%%%%%%%%%%%%%%%
%%   Copyright (C) 1993 University of Hawaii
%%%%%%%%%%%%%%%%%%%%%%%%%%%%%%%%%%%%%%%%%%%%%%%%%%%%%%%%%%%%%%%%%%%%%%%%%%%%%%%
%% 
%% History
%% 11-Aug-1993          Philip Johnson  
%%    

\documentstyle[times,11pt,/Users/johnson/tex/lmacros,/Users/johnson/tex/definemargins]{article}

\begin{document}

%%      Command to define new categories:
\newcommand{\newcategory}[1]{\newenvironment{#1}
 {\sectionheading{#1}\par\vspace*{0.03in}\par\hrule\par\begin{description}}{\end{description}\par}}
\newcommand{\sectionheading}[1]{\medskip\pagebreak[2]\par\noindent
 {\bf #1}\nopagebreak}

\newcategory{Degrees}
\newcategory{Tutorial Presentations}
\newcategory{Journal Publications}
\newcategory{Conference Publications}
\newcategory{Workshop Publications}
\newcategory{Publications}
\newcategory{Book Chapters}
\newcategory{Invited Talks}
\newcategory{Awarded Grant Support}
\newcategory{Pending Grant Support}
\newcategory{Professional Activities}
\newcategory{Research and Teaching Experience}
\newcategory{Industry Experience}
\newcategory{Awards and Honors}

\definemargins{1in}{1in}{1in}{1in}{.3in}{.3in}


\begin{center}
{\bf Philip M. Johnson}\\
Information and Computer Sciences \hfill (808) 956-3489

University of Hawaii              \hfill fax: (808) 956-3548

1680 East-West Road               \hfill johnson@hawaii.edu

Honolulu, HI~~~96822              \hfill http://csdl.hawaii.edu/$\sim$johnson/

\end{center}

\begin{Degrees} 
\item Ph.D.~in Computer Science, University of Massachusetts, Amherst. 1990
\item M.S.~in Computer Science, University of Massachusetts, Amherst. 1985 
\item B.S.~in Computer Science,  University of Michigan, Ann Arbor. 1980 
\item B.S.~in Biology, University of Michigan, Ann Arbor. 1980  
\end{Degrees}

\begin{Research and Teaching Experience}
  
\item {\em Professor} \hfill 2001---present 
\vspace*{-10pt}
\item Department of Information and Computer Sciences, University of Hawaii.  
  
 $\bullet$  Director, Collaborative Software Development Laboratory \newline


\item {\em Visiting Professor} \hfill   2006
\vspace*{-10pt}
\item School of Engineering, Blekinge Institute of Technology, Sweden.

 $\bullet$  Research on software measurement. \newline


\end{Research and Teaching Experience}

\begin{Industry Experience}

\item {\em Member, Technical Advisory Board} \hfill 2006-present
\vspace*{-10pt}
\item Sixth Sense Analytics, Raleigh, North Carolina.

  $\bullet$ Sixth Sense Analytics provides software measurement
and analysis services. 

\item {\em Member, Board of Directors} \hfill 1999-2007
\vspace*{-10pt}
\item Hawaii Strategic Development Corporation, Honolulu, Hawaii.

  $\bullet$ HSDC is a State-sponsored organization that works
to support the growth of the venture capital industry in Hawaii.

\item {\em Member, Board of Directors} \hfill 2003-2005
\vspace*{-10pt}
\item Tiki Technologies, Inc., Honolulu, Hawaii.

  $\bullet$ Tiki Technologies develops internet software including
spam detection systems. 

\item {\em Member, Board of Directors} \hfill 2002-2005
\vspace*{-10pt}
\item Lavanet, Inc., Honolulu, Hawaii.

  $\bullet$ Lavanet is an Internet Service Provider, 
Network Engineering, and Web Development Services company.



\item {\em Member, Professional Advisory Board} \hfill 2000-present
\vspace*{-10pt}
\item BreastCancer.org, Philadelphia, PA.

  $\bullet$ BreastCancer.org is a non-profit organization dedicated 
  to helping those living with breast cancer. 

\item {\em Consulting Software Engineer} \hfill 1994---present
\vspace*{-10pt}
\item Honolulu, Hawaii.

  $\bullet$ Providing project management and software engineering services to local and national companies.

\end{Industry Experience}


\begin{Journal Publications}

\item V. Basili and M. Zelkowitz and D. Sjoberg and P. Johnson and T. Cowling,
{\em Protocols in the use of Empirical Software Engineering Artifacts}, 
Empirical Software Engineering, Volume 12, February, 2007.

\item L. Hochstein and T. Nakamura and V. Basili and S. Asgari and 
M. Zelkowitz and J. Hollingsworth and F. Shull and J. Carver and 
M. Voelp and N. Zazworka and P. Johnson, {\em Experiments to 
understand HPC time to development}, CTWatch Quarterly, 
November, 2006.

\item P.~M.~Johnson and H.~Kou and M.~Paulding and Q.~Zhang and
and A.~Kagawa and T.~Yamashita, {\em Improving software development
management through software project telemetry}, 
IEEE Software, Vol. 22, No. 4, July 2005.

\item S.~Faulk and J.~Gustafson and P.~Johnson and A.~Porter and W.~Tichy 
and L.~Votta, {\em Measuring {HPC} Productivity}, 
International Journal of High Performance Computing Applications, December 2004. 

\item P.~M.~Johnson and M.~L.~Moffett and B.~T.~Pentland, {\em
Lessons learned from VCommerce: A virtual
environment for interdisciplinary learning about software entrepreneurship}.
Communications of the ACM, Vol. 46, No. 12, December, 2003.


\end{Journal Publications}  

\begin{Conference Publications}

\item P. Johnson, {\em Requirement and Design Trade-offs in Hackystat: An
in-process software engineering measurement and analysis system},
Proceedings of the 2007 International Symposium on Empirical Software
Engineering and Measurement, Madrid, Spain, September, 2007.

\item P. Johnson and H. Kou, {\em Automated Recognition of Test-Driven Development with Zorro}, 
Proceedings of Agile 2007, Washington, D.C., August, 2007. 

\item H. Scott and P. Johnson, {\em Generalizing fault contents from a few
classes}, Proceedings of the 2007 International Symposium on Empirical
Software Engineering and Measurement, Madrid, Spain, September, 2007.

\item P.~M.~Johnson, H.~Kou, J.~Agustin, Q.~Zhang, A.~Kagawa, T.~Yamashita,
{\em Practical automated process and product metric collection and analysis
in a classroom setting: {L}essons learned from {Hackystat-UH}}, In
Proceedings of the 2004 Symposium on Empirical Software Engineering, Los
Angeles, CA., August 2004.

\item P.~M.~Johnson, H.~Kou, J.~Agustin, C.~Chan, C.~Moore, 
J.~Miglani, S.~Zhen, and W.~Doane, {\em Beyond
the Personal Software Process: Metrics collection and analysis for the
differently disciplined}, In Proceedings of the 2003 International
Conference on Software Engineering, Portland, OR., May, 2003.


\end{Conference Publications}  


\begin{Workshop Publications}

\item P. Johnson, {\em Ultra-automation and ultra-autonomy for software
engineering management of ultra-large-scale systems}, Proceedings of the
2007 Workshop on Ultra Large Scale Systems, Minneapolis, Minnesota, May,
2007.

\item H. Kou, P.~M.~Johnson, 
{\em Automated recognition of low-level process: A pilot validation study
of Zorro for test-driven development}, Proceedings of the 2006 
International Workshop on Software Process, Shanghai, China, May 2006.

\item P.~M.~Johnson, M.~G.~Paulding, 
  {\em Understanding HPC Development through Automated Process and Product Measurement with Hackystat},
  Proceedings of the Second Workshop on Productivity and Performance in High-End Computing (P-PHEC), 
 February, 2005.


\end{Workshop Publications}



\begin{Awarded Grant Support}

\item {\em CSDL Grant}, P.~M.~Johnson, Principal Investigator. Expedia, Inc..  \$25,000. 2008.

\item {\em CSDL Grant}, P.~M.~Johnson, Principal Investigator. Sixth Sense Analytics, Inc.  \$25,000. 2006.

\item {\em Student Engagement Grant}, P.~M.~Johnson, Principal
Investigator. University of Hawaii and Maui High Performance Computing Center.  \$42,000. 2004, 2005.

\item {\em Eclipse Innovation Grant Award}, P.~M.~Johnson, Principal
Investigator.  IBM Corporation. \$15,000. 2004.

\item {\em Supporting development of highly dependable software through
continuous, automated, in-process, and individualized software measurement
validation.}  P.~M.~Johnson, Principal Investigator.  Joint NSF/NASA 
Highly Dependable Computing Program.  \$638,000.  2002-2006.


\end{Awarded Grant Support}

\begin{Professional Activities}

\item {\em Program Co-Chair}, First International Workshop on In-Process
Software Engineering Measurement and Analysis, Montreal, Canada, October, 2007.

\item {\em Program Committee}, Third International Workshop on Software
Engineering for High Performance Computing System Applications, 
Minneapolis, MN,  May, 2007.

\item {\em Editorial Board}, Journal of Empirical Software Engineering, 2004-2007.

\item {\em Program Committee Member}, PROFES 2005-present.

\item {\em Program Committee Member}, Workshop on Productivity and
Performance in High-End Computing, 2005-2006.

\item {\em Program Chair}, Second International Workshop on Software
Engineering for High Performance Computing System Applications, St. Louis,
MO, May, 2005.

\item {\em Program Chair}, First International Workshop on Software
Engineering for High Performance Computing System Applications, Edinburgh,
Scotland, May, 2004.

\item {\em Program Committee Member}, XP/Agile Universe,  Calgary, CA, August 2004.

\item {\em Program Committee Member}, International Software Metrics Symposium, 2003-2004. 

\item {\em Program Committee Member}, International Symposium on Empirical Software Engineering, 2002-2004.

\item {\em Editorial Board}, IEEE Transactions on Software Engineering, 2000-2004.

\item {\em Judge}, 
  Hawaii State Science Fair,
  Honolulu, Hawaii, 1998-present.

\item {\em Founder and Chair}, 
  Hawaii Java Users Group,
  Honolulu, Hawaii, 1996-present.

\item {\em Member}, International Software Engineering Research Network
(ISERN), 1996-present. 

\end{Professional Activities}


\end{document}






