%%%%%%%%%%%%%%%%%%%%%%%%%%%%%% -*- Mode: Latex -*- %%%%%%%%%%%%%%%%%%%%%%%%%%%%
%% 09-01.tex --     Automated Software Engineering submission
%% Author          : Philip Johnson 
%% Created On      : Tue Jan 06 10:41:51 2009
%% Last Modified By: Philip Johnson
%% Last Modified On: Tue Jan 06 11:42:13 2009
%%%%%%%%%%%%%%%%%%%%%%%%%%%%%%%%%%%%%%%%%%%%%%%%%%%%%%%%%%%%%%%%%%%%%%%%%%%%%%%
%%   Copyright (C) 2009  Philip Johnson
%%%%%%%%%%%%%%%%%%%%%%%%%%%%%%%%%%%%%%%%%%%%%%%%%%%%%%%%%%%%%%%%%%%%%%%%%%%%%%%
%% 

\documentclass[smallextended]{svjour3}     % onecolumn (second format)
\usepackage{graphicx}
\usepackage{natbib}
\journalname{Automated Software Engineering}

\begin{document}
\title{Operational Definition and Automated Inference of Test-Driven Design}
\author{Hongbing Kou \and Philip M. Johnson}
\institute{Hongbing Kou and Philip M. Johnson \at 
           Collaborative Software Development Laboratory  \\
           Department of Information and Computer Sciences \\
           University of Hawaii \\
           Honolulu, HI 96822 \\
           Tel.: 808-956-3489\\
           Fax:  808-956-3548\\
           \email{hongbing@hawaii.edu} \\
           \email{johnson@hawaii.edu} \\
}

\date{Received: date / Accepted: date}

\maketitle

\begin{abstract}
Abstract will go here. 
\keywords{Test Driven Design \and Hackystat }
\end{abstract}

\section{Introduction}
\label{intro}
A recent focus of interest in software engineering research is on low-level
software processes, which define how software developers or development
teams should carry on development activities in short phases that last from
several minutes to a few hours. Anecdotal evidence exists for the positive
impact on quality and productivity of certain low-level software processes
such as test-driven development and continuous integration. However,
empirical research on low-level software processes often yields conflicting
results. A significant threat to the validity of the empirical studies on
low-level software processes is that they lack the ability to rigorously
assess process conformance. That is to say, the degree to which developers
follow the low-level software processes can not be evaluated. In order to
improve the quality of empirical research on low-level software processes,
I developed a technique called Software Development Stream Analysis (SDSA)
that can infer development behaviors using automatically collected
in-process software metrics. The collection of development activities is
supported by Hackystat, a framework for automated software process and
product metrics collection and analysis. SDSA abstracts the collected
software metrics into a software development stream, a time-series data
structure containing time-stamped development events. It then partitions
the development stream into episodes, and then uses a rule-based system to
infer low-level development behaviors exhibited in episodes. With the
capabilities provided by Hackystat and SDSA, I developed the Zorro software
system to study a specific low-level software process called Test-Driven
Development (TDD). Experience reports have shown that TDD can greatly
improve software quality with increased developer productivity, but
empirical research findings on TDD are often mixed. An inability to
rigorously assess process conformance is a possible explanation. Zorro can
rigorously assess process conformance to a specific operational definition
for TDD, and thus enable more controlled, comparable empirical studies. My
research has demonstrated that Zorro can recognize the low-level software
development behaviors that characterize TDD. Both the pilot and classroom
case studies support this conclusion. The industrial case study shows that
the automated data collection and development behavior inference has the
potential to be useful for researchers.

Applications of the Hackystat Framework in addition to our work on SDSA and
Zorro include in-process project management \cite{csdl2-04-11}, high
performance computing \cite{csdl2-04-22}, and software engineering
education \cite{csdl2-03-12}.

\begin{acknowledgements}
We gratefully acknowledge the members of the Collaborative Software Development Laboratory (etc.).
\end{acknowledgements}

\bibliographystyle{spbasic}      % basic style, author-year citations
\bibliography{tdd,zorro,csdl-trs,hackystat,psp}
\end{document}


