%%%%%%%%%%%%%%%%%%%%%%%%%%%%%% -*- Mode: Latex -*- %%%%%%%%%%%%%%%%%%%%%%%%%%%%
%% ptr.tex         : 2009 Post Tenure Review
%% Author          : Philip Johnson
%% Created On      : Tue Mar 31 11:16:58 2009
%% Last Modified By: Philip Johnson
%% Last Modified On: Tue Dec 29 08:44:31 2009
%%%%%%%%%%%%%%%%%%%%%%%%%%%%%%%%%%%%%%%%%%%%%%%%%%%%%%%%%%%%%%%%%%%%%%%%%%%%%%%
%%   Copyright (C) 2009 
%%%%%%%%%%%%%%%%%%%%%%%%%%%%%%%%%%%%%%%%%%%%%%%%%%%%%%%%%%%%%%%%%%%%%%%%%%%%%%%
%% 
 
\documentclass[11pt]{article}
\usepackage[final]{graphicx}

%% Make subsubsections numbered and included in ToC
\setcounter{secnumdepth}{3}
\setcounter{tocdepth}{3}

\usepackage{multirow}

%% URLs
\usepackage{url}
\usepackage[colorlinks, bookmarks=true]{hyperref}

%% Define a new 'smallurl' style for the package that will use a smaller font.
\makeatletter
\def\url@smallurlstyle{%
  \@ifundefined{selectfont}{\def\UrlFont{\sf}}{\def\UrlFont{\small\ttfamily}}}
\makeatother
%% Now actually use the newly defined style.
\urlstyle{smallurl}

%% CO2 
\usepackage{xspace}
\newcommand{\COtwo}{CO\ensuremath{_2}\xspace}

%% Make margins less ridiculous
\usepackage{fullpage}

%% Since I'm using the LaTeX Makefile that uses dvips, I need this
%% package to make URLs break nicely
\usepackage{breakurl}

\begin{document}
\title{{\bf Philip M. Johnson \\ 
       Post Tenure Review \\ 
       January 2005 to December 2009}}

\author{Philip M. Johnson \\
      Collaborative Software Development Laboratory \\
      Department of Information and Computer Sciences \\
      University of Hawai'i,  Honolulu, HI 96822 \\
      johnson@hawaii.edu 
}

\maketitle

\section{Research}

My research activities during the past five years involved the following major projects: the Collaborative Software Development Laboratory (CSDL), the Hackystat Project, the Renewable Energy and Island Sustainability Project (REIS), WattDepot, and eSpheres.

\subsection{CSDL}

I established the Collaborative Software Development Laboratory (CSDL) in 1991, shortly after joining the University of Hawaii faculty.  CSDL occupies a suite of offices in the POST Building (Room 307). The goal of CSDL is ``to create a physical, organizational, technological, and intellectual environment conducive to collaborative development of world-class software engineering skills. Through research, education, and technology transfer, we pursue this goal for CSDL members, the University of Hawaii, our affiliates, and the Hawaiian, U.S., and international software research and development communities.''

\paragraph{Students.} Since 2005, I have provided direct research supervision for 6 Ph.D. students, 7 M.S. students, and 4 B.S. students under the auspices of CSDL, as detailed in Figure \ref{fig:students}.

\begin{figure}[ht]
\small
\begin{tabular}{p{2in}p{2in}p{2in}} \hline
Robert Brewer (Ph.D., current) & Pavel Senin (Ph.D., current)  & Shaoxuan Zhang (M.S., 2009) \\
Alexey Olkov (M.S., 2009) & Ka Yee Leung (B.S., 2009) & Cedric Zhang (Ph.D., 2007) \\
Hongbing Kou (Ph.D., 2007) & Mike Paulding (M.S, 2006)  & Takuya Yamashita (M.S., 2005) \\
James Wang (B.S., 2005) & Aaron Kagawa (M.S., 2005) & Christoph Lofi (M.S., 2005) \\
Austin Ito (B.S., 2005) & Julie Sakuda (B.S., 2005) & Randy Cox (M.S., unfinished) \\
David Nickles (Ph.D., unfinished) & \multicolumn{2}{l}{Tryggvi Bjorgvinsson (Ph.D., unfinished)}  \\ \hline
\end{tabular} \\ 
\normalsize
\caption{Students in CSDL since 2005}
\label{fig:students}
\end{figure}

\paragraph{Website.} CSDL has an online presence at
\url{http://csdl.ics.hawaii.edu}. In 2008, we redesigned and reimplemented 
the CSDL website over the course of 12 months with the
assistance of Pam Scott, an LIS student.  The Plone-based CSDL website 
now provides extensive information about the lab, including overviews of
29 current and completed research projects, our online tech
report library with 193 publications, 35 current and former CSDL
members, 15 affiliate organizations, and dozens of news items regarding the lab, as
illustrated in Figure \ref{fig:csdl}.

\begin{figure*}[th]
  \center
  \includegraphics[width=\textwidth]{csdl.eps}
  \caption{The CSDL Home Page}
  \label{fig:csdl}
\end{figure*} 

\newpage
\paragraph{Funding.} Figure \ref{fig:funding} summarizes the 12 funding opportunities (1 pending, 6 awarded, 5 declined) that I have pursued since 2005. All of the funding opportunities were extramural, with the sole exception of Renewable Energy and Island Sustainability, which was a University of Hawaii intramural funding competition involving approximately 15 UH faculty. 

\begin{figure}[ht]
\small
\begin{tabular}{p{4.5in}rrr} \hline
{\bf Title, Organization (Year)} & {\bf Role} & {\bf Amount} & {\bf Status} \\
Human Centered Information Integration for the Smart Grid, NSF (2009) & PI & 381,468 & Pending \\
Renewable Energy and Island Sustainability, UH (2009) & co-PI & 1,000,000 & Awarded  \\
Empirical Computational Thinking, NSF (2009) & PI & 298,237 & Declined \\
Google Summer of Code, Google (2009) & PI & 25,000 & Awarded \\
Google Summer of Code, Google (2008) & PI & 20,000 & Awarded \\
Hackystat Development Donation, Expedia, Inc. (2008) & PI & 25,000 & Awarded \\
A Science of Design Empirical Testbed, NSF (2007) & PI & 367,672 & Declined \\
Discovery and evaluation of software engineering best practices, NSF (2007) & PI & 371,592 & Declined \\
Hackystat Development Donation, Sixth Sense Analytics (2006) & PI & 25,000 & Awarded \\
A continuous, evidence-based approach to discovery and assessment of software engineering best practices, NSF (2005) & PI & 491,096 & Declined\\
Cyberinfrastructure for Empirical Data Analysis and Reuse, NSF (2005) & co-PI & 629,675 & Declined \\
Supporting Development of Highly Dependable Software Through Continuous, Automated, In-process, and Individualized Software Measurement Validation (REU), NSF (2005) & PI & 15,000 & Awarded \\ \hline
\end{tabular} \\ 
\normalsize
\caption{Funding since 2005}
\label{fig:funding}
\end{figure}



\paragraph{Publications.}  Since 2005, members of CSDL have produced a total of 31 publications: 4 journal articles, 4 conference publications, 2 Ph.D. theses, 3 Ph.D thesis proposals, 4 M.S. theses, 3 Workshop publications, and 11 technical reports.  All of these are available online through the CSDL website.  Figure \ref{fig:pubs} lists these publications. 

My C.V. provides citations for my personal publications.  Since 2005, I have authored or co-authored 4 journal articles, 4 conference articles, and 3 Workshop articles. In addition, I have authored 6 technical reports not listed in my C.V.


\begin{figure}[!ht]
\small
\begin{tabular}{p{5.5in}r} \hline
{\bf Title} & {\bf Type}  \\
{\em 2009} & \\
Operational Definition and Automated Inference of Test-Driven Development, H. Kou et al  & Journal Pub. \\
Proposal for Electricity Conservation Experiments in Saunders Hall, R. Brewer &  Tech Report \\
Learning Empirical Software Engineering Using Software Intensive Care Unit, S. Zhang  & M.S. Thesis \\
Software Trajectory Analysis: An empirically based method for process discovery, P. Senin  & Ph.D. Prop. \\
Experiences with Hackystat as a service-oriented architecture, P. Johnson et al  & Tech Report \\
Literature review on carbon footprint collection and analysis, R. Brewer  & Tech Report \\
Results from the 2008 Classroom Evaluation of Hackystat, S. Zhang, P. Johnson  & Tech Report \\
We need more coverage, stat! Experience with the Software ICU, P. Johnson et al  & Conf. Pub. \\
{\em 2008} & \\
Using simulation to investigate {IT} micro-processes, A. Olkov, D. Port  & Tech Report \\
Dynamic Time Warping Algorithm Review, P. Senin  & Tech Report \\
Carbon Metric Collection and Analysis with the Personal Environmental Tracker, R. Brewer  & Conf. Pub \\
{\em 2007} & \\
Automated Inference of Software Development Behaviors, H. Kou  & Ph.D. Thesis \\
Ultra-automation and ultra-autonomy for software engineering management, P. Johnson  & Workshop Pub \\
Automated Recognition of Test-Driven Development with Zorro, P. Johnson, H. Kou  & Conf. Pub. \\
Protocols in the use of Empirical Software Engineering Artifacts, V. Basili et al  & Journal Pub. \\
Requirement and Design Trade-offs in Hackystat, P. Johnson  & Conf. Pub \\
{\em 2006} & \\
Evaluation of Jupiter: A Lightweight Code Review Framework, T. Yamashita & M.S. Thesis \\
Experiments to understand HPC time to development, L. Hochstein et al  & Journal Pub. \\
Automated Inference of Software Development Behaviors, H. Kou  & Ph.D. Prop \\
Results from the 2006 Classroom Evaluation of Hackystat-UH, P. Johnson  & Tech Report \\
Improving Software Development Process and Product Management, Q. Zhang  & Ph.D. Thesis \\
Automated recognition of low-level process, H. Kou, P. Johnson  & Workshop Pub. \\
Actual Process: A Research Program, L. Prechalt et al.  & Tech Report \\
{\em 2005} & \\
Continuous GQM, C. Lofi  & M.S. Thesis \\
Telemetry Plate Lunch Contest Results, P. Johnson  & Tech Report \\
Readings in Empirical Evaluation for Budding Software Engineering Researchers, P. Johnson  & Tech Report \\
Studying Micro-Processes in Software Development Stream, H. Kou  & Tech Report \\
Priority Ranked Inspection, A. Kagawa  & M.S. Thesis \\
Understanding HPCS development, P. Johnson, M. Paulding  & Workshop Pub \\
Improving Software Development Management with Software Project Telemetry, Q. Zhang  & Ph.D. Prop. \\
Improving Software Development Management through Software Project Telemetry, P. Johnson et al  & Journal Pub \\ \hline
\end{tabular} \\ 
\normalsize
\caption{CSDL publications since 2005. Some titles elided to improve formatting}
\label{fig:pubs}
\end{figure}

\newpage
\subsection{Hackystat}

\paragraph{Overview.} Hackystat is an open source, scalable, extensible framework for collection,
analysis, interpretation, and dissemination of software engineering process
and product data.  It has been under continuous development under my
supervision for almost a decade with software contributions from over 30
developers and financial support from NSF, IBM, Google, SUN Microsystems,
NASA, Expedia, and others.  Figure \ref{fig:hackystat} illustrates the
Hackystat Project Home Page.


\begin{figure*}[ht]
  \center
  \includegraphics[width=\textwidth]{hackystat.eps}
  \caption{The Hackystat Project Home Page}
  \label{fig:hackystat}
\end{figure*} 

\paragraph{Recent research and development.} During the past five years, we
have worked extensively on improving the design, implementation, and
functionality of the system.  Hackystat currently comprises over 250,000
lines of code, written in a variety of languages, and spread over more than
40 interdependent software components.  Two years ago, we reimplemented the
system in order to implement a service-oriented architecture.  In
Hackystat, data is first gathered by ``sensors'' and sent to a low-level
repository for storage.  The data is abstracted by one or more middleware
analysis components, and finally presented to the user through one or more
user interface components.  Hackystat components typically implement a
RESTful API, and user interface components have included email, text
messages, ambient devices (Orb, Nabaztag), web applications (Wicket),
social networks (Facebook), and micro-blogging platforms (Twitter).

As indicated by Figure \ref{fig:pubs}, Hackystat has supported a
substantial amount of research during the past five years, including 2
Ph.D. theses, 3 M.S. theses, 4 journal articles, 3 conference articles, and
2 workshop articles.

\paragraph{Google Summer of Code.} Google selected Hackystat as a
participating project for its Google Summer of Code program in both 2008
and 2009.  Google Summer of Code is a global program that offers student
developers stipends to write code for various open source software
projects.  Participation in the program is highly competitive;
approximately 140 projects world-wide are selected each year.  Our
participation enabled 4 students in 2008 and five students in 2009 to work
on Hackystat with funding from Google.



%% \newpage
%% \bibliography{csdl-trs}
%% \bibliographystyle{plain}


\end{document}


