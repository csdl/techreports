\chapter{Hackystat-Trajectory - software process mining framework.} \label{trajectory}
In my work for this thesis I focus on the mining of the recurrent behaviour patterns from the observed software process. As we seen in the Chapter \ref{tconcepts}, it is possible to construct symbolic time-point and time-interval series directly from the real-valued telemetry streams provided by Hackystat and find a set of frequently occuring patterns. By characterizing these frequent patterns through a common taxonomy and successively applying process inference techniques shown in the Chapter \ref{related.work} it is possible to reconstruct and formalize software process (however it will be out of the scope of this thesis).

\section{Framework architecture overview}
The overview of the framework I am working on is shown at the Figure \ref{fig:system_overview} and resembles the flow of the Knowledge Discovery in Database process discussed in Han et al \cite{citeulike:709476}. As shown, the data collected by Hackystat is getting transformed into symbolic format and indexed for further use in the data mining process later. The data-mining tools I am designing are restricted in the search space with the domain and context specific knowledge in order to limit the amount of reported to the user patterns to the useful ones. I am planning to design a GUI in the way which allows easy acces and modification of such rules. 

The Figure \ref{fig:data_flow} shows the data abstraction process within the Hackystat Trajectory framework in a greater detail. Collected and aggregated by Hackystat, raw sensor data and Hackystat Telemetry streams are used as the data sources. By using a user-defined taxonomy mapping, the stream of individual events retrieved from Hackystat Sensorbase getting sorted by the activities, tokenized and converted into symbolic time point and time-interval (Episodes) series. By performing a user-configured PAA and successive SAX approximations Hackystat telemetry streams are getting converted into the same temporal symbolic format.

The symbolic temporal data from two data sources than used in the unsupervised pattern mining with a user-specified restrictions. 

\begin{figure}[tbp]
   \centering
   \includegraphics[height=100mm]{data_flow.eps}
   \caption{The overview of the data abstraction from the the low-level process and product artifacts collected by Hackystat (left side) to the high-level symbolic time-point and time-interval series stored in the Trajectory data repository.}
   \label{fig:data_flow}
\end{figure}

\section{Database schema design}

\section{Roadmap}
