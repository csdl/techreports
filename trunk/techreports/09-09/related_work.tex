\chapter{Related work} \label{related.work}
Although, process mining in the business domain is a well-established field with many work done and software developed up to date (ERP, WFM and other systems), the Business Process Intelligence tools usually do not allow to perform a process discovery and typically offer relatively simple analyses that depend upon a correct a-priori process model \cite{citeulike:3718014} \cite{citeulike:5044991}. This fact restricts a direct application of the business domain process mining techniques to the general process mining and especially to the software engineering, where processes are usually performed concurrently by many agents, are more complex and typically have a higher level of noise. Taking this fact in account, I will review only some of the existing approaches to the general process mining for which applicability to the software process mining was expressed. 

Three papers are reviewed in this chapter: 
\begin{itemize}
	\item Cook \& Wolf in \cite{citeulike:328044} discuss an event-based framework for the process discovery based on the grammar inference and finite state machines. Authors directly applied their framework to the Software Configuration Management (SCM) logs demonstrating satisfiable results. 
	\item Van der Aalst et al \cite{citeulike:3718014} demonstrate the applicability of the Transition Systems and labeled Petri nets to the process discovery in general. While this paper is not inferring an application to the software process, the subsequent work by van der Aalst and Rubin \cite{citeulike:1885717} discusses software process application.
	\item The third paper, by Jensen \& Scacchi \cite{citeulike:5043664}, while not presenting a pattern mining strategy describes an interesting framework built upon an universal generic meta-model and specific to the observed processes models which are iteratively built and revised during case studies. The value of this paper is in the demonstration of the importance of the correct mapping between process artifacts and process entities as well as a demonstration of iterative, human-involved technique of process revision which is emphasizing importance of pre-existing domain knowledge in the effective pruning of the search space.
\end{itemize}
As pointed by authors in the reviewed papers, proposed methods have difficulties while dealing with concurrency, which, in turn, is inevitable in the software process usually performed by many agents. Many successive work has been done extending reviewed approaches to the concurrent processes. Among others, Weijters \& van der Aalst in \cite{citeulike:5128101} propose heuristics application to handle concurrency and noise issues, while van der Aalst et al in \cite{citeulike:5128110} discuss a genetic programming application. 
