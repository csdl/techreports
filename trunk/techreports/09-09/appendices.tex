\chapter{Appendices} \label{appendix}

\section{Classroom case study interview design} \label{survey}
I plan to conduct at least one classroom interview session evaluating Software Trajectory framework. I am targeting graduate students as my primary responders. These interviews will be conducted under approved by Committee on Human Subjects application: \textit{CHS \#16520 - Evaluation of Hackystat}. By performing interviews, I expect to collect evidence about the ability of Software Trajectory framework to capture recurrent behaviors from software process performed on the team and individual level. My secondary goal is the evaluation of the significance of these behaviors for software process understanding and improvement. 

I plan to administer interviews during the last week of instructions. I will select responders demonstrating recurrent behaviors by balancing two factors: first is to cover as much various discovered patterns as possible, and second is to cover patterns demonstrated by the most of students. Taking in account the possible students' concern about final grading, I am crafting my interview questionnaire in order to make it as neutral as possible to curriculum. The current version of the interview questionnaire is presented further. 

First three questions are designed with a purpose of assessing of respondents' level of education, area of expertise, and knowledge of software development patterns. Fourth question brings a general discussion to the classroom experience and uncovers respondents' opinion about performed and observed by me process. By introducing my research in greater detail at this point, and by walking through the last three questions I will try to encourage responder to discuss with me my findings. Depending on the responders' reaction and willingness to continue, I will finish the interview after two to five iterations over observed patterns.

The interviews will be tape-recorded, later they will be transcribed, coded, and statistically analyzed. I am not very familiar with a methodology of developing of coding procedures yet, moreover, at this point of time it is impossible to foresee any of the results of Software Trajectory analyses. Thus, interview coding scheme will be constructed after interview-discussed (``candidate'') patterns will be identified.

\subsection{Software Trajectory evaluation interview questionnaire}
\begin{enumerate}
	\item What is your level of education and major?
	\item Do you have any previous experience with software development?
	\item Do you aware about recurrent behaviors (philosophies/approaches/styles) in the software development? If so, which ones can you recall? 
	\item Did you ever follow any of the formal approaches or styles in your own, professional, or the classroom development?
	\item By applying Software Trajectory framework I was able to capture a behavioral pattern $P$ which looks like $P_{1} \rightarrow P_{2} \rightarrow ...$. My interpretation of this is .... . Does my interpretation of this recurrent behavior match your own understanding of your development behavior? 
	\item Is there another interpretation of this pattern that you believe more accurately reflects your behavior?
	\item When you think about this pattern, can you think of changes you might want to make to your development behaviors based upon it? 
\end{enumerate}

\clearpage

\subsection{Software Trajectory consent form}
Thank you for agreeing to participate in our research on software process discovery.  
This research experiment is being conducted by Pavel Senin as a part of his Ph.D. research in Computer Science under the supervision of Professor Philip Johnson.

As a part of this research experiment, you will be asked to provide a feedback about automatically discovered behavioral patterns in your software development process. These patterns will be automatically discovered by Software Trajectory framework which applies data mining techniques to the telemetry streams collected by Hackystat. The primary goal of this research is to evaluate the ability of Software Trajectory to discover and classify recurrent behavioral patterns; while the secondary goal is to assess the usefulness of discovered behaviors. 

The data that we collect will be kept anonymously, and there will be no identifying information about you in any analyses of this data. 

Your participation is voluntary, and you may decide to stop participation at any time, including after your data has been collected. If you are doing this task as part of a course, your participation or lack of participation will not affect your grade.

If you have questions regarding this research, you may contact Professor Philip Johnson, Department of Information and Computer Sciences, University of Hawaii, 1680 East-West Road, Honolulu, HI 96822, 808-956-3489.  If you have questions or concerns related to your treatment as a research subject, you can contact the University of Hawaii Committee on Human Studies, 2540 Maile Way, Spalding Hall 253, University of Hawaii, Honolulu, HI 96822, 
808-539-3955.

Please sign below to indicate that you have read and agreed to these conditions. 


Thank you very much! \\[2.0cm]


Your name/signature

Cc: A copy of this consent form will be provided to you to keep. 
