\chapter{Introduction}

\section{Software development process as a human activity}
I would agree with Gerhard Fischer \cite{citeulike:4913213} that Software Engineering ``\textit{should be considered a human activity}'' as well as with his arguing about primary source of difficulties in the system engineering - social interaction between users and system developers along with communication between developers. Clearly, as the software system is growing and getting more complex, the specialization of individual developers increases and inter-developers collaboration becomes everyday necessety. From other hands, too often, minor changes in the mature system requested by users result in a major re-engineering effort and associated costs, which emphasizes the value of early communication between users and developers.

Another thought expressed by Fischer and inspired by work of Dawkins \cite{citeulike:606469} and Simon \cite{citeulike:143101} is that the software engineering is an evoultion process. The specifics of this lie in the iterative process of refinment of both: user requirerements for a software system and the software system design and implementation. Usually multiple stakeholders closely interact with system developers refining system requirements but this process never completes until the system actually developed. Once developed, a software system usually goes into production phase and later returns to developers for re-engineering with new, evolved, requirements. This iterative process demonstartes a continous co-evolution of software and requirements.

Fischer argues that observations of the importance of communications along with understanding of underlying evolution of requirements and design led to development of the ``Design for Evolution'' model

