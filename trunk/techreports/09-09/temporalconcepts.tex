\section{Temporal concepts} \label{tconcepts}
The SAX transformation procedure described in the previous section yields symbolic time-series based on the daily aggregated data from telemetry streams. Having such a symbolic representation which called in the literature as ``symbolic temporal data'' is very adventageous comparing to the real-valued data in terms of in-memory manipulations, pattern search, indexing and clustering. In this section we will follow the technical report by Fabian M\"orchen \cite{citeulike:1748833} which aggregates many of the work done in the field of data mining from symbolic temporal data.

It is important also to understand a time-intervals temporal data model before introducing concepts and operators. The time-intervals are continous groups of discrete time instants and some applications operate with them instead of individual points. Time-intervals essentially are sets of two or more points. Two time points define a minimal interval which starts at the earlier point and continues to the latter point inclusevily.

Two figures taken from the original work constitute the Figure \ref{fig:concepts1} and depict hierarchy of the time-points and operators (left panel) along with time-intervals and applicable operators (right panel).

The concept of \textit{concurrency} as described by the author explains the closeness of two time-points in time without considering their ordering, what is most important here is that a coincidence of events in time. The \textit{synchronicity} is a special case of concurrency relation where events occur synchonously in time.

The \textit{order}, and \textit{synchronicity} concepts in the Time intervals model are anlogous ones in the Time points model whether the \textit{coincidence} describes an intersection of intervals in time.

The time point operators from the Figure \ref{fig:concepts1}: \textit{before}, \textit{after} and \textit{equals} precizely define the relation of points in time whether the \textit{close} operator is a ``fuzzy extension for temporal reasoning'' since it encapsulates other three. Note that some threshold can be used to relax or constrain these operators, for example we can consider points equal to each other even if they are less than $k$ time units apart.


\begin{figure}[tbp]
   \centering
   \includegraphics[height=45mm]{concepts1.eps}
   \caption{The temporal concepts and operators from \cite{citeulike:1748833} for both: time point and time interval data models.}
   \label{fig:concepts1}
\end{figure}