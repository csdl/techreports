\section{Temporal patterns and indexing} \label{tpatterns}
In previous sections of this chapter we have shown the PAA and SAX algorithms for conversion of real valued time series into the symbolic representation. Later we introduced temporal data models, concepts and operators. All these was a necessary background in order to continue with review of approaches for unsupervised knowledge mining from a symbolic temporal data. In this section we will review sequential pattern mining algorithms from time points and time-intervals data. We will base the review on the univariate data extending it to the multivariate.

Before discussing algorithms we need to formally define patterns of interest we are looking for. There are two well-established categories of patterns. First kind of patterns is very important in many data mining areas such as medicine, motion-capture, robotics, video surveillance, meteorology and others detection of \textit{repeated} or \textit{approximately repeated} patterns plays essential role. This kind of frequently occuring, repeated patterns called \textit{motif}. Note, that the temporal motif finding problem is very similar to one of the central problems in the field of computation biology \cite{citeulike:465665} and many algorithms are very similar, but, from other hands, while in the biology motifs are usually informative and bear some information about evolutionary artifacts, it is not true in the field of time-series analysis \cite{citeulike:3978085}. Second kind of patterns important in the mentioned fields as well as in other areas of time-series analysis. These patterns called \textit{surprise} or (\textit{novelty}) patterns. Usually they are appearing in the data set uniquely or with some frequency highly different from the expected one. These patterns have a great value for many applications: for example it is important to detect unusual semi-repeated pattern in the ECG data diagnosing heartbeat abnormalities, or detecting unusual activity patterns in video surveillance recognizing suspicious activity.

\subsection{Time points patterns}
According to M\"orchen, the most commonly searched pattern within univariate symbolic time series is order. This search for particular order of symbols within subsequence is called \textit{sequential pattern mining} \cite{citeulike:775528} and not necesserely requires symbols to be consecutive, usually gaps and substitution allowed.

The classical suffix tree algorithm \cite{citeulike:707616} with some modifications is a standard approach for pattern discovery from the string time-series according to Palopoli et al \cite{citeulike:5003338}, in this paper authors discussing algorithms of automatic discovery of frequent structured patterns (\textit{motifs}) in ``exact'' or ``approximate'' forms. There are two approaches usually used for the suffix tree building used: \textit{generative}, when algorithms generates all possible patterns and tests their appearence frequency \cite{citeulike:5012661} and \textit{scanning}, when the sliding window used to scan over the sequences available and construct the tree on the fly. 

citeulike:5012661 The certain limitation of the algorithm is that the maximum length of a pattern needs to be specified upon tree construction since all sub-sequences of this length are extracted from the time series with a sliding window. In \cite{citeulike:5003404} Jiang \& Hamilton compare traversal strategies for suffix trees: breadth-first, depth-first and the heuristic depth-first algorithms implementations for temporal data mining.

As per \textit{surprise pattern} finding problem, Keogh et al in \cite{citeulike:3025877} discuss methods of finding a surprise patterns from the temporal data and propose their ``TARZAN'' algorithm which is based upon suffix tree and Markov model and reporting surprising patterns occurring with a frequency substantially different from that one expected by a chance.

