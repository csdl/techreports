\section{Temporal patterns and indexing} \label{tpatterns}
In previous sections of this chapter we have shown the PAA and SAX algorithms for conversion of real value time series into symbolic representation. Later we introduced temporal data models, concepts and operators. All these was a necessary background in order to continue with review of approaches for unsupervised knowledge mining from a symbolic temporal data. In this section we will review sequential pattern mining algorithms from time points and time-intervals data. We will base the review on the univariate data extending it to the multivariate.

Before discussing algorithms we need to define patterns of interest we are looking for. There are two well-established categories of patterns. For many data mining areas such as medicine, motion-capture, robotics, video surveillance, meteorology and others detection of \textit{repeated} or \textit{approximately repeated} patterns plays essential role. These repeated patterns are named as \textit{motifs}. This problem is very similar to one of the central problems in the field of computation biology \cite{citeulike:465665} and many algorithms are very similar, but, from other hands, while in the biology motifs are usually informative and bear some information about evolutionary artifacts, it is not true in the field of time-series analysis \cite{citeulike:3978085}. Another problem of patterns detection in the mentioned and other areas of time-series analysis is a discovery of the \textit{unusual} (\textit{novelty}) patterns. These patterns have a great value for many applications: for example it is important to detect unusual semi-repeated pattern in the ECG data diagnosing heartbeat abnormalities, or detecting unusual activity patterns in video surveillance recognizing suspicious activity. These patterns, which appear uniquely or with some frequency highly different from expected, named as \textit{surprise patterns} \cite{citeulike:3025877}.

\subsection{Time points patterns}
According to M\"orchen, the most commonly searched pattern within univariate symbolic time series is order. This search for particular order of symbols within subsequence is not necesserely requires symbols to be consecutive, usually gaps and substitution allowed. 

The classical suffix tree algorithm \cite{citeulike:707616} with some modifications is a standard approach for pattern discovery from the string time-series according to Palopoli et al \cite{citeulike:5003338}, in this paper authors discussing algorithms of automatic discovery of frequent structured patterns (\textit{motifs}) in ``exact'' or ``approximate'' forms. 

As per \textit{surprise pattern} finding problem, Keogh et al in \cite{citeulike:3025877} discuss methods of finding a surprise patterns from the temporal data and propose their ``TARZAN'' algorithm which is reporting surprising patterns occurring with a frequency substantially different from that expected by chance.