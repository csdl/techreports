\chapter{Experimental evaluation} \label{experiments}
Three case studies: a pilot study, a classroom case study, and a public data case study will be conducted in order to empirically evaluate capabilities and performance of the Hackystat Trajectory framework. The primary goal of these studies will be to assess the ability of the framework to reproduce well known recurrent behavioral patterns (for example TDD), as well as the ability to discover novel ones.

\section{Pilot project evaluation}
In order to demonstrate the ability of the current system implementation to perform telemetry indexing and temporal recurrent patterns extraction I have conducted two experiments. 

For the first experiment, aiming a classification of Telemetry streams, I was using the real data collected during the Spring'09 software engineering class. This dataset represents Hackystat metrics collected during sixty days of the classroom project development conducted by eight students. Following experiments were conducted:
\begin{itemize}
	\item Clustering of the software process related telemetry streams collected from individual developers by using motif frequencies and various PAA and SAX approximation settings. As results of this experiment I was able to cluster developers with the similar behavioral patterns, which proved the correctness of the approach to measurements through motif frequencies.
	\item Clustering of the software product-related telemetry streams by using motif frequencies and various PAA and SAX approximation settings. As results of this experiment I was able to cluster telemetry streams and select most optimal PAA and SAX settings.
\end{itemize}

\begin{figure}[tbp]
   \centering
   \includegraphics[height=80mm]{sequential_growth.eps}
   \caption{The illustration of finding of sequential $growth \; pattern$ in two DevTime telemetry streams. Panel $a$: The Hackystat ProjectBrowser showing telemetry streams. Panel $b$: the TrajectoryBrowser showing same telemetry streams along with identified pattern. Panel $c$: the symbolic representation of streams with highlighted pattern.}
   \label{fig:sequential_growth}
\end{figure}

The second experiment, aiming a discovery of sequential patterns was also conducted by using a real data from my own concurrent development of two software projects. In my development I decided to separate the development of the algorithms and GUI into two pieces, where TrajectoryBrowser is heavily depends on the JMotif library. The JMotif library implements variations of DTW, PAA and SAX algorithms along with defining data structures for indexing and clustering. Following iterative pattern in my development I was changing JMotif public API three times, which consequently involved extensive refactoring in the ProjectBrowser code, see Figure \ref{fig:sequential_growth} panel $a$.

In order to capture this dependency pattern in two Telemetry streams representing daily amount of development time spent on the TrajectoryBrowser and JMotif projects I have defined a synthetic \textit{growth pattern} which corresponds to the sudden growth in the development time (large positive delta value between previous and current day). By transforming the Telemetry streams with this simple rule in the symbolic form I have obtained a two dimensional symbolic time series, where letter $G$ represents a growth pattern, see Figure \ref{fig:sequential_growth} panel $c$. I have defined a formal rule for sequential growth search in this streams: \textit{sequential growth} pattern is the pattern like $G_{JMotif}\; \rightarrow \; G_{TrajectoryBrowser}$ where distance between these $G$s is less than three days. By application of this rule I have identified a pattern which exactly corresponds to my experience.
