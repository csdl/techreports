\chapter{Methods} \label{methods}
My work in the process discovery mainly rests on a behavioral representation of software processes as a sequence of events performed by individual developers or automaton. This event-based model of process actions, where an event characterizes the dynamic behavior of a software process such as invoking a build tool or performing a test, extended with a temporal data of effort applied to the software development as well as other low-level artifacts of the development process such as buffer transfers or compilation activities. 

The collection of development activities is performed by Hackystat, a framework for automated software process and product metrics collection and analysis. My system extracts the data with desired granularity from the Hackystat and converts it into the symbolic representation by performing Piecewise Aggregate Approximation (\ref{paa}) and Symbolic Aggregate approXimation.  Both approximation methods are discussed in the Section \ref{sax}.

The Temporal Concepts section (\ref{tconcepts}) introduces data models of \textit{time-points} and \textit{time-intervals} on the symbolic sequences along with \textit{temporal concepts} and applicable \textit{temporal operators}. 

The last section of this chapter, Temporal patterns  and indexing (\ref{tpatterns}), defines \textit{temporal patterns} (\textit{motifs} and \textit{surprise}) along with discussing relevant pattern search algorithms and data structures used for patterns indexing.