\documentclass[11pt,oneside]{article}
\begin{document}
\title{Dissertation proposal abstract}
\author{Pavel Senin \\
 \texttt{senin@hawaii.edu}
}
\date{July 2009}
\maketitle

\section{Motivation}
Since the first computers were build and first programs written many research was done on understanding of both: \textit{programming} and \textit{programs}. In contemporary understanding the human activity called ``\textit{programming}'' consists of many iterative phases and interleaving activities such as planning, writing code, testing, debugging, and maintaining the source code of computer programs. All these high-level phases are also aggregating many of low-level processes and episodes. In addition to that, the social interactions among developers and between developers and users are adding even more complexity into each phase of the programming. The computer program (\textit{system}, or \textit{software}) itself, from other hands, has it's own lifecycle which obviously directed by the programming effort but also, and in many cases, program, especially large software systems, can be found ``orchestrating'' needed programming activities and social interactions. All of this creates great process complexity and spurious connections between different activities making understanding of software development process and software evolution difficult.

The process of human ``understanding'' involves measurements, comparisons and various experiments. Over the years many work has been done in order to discover and standardize metrics for programming activity and for the computer systems. Currently we have many software utilities aiding the collection and analysis of software metrics. The Hackystat system, originated from University of Hawaii, is one of them and provides users with an ``one-stop shopping place'' for metrics collection utilities, storage database, analysis engine and visualization modules. The latest Hackystat implementation is a sophisticated distributed, service oriented system which provides users not only with all the metrics, but aids the understanding of metrics trends dynamic through the system of rules and indicators.

All of these creates a rich background for a further investigation 




\end{document}
