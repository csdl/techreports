% 11-07.tex
% FDG 2012 conference paper
% WORKS WITH V3.2SP OF ACM_PROC_ARTICLE-SP.CLS
% November 2011
% Author: Yongwen Xu, George Lee, Philip Johnson
%
% ----------------------------------------------------------------------------------------------------------------
% This .tex file (and associated .cls V3.2SP) *DOES NOT* produce:
%       1) The Permission Statement
%       2) The Conference (location) Info information
%       3) The Copyright Line with ACM data
%       4) Page numbering
% ---------------------------------------------------------------------------------------------------------------
% It is an example which *does* use the .bib file (from which the .bbl file
% is produced).
% REMEMBER HOWEVER: After having produced the .bbl file,
% and prior to final submission,
% you need to 'insert'  your .bbl file into your source .tex file so as to provide
% ONE 'self-contained' source file.
%

\documentclass{acm_proc_article-sp}

\begin{document}

\title{Makahiki: An Open Source Game Engine for Energy Education and Conservation}

%\numberofauthors{1} 

\author{
\smallskip
George E. Lee\\ 
\smallskip
Yongwen Xu\\ 
\smallskip
Robert S. Brewer\\ 
\smallskip
Philip M. Johnson\\
       \affaddr{Information and Computer Sciences}\\
       \affaddr{University of Hawai`i at M\=anoa}\\
       \affaddr{Honolulu, HI 96822}\\
       \email{[gelee, rbrewer, yxu, johnson]@hawaii.edu}
}

\maketitle
\begin{abstract}
  The rising cost, increasing scarcity, and climate impact of fossil
  fuels as an energy source makes a transition to cleaner, renewable energy
  sources an international imperative.  This paper presents Makahiki, an
  open source game engine for energy education and conservation.  Makahiki
  facilitates the implementation of ``serious games'' that motivate players
  to learn about energy issues, improve their intuition about the energy
  impact of appliances and behaviors, and enable them to discover how to
  use energy more efficiently in their normal life.  Makahiki has been
  used to implement ``The Quest for the Kukui Cup'', a three week energy
  challenge for over 1,000 first year students living in residence halls at
  the University of Hawaii in Fall, 2011.   Evaluation of this initial
  deployment of Makahiki has revealed useful insights into its game
  mechanics, ways to improve the next Kukui Cup challenge, and the
  challenges when adapting it to other energy contexts.
\end{abstract}

% A category with the (minimum) three required fields
\category{L.5.1}{Game-based Learning}{Gaming}

\terms{Human Factors, Games, Education, Motivation}

\keywords{Serious Games, Education, Gamification}% NOT required for Proceedings

\section{Introduction}

The rising cost, increasing scarcity, and climate impact of fossil fuels as
an energy source makes a transition to cleaner, renewable energy sources an
international imperative.  One barrier to this transition is the relatively
inexpensive cost of current energy, making financial incentives less
effective.  Another barrier is the success that electrical utilities have
had in making energy ubiquitous, reliable, and easy to access, thus
enabling widespread ignorance in the general population about basic energy
principles and trade-offs.  In Hawaii, the need for transition is
especially acute, as the state leads the nation both in the price of energy
(over \$0.30/kWh) and reliance on fossil fuels as an energy source (over
90\% from oil and coal).

Moving away from petroleum is a technological, political, and social
paradigm shift, requiring citizens to think differently about energy
policies, methods of generation, and their own consumption than they have
in the past.  Unfortunately, unlike other civic and community issues,
energy has been almost completely absent from the educational system. To
give a sense for this invisibility, public schools in the United States
generally teach about the structure and importance of our political system
(via classes like ``social studies''), monetary issues (though ``(home)
economics''), nutrition and health (through ``health''), and even sports
(through ``physical education'').  But there is no tradition of teaching
``energy'' as a core subject area for an educated citizenry, even though energy
appears to be one of the emergent issues of the 21st century.

Another emergent issue, at least for the first part of the 21st
century, is the explosive spread of game techniques, not only in its
traditional form of entertainment, but across the entire cultural spectrum.
The adoption of game techniques to non-traditional areas such as finance,
sales, and education has become such a phenomenon that the Gartner Group
included ``gamification'' on its 2011 Hype List, positioning it near the
summit of ``inflated expectations'' (after which it is projected to fall
into the ``trough of disillusionment''). 

This paper describes Makahiki, an open source game engine for energy
conservation and education, in which we attempt to create synergy between
these two emergent issues.  The result of over two years of research and
iterative development, Makahiki explores one section of the design space
where virtual world game mechanics are employed to affect real world energy
behaviors.  The ultimate goal of the Makahiki project is to learn how to
not just affect energy behaviors during the course of the game, but to
produce more long lasting, sustained change in energy behaviors and
outlooks by participants. 

We used Makahiki to create an energy challenge called The Quest for the
Kukui Cup for approximately 1,000 first year students living in the
residence halls at the University of Hawaii in Fall, 2011.  During the
three weeks of the competition, over 400 of the eligible students played
the game, for a total of 850 game play hours.  In addition to online play,
the Quest for the Kukui Cup integrated 24 real world events, including
workshops on energy-related matters, excursions to wind farms and other
energy related locations, and energy-related activities on campus. The game
mechanics were designed to create a self-reinforcing ``virtuous circle''
between the real world and virtual world activities.  The challenge was
very successfully received and plans are already underway to both repeat
the challenge in 2012 for University of Hawaii first year students, and 
adapt it to other residence halls and other universities in Hawaii.

The next section briefly reviews the research on which we based Makahiki
and the Quest for the Kukui Cup.   We then discuss the architecture and
game mechanics of the system, followed by the lessons we learned from its
first deployment. We conclude with our current goals for improvements to
the system.

\section{Related Work (Yongwen)}
Our research draws on work we've done previously
\cite{csdl2-10-05,csdl2-10-07,csdl2-11-02,csdl2-11-03}, and well as from
work done by others \cite{Lazzaro2010}.  But mostly we just make stuff up.

\section{System Design (George)}

\subsection{Architecture}
The architecture is ...

\subsection{Game Mechanics}
The game mechanics is ...

\subsubsection{Smart Grid Game}
\subsubsection{Daily Energy Goal Game}
\subsubsection{Raffle Game}
\subsubsection{Social and Referral Bonuses}
\subsubsection{Confirmation Code}
\subsubsection{Quest Engine}

\section{Lessons Learned (Philip)}
In Kukuicup.....

\section{Future Directions (Philip)}
From the above .....

\section{Acknowledgments}
Makahiki has been supported in part by grant IIS-1017126 from the National
Science Foundation, and by funding from the University of Hawaii Office of Facilities
Management. 

\bibliographystyle{abbrv}
\bibliography{csdl-trs,gamification}  

\balancecolumns

\end{document}
