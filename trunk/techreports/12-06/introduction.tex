\section{Introduction}
Several characteristics of the traditional electrical energy infrastructure of industrial societies have remained unchanged for almost 100 years.  First, energy production has been centralized in power plants using ``firm'' energy sources such as coal, oil, nuclear, or hydro.  Second, centralized production has promoted centralized control of the grid, typically through a single or small number of utilities with public regulation over their policies and rates.  Third, centralized production and control has led to the predominence of ``macro-grids'', or grid infrastructures designed to service hundreds of thousands to millions of consumers. Finally, traditional grids have been designed to minimize the information about energy required by consumers to utilize the service.  The typical consumer needs to know almost nothing more than how to plug an appliance into an outlet, and can assume that this exceedingly simple ``user interface'' will provide virtually unlimited amounts of high quality energy at any time for a relatively small cost.

Unfortunately, the accelerating world-wide growth in demand for energy is leading to a breakdown in this approach to electrical energy infrastructure.  Petroleum products such as coal and oil are nonrenewable, are no longer reliably inexpensive, and have been found to have a variety of adverse environmental effects.  Nuclear energy, while low in emissions, has risks that have caused countries such as Japan and Germany to reevaluate this approach to energy generation.

Addressing these emergent problems has led to the conceptualization of a ``smart grid'', where a variety of decentralized, intermittent, renewable energy sources (for example, wind, solar, and wave) would provide most or all of the power required by small-scale ``micro-grids'' servicing hundreds to thousands of consumers. Such a smart grid will require consumers to transition from passive to active participation in maintaining efficient and effective use of the grid's electrical capabilities.  For example, these ``smart consumers'' should be able to tailor their use of electrical energy to the types and amount of energy available in the grid at any point in time; minimizing the overall use of non-renewable resources as well as peak loads on the grid.

Satisfying the radically different requirements and operating assumptions of this next generation grid requires new kinds of software that enable research and experimentation into the ways that electrical energy production and consumption can be collected, analyzed, visualized, and provided to consumers in a way that enables a transition from passive to active participation.  Since 2009, we have been designing, implementing, and evaluating an open source software ``stack'' to facilitate this research.  This software stack consists of two custom systems called WattDepot\footnote{http://wattdepot.googlecode.com} and Makahiki\footnote{http://github.com/csdl/makahiki}, along with the open source components they rely upon (Java, Restlet, Postgres, Python, Django, Memcache).

In this paper, we detail the novel features of WattDepot and Makahiki,  along with our initial experiences using
them to implement an energy challenge called the Kukui Cup.
