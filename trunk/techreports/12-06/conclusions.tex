\section{Conclusions/Future Directions}

Our experiences thus far with the Makahiki+WattDepot software stack have shown it to be a
reliable and performant system for the provision of sophisticated energy challenges to
address the need for informed consumers in the development of the Smart Grid. Its
tailorability and game analytics provides a useful platform for research on gamification,
energy education, and behavior change.


One future direction involves the development of a consortium of local organizations in
order to explore the use of this software stack in new settings.  This will create
challenges at both the design and implementation levels.   Moving outside of the context
of either Hawaii or college-aged users will necessitate development of significant new
forms of content, including activities, workshops, events, and videos. This challenge,
while significant, does not necessitate significant changes to the Makahiki+WattDepot
software stack.

A far more challenging future direction is an island-wide Kukui Cup, in which all of the residents of
Oahu would be able to login to the system and play the game. This creates challenges on
multiple levels: providing content appropriate to the user (an elementary school student should
have activities different from her mother and father), obtaining energy data for
residential users from the local utility in a manner appropriate for the challenge and
interfacing this data with the system, and finally resolving the scalability problem
identified in the last section.  We will require the engagement of multiple stakeholders
from across the community spectrum to identify and resolve these issues. 
