%%%%%%%%%%%%%%%%%%%%%%%%%%%%%% -*- Mode: Latex -*- %%%%%%%%%%%%%%%%%%%%%%%%%%%%
%% 12-06.tex --      ICT4S Paper
%% Author          : Philip Johnson
%% Created On      : Mon Sep 23 11:52:28 2002
%% Last Modified By: Philip Johnson
%% Last Modified On: Fri Apr 16 15:19:50 2010
%%%%%%%%%%%%%%%%%%%%%%%%%%%%%%%%%%%%%%%%%%%%%%%%%%%%%%%%%%%%%%%%%%%%%%%%%%%%%%%
%%   Copyright (C) 2009 Philip Johnson
%%%%%%%%%%%%%%%%%%%%%%%%%%%%%%%%%%%%%%%%%%%%%%%%%%%%%%%%%%%%%%%%%%%%%%%%%%%%%%%
%% 

%% Home page:http://www.ict4s.org/

\documentclass{acm_proc_article-sp}
%\usepackage[final]{graphicx}
\usepackage{cite}
\usepackage{url}
% uncomment the % away on next line to produce the final camera-ready version
% and uncomment the \thispagestyle{empty} following \maketitle
%\pagestyle{empty}

\begin{document}

\title{Makahiki+WattDepot: An open source software stack for 
next generation energy research and education}
\subtitle{[Extended Abstract]}

\author{Philip M. Johnson\\
        Yongwen Xu\\
        Robert S. Brewer\\
        George E. Lee\\
        Andrea Connell\\
        Collaborative Software Development Laboratory\\
        Department of Information and Computer Sciences\\
        University of Hawai`i at M\=anoa\\
        Honolulu, HI 96822\\
        \{johnson, yxu, rbrewer, gelee, connell4\}@hawaii.edu\\
}


\maketitle

\section{Introduction}

There are a number of characteristics of the traditional electrical energy infrastructure
of industrial societies that have remained unchanged for almost 100 years.  First, energy
production is centralized in power plants using ``firm'' energy sources such as coal, oil,
nuclear, or water.  Second, centralized production
facilitates centralized control of the grid, typically through a single or small number of
utilities with public regulation over their policies and rates.  Third, centralized
production and control leads to the predominence of ``macro-grids'', or grid
infrastructures that are designed to service hundreds of thousands to millions of
consumers. Finally, these macro-grids are designed to minimize the information about
energy required by consumers to utilize the service.  The typical consumer needs to know
almost nothing more than how to plug an appliance into an outlet, and can operate under the
assumption that this exceedingly simple ``user interface'' can provide virtually unlimited
amounts of high quality energy at any time for a relatively small cost.

Unfortunately, the rapid growth in demand for energy world-wide over the past 100 years is
leading to a breakdown in this model for electrical energy infrastructure.  Petroleum
products such as coal and oil are nonrenewable, are no longer reliably inexpensive, and
have been found to have a variety of adverse environmental effects.  Nuclear energy, while
relatively ``clean'', has risks that have caused countries such as Japan and Germany to
discontinue use of this form of energy generation.  Addressing these issues has led
to the conceptualization of a ``smart grid'', where a variety of decentralized,
intermittent, renewable energy sources (for example, wind, solar, and wave) would provide
most or all of the power required by small-scale ``micro-grids'' of hundreds to thousands
of consumers. Finally, such a ``smart grid'' requires consumers to transition from passive
to active participation in maintaining efficient and effective use of the grid's
electrical capabilities.  For example, such ``smart consumers'' should be able to tailor
their use of electrical energy to the types and amount of energy available in the grid at
any point in time; minimizing the overall use of non-renewable resources as well as peak
loads on the grid.

Satisfying the radically different requirements and operating assumptions of this next
generation ``smart grid'' require new kinds of software that enable research and
experimentation into the ways that electrical energy production and consumption can be
collected, analyzed, visualized, and provided to consumers in a way that enables a
transition from passive to active participation.  Since 2009, we have been developing an
open source software ``stack'' to facilitate this research.  Currently, the software stack
consists of two custom systems we have developed:
WattDepot\footnote{http://wattdepot.googlecode.com} and
Makahiki\footnote{http://github.com/csdl/makahiki}, along with the open source systems
they rely upon (Java, Restlet, Postgres, Python, Django, Memcache).  In the full version
of this paper, we will detail the novel features of WattDepot and Makahiki, our
experiences using them for research and education, and additional ways they can be used
for academic and industrial research on next generation energy research and education.
The remainder of this extended abstract briefly introduces WattDepot and Makahiki and our
research experiences to date.

\section{WattDepot}

Software for energy collection, storage, and analysis tends to come in two flavors that
support two ends of the scalability spectrum.  At one end are utility-scale SCADA systems
which are intended to manage macro-grid data.  At the other end are ``personal scale''
systems such as those provided by energy meter or solar panel manufacturers which are
intended to manage information about single households.  We designed WattDepot to support
a middle ground that we refer to as ``enterprise-level'' energy management, in which data
concerning energy production and consumption of hundreds to thousands of households can be
usefully managed.  Our use of WattDepot has led to a novel set of capabilities to support
this middle ground.

First, unlike personal-scale systems that are typically tied to a particular
manufacturer's product, WattDepot is agnostic about the kinds of meters used to monitor
energy production and consumption data, and whether the data is personal-scale or
utility-scale. It provides a REST protocol for data transmission that can be used to
implement clients for a wide variety of devices; the major constraint is that these
devices need to have Internet access. WattDepot clients can be written in any language
that supports the HTTP protocol. We provide a high-level client library for Java.

Second, WattDepot can represent aggregations of power sou\-rces. For example, a building
might have multiple meters monitoring energy consumption, one per floor. WattDepot can
represent the power consumed by individual floors, as well as an aggregate source
representing the building as a whole. Aggregations can be nested, so that floors can be
aggregated into buildings, buildings into neighborhoods, and neighborhoods into cities.

Third, WattDepot automatically performs data interpolation when necessary. For example, a
meter might provide a snapshot of energy usage once per hour for a given device. Clients
can request the power consumed by this device at any time instant, and WattDepot will
automatically provide interpolation when the requested time does not match a time for
which actual sensor data is available.

Fourth, WattDepot is architecturally decoupled from the underlying data storage
technology. This supports experimentation with both traditional relational as well as
NoSQL technologies, and facilitates scalability. Currently, WattDepot implements support
for Derby, Postgres, and BerkeleyDB storage systems.

Fifth, WattDepot is designed to support both PaaS and local installation. We have deployed
WattDepot to the Heroku cloud-based hosting service.

Sixth, WattDepot implements support for "ephemeral" data. In some application scenarios,
it is useful to send energy data to the WattDepot server quite frequently (i.e. every few
seconds) so that clients can monitor current energy consumption with low latency. However,
that rate of data sampling is not necessary for historical analyses, which may only
require energy data sampling at the rate of every few minutes. WattDepot supports this
situation through "ephemeral" data, which creates an in-memory "window" during which all
recently received energy data is available for retrieval, but stored in the repository
only at a much lower sampling rate.

Seventh, WattDepot can be effectively used for simulation and what-if scenario
development, as well as for management of live energy data.  This makes it appropriate as
a kind of technological ``scaffolding'' for smart grid applications, where WattDepot can
provide clients with simulated production and consumption data early in development, with
the simulated data transitioning to live data as these sources go online later in development.

\section{Makahiki}

The feature set of WattDepot creates attractive infrastructure for management of energy
data, but research suggests that effective participation of consumers in a next generation
smart grid requires more than simple feedback to consumers about their consumption,
particularly given the passive nature of their involvement for the past 100 years. 

The move to a smart grid is not only technological, but also a political and social
paradigm shift, requiring citizens to think differently about energy policies, methods of
generation, and their own consumption than they have in the past. Unfortunately, unlike
other civic and community issues, energy has been almost completely absent from the
educational system. To give a sense for this invisibility, public schools in the United
States generally teach about the structure and importance of our political system (via
classes like ``social studies''), nutrition and health (through ``health''), and even sports
(through ``physical education''). But there is no tradition of teaching ``energy'' as a core
subject area for an educated citizenry, even though energy appears to be one of the most
important emergent issues of the 21st century.

Recent years have seen the adoption of game techniques, not only to their traditional form
of entertainment, but across the entire cultural spectrum. The adoption of game techniques
to non-traditional areas such as finance, sales, and education has become such a phenomenon
that the Gartner Group included ``gamification'' on its 2011 Hype List.

The second component of our open source software stack, Makahiki, represents research
intended to create synergy between the need to facilitate education regarding energy and
the ability of game techniques to create engagement and participation.  In Makahiki,
online world game mechanics are employed with the goal of affecting real-world energy
behaviors.  The ultimate goal is to not just affect energy behaviors during the course of
the game, but to produce long lasting, sustained change in energy behaviors and outlooks
by participants.

Makahiki consists of a configurable game engine that can be customized to the needs of
different organizations.  It includes the following set of pre-built game ``widgets'' that
provide a variety of game mechanics.  Using the widgets, an organization can create an
energy challenge in which players can compete individually and in teams to earn the most
points by learning about energy concepts in general and the way their behaviors affect
their energy consumption in particular. A few of these widgets are introduced below:

%% 1500 words here.

The Smart Grid Game widget is the primary place players go to learn about energy issues and earn
points.  The Smart Grid Game supports four different kinds of tasks: activities,
commitments, events, and excursions. 

The Daily Energy Goal Game widget provides a way for players to earn points by reducing their
current consumption from a baseline that is typically determined prior to the challenge.
Both the historical baseline data and the current consumption is typically provided
through calls by Makahiki to an underlying WattDepot server.

The Raffle Game widget provides a way to incentivize participation from all individuals, even
those who are not in the running for a top prize. For every 25 points a player earns, they
receive one virtual raffle ticket. Players can dynamically allocate their tickets to any
raffle prizes they are interested in at any time, up to the end of the raffle.

The Social and Referral Bonus widgets provide game mechanics that help encourage
participation by providing additional points to users who participate in activities with
other users and/or facilitate the entry of new users into an energy challenge. 

\section{Experiences}

To assess the Makahiki+WattDepot software stack, we designed and implemented an energy
challenge called the ``Ku\-kui Cup'' for over 1,000 first year students living in the
residence halls at the University of Hawaii in Fall, 2011.  We installed smart meters on each
residential floor of the towers, but due to the electrical infrastructure, energy use could
only be tracked by pairs of floors. These pairs of floors share a common lounge space and
elevator, so the pairs are called ``lounges''. The lounges across all four towers competed
to minimize energy use during the competition, measured in kilowatt-hours. In addition,
individuals and lounges competed to earn the most points by engaging in activities that
helped them learn about energy and the impact on their behaviors on it.

















 






\end{document}
