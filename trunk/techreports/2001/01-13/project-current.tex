%%%%%%%%%%%%%%%%%%%%%%%%%%%%%% -*- Mode: Latex -*- %%%%%%%%%%%%%%%%%%%%%%%%%%%%
%% project-current.tex -- 
%% Author          : Philip Johnson
%% Created On      : Thu Oct  4 08:28:29 2001
%% Last Modified By: Philip Johnson
%% Last Modified On: Mon Nov  5 16:25:43 2001
%% RCS: $Id$
%%%%%%%%%%%%%%%%%%%%%%%%%%%%%%%%%%%%%%%%%%%%%%%%%%%%%%%%%%%%%%%%%%%%%%%%%%%%%%%
%%   Copyright (C) 2001 Philip Johnson
%%%%%%%%%%%%%%%%%%%%%%%%%%%%%%%%%%%%%%%%%%%%%%%%%%%%%%%%%%%%%%%%%%%%%%%%%%%%%%%
%% 

\subsection{Current project status}

\begin{figure}[t]
 {\centerline {\psfig{figure=dataflow.eps}}}
 \caption{{\em Basic communication pathways in Hackystat.}}
 \label{fig:dataflow}
\end{figure}


Development of the Hackystat system began in May, 2001, and the first
functional public release was made in July, 2001. This release included
source and binaries for the Hackystat server, a set of sensors for
detecting time and activity data in JBuilder and Emacs, and server-side
analysis tools for summarizing time and activity data.  In addition, a
publically accessible Hackystat server was deployed on the Internet and has
been available and regularly updated with new releases since then.  Since
the initial release, additional sensors and analysis tools have been
released at monthly intervals.  The active development team consists of
five members, three from the University of Hawaii and one member from each
of two commercial software development companies.

The Hackystat server code is implemented in Java and contains approximately
14 packages, 62 classes, and 4,200 non-comment source lines of code.  The
code leverages over a dozen open source components including Tomcat for web
services, Cocoon for XML/XSLT processing, JUnit for testing, and SOAP for
sensor-server communication over the HTTP protocol.

Hackystat sensor code is much smaller and simpler than that required for
the server.  For example, the JBuilder activity and session sensors consist
of a total of approximately 200 non-comment source lines of Java, with over
half of that code consisting of "empty" method declarations required to
fulfill plug-in API requirements.  The Emacs activity and session sensors
are written in Lisp and are similar in size.

Currently implemented sensors include: time and activity data for the Emacs
and JBuilder development environments, and defect data for JUnit.  An
initial implementation of a size data sensor for Java and C++ using the
LOCC grammar-based size counting tool will be available in November, 2001.
Currently implemented analysis tools include active time data, idle time
threshold analysis, JUnit summary data, and data integrity checking for all
data types.

Hackystat does not use a relational database back-end. Instead, data for
each user for each sensor data type is stored in a ``log'' file in XML
format.  Currently, some developer log files store only a day's worth of
data (in the case of potentially high volume data such as activity and
testing data), while other logs store a month's worth of data (in the case
of low volume data such as session data).  Note that different sensors can
contribute data to the same log file; for example, activity data for a
given developer on a given day is stored in the same log file, even if some
of that data was sent from an Emacs sensor and other data was sent from a
JBuilder sensor. Each log file type includes data integrity code which
checks incoming data from sensors.  If bad data is detected, it is not
added to the sensor log. Instead, it is added to a special ``BadData'' log
and the user and system administrator gets information concerning this in
their next daily email.  Bad data should not be sent by correctly
implemented sensors under normal circumstances, of course.  We use the JATO
COTS component to support automated translation between the XML log file
and an in-memory Java object representation, which simplifies the
implementation of both log files and analysis functions.  While we are
pleased with both the flexibility and performance of our XML-based approach
so far, we could easily migrate to a relational database approach in future
if that becomes necessary.

Initial performance evaluation results are encouraging. There are four
important costs that impact upon performance: the cost of collecting data
in the sensor, the cost of sending sensor data to the server, the cost of
performing analyses on the data, and the cost of storing data. 

If any of
these costs become excessive, they affect the usability or scalability of
the system. First, the cost of collecting data by the sensor is so far
negligible for all sensors we have implemented.  Second, sensor data is
typically aggregated and sent to the server at the end of tool sessions.
Currently, this means that Hackystat sends data from a developer less than
a dozen times a day, and each notification normally requires less than two
seconds for the http interaction.  Third, analysis computations are
performed once a day during ``off-hours'' such as 2 am, and current
analyses require only a few seconds in total to perform for each user.  We
believe that none of these first three costs prevent the current system
architecture from scaling to at least several hundred users per server with
many additional sensors, sensor data types, and analyses.  

The fourth and
final cost, data storage, is the only one which currently appears to pose
significant scalability threats. Currently, a typical Hackystat user
accumulates data in XML log files at the rate of 500 KB per month. As the
set of sensors and thus data gathering potential increases, we believe that
data accumulation could easily double to 1MB/user/month, which could create
performance problems as the number of users rises to hundreds or more. We
know of one relatively simple potential solution: use built-in Java
utilities to read and write the XML data as compressed zip files. Tests
indicate that Hackystat logs are compressed by an average of 90\% using zip 
encoding. 


Figure \ref{fig:dataflow} illustrates the basic communication pathways and
processing in Hackystat.  First, the developer downloads one or more
sensors from the server and installs them into the development environment.
Next, all sensors require some initial configuration; at a minimum, the
developer must specify the URL of the Hackystat host server to which data
should be sent and an email address identifying them as the user. At that
point, sensors begin sending data to the server, and the server notifies
the developer via email when potentially ``interesting'' analyses occur.
The developer then has the option of retrieving additional pages from the
server containing analysis details.



Our progress so far indicates to us that Hackystat represents a feasible
approach to developer-centric, in-process, non-disruptive software project
data collection and analysis, and that our research group has the technical
skills required to build and enhance such a system.  However, our work so
far has only begun laying the groundwork for answering important questions
about this approach.  For example, we know we can collect data
automatically, but can we collect the right data in the right way?  We know
we can send the developer an email with analysis results, but can we send
the developer an email at the right time with the right analysis results?
The next section presents our approach to addressing these issues.




 




