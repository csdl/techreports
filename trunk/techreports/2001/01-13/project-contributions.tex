%%%%%%%%%%%%%%%%%%%%%%%%%%%%%% -*- Mode: Latex -*- %%%%%%%%%%%%%%%%%%%%%%%%%%%%
%% project-contributions.tex -- 
%% Author          : Philip Johnson
%% Created On      : Thu Oct  4 08:28:29 2001
%% Last Modified By: Philip Johnson
%% Last Modified On: Mon Nov  5 16:25:07 2001
%% RCS: $Id$
%%%%%%%%%%%%%%%%%%%%%%%%%%%%%%%%%%%%%%%%%%%%%%%%%%%%%%%%%%%%%%%%%%%%%%%%%%%%%%%
%%   Copyright (C) 2001 Philip Johnson
%%%%%%%%%%%%%%%%%%%%%%%%%%%%%%%%%%%%%%%%%%%%%%%%%%%%%%%%%%%%%%%%%%%%%%%%%%%%%%%
%% 

\subsection{Summary of anticipated contributions}

We expect this research on non-disruptive, developer-centric, in-process
software project data collection and improvement to make the following
contributions to the academic and industrial software engineering
communities. 

\begin{smallenum2}

\item We will provide an open-source, extensible, community-supported
environment for sensor-based data collection and web service-based analysis
of software project data. 
If Hackystat is successful, we anticipate that
tool vendors will begin developing and maintaining their own sensors. 

\item We will provide and maintain a publically available Hackystat web
server.  This will substantially lower the cost of
evaluation of the services for initial users and provide developers an
external ``safe haven'' for their process and product data.

\item We will replicate previous case study results on the
relative accuracy of various approaches to individual project
estimation. This will provide additional data on the merits of simpler
approaches to estimation than PROBE.

\item We will contribute empirical results regarding the relative utility
and effectiveness of the PSP and Hackystat approaches to data collection
and analysis. This will provide insight into the strengths and weaknesses
of disruptive and non-disruptive approaches to developer-centric software
project data.

\item We will contribute case study results regarding the utility and
adoptability of the Hackystat approach to data collection and analysis
within the XP development method and community.

\item We will contribute insights into the use and applicability of
statistical process control techniques to developer-centric in-process
project data.

\item We will obtain empirical in-process data as students progress from
beginning to advanced programming capabilities, and explore how this data
can be used to improve education.
\end{smallenum2}











