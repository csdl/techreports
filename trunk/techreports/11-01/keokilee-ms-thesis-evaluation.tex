\chapter{Evaluation}
\label{eval}

When we decided to redesign the system in October of 2010, we also decided to hold user evaluations in order to get feedback earlier in the process.  The overall goals of the evaluation were:

\begin{enumerate}
  \item To determine if our interface is intuitive.  Are users able to find their way through the Makahiki system when they have not used it before?  How do users navigate through the system?
  \item To observe how understandable the different aspects of the system are.  After users interact with the different components, do they understand its function and what it means?
  \item To assess our ability to make a game.  Because the actual competition is a ``serious game'' with incentives, we want to integrate some incentives into our evaluation.  Do these users understand how to play the game?  Are they motivated by the incentives to ``win'' in the evaluation?
\end{enumerate}  

Since December of 2010, we held three user evaluations: one mockup evaluation and two onboarding evaluations.  

\section{Mockup Evaluation}
\label{eval-mockup}

Before we stared implementing the beta version of Makahiki, we created mockups of the system in order to plan out what the new beta interface would look like. To create the mockups, we used a program called Balsamiq Mockups\cite{balsamiq-mockup}.  The mockups created by Balsamiq are visually simplistic by design; it conveys the idea that the mockups are not final and are subject to change.  At the same time, Balsamiq allows us to link together mockups, thus creating a mockup representation of the actual system's navigation.

After we finished creating our initial set of mockups, we held a user evaluation in December, 2010 of our ``mockup'' system.  The evaluation involved three scenarios.  The first scenario involved the user coming to the system for the first time, setting up their profile, making a commitment, and signing up for an event.  The second scenario was then having the user redeem a confirmation code for the event they said they would attend in the previous scenario and visit the ``Go Low'' page for the first time.  The final scenario involves viewing the list of prize winners and also seeing if they had earned any badges.

\section{Onboarding Evaluations}
\label{eval-onboarding}

