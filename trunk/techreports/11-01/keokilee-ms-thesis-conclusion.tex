Makahiki is a system designed to allow individuals to easily create energy competitions for their organizations. We started with concepts from other energy competitions and provided a more personalized view for users when the log in. We also incorporated a serious game designed around energy conservation in order to help individuals become more aware of their energy usage. Furthermore, we implemented logging of users as they navigated through the system. This allows researchers to gain insight into the behaviors of participants in an energy competition. Finally, this platform is open source and able to be used by other organizations to support energy competitions.

To prepare for our first deployment of Makahiki in the 2011 Kukui Cup, we needed to evaluate the design of the system. We started with a mockup evaluation to validate our design. We then held individual onboarding evaluations when the system had most of its basic features implemented. In these evaluations, we observed the subject to gain insight into how an individual might interact with the system by themselves. Finally, we held a beta evaluation that involved subjects interacting with the system outside of a laboratory setting. This allowed us to observe how the time-dependent aspects of the system (rounds, events, etc.) may work in an actual competition. The beta evaluation also provided insight into social interactions that may take place in the system.

During the 2011 Kukui Cup, Makahiki performed very well with 413 individuals logging into the site during the competition period. Based on our logging, we found that individuals spent a lot of time on the smart grid and energy usage pages, which were tested in each of our evaluations. However, our Canopy page, which was barely tested in any of the evaluations, turned out to be our most significant failure of the system. Based on the feedback from a survey of users during the competition, they were very satisfied with the system overall.

\section{Future Work}
\label{future-work}

One of our original goals was to create a system that would more customizable for use by other energy competitions. For example, some organizations may not be able to install smart meters and support near-real time power and energy displays. Others may choose to omit the Smart Grid Game because they do not have the time to develop the content. We need to make it more configurable to support other organizations that may want to use our software. We also want the front-end graphical interface to be more customizable for other organizations.

Makahiki is also a research platform that was used in 2011 to support tracking of user behaviors within the system. This was accomplished through low level logging that tracked every action that users took within the system. In the future, we want to provide higher level analytics of user interactions that can be viewed in real time during the competition. We also want the system to support A/B testing, a usability technique that allows a researcher to create multiple interfaces to a component. Certain participants within the system will see one interface others will see another one. This allows researchers to observe how users interact with different interfaces and to evaluate which ones are more successful.

For the 2011 Kukui Cup, our research lab used its own web server. Competition organizers may not be tech-savvy enough to be able to set up Makahiki on their web server. Furthermore, many organizers may not have direct access to a web server at all. With the advent of cloud computing and platforms as a service (PaaS), organizers no longer need to have a web server of their own. We want to make it as easy as possible for organizers to deploy an instance of Makahiki to the cloud, configure it, and then run their competition.

Our initial design of Makahiki focused on energy. However, we can support additional reduction competitions. Meters that track water usage are currently available. Trash and recycling can be tracked manually by organizers as well. We are considering making components within Makahiki to support these competitions. However, since Makahiki is open source, we would also like to make it easier for other developers to create modules for the system.

\section{Contributions}
\label{contributions}

We claim the following contributions:

\begin{enumerate}
  \item An open source system for creating serious games for energy competitions.
  \item A research platform on which researchers can observe user behavior during energy competitions.
	\item A methodology for evaluating and testing serious games that involve competitions over a period of time.
\end{enumerate}

Over the course of this thesis, we developed an open source system for creating serious games for energy competitions. Development still continues and the system will evolve to support other sustainability-related competitions. Furthermore, the availability of detailed logging allows researchers to investigate the behavior of participants in a competition as they participate. Finally, in order to properly evaluate the usability of the system, we developed a procedure for evaluating our system using a series of evaluations involving separate sets of users.

