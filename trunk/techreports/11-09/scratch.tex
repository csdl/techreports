\documentclass[10pt]{article}
\usepackage[utf8x]{inputenc}

%% Package to linebreak URLs in a sane manner.
\usepackage{hyperref}
\usepackage{url}
%% Define a new 'smallurl' style for the package that will use a smaller font.
\makeatletter
\def\url@smallurlstyle{%
  \@ifundefined{selectfont}{\def\UrlFont{\sf}}{\def\UrlFont{\small\ttfamily}}}
\makeatother
%% Now actually use the newly defined style.
\urlstyle{smallurl}
%% Define 'tinyurl' style for even smaller URLs (such as in tables)
\makeatletter
\def\url@tinyurlstyle{%
  \@ifundefined{selectfont}{\def\UrlFont{\sf}}{\def\UrlFont{\scriptsize\ttfamily}}}
\makeatother
%%
%%
%% Make margins less ridiculous
\usepackage{fullpage}
%%%% Turned off for tech report, should be turned on for research portfolio
%% Turn on double spacing
\usepackage{setspace}
\doublespacing
%%
%%

% Title Page
\title{Behavior discovery workflow primer}
\author{Pavel Senin \\ \href{mailto:senin@hawaii.edu}{senin@hawaii.edu}}


\begin{document}
\maketitle

\begin{abstract}
This explains the motivation which driven the data indexer design and implementation, provides theoretical background, and 
illustrates the data flow on the real-life example.
\end{abstract}

\section{Data description}
The data stored in SCM can be characterized by the generating activities. 
\begin{itemize}
 \item coding
 \item testing
 \item debugging
 \item documenting
 \item organizing files, data, folders, etc.
 \item email exchange
 \item issue creation/assignment/closing
 \item issue comment
\end{itemize}
All these are unevenly distributed in time. This data is uniformly re-sampled 
for the purpose of further analyzes using a variety of time-intervals ranging from
one hour to one month.

\section{Data collection}

\section{Data preprocessing}
According to \cite{citeulike:825581} and \cite{citeulike:3000416}, there is a number of highly desirable 
properties for any indexing scheme:
\begin{itemize}
 \item It should be much faster than sequential scanning.
 \item The method should require little space overhead.
 \item The method should be able to handle queries of various lengths.
 \item The method should allow insertions and deletions without requiring the index to be rebuilt.
 \item It should be correct, i.e. there should be no false dismissals.
 \item It should be possible to build the index in "reasonable time".
 \item The index should be able to handle different distance measures, where appropriate.
\end{itemize}

\subsection{Data properties}
The data collected from Software Change Management system bear high complexity due to its 
high dimensionality, inconsistency, noise and incompleteness. Another problem inherent in 
its nature is sparseness. 

Related to the complexity are next issues:
\begin{itemize}
 \item First of all, the data entities stored in an 
   SCM system are not equidistantly placed in time and mostly contain only information 
   related to activities preceeding an entity's time-stamp. This fact greatly affects 
   the behaviors recognition process as well as impacts our ability to place any of 
   inferred behaviors in time.
 \item Secondly, these data entities are, in fact, a highly compressed summaries of an 
   actual contributor's activities within a project development activities. These activities 
   include coding, testing, debugging, designing, planning, code and data management 
   and many others. Every of these activities is in turn modulated by assigned to developer role, 
   project phase, daily activity patterns, geographical and physical location, internet 
   connection availability, expertise level, employment constraints, and many others.
   Inconsistency and noise? Lokk on the single user stream - it's due to the nature of 
    committ and individual development behaviors. There are
   contributors committing regularly by small increments, others rarely by large incerements. 
 \item Thirdly, the information encoded within SCM trails is incomplete. For example it is
   impossible to assess the time when the actual software change occured nor the effort it 
   required.
\end{itemize}

Sparseness. The individual contributor' stream consists of a number of one-dimensional time series
which are often short and no longer prolongable. Moreover, they may be very sparse in the sense 
that the underlying commits events are too rare and not homogeneously distributed in time and 
in-project location, thus many of daily values are null.

\subsection{Indexing}

\subsection{Normalization}

\section{Data preprocessing}

\section{Behaviors dictionary building primer}
\bibliographystyle{IEEEtran}
% argument is your BibTeX string definitions and bibliography database(s)
%\bibliographystyle{abbrv}
%\bibliography{seninp}

\bibliography{IEEEabrv,seninp}
\end{document}          
