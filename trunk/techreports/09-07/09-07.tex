%%%%%%%%%%%%%%%%%%%%%%%%%%%%%% -*- Mode: Latex -*- %%%%%%%%%%%%%%%%%%%%%%%%%%%%
%% 09-07.tex -- IEEE Software paper
%% Author          : Hongbing Kou
%% Created On      : Mon Sep 23 11:52:28 2002
%% Last Modified By: Philip Johnson
%% Last Modified On: Thu Jan 22 07:47:45 2009
%%%%%%%%%%%%%%%%%%%%%%%%%%%%%%%%%%%%%%%%%%%%%%%%%%%%%%%%%%%%%%%%%%%%%%%%%%%%%%%
%%   Copyright (C) 2005 Hongbing Kou
%%%%%%%%%%%%%%%%%%%%%%%%%%%%%%%%%%%%%%%%%%%%%%%%%%%%%%%%%%%%%%%%%%%%%%%%%%%%%%%
%% SWSI-0197-1106

\documentclass[conference,12pt]{IEEEtran}
%\usepackage{IEEEconf}
\usepackage[final]{graphicx}
% uncomment the % away on next line to produce the final camera-ready version
% and uncomment the \thispagestyle{empty} following \maketitle
\pagestyle{empty}

\begin{document}

\title{Experiences with Hackystat as a service-oriented architecture}

\author{Philip M. Johnson \\
        Shaoxuan Zhang \\
        Randy Cox \\
        Pavel Senin \\
\em  Collaborative Software Development Laboratory \\
\em  Department of Information and Computer Sciences \\
\em  University of Hawai'i \\
\em  Honolulu, HI 96822 \\
\em  johnson@hawaii.edu \\
}


\maketitle
%\thispagestyle{empty}

\begin{abstract}  % 200 words
Hackystat is an open source framework for automated collection and analysis of software engineering process and product data.  Hackystat has been under development since 2001, and has gone through eight major architectural revisions during that time.  In 2007, we began the latest architectural revision, whose primary goal was to 
reimplement Hackystat as a service-oriented architecture (SOA).  Hackystat Version 8 has now been in public release for over a year, and this paper reports on our 
experiences:  the motivations that led us to reimplement the system as a SOA, the benefits we have experienced from that conversion, and the new challenges we now
face as a result. 
\end{abstract}

\section{Introduction}
\label{sec:intro}


Finally, we want to emphasize the open source nature of the Zorro system and 
the research process.   We encourage you to download the system and try it out,
or contact us if you wish to participate in the research process. 

\bibliographystyle{IEEEconf}
\bibliography{tdd,zorro,csdl-trs,hackystat,psp}
\end{document}











