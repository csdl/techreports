%%%%%%%%%%%%%%%%%%%%%%%%%%%%%% -*- Mode: Latex -*- %%%%%%%%%%%%%%%%%%%%%%%%%%%%
%% proposal-abstract.tex -- 
%% Author          : Carleton Moore
%% Created On      : Fri Jun  9 09:43:42 1995
%% Last Modified By: Robert Brewer
%% Last Modified On: Fri Sep 18 15:21:59 1998
%% Status          : Unknown
%% RCS: $Id: proposal-abstract.tex,v 1.1 1998/09/19 01:24:42 rbrewer Exp $
%%%%%%%%%%%%%%%%%%%%%%%%%%%%%%%%%%%%%%%%%%%%%%%%%%%%%%%%%%%%%%%%%%%%%%%%%%%%%%%
%%   Copyright (C) 1995 University of Hawaii
%%%%%%%%%%%%%%%%%%%%%%%%%%%%%%%%%%%%%%%%%%%%%%%%%%%%%%%%%%%%%%%%%%%%%%%%%%%%%%%
%% 

%for review purposes
%\ls{1}

\abstract{
  
  Electronic mailing lists are popular Internet information sources. Many
  mailing lists maintain an archive of all messages sent to the list which is
  often searchable using keywords. While useful, these archives suffer from the
  fact that they include all messages sent to the list. Because they include
  all messages, the ability of users to rapidly find the information they want
  in the archive is hampered. To solve the problems inherent in current mailing
  list archives, I propose a process called condensation whereby one can strip
  out all the extraneous, conversational aspects of the data stream leaving
  only the pearls of interconnected wisdom.
  
  To explore this idea of mailing list condensation and to test whether or not
  a condensed archive of a mailing list is actually better than traditional
  archives, I propose the construction and evaluation of a new software system.
  I name this system the Mailing list Condensation System or MCS. MCS will have
  two main parts: one which is dedicated to taking the raw material from the
  mailing list and condensing it, and another which stores the condensed
  messages and allows users to retrieve them.
  
  The condensation process is performed by a human editor (assisted by a tool),
  not an AI system. While this adds a certain amount of overhead to the
  maintenance of the MCS-generated archive when compared to a traditional
  archive, it makes the system implementation feasible.
  
  I believe that an MCS-generated mailing list archive maintained by an
  external researcher will be adopted as a information resource by the
  subscribers of that mailing list.  Furthermore, I believe that subscribers
  will prefer the MCS-generated archive over existing traditional archives of
  the mailing list. This thesis will be tested by a series of quantitative and
  qualitative measures.

}











