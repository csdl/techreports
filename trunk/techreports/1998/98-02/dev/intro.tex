%%%%%%%%%%%%%%%%%%%%%%%%%%%%%% -*- Mode: Latex -*- %%%%%%%%%%%%%%%%%%%%%%%%%%%%
%% intro.tex -- 
%% Author          : Carleton Moore
%% Created On      : Fri Oct 18 16:20:55 1996
%% Last Modified By: Carleton Moore
%% Last Modified On: Mon Sep 21 10:54:42 1998
%% RCS: $Id: intro.tex,v 1.4 1996/11/21 02:42:55 cmoore Exp $
%%%%%%%%%%%%%%%%%%%%%%%%%%%%%%%%%%%%%%%%%%%%%%%%%%%%%%%%%%%%%%%%%%%%%%%%%%%%%%%
%%   Copyright (C) 1996 Carleton Moore
%%%%%%%%%%%%%%%%%%%%%%%%%%%%%%%%%%%%%%%%%%%%%%%%%%%%%%%%%%%%%%%%%%%%%%%%%%%%%%%
%% 

\chapter{Introduction}
\label{sec:intro}

\section{Motivation}

\subsection{Software Quality}

\subsubsection{Quality of Software Development}

\subsubsection{Software Developer Improvement}

\subsubsection{Personal Software Process Improvement (PSPI)}

\paragraph{Time recording}

\paragraph{Size recording}

\paragraph{Defect recording}

\paragraph{Analysis}

\subsection{Measurement Dysfunction}

\subsection{Transfer of PSPI Technology}



\section{Thesis}
\begin{enumerate}
  
\item {Training in a LEAP compliant personal software process improvement
    toolset will lead to adoption of the toolset in professional practice.}
  
\item {The LEAP casestudies will provide valuable insights into the strengths
    and weaknesses of this technology transfer method.}

\end{enumerate}

\subsection{Hypotheses}
\begin{enumerate}
\item[H1.1:]{.}
\item[H1.2:]{.}
\item[H1.3:]{.}
\item[H1.4:]{.}
\item[H2.1:]{.}
\item[H2.2:]{.}
\end{enumerate}

\subsection{Research Questions}
\begin{itemize}
\item{How do we develop a system that supports Personal Software Process Improvement without introducing
dysfunction?}
\item{Does a system that measures FTR without introducing dysfunction,
provide management with the answers they want?}
\item{Will a FTR system without dysfunction be adopted?}
\end{itemize}


This proposal is organized as following: Section \ref{sec:related} relates
the current research to the broader context of existing work.  Section
\ref{sec:method} introduces our method for introducing automated FTR into
an organization.  Section \ref{sec:manual} describes the manual FTR method
being used.  Section \ref{sec:LEAP} depicts the main design features and
architecture of LEAP.  Section \ref{sec:experiment} outlines the
experimental case study I plan to conduct to evaluate LEAP.  Finally,
Section \ref{sec:plan} presents the current research plan.

