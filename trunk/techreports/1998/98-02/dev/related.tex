%%%%%%%%%%%%%%%%%%%%%%%%%%%%%% -*- Mode: Latex -*- %%%%%%%%%%%%%%%%%%%%%%%%%%%%
%% related.tex -- 
%% Author          : Carleton Moore
%% Created On      : Fri Oct 18 17:06:48 1996
%% Last Modified By: Carleton Moore
%% Last Modified On: Fri Sep 18 09:53:58 1998
%% RCS: $Id: related.tex,v 1.2 1996/11/06 23:05:00 cmoore Exp cmoore $
%%%%%%%%%%%%%%%%%%%%%%%%%%%%%%%%%%%%%%%%%%%%%%%%%%%%%%%%%%%%%%%%%%%%%%%%%%%%%%%
%%   Copyright (C) 1996 Carleton Moore
%%%%%%%%%%%%%%%%%%%%%%%%%%%%%%%%%%%%%%%%%%%%%%%%%%%%%%%%%%%%%%%%%%%%%%%%%%%%%%%
%% 
\newpage
\chapter{Related Work}
\label{sec:related}
%\ls{1.5}

This section discusses the background for this research.  I begin with two
software engineering research methodologies.  These methodologies involve
the adoption of new software engineering technology, such as inspections.
The adoption of new technology and Inspections can generate metrics.  I
discuss some of the possible dysfunctions in collecting metrics.  Finally,
I review some of the background on the actual transfer of technology.

\section{Software Quality Improvement}
\subsection{Software Process Improvement}
\subsection{Software Review}
\subsection{Testing}
\subsection{Software Metrics}

\section{Personal Software Process}

\section{Metrics Collection}
An important aspect of adopting best practices is measuring the process.



Dysfunction:
\begin{quote}
Results of an organizational activity that are opposite or interfere with
intended results.  Results that are dysfunctional often fulfill the letter
of stated intentions but dramatically depart from the spirit of an
organizational directive or objective.\cite{Austin96}(p. 192)
\end{quote}


There are two intended uses of measurements in organizations, for
motivation and for information.  ``{\em Motivational measurements} are
explicitly intended to affect the people who are being measured, to provoke
greater expenditure of effort in pursuit of organizational goals.  {\em
  Informational measurements} are valued primarily for the logistical,
status, and research information they convey, which provides insights and
allows better short-term management and long-term improvement of
organizational processes.''\cite{Austin96}(p. 21)  The distinctions between
the two uses of measurements are distinctions between the decisions being
made based upon the measurements.  A single measure may be purely used for
informational purposes for one decision, while being used purely for
motivational purposes in another decision.  


{\em Motivational measurements} may lead to dysfunction.  {\em
  Informational measurements} do not.  Before the measure is taken, we
cannot say that the measure will be for purely informational purposes.


\begin{quote}
When benefits associated with the direction of a particular measure are
obvious (such as high quantity or low defect rates), agents become
sensitive to a competitive dynamic that is not represented in models that
feature one principal and one agent.  As agents become familiar with the
system of measurement and discover ways to exploit it, the realize that
their coworkers are also discovering the means of exploitation.  A dilemma
arises.  If coworkers don not exploit the system, then a given worker will
benefit from exploiting the system because he will look better by measured
criteria than his more honest coworkers.  If coworkers do exploit the
system, the given worker will still benefit form exploiting the system
since he will not seem to lag behind his less honest coworkers.  This logic
applies to all workers in the group.  Exploiting the system is, then, a
dominating strategy for all workers.
\end{quote}\cite{Austin96}


Ways to reduce dysfunction:
\begin{itemize}
\item{Keep all individual data private}
\item{Compare personal data to aggregate data}
\item{Make data anonymous}
\item{Keep aggregate group data private to group}
\item{Compare group data to aggregate groups' data}
\end{itemize}


\section{Software Engineering Research}

Colin Potts describes two different methods for researching software
engineering problems; research-then-transfer and
industry-as-laboratory\cite{Potts93}.  Both of these methods involve
transferring new research to industry.  The difference between them is the
coupling between the industry and researchers.  In Research-then-transfer
the research and industry are loosely coupled, while in
industry-as-laboratory they are tightly coupled.

\subsection{Research-then-Transfer}

Research-then-transfer is a more common method for conducting software
engineering research and technology transfer. Figure \ref{fig:r-t-t}
illustrates the research-then-transfer method.  Researchers begin by
identifying some problems to address.  Usually these problems are related
back to studies of industry.  Often the researcher has a solution in mind
and is looking for a problem to address.  Once the problem is identified,
the researcher develops their solution.  The researcher develops the
solution independently from the industry's problem.  The researcher's
attention is solution oriented not problem oriented.  After some time the
researcher decides the research is ready to be applied to a real problem.
He now looks for an organization willing and able to try the new technology
on their problem.  This search is often difficult since the organizations
may not have the equipment or software to support the research solution.
By delaying the evaluation of the research, the researcher runs the risk of
being on the wrong track and his solution is not suitable.  Another major
issue with research-then-transfer is while the research on the solutions is
occurring independent of the problem, the problem may be evolving.  So when
the researcher has their solution the problem may not be the same and the
solution is not appropriate for the new problem.



\begin{figure}[htb]
  \begin{center}
    \setlength{\unitlength}{1.0cm}
    \begin{picture}(12,6)
      \put(0,5.5){\makebox(3.2,1){Application-problem}}
      \put(1,5){\makebox(1,1){domain}}
      \put(8,5.5){\makebox(3.2,1){Research-solutions}}
      \put(9,5){\makebox(1,1){domain}}
      \put(1.5,1){\framebox(2.2,1){Problem v.4}}
      \put(1,2){\framebox(2.2,1){Problem v.3}}
      \put(0.5,3){\framebox(2.2,1){Problem v.2}}
      \put(0,4){\framebox(2.2,1){Problem v.1}}
      \put(8,4){\framebox(2.2,1){Research v.1}}
      \put(8.5,3){\framebox(2.2,1){Research v.2}}
      \put(9,2){\framebox(2.2,1){Research v.3}}
      \put(9.5,1){\framebox(2.2,1){Research v.4}}
      \put(2.2,4.5){\vector(1,0){5.8}}
      \put(2.5,4.6){\makebox(5,1){Wide gulf bridged by indirect,}}
      \put(2.5,4.2){\makebox(5,1){anecdotal knowledge}}
      \put(9.5,1.5){\vector(-1,0){5.8}}
      \put(4,1.2){\makebox(5,1){Technology-transfer gap bridged}}
      \put(4,0.8){\makebox(5,1){by hard, but frequently inappropriate}}
      \put(4,0.4){\makebox(5,1){technology}}
    \end{picture}
  \end{center}
  \caption{Research-then-transfer approach}
  \label{fig:r-t-t}
\end{figure}

\paragraph{Advantages}

The primary advantage of research-then-transfer is its simplicity for the
researcher.  The researcher is free to find an interesting problem and
research a solution.  Once a solution is found he can attempt to find an
organization to use the solution or find someone else to transfer the
technology to industry.  This method allows the researcher to focus on
research not coordinating his effort with another organization.
Researchers would rather find solutions to interesting problems then try
and sell their research to an organization.  The problems of coordination
between researchers and industry are great.  Some of these coordination
issues are security, publishing, deliverables and profits.  Security and
publishing are tightly coupled.  Researchers need to publish while the
company needs to protect their privacy and competitive advantage.  These
conflicting motivations need to be coordinated and resolved.  By
researching independently the researcher can publish freely without the
coordination issues.  This is the primary reason why research-then-transfer
is the most common method for technology transfer.

Another advantage of research-then-transfer is the freedom that it gives to
researchers.  They can look for the next revolutionary technology and
research it without worrying about how industry will adopt and use the
technology.  Without this freedom many revolutionary improvements may not
have occurred.

\paragraph{Disadvantages}

There are several problems with research-then-transfer, often the research
is not important to industry, often the solution is impractical, and the
solution may not solve the current problem in the industry.  First
researchers often look for problems that their research can solve.  These
problems may not be very important to industry.  A solution to a very minor
problem may not be worth the effort to implement.  Second the solution to
the problem may require equipment and expertise that industry cannot afford
or is unwilling to acquire.  Often the researcher uses different equipment
than industry and non-standard software.  Third since the researcher is
working independently from industry, the problem may evolve without the
knowledge of the researcher.  When the researcher develops the solution the
problem may have evolved enough that the solution is not appropriate.

To the researcher the advantages of simplicity and freedom out weigh the
disadvantage of transferring the solutions.  Perhaps this is why Potts has
found that two-thirds of the projects in his field of study, requirements
validation, had no customers from whom to elicit
requirements\cite{Potts93}. Researchers are not looking at actual industry
problems.  This lack of industry focus explains Bill Riddle's findings.
``Bill Riddle discovered that for software environments, the average
transition time form research innovation to state of the practice was 18
years.\cite{Riddle84}''\cite{Potts93} To increase the efficiency of problem
solving, Potts suggests the Industry-as-Laboratory methodology.

\subsection{Industry-as-Laboratory} 

Potts's suggested method for software engineering research is
industry-as-laboratory.  Figure \ref{fig:i-a-l} illustrates the
industry-as-laboratory method.  This method grounds research ideas in
practical problems.  The researcher conducts many case studies to keep his
research tied to the industry problem.  Since the research is closely tied
to industry, the researcher gains insight to the actual industry problem.
The tight coupling helps both industry and the researcher better understand
the problem domain and the solution domain.  Further research helps define
the problem and the problem helps lead the research.

\begin{figure}[htb]
  \begin{center}
    \setlength{\unitlength}{1.0cm}
    \begin{picture}(12,6)
      \put(0,5.5){\makebox(3.2,1){Application-problem}}
      \put(1,5){\makebox(1,1){domain}}
      \put(6,5.5){\makebox(3.2,1){Research-solutions}}
      \put(7,5){\makebox(1,1){domain}}
      \put(2,0){\framebox(2.2,1){Problem v.5}}
      \put(1.5,1){\framebox(2.2,1){Problem v.4}}
      \put(1,2){\framebox(2.2,1){Problem v.3}}
      \put(0.5,3){\framebox(2.2,1){Problem v.2}}
      \put(0,4){\framebox(2.2,1){Problem v.1}}
      \put(6,4){\framebox(2.2,1){Research v.1}}
      \put(6.5,3){\framebox(2.2,1){Research v.2}}
      \put(7,2){\framebox(2.2,1){Research v.3}}
      \put(7.5,1){\framebox(2.2,1){Research v.4}}
      \put(2.2,4.5){\vector(1,0){3.8}}
      \put(2.7,3.5){\vector(1,0){3.8}}
      \put(3.2,2.5){\vector(1,0){3.8}}
      \put(3.7,1.5){\vector(1,0){3.8}}
      \put(6,4.4){\vector(-4,-1){3.3}}
      \put(6.5,3.4){\vector(-4,-1){3.3}}
      \put(7,2.4){\vector(-4,-1){3.3}}
      \put(7.5,1.4){\vector(-4,-1){3.3}}
    \end{picture}
  \end{center}
  \caption{Industry-as-laboratory approach}
  \label{fig:i-a-l}
\end{figure}

\paragraph{Advantages}

Some advantages of industry-as-laboratory are the solutions match the
problem, the solutions can be used by industry, and the researchers and
industry learn to work together.  By focusing on an individual
organization's problem the researcher's solution will solve the problem.
The solution will be tailored to the organization and thus will be able to
be used by the organization.  The researcher will take into account the
organization's resources.  Working together will provide researchers with a
better understanding of industry's requirements and provide industry with a
better understanding of how researchers can improve industry's
productivity.


\paragraph{Disadvantages}

The some disadvantages for industry-as-laboratory are overemphasis on the
short term, the perceived weakness of evolutionary change, and loss of
privacy.  By focusing on real-world problems, the researcher may
overemphasize short-term solutions that cannot be generalized. The
short-term solution may not be the most efficient or effective solution for
the entire organization.  The focus on real-world problems leads to
evolutionary changes not revolutionary changes.  Evolutionary change is not
exciting and is often difficult to publish.  The last disadvantage, the
loss of privacy, can cause problems for researchers.  Most organizations do
not want their private data published.  Researchers, who often need to
publish, may not be permitted to publish the results of their research.


\section{Technology Transfer}

Doheny-Farina conducted three case studies of technology transfers.  His
interest is in the role of rhetoric in technology transfer.     \cite{Doheny92} 

Technology transfer is constructing the technology in the target
organization.   This includes the organization learning how to use the new
technology, adopting the new technology into the organization's
operations, adopting the organization's operations to the new technology.  

This adoption of new technology is primarily done through learning how to
use the technology.  During this learning process the organization changes
the technology from the researchers.  They adopt and the technology adopts
to the new situation.

How do organizations learn about new technology?  Primarily through the
manuals and training materials provided by the transfer agents.  The
manuals and training materials are a rhetorical method for transferring
knowledge.  However, the knowledge is not so much transferred as built in
the users.  The users use their experience to build up understanding of the
new technology.  This process changes the technology.  New understandings
and realizations about what the technology is occur.

Documentation is very important since that is what will be left after the
transfer is complete.  




\section{Adoption of Formal Technical Review}

Potts does not address the actual transfer of the solution from the
researcher to industry, except to say that it must be planned.  Grady and
Van Slack do discuss the adoption of a new technology, inspections, at
Hewlett-Packard\cite{Grady94}.  They developed a Model of successful
technology adoption and a metric for measuring where it stands.

\subsection{Model for technology adoption}

The adopting organization goes through four stages while adopting the new
technology; 1) the experimental stage, 2) the initial guidelines stage, 3)
the widespread belief and adoption stage, and 4) the standardization stage.
Figure \ref{fig:tech-adopt} illustrates the four levels of the Technology
Adoption Model.

\begin{figure}[htb]
  \begin{center}
    \setlength{\unitlength}{1.0cm}
    \begin{picture}(11,9)
      \put(1,1){\framebox(4,1){Experimental Stage}}
      \put(2.5,3){\framebox(4,1){Initial Guidelines Stage}}
      \put(4,5){\framebox(4,1){Widespread Belief Stage}}
      \put(5.5,7){\framebox(4,1){Standardization Stage}}
      \put(2.5,2){\vector(1,1){1}}
      \put(4.5,4){\vector(1,1){1}}
      \put(6.5,6){\vector(1,1){1}}
      \put(0.5,0.5){\vector(0,1){9}}
      \put(0.5,0.5){\vector(1,0){10}}
      \put(5,0){\makebox(2,0.5){Time}}
      \put(0,7){\makebox(3.5,0.5){Increasing use}}
      \put(0,6.5){\makebox(3.5,0.5){and savings}}        
    \end{picture}
  \end{center}
  \caption{Technology Adoption Model}
  \label{fig:tech-adopt}
\end{figure}


\paragraph{1) Experimental Stage}

The experimental stage is characterized by a small group of visionaries
bringing in a new technology to solve a local problem.  In the traditional
research-then-transfer methodology visionary people in the organization see
someone else's results and apply the new procedures or tools locally.  The
visionary personnel are trying to reproduce the success locally.  In
industry-as-laboratory, the visionary people bring in the researcher to
help solve the problem and support the local installation of the
researcher's solution.

Grady and Van Slack's lessons
learned are:
\begin{itemize}
\item{Having the right person at the right place is important to initiate
the adoption.}
\item{Management must foster these visionary attempts and not penalize
failures.} 
\item{Early success if fragile.  A supporting infrastructure can make the
difference between success and failure.}
\end{itemize}

\paragraph{2) Initial Guideline Stage}
In the initial guidelines stage the organization creates some internal
guidelines for using the new technology.  Training is established to train
new users.  
\begin{itemize}
\item{Communicating successes speeds both adoption and improvement of
procedures.}
\item{Clearly defining who is responsible for process improvement speeds
adoption of best practices}
\item{Management training contributes to strong, sustained sponsorship.}
\item{A high-level, compelling vision, like 10X goals that are directly
tied to business challenges, helps ensure strong management sponsorship.}
\item{Readily available training is necessary to speed technology adoption,
but it is not sufficient to sustain use.}
\end{itemize}

\paragraph{3) Widespread Belief and Adoption Stage}
In the widespread belief stage the organization believes in the new
technology and is adopting the technology.  The organization has strong
confidence that the technology has improved their productivity.  The
technology is still tailored for each new division that adopts it.  Often
the organization will assign personnel to aid other divisions in the
adoption of the technology.  These personnel act as consultants.  

Consulting model structured to match and meet customer needs, provide
appropriate training, and offer follow-up until the customer succeeds.  The
model has five steps:
\begin{itemize}
\item{Define the organizational business objective for doing
``inspection''}
\item{Evaluate and influence the organization's readiness to do
inspections.}
\item{Create an infrastructure for success by identifying a local
interested person as ``chief moderator'' (who will act as a champion)}
\item{Benchmark the current process.}
\item{Adjust the current process, train people, and consult to ensure
success.}
\end{itemize}

\paragraph{4) Standardization Stage}
In the standardization stage the organization does not need a single
``standard'' for the technology.  It must use some variation of the
technology in every project in a efficient, cost-effective way.

Standardization plan
\begin{enumerate}
\item{Pro-actively identify and support champions and sponsors.}
\item{Reinforce management awareness with a strong business case.}
\item{Continue building an infrastructure strong enough to achieve and hold
software core competence.}
\item{Measure the extent of adoption.}
\end{enumerate}

\subsection{Adoption Metric}

The adoption metric measures three adoption components.
\begin{itemize}
\item{{\em Depth} percentage of projects using ``inspections''}
\item{{\em Breadth} weighted percentage of documents ``inspected''}
\item{{\em Inspection-process maturity}}
  \begin{itemize}
  \item{Level 1: Initial/ad hoc (weight = 1)}
  \item{Level 2: Emerging (weight = 3)}
  \item{Level 3: Defined (weight = 10)}
  \item{Level 4: Managed (weight = 12)}
  \item{Level 5: Optimizing (weight = 14)}
  \end{itemize}
\end{itemize}

extent of adoption = inspection-process maturity * (percentage of projects
using inspections + weighted percentage of documents inspected) * constant


\cite{Grady94}




%\ls{1.0}