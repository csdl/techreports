%%%%%%%%%%%%%%%%%%%%%%%%%%%%%% -*- Mode: Latex -*- %%%%%%%%%%%%%%%%%%%%%%%%%%%%
%% diss-chp4.tex -- 
%% Author          : Carleton Moore
%% Created On      : Mon Oct  5 11:02:21 1998
%% Last Modified By: Carleton Moore
%% Last Modified On: Tue Apr 20 15:46:27 1999
%% RCS: $Id: diss-chp4.tex,v 1.1 1998/10/05 21:02:46 cmoore Exp cmoore $
%%%%%%%%%%%%%%%%%%%%%%%%%%%%%%%%%%%%%%%%%%%%%%%%%%%%%%%%%%%%%%%%%%%%%%%%%%%%%%%
%%   Copyright (C) 1998 Carleton Moore
%%%%%%%%%%%%%%%%%%%%%%%%%%%%%%%%%%%%%%%%%%%%%%%%%%%%%%%%%%%%%%%%%%%%%%%%%%%%%%%
%% 

\chapter{Post Dissertation Goals}

\section{Contributions}
We expect the following contributions from this research:
\begin{itemize}
  
\item{The Leap Tool Set provides a novel form of automated support for
    empirically based process improvement, including time, defect, size,
    and pattern recording and analysis. It is implemented in Java and runs
    on Windows, Unix, and Macintosh. You may download it at
    \url{http://csdl.ics.hawaii.edu/Tools/LEAP/LEAP.html}.}

\item{Insight into empirical process improvement.  The use of Leap helps us
    learn what improvements we can make using empirical measurement.  The
    case studies may provide insights into the limits of empirically based
    approaches to process improvement.}
  
\item{Insight into process improvement issues.  The PSP uses the classic
    waterfall software development model, fixed forms, and a single size
    definition. Leap relaxes many of these constraints.  This research
    should help us learn if these constraints are required for effective
    process improvement.}


\end{itemize}
