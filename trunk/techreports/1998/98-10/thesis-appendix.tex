%%%%%%%%%%%%%%%%%%%%%%%%%%%%%% -*- Mode: Latex -*- %%%%%%%%%%%%%%%%%%%%%%%%%%%%
%% thesis-appendix.tex -- 
%% Author          : Robert Brewer
%% Created On      : Tue Jan 10 12:04:50 1995
%% Last Modified By: Robert Brewer
%% Last Modified On: Wed Mar 15 17:09:14 2000
%% Status          : Unknown
%% RCS: $Id: thesis-appendix.tex,v 1.3 2000/03/17 21:28:36 rbrewer Exp $
%%%%%%%%%%%%%%%%%%%%%%%%%%%%%%%%%%%%%%%%%%%%%%%%%%%%%%%%%%%%%%%%%%%%%%%%%%%%%%%
%%   Copyright (C) 1998 Robert Brewer
%%%%%%%%%%%%%%%%%%%%%%%%%%%%%%%%%%%%%%%%%%%%%%%%%%%%%%%%%%%%%%%%%%%%%%%%%%%%%%%
%% 

\appendix

\chapter{``jcvs'' Mailing List Subject Lines}
\label{cha:subject-appendix}
In this appendix I present some actual Subject lines extracted from a few days
of traffic on the jCVS mailing list. I provide this to demonstrate the chatty
an disorganized nature of most mailing lists.

\begin{alltt}
{\small{}Subject: Re: [jcvs] HELP--authentication error
Subject: Re: [jcvs] Examples of Implementation
Subject: Re: [jcvs] failed authentication with the user name
Subject: [jcvs] Analog to cvs update -n -q?
Subject: Re: [jcvs] Analog to cvs update -n -q?
Subject: Re: [jcvs] Analog to cvs update -n -q?
Subject: Re: [jcvs] Analog to cvs update -n -q?
Subject: Re: [jcvs] Analog to cvs update -n -q?
Subject: Re: [jcvs] Analog to cvs update -n -q?
Subject: [jcvs] connect doesn't seem to work
Subject: Re: [jcvs] connect doesn't seem to work
Subject: [jcvs] admin -m / change log entries
Subject: [jcvs] cannont chdir(/root.home)
Subject: Re: [jcvs] Analog to cvs update -n -q?
Subject: [jcvs] License question..
Subject: [jcvs] Browsing projects with jCVSlet
Subject: [jcvs] Direct Connection with Kerberos
Subject: Re: [jcvs] admin -m / change log entries
Subject: [jcvs] Where WinInstaller
Subject: [jcvs] JCVS bug
Subject: Re: [jcvs] JCVS bug
Subject: Re: [jcvs] Where WinInstaller
Subject: [jcvs] kinda stuck...
Subject: RE: [jcvs] kinda stuck...
Subject: Re: [jcvs] cannont chdir(/root.home)
Subject: Re: [jcvs] Browsing projects with jCVSlet
Subject: Re: [jcvs] Direct Connection with Kerberos
Subject: [jcvs] Installation woes or ClassNotFoundException
Subject: Re: [jcvs] Installation woes or ClassNotFoundException}
\end{alltt}

\chapter{Introductory Email to jCVS List}
\label{cha:intro-email}
The following email was sent to the jCVS list as an introduction to the
existence of the condensed jCVS list archive:

\begin{alltt}
{\footnotesize{}Date: Mon, 24 Jan 2000 17:18:34 -1000
From: Robert Brewer <rbrewer@lava.net>
To: jcvs@mail.gjt.org
Subject: Problem Solving jCVS archive
Message-ID: <3841508885.948734314@sabrina.ics.Hawaii.Edu>
X-Mailer: Mulberry (Win32) [1.4.5, s/n U-300878]
MIME-Version: 1.0
Content-Type: text/plain; charset=us-ascii
Content-Transfer-Encoding: 7bit
Content-Disposition: inline

Mahalo for that kind introduction Tim!

As Tim mentioned, I'm working on my Masters degree in ICS here at the
University of Hawaii at Manoa.

My research is in the area of improving the archives of product support
mailing lists like this one. My basic thesis is that when people go to an
archive of a product support mailing list, it is usually because they are
having some sort of problem with the product and they want to find a
solution. Therefore the archive should be designed to maximize the
efficiency with which users can find solutions to their problems.

I have created a system called MCS (Mailinglist Condensation System) which
takes existing mailing list archives and turns them into problem solving
archives. This is done through a process called condensation which takes
the verbose content of a mailing list and removes the messages which don't
have long-term relevance. The condensation is done by a human editor (for
the jCVS archive this was me) who leaves out all the messages which are
irrelevant to problem solving. The editor annotates the messages with
keywords, writes a one-line summary of each message, and even removes
extraneous text from the body of messages. The result is an archive which
is much smaller with a higher information density and four methods by which
searches can be performed: keyword search, symptom search, 2D search, and
full-text search.

I hope you'll take the time to try out the archive and maybe it will save
you some time by tracking down a problem you've been having with jCVS. In
about two weeks I will be putting up an online survey at the MCS archive to
gauge people's reaction to the system. Feel free to send comments about the
system using the Feedback page.

The MCS archive is located at <http://csdl.ics.hawaii.edu:8100/> and Tim
has graciously added it to the footer of each message to the list for easy
access. I'll be continuing to condense new messages and add them to the
archive.

Thanks for your time, and I hope the archive can become a valuable resource
for the list which lets us spend more time discussing new issues and less
time answering frequently asked questions. :)}
\end{alltt}


\chapter{Raw Web Server Log Analysis}
\label{cha:raw-log-analysis}
This is the report generated by version 4.03 of the {\it analog} log file
analysis program \cite{analog-program} when supplied with the log file from the
web server running the condensed archive of the jcvs mailing list.

\begin{alltt}
{\footnotesize{}Web Server Statistics for jcvs Condensed Mailing List Archive
=============================================================

Program started at Wed-23-Feb-2000 17:46.
Analysed requests from Sat-22-Jan-2000 08:57 to Wed-23-Feb-2000 17:18 (32.3
  days).
----------------------------------------------------------------------------

General Summary
---------------
(Figures in parentheses refer to the 7 days to 23-Feb-2000 17:46).
Successful requests: 949 (281)
Average successful requests per day: 29 (40)
Successful requests for pages: 240 (76)
Average successful requests for pages per day: 7 (10)
Failed requests: 11 (0)
Distinct files requested: 339 (135)
Distinct hosts served: 99 (31)
Unwanted logfile entries: 278
Data transferred: 2.626 Mbytes (837.015 kbytes)
Average data transferred per day: 83.159 kbytes (119.573 kbytes)
----------------------------------------------------------------------------

Monthly Report
--------------
Each unit (+) represents 4 requests for pages or part thereof.

   month: reqs: pages: 
--------: ----: -----: 
Jan 2000:  262:    63: ++++++++++++++++
Feb 2000:  687:   177: +++++++++++++++++++++++++++++++++++++++++++++

Busiest month: Feb 2000 (177 requests for pages).
----------------------------------------------------------------------------

Daily Summary
-------------
Each unit (+) represents 2 requests for pages or part thereof.

day: reqs: pages: 
---: ----: -----: 
Sun:   68:    30: +++++++++++++++
Mon:  167:    37: +++++++++++++++++++
Tue:  150:    32: ++++++++++++++++
Wed:  336:    67: ++++++++++++++++++++++++++++++++++
Thu:   67:    23: ++++++++++++
Fri:  110:    38: +++++++++++++++++++
Sat:   51:    13: +++++++
----------------------------------------------------------------------------

Hourly Summary
--------------
Each unit (+) represents 1 request for a page.

hr: reqs: pages: 
--: ----: -----: 
 0:   12:     2: ++
 1:   48:     8: ++++++++
 2:    7:     3: +++
 3:   40:     4: ++++
 4:   68:    28: ++++++++++++++++++++++++++++
 5:    0:     0: 
 6:   80:    18: ++++++++++++++++++
 7:   50:    14: ++++++++++++++
 8:  146:    11: +++++++++++
 9:   73:    13: +++++++++++++
10:   14:     5: +++++
11:   46:    16: ++++++++++++++++
12:   34:    10: ++++++++++
13:   47:    17: +++++++++++++++++
14:   33:     9: +++++++++
15:    0:     0: 
16:   48:    16: ++++++++++++++++
17:   50:    16: ++++++++++++++++
18:    4:     4: ++++
19:    0:     0: 
20:    8:     7: +++++++
21:   82:    22: ++++++++++++++++++++++
22:   43:     9: +++++++++
23:   16:     8: ++++++++
----------------------------------------------------------------------------

Domain Report
-------------
Listing domains, sorted by the amount of traffic.

reqs: %bytes: domain
----: ------: ------
 360: 37.08%: .com (Commercial)
 205: 21.56%: .net (Network)
 176: 18.77%: [unresolved numerical addresses]
  85: 10.37%: .de (Germany)
  26:  2.91%: .fr (France)
  25:  1.79%: .gov (USA Government)
  12:  1.39%: .dk (Denmark)
  12:  1.19%: .uk (United Kingdom)
  13:  1.15%: .au (Australia)
  12:  1.04%: .at (Austria)
   4:  0.66%: .edu (USA Educational)
   5:  0.65%: .se (Sweden)
   4:  0.41%: .fi (Finland)
   3:  0.29%: .br (Brazil)
   3:  0.27%: .arpa (Old style Arpanet)
   2:  0.23%: .be (Belgium)
   1:  0.12%: .mil (USA Military)
   1:  0.11%: .za (South Africa)
----------------------------------------------------------------------------

Organisation Report
-------------------
Listing organisations, sorted by the number of requests.

reqs: %bytes: organisation
----: ------: ------------
 176: 18.77%: [unresolved numerical addresses]
  70:  6.81%: moorebcs.com
  39:  4.35%: collab.net
  35:  3.74%: uu.net
  33:  3.63%: eumetsat.de
  33:  3.31%: ses-astra.com
  28:  2.80%: earthlink.net
  28:  2.53%: pacbell.net
  25:  2.66%: nucleus.com
  24:  2.08%: best.com
  24:  2.87%: pacoffee.com
  24:  1.68%: lanl.gov
  24:  2.64%: ara.com
  22:  3.14%: dresdnerbank.de
  21:  2.27%: novell.com
  17:  1.61%: mindspring.net
  17:  1.50%: trustice.com
  14:  1.56%: ihost.com
  13:  1.15%: monash.edu.au
  13:  1.77%: multipath.com
  13:  1.06%: digex.com
  13:  1.64%: rmc.de
  13:  1.56%: cnc.net
  12:  1.19%: demon.co.uk
  12:  1.30%: mp3.com
  12:  1.59%: ubs.com
  12:  1.04%: tuwien.ac.at
   9:  0.92%: fedex.com
   9:  1.02%: fast-search.net
   9:  1.12%: fzk.de
   9:  1.03%: prserv.net
   8:  1.01%: kia.dk
   8:  0.95%: dialups.net
   7:  0.68%: codiciel.fr
   7:  0.54%: home.com
   7:  0.70%: hp.com
   7:  0.58%: kuit.com
   6:  0.27%: pncbank.com
   6:  0.94%: betasys.com
   6:  0.86%: silicomp.fr
   5:  0.45%: snap.com
   5:  0.65%: volvo.se
   5:  0.68%: univ-angers.fr
   4:  0.49%: bellsouth.net
   4:  0.21%: emn.fr
   4:  0.47%: uhc.com
   4:  0.41%: regex.fi
   4:  0.45%: uni-sb.de
   4:  0.37%: mediaone.net
   4:  0.38%: bfc.dk
   3:  0.37%: mich.net
   3:  0.33%: voyager.net
   3:  0.36%: univ-nantes.fr
   3:  0.33%: cnet.com
   3:  0.29%: acessonet.com.br
   3:  0.16%: cyrus.net
   3:  0.28%: rwth-aachen.de
   3:  0.27%: arpa
   2:  0.33%: msus.edu
   2:  0.23%: skynet.be
   2:  0.23%: eu.net
   2:  0.33%: rpi.edu
   1:  0.12%: af.mil
   1:  0.12%: capgemini.fr
   1:  0.12%: nasa.gov
   1:  0.11%: mweb.co.za
   1:  0.12%: digital.de
   1:  0.12%: verity.com
   1:  0.12%: brixnet.com
   1:  0.11%: lanxtra.com
   1:  0.11%: googlebot.com
----------------------------------------------------------------------------

Directory Report
----------------
Listing directories with at least 0.01% of the traffic, sorted by the amount
  of traffic.

reqs: %bytes: directory
----: ------: ---------
 707: 70.26%: /servlet/
 215: 26.69%: [root directory]
  27:  3.06%: /help/
----------------------------------------------------------------------------

File Type Report
----------------
Listing extensions with at least 0.1% of the traffic, sorted by the amount
  of traffic.

reqs: %bytes: extension
----: ------: ---------
 707: 70.26%: [no extension]
 167: 19.04%: [directories]
  73: 10.69%: .html [Hypertext Markup Language]
   2:  0.02%: [not listed: 1 extension]
----------------------------------------------------------------------------

File Size Report
----------------

      size: reqs: %bytes: 
----------: ----: ------: 
         0:    6:       : 
  1b-  10b:    0:       : 
 11b- 100b:    0:       : 
101b-  1kb:   14:  0.29%: 
 1kb- 10kb:  925: 97.99%: 
10kb-100kb:    4:  1.71%: 
----------------------------------------------------------------------------

Status Code Report
------------------
Listing status codes, sorted numerically.

reqs: status code
----: -----------
 948: 200 OK
   1: 304 Not modified since last retrieval
  11: 404 Document not found
----------------------------------------------------------------------------

Request Report
--------------
Listing files with at least 20 requests, sorted by the number of requests.

reqs: %bytes:       last date: file
----: ------: ---------------: ----
 305: 23.94%: 23/Feb/00 17:17: /servlet/MCSSearch
  53:  2.60%: 23/Feb/00 17:16:   /servlet/MCSSearch?mode=keyword
  37:  1.77%: 23/Feb/00 11:23:   /servlet/MCSSearch?mode=twod
  28:  1.68%: 23/Feb/00 11:23:   /servlet/MCSSearch?mode=symptom
  27:  1.52%: 23/Feb/00 09:55:   /servlet/MCSSearch?mode=fulltext
 298: 38.48%: 23/Feb/00 17:17: /servlet/MCSKeywordSelector
  38:  4.32%: 23/Feb/00 17:16:   /servlet/MCSKeywordSelector?java=false&mode=\\keyword
  30:  4.94%: 23/Feb/00 09:19:   /servlet/MCSKeywordSelector?java=false&mode=\\category
 167: 19.04%: 23/Feb/00 17:18: /
  80:  6.76%: 23/Feb/00 09:20: /servlet/MCSDisplay
  99: 11.79%: 23/Feb/00 17:15: [not listed: 16 files]
----------------------------------------------------------------------------

This analysis was produced by analog4.03/Unix.
Running time: 1 second.}
\end{alltt}

\chapter{Online User Questionnaire}
\label{cha:questionnaire-appendix}
I made the following questionnaire available to users of the system through a
web form. The questions were numbered, and each possible response is listed
with the numerical value which was used to denote that choice when recording
the data from the web form. Note that every multiple choice question contains
the option ``Not applicable''. This value was set as the default for every
question in the web form so that any bias towards default values would not skew
the results.

\section{MCS Two Minute Questionnaire}
Once you have used the this problem-solving archive, we would appreciate it if
you would take the time to fill out this brief questionnaire. It should only
take two minutes of your time. Mahalo!

If you have not used this archive yet, please check it out and then fill out
the questionnaire when you have experienced it.

\begin{enumerate}
\item How long have you used the jCVS software?
  \begin{itemize}
  \item 1. Never
  \item 2. Downloaded, installed, or read documentation but never actually used
  \item 3. Less than 3 months
  \item 4. 3-12 months
  \item 5. More than 12 months
  \item 0. Not applicable
  \end{itemize}
\item Are you subscribed to the jcvs mailing list?
  \begin{itemize}
  \item 1. Yes
  \item 2. No
  \item 0. Not applicable
  \end{itemize}  
\item If you are subscribed, on average, how often do you read list messages?
  \begin{itemize}
  \item 1. Whenever an email arrives
  \item 2. Once a day
  \item 3. About three times a week
  \item 4. Once a week
  \item 5. Once a month
  \item 6. Almost never
  \item 7. I only read it when I have a problem that needs solving
  \item 0. Not applicable
  \end{itemize}
\item If you are subscribed, on average, what fraction of list messages do
  you actually read?
  \begin{itemize}
  \item 1. Zero
  \item 2. Less than a third
  \item 3. Between one and two thirds
  \item 4. More than two thirds
  \item 5. Every message
  \item 0. Not applicable
  \end{itemize}
\item Have you used the old archive of the jcvs mailing list?
  \begin{itemize}
  \item 1. Yes
  \item 2. No
  \item 0. Not applicable
  \end{itemize}  
\item If so, roughly how many times have you used it?
  \begin{itemize}
  \item 1. More than 10 times
  \item 2. 6-10 times
  \item 3. 2-5 times
  \item 4. Once
  \item 0. Not applicable
  \end{itemize}
\item How often did you find what you were looking for in the old archive?
  \begin{itemize}
  \item 1. Never
  \item 2. Rarely
  \item 3. Sometimes
  \item 4. Usually
  \item 5. Always
  \item 0. Not applicable
  \end{itemize}
\item Roughly how many times have you used this new problem solving archive?
  \begin{itemize}
  \item 1. More than 10 times
  \item 2. 6-10 times
  \item 3. 2-5 times
  \item 4. Once
  \item 5. Never
  \item 0. Not applicable
  \end{itemize}
\item How often did you find what you were looking for in this new archive?
  \begin{itemize}
  \item 1. Never
  \item 2. Rarely
  \item 3. Sometimes
  \item 4. Usually
  \item 5. Always
  \item 0. Not applicable
  \end{itemize}
\item Since the problem-solving archive has been available, do you find
  yourself using it instead of the existing archive?
  \begin{itemize}
  \item 1. Yes
  \item 2. Somewhat
  \item 3. No
  \item 0. Not applicable
  \end{itemize}
\item Overall, how would you rate your satisfaction with this new archive?
  \begin{itemize}
  \item 1. Completely Satisfied
  \item 2. Somewhat Satisfied
  \item 3. Somewhat Unsatisfied
  \item 4. Completely Unsatisfied
  \item 5. Undecided
  \item 0. Not applicable
  \end{itemize}
\item The messages in this archive were condensed by a human editor, requiring
  some effort. Would you be willing to help maintain this archive as one of
  many editors on a volunteer basis? [{\bf Note}: this information is for
  research purposes so answering ``Yes'' will not commit you to anything]
  \begin{itemize}
  \item 1. Yes
  \item 2. Not sure
  \item 3. No
  \item 0. Not applicable
  \end{itemize}
\item Would any other mailing lists you are interested in benefit from having
  this kind of archive? If so, please list them below:
\item If you have any other comments or suggestions about this archive, please
  let us know!
\end{enumerate}


\chapter{Raw Questionnaire Results}
\label{cha:raw-questionnaire-results}
In this appendix, I provide all the raw data from the web questionnaire
provided to users. See Appendix \ref{cha:questionnaire-appendix} for the list
of questions and what the answer values correspond to. In Table
\ref{tab:multiple-choice-answers} I show the answers given to the multiple
choice questions, and Table \ref{tab:open-answer} shows the results of the
open-answer questions.

\begin{table}[htbp]
  \begin{center}
    \begin{tabular} {|c||c|c|c|c|c|c|c|c|c|c|c|c|} \hline
      {\bf Survey \#} & {\bf Q 1} & {\bf Q 2} & {\bf Q 3} & {\bf Q 4} &
      {\bf Q 5} & {\bf Q 6} & {\bf Q 7} & {\bf Q 8} & {\bf Q 9} & {\bf Q 10} &
      {\bf Q 11} & {\bf Q 12}\\ \hline\hline
      {\bf 1} & 5 & 1 & 6 & 2 & 2 & 0 & 0 & 0 & 0 & 0 & 0 & 0\\ \hline
      {\bf 2} & 2 & 2 & 0 & 0 & 0 & 0 & 0 & 0 & 0 & 0 & 0 & 2\\ \hline
      {\bf 3} & 5 & 1 & 3 & 2 & 1 & 3 & 3 & 4 & 5 & 1 & 1 & 1\\ \hline
      {\bf 4} & 5 & 1 & 0 & 5 & 1 & 1 & 3 & 3 & 5 & 1 & 1 & 1\\ \hline
      {\bf 5} & 4 & 1 & 1 & 5 & 2 & 0 & 0 & 4 & 3 & 0 & 2 & 3\\ \hline
      {\bf 6} & 4 & 1 & 1 & 3 & 2 & 0 & 0 & 4 & 2 & 0 & 2 & 2\\ \hline
    \end{tabular}
    \caption{Raw response data from questionnaire's multiple choice questions}
    \label{tab:multiple-choice-answers}
  \end{center}
\end{table}

\begin{table}[htbp]
  \begin{center}
    \begin{tabular} {|c|p{1.3in}|p{3.4in}|} \hline
      {\bf Survey \#} & {\bf Q 13} & {\bf Q 14}\\ \hline\hline
      {\bf 1} &  & if there were a jcvs-announce list, i would unsub from the jcvs list, join the announce list, and make use of your archive when it was announced. because the noise-to-signal ratio on the jcvs list is so high, i never noticed your archive announcement. i'm looking forward to checking it out now :-)\\ \hline
      {\bf 2} & Yes, will discuss it with them. & Your product looks good. As a concept I would be VERY interested in a page giving stats (time devoted to editing, etc...) by those that condense, and some personal feedback from those people as far as how difficult they percieve their task. Keep me on what ever mailing list I recieved your notice and I will watch the development of your program.  Thank you.\\ \hline
      {\bf 3} & The webmacro mailing list (\url{www.webmacro.org}) & Maybe you could consider using tools that help you in summarizing and tagging text. There are a lot of scientific projects out there, but I'm not sure if it would be easy to get them on an open source basis. \\ \hline
      {\bf 4} & GNUJSP, GJT-DEV & I think the effort is tremendous, and the tool a great improvement over what was available. I would be very excited to see further development and support.\\ \hline
      {\bf 5} & PHP Mailing List -$>$ \url{www.php.net} & Neat concept - could be very useful for larger lists like the php or perl or mysql list.  A significant part of the message volume is from people asking common questions.  People are usually asking the question because they don't know what keywords to use to search the archives for the solution to their problem. (i.e. If you know how to grok widgets, then you probably know that you need to use the foo() function to do this.  However, if you don't know how to grok widgets, then you would have to use 'grok and widget' as your keys for searching the archives.  Chances are that whoever asked the question last phrased their question in a different fashion - so the previous exchange on the list regarding this topic is very little help to the new user. ) ..uh.. why I am bothering to write this - you obviously know this already - that is why you built the system... :P  One note - I found that the interface was pretty clunky.  Good Luck!  PS Remember that the best defense is a good offence - try using foul language in your thesis defense ;)\\ \hline
      {\bf 6} &  & \\ \hline
    \end{tabular}
    \caption{Raw response data from questionnaire's open answer questions}
    \label{tab:open-answer}
  \end{center}
\end{table}

% LocalWords:  jcvs cvs admin cannont chdir jCVSlet Kerberos WinInstaller kinda
% LocalWords:  ClassNotFoundException charset ascii inline Mailinglist Analysed
% LocalWords:  Mbytes kbytes reqs hr com de fr gov dk uk au edu se fi br arpa
% LocalWords:  za Organisation organisations organisation moorebcs collab uu mp
% LocalWords:  eumetsat ses astra earthlink pacbell pacoffee lanl ara novell hp
% LocalWords:  dresdnerbank mindspring trustice ihost multipath digex rmc cnc
% LocalWords:  ubs fedex fzk prserv kia dialups codiciel kuit pncbank betasys
% LocalWords:  silicomp volvo univ bellsouth emn uhc regex uni sb mediaone bfc
% LocalWords:  mich nantes cnet cyrus rwth aachen msus skynet eu rpi af nasa kb
% LocalWords:  capgemini brixnet lanxtra googlebot html Markup twod fulltext uh
% LocalWords:  java eyword ategory unsub i'm stats percieve recieved webmacro
% LocalWords:  GNUJSP GJT DEV PHP php mysql foo PS offence
