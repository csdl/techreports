\documentstyle[times,12pt,/group/csdl/tex/definemargins]{article}
\definemargins{1in}{1in}{1in}{1in}{0.3in}{0.3in}
\input{/group/csdl/tex/psfig/psfig}

\begin{document}

\title{A Cautionary Case Study of the Personal Software Process}

\author{Philip M. Johnson\\
        Anne M. Disney\\
        Dept. of Information and Computer Sciences\\
        University of Hawaii\\
        Honolulu, HI  96822 USA\\
        +1 808-956-6920\\
        johnson@hawaii.edu}

\maketitle

\section{The Personal Approach to Software Process}

The Personal Software Process (PSP) was introduced in 1995 in the book, ``A
Discipline for Software Engineering'' \cite{Humphrey95}.  This text
describes a four month curriculum that teaches concepts in
empirically-guided software process improvement. The text has been used 
as a basis for both university and academic courses to
teach and improve software engineering skills. 

Programmers using the PSP gather measurements related to their own work
products and the process by which they were developed, and use these
measures to drive changes to their development behavior.  PSP focuses on
defect reduction and estimation accuracy improvement as the two primary
goals of personal process improvement. Through individual collection and
analysis of personal data, the PSP illustrates how empirically-guided
software process improvement can be implemented by individuals. The full
PSP curriculum leads practitioners through a sequence of seven personal
processes.  The first and most simple PSP process, PSP0, requires practitioners
to keep track of time and defect data using a ``Time Recording Log'' and
``Defect Recording Log'' similar to those illustrated in Figures \ref{time}
and \ref{defects}, and then fill out a ``Project Summary Report'' similar
to that illustrated in Figure \ref{psp0}.  Later processes become more
complicated, and introduce size and time estimation, scheduling, and
quality management practices such as defect density predication and cost of
quality analyses. Figure \ref{psp2} is similar to one of the more
advanced ``Project Summary Forms'' and illustrates aspects of the sophistication
present in later processes.

  \begin{figure} [tp]
    {\centerline{\psfig{figure=/group/csdl/techreports/98-11/time.eps}}}
    \caption{\label{time}A sample portion of a time recording log.}
  \end{figure}

  \begin{figure} [tp]
    {\centerline{\psfig{figure=/group/csdl/techreports/98-11/defects.eps}}}
    \caption{\label{defects}A sample portion of a defect recording log.}
  \end{figure}

  \begin{figure} [tp]
    {\centerline{\psfig{figure=/group/csdl/techreports/98-11/psp0.eps}}}
    \caption{\label{psp0}A sample portion of the PSP0 process summary form.}
  \end{figure}

  \begin{figure} [tp]
    {\centerline{\psfig{figure=/group/csdl/techreports/98-11/psp2.eps}}}
    \caption{\label{psp2}A sample portion of the PSP2 process summary form.}
  \end{figure}


\section{Benefits of the PSP}

Since its introduction, positive experiences with the PSP have been
reported in several case studies \cite{Ramsey96,Shostak96,Ferguson97}.  For
example, in one broad-ranging study, researchers from the Software
Engineering Institute analyzed data submitted to them by instructors of 23
PSP classes at both academic and industrial sites.  The report concludes
that the PSP improved performance in size and effort estimation accuracy,
product quality, process quality, and personal productivity, without any
loss on productivity \cite{CMU97}.

At the University of Hawaii, we have taught and used the PSP for over two
years.  Our experiences with the PSP have also been positive, and the
results of our analyses mirror those reported by other researchers. For
example, our final PSP project assignment typically requires each student
to design and develop a 500-1000 line Java program. Using the
personal data they collect on their first eight projects, many students
successfully estimate both the size and time required for this final project
with at least 95\% accuracy.  For students who typically enter the course
believing that ``software estimation'' is an oxymoron, this experience is
revelatory. Some typical post-course evaluation comments are:

\begin{itemize}
\item ``I thought I was a good programmer, but after using PSP I realized
      that I was nothing back then.''
\item ``I must admit, when I started this course, I understood what we were
        supposed to do in good software engineering, but I never really did
        it.  Now I understand the reasons behind these practices and the
        benefits of actually following a process instead of just jumping
        right into coding.''
\item ``By executing and learning this process
       I know way more about software engineering than when I
       started this course."
\end{itemize}

We also practice what we preach...and teach.  We applied PSP concepts to
the development of a ``Personal Thesis Process'' for use by graduate
students.  One of us (Anne Disney) has consistently used PSP in her
workplace for over two years, and has recorded PSP data for over 120
commercial database development projects---likely the largest
collection of PSP data on a single developer's industrial practice in
existance.


\section{Issues with manual PSP}

Although all of these results undeniably speak well of the PSP, we have
long been concerned with the amount of paperwork and manual calculation
involved in the original PSP curriculum.  For a software system developed
using PSP2.0, for example, students fill out twelve separate paper forms,
including a project plan summary, time recording log, defect recording log,
process improvement proposal, size estimation template, time estimation
template, object categories worksheet, test report template, task planning
template, schedule planning template, design checklist, and code checklist!
These dozen forms typically yield over 500 distinct values that each
student calculates and fills in manually for a single project.  If you
think this sounds tedious, consider the fate of the instructor who has to
check all of these values!  We estimated that in a recent PSP course, the
instructor manually checked over 31,000 PSP data values for the nine
projects developed by ten students, in addition to checking the actual
software projects themselves! 

There is nothing wrong with a heavy workload, of course, when the means
justify the ends. In the case of manual PSP, however, we began to wonder if
the clerical overhead involved in completing and checking the forms by hand
might significantly impact upon the PSP itself?  To better understand this
issue, consider the simple ``model of PSP data quality'' in Figure
\ref{fig:model}. Through a process of {\em collection}, the developer
generates an initial empirical representation (``Records of Work'') of her
personal process (``Actual Work'').  Through additional {\em analyses}, the
developer augments her initial empirical representation with derived data
(``Analyzed Work'') intended to enable process improvement through
``Insights about Work''.


\begin{figure} [tp]
    {\centerline{\psfig{figure=/group/csdl/techreports/98-04/PSPmodel.eps}}}
    \caption{\label{fig:model} A simple model for PSP data quality.
      }
\end{figure}

In other words, the goal of the PSP is to generate an empirical profile, or
model, of the user's actual software development behavior that is accurate
enough to support process improvement.  In our initial experiences with the 
manual PSP, we found we were catching dozens upon dozens of clerical and 
other errors made by students during the course. If we were catching so
many errors, could many more be slipping through, and if so, were these
errors making an impact upon the behavioral model used to support
process improvement?

\section{A Case Study to Check PSP Data Quality}

To answer this question, we performed a case study as part of a class of
ten students learning the PSP in 1996. Our case study
design involved teaching the class using the manual PSP method, augmented
with certain curriculum modifications designed to improve data quality. For
example, we instituted technical reviews of the PSP data, and generated
supplemental forms to clarify the process of certain multi-step estimation
techniques.  Next, we designed and implemented a database system that we
used to enter each of the over 30,000 data values from the paper forms, and
compare the results of each student's calculations with those computed by
the database package.  We used this to compare the student-generated
empirical models of their behavior with the database-generated empirical
models, and determine if any differences existed.

At the start of this case study, we anticipated that the database program
might uncover 50 to 100 errors. To our astonishment, the program discovered
1539 errors!  We also found that in some cases, these errors were not
simply ``noise'' in the model, but did appear to impact upon some of the
important PSP measures, such as Yield (the percentage of defects discovered
that were removed before the first compile) and the Cost-Performance Index
(a measure of how well the planned effort predicts actual effort).
Figures \ref{compareCPI} and \ref{compareYield} illustrate the differences
between the original values calculated by the students and the
``corrected'' values calculated by our database program.

  \begin{figure} [tp]
    {\centerline{\psfig{figure=/group/csdl/techreports/98-04/8cpi2.eps}}}
    \caption{\label{compareCPI}Effect of Correction on CPI}
  \end{figure}

  \begin{figure} [tp]
    {\centerline{\psfig{figure=/group/csdl/techreports/98-04/8yield2.eps}}}
    \caption{\label{compareYield}Effect of Correction on Yield}
\end{figure}
  
One of the first questions we asked ourselves upon obtaining such a high
number of errors was: Could this poor data quality be a simple result of
poor instruction and/or project correction?  As tempting as this
explanation might be, the data does not appear to support it.  The 1539
incorrect data values represent only 4.8\% of the total number of data
values, which means that the instructor checked over 31,000 data values by
hand with over 95\% accuracy.  Although there is always room for
improvement in any instructional context, 95\% accuracy does not support
the idea that the results are due to poor instruction.


\section{Recommendations}

Do our results indicate that the PSP should be abandoned?  Certainly not,
and we intend to continue using it for teaching, research, and our own
industrial software development activities. Furthermore, these results are
based upon data from a single course, and a controlled experimental design
was not used.  Nevertheless, we recommend you consider the following
when deploying the PSP within your academic or industrial organization:

\begin{enumerate}
  
\item {\em Despite the potential for data quality problems, the PSP still
    teaches useful skills.} Some detractors claim that the improvements in
  product and process quality attributed to the PSP over the four months of
  the course would occur in any programmer doing the ten programming projects,
  regardless of whether the PSP was usedor not. We disagree
  emphatically.  We have taught both introductory Java programming
  (involving completion of 10 or so Java programming assignments) and advanced
  software engineering (involving completion of 10 or so Java programming
  assignments using the PSP).  Although the results are not controlled,
  our anecdotal experience suggests that the
  the PSP adds substantial value. While students in both
  contexts do improve with respect to syntax and programming idioms, only
  the PSP students acquire concrete software engineering skills involving
  time and size estimation, defect removal costs, design quality, and so
  forth. These skills, in turn, produce insights concerning process and
  product quality improvement not available to programmers who just hack
  code.
  
\item {\em Avoid teaching or adopting a manual, paper-based version of the
    PSP.} Fortunately, since the time of our case study, a few automated
  tools for the PSP have become available. For example, East Tennessee
  State University makes a package called the ``PSP Design Studio''
  available on-line \cite{PSPS97}.  In our opinion, it is vital that any
  PSP toolset used must {\em replace} the hardcopy forms from the original
  PSP curriculum with on-line equivalents, rather than simply compute
  values that must be transferred to the forms by hand. We say this because
  we provided the students in our study with tools including spreadsheets
  and Java-based code counting and estimation applets, but these
  (unintegrated) tools still did not prevent over 1500 errors from
  occurring.
  
\item {\em Avoid PSP measures when attempting to evaluate the success of
    PSP use.} Since the PSP produces measures of product and process
  quality, it is quite tempting to use changes in these measures from the
  beginning of the course to the end of the course as evidence of the
  success of the method. This approach to evaluation is widespread in
  current PSP case studies, which frequently cite conclusions such as:
  ``The improvement in average defect levels for engineers who complete the
  course is 58\% for total defects per KLOC...''.  The problem is that such
  numbers are derived from the {\em model} of programmer behavior built by
  the PSP, which as our case study shows, may not always accurately
  represent {\em actual} programmer behavior.  Instead of using PSP
  measures to evaluate the PSP, case studies should use external, non-PSP
  measures.  As an example of this approach, one report on industrial use
  of PSP found that acceptance test defect density fell after the
  introduction of PSP into selected development groups \cite{Ferguson97}.
  
\item {\em Automated support is not a silver bullet for the problem of PSP
    data quality.} As always in software engineering, automated tool
  support is not a panacea. Looking again at our model in Figure
  \ref{fig:model}, automated support has the potential to dramatically
  reduce or eliminate the ``analysis'' errors that occur while transforming
  ``Records of Work'' into ``Analyzed Work''. Unfortunately, automated
  support can do little to guarantee high quality ``collection''---in other
  words, that the programmer's ``Records of Work'' accurately reflect her
  ``Actual Work''.  Thus, even if automated support is used to provide
  defect-free analysis, collection-stage errors could still lead to an
  inaccurate PSP model of programmer behavior.

\end{enumerate}


For more details on this case study, including a discussion of the types
of errors we found, their severity, and their origin, see either our
technical report \cite{csdl-98-04} or the thesis discussing this
research \cite{csdl-98-08}


\bibliographystyle{plain}
\bibliography{/group/csdl/bib/csdl-trs,/group/csdl/bib/psp,/group/csdl/bib/ftr} 
\end{document}



