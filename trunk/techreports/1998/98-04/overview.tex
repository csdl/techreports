%%%%%%%%%%%%%%%%%%%%%%%%%%%%%% -*- Mode: Latex -*- %%%%%%%%%%%%%%%%%%%%%%%%%%%%
%% overview.tex -- 
%% Author          : Philip Johnson
%% Created On      : Wed Apr  8 14:23:39 1998
%% Last Modified By: Philip Johnson
%% Last Modified On: Mon Aug 10 14:36:38 1998
%% RCS: $Id$
%%%%%%%%%%%%%%%%%%%%%%%%%%%%%%%%%%%%%%%%%%%%%%%%%%%%%%%%%%%%%%%%%%%%%%%%%%%%%%%
%%   Copyright (C) 1998 Philip Johnson
%%%%%%%%%%%%%%%%%%%%%%%%%%%%%%%%%%%%%%%%%%%%%%%%%%%%%%%%%%%%%%%%%%%%%%%%%%%%%%%
%% 

\section{OVERVIEW OF THE PSP}


In the PSP curriculum presented in ``A Discipline for Software
Engineering'', each student develops 10 small programs over the course of a
semester using a sequence of seven increasingly sophisticated software
development processes, labeled PSP0 to PSP3.  For every program, the students record various
measurements related to their personal development activities. Such
measures include, for example, the time spent in each phase of development,
the numbers of defects injected and removed during each phase, and the size
of the resulting work product.

The initial programs use relatively simple processes that establish a
baseline set of process measures for time, size, and defects. Later
programs employ more advanced processes that extend these baseline
process statistics.  Although there are a myriad of individual extensions, 
most fall into two conceptual categories.

First, the planning phase is expanded to include estimates of the program's
projected size, the projected time required to complete each of the phases,
and the number of anticipated defects that will be injected and removed
during development.  The process by which these estimates are produced
involves statistical analysis of historical correlations between designs
(i.e. class and method counts) and actual size (in lines of code), between
estimated size and actual time, between actual size and actual time, and
between size and defects injected and removed.  (While lines of code as a
metric of size at the organizational level is almost uniformly excoriated
in the measurement literature, it seems to work surprisingly well in the
PSP, since the measure is collected and applied to a single individual working
in a single language in a relatively uniform domain.)

Second, by the middle of the course, each student has typically recorded a
hundred or more defects made during development.  Later processes
include mechanisms to help students understand the impact of various kinds
of defects and to drive process improvements intended to reduce future
occurrence of important defect types. For example, since students record
the phase each defect was injected and removed and the time required to fix
it, it is possible to analyze the relationship between fix time and various
characteristics of defects.  One relationship nearly
always present in student data is that the ``longer'' a defect is present,
the more time it takes to remove it.  Thus, defects injected during design
and not removed until testing are nearly always more expensive to remove
than, for example, defects injected during coding and removed during
compiling. This outcome typically motivates students to put more effort and
care into design activities. Later processes support such behaviors by
providing active defect management mechanisms. For example, by analyzing
defect data to determine the types of design defects made on prior
projects, a student can generate a checklist to be used as part of a
personal design review to ensure that those defects do not escape into
code, compile, or test phases.

The final stages of the course further extend the basic PSP para\-digm.  The
last PSP process provides a way to scale the method to support larger
projects using a cyclic development method. In addition, PSP includes a
meta-level process for defining personal processes in non-software domains
or for specific software organizational contexts.

