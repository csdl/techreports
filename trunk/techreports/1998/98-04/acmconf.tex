\documentstyle[]{acmconf}

\begin{document}

\title{Assessing software review meetings: \\ A controlled experimental study using CSRS}

\author{
        Philip M. Johnson\\
        Danu Tjahjono\\
        Department of Information and Computer Sciences\\
        University of Hawaii\\
        Honolulu, HI, 96822 USA\\
        johnson@hawaii.edu
       }


\maketitle

\section{ABSTRACT}

Software review is a fundamental component of the software quality
assurance process, yet significant controversies exist concerning
efficiency and effectiveness of various review methods. A central question
surrounds the use of meetings: traditional review practice views them as
essential, while more recent findings question their utility.

We conducted a controlled experimental study to assess several measures of
cost and effectiveness for a meeting and non-meeting-based review method.
The experiment used CSRS, a computer mediated collaborative software review
environment, and 24 three person groups.  Some of the data we collected
included: the numbers of defects discovered, the effort required, the
presence of synergy in the meeting-based groups, the occurrence of false
positives in the non-meeting-based groups, and qualitative questionnaire
responses.

This paper presents the motivation for this experiment, its design and
implementation, our empirical findings, conclusions, and future directions.


\subsection{Keywords}
Formal technical review, inspection, experimental study, CSRS.

\section{INTRODUCTION}

Formal technical review is an umbrella term for a variety of structured
group processes designed to assess and improve the quality of a software

\end{document}

