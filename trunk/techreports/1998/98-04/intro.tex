%%%%%%%%%%%%%%%%%%%%%%%%%%%%%% -*- Mode: Latex -*- %%%%%%%%%%%%%%%%%%%%%%%%%%%%
%% intro.tex -- 
%% Author          : Philip Johnson
%% Created On      : Wed Apr  8 14:22:56 1998
%% Last Modified By: Philip Johnson
%% Last Modified On: Mon Aug 10 12:00:22 1998
%% RCS: $Id$
%%%%%%%%%%%%%%%%%%%%%%%%%%%%%%%%%%%%%%%%%%%%%%%%%%%%%%%%%%%%%%%%%%%%%%%%%%%%%%%
%%   Copyright (C) 1998 Philip Johnson
%%%%%%%%%%%%%%%%%%%%%%%%%%%%%%%%%%%%%%%%%%%%%%%%%%%%%%%%%%%%%%%%%%%%%%%%%%%%%%%
%% 

\section{INTRODUCTION}

\begin{quotation}
\noindent {\em The actual process is what you do, with all its omissions, mistakes, and
oversights. The official process is what the book says you are supposed to
do.} \cite{Humphrey95}
\end{quotation}

The Personal Software Process (PSP) was introduced in 1995 in the book, ``A
Discipline for Software Engineering'' \cite{Humphrey95}.
This text
describes a one-semester curriculum for advanced undergraduates or graduate
students in computer science that teaches concepts in empirically-guided
software process improvement. Since its introduction, experience with
the PSP has been reported on in several case studies \cite{Ceberio-Verghese96,Ferguson97,Humphrey96,Tomayko96,Humphrey97}.   Although empirically-guided software process
improvement is a key feature of other software engineering initiatives,
such as the Capability Maturity Model (CMM) \cite{Paulk95}, ISO-9000, and
Inspection \cite{Gilb93}, the PSP differs from these other approaches in
important ways.

The CMM, ISO-9000, and Inspection discuss empirical software process
improvement in the context of a large organization.  Process improvement in
this context requires the gathering and analysis of large amounts of data,
within and across departments, generated by different people at different
times.  Indeed, inevitable personnel turnover means that the data collected
from the working procedures of one set of people tend to generate
measurements leading to process changes that affect the working procedures
of a potentially different set of people.  The substantial effort required
to collect, interpret, and introduce organizational change based upon the
measurements for a large organization leads to the need for an explicit
software engineering process group (SEPG) whose mission is to manage
empirically guided improvement. Although the utility of these approaches
have been repeatedly validated, they leave the unfortunate impression that
empirically-guided software process improvement is the sole province of
large organizations who can dedicate teams of people to its enactment.

The PSP provides an alternative, complementary approach in which
empirically guided software process improvement is ``scaled down'' to the
level of an individual developer.  In the PSP, individuals gather
measurements related to their own work products and the process by which
they were developed, and use these measures to drive changes to their
development behavior.  PSP focuses on defect reduction and
estimation accuracy improvement as the two primary goals of personal
process improvement. Through individual collection and analysis of personal
data, the PSP provides a novel example of how empirically-guided software
process improvement can be implemented by individuals regardless of the
surrounding organizational context and the availability of institutional
infrastructure support.

Since PSP is a new technique, relatively little data exists on its use and
effectiveness.  Those studies of which we are aware all report positive
%PJ
results, usually based upon measurements obtained during enactment of the PSP
curriculum. For example, one case study states that ``during the
course, productivity improvements average around 20\% and product quality,
as measured by defects, generally improves by five times or more''
\cite{Ferguson97}. Another study states that ``the improvement in average
defect levels for engineers who complete the course is 58.0\% for total
defects per KLOC and 71.9\% for defects per KLOC found in test.''  Indeed,
our own PSP data yields similarly positive measurements for process and
products.

In this paper, we report on a case study performed to assess the quality
of PSP data---the data often used in evaluations of the effectiveness of
the PSP as shown above.  Our case study was motivated by
our experiences teaching and using the PSP, which led us to suspect
that the empirical measures gathered by the PSP may not, in all cases,
reflect the true underlying process or products of development.  
%PJ
We hypothesized that one class of problems---data analysis errors---could
significantly change at least some of the measures produced by the PSP that
are commonly used to evaluate its effectiveness.  To test this hypothesis,
we taught a modified version of the PSP curriculum augmented with
mechanisms to ameliorate potential PSP data quality problems.  We then
entered the PSP measures into a database and subjected them to a variety of
data quality analyses.  These analyses uncovered over 1500 errors in the
PSP data generated by the ten students in the class during nine projects.
Additional analysis yielded a seven part classification scheme for PSP data
errors.  Although we were not always able to generate corrected values for
the data errors, partial correction lead to substantially different values
for certain PSP measurements, confirming our hypothesis.

The remainder of the paper is organized as follows. The next two sections
present a brief overview of the PSP and a simple model we developed to
organize our exploration of PSP data quality problems. The following three
sections present the case study, its results, and our conclusions.

