%%%%%%%%%%%%%%%%%%%%%%%%%%%%%% -*- Mode: Latex -*- %%%%%%%%%%%%%%%%%%%%%%%%%%%%
%% problems.tex -- 
%% Author          : Philip Johnson
%% Created On      : Wed Apr  8 14:23:59 1998
%% Last Modified By: Philip Johnson
%% Last Modified On: Mon Aug 10 14:19:37 1998
%% RCS: $Id$
%%%%%%%%%%%%%%%%%%%%%%%%%%%%%%%%%%%%%%%%%%%%%%%%%%%%%%%%%%%%%%%%%%%%%%%%%%%%%%%
%%   Copyright (C) 1998 Philip Johnson
%%%%%%%%%%%%%%%%%%%%%%%%%%%%%%%%%%%%%%%%%%%%%%%%%%%%%%%%%%%%%%%%%%%%%%%%%%%%%%%
%% 

\section{A MODEL OF PSP DATA QUALITY}

\begin{figure*}
%   {\centerline{\psfig{figure=graphic2.eps}}}
    {\centerline{\psfig{figure=PSPmodel.eps}}}
    \caption{\label{fig:model} A simple model for PSP data quality. Through 
      a process of {\em collection}, the developer generates an initial
      empirical representation (``Records of Work'') of her personal process
      (``Actual Work'').  Through additional {\em analyses}, the developer
      augments her initial empirical representation with
      derived data (``Analyzed Work'') intended to enable process improvement
      through ``Insights about Work''.
      }
\end{figure*}
     
To guide our understanding of data quality problems in the PSP, we devised
a simple two stage model of PSP data, as illustrated in Figure
\ref{fig:model}.  The model begins with ``Actual Work''---the actual developer
efforts devoted to a software development project.  As part of these
efforts, the developer {\em collects} a set of primary measures on defects,
time, and work product characteristics---the ``Records of Work''. Based on
these primary measures, the developer performs additional {\em analyses},
many of which result in secondary (i.e.  derived) measures which are
themselves inputs for further analyses.  The secondary, derived measures
and associated analyses are presented in various PSP forms---the ``Analyzed
Work''---and hopefully yield ``Insights about Work'' to improve future
software development activities.

% Added by AD, 08/06/98
We based this model upon the PSP as presented in {\it A Discipline for
  Software Engineering} \cite{Humphrey95}
--- what we term ``manual PSP''.  
Manual PSP refers to a situation in which the PSP forms must be filled out
by hand, either by editing a copy of the form on-line, or by filling out
out a printed copy with pen or pencil.  Even if tools such as spreadsheets
are used to collect historical data and to provide various computations, if
they do not automatically insert and maintain the correct calculated values
in the appropriate places in the forms, then we define the technique as
``manual''.  We define partially or fully ``automated'' PSP as one in which 
some or all of the derived measures are calculated and placed into the forms
automatically.  In other words, in automated PSP, the analysis tools and
forms presenting the PSP reports are tightly {\it integrated}. 
Although ``automated'' PSP can essentially automate all of the analysis stage
calculations, there are limits to its ability to automate the collection
stage work.  The collection stage is still quite ``manual'' in nature.

At the time we performed this case study, there was no {\em integrated}
software support for the PSP.  Thus, the case study employed what we call the
``manual'' version of the PSP, despite our extensive use of spreadsheets, program
size counting tools, and statistical tools during the course.  Since then,
integrated tools have become available, including spreadsheets available at
the Addison-Wesley FTP site which print out the project summary forms, and
the Personal Software Process Studio tool produced by East Tennessee State
%PJ
University \cite{PSPS97}.
% end add by AD

There are three basic ways to affect PSP data quality in the collection
stage: errors of omission, errors of addition, and errors of transcription.
Errors of omission occur when the developer does not record a primary
measure related to defects, time, or the work product itself.  If a defect
occurring during ``Actual Work'' does not appear in the ``Records of Work'', then, for
example, the PSP model of that work product's defect density will be lower
than its actual defect density. If time spent on the work product is not
recorded, then the PSP model of that developer's productivity will be
higher than her actual productivity. Errors of addition occur when the
developer augments the ``Records of Work'' with data not reflecting
actual practice. For example, a developer, having made an error of omission
to the point of having no time or defect data, may recover by simply
inventing enough time and defect entries to make his or her PSP data appear
plausible. Finally, errors of transcription occur when the developer does
intend to record their ``Actual Work'' in the ``Records of Work'' but makes a
mistake during this process.

The presence of collection stage data quality problems is typically
difficult to ascertain and difficult or impossible to rectify. In the
PSP, primary data collection often feels both time consuming and
psychologically disruptive.  Many students complain that stopping to record
defects disrupts their ``flow'' state, and that the time spent recording a
defect---particularly for compilation stage errors---often exceeds the time
spent correcting the defect.  The PSP requires users to learn to constantly
interleave ``doing work'' with ``recording what work you are doing''.

There are also three basic ways to affect PSP data quality in the analysis
stage of manual PSP: errors of omission, errors of calculation, and errors of
transcription.  Errors of omission occur when the developer does not
perform a required analysis of the primary data. Errors of calculation
occur when the developer attempts to perform an analysis but does so
incorrectly. For example, a developer might use a regression-based
estimation method when the historical data is so uncorrelated that this
method is invalid. Finally, errors of transcription occur when the
developer makes a clerical error when moving data from one form to another.

Unlike the collection stage, analysis stage data quality problems are much
easier to ascertain and correct, {\em provided that errors did not occur
  during the collection stage}.  In other words, if one assumes that the
work records accurately reflect the underlying work, then appropriate use
of automated tools can reduce or eliminate analysis errors of
omission, calculation, and transcription.  On the other hand, since the
quality of these analyses are totally dependent upon the quality of the
work records, overall PSP data quality could be quite low even if the
analysis stage is totally automated to eliminate all of its potential data
quality errors.







