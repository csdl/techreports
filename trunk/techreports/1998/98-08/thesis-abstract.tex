%%%%%%%%%%%%%%%%%%%%%%%%%%%%%% -*- Mode: Latex -*- %%%%%%%%%%%%%%%%%%%%%%%%%%%%
%% thesis-abstract.tex -- 
%% Author          : Carleton Moore
%% Created On      : Fri Jun  9 09:43:42 1995
%% Last Modified By: Anne Disney
%% Last Modified On: Mon Aug  3 16:27:27 1998
%% Status          : Unknown
%% RCS: $Id: thesis-abstract.tex,v 1.4 1995/07/07 03:26:19 cmoore Exp $
%%%%%%%%%%%%%%%%%%%%%%%%%%%%%%%%%%%%%%%%%%%%%%%%%%%%%%%%%%%%%%%%%%%%%%%%%%%%%%%
%%   Copyright (C) 1995 University of Hawaii
%%%%%%%%%%%%%%%%%%%%%%%%%%%%%%%%%%%%%%%%%%%%%%%%%%%%%%%%%%%%%%%%%%%%%%%%%%%%%%%
%% 


\abstract{
   
  The Personal Software Process (PSP) is used by software engineers to
  gather and analyze data about their work and to produce empirically
  based evidence for the improvement of planning and quality in future
  projects.  Published studies have suggested that adopting the PSP results
  in improved size and time estimation and in reduced numbers of defects
  found in the compile and test phases of development.  However, personal
  experience using PSP in both industrial and academic settings caused me
  to wonder about the quality of two areas of PSP practice: collection and
  analysis.  To investigate this I built a tool to automate the PSP and
  then examined 89 projects completed by nine subjects using the PSP in an
  educational setting.  I discovered 1539 primary errors and analyzed them
  by type, subtype, severity, and age.  To examine the collection problem
  I looked at the 90 errors that represented impossible combinations of
  data and at other less concrete anomalies in Time Recording Logs and
  Defect Recording Logs.  To examine the analysis problem I developed a
  rule set, corrected the errors as far as possible, and compared the
  original and corrected data.  This resulted in substantial
  differences for numbers such as yield and the cost-performance ratio.
  The results raise questions about the accuracy of published data on the
  PSP and directions for future research.
}










