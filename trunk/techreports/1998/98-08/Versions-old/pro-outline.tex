\documentstyle[nftimes,11pt,/group/csdl/tex/definemargins,
/group/csdl/tex/lmacros]{report} 
\input{/group/csdl/tex/psfig/psfig}

\begin{document}

\title{Evaluating Automated Support for the PSP}
\author{Anne Disney\\
Collaborative Software Development Laboratory,\\
Department of Information and Computer Sciences\\
2565 The Mall\\
University of Hawaii, Manoa\\
Honolulu, Hawaii   96822\\
{\tt anne@uhics.ics.hawaii.edu}}
\maketitle

\setlength{\parindent}{0em}
\setlength{\parskip}{1ex}

\tableofcontents
\chapter{Introduction}
\section{What is PSP}
\subsection{Basic description}
\subsection{Who developed it and when?}
\subsection{Current environments where it is used}
\subsection{How widespread is it?}
\subsection{Why do people want to use it?}
\section{PSP can be difficult to learn}
\subsection{Many forms}
\subsection{Multiple processes, so as user learns PSP, familiar forms change}
\subsection{Complicated computations with higher PSP levels}
\subsection{Instructions in textbook not always clear}
\subsection{Data interdependencies between forms complicated in higher PSP levels}
\section{PSP is time-consuming to use even after learning it}
\subsection{Many forms with many fields to fill out by hand, at the same time as coding}
\subsection{Tedious computations at the end of even minor projects}
\subsection{When computations rely on historical data, finding the data takes time}
\section{Analyzing PSP data is time-consuming}
\subsection{Finding the right data on the right form of each project takes time}
\subsection{Even when using spreadsheets, an extra step is added to each project}
\subsection{Doing even part of the computations by hand is a big job}
\subsection{Users tend to do only a few kinds of data analysis because of the work involved}
\section{Thesis of this research}
\subsection{Automation of the PSP will result in faster and more accurate collection of PSP data.  Analysis of this data will also be faster and users will look at the data 
in more ways.}
\chapter{Other related work}

Most processes for producing high-quality software are designed to be used by groups of
software developers or even organizations.  Examples are 

* CMM - get information from The Capability Maturity Model, Carnegie Mellon University, SEI

  - basic idea

  - how is high quality software produced?

* FTR  - source?

  - basic idea

  - how is high quality software produced?

* Others?


Watts Humphrey's Personal Software Process, however, focuses on the software development
process for an individual software engineer.

* main ideas described in introduction.  Restate?

* The main focus of the PSP is the individual programmer.  However, it can be used as 
Engineeringa tool in group development as well.  (9th conf on Software Engineering Education, p119 (125)

* Like other software processes, PSP requires significant investment of time and effort by the
  software engineer.  However, it directly benefits the person who makes the effort with 
  information about their own process and products. (9th conf on Software  Education,   
  p 52)
 
\chapter{Describe how PSP is done now}
\section{Paper forms}
\section{Mixed group of helping tools for computations and analysis}
\section{Helping tools typically don't communicate}
\chapter{Describe the improvement/change I have in mind}
\section{Describe package}
\chapter{Case study to address important problems}
\chapter{Research plan}
\section{Look at a group of students who have learned PSP}
\subsection{Find out how long it takes them to fill out forms per project}
\subsection{Find out how many mistakes they make in filling out the forms or doing computations}
\subsection{Find out how long it takes them to do data analysis}
\subsection{Find out how many ways they view analyzed data}
\section{Look at a group of students who have learned PSP, and are using an automated version}
\subsection{Find out how long it takes them to fill out forms per project}
\subsection{Find out how many mistakes they make in filling out the forms
or doing computations}
\subsubsection{Defect Counting Procedure}

When an incorrect value was found, a defect was recorded for the field,
but from that point on, the incorrect data value was considered to be 
correct.  For example, consider this sample time recording log: \\ \\

\begin{center}
\begin{tabular}{|l|l|l|r|l|l|}\hline
\multicolumn{6}{|c|}{\bf Time Recording Log}\\ \hline
Date & Start & Stop & Minutes & Phase & Comment \\ \hline\hline
12/01/97 & 10:40 & 11:20 & 30 & plan       & total minutes should be 40 \\ \hline
12/01/97 & 11:20 & 11:30 & 10 & design     & \\ \hline
12/01/97 & 12:30 & 12:50 & 20 & code       & \\ \hline
12/01/97 & 12:50 & 12:55 &  5 & compile    & \\ \hline
12/02/97 & 09:00 & 09:15 & 15 & test       & \\ \hline
12/02/97 & 09:15 & 09:25 & 10 & postmortem & \\ \hline 
\end{tabular}
\end{center}

Since the user incorrectly subtracted {\it stop time} from {\it start time}
when calculating the number of minutes spent in planning, a defect
is recorded for the {\it minutes} field.  But when looking at {\it total actual
minutes} on the Project Plan Summary form, 90 is considered to be the
correct value, even though the time spent in planning was
actually 40 minutes, making the true value of {\it total actual minutes}
100 minutes.

\subsubsection{Defect Types}
I divided defects into eight types:\newline

{\bf Blank Field:} 
A data field that should contain a value, such as {\it start time},
was left blank.  This does not apply to fields where a value is 
optional, such as comment fields.

{\bf Calculation Done Incorrectly:}
Applies to data fields whose value is derived using any
sort of calculation from addition to linear regression.  If the calculation
was not done correctly, a defect wass counted.  This does not apply to
values that are incorrect because other fields used in the calculation
contain bad numbers.

{\bf Entry Error:}
Used when the student clearly did not understand the 
purpose of a field or used an incorrect method in selecting data.  For
example, filling in the {\it defect fixed} field in the Defect Recording
Log with a phase name, or having the defect to-date values in the 
Project Plan Summary originate from a different project than the LOC
to-date values.

{\bf Fields for a More Advanced Process Filled In:}
Included because on the first project, one student filled in values for the
{\it Defects Injected, Plan} and {\it Defects Removed, Plan} fields even
though there was no entry area provided for them on the form and
they are not part of the PSP0 Project Plan Summary. 

{\bf Impossible Values:} Used when two values are mutually exclusive.  For
example: overlapping time log entries, defect fix times for a phase adding
up to more time that the time log entry for the phase, or phases occurring
in the Defect Recording Log in a different order than those in the Time
Recording Log. 

{\bf Process Sequence not Followed:} Used when the Time Recording Log showed a student
moving back and forth between phases instead of sequentially moving through
the phases appropriate for the process.

{\bf Transfer of Data Between Projects Incorrect:}
Used for incorrect values in fields that involve data from a prior
project.  Typically these fields are ``to-date'' fields that involve adding a
to-date value from a prior project with a similar value in the current project.
Unfortunately, it is often impossible to determine in these cases if the
error arose from bringing forward a bad number, incorrectly adding two good
numbers, or bringing forward the correct number and correctly adding it
to the wrong number from the current form.  However, in two important
areas, time and size estimation, the forms were modified so that students
were required to fill in the prior values to be used in the estimation
calculations. In these cases any incorrect values obviously originated in
the transfer.

{\bf Transfer of Data Within Project Incorrect:}
This is the same as the previous defect type, except that it refers to values
being transferred from one form to another within the current project.  For
example, filling in 172 for {\it Estimated New and Changed LOC} on the Size
Estimating Template, but using 190 for {\it Total New and Changed, Plan} on
the Project Plan Summary.\\ \\

\begin{tabular}{|l|r|}\hline
\multicolumn{2}{|c|}{\bf Defects by Type}\\ \hline
Description & \# \\ \hline\hline
Calculation done incorrectly                 & 699 \\ \hline
Blank field                                  & 262 \\ \hline
Transfer of data between projects incorrect  & 175 \\ \hline
Entry error                                  & 146 \\ \hline
Transfer of data within project incorrect    & 100 \\ \hline
Impossible values                            &  88 \\ \hline
Process sequence not followed                &  16 \\ \hline
Fields for a more advanced process filled in &   2 \\ \hline

\end{tabular}


\subsubsection{Defect Severity Levels}
I divided defects into five severity levels:\newline

{\bf Defect has no impact on PSP data:} This includes errors such as
missing header data, incorrect dates in the time recording log, and filling
in fields for a more advanced process.
{\bf Results in a single bad value, single form:} Used if a significant
field which affects no other fields, such as {\it LOC, actual},
is blank or incorrect. 
{\bf Results in multiple bad values, single form:}
{\bf}
{\bf}

\subsection{Find out how long it takes them to do data analysis}
\subsection{Find out how many ways they view analyzed data}
\end{document}











