%%%%%%%%%%%%%%%%%%%%%%%%%%%%%% -*- Mode: Latex -*- %%%%%%%%%%%%%%%%%%%%%%%%%%%%
%% abstract.tex -- 
%% Author          : Carleton Moore
%% Created On      : Wed Dec 30 09:46:12 1998
%% Last Modified By: Carleton Moore
%% Last Modified On: Mon Jan  4 09:49:47 1999
%% RCS: $Id$
%%%%%%%%%%%%%%%%%%%%%%%%%%%%%%%%%%%%%%%%%%%%%%%%%%%%%%%%%%%%%%%%%%%%%%%%%%%%%%%
%%   Copyright (C) 1998 Carleton Moore
%%%%%%%%%%%%%%%%%%%%%%%%%%%%%%%%%%%%%%%%%%%%%%%%%%%%%%%%%%%%%%%%%%%%%%%%%%%%%%%
%% 
\documentclass[12pt]{article}
\RequirePackage{times}

\usepackage{url}

\oddsidemargin 0in   %   Note that \oddsidemargin = \evensidemargin
\evensidemargin 0in
\marginparwidth 0pt
\marginparsep 10pt        % Horizontal space between outer margin and 
\textheight = 8.45in
\textwidth 6.5truein     % Width of text line.
\topmargin 0.0in        %    Nominal distance from top of page to top of
                        %    box containing running head.
\headheight 0pt         %    Height of box containing running head.
\headsep 0pt            %    Space between running head and text.
% for some reason setting topskip to 0 creates a extra blank page in front!
% So we use 1 pt. RSB 9/28/98
\topskip = 1pt          %    '\baselineskip' for first line of page.

\begin{document}
\title{Project LEAP: Personal Process Improvement for the Differently Disciplined}
\author{Carleton A. Moore}
\maketitle

\section*{Research Area}
Process Improvement, Measurement, Personal Software Process
\section*{Problem}

Software developers and managers have faced the problem of producing
quality software since the beginning of the computer age.  Many
people have studied the software quality problem and have proposed
solutions, including better testing, better organizations, better
practices, better project tracking, better programming environments, and
many other factors that potentially affect the development of software.  We
can categorize these different solutions into two groups: (1) solutions
that focus on software development as a group effort and (2) solutions that
focus on the individual software developer.  Some of the many suggestions
that involve groups of software developers include: the Capability Maturity
Model, Clean Room development, software quality assurance groups, and
Formal Technical Review.  These organization level methods help improve the
quality of the software, however they may not be enough.

In the past four years, there has arisen a new focus on the individual software
developer.  In ``A Discipline for Software Engineering''\cite{Humphrey95}
Watts Humphrey introduced the Personal Software Process, also known as PSP.
\begin{quote}
  PSP is a self-improvement process designed to help you control, manage,
  and improve the way you work.  It is a structured framework of forms,
  guidelines, and procedures for developing software.  Properly used, the
  PSP provides the historical data you need to better make and meet
  commitments and it makes the routine elements of your job more
  predictable and more efficient.\cite{Humphrey95}
\end{quote}
PSP, an empirically based process improvement method, focuses on the
individual software engineer.  In PSP, software engineers record the time
they spend programming, the defects they find in their software and the
size of the software.  Based upon these measurements, the engineer can track
their productivity, make better predictions for future projects, gain
insight to what types of errors they make, and learn how to remove defects
earlier in their development process.  The PSP, as described by Humphrey,
is a manual process.  The engineer records, transfers and analyses the data
all on paper forms.  After many projects, the engineer accumulates a large
paper database of their historical data.

After using the PSP for two years, we noticed three general problems.  First, we
started to question the quality of the data recorded.  We noticed that we
did not record all of our defects, in part because the overhead of recording
each defect is too expensive.  Anne Disney and Philip Johnson conducted a
study to look at the data quality of PSP data.  They found that there are
significant data quality issues with manual PSP.\cite{Disney98, Disney98a}

Second, our experiences with industrial partners and management practices
and Robert Austin's book ``Measuring and Managing Performance in
Organizations''\cite{Austin96} made us think about the issues of
measurement dysfunction in PSP and review data.  An organization may
pressure their members to produce ``good'' results.  There are many ways
that the members can manipulate the personal data collected in the PSP to
get the ``right'' results.

Third, after four years, the results with long term adoption of PSP are
mixed.  Pat Ferguson and others report excellent results with PSP adoption
at Advanced Information Services, Motorola and Union Switch and
Signal\cite{Ferguson97}.  However, Barry Shostak and others report poor
adoption of PSP in industry\cite{Shostak96,Emam96}.

These issues started us thinking about designing an automated, empirically
based, personal process improvement tool.  Our goal is to reduce the
collection and analysis overhead for the engineer, and the measurement
dysfunction of the collection process.  This should improve the benefits to
the engineer and the long term adoption of empirically based process
improvement.  To pursue this work, we initiated Project LEAP,
\url{<http://csdl.ics.hawaii.edu/Research/LEAP/LEAP.html>}, and began
developing the Leap Tool Set,
\url{<http://csdl.ics.hawaii.edu/Tools/LEAP/LEAP.html>}.

We designed the Leap Tool Set to automate much of the collection and
analysis of PSP data.  Based upon our design goal to reduce measurement
dysfunction, we allow the user to control who sees what data and change the
data that is shared.  We also reduce the process constraints on the user.
Leap does not require the user to follow a predefined, fixed process.  Leap
also incorporates collaborative review support.  This allows the developer
to gain insight from other developers.  This group input is an important
feature lacking in the PSP.  We made these design decisions in an attempt
to improve the usability and adoption of the Leap Tool Set.



\section*{Research Questions}

We intend to deploy the Leap Tool Set in both academic and industry
settings in order to investigate the following research questions:
\begin{itemize}

\item{What are the strengths and weaknesses of the Leap Tool Set?}
  
\item{What are the strengths and weaknesses of empirically based process
    improvement?}
  
\item{What types of insights can we gain through empirically based
    process improvement?  What types of insights can we not get through
    empirically based process improvement?}

\item{What are the barriers to adoption of the Leap Tool Set}
  
\item{What are the benefits to the users of Leap?}

\item{How do users improve after using Leap?}
  
\item{What are the kinds of improvements we can make using Leap beyond
    improved estimation?}

\item{Is the integration of collaborative review and personal data
    collection appropriate?}
  
\item{Is Leap an appropriate form of automated support for personal process
    improvement?}

\end{itemize}

Based upon these research questions we have developed the following 
testable hypotheses:
\begin{itemize}
\item{Automating the PSP will lead to improved adoption of personal
    software process improvement.}

\item {Reducing the constraints on developers imposed by PSP will lead to
    improved adoption.}
  
\item {Training in a LEAP compliant personal software process improvement
    tool set will lead to adoption of the tool set in professional practice.}
  
\item {The LEAP case studies will provide valuable insights into the strengths
    and weaknesses of this technology.}

\end{itemize}

\section*{Evaluation}

To evaluate these hypotheses we plan on conducting two case studies.  We
will conduct the first case study on senior level undergraduates in Spring,
1999.  The undergraduates will learn about empirically based process
improvement and software engineering processes.  We will train the students
how to use the Leap Tool Set and they will use it to record their software
development of several projects.  This process is similar to the training
of the PSP.  The students will submit reports on their progress and
findings.  We will interview the students to determine their feelings and
attitudes toward process improvement and the Leap Tool Set.  We will also
conduct a survey of their perceptions of the Leap Tool Set and their
environment.  The interviews and the surveys will allow us to predict the
students' adoption of the Leap Tool Set and empirically base process
improvement practices.  A few months after they finish the class we will
contact the students and ask them to fill out a survey.  This survey will
determine if they are still using Leap, and/or any of the empirically base
process improvement concepts in their work.  This survey will allow us to
determine the level of adoption of Leap and empirically based process
improvement, and any barriers to adoption of Leap.

We plan on conducting the second case study in an industrial research group
during Summer, 1999.  We will introduce the Leap Tool Set as automated
support for collaborative review and personal process improvement.  We will
train the members on the use of Leap and help them conduct reviews and
analyses.  We will also train them on personal process improvement.  During
the training we will conduct a survey and interviews to determine their
perceptions of Leap and personal process improvement.  Again the surveys
and interviews will allow us to predict the adoption of Leap.  A few months
after the training we will contact the members and conduct a survey and
interviews.  This survey will determine the level of adoption of Leap, any
barriers to adoption, and the member's attitude toward personal process
improvement.


\section*{Contributions}

We expect the following contributions from this research:
\begin{itemize}
\item{We have made the Leap Tool set available for down loading at
    \url{<http://csdl.ics.hawaii.edu/Tools/LEAP/LEAP.html>}.  Leap is
    implemented in Java and runs on Windows, Unix, and Macintosh.  The Leap
    Tool Set will provide a novel form of automated support for empirically
    based process improvement, including time, defect, size, and pattern
    recording and analysis.  We will soon release the Leap Tool set as an
    Open Source system.}
  
\item{Insight into adoption of personal process improvement.  The results
    of this research will provide new insight into barriers to adoption of
    personal process improvement.  The research will provide new insight
    into why users adopt or refuse to adopt personal process improvement in
    their work.  We designed the Leap Tool Set to overcome some of the
    known barriers to adoption.  If Leap is not adopted, then this suggests
    that these barriers are not key barriers.}
  
\item{Insight into empirical process improvement.  The use of Leap helps us
    learn what improvements we can make using empirical measurement.  The
    case studies may provide insights into the limits of empirically based
    approaches to process improvement.}
  
\item{Insight into process improvement issues.  PSP uses the classic
    waterfall software development model, fixed forms, and a single size
    definition. Leap relaxes many of the constraints that PSP imposes.  The
    case studies should help us learn if these constraints are required for
    process improvement. }

\end{itemize}

%%% Input file for bibliography
\bibliography{/group/csdl/bib/psp}
%% Use this for an alphabetically organized bibliography
\bibliographystyle{plain}

\end{document}