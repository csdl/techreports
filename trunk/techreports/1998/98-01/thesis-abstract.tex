%%%%%%%%%%%%%%%%%%%%%%%%%%%%%% -*- Mode: Latex -*- %%%%%%%%%%%%%%%%%%%%%%%%%%%%
%% thesis-abstract.tex -- 
%% Author          : Carleton Moore
%% Created On      : Fri Jun  9 09:43:42 1995
%% Last Modified By: Jennifer Geis
%% Last Modified On: Fri Mar 13 15:59:38 1998
%% Status          : Unknown
%% RCS: $Id: thesis-abstract.tex,v 1.4 1995/07/07 03:26:19 cmoore Exp $
%%%%%%%%%%%%%%%%%%%%%%%%%%%%%%%%%%%%%%%%%%%%%%%%%%%%%%%%%%%%%%%%%%%%%%%%%%%%%%%
%%   Copyright (C) 1995 University of Hawaii
%%%%%%%%%%%%%%%%%%%%%%%%%%%%%%%%%%%%%%%%%%%%%%%%%%%%%%%%%%%%%%%%%%%%%%%%%%%%%%%
%% 

%for review purposes
%\ls{1}

\abstract{ 
  
  This thesis presents a study designed to investigate the occurrence of
  certain kinds of errors in Java\cite{Grand97} programs using JavaWizard
  (JWiz), a static analysis mechanism for Java source code.  JWiz is an
  extensible tool that supports detection of certain commonly occurring
  semantic errors in Java programs.  For this thesis, I used JWiz within a
  research framework designed to reveal (1) knowledge about the kinds of
  errors made by Java programmers, (2) differences among Java programmers
  in the kinds of errors made, and (3) potential avenues for improvement in
  the design and/or implementation of the Java language or environment.
  
  I performed a four week case study, collecting data from 14 students over
  three programming projects which produced approximately 12,800 lines of
  code.  The JWiz results were categorized into three types: functional
  errors (must be fixed for the program to work properly, maintenance
  errors (program will work, but considered to be bad style), and false
  positives (intended by the developer).  Out of 235 JWiz warnings, there
  were 69 functional errors, 100 maintenance errors, and 66 false
  positives.  The fix times for the functional errors added up to five and
  a half hours, or 7.3 percent of the total amount of time spent debugging
  in test.
  
  I found that all programmers inject a few of the same mistakes into their
  code, but these are only minor, non-defect causing errors.  I found that
  the types of defects injected vary drastically with no correlation to
  program size or developer experience.  I also found that for those
  developers who make some of the mistakes that JWiz is designed for, JWiz
  can be a great help, saving significant amounts of time ordinarily spent
  tracking down defects in test.

}











