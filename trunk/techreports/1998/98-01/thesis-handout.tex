%%%%%%%%%%%%%%%%%%%%%%%%%%%%%% -*- Mode: Latex -*- %%%%%%%%%%%%%%%%%%%%%%%%%%%%
%% thesis-handout.tex -- 
%% Author          : Carleton Moore
%% Created On      : Tue Jan 10 12:04:50 1995
%% Last Modified By: Jennifer Geis
%% Last Modified On: Wed Apr 29 11:28:22 1998
%% Status          : Unknown
%% RCS: $Id: thesis-apendix.tex,v 1.7 1995/07/07 04:09:55 cmoore Exp cmoore $
%%%%%%%%%%%%%%%%%%%%%%%%%%%%%%%%%%%%%%%%%%%%%%%%%%%%%%%%%%%%%%%%%%%%%%%%%%%%%%%
%%   Copyright (C) 1995 University of Hawaii
%%%%%%%%%%%%%%%%%%%%%%%%%%%%%%%%%%%%%%%%%%%%%%%%%%%%%%%%%%%%%%%%%%%%%%%%%%%%%%%
%% 

%for review purposes
%\ls{1}

\appendix
%\newpage
\chapter{ICS311 Handout}
\begin{center}
    JavaWizard\\
    Jennifer Geis\\
    Collaborative Software Development Laboratory\\
    jgeis@uhics.ics.hawaii.edu\\
    Post 307B\\
    956-6920\\
\end{center}

    JavaWizard (jwiz) is an automated code checker for the Java programming
    language.  Basically, jwiz scans through your code and looks for things 
    which will likely cause unintended behavior during run-time.  Jwiz
    assumes your code is syntactically legal.  In other words, your code
    must compile correctly before you can run jwiz on it.  When 
    jwiz finds something that might be an error, it writes out a warning
    message that contains the name of the file, the line number on which
    the error occurred, and a description of the error. 

    An example of the kind of error that jwiz can detect is the use of a
    double-equals sign instead of the 'equals' method to compare two
    strings. Jwiz will give you the warning message like: 

      'FileName.java:43:Comparing two string using '==' instead of
      'equals'.'
      
      There are several versions of JWiz: a gui application, a text
      application, and an applet.

    To run the gui application you must do one of the following:
    \begin{enumerate}
    \item Include '/home/35/csdl/bin' in your path.
    \end{enumerate}
    or, 
    \begin{enumerate}
      \item Copy the file '/home/35/csdl/bin/jwiz' into your home/bin directory.
      \item Make sure your path includes your home/bin directory.
    \end{enumerate}
    or, 
    \begin{enumerate}
    \item Type '/home/35/csdl/bin/jwiztext'
    \end{enumerate}

      To run the text only application (ideal if you are telneting into
      uhunix):
    \begin{enumerate}
    \item Include '/home/35/csdl/bin' in your path.
    \end{enumerate}
    or, 
    \begin{enumerate}
    \item Copy the file '/home/35/csdl/bin/jwiztext' into your home/bin directory.
    \item Make sure your path includes your home/bin directory.
    \end{enumerate}
    or, 
    \begin{enumerate}
    \item Type '/home/35/csdl/bin/jwiztext'
    \end{enumerate}

      To run the applet:
    \begin{enumerate}
    \item Go to 'http://bianca.ics.hawaii.edu/~csdl/applets/JwizHomePage.html
    \item Enter in a url to your file and hit the 'run jwiz' button (due to 
      applet security restrictions, you must enter a url, not a path).
    \end{enumerate}

     Now that you have jwiz available in your environment, to run jwiz,
     type 'jwiz' at the command prompt (if you are using the gui
     application). At this point, a window will appear showing the files in 
     the current directory.

     \begin{figure}[htb] 
       {\centerline{\psfig{figure=images/HandoutFileDialog.ps}}}
       \caption{\label{HandoutFileDialog} FileDialog}
     \end{figure}

      Select the file you want to run Jwiz on and press the OK button.
      You can also start jwiz by giving the file name as an argument: For
      example 'jwiz FileName.java' for a specific file, or 'jwiz *.java' if
      you want to run jwiz on everything in the current directory.

      Jwiz will give you some status messages while it's running, then
      another window will be displayed.  If no warnings were generated, a
      window will be shown with the message "No warnings were generated."
      If jwiz did find anything however, the window will contain the code
      of the file you ran jwiz on, the warnings generated, and a few survey 
      questions.

      \begin{figure}[htb] 
        {\centerline{\psfig{figure=images/HandoutExample2.ps,width=6in}}}
        \caption{\label{HandoutExample2} JWiz Results}
      \end{figure}

      By moving the cursor over the warning message, the line that
      generated the warning will automatically be highlighted.  Go through
      the list and verify which are valid and which
      are not.  Since jwiz looks for things which are usually errors but not
      always, sometimes it flags things that you really meant to do.

      Each warning is assumed to be valid and so is selected by default,
      if a warning is not valid, please deselect it before quitting the
      program.

      For the purpose of this experiment, you will be
      required to fill out the survey data each time you run
      the program. Since your answers about programming
      experience do not change, JWiz writes out a file to
      the current directory with these answers and uses them
      as defaults the next time you run the
      program. However, there is one survey question you
      need to check each time you run JWiz, since it could
      change between executions of JWiz. This is the "development phase"
      question. For example, say you run JWiz for the
      first time on a program right after it compiles
      correctly and before you've actually run it on some
      data to test it.  At this point, you'd mark the
      "Development Phase" as pre-test.  Then, if you used
      JWiz again after running your program a few times to
      test it on some data, you would mark the Development
      Phase as "post-test".  You would mark the phase as
      "Post-test" for this and all future runs of JWiz even
      though you might not yet be done with testing.

      The survey questions are:

      \begin{enumerate}
      \item Programming experience:  I would like you to indicate how long
      you have been programming (any languages), less than six months,
      between six months and two years, or more than two years.
      \item Java programming experience: How long you have been programming
      with Java, less than six months, between six months and two years, or
      more than two years.
      \item Development phase: This is the phase of development you were in
      when you ran jwiz.  "Pre-Test" means that you are
      running JWiz after the code compiles cleanly for the
      first time, but
      before you have started testing it by running the
      program on some data.  "Post-Test" means that you are
      running JWiz after having started to test your program 
      by running it on data.  Once you've run your program
      on some data for the first time, you will always
      indicate the phase as "Post-Test", even if you need to 
      change and recompile your program again.
      \item Number of languages:  Please indicate the number of programming
      languages that you have experience with. One to two, three to four,
      five to six, or more than six languages.
    \end{enumerate}

    The buttons at the bottom of the window are as follows:

      Make HTML: Writes out whatever warnings you have indicated to be
      valid in HTML format.

      Comments: Pops up a window through which you can send me any
      questions, comments, or bug reports. 

      OK: Sends me the list of valid/invalid warnings, a
      copy of the source (if the "Include source in
      submission" checkbox was checked), and the survey
      data.  In the interest of privacy, I will not report
      any details identifying you, your program, or your
      errors.  I request your email address solely in order
      to contact you if there are questions about your
      data. 


\normalsize

\newpage


