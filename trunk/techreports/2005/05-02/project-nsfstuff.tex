%%%%%%%%%%%%%%%%%%%%%%%%%%%%%% -*- Mode: Latex -*- %%%%%%%%%%%%%%%%%%%%%%%%%%%%
%% project-nsfstuff.tex -- 
%% Author          : Philip Johnson
%% Created On      : Thu Oct  4 08:05:31 2001
%% Last Modified By: Philip M. Johnson
%% Last Modified On: Mon May 16 12:01:54 2005
%% RCS: $Id$
%%%%%%%%%%%%%%%%%%%%%%%%%%%%%%%%%%%%%%%%%%%%%%%%%%%%%%%%%%%%%%%%%%%%%%%%%%%%%%%
%%   Copyright (C) 2001 Philip Johnson
%%%%%%%%%%%%%%%%%%%%%%%%%%%%%%%%%%%%%%%%%%%%%%%%%%%%%%%%%%%%%%%%%%%%%%%%%%%%%%%
%% 
\newpage
\section{Excerpts from the NSF Solicitation}

We need to both ensure
that we are achieving the goals of this solicitation, as well as clearly
addressing the standard NSF review criteria.  To help us get there, the
next two sections provide copies of text from the solicitation that
we will need to keep in mind as we proceed.

\subsection{Solicitation Goals}
\label{sec:solicitation}

Some of the solicitation goals include:
\begin{enumerate}

\item The development of tools that facilitate the integration of
qualitative and quantitative information from heterogeneous sources,
multiple media, and/or multiple modes;

\item Investment in basic research that addresses the protection of the
confidentiality of respondents in computerized, widely accessible
databases; and

\item The development of incentives, standards and policies for collecting,
storing, archiving, accessing, and publishing research results using
organization-relevant information.

\item Testbed I. information collected on organizations from a variety of
heterogeneous, independently developed data sources, such as administrative
and survey data, temporal, spatial and image data or textual data. The goal
is to free users from having to locate the data sources, interact with each
data source in isolation, and manually combine data from multiple formats
and multiple sources. This could be achieved through the creation of new
and more accurate and efficient ways to collect, code and analyze
qualitative information from case studies, and other sources, and to enable
the linking of this information with repositories of quantitative data,
while protecting fundamental privacy and confidentiality concerns. The
research should be designed to show how appropriate cybertools can lead to
multiple advances in the empirical understanding of how organizations
emerge, develop, thrive or weaken.

\item Proposals must address the protection of data providers from
identification, exploitation, and other misuses of personal or
organizational information. Such misuses present a perpetual challenge to
the melding of data and media of different types in a tool for widespread
use. Proposals in response to this solicitation must show a sophisticated
understanding of this sociotechnical problem and must propose to advance
fundamental knowledge of effective privacy protections during the
development of the analytical tools and in their later use by various
research communities.

\item Proposals must demonstrate potential long-term sustainability,
usability, and impact. This could be achieved for the organizational
"testbed", for example, by documenting proposed collaboration with firms in
an industry, attracting support from foundations or developing replicable
incentive-compatible policies for collecting, storing, accessing, and
disseminating data while continuing to utilize and advance relevant
cybertechnology.

\item Unifying Data Models and System Descriptions: There is a need to
develop stronger theoretical foundations for the representation and
integration of information of various types from extant data models (e.g.,
temporal, spatial and image data, textual data, administrative and survey
data) as well as the scientific literature into conceptually coherent
views.

\item Reconciling heterogeneous formats schemas and ontologies: The
fundamental problem in any data sharing application is that systems are
heterogeneous in many different aspects, such as different ways of
representing data and/or knowledge about the world, different
representation mechanisms (e.g., relational databases, legacy systems, XML
schemas, ontologies), different access methods and policies. In order to
share data among heterogeneous sources, approaches to form a semantic
mapping of their respective representations are needed to avoid manual
intervention in each step of converting and merging data resources.

\item Web semantics: Data on the web needs to be defined and linked in a
way that it can be used by machines not just for display purposes, but also
for automation, integration and reuse of data across various
applications. Supported research topics will include frameworks for
describing resources, methods of automating inferences about web data and
resources, and the development of interoperable ontologies, mark up
languages and representations for specific social, behavioral and other
scientific domains.

\item Decentralized data-sharing: Traditional data integration systems use
a centralized mediation approach, in which a centralized mediator,
employing a mediated schema, accepts user queries and reformulates them
over the schemas of the different sources. However, mediated schemas are
often hard to agree upon, construct and maintain. For example, researchers
conducting social and behavioral research share their experimental results
with each other, but may do it in an ad hoc fashion. A similar scenario is
found in data sharing among government agencies. Architectures and
protocols that enable large-scale sharing of data with no central control
are needed.

\item On-the-fly integration: Currently, data integration systems rely on
relatively static configurations with a set of long-lived data
sources. On-the-fly integration refers to scenarios where one wants to
integrate data from a source immediately after discovering it. The
challenge is to significantly reduce the time and skill needed to integrate
data sources so that scientists can focus on domain problems instead of
information technology challenges.

\end{enumerate}

\subsection{NSF Review Guidelines}

The generic ones are:

\begin{enumerate}
\item What is the intellectual merit of the proposed activity?  How
important is the proposed activity to advancing knowledge and understanding
within its own field or across different fields? How well qualified is the
proposer (individual or team) to conduct the project? (If appropriate, the
reviewer will comment on the quality of the prior work.) To what extent
does the proposed activity suggest and explore creative and original
concepts? How well conceived and organized is the proposed activity? Is
there sufficient access to resources?

\item What are the broader impacts of the proposed activity?  How well does
the activity advance discovery and understanding while promoting teaching,
training, and learning? How well does the proposed activity broaden the
participation of underrepresented groups (e.g., gender, ethnicity,
disability, geographic, etc.)? To what extent will it enhance the
infrastructure for research and education, such as facilities,
instrumentation, networks, and partnerships? Will the results be
disseminated broadly to enhance scientific and technological understanding?
What may be the benefits of the proposed activity to society?

\item Integration of Research and Education One of the principal strategies
in support of NSF's goals is to foster integration of research and
education through the programs, projects, and activities it supports at
academic and research institutions. These institutions provide abundant
opportunities where individuals may concurrently assume responsibilities as
researchers, educators, and students and where all can engage in joint
efforts that infuse education with the excitement of discovery and enrich
research through the diversity of learning perspectives.

\item Integrating Diversity into NSF Programs, Projects, and Activities
Broadening opportunities and enabling the participation of all citizens --
women and men, underrepresented minorities, and persons with disabilities
-- is essential to the health and vitality of science and engineering. NSF
is committed to this principle of diversity and deems it central to the
programs, projects, and activities it considers and supports.

\end{enumerate}

There are several final review criteria specific to this solicitation:

\begin{enumerate}

\item  Possession of the scientific expertise and resources needed for tool development.
\item Possession of the scientific expertise and resources needed for the creation and analysis of databases on organizations and individuals.
\item  Cohesion of technology, tools and data within each "testbed".
\item  Documented outreach and dissemination plan.
\item Evidence of applicability to a broad range of sciences.
\item Quality of coordination plan.
\item Demonstration of scalability to, for example, additional organizations or other large-scale databases.
\item Evidence of long-term sustainability and impact.

\end{enumerate}














