%%%%%%%%%%%%%%%%%%%%%%%%%%%%%% -*- Mode: Latex -*- %%%%%%%%%%%%%%%%%%%%%%%%%%%%
%% project-contributions.tex -- 
%% Author          : Philip Johnson
%% Created On      : Thu Oct  4 08:05:31 2001
%% Last Modified By: Philip M. Johnson
%% Last Modified On: Wed May 25 13:15:26 2005
%% RCS: $Id$
%%%%%%%%%%%%%%%%%%%%%%%%%%%%%%%%%%%%%%%%%%%%%%%%%%%%%%%%%%%%%%%%%%%%%%%%%%%%%%%
%%   Copyright (C) 2001 Philip Johnson
%%%%%%%%%%%%%%%%%%%%%%%%%%%%%%%%%%%%%%%%%%%%%%%%%%%%%%%%%%%%%%%%%%%%%%%%%%%%%%%
%% 
\section{Conclusions}

We believe the Cedar project has substantial intellectual merit: it brings
together not only qualitative and quantitative data, but also researchers
from multiple disciplines to synthesize their knowledge and capabilities to
produce a system with unique capabilities.  The four Principle
Investigators in this project bring a diverse, but complementary set of
skills regarding qualitative and quantitative data collection and analysis;
software development; and empirical data repository management. Our
application of narrative and network theories to integrating empirical data
is both novel and promising, and our prior development of the Hackystat system 
enables us to jump-start the Cedar implementation.

We have designed the Cedar project with the intention that it be broadly
applicable as a tool for collecting, analyzing, and disseminating
qualitative and quantitative data.  As the University of Hawaii is a
university with 75\% minority students in an EPSCOR state, this project
will provide novel research opportunities to underrepresented groups. 
The development of curriculum materials, classroom evaluation, case studies 
in HPCS, and technology transfer through the Cedar Consortium will all result
in enhancements to infrastructure for research and education. 

We would like to conclude by noting that the ability to gather and access
data of this sort brings with it a duty to develop ways of interpreting
these data responsibly.  We have already pointed out that existing data
bases have been subject to misuse and misrepresentation.  People who are
marginal in society are particularly vulnerable to misinterpretation of
their actions.  Poor people, for instance, find a variety of ways of coping
with the lack of money that may make them look like irresponsible parents
or even criminals when neither may be the case \cite{Crampton01, Edin91,
Stack74}.  

%% As the use of data bases become even more widespread there will be more
%% times when in the course of normal events, people are asked to abstract
%% from a series of actions to a pattern that has a particular meaning.  The
%% ability to generalize from a series of actions to either an intent or a
%% predicted outcome is fraught with difficulty and must be tempered by a
%% connection to the context in which the person is taking the actions.  This
%% is particularly true when people are attempting to abstract across cultural
%% lines as the same action may have different meanings in different cultural
%% contexts.  As we have the ability to keep track of more and more things
%% (which is the intent of this proposal), being able to tell when action
%% should be taken and what actions should be taken will be more and more
%% important.

%% No matter how fine-grained and complete the data are, there are always
%% multiple ways in which these data add up to something that has meaning.
%% For instance, our credit card companies are already very good at taking the
%% specific purchases on an existing credit card and generalizing to "someone
%% has stolen this credit card number".  They are not always right, however,
%% and sometimes a person has just gone on a shopping spree.  Moreover, even
%% if the purchases don't fit the normal pattern and even if it turns out that
%% there is someone else using the credit card, it could be the spouse of the
%% credit card holder (whose last name, of course, may not be the same as the
%% credit card holder) and the use may be entirely legitimate.  Credit card
%% companies have learned to contact the customer to find out what is going
%% on.

%% Not only are there multiple ways to interpret anything, but there are also
%% multiple things happening at the same time. A waiter often serves more than
%% one table at a time.  Thus, when the waiter goes to the kitchen to pick up
%% a salad for one table, he may also pick up the desert for another table.
%% What may seem to be one action (going to the kitchen to pick up an order)
%% is actually two actions in two active performances. Pentland observed the
%% same thing in software support hot lines, where individual support
%% engineers could have as many as 60 open calls in their queue at one time
%% \cite{Pentland92}.  Identifying a solution for one customer may solve a
%% similar problem for other customers.  Such concurrency is an every day
%% event in most organizations.  Hiring routines overlap with training
%% routines or with budgeting routines.

An important goal of this research is to help people
incorporate, rather than bypass, context so that interpretations are
smarter.  Cedar will enable data analysts to identify some of the multiple
stories (or at least be aware of the multiple stories) and think about what
questions they need to ask and who they need to ask them of in order to
sort through which stories are more likely than others.  We view this
project as an opportunity not only to be more precise in the data we are
gathering but also as a way to incorporate the intrinsic diversity of
meanings in any set of actions.  If we succeed, this project will allow us
to make the complexity of life more accessible rather than to obscure it.


