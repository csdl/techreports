%%%%%%%%%%%%%%%%%%%%%%%%%%%%%% -*- Mode: Latex -*- %%%%%%%%%%%%%%%%%%%%%%%%%%%%
%% summary.tex -- 
%% RCS:            : $Id: nsf93-summary.tex,v 1.11 93/10/06 16:52:35 johnson Exp $
%% Author          : Philip Johnson
%% Created On      : Wed Aug 11 12:55:46 1993
%% Last Modified By: Philip M. Johnson
%% Last Modified On: Tue May 24 09:23:33 2005
%% Status          : Unknown
%%%%%%%%%%%%%%%%%%%%%%%%%%%%%%%%%%%%%%%%%%%%%%%%%%%%%%%%%%%%%%%%%%%%%%%%%%%%%%%
%%   Copyright (C) 1993 University of Hawaii
%%%%%%%%%%%%%%%%%%%%%%%%%%%%%%%%%%%%%%%%%%%%%%%%%%%%%%%%%%%%%%%%%%%%%%%%%%%%%%%
%% 
%% History
%% 11-Aug-1993          Philip Johnson  
%%    

\documentclass[11pt]{article} 
\usepackage{/export/home/csdl/tex/icse2003/latex8}
\usepackage{times}

\pagestyle{empty}

\begin{document}
\section*{Project Summary}

In this research, we propose to design, implement, and evaluate Cedar: a
CyberInfrastructure for Empirical Data Analysis and Reuse, to satisfy the
requirements for Testbed I.  Cedar is intended to be an open source
information infrastructure architecture coupled with a data management
policy mechanism that supports scalable and collaborative, qualitative and
quantitative organizational research data collection, analysis,
dissemination, and archiving. Our project involves the following components:

(1) {\em Infrastructure technology research and development.}  Through the
Hackystat Project, Principle Investigator (PI) Johnson has
developed expertise in the development of open source
collaborative systems for collection and analysis of quantitative data for
software engineeering research and experimentation.  The Hackystat system
and experiences provide a base for extension into qualitative data
collection and analysis, as well as to a peer-to-peer network of federated
servers.

(2) {\em Research on and development of policies and procedures for data
  privacy and dissemination.} PI Basili is leading a task force of
software researchers with experience in developing and maintaining software
engineering empirical data repositories with the goal of articulating prior 
problems and proposing improvements for management of future repositories. 
We will leverage this initial research and incorporate related research in 
privacy policies and technologies for integration into the Cedar infrastructure. 

(3) {\em Research on and development of models and mechanisms for
representation and integration of qualitative and quantitative
information.}  PI Pentland and PI Feldman have carried out a variety of
research on the theoretical underpinnings of qualitative and quantitative
empirical data and its appropriate interpretation.  Cedar will leverage
these insights with technological infrastructure for collection, analysis,
and dissemination of empirical data according to narrative and network theories 
for representation and analysis of qualitative and quantitative data. 

(4) {\em Case study evaluation of Cedar.}  The four PIs (Johnson, Basili,
Pentland, Feldman) have substantial prior experience in the design and
implementation of case studies across a variety of application domains and
organizational types. To test the validity of Cedar, and to understand its
strengths and limitations, we will perform a case study with selected
organizations involved in the DARPA HPCS program.  As part of its
evaluation, we will also develop Cedar-based curriculum materials to
support education and technology transfer.

The intellectual merit of this research includes the application of novel
data gathering and analysis techniques for the collection, integration,
analysis, and dissemination of qualitative and quantitative data, and the
application of this framework to a real-world organization and resulting
evaluation.

The broader impact of this research includes the development of a
sophisticated, freely available, open source software system for use by
scientists for collection, analysis, integration, and dissemination of
qualitative and quantitative data, along with associated curriculum
materials to support education and technology transfer.  As the University
of Hawaii is a university with 75\% minority students in an EPSCOR state,
this project will provide novel research opportunities to underrepresented
groups.

\end{document}

 







