%%%%%%%%%%%%%%%%%%%%%%%%%%%%%% -*- Mode: Latex -*- %%%%%%%%%%%%%%%%%%%%%%%%%%%%
%% 04-18.tex -- Improving review process quality with Jupiter-Hackystat paradigm
%% Author          : Takuya Yamashita
%% Created On      : Mon Sep 23 11:52:28 2002
%% Last Modified By: Takuya Yamashita
%% Last Modified On: Fri Sep 24 16:22:32 2004
%% RCS: $Id$
%%%%%%%%%%%%%%%%%%%%%%%%%%%%%%%%%%%%%%%%%%%%%%%%%%%%%%%%%%%%%%%%%%%%%%%%%%%%%%%
%%   Copyright (C) 2002 Takuya Yamashita
%%%%%%%%%%%%%%%%%%%%%%%%%%%%%%%%%%%%%%%%%%%%%%%%%%%%%%%%%%%%%%%%%%%%%%%%%%%%%%%
%%

\chapter{Experiment Evaluation}
\label{sec:evaluation}

In order to evaluate if the automated review framework provides the
better understanding of inspection process as a result of the fact
that developers continuously use the lightweight functional tool
rather than text editor based tool, I propose the two hypotheses:

\begin{enumerate}
\item Claim 1: Jupiter provides the useful functionalities enough to
adopt the lightweight code review tool rather than the text editor
tool.

\item Claim 2: Jupiter and Hackystat automated review process provides
better understanding of inspection process.
\end{enumerate}

To test the hypotheses, I propose two stages to evaluate claim 1 and
2:  1) evaluation of the usefulness of Jupiter review tool and 2)
evaluation of better understanding of the automated review process
with Jupiter and Hackystat. The second evaluation is supposed to be
conducted after the significant result for the evaluation of Jupiter
review tool.

\subsection{Experiment Design for Jupiter review tool}

To evaluate the usefulness of Jupiter plug-in for Eclipse (Claim 1),
I propose that the evaluation is considered to be based upon
qualitative approach: Questionnaire. The questionnaire will be
conducted before and after the use of Jupiter.

\subsubsection{Subjects}

The experiment will be carried out during the spring semester in
2004 from January to May as a part of both introductory Software
Engineering course for approximately 40 undergraduates and graduates
in University of Hawaii at Manoa's information and Computer Sciences
department. It is assumed that the subjects has the background of
the required computer science such as programming language,
algorithm. However, there will be a room that different students
have different ability of skills. To minimize the deficiency of
skills, students are supposed to review the basic knowledge which
would be necessary to evaluate the claim 1. That is to say, the
basic knowledge of Java language, Text editor, Eclipse IDE, and so
forth will be reviewed. All students are supposed to have the
identical educational tutorial on how to use the text editor in
Eclipse IDE before the post questionnaire for Text editor is
conducted. They are also supposed to have the identical educational
tutorial on how to use the Jupiter after the text editor review
process is mastered and before the post questionnaire for the
Jupiter editor is conducted.

\subsubsection{Materials}

The material to be used for review is the code actually students
wrote in Java during the course. For each session of review
including preparation, individual, and rework phase, the same
material is used for all students. The appropriate material to be
used in each session will be determined by the course instructor
and/or me in such a way that it contains if 1) there are proper
lines of code, 2) there exist enough defects, and 3) there are
enough parts that educate students. I will also prepare for each
session the backup materials to have all of them just in case that
there exist no materials as such. It is considered that this
approach would be better way than the approach that all materials
are prepared before course starts. It is because 1) Instructor has
own plan to organize his class and 2) the source codes students
wrote are more useful for them to understand and recognize the
defects.

\subsubsection{Instruments}

The main instrument is Eclipse IDE, which is the integrated software
development tool, and Jupiter, which is the code review plug-in to
the Eclipse. As student experience software engineering course, some
software engineering tools such as unit test framework, ant build
too, configuration management tool, and web application tool are
introduced step by step.

To eliminate the external factor of usability of editor, Text editor
embedded in Eclipse is used for the text based review. So students
are supposed to experience both text editor review and Jupiter based
review within Eclipse IDE.

To see the estimated review time in individual phase, Jupiter sensor
for Hackystat is used. This sensor records the review time without
any special action such as writing start time and end time or even
pushing start and end button in the tool. Text editor used is a
special editor for review, whose most function is the same as
regular Eclipse editor, but it can be opened by Jupiter so that the
review time for text editor and Jupiter is recorded.


\subsubsection{Experiment Execution}

Before the experiment starts, subjects are given lecture on how to
conduct review process. The lecture includes the explanation of the
research purpose, review process, review iteration period, review
tool, and so forth.

To iterate though review sessions, the first round is set as three
review sessions. The three sessions include two general review
process and one group review process. A general review process is
determined as the process that one review material would be reviewed
in individual phase and team phase by all students. On the contrary,
an individual group review process is determined as the process that
each small groups has own individual materials to be reviewed in the
individual, team and rework phases. The reason why there are two
different type of review process is that only individual review
phase can conduct the rework phase since the author of the materials
would be the rework person. A session is determined as the process
that includes individual phase, team phase, and rework phase.  The
small groups conduct the first round with the text editor in
Eclipse.

In individual phase, students are given a source code and supposed
to review for just an hour. They are supposed to finish the review
regardless of any reason. Jupiter review sensor helps seeing the
review time for students. If there is a significant time difference
from the time period, students are asked for the reason, and the
problem would be solved in the next iteration. After the review
done, the review files for students are emailed to Instructor or me.

In team phase, students conduct the team review with the small team
members. Instructor or I measure one hour with physical timer in the
class. They review all raised issues and validate issues as much as
time is allowed. They can the validate issues by setting the
resolution status such as "Vaid-NeedsFixing", "Valid-FixedLater",
"Valid-WontFix", and so forth. They are supposed to finish the team
review in an hour regardless of any reason and send the review file
to Instructor or me.

In rework phase, students conduct the rework review only with the
individual group review process. The time spent for the rework phase
is not calculated because it would be hard to measure the time for
rework phase. However, students are asked to fix the all confirmed
review issues in a specific period of time. This phase is especially
evaluated by the questionnaire.

The review process including as individual, team, and rework phase
is the same for the text editor based review and Jupiter based
review.

To evaluate the distinct usefulness of Jupiter compared to text
editor in Eclipse, I will ask students to have questionnaire after
both tool are used. The first questionnaire will be conducted after
the first round with the Text editor is done. This would be the
evaluation for the usefulness of the Text editor and its review. The
second questionnaire will be conducted after the second round with
Jupiter is done. This would be the evaluation for the usefulness of
the Jupiter and its review.


\subsection{Experiment Design for Automated Review System}
