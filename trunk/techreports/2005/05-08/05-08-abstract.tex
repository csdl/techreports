%%%%%%%%%%%%%%%%%%%%%%%%%%%%%% -*- Mode: Latex -*- %%%%%%%%%%%%%%%%%%%%%%%%%%%%
%% 04-18.tex -- Improving review process quality with Jupiter-Hackystat paradigm
%% Author          : Takuya Yamashita
%% Created On      : Mon Sep 23 11:52:28 2002
%% Last Modified By: Takuya Yamashita
%% Last Modified On: Fri Sep 24 16:22:32 2004
%% RCS: $Id$
%%%%%%%%%%%%%%%%%%%%%%%%%%%%%%%%%%%%%%%%%%%%%%%%%%%%%%%%%%%%%%%%%%%%%%%%%%%%%%%
%%   Copyright (C) 2002 Takuya Yamashita
%%%%%%%%%%%%%%%%%%%%%%%%%%%%%%%%%%%%%%%%%%%%%%%%%%%%%%%%%%%%%%%%%%%%%%%%%%%%%%%
%%

\begin{abstract}  % 200 words
Over many years, there is general agreement that software inspection
reduces development costs and improves product quality by finding
defects in early software development, and these software tools help
the inspection process efficiently and effectively. Even though the
agreement, developers have not uses one review tool for a long time.
Or they even have gave up the functional tool any more and went back
to use extremely lightweight text editor tool.

By using Jupiter (review tool) and Hackystat (automated metric
collection system), I proposed that the issue is addressed if the
automated review framework provides the better understanding of
inspection process and as a result, helps developers to continuously
use the lightweight functional tool rather than text editor based
tool. By providing the transparent inspection process and tool,
reviewers and team can be aware of the efficiency use of inspection
tool.

To investigate my research questions, I will set a controlled review
experiment in software engineering class to give Java based
assignments to around 20 students. In first several round, students
are encouraged to learn both text editor based and Jupiter based
review. Qualitative evaluation will be conducted before and after
the round to see the usefulness of review tool. In the next several
round, Hackystat metric collection system will be introduced to
gather review metrics. It could provide the review analysis such as
prepared reviewers, categorized defect types and severity, number of
confirmed issues, and so forth. The second qualitative evaluation
will be conducted before and after the introduction to see the
usefulness of automated review framework.

The expected results would be that automated review framework,
including the review tool, provides some useful aspects for
inspection process and tool. This result would be one milestone for
empirical software inspection community.

Finally, this thesis is going to be done by not later than May, 2005
with first milestone for the implementation of analysis tool by
January 2005, second milestone for the evaluation of the review tool
by March 2005, and final milestone for the evaluation of the
automated review system by May 2005.
\end{abstract}
