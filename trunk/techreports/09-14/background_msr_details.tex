\section{Mining software repositories}\label{evolution.discovery}
According to Kagdi et al. \cite{citeulike:4534888} the term \textit{mining
software repositories (MSR)} ``... has been coined to describe a broad class of
investigations into the examination of software repositories.'' The ``software
repositories'' here refer to various sources containing artifacts produced by
software process. Examples of such sources are version-control systems (CVS,
SVN, etc.), requirements/change/bug control systems (Bugzilla, Trac etc.),
mailing lists archives and social networks. These repositories have different
purposes but they support a single goal - a software change which is the single
unit of the software evolution. 

In the literature, \textit{software change} defined as an addition, deletion or
modification of any software artifact such as requirement, design document, test
case, function in the source code, etc. Typically, software change is realized
as the source code modification; and while version control system keeps track of
actual source code changes, other repositories track various artifacts (called
\textit{metadata}) about these changes: a description of a rationale behind a
change, tracking number assigned to a change, assignment to a particular
developer, communications among developers about a change, etc.

Researchers mine this wealth of data from repositories in order to extract
relevant information and discover relationships about a particular evolutionary
characteristic. For example, one may be interested in the growth of a system
during each change, or reuse of components from version to version. In this
section I will review some MSR research literature which is relevant to my
research and based on the mining of temporal patterns from SCM audit trails.

\subsection{Mining evolutionary coupling and changes}
One of the approaches in MSR mining relevant to my research is built upon mining
of the simultaneous changes occurring in software evolution. This type of mining
considers changes in the code within a short time-window interval which occur
recurrently. Such changes are revealing logical coupling within the code which
can not be captured by the static code analysis tools. This knowledge allows
researcher and analysts predict the required effort and impact of changes with a
higher precision. 

Mining of evolutionary coupling is typically performed on different levels of
code abstraction: Zimmermann et al. in \cite{citeulike:4406375} discuss mining
of version archives on the level of the lines of source-code using annotation
graphs; Ying et al. in \cite{citeulike:983796} discuss mining of version
archives for \textit{co-change} patterns among files by employing association
rule mining algorithm, and refining results by introducing
\textit{interestingness} measure, which based on the call and usage patterns
along with inheritance; Gall et al. in \cite{citeulike:5397994} use a
window-based heuristics on CVS logs for uncovering logical couplings and change
patterns on the module/package level. Kim et al. in \cite{citeulike:5375867}
taking a different approach by mining \textit{function signature change} and
introducing kinds of signature changes and its metrics in order to understand
and predict future evolution patterns and aid software evolution analysis.

The fine-grain mining of changes on the level of lines of source code is usually
implemented with the use of \textit{diff} utilities family which report
differences between versions of the same file. For capturing temporal properties
the sliding-window approach is used if mining CVS logs, while Subversion is able
to report co-changed filesets (\textit{change-sets}). Use of the information
extracted by parsing issue/bug tracking logs and developer comments from version
control logs allows to capture co-occurring changes with higher precision.

What is common among all this work is that while researchers use different
sources and abstraction levels of information, they are extracting only the
relevant to a specific question data (using filters and taxonomy mappings) and
compose data sets suitable for KDD algorithms. In order to refine and classify
(prune) reported results, various support functions proposed.

The main contribution of this type of mining is in the discovery of patterns in
software changes which are improving our  understanding of the software and
allowing estimation of effort and impact of new changes with higher precision.

\subsection{Ordered change patterns}
A step ahead in the analysis of co-occurring changes in source code entities was
shown by Kagdi et al. in \cite{citeulike:3929070}. The authors investigated a
problem of mining ordered sequences of changed files from change-sets. Six
heuristics (\textit{Day, Author, File, Author-date, Author-file, and Day-file})
based on the version control transaction properties were developed and
implemented. Abstracted sequences were mined with Apriori algorithm (see
\ref{apriori}) discovering recurrent sequential patterns. The authors proposed a
higher specificity and effectiveness of such approach to software change
prediction than by using convenient (un-ordered) change patterns mining.

\subsection{Usage patterns}
Another interesting approach for MSR, relevant to my work, is the mining of
usage patterns proposed by Livshits \& Zimmermann in \cite{citeulike:5398684}.
In this work, the authors approach a problem of finding violations of
application-specific coding rules which are ultimately responsible for a number
of errors. They designed approach to find ``surprise patterns'' (see Subsection
\ref{tpatterns}) of the API and function usage in SCM audit trail by
implementing a preprocessing of the functional calls and mining aggregated data
with a customized Apriori algorithm (see \ref{apriori}) implementation. By
considering past changes and bug fixes, authors were able to classify patterns
into three categories: \textit{valid patterns}, \textit{likely error} patterns,
and \textit{unlikely} patterns. Candidate patterns found with Apriori algorithms
were considered to be a valid pattern if they were found a specified number of
times and an unlikely patterns otherwise. Similarly, if a previously labeled as
valid pattern was later violated a certain number of times, it was considered as
an error pattern. The authors validated their approach on mining publicly
available repositories effectively reporting error patterns.
