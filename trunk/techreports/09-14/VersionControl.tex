\documentclass[12pt,oneside]{report}

%%% Load some useful packages:
%% "New" LaTeX2e graphics support.
\usepackage{graphicx}
%%	using final option to force graphics to be included even in draft mode
%\usepackage[final]{graphicx}
%% Tell graphicx the default directory for all figures
\graphicspath{{figures/}}

% Enable subfigure support
\usepackage{subfigure}

%% Make subsubsections numbered and included in ToC
\setcounter{secnumdepth}{3}
\setcounter{tocdepth}{2}

%% Package to linebreak URLs in a sane manner.
\usepackage{url}

%% Define a new 'smallurl' style for the package that will use a smaller font.
\makeatletter
\def\url@smallurlstyle{%
  \@ifundefined{selectfont}{\def\UrlFont{\sf}}{\def\UrlFont{\small\ttfamily}}}
\makeatother
%% Now actually use the newly defined style.
\urlstyle{smallurl}

%% Define 'tinyurl' style for even smaller URLs (such as in tables)
\makeatletter
\def\url@tinyurlstyle{%
  \@ifundefined{selectfont}{\def\UrlFont{\sf}}{\def\UrlFont{\scriptsize\ttfamily}}}
\makeatother

%% Provides additional functionality for tabular environments
\usepackage{array}

%% Puts space after macros, unless followed by punctuation
\usepackage{xspace}

%% Make margins less ridiculous
\usepackage{fullpage}

%% Allows insertion of fixme notes for future work
\usepackage[footnote, nomargin]{fixme}

%%%% Turned off for tech report, should be turned on for research portfolio
%% Turn on double spacing
\usepackage{setspace}
\usepackage{mdwlist}
\doublespacing

%% Make URLs clickable
%\usepackage[colorlinks, bookmarks=false]{hyperref}
\usepackage[colorlinks, bookmarks=true]{hyperref}

%% Since I'm using the LaTeX Makefile that uses dvips, I need this
%% package to make URLs break nicely
\usepackage{breakurl}

\usepackage{amsmath,amsfonts}
\numberwithin{equation}{subsection}
%%\usepackage{nonfloat}
\usepackage{bbm}
\usepackage{setspace}
\onehalfspacing
\usepackage{tabularx}

%%% End of preamble
\begin{document}

\begin{titlepage}

\begin{center}

\tiny{.}\\ [1.2cm]

\Large{Dissertation draft:} \\ [0.5cm]
\LARGE{\textsc{Software Trajectory Analysis:}} \\
\LARGE{\textsc{An empirically based method for automated software process discovery}} \\ [0.6cm]

\large{
Pavel Senin \\
Collaborative Software Development Laboratory \\
Department of Information and Computer Sciences \\
University of Hawaii \\
\texttt{senin@hawaii.edu} \\ [1.0cm]


\emph{Committee:} \\
Philip M. Johnson, Chairperson \\
%%Kyungim Baek \\
%%Guylaine Poisson \\
%%Henri Casanova \\
%%Daniel Port \\ [1.5cm]
}

\normalsize{
CSDL Technical Report 09-14 \\
\url{http://csdl.ics.hawaii.edu/techreports/09-14/09-14.pdf} \\ [1.5cm]
}

\large{2012}

\end{center}

\end{titlepage}


%% Philip suggests it needs a ToC
\tableofcontents

\begin{abstract}
%Abstract goes here if needed.
Software process, team (TSP) or personal (PSP) is a structured human activity resulting in a software. Many well-defined software processes were designed. However, they are not proven to deliver consistently and fail with equal probability. This phenomena was widely observed and defined as a ``software crisis'' in 1968 and since then studied by generations of researchers. Over the years, the research in software process matured and yielded a number of standards and principles aiming the crisis resolution, however, the overall picture changed a little. As an alternative opinion - the ``software development process as a craft'' idea emerged and found some strong supporters in the industrial and research communities. While both approaches for the software process design are supported by excellent research work and industrial success stories, in general, software processes remain designed ``top-down'' - i.e. first someone has to invent a process, design and implement its building blocks and try it after. In my work I explore an alternative approach for the process design - through the discovery of recurrent behaviors from the software process artifacts trails, which may shed light on the larger building blocks of software process.
\end{abstract}

% my main text chapters
\chapter{Software Process Recovery}
Software development process was always being under focus of various stakeholders due to the number of reasons ranging from standards compliance to business and security intelligence. When no live observations are made on the performers, due to the availability, timing, cost, or privacy issues, recovering software process from artifacts could be complex and expensive process. Researchers have suggested a possibility of software process recovery by interviewing of developers and managers and by analysis of process artifacts: such as printed documents - designs, use-cases, software inspections or electronic artifact trails: version, bug and issue control systems and mailing lists. This research resulted in many published work: 
\begin{itemize}
\item{Cook \& Wolf in \cite{citeulike:328044} discuss an event-based framework for process discovery based on grammar inference and finite state machines. The authors directly applied their framework to Software Configuration Management (SCM) logs demonstrating satisfactory results.}
\item{Jensen \& Scacchi \cite{citeulike:5043664} describe an interesting framework built upon mapping between process artifacts and process entities into an universal generic meta-model. Application of their human-involved technique leveraged a pre-existing domain knowledge for the effective pruning and iterative process revision resulted in ``workflows discovery''.}
\item{German in \cite{citeulike:421438} performed a manual mining of the GNOME process artifacts: documentations, CVS logs, and one hundred and four mailing-lists in order to describe the development processes from GSD (Global Software Development) point of view.
}
\item{Ripoche tried a more automatic approach in \cite{citeulike:9112798} by developing a generic model for process-based explanation of bugs persistence using state diagrams and probabilistic choices.}
\end{itemize}
All these suggests that software process artifacts bear enough information about performed process for its recontsruction. In my study I havily relying on this fact. This not only serves as a foundation of my hypothesis, but also partially assures the validity of my approach to the software process reconstruction. Further in this chapter I will introduce a novel technique of pattern mining form the software process artifacts trails. After introduction of the technique, I will walk through the performed case studies in which I have applied this technique to the various types of the software process artifact trails. 

\section{Symbolic Aggregate approXimation (SAX)}

\section{SCM, Software configuration management system}

\section{Software process recovery from SCM system}
As Ball et al. \cite{citeulike:9004378} and Zimmermann \& WeiBgerber \cite{citeulike:5058462} point out - all of the contemporary version control systems provide considerably large amount of auxiliary information about software change. In particular, version control system, when coupled with a mailing lists and (or) bug and issue tracking system is capable of providing information \textit{who} changed \textit{what} and \textit{why}. Which seems to be a fair amount of information needed for one's opinion about the change. It is possible to get an overall understanding of the change necessity through the analysis of bug and issue reports. Version control itself provides quantitative data about files and LOC and changed, added or deleted. The analysis of code snapshots (versions) allows to quantify the change in terms of various software metrics like complexity, cohesion etc. When considered in time all this data provides a solid background for a software evolution research.

However, what is very difficult to know from any contemporary SCM system is that a software process behind the changes. Nevertheless many research in the field of MSR was done in order to shed a light on the software process itself. \cite{citeulike:9007622} There are only traces of such information present in version control transactions. In order to recover some insights about the performed software process information behind a software change statistics behind the change can show us some behavioral patterns blanks in the single transactions can be restored by statistics
outliers effect can be diminished by statistics

\chapter{Use case 1 - Eclipse software process analysis}
\input{eclipse-report}

%%% Input file for bibliography
\bibliography{seninp}
%% Use this for an alphabetically organized bibliography
\bibliographystyle{plain}
%% Use this for a reference order organized bibliography
%\bibliographystyle{unsrt}
%% Try using this BibTeX style that hopefully will print annotations in
%% the bibliography. This will allow me to make notes on papers in the
%% BibTeX file and have them readable in the references section until
%% I turn them into a conceptual literature review 
%\bibliographystyle{annotation}

\end{document}