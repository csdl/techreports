\chapter{Software Process Recovery}
Software development process was always being under focus of various stakeholders due to the number of reasons ranging from standards compliance to business and security intelligence. When no live observations are made on the performers, due to the availability, timing, cost, or privacy issues, recovering software process from artifacts could be complex and expensive process. Researchers have suggested a possibility of software process recovery by interviewing of developers and managers and by analysis of process artifacts: such as printed documents - designs, use-cases, software inspections or electronic artifact trails: version, bug and issue control systems and mailing lists. This research resulted in many published work: 
\begin{itemize}
\item{Cook \& Wolf in \cite{citeulike:328044} discuss an event-based framework for process discovery based on grammar inference and finite state machines. The authors directly applied their framework to Software Configuration Management (SCM) logs demonstrating satisfactory results.}
\item{Jensen \& Scacchi \cite{citeulike:5043664} describe an interesting framework built upon mapping between process artifacts and process entities into an universal generic meta-model. Application of their human-involved technique leveraged a pre-existing domain knowledge for the effective pruning and iterative process revision resulted in ``workflows discovery''.}
\item{German in \cite{citeulike:421438} performed a manual mining of the GNOME process artifacts: documentations, CVS logs, and one hundred and four mailing-lists in order to describe the development processes from GSD (Global Software Development) point of view.
}
\item{Ripoche tried a more automatic approach in \cite{citeulike:9112798} by developing a generic model for process-based explanation of bugs persistence using state diagrams and probabilistic choices.}
\end{itemize}
All these suggests that software process artifacts bear enough information about performed process for its recontsruction. In my study I havily relying on this fact. This not only serves as a foundation of my hypothesis, but also partially assures the validity of my approach to the software process reconstruction. Further in this chapter I will introduce a novel technique of pattern mining form the software process artifacts trails. After introduction of the technique, I will walk through the performed case studies in which I have applied this technique to the various types of the software process artifact trails. 

\section{Symbolic Aggregate approXimation (SAX)}
The last approach for the time-series similarity problem we are reviewing in this writing is the current state of the art time-series representation and dimensionality reduction method called Symbolic Aggregate approXimation (SAX) which transforms original time-series data into symbolic strings. This method, proposed by Lin et al \cite{citeulike:2821475}, turns out to be not only extremely simple and computationally cheap, but also fast and precise in the range-query processing. Moreover, the use of the symbolic representation opens door to the existing wealth of data-structures and string-manipulation algorithms in computer science such as hashing, suffix trees, regular expression pattern matching, etc.

SAX transforms a time-series $X$ of length $n$ into the string of arbitrary length $\omega$ where typically $\omega << n$, using an alphabet $A$ of size $ a \geq 2$. The SAX algorithm consist of two steps: at the first step it transforms the original time-series into PAA representation and this intermediate representation than converted into the string during the second step. Use of the PAA at the first step brings an advantage of simple and efficient dimensionality reduction while providing the important lower bounding property. Second step, actual conversion of PAA coefficients into letters also computationally efficient and the lower bounding of symbolic distance was proven by Lin et al.

Discretization of the PAA representation of the time-series into SAX implemented in a way which produces symbols corresponding to the time-series features with equal probability. The rigorous analysis of the time-series datasets available for authors shows that normalized by the zero mean and unit of energy time-series follow the Normal distribution law. By using the Gaussian distribution properties \cite{citeulike:167581} it's easy to pick $a$ equal-sized areas under the Normal curve. The points of the cut lines slicing the the under-the-Gaussian-curve area called ``breakpoints''.
\begin{figure}[tbp]
   \centering
   \includegraphics[height=45mm]{sax_intro.eps}
   \caption{The illustration of the SAX approach taken from \cite{citeulike:2821475} depicts two pre-determined breakpoints for the three-symbols alphabet and the conversion of the time-series of length $n=128$ into PAA representation first and following mapping of the PAA coefficients into SAX symbols with $w=8$ and $a=3$ resulting in the string \textbf{baabccbc}.}
   \label{fig:sax_intro}
\end{figure}

Extending Euclidean \ref{eq:euclidean_distance} and PAA \ref{eq:paa_distnace} distances, the function returning the minimal distance between two string representations of original time series $\hat{Q}$ and $\hat{C}$ defined as
\begin{equation}
MINDIST(\hat{Q},\hat{C}) \equiv \sqrt{ \frac{n}{w} } \sqrt{ \sum_{i=1}^{w} ( dist( \hat{q}_{i}, \hat{c}_{i} ) )^{2}}
\label{eq:sax_mindist}
\end{equation} 
where the $dist$ function is implemented by using the lookup table for the particular set of the breakpoints as shown in table \ref{tbl:sax_lookup} where the singular value for each cell $(r,c)$ is computed as 
\begin{equation}
cell_{(r,c)} = 
\begin{cases} 
0, \text{ if }\left| r-c \right| \leq 1 \\
\beta_{\max(r,c) - 1} - \beta_{\min(r,c) - 1}, \text{ otherwise}
\end{cases}
\label{eq:cell}
\end{equation}
\begin{table}
\begin{tabularx}{400pt}{X X X X X}
\hline
   & a   & b    & c    & d    \\
\hline
a & 0    & 0    & 0.67 & 1.34 \\
b & 0    & 0    & 0    & 0.67 \\
c & 0.67 & 0    & 0    & 0    \\
d & 1.34 & 0.67 & 0    & 0    \\
\hline
\end{tabularx}
\caption{A lookup table used by the MINDIST function for the $a=4$}
\label{tbl:sax_lookup}
\end{table}
\begin{figure}[tbp]
   \centering
   \includegraphics[height=47mm]{sax_distance.eps}
   \caption{The visual representation of the two time-series $Q$ and $C$ and three distances between their representation: Euclidean distance between raw time-series (A), the distance defined for PAA coefficients (B) and the distance between two SAX representations (C).}
   \label{fig:sax_distance}
\end{figure}


As shown by Li et al, the introduced SAX distance measure lower-bounds the PAA distance, i.e.
\begin{equation}
\sum_{i=1}^{n} (q_{i} - c_{i})^{2} \geq n(\bar{Q} - \bar{C})^{2} \geq n(dist(\hat{Q},\hat{C}))^2
\label{eq:sax_bounding}
\end{equation}


\section{SCM, Software configuration management system}

\section{Software process recovery from SCM system}
As Ball et al. \cite{citeulike:9004378} and Zimmermann \& WeiBgerber \cite{citeulike:5058462} point out - all of the contemporary version control systems provide considerably large amount of auxiliary information about software change. In particular, version control system, when coupled with a mailing lists and (or) bug and issue tracking system is capable of providing information \textit{who} changed \textit{what} and \textit{why}. Which seems to be a fair amount of information needed for one's opinion about the change. It is possible to get an overall understanding of the change necessity through the analysis of bug and issue reports. Version control itself provides quantitative data about files and LOC and changed, added or deleted. The analysis of code snapshots (versions) allows to quantify the change in terms of various software metrics like complexity, cohesion etc. When considered in time all this data provides a solid background for a software evolution research.

However, what is very difficult to know from any contemporary SCM system is that a software process behind the changes. Nevertheless many research in the field of MSR was done in order to shed a light on the software process itself. \cite{citeulike:9007622} There are only traces of such information present in version control transactions. In order to recover some insights about the performed software process information behind a software change statistics behind the change can show us some behavioral patterns blanks in the single transactions can be restored by statistics
outliers effect can be diminished by statistics

\chapter{Use case 1 - Eclipse software process analysis}