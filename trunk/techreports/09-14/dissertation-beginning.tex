\documentclass[12pt,oneside]{report}

%%% Load some useful packages:
%% "New" LaTeX2e graphics support.
\usepackage{graphicx}
%%	using final option to force graphics to be included even in draft mode
%\usepackage[final]{graphicx}
%% Tell graphicx the default directory for all figures
\graphicspath{{figures/}}

% Enable subfigure support
\usepackage{subfigure}

%% Make subsubsections numbered and included in ToC
\setcounter{secnumdepth}{3}
\setcounter{tocdepth}{2}

%% Package to linebreak URLs in a sane manner.
\usepackage{url}

%% Define a new 'smallurl' style for the package that will use a smaller font.
\makeatletter
\def\url@smallurlstyle{%
  \@ifundefined{selectfont}{\def\UrlFont{\sf}}{\def\UrlFont{\small\ttfamily}}}
\makeatother
%% Now actually use the newly defined style.
\urlstyle{smallurl}

%% Define 'tinyurl' style for even smaller URLs (such as in tables)
\makeatletter
\def\url@tinyurlstyle{%
  \@ifundefined{selectfont}{\def\UrlFont{\sf}}{\def\UrlFont{\scriptsize\ttfamily}}}
\makeatother

%% Provides additional functionality for tabular environments
\usepackage{array}

%% Puts space after macros, unless followed by punctuation
\usepackage{xspace}

%% Make margins less ridiculous
\usepackage{fullpage}

%% Allows insertion of fixme notes for future work
\usepackage[footnote, nomargin]{fixme}

%%%% Turned off for tech report, should be turned on for research portfolio
%% Turn on double spacing
\usepackage{setspace}
\usepackage{mdwlist}
\doublespacing

%% Make URLs clickable
%\usepackage[colorlinks, bookmarks=false]{hyperref}
\usepackage[colorlinks, bookmarks=true]{hyperref}

%% Since I'm using the LaTeX Makefile that uses dvips, I need this
%% package to make URLs break nicely
\usepackage{breakurl}

\usepackage{amsmath,amsfonts}
\numberwithin{equation}{subsection}
%%\usepackage{nonfloat}
\usepackage{bbm}
\usepackage{setspace}
\onehalfspacing
\usepackage{tabularx}

%%% End of preamble
\begin{document}

\begin{titlepage}
\vspace*{1in}
\begin{center}
   
\Large


{\bf Project Hackystat: Accelerating adoption of empirically guided software development through
  non-disruptive, developer-centric, in-process data collection and analysis}

\bigskip

\normalsize

Philip Johnson                           \medskip\par
Department of Information and Computer Sciences\\ 
University of Hawaii\\ 
Honolulu, HI 96822\\                       
(808) 956-3489\\
(808) 956-3548 (fax)\\
{\tt johnson@hawaii.edu}                 \bigskip\par

\today                                   \bigskip\par



\end{center}
\end{titlepage}


%% Philip suggests it needs a ToC
\tableofcontents

\begin{abstract}
%Abstract goes here if needed.
Software process defines a structure imposed on a human activity resulting in a software. Many different kinds of software processes were designed up today. However, they are not proven to deliver consistently and fail with equal probability. This phenomena was widely observed and defined as a ``software crisis'' in 1968. Since then it was studied by generations of researchers. Over these years the research in software process matured and yielded a number of standardized approaches for the process design mechanisms and general principles aimed a ``crisis'' resolution. However, the overall picture changed a little. As an alternative opinion - the ``software development process as a craft'' idea, which emphasize unique roles of developers in creating software, emerged and found some strong supporters in the industrial and research communities. While both views on the software process composition are supported by excellent research work and industrial success stories, in general, software processes remain designed ``top-down'' - i.e. first someone has to invent a process, design and implement its building blocks and try it after. In this work I explore an alternative approach for the software process analysis - through the discovery of recurrent behaviors from software process artifacts trails. These behaviors are thought to be the finest building blocks of larger processes and their understanding may potentially shed light on the larger scale blocks and their interactions within a software process and its formal models.
\end{abstract}

% my main text chapters
\chapter{Introduction}
Delivering high quality software products within the budget and in time is the main goal and the most 
challenging task of Software Engineering. Years of scientific research in this area resulted in a 
number of software processes providing detailed guidelines on how to reach 
the goal efficiently. These processes manifested themselfs as the means for improvements in terms 
of quality, speed and cost over existing practices. Many were implemented and tested within academic 
and industrial settings and proved proposed superiority. Some of these processes were successfully 
adopted and standardized in industry shaping the best practices of contemporary software development 
\cite{citeulike:9962021}. Moreover, there are plethora of processes for improving existing processes 
of software development on the team \cite{citeulike:9962027} and personal 
levels \cite{citeulike:9962022}.

The processes I am mentioning here are the well-known large formal models such as Waterfall and Spiral, 
as well as more flexible iterative agile approaches like XP, SCRUM or FDD. These are also sets of 
rules and recommendations which can be applied to certain stages of the software processes 
such as Test Driven Development or Pair Programming; there are general guidelines helping 
to improve the correctness of a product and standards, like CMMI or ISO 9000; guidlines for testing 
and measurements, code syntax rules and formatting styles, code comments 
recomendations \cite{citeulike:900855}. 

From the first sight, taking all this in account,  one would guess that 
the area of software processes is thoroughly explored and there are clear choices of processes 
and models for the one in charge making decision... But it is not true - despite many choices 
one can make, no one can foretell what is the ``best'' process to choose for certain constraints.
What managers are left with are the equal alternatives and vague promises. 
This deficiency in knowledge is the main coause of the ``software crisis'' phenomena point is supported by the fact that according to ``Chaos Report'' from the Standish 
Group (Rubinstein) \cite{SDTimes} only ``35\% of software projects in 2006 can be categorized as successful - meaning 
they were completed on time, on budget and met user requirements''. 
These thirty five percent of success clearly saying that it is somewhat difficult to make 
a statement that we are fully understand and able to control software processes. 
Moreover, over years, while this idea of a software process formalizations shaped the 
programming practices, which once thought to be a creative human activity accessible by amateurs 
and hobbyists \cite{citeulike:9958822} into a serious engineering discipline, bounded 
by requirements for education, standardized processes, rules, certifications, and strict 
financial requirements from stakeholders the opposite idea was born - the idea of 
software development as a craft. Interesting that such a duality of views can be found 
in the work of a single person \cite{citeulike:5203446}.

Clearly, there is a great room for research and improvement of our understanding of software processes.
This exploratory study is yet another attempt at the understanding. In my research work I am 
exploring techniques aiming the understanding of small processes which are 
rather the reflection of personal behaviors or habits of software development rather than a 
formalized constructs. Also, I would like to emphasize, that in this work I will not 
address the need and means of the process synthesis, its quality assessment, productiveness
or any topics related to the software product itself; I would rather focus on the specific issue - 
uncovering an existence and studying the programming habits. 

This thesis presents a methodology for finding recurrent behaviors through the 
analysis of the variety of software process artifacts left after performing a 
software process. I have called this methodology ``Software Trajectory'' and it consists 
of four distinct steps. Each of these steps has a specific goal and compromising variety of 
means to reach it. 
At first software process artifacts are identified and collected. 
At second, they are cleaned, organized and classified. 
On the third step particular research questions are formulated and data are organized and indexed. 
And finally, a set of KDD techniques is applied in order to undercover recurrent behaviors which 
could potentially shed a light on the performed process details. 

My personal motivation for performing this work is coming from the recognition of the 
importance of the software in our lives and the severity of issues with its development. 
Through my everyday experiences with software development and use I have stumble upon 
a number of issues which made me realize that mentioned ``software crisis'' phenomena is very real.
As a user in industrial and academic settings I often find myself facing software failures 
which create numerous difficulties for reaching production or research goals. As a developer, 
in an attempt to be productive and in order to deliver a better software I have studied and 
explored a number of formal processes, however, sometime I found myself seeing a very little of 
rationale behind their application, and moreover, in this exploration, when facing the process
application failing to help I was unable to comprehend what exactly went wrong and what need 
to be changed. All of these experiences made me studying software process research and exploring
novel approaches to software process recovery on my own in order to understand software process better.

\section{Research area overview}
As mentioned above, in this thesis I am focusing on a very narrow subject - exploring approaches
for uncovering of recurrent behaviors or ``programming habits'' out of software process artifacts.
Before narrowing further 


Software is usually coded by teams. Members of these teams are agreed and bound to use 
a particular technologies and development tools, they also agree on following well defined 
development process which is constrained by a timeline and budget. These are necessary 
constraints to keep work organized, however there is a great freedom in what they actually 
do in every single moment of time in order to progress towards lines of code which eventually 
will result in software. For example one developer may follow test first process while
another writes tests at last.  This freedom of choice in ordering of development activities 
while being much appreciated by talented and creative individuals creates an impression 
of chaotic and unordered activities for random observers, newbies and people in 
charge - so there we have all the attempts of imposing an order 
(or control) on all of the development activities. Metrics and models of processes





\chapter{Software Process Recovery}
Software development process was always being under focus of various stakeholders due to the number of reasons ranging from standards compliance to business and security intelligence. When no live observations are made on the performers, due to the availability, timing, cost, or privacy issues, recovering software process from artifacts could be complex and expensive process. Researchers have suggested a possibility of software process recovery by interviewing of developers and managers and by analysis of process artifacts: such as printed documents - designs, use-cases, software inspections or electronic artifact trails: version, bug and issue control systems and mailing lists. This research resulted in many published work: 
\begin{itemize}
\item{Cook \& Wolf in \cite{citeulike:328044} discuss an event-based framework for process discovery based on grammar inference and finite state machines. The authors directly applied their framework to Software Configuration Management (SCM) logs demonstrating satisfactory results.}
\item{Jensen \& Scacchi \cite{citeulike:5043664} describe an interesting framework built upon mapping between process artifacts and process entities into an universal generic meta-model. Application of their human-involved technique leveraged a pre-existing domain knowledge for the effective pruning and iterative process revision resulted in ``workflows discovery''.}
\item{German in \cite{citeulike:421438} performed a manual mining of the GNOME process artifacts: documentations, CVS logs, and one hundred and four mailing-lists in order to describe the development processes from GSD (Global Software Development) point of view.
}
\item{Ripoche tried a more automatic approach in \cite{citeulike:9112798} by developing a generic model for process-based explanation of bugs persistence using state diagrams and probabilistic choices.}
\end{itemize}
All these suggests that software process artifacts bear enough information about performed process for its recontsruction. In my study I havily relying on this fact. This not only serves as a foundation of my hypothesis, but also partially assures the validity of my approach to the software process reconstruction. Further in this chapter I will introduce a novel technique of pattern mining form the software process artifacts trails. After introduction of the technique, I will walk through the performed case studies in which I have applied this technique to the various types of the software process artifact trails. 

\section{Symbolic Aggregate approXimation (SAX)}
The last approach for the time-series similarity problem we are reviewing in this writing is the current state of the art time-series representation and dimensionality reduction method called Symbolic Aggregate approXimation (SAX) which transforms original time-series data into symbolic strings. This method, proposed by Lin et al \cite{citeulike:2821475}, turns out to be not only extremely simple and computationally cheap, but also fast and precise in the range-query processing. Moreover, the use of the symbolic representation opens door to the existing wealth of data-structures and string-manipulation algorithms in computer science such as hashing, suffix trees, regular expression pattern matching, etc.

SAX transforms a time-series $X$ of length $n$ into the string of arbitrary length $\omega$ where typically $\omega << n$, using an alphabet $A$ of size $ a \geq 2$. The SAX algorithm consist of two steps: at the first step it transforms the original time-series into PAA representation and this intermediate representation than converted into the string during the second step. Use of the PAA at the first step brings an advantage of simple and efficient dimensionality reduction while providing the important lower bounding property. Second step, actual conversion of PAA coefficients into letters also computationally efficient and the lower bounding of symbolic distance was proven by Lin et al.

Discretization of the PAA representation of the time-series into SAX implemented in a way which produces symbols corresponding to the time-series features with equal probability. The rigorous analysis of the time-series datasets available for authors shows that normalized by the zero mean and unit of energy time-series follow the Normal distribution law. By using the Gaussian distribution properties \cite{citeulike:167581} it's easy to pick $a$ equal-sized areas under the Normal curve. The points of the cut lines slicing the the under-the-Gaussian-curve area called ``breakpoints''.
\begin{figure}[tbp]
   \centering
   \includegraphics[height=45mm]{sax_intro.eps}
   \caption{The illustration of the SAX approach taken from \cite{citeulike:2821475} depicts two pre-determined breakpoints for the three-symbols alphabet and the conversion of the time-series of length $n=128$ into PAA representation first and following mapping of the PAA coefficients into SAX symbols with $w=8$ and $a=3$ resulting in the string \textbf{baabccbc}.}
   \label{fig:sax_intro}
\end{figure}

Extending Euclidean \ref{eq:euclidean_distance} and PAA \ref{eq:paa_distnace} distances, the function returning the minimal distance between two string representations of original time series $\hat{Q}$ and $\hat{C}$ defined as
\begin{equation}
MINDIST(\hat{Q},\hat{C}) \equiv \sqrt{ \frac{n}{w} } \sqrt{ \sum_{i=1}^{w} ( dist( \hat{q}_{i}, \hat{c}_{i} ) )^{2}}
\label{eq:sax_mindist}
\end{equation} 
where the $dist$ function is implemented by using the lookup table for the particular set of the breakpoints as shown in table \ref{tbl:sax_lookup} where the singular value for each cell $(r,c)$ is computed as 
\begin{equation}
cell_{(r,c)} = 
\begin{cases} 
0, \text{ if }\left| r-c \right| \leq 1 \\
\beta_{\max(r,c) - 1} - \beta_{\min(r,c) - 1}, \text{ otherwise}
\end{cases}
\label{eq:cell}
\end{equation}
\begin{table}
\begin{tabularx}{400pt}{X X X X X}
\hline
   & a   & b    & c    & d    \\
\hline
a & 0    & 0    & 0.67 & 1.34 \\
b & 0    & 0    & 0    & 0.67 \\
c & 0.67 & 0    & 0    & 0    \\
d & 1.34 & 0.67 & 0    & 0    \\
\hline
\end{tabularx}
\caption{A lookup table used by the MINDIST function for the $a=4$}
\label{tbl:sax_lookup}
\end{table}
\begin{figure}[tbp]
   \centering
   \includegraphics[height=47mm]{sax_distance.eps}
   \caption{The visual representation of the two time-series $Q$ and $C$ and three distances between their representation: Euclidean distance between raw time-series (A), the distance defined for PAA coefficients (B) and the distance between two SAX representations (C).}
   \label{fig:sax_distance}
\end{figure}


As shown by Li et al, the introduced SAX distance measure lower-bounds the PAA distance, i.e.
\begin{equation}
\sum_{i=1}^{n} (q_{i} - c_{i})^{2} \geq n(\bar{Q} - \bar{C})^{2} \geq n(dist(\hat{Q},\hat{C}))^2
\label{eq:sax_bounding}
\end{equation}


\section{SCM, Software configuration management system}

\section{Software process recovery from SCM system}
As Ball et al. \cite{citeulike:9004378} and Zimmermann \& WeiBgerber \cite{citeulike:5058462} point out - all of the contemporary version control systems provide considerably large amount of auxiliary information about software change. In particular, version control system, when coupled with a mailing lists and (or) bug and issue tracking system is capable of providing information \textit{who} changed \textit{what} and \textit{why}. Which seems to be a fair amount of information needed for one's opinion about the change. It is possible to get an overall understanding of the change necessity through the analysis of bug and issue reports. Version control itself provides quantitative data about files and LOC and changed, added or deleted. The analysis of code snapshots (versions) allows to quantify the change in terms of various software metrics like complexity, cohesion etc. When considered in time all this data provides a solid background for a software evolution research.

However, what is very difficult to know from any contemporary SCM system is that a software process behind the changes. Nevertheless many research in the field of MSR was done in order to shed a light on the software process itself. \cite{citeulike:9007622} There are only traces of such information present in version control transactions. In order to recover some insights about the performed software process information behind a software change statistics behind the change can show us some behavioral patterns blanks in the single transactions can be restored by statistics
outliers effect can be diminished by statistics

\chapter{Use case 1 - Eclipse software process analysis}
\input{eclipse-report}

%%% Input file for bibliography
\bibliography{seninp}
%% Use this for an alphabetically organized bibliography
\bibliographystyle{plain}
%% Use this for a reference order organized bibliography
%\bibliographystyle{unsrt}
%% Try using this BibTeX style that hopefully will print annotations in
%% the bibliography. This will allow me to make notes on papers in the
%% BibTeX file and have them readable in the references section until
%% I turn them into a conceptual literature review 
%\bibliographystyle{annotation}

\end{document}