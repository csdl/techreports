\section{Research problem statement and Scope of the dissertation}
Software is coded by humans. Whether in team or individually, humans perform relevant 
daily activities in order to reach the goal - deliver the software. Understanding of these
human activities spanning through the life-cycle of the software, in connection with personal 
and team's motivations, environment settings and constraints essentially enables one to
comprehend the software process. It is worth noting, that roughly \todo[inline]{put here something 
about two components - the human-driven, non-recurrent and creative activity - 
the behavioral component - and the process and the technology/toolkit component which
provides a measurable marginal effect}

\todo[inline]{Here, put stuff about observing the process and artifacts availability}

\todo[inline]{Here, put stuff about time-series analysis flexibility and its difference from 
convenient process mining.}

While it is shown by the large body of previous research, that it is technically possible 
to factor out the impact of the technology and the process, the impact of the behavioral 
component is yet to be studied. Hence, my primary research question is this:
\begin{myindentpar}{0.07\linewidth}
 \textbf{Can we discover recurrent behaviors in software processes from project
  repository artifacts?}
\end{myindentpar}

Obviously, in order to answer this question correctly two interconnected, secondary, question must be resolved:
\begin{myindentpar}{0.07\linewidth}
 \textbf{Which kind of software process artifacts reflect recurrent behaviors?}
\end{myindentpar}
\begin{myindentpar}{0.07\linewidth}
 \textbf{Which data-mining method is sensitive and selective enough to recover recurrent behaviors
from software process artifacts?}
\end{myindentpar}
These two can be broken down further to the problem of studying and classifying of software process artifacts,
methods of their extraction and partitioning, and, of course, the problem of recurrent behaviors discovery.

As was mentioned above, in this thesis I am focusing on a very narrow subject - exploring approaches
for uncovering of recurrent behaviors or ``programming habits'' out of software process artifacts.
While I will show, that, potentially, many software-development activity behaviors can be recovered,
their classification, impact and performance studies are beyond the scope of this thesis.

