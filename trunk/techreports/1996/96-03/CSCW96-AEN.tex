%%%%%%%%%%%%%%%%%%%%%%%%%%%%%% -*- Mode: Latex -*- %%%%%%%%%%%%%%%%%%%%%%%%%%%%
%% CSCW96-AEN.tex -- 
%% Author          : Carleton Moore
%% Created On      : Sat Mar  9 13:42:23 1996
%% Last Modified By: Carleton Moore
%% Last Modified On: Sat Mar  9 14:35:40 1996
%% RCS: $Id$
%%%%%%%%%%%%%%%%%%%%%%%%%%%%%%%%%%%%%%%%%%%%%%%%%%%%%%%%%%%%%%%%%%%%%%%%%%%%%%%
%%   Copyright (C) 1996 Carleton Moore
%%%%%%%%%%%%%%%%%%%%%%%%%%%%%%%%%%%%%%%%%%%%%%%%%%%%%%%%%%%%%%%%%%%%%%%%%%%%%%%
%% 

\documentstyle[nftimes,/group/csdl/tex/CHI95]{article}
\input{/group/csdl/tex/psfig/psfig}

\begin{document}

\title{Exploring Strong Collaboration with AEN}
\author{Carleton Moore\\
Collaborative Software Development Laboratory,\\
Department of Information and Computer Sciences\\
2565 The Mall\\
University of Hawaii, Manoa\\
Honolulu, Hawaii   96822\\
{\tt cmoore@uhics.ics.hawaii.edu}}
\maketitle
\abstract

This paper overviews the Annotated Egret Navigator (AEN), a system designed
to support {\em strong collaboration} among a group as they cooperatively
build, review, improve, and learn from a structured hypertext document.
AEN explores how to support strong collaboration through computer
mediation. It enables collaborative creation of hypertext documents and
instruments the actions of its users to understand how such creation
occurs.  Base upon our evaluation of and experience with AEN, I recommend
the following to encourage strong collaboration: provide direct and
indirect authoring mechanisms, provide access to intermediate work
products, provide context-sensitive ``what's new,'' and provide mechanisms
to allow users as well as documents to be visible.

\paragraph{KEYWORDS:} Hypertext, Strong Collaboration, Authoring

\section{INTRODUCTION}

\section{AEN}

\section{Experiment in Collaboration}

\section{Related Work}


\subsection{What do the results say in general?}

Pure strong collaboration is not necessary for quality work.  Not necessary
for all the scores to be high for strong collaboration.  Certain tools are
needed for strong collaboration?  Discussion among group members to resolve
issues?  Mechanism for revision/co authorship? 

Strong collaboration can be supported.  It is not easy to do.  Maybe can't
be done for large groups.  People can control/interact with only 3-7
people.  Does this also go for collaboration?  Strong collaboration yes.
You have to deal with the other members of the team

\subsection{Is strong collaboration good?}

Not always.  Takes time and effort.  if time sensitive maybe not.  Yes
improves not only artifact but group.  All can learn from the process.  All
can learn about the other aspects of the artifact.  Anecdotally, should be
small groups.  Group process may limit sizes.
Not for groups who don't want to come together.  If there is no desire to
come together and share ideas then strong collaboration is impossible.

Yes it is good for improving the group.  They come to a common level of
understanding.  There is consensus....I believe the document represents
consensus.  In the creation of the document each member learns/inputs their
views/ideas to improve the document/themselves.  They get to see what the
others think and believe.  Great way of building Esprit de Corps.  They are
a team and they have a document proving they worked together something they
can all claim ownership of.  We did this and it is of high quality.


Strong collaboration can't be used for large projects.  Must only be used
for ``small'' sections of a project.


\subsection{How would I change AEN to make it better/support strong
collaboration better.}

Allow for two people to edit the same node at the same time.  This way they
can gain access and have a finer level of interaction.  Allow commenting
while editing of the node is going on.  Allow people to see the changes as
they occur.  This could allow mentoring...  Watching over the shoulder and
making suggestions while the artifact is being created/revised/commented
on.  Simplify the node creation and access control mechanisms.  Improve the
off line access, allow printing of the document.  More reliable system.
clean up the metric collection mechanism.  

\subsection{If the WWW could be made to support strong collaboration how
could it be done?}

Some sort of presence for the users of the Web.  A way of communicating
with your group members.  Way of seeing/going to the location that your
group member is at.  Allow editing of the Web pages remotely.  Some sort of
access control mechanism that allows privacy from outsiders.  Some way of
dynamically modifying the group/process of development.  A way of blocking
out the rest of the world until the document is ready for review / release.
Better real time communications.  A sense of group. How can I find my group
and talk to them?  How do I not get outsiders butting their noses into our
work?  Do I want input from outsiders?  They could have good ideas/They
could overload the system and be a constant pain.

\end{document}