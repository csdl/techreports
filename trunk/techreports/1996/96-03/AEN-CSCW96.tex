%%%%%%%%%%%%%%%%%%%%%%%%%%%%%% -*- Mode: Latex -*- %%%%%%%%%%%%%%%%%%%%%%%%%%%%
%% AEN-CSCW96.tex -- 
%% Author          : Carleton Moore
%% Created On      : Sun Oct 29 13:13:22 1995
%% Last Modified By: Carleton Moore
%% Last Modified On: Wed Mar  6 18:29:07 1996
%% RCS: $Id: AEN-CSCW96.tex,v 1.1 1996/02/26 00:03:33 cmoore Exp cmoore $
%%%%%%%%%%%%%%%%%%%%%%%%%%%%%%%%%%%%%%%%%%%%%%%%%%%%%%%%%%%%%%%%%%%%%%%%%%%%%%%
%%   Copyright (C) 1995 Carleton Moore
%%%%%%%%%%%%%%%%%%%%%%%%%%%%%%%%%%%%%%%%%%%%%%%%%%%%%%%%%%%%%%%%%%%%%%%%%%%%%%%
%% 

\documentstyle[nftimes,/group/csdl/tex/cscw]{article}
% Psfig/TeX 
\def\PsfigVersion{1.9}
% dvips version
%
% All psfig/tex software, documentation, and related files
% in this distribution of psfig/tex are 
% Copyright 1987, 1988, 1991 Trevor J. Darrell
%
% Permission is granted for use and non-profit distribution of psfig/tex 
% providing that this notice is clearly maintained. The right to
% distribute any portion of psfig/tex for profit or as part of any commercial
% product is specifically reserved for the author(s) of that portion.
%
% *** Feel free to make local modifications of psfig as you wish,
% *** but DO NOT post any changed or modified versions of ``psfig''
% *** directly to the net. Send them to me and I'll try to incorporate
% *** them into future versions. If you want to take the psfig code 
% *** and make a new program (subject to the copyright above), distribute it, 
% *** (and maintain it) that's fine, just don't call it psfig.
%
% Bugs and improvements to trevor@media.mit.edu.
%
% Thanks to Greg Hager (GDH) and Ned Batchelder for their contributions
% to the original version of this project.
%
% Modified by J. Daniel Smith on 9 October 1990 to accept the
% %%BoundingBox: comment with or without a space after the colon.  Stole
% file reading code from Tom Rokicki's EPSF.TEX file (see below).
%
% More modifications by J. Daniel Smith on 29 March 1991 to allow the
% the included PostScript figure to be rotated.  The amount of
% rotation is specified by the "angle=" parameter of the \psfig command.
%
% Modified by Robert Russell on June 25, 1991 to allow users to specify
% .ps filenames which don't yet exist, provided they explicitly provide
% boundingbox information via the \psfig command. Note: This will only work
% if the "file=" parameter follows all four "bb???=" parameters in the
% command. This is due to the order in which psfig interprets these params.
%
%  3 Jul 1991	JDS	check if file already read in once
%  4 Sep 1991	JDS	fixed incorrect computation of rotated
%			bounding box
% 25 Sep 1991	GVR	expanded synopsis of \psfig
% 14 Oct 1991	JDS	\fbox code from LaTeX so \psdraft works with TeX
%			changed \typeout to \ps@typeout
% 17 Oct 1991	JDS	added \psscalefirst and \psrotatefirst
%

% From: gvr@cs.brown.edu (George V. Reilly)
%
% \psdraft	draws an outline box, but doesn't include the figure
%		in the DVI file.  Useful for previewing.
%
% \psfull	includes the figure in the DVI file (default).
%
% \psscalefirst width= or height= specifies the size of the figure
% 		before rotation.
% \psrotatefirst (default) width= or height= specifies the size of the
% 		 figure after rotation.  Asymetric figures will
% 		 appear to shrink.
%
% \psfigurepath#1	sets the path to search for the figure
%
% \psfig
% usage: \psfig{file=, figure=, height=, width=,
%			bbllx=, bblly=, bburx=, bbury=,
%			rheight=, rwidth=, clip=, angle=, silent=}
%
%	"file" is the filename.  If no path name is specified and the
%		file is not found in the current directory,
%		it will be looked for in directory \psfigurepath.
%	"figure" is a synonym for "file".
%	By default, the width and height of the figure are taken from
%		the BoundingBox of the figure.
%	If "width" is specified, the figure is scaled so that it has
%		the specified width.  Its height changes proportionately.
%	If "height" is specified, the figure is scaled so that it has
%		the specified height.  Its width changes proportionately.
%	If both "width" and "height" are specified, the figure is scaled
%		anamorphically.
%	"bbllx", "bblly", "bburx", and "bbury" control the PostScript
%		BoundingBox.  If these four values are specified
%               *before* the "file" option, the PSFIG will not try to
%               open the PostScript file.
%	"rheight" and "rwidth" are the reserved height and width
%		of the figure, i.e., how big TeX actually thinks
%		the figure is.  They default to "width" and "height".
%	The "clip" option ensures that no portion of the figure will
%		appear outside its BoundingBox.  "clip=" is a switch and
%		takes no value, but the `=' must be present.
%	The "angle" option specifies the angle of rotation (degrees, ccw).
%	The "silent" option makes \psfig work silently.
%

% check to see if macros already loaded in (maybe some other file says
% "\input psfig") ...
\ifx\undefined\psfig\else\endinput\fi

%
% from a suggestion by eijkhout@csrd.uiuc.edu to allow
% loading as a style file. Changed to avoid problems
% with amstex per suggestion by jbence@math.ucla.edu

\let\LaTeXAtSign=\@
\let\@=\relax
\edef\psfigRestoreAt{\catcode`\@=\number\catcode`@\relax}
%\edef\psfigRestoreAt{\catcode`@=\number\catcode`@\relax}
\catcode`\@=11\relax
\newwrite\@unused
\def\ps@typeout#1{{\let\protect\string\immediate\write\@unused{#1}}}
\ps@typeout{psfig/tex \PsfigVersion}

%% Here's how you define your figure path.  Should be set up with null
%% default and a user useable definition.

\def\figurepath{./}
\def\psfigurepath#1{\edef\figurepath{#1}}

%
% @psdo control structure -- similar to Latex @for.
% I redefined these with different names so that psfig can
% be used with TeX as well as LaTeX, and so that it will not 
% be vunerable to future changes in LaTeX's internal
% control structure,
%
\def\@nnil{\@nil}
\def\@empty{}
\def\@psdonoop#1\@@#2#3{}
\def\@psdo#1:=#2\do#3{\edef\@psdotmp{#2}\ifx\@psdotmp\@empty \else
    \expandafter\@psdoloop#2,\@nil,\@nil\@@#1{#3}\fi}
\def\@psdoloop#1,#2,#3\@@#4#5{\def#4{#1}\ifx #4\@nnil \else
       #5\def#4{#2}\ifx #4\@nnil \else#5\@ipsdoloop #3\@@#4{#5}\fi\fi}
\def\@ipsdoloop#1,#2\@@#3#4{\def#3{#1}\ifx #3\@nnil 
       \let\@nextwhile=\@psdonoop \else
      #4\relax\let\@nextwhile=\@ipsdoloop\fi\@nextwhile#2\@@#3{#4}}
\def\@tpsdo#1:=#2\do#3{\xdef\@psdotmp{#2}\ifx\@psdotmp\@empty \else
    \@tpsdoloop#2\@nil\@nil\@@#1{#3}\fi}
\def\@tpsdoloop#1#2\@@#3#4{\def#3{#1}\ifx #3\@nnil 
       \let\@nextwhile=\@psdonoop \else
      #4\relax\let\@nextwhile=\@tpsdoloop\fi\@nextwhile#2\@@#3{#4}}
% 
% \fbox is defined in latex.tex; so if \fbox is undefined, assume that
% we are not in LaTeX.
% Perhaps this could be done better???
\ifx\undefined\fbox
% \fbox code from modified slightly from LaTeX
\newdimen\fboxrule
\newdimen\fboxsep
\newdimen\ps@tempdima
\newbox\ps@tempboxa
\fboxsep = 3pt
\fboxrule = .4pt
\long\def\fbox#1{\leavevmode\setbox\ps@tempboxa\hbox{#1}\ps@tempdima\fboxrule
    \advance\ps@tempdima \fboxsep \advance\ps@tempdima \dp\ps@tempboxa
   \hbox{\lower \ps@tempdima\hbox
  {\vbox{\hrule height \fboxrule
          \hbox{\vrule width \fboxrule \hskip\fboxsep
          \vbox{\vskip\fboxsep \box\ps@tempboxa\vskip\fboxsep}\hskip 
                 \fboxsep\vrule width \fboxrule}
                 \hrule height \fboxrule}}}}
\fi
%
%%%%%%%%%%%%%%%%%%%%%%%%%%%%%%%%%%%%%%%%%%%%%%%%%%%%%%%%%%%%%%%%%%%
% file reading stuff from epsf.tex
%   EPSF.TEX macro file:
%   Written by Tomas Rokicki of Radical Eye Software, 29 Mar 1989.
%   Revised by Don Knuth, 3 Jan 1990.
%   Revised by Tomas Rokicki to accept bounding boxes with no
%      space after the colon, 18 Jul 1990.
%   Portions modified/removed for use in PSFIG package by
%      J. Daniel Smith, 9 October 1990.
%
\newread\ps@stream
\newif\ifnot@eof       % continue looking for the bounding box?
\newif\if@noisy        % report what you're making?
\newif\if@atend        % %%BoundingBox: has (at end) specification
\newif\if@psfile       % does this look like a PostScript file?
%
% PostScript files should start with `%!'
%
{\catcode`\%=12\global\gdef\epsf@start{%!}}
\def\epsf@PS{PS}
%
\def\epsf@getbb#1{%
%
%   The first thing we need to do is to open the
%   PostScript file, if possible.
%
\openin\ps@stream=#1
\ifeof\ps@stream\ps@typeout{Error, File #1 not found}\else
%
%   Okay, we got it. Now we'll scan lines until we find one that doesn't
%   start with %. We're looking for the bounding box comment.
%
   {\not@eoftrue \chardef\other=12
    \def\do##1{\catcode`##1=\other}\dospecials \catcode`\ =10
    \loop
       \if@psfile
	  \read\ps@stream to \epsf@fileline
       \else{
	  \obeyspaces
          \read\ps@stream to \epsf@tmp\global\let\epsf@fileline\epsf@tmp}
       \fi
       \ifeof\ps@stream\not@eoffalse\else
%
%   Check the first line for `%!'.  Issue a warning message if its not
%   there, since the file might not be a PostScript file.
%
       \if@psfile\else
       \expandafter\epsf@test\epsf@fileline:. \\%
       \fi
%
%   We check to see if the first character is a % sign;
%   if so, we look further and stop only if the line begins with
%   `%%BoundingBox:' and the `(atend)' specification was not found.
%   That is, the only way to stop is when the end of file is reached,
%   or a `%%BoundingBox: llx lly urx ury' line is found.
%
          \expandafter\epsf@aux\epsf@fileline:. \\%
       \fi
   \ifnot@eof\repeat
   }\closein\ps@stream\fi}%
%
% This tests if the file we are reading looks like a PostScript file.
%
\long\def\epsf@test#1#2#3:#4\\{\def\epsf@testit{#1#2}
			\ifx\epsf@testit\epsf@start\else
\ps@typeout{Warning! File does not start with `\epsf@start'.  It may not be a PostScript file.}
			\fi
			\@psfiletrue} % don't test after 1st line
%
%   We still need to define the tricky \epsf@aux macro. This requires
%   a couple of magic constants for comparison purposes.
%
{\catcode`\%=12\global\let\epsf@percent=%\global\def\epsf@bblit{%BoundingBox}}
%
%
%   So we're ready to check for `%BoundingBox:' and to grab the
%   values if they are found.  We continue searching if `(at end)'
%   was found after the `%BoundingBox:'.
%
\long\def\epsf@aux#1#2:#3\\{\ifx#1\epsf@percent
   \def\epsf@testit{#2}\ifx\epsf@testit\epsf@bblit
	\@atendfalse
        \epsf@atend #3 . \\%
	\if@atend	
	   \if@verbose{
		\ps@typeout{psfig: found `(atend)'; continuing search}
	   }\fi
        \else
        \epsf@grab #3 . . . \\%
        \not@eoffalse
        \global\no@bbfalse
        \fi
   \fi\fi}%
%
%   Here we grab the values and stuff them in the appropriate definitions.
%
\def\epsf@grab #1 #2 #3 #4 #5\\{%
   \global\def\epsf@llx{#1}\ifx\epsf@llx\empty
      \epsf@grab #2 #3 #4 #5 .\\\else
   \global\def\epsf@lly{#2}%
   \global\def\epsf@urx{#3}\global\def\epsf@ury{#4}\fi}%
%
% Determine if the stuff following the %%BoundingBox is `(atend)'
% J. Daniel Smith.  Copied from \epsf@grab above.
%
\def\epsf@atendlit{(atend)} 
\def\epsf@atend #1 #2 #3\\{%
   \def\epsf@tmp{#1}\ifx\epsf@tmp\empty
      \epsf@atend #2 #3 .\\\else
   \ifx\epsf@tmp\epsf@atendlit\@atendtrue\fi\fi}


% End of file reading stuff from epsf.tex
%%%%%%%%%%%%%%%%%%%%%%%%%%%%%%%%%%%%%%%%%%%%%%%%%%%%%%%%%%%%%%%%%%%

%%%%%%%%%%%%%%%%%%%%%%%%%%%%%%%%%%%%%%%%%%%%%%%%%%%%%%%%%%%%%%%%%%%
% trigonometry stuff from "trig.tex"
\chardef\psletter = 11 % won't conflict with \begin{letter} now...
\chardef\other = 12

\newif \ifdebug %%% turn me on to see TeX hard at work ...
\newif\ifc@mpute %%% don't need to compute some values
\c@mputetrue % but assume that we do

\let\then = \relax
\def\r@dian{pt }
\let\r@dians = \r@dian
\let\dimensionless@nit = \r@dian
\let\dimensionless@nits = \dimensionless@nit
\def\internal@nit{sp }
\let\internal@nits = \internal@nit
\newif\ifstillc@nverging
\def \Mess@ge #1{\ifdebug \then \message {#1} \fi}

{ %%% Things that need abnormal catcodes %%%
	\catcode `\@ = \psletter
	\gdef \nodimen {\expandafter \n@dimen \the \dimen}
	\gdef \term #1 #2 #3%
	       {\edef \t@ {\the #1}%%% freeze parameter 1 (count, by value)
		\edef \t@@ {\expandafter \n@dimen \the #2\r@dian}%
				   %%% freeze parameter 2 (dimen, by value)
		\t@rm {\t@} {\t@@} {#3}%
	       }
	\gdef \t@rm #1 #2 #3%
	       {{%
		\count 0 = 0
		\dimen 0 = 1 \dimensionless@nit
		\dimen 2 = #2\relax
		\Mess@ge {Calculating term #1 of \nodimen 2}%
		\loop
		\ifnum	\count 0 < #1
		\then	\advance \count 0 by 1
			\Mess@ge {Iteration \the \count 0 \space}%
			\Multiply \dimen 0 by {\dimen 2}%
			\Mess@ge {After multiplication, term = \nodimen 0}%
			\Divide \dimen 0 by {\count 0}%
			\Mess@ge {After division, term = \nodimen 0}%
		\repeat
		\Mess@ge {Final value for term #1 of 
				\nodimen 2 \space is \nodimen 0}%
		\xdef \Term {#3 = \nodimen 0 \r@dians}%
		\aftergroup \Term
	       }}
	\catcode `\p = \other
	\catcode `\t = \other
	\gdef \n@dimen #1pt{#1} %%% throw away the ``pt''
}

\def \Divide #1by #2{\divide #1 by #2} %%% just a synonym

\def \Multiply #1by #2%%% allows division of a dimen by a dimen
       {{%%% should really freeze parameter 2 (dimen, passed by value)
	\count 0 = #1\relax
	\count 2 = #2\relax
	\count 4 = 65536
	\Mess@ge {Before scaling, count 0 = \the \count 0 \space and
			count 2 = \the \count 2}%
	\ifnum	\count 0 > 32767 %%% do our best to avoid overflow
	\then	\divide \count 0 by 4
		\divide \count 4 by 4
	\else	\ifnum	\count 0 < -32767
		\then	\divide \count 0 by 4
			\divide \count 4 by 4
		\else
		\fi
	\fi
	\ifnum	\count 2 > 32767 %%% while retaining reasonable accuracy
	\then	\divide \count 2 by 4
		\divide \count 4 by 4
	\else	\ifnum	\count 2 < -32767
		\then	\divide \count 2 by 4
			\divide \count 4 by 4
		\else
		\fi
	\fi
	\multiply \count 0 by \count 2
	\divide \count 0 by \count 4
	\xdef \product {#1 = \the \count 0 \internal@nits}%
	\aftergroup \product
       }}

\def\r@duce{\ifdim\dimen0 > 90\r@dian \then   % sin(x+90) = sin(180-x)
		\multiply\dimen0 by -1
		\advance\dimen0 by 180\r@dian
		\r@duce
	    \else \ifdim\dimen0 < -90\r@dian \then  % sin(-x) = sin(360+x)
		\advance\dimen0 by 360\r@dian
		\r@duce
		\fi
	    \fi}

\def\Sine#1%
       {{%
	\dimen 0 = #1 \r@dian
	\r@duce
	\ifdim\dimen0 = -90\r@dian \then
	   \dimen4 = -1\r@dian
	   \c@mputefalse
	\fi
	\ifdim\dimen0 = 90\r@dian \then
	   \dimen4 = 1\r@dian
	   \c@mputefalse
	\fi
	\ifdim\dimen0 = 0\r@dian \then
	   \dimen4 = 0\r@dian
	   \c@mputefalse
	\fi
%
	\ifc@mpute \then
        	% convert degrees to radians
		\divide\dimen0 by 180
		\dimen0=3.141592654\dimen0
%
		\dimen 2 = 3.1415926535897963\r@dian %%% a well-known constant
		\divide\dimen 2 by 2 %%% we only deal with -pi/2 : pi/2
		\Mess@ge {Sin: calculating Sin of \nodimen 0}%
		\count 0 = 1 %%% see power-series expansion for sine
		\dimen 2 = 1 \r@dian %%% ditto
		\dimen 4 = 0 \r@dian %%% ditto
		\loop
			\ifnum	\dimen 2 = 0 %%% then we've done
			\then	\stillc@nvergingfalse 
			\else	\stillc@nvergingtrue
			\fi
			\ifstillc@nverging %%% then calculate next term
			\then	\term {\count 0} {\dimen 0} {\dimen 2}%
				\advance \count 0 by 2
				\count 2 = \count 0
				\divide \count 2 by 2
				\ifodd	\count 2 %%% signs alternate
				\then	\advance \dimen 4 by \dimen 2
				\else	\advance \dimen 4 by -\dimen 2
				\fi
		\repeat
	\fi		
			\xdef \sine {\nodimen 4}%
       }}

% Now the Cosine can be calculated easily by calling \Sine
\def\Cosine#1{\ifx\sine\UnDefined\edef\Savesine{\relax}\else
		             \edef\Savesine{\sine}\fi
	{\dimen0=#1\r@dian\advance\dimen0 by 90\r@dian
	 \Sine{\nodimen 0}
	 \xdef\cosine{\sine}
	 \xdef\sine{\Savesine}}}	      
% end of trig stuff
%%%%%%%%%%%%%%%%%%%%%%%%%%%%%%%%%%%%%%%%%%%%%%%%%%%%%%%%%%%%%%%%%%%%

\def\psdraft{
	\def\@psdraft{0}
	%\ps@typeout{draft level now is \@psdraft \space . }
}
\def\psfull{
	\def\@psdraft{100}
	%\ps@typeout{draft level now is \@psdraft \space . }
}

\psfull

\newif\if@scalefirst
\def\psscalefirst{\@scalefirsttrue}
\def\psrotatefirst{\@scalefirstfalse}
\psrotatefirst

\newif\if@draftbox
\def\psnodraftbox{
	\@draftboxfalse
}
\def\psdraftbox{
	\@draftboxtrue
}
\@draftboxtrue

\newif\if@prologfile
\newif\if@postlogfile
\def\pssilent{
	\@noisyfalse
}
\def\psnoisy{
	\@noisytrue
}
\psnoisy
%%% These are for the option list.
%%% A specification of the form a = b maps to calling \@p@@sa{b}
\newif\if@bbllx
\newif\if@bblly
\newif\if@bburx
\newif\if@bbury
\newif\if@height
\newif\if@width
\newif\if@rheight
\newif\if@rwidth
\newif\if@angle
\newif\if@clip
\newif\if@verbose
\def\@p@@sclip#1{\@cliptrue}


\newif\if@decmpr

%%% GDH 7/26/87 -- changed so that it first looks in the local directory,
%%% then in a specified global directory for the ps file.
%%% RPR 6/25/91 -- changed so that it defaults to user-supplied name if
%%% boundingbox info is specified, assuming graphic will be created by
%%% print time.
%%% TJD 10/19/91 -- added bbfile vs. file distinction, and @decmpr flag

\def\@p@@sfigure#1{\def\@p@sfile{null}\def\@p@sbbfile{null}
	        \openin1=#1.bb
		\ifeof1\closein1
	        	\openin1=\figurepath#1.bb
			\ifeof1\closein1
			        \openin1=#1
				\ifeof1\closein1%
				       \openin1=\figurepath#1
					\ifeof1
					   \ps@typeout{Error, File #1 not found}
						\if@bbllx\if@bblly
				   		\if@bburx\if@bbury
			      				\def\@p@sfile{#1}%
			      				\def\@p@sbbfile{#1}%
							\@decmprfalse
				  	   	\fi\fi\fi\fi
					\else\closein1
				    		\def\@p@sfile{\figurepath#1}%
				    		\def\@p@sbbfile{\figurepath#1}%
						\@decmprfalse
	                       		\fi%
			 	\else\closein1%
					\def\@p@sfile{#1}
					\def\@p@sbbfile{#1}
					\@decmprfalse
			 	\fi
			\else
				\def\@p@sfile{\figurepath#1}
				\def\@p@sbbfile{\figurepath#1.bb}
				\@decmprtrue
			\fi
		\else
			\def\@p@sfile{#1}
			\def\@p@sbbfile{#1.bb}
			\@decmprtrue
		\fi}

\def\@p@@sfile#1{\@p@@sfigure{#1}}

\def\@p@@sbbllx#1{
		%\ps@typeout{bbllx is #1}
		\@bbllxtrue
		\dimen100=#1
		\edef\@p@sbbllx{\number\dimen100}
}
\def\@p@@sbblly#1{
		%\ps@typeout{bblly is #1}
		\@bbllytrue
		\dimen100=#1
		\edef\@p@sbblly{\number\dimen100}
}
\def\@p@@sbburx#1{
		%\ps@typeout{bburx is #1}
		\@bburxtrue
		\dimen100=#1
		\edef\@p@sbburx{\number\dimen100}
}
\def\@p@@sbbury#1{
		%\ps@typeout{bbury is #1}
		\@bburytrue
		\dimen100=#1
		\edef\@p@sbbury{\number\dimen100}
}
\def\@p@@sheight#1{
		\@heighttrue
		\dimen100=#1
   		\edef\@p@sheight{\number\dimen100}
		%\ps@typeout{Height is \@p@sheight}
}
\def\@p@@swidth#1{
		%\ps@typeout{Width is #1}
		\@widthtrue
		\dimen100=#1
		\edef\@p@swidth{\number\dimen100}
}
\def\@p@@srheight#1{
		%\ps@typeout{Reserved height is #1}
		\@rheighttrue
		\dimen100=#1
		\edef\@p@srheight{\number\dimen100}
}
\def\@p@@srwidth#1{
		%\ps@typeout{Reserved width is #1}
		\@rwidthtrue
		\dimen100=#1
		\edef\@p@srwidth{\number\dimen100}
}
\def\@p@@sangle#1{
		%\ps@typeout{Rotation is #1}
		\@angletrue
%		\dimen100=#1
		\edef\@p@sangle{#1} %\number\dimen100}
}
\def\@p@@ssilent#1{ 
		\@verbosefalse
}
\def\@p@@sprolog#1{\@prologfiletrue\def\@prologfileval{#1}}
\def\@p@@spostlog#1{\@postlogfiletrue\def\@postlogfileval{#1}}
\def\@cs@name#1{\csname #1\endcsname}
\def\@setparms#1=#2,{\@cs@name{@p@@s#1}{#2}}
%
% initialize the defaults (size the size of the figure)
%
\def\ps@init@parms{
		\@bbllxfalse \@bbllyfalse
		\@bburxfalse \@bburyfalse
		\@heightfalse \@widthfalse
		\@rheightfalse \@rwidthfalse
		\def\@p@sbbllx{}\def\@p@sbblly{}
		\def\@p@sbburx{}\def\@p@sbbury{}
		\def\@p@sheight{}\def\@p@swidth{}
		\def\@p@srheight{}\def\@p@srwidth{}
		\def\@p@sangle{0}
		\def\@p@sfile{} \def\@p@sbbfile{}
		\def\@p@scost{10}
		\def\@sc{}
		\@prologfilefalse
		\@postlogfilefalse
		\@clipfalse
		\if@noisy
			\@verbosetrue
		\else
			\@verbosefalse
		\fi
}
%
% Go through the options setting things up.
%
\def\parse@ps@parms#1{
	 	\@psdo\@psfiga:=#1\do
		   {\expandafter\@setparms\@psfiga,}}
%
% Compute bb height and width
%
\newif\ifno@bb
\def\bb@missing{
	\if@verbose{
		\ps@typeout{psfig: searching \@p@sbbfile \space  for bounding box}
	}\fi
	\no@bbtrue
	\epsf@getbb{\@p@sbbfile}
        \ifno@bb \else \bb@cull\epsf@llx\epsf@lly\epsf@urx\epsf@ury\fi
}	
\def\bb@cull#1#2#3#4{
	\dimen100=#1 bp\edef\@p@sbbllx{\number\dimen100}
	\dimen100=#2 bp\edef\@p@sbblly{\number\dimen100}
	\dimen100=#3 bp\edef\@p@sbburx{\number\dimen100}
	\dimen100=#4 bp\edef\@p@sbbury{\number\dimen100}
	\no@bbfalse
}
% rotate point (#1,#2) about (0,0).
% The sine and cosine of the angle are already stored in \sine and
% \cosine.  The result is placed in (\p@intvaluex, \p@intvaluey).
\newdimen\p@intvaluex
\newdimen\p@intvaluey
\def\rotate@#1#2{{\dimen0=#1 sp\dimen1=#2 sp
%            	calculate x' = x \cos\theta - y \sin\theta
		  \global\p@intvaluex=\cosine\dimen0
		  \dimen3=\sine\dimen1
		  \global\advance\p@intvaluex by -\dimen3
% 		calculate y' = x \sin\theta + y \cos\theta
		  \global\p@intvaluey=\sine\dimen0
		  \dimen3=\cosine\dimen1
		  \global\advance\p@intvaluey by \dimen3
		  }}
\def\compute@bb{
		\no@bbfalse
		\if@bbllx \else \no@bbtrue \fi
		\if@bblly \else \no@bbtrue \fi
		\if@bburx \else \no@bbtrue \fi
		\if@bbury \else \no@bbtrue \fi
		\ifno@bb \bb@missing \fi
		\ifno@bb \ps@typeout{FATAL ERROR: no bb supplied or found}
			\no-bb-error
		\fi
		%
%\ps@typeout{BB: \@p@sbbllx, \@p@sbblly, \@p@sbburx, \@p@sbbury} 
%
% store height/width of original (unrotated) bounding box
		\count203=\@p@sbburx
		\count204=\@p@sbbury
		\advance\count203 by -\@p@sbbllx
		\advance\count204 by -\@p@sbblly
		\edef\ps@bbw{\number\count203}
		\edef\ps@bbh{\number\count204}
		%\ps@typeout{ psbbh = \ps@bbh, psbbw = \ps@bbw }
		\if@angle 
			\Sine{\@p@sangle}\Cosine{\@p@sangle}
	        	{\dimen100=\maxdimen\xdef\r@p@sbbllx{\number\dimen100}
					    \xdef\r@p@sbblly{\number\dimen100}
			                    \xdef\r@p@sbburx{-\number\dimen100}
					    \xdef\r@p@sbbury{-\number\dimen100}}
%
% Need to rotate all four points and take the X-Y extremes of the new
% points as the new bounding box.
                        \def\minmaxtest{
			   \ifnum\number\p@intvaluex<\r@p@sbbllx
			      \xdef\r@p@sbbllx{\number\p@intvaluex}\fi
			   \ifnum\number\p@intvaluex>\r@p@sbburx
			      \xdef\r@p@sbburx{\number\p@intvaluex}\fi
			   \ifnum\number\p@intvaluey<\r@p@sbblly
			      \xdef\r@p@sbblly{\number\p@intvaluey}\fi
			   \ifnum\number\p@intvaluey>\r@p@sbbury
			      \xdef\r@p@sbbury{\number\p@intvaluey}\fi
			   }
%			lower left
			\rotate@{\@p@sbbllx}{\@p@sbblly}
			\minmaxtest
%			upper left
			\rotate@{\@p@sbbllx}{\@p@sbbury}
			\minmaxtest
%			lower right
			\rotate@{\@p@sbburx}{\@p@sbblly}
			\minmaxtest
%			upper right
			\rotate@{\@p@sbburx}{\@p@sbbury}
			\minmaxtest
			\edef\@p@sbbllx{\r@p@sbbllx}\edef\@p@sbblly{\r@p@sbblly}
			\edef\@p@sbburx{\r@p@sbburx}\edef\@p@sbbury{\r@p@sbbury}
%\ps@typeout{rotated BB: \r@p@sbbllx, \r@p@sbblly, \r@p@sbburx, \r@p@sbbury}
		\fi
		\count203=\@p@sbburx
		\count204=\@p@sbbury
		\advance\count203 by -\@p@sbbllx
		\advance\count204 by -\@p@sbblly
		\edef\@bbw{\number\count203}
		\edef\@bbh{\number\count204}
		%\ps@typeout{ bbh = \@bbh, bbw = \@bbw }
}
%
% \in@hundreds performs #1 * (#2 / #3) correct to the hundreds,
%	then leaves the result in @result
%
\def\in@hundreds#1#2#3{\count240=#2 \count241=#3
		     \count100=\count240	% 100 is first digit #2/#3
		     \divide\count100 by \count241
		     \count101=\count100
		     \multiply\count101 by \count241
		     \advance\count240 by -\count101
		     \multiply\count240 by 10
		     \count101=\count240	%101 is second digit of #2/#3
		     \divide\count101 by \count241
		     \count102=\count101
		     \multiply\count102 by \count241
		     \advance\count240 by -\count102
		     \multiply\count240 by 10
		     \count102=\count240	% 102 is the third digit
		     \divide\count102 by \count241
		     \count200=#1\count205=0
		     \count201=\count200
			\multiply\count201 by \count100
		 	\advance\count205 by \count201
		     \count201=\count200
			\divide\count201 by 10
			\multiply\count201 by \count101
			\advance\count205 by \count201
			%
		     \count201=\count200
			\divide\count201 by 100
			\multiply\count201 by \count102
			\advance\count205 by \count201
			%
		     \edef\@result{\number\count205}
}
\def\compute@wfromh{
		% computing : width = height * (bbw / bbh)
		\in@hundreds{\@p@sheight}{\@bbw}{\@bbh}
		%\ps@typeout{ \@p@sheight * \@bbw / \@bbh, = \@result }
		\edef\@p@swidth{\@result}
		%\ps@typeout{w from h: width is \@p@swidth}
}
\def\compute@hfromw{
		% computing : height = width * (bbh / bbw)
	        \in@hundreds{\@p@swidth}{\@bbh}{\@bbw}
		%\ps@typeout{ \@p@swidth * \@bbh / \@bbw = \@result }
		\edef\@p@sheight{\@result}
		%\ps@typeout{h from w : height is \@p@sheight}
}
\def\compute@handw{
		\if@height 
			\if@width
			\else
				\compute@wfromh
			\fi
		\else 
			\if@width
				\compute@hfromw
			\else
				\edef\@p@sheight{\@bbh}
				\edef\@p@swidth{\@bbw}
			\fi
		\fi
}
\def\compute@resv{
		\if@rheight \else \edef\@p@srheight{\@p@sheight} \fi
		\if@rwidth \else \edef\@p@srwidth{\@p@swidth} \fi
		%\ps@typeout{rheight = \@p@srheight, rwidth = \@p@srwidth}
}
%		
% Compute any missing values
\def\compute@sizes{
	\compute@bb
	\if@scalefirst\if@angle
% at this point the bounding box has been adjsuted correctly for
% rotation.  PSFIG does all of its scaling using \@bbh and \@bbw.  If
% a width= or height= was specified along with \psscalefirst, then the
% width=/height= value needs to be adjusted to match the new (rotated)
% bounding box size (specifed in \@bbw and \@bbh).
%    \ps@bbw       width=
%    -------  =  ---------- 
%    \@bbw       new width=
% so `new width=' = (width= * \@bbw) / \ps@bbw; where \ps@bbw is the
% width of the original (unrotated) bounding box.
	\if@width
	   \in@hundreds{\@p@swidth}{\@bbw}{\ps@bbw}
	   \edef\@p@swidth{\@result}
	\fi
	\if@height
	   \in@hundreds{\@p@sheight}{\@bbh}{\ps@bbh}
	   \edef\@p@sheight{\@result}
	\fi
	\fi\fi
	\compute@handw
	\compute@resv}

%
% \psfig
% usage : \psfig{file=, height=, width=, bbllx=, bblly=, bburx=, bbury=,
%			rheight=, rwidth=, clip=}
%
% "clip=" is a switch and takes no value, but the `=' must be present.
\def\psfig#1{\vbox {
	% do a zero width hard space so that a single
	% \psfig in a centering enviornment will behave nicely
	%{\setbox0=\hbox{\ }\ \hskip-\wd0}
	%
	\ps@init@parms
	\parse@ps@parms{#1}
	\compute@sizes
	%
	\ifnum\@p@scost<\@psdraft{
		%
		\special{ps::[begin] 	\@p@swidth \space \@p@sheight \space
				\@p@sbbllx \space \@p@sbblly \space
				\@p@sbburx \space \@p@sbbury \space
				startTexFig \space }
		\if@angle
			\special {ps:: \@p@sangle \space rotate \space} 
		\fi
		\if@clip{
			\if@verbose{
				\ps@typeout{(clip)}
			}\fi
			\special{ps:: doclip \space }
		}\fi
		\if@prologfile
		    \special{ps: plotfile \@prologfileval \space } \fi
		\if@decmpr{
			\if@verbose{
				\ps@typeout{psfig: including \@p@sfile.Z \space }
			}\fi
			\special{ps: plotfile "`zcat \@p@sfile.Z" \space }
		}\else{
			\if@verbose{
				\ps@typeout{psfig: including \@p@sfile \space }
			}\fi
			\special{ps: plotfile \@p@sfile \space }
		}\fi
		\if@postlogfile
		    \special{ps: plotfile \@postlogfileval \space } \fi
		\special{ps::[end] endTexFig \space }
		% Create the vbox to reserve the space for the figure.
		\vbox to \@p@srheight sp{
		% 1/92 TJD Changed from "true sp" to "sp" for magnification.
			\hbox to \@p@srwidth sp{
				\hss
			}
		\vss
		}
	}\else{
		% draft figure, just reserve the space and print the
		% path name.
		\if@draftbox{		
			% Verbose draft: print file name in box
			\hbox{\frame{\vbox to \@p@srheight sp{
			\vss
			\hbox to \@p@srwidth sp{ \hss \@p@sfile \hss }
			\vss
			}}}
		}\else{
			% Non-verbose draft
			\vbox to \@p@srheight sp{
			\vss
			\hbox to \@p@srwidth sp{\hss}
			\vss
			}
		}\fi	



	}\fi
}}
\psfigRestoreAt
\let\@=\LaTeXAtSign





\begin{document}
\title{Exploring Strong Collaboration with AEN}
\author{Carleton Moore\\
Collaborative Software Development Laboratory,\\
Department of Information and Computer Sciences\\
2565 The Mall\\
University of Hawaii, Manoa\\
Honolulu, Hawaii   96822\\
{\tt cmoore@uhics.ics.hawaii.edu}}
\maketitle
\abstract

This paper overviews the Annotated Egret Navigator (AEN), a system designed
to support {\em strong collaboration} among a group as they cooperatively
build, review, revise, improve and learn from a structured hypertext
document.  AEN addresses how strong collaboration can be supported through
computer mediation.  It is designed to support collaborative creation of
hypertext and to instrument the actions of its users in order to understand
how such creation occurs.

\paragraph{KEYWORDS:} Hypertext, Strong Collaboration, Authoring
\section{INTRODUCTION}

With the rapid expansion of the World Wide Web (WWW) \cite{Berners-Lee94},
the use of hypertext as a structuring mechanism for richly interdependent
information is becoming widespread.  Indeed, hypertext is almost certain to
be the `lingua franca' of the information superhighway.  The WWW allows
authors to collaborate by including sections of other authors' documents.
Unfortunately, WWW provides little support for collaborative authoring
beyond including someone else's work.

The issues associated with collaborative authoring have been investigated
by many researchers.  Baecker et. al. \cite{Baecker93} provide several
requirements for collaborative writing systems.  They include: preservation
of identities, enhanced communications, enhanced collaborator awareness,
annotations, undo, session control, explicit roles, variety of activities,
transition between activities, several document access methods, separate
document segments, version control, one or several writers, and synchronous
and asynchronous writing.

This paper reports on my investigation into the issues related to
collaborative hypertext document construction.  Specifically, my
investigation and my characterization of a form of collaboration termed
{\em strong collaboration}.  In strong collaboration, each participant
contributes to the construction of the document, and gains new knowledge as
a direct result of this construction.  The constructed document is not
simply a patchwork of individual contributions, but instead, an
incremental, emergent synthesis that reflects the knowledge created by the
group as a whole.  In general, strong collaboration most often (though not
always) occurs in non-computer mediated, face-to-face contexts such as a
group software design session or an interactive classroom setting.  A
distinguishing characteristic of strong collaboration is a sense of
collective authorship over most, if not all of the components of the
document.  Documents produced in this way should be of higher quality,
because more perspectives have combined to produce, review, and correct
errors in it.

Strong collaboration is at one end of a collaborative spectrum.  I call
the other end of the spectrum {\em weak collaboration}.  Weak collaboration
uses a divide and conquer process --- participants divide up the document
and work independently on their section.  There is very little interaction
between the participants.  When all of the sections are finished, the
document is created by combining the sections into a whole.  A
distinguishing characteristic of weak collaboration is distinct authorship
of components of the document.  

Computational support for weak collaboration is well established.  For
example, word processors or editors allow different authors to create
subsections of a document and combine them together.  E-mail can deliver
different sections of a document to an editor who compiles them.  In
contrast, the computational needs of strong collaboration are not well
established.  For example, very few editors allow more than one author to
edit the same document at the same time.  In order to understand strong
collaboration, I implemented a collaborative hypertext authoring
environment called the Annotated Egret\footnote{Exploratory Group Research
EnvironmenT} Navigator (AEN).  This environment provides tools designed to
encourage strong collaboration and measurement facilities designed to
detect strong collaboration, if it were to occur.

AEN is built upon Egret \cite{csdl-93-09}, a client-server hypertext
database manager.  Authors/learners use AEN's typed nodes and links to give
the hypertext document structure.  They also use an access control model to
dynamically restrict the type of access to the individual nodes in the
database.  Several autonomous agents help AEN manage all of the information
needed to support strong collaboration.

I conducted a case study during the Spring of 1995 to answer two questions:
(1) Does AEN support strong collaboration? and (2) What computational
support is needed for strong collaboration?

One approach to answering these question would have been a controlled,
experimental study where the interactions, and authoring styles of a group
using AEN is compared to that of a group not using AEN.  However,
methodological and resource problems prevented that approach.  Instead, I
decided to first operationalize the definition of strong collaboration as a
set of measurable collaborative behaviors, then test to see to what degree
a group of AEN users exhibit those behaviors.  

\section{Operationalized Definition of Strong Collaboration}
\label{sec:operationalized-definition}
 
The operationalized definition of strong collaboration consists of four
classes of observable collaborative behaviors.  For each behavior, I have
defined one or more metrics.

\begin{enumerate}
\item{\em Members read each other's nodes.}  Any form of collaboration
  requires that the members of the group know what other members have
  contributed. 
  
  % Without reading the other nodes members have no way of
  %  contributing to the nodes.
  
  {\bf Metric:} {\em Readers per node (RPN)}.  This metric calculates the
  average number of readers per node for a group. For example, for a
  group of five, the lowest RPN value is one, indicating that no member
  of the group read any nodes other than those he created.  The highest
  possible RPN value for this group of five is five, indicating that every member
  read every node.
  
  
  
%\item{\em Members edit nodes that were also edited by others.}  This
%  is the most direct way that members can collaborate on a node.  One author
%  creates a node, then another author adds to it.
%  
%  {\bf Metric:} {\em Member Co-editing (MCE)}.  This metric calculates
%  the average number of members that each member co-edits with. For a
%  group of five, the lowest MCE value is one, indicating that no member
%  of the group edited nodes that any other member edited.  The highest
%  possible MCE value for this group is five, indicating that every member
%  edited every node with every other member.

\item{\em Document nodes are edited by more than one person.}  This
  component of the operationalized definition assesses the degree to
  which multiple authorship occurs.  Only document nodes are included in
  this assessment, since other node types (comment, quicky quiz, and
  quicky quiz answers) are by definition singly authored.  

  {\bf Metric:} {\em Editors per node (EPN)}.  This metric calculates the
  average number of editors per node.  The lowest EPN value possible, for
  a group of five, is one, indicating that each node was only edited by
  one member.  The highest possible EPN value for this group is five,
  indicating that every node was edited by all five group members.

\item{\em Members create feedback nodes.}  This is another way for group
  members to collaborate.  They can create feedback on the contents of a
  node.

  {\bf Metric:} {\em Feedback Node Creation (FNC)}.  This metric
  calculates the percentage of document nodes that have been commented
  on.  The lowest possible FNC value is 0\%, indicating that no document
  nodes were commented on.  The highest FNC value is 100\%, indicating that
  all document nodes were commented on.


\item{\em Members manipulate access control to publish/protect documents
  under development.}  By changing the access control members are able to
  dynamically control the process of collaboration.  Changing the access
  control from the default allows collaboration on a node.  Changing the
  access control for a node more than once indicates that the mode of
  collaboration changed for that node.

  {\bf Metric:} {\em Non-default Access Control (NAC)}.  This metric
  calculates the percentage of nodes that have had their access control
  changed from the default.  The lowest NAC value possible is 0\%, indicating
  that no nodes had their access control changed from the default.
  This also indicates that there was no interaction among the members
  since the default access control is no access for other members.  The
  highest possible NAC value is 100\%, indicating that every node has had its
  access control modified.

  {\bf Metric:} {\em Evolving Access Control (EAC)}.  This metric
  calculates the percentage of document nodes that have had their access
  control changed more than once. This metric measures the degree to
  which the groups' collaboration changed. The lowest EAC value possible
  is 0\%, indicating that no nodes had their access control changed more
  than once.  The highest possible EAC value is 100\%, indicating that
  all nodes had their access control changed more than once.

\end{enumerate}


These metrics characterize many aspects of collaborative behaviors.
Specifically, a group that exhibits uniformly high values for each of these
metrics can be said to be exhibiting ``pure'' strong collaboration.  Of
course, intermediate levels of these metrics still indicate at least a
partial presence of strong collaboration.  In order to evaluate
AEN's set of tools, tool use will be monitored and a post study survey will
ask the participants their impressions of the tools.



%\small
%\begin{table}[htb]
%  \caption{Summary of Operationalized Definitions and Metrics.}
%  \begin{center}
%    \begin{tabular}{|l|l|}
%      \hline
%      {\rule[-3mm]{0mm}{8mm}{\bf Operationalized
%      Definition }}& {\rule[-3mm]{0mm}{8mm}{\bf Corresponding Metric(s)}}\\ \hline
%      \hline
%      Members read each other's nodes&Readers per node (RPN)\\ \hline
%%      Members edit nodes that were also edited by others&Member Co-editing
%%      (MCE)\\ \hline
%      Document nodes are edited by more&Editors per node
%      (EPN)\\
%      than one person&\\\hline
%      Members create feedback nodes&Feedback Node Creation (FNC)\\ \hline
%      Members manipulate access control &Non-default
%      Access Control (NAC) \&\\
%      to publish/protect documents &Evolving Access Control (EAC)\\ 
%      under development&\\\hline
%    \end{tabular}
%    \label{tab:op-metrics}
%  \end{center}
%\end{table}
%\normalsize

%In order to detect strong collaboration, I developed a metric for each
%component of the operationalized definition of strong collaboration. Table
%\ref{tab:op-metrics} summarizes the operationalized definitions and their
%corresponding metrics.

I used these collaborative metrics to detect occurrences of 
collaborative behaviors.  In order to evaluate AEN's set of tools, tool use
was monitored and a post study survey asked the participants their
impressions of the tools.

The data analysis reveals that a high score for each metric was recorded by
at least one of the groups studied.  However, no single group scored high
on all five metrics.  These results indicate that AEN does support the
various components of strong collaboration, but the results do not
demonstrate that AEN allows a single group to collaborate strongly with
respect to all metrics simultaneously.  Section \ref{sec:evaluation} explores
the results of the case study in detail.

The remainder of this paper is organized as follows.  Section
\ref{sec:related-work} examines other collaborative learning and authoring
systems.  Section \ref{sec:AEN} introduces AEN.  Section
\ref{sec:evaluation} is a discussion of how I evaluated AEN's design and
its support for strong collaboration.  Section \ref{sec:conclusions}
presents some conclusions and recommendations about strong collaboration.
%Section \ref{sec:future} discusses future directions for AEN.


\section{Related Work}
\label{sec:related-work}

This section surveys major existing hypertext or collaborative editors,
with a comparison of AEN to these systems.  Subsection \ref{sec:collab-sys}
discusses different collaborative writing systems.  Subsection
\ref{sec:hypermedia} compares several hypermedia systems with AEN.

AEN's set of features were designed to explore and support strong
collaboration. Subsection \ref{sec:mechanisms} provides a detailed description
of AEN's features.  AEN provides a typed hypertext document, dynamic user
defined table of contents, real-time communications with other users of the
system, real-time knowledge of what the other users of the system are doing,
and access control for each node.  Other collaborative writing systems or
hypermedia systems provide different combinations of features.

\subsection{Collaborative Writing Systems}
\label{sec:collab-sys}

These systems support the process of collaborative writing.  Not all of the
systems in this subsection are pure collaborative writing system, but they all
support the process of collaborative writing.  I have divided the
collaborative writing systems into three categories, weakly collaborative,
additive, and synchronous.

The weakly collaborative systems are MILO \cite{Jones93} and NSF Express
\cite{Greenberg91}.  They appear to only support the combination of
different types of documents from different authors.  They do not appear to
support co-editing of other authors' work while they are still working on
it.

The additive systems are CLARE \cite{csdl-93-21}, ConversationBuilder
\cite{Kaplan92}, ForComment \cite{Ellis91,Opper88}, InterNote
\cite{Malcolm91}, gIBIS \cite{Conklin88}, and MarkUp! \cite{Allen93}.
These systems allow authors to collaboratively add to a document.  Three
different systems, CLARE, ConversationBuilder and gIBIS will be discussed
briefly.  They provide a good cross section of these types of tools.

The last category of collaborative writing systems, is synchronous.  The
following are examples of this category of system: Aspects
\cite{Allen93,Baecker93}, Augment \cite{Engelbart84}, CES
\cite{Ellis91,Greif87}, CoAUTHOR \cite{Bowers91}, CoEd \cite{Ellis89},
COOPerator \cite{Michels95}, EHTS: Emacs HyperText System \cite{Wiil92},
GroupWriter \cite{Malcolm91,Malcolm93}, PREP \cite{Neuwirth90,Neuwirth92},
Quilt \cite{Fish88}, SASE and SASSE \cite{Baecker93}, SEPIA \cite{Haake92},
and ShrEdit \cite{Cogn92}.  All of these systems appear to support
synchronous editing of documents by multiple authors.  In general, they do
not support different modes of access control.  Many do have a presence for
the authors of the document. 

\subsection{Hypermedia Systems}
\label{sec:hypermedia}

This subsection reviews several other hypermedia systems that are not directly
oriented toward collaborative writing.  These systems include: CoMEdiA
\cite{Santos92a,Santos92b,Santos93a,Santos93b}, Intermedia
\cite{Conklin87},  NoteCards \cite{Halasz87},  The
Virtual Notebook System \cite{Shipman89}, and the World Wide Web
\cite{Berners-Lee94}.

\subsection{Summary}

One key feature that distinguishes AEN from the other systems is its
specific intent to support strong collaboration and to be a tool to help
investigate collaborative construction of hypertext documents.  AEN's set of
features is guided by the desire to support and explore collaboration.




\section{AEN}
\label{sec:AEN}


\subsection{Architecture of AEN}
\label{sec:architectureAEN}

AEN is a collaborative, instrumented learning/authoring tool implemented
with the Egret collaborative system development
environment\cite{csdl-93-09}.  As with all Egret domain-specific
applications, AEN is a client-server system running over TCP/IP, where
clients (implemented as customized versions of XEmacs) interact with a
central server called HBS (implemented in C++) to store and retrieve
information or communicate with other clients.  AEN makes substantial use
of Egret's facilities for agents, or autonomous client processes that
extend the system's information processing capabilities beyond those
provided by HBS or the clients.  Figure \ref{fig:aen-architecture}
illustrates AEN's basic client-server architecture.

\begin{figure}[htb]
  \centerline{\psfig{figure=client-server.eps,width=3.5in}} 
  {\em AEN consists of a central server called HBS that provides basic
  storage and concurrency control mechanisms, augmented with several agent
  processes to support strong collaboration.}
 \caption{AEN's architecture from an OS process perspective.}
\label{fig:aen-architecture}
\end{figure}

\subsection{The design of AEN}
\label{sec:designAEN}
\subsubsection{Structure of Data in AEN}

AEN provides facilities for storage, retrieval, and analysis of a hypertext
network of node and links. The hypertext network represents the text from
which the students learn and the product of collaboration among the
participants.  AEN provides a static, predefined set of node and link
types.

\paragraph{Node Types}

At the user-level, AEN provides two type hierarchies of nodes: artifact
nodes and figure nodes.  Nodes of type {\em artifact} represent textual
information in the hypertext document.  Nodes of type {\em figure}
represent graphical information.

Artifact nodes contain ETML\footnote{A primary difference between the Egret
Text Markup Language (ETML) and HTML is that ETML supports {\tt egret::}
URL specifier for link and node objects within an HBS server.} code that
provides formating directives and inter-node links.  There are four
subclasses of artifacts:

\begin{itemize}

\item{\bf Document:} Nodes that contain the text of the document created by
the participants.

\item{\bf Comment:} Nodes containing the reactions of the participants to
the contents of other nodes.

\item{\bf Quicky Quiz:} Nodes that contain exercises for the participants
to complete.

\item{\bf Quicky Quiz Answer:} Nodes that contain the participant's
solutions to the quicky quiz nodes.

\end{itemize}

Quicky Quiz and Quicky Quiz Answer nodes are used in AEN to support
learning.  They can be ignored for collaborative authoring.

%\begin{figure}[htb]
%  \centerline{\psfig{figure=Data.eps,width=3.5in}}
%  \caption{Data Relationships in AEN}
%  \label{fig:Data}
%\end{figure}

Unlike artifact nodes, figure nodes contain the graphics data used by Xview
to display the figures.  Users can create links from artifact nodes to
figure nodes or other artifact nodes by a simple menu-based command.  Such
an operation results in both the creation of a new link object within the
server and the creation of an ETML link anchor as part of the contents of
the artifact node.  However, since figure nodes contain graphical
information, compatible with Xview, rather than ETML code, users cannot
create links from figure nodes. % Figure \ref{fig:Data} shows the
%relationships between the node types.

\paragraph{Link Types}

AEN defines six types of links: Include, Xref, see\_Quicky\_\-Quiz,
see\_\-Quicky\_\-Quiz\_\-Answer, see\_\-Comment, and see\_Figure.  Include
Links link two document nodes together and forms the Backbone structure of
the hypertext document.  Xref Links also link two document nodes together
but are used for cross references not Backbone structure.  See\_Comment
Links link a comment node to a document node.  See\_Quicky\_Quiz Links link
a Quicky Quiz node to a document node.  See\_Quicky\_Quiz\_Answer Links
link Quicky Quiz Answers to Quicky Quizzes.  AEN restricts the type of the
source and destination nodes for each type of link.  The different types of
links allows the authors to create different
structures in the hypertext document. 

\paragraph{The AEN Backbone}
\label{sec:aen-backbone}

Given the set of node and link types in AEN, there are many possible ways
to organize the resulting hypertext network.  However, given the intended
usage of AEN, it is natural and appropriate to provide a fundamental
organization of the hypertext as an ordered sequence of interlinked,
hypertext ``chapters'', each hypertext chapter having as its root a single
document node.

Each document node with a link to it from a chapter node represents a
``section'' within that chapter. In the same manner, each document node
with a link to it from a section node represents a subsection within that
section, and so forth ad infinitum.

Any document node can, of course, have a link to it from more than one
other document node in the hypertext, with the result that the same node
can be a ``section'' with respect to one node, but a ``sub-subsection''
with respect to another.  As a result, the AEN hypertext is organized as a
directed, hierarchical, but potentially cyclic graph structure.

So far, this description seems to indicate that there is a one-to-one
correspondence between each document node and each chapter, section,
subsection, and so forth in the document.  The reality is slightly more
complex.  Each document node does contribute at least one new chapter, or
section, or subsection, etc. to the hypertext.  However, if a document
node contains the HTML {\em heading} formatting directive, then the label for
this heading indicates a new section (or subsection, etc. depending upon
context) within the node.

To bootstrap this organization, some distinguished ``root'' document node
is required, which I call the AEN Backbone.  This node contains a set of
links to document nodes which define the set of top-level chapters in AEN.

The AEN Backbone and the nested structure of document nodes defines the
major organizing principle for the AEN hypertext, and leads to the first
data manipulation tool: the table of contents screen, as shown in Figure
\ref{fig:Screen1}.  The next section discusses the table of contents screen
and the other data manipulation tools. 



\begin{figure}[htb]
 \centerline{\psfig{figure=screen1.ps,width=3.5in}}
 {\em This screen shot shows a portion of this 
 user's table of contents in the upper left screen, and a retrieved
 document node called ``Design'' in the upper right screen.  The Design
 node contains within it both a link to a figure called ``Figure 1'' and a
 comment called ``Regions'' immediately after it (represented in the
 Design node by the textual anchor ``[??]''.  Both the figure and the
 comment are retrieved and displayed in the screens in the lower left and
 lower right hand corners, respectively.}
 \caption{Screen Shot of AEN's Main display, a comment node, a figure node,
 and TOC.}
 \label{fig:Screen1}
\end{figure}


\subsubsection{Mechanisms for Data Manipulation in AEN}
\label{sec:mechanisms}

For browsing, AEN emulates many of the facilities of WWW browsers such as
Mosaic or Netscape.  Each node is displayed in a separate screen, and
multiple nodes can be displayed simultaneously.  The contents of nodes are
automatically reflowed to conform to the size of the screen.
Middle-clicking over a link anchor retrieves the corresponding node from
the server.  As noted above, AEN supports storage and retrieval of both
text and graphics, and although audio is not currently available in AEN, it
has been implemented in other Egret-based applications, and could be easily
added.  Due to a limitation in the current release of XEmacs, in-line
graphics are not currently available in AEN.

Where AEN begins to depart from WWW systems is in its support for
authoring and collaboration.  First, AEN provides an authoring mode
for nodes, including menu and command key-based formatting directives
for headings, lists, and fontification, link anchor creation, and so
forth.  Second, all artifact nodes must be locked before they can be
modified.  Locking preserves the integrity of information when
multiple authors are working on a document, but provides only basic
support for collaboration.  AEN augments locking with an access
control mechanism, discussed next.

\paragraph{Access Control}

AEN implements three forms of access to individual nodes: read access,
write access, and annotation access.  The administrator(s) of a node can set each
form of access independently for each user of AEN.

When no access is provided to a node, then that node is essentially
private and can be retrieved and edited only by the administrator(s) of the
node. (The creator of a node is given the status of ``administrator'', which
has the effect of irrevocably providing all forms of access. Only the
administrator[s] of a node can delete the node.)

When a user has read access for a node, they can retrieve it but they
cannot add a link to it or edit its contents. Providing only read access to
a node to others indicates that the information is available for use, but
not for review or feedback.

When a user has both read and annotation access for a node, they can both
retrieve it and add a link from the node to a newly created comment
node. Providing both read and annotation access indicates that the
information is available for use and that comments from others about the
content of the node are welcomed.  Annotation access allows the administrator of
the node to solicit suggestions for changes to the contents of the node,
while still retaining control over whether those changes are actually
applied to the document.

When a user has both read and write access for a node, they can both
retrieve it and lock it for editing.  Providing both read and write access
signifies the desire for the closest form of collaborative development, in
which those with write access can make arbitrary changes to the contents of
a node.

Of course, the administrator of a node may grant all three access rights to another
user, which allows them to select which form of feedback to provide.  It is
also interesting to note that without read access, neither annotation nor
write access is useful, since one cannot annotate or lock a node without
first retrieving it.  This is actually quite useful in practice, since it
allows the administrator of a node to temporarily ``hide'' a node simply by
disabling read access to the group.  By re-enabling read access, the
previous annotation, and write access are all reinstated.


\begin{figure}%[htb]
 \centerline{\psfig{figure=screen2.ps,width=3.5in}}
 {\em This screen shot illustrates some of the real-time support
for strong collaboration in AEN.  The Snoopy window in the lower right hand
corner shows, for each user of the system, whether or not they are logged
in currently and if so, where they are working in the document.  Both user
cmoore and user jeremyt are working on the document node called ``Shared
Emacs''.  The Partyline screen, in the lower right hand corner, allows these
users (as well as all others) to talk to each other while they work.  This
figure also shows a local TOC for the Shared Emacs chapter, and a screen
displaying the current set of unread document nodes for this user.}
 \caption{Screen Shot of AEN's Main display, Snoopy, Partyline,  Node List
 and TOC.}
 \label{fig:Screen2}
\end{figure}

\paragraph{Context-Sensitive Table of Contents}

The access control mechanism determines which nodes a user may read and
what they may do to the node once they have read it, but it does not help
the user determine which node they want to read.

The primary navigation mechanism  provided by AEN is {\em tables of
contents}.  Each table of contents (TOC) is a dynamically generated, linear
traversal of document nodes in the database, organized according to the AEN
Backbone structure described in Subsection \ref{sec:aen-backbone}.  Figures
\ref{fig:Screen1} and \ref{fig:Screen2} each contain an example of a table
of contents.  Each line in a TOC is mouse-sensitive, and middle-clicking on
it retrieves the corresponding node (or section within a node) from the
server.  The indentation of headings provides a visual cue to the
hierarchical structure of the hypertext.  Once the TOC is displayed the
user can collapse or expand the listing to reduce or enlarge the number of
listings present.


AEN generates a TOC based upon a starting node that the user chooses.  The
TOC algorithm only analyzes document nodes for which the user has read
access, thus, the TOC is both location and user-specific.  The algorithm
only visits the children of a node once, which breaks cycles in the
hypertext. Finally, each time a document node is written to the server,
structural information relevant to the TOC is extracted, cached, and
propagated to all clients if it has changed. Such a mechanism is crucial to
the usability of the system; one cannot retrieve every document node from
the database each time a TOC is generated.


\paragraph{Node Lists}

The tables of contents provide the most important perspective on the
hypertext---that of the document-level structure, but other viewpoints are
also useful.  Users can make queries for nodes satisfying various criteria
with the node-list mechanism. In AEN, useful criteria so far include the
type of nodes, whether or not the nodes have been retrieved by this user
yet, and the nodes owned by the user.

Each node-list returned from a query consists of a mouse-sensitive
list of node names appearing in a single screen. Middle-clicking
on the node retrieves it in the standard fashion for AEN navigation.

The Hyperstar Bulletin \footnote{The name of this mechanism is a pun on the
name of Honolulu's evening newspaper: the Honolulu Star-Bulletin.} is an
agent-based mechanism intended to support collaboration via a daily
``snapshot'' of what has changed in the database since each user last
visited it.  The Hyperstar Bulletin consists of an agent process that wakes
up at approximately 4 a.m. each morning and sends a daily electronic
``newspaper'' to each participant containing a listing of which nodes (for
which the user has read access) have been created or changed since the last
time the user logged in.


\paragraph{Snoopy and Partyline}

The mechanisms described above support the construction of a hypertext by
multiple authors, but falls short of support for strong collaboration in
one crucial way.  With these above mechanisms alone, the actual people
involved in the collaborative activity are ``invisible'' within the system:
their effects upon hypertext document are visible, but they themselves are
not.  To create a physical presence for participants within the system, AEN
provides two additional facilities: Snoopy and Partyline, as illustrated in
Figure \ref{fig:Screen2}.

The Snoopy mechanism allows the user to see who else is at work and have an
idea of what they are working on.  It displays basic information about the
connection status and last read node of the users of AEN. A Snoopy screen
is created automatically each time a user connects to AEN, and is updated
every 30 seconds.  Through Snoopy, each user knows who else is currently in
AEN and where they are in the hypertext.

Partyline is a mechanism for passing real time messages to other users of
AEN.  Partyline allows the user to send a textual message to all the users
currently connected to AEN or send a private message to one other user
currently connected to Partyline.  Partyline is similar to an Internet
``chat'' mechanism.  Typically, users greet the other people when they
log in to AEN with Partyline messages.

Both facilities provide access to the users of AEN.  In order to help
support strong collaboration, users of AEN need access to the other users.
Snoopy and Partyline provide this service.


\section{Evaluation of AEN}
\label{sec:evaluation}


AEN was used during the Fall semester of 1994 in a graduate seminar on
collaborative systems in the Department of Information and Computer
Sciences at the University of Hawaii. The initial requirements document for
AEN \cite{csdl-94-06} was developed in June, 1994. Precise and accurate
metrics data for only a small portion of the Fall semester was collected by
AEN, due to volatility in both AEN and the metrics system.  Approximately
285 hours of active use of AEN were logged during the second half of the
Fall semester by ten participants.  Approximately 800 nodes and 800 links
were created during the semester by the class.

In Spring, 1995, I conducted an evaluation of AEN's support of strong
collaboration for my Master's thesis.  Three groups of five undergraduate
and graduate students used AEN to develop requirements documents for an
undergraduate software engineering course.  Over the course of the study,
the three groups spent over 180 hours using AEN. They created 100 document
nodes, 111 comment nodes, and 5 figure nodes.  All of the groups in the
study demonstrated some degree of strong collaboration in their
requirements documents.



This section presents the results from the evaluation of AEN during Spring,
1995.  Subsection \ref{sec:data-overview} briefly overviews the results of the
groups' usage of AEN.  Subsection \ref{sec:AEN-supports} presents the
evaluation of AEN's support of strong collaboration.  Subsection
\ref{sec:Tools} discusses the major features of AEN that were designed to
support strong collaboration.  Subsection \ref{sec:eval-AEN} presents the
evaluation of AEN as a collaborative tool.  Subsection \ref{sec:bottom-up}
closes this section with a brief description of the three groups'
procedures for collaboration.


\subsection{Overview of the Collected Data}
\label{sec:data-overview}


At the end of the study, I analyzed the database, metrics, bug reports,
user suggestions and user surveys.  Over the course of the study, the three
groups spent over 180 hours using AEN. They created 100 document nodes, 111
comment nodes, and 5 figure nodes.  There were large differences in how the
different groups and members used AEN.  

This study investigated two questions. First, does AEN support strong
collaboration? Second, does AEN's set of tools promote collaboration?  The
next section discusses how well AEN supports strong collaboration.

\subsection{Does AEN support strong collaboration?}
\label{sec:AEN-supports}

To answer this question, each of the metrics for each component of the
operationalized definition are examined in turn.  The analysis of each
metric presents two sets of numbers, one relating to all of the nodes owned
by the group, the other focusing on the only the nodes that appear in the
group's final requirements document.  For the purpose of the this
evaluation, I will look only at the groups' document nodes, since the
comment nodes are for feedback purposes.  I do not expect collaborative
editing on a comment intended as feedback.  In order to say that AEN does
support the strongest collaboration possible, a single group must
demonstrate all four components of the operationalized definition of strong
collaboration presented in Subsection \ref{sec:operationalized-definition}.

\subsubsection{RPN: Readers per Node}

The first metric I will look at, is the readers per node (RPN) metric. For
this study, the lowest RPN value is one, indicating that no member of the
group read any nodes other than those he created.  The highest possible RPN
value is five, indicating that every member read every node.  Each group
in the study behaved differently with respect to RPN.


\small
\begin{table}[htb]
  \caption{Summary of Groups RPN Metric.}
  \begin{center}
    \begin{tabular}{|c|c|c|}
      \hline
      \multicolumn{3}{|c|}{\rule[-3mm]{0mm}{8mm}\bf RPN metric}\\ \hline 
      Group&All Nodes&Final Doc Nodes\\ \hline 
      \hline
      A&2.5 Readers/Node&3.6 Readers/Node\\ \hline
      B&2.9 Readers/Node&4.8 Readers/Node\\ \hline
      C&3.0 Readers/Node&5.0 Readers/Node\\ \hline
    \end{tabular}
  \end{center}
  \label{tab:RPN}
\end{table}
\normalsize

Table \ref{tab:RPN} shows the RPN values for the three groups.  Group A's
RPN value for their final document is 3.6, the lowest RPN value for the
three groups.  However, this value is still in the top half of the RPN
range.  In Group B's final document, Group B did not use any nodes with
fewer than four readers per node.  Their RPN value, for their final
document, was 4.8.  Their RPN value is much higher than Group A's. It is in
the top 4\% of the RPN range.  In their final document Broup C only chose
nodes that had five readers per node.  They demonstrated the highest
possible RPN value (5) in their final document.

All three groups had RPN values in the top half of the RPN scale.
The next metric looks at collaborative editing of nodes. 

\subsubsection{EPN: Editors per Node}

The Editors per node (EPN) metric calculates the average number of editors
per node.  The lowest EPN value possible for this study is one, indicating
that each node was only edited by one member.  The highest possible EPN
value is five, indicating that every node was edited by all five group
members.

\small
\begin{table}[htb]
  \caption{Summary of Groups EPN Metric.}
  \begin{center}
    \begin{tabular}{|c|c|c|}
      \hline
      \multicolumn{3}{|c|}{\rule[-3mm]{0mm}{8mm}\bf RPN metric}\\ \hline 
      Group&All Nodes&Final Doc Nodes\\ \hline 
      \hline
      A&1.7 Editors/Node&2.7 Editors/Node\\ \hline
      B&2.2 Editors/Node&4.3 Editors/Node\\ \hline
      C&2.3 Editors/Node&3.6 Editors/Node\\ \hline
    \end{tabular}
  \end{center}
  \label{tab:EPN}
\end{table}
\normalsize


Table \ref{tab:EPN} shows the EPN values for the three groups.  Even though
Group A had the lowest EPN values their EPN value for the final document is
in the top half of the EPN range.  In Group B's final requirements
document, Group B had an EPN value of 4.3.  This value is in the top 20\%
of the EPN range.  Group B's EPN value is the highest EPN value of all
three groups.  Group C also had high values for their editors per node.
Group C's EPN value for their final document was 3.8, which is in the upper
half of the EPN range.  All three groups had EPN values in the top half of
the EPN range.


This subsection discussed one way to provide input on a node of the document ---
direct editing.  Another way to provide input is to create feedback or
comment nodes.  The next section discusses the feedback creation metric of strong
collaboration, FNC.

\subsubsection{FNC: Feedback Node Creation}

The Feedback Node Creation (FNC) metric in combination with EPN helps define
how much collaborative input occurred. The creation of feedback nodes gives
an indication how much discussion is going on among the authors of the
document.  FNC calculates the percentage of document nodes that have been
commented on. The lowest possible FNC value is 0\%, indicating that no
document nodes were commented on.  The highest FNC value is 100\%,
indicating that all document nodes were commented on.  

\small
\begin{table}[htb]
  \caption{Feedback Nodes.}
  \begin{center}
    \begin{tabular}{|c|c|c||c|c|}
      \hline
      \multicolumn{5}{|c|}{\rule[-3mm]{0mm}{8mm}\bf FNC metric for
      all groups}\\ \hline
      Group&All Nodes&FNC&Final Doc Nodes&FNC\\ \hline
      \hline
      A&10&29.4\%&8&50.0\%\\ \hline
      B&8&19.5\%&6&42.9\%\\ \hline
      C&4&16.0\%&2&20.0\%\\ \hline
    \end{tabular}
  \end{center}
  \label{tab:feedback}
\end{table}
\normalsize

Table \ref{tab:feedback} shows the number of comment nodes and the FNC
values for each group.  Different collaboration styles can easily be seen
in the different FNC values.  Group A, with an FNC of 50\%, used comments
widely, while Group C, with an FNC of 20\%, hardly used any comments
at all. 
% These higher FNC values might explain group A's lower EPN scores

The next component of the definition is how the users control the access to
the nodes.

\subsubsection{NAC: Non-default Access Control}

The Non-default Access Control (NAC) metric is the first of two access
control metrics.  It calculates the percentage of nodes that have had their
access control changed from the default.  The lowest NAC value possible is
0\%, indicating that no nodes had their access control changed from the
default.  This also indicates that there was no interaction among the
members since the default access control is no access for other members.
The highest possible NAC value is 100\%, indicating that every node has had
its access control modified.

\small
\begin{table}[htb]
  \caption{NAC Breakdown.}
  \begin{center}
    \begin{tabular}{|c|c|c|c|c|}
      \hline
      \multicolumn{5}{|c|}{\rule[-3mm]{0mm}{8mm}\bf NAC metric by group}\\ \hline
      Group&All Nodes&NAC&Final Doc Nodes&NAC\\ \hline
      \hline
      A&26&76.5\%&16&100.0\%\\ \hline
      B&23&56.1\%&14&100.0\%\\ \hline
      C&16&64.0\%&10&100.0\%\\ \hline
    \end{tabular}
  \end{center}
  \label{tab:access-control}
\end{table}
\normalsize

Table \ref{tab:access-control} shows the NAC values for each group.  In the
final document, 100\% of the nodes for all groups had their access control
changed.  This is not surprising.  Since all groups had RPN values for their
final documents greater than one, the access control for all those nodes
must have been changed to allow other members access to the node.  The
relatively high NAC value for A's total document implies that they
created fewer ``private'' nodes.  Private nodes are nodes whose access
control does not let any one else read or change them.

\subsubsection{EAC: Evolving Access Control}

The second metric dealing with access control is Evolving Access Control
(EAC). It calculates the percentage of document nodes that have had their
access control changed more than once. This metric measures the degree to
which the groups' collaboration changed. The lowest EAC value possible is
0\%, indicating that no nodes had their access control changed more than
once.  The highest possible EAC value is 100\%, indicating that all nodes
had their access control changed more than once.

\small
\begin{table}[htb]
  \caption{EAC Breakdown.}
  \begin{center}
    \begin{tabular}{|c|c|c||c|c|}
      \hline
      \multicolumn{5}{|c|}{\rule[-3mm]{0mm}{8mm}\bf EAC metric by group}\\ 
      \hline
      Group&All Nodes&NAC&Final Doc Nodes&NAC\\ \hline
      \hline
      A&7&21\%&7&44\%\\ \hline
      B&2&5\%&1&7\%\\ \hline
      C&2&8\%&1&10\%\\ \hline
    \end{tabular}
  \end{center}
  \label{tab:d-access-control}
\end{table}
\normalsize

Table \ref{tab:d-access-control} shows the results for the EAC metric.  All
three groups evolved the access control for a few of their nodes.  Group
A, which so far has had the lower metric scores in RPN and EPN, greatly
out scores the other two groups in EAC.  The absolute values for EAC are
low, 44\%, 7\%, and 10\%.  However, a high score in EAC represents a
dynamic collaborative process.  I did not expect the three groups to have
very dynamic collaborative process, because of their low experience with
group collaboration and because they had never collaborated together.

\subsubsection{Summary}


Does AEN support strong collaboration?  In order to answer positively, at
least one group must demonstrate strong collaboration in their hypertext
document. Figure \ref{fig:op-def} shows a view of all the metrics for each
group.

\begin{figure}[htbp]
  \centerline{\psfig{figure=metrics.eps,width=3.5in}}
 \caption{Summary of Groups' Collaboration Metrics.}
 \label{fig:op-def}
\end{figure}

Since no one group had uniformly high values for their five metrics, I
cannot say that any group collaborated in a ``purely'' strong manner on their
document.  This does not mean that AEN does not support strong
collaboration. There could be some obscure interaction between the five
major features of AEN that prevents a single group from having high scores
in all five metrics, but I do not believe this is true.

The one or more of the three groups showed high values for the RPN, EPN and
NAC metrics.  For the FNC and EAC metrics, group A scored 50\% and 44\%,
indicating that they used AEN to comment on and change the collaborative
style of their document.  Based upon the high scores in RPN, EPN and NAC
and Group A's ability to demonstrate the other two components of the
operationalized definition of strong collaboration, I claim that AEN is
capable of supporting strong collaboration.

Let us look at the specific features that AEN uses to support strong
collaboration.

\subsection{AEN's Features}
\label{sec:Tools}

AEN's set of features was designed to support and encourage collaboration
between members.  The features were developed during the Fall, 1994 semester
and evaluated in the Spring, 1995 semester.  This section discusses the
five major features provided by AEN in support of collaboration: access
control, tables of contents, node lists, Snoopy and Partyline.


\subsubsection{Access Control}

As discussed above, each group used the access control mechanisms during
the case study.  I hoped the design of access control would promote
evolving access control.  In other words, the access to a node would change
over the lifetime of the node reflecting the different states of
completeness and collaboration.  AEN's access control model was a partial
success.  All the groups did use the control mechanism to restrict access
to the node, but they did not use its evolutionary nature to significantly
change their mode of collaboration.


For each of the remaining four major features, I will present the usage
data and the survey results.  The usage data will compare the number of
sessions each user had and the number of instances they used the tool (when
this information is available).  The survey results indicate the tools'
helpfulness and their frequency of use.  The survey questions asked the
user to rate different aspects of AEN from a scale of one to nine, with one
being poor and nine being good.

The first two features, tables of contents and node lists, are
navigation aids that help the users find their way around the hypertext
document.  The tools should help the users get to the interesting nodes of
their documents.  The last two tools, Snoopy and Partyline, are designed to
provide a sense of physical presence in the hypertext.  Each will be
discussed in turn.

\small
\begin{table}[htb]
  \caption{Context-Sensitive Table of Contents}
  \begin{center}
    \begin{tabular}{|c|c|}
      \hline
      \multicolumn{2}{|c|}{\rule[-3mm]{0mm}{8mm}\bf Context-Sensitive TOC Creation by group}\\ 
      \hline
      Group&Ave \# TOCs/Session\\ \hline
      \hline
      A&1.24\\ \hline
      B&1.17\\ \hline
      C&1.17\\ \hline
    \end{tabular}
  \end{center}
  \label{tab:TOC}
\end{table}
\normalsize

The table of contents (TOC) was the most used tool provided by AEN.  Table
\ref{tab:TOC} shows the average number of times each user created a TOC per
session.  Since AEN does not automatically create a TOC, the metrics
indicate that TOCs were widely used.  Each group usage pattern of the TOCs
is different.  Group A created TOCs more often than the other groups.
Group C and B are very close in their numbers.


\small
\begin{table}
  \caption{Survey Questions about Table of Contents. (Scale 1 to 9)}
  \begin{center}
    \begin{tabular}{|l|c|c|}
      \hline
      \multicolumn{3}{|c|}{\rule[-3mm]{0mm}{8mm}\bf Survey Questions:
      Table of Contents}\\ \hline
      &Average&Std. Dev.\\ \hline
      I used the Table of Contents:&8.7&0.7\\ \hline
      The Table of Contents is helpful:&8.3&1.3\\ \hline
    \end{tabular}
  \end{center}
  \label{tab:survey-TOC}
\end{table}
\normalsize

The survey results in Table \ref{tab:survey-TOC} show that the users felt
they used the TOC very often (8.7 out of 9, with a very small standard
deviation).  The users also thought that the TOC was a very helpful tool,
average score of 8.3.  

The next feature, node lists, complements tables of contents as a navigation
aid.

\subsubsection{Node Lists}
\small
\begin{table}[htb]
  \caption{Node List Creations}
  \begin{center}
    \begin{tabular}{|c|c|}
      \hline
      \multicolumn{2}{|c|}{\rule[-3mm]{0mm}{8mm}\bf Node List Creation by group}\\ 
      \hline
      Group&Ave \# Node Lists/Session\\ \hline
      \hline
      A&0.27\\ \hline
      B&0.63\\ \hline
      C&1.12\\ \hline
    \end{tabular}
  \end{center}
  \label{tab:NL}
\end{table}
\normalsize

The groups' use and feelings towards node lists vary greatly.  Some group
members did not create any node lists, while others created more node lists
than tables of contents.  Table \ref{tab:NL} shows the average number of
node lists each member created per session.


\small
\begin{table}[htbp]
  \caption{Survey Questions about Node Lists. (Scale 1 to 9)}
  \begin{center}
    \begin{tabular}{|l|c|c|}
      \hline
      \multicolumn{3}{|c|}{\rule[-3mm]{0mm}{8mm}\bf Survey Questions:
      Node Lists}\\ \hline
      &Average&Std. Dev.\\ \hline
      I used the Unread Nodes:&5.6&2.8\\\hline
      I used the Nodes by type:&4.9&2.6\\\hline
      I used the Owned Nodes:&4.1&2.2\\\hline
      Unread Nodes is helpful helpful:&6.8&2.2\\\hline
      Nodes by type is helpful:&5.3&2.2\\\hline
      Owned Nodes is helpful:&5.1&2.2\\\hline
    \end{tabular}
  \end{center}
  \label{tab:survey-Nodelist}
\end{table}
\normalsize


The survey results (see Table \ref{tab:survey-Nodelist}) reflect the wide
variety in the metrics data.  The users did not feel that they had used the
node lists as much as the table of contents (Avg. 5.6 vs. Avg. 8.7).  The
metrics data shows that some users created more node lists than TOCs.  The
standard deviations of the node list surveys show this large variation.
The users thought that the Unread Nodes list was almost as helpful as the
TOC (Avg. 6.8 vs. Avg. 8.3).



The last two features, Snoopy and Partyline were designed to promote
collaboration in AEN.  They are discussed next.

\subsubsection{Snoopy}

\small
\begin{table}[htbp]
  \caption{Survey Questions about Snoopy. (Scale 1 to 9)}
  \begin{center}
    \begin{tabular}{|l|c|c|}
      \hline
      \multicolumn{3}{|c|}{\rule[-3mm]{0mm}{8mm}\bf Survey Questions:
      Snoopy}\\ \hline
      &Average&Std. Dev.\\ \hline
      I used Snoopy:&6.3&2.6\\\hline
      Snoopy is helpful:&6.8&2.4\\\hline
    \end{tabular}
  \end{center}
  \label{tab:survey-Snoopy}
\end{table}
\normalsize

The Snoopy feature is started with each AEN session.  To use the tool, the
user simply looks at a window that displays the usage information of all
users.  Thus there were no meaningful metrics collected about Snoopy.  The
survey results (see Table \ref{tab:survey-Snoopy}) show that the users
generally used and liked Snoopy.  The large standard deviation might be
attributed to the different user's method of collaboration.  Snoopy's
information is not very interesting if there are no other users logged in
or if the other members are present and accessible in the room.


The last feature evaluated in AEN is Partyline.
\subsubsection{Partyline} 

\small
\begin{table}[htb]
  \caption{Partyline Messages}
  \begin{center}
    \begin{tabular}{|c|c|}
      \hline
      \multicolumn{2}{|c|}{\rule[-3mm]{0mm}{8mm}\bf Partyline Messages by group}\\ 
      \hline
      Group&Ave \# Partyline Messages/Session\\ \hline
      \hline
      A&2.71\\ \hline
      B&1.67\\ \hline
      C&0.53\\ \hline
    \end{tabular}
  \end{center}
  \label{tab:Partyline}
\end{table}
\normalsize


The metrics for Partyline use (see Table \ref{tab:Partyline}) show that
there is a wide variety in the number of Partyline messages sent.
Comparing the times Partyline messages were sent and when users were
simultaneously editing node, it does not appear that Partyline was used
more than 15 times to coordinate editing.

\small
\begin{table}
  \caption{Survey Questions about Partyline. (Scale 1 to 9)}
  \begin{center}
    \begin{tabular}{|l|c|c|}
      \hline
      \multicolumn{3}{|c|}{\rule[-3mm]{0mm}{8mm}\bf Survey Questions:
      Partyline}\\ \hline
      &Average&Std. Dev.\\ \hline
      I used Partyline:&5.2&2.6\\\hline
      Partyline is helpful helpful:&6.2&2.4\\\hline
    \end{tabular}
  \end{center}
  \label{tab:survey-Partyline}
\end{table}
\normalsize

One possible reason for this low use of Partyline is referenced in the
limitations section of the experimental design.  The users could only
access AEN from two rooms.  This meant that they were frequently co-located
in the same room, and thus able to speak directly to each other instead of
using Partyline.  Since it is easier talk than type, I believe they
collaborated verbally instead of using Partyline.  This could lead to the
low score for Partyline use in the survey.  Table
\ref{tab:survey-Partyline} shows the results of the survey for Partyline.
The users felt that Partyline was fairly helpful even though the did not
use it very often.


\small
\begin{table}[htbp]
  \caption{Survey Question about Sense of Presence. (Scale 1 to 9)}
  \begin{center}
    \begin{tabular}{|l|c|c|}
      \hline
      \multicolumn{3}{|c|}{\rule[-3mm]{0mm}{8mm}\bf Survey Questions:
      Sense of Presence}\\ \hline
      &Average&Std. Dev.\\ \hline
      AEN provides a sense of:&6.5&1.9\\
      physical presence for&&\\
      the other users.&&\\\hline
    \end{tabular}
  \end{center}
  \label{tab:survey-Presence}
\end{table}
\normalsize

The above two features are designed to promote a sense of physical presence
for each of the users.  The survey asked the users about their sense of
physical presence.  Table \ref{tab:survey-Presence} shows the results of
the sense of presence survey question.  They felt that AEN does provide a
sense of physical presence for the other users.

\subsection{Evaluation of AEN}
\label{sec:eval-AEN}

The evaluation of the set of features provided by AEN shows that they
generally were used and liked by the group members.  There was a wide
variation in the feelings of the group members and different features were
used to different degrees.  A goal of this research is to evaluate AEN as a
collaborative editing tool.

There are three different sets of evidence that lead me to believe AEN is a
successful collaborative tool. One, all three groups collaboratively
created quality requirements documents. Two groups received an A on their
document and one received a B.  Two, all three requirements documents show
high values in at least three of their collaborative metrics.  Three, the
survey results show that the users liked AEN.  Table \ref{tab:survey-AEN}
shows the results of the survey questions about AEN.

\small
\begin{table}[htb]
  \caption{Survey Questions about AEN. (Scale 1 to 9)}
  \begin{center}
    \begin{tabular}{|l|c|c|}
      \hline
      \multicolumn{3}{|c|}{\rule[-3mm]{0mm}{8mm}\bf Survey Questions:
      Overall reactions to AEN:}\\ \hline
      &Average&Std. Dev.\\ \hline
      terrible---wonderful&5.5&1.6\\\hline
      frustrating---satisfying&5.0&1.8\\\hline
      dull---stimulating&5.0&1.7\\\hline
      difficult---easy&6.2&1.9\\\hline
      rigid---flexible&5.2&1.9\\\hline\hline
      AEN promotes collaboration:&7.2&2.0\\\hline
    \end{tabular}
  \end{center}
  \label{tab:survey-AEN}
\end{table}
\normalsize

The users felt that AEN promoted collaboration, but there was disagreement
about that.  One user did not think AEN promoted collaboration.  That user
rated AEN a two out of nine.


\section{Recommendations}
\label{sec:conclusions}


Based upon the my experiences with AEN, I recommend the following to
encourage strong collaboration: provide direct and indirect authoring
mechanisms, provide access to intermediate work products, provide
context-sensitive ``what's new,'' and provide mechanisms to allow users as
well as documents to be visible.

My first recommendation is to provide direct and indirect authoring
mechanisms.  AEN supports two methods for collaborative authoring, which I
term {\em proof-reading}(indirect) and {\em trading the lock}(direct).  In
the proof-reading method, the author creates a node and publishes it by
allowing read and annotate access.  This allows others to make comments
suggesting changes, new ideas, or just general comments on the subject.
The author can read the comments and make changes to the document or
comment on the comments.  In the trading the lock method, the author
creates a node and allows other authors to edit it by providing read and
write access.  Each author can lock the node, make a change, save the node,
and unlock it, which updates the contents of the node displayed on each
author's screen.  Strong collaboration is enhanced by providing authors
with these degrees of control over the document and styles of interaction
with others.

My second recommendation is to provide access to intermediate work
products.  One of the strongest enablers of strong collaboration is easy
accessibility to intermediate work products.  Synergy is nurtured by
permission to review another's admittedly rough, first pass at an idea,
where comments and suggestions are aimed at refining and enhancing, rather
than confirming or denying, as is frequently the case with final, polished
presentations.

My third recommendation is to provide context-sensitive ``what's new.''
When a group divides into subgroups and is actively and incrementally
building hypertext documents, context-sensitive mechanisms to automatically
inform users of what has changed are essential. Otherwise, the users will
suffer from either lack of knowledge about what is changing (if no
change-related mechanisms exist) or a low signal-to-noise ratio (if
context-free change-related mechanisms are used, in which case users are
informed of many changes that are irrelevant to them.)  In AEN, the
combination of access control, unread nodes, and Hyperstar Bulletin
provides a very nice means of selectively propagating change-related
information across groups.  The daily Hyperstar Bulletin encourages users
to log in to AEN only when necessary to see changes, and access control
allows users to control the visibility of their changes.

My last recommendation is to provide a mechanism to allow users as well as
artifacts to be visible.  In AEN, providing knowledge of who was using AEN,
where they were, and a means to communicate with them created many new
opportunities for collaboration without requiring face-to-face interaction.

%\section{Experiences}

%\section{RECOMMENDATIONS}

\section{Availability}

The current release of AEN is publicly available through anonymous ftp from 
ftp.ics.Hawaii.Edu/pub/csdl/egret/aen/.  The uncompressed distribution
requires 2MB of disk space.  AEN requires XEmacs and EGRET to run.
\bibliography{/group/csdl/bib/aen,/group/csdl/bib/csdl-trs}
\bibliographystyle{plain}
\end{document}




