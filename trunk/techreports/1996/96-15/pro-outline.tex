\documentstyle[nftimes,11pt,/group/csdl/tex/definemargins,
/group/csdl/tex/lmacros]{report} 
\input{/group/csdl/tex/psfig/psfig}

\begin{document}

\title{Evaluating Automated Support for the Personal Software Process}
\author{Anne Disney\\
Collaborative Software Development Laboratory,\\
Department of Information and Computer Sciences\\
2565 The Mall\\
University of Hawaii, Manoa\\
Honolulu, Hawaii   96822\\
{\tt anne@uhics.ics.hawaii.edu}}
\maketitle

\tableofcontents
\chapter{Introduction}
\section{What is PSP}
\subsection{Basic description}
\subsection{Who developed it and when?}
\subsection{Current environments where it is used}
\subsection{How widespread is it?}
\subsection{Why do people want to use it?}
\section{PSP can be difficult to learn}
\subsection{Many forms}
\subsection{Multiple processes, so as user learns PSP, familiar forms change}
\subsection{Complicated computations with higher PSP levels}
\subsection{Instructions in textbook not always clear}
\subsection{Data interdependencies between forms complicated in higher PSP levels}
\section{PSP is time-consuming to use even after learning it}
\subsection{Many forms with many fields to fill out by hand, at the same time as coding}
\subsection{Tedious computations at the end of even minor projects}
\subsection{When computations rely on historical data, finding the data takes time}
\section{Analyzing PSP data is time-consuming}
\subsection{Finding the right data on the right form of each project takes time}
\subsection{Even when using spreadsheets, an extra step is added to each project}
\subsection{Doing even part of the computations by hand is a big job}
\subsection{Users tend to do only a few kinds of data analysis because of the work involved}
\section{Thesis of this research}
\subsection{Automation of the PSP will result in faster and more accurate collection of PSP data.  Analysis of this data will also be faster and users will look at the data 
in more ways.}
\chapter{Other related work}

Most processes for producing high-quality software are designed to be used by groups of
software developers or even organizations.  Examples are
* CMM <get information from The Capability Maturity Model, Carnegie Mellon University, SEI>
  - basic idea
  - how is high quality software produced
* FTR <source?>
  - basic idea
  - how is high quality software produced
* <Others?>

Watts Humphrey's Personal Software Process, however, focuses on the software development
process for an individual software engineer.
* main ideas described in introduction.  Restate?
* The main focus of the PSP is the individual programmer.  However, it can be used as 
a tool in group development as well.  <9th conf on Software Engineering Education, p119 (125)
* Like other software processes, PSP requires significant investment of time and effort by the
  software engineer.  However, it directly benefits the person who makes the effort with 
  information about their own process and products. <9th conf on Software Enginering Eduction,   
  p 52>
 
\chapter{Describe how PSP is done now}
\section{Paper forms}
\section{Mixed group of helping tools for computations and analysis}
\section{Helping tools typically don't communicate}
\chapter{Requirements for automating the Personal Software Process}
\section{Should be implemented using java}
\subsection{will allow product to run multiple operating systems and
hardware platforms}
\subsection{Java is an object-oriented language}
\subsection{will allow a GUI}
\section{Should follow the PSP forms as outlined by Watts Humphrey as
closely as possible}
\subsection{will make it easier for users to understand and follow along in
Humphrey's book}
\subsection{will make comparisons between the two ways of doing PSP as
meaningful as possible}
\subsection{should implement multiple levels of PSP}
\section{Should eliminate data redundancy}
\subsection{example: user should only fill in estimated lines deleted once,
in the Size Estimating Template, and not again in the Project Plan Summary}
\subsection{this should be a goal not only in the user interface, but in
the table design}
\section{Should add fields and forms as needed to fully implement
Humphrey's suggestions}
\subsection{example: a field is needed to indicate whether or not a project
should be included in data analysis.  Humphrey suggests eliminating
outliers for certain computations, but there is nowhere to permanently
indicate this.}
\subsection{example: need a field to categorize projects by type, so that
new development projects don't get mixed up with bug-fixing projects in data analysis}
\subsection{example: one input to all the PSP processes is a problem description, but
there is no form to record this.  With a hardcopy system, this could be
attached to the front of the packet, but an on-line system needs a place to
record this}
\subsection{example: defect models can be implemented so that the user only
needs to record a defect, and is not required to remember the class of
defect to which it it belongs}
\section{Should guide user through the process as much as possible}
\subsection{user should not have to remember what to do next}
\subsection{this should apply to the order of phases and the order of forms
within a phase}
\subsection{on any particular form, only fields appropriate to the current
phase should be in ``update'' mode}
\section{Online help at both the field and form level should be provided}
\section{Should relieve the user of the burden of calculations}
\subsection{should automatically do regression, splitting estimated time
across phases based on past projects, adding up of defects and times, etc.,
and at the appropriate times in the process use the data to fill in the
correct fields on the forms}
\subsection{should be smart enough to look at the historical data available and decide
which method of calulating to do (for example: time estimation)}
\subsection{should automatically keep track of to-date figures}
\section{Should perform the basic data analyses as outlined by Watts
Humphrey}
\subsection{should present in numerical format}
\subsection{should present in graphical format}
\subsection{should provide a user interface to select project types,
languages, processes, and dates to be included}
\section{Should be table based}
\subsection{for menu options}
\subsection{for constructs like phases, forms, defect types, processes}
\subsection{for user preferences}
\subsection{a goal should be to hardcode by process and phase as little as
possible}
\section{Should address process problems identified in first phase of the research}
\chapter{Case study to address important problems}
\chapter{Research plan}
\section{Look at a group of students who have learned PSP}
\subsection{Find out how long it takes them to fill out forms per project}
\subsection{Find out how many mistakes they make in filling out the forms or doing computations}
\subsection{Find out how long it takes them to do data analysis}
\subsection{Find out how many ways they view analyzed data}
\section{Look at a group of students who have learned PSP, and are using an automated version}
\subsection{Find out how long it takes them to fill out forms per project}
\subsection{Find out how many mistakes they make i
n filling out the forms or doing computations}
\subsection{Find out how long it takes them to do data analysis}
\subsection{Find out how many ways they view analyzed data}
\end{document}
