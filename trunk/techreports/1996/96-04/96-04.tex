%%%%%%%%%%%%%%%%%%%%%%%%%%%%%% -*- Mode: Latex -*- %%%%%%%%%%%%%%%%%%%%%%%%%%%%
%% 96-04.tex -- 
%% Author          : Philip Johnson
%% Created On      : Sun Mar 10 07:07:57 1996
%% Last Modified By: Philip Johnson
%% Last Modified On: Mon Mar 18 11:25:10 1996
%% RCS: $Id: 96-04.tex,v 1.6 1996/03/18 22:34:11 johnson Exp $
%%%%%%%%%%%%%%%%%%%%%%%%%%%%%%%%%%%%%%%%%%%%%%%%%%%%%%%%%%%%%%%%%%%%%%%%%%%%%%%
%%   Copyright (C) 1996 Philip Johnson
%%%%%%%%%%%%%%%%%%%%%%%%%%%%%%%%%%%%%%%%%%%%%%%%%%%%%%%%%%%%%%%%%%%%%%%%%%%%%%%
%% 

\documentstyle[nftimes,/group/csdl/tex/CHI95]{article}
\input{/group/csdl/tex/psfig/psfig}

\begin{document}

\title{Requiem for the Project HI-TIME Collaborative Process}

\author{
David Brauer\\
Philip Johnson\\
Carleton Moore\\
Collaborative Software Development Laboratory\\
Department of Information and Computer Sciences\\
University of Hawaii\\
Honolulu, Hawaii 96822\\
(808) 956-3489\\
\{dave, johnson, cmoore\}@uhics.ics.hawaii.edu}

\maketitle

\abstract 

In early 1995, the State of Hawaii began work on an ambitious revision to
its telecommunications policy planning process.  A multidisciplinary team
was commissioned to develop a proposal for an iterative, interactive,
computer-mediated collaborative planning process whereby the State's
telecommunications infrastructure plan could be developed and periodically
upgraded to reflect technology and policy shifts in local communities. The
proposal included a sophisticated, CSCW software system called HI-TIME
which would both enact the planning process as well as provide access and
visibility into the planning process for the general public.  In early
1996, the ambitious collaborative planning process, including the
implemented, deployed HI-TIME system, was abandoned in favor of a more
traditional approach. This paper explores the rise and fall of Project
HI-TIME and the lessons it holds for developers of CSCW systems.


%%%%%%%%%%%%%%%%%%%%%%%%%%%%%% -*- Mode: Latex -*- %%%%%%%%%%%%%%%%%%%%%%%%%%%%
%% 96-04-intro.tex -- 
%% Author          : Philip Johnson
%% Created On      : Sun Mar 10 07:07:40 1996
%% Last Modified By: Philip Johnson
%% Last Modified On: Mon Mar 18 10:54:38 1996
%% RCS: $Id: 96-04-intro.tex,v 1.4 1996/03/18 20:54:49 johnson Exp $
%%%%%%%%%%%%%%%%%%%%%%%%%%%%%%%%%%%%%%%%%%%%%%%%%%%%%%%%%%%%%%%%%%%%%%%%%%%%%%%
%%   Copyright (C) 1996 Philip Johnson
%%%%%%%%%%%%%%%%%%%%%%%%%%%%%%%%%%%%%%%%%%%%%%%%%%%%%%%%%%%%%%%%%%%%%%%%%%%%%%%
%% 

\section{INTRODUCTION}


The explosive growth of the Internet and WWW has provided new opportunities
to experiment with collaborative technologies formerly limited to research
systems and sophisticated user communities.  This paper describes an
attempt to exploit the opportunity of the Internet and WWW to provide a
collaborative system for use by the general public to support
telecommunications policy planning in the State of Hawaii.

This computer-mediated, collaborative planning process must ultimately be
considered a failure, even though the design appeared sound, the
implementation was solid, and preliminary use seemed acceptable.  This
paper presents our effort to explain why this happened and how future
developers of public CSCW systems can benefit from our experiences.
 
The paper is organized as follows. The next section provides a background
on the telecommunications policy planning process surrounding Project
HI-TIME. The following section briefly describes the design and
implementation of the Project HI-TIME system.  The next section details the
operational life of the Project HI-TIME system. The final section
presents our hypotheses as to why Project HI-TIME did not succeed, and 
recommendations to designers of public CSCW systems.


%%%%%%%%%%%%%%%%%%%%%%%%%%%%%% -*- Mode: Latex -*- %%%%%%%%%%%%%%%%%%%%%%%%%%%%
%% 96-04-telecom.tex -- 
%% Author          : Philip Johnson
%% Created On      : Sun Mar 10 07:15:06 1996
%% Last Modified By: Philip Johnson
%% Last Modified On: Mon Mar 18 11:06:40 1996
%% RCS: $Id: 96-04-telecom.tex,v 1.5 1996/03/18 21:06:45 johnson Exp $
%%%%%%%%%%%%%%%%%%%%%%%%%%%%%%%%%%%%%%%%%%%%%%%%%%%%%%%%%%%%%%%%%%%%%%%%%%%%%%%
%%   Copyright (C) 1996 Philip Johnson
%%%%%%%%%%%%%%%%%%%%%%%%%%%%%%%%%%%%%%%%%%%%%%%%%%%%%%%%%%%%%%%%%%%%%%%%%%%%%%%
%% 

\section{TELECOMMUNICATIONS POLICY PLANNING}

The Hawaii Telecommunications Infrastructure Modernization and Expansion
(HI-TIME) project sought to establish a Strategic Planning Process for the
State of Hawaii that can serve as a model for Telecommunications and
Information Infrastructure planning throughout the nation \cite{TIIAP95}.
The primary goal was to provide the government of the State of Hawaii with
public policy principles, strategic objectives, and a plan of recommended
actions to guide decision making.  The two key features of the proposed
strategic planning process were:

\begin{itemize} 

\item Methods and mechanisms for engaging the general public in a
  meaningful dialog to develop appropriate public and social policies with
  respect to the telecommunications and information infrastructure.

\item A methodology and supporting information infrastructure for an
  on-going strategic planning process which can take into account the rapid
  changes in technology and regulation in the telecommunications field and
  which can evolve over time to effectively address new issues and
  opportunities which arise.

\end{itemize} 
 
Telecommunications policy planning has historically been the exclusive
domain of the State Public Utilities Commission (PUC), the Federal
Communications Commission (FCC), and various legislative and regulatory
bodies in conjunction with the private telecommunications carriers they
regulate.  The process of planning and implementing a telecommunications
infrastructure has typically been carried out in adversarial legal
proceedings before the PUC and FCC.  While the general public is usually
afforded some opportunity for comment on the results of a proceeding, they
seldom have any real opportunity to influence the proceeding itself.
 
Lack of widespread and informed public input to telecommunications
infrastructure planning has made it more difficult to resolve such critical
issues as \cite{NII95}:
 
\begin{itemize} 
\item the definition of universal service, 
\item how to deal with information have-nots, 
\item which government services should be freely provided via the infrastructure, 
\item equal access for rural and disadvantaged communities, and 
\item low-cost access for educational, health and community service organizations. 
\end{itemize} 
 
An expression of public will with respect to these issues is necessary to
formulate appropriate levels of public funding (including tax base support)
as well as policy to ensure telecommunications providers work towards the
desired social goals.  Informed, organized public input will help guide
decision makers towards actions which meet the true needs of their
constituents. The Public Utilities Commission, the Consumer Advocate's
Office, the State Administration and the Legislature are all intended
beneficiaries of the outputs from this process.
 
Another frequently cited problem with telecommunications planning through regulatory proceedings is the length of the process.  Proceedings seldom yield a decision from the commission (a order) within a few months and can frequently drag on for one (or several) years.  Without an on-going, evolutionary strategic planning process, decision makers cannot hope to effectively respond to rapid changes in telecommunications regulation and technology.  A one-time "Strategic Plan" report will be obsolete almost as soon as it is produced, requiring a subsequent effort to update the report.  A more cost effective approach is to recognize the need for an on-going planning process and to develop mechanisms, both automated and organizational, for supporting that process. 
 
In May 1993, responding to legislative mandate, the Public Utilities
Commission of the State of Hawaii, instituted PUC docket 7702 \cite{PUC93},
a proceeding on telecommunications infrastructure for the
State of Hawaii. In conducting these proceedings,
the PUC elected to convene a series of facilitated collaborative sessions
involving all registered intervenors and participants in the docket.  These
collaborative sessions proved to be highly fruitful in identifying and
proposing solutions for key problems in the deployment of Hawaii's
communications infrastructure.  By June of 1994, the "Communications
Infrastructure Collaborative" issued its final report to the PUC detailing
a framework for introducing competition in the Hawaii telecommunications
market and for accelerating the deployment of a "world-class"
infrastructure \cite{PUC95}.  The report was the foundation for landmark legislation in
the 1995 Hawaii legislature which moved Hawaii to the forefront in
innovative telecommunications regulatory policy \cite{HI95}. 
 
The success of the Communications Infrastructure Collaborative led the
State Administration to believe that telecommunications planning was an
area of public discourse that would greatly benefit from improved
mechanisms for collaboration.  A hierarchy of telecommunications
commmittees was formed and new initiatives were launched to further
accelerate improvements in Hawaii's infrastructure.  Project HI-TIME
(Hawaii Information and Telecommunications Infrastructure Modernization and
Expansion) was formed to increase the involvement of the general public in
the planning process and to produce an initial set of planning documents
which would stimulate on-going strategic planning in the telecommunications
domain.
 

 




%%%%%%%%%%%%%%%%%%%%%%%%%%%%%% -*- Mode: LaTeX -*- %%%%%%%%%%%%%%%%%%%%%%%%%%%%
%% 96-04-process.tex -- 
%% Author          : David C. Brauer
%% Created On      : Fri Mar 15 15:25:29 1996
%% Last Modified By: David C. Brauer
%% Last Modified On: Mon Mar 18 13:48:39 1996
%% RCS: $Id: 96-04-process.tex,v 1.7 1996/03/18 23:49:04 dave Exp $
%%%%%%%%%%%%%%%%%%%%%%%%%%%%%%%%%%%%%%%%%%%%%%%%%%%%%%%%%%%%%%%%%%%%%%%%%%%%%%%
%%   Copyright (C) 1996 David C. Brauer
%%%%%%%%%%%%%%%%%%%%%%%%%%%%%%%%%%%%%%%%%%%%%%%%%%%%%%%%%%%%%%%%%%%%%%%%%%%%%%%
%% 

\section{THE HI-TIME PROCESS}


With the HI-TIME Process, we were attempting to draw a picture of 
Hawaii's future with an advanced information and telecommunications 
infrastructure or "Information Superhighway" and to answer questions 
like:

\begin{itemize}
\item How will we use this new technology to enrich our lives?

\item What are the fundamental principles we should use to guide its 
development and use?

\item What role should the government play in its development and use?

\item How should the government proceed with respect to protecting 
fundamental rights and resolving critical issues?

\end{itemize}

A schematic of the HI-TIME Strategic Planning Process is shown in Figure
\ref{fig:hi-time-flow}.  The process began with the collaborative
development of the HI-TIME Vision.  The Vision was expressed as high level
goals and principles which were indended to focus subsequent activities in the
planning process.  Initial concepts for the HI-TIME Vision were developed by
drawing upon a variety of efforts already underway within the State of
Hawaii.  The Vision was then reviewed and refined by the "Process
Participants" (general public, key user groups, telecommunications
stakeholders, government decision makers, and providers of
telecommunications and information infrastructure both public and private.)
Along with the goals and principles, general metrics were developed to
ascertain progress towards meeting the goals as well as their compliance
with the guiding principles.

\begin{figure*}[htbp]
  \centerline{\psfig{figure=hi-time-flow.ps}} 
\caption{The Project HI-TIME Strategic Planning Process.}
\label{fig:hi-time-flow}
\end{figure*}


A Status assessment of the current telecommunications and information 
infrastructure was planned to be conducted to determine how well it meets the 
Vision.  A variety of mechanisms, including surveys, document reviews, 
and field research, were intended to be used along with input, review and commentary 
from Process Participants.  For the first iteration, a preliminary 
Status Assessment was derived from existing sources to be used as a starting 
point.

Process Participants were then to Identify Disparities between the HI-TIME 
Vision and the current Status.  Reviews and discussions with 
stakeholders/key users, decision makers, and providers would serve to identify 
gaps and shortfalls.

The next step was for stakeholders/key users, decision makers, and 
providers to articulate a Strategy to address the gaps and shortfalls. 
This Strategy was to be expressed as a set of measurable objectives to be 
accomplished in order to meet the goals and principles of the Vision.  
All Process Participants were to have the opportunity to review the Strategy.  
Metrics for determining the progress towards and achievement of each 
strategic objective were also to be defined.

Decision makers and providers in the public and private sector were then 
to develop a detailed Plan which recommends specific, measurable, 
agreed-upon, realistic and time-framed tasks to be accomplished in the 
following year.  The Plan would also recommend responsibility and resource 
assignments.  All Process Participants were to have the opportunity to review 
and comment on the recommended tasks.  The tasks, by their nature, would serve 
as metrics to help determine progress on the Plan.

Implementation of the plan was to be accomplished by the responsible parties 
using designated resources.  As implementation proceeded, data on 
progress and results would be fed back into the process for evaluation 
against the Vision, Strategy and Plan metrics.  On an annual basis the 
current status of the implementation of the plan and data on all metrics 
would be fed back into the initial stages of the Strategic Planning Process.  
Status assessment and Identification of Disparities the would reflect the new 
state of the telecommunications and information infrastructure and the 
Vision, Strategy and Plan would be updated accordingly.  Through the 
application of this process, all parties (the general public, 
stakeholders/key users, decision makers and providers) would become better 
informed with respect to the issues, technologies, and most promising 
applications of the telecommunications infrastructure.  Throughout this 
process, metrics arewould be used to evaluate how the process itself was working 
and to determine how it could be improved.




%%%%%%%%%%%%%%%%%%%%%%%%%%%%%% -*- Mode: Latex -*- %%%%%%%%%%%%%%%%%%%%%%%%%%%%
%% 96-04-hitime.tex -- 
%% Author          : Philip Johnson
%% Created On      : Sun Mar 10 07:11:50 1996
%% Last Modified By: Philip Johnson
%% Last Modified On: Mon Mar 18 11:13:00 1996
%% RCS: $Id: 96-04-hitime.tex,v 1.3 1996/03/18 21:13:08 johnson Exp $
%%%%%%%%%%%%%%%%%%%%%%%%%%%%%%%%%%%%%%%%%%%%%%%%%%%%%%%%%%%%%%%%%%%%%%%%%%%%%%%
%%   Copyright (C) 1996 Philip Johnson
%%%%%%%%%%%%%%%%%%%%%%%%%%%%%%%%%%%%%%%%%%%%%%%%%%%%%%%%%%%%%%%%%%%%%%%%%%%%%%%
%% 

\section{THE HI-TIME SYSTEM}

Project HI-TIME was intended to provide an unprecedented level of public
involvement in the telecommunications strategic planning process.  It was
also intended to change the paradigm of telecommunications strategic
planning from a linear start/stop process to a continuous, incremental
process.

To make the planning process as visible as possible, various intermediate
work products had to be made widely accessable. To make the planning process
open to the broadest spectrum of stakeholders as possible, comments and
critiques of these work products had to be supported. Finally, such a process
had to maintain a high signal-to-noise ratio and be process-driven, unlike
current public mechanisms for collaboration like USENET.  To satisfy these
requirements, we designed an architecture called CA/M (for Coordinated
Agent/Mailbox) which was intended to support moderated input into the
process in the most cost-effective manner possible.  In the initial
phases of design, we made two crucial design decisions which affected
the remainder of the project:

\begin{itemize}
\item {\em The Internet must provide the backbone for collaboration.}

  First, we decided that supporting the level of sustained, active, and
  geographically distributed public involvement through manual
  (non-networked, non-computerized) mechanisms would be prohibitively
  costly.  We decided that the only cost-effective manner to carry out
  the collaborative goals for Project HI-TIME would be to employ the
  Internet as the primary transport, access, storage, and retrieval
  mechanism. The State of Hawaii has dozens of 
  Internet service provider businesses, which provide reliable and
  competitive access to the Internet through a local phone call on Oahu,
  Maui, Kauai, and the Big Island. Residents of other islands could reach
  the Internet through a long distance phone call to these other islands.

  While the Internet was intended to be the {\em primary} infrastructure for input
  and access, Project HI-TIME did not intend that
  participation be {\em restricted} to only
  those stake-holders with Internet access.  Community meetings, phone
  surveys, print and television media, and other mechanisms for
  participation were intended to be considered as additional channels for
  input.  However, all alternative channels would need to be transcribed 
  and published on the Internet in order to be included as part of the
  public record.

\item {\em The World Wide Web must provide the backbone for publication.}

  Given the choice of the Internet as the primary networking
  infrastructure, our second design decision followed naturally: the
  World Wide Web (WWW) would be the primary user interface.  documents
  related to Project HI-TIME.  Similar to the Internet, the WWW was
  viewed as the primary but not exclusive format for information
  publication. Hardcopy, fax, e-mail, and even oral transcription
  mechanisms were envisioned as important secondary avenues for
  publication of HI-TIME information.

\end{itemize}

Having decided upon the Internet and WWW, we then began work on 
mechanism and process-level support for Project HI-TIME. The result
was the CA/M architecture. 


The CA/M architecture consists of two orthogonal components: an architecture
for communication, and an architecture for information.  

\subsection{The CA/M Communication Architecture}

The communication component of the CA/M architecture defined how
information flows between the various participants in Project HI-TIME.
Figure \ref{fig:communication-architecture} illustrates the {\em canonical
  information cycle\/} in the CA/M communication architecture. In this
canonical form of interaction, users learn about Project HI-TIME by
browsing the contents of its WWW server.  They then submit new documents,
critiques, comments, suggestions, or other forms of dialogue to Project
HI-TIME via a WWW form retrieved from the server.  This form activates a
script that converts the information to an e-mail message that is sent to
the Project HI-TIME mailbox.  The CA/M mail agent processes the submission
and forwards it on to one or more of the HI-TIME moderators for review via
e-mail.  The moderators review and possibly revise the submission, then
forward it to the CA/M WWW Agent for inclusion into the server.  The
submission is then publically available to users, which may stimulate
additional commentary.


\begin{figure*}[htbp]
  \centerline{\psfig{figure=cam-architecture2.ps}} 
\caption{The {\em canonical information cycle} in the CA/M Communication Architecture} 
\label{fig:communication-architecture}
\end{figure*}

Figure \ref{fig:communication-architecture} illustrates the most typical
mechanisms for communication in Project HI-TIME, but others are also
possible.  For example, users may obtain information about Project HI-TIME
by mechanisms other than the WWW Server.  For example, a Fax Agent could be
incorporated into the mechanism to work in parallel with the Mail and WWW
Agent, and whose responsibility is to (a) send fax versions of submissions
and updates to organizations without Internet/WWW Access, and (b) receive
faxes, convert to ASCII format, and forward via e-mail to the CA/M Internet
mailbox.

Other communication paths are possible. For example, 
the HI-TIME moderators could reject the submission, in which case it would be
returned to the submitting user with a reason why it is not being included.
Such rejections are also mediated through the CA/M mail agent---the
moderators do not reply to the user directly, but rather forward the
submission back to the CA/M Internet mailbox with an included directive to
the Mail Agent to return the submission to the sender with the explanatory
comment.  By going through the Mail Agent, the system can automatically
maintain an audit trail of all interactions between Project HI-TIME and the
user community.

There are several important design properties of the CA/M architecture
that influenced the course of Project HI-TIME:

\begin{itemize}


\item{\em Agent-based support.}

Effective coordination of the participants in Project HI-TIME 
appeared to demand that
information be communicated reliably and efficiently.  In addition, an
audit trail of interactions seemed to be required to provide
accountability for actions and data necessary for
process improvement and optimization.  To  support this,  the CA/M
architecture used agents. 

\item {\em Moderator-based evaluation of input.} 

  If e-mail and WWW forms are used to obtain input to Project HI-TIME from
  the community, we worried about the ``signal-to-noise'' ratio within
  Project HI-TIME.  While democratic ideals would seem to argue for
  unrestricted and uncensored publication of viewpoints, experience with
  USENET demonstrated to us that the quality of unmoderated information
  channels can quickly deteriorate without editorial supervision.  Therefore,
  the CA/M communication architecture put moderators ``in the loop'', who
  were to ensure that postings were of sufficient quality. If they weren't,
  the moderator was supposed to send them back to the poster for editorial
  revision when necessary.  The goal was to create a forum for serious, high
  quality discussion on telecommunications policy planning that excludes
  frivolity and gratuitous ``flaming''.  Since agents maintain an audit trail
  of moderator/user interaction, claims of abuse of moderator power could be
  investigated.


\item{\em Security and Authentication.}  

  An important concern in Project HI-TIME was authentication: how can the
  identity of users and HI-TIME moderators be guaranteed?  It is relatively
  easy for Internet crackers to ``spoof'' e-mail, making it appear to come
  from other users.  As a result, it would be possible for unscrupulous users to
  submit postings that appear to come from high-ranking state government or
  telecommunications officials. Worse, it would be  equally possible for users to
  spoof the HI-TIME moderators, and thus direct the CA/M agents to perform
  unauthorized actions.

  We settled upon a two tier authentication scheme in the CA/M architecture,
  one for the users, and one for the moderators.  In the case of users,
  certain types of comments could be made anonymously with all identifying
  information stripped. In these cases, authentication would not be an issue.
  In situations in which identity is preserved and authentication prior to
  publication is required, then some alternate means of authentication would
  be provided.  Manual approaches include authentication by fax, phone, US
  mail, or other mechanisms.

  Spoofed postings by users could be irritating, but the damage would be
  restricted to misattributed publication of information which could be
  later retracted. On the other hand, spoofed moderator commands to the
  CA/M agents could potentially result in complete destruction of the
  server's information database, requiring total reconstruction from
  backup and loss of recent submissions.  Stronger safeguards against
  spoofing of HI-TIME moderators were provided to prevent this from
  occurring.  We decided upon a digital signature using PGP that would be
  attached to each message to an Agent from a HI-TIME moderator.  Digital
  signatures could also be employed by users who wish to prevent spoofing
  and who do not wish to employ the more time-consuming forms of manual
  authentication.


\item {\em Roles.}  

  Figure \ref{fig:communication-architecture} shows two kinds of
  participants in Project HI-TIME: ``users'' and ``HI-TIME moderators''.
  These two forms of participants are distinguished by the nature of
  their interaction with the system: users browse and submit information,
  while moderators review information and interact with the Agents.
  ``User'' and ``HI-TIME moderator'' are two instances of {\em roles\/}
  in the CA/M architecture. 

  We expected that the User and HI-TIME moderator roles would each form
  the root of a hierarchy of role types.  Some roles would be based upon
  the organizational perspective or interest of the participant. For
  example, participants who work for an internet service provider would
  wish to be classified in that user role, while participants from cable
  companies might desire their own classification.  Role types would be
  used to tailor information or access in a manner appropriate to the
  individual.

\end{itemize}

\subsection{The CA/M Information Architecture}

The communication architecture described above provides the general design
for flow of information in Project HI-TIME, but does not specify the
structure or content of this information.  The CA/M information
architecture specified how information was structured within the system.
The information architecture is hypertext, and consists at the most
abstract level of entities and links.

\subsubsection{Entities}

The content-bearing entities in the CA/M information architecture were
classified into two non-overlapping hierarchies of information objects:
{\em document\/} and {\em comment\/}.

Documents are incrementally generated,
reviewed, refined, and modified over time.  For example, a ``Project
HI-TIME Vision Statement'' artifact would be stored in the system as a document
object.  All document objects had the following essential properties:

\begin{itemize}

\item {\em Mutability.\/} Document objects can be edited.  Simple
  concurrency control mechanisms (such as RCS) would support shared
  editing. 

\item {\em Versioning.\/} Management of evolving documents requires the
  ability to represent and maintain a sequence of versions.  Each
  document object exists as a sequence of versions, numbered sequentially
  starting with 1.  Only the highest numbered version of a document can
  be changed.  The current highest numbered version of a document is
  referred to as the {\em development\/} version.

  The {\em freeze\/} operation can be applied to the development version
  of a document object.  This produces a new development version, which
  is simply a copy of the previous one with an incremented version
  number. Freezing also disallows any further changes to the previous
  object.  Only specially designated HI-TIME editors can edit or freeze
  document objects.  Freezing can be done at any time by qualified
  personnel, and may occasionally be done automatically by CA/M agents.

\end{itemize}


Comment entities, however, provided annotations, feedback,
enhancements, and suggestions. They could exist either as stand-alone entities
within Project HI-TIME, or as links to pre-existing document or comment
entities. Comments contrast with documents by being non-mutable and
non-versioning. 

\subsubsection{Links}

Every document and comment entity in the Project HI-TIME WWW server contained
a clickable line called ``Make new comment on this page.''  When clicked,
a form was presented that allows the user to provide commentary on the
entity.  Submission of this form activates a script that results in an
e-mail to HI-TIME moderators with the contents of the comment.  When 
approved, the HI-TIME WWW agent added the comment to the WWW server, 
and updated both the corresponding document and the comment itself
with links to each other.  




%%%%%%%%%%%%%%%%%%%%%%%%%%%%%% -*- Mode: Latex -*- %%%%%%%%%%%%%%%%%%%%%%%%%%%%
%% 96-04-exper.tex -- 
%% Author          : Philip Johnson
%% Created On      : Sun Mar 10 07:13:19 1996
%% Last Modified By: Philip Johnson
%% Last Modified On: Mon Mar 18 11:17:57 1996
%% RCS: $Id: 96-04-exper.tex,v 1.3 1996/03/18 21:18:01 johnson Exp $
%%%%%%%%%%%%%%%%%%%%%%%%%%%%%%%%%%%%%%%%%%%%%%%%%%%%%%%%%%%%%%%%%%%%%%%%%%%%%%%
%%   Copyright (C) 1996 Philip Johnson
%%%%%%%%%%%%%%%%%%%%%%%%%%%%%%%%%%%%%%%%%%%%%%%%%%%%%%%%%%%%%%%%%%%%%%%%%%%%%%%
%% 

\section{EXPERIENCES WITH HI-TIME}

The initial HI-TIME process plan was to create a Vision document.  This
document was intended to be generic, to solicit the public's input, and to
reflect the desires of Hawaii's people for the future of Telecommunications
infrastructure.

\subsection{September 1995: Initial Release --- The HI-TIME Vision}

To seed the HI-TIME process, we created a rough outline for the
telecommunications strategic plan.  This outline consisted of four main
sections

\begin{itemize}
\item Future Scenarios --- This section was an area for the public to
  input their ideas and visions for the future of Hawaii's
  telecommunications.
\item Guiding Principles --- This section was for proposing and
  discussing the different principles that would guide the
  telecommunications plan.
\item Issues --- This section was intended for soliciting different issues
  about the future scenarios, guiding principles or any other issues that the
  public might have.
\item Background --- This section was intended to be a store house of
  information about Telecommunications.  It would hold pointers to different
  documents and information sources about telecommunications infrastructure
  at the State, Nationals and International levels.
\end{itemize}

When the initial HI-TIME vision was released we received thirteen user
registrations.  These thirteen registrations were primarily the members of
the HI-TIME planning committee.  

From the beginning, we had trouble obtaining comments.  Most of the
comments we received were about the format of the document and not about
telecommunications infrastructure.  After the initial rush of comments
about the structure of the document was complete, we redesigned the HI-TIME
document.

\subsection{October - November 1995: Building the HI-TIME Strategy}

The HI-TIME Strategy's structure was built out of the Vision's building
blocks.  The Guiding Principles were broken down into Principles,
Objectives, and Metrics.  HI-TIME's hypertext structure allowed us to
rearrange the different components into new organizations easily.  The new
document structure was as follows:
\begin{itemize}
\item Introduction --- A brief introduction to the HI-TIME strategic
  planning process.
\item HI-TIME Strategy --- A hierarchical organization of Principles,
  Objectives, Metrics and Results.  This section is the main body of the
  HI-TIME document.  When filled out it would act as the strategic plan.
\item Appendix A. Issues --- This section is identical to the previous
  version.  It was moved to an appendix.
\item Appendix B. Future Scenarios --- The future scenarios was also
  moved to an appendix since it did not directly relate to the strategic
  plan.
\end{itemize}
We received initial feed back from the HI-TIME committee saying that the
format of just asking for comments did not get the public involved enough.
The committee worried that the general public would not know what to
comment on.  We added a few tickler questions to get the reader of the
HI-TIME document to think about the issues.  These questions were supposed
to encourage the reader to provide input to the HI-TIME document. 

At the end of November we submitted the current state of the document to
the HI-TIME committee.  They did not like the lack of input and the lack of
a detailed concrete plan to implement the telecommunications strategy.

\subsection{November 1995 - March 1996: Lack of Input --- Increasing Hits}

Since the end of November the HI-TIME document has been available on the
web.  The HI-TIME site was announced to several World Wide Web indexes and
search engines in late November.  The number of hits to the HI-TIME web
pages has been increasing since November 1995 while the number of comments
and inputs have been falling.  Prior to December 1, 1995 the HI-TIME web
pages were averaging 152 accesses per month.  After December 1, 1995 the
pages are averaging 553 accesses per month.


Even with a reasonable amount of ``hits'', we are not receiving much public
input.  Since the HI-TIME system went on-line we have accepted only 22
comments and 23 extensions to the base HI-TIME document. 

In February 1996, the State Government decided to change the HI-TIME
process.  Instead of using the HI-TIME system to create the
telecommunications strategic plan from the public's input, the State would
form a committee to write the initial draft of the plan, which may
eventually be presented for public comment through the HI-TIME Web pages.



%%%%%%%%%%%%%%%%%%%%%%%%%%%%%% -*- Mode: Latex -*- %%%%%%%%%%%%%%%%%%%%%%%%%%%%
%% 96-04-lessons.tex -- 
%% Author          : Philip Johnson
%% Created On      : Sun Mar 10 07:14:24 1996
%% Last Modified By: David C. Brauer
%% Last Modified On: Mon Mar 18 13:41:06 1996
%% RCS: $Id: 96-04-lessons.tex,v 1.6 1996/03/18 23:41:24 dave Exp $
%%%%%%%%%%%%%%%%%%%%%%%%%%%%%%%%%%%%%%%%%%%%%%%%%%%%%%%%%%%%%%%%%%%%%%%%%%%%%%%
%%   Copyright (C) 1996 Philip Johnson
%%%%%%%%%%%%%%%%%%%%%%%%%%%%%%%%%%%%%%%%%%%%%%%%%%%%%%%%%%%%%%%%%%%%%%%%%%%%%%%
%% 
   
\section{LESSONS LEARNED}

The high level of collaboration experienced in docket 7702, which was the
impetus behind Project HI-TIME, never materialized as expected. Reliance on
the HI-TIME Web site as the primary mechanism for input to the strategic
planning process proved to be a serious mistake. The very low participation
of the general public on the HI-TIME Web Site was not surprising since
access to the Internet and the World Wide Web is still not widespread.
What did surprise the project team was that members of the community and
representatives of the State government directly involved in
telecommunications and information infrastructure issues, who did have
internet access, still chose to not participate.  The majority of the input
to the Web site was generated by only a few individuals (despite the fact
that Web site statistics show that HI-TIME was accessed from several
hundred unique addresses).
 
The cause of this low level of participation has been the subject of much
speculation.  We have developed two general categories of hypotheses
regarding factors which may
have contributed to the demise of the Project HI-TIME process.  The
first category covers problems regarding the execution of the process.  The second
category examines potential problems with the underlying collaborative
tools.  The next two  sections present these hypotheses in the hope that they may serve
to guide future collaborative efforts.


\subsection{Problems with the HI-TIME Process}

\begin{itemize}

\item {\bf The topic was not intrinsically motivating}

Telecommunications policy planning is a rather arcane topic, whose impact
upon the general public is not necessarily obvious. Furthermore, the
general public does not necessarily know what they can contribute to
this topic that will be of genuine use.  One goal of this project was to
enable the public to participate in the process by increasing the
visibility of the process, and lowering the barrier to entry into it.

However, the public did not make much use of the system during its brief
lifetime. We suspect that we did not put enough emphasis into educating
the public on why this topic was of importance to them. We assumed that
if they had access, they would use it.  This was not the case.  The level
of contributory participation may increase as community planning centers
are opened and initial training/review sessions are held.

We are also testing this assumption by fielding another system based upon
the CA/M architecture which has a much broader appeal.  Ke Ala Hoku:
Community Benchmarking for Hawaii, is an initiative which began with a
vision of a preferred future for Hawaii articulated by over 3000 K-12
students from across the State.  Ke Ala Hoku now seeks to translate that
vision into measurable benchmarks, which will be used to guide the
allocation of public and private resources to specific actions in pursuit
of the vision.  The benchmarks touch upon all aspects of modern society;
the environment, drug abuse and crime, education, intolerance, appropriate
application of technology, sustainability, etc.  This breadth of topics
gives us confidence that we can isolate the variable of motivation to comment.

\item {\bf Incentives were neither defined nor maintained}

When we began this project, we also assumed that the use of this tool would
be mandated, or at least supported vigorously by the State government.  
This official coercion (or at least blessing) from the political
community would help overcome any natural aversion that the 
telecommmunications industry might have toward working together. 
Unfortunately, due to the low level of "official" participation by State
Government policy personnel, telecommunications stakeholders may not
consider HI-TIME as the right vehicle to raise their concerns. There was little motivation for them
to invest the time (and reveal their strategies) through 
involvement in this system.

\item {\bf The project was not promoted}

The root cause may be simply that the HI-TIME Web site has not yet been heavily
promoted both to the on-line community and in the local media.  Informal
surveys of community members and media reveals that very few people even
know project HI-TIME exists.  Several recommendations were made to the
State of Hawaii officials in charge of this project that a high level media
event be used to kick-off the system.  In fact, two press releases were
drafted but never released for reasons unknown to the authors. Promotion of
a world wide web collaborative tool only on the world wide web is clearly
insufficient to bring an audience from the 'general public' to the site.

\item {\bf The target was not articulated clearly enough}

Because the process was oriented toward an incremental, emergent approach
to telecommunications policy plan generation, there was little need felt
to come up with a concrete, exemplary telecommunications planning
document as an example of what the system should strive to create.  The
lack of a clear "target" led to miscommunication between the State
Agency funding the project and the project team as to the ultimate goal
of the system and the process.  It may also have contributed to some degree
of confusion regarding how to interact with the web site.  Some of the
early feedback from HI-TIME participants indicated that they
didn't know how to respond.  This led to the development of tickler
questions and a more structured document, however, these modifications did
not yield any greater degree of participation.  In fact, participation on
the site was even lower after these enhancments.

We are now in the process of testing this assumption.  The State of
Hawaii has elected to have a blue ribbon panel draft the Strategic Plan
which will the be disseminated for public comment.  The HI-TIME system will
be used as the vehicle to collect public commentary.  If we experience a
much higher degree of participation, then it may be due in part to a
clearly articulated target, although political incentives and promotion of
the site will also be contributory factors.

\end{itemize}

\subsection{Problems with the HI-TIME Collaborative Tools}

\begin{itemize}

\item {\bf Collaboration was developed too independently of content}

One mistake made in this project was the creation of a division between the
"collaboration architecture" people and the "telecommunications planning"
people.  The architecture people worked relatively independently of the
telecommunications planning people on the system. The telecommunications
planning people waited until the system was running to try to introduce any
content.  The architecture people proceeded from the assumption that a
generic CSCW architecture could be instantiated for this domain, which was
not necessarily true.  We discovered soon after the launch of the Web site
that the
telecommunications planners in the State preferred a much more structured
document and more clearly defined areas for comment.  Had the
telecommunications planners been more intimately involved in the initial
design of the system, an entirely different form of collaboration may have
emerged.  To our embarassment as software engineers, we negleted the time
tested truth -- know and communicate with your users.

\item {\bf Tool support was an insufficient condition for collaboration}

When we began this project, we assumed that the existance of 
the system would precipitate extensive involvement from at least the 
major players in telecommunications.  Because the process of
planning and the artifacts were publically visible over the web, 
we then expected to draw in the general public. 

However, the telecommunications industry is intrinsically competitive, not
cooperative. Just having a tool does not instantly create a collaborative
community where one did not previously exist.  Although we had evidence
that this community could collaborate by virtue of the results of PUC 7702,
we overlooked one glaring fact.  The ``Communication Infrastructure
Collaborative'' came about because of a PUC order.  The participants and
intervenors in PUC 7702 had to participate or risk detrimental outcomes in
the formal proceedings.  They had a strong legal and economic incentive to
collaborate.  HI-TIME, on the other hand, asked them to collaborate ``for the
good of the general public''.

\item {\bf The latency between submission and publication was too high}

By placing a moderator in the loop to protect the information integrity of
the evolving project HI-TIME, we introduced a rather significant latency
time between the submission of a comment and its appearance in the on-line
document.  Most modern computer users have come to expect
immediate (or at least timely) feedback from interactive information
systems.  In HI-TIME, a user does receive immediate feedback that their
comment has been forwarded to the moderator for review.  In most cases,
this type of feedback would seem to be sufficient. However, we discovered
through our own use of the system that this latency time actually made it
difficult to execute the HI-TIME process as defined.  The process dictated
that principles were further defined by objectives which were then
amplified by metrics.  Because of the latency time, it was difficult to
propose a new principle, then follow up immediately with related objectives
and metrics.  There was no way of referring to the principle you just
submitted!  We were forced by the collaborative tool to work the process
breadth first rather than depth first.  

\item {\bf Reducing the signal to noise ratio also reduced the signal}

Early on in the design of the CSCW system, we decided for a moderator-based
system to prevent the low signal-to-noise ratio often seen on public
systems such as USENET. In addition, we wanted to make sure that we could
block any attempts to impersonate or misrepresent the positions of State of
Hawaii officials, Telecommuncations companies, and others.  The moderator
mechanism was our solution to these potential problems.  But these sorts of
problems usually only emerge in heavily used systems like UseNet.  Our
system had a very limited number of users.

In retrospect, our concerns may have been better addressed by having two
areas for feedback.  The ``unofficial'' feedback area would immediately
display user's comments and ideas, perhaps stimulating dialog with other
users.  At regular intervals, moderators would then review the ``unofficial''
area, filtering out inappropriate feedback and condensing the information,
and place the filtered information in the ``official'' feedback area.


\end{itemize}

\section{SUMMARY}

Unfortunately, with respect to the Project HI-TIME collaborative planning
process, we will never have the opportunity to determine which of our
hypotheses actually led to its demise. The State of Hawaii
Telecommunications Strategic Plan must be completed in a timely fashion.
It is our hope that these experiences will support the design of better
public collaborative processes and tools in the future.



\newpage
\bibliography{/group/csdl/bib/csdl-trs,96-04}
\bibliographystyle{plain}

\end{document}

