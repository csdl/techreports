%%%%%%%%%%%%%%%%%%%%%%%%%%%%%% -*- Mode: Latex -*- %%%%%%%%%%%%%%%%%%%%%%%%%%%%
%% 96-04-intro.tex -- 
%% Author          : Philip Johnson
%% Created On      : Sun Mar 10 07:07:40 1996
%% Last Modified By: Philip Johnson
%% Last Modified On: Mon Mar 18 10:54:38 1996
%% RCS: $Id: 96-04-intro.tex,v 1.4 1996/03/18 20:54:49 johnson Exp $
%%%%%%%%%%%%%%%%%%%%%%%%%%%%%%%%%%%%%%%%%%%%%%%%%%%%%%%%%%%%%%%%%%%%%%%%%%%%%%%
%%   Copyright (C) 1996 Philip Johnson
%%%%%%%%%%%%%%%%%%%%%%%%%%%%%%%%%%%%%%%%%%%%%%%%%%%%%%%%%%%%%%%%%%%%%%%%%%%%%%%
%% 

\section{INTRODUCTION}


The explosive growth of the Internet and WWW has provided new opportunities
to experiment with collaborative technologies formerly limited to research
systems and sophisticated user communities.  This paper describes an
attempt to exploit the opportunity of the Internet and WWW to provide a
collaborative system for use by the general public to support
telecommunications policy planning in the State of Hawaii.

This computer-mediated, collaborative planning process must ultimately be
considered a failure, even though the design appeared sound, the
implementation was solid, and preliminary use seemed acceptable.  This
paper presents our effort to explain why this happened and how future
developers of public CSCW systems can benefit from our experiences.
 
The paper is organized as follows. The next section provides a background
on the telecommunications policy planning process surrounding Project
HI-TIME. The following section briefly describes the design and
implementation of the Project HI-TIME system.  The next section details the
operational life of the Project HI-TIME system. The final section
presents our hypotheses as to why Project HI-TIME did not succeed, and 
recommendations to designers of public CSCW systems.

