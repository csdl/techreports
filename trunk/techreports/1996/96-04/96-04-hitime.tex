%%%%%%%%%%%%%%%%%%%%%%%%%%%%%% -*- Mode: Latex -*- %%%%%%%%%%%%%%%%%%%%%%%%%%%%
%% 96-04-hitime.tex -- 
%% Author          : Philip Johnson
%% Created On      : Sun Mar 10 07:11:50 1996
%% Last Modified By: Philip Johnson
%% Last Modified On: Mon Mar 18 11:13:00 1996
%% RCS: $Id: 96-04-hitime.tex,v 1.3 1996/03/18 21:13:08 johnson Exp $
%%%%%%%%%%%%%%%%%%%%%%%%%%%%%%%%%%%%%%%%%%%%%%%%%%%%%%%%%%%%%%%%%%%%%%%%%%%%%%%
%%   Copyright (C) 1996 Philip Johnson
%%%%%%%%%%%%%%%%%%%%%%%%%%%%%%%%%%%%%%%%%%%%%%%%%%%%%%%%%%%%%%%%%%%%%%%%%%%%%%%
%% 

\section{THE HI-TIME SYSTEM}

Project HI-TIME was intended to provide an unprecedented level of public
involvement in the telecommunications strategic planning process.  It was
also intended to change the paradigm of telecommunications strategic
planning from a linear start/stop process to a continuous, incremental
process.

To make the planning process as visible as possible, various intermediate
work products had to be made widely accessable. To make the planning process
open to the broadest spectrum of stakeholders as possible, comments and
critiques of these work products had to be supported. Finally, such a process
had to maintain a high signal-to-noise ratio and be process-driven, unlike
current public mechanisms for collaboration like USENET.  To satisfy these
requirements, we designed an architecture called CA/M (for Coordinated
Agent/Mailbox) which was intended to support moderated input into the
process in the most cost-effective manner possible.  In the initial
phases of design, we made two crucial design decisions which affected
the remainder of the project:

\begin{itemize}
\item {\em The Internet must provide the backbone for collaboration.}

  First, we decided that supporting the level of sustained, active, and
  geographically distributed public involvement through manual
  (non-networked, non-computerized) mechanisms would be prohibitively
  costly.  We decided that the only cost-effective manner to carry out
  the collaborative goals for Project HI-TIME would be to employ the
  Internet as the primary transport, access, storage, and retrieval
  mechanism. The State of Hawaii has dozens of 
  Internet service provider businesses, which provide reliable and
  competitive access to the Internet through a local phone call on Oahu,
  Maui, Kauai, and the Big Island. Residents of other islands could reach
  the Internet through a long distance phone call to these other islands.

  While the Internet was intended to be the {\em primary} infrastructure for input
  and access, Project HI-TIME did not intend that
  participation be {\em restricted} to only
  those stake-holders with Internet access.  Community meetings, phone
  surveys, print and television media, and other mechanisms for
  participation were intended to be considered as additional channels for
  input.  However, all alternative channels would need to be transcribed 
  and published on the Internet in order to be included as part of the
  public record.

\item {\em The World Wide Web must provide the backbone for publication.}

  Given the choice of the Internet as the primary networking
  infrastructure, our second design decision followed naturally: the
  World Wide Web (WWW) would be the primary user interface.  documents
  related to Project HI-TIME.  Similar to the Internet, the WWW was
  viewed as the primary but not exclusive format for information
  publication. Hardcopy, fax, e-mail, and even oral transcription
  mechanisms were envisioned as important secondary avenues for
  publication of HI-TIME information.

\end{itemize}

Having decided upon the Internet and WWW, we then began work on 
mechanism and process-level support for Project HI-TIME. The result
was the CA/M architecture. 


The CA/M architecture consists of two orthogonal components: an architecture
for communication, and an architecture for information.  

\subsection{The CA/M Communication Architecture}

The communication component of the CA/M architecture defined how
information flows between the various participants in Project HI-TIME.
Figure \ref{fig:communication-architecture} illustrates the {\em canonical
  information cycle\/} in the CA/M communication architecture. In this
canonical form of interaction, users learn about Project HI-TIME by
browsing the contents of its WWW server.  They then submit new documents,
critiques, comments, suggestions, or other forms of dialogue to Project
HI-TIME via a WWW form retrieved from the server.  This form activates a
script that converts the information to an e-mail message that is sent to
the Project HI-TIME mailbox.  The CA/M mail agent processes the submission
and forwards it on to one or more of the HI-TIME moderators for review via
e-mail.  The moderators review and possibly revise the submission, then
forward it to the CA/M WWW Agent for inclusion into the server.  The
submission is then publically available to users, which may stimulate
additional commentary.


\begin{figure*}[htbp]
  \centerline{\psfig{figure=cam-architecture2.ps}} 
\caption{The {\em canonical information cycle} in the CA/M Communication Architecture} 
\label{fig:communication-architecture}
\end{figure*}

Figure \ref{fig:communication-architecture} illustrates the most typical
mechanisms for communication in Project HI-TIME, but others are also
possible.  For example, users may obtain information about Project HI-TIME
by mechanisms other than the WWW Server.  For example, a Fax Agent could be
incorporated into the mechanism to work in parallel with the Mail and WWW
Agent, and whose responsibility is to (a) send fax versions of submissions
and updates to organizations without Internet/WWW Access, and (b) receive
faxes, convert to ASCII format, and forward via e-mail to the CA/M Internet
mailbox.

Other communication paths are possible. For example, 
the HI-TIME moderators could reject the submission, in which case it would be
returned to the submitting user with a reason why it is not being included.
Such rejections are also mediated through the CA/M mail agent---the
moderators do not reply to the user directly, but rather forward the
submission back to the CA/M Internet mailbox with an included directive to
the Mail Agent to return the submission to the sender with the explanatory
comment.  By going through the Mail Agent, the system can automatically
maintain an audit trail of all interactions between Project HI-TIME and the
user community.

There are several important design properties of the CA/M architecture
that influenced the course of Project HI-TIME:

\begin{itemize}


\item{\em Agent-based support.}

Effective coordination of the participants in Project HI-TIME 
appeared to demand that
information be communicated reliably and efficiently.  In addition, an
audit trail of interactions seemed to be required to provide
accountability for actions and data necessary for
process improvement and optimization.  To  support this,  the CA/M
architecture used agents. 

\item {\em Moderator-based evaluation of input.} 

  If e-mail and WWW forms are used to obtain input to Project HI-TIME from
  the community, we worried about the ``signal-to-noise'' ratio within
  Project HI-TIME.  While democratic ideals would seem to argue for
  unrestricted and uncensored publication of viewpoints, experience with
  USENET demonstrated to us that the quality of unmoderated information
  channels can quickly deteriorate without editorial supervision.  Therefore,
  the CA/M communication architecture put moderators ``in the loop'', who
  were to ensure that postings were of sufficient quality. If they weren't,
  the moderator was supposed to send them back to the poster for editorial
  revision when necessary.  The goal was to create a forum for serious, high
  quality discussion on telecommunications policy planning that excludes
  frivolity and gratuitous ``flaming''.  Since agents maintain an audit trail
  of moderator/user interaction, claims of abuse of moderator power could be
  investigated.


\item{\em Security and Authentication.}  

  An important concern in Project HI-TIME was authentication: how can the
  identity of users and HI-TIME moderators be guaranteed?  It is relatively
  easy for Internet crackers to ``spoof'' e-mail, making it appear to come
  from other users.  As a result, it would be possible for unscrupulous users to
  submit postings that appear to come from high-ranking state government or
  telecommunications officials. Worse, it would be  equally possible for users to
  spoof the HI-TIME moderators, and thus direct the CA/M agents to perform
  unauthorized actions.

  We settled upon a two tier authentication scheme in the CA/M architecture,
  one for the users, and one for the moderators.  In the case of users,
  certain types of comments could be made anonymously with all identifying
  information stripped. In these cases, authentication would not be an issue.
  In situations in which identity is preserved and authentication prior to
  publication is required, then some alternate means of authentication would
  be provided.  Manual approaches include authentication by fax, phone, US
  mail, or other mechanisms.

  Spoofed postings by users could be irritating, but the damage would be
  restricted to misattributed publication of information which could be
  later retracted. On the other hand, spoofed moderator commands to the
  CA/M agents could potentially result in complete destruction of the
  server's information database, requiring total reconstruction from
  backup and loss of recent submissions.  Stronger safeguards against
  spoofing of HI-TIME moderators were provided to prevent this from
  occurring.  We decided upon a digital signature using PGP that would be
  attached to each message to an Agent from a HI-TIME moderator.  Digital
  signatures could also be employed by users who wish to prevent spoofing
  and who do not wish to employ the more time-consuming forms of manual
  authentication.


\item {\em Roles.}  

  Figure \ref{fig:communication-architecture} shows two kinds of
  participants in Project HI-TIME: ``users'' and ``HI-TIME moderators''.
  These two forms of participants are distinguished by the nature of
  their interaction with the system: users browse and submit information,
  while moderators review information and interact with the Agents.
  ``User'' and ``HI-TIME moderator'' are two instances of {\em roles\/}
  in the CA/M architecture. 

  We expected that the User and HI-TIME moderator roles would each form
  the root of a hierarchy of role types.  Some roles would be based upon
  the organizational perspective or interest of the participant. For
  example, participants who work for an internet service provider would
  wish to be classified in that user role, while participants from cable
  companies might desire their own classification.  Role types would be
  used to tailor information or access in a manner appropriate to the
  individual.

\end{itemize}

\subsection{The CA/M Information Architecture}

The communication architecture described above provides the general design
for flow of information in Project HI-TIME, but does not specify the
structure or content of this information.  The CA/M information
architecture specified how information was structured within the system.
The information architecture is hypertext, and consists at the most
abstract level of entities and links.

\subsubsection{Entities}

The content-bearing entities in the CA/M information architecture were
classified into two non-overlapping hierarchies of information objects:
{\em document\/} and {\em comment\/}.

Documents are incrementally generated,
reviewed, refined, and modified over time.  For example, a ``Project
HI-TIME Vision Statement'' artifact would be stored in the system as a document
object.  All document objects had the following essential properties:

\begin{itemize}

\item {\em Mutability.\/} Document objects can be edited.  Simple
  concurrency control mechanisms (such as RCS) would support shared
  editing. 

\item {\em Versioning.\/} Management of evolving documents requires the
  ability to represent and maintain a sequence of versions.  Each
  document object exists as a sequence of versions, numbered sequentially
  starting with 1.  Only the highest numbered version of a document can
  be changed.  The current highest numbered version of a document is
  referred to as the {\em development\/} version.

  The {\em freeze\/} operation can be applied to the development version
  of a document object.  This produces a new development version, which
  is simply a copy of the previous one with an incremented version
  number. Freezing also disallows any further changes to the previous
  object.  Only specially designated HI-TIME editors can edit or freeze
  document objects.  Freezing can be done at any time by qualified
  personnel, and may occasionally be done automatically by CA/M agents.

\end{itemize}


Comment entities, however, provided annotations, feedback,
enhancements, and suggestions. They could exist either as stand-alone entities
within Project HI-TIME, or as links to pre-existing document or comment
entities. Comments contrast with documents by being non-mutable and
non-versioning. 

\subsubsection{Links}

Every document and comment entity in the Project HI-TIME WWW server contained
a clickable line called ``Make new comment on this page.''  When clicked,
a form was presented that allows the user to provide commentary on the
entity.  Submission of this form activates a script that results in an
e-mail to HI-TIME moderators with the contents of the comment.  When 
approved, the HI-TIME WWW agent added the comment to the WWW server, 
and updated both the corresponding document and the comment itself
with links to each other.  



