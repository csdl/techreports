%%%%%%%%%%%%%%%%%%%%%%%%%%%%%% -*- Mode: Latex -*- %%%%%%%%%%%%%%%%%%%%%%%%%%%%
%% intro.tex -- 
%% Author          : Carleton Moore
%% Created On      : Fri Oct 18 16:20:55 1996
%% Last Modified By: Carleton Moore
%% Last Modified On: Wed Nov 20 16:41:34 1996
%% RCS: $Id: intro.tex,v 1.4 1996/11/21 02:42:55 cmoore Exp $
%%%%%%%%%%%%%%%%%%%%%%%%%%%%%%%%%%%%%%%%%%%%%%%%%%%%%%%%%%%%%%%%%%%%%%%%%%%%%%%
%%   Copyright (C) 1996 Carleton Moore
%%%%%%%%%%%%%%%%%%%%%%%%%%%%%%%%%%%%%%%%%%%%%%%%%%%%%%%%%%%%%%%%%%%%%%%%%%%%%%%
%% 

\section{Introduction}
\label{sec:intro}

\subsection{Motivation}

\subsection{Theses}
\begin{enumerate}
\item {An appropriately designed computer-mediated formal technical review
    environment will be adopted in an industrial environment and will yield
    substantial improvements in measures of review efficiency and
    effectiveness without introducing metrics dysfunction.}

\item {The AFTR technology transfer method's instrumentation will provide valuable
    insights into the strengths and weaknesses of this technology transfer
    method.}

\end{enumerate}

\subsubsection{Hypotheses}
\begin{enumerate}
\item[H1.1:]{There will be observable differences in review
effectiveness between the manual process and the automated process.}
\item[H1.2:]{There will be a observable difference in cost between
the manual FTR process and the automated FTR process.}
\item[H1.3:]{The organization will adopt AFTR as their FTR tool.}
\item[H1.4:]{The participants will prefer to use the automated FTR
process over the manual FTR process.}
\item[H2.1:]{Analysis of the process defects will lead to Process
Improvement Proposals (PIPs).}
\item[H2.2:]{Process Improvement Proposals (PIPs) from the method will
provide insight to improving the technology transfer method.}
\end{enumerate}

\subsubsection{Research Questions}
\begin{itemize}
\item{How do we develop a system that measures FTR without introducing
dysfunction?}
\item{Does a system that measures FTR without introducing dysfunction,
provide management with the answers they want?}
\item{Will a FTR system without dysfunction be adopted?}
\end{itemize}

\subsubsection{Definition of Improvement:}
\begin{itemize}
\item{less effort}
  \begin{itemize}
  \item{overall time spent in TekInspect}
  \item{individuals spend less time doing AFTR than TekInspect}
  \item{less rework time per defect}
  \item{reduces scribe's work}
  \item{makes group meetings more effective}
  \end{itemize}
\item{more defects found}
  \begin{itemize}
  \item{allows reviewers to focus on review not process}
  \item{more severe errors than minor}
  \item{defects per hour goes up}
  \end{itemize}
\item{participants like new automated system}
  \begin{itemize}
  \item{participants feelings toward FTR improves}
  \item{participants want to have their work reviewed}
  \item{Esprit de corps}
  \item{team building}
  \end{itemize}
\item{more groups use FTR with AFTR than with out}
  \begin{itemize}
  \item{number of groups using TekInspect vs AFTR}
  \item{more groups asking to use AFTR}
  \end{itemize}
\item{Management sees benefits of FTR from AFTR}
  \begin{itemize}
  \item{cost savings}
  \item{more predictability for reviews using AFTR}
  \item{reduced time for projects using AFTR}
  \item{Management want teams to use AFTR}
  \end{itemize}
\end{itemize}

This proposal is organized as following: Section \ref{sec:related} relates
the current research to the broader context of existing work.  Section
\ref{sec:method} introduces our method for introducing automated FTR into
an organization.  Section \ref{sec:manual} describes the manual FTR method
being used.  Section \ref{sec:AFTR} depicts the main design features and
architecture of AFTR.  Section \ref{sec:experiment} outlines the
experimental case study I plan to conduct to evaluate AFTR.  Finally,
Section \ref{sec:plan} presents the current research plan.

