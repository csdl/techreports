%%%%%%%%%%%%%%%%%%%%%%%%%%%%%% -*- Mode: Latex -*- %%%%%%%%%%%%%%%%%%%%%%%%%%%%
%% cscw96.tex -- 
%% Author          : Rosemary Andrada
%% Created On      : Mon Feb 19 12:11:32 1996
%% Last Modified By: Rosemary Andrada
%% Last Modified On: Sun Mar  3 15:35:43 1996
%% RCS: $Id$
%%%%%%%%%%%%%%%%%%%%%%%%%%%%%%%%%%%%%%%%%%%%%%%%%%%%%%%%%%%%%%%%%%%%%%%%%%%%%%%
%%   Copyright (C) 1996 Rosemary Andrada
%%%%%%%%%%%%%%%%%%%%%%%%%%%%%%%%%%%%%%%%%%%%%%%%%%%%%%%%%%%%%%%%%%%%%%%%%%%%%%%
%% 

\documentstyle[nftimes,/group/csdl/tex/CHI95]{article}

\begin{document}

\title{Building Community in an Organization by Appealing to its Online
Counterpart:\\  
A Look at a Community-Based Web Site}

\author{
\bf{Rosemary Andrada} \\
Collaborative Software Development Laboratory \\
2565 The Mall \\
Honolulu, Hawaii 96822 \\
(808) 845-9291 \\
rosea@hcc.Hawaii.Edu}

\maketitle

\abstract 

This paper presents a case study designed to assess the strengths and
weaknesses of a computer-based approach to improving the sense of community
within one organization, the Department of Information and Computer Sciences at
the University of Hawaii.  The relatively high usage of the system was a
function of its novelty rather than its role as a tool for community building.
This mode of communication increased self-awareness in the organization to a
level which had a polarizing effect.  However, users found it to be a useful
method for communicating static, referential information.

\paragraph{KEYWORDS:} World Wide Web (WWW), building community

\section{INTRODUCTION}

%Brief audience on entire set of events making sure they WANT to read more
I'm still reading Network Nation and other sociology texts for their views on
the influence of technology on organizations.

\section{EXPERIMENTAL DESIGN}

%How did I conduct my study?  Looked at empirical evidence: questionnaire
%responses
This study used pre-test and post-test questionnaires to assess the level of
community before and after the introduction of the department Web site.  The
experiment lasted the length of a semester from January 10, 1995 to April 30,
1995.  Seminars introducing the World Wide Web and classes teaching users how
to create web pages were conducted for the duration of the experiment.  Logs of
each web page request were kept to evaluate system usage.

\subsection{Questionnaires}
The operational definition of community in this study is based on feelings
people have about themselves in relation to their organization and on their
awareness of other people and their group as a whole.  Evaluating community
involves asking people about their feelings and testing their knowledge of the
organization.  Administering questionnaires was the method of choice due to its
efficiency in reaching as many people as possible with minimal resources.  

The pre-test questionnaire follows:

Please answer the following questions.  Different people may interpret the
meaning of some phrases (such as ``to know personally'') in different
ways. Simply respond in the way that feels best to you.

\begin{enumerate}
\item{Which group do you work with most?}\\
  \begin{tabular}{ll}
    \underline{  }\underline{  }\underline{  }  & Faculty \\
    \underline{  }\underline{  }\underline{  }  & Staff \\
    \underline{  }\underline{  }\underline{  }  & Graduate students \\
    \underline{  }\underline{  }\underline{  }  & Undergraduate students \\
  \end{tabular}

\item{How many [ICS students/faculty members] do you feel you know
  personally?}\\
  \begin{tabular}{ll}
    \underline{  }\underline{  }\underline{  }  & 0-10 \\
    \underline{  }\underline{  }\underline{  }  & 10-20 \\
    \underline{  }\underline{  }\underline{  }  & 20-30 \\
    \underline{  }\underline{  }\underline{  }  & over 30 \\
  \end{tabular}

\item{How many [ICS students/faculty members] do you think you could name,
  given their face?}\\
  \begin{tabular}{ll}
    \underline{  }\underline{  }\underline{  }  & 0-10 \\
    \underline{  }\underline{  }\underline{  }  & 10-20 \\
    \underline{  }\underline{  }\underline{  }  & 20-30 \\
    \underline{  }\underline{  }\underline{  }  & over 30 \\
  \end{tabular}

\item{Please provide three numeric estimates of how many faculty, graduate
  students and  undergraduates you think are in the department:}\\
  \begin{tabular}{ll}
    \underline{  }\underline{  }\underline{  }  & Faculty \\
    \underline{  }\underline{  }\underline{  }  & Graduate students \\
    \underline{  }\underline{  }\underline{  }  & Undergraduate students \\
  \end{tabular}

\item{Do you feel you play a significant role in the ICS department?  Why
  or why not?}
\\ 

\item{Do you feel a sense of belonging in the ICS department?  Why or why not?}
\\ 

\item{What ICS research projects are you aware of?  Please briefly list the
  projects you can think of immediately and any faculty, staff, or students
  you know who are involved in them. (If you know many people involved in a
  particular project, then simply estimate the number of involved people
  and provide that number.)}\\
  \begin{tabular}{ccc|ccc|c}
    & research project & & &  people involved    & & \# involved    \\ \hline
    &&&&&&\\ &&&&&&\\ &&&&&&\\ &&&&&&\\ &&&&&&\\ &&&&&&\\ &&&&&&\\ &&&&&&\\
    &&&&&&\\ &&&&&&\\ &&&&&&\\ &&&&&&\\ &&&&&&\\ &&&&&&\\
  \end{tabular}

\item{Assume you had a question about one of the following topics.  For
  each of the topics, name one or two people in the department who you
  would want to ``just stop by'' to talk with about it, or leave it blank
  if you can't think of anybody.}\\
  \begin{tabular}{l}
  Departmental rules: \\
  Artificial intelligence: \\
  Software Engineering: \\
  Computer networks: \\
  Hypertext and multimedia: \\
  Cognitive Science: \\
  Computer Programming: \\
  Human Computer Interaction: \\
  Computer games: \\
  Employment opportunities: \\
  \end{tabular}

\item{For each of the following types of communication, indicate whether
  you use it Daily, Weekly, Monthly, or Never, to communicate with other
  people in the department.}\\
\\
  \begin{tabular}{ll}
    \underline{  }\underline{  }\underline{  }  & Email\\
    \underline{  }\underline{  }\underline{  }  & Telephone\\
    \underline{  }\underline{  }\underline{  }  & Fax\\
    \underline{  }\underline{  }\underline{  }  & Informal meetings (lunch, etc.)\\
    \underline{  }\underline{  }\underline{  }  & Formal meetings\\
    \underline{  }\underline{  }\underline{  }  & Other (please specify)\\
  \end{tabular}

\item{Do you use Mosaic, or any other Web browser to access the World Wide
  Web?}\\
  \begin{tabular}{ll}
  \underline{  }\underline{  }\underline{  }  & Yes\\
  \underline{  }\underline{  }\underline{  }  & No\\
  \end{tabular}

\item{(Optional) Please write any other comments you have about the sense
  of community in the department below.}

\end{enumerate}

The post-test questionnaire follows:

Please answer the following questions.  Different people may interpret the
meaning of some phrases (such as "to know personally") in different
ways. Simply respond in the way that feels best to you.

\begin{enumerate}

\item{Which group do you work with most?}\\
  \begin{tabular}{ll}
    \underline{  }\underline{  }\underline{  }  & Faculty \\
    \underline{  }\underline{  }\underline{  }  & Staff \\
    \underline{  }\underline{  }\underline{  }  & Graduate students \\
    \underline{  }\underline{  }\underline{  }  & Undergraduate students \\
  \end{tabular}

\item{How many [ICS students/faculty members] do you feel you know
  personally?}\\
  \begin{tabular}{ll}
    \underline{  }\underline{  }\underline{  }  & 0-10 \\
    \underline{  }\underline{  }\underline{  }  & 10-20 \\
    \underline{  }\underline{  }\underline{  }  & 20-30 \\
    \underline{  }\underline{  }\underline{  }  & over 30 \\
  \end{tabular}

\item{How many [ICS students/faculty members] do you think you could name,
  given their face?}\\
  \begin{tabular}{ll}
    \underline{  }\underline{  }\underline{  }  & 0-10 \\
    \underline{  }\underline{  }\underline{  }  & 10-20 \\
    \underline{  }\underline{  }\underline{  }  & 20-30 \\
    \underline{  }\underline{  }\underline{  }  & over 30 \\
  \end{tabular}

\item{Please provide three numeric estimates of how many faculty, graduate
  students and  undergraduates you think are in the department:}\\
  \begin{tabular}{ll}
    \underline{  }\underline{  }\underline{  }  & Faculty \\
    \underline{  }\underline{  }\underline{  }  & Graduate students \\
    \underline{  }\underline{  }\underline{  }  & Undergraduate students \\
  \end{tabular}

\item{Do you feel you play a significant role in the ICS department?  Why
  or why not?}
\\

\item{Do you feel a sense of belonging in the ICS department?  Why or why not?}
\\ 

\item{What ICS research projects are you aware of?  Please briefly list the
  projects you can think of immediately and any faculty, staff, or students
  you know who are involved in them. (If you know many people involved in a
  particular project, then simply estimate the number of involved people
  and provide that number.)}\\
  \begin{tabular}{ccc|ccc|c}
    & research project & & &  people involved    & & \# involved    \\ \hline
    &&&&&&\\ &&&&&&\\ &&&&&&\\ &&&&&&\\ &&&&&&\\ &&&&&&\\ &&&&&&\\ &&&&&&\\
    &&&&&&\\ &&&&&&\\ &&&&&&\\ &&&&&&\\ &&&&&&\\ &&&&&&\\
  \end{tabular}

\item{Assume you had a question about one of the following topics.  For
  each of the topics, name one or two people in the department who you
  would want to ``just stop by'' to talk with about it, or leave it blank
  if you can't think of anybody.}\\
  \begin{tabular}{l}
  Departmental rules: \\
  Artificial intelligence: \\
  Software Engineering: \\
  Computer networks: \\
  Hypertext and multimedia: \\
  Cognitive Science: \\
  Computer Programming: \\
  Human Computer Interaction: \\
  Computer games: \\
  Employment opportunities: \\
  \end{tabular}

\item{For each of the following types of communication, indicate whether
  you use it Daily, Weekly, Monthly, or Never, to communicate with other
  people in the department.}\\
\\
  \begin{tabular}{ll}
    \underline{  }\underline{  }\underline{  }  & Email\\
    \underline{  }\underline{  }\underline{  }  & Telephone\\
    \underline{  }\underline{  }\underline{  }  & Fax\\
    \underline{  }\underline{  }\underline{  }  & Informal meetings (lunch, etc.)\\
    \underline{  }\underline{  }\underline{  }  & Formal meetings\\
    \underline{  }\underline{  }\underline{  }  & Other (please specify)\\
  \end{tabular}

\item{Did you know the ICS Department has a Web site? If so, how did you
  discover this?}\\
  \begin{tabular}{ll}
    \underline{  }\underline{  }\underline{  }  & Flyers\\
    \underline{  }\underline{  }\underline{  }  & Email announcements\\
    \underline{  }\underline{  }\underline{  }  & Colleague or friend\\
    \underline{  }\underline{  }\underline{  }  & My professor\\
    \underline{  }\underline{  }\underline{  }  & I stumbled on it by accident\\
  \end{tabular}

\item{The WWW training sessions offered by the department were...}\\
  \begin{tabular} {ll}
  \underline{  }\underline{  }\underline{  }  & Useful\\
  \underline{  }\underline{  }\underline{  }  & Not very useful\\
  \underline{  }\underline{  }\underline{  }  & Frequently and flexibly scheduled\\
  \underline{  }\underline{  }\underline{  }  & Not easily accessible\\
  \underline{  }\underline{  }\underline{  }  & What training sessions?\\
  \underline{  }\underline{  }\underline{  }  & Other comments:\\
  \end{tabular}

\item{How often have you been accessing information from the ICS
  Department's Web site?}\\
  \begin{tabular}{ll}
  \underline{  }\underline{  }\underline{  }  & Rarely or never\\
  \underline{  }\underline{  }\underline{  }  & A few times a month\\
  \underline{  }\underline{  }\underline{  }  & A few times a week\\
  \underline{  }\underline{  }\underline{  }  & Amost everyday\\
  \end{tabular}

\item{I feel I am a competent...}\\
  \begin{tabular}{ll}
  \underline{  }\underline{  }\underline{  }  & Web surfer\\
  \underline{  }\underline{  }\underline{  }  & Web publisher\\
  \underline{  }\underline{  }\underline{  }  & neither\\
  \end{tabular}

\item{Did you find the information presented about the department useful?
  If so, in what ways?  If not, why not and how would you improve it?}
\\ 

\item{I use the World Wide Web because...}\\
  \begin{tabular}{ll}
  \underline{  }\underline{  }\underline{  }  & It was required by my professor\\
  \underline{  }\underline{  }\underline{  }  & The Web is fun and exciting\\
  \underline{  }\underline{  }\underline{  }  & Everyone else seems to be using it\\
  \underline{  }\underline{  }\underline{  }  & It makes it easy for me to
  navigate the Internet\\
  \underline{  }\underline{  }\underline{  }  & I don't use it\\
  \underline{  }\underline{  }\underline{  }  & Other\\
  \end{tabular}

\item{Did you complete and turn in a similar questionnaire regarding this
  research administered at the beginning of the semester?}\\
  \begin{tabular}{ll}
  \underline{  }\underline{  }\underline{  }  & Yes\\
  \underline{  }\underline{  }\underline{  }  & No\\
  \end{tabular}

\item{(Optional) Please write any other comments you have about the sense of
  community in the department below.}
\\ 
\end{enumerate}

\subsection{Training}


\subsection{System Logs}


\section{PROPERTIES OF A COMMUNITY-ENHANCING WEB SITE}

%Focus is not on World-Wide audience, but internal one.  Includes small town
%ideals: pictures and hobbbies.


\section{RESULTS}

%Participation across three groups.  Questionnaire responses indicate feelings
%of community.


\section{CONCLUSIONS}

%What are strenghts and weaknesses of this approach?
%Strenghts
%1. participation augmented by technology (surging in popularity)
%2. technology available to the commoner (easy to use)
%3. can be conducted asynchronously
%4. system is diagnostic tool for building community
%5. does not require social interaction *insight*
%Weaknesses
%1. lack of interaction
%2. no moderator (no leader, no guidance) *this case*
%3. system cannot be standalone tool

\bibliography{/group/csdl/bib/www-ics,/group/csdl/bib/csdl-trs}
\bibliographystyle{plain}

\end{document}
