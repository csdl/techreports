%%%%%%%%%%%%%%%%%%%%%%%%%%%%%% -*- Mode: Latex -*- %%%%%%%%%%%%%%%%%%%%%%%%%%%%
%% 96-05.tex -- 
%% Author          : Philip Johnson
%% Created On      : Mon Mar 12 07:25:52 1996
%% Last Modified By: Philip Johnson
%% Last Modified On: Fri May 31 10:31:27 1996
%% RCS: $Id$
%%%%%%%%%%%%%%%%%%%%%%%%%%%%%%%%%%%%%%%%%%%%%%%%%%%%%%%%%%%%%%%%%%%%%%%%%%%%%%%
%%   Copyright (C) 1996 Philip Johnson
%%%%%%%%%%%%%%%%%%%%%%%%%%%%%%%%%%%%%%%%%%%%%%%%%%%%%%%%%%%%%%%%%%%%%%%%%%%%%%%
%% 

\documentstyle[nftimes,/group/csdl/tex/CHI95]{article}
% Psfig/TeX 
\def\PsfigVersion{1.9}
% dvips version
%
% All psfig/tex software, documentation, and related files
% in this distribution of psfig/tex are 
% Copyright 1987, 1988, 1991 Trevor J. Darrell
%
% Permission is granted for use and non-profit distribution of psfig/tex 
% providing that this notice is clearly maintained. The right to
% distribute any portion of psfig/tex for profit or as part of any commercial
% product is specifically reserved for the author(s) of that portion.
%
% *** Feel free to make local modifications of psfig as you wish,
% *** but DO NOT post any changed or modified versions of ``psfig''
% *** directly to the net. Send them to me and I'll try to incorporate
% *** them into future versions. If you want to take the psfig code 
% *** and make a new program (subject to the copyright above), distribute it, 
% *** (and maintain it) that's fine, just don't call it psfig.
%
% Bugs and improvements to trevor@media.mit.edu.
%
% Thanks to Greg Hager (GDH) and Ned Batchelder for their contributions
% to the original version of this project.
%
% Modified by J. Daniel Smith on 9 October 1990 to accept the
% %%BoundingBox: comment with or without a space after the colon.  Stole
% file reading code from Tom Rokicki's EPSF.TEX file (see below).
%
% More modifications by J. Daniel Smith on 29 March 1991 to allow the
% the included PostScript figure to be rotated.  The amount of
% rotation is specified by the "angle=" parameter of the \psfig command.
%
% Modified by Robert Russell on June 25, 1991 to allow users to specify
% .ps filenames which don't yet exist, provided they explicitly provide
% boundingbox information via the \psfig command. Note: This will only work
% if the "file=" parameter follows all four "bb???=" parameters in the
% command. This is due to the order in which psfig interprets these params.
%
%  3 Jul 1991	JDS	check if file already read in once
%  4 Sep 1991	JDS	fixed incorrect computation of rotated
%			bounding box
% 25 Sep 1991	GVR	expanded synopsis of \psfig
% 14 Oct 1991	JDS	\fbox code from LaTeX so \psdraft works with TeX
%			changed \typeout to \ps@typeout
% 17 Oct 1991	JDS	added \psscalefirst and \psrotatefirst
%

% From: gvr@cs.brown.edu (George V. Reilly)
%
% \psdraft	draws an outline box, but doesn't include the figure
%		in the DVI file.  Useful for previewing.
%
% \psfull	includes the figure in the DVI file (default).
%
% \psscalefirst width= or height= specifies the size of the figure
% 		before rotation.
% \psrotatefirst (default) width= or height= specifies the size of the
% 		 figure after rotation.  Asymetric figures will
% 		 appear to shrink.
%
% \psfigurepath#1	sets the path to search for the figure
%
% \psfig
% usage: \psfig{file=, figure=, height=, width=,
%			bbllx=, bblly=, bburx=, bbury=,
%			rheight=, rwidth=, clip=, angle=, silent=}
%
%	"file" is the filename.  If no path name is specified and the
%		file is not found in the current directory,
%		it will be looked for in directory \psfigurepath.
%	"figure" is a synonym for "file".
%	By default, the width and height of the figure are taken from
%		the BoundingBox of the figure.
%	If "width" is specified, the figure is scaled so that it has
%		the specified width.  Its height changes proportionately.
%	If "height" is specified, the figure is scaled so that it has
%		the specified height.  Its width changes proportionately.
%	If both "width" and "height" are specified, the figure is scaled
%		anamorphically.
%	"bbllx", "bblly", "bburx", and "bbury" control the PostScript
%		BoundingBox.  If these four values are specified
%               *before* the "file" option, the PSFIG will not try to
%               open the PostScript file.
%	"rheight" and "rwidth" are the reserved height and width
%		of the figure, i.e., how big TeX actually thinks
%		the figure is.  They default to "width" and "height".
%	The "clip" option ensures that no portion of the figure will
%		appear outside its BoundingBox.  "clip=" is a switch and
%		takes no value, but the `=' must be present.
%	The "angle" option specifies the angle of rotation (degrees, ccw).
%	The "silent" option makes \psfig work silently.
%

% check to see if macros already loaded in (maybe some other file says
% "\input psfig") ...
\ifx\undefined\psfig\else\endinput\fi

%
% from a suggestion by eijkhout@csrd.uiuc.edu to allow
% loading as a style file. Changed to avoid problems
% with amstex per suggestion by jbence@math.ucla.edu

\let\LaTeXAtSign=\@
\let\@=\relax
\edef\psfigRestoreAt{\catcode`\@=\number\catcode`@\relax}
%\edef\psfigRestoreAt{\catcode`@=\number\catcode`@\relax}
\catcode`\@=11\relax
\newwrite\@unused
\def\ps@typeout#1{{\let\protect\string\immediate\write\@unused{#1}}}
\ps@typeout{psfig/tex \PsfigVersion}

%% Here's how you define your figure path.  Should be set up with null
%% default and a user useable definition.

\def\figurepath{./}
\def\psfigurepath#1{\edef\figurepath{#1}}

%
% @psdo control structure -- similar to Latex @for.
% I redefined these with different names so that psfig can
% be used with TeX as well as LaTeX, and so that it will not 
% be vunerable to future changes in LaTeX's internal
% control structure,
%
\def\@nnil{\@nil}
\def\@empty{}
\def\@psdonoop#1\@@#2#3{}
\def\@psdo#1:=#2\do#3{\edef\@psdotmp{#2}\ifx\@psdotmp\@empty \else
    \expandafter\@psdoloop#2,\@nil,\@nil\@@#1{#3}\fi}
\def\@psdoloop#1,#2,#3\@@#4#5{\def#4{#1}\ifx #4\@nnil \else
       #5\def#4{#2}\ifx #4\@nnil \else#5\@ipsdoloop #3\@@#4{#5}\fi\fi}
\def\@ipsdoloop#1,#2\@@#3#4{\def#3{#1}\ifx #3\@nnil 
       \let\@nextwhile=\@psdonoop \else
      #4\relax\let\@nextwhile=\@ipsdoloop\fi\@nextwhile#2\@@#3{#4}}
\def\@tpsdo#1:=#2\do#3{\xdef\@psdotmp{#2}\ifx\@psdotmp\@empty \else
    \@tpsdoloop#2\@nil\@nil\@@#1{#3}\fi}
\def\@tpsdoloop#1#2\@@#3#4{\def#3{#1}\ifx #3\@nnil 
       \let\@nextwhile=\@psdonoop \else
      #4\relax\let\@nextwhile=\@tpsdoloop\fi\@nextwhile#2\@@#3{#4}}
% 
% \fbox is defined in latex.tex; so if \fbox is undefined, assume that
% we are not in LaTeX.
% Perhaps this could be done better???
\ifx\undefined\fbox
% \fbox code from modified slightly from LaTeX
\newdimen\fboxrule
\newdimen\fboxsep
\newdimen\ps@tempdima
\newbox\ps@tempboxa
\fboxsep = 3pt
\fboxrule = .4pt
\long\def\fbox#1{\leavevmode\setbox\ps@tempboxa\hbox{#1}\ps@tempdima\fboxrule
    \advance\ps@tempdima \fboxsep \advance\ps@tempdima \dp\ps@tempboxa
   \hbox{\lower \ps@tempdima\hbox
  {\vbox{\hrule height \fboxrule
          \hbox{\vrule width \fboxrule \hskip\fboxsep
          \vbox{\vskip\fboxsep \box\ps@tempboxa\vskip\fboxsep}\hskip 
                 \fboxsep\vrule width \fboxrule}
                 \hrule height \fboxrule}}}}
\fi
%
%%%%%%%%%%%%%%%%%%%%%%%%%%%%%%%%%%%%%%%%%%%%%%%%%%%%%%%%%%%%%%%%%%%
% file reading stuff from epsf.tex
%   EPSF.TEX macro file:
%   Written by Tomas Rokicki of Radical Eye Software, 29 Mar 1989.
%   Revised by Don Knuth, 3 Jan 1990.
%   Revised by Tomas Rokicki to accept bounding boxes with no
%      space after the colon, 18 Jul 1990.
%   Portions modified/removed for use in PSFIG package by
%      J. Daniel Smith, 9 October 1990.
%
\newread\ps@stream
\newif\ifnot@eof       % continue looking for the bounding box?
\newif\if@noisy        % report what you're making?
\newif\if@atend        % %%BoundingBox: has (at end) specification
\newif\if@psfile       % does this look like a PostScript file?
%
% PostScript files should start with `%!'
%
{\catcode`\%=12\global\gdef\epsf@start{%!}}
\def\epsf@PS{PS}
%
\def\epsf@getbb#1{%
%
%   The first thing we need to do is to open the
%   PostScript file, if possible.
%
\openin\ps@stream=#1
\ifeof\ps@stream\ps@typeout{Error, File #1 not found}\else
%
%   Okay, we got it. Now we'll scan lines until we find one that doesn't
%   start with %. We're looking for the bounding box comment.
%
   {\not@eoftrue \chardef\other=12
    \def\do##1{\catcode`##1=\other}\dospecials \catcode`\ =10
    \loop
       \if@psfile
	  \read\ps@stream to \epsf@fileline
       \else{
	  \obeyspaces
          \read\ps@stream to \epsf@tmp\global\let\epsf@fileline\epsf@tmp}
       \fi
       \ifeof\ps@stream\not@eoffalse\else
%
%   Check the first line for `%!'.  Issue a warning message if its not
%   there, since the file might not be a PostScript file.
%
       \if@psfile\else
       \expandafter\epsf@test\epsf@fileline:. \\%
       \fi
%
%   We check to see if the first character is a % sign;
%   if so, we look further and stop only if the line begins with
%   `%%BoundingBox:' and the `(atend)' specification was not found.
%   That is, the only way to stop is when the end of file is reached,
%   or a `%%BoundingBox: llx lly urx ury' line is found.
%
          \expandafter\epsf@aux\epsf@fileline:. \\%
       \fi
   \ifnot@eof\repeat
   }\closein\ps@stream\fi}%
%
% This tests if the file we are reading looks like a PostScript file.
%
\long\def\epsf@test#1#2#3:#4\\{\def\epsf@testit{#1#2}
			\ifx\epsf@testit\epsf@start\else
\ps@typeout{Warning! File does not start with `\epsf@start'.  It may not be a PostScript file.}
			\fi
			\@psfiletrue} % don't test after 1st line
%
%   We still need to define the tricky \epsf@aux macro. This requires
%   a couple of magic constants for comparison purposes.
%
{\catcode`\%=12\global\let\epsf@percent=%\global\def\epsf@bblit{%BoundingBox}}
%
%
%   So we're ready to check for `%BoundingBox:' and to grab the
%   values if they are found.  We continue searching if `(at end)'
%   was found after the `%BoundingBox:'.
%
\long\def\epsf@aux#1#2:#3\\{\ifx#1\epsf@percent
   \def\epsf@testit{#2}\ifx\epsf@testit\epsf@bblit
	\@atendfalse
        \epsf@atend #3 . \\%
	\if@atend	
	   \if@verbose{
		\ps@typeout{psfig: found `(atend)'; continuing search}
	   }\fi
        \else
        \epsf@grab #3 . . . \\%
        \not@eoffalse
        \global\no@bbfalse
        \fi
   \fi\fi}%
%
%   Here we grab the values and stuff them in the appropriate definitions.
%
\def\epsf@grab #1 #2 #3 #4 #5\\{%
   \global\def\epsf@llx{#1}\ifx\epsf@llx\empty
      \epsf@grab #2 #3 #4 #5 .\\\else
   \global\def\epsf@lly{#2}%
   \global\def\epsf@urx{#3}\global\def\epsf@ury{#4}\fi}%
%
% Determine if the stuff following the %%BoundingBox is `(atend)'
% J. Daniel Smith.  Copied from \epsf@grab above.
%
\def\epsf@atendlit{(atend)} 
\def\epsf@atend #1 #2 #3\\{%
   \def\epsf@tmp{#1}\ifx\epsf@tmp\empty
      \epsf@atend #2 #3 .\\\else
   \ifx\epsf@tmp\epsf@atendlit\@atendtrue\fi\fi}


% End of file reading stuff from epsf.tex
%%%%%%%%%%%%%%%%%%%%%%%%%%%%%%%%%%%%%%%%%%%%%%%%%%%%%%%%%%%%%%%%%%%

%%%%%%%%%%%%%%%%%%%%%%%%%%%%%%%%%%%%%%%%%%%%%%%%%%%%%%%%%%%%%%%%%%%
% trigonometry stuff from "trig.tex"
\chardef\psletter = 11 % won't conflict with \begin{letter} now...
\chardef\other = 12

\newif \ifdebug %%% turn me on to see TeX hard at work ...
\newif\ifc@mpute %%% don't need to compute some values
\c@mputetrue % but assume that we do

\let\then = \relax
\def\r@dian{pt }
\let\r@dians = \r@dian
\let\dimensionless@nit = \r@dian
\let\dimensionless@nits = \dimensionless@nit
\def\internal@nit{sp }
\let\internal@nits = \internal@nit
\newif\ifstillc@nverging
\def \Mess@ge #1{\ifdebug \then \message {#1} \fi}

{ %%% Things that need abnormal catcodes %%%
	\catcode `\@ = \psletter
	\gdef \nodimen {\expandafter \n@dimen \the \dimen}
	\gdef \term #1 #2 #3%
	       {\edef \t@ {\the #1}%%% freeze parameter 1 (count, by value)
		\edef \t@@ {\expandafter \n@dimen \the #2\r@dian}%
				   %%% freeze parameter 2 (dimen, by value)
		\t@rm {\t@} {\t@@} {#3}%
	       }
	\gdef \t@rm #1 #2 #3%
	       {{%
		\count 0 = 0
		\dimen 0 = 1 \dimensionless@nit
		\dimen 2 = #2\relax
		\Mess@ge {Calculating term #1 of \nodimen 2}%
		\loop
		\ifnum	\count 0 < #1
		\then	\advance \count 0 by 1
			\Mess@ge {Iteration \the \count 0 \space}%
			\Multiply \dimen 0 by {\dimen 2}%
			\Mess@ge {After multiplication, term = \nodimen 0}%
			\Divide \dimen 0 by {\count 0}%
			\Mess@ge {After division, term = \nodimen 0}%
		\repeat
		\Mess@ge {Final value for term #1 of 
				\nodimen 2 \space is \nodimen 0}%
		\xdef \Term {#3 = \nodimen 0 \r@dians}%
		\aftergroup \Term
	       }}
	\catcode `\p = \other
	\catcode `\t = \other
	\gdef \n@dimen #1pt{#1} %%% throw away the ``pt''
}

\def \Divide #1by #2{\divide #1 by #2} %%% just a synonym

\def \Multiply #1by #2%%% allows division of a dimen by a dimen
       {{%%% should really freeze parameter 2 (dimen, passed by value)
	\count 0 = #1\relax
	\count 2 = #2\relax
	\count 4 = 65536
	\Mess@ge {Before scaling, count 0 = \the \count 0 \space and
			count 2 = \the \count 2}%
	\ifnum	\count 0 > 32767 %%% do our best to avoid overflow
	\then	\divide \count 0 by 4
		\divide \count 4 by 4
	\else	\ifnum	\count 0 < -32767
		\then	\divide \count 0 by 4
			\divide \count 4 by 4
		\else
		\fi
	\fi
	\ifnum	\count 2 > 32767 %%% while retaining reasonable accuracy
	\then	\divide \count 2 by 4
		\divide \count 4 by 4
	\else	\ifnum	\count 2 < -32767
		\then	\divide \count 2 by 4
			\divide \count 4 by 4
		\else
		\fi
	\fi
	\multiply \count 0 by \count 2
	\divide \count 0 by \count 4
	\xdef \product {#1 = \the \count 0 \internal@nits}%
	\aftergroup \product
       }}

\def\r@duce{\ifdim\dimen0 > 90\r@dian \then   % sin(x+90) = sin(180-x)
		\multiply\dimen0 by -1
		\advance\dimen0 by 180\r@dian
		\r@duce
	    \else \ifdim\dimen0 < -90\r@dian \then  % sin(-x) = sin(360+x)
		\advance\dimen0 by 360\r@dian
		\r@duce
		\fi
	    \fi}

\def\Sine#1%
       {{%
	\dimen 0 = #1 \r@dian
	\r@duce
	\ifdim\dimen0 = -90\r@dian \then
	   \dimen4 = -1\r@dian
	   \c@mputefalse
	\fi
	\ifdim\dimen0 = 90\r@dian \then
	   \dimen4 = 1\r@dian
	   \c@mputefalse
	\fi
	\ifdim\dimen0 = 0\r@dian \then
	   \dimen4 = 0\r@dian
	   \c@mputefalse
	\fi
%
	\ifc@mpute \then
        	% convert degrees to radians
		\divide\dimen0 by 180
		\dimen0=3.141592654\dimen0
%
		\dimen 2 = 3.1415926535897963\r@dian %%% a well-known constant
		\divide\dimen 2 by 2 %%% we only deal with -pi/2 : pi/2
		\Mess@ge {Sin: calculating Sin of \nodimen 0}%
		\count 0 = 1 %%% see power-series expansion for sine
		\dimen 2 = 1 \r@dian %%% ditto
		\dimen 4 = 0 \r@dian %%% ditto
		\loop
			\ifnum	\dimen 2 = 0 %%% then we've done
			\then	\stillc@nvergingfalse 
			\else	\stillc@nvergingtrue
			\fi
			\ifstillc@nverging %%% then calculate next term
			\then	\term {\count 0} {\dimen 0} {\dimen 2}%
				\advance \count 0 by 2
				\count 2 = \count 0
				\divide \count 2 by 2
				\ifodd	\count 2 %%% signs alternate
				\then	\advance \dimen 4 by \dimen 2
				\else	\advance \dimen 4 by -\dimen 2
				\fi
		\repeat
	\fi		
			\xdef \sine {\nodimen 4}%
       }}

% Now the Cosine can be calculated easily by calling \Sine
\def\Cosine#1{\ifx\sine\UnDefined\edef\Savesine{\relax}\else
		             \edef\Savesine{\sine}\fi
	{\dimen0=#1\r@dian\advance\dimen0 by 90\r@dian
	 \Sine{\nodimen 0}
	 \xdef\cosine{\sine}
	 \xdef\sine{\Savesine}}}	      
% end of trig stuff
%%%%%%%%%%%%%%%%%%%%%%%%%%%%%%%%%%%%%%%%%%%%%%%%%%%%%%%%%%%%%%%%%%%%

\def\psdraft{
	\def\@psdraft{0}
	%\ps@typeout{draft level now is \@psdraft \space . }
}
\def\psfull{
	\def\@psdraft{100}
	%\ps@typeout{draft level now is \@psdraft \space . }
}

\psfull

\newif\if@scalefirst
\def\psscalefirst{\@scalefirsttrue}
\def\psrotatefirst{\@scalefirstfalse}
\psrotatefirst

\newif\if@draftbox
\def\psnodraftbox{
	\@draftboxfalse
}
\def\psdraftbox{
	\@draftboxtrue
}
\@draftboxtrue

\newif\if@prologfile
\newif\if@postlogfile
\def\pssilent{
	\@noisyfalse
}
\def\psnoisy{
	\@noisytrue
}
\psnoisy
%%% These are for the option list.
%%% A specification of the form a = b maps to calling \@p@@sa{b}
\newif\if@bbllx
\newif\if@bblly
\newif\if@bburx
\newif\if@bbury
\newif\if@height
\newif\if@width
\newif\if@rheight
\newif\if@rwidth
\newif\if@angle
\newif\if@clip
\newif\if@verbose
\def\@p@@sclip#1{\@cliptrue}


\newif\if@decmpr

%%% GDH 7/26/87 -- changed so that it first looks in the local directory,
%%% then in a specified global directory for the ps file.
%%% RPR 6/25/91 -- changed so that it defaults to user-supplied name if
%%% boundingbox info is specified, assuming graphic will be created by
%%% print time.
%%% TJD 10/19/91 -- added bbfile vs. file distinction, and @decmpr flag

\def\@p@@sfigure#1{\def\@p@sfile{null}\def\@p@sbbfile{null}
	        \openin1=#1.bb
		\ifeof1\closein1
	        	\openin1=\figurepath#1.bb
			\ifeof1\closein1
			        \openin1=#1
				\ifeof1\closein1%
				       \openin1=\figurepath#1
					\ifeof1
					   \ps@typeout{Error, File #1 not found}
						\if@bbllx\if@bblly
				   		\if@bburx\if@bbury
			      				\def\@p@sfile{#1}%
			      				\def\@p@sbbfile{#1}%
							\@decmprfalse
				  	   	\fi\fi\fi\fi
					\else\closein1
				    		\def\@p@sfile{\figurepath#1}%
				    		\def\@p@sbbfile{\figurepath#1}%
						\@decmprfalse
	                       		\fi%
			 	\else\closein1%
					\def\@p@sfile{#1}
					\def\@p@sbbfile{#1}
					\@decmprfalse
			 	\fi
			\else
				\def\@p@sfile{\figurepath#1}
				\def\@p@sbbfile{\figurepath#1.bb}
				\@decmprtrue
			\fi
		\else
			\def\@p@sfile{#1}
			\def\@p@sbbfile{#1.bb}
			\@decmprtrue
		\fi}

\def\@p@@sfile#1{\@p@@sfigure{#1}}

\def\@p@@sbbllx#1{
		%\ps@typeout{bbllx is #1}
		\@bbllxtrue
		\dimen100=#1
		\edef\@p@sbbllx{\number\dimen100}
}
\def\@p@@sbblly#1{
		%\ps@typeout{bblly is #1}
		\@bbllytrue
		\dimen100=#1
		\edef\@p@sbblly{\number\dimen100}
}
\def\@p@@sbburx#1{
		%\ps@typeout{bburx is #1}
		\@bburxtrue
		\dimen100=#1
		\edef\@p@sbburx{\number\dimen100}
}
\def\@p@@sbbury#1{
		%\ps@typeout{bbury is #1}
		\@bburytrue
		\dimen100=#1
		\edef\@p@sbbury{\number\dimen100}
}
\def\@p@@sheight#1{
		\@heighttrue
		\dimen100=#1
   		\edef\@p@sheight{\number\dimen100}
		%\ps@typeout{Height is \@p@sheight}
}
\def\@p@@swidth#1{
		%\ps@typeout{Width is #1}
		\@widthtrue
		\dimen100=#1
		\edef\@p@swidth{\number\dimen100}
}
\def\@p@@srheight#1{
		%\ps@typeout{Reserved height is #1}
		\@rheighttrue
		\dimen100=#1
		\edef\@p@srheight{\number\dimen100}
}
\def\@p@@srwidth#1{
		%\ps@typeout{Reserved width is #1}
		\@rwidthtrue
		\dimen100=#1
		\edef\@p@srwidth{\number\dimen100}
}
\def\@p@@sangle#1{
		%\ps@typeout{Rotation is #1}
		\@angletrue
%		\dimen100=#1
		\edef\@p@sangle{#1} %\number\dimen100}
}
\def\@p@@ssilent#1{ 
		\@verbosefalse
}
\def\@p@@sprolog#1{\@prologfiletrue\def\@prologfileval{#1}}
\def\@p@@spostlog#1{\@postlogfiletrue\def\@postlogfileval{#1}}
\def\@cs@name#1{\csname #1\endcsname}
\def\@setparms#1=#2,{\@cs@name{@p@@s#1}{#2}}
%
% initialize the defaults (size the size of the figure)
%
\def\ps@init@parms{
		\@bbllxfalse \@bbllyfalse
		\@bburxfalse \@bburyfalse
		\@heightfalse \@widthfalse
		\@rheightfalse \@rwidthfalse
		\def\@p@sbbllx{}\def\@p@sbblly{}
		\def\@p@sbburx{}\def\@p@sbbury{}
		\def\@p@sheight{}\def\@p@swidth{}
		\def\@p@srheight{}\def\@p@srwidth{}
		\def\@p@sangle{0}
		\def\@p@sfile{} \def\@p@sbbfile{}
		\def\@p@scost{10}
		\def\@sc{}
		\@prologfilefalse
		\@postlogfilefalse
		\@clipfalse
		\if@noisy
			\@verbosetrue
		\else
			\@verbosefalse
		\fi
}
%
% Go through the options setting things up.
%
\def\parse@ps@parms#1{
	 	\@psdo\@psfiga:=#1\do
		   {\expandafter\@setparms\@psfiga,}}
%
% Compute bb height and width
%
\newif\ifno@bb
\def\bb@missing{
	\if@verbose{
		\ps@typeout{psfig: searching \@p@sbbfile \space  for bounding box}
	}\fi
	\no@bbtrue
	\epsf@getbb{\@p@sbbfile}
        \ifno@bb \else \bb@cull\epsf@llx\epsf@lly\epsf@urx\epsf@ury\fi
}	
\def\bb@cull#1#2#3#4{
	\dimen100=#1 bp\edef\@p@sbbllx{\number\dimen100}
	\dimen100=#2 bp\edef\@p@sbblly{\number\dimen100}
	\dimen100=#3 bp\edef\@p@sbburx{\number\dimen100}
	\dimen100=#4 bp\edef\@p@sbbury{\number\dimen100}
	\no@bbfalse
}
% rotate point (#1,#2) about (0,0).
% The sine and cosine of the angle are already stored in \sine and
% \cosine.  The result is placed in (\p@intvaluex, \p@intvaluey).
\newdimen\p@intvaluex
\newdimen\p@intvaluey
\def\rotate@#1#2{{\dimen0=#1 sp\dimen1=#2 sp
%            	calculate x' = x \cos\theta - y \sin\theta
		  \global\p@intvaluex=\cosine\dimen0
		  \dimen3=\sine\dimen1
		  \global\advance\p@intvaluex by -\dimen3
% 		calculate y' = x \sin\theta + y \cos\theta
		  \global\p@intvaluey=\sine\dimen0
		  \dimen3=\cosine\dimen1
		  \global\advance\p@intvaluey by \dimen3
		  }}
\def\compute@bb{
		\no@bbfalse
		\if@bbllx \else \no@bbtrue \fi
		\if@bblly \else \no@bbtrue \fi
		\if@bburx \else \no@bbtrue \fi
		\if@bbury \else \no@bbtrue \fi
		\ifno@bb \bb@missing \fi
		\ifno@bb \ps@typeout{FATAL ERROR: no bb supplied or found}
			\no-bb-error
		\fi
		%
%\ps@typeout{BB: \@p@sbbllx, \@p@sbblly, \@p@sbburx, \@p@sbbury} 
%
% store height/width of original (unrotated) bounding box
		\count203=\@p@sbburx
		\count204=\@p@sbbury
		\advance\count203 by -\@p@sbbllx
		\advance\count204 by -\@p@sbblly
		\edef\ps@bbw{\number\count203}
		\edef\ps@bbh{\number\count204}
		%\ps@typeout{ psbbh = \ps@bbh, psbbw = \ps@bbw }
		\if@angle 
			\Sine{\@p@sangle}\Cosine{\@p@sangle}
	        	{\dimen100=\maxdimen\xdef\r@p@sbbllx{\number\dimen100}
					    \xdef\r@p@sbblly{\number\dimen100}
			                    \xdef\r@p@sbburx{-\number\dimen100}
					    \xdef\r@p@sbbury{-\number\dimen100}}
%
% Need to rotate all four points and take the X-Y extremes of the new
% points as the new bounding box.
                        \def\minmaxtest{
			   \ifnum\number\p@intvaluex<\r@p@sbbllx
			      \xdef\r@p@sbbllx{\number\p@intvaluex}\fi
			   \ifnum\number\p@intvaluex>\r@p@sbburx
			      \xdef\r@p@sbburx{\number\p@intvaluex}\fi
			   \ifnum\number\p@intvaluey<\r@p@sbblly
			      \xdef\r@p@sbblly{\number\p@intvaluey}\fi
			   \ifnum\number\p@intvaluey>\r@p@sbbury
			      \xdef\r@p@sbbury{\number\p@intvaluey}\fi
			   }
%			lower left
			\rotate@{\@p@sbbllx}{\@p@sbblly}
			\minmaxtest
%			upper left
			\rotate@{\@p@sbbllx}{\@p@sbbury}
			\minmaxtest
%			lower right
			\rotate@{\@p@sbburx}{\@p@sbblly}
			\minmaxtest
%			upper right
			\rotate@{\@p@sbburx}{\@p@sbbury}
			\minmaxtest
			\edef\@p@sbbllx{\r@p@sbbllx}\edef\@p@sbblly{\r@p@sbblly}
			\edef\@p@sbburx{\r@p@sbburx}\edef\@p@sbbury{\r@p@sbbury}
%\ps@typeout{rotated BB: \r@p@sbbllx, \r@p@sbblly, \r@p@sbburx, \r@p@sbbury}
		\fi
		\count203=\@p@sbburx
		\count204=\@p@sbbury
		\advance\count203 by -\@p@sbbllx
		\advance\count204 by -\@p@sbblly
		\edef\@bbw{\number\count203}
		\edef\@bbh{\number\count204}
		%\ps@typeout{ bbh = \@bbh, bbw = \@bbw }
}
%
% \in@hundreds performs #1 * (#2 / #3) correct to the hundreds,
%	then leaves the result in @result
%
\def\in@hundreds#1#2#3{\count240=#2 \count241=#3
		     \count100=\count240	% 100 is first digit #2/#3
		     \divide\count100 by \count241
		     \count101=\count100
		     \multiply\count101 by \count241
		     \advance\count240 by -\count101
		     \multiply\count240 by 10
		     \count101=\count240	%101 is second digit of #2/#3
		     \divide\count101 by \count241
		     \count102=\count101
		     \multiply\count102 by \count241
		     \advance\count240 by -\count102
		     \multiply\count240 by 10
		     \count102=\count240	% 102 is the third digit
		     \divide\count102 by \count241
		     \count200=#1\count205=0
		     \count201=\count200
			\multiply\count201 by \count100
		 	\advance\count205 by \count201
		     \count201=\count200
			\divide\count201 by 10
			\multiply\count201 by \count101
			\advance\count205 by \count201
			%
		     \count201=\count200
			\divide\count201 by 100
			\multiply\count201 by \count102
			\advance\count205 by \count201
			%
		     \edef\@result{\number\count205}
}
\def\compute@wfromh{
		% computing : width = height * (bbw / bbh)
		\in@hundreds{\@p@sheight}{\@bbw}{\@bbh}
		%\ps@typeout{ \@p@sheight * \@bbw / \@bbh, = \@result }
		\edef\@p@swidth{\@result}
		%\ps@typeout{w from h: width is \@p@swidth}
}
\def\compute@hfromw{
		% computing : height = width * (bbh / bbw)
	        \in@hundreds{\@p@swidth}{\@bbh}{\@bbw}
		%\ps@typeout{ \@p@swidth * \@bbh / \@bbw = \@result }
		\edef\@p@sheight{\@result}
		%\ps@typeout{h from w : height is \@p@sheight}
}
\def\compute@handw{
		\if@height 
			\if@width
			\else
				\compute@wfromh
			\fi
		\else 
			\if@width
				\compute@hfromw
			\else
				\edef\@p@sheight{\@bbh}
				\edef\@p@swidth{\@bbw}
			\fi
		\fi
}
\def\compute@resv{
		\if@rheight \else \edef\@p@srheight{\@p@sheight} \fi
		\if@rwidth \else \edef\@p@srwidth{\@p@swidth} \fi
		%\ps@typeout{rheight = \@p@srheight, rwidth = \@p@srwidth}
}
%		
% Compute any missing values
\def\compute@sizes{
	\compute@bb
	\if@scalefirst\if@angle
% at this point the bounding box has been adjsuted correctly for
% rotation.  PSFIG does all of its scaling using \@bbh and \@bbw.  If
% a width= or height= was specified along with \psscalefirst, then the
% width=/height= value needs to be adjusted to match the new (rotated)
% bounding box size (specifed in \@bbw and \@bbh).
%    \ps@bbw       width=
%    -------  =  ---------- 
%    \@bbw       new width=
% so `new width=' = (width= * \@bbw) / \ps@bbw; where \ps@bbw is the
% width of the original (unrotated) bounding box.
	\if@width
	   \in@hundreds{\@p@swidth}{\@bbw}{\ps@bbw}
	   \edef\@p@swidth{\@result}
	\fi
	\if@height
	   \in@hundreds{\@p@sheight}{\@bbh}{\ps@bbh}
	   \edef\@p@sheight{\@result}
	\fi
	\fi\fi
	\compute@handw
	\compute@resv}

%
% \psfig
% usage : \psfig{file=, height=, width=, bbllx=, bblly=, bburx=, bbury=,
%			rheight=, rwidth=, clip=}
%
% "clip=" is a switch and takes no value, but the `=' must be present.
\def\psfig#1{\vbox {
	% do a zero width hard space so that a single
	% \psfig in a centering enviornment will behave nicely
	%{\setbox0=\hbox{\ }\ \hskip-\wd0}
	%
	\ps@init@parms
	\parse@ps@parms{#1}
	\compute@sizes
	%
	\ifnum\@p@scost<\@psdraft{
		%
		\special{ps::[begin] 	\@p@swidth \space \@p@sheight \space
				\@p@sbbllx \space \@p@sbblly \space
				\@p@sbburx \space \@p@sbbury \space
				startTexFig \space }
		\if@angle
			\special {ps:: \@p@sangle \space rotate \space} 
		\fi
		\if@clip{
			\if@verbose{
				\ps@typeout{(clip)}
			}\fi
			\special{ps:: doclip \space }
		}\fi
		\if@prologfile
		    \special{ps: plotfile \@prologfileval \space } \fi
		\if@decmpr{
			\if@verbose{
				\ps@typeout{psfig: including \@p@sfile.Z \space }
			}\fi
			\special{ps: plotfile "`zcat \@p@sfile.Z" \space }
		}\else{
			\if@verbose{
				\ps@typeout{psfig: including \@p@sfile \space }
			}\fi
			\special{ps: plotfile \@p@sfile \space }
		}\fi
		\if@postlogfile
		    \special{ps: plotfile \@postlogfileval \space } \fi
		\special{ps::[end] endTexFig \space }
		% Create the vbox to reserve the space for the figure.
		\vbox to \@p@srheight sp{
		% 1/92 TJD Changed from "true sp" to "sp" for magnification.
			\hbox to \@p@srwidth sp{
				\hss
			}
		\vss
		}
	}\else{
		% draft figure, just reserve the space and print the
		% path name.
		\if@draftbox{		
			% Verbose draft: print file name in box
			\hbox{\frame{\vbox to \@p@srheight sp{
			\vss
			\hbox to \@p@srwidth sp{ \hss \@p@sfile \hss }
			\vss
			}}}
		}\else{
			% Non-verbose draft
			\vbox to \@p@srheight sp{
			\vss
			\hbox to \@p@srwidth sp{\hss}
			\vss
			}
		}\fi	



	}\fi
}}
\psfigRestoreAt
\let\@=\LaTeXAtSign





\begin{document}

\title{State as an Organizing Principle for CSCW Architectures}

\author{
Philip Johnson\\
Collaborative Software Development Laboratory\\
Department of Information and Computer Sciences\\
University of Hawaii\\
Honolulu, Hawaii 96822\\
(808) 956-3489\\
johnson@hawaii.edu}

\maketitle

\abstract 

A useful way to gain insight into collaborative architectures is by
analyzing how they collect, represent, store, analyze, and distribute state
information.  This paper presents state as an organizing principle for
collaborative architectures. It uses a framework with eight dimensions to
analyze four systems: WWW, GroupKit, Lotus Notes, and Egret. The analysis
illuminates similarities and differences between these architectures, and
yields two conjectures: that no single collaborative architecture can fully
support both collaboration-in-the-small and collaboration-in-the-large, and
that flexible and efficient support for state management requires
architectural support for agents as first-class users.

\section{INTRODUCTION}

Over the past five years, the Collaborative Software Development Laboratory
at the University of Hawaii has designed, implemented, and evaluated a
variety of CSCW systems.  This work includes CLARE, a collaborative
learning environment \cite{csdl-93-21}; CSRS, a collaborative software
review environment \cite{csdl-92-07}; AEN, a collaborative hypertext
authoring environment \cite{csdl-94-20}, HI-TIME, a collaborative system
for telecommunications policy planning; Shemacs, a collaborative editor
\cite{csdl-95-16}; and Flashmail, a collaborative messaging facility.  All
of these environments were built using Egret \cite{csdl-92-01}, an architecture
and development environment for CSCW applications. 


As we built these systems, we accumulated general requirements for
collaboration from these various domains that we pushed downward into the
framework.  Egret's current design and implementation results from three
complete redesigns and over a hundred separate releases during its five
years of active development to date.  The current architecture of Egret
embodies much of what we find to be essential for computer-mediated
collaboration across our chosen application domains.

Although the specifics differ, we find that there are general classes of
questions that our collaborative applications must often answer.  For
example: Who are the other users?  What artifacts are involved in this
collaborative activity?  What has changed since yesterday?  What is
important for me to focus on right now?  Who should be informed about this
change?  Is the information in the system up to date?  Does user X know
about event Y yet? What is user Z doing right now?  We also find that the
usability of our applications often depends upon the efficiency with which
it answers one or more of these types of questions, although which types of
questions are time-critical and their specific form vary from application
to application.


In general, answering these questions requires an application to collect,
represent, store, modify, analyze, and distribute state information.  In
fact, Egret's development history can be profitably viewed as an
exploration of the architectural services required for a CSCW application
to manage state information appropriately.  In Egret, state is an
organizing principle.

This paper presents state as an organizing principle for CSCW
architectures, and illustrates this perspective using four exemplary
systems: the World Wide Web, GroupKit, Lotus Notes and Egret.  The next
section presents a framework to aid in understanding the role of state in
CSCW architectures.  Subsequent sections apply this framework to the chosen
architectures. The paper concludes with two conjectures supported
by this viewpoint.  First, that the requirements for architectures
supporting ``collaboration-in-the-small'', such as Egret, are fundamentally
different from those supporting ``collaboration-in-the-large'', such as
WWW. Thus, a single system can provide effective support for one but not
both of these extremes.  Second, that collaborative architectures must
support autonomous agents as ``first class'' users if both flexible and
efficient support for state management is desired.


\section{STATE AS AN ORGANIZING PRINCIPLE}

If a system is to effectively support answers to a question like ``Who
should be informed about this change?'', it must potentially represent
information about users, artifacts, types of changes, the context of
collaboration, and communication mechanisms.  To aid in understanding
the various aspects of state management, we identified a useful set 
of characteristics, or dimensions of state management.
The following table summarizes them:  


\begin{center}
\begin{tabular} {|p{1in}|p{2in}|} \hline   
{\em Dimension} & {\em Exemplary Attributes} \\ \hline
Producer & Artifacts,  Individuals,  Process, Environment     \\ \hline
Consumer & Artifacts,  Individuals,  Process, Environment     \\ \hline
Trigger & User actions, Artifact content, External events   \\ \hline
Representation & Forms, Text, Arbitrary data structures      \\ \hline
Persistence & Non-persistent, In-session, Between-session    \\ \hline
Location & Local, Central, Distributed, Replicated           \\ \hline
Distribution & Queries, Signals                              \\ \hline
Time         & Current, Past, Future                         \\ \hline
\end{tabular}
\end{center}

\subsection{Producer}

The producer dimension focuses on what classes of objects in the
collaborative system have state information collected about them.

One producer of state information is the artifacts, or work
products of the collaborative system. For example, state information can
be represented about the structure of artifacts or their relationship to
each other, or the manner in which the artifact evolved to its current
state.

The individuals engaging in collaboration also produce state information,
which can model each individual's capabilities, their prior involvement in
the collaboration, or their future commitments.


The collaborative process itself can also produce state information.  For
example, some systems represent a plan for collaboration, and collect
state information concerning where in the plan the collaboration has
reached, any exceptions or problems with the plan, and so forth.

Finally, the context surrounding the use of the system can produce state
information. For example, some systems collect data on the organizational
environment in which the group works that affects the productivity and
quality of collaboration. Others collect network
statistics on message transmission latency and its effects on
collaboration.

\subsection{Consumer}

The consumer dimension parallels the producer dimension except
that it characterizes what objects can be affected in the 
system by state information collected from the producers. 

For example, a process object can be affected by the state of artifacts in
the system. Systems supporting software engineering process modeling
typically include rules such as: the testing phase cannot be entered until
the state of the code artifact includes ``clean compile.''


\subsection{Trigger}

The trigger dimension focuses on what events or constructs in the system 
trigger the collection of state information.  

The most obvious trigger is a user action. Examples of user actions
that can trigger the collection of state information include mouse
clicks, artifact retrieval, artifact creation, interaction with another
user, and so forth. 

Another potential trigger is the structure and/or content of artifacts.
For example, a system can react to the creation of an artifact of type
``e-mail'' by creating state information concerning when, where, and by who
this e-mail was created, and what it's intent was. (Such state information
is useful to a system implementing a ``speech act'' paradigm, in
which the state of the system must include a representation of communication
between users and the state of the conversation, though not the actual
e-mail itself.)

External events also serve as triggers to state information collection.
For example, some systems ``wake up'' each day at midnight to perform
certain supportive analyses on the state of collaboration. 

\subsection{Representation}

The representation dimension provides perspective on how state
information, once collected, is represented.  Some 
systems, for example, have no mechanisms to represent state 
explicitly: state is represented and manipulated implicitly (such
as when the artifacts under collaboration themselves are the 
only representation of the state of the system).  

Other systems support construction of state information
separate from the artifacts themselves.  Depending upon the 
paradigm for collaboration and the underlying infrastructure,
these representations are restricted to text, or structured
forms, or other data structures. 

\subsection{Persistence}

The persistence dimension specifies whether and for what period of time the
system preserves state information.  A system that is totally
non-persistent with respect to state can nevertheless collect a great deal
of state information, but the information is either used immediately or
discarded.

Other systems differ with respect to persistence of state information
across sessions.  Some systems collect state only within a session: when a
user logs out of the system, any state information collected about that
user is lost. Others preserve state information across sessions.

\subsection{Location}

The location dimension indicates where in the system state information is
held.  

Some systems store all state information on a central server, and
require clients to request each value individually.  Other systems
store state information in a distributed manner, where each client
maintains its own state information locally. Still others replicate state
information across the clients and servers.

In our experience, the ability to control where state information
is located is a major determinant of whether or not the system
can be simultaneously expressive and responsive. 

\subsection{Distribution}

The distribution dimensions characterizes how a system communicates state
information from a producer to its consumer or consumers.  There are two
basic mechanisms: queries, in which the consumer requests state information
explicitly, and signals, in which the system distributes state information
to clients without their having asked for it.

In general, the query distribution mechanism is better for 
rarely used but frequently changing state information, since
the cost of sending signals to maintain the correct value 
at a client is not justified by the use of this information.
For example, the idle time value associated with a user
tends to vary frequently, but this state information might
be rarely useful to other clients. 

Conversely, state information which changes infrequently and
unpredictably, but is used often by clients is a good candidate for
signals. For example, state information concerning the set of users
currently connected to a client might be used every few seconds by
a system, though it changes only every few minutes or hours.
Such information is best communicated through signals.

CSCW architectures can support just one or both of these
distribution mechanisms.  A further distinction can be made
between those systems that hardwire the distribution mechanism
for a particular type of state information, and those that
allow applications to choose the distribution mechanism best
suited to their purposes for the state information in question.

\subsection{Time}

The final dimension is time.  This dimension characterizes the period over
which state information is preserved and manipulated.

Some architectures provide an explicit representation of time, and can thus
represent change in state over time. Other systems support only a
representation of the current state.  Finally, an architecture can provide
explicit support for predictive state, where the possible future state or
states of the system are represented and used to influence its current
behavior.


\section{STATE IN CSCW ARCHITECTURES} 

The dimensions of state identified above are not a checklist, such that the
more facilities for state representation and manipulation an architecture
supports, the better and more generic it is.  In fact, we have found the
opposite: the inclusion of certain state mechanisms appears to make an
architecture more specialized and less generic. Further, the ultimate
utility and value of a system depends upon much more than its state
representation mechanisms.  A CSCW architecture may consciously and
correctly sacrifice certain state facilities to increase its overall
utility, as we shall see below.

The benefit of using state as an organizing principle for CSCW
architectures is that it helps illuminate the tradeoffs between
expressiveness, responsiveness, scalability, and usability that must be
made.  It also provides insight into opportunities for improvement, by
illuminating mechanisms in one architecture that may prove valuable to
another. Finally, it tends to shed light onto the underlying, often
implicit assumptions about the types of applications, process, or users
that guided the designers of the architecture.

This section revisits the dimensions of state, this time using them as a
mechanism for focused inquiry into the nature of several CSCW
architectures: WWW, GroupKit, Lotus Notes, and Egret.  These architectures
were chosen because they form an interesting cross-section of design
choices; analysis of the state properties of other CSCW architectures, such
as Hyper-G \cite{Andrews95}, would also be interesting and useful.


The WWW needs no introduction for the CSCW community, of course.  In this
discussion, however, the architectural model of WWW is viewed in a somewhat
narrow fashion.  For example, although many agents (a.k.a. ``spiders'') run
over the WWW, we do not view this architecture as agent-based. The WWW is
unlike Egret, for example, which requires autonomous agents as part of its
core implementation.  Agents in WWW are incidental, not essential to its
architecture. Most interestingly, the http protocol underlying the WWW is
defined as ``stateless'' \cite{Berners-Lee94}.  This design decision
provides superb scalability, but greatly constrains the possibilities for
state collection and manipulation.

GroupKit is ``a groupware toolkit that lets developers build applications
for synchronous and distributed computer-based conferencing''
\cite{Roseman95}. This CSCW architecture includes a run-time infrastructure
for process management, groupware programming abstractions, user
interface widgets, and session management mechanisms.

Lotus Notes is a toolkit for building distributed, primarily
asynchronous conferences \cite{LotusNotes}.  In this architecture,
developers build structured forms (or ``notes'') which members of the
collaborating group create, view, and modify.  The database of notes can be fully
replicated to remote sites, or accessed via a centralized server in a
client-server architecture. Version 4.0 of Lotus Notes is expected to
support agents for both servers and clients.

Egret is a ``multi-client, multi-server, multi-agent'' architecture for
building collaborative systems. The Egret architecture
includes six major subsystems, including: a server subsystem for
inter-process communication and low-level storage; a type subsystem for
building domain-specific typed hypertext nodes and links; a generic
interface subsystem implementing facilities for manipulating ETML (the
Egret analogue to HTML); an applets subsystem providing small, reusable,
Egret devices; and a metrics subsystem providing instrumentation facilities.


The following sections take a closer look at these systems, using 
state as an organizing principle.

\subsection{Production and Consumption of State Information} 

Some very useful CSCW architectures provide little explicit support for
either production or consumption of explicit state information.  The WWW is
an excellent example.  At the architectural level, the WWW is
a mechanism allowing users to make documents that are globally
accessible via servers, and that may contain embedded links to other
documents. At this level, there is no explicit support for collection or
manipulation of state information.  The WWW architecture does not build in
mechanisms, for example, to represent which people are currently using a
document, which documents have been created by a particular user, why a
document is being manipulated by a user, and so forth.  The cgi script
mechanism does enable primitive support for state production and
consumption, since it creates a kind of ``trap door'' in the architecture
for incorporation of external storage, retrieval, and information 
processing mechanisms. 

The paucity of WWW mechanisms for state production and consumption should
not be construed as a bug, but rather as a feature. By eliminating all
requirements for state representation and manipulation, explosive growth of
WWW servers, documents, and use can occur without a corresponding explosion
in state information processing.  The WWW represents a canonical form of
``collaboration-in-the-large'', and has enabled an unprecedented number of
people of divergent backgrounds and interests to collaborate 
together. 

Lotus Notes and GroupKit both provide greater support for production and
consumption of state information.  The Lotus Notes architecture
provides a document repository that supports wide-area distribution in
the presence of potentially unreliable and/or intermittent network
connections.  It also provides mechanisms to collect basic forms of
state information about the artifacts under development, the people
involved in collaboration, and the process of collaboration itself.
Architecturally, however, Lotus Notes does not distinguish between state
information and the artifacts of collaboration.

In GroupKit, there is such a distinction: there is the shared artifact
under construction (such as an illustration in a shared white board) as well
as a separate, shared data structure called an ``environment'' that allows
an explicit and independent representation of state information.

Egret provides extensive architectural support for production and
consumption of state information concerning artifacts, individuals,
process, and the external environment.  Egret makes an architectural
distinction between the artifacts involved in collaboration and state
information by partitioning storage into separate areas. Some of these
areas for designed for ``user-level'' storage and manipulation of
artifacts, while other areas are designed for management of state
information.  Each of these latter areas differ along the dimensions of
representation, persistence, location, distribution, and time.

\subsection{Triggers and Agents}

Analyzing architectural support for the trigger dimension in a
CSCW architecture tends to illuminate its support for autonomous
software agents.  If an architecture provides any support for state
production at all, then it supports at least triggering based upon 
user action. More advanced forms of triggering require agents. 

Within architectures supporting user action triggers, there is wide
variation in the spectrum of user actions that trigger state collection.
For example, the cgi script mechanism in WWW allows essentially arbitrary
actions to be carried out based upon the contents of a web form.  This
allows a single, simple user action trigger---pressing the infamous
``submit'' button---that supports the collection of any state information
present on the form.

Lotus Notes provides more support for user action triggers than the WWW.
In Notes (when using the centralized server mechanism), virtually any
change to the database can trigger an event, which can communicate
information back to that client or any other connected client. However, the
architecture of Lotus Notes does not provide explicit support for user
action triggers that do not modify the database.  For example, one client
cannot be notified in real-time whenever another client pulls down a 
specific menu, since this user action does not affect the database in 
any fashion.

GroupKit and Egret provide more comprehensive support for user action
triggers. Both GroupKit and Egret can attach state collection functions to
most user gestures, such as key presses, menu pulls, mouse clicks, and so
forth, regardless of whether the gesture modifies the central server
process. In both GroupKit and Egret, user action triggers are limited to
those user gestures occurring while the system is the active process.

There is more to architectural support for fine-grained user action
triggers (such as menu pulls or mouse drags) than simply allowing
invocation of an application-specific function when these events occur.  As
we learned with Egret, fine-grained user action triggers can produce a
great deal of state information quite quickly, with a potential for
substantially negative impact upon the system's responsiveness.

For example, in an application built using the original design of Egret, we
associated various attributes, such as the user's name and a time-stamp,
with several fine-grained user actions.  Each time we used it, the
application's performance and responsiveness would degrade terribly after
only a few days.  The problem was that the stored state information would
quickly overwhelm the actual artifact information in the database.
This situation eventually led us to a complete redesign of Egret in which
many forms of high volume, fine-grained state information could be
collected with minimal impact upon the responsiveness of the
application. (As a humorous aside, while evaluating that early application,
we discovered that over 70\% of the database was occupied by time-stamps
alone! This occurred because we represented time using the Emacs {\tt
current-time-string} function, which produced very space-inefficient
time-stamps such as ``Wed Mar 13 08:18:50 1996''. We solved this particular
problem with a custom date package that reduced time-stamps to a fraction
of their former size.)

To support triggers based upon anything other than user action, agents
are necessary.  For example, the Egret application AEN has an agent that
composes and delivers a ``daily newspaper'' by e-mail to the group members
each morning summarizing the changes to the document since the day
before. The e-mail also contains a customized to-do list for each member
containing, for example, which parts of the artifact were changed by some
other group member since the last time the recipient looked at them.  This
state manipulation is triggered by several factors: the time of day (an
external trigger) begins the process, and the outcome is influenced by the
structure of the artifact as well as the state information associated with
each component.  

\subsection{Representing State}

CSCW architectures provide a great variety of sophisticated representations
for the {\em products} of collaboration. For example, GroupKit represents a
graphical image constructed by a group as the sequence of Tcl commands used
to construct it. (When a new user joins a group in-progress, the system
``replays'' the commands used to construct the artifact for the new user,
synchronizing them with the group.) The WWW architecture supports an
extensible set of representations for artifacts, including HTML,
postscript, audio, and so forth. Lotus Notes also supports many different
representations for the products of collaboration.

However, CSCW architectures provide much less variety in representations
for the {\em state} of collaboration.  The WWW architecture, for example,
defines no representations for state.  Most http servers do support log
files, which provide a kind of ad-hoc state representation, but the
provision and use of this state information falls well outside the province
of the Web's architecture.

Similarly, Lotus Notes does not provide explicit mechanisms or support for
state representation independent from artifact data.  State information
is represented and manipulated exactly like artifact information: as a Lotus
Note or Note field.  Further, a Lotus Note is a restricted kind of data structure
implementing a kind of ``form''.  The Notes architecture does not provide
explicit support for common data abstractions (such as lists, sets, trees,
graphs, etc.)  useful for state representation and manipulation.  While
Notes developers demonstrate great ingenuity in simulating
higher-level data abstractions using interconnected collections of Notes,
they succeed in spite of the architecture, not because of it.

GroupKit provides a single, yet sophisticated mechanism to represent state,
called an environment. Environments are table-based data structures that
allow users to associate keys with values (which in turn may be another
environment object).  Furthermore, functions can be attached to the add,
change, and delete operations on an environment. These functions can be
used to implement shared environments which maintain common values across
all clients, or support other forms of coordination.  Environments provide
better support for arbitrary state data structure definition than that
provided in Notes, although certain data structures require some twisting
to fit within the table representation.

The Egret architecture implements state storage and retrieval mechanisms
that provide explicit and high-level support for common patterns across our
various application development efforts. For example, the gtable class
implements a shared, persistent global table which is locally
replicated. Gtables are similar to GroupKit's environments, except that
gtables are persistent across sessions while environments are not.  Gtables
provide high-level support for situations in which each client needs fast
access to state data (thus the local replication), some of which may be
produced or used by other clients as well (thus the shared structure), and
which must persist across sessions.

However, gtables are not always the right answer. In other situations,
clients require fast access to state data that they themselves generate
(perhaps from previous sessions), but which other clients need rarely or
never.  In this case, gtables are inappropriate: each client will suffer
the overhead of real-time maintenance of (potentially large amounts) of
useless state information concerning other clients. For this situation,
Egret provides the ptable class, which implements a separate table for each
client that is locally cached and persistent across sessions.  In addition
to gtables and ptables, Egret provides other state storage facilities,
including server nodes, system nodes, personal nodes, and system tables.
Yet others result from the combination of these primitive building blocks.

Finally, in addition to the above mechanisms, Egret supports any data
structure that can be easily passed to a Lisp reader.  Thus, linked lists,
sets, trees, and so forth are simple to construct and manipulate in Egret.

\subsection{State persistence across sessions and conferences}

In this paper, a ``session'' is the contiguous period of time that
a user is ``connected'' to a system. A ``conference'' is the period of time
during which a group of interacting users are connected simultaneously.
Interestingly, CSCW architectures interpret sessions and conferences quite
differently, which affects their support for state persistence.

In the WWW architecture, a session is best viewed as a single http request,
which is typically less than a second long. (Server push/client pull
mechanisms enable much longer sessions, of course.) This interpretation of
a session as a single http request aligns well with the intent of the
designers of the WWW architecture, who state:


{\em 
[HTTP] is stateless... The stateless model is efficient when a link
from one object may lead equally well to an object stored on the same
server, or to another distant object.  [A URI] should always refer
to the ``same'' object. This also makes a stateless protocol appropriate,
as it returns results based on the URI that are irrelevant of any previous
operations performed by the client. \cite{Berners-Lee94}
}

More colloquially speaking, the WWW architecture has no explicit
representation of users who ``visit'' a single web site and ``surf''
through the artifacts provided by its server over the course of minutes or hours.
Similarly, the WWW architecture does not represent or support conferences.
Given the clear intents of the WWW designers with respect to state (and its
lack thereof), and the paucity of support for sessions and conferences,
there is obviously no support for persistent state across sessions or
conferences.

In the replicated database version of Lotus Notes, a session
logically consists of the period of contiguous time during which the user
is connected to the Notes database.  Any state changes stored in
this database will persist until the next session. However, the fact
that databases are locally replicated eliminates the possibility of 
conferences, and also means that state information collected on one 
database from one user is not available to another user until 
synchronization occurs.  The central server database version of Notes
does support sessions, conferences, and persistent state across 
sessions and conferences. 

Computer-based conferencing is the raison d'\^{e}tre of GroupKit, so the
architecture naturally includes well-defined and explicit representations
and support for both sessions and conferences.  However, users, artifacts,
sessions, and conferences are ephemeral in GroupKit: nothing persists
beyond or between conferences. ``The concept of users exists only within
the current real-time conference session.  No information is held about
past users in a conference, nor any notion of a long term community
\cite{Roseman95a}.''

The Egret architecture provides both representations and support for
sessions, and support (such as chat tools) but no explicit representations
for conferences of simultaneously connected, interacting users.  State
persists across sessions and conferences. 

In our experience, support for state manipulation within and between
sessions and conferences involves the following four requirements:

\begin{itemize}

\item Users need to be aware of and react to the state of other users
  within a session and/or conference.

\item  Users joining a conference need to get ``caught up'' to other
  users.  (In other words, the state of the conference needs to be shared
  and synchronized between all participating users.)

\item  Users need to be able to ``pick up where they left off last time.''
  In other words, they need to be able to recover their final state from
  the last session or conference. 

\item  Users need to be aware of and react to the state changes of other
  users that occurred since their last session and/or conference.

\end{itemize}


The WWW architecture supports none of these requirements; GroupKit supports
the first two; Lotus Notes and Egret support all four.  The last
requirement, informing users of what happened of interest while they were
absent, can be particularly problematic to support in an efficient and
responsive fashion.  In Egret, some of our architectural support for agents
came specifically from the fourth requirement, because it is much simpler to satisfy with
an agent that remains continuously connected and serves as a kind of
``virtual participant'' in all sessions and conferences. Figure
\ref{fig:snoopy} illustrates one agent-based mechanism in Egret that maintains
state information within and between sessions. 

\begin{figure}[t]
  \centerline{\psfig{figure=aen-snoopy.ps}} 
\caption{The ``Snoopy'' mechanism in Egret. This simple agent-based mechanism
observes the collaborative process, collects state information about each
participant, and posts a summary periodically to each connected user.}
\label{fig:snoopy}
\end{figure}



\subsection{Location of state information}

One lesson from our experiences with CSCW application development is that
it is possible to collect a lot of {\em potentially} useful state information
about the artifacts, individuals, processes, and the external environment
surrounding collaboration. Whether this state information is {\em actually}
useful depends upon whether it is located at the right place at the right
time.  The right place, for some users at some points in time, may be the
server, while the right place for other users at other times may be the
client.  Different CSCW architectures make different choices about where
state information is maintained, and whether or not the decision can be
changed dynamically as circumstances require.

An example from our development experiences may make these issues more
clear. The AEN collaborative hypertext authoring application supports a
dynamically generated table of contents that represents a linear traversal of
the accessible portions of a document's sectional structure.
(The sectional, or logical structure of the document is not usually
isomorphic to the document's hypertext storage structure.)  The TOC is
context-sensitive to both the state of the artifact and the state of the
user.  First, only those sections for which the user has read access are
displayed in the TOC. Second, the sections which are newly created or which
have been modified since the last time this user retrieved them are
represented in bold font.  This sensitivity to the state of the artifact
and the state of the user means that the TOC can potentially change each
time any node is written by any user or an access control value is changed.
Figure \ref{fig:toc} illustrates one such table of contents.

\begin{figure}[t]
  \centerline{\psfig{figure=aen-toc4.ps}} 
\caption{The ``TOC'' mechanism in AEN displaying the hierarchical section-based
structure 
of a sample document. Each hypertext node can contain an arbitrary number
of sections, which could be actual text or links to other nodes. 
Each user sees only those sections for which they have read access, and 
sections which have changed since this user last read them are bold faced.}
\label{fig:toc}
\end{figure}

A responsive TOC mechanism requires a CSCW architecture to provide flexible
support for location of state information.  For example, if state
information is located only on the server side, then each time a node is
written by one user, all other users must retrieve the node from the
server, parse it to determine the logical structure, and finally rebuild
their TOC.  (Due to embedded links, the logical structure of an AEN
document can change almost arbitrarily when a hypertext node is written.)
The overhead of these three activities on clients is unacceptable during
active collaboration.


AEN solves this using Egret facilities that allow portions of the document
structure state to be replicated at the clients.  When an AEN client writes
out a document node, it also parses the document to determine if any
structural changes have occurred since it was last written.  If changes
occurred, it ``piggy-backs'' information onto the write event broadcast to
all clients that adds information specifying exactly what structural
changes occurred in this node.  This approach reduces the overhead of table of contents
maintenance for non-authoring clients to only the third activity:
rebuilding the table.  Furthermore, rebuilding occurs only when the
TOC has actually changed. There is a minor amount of new overhead on
the authoring client to parse the node for structural changes, but this
activity can be done quite efficiently because the node is already local to
the client.  Were the parsing overhead substantial, it could be off-loaded
to an agent quite easily.

The TOC facility demonstrates a situation in which the
availability of a context-sensitive, time-dependent service depends upon
whether or not the architecture can support locating ``just the right
amount'' of state information at each client, keeping the rest on the
server. We hypothesize that neither the WWW, nor GroupKit, nor Lotus Notes
architectures can support this mechanism responsively.  The WWW cannot
because it does not support local state replication. GroupKit cannot
because it does not support persistent state across sessions.  Lotus Notes
cannot because it lacks the state data abstraction mechanisms required to
collect, process and represent this information efficiently.


\subsection{Distributing state}

Distributing state information is conceptually straightforward: clients
can either query (poll) for data or changes,  or else be signaled
(interrupted) with new data or modifications which are sent by other 
clients. 

As should be obvious by now, the WWW does not support either queries or
signals with respect to state information. GroupKit, Lotus Notes, and 
Egret all support both.  

The GroupKit architecture differs from both Lotus Notes and Egret by
supporting both peer-to-peer inter-process communication as well as the
more traditional centralized server paradigm.  Peer-to-peer communication
can prevent server bottlenecks, as well as removing a source of single
point failure.  

Lotus Notes, as noted before, supports both a replicated server and
centralized server mechanism.  

Egret supports only the centralized server mechanism.

\subsection{Time}

The final dimension concerns time.  Does the CSCW architecture provide
explicit support for representing both current, past, and future state 
information?  

Egret is alone among the CSCW architectures discussed here with
respect to explicit support for characterizing time of state. Egret
provides a ``metrics'' subsystem that supports efficient, low-overhead
collection and storage of time-stamped event data that allows applications
to generate a historical record of the state of the system over time.
While Lotus Notes supports storage of state information, the architecture
provides no explicit facilities for the time-based state representation and
manipulation.  As noted above, GroupKit does not store state information
between conferences at all, so it naturally does not manipulate the time
dimension. The WWW architecture also does not support this dimension.

\section{CONJECTURES}

A discussion of the collection, representation, storage, modification, and
distribution of state information in CSCW architectures may be useful from
a purely intellectual standpoint for some, but others may desire some more
practical outcome.  Even though Egret ``scores high'' with respect to state
manipulation, one would certainly not conclude that it is the ``best'' CSCW
architecture.  Conversely, the WWW architecture exhibits extremely poor
facilities for state management, yet this architecture has produced global
changes in who uses computers and the way in which they are used.  
Other CSCW architectures have produced little more than a ripple compared
to the collaborative tidal wave of the Web.

While state does not serve to differentiate the utility of CSCW
architectures, it still has great value as a means to illuminate certain
fundamental issues and trade-offs made by any design.  We conclude by
offering two conjectures to the CSCW community for reflection and debate.
First, that ``collaboration-in-the-small'' requires fundamentally different
support at the architectural level from ``collaboration-in-the-large''.
Second, that collaborative architectures should be designed with the view
that the interface and processing requirements of autonomous software
agents are just as important than those of human users. Both of these
conjectures arise directly from consideration of state as an organizing
principle.

\subsection{CITS vs. CITL}

To gain a visceral sense for the differences between
collaboration-the-large (CITL) and collaboration-in-the-small (CITS),
consider the following two activities of the Collaborative Software
Development Laboratory, which are typical in many organizations.  

First, we maintain the WWW Archive on Formal Technical Review (FTR), which
is a structured software engineering group process for improving software
quality.  This archive receives a substantial amount of traffic and
receives contributions from many groups world-wide.  The mode of
collaboration for the FTR Archive is one of collaboration-in-the-large:
while we gather occasional statistics on aggregate usage, it is unimportant
(in fact, it would be distracting) for us to be continuously aware of who
is using the archive, where they are in it, what they have seen before,
what they should look at next, what their reaction is to the current
material they are looking at, and so forth.  For systems such as the FTR
Archive, collecting, representing, and responsively presenting state
information about users, artifacts, and processes would not only be
computationally expensive, it would be counter-productive to the
collaborative activity for both us and our users.  For a broadly popular
web-site, such as Yahoo, such state processing would be prohibitive.

In contrast, consider Flashmail, a very simple collaborative system used
extensively by members of CSDL. Flashmail attaches a menu to an editor
window which, when pulled down, displays a list of members, whether they're
logged on, which workstation they're using, and the idle time associated
with that workstation. For any logged in member, one can send a textual
message that appears immediately in a window covering their current screen,
regardless of which application it is. Flashmail is thus a deliberately
intrusive communication facility based upon real-time awareness of the
location and activity of others in a group. The canonical Flashmail message
is, ``I'm hungry. Do you want to go to lunch?''  Flashmail is an excellent
example of collaboration-in-the-small, because there is an intrinsically
limited number of people for whom a user might desire such real-time
awareness, much less allowing them to interrupt in the midst of other
activities. 

Our experiences with state lead us to conjecture that there is a
fundamental difference between systems designed to support
collaboration-in-the-large, involving thousands or millions of people, and
collaboration-in-the-small, involving from a handful to dozens of people.
This difference manifests itself at both the conceptual and architectural
levels. Conceptually, CITL appears to demand a ``collaborative open world
assumption''. Users cannot know, with any level of accuracy, the boundaries
and contents of the system (including users, artifacts, and processes) at
any single point in time. CITS, on the other hand, appears to demand the
opposite: a closed world in which relatively complete knowledge about
users, artifacts, and processes is explicitly represented and immediately
accessible.  The closed world assumption permeates the Egret architecture
at every level, as does the open world assumption in the WWW's. We
conjecture that this relationship is not incidental, but essential. We
invite the CSCW community to consider with us the following question: what
would constitute appropriate architectural support for
collaboration-in-the-medium?

\subsection{Agents as First Class Users}

A second outcome from our reflections upon the role of state is that
architectures must be designed with as much concern for the needs of
autonomous agents as for the needs of human users.  This conjecture results
from our finding that state information tends to be very expensive to
collect, manage, and distribute, and placing this computational load on
either user-level clients or the server often degrades system
responsiveness to an acceptable level.  In addition, the storage allocated
for state information can far exceed that of the artifacts themselves.

It is not enough, in our experience, to simply enable basic autonomous
interaction, such as the ability of the WWW to support spiders.  While
the WWW architecture was not designed with the needs of its agents in mind,
the current architecture of Egret reflects evolution driven as much by the
needs of its agents as its users.

For example, URN, a collaborative news reader, was built using the second
major release of Egret. It utilized an agent that woke up each night to
``read'' news articles and prepare them for the users. Unfortunately, the
high-level Egret {\em user} interface used by the agent formatted each
article for human readability, even though the agent treated it as simple
ASCII text. The formatting, unfortunately, slowed the agent down by a
factor of 10 to 100, and placed an artificially low upper limit on the
number of articles that a single agent could process in a night.  This
architectural flaw forced the designer of the agent to choose between using
a low-level system interface and reprogramming a great deal of higher level
functionality, or using the high-level user interface and sacrificing time
to unnecessary processing. The flaw occurred because we had neglected to
take into account the differences between agents and users. 

Eliciting requirements from software agents is a poorly understood area of
systems design.  We have, nevertheless, found the following heuristics to
be useful during application design. We invite the CSCW community to 
contribute additional insights into requirements for agent-based systems:

\begin{itemize}
\item {\em Agents should be able to do virtually anything a user is
  expected to do.}

  If the CSCW application is primarily textual, does it provide an
  interface to allow an agent to contribute new text?  If the application
  is primarily graphical, is an interface provided that allows an agent to
  draw?  In other words, a CSCW architecture should  make as few {\em a
  priori} assumptions as possible about which behaviors can only be
  provided by humans.  Agents can be surprisingly capable, if they are
  provided with an opportunity.

\item {\em The agent and user interfaces, if different, should be based
  upon a common, software/wetware-independent substrate.}

  Although agents should be able to do anything users do, they might do
  it in a slightly different way. For example, most software agents
  cannot physically move a mouse, so an agent that contributes to a
  drawing must be provided with an interface that allows it to create
  graphical elements on the screen.  For basic software engineering reasons, 
  these interfaces should merge as quickly as possible in the architecture.  

\item {\em Even though agents can mimic humans, don't disregard their
  special properties.}

  As the example of agent-based processing in URN illustrates, the system
  may need to support behaviors from agents that might never be required
  for a system whose users are always humans. When designing interfaces and
  subsystems, after ensuring that they are both human and agent accessible,
  consider whether a supplemental interface designed purely around the 
  agent might not provide useful additional capabilities. 

\end{itemize}


\section{ACKNOWLEDGMENTS}

The Egret architecture and these insights into state as an organizing
principle have emerged from hard work by the members of CSDL over the
years, including: Dadong Wan, Danu Tjahjono, Cam Moore, Rosemary Andrada,
Robert Brewer, Julio Polo, Russ Tokuyama, and Jennifer Geis.

Research on the Egret Framework has been supported by funding from National
Science Foundation, Tektronix, Inc., the University of Hawaii, and the
Pacific International Center for High Technology Research.

\section{EGRET AVAILABILITY}

The Egret source, binaries, and documentation are available free of charge.
To obtain the latest information
on the Egret release, and to obtain downloading instructions, please 
consult our WWW page at: \newline 
http://www.ics.hawaii.edu/$\sim$csdl/egret/release-notes.html


\newpage
\bibliography{/group/csdl/bib/csdl-trs,96-05}
\bibliographystyle{plain}

\end{document}

