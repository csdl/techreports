%%%%%%%%%%%%%%%%%%%%%%%%%%%%%% -*- Mode: Latex -*- %%%%%%%%%%%%%%%%%%%%%%%%%%%%
%% scharding-conclusion.tex -- 
%% Author          : Philip Johnson
%% Created On      : Thu Jun 16 09:44:35 1994
%% Last Modified By: Philip Johnson
%% Last Modified On: Sun Jun 19 10:23:21 1994
%% Status          : Unknown
%% RCS: $Id$
%%%%%%%%%%%%%%%%%%%%%%%%%%%%%%%%%%%%%%%%%%%%%%%%%%%%%%%%%%%%%%%%%%%%%%%%%%%%%%%
%%   Copyright (C) 1994 University of Hawaii
%%%%%%%%%%%%%%%%%%%%%%%%%%%%%%%%%%%%%%%%%%%%%%%%%%%%%%%%%%%%%%%%%%%%%%%%%%%%%%%
%% 

\section{Conclusions}

This paper has described the evolution of our computer-support for FTR from
an initial, hard-wired model that was based upon research findings, to a
more flexible, ``fourth generation'' programming language for FTR that
allows the organization to design for themselves what FTR method will be
useful and how it should evolve over time.  We find an interesting parallel
between our own evolution and that of our characterization of the FTR
research literature. Although we viewed the ``prescriptive'' literature as
over-constraining, we now see that our early publications on CSRS would be
properly characterized in that manner. The evolution in our design also
reflects an evolution from a prescriptive to a descriptive viewpoint.  This
view, although perhaps as under-constraining as the prescriptive
perspective is over-constraining, is what our interaction with reality has
pushed us toward.

As we reflect upon this process, however, we do not believe that the
appropriate ``lessons learned'' for the design of computer-based support
environments are to (a) ignore the research literature and simply interview
100 organizations to find out their requirements; or (b) generate a general
fourth generation framework from scratch that satisfies everyone's
requirements.

Interestingly, we currently believe that our process was not only a
rational, but perhaps an inevitable one.  The comments and reactions we
were able to solicit from companies were so constructive and helpful
exactly because they were based upon a reaction to a clearly designed,
specified, and implemented paradigm for computer-supported FTR.  Had we
merely interviewed several companies about what they would desire from a
hypothetical computer-supported FTR environment, we doubt that the answers
would have had nearly the specificity or quality of those we elicited with
CSRS/FTArm.

We suspect that in experimental and emergent domains, our process reflects
a natural form of ``bootstrapping''.  When you cannot, by definition, know
what the impact of introducing computer-support on an environment will be,
your best strategy is to learn all you can about the non-computer supported
task environment, design a computer-based system to reflect this knowledge,
and then provide this implemented, proof-of-concept to organizations to
assess its viability.  No organization categorically rejected any single
feature of FTArm as inappropriate under all circumstances.  What
organizations revealed to us was the impact of certain combinations of
design decisions.  CSRS 3.0, in response, unbundles certain design
decisions that were linked in CSRS/FTArm.  

The next stage in the evolution of this research will be to refine the FTR
method definition paradigm through close collaboration with organizations.
We hope to develop and publicize various ``canonical'' methods built using
our language that organizations can use as further illustrations of the
space of possible computer-based FTR methods.  We are certain, however,
that there await many more surprising and unanticipated problems in the
design of computer-supported formal technical review.





\section*{Acknowledgments}

The author gratefully acknowledges the other members of Collaborative
Software Development Laboratory: Danu Tjahjono, Rosemary Andrada, Carleton
Moore, Dadong Wan, and Robert Brewer for their contributions to the
development of CSRS. Support for this research was partially provided by
the National Science Foundation Research Initiation Award CCR-9110861.

