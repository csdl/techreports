%%%%%%%%%%%%%%%%%%%%%%%%%%%%%% -*- Mode: Latex -*- %%%%%%%%%%%%%%%%%%%%%%%%%%%%
%% scharding-intro.tex -- 
%% RCS:            : $Id: icse94-intro.tex,v 1.4 94/02/14 15:39:04 johnson Exp Locker: johnson $
%% Author          : Philip Johnson
%% Created On      : Thu Aug 12 11:14:18 1993
%% Last Modified By: Philip Johnson
%% Last Modified On: Sun Jun 19 07:36:43 1994
%% Status          : Unknown
%%%%%%%%%%%%%%%%%%%%%%%%%%%%%%%%%%%%%%%%%%%%%%%%%%%%%%%%%%%%%%%%%%%%%%%%%%%%%%%
%%   Copyright (C) 1993 University of Hawaii
%%%%%%%%%%%%%%%%%%%%%%%%%%%%%%%%%%%%%%%%%%%%%%%%%%%%%%%%%%%%%%%%%%%%%%%%%%%%%%%
%% 
%% History
%% 12-Aug-1993		Philip Johnson	
%%    

\section{Introduction}

Design for novel domains and in novel environments suffers from a
bootstrapping problem: domain requirements should drive the design, but the
design makes a direct impact upon the domain requirements.  This happens
most frequently in domains where computer-support is being introduced to
replace a manual process.  Simple automation of the manual process steps
frequently leads to a poor design, since the introduction of automation may
fundamentally change the nature of the requirements or and/or task
environment.  The introduction of spreadsheet technology is a classic
example of this latter phenomenon: spreadsheets were originally intended to
support accounting department personnel by automating their tasks, but
resulted in fundamentally changing the nature of work performed by
accounting departments, since spreadsheets enabled certain accounting tasks
to be distributed throughout the organization.

Such design situations have been termed ``experimental and embedded''
\cite{Giddings84}.  Experimental means that the appropriate structure and
process for problem solving in the domain cannot be well determined except
through experimentation with a variety of techniques.  Embedded means that
the introduction of new support may impact upon the domain in unpredictable
ways.  The more prosaic ``wicked problem'' has also been suggested as 
terminology to describe these kinds of design \cite{Partridge85}.

This paper presents a case study of one such wicked problem: the design of
computer support for formal technical review (FTR).  The domain of
computer-supported formal technical review contains all of the classic
ingredients for an experimental and embedded design: there is little
understanding of what constitutes effective computer support for formal
technical review; existing FTR procedures are designed to exploit the
strengths and minimize the weaknesses of non-computer supported
environments; and introduction of computer-supported FTR can lead to
forms of FTR that would be impossible to carry out in a non-automated 
context. 

For almost three years, we have been designing, evaluating, and redesigning
a system called CSRS which provides computer-supported formal technical
review.  Over the three years and three major versions of CSRS, we have
undergone a dramatic evolution in our conceptualization of what the issues
are in computer supported formal technical review.  We view this evolution
in general as a change from a {\em principle-centered} to an {\em
organization-centered} design philosophy, and the existance of this change
and its nature is a direct result of the experimental and embedded nature
of our problem domain.

When we began our research on computer-supported FTR, there were just a
handful of systems providing only prototypes of computer support (this also
reflects the current state of the field.)  In contrast, there was fairly
substantial research literature on the theory and practice of manual FTR,
and this literature revealed a set of common issues that appeared
responsible for much of the adverse outcomes in FTR practice.  Furthermore,
there was literature from another domain (electronic meeting rooms) which
appeared to suggest ways in which computer support could address these
types of issues.  We ``put two and two together'', and the first two
versions of CSRS resulted.  These designs were principle-centered in that they
embodied the following argument: (a) given a set of problems currently arising in
the practice of FTR, and (b) given research on computer-supported electronic
meeting rooms that appears to address these problems, then (c) an effective
solution will result from designing a computer-supported FTR environment
that resolves these problems.

Interestingly, the logic of this argument did not survive reality.  We have
discovered that the set of problems arising in computer-supported FTR is
different from the set of problems arising in manual FTR.  Initially we
believed that by resolving the set of problems occurring in manual FTR, we
were resolving generic FTR problems.  Now we understand that by providing
computer-support, we fundamentally changed the problem domain and created
an entirely new set of problems which we did not anticipate and which do
not occur in manual FTR.

In our current design of CSRS, we have abandoned the notion that we can
``design in'' solutions to problems in FTR.  Instead, the system now
embodies a ``design space'' of possiblities for FTR systems, and we now
perceive our task as one in which we help an organization to discover what
form of FTR is effective for their particular organizational context.  We
call this new perspective organization-centered, because we have abandoned
the promotion of any particular form of FTR, and instead believe that an
appropriate FTR method is one which organically grows over time within an
organization, and the design of computer-support must explicitly reflect
the individualized and emergent nature of the method.

The next several sections of this paper discuss the history of our design
for computer supported FTR.  Following this, we discuss some implications
of this case study and its relationships to other works on design.









 


