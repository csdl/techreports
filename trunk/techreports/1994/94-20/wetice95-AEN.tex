%%%%%%%%%%%%%%%%%%%%%%%%%%%%%% -*- Mode: Latex -*- %%%%%%%%%%%%%%%%%%%%%%%%%%%%
%% wetice95-AEN.tex -- 
%% Author          : Carleton Moore
%% Created On      : Thu Dec  8 10:54:14 1994
%% Last Modified By: Philip Johnson
%% Last Modified On: Thu May 11 08:54:27 1995
%% Status          : Unknown
%% RCS: $Id: wetice95-AEN.tex,v 1.17 1995/05/11 20:26:54 johnson Exp $
%%%%%%%%%%%%%%%%%%%%%%%%%%%%%%%%%%%%%%%%%%%%%%%%%%%%%%%%%%%%%%%%%%%%%%%%%%%%%%%
%%   Copyright (C) 1994 University of Hawaii
%%%%%%%%%%%%%%%%%%%%%%%%%%%%%%%%%%%%%%%%%%%%%%%%%%%%%%%%%%%%%%%%%%%%%%%%%%%%%%%
%% 

\section{An Overview of AEN}

AEN is implemented with the Egret collaborative system development
environment \cite{csdl-www-egret,csdl-93-09}.  As with all Egret
domain-specific applications, AEN is a client-server system running over
TCP/IP, where clients (implemented as customized versions of
XEmacs) interact with a central server called HBS (implemented in C++) to
store and retrieve information or communicate with other
clients.  AEN makes substantial use of Egret's facilities for agents,
or autonomous client processes that extend the system's information
processing capabilities beyond those provided by HBS or the clients.  Figure
\ref{fig:aen-architecture} illustrates AEN's basic client-server-agent
architecture.


\begin{figure*}[htb]
 \centerline{\psfig{figure=client-server.eps,width=4in}}
 \caption{\ls{1.0} AEN's architecture from an OS process perspective. AEN consists
of a central server called HBS that provides basic storage and concurrency
control mechanisms, augmented with several agent processes to support
strong collaboration.  AEN was used with anywhere between one and twelve simultaneous
user-level client connections during Fall, 1994.}
\label{fig:aen-architecture}
\end{figure*}


AEN was used during the Fall semester of 1994 in a graduate seminar on
collaborative systems in the Department of Information and Computer Sciences
at the University of Hawaii.  AEN was used both as an example of a
collaborative system for the class, as well as the principle mechanism for
instruction and interaction among class participants.  Nine students enrolled
in this seminar, and two other colleagues participated in the seminar from
geographically dispersed locations (the Big Island of Hawaii and Berkeley,
CA.).  Egret's architecture supports client connections from any Internet
location supporting socket-based connections.

An initial design goal of AEN was to support virtually all lecture material
and participant interactions for the seminar.  Participant interaction
would consist of reading and reacting to material posted by students and
the instructor.  Based upon the reactions, the material would either be
revised and improved, or new material would be added. On-line quizzes, as
well as semester projects, would assess the learning process. A small
number of face-to-face classroom meetings would provide supplementary
forums to discuss things learned or questioned through on-line activities.
Through AEN, groups of students were expected to learn how to develop
software in Emacs Lisp, use Egret to implement a collaborative system,
design and present a project proposal for a collaborative system, and
create a well-structured report on the results of their effort.

Given this context of use, AEN must support an environment in which
hypertext information is continuously generated, read, annotated, and
modified by groups of individuals.  To support coordination in a
distributed classroom environment, AEN also provides real-time (textual)
communication facilities.  Our experiences with AEN lead us to believe that
such facilities are essential for computer-mediated strong collaboration.

AEN is similar to several other collaborative systems.  SEPIA
\cite{Haake92}, a hypertext authoring system that supports collaborative
authoring, does not appear to provide restricted read or write access in
order to support temporary subgroups like AEN does.  SEPIA does have a
strong sense of physical presence like AEN.  Each user has a location in
the hypertext document visible to the other users.

ConversationBuilder \cite{Kaplan92} is a collaborative work tool that supports
more than collaborative writing.  It also supports a strong, flexible
process model of obligations that the users must complete before they can
move on to another task in ConversationBuilder.  AEN is not designed to
have a strong process model since we do not wish to constrain any potential
styles of collaborative authoring and learning.

gIBIS \cite{Conklin88} is a design rationale capturing system.  The process
of collecting and storing design rationale is essentially additive in
nature.  New issues, positions and arguments are added to the existing
database to keep track of the design process.  The data stored in gIBIS is
used for reference and is not intended to be edited at a later date.  Unlike gIBIS,
AEN provides explicit facilities, such as unread nodes, access control, and
locking to support changes to existing information within the system.

% NoteCards:  I don't have any papers at this time

% Intermedia:  I don't have any papers at this time

CLARE \cite{csdl-93-21}, a computer-based collaborative learning
environment, has a very strong process and data model used to learn about
scientific text.  Users of CLARE evolve a group knowledge base based upon
comparing and debating their interpretations of a scientific text.  These
comparisons and debates are additive.  The original statements are not
edited to reflect the new consensus.  In AEN, users edit the original
document or arguments as part of building consensus.  Unlike AEN, CLARE
does not provide synchronous communication between the participants and
there is no representation of user presence in the hypertext document.

% ShrEDIT:  I don't have any papers at this time

% Lotus Notes:  I don't have any papers at this time

WWW \cite{Berners-Lee94} is designed for global, Internet-based storage and
dissemination of hypertext documents, multi-media data.  The WWW provides
crude, password-based access control, no locking or other form of
concurrency control, and virtually no representation of user state.  These
design decisions facilitated the rapid expansion of the WWW to its current
multi-million user population.  AEN, on the other hand, is designed
to accommodate collaboration among small groups (on the order of dozens).  This
radical difference in scale enables AEN to provide many facilities
not conceivable in the WWW (such as snoopy, or unread nodes).  However,
these same facilities impose a ``ceiling'' on the maximum group size
supported by AEN.


The next section goes into detail on the current design of AEN.  Following
this, we present results from our initial use of AEN, and the general
lessons learned from these experiences.  The paper concludes with our 
future directions.

\section{The design of AEN}

First and foremost, AEN provides facilities for storage, retrieval, and
analysis of a hypertext network of nodes and links.  AEN provides a static,
predefined set of node and link types.

\subsection{Node Types}

At the user-level, AEN provides two type hierarchies of nodes: artifact
nodes and figure nodes.  Nodes of type {\em artifact} represent textual
information in the hypertext document.  Nodes of type {\em figure}
represent graphical information.

All artifact nodes are structured in ETML\foot{A primary difference between
the Egret Text Mark up Language (ETML) and HTML is that ETML supports an
{\tt egret::} URL specifier for link and node objects within an HBS
server.}, a minor variant of HTML that provides formatting directives and
anchors for inter-node links. There are four subtypes of artifact:

\begin{itemize}
\item{\em Document:} This node type contains the text of the document created by
the participants.
\item{\em Comment:} This node type contains the reactions of the participants to
the contents of other nodes.
\item{\em Quicky Quiz:} This node type contains exercises for the participants
to complete.
\item{\em Quicky Quiz Answer:} This node type contains participant's
solutions to the quicky quiz nodes.
\end{itemize}


%\begin{figure}[htb]
% \centerline{\psfig{figure=Data.eps,width=6in}}
% \caption{Node type hierarchy in AEN}
% \label{fig:Data}
%\end{figure}

Unlike the four subtypes of artifact nodes, nodes of type figure
contain graphic data in any format compatible with Xview.  Users can create
links from artifact nodes to other artifact or figure nodes by a
simple menu-based command.  Such an operation results in both the
creation of a new link object within the server and the creation of an
ETML link anchor as part of the contents of the artifact node.
However, since figure nodes contain graphic information compatible with XView
and does not support link anchors, users cannot create links from
figure nodes to other nodes.


\subsection{Link Types}

AEN defines six types of links: Include, Xref,
see\_Quicky\_Quiz, see\_Quicky\_Quiz\_Answer, see\_Comment, and
see\_Figure.  AEN restricts the type of the source and destination nodes
for each type of link. Figure \ref{table:link-types} describes the different
links.  

\begin{figure*}
\begin{center}
\begin{tabular}{|l|l|l|l|}
\hline
Link Type&Source Node&Destination Node&Description\\ \hline \hline
Include&document&document&Backbone structure \\
&&&of hypertext document\\ \hline
Xref&document&document&Cross reference\\ \hline
See\_Comment&Any&comment&Points to a Comment\\ \hline
See\_Quicky\_Quiz&Any&Quicky Quiz&Points to a Quicky Quiz\\ \hline
See\_Quicky\_Quiz\_Answer&Quicky Quiz&Quicky Quiz Answer&Points to a \\
&&&Quicky Quiz Answer\\ \hline
See\_Figure&Any&Figure&Points to a Figure\\ \hline
\end{tabular}
\end{center}
\label{table:link-types}
\caption{Link Types}
\end{figure*}


\subsection{The AEN Backbone}
\label{sec:aen-backbone}

Given the set of node and link types in AEN, there are many possible ways
to organize the resulting hypertext network.  However, given the intended
usage of AEN, it is natural and appropriate to provide a fundamental
organization of the hypertext document as an ordered sequence of
interlinked, hypertext ``chapters'', each hypertext chapter having as its
root a single document node.  Each chapter is organized as a sequence of
interlinked ``sections'', each section as a sequence of interlinked
``subsections'' and so forth.

Any document node can, of course, have a link to it from more
than one other document node in the hypertext document, with the result that the
same node can be a ``section'' with respect to one node, but a
``sub-subsection'' with respect to another.  As a result, the AEN hypertext document
is organized as a directed, hierarchical, but potentially cyclic graph structure.

So far, this description seems to indicate that there is a one-to-one
correspondence between each document node and each chapter, section,
subsection, and so forth in the document.  The reality is slightly more
complex.  Each document node does contribute at least one new chapter, or
section, or subsection, etc. to the hypertext document.  However, if a document
node contains the HTML {\em heading} formatting directive, then the label for
this heading indicates a new section (or subsection, etc. depending upon
context) within the node.

To bootstrap this organization, some distinguished ``root'' document node
is required, which we call the AEN Backbone.  This node contains a set of
links to document nodes which define the set of top-level chapters in AEN.
The AEN Backbone node was modified rarely during the semester; the initial
set of chapters appeared to provide a sufficient number of ``jumping off
points'' for users and groups to create their own hypertext documents.

The AEN Backbone and the nested structure of document nodes defines the
major organizing principle for the AEN hypertext document, and leads directly to a
browsing mechanism: the table of contents screen, as shown in Figure
\ref{fig:Screen1}.

\begin{figure*}[htb]
 \centerline{\psfig{figure=screen1.ps,width=6in}}
 \caption{\ls{1} This screen shot shows a portion of this 
 user's table of contents in the upper left screen, and a retrieved
 document node called ``Design'' in the upper right screen.  The Design
 node contains within it both a link to a figure called ``Figure 1'' and a
 comment called ``Regions'' immediately after it (represented in the
 Design node by the textual anchor ``[??]''.  Both the figure and the
 comment are retrieved and displayed in the screens in the lower left and
 lower right hand corners, respectively.}
 \label{fig:Screen1}
\end{figure*}


\subsection{Primitive Node and Link Operations}

For browsing, AEN emulates many of the facilities of WWW browsers such as
Mosaic or Netscape.  Each node is displayed in a separate screen, and
multiple nodes can be displayed simultaneously.  The contents of nodes are
automatically reflowed to conform to the size of the screen.
Middle-clicking over a link anchor retrieves the corresponding node from
the server.  As noted above, AEN supports storage and retrieval of both
text and graphics, and although audio is not currently
available in AEN, it has been implemented in other Egret-based
applications, and could be easily added.  Due to a limitation in the
current release of XEmacs, in-line graphics are not currently available in
AEN. 

AEN begins to depart from WWW systems in its support for authoring and
collaboration.  First, AEN provides an authoring mode for nodes, including
menu and command key-based formatting directives for headings, lists, and
fontification, link anchor creation, and so forth.  Second, all artifact
nodes must be locked before they can be modified.  Locking preserves the
integrity of information when multiple authors are working on a document,
but provides only basic support for collaboration.  AEN augments locking
with an access control mechanism, discussed next. Other differences between
AEN and WWW will be discussed later.


\subsection{Access Control}

A predominant use of AEN was in the creation of mid-semester project
proposals and final reports, in which the class split up into small groups
of two to four participants each.  Supporting strong collaboration in this 
situation required a means to indicate what newly created information was
of interest to others, both within and between groups.  Furthermore, when 
information was provided for review and critique by others, a mechanism
was needed to indicate how readers should respond.  AEN's access control
mechanism supports a strongly collaborative approach to this issue.

AEN implements three forms of access to each individual node: read access,
write access, and annotation access. The owner(s) of a node can set each
form of access independently for each user of AEN by a menu-based
operation.

When no access is provided to a node, then that node is essentially
private and can be retrieved and edited only by the owner of the
node. (The creator of a node is given the status of ``owner'', which
has the effect of irrevocably providing all forms of access. Only the
owner of a node can delete the node.)

When a user has read access for a node, they can retrieve it but they
cannot add a link to it or edit its contents. Providing only read access to
a node to others indicates that the information is available for use, but
not for review or feedback.

When a user has both read and annotation access for a node, they can both
retrieve it and add a link from the node to a newly created comment
node. Providing both read and annotation access indicates that the
information is available for use and that comments from others about the
content of the node are welcomed.  Annotation access allows the owner of
the node to solicit suggestions for changes to the contents of the node,
while still retaining control over whether those changes are actually
applied to the document.

When a user has both read and write access for a node, they can both
retrieve it and lock it for editing.  Providing both read and write access
signifies the desire for the closest form of collaborative development, in
which those with write access can make arbitrary changes to the contents of
a node.

Of course, the owner of a node may grant all three access rights to another
user, which allows them to select which form of feedback to provide.  It is
also interesting to note that without read access, neither annotation nor
write access is useful, since one cannot annotate or lock a node without
first retrieving it.  This is actually quite useful in practice, since it
allows the owner of a node to temporarily ``hide'' a node simply by
disabling read access to the group.  By re-enabling read access, the
previous state of read, annotation, and write access are all reinstated.


\begin{figure*}[htb]
 \centerline{\psfig{figure=screen2.ps,width=6in}}
 \caption{\ls{1.0} This screen shot illustrates some of the real-time support
for strong collaboration in AEN.  The Snoopy window in the lower right hand
corner shows, for each user of the system, whether or not they are logged
in currently and if so, where they are working in the document.  Both user
cmoore and user jeremyt are working on the document node called ``Shared
Emacs''.  The Partyline screen, in the lower left hand corner, allows these
users (as well as all others) to talk to each other while they work.  This
figure also shows a local TOC for the Shared Emacs chapter, and a screen
displaying the current set of unread document nodes for this user.}
 \label{fig:Screen2}
\end{figure*}

\subsection{Context-sensitive tables of contents}

The access control mechanism determines which nodes a user may read and
what they may do to the node once they have read it, but it
does not help the user determine which node they want to read. 

The primary navigation mechanism provided by AEN is {\em tables of
contents}. Each table of contents (TOC) is a dynamically generated, linear
traversal of document nodes in the database, organized according to the AEN
Backbone structure described in Section \ref{sec:aen-backbone}.  Figures
\ref{fig:Screen1} and \ref{fig:Screen2} each contain an example of a table
of contents. Each line in a TOC is mouse-sensitive,
and middle-clicking on it retrieves the corresponding node (or section
within a node) from the server. The indentation of headings provides a
visual cue to the hierarchical structure of the hypertext document.  Once the TOC is
displayed the user can collapse or expand the listing to reduce or enlarge
the number of listings present.

AEN generates a TOC based upon a starting node that the user chooses.
The TOC algorithm only analyzes document nodes for which the user has
read access, thus, the TOC is both location and user-specific.  The
algorithm only visits the children of a node once, which breaks cycles
in the hypertext document. Finally, each time a document node is written 
to the server, structural information 
relevant to the TOC is extracted, cached, and propagated to all
clients if it has changed. Such a mechanism is crucial to the 
usability of the system; one cannot retrieve every document node
from the database each time a TOC is generated.


\subsection{Node Lists and Hyperstar Bulletin}

The tables of contents provide the most important perspective on the
hypertext document---that of the document-level structure, but other viewpoints
are also useful.  Users can build customized queries for nodes
satisfying various criteria with the node-list mechanism. In AEN,
useful criteria so far include the type of nodes, whether or not the
nodes have been retrieved by this user yet, and so forth.

Each node-list returned from a query consists of a mouse-sensitive
list of node names appearing in a single screen. Middle-clicking
on the node retrieves it in the standard fashion for AEN navigation.

The Hyperstar Bulletin\foot{The name of this mechanism is a pun on the
name of Honolulu's evening newspaper: the Star-Bulletin.} is an agent-based
mechanism intended to support collaboration via a daily ``snapshot'' of
what's changed in the database since each user last visited it.  The
Hyperstar Bulletin consists of an agent process that wakes up each morning
and sends a daily electronic ``newspaper'' to each participant containing a
listing of which nodes (for which the user has read access) have been
created or changed since the last time the user logged in.

\subsection{Snoopy and Partyline}

The mechanisms described above support the construction of a hypertext documents by
multiple authors, but falls short of support for strong collaboration in
one crucial way.  With these above mechanisms alone, the actual people
involved in the collaborative activity are ``invisible'' within the system:
their effects upon hypertext document are visible, but they themselves are
not.  To create a physical presence for participants within the system, AEN
provides two additional facilities: Snoopy and Partyline, as illustrated in
Figure \ref{fig:Screen2}.

The Snoopy mechanism allows the user to see who else is at work and have an
idea of what they are working on.  It displays basic information about the
connection status and last read node of the users of AEN.  A Snoopy screen
is created automatically each time a user connects to AEN, and is updated
every 30 seconds.  Through Snoopy, each user knows who else is currently in
AEN and where they are in the hypertext document.

Partyline is a mechanism for passing real time messages to other users of
AEN.  Partyline allows the user to send a textual message to all the users
currently connected to AEN or send a private message to one other user
currently connected to Partyline.  Partyline is similar to an Internet
``chat'' mechanism.  Typically, users use Partyline messages to greet the
other people logged in to AEN.







