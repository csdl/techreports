%%%%%%%%%%%%%%%%%%%%%%%%%%%%%% -*- Mode: Latex -*- %%%%%%%%%%%%%%%%%%%%%%%%%%%%
%% wetice95-experiences.tex -- 
%% Author          : Philip Johnson
%% Created On      : Thu Dec 22 09:46:32 1994
%% Last Modified By: Carleton Moore
%% Last Modified On: Wed May 10 14:40:37 1995
%% Status          : Unknown
%% RCS: $Id: wetice95-experiences.tex,v 1.3 1995/05/11 00:40:46 cmoore Exp $
%%%%%%%%%%%%%%%%%%%%%%%%%%%%%%%%%%%%%%%%%%%%%%%%%%%%%%%%%%%%%%%%%%%%%%%%%%%%%%%
%%   Copyright (C) 1994 University of Hawaii
%%%%%%%%%%%%%%%%%%%%%%%%%%%%%%%%%%%%%%%%%%%%%%%%%%%%%%%%%%%%%%%%%%%%%%%%%%%%%%%
%% 


\section{Experiences}

The initial requirements document for AEN \cite{csdl-94-06} was developed
in June, 1994, and an alpha Version 1.0 was released for classroom use in
September, 1994.  By mid-October, we finished a substantial redesign of AEN
and released Version 2.2.0.  As of December, 1994, the current version of
AEN is 2.3.13, representing the result of many bug fix and minor
enhancement releases, but no major new redesigns. Version 2.3.13 contains
approximately 12 KLOC in Emacs Lisp.

Precise and accurate metrics data for only a small portion of the Fall
semester was collected by AEN, due to volatility in both AEN and the metrics
system.  However, the collected data does provide some insights into the
use of AEN.  First, approximately 285 hours of active use of AEN were
logged during the second half of the Fall semester by ten participants, for
an average of 28 hours per person.  Approximately 800 nodes and 800 links
were created during the semester by the class.  As a class project, one
group designed an interface to the WWW from AEN, and a snapshot of the AEN
database as it existed in December, 1994, is available for perusal from
the AEN Home Page \cite{csdl-www-aen}.

In general, our experience with using AEN as the principal instructional
format for a class on collaborative systems was extremely successful.
Students acquired a ``visceral'' sense for the strengths and weaknesses of
collaborative systems, and their end-of-semester papers, in which they
answered such questions as ``what is a collaborative system'' showed
substantial sophistication.  For example, by the end of the semester, no
one in the seminar viewed the World Wide Web as a ``real'' collaborative
system.

The most significant research problem we encountered was entwining the
actual design and implementation of the system with its experimental
use. While this was very effective in rapidly evolving the design toward a
highly usable state and in discovering and adding many helpful features, it
also led to significant volatility, coarse measurements of usage, several
disastrous system crashes, and lack of user confidence in the system.  By
the end of the semester, the class viewed AEN as both powerful and
dangerous: while it provided far superior facilities for collaboration, it
was also far more likely to lose their work.

For class projects, students created: a shared emacs editor; a regular
expression search facility for AEN; an AEN to WWW converter; a design for
MUD facility using Egret; and a help/tutorial facility for AEN.  Details on
these projects are accessible from the AEN Home Page.  In general, students
found that AEN and Egret provided an excellent foundation for the
implementation of collaboration mechanisms. For example, the shared emacs
authors noted the following: ``The Shemacs code is a mere 1267 lines long!
One would expect a program of this nature to take 50,000 lines or so.''
All of the class projects are essentially at the alpha release stage, and
serve primarily to explore the requirements for such mechanisms.
Fortunately, several of the students in the class have decided to continue
working on these projects in future semesters.

Perhaps the most fundamental benefit that accrued from the use of AEN was
the increased access by participants to the intermediate work products of
others.  During the mid-semester and final project preparation, students
frequently provided annotation access to the entire class of their
hypertext documents as they prepared them.  This did, of course, provide
students with commentary from others, but much more importantly, it
provided students with examples of how other students were approaching
their projects, and the chance to compare notes while time remained to act
upon the insights gained.  We believe that such access was AEN's strongest
support for strong collaboration.











