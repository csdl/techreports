%%%%%%%%%%%%%%%%%%%%%%%%%%%%%% -*- Mode: Latex -*- %%%%%%%%%%%%%%%%%%%%%%%%%%%%
%% wetice95-intro.tex -- 
%% Author          : Carleton Moore
%% Created On      : Thu Dec  8 10:53:36 1994
%% Last Modified By: Carleton Moore
%% Last Modified On: Wed May 10 13:07:47 1995
%% Status          : Unknown
%% RCS: $Id: wetice95-intro.tex,v 1.8 1995/05/11 00:14:43 cmoore Exp $
%%%%%%%%%%%%%%%%%%%%%%%%%%%%%%%%%%%%%%%%%%%%%%%%%%%%%%%%%%%%%%%%%%%%%%%%%%%%%%%
%%   Copyright (C) 1994 University of Hawaii
%%%%%%%%%%%%%%%%%%%%%%%%%%%%%%%%%%%%%%%%%%%%%%%%%%%%%%%%%%%%%%%%%%%%%%%%%%%%%%%
%% 

\section{Introduction}

Hypertext has become a preferred medium for electronic presentation of
textual information.  Virtually all on-line documentation for commercial PC
and mainframe applications appears in hypertext format, and the exploding
popularity of the World Wide Web \cite{Berners-Lee94} and its HTML protocol 
virtually guarantees that the ``information superhighway'' will be
hypertextual.

Typically, hypertext document development and use is only {\em weakly\/}
collaborative in nature.  The usual paradigm involves a single person
constructing a hypertext document to be browsed at some future point by
others. Some systems support links between separately authored documents,
and some allow one user to annotate another's document.  However, it is
rare for a hypertext system to explicitly support multiple authors in the
construction of a single hypertext node, rarer still for the system to
explicitly entwine the construction, browsing, critique, and improvement of
hypertext documents by authors and their audiences.

This paper reports on our research into requirements for systems to
support {\em strongly\/} collaborative hypertext document construction.  By ``strong
collaboration'', we mean to evoke situations in which a group
synergistically develops and improves a structured artifact more
efficiently and effectively than would be possible by the same group of
people working independently.  In general, strong collaboration most often
(though not always) occurs in non-computer mediated, face-to-face contexts such as a group
software design session or an interactive classroom setting.

Characteristics of strong collaboration are collective authoring,
collective learning, and collective ownership. In other words, during the
process of strong collaboration, each participant is both contributing to
the construction of the artifact, and gaining new knowledge as a direct
result of this construction.  The constructed artifact is not simply a
patchwork of individual contributions, but instead an incremental, emergent
synthesis that reflects the knowledge created by the group as a whole.

Whether or not strong collaboration occurs within a group depends upon many
factors, of which a supportive computational infrastructure is only
one.  We have learned from this research that appropriate technology can
support and even encourage strong collaboration, but cannot guarantee it.

To better understand the computational requirements for support of strong collaboration
in the construction of hypertext documents, we designed, implemented, and 
evaluated a system
called AEN\foot{An acronym for Annotated Egret Navigator.}.  The remainder
of this paper will discuss the design of AEN, our experiences with its use,
and the lessons we learned.  We hope that our successes and failures with
the use of AEN to support strong collaboration will provide helpful insight
to designers of current and future collaborative systems.















