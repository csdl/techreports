	LATEX MACROS
%template for producing IEEE-format articles using 
   LaTeX.
%written by Matthew Ward, CS Department, Worcester 
   Polytechnic Institute. %use at your own risk.  
   Complaints to /dev/null.
%make two column with no page numbering, default is 10 
   point 
\documentstyle[twocolumn]{article}
\pagestyle{empty}
%set dimensions of columns, gap between columns, and 
   paragraph indent 
\setlength{\textheight}{8.75in}
\setlength{\columnsep}{2.0pc}
\setlength{\textwidth}{6.8in}
\setlength{\footheight}{0.0in}
\setlength{\topmargin}{0.25in}
\setlength{\headheight}{0.0in}
\setlength{\headsep}{0.0in}
\setlength{\oddsidemargin}{-.19in}
\setlength{\parindent}{1pc}
%I copied stuff out of art10.sty and modified them to 
   conform to IEEE format
\makeatletter
%as Latex cosiders descenders in its calculation of 
   interline spacing,
%to get 12 point spacing for normalsize text, must set it 
   to 10 points 
\def\@normalsize{\@setsize\normalsize{10pt}\xpt\@xpt
\abovedisplayskip 10pt plus2pt minus5pt\belowdisplayskip 
\abovedisplayskip \abovedisplayshortskip \z@ 
plus3pt\belowdisplayshortskip 6pt plus3pt 
minus3pt\let\@listi\@listI}
%need an 11 pt font size for subsection and abstract 
   headings 
\def\subsize{\@setsize\subsize{12pt}\xipt\@xipt}
%make section titles bold and 12 point, 2 blank lines 
   before, 1 after 
\def\section{\@startsection {section}{1}{\z@}{1.0ex plus 
1ex minus .2ex}{.2ex plus .2ex}{\large\bf}}
%make subsection titles bold and 11 point, 1 blank line 
   before, 1 after 
\def\subsection{\@startsection {subsection}{2}{\z@}{.2ex 
plus 1ex} {.2ex plus .2ex}{\subsize\bf}}
\makeatother
\begin{document}
%don't want date printed
\date{}
%make title bold and 14 pt font (Latex default is non-
   bold, 16pt) 
\title{\Large\bf My Wonderful Article in IEEE Format}
%for single author (just remove % characters)
%\author{I. M. Author \\
%  My Department \\
%  My Institute \\
%  My City, STATE, zip}
%for two authors (this is what is printed) 
\author{\begin{tabular}[t]{c@{\extracolsep{8em}}c} 
I. M. Author  & M. Y. Coauthor \\
 \\
	My Department & Coauthor Department \\
	My Institute & Coauthor Institute \\
	City, STATE~~zipcode & City, STATE~~zipcode
\end{tabular}}
\maketitle
%I don't know why I have to reset thispagestyle, but 
   otherwise get page numbers 
\thispagestyle{empty}
\subsection*{\centering Abstract}
%IEEE allows italicized abstract
{\em
This is the abstract of my paper.  It must fit within the 
size allowed, which is about 3 inches, including section 
title, which is 11 point bold font.  If you don't want 
the text in italics, simply remove the 'em' command and 
the curly braces which bound the abstract text.  If you 
have em commands within an italicized abstract, the text 
will come out as normal (non-italicized) text. 
%end italics mode
}
\section{Introduction}
Here is my introduction text.  There are 2 blank lines 
before the section heading and one afterwards.  Heading 
text is 12 point bold font.  Paragraphs are not indented. 
I may want a numbered subsection, which is done as 
follows.
\subsection{Previous Work}
In subsections there is 1 blank line before the section 
heading and one afterwards.  Heading text is 11 point 
bold font.  Paragraphs are indented one pica.  There is 
no blank line between paragraphs.
Throughout I may cite references of the form 
\cite{key:foo} or \cite{foo:baz}, and LaTeX will keep 
track of numbering.  The numbers are based on the order 
you place them in the bibliography, not the order they 
appear in the text.  They should (I believe) be in 
alphabetical order.  LaTex will put square brackets about 
the number within the text of your paper.  For those of 
you new to the bibliography package, you may have to run 
the latex process twice to allow all references to be 
resolved. You will get a warning about a missing .aux 
file.  Just rerun latex and it will be ok.
\section{Summary and Conclusions}
This template will get you through the minimum article, 
i.e. no figures or equations.  To include those, please 
refer to your LaTeX manual and the IEEE publications 
guidelines.  Good Luck!
%this is how to do an unnumbered subsection
\subsection*{Acknowledgements}
This is how to do an unnumbered subsection, which comes 
out in 11 point bold font.  Here I thank my colleagues, 
especially Mike Gennert, who know more about Tex and 
Latex than I.
\begin{thebibliography}{9}
\bibitem{key:foo}
I. M. Author,
``Some Related Article I Wrote,''
{\em Some Fine Journal}, Vol. 17, pp. 1-100, 1987.
\bibitem{foo:baz}
A. N. Expert,
{\em A Book He Wrote,}
His Publisher, 1989.
\end{thebibliography}
\end{document}

