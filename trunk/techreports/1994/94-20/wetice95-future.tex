%%%%%%%%%%%%%%%%%%%%%%%%%%%%%% -*- Mode: Latex -*- %%%%%%%%%%%%%%%%%%%%%%%%%%%%
%% wetice95-future.tex -- 
%% Author          : Philip Johnson
%% Created On      : Mon Dec 26 10:44:09 1994
%% Last Modified By: Philip Johnson
%% Last Modified On: Thu May 11 09:26:58 1995
%% Status          : Unknown
%% RCS: $Id: wetice95-future.tex,v 1.2 1995/05/11 20:26:43 johnson Exp $
%%%%%%%%%%%%%%%%%%%%%%%%%%%%%%%%%%%%%%%%%%%%%%%%%%%%%%%%%%%%%%%%%%%%%%%%%%%%%%%
%%   Copyright (C) 1994 University of Hawaii
%%%%%%%%%%%%%%%%%%%%%%%%%%%%%%%%%%%%%%%%%%%%%%%%%%%%%%%%%%%%%%%%%%%%%%%%%%%%%%%
%% 

\section{Future Directions}

While we have learned a tremendous amount about strong collaboration
through our use of AEN thus far, we believe there is far more to learn.
For example, we would like to know in more detail about the styles
of collaborative authoring and learning enabled by AEN.  This detailed 
knowledge could not be provided by our initial use of AEN, since the
system was too volatile and appropriate instrumentation was not yet in
place. 

We have pursued this line of inquiry during Spring semester, 1995, when
AEN was used in a Software Engineering class at the Department of
Information and Computer Sciences at the University of Hawaii.  This second
study is intended to reveal the nature of strong collaboration in more
depth, and provide further insight into its computational support.

\section{Acknowledgements}

We wish to thank the following current and prior members of the
Collaborative Software Development Laboratory who provided assistance with
the design and implementation of AEN during this study: Rosemary Andrada,
Robert Brewer, Sang-Woo Han, Jeremy Harrison, John Johnson, Danu Tjahjono,
Dadong Wan, and Tae Ho Yum. 
