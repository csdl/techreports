%%%%%%%%%%%%%%%%%%%%%%%%%%%%%% -*- Mode: Latex -*- %%%%%%%%%%%%%%%%%%%%%%%%%%%%
%% jmis-usenet.tex -- 
%% RCS:            : $Id: jmis-usenet.tex,v 1.3 94/02/18 11:10:38 johnson Exp $
%% Author          : Robert Brewer
%% Created On      : Thu Feb  3 10:21:04 1994
%% Last Modified By: Philip Johnson
%% Last Modified On: Fri Feb 18 11:10:18 1994
%% Status          : Unknown
%%%%%%%%%%%%%%%%%%%%%%%%%%%%%%%%%%%%%%%%%%%%%%%%%%%%%%%%%%%%%%%%%%%%%%%%%%%%%%%
%%   Copyright (C) 1994 University of Hawaii
%%%%%%%%%%%%%%%%%%%%%%%%%%%%%%%%%%%%%%%%%%%%%%%%%%%%%%%%%%%%%%%%%%%%%%%%%%%%%%%
%% 
%% History
%% 3-Feb-1994		Robert Brewer	
%%    Created

\section{Usenet as an information system}
\label{sec:usenet}

%%%   (This section overviews Usenet (very briefly in CSCW, at length in JMIS).
%%%   It can talk about the various uses of Usenet, from simple
%%%   request/response, to long discussions of topics, to dissemination of
%%%   information (Tianniman (sic) Square, Collapse of USSR, Cold Fusion, LA
%%%   Earthquake.) 

\subsection{Background on Usenet}

Usenet (standing for Users' Network) is a massive but loosely connected
network of computers that exchange `netnews' which can be thought of as a
kind of `public' email. Any user on a Usenet node can post an article to
Usenet by simply typing in some text and submitting it to a program on the
local computer. This local computer then forwards the article to a few
close-by Usenet nodes, who in turn forward it to other nodes. In this
manner news is propagated around the world, yet the original posting
machine need only send it to a few near-by machines.

Although Usenet started in 1979 with only a few nodes, its growth has been
incredible.  As of March 1993, an estimated 76,000 Usenet sites existed with a
total of over 2.4 million Usenet users \cite{reid-usenet-93}. In the two week
period from January 10, 1994 to January 24, 1994, users generated over a
gigabyte of data, consisting of approximately 673,000 separate articles
\cite{uunet-usenet-94}.

\subsection{Syntactic Structure of Usenet}

Usenet articles are categorized into thousands of `newsgroups' (almost 9000
newsgroups exist as of January 24, 1994 \cite{uunet-usenet-94}). Newsgroup
membership is the primary way to classify articles by subject area.  Newsgroups
are hierarchically named where `.' separates the levels of hierarchy. For
example, the newsgroup about Macintosh hardware is called
`comp.sys.mac.hardware'. The subject areas of Usenet newsgroups is diverse,
ranging from groups about software engineering (comp.software-eng) to groups
about dogs (rec.pets.dogs) to groups about abortion (talk.abortion).

All articles on Usenet are ASCII text, but not all articles posted to
Usenet are human-readable text; some are encoded versions of binary files:
applications, pictures, and sounds. Regardless, all articles contain a `subject line', which is intended to summarize the content of the
article, other header information (such as the e-mail address of the
posters, and the date of posting), and finally the `body' of the
posting containing the actual content.

The human-readable articles, as in all communications media, can be statements,
questions, comments, replies to questions, poems, or any other textual object.
After an article is posted, other users may choose to `followup' that article
with an article containing a reference to the original article. This process is
recursive, with followups often generating new followups. A set of articles
linked together in this way on a common topic is called a `newsthread' or
simply `thread'. Sophisticated newsreading software (such as trn or GNUS)
allows the user to navigate through Usenet newsgroups by following these
threads.

Followup articles often include quotes, or partial copies of text from the
original article.  This quoting is usually done in an automated fashion so
that readers can distinguish between quoted and original text. However,
this creates a significant amount of redundant text, especially in long
threads.  In March 1993, for example, quotes represented more than 9\% of
Usenet's volume.

\subsection{Semantic Structure of Usenet}

Usenet is used for many different purposes, but three of the most common
uses are question and answer, discussion, and dissemination. Many
newsgroups consist partially or primarily of articles that ask questions
about the topic, leading to followup articles by other users containing
answers. This question and answer format can lead to problems, such as new
users posting a question that has been posted and answered previously.  One
technique employed to reduce this problem is a set of Frequently Asked
Questions (FAQ). A list of FAQs with standard answers is posted to the
newsgroup on a regular basis to ward off such questions. 

Another use of Usenet is for discussion. These discussions are often quite long
and involved, and can last for weeks or months. Due to their length and the
number of users who participate, the topic of a thread can wander and evolve.
In many cases there can be several topics under discussion simultaneously in
the same thread. It would not be unusual for a thread discussing a software
package to evolve into a debate on software patent law or even into a debate on
the location of the best pizza parlor in Silicon Valley.  Since newsreaders
simply copy the subject line when creating a followup posting, the content of
an article in a discussion thread frequently evolves quite far from that
indicated by its subject line.

A final use of Usenet is for dissemination of timely information. For
example, Usenet was used to disseminate information during such world
events as the collapse of the Soviet Union, the initial reports on the
`discovery' of cold fusion, and the recent Los Angeles earthquake.  These
global events precipitate a deluge of articles on the subject, and usually
a new newsgroup for the subject is created immediately.

\subsection{Usenet vs. Conventional Databases}

Many explanations for the problems in effectively utilizing the wealth of
information generated by Usenet focus upon its immense volume and numbers of
users, its global constituency, or even the mixture of information postings of
transient interest (for example, an advertisement of an upcoming conference)
with postings of more permanent interest (for example, a comparison of two
programming paradigms.)  However, such features alone do not explain why
conventional database techniques cannot be applied to Usenet with equal
success.

To illustrate this, note that airline reservation systems rival Usenet in
volume of traffic, global constituency, number of users, and in mixing
transiently useful information (single reservations) with more permanently
useful information (airline routes and schedules).  However, users of airline
reservation systems do not share the problems faced by users of Usenet with
respect to information access.  Despite their immense size and complexity,
airline reservation system users can almost always efficiently find all useful
information in the system related to a particular topic.

%% Robert: This last sentence contains a syntactic joke: strictly speaking, 
%% it says that the *users* are immense and complex, not the systems.
%% My vote is to keep it in for grins, but I accede to your (perhaps better
%% developed) sense of propriety.

The essential difference between airline reservation systems and Usenet, (and,
indeed, between database systems in general and Usenet) is a fundamentally
different notion of {\em structure}.  The structure of an airline reservation
system consists of a fixed set of schemas that represent all of the information
contained in the system in a fine-grained, machine-processable form.  This
structure was decided upon before any information entered the system, and is
fixed for the life of the system.  The structure of airline reservation systems
was designed after close analysis of this domain to ensure that all important
forms of information access would be readily available.  This combination of
features enables airline reservation systems to scale up in size, complexity,
and number of users without creating information overload or other problems
manifested in Usenet.

In Usenet, however, both ``structure'' and ``domain'' have qualitatively
more dynamic, emergent, and coarse-grained meanings.  The structure of
Usenet in database terms is its syntactic structure as described above: a
set of newsgroups, each broken down into a set of articles, where each
article is broken down into a set of generic, domain-independent fields.
This structure is imposed solely to enable information {\em transport}
across networks, hosts, and newsreader systems.  It was not designed to
support information retrieval, filtering, or analysis.  In conventional
database design, supporting information retrieval, filtering, or analysis
is based upon careful domain analysis, followed by definition of
specialized schemas and operations for that domain.  However, the structure
of Usenet must support not only the current 9000 domains, but also the
thousands upon thousands of others to be created in the future. 

Therefore, applying conventional automated database information retrieval
mechanisms to dynamically structured databases in ill-defined domains is
bound to fail.  Systems such as Usenet simply cannot provide enough
structural and domain-level information for conventional IR mechanisms to
exploit. As a result, their performance will be of low quality: because
they will be unable to classify information correctly, they will fail to
filter irrelevent information and fail to retrieve all relevent
information.

If the automated techniques of conventional database systems for effective
information management cannot apply to Usenet, what techniques do apply?
The Usenet community has developed several ad-hoc solutions, as discussed
next.

\subsection{Current Usenet Information Management}

The users of Usenet are well aware of their information overload problem,
and a variety of information management mechanisms have evolved in
response. Some examples are FAQ files, subscription, kill files, and
quotation ratios.

As mentioned above, many groups address the problem of repetitious postings
through the creation and periodic posting of FAQ files.  FAQs probably do
not substantially reduce the total volume of Usenet (although they do
reduce the frequency of some of the most irritating postings for long-term
users). Rather, they provide a means for novice users to quickly acquire
some of the permanent, relatively slowly changing forms of information
discussed in the group.  FAQs never store information that becomes outdated
quickly, nor do they store important or useful information that is not of
general relevance.  Finally, FAQs are typically maintained by a single
person, and their quality, content, and currency is strictly dependent upon
the commitment of that person.

A second mechanism for information management is newsgroup subscription. Since
each newsgroup has a subject area, one can merely subscribe only to those
newsgroups that one finds interesting, and leave the uninteresting groups
unsubscribed. This technique works well only when the user has a small number
of interests which are only discussed within a small number of newsgroups with
a small daily volume.  Most users have interests that potentially span many
newsgroups, but subscribe only to those groups which most directly address
issues of interest.  As a simple example, a user interested in software
engineering might read the newsgroup comp.software-eng.  However, software
engineering issues crop up frequently in hundreds of other newsgroups,
including the comp.lang.* newsgroups, the news.software.* newsgroups, the
comp.soft-sys.* newsgroups, and so forth.  Subscription reduces the apparent
volume of Usenet by simply reducing the wealth of information available.

A third mechanism is ``kill files''. A kill file is a list of patterns that
are designed to match fields in the header of an article such as its
subject or author. Newsreaders do not present articles to users
whose headers match one of the patterns in the kill file.  In this way
subjects or authors that a user find uninteresting are removed from his or
her view. 

There are three significant problems with kill files:
\begin {itemizenoindent}
\item {\em Kill files make all-or-nothing decisions.} The patterns for kill
  files must be chosen very carefully, or the user risks killing articles
  that are relevant to their interests. For example, a hypothetical user
  might dislike IBM, and therefore create a kill file entry that kills all
  articles related to the subject of IBM.  However, this hypothetical user
  might also love Apple Computer. In this case, if an article is posted about
  a partnership between Apple and IBM the kill file would prevent the user
  from seeing it.
  
\item {\em Kill files only deal with uninteresting articles.} The second
  problem with kill files is that they can only exclude articles; they cannot
  bring articles to the users' attention. Kill files only provide a structure
  for users to list patterns that exclude everything that they do not like,
  when it is easier for users to think about patterns that they do like.
  
\item {\em Kill files are brittle.} Finally, kill files are
  difficult to create and maintain \cite{cacm-infoscope-92}.  It is difficult,
  for example, to know when to take a inappropriate pattern {\em off} the list,
  since the user will not see the interesting articles deleted by the kill
  file.

\end {itemizenoindent}

A final method currently used for information management is the quotation-ratio
restriction present in some newsreaders.  This mechanism is designed to
reduce the number of ``me-too'' comments by preventing articles from being
posted unless the number of new lines of commentary in the article is
greater than the number of quoted lines.  This mechanism has completely
backfired, since users quickly learn to simply ``pad'' their articles with
lines of junk characters in order to trick the newsreader---thus
exacerbating the very problem that the mechanism was intended to resolve.







%%% \foot{Data from Brian Reid's {\tt $<$reid@decwrl.dec.com$>$}
%%% postings in the news.lists newsgroup with message-ID {\tt
%%% $<$1ptj0d\$2h1@usenet.pa.dec.com$>$}.}
