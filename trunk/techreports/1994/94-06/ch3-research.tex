%%%%%%%%%%%%%%%%%%%%%%%%%%%%%% -*- Mode: Latex -*- %%%%%%%%%%%%%%%%%%%%%%%%%%%%
%% ch3-research.tex -- 
%% Author          : Philip Johnson
%% Created On      : Fri Jun 17 10:33:38 1994
%% Last Modified By: Philip Johnson
%% Last Modified On: Fri Jun 17 11:17:35 1994
%% Status          : Unknown
%% RCS: $Id$
%%%%%%%%%%%%%%%%%%%%%%%%%%%%%%%%%%%%%%%%%%%%%%%%%%%%%%%%%%%%%%%%%%%%%%%%%%%%%%%
%%   Copyright (C) 1994 University of Hawaii
%%%%%%%%%%%%%%%%%%%%%%%%%%%%%%%%%%%%%%%%%%%%%%%%%%%%%%%%%%%%%%%%%%%%%%%%%%%%%%%
%% 

\chapter{The CSDL Way: Research}

\silentfootnote{This chapter is a revised version of CSDL-RO-93-01.}

\section{Introduction}
\label{sec:motivation}

This chapter discusses how CSDL is applying software process modelling
ideas to improve the efficiency and quality of software engineering
research.  The long term goals of the CSDL research paradigm is to 
provide information about the following four questions:

\begin{enumerate}
\item {\em Is  software engineering research a ``special'' kind of
  activity, to which software process modelling techniques are inapplicable?}

\item {\em Can the activities of software engineering researchers be 
usefully measured in a low-overhead, fine-grained fashion?}

\item {\em Can these measurements be used to improve the efficiency and 
quality of research activities?}

\item {\em Can these process modelling and measurement techniques enable
  some aspects of software engineering research to become ``repeatable''?}
\end{enumerate}

These issues are important to the field, since software research has been
criticized as ad-hoc, unscientific, and wasteful of human and monetary
resources \cite{Berry92,Tichy93,Cohen88}. It is important on a human basis,
as well: a concrete, empirically-based method for improving the quality of
software engineering research can substantially reduce the stress and risk
associated with graduate, undergraduate, and faculty work.

The central problem, stated in Question 1 above, bears immediate
amplification. The CSDL research paradigm attempts to determine to what
extent the research process can be formalized, managed, and eventually
predicted.  It attempts this through a software process modelling method
similar the SEI Capability Maturity Model \cite{Paulk93}, where empirical
measures are used to incrementally refine and standardize an initial set of
constraints on the research process. These constraints are concerned with
defining intermediate outcomes and measuring related activities.  They are
very different from other software process modelling methods, such as
``process programs'' \cite{Osterweil87}.

The CSDL research paradigm is intended to be compatible with the ``prime
directive'' for academic research:

\begin{quotation}
  {\em Academic software engineering researchers are extremely time and
  resource constrained. New measurement or analysis related overhead on
  their work activities is difficult to impossible to impose.  If any
  such overhead is imposed, there must a direct pay-off, in proportion to
  the size of the overhead.  This pay-off must occur in the near term and
  be visible.}
\end{quotation}

The remainder of this chapter presents the CSRS research paradigm, and is
organized as follows.  The next section presents the initial process model.
The following section presents the experimental design.  The final section
discusses the data to be collected, the range of possible outcomes, and
their significance.  The final section discusses the activities to be
carried out in more detail.  The organization and contents of this chapter
is designed to not only describe the CSDL research paradigm, but also to
serve as an example of a CSDL research definition.

\section{The CSDL Research Process Model}

The CSDL research process model is composed of three distinct phases:
research definition, research execution, and research publication.  In
general, research begins with the creation of a research definition
document, which includes a statement of the research objective and the
research plan.  That document bootstraps the research execution phase by
providing justification for what will be done and an initial strategy for
research procedures.  Thereafter, research execution and research
definition is interleaved.  At some future point, research publication
begins to be interleaved with research definition and research execution.
The next sections discuss each of the three phases in more detail.

%%% 
%%% One general goal of the CSDL research paradigm is to instrument the
%%% activities of researchers in such a way as to provide new data about the
%%% nature of interleaving between these phases and the types of durations
%%% exhibited for these phases in different projects.  Such data is intended to
%%% be used to improve the quality of the process model and thus the quality of
%%% the research. 


\subsection{Research definition}

\subsubsection*{Process}

During this phase, the researcher defines and refines a precise statement
of the objectives and activities for a research project.  The research
objective is a written document that is generally developed as a new CSDL
technical report. Thus, an ICS tr number must be requested from {\tt
tr@uhics} and a new subdirectory in {\tt csdl/techreports} must be created.
This document is placed under RCS control.

The research definition is circulated, reviewed, and improved
until it contains two basic features.  

First, it must describe an important problem, along with a well-defined
experimental method that will uncover new, publishable information about
the problem.  Figure \ref{fig:objective-questions} contains a set of
questions about a research problem that are typically answered by a high
quality research definition.


\begin{figure}
\hrule
\small
\begin{enumerate}
\item What is the purpose of this research?  What is the central problem being
  addressed?  Why is this problem important?
\item What new insight about the problem will be provided through this
  research? 
\item What method will be used to provide new insight into the problem?
\item What data will be collected?  What are example instances of this
  data?  How will the data be analyzed?
\item What are the range of outcomes from this research?  What is the
  significance of each outcome?
\end{enumerate}
\hrule
\caption {Questions answered by a high quality research objective statement.}
\label{fig:objective-questions}
\end{figure}


Second, the research definition must describe the set of activities that
will be required to fulfill the research objective.
Each research activity has four  components:
\begin{itemize}
\item A description of the activity.
\item The definition (or naming) of an activity-ID function that will be
used to track the effort associated with the activity.
\item The apparent risk of failure associated with the activity.
\item The projected duration of the activity.
\end{itemize}

As an approximate guideline, research definitions typically begin with
about a half dozen activities (including the publication writing
activities).  The intent is to provide a reasonable decomposition of
activities without introducing ``micro-management'' and untoward overhead
in task definition.

Characteristics of high quality activity descriptions are those with
well-understood risks, easily measured activities, and durations that
appear consistent with the overall size and complexity of the task.

The research should be designed to lead to publishable data within four
to eight months.  Details on the experimental methodology must be provided,
including: the technique used to collect data, the type of data collected,
the analyses to be performed, and the range of possible conclusions.  The
research objective statement should be able to discuss the range and
significance of different outcomes.

Each time the researcher circulates a version of the research definition
for review, he invokes the Emacs command {\tt rcs-freeze} to create a
configuration to distinguish this checked-in version as one circulated for
external review.

When the document satisfies criteria for a high quality research definition
(as judged by the CSDL Director), then the researcher sets the RCS state of
the latest version is set to Approved.  This is accomplished through the
Emacs command {\tt m-0 m-x rcs-status} which prompts for the file name and
the switches to be passed to rcs.  The value supplied to this second prompt
should be {\tt -sApproved}.


\subsubsection*{Measurements}

\small \begin{figure}
\hrule

\medskip
\begin{verbatim}
Activitylog information for user: johnson

Activitylog Report for: Jul-30

Activity Classes (5% or greater effort)                         N         %
  Mail                                                         92      40.0
  Research Project 93-01                                       86      38.0
  Usenet                                                       17       7.0

Timestamp info:
  First timestamp:  Jul 30, 1993 08:16:45
  Last timestamp:   Jul 30, 1993 15:17:52
  Total recorded:   226
  Total displayed:  195 (86.0%)
  Timestamps per hour: 32.0
\end{verbatim}
\hrule
\label{fig:activitylog}
\caption{An example activitylog printout for one day's activities.}
\end{figure}\normalsize

\noindent {\em Effort measures.} During this phase, the activitylog system
will be used to automatically measure the effort associated with the
creation of a research objective statement.  Effort will be measured as an
absolute value (i.e.  the number N of samples taken within the research
objective's activityclass).  It will also be measured relative to the total
number of timestamps obtained, to measure the proportion of the
researcher's time spent on research objective development. Figure
\ref{fig:activitylog} illustrates the kind of data provided by activitylog.

\noindent {\em Duration measures.} A second set of measurements can be
automatically provided by analysis of RCS data.  Since each RCS
configuration corresponds to a released version of the objective statement
for review, it is possible to automatically determine: (1) the total number
of reviewed versions, (2) the time and date of release of each reviewed
version, (3) the elapsed time between each reviewed version, (4) whether or not
the objective was approved; (5) How many review cycles were required until
approval; (6) whether or not reviews following approval occurred, along
with their elapsed time, frequency of occurrence, effort, and so forth.

\noindent {\em Release-effort measures.} A third set of measurements 
that correlate the effort expended upon objective development with
a particular release can be automatically obtained by combining together the 
first two measurements.


\subsection{Research execution}

At all points in time, researchers will be carrying out research activities,
both those stated explicitly in the research activities description document, as
well as other related and unrelated activities.

Some types of research activities and the effort expended upon them will be
tracked automatically via Activitylog, although others (such as library
research, non-Unix/Emacs work, and so forth) will not show up in
Activitylog reports.

Research activity execution is formally monitored once per week during a
group meeting.  At this meeting, the following data collection occurs:

\begin{itemize}
\item All researchers submit a single hard-copy version of their weekly
  Activitylog report (from the preceding Monday to the end of Sunday) to
  the CSDL Director.
  
\item Each Activitylog report is reviewed immediately and assessed in the
  following way:
  \begin{itemize}
  \item Does the report appear to contain major inaccuracies?
  \item Are there significant activities that the researcher performed that
  does not appear in the  report? If so, these activities and
  their approximate durations are recorded by hand.
  \item Does the report clearly state relationships between Activitylog
  activity classes and Research objective activities?  If not, then these
  relationships are recorded by hand.
  \item Are there any proposals for changes to the process model or
  Activitylog collection mechanism that will improve the quality and/or
  efficiency of the CSDL research process?  If so, then these suggestions
  are recorded by hand.
  \end{itemize}
\end{itemize}

Also to facilitate post-experiment data analysis, researchers should agree
not to destroy their personal metrics files, since new analysis methods
requiring the original raw data may be developed at some future point.
However, no access to any personal metrics files will ever be made without
the prior consent of the researchers.


\subsection{Research publication}

The third phase of the CSDL research process is publication of the results
of the experiment and the analysis of the data.  The publication is
developed as a separate CSDL technical report, which facilitates
activitylog capture and conforms to existing CSDL guidelines.  Otherwise,
the process corresponds to that described for the research definition
phase, with RCS control, explicit configuration freezing to mark reviewed
versions, and reviews until approved.

In this phase, several additional measures are collected: (1) the place to
which the publication is submitted, (2) the type of submission (journal,
conference, workshop), (3) whether it was accepted or rejected, and (4) the
published acceptance rate.

\section{An Experiment}

In order to provide data to address the questions raised in Section
\ref{sec:motivation}, the following experiment will be performed. 

\subsection{Duration}

In July, 1993, the initial process model for research will be proposed for
implementation in the Collaborative Software Development Laboratory at the
University of Hawaii.  Members of CSDL will be asked whether or not they
voluntarily wish to participate in this experiment.  Only those members
volunteering will be involved in this experiment.

The experiment will start in August, 1993, and continue until February
1994.  However, the experiment will be terminated prematurely if it is
perceived to introduce excessive new overhead, be overly invasive, or have
some other fatally undesirable quality.

The proposed duration is intended to be long enough for the experiment to
collect data from at least one complete objective-to-publication cycle.
Such a publication submission is projected to be available six months
following the start of the experiment, if any research under measurement
succeeds in running through one complete cycle in four months.

\subsection{Method}

The method for this experiment is to implement the process model described
above within CSDL, and assess the data collected both during and at the end of the
experimental period.  More specifically:

\begin{itemize}
\item Activitylog will be used by all participating members.
\item A portion of each Monday CSDL meeting will be allocated to data
collection as specified above.
\item Participating members will be encouraged to develop research
  objective and activity documents, and in general conduct their research
  in a manner consistent with the proposed process model.
\end{itemize}


\subsection{Data analysis}

To summarize, the data to be collected consists of the annotated
Activitylog reports collected each Monday, the RCS data, and the actual
products of the research activities.  Let us now revisit the four central
questions of this research objective, and discuss how this data can be
used to assess them.

\begin{enumerate}

\item {\em Is  software engineering research a ``special'' kind of
  activity, to which software process modelling techniques are inapplicable?}
  
  It might be argued, based upon observation of extant software engineering
  research, that software engineering research is simply different and
  cannot be straight-jacketed into a process-driven approach.
  
  Upon conclusion of the experiment, the data can be analyzed to
  determine whether or not a formal software process was {\em adopted} by
  the group.  This adoption will be assessed in two ways: (1) existance
  of a formal document (such as a revision of this research objective
  statement) that specifies a process for research within CSDL, and (2)
  compliance with this research process, as evidenced through the
  Activitylog data and annotations.
  
  If a formal software process has been adopted by CSDL, then this
  provides evidence that there exists at least one style of software
  engineering research that is amenable to instrumented software process
  modelling techniques.  Additional research must be performed to assess
  the generality and applicability of the process model.
  
  If, upon conclusion of the experiment, there is no formal document
  specifying a process, or no evidence from activitylog data that the
  formally specified process is being followed, then the experiment may
  suggest that either software engineering research does have special
  aspects, or the experimental method was flawed in such a manner that
  this conclusion cannot be drawn.  Additional analysis of the data will
  be performed to attempt to discover why the process modelling approach
  failed.
  
  
\item {\em Can the activities of software engineering researchers be
  usefully measured in a low-overhead, fine-grained fashion?}
  
  This question will be answered by comparison of the actual Activitylog
  data to the annotations provided during the weekly meeting.
  
  If the activities can be usefully measured in a fine-grained fashion,
  then the data produced by Activitylog should (over time) become all the
  information needed by researchers to monitor their process and make
  improvements.

  If the activities cannot be measured in a fine-grained fashion, then the
  annotations provided by researchers will continue to play an important
  (if not the most important) role in assessing their process and progress;
  the Activitylog data will not be viewed as useful.  This could happen,
  for example, if major and important aspects of research activities do not
  involve Emacs. 
 

\item {\em Can these measurements be used to improve the efficiency and 
quality of research activities?}

  Collecting data is one thing; using it effectively is another.  
  
  If these measures are useful for quality assessment and improvement,
  then by the conclusion of the study, additional constraints on the
  software research process will have been empirically derived from the
  experience of analyzing Activitylog data each week.  An example rule
  might be: ``devote at least 50\% of your Activitylog time to objective
  definition until it has been approved'', or ``never devote less than
  50\% of your time to programming.''  These constraints might be
  personal for a particular researcher, or adopted by the entire group.
  
  An alternative outcome is to find that the measurements are interesting
  in an abstract sense, but not of any real use in modifying behavior
  toward increased quality and productivity.


\item {\em Can these process modelling and measurement techniques enable
  some aspects of software engineering research to become ``repeatable''
  (in the Capability Maturity Model sense)?}

  This question addresses whether or not one can use prior measurements to
  predict the effort or activities involved with future activities.  If so,
  then by the end of the experiment, it should be possible to analyze the
  data and derive heuristic rules which provide guidelines for the time and
  effort required to perform research tasks. 
 
\end{enumerate}

It is important to note that this experiment is not designed to yield
statistically significant conclusions concerning the research objective
questions discussed above.  Instead, it is concerned with generating
descriptive data about the implementation of the process model and
instrumentation in a functioning research group, and using this
empirical evidence to discuss the research objective questions.

Post-experiment analysis of weekly Activitylog can allow investigation into
the patterns of effort allocated to the research phases.  Among the many
possible patterns of behavior are the following: (1) a ``waterfall''
sequence, where all activity focusses on objectives, followed by a focus on
activity description, and so forth; (2) a ``concurrent'' sequence, where
activity is relatively evenly distributed across all phases at all times;
(3) an ``iterative'' sequence, where all phases cyclically follow each
other many times before the conclusion of the objective.

\section{Future Directions}
\label{sec:future}

One possible outcome of this research is an exportable process model and
instrumentation support for use by other software engineering researchers
in similar hardware/software platforms.  The constraints imposed by the
process model and instrumentation support appear to be rather small: the
use of Unix and the use of Emacs being primary among them.

Under this outcome, a second research objective can be generated involving
the external distribution of this process model and instrumentation through
the Internet after conclusion of its use in one complete research cycle and
the publication submission.  The goal of this distribution will be to
gather additional data on the process model, validate it externally, and
increase the number of research cycles for which process data has been
collected.



