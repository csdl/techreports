%%%%%%%%%%%%%%%%%%%%%%%%%%%%%% -*- Mode: Latex -*- %%%%%%%%%%%%%%%%%%%%%%%%%%%%
%% objective.tex -- 
%% Author          : Carleton Moore
%% Created On      : Mon Sep 19 11:14:03 1994
%% Last Modified By: Carleton Moore
%% Last Modified On: Tue Dec 27 16:48:11 1994
%% Status          : Unknown
%% RCS: $Id: objective.tex,v 1.1 1994/09/20 01:40:15 cmoore Exp cmoore $
%%%%%%%%%%%%%%%%%%%%%%%%%%%%%%%%%%%%%%%%%%%%%%%%%%%%%%%%%%%%%%%%%%%%%%%%%%%%%%%
%%   Copyright (C) 1994 University of Hawaii
%%%%%%%%%%%%%%%%%%%%%%%%%%%%%%%%%%%%%%%%%%%%%%%%%%%%%%%%%%%%%%%%%%%%%%%%%%%%%%%
%% 

\documentstyle [11pt,/group/csdl/tex/definemargins]{article}      % Unix
% Psfig/TeX 
\def\PsfigVersion{1.9}
% dvips version
%
% All psfig/tex software, documentation, and related files
% in this distribution of psfig/tex are 
% Copyright 1987, 1988, 1991 Trevor J. Darrell
%
% Permission is granted for use and non-profit distribution of psfig/tex 
% providing that this notice is clearly maintained. The right to
% distribute any portion of psfig/tex for profit or as part of any commercial
% product is specifically reserved for the author(s) of that portion.
%
% *** Feel free to make local modifications of psfig as you wish,
% *** but DO NOT post any changed or modified versions of ``psfig''
% *** directly to the net. Send them to me and I'll try to incorporate
% *** them into future versions. If you want to take the psfig code 
% *** and make a new program (subject to the copyright above), distribute it, 
% *** (and maintain it) that's fine, just don't call it psfig.
%
% Bugs and improvements to trevor@media.mit.edu.
%
% Thanks to Greg Hager (GDH) and Ned Batchelder for their contributions
% to the original version of this project.
%
% Modified by J. Daniel Smith on 9 October 1990 to accept the
% %%BoundingBox: comment with or without a space after the colon.  Stole
% file reading code from Tom Rokicki's EPSF.TEX file (see below).
%
% More modifications by J. Daniel Smith on 29 March 1991 to allow the
% the included PostScript figure to be rotated.  The amount of
% rotation is specified by the "angle=" parameter of the \psfig command.
%
% Modified by Robert Russell on June 25, 1991 to allow users to specify
% .ps filenames which don't yet exist, provided they explicitly provide
% boundingbox information via the \psfig command. Note: This will only work
% if the "file=" parameter follows all four "bb???=" parameters in the
% command. This is due to the order in which psfig interprets these params.
%
%  3 Jul 1991	JDS	check if file already read in once
%  4 Sep 1991	JDS	fixed incorrect computation of rotated
%			bounding box
% 25 Sep 1991	GVR	expanded synopsis of \psfig
% 14 Oct 1991	JDS	\fbox code from LaTeX so \psdraft works with TeX
%			changed \typeout to \ps@typeout
% 17 Oct 1991	JDS	added \psscalefirst and \psrotatefirst
%

% From: gvr@cs.brown.edu (George V. Reilly)
%
% \psdraft	draws an outline box, but doesn't include the figure
%		in the DVI file.  Useful for previewing.
%
% \psfull	includes the figure in the DVI file (default).
%
% \psscalefirst width= or height= specifies the size of the figure
% 		before rotation.
% \psrotatefirst (default) width= or height= specifies the size of the
% 		 figure after rotation.  Asymetric figures will
% 		 appear to shrink.
%
% \psfigurepath#1	sets the path to search for the figure
%
% \psfig
% usage: \psfig{file=, figure=, height=, width=,
%			bbllx=, bblly=, bburx=, bbury=,
%			rheight=, rwidth=, clip=, angle=, silent=}
%
%	"file" is the filename.  If no path name is specified and the
%		file is not found in the current directory,
%		it will be looked for in directory \psfigurepath.
%	"figure" is a synonym for "file".
%	By default, the width and height of the figure are taken from
%		the BoundingBox of the figure.
%	If "width" is specified, the figure is scaled so that it has
%		the specified width.  Its height changes proportionately.
%	If "height" is specified, the figure is scaled so that it has
%		the specified height.  Its width changes proportionately.
%	If both "width" and "height" are specified, the figure is scaled
%		anamorphically.
%	"bbllx", "bblly", "bburx", and "bbury" control the PostScript
%		BoundingBox.  If these four values are specified
%               *before* the "file" option, the PSFIG will not try to
%               open the PostScript file.
%	"rheight" and "rwidth" are the reserved height and width
%		of the figure, i.e., how big TeX actually thinks
%		the figure is.  They default to "width" and "height".
%	The "clip" option ensures that no portion of the figure will
%		appear outside its BoundingBox.  "clip=" is a switch and
%		takes no value, but the `=' must be present.
%	The "angle" option specifies the angle of rotation (degrees, ccw).
%	The "silent" option makes \psfig work silently.
%

% check to see if macros already loaded in (maybe some other file says
% "\input psfig") ...
\ifx\undefined\psfig\else\endinput\fi

%
% from a suggestion by eijkhout@csrd.uiuc.edu to allow
% loading as a style file. Changed to avoid problems
% with amstex per suggestion by jbence@math.ucla.edu

\let\LaTeXAtSign=\@
\let\@=\relax
\edef\psfigRestoreAt{\catcode`\@=\number\catcode`@\relax}
%\edef\psfigRestoreAt{\catcode`@=\number\catcode`@\relax}
\catcode`\@=11\relax
\newwrite\@unused
\def\ps@typeout#1{{\let\protect\string\immediate\write\@unused{#1}}}
\ps@typeout{psfig/tex \PsfigVersion}

%% Here's how you define your figure path.  Should be set up with null
%% default and a user useable definition.

\def\figurepath{./}
\def\psfigurepath#1{\edef\figurepath{#1}}

%
% @psdo control structure -- similar to Latex @for.
% I redefined these with different names so that psfig can
% be used with TeX as well as LaTeX, and so that it will not 
% be vunerable to future changes in LaTeX's internal
% control structure,
%
\def\@nnil{\@nil}
\def\@empty{}
\def\@psdonoop#1\@@#2#3{}
\def\@psdo#1:=#2\do#3{\edef\@psdotmp{#2}\ifx\@psdotmp\@empty \else
    \expandafter\@psdoloop#2,\@nil,\@nil\@@#1{#3}\fi}
\def\@psdoloop#1,#2,#3\@@#4#5{\def#4{#1}\ifx #4\@nnil \else
       #5\def#4{#2}\ifx #4\@nnil \else#5\@ipsdoloop #3\@@#4{#5}\fi\fi}
\def\@ipsdoloop#1,#2\@@#3#4{\def#3{#1}\ifx #3\@nnil 
       \let\@nextwhile=\@psdonoop \else
      #4\relax\let\@nextwhile=\@ipsdoloop\fi\@nextwhile#2\@@#3{#4}}
\def\@tpsdo#1:=#2\do#3{\xdef\@psdotmp{#2}\ifx\@psdotmp\@empty \else
    \@tpsdoloop#2\@nil\@nil\@@#1{#3}\fi}
\def\@tpsdoloop#1#2\@@#3#4{\def#3{#1}\ifx #3\@nnil 
       \let\@nextwhile=\@psdonoop \else
      #4\relax\let\@nextwhile=\@tpsdoloop\fi\@nextwhile#2\@@#3{#4}}
% 
% \fbox is defined in latex.tex; so if \fbox is undefined, assume that
% we are not in LaTeX.
% Perhaps this could be done better???
\ifx\undefined\fbox
% \fbox code from modified slightly from LaTeX
\newdimen\fboxrule
\newdimen\fboxsep
\newdimen\ps@tempdima
\newbox\ps@tempboxa
\fboxsep = 3pt
\fboxrule = .4pt
\long\def\fbox#1{\leavevmode\setbox\ps@tempboxa\hbox{#1}\ps@tempdima\fboxrule
    \advance\ps@tempdima \fboxsep \advance\ps@tempdima \dp\ps@tempboxa
   \hbox{\lower \ps@tempdima\hbox
  {\vbox{\hrule height \fboxrule
          \hbox{\vrule width \fboxrule \hskip\fboxsep
          \vbox{\vskip\fboxsep \box\ps@tempboxa\vskip\fboxsep}\hskip 
                 \fboxsep\vrule width \fboxrule}
                 \hrule height \fboxrule}}}}
\fi
%
%%%%%%%%%%%%%%%%%%%%%%%%%%%%%%%%%%%%%%%%%%%%%%%%%%%%%%%%%%%%%%%%%%%
% file reading stuff from epsf.tex
%   EPSF.TEX macro file:
%   Written by Tomas Rokicki of Radical Eye Software, 29 Mar 1989.
%   Revised by Don Knuth, 3 Jan 1990.
%   Revised by Tomas Rokicki to accept bounding boxes with no
%      space after the colon, 18 Jul 1990.
%   Portions modified/removed for use in PSFIG package by
%      J. Daniel Smith, 9 October 1990.
%
\newread\ps@stream
\newif\ifnot@eof       % continue looking for the bounding box?
\newif\if@noisy        % report what you're making?
\newif\if@atend        % %%BoundingBox: has (at end) specification
\newif\if@psfile       % does this look like a PostScript file?
%
% PostScript files should start with `%!'
%
{\catcode`\%=12\global\gdef\epsf@start{%!}}
\def\epsf@PS{PS}
%
\def\epsf@getbb#1{%
%
%   The first thing we need to do is to open the
%   PostScript file, if possible.
%
\openin\ps@stream=#1
\ifeof\ps@stream\ps@typeout{Error, File #1 not found}\else
%
%   Okay, we got it. Now we'll scan lines until we find one that doesn't
%   start with %. We're looking for the bounding box comment.
%
   {\not@eoftrue \chardef\other=12
    \def\do##1{\catcode`##1=\other}\dospecials \catcode`\ =10
    \loop
       \if@psfile
	  \read\ps@stream to \epsf@fileline
       \else{
	  \obeyspaces
          \read\ps@stream to \epsf@tmp\global\let\epsf@fileline\epsf@tmp}
       \fi
       \ifeof\ps@stream\not@eoffalse\else
%
%   Check the first line for `%!'.  Issue a warning message if its not
%   there, since the file might not be a PostScript file.
%
       \if@psfile\else
       \expandafter\epsf@test\epsf@fileline:. \\%
       \fi
%
%   We check to see if the first character is a % sign;
%   if so, we look further and stop only if the line begins with
%   `%%BoundingBox:' and the `(atend)' specification was not found.
%   That is, the only way to stop is when the end of file is reached,
%   or a `%%BoundingBox: llx lly urx ury' line is found.
%
          \expandafter\epsf@aux\epsf@fileline:. \\%
       \fi
   \ifnot@eof\repeat
   }\closein\ps@stream\fi}%
%
% This tests if the file we are reading looks like a PostScript file.
%
\long\def\epsf@test#1#2#3:#4\\{\def\epsf@testit{#1#2}
			\ifx\epsf@testit\epsf@start\else
\ps@typeout{Warning! File does not start with `\epsf@start'.  It may not be a PostScript file.}
			\fi
			\@psfiletrue} % don't test after 1st line
%
%   We still need to define the tricky \epsf@aux macro. This requires
%   a couple of magic constants for comparison purposes.
%
{\catcode`\%=12\global\let\epsf@percent=%\global\def\epsf@bblit{%BoundingBox}}
%
%
%   So we're ready to check for `%BoundingBox:' and to grab the
%   values if they are found.  We continue searching if `(at end)'
%   was found after the `%BoundingBox:'.
%
\long\def\epsf@aux#1#2:#3\\{\ifx#1\epsf@percent
   \def\epsf@testit{#2}\ifx\epsf@testit\epsf@bblit
	\@atendfalse
        \epsf@atend #3 . \\%
	\if@atend	
	   \if@verbose{
		\ps@typeout{psfig: found `(atend)'; continuing search}
	   }\fi
        \else
        \epsf@grab #3 . . . \\%
        \not@eoffalse
        \global\no@bbfalse
        \fi
   \fi\fi}%
%
%   Here we grab the values and stuff them in the appropriate definitions.
%
\def\epsf@grab #1 #2 #3 #4 #5\\{%
   \global\def\epsf@llx{#1}\ifx\epsf@llx\empty
      \epsf@grab #2 #3 #4 #5 .\\\else
   \global\def\epsf@lly{#2}%
   \global\def\epsf@urx{#3}\global\def\epsf@ury{#4}\fi}%
%
% Determine if the stuff following the %%BoundingBox is `(atend)'
% J. Daniel Smith.  Copied from \epsf@grab above.
%
\def\epsf@atendlit{(atend)} 
\def\epsf@atend #1 #2 #3\\{%
   \def\epsf@tmp{#1}\ifx\epsf@tmp\empty
      \epsf@atend #2 #3 .\\\else
   \ifx\epsf@tmp\epsf@atendlit\@atendtrue\fi\fi}


% End of file reading stuff from epsf.tex
%%%%%%%%%%%%%%%%%%%%%%%%%%%%%%%%%%%%%%%%%%%%%%%%%%%%%%%%%%%%%%%%%%%

%%%%%%%%%%%%%%%%%%%%%%%%%%%%%%%%%%%%%%%%%%%%%%%%%%%%%%%%%%%%%%%%%%%
% trigonometry stuff from "trig.tex"
\chardef\psletter = 11 % won't conflict with \begin{letter} now...
\chardef\other = 12

\newif \ifdebug %%% turn me on to see TeX hard at work ...
\newif\ifc@mpute %%% don't need to compute some values
\c@mputetrue % but assume that we do

\let\then = \relax
\def\r@dian{pt }
\let\r@dians = \r@dian
\let\dimensionless@nit = \r@dian
\let\dimensionless@nits = \dimensionless@nit
\def\internal@nit{sp }
\let\internal@nits = \internal@nit
\newif\ifstillc@nverging
\def \Mess@ge #1{\ifdebug \then \message {#1} \fi}

{ %%% Things that need abnormal catcodes %%%
	\catcode `\@ = \psletter
	\gdef \nodimen {\expandafter \n@dimen \the \dimen}
	\gdef \term #1 #2 #3%
	       {\edef \t@ {\the #1}%%% freeze parameter 1 (count, by value)
		\edef \t@@ {\expandafter \n@dimen \the #2\r@dian}%
				   %%% freeze parameter 2 (dimen, by value)
		\t@rm {\t@} {\t@@} {#3}%
	       }
	\gdef \t@rm #1 #2 #3%
	       {{%
		\count 0 = 0
		\dimen 0 = 1 \dimensionless@nit
		\dimen 2 = #2\relax
		\Mess@ge {Calculating term #1 of \nodimen 2}%
		\loop
		\ifnum	\count 0 < #1
		\then	\advance \count 0 by 1
			\Mess@ge {Iteration \the \count 0 \space}%
			\Multiply \dimen 0 by {\dimen 2}%
			\Mess@ge {After multiplication, term = \nodimen 0}%
			\Divide \dimen 0 by {\count 0}%
			\Mess@ge {After division, term = \nodimen 0}%
		\repeat
		\Mess@ge {Final value for term #1 of 
				\nodimen 2 \space is \nodimen 0}%
		\xdef \Term {#3 = \nodimen 0 \r@dians}%
		\aftergroup \Term
	       }}
	\catcode `\p = \other
	\catcode `\t = \other
	\gdef \n@dimen #1pt{#1} %%% throw away the ``pt''
}

\def \Divide #1by #2{\divide #1 by #2} %%% just a synonym

\def \Multiply #1by #2%%% allows division of a dimen by a dimen
       {{%%% should really freeze parameter 2 (dimen, passed by value)
	\count 0 = #1\relax
	\count 2 = #2\relax
	\count 4 = 65536
	\Mess@ge {Before scaling, count 0 = \the \count 0 \space and
			count 2 = \the \count 2}%
	\ifnum	\count 0 > 32767 %%% do our best to avoid overflow
	\then	\divide \count 0 by 4
		\divide \count 4 by 4
	\else	\ifnum	\count 0 < -32767
		\then	\divide \count 0 by 4
			\divide \count 4 by 4
		\else
		\fi
	\fi
	\ifnum	\count 2 > 32767 %%% while retaining reasonable accuracy
	\then	\divide \count 2 by 4
		\divide \count 4 by 4
	\else	\ifnum	\count 2 < -32767
		\then	\divide \count 2 by 4
			\divide \count 4 by 4
		\else
		\fi
	\fi
	\multiply \count 0 by \count 2
	\divide \count 0 by \count 4
	\xdef \product {#1 = \the \count 0 \internal@nits}%
	\aftergroup \product
       }}

\def\r@duce{\ifdim\dimen0 > 90\r@dian \then   % sin(x+90) = sin(180-x)
		\multiply\dimen0 by -1
		\advance\dimen0 by 180\r@dian
		\r@duce
	    \else \ifdim\dimen0 < -90\r@dian \then  % sin(-x) = sin(360+x)
		\advance\dimen0 by 360\r@dian
		\r@duce
		\fi
	    \fi}

\def\Sine#1%
       {{%
	\dimen 0 = #1 \r@dian
	\r@duce
	\ifdim\dimen0 = -90\r@dian \then
	   \dimen4 = -1\r@dian
	   \c@mputefalse
	\fi
	\ifdim\dimen0 = 90\r@dian \then
	   \dimen4 = 1\r@dian
	   \c@mputefalse
	\fi
	\ifdim\dimen0 = 0\r@dian \then
	   \dimen4 = 0\r@dian
	   \c@mputefalse
	\fi
%
	\ifc@mpute \then
        	% convert degrees to radians
		\divide\dimen0 by 180
		\dimen0=3.141592654\dimen0
%
		\dimen 2 = 3.1415926535897963\r@dian %%% a well-known constant
		\divide\dimen 2 by 2 %%% we only deal with -pi/2 : pi/2
		\Mess@ge {Sin: calculating Sin of \nodimen 0}%
		\count 0 = 1 %%% see power-series expansion for sine
		\dimen 2 = 1 \r@dian %%% ditto
		\dimen 4 = 0 \r@dian %%% ditto
		\loop
			\ifnum	\dimen 2 = 0 %%% then we've done
			\then	\stillc@nvergingfalse 
			\else	\stillc@nvergingtrue
			\fi
			\ifstillc@nverging %%% then calculate next term
			\then	\term {\count 0} {\dimen 0} {\dimen 2}%
				\advance \count 0 by 2
				\count 2 = \count 0
				\divide \count 2 by 2
				\ifodd	\count 2 %%% signs alternate
				\then	\advance \dimen 4 by \dimen 2
				\else	\advance \dimen 4 by -\dimen 2
				\fi
		\repeat
	\fi		
			\xdef \sine {\nodimen 4}%
       }}

% Now the Cosine can be calculated easily by calling \Sine
\def\Cosine#1{\ifx\sine\UnDefined\edef\Savesine{\relax}\else
		             \edef\Savesine{\sine}\fi
	{\dimen0=#1\r@dian\advance\dimen0 by 90\r@dian
	 \Sine{\nodimen 0}
	 \xdef\cosine{\sine}
	 \xdef\sine{\Savesine}}}	      
% end of trig stuff
%%%%%%%%%%%%%%%%%%%%%%%%%%%%%%%%%%%%%%%%%%%%%%%%%%%%%%%%%%%%%%%%%%%%

\def\psdraft{
	\def\@psdraft{0}
	%\ps@typeout{draft level now is \@psdraft \space . }
}
\def\psfull{
	\def\@psdraft{100}
	%\ps@typeout{draft level now is \@psdraft \space . }
}

\psfull

\newif\if@scalefirst
\def\psscalefirst{\@scalefirsttrue}
\def\psrotatefirst{\@scalefirstfalse}
\psrotatefirst

\newif\if@draftbox
\def\psnodraftbox{
	\@draftboxfalse
}
\def\psdraftbox{
	\@draftboxtrue
}
\@draftboxtrue

\newif\if@prologfile
\newif\if@postlogfile
\def\pssilent{
	\@noisyfalse
}
\def\psnoisy{
	\@noisytrue
}
\psnoisy
%%% These are for the option list.
%%% A specification of the form a = b maps to calling \@p@@sa{b}
\newif\if@bbllx
\newif\if@bblly
\newif\if@bburx
\newif\if@bbury
\newif\if@height
\newif\if@width
\newif\if@rheight
\newif\if@rwidth
\newif\if@angle
\newif\if@clip
\newif\if@verbose
\def\@p@@sclip#1{\@cliptrue}


\newif\if@decmpr

%%% GDH 7/26/87 -- changed so that it first looks in the local directory,
%%% then in a specified global directory for the ps file.
%%% RPR 6/25/91 -- changed so that it defaults to user-supplied name if
%%% boundingbox info is specified, assuming graphic will be created by
%%% print time.
%%% TJD 10/19/91 -- added bbfile vs. file distinction, and @decmpr flag

\def\@p@@sfigure#1{\def\@p@sfile{null}\def\@p@sbbfile{null}
	        \openin1=#1.bb
		\ifeof1\closein1
	        	\openin1=\figurepath#1.bb
			\ifeof1\closein1
			        \openin1=#1
				\ifeof1\closein1%
				       \openin1=\figurepath#1
					\ifeof1
					   \ps@typeout{Error, File #1 not found}
						\if@bbllx\if@bblly
				   		\if@bburx\if@bbury
			      				\def\@p@sfile{#1}%
			      				\def\@p@sbbfile{#1}%
							\@decmprfalse
				  	   	\fi\fi\fi\fi
					\else\closein1
				    		\def\@p@sfile{\figurepath#1}%
				    		\def\@p@sbbfile{\figurepath#1}%
						\@decmprfalse
	                       		\fi%
			 	\else\closein1%
					\def\@p@sfile{#1}
					\def\@p@sbbfile{#1}
					\@decmprfalse
			 	\fi
			\else
				\def\@p@sfile{\figurepath#1}
				\def\@p@sbbfile{\figurepath#1.bb}
				\@decmprtrue
			\fi
		\else
			\def\@p@sfile{#1}
			\def\@p@sbbfile{#1.bb}
			\@decmprtrue
		\fi}

\def\@p@@sfile#1{\@p@@sfigure{#1}}

\def\@p@@sbbllx#1{
		%\ps@typeout{bbllx is #1}
		\@bbllxtrue
		\dimen100=#1
		\edef\@p@sbbllx{\number\dimen100}
}
\def\@p@@sbblly#1{
		%\ps@typeout{bblly is #1}
		\@bbllytrue
		\dimen100=#1
		\edef\@p@sbblly{\number\dimen100}
}
\def\@p@@sbburx#1{
		%\ps@typeout{bburx is #1}
		\@bburxtrue
		\dimen100=#1
		\edef\@p@sbburx{\number\dimen100}
}
\def\@p@@sbbury#1{
		%\ps@typeout{bbury is #1}
		\@bburytrue
		\dimen100=#1
		\edef\@p@sbbury{\number\dimen100}
}
\def\@p@@sheight#1{
		\@heighttrue
		\dimen100=#1
   		\edef\@p@sheight{\number\dimen100}
		%\ps@typeout{Height is \@p@sheight}
}
\def\@p@@swidth#1{
		%\ps@typeout{Width is #1}
		\@widthtrue
		\dimen100=#1
		\edef\@p@swidth{\number\dimen100}
}
\def\@p@@srheight#1{
		%\ps@typeout{Reserved height is #1}
		\@rheighttrue
		\dimen100=#1
		\edef\@p@srheight{\number\dimen100}
}
\def\@p@@srwidth#1{
		%\ps@typeout{Reserved width is #1}
		\@rwidthtrue
		\dimen100=#1
		\edef\@p@srwidth{\number\dimen100}
}
\def\@p@@sangle#1{
		%\ps@typeout{Rotation is #1}
		\@angletrue
%		\dimen100=#1
		\edef\@p@sangle{#1} %\number\dimen100}
}
\def\@p@@ssilent#1{ 
		\@verbosefalse
}
\def\@p@@sprolog#1{\@prologfiletrue\def\@prologfileval{#1}}
\def\@p@@spostlog#1{\@postlogfiletrue\def\@postlogfileval{#1}}
\def\@cs@name#1{\csname #1\endcsname}
\def\@setparms#1=#2,{\@cs@name{@p@@s#1}{#2}}
%
% initialize the defaults (size the size of the figure)
%
\def\ps@init@parms{
		\@bbllxfalse \@bbllyfalse
		\@bburxfalse \@bburyfalse
		\@heightfalse \@widthfalse
		\@rheightfalse \@rwidthfalse
		\def\@p@sbbllx{}\def\@p@sbblly{}
		\def\@p@sbburx{}\def\@p@sbbury{}
		\def\@p@sheight{}\def\@p@swidth{}
		\def\@p@srheight{}\def\@p@srwidth{}
		\def\@p@sangle{0}
		\def\@p@sfile{} \def\@p@sbbfile{}
		\def\@p@scost{10}
		\def\@sc{}
		\@prologfilefalse
		\@postlogfilefalse
		\@clipfalse
		\if@noisy
			\@verbosetrue
		\else
			\@verbosefalse
		\fi
}
%
% Go through the options setting things up.
%
\def\parse@ps@parms#1{
	 	\@psdo\@psfiga:=#1\do
		   {\expandafter\@setparms\@psfiga,}}
%
% Compute bb height and width
%
\newif\ifno@bb
\def\bb@missing{
	\if@verbose{
		\ps@typeout{psfig: searching \@p@sbbfile \space  for bounding box}
	}\fi
	\no@bbtrue
	\epsf@getbb{\@p@sbbfile}
        \ifno@bb \else \bb@cull\epsf@llx\epsf@lly\epsf@urx\epsf@ury\fi
}	
\def\bb@cull#1#2#3#4{
	\dimen100=#1 bp\edef\@p@sbbllx{\number\dimen100}
	\dimen100=#2 bp\edef\@p@sbblly{\number\dimen100}
	\dimen100=#3 bp\edef\@p@sbburx{\number\dimen100}
	\dimen100=#4 bp\edef\@p@sbbury{\number\dimen100}
	\no@bbfalse
}
% rotate point (#1,#2) about (0,0).
% The sine and cosine of the angle are already stored in \sine and
% \cosine.  The result is placed in (\p@intvaluex, \p@intvaluey).
\newdimen\p@intvaluex
\newdimen\p@intvaluey
\def\rotate@#1#2{{\dimen0=#1 sp\dimen1=#2 sp
%            	calculate x' = x \cos\theta - y \sin\theta
		  \global\p@intvaluex=\cosine\dimen0
		  \dimen3=\sine\dimen1
		  \global\advance\p@intvaluex by -\dimen3
% 		calculate y' = x \sin\theta + y \cos\theta
		  \global\p@intvaluey=\sine\dimen0
		  \dimen3=\cosine\dimen1
		  \global\advance\p@intvaluey by \dimen3
		  }}
\def\compute@bb{
		\no@bbfalse
		\if@bbllx \else \no@bbtrue \fi
		\if@bblly \else \no@bbtrue \fi
		\if@bburx \else \no@bbtrue \fi
		\if@bbury \else \no@bbtrue \fi
		\ifno@bb \bb@missing \fi
		\ifno@bb \ps@typeout{FATAL ERROR: no bb supplied or found}
			\no-bb-error
		\fi
		%
%\ps@typeout{BB: \@p@sbbllx, \@p@sbblly, \@p@sbburx, \@p@sbbury} 
%
% store height/width of original (unrotated) bounding box
		\count203=\@p@sbburx
		\count204=\@p@sbbury
		\advance\count203 by -\@p@sbbllx
		\advance\count204 by -\@p@sbblly
		\edef\ps@bbw{\number\count203}
		\edef\ps@bbh{\number\count204}
		%\ps@typeout{ psbbh = \ps@bbh, psbbw = \ps@bbw }
		\if@angle 
			\Sine{\@p@sangle}\Cosine{\@p@sangle}
	        	{\dimen100=\maxdimen\xdef\r@p@sbbllx{\number\dimen100}
					    \xdef\r@p@sbblly{\number\dimen100}
			                    \xdef\r@p@sbburx{-\number\dimen100}
					    \xdef\r@p@sbbury{-\number\dimen100}}
%
% Need to rotate all four points and take the X-Y extremes of the new
% points as the new bounding box.
                        \def\minmaxtest{
			   \ifnum\number\p@intvaluex<\r@p@sbbllx
			      \xdef\r@p@sbbllx{\number\p@intvaluex}\fi
			   \ifnum\number\p@intvaluex>\r@p@sbburx
			      \xdef\r@p@sbburx{\number\p@intvaluex}\fi
			   \ifnum\number\p@intvaluey<\r@p@sbblly
			      \xdef\r@p@sbblly{\number\p@intvaluey}\fi
			   \ifnum\number\p@intvaluey>\r@p@sbbury
			      \xdef\r@p@sbbury{\number\p@intvaluey}\fi
			   }
%			lower left
			\rotate@{\@p@sbbllx}{\@p@sbblly}
			\minmaxtest
%			upper left
			\rotate@{\@p@sbbllx}{\@p@sbbury}
			\minmaxtest
%			lower right
			\rotate@{\@p@sbburx}{\@p@sbblly}
			\minmaxtest
%			upper right
			\rotate@{\@p@sbburx}{\@p@sbbury}
			\minmaxtest
			\edef\@p@sbbllx{\r@p@sbbllx}\edef\@p@sbblly{\r@p@sbblly}
			\edef\@p@sbburx{\r@p@sbburx}\edef\@p@sbbury{\r@p@sbbury}
%\ps@typeout{rotated BB: \r@p@sbbllx, \r@p@sbblly, \r@p@sbburx, \r@p@sbbury}
		\fi
		\count203=\@p@sbburx
		\count204=\@p@sbbury
		\advance\count203 by -\@p@sbbllx
		\advance\count204 by -\@p@sbblly
		\edef\@bbw{\number\count203}
		\edef\@bbh{\number\count204}
		%\ps@typeout{ bbh = \@bbh, bbw = \@bbw }
}
%
% \in@hundreds performs #1 * (#2 / #3) correct to the hundreds,
%	then leaves the result in @result
%
\def\in@hundreds#1#2#3{\count240=#2 \count241=#3
		     \count100=\count240	% 100 is first digit #2/#3
		     \divide\count100 by \count241
		     \count101=\count100
		     \multiply\count101 by \count241
		     \advance\count240 by -\count101
		     \multiply\count240 by 10
		     \count101=\count240	%101 is second digit of #2/#3
		     \divide\count101 by \count241
		     \count102=\count101
		     \multiply\count102 by \count241
		     \advance\count240 by -\count102
		     \multiply\count240 by 10
		     \count102=\count240	% 102 is the third digit
		     \divide\count102 by \count241
		     \count200=#1\count205=0
		     \count201=\count200
			\multiply\count201 by \count100
		 	\advance\count205 by \count201
		     \count201=\count200
			\divide\count201 by 10
			\multiply\count201 by \count101
			\advance\count205 by \count201
			%
		     \count201=\count200
			\divide\count201 by 100
			\multiply\count201 by \count102
			\advance\count205 by \count201
			%
		     \edef\@result{\number\count205}
}
\def\compute@wfromh{
		% computing : width = height * (bbw / bbh)
		\in@hundreds{\@p@sheight}{\@bbw}{\@bbh}
		%\ps@typeout{ \@p@sheight * \@bbw / \@bbh, = \@result }
		\edef\@p@swidth{\@result}
		%\ps@typeout{w from h: width is \@p@swidth}
}
\def\compute@hfromw{
		% computing : height = width * (bbh / bbw)
	        \in@hundreds{\@p@swidth}{\@bbh}{\@bbw}
		%\ps@typeout{ \@p@swidth * \@bbh / \@bbw = \@result }
		\edef\@p@sheight{\@result}
		%\ps@typeout{h from w : height is \@p@sheight}
}
\def\compute@handw{
		\if@height 
			\if@width
			\else
				\compute@wfromh
			\fi
		\else 
			\if@width
				\compute@hfromw
			\else
				\edef\@p@sheight{\@bbh}
				\edef\@p@swidth{\@bbw}
			\fi
		\fi
}
\def\compute@resv{
		\if@rheight \else \edef\@p@srheight{\@p@sheight} \fi
		\if@rwidth \else \edef\@p@srwidth{\@p@swidth} \fi
		%\ps@typeout{rheight = \@p@srheight, rwidth = \@p@srwidth}
}
%		
% Compute any missing values
\def\compute@sizes{
	\compute@bb
	\if@scalefirst\if@angle
% at this point the bounding box has been adjsuted correctly for
% rotation.  PSFIG does all of its scaling using \@bbh and \@bbw.  If
% a width= or height= was specified along with \psscalefirst, then the
% width=/height= value needs to be adjusted to match the new (rotated)
% bounding box size (specifed in \@bbw and \@bbh).
%    \ps@bbw       width=
%    -------  =  ---------- 
%    \@bbw       new width=
% so `new width=' = (width= * \@bbw) / \ps@bbw; where \ps@bbw is the
% width of the original (unrotated) bounding box.
	\if@width
	   \in@hundreds{\@p@swidth}{\@bbw}{\ps@bbw}
	   \edef\@p@swidth{\@result}
	\fi
	\if@height
	   \in@hundreds{\@p@sheight}{\@bbh}{\ps@bbh}
	   \edef\@p@sheight{\@result}
	\fi
	\fi\fi
	\compute@handw
	\compute@resv}

%
% \psfig
% usage : \psfig{file=, height=, width=, bbllx=, bblly=, bburx=, bbury=,
%			rheight=, rwidth=, clip=}
%
% "clip=" is a switch and takes no value, but the `=' must be present.
\def\psfig#1{\vbox {
	% do a zero width hard space so that a single
	% \psfig in a centering enviornment will behave nicely
	%{\setbox0=\hbox{\ }\ \hskip-\wd0}
	%
	\ps@init@parms
	\parse@ps@parms{#1}
	\compute@sizes
	%
	\ifnum\@p@scost<\@psdraft{
		%
		\special{ps::[begin] 	\@p@swidth \space \@p@sheight \space
				\@p@sbbllx \space \@p@sbblly \space
				\@p@sbburx \space \@p@sbbury \space
				startTexFig \space }
		\if@angle
			\special {ps:: \@p@sangle \space rotate \space} 
		\fi
		\if@clip{
			\if@verbose{
				\ps@typeout{(clip)}
			}\fi
			\special{ps:: doclip \space }
		}\fi
		\if@prologfile
		    \special{ps: plotfile \@prologfileval \space } \fi
		\if@decmpr{
			\if@verbose{
				\ps@typeout{psfig: including \@p@sfile.Z \space }
			}\fi
			\special{ps: plotfile "`zcat \@p@sfile.Z" \space }
		}\else{
			\if@verbose{
				\ps@typeout{psfig: including \@p@sfile \space }
			}\fi
			\special{ps: plotfile \@p@sfile \space }
		}\fi
		\if@postlogfile
		    \special{ps: plotfile \@postlogfileval \space } \fi
		\special{ps::[end] endTexFig \space }
		% Create the vbox to reserve the space for the figure.
		\vbox to \@p@srheight sp{
		% 1/92 TJD Changed from "true sp" to "sp" for magnification.
			\hbox to \@p@srwidth sp{
				\hss
			}
		\vss
		}
	}\else{
		% draft figure, just reserve the space and print the
		% path name.
		\if@draftbox{		
			% Verbose draft: print file name in box
			\hbox{\frame{\vbox to \@p@srheight sp{
			\vss
			\hbox to \@p@srwidth sp{ \hss \@p@sfile \hss }
			\vss
			}}}
		}\else{
			% Non-verbose draft
			\vbox to \@p@srheight sp{
			\vss
			\hbox to \@p@srwidth sp{\hss}
			\vss
			}
		}\fi	



	}\fi
}}
\psfigRestoreAt
\let\@=\LaTeXAtSign




\begin{document}

\title{Supporting authoring and learning\\ in a collaborative hypertext
system:\\ The Annotated Egret Navigator.}

\author {Carleton Moore\\
\\ Collaborative Software Development Laboratory\\ Department of
Information and Computer Sciences\\ 2565 The Mall\\ University of Hawaii\\
Honolulu, Hawaii 96822\\ (808) 956-6920\\ {\tt
cmoore@uhics.ics.Hawaii.edu}} \date{Techreport CSDL-TR-94-16\\ \today}

\maketitle

\tableofcontents

\newpage

\section{Introduction}
\label{sec:introduction}


%\subsection{Background}
%\subsection{Issues}
This research is concerned with how people collaboratively author and
learn.  More specifically, it is concerned with how to design and implement
a hypertext system to support collaborative authoring and learning. It will
provide information on the following questions:\begin{enumerate}
\item{What breakdowns occur in collaborative authoring and learning?}
  \begin{enumerate}
  \item{What types of breakdowns are due to moving collaboration on line?}
  \item{What types of breakdowns are due to our collaborative methods?}
  \end{enumerate} 
\item{How does hypertext support collaborative learning?}
  \begin{enumerate}
  \item{What kinds of hypertext structure are necessary for collaborative learning?}
  \item{What additional tools are needed for collaborative learning?}
  \end{enumerate} 
\item{How do authors create hypertext documents?}
  \begin{enumerate}
  \item{What kinds of hypertext structure are necessary for collaborative authoring?}
  \item{What additional tools are needed for collaborative authoring?}
  \end{enumerate} 
\item{What are the reading patterns of hypertext users?}
\item{What tools help the readers navigate in the hypertext document?}
\end{enumerate}

We are investigating these issues through the design, implementation, and
evaluation of AEN \footnote{Annotated Egret Navigator}, a hypertext collaborative authoring and learning tool.

These issues are becoming more and more important each day.  The use of
hypertext as a structuring mechanism for richly interdependent information
is becoming more widespread, and finds application in many areas of
software engineering and other fields.

The remainder of this paper is structured as follows, Section
\ref{sec:collaborative-l} introduces collaborative learning and authoring
in AEN. Section \ref{sec:evaluation} is a discussion of how we are going to
answer the issues raised above.  Section \ref{sec:related-work} examines
other collaborative learning and authoring systems and how they compare to
AEN. Section \ref{sec:future} discusses future directions for AEN.


\section{Collaborative Learning and Authoring in AEN}
\label{sec:collaborative-l}
Over the past five months, we have experimented with and redesigned AEN, a
reader and authoring system for hypertext, that places equal weight on
supporting the following: (1) on-line collaborative learning, (2)
collaborative authoring facilities and (3) collecting data about user
interaction with the hypertext document.  A fourth requirement of the
system is rapid response time. This system has undergone extensive
evolution as we learned more about the implications of these requirements
and their often conflicting natures.

AEN is a collaborative, instrumented learning/authoring tool built upon the
Egret hypertext database system \cite{csdl-93-09}.  AEN consists of a
hypertext document and several tools to support collaboration.  The
hypertext document represents the text from which the students learn and
the product of collaboration among the participants in the class.

The initial requirements of AEN were to create a virtual classroom and move
nearly all of the lectures for a class on-line into a hypertext document
called {\em The Annotated Egret}.  The class' subject is how to design and
construct collaborative systems using Egret.  This class would meet both
on-line and off-line for discussions about issues.  Most of the interaction
in the class would occur through annotation of the lecture material with
new hypertext links to questions, comments, and insights.  A few actual
classroom meetings would be supplementary forums to discuss things learned
or questioned through on-line activities \cite{John94}.  The Fall 94 ICS613
class is currently using AEN as the priciple learning environment.

The following three sections provide a detailed look at AEN's data model,
tool support and process model for collaborative authoring and learning.

\subsection{Structure of Data in AEN}

\subsubsection{Node Types}
AEN has two different classes of nodes: artifact
nodes and figure nodes.  Artifact nodes represent textual information in
the hypertext document.  Figure nodes represent graphical information. 

Artifact nodes contain HTML code that provides formating directives and
inter-node links.  There are four subclasses of artifacts:
\begin{itemize}
\item{\bf document:} Nodes that contain the text of the document created by
the participants.
\item{\bf comment:} Nodes containing the reactions of the participants to
the contents of other nodes.
\item{\bf quicky quiz:} Nodes that contain exercises for the participants
to complete.
\item{\bf quicky quiz answer:} Nodes that contain the participant's
solutions to the quicky quiz nodes.
\end{itemize}


\begin{figure}[htb]
  \centerline{\psfig{figure=Data.eps,width=6in}}
  \caption{{\bf Data Relationships in AEN}}
  \label{fig:Data}
\end{figure}

Figure nodes contain the graphics used by xview to display the figures.
Users can link figure nodes into artifact nodes.  Since figure nodes
contain graphical information rather than HTML code, users cannot annotate
them like artifact nodes.  Figure \ref{fig:Data} shows the relationships
between the node types.

\subsubsection{Link Types}
AEN also has 6 types of links: Include, Xref,
see\_Quicky\_Quiz, see\_Quicky\_Quiz\_Answer, see\_Comment, and
see\_Figure.  AEN restricts the type of the source and destination nodes
for each type of link. Table \ref{table:link types} describes the different
links.  These restrictions allow AEN to support a structure known as the
AEN Backbone.

\begin{table}
\begin{tabular}{|l|l|l|l|}
\hline
Link Type&Source Node&Destination Node&Description\\ \hline \hline
Include&document&document&Backbone structure \\
&&&of hypertext document\\ \hline
Xref&document&document&Cross reference\\ \hline
See\_Comment&Any&comment&Points to a Comment\\ \hline
See\_Quicky\_Quiz&Any&Quicky Quiz&Points to a Quicky Quiz\\ \hline
See\_Quicky\_Quiz\_Answer&Quicky Quiz&Quicky Quiz Answer&Points to a \\
&&&Quicky Quiz Answer\\ \hline
See\_Figure&Any&Figure&Points to a Figure\\ \hline
\end{tabular}
\label{table:link types}
\caption{Link Types}
\end{table}


\subsubsection{The AEN Backbone}
The AEN Backbone is the basic structure that defines {\em The Annotated
Egret} document.  It begins with a Document node that contains Include links to the
different ``chapters'' in {\em The Annotated Egret} document.  The Backbone
is the skeleton on which {\em The Annotated Egret} is built.  It provides a
spine for the hypertext.


\subsection{Mechanisms for Data Manipulation in AEN}
The previous section discussed the Data Model for AEN.  This section
discusses mechanisms to manipulate artifacts: access control, table of
contents, and nodelist.  The section concludes with two mechanisms to
support real time communication in AEN, Snoopy and Partyline.

\subsubsection{Access Control}
AEN's access control mechanisms implement three forms of access to
individual nodes: read access, write access, and annotation access. Users
can set each form of access for individual users or for all users.
When a user has read access for a specific node, they can retrieve and view
it.  Similarly, when a user has annotation access they can comment on the
node and make links from the node, but not edit its contents. Finally, when
a user has write access to a node, they can change the text in the node.

The read access mechanism is the primary access control mechanism.  The
owner of a node can restrict all access by denying other users read
access.  Users cannot annotate, or change the contents of nodes they cannot
read.
%, even if they have annotate or write access to the node.


\subsubsection{Table of Contents}
The access control mechanism determines which nodes a user may read, but it
provides no navigational aids to the user.  To help the user navigate
through the hypertext document AEN has a dynamically generated table of
contents (TOC).  The user can ask for a new TOC at any time.  AEN
calculates the new TOC from a starting node that the user chooses.  The TOC
generation algorithm only looks at document nodes the user has read access
to and include links.  It will only visit the children of
a node once.  This rule breaks any cycles in the traversal of a hypertext
network.  Once the TOC is displayed the user can decide what level of
detail they want to view.  Figure \ref{fig:TOC} shows what an example TOC
looks like.  The user can create multiple TOCs starting from different
document nodes.  The table of contents should help users find their way
through the hypertext document.  The TOC provides orienting
information about the hypertext network and prevents the
``lost-in-hyperspace'' phenomenon.  Another mechanism AEN has to help the
user navigate in the hypertext document is node lists.


\begin{figure}[htb]
  \centerline{\psfig{figure=TOC.eps,height=4in}}
  \caption{{\bf AEN's Table of Contents}}
  \label{fig:TOC}
\end{figure}


\subsubsection{Node Lists}
Node lists are a list of nodes based upon certain
criteria.  AEN provides the user with several built in criteria for
building these lists.  First, AEN allows the user to get a list of all the
nodes of any type.  Second, the user can get a list of all the nodes they
have not read yet or that have changed since the last time they were on
AEN.  Finally, AEN allows the user to get a list of all the nodes they own.
All of the node lists provide the user with a list of nodes from which 
they may select and view. Figure \ref{fig:nodelist} shows what a node list
of unread document nodes would look like.


\begin{figure}[htb]
  \centerline{\psfig{figure=nodelist2.eps,height=2.5in}}
  \caption{{\bf Unread Document Node List}}
  \label{fig:nodelist}
\end{figure}



All of the above mechanisms build upon the basic hypertext document, but we
believe that a plain hypertext document is not good enough for effective
collaboration.  To fully support collaboration there should be a
``physical'' presence in the hypertext document.  Users can
see who else is reading the document, what section they are in and
communicate to any one else using AEN.  These features are difficult to
support by just using hypertext.  The following two mechanisms provide
these features in AEN.

%With out real time communications we feel
%that a virtual classroom is not possible.  Virtual classrooms with out real
%time communications seem foreign to the us and we believe the lack of
%communication will be a hindrance.  For the classroom we also want a way of
%``seeing'' who was present and what they were doing.  This allows the user
%to ask questions of the people that are present and that have been looking
%at the same piece of the hypertext.  For collaborative authoring we also
%want to be able to see who else is using AEN.  This way authors will be
%able to meet other authors and talk to them. We have added two tools to
%provide this presence and communication in AEN, Snoopy and Partyline.

\subsubsection{Snoopy}
The Snoopy mechanism allows the user to see who else is at work and have an
idea of what they are working on.  It displays basic information about the
connection status and last read node of the users of AEN.  Figure
\ref{fig:snoopy} shows a view of the snoopy buffer.

\begin{figure}[htb]
  \centerline{\psfig{figure=snoopy.eps,height=2in}}
  \caption{{\bf AEN's Snoopy Buffer}}
  \label{fig:snoopy}
\end{figure}

\newpage
\subsubsection{Partyline}
Partyline is a mechanism for passing real time messages to other users of
AEN.  Partyline allows the user to send a textual message to all the users
currently connected to AEN or send a private message to one other user
currently connected to Partyline.  Figure \ref{fig:Partyline} shows a view
of the Partyline buffer.

\begin{figure}[htb]
  \centerline{\psfig{figure=partyline.eps,height=2in}}
  \caption{{\bf AEN's Partyline}}
  \label{fig:Partyline}
\end{figure}

Both facilities provide services that are similar to walking into
a room containing the other people using AEN.  There are some significant
differences between AEN's Snoopy/Partyline and the walking into a room.
When you walk into a room, you can only see the people in the room.  You
have no idea how long the people have been in the room and when people left
the room.  Snoopy gives you this information.  You can see how long ago
people left the room and how long people have been working on their current
task.

In a room many conversations can go on simultaneously and you can listen to
any of the conversations in the room.  Partyline allows you to ``listen''
to conversations that are sent to the entire room. In a real room it is
easy to have two conversations at the same time since we can tune out other
noises.  Partyline does not support two ``global'' conversations well.
Since Partyline is text-based, users have to read the messages.  When the
user receives a message they have to see who sent it.  This does not allow
them to totally ignore the other conversations in the room.  If they
actually read the message they focus their attention on the other
conversation and may loose their train of thought of their conversation.

One special feature that Partyline does allow is to have private
conversations.  In a real room you can tell when people are ``whispering''
to each other.  In Partyline there is no indication that two people are
``whispering''.

Another special feature of Partyline is that it keeps track of
conversations.  If a user leaves their workstation and don't log out of AEN
the Partyline buffer records the Partyline conversations.  When they return,
the conversations are still accessible in the Partyline buffer.  The other
users in AEN will know that they still can ``hear'' the conversations
since Snoopy tells them that the user is asleep.  Snoopy displays the
asleep message if the user has not done anything in a while.  Partyline
allows a user to be paged.  Even though the user might be asleep the
persistence of Partyline will tell them they have been paged.  They can
respond to the page when they return.

\subsection{AEN's Process Model}
All of the above mechanisms are used by AEN to support collaborative
learning and authoring.  The next sections discuss how these mechanisms
are used for collaborative learning and authoring.

\subsubsection{Collaborative Learning}
For collaborative learning the hypertext document represents the textbook
for the class and the instructor's lectures. When students have questions
about the textbook or ``lectures'', they can ask their questions by
annotating the text.  The instructor or other students can answer
the questions by making annotations on the original question.  This way
``in-class discussions'' can take place about different topics.  By using the
access control mechanism students or instructors can decide who gets to
read the node and who can participate in the discussions.  This allows
groups to collaborate on a question or project easily.  When the group is
ready, they can allow the rest of the class to view their answer or project.
To find out if anything has changed on their project the students can use
the node lists to see what nodes have changed.  This way they will waste
less time searching for new information.

The Snoopy and Partyline features expand AEN's textbook by providing
synchronous communications. Students can get together in AEN and discuss the
concepts presented in the text.  They can work together on solving the
quizzes or projects presented in AEN.  Snoopy and Partyline allow the
instructor to hold office hours by being ``present'' in AEN at given
times.  The students can ask the instructor questions and receive
answers immediately.  The instructor could have a group discussion with all
the students present in AEN. If anyone decides that they want to save the
discussion they can create a node in the hypertext document and paste the
conversation into it.  This way a permanent record of the discussion can
be kept. 

The combination of hypertext document and synchronous communications allows
the students to learn in many ways.  They can learn by exploring the text,
asking questions, answering questions raised by their fellow students,
building their own hypertext sections or by answering the exams and quizzes
added by the instructor.  As the students learn, the hypertext document
will be evolving.  The instructor will change the text to answer the
questions raised by the students.  The students will be creating
sub-documents in the hypertext document to explore the subject.  
 

\subsubsection{Collaborative Authoring}

AEN supports two methods for collaborative authoring: {\em proofreading \&
trading the lock}.  In the proofreading method the author can create a node
and ``publish'' it by changing the read and annotate access for the
document.  Then other co-authors can make comments as simple as
proofreading comments or as complex as philosophical discussions.  These
comments could be thought of as marginal notes suggesting changes, new
ideas, or just general comments on the subject.  The author can read the
comments and make changes to the document or comment on the comments.  This
can lead to discussions on the contents of the document.

In the trading the lock method the author can create a node and
allow other authors to edit the node by giving them read and write access to the
node.  When a node is saved, all connected users viewing the node
see the changes.  In this way two or more co-authors could trade the lock and
make changes that all the co-authors will see.  

%Philip wants to collaboratively author a document with Dadong.  To do this
%Philip would create a new node to store part of the text for the document.
%Philip then must decide how he wants to collaborate with Dadong.  
%
%If he want to get Dadong's comments on the document he could write the
%document and then give Dadong read and annotate access to the node.  Philip
%should then inform Dadong that he can read the node.  Dadong's Table of
%Contents will reflect the change in read access if Dadong refreshes the
%TOC.  Dadong then could read the node and create as many comments as he
%wants.  When Dadong is finished making a comment he should set the read and
%annotate access of the comment to allow Philip to read the comment.  Philip
%can look at the document node and see what comments Dadong has made.  If
%Dadong has finished the comment Philip will be able to read the comment.
%If Philip agrees with the comment he can change the text of the document
%node.  If Philip does not agree with the comment he can make a comment
%about the comment.  If Philip and Dadong are using AEN at the same time
%they could discuss their differences using Partyline.
%
%Another way Philip could collaborate with Dadong is to start the document
%node and then give Dadong read and write access.  Then when Philip wants
%Dadong to be able to edit the node he could unlock the node.  Dadong then
%could get the lock and edit the node.  Each time Dadong saves the node
%Philip will see the changes if he is viewing that node.  They could again
%have a conversation about the contents of the node using Partyline.
%
%A third method of collaboration is a combination of these two.  Philip
%could give Dadong read, write and annotation access to the node.  Then
%Dadong could make changes and/or comments on the text.  This is the most
%flexible method AEN supports for collaborative authoring.  In all of these
%methods Dadong cannot change the access of the node.  If Dadong wanted
%someone else to see the node he would have to ask Philip to change the read
%access for the node.  Philip could add Dadong to the owner's list of the
%node.  This way Dadong also has control over the access of the node.
%Owners cannot lose their access to a node.  This is the most powerful way
%of collaboratively authoring available in AEN.

\section{Evaluation}
\label{sec:evaluation}
In order to provide data to address the questions raised in Section
\ref{sec:introduction}, the following case study in the usage of AEN is
being performed.

\subsection{Duration}
Starting in the Fall of 1994 12 students in the graduate seminar ICS 613
are using AEN to learn about designing and implementing collaborative
systems using Egret.  The experiment started in September 1994 and is
continuing through December 1994.  Which corresponds to the Fall semester.
During this period the contents of the hypertext document will be
incrementally constructed while the system is in use.

\subsection{Method}

The method for this experiment is to collect metrics data generated by both
authors and readers of the hypertext, collect bug reports and suggestions,
and conduct a user survey.  The data will be analyzed both during and at
the end of the experiment.  The metrics will be collected
on:\begin{itemize}
\item{node creation}
\item{reading the contents of a node}
\item{changing the contents of a node}
\item{locking of nodes}
\item{changing the access privileges for a node}
\item{using different tools (ie Table of Contents, Node list, Unread
  Nodes)}
\end{itemize}

A questionnaire will be developed to gain information about the users of the
systems and their evaluation of the system.  More
specifically:\begin{itemize}
\item{How much hypertext experience the users had before the class}
\item{Which tools were the most helpful}
\item{How much each student learned from the class}
\item{Which features were the most convenient to use}
\end{itemize}

AEN has a Feedback menu that allows the users to send suggestions and bug
reports.  These suggestions and bug reports will be analyzed to help answer
the research questions.  The ICS 613 class has periodic meetings.
Suggestions or comments raised in these meetings will also be used to help
answer the research questions.  These suggestions and bug reports will be
classified in the following manner.\begin{itemize}
\item{Are they inside or outside our paradigm.}
\item{Do they involve a major or minor redesign of the system to
implement.}
\item{Are they important or nonessential.}
\end{itemize}

\subsection{Data Analysis}
The data collected will consist of metrics data, user suggestions, bug
reports and user questionnaires.  Let's revisit the central questions of
this research objective, and discuss how this data can be used to assess
them.
\begin{enumerate}
\item{\em What breakdowns occur in collaborative authoring and learning?}

  By looking at the questionnaires we can see how the users feel about
  AEN as a collaborative learning and authoring tool.  The user's input
  will help us understand which tools provided by AEN they disliked.
  The bug reports and suggestions will also tell us what problems they
  had using AEN.  We can also look at the answers to the exercises and
  projects to guage the knowledge learned by the users.  By looking at the
  metrics we will see what patterns the users used and what patterns lead
  to difficulties.

  \begin{enumerate}
  \item{\em What types of breakdowns are due to moving collaboration on
    line?}

  The comments raised in class will help us find out if face to face
  meetings are required for collaboration.  The questionnaire will also ask
  about how much time was spent in face to face meetings outside the
  context of the class.  By looking at the bug reports and the suggestions
  we can get an idea what the users liked and what the users did not like
  about AEN.  This should give us an idea of what problems the users had in
  collaborating on line.
\item{\em What types of breakdowns are due to our collaborative methods?}

  The user questionnaires and suggestions made by the users will help us
  determine if our methods are correct for this group of students.  The
  suggestions will tell us what features the users want to have.  The
  questionnaires will tell us which features of AEN were not helpful or were
  misleading.
\end{enumerate} 
\item{\em How does hypertext support collaborative learning?}  

  The metrics will help us find out what types of nodes are used.  The
  metrics will also tell us how the users answered ``collaborative''
  assignments.  We can tell how a group answered the question.
  Some of the possible ways they could collaboratively solve problems
  are:\begin{itemize}
\item{Alternatively Edit: They could create a solution and then take turns
  editing the solution.}
\item{Proof Read: They could create a solution and then proof read the
  solution}
\item{Discuss and Solve: They could create a discussion tree about the
  solution then create the solution based upon the discussion}
\item{Combination of the above methods}
\end{itemize}
\begin{enumerate}
\item{\em What kinds of hypertext structure are necessary for collaborative
  learning?}

  By looking at the metrics and the types of nodes created we can see how
  the users collaborated.  Did they create discussions through comments
  or did they create separate collaborative documents?

\item{\em What additional tools are needed for collaborative learning?}

  The questionnaire and suggestions will give us an idea of how much the
  other tools like Partyline and Snoopy were used to aid collaboration.
  The user's suggestions will give us a good idea of what tools are missing
  or are not necessary in AEN.  The classifications of the suggestions
  should help us determine how good AEN's paradigm is.
\end{enumerate} 
\item{\em How do authors create hypertext documents?}

  The metrics data will tells us how the users created new nodes and when
  they changed the access privileges to allow the other users to read the
  nodes.  We should be able to determine general strategies for authoring
  hypertext document.  Some possible strategies are:\begin{itemize}
\item{Build content then structure: The author creates the contents of the
  nodes in a single node then creates the separate nodes in the network and
  cuts and pastes the data into them.}
\item{Build structure then fill content: The author could ``outline'' the
  document by creating the structure of the nodes and then fill in their
  contents.}
\item{Iterative: The author could start filling the contents of a node
  then build another node when the contents dictate a new node and then
  start filling the contents of the new node.}\end{itemize}  The user's suggestions will also give us insight
  into what tools they want to aid in the authoring process.

  \begin{enumerate}
  \item{\em What kinds of hypertext structure are necessary for
    collaborative authoring?}  The metrics will tell us how authors
    collaboratively created a document.  Did they take turns editing the
    document or did they use the proof reading method?

  \item{\em What additional tools are needed for collaborative
    authoring?}

    The metrics data will tell us which tools were used the most.  The
    user questionnaires will tell us which tools the author's liked the
    most.  The user's suggestions will show us what addtional tools to
    include in AEN's design. The metrics and questionnaires should
    correlate well.  If they do then we can conclude that those tools
    are helpful in authoring a hypertext document.

  \end{enumerate} 
\item{\em What are the reading patterns of hypertext users?}

  The metrics data will tell us exactly in what order the users read the
  individual nodes in the hypertext document.  We will generate a ``map''
  of activity for each user.  We hope to determine general patterns for
  the users.  Some general patterns might be,
  
  \begin{itemize}
  \item{Linear: The user generally reads through the document once from
    one node to the next.}

  \item{Hunt and peck: The user jumps from one node to another that is
    not connected by a link.}

  \item{Combination: The user might hunt and peck until a particular area
    of the hypertext network is found then read linearly.}
  \end{itemize}

    
\item{\em What tools help the readers navigate in the hypertext document?}

  The metrics data will tell us which tools were used the most.  The user
  questionnaires will tell us which tools the user's liked the most.  The
  user's suggestions will tell us what new tools to include in AEN's
  design.  The metrics and questionnaire should correlate well.  If they
  do then we can conclude that those tools are helpful in navigating in a
  hypertext document.

\end{enumerate}


\section{Related Work}
%\subsection{Collaborative Systems}
\label{sec:related-work}
AEN's set of features were designed to explore the issues presented in the
introduction.  AEN provides a typed hypertext document, dynamic user
defined table of contents, real-time communications with other users of the
system, real-time knowledge of what the other users of the system are doing
and access control for each node.  Other hypertext or collaborative editors
provide different combinations of these features.  The final thesis will
look closely at the following collaborative or hypertext
systems abilities to explore the issues raised in Section \ref{sec:introduction}.:\begin{itemize}
\item{SEPIA \cite{Haak92}: a collaborative hypermedia authoring environment.}
\item{ShrEdit \cite{Cogn90}: a collaborative editor.}
\item{HyperCard: a Hypertext system that does not support collaboration.}
\item{WWW \cite{WWW94}: a global Hypertext system.}
\item{Intermedia: one of the largest and oldest hypermedia systems designed
to support learning.}
\item{GROVE \cite{Elli88}:}
\item{Quilt \cite{Fish88}:}
\item{NodeCards \cite{Hal87,Tri88,MI89}:}
\item{ENFI \cite{BPB93}:}
\item{PREP \cite{NKCM90,NCK+92}:}
\item{SASE \cite{BNPM93}:}
\item{WE \cite{SWF87}:}
\end{itemize}

%\subsection{Collaborative Learning}


%
%\subsection{SEPIA}
%SEPIA is an exception to this.  It is a collaborative hypermedia authoring
%environment.  It provides support for synchronous as well as asynchronous
%authoring.  It uses a graphical interface to represent the hypermedia
%document.  I provides support for collaboration by informing the user when
%another user is in the same area of the hypermedia network.  SEPIA provides
%both audio and video communications between users.  In the synchronous mode
%the users can see each others mouse cursors.  The controller controlls the
%scrolling and node selection for the group in closely coupled mode.  The
%users of SEPIA can change the structure of the hypermedia document
%collaboratively.  I do not know if they are able to lock a node and edit
%the individual nodes or not.  I know that if one user does change the
%contents of a graphics node all other users see the change immediately.
%The goals of SEPIA are more oriented toward supporting collaboration not
%collaborative editing of hypertext.  SEPIA was not built to study
%collaborative learning.\cite{Haak92}
%
%\subsection{ShrEdit}
%\subsection{USENET}
%\subsection{WWW}
%\subsection{Lotus Notes}
%\subsection{HyperCard}
%
%
%With out the support for collaborative authoring or changing the hypertext
%document collaborative learning cannot take place.
%
%In today's rapidly changing world the ability to author new hypertext
%documents or modify existing hypertext documents is becoming more
%important.  


%\section{New insights to collaborative learning and authoring}
%
%AEN has been evolving over the course of the experiment.  As users are
%using the system we have been making ``improvements'' to the system.  This
%has lead to some significant problems.  The ``improvements'' have often
%lead to consistency errors in the database.  The changes made have to be
%propagated through the database.  One significant change has been the
%complete re-design of AEN's class system.  This re-design caused major
%changes in the nodes of the database.  We were able to minimize the cost by
%just adding information to the nodes.  This allowed us to use the old
%representation and create new nodes with out much difficulty.  These
%changes have lead to some problems also.  Several times the users were
%unable to read old comments because they had not been updated to the newest
%version of AEN.  This evolution has allowed the users to continue to use
%AEN while the developers have been able to modify the system.  
%
%%AENs design/implementation cycle has been significantly different than most
%%of CSDL's other projects.  The typical CSDL project goes through intensive
%%design for a few months then a short usage to test the application.  After
%%the user's testing the application goes through more re-design and
%%implementation.  After the fixes have been implemented the application is
%%given to the users to test for a short period.  This cycle is repeated
%%several times.  
%
%AENs cycle has been very different.  There was an intense
%design/implementation period that lasted for about three months.  Then the
%application was ``released'' and used by ICS613.  While it was in use
%re-design and bug fixes were continually happening.  AEN has had only one
%``cycle'' of pure design/implementation.  The rest of the changes are due
%to user input.  They were implemented as the users used the system.
%
%{\em This section is under development.  It does not belong in the
%proposal.}
%\subsection{AEN's Current Data Model}
%\subsection{AEN's Current Process Model}
%
%%\subsubsection{Collaborative Learning}
%%\subsubsection{Collaborative Authoring}
%%\subsection{AEN's Current Tools for Real Time Collaboration Support}
%\subsection{Specific support for Process Model}
%
%Another feature added to AEN during the semester is the ability to create a
%collection of nodes by type.  The user can ask to see a list of all of the
%comment nodes in the hypertext document.  The user is then able to select
%the comment they want to read and the comment will be displayed.
%
%%\item{Unread Nodes,}  Records the status of the nodes.  It keeps track of
%%which nodes the user has read and which ones have been changed since the
%%last time the user was logged on.
%%\item{History,}  Records what nodes have been visited.  This allows the user
%%to easily return to a node they had visited.
%%\item{Node List,}  Displays different lists of nodes such as unread and
%%types.  This allows AEN to display customized list of nodes.  

\section{Future Directions}
\label{sec:future}

An area that this research does not look at is the difference between
collaboration in the small and collaboration in the large.  Also the
difference between a static hypertext reader like Mosaic and AEN is not
investigated.  AEN dynamically generates many navigation aids that Mosaic
cannot at this time.  What is the effect of these navigation aids on the
reader and author of hypertext?


\appendix
\newpage
\section{User Questionnaire}

The following questionnaire will be given to each of the users of AEN
during the experiment.  The questionnaires will be anonymous.

Please circle the answer or number which most appropriately reflect your
impressions.  Not Applicable = NA.  There is room on the last page for your
written comments.\\
\begin{tabular}{llcll}\\
\hline
Previous Experience:&&\\
Emacs:&\underline{    } $ < $ 3 months&  &Hypertext:&\underline{    } $ < $ 3 months\\\
&\underline{    } 3 -  $ < $ 6 months&  &&\underline{    } 3 -  $ < $ 6 months\\
&\underline{    } 6 - $ < $ 1 year&  &&\underline{    } 6 - $ < $ 1 year\\
&\underline{    } 1 year - $ < $ 2 years&  &&\underline{    } 1 year - $ < $ 2 years\\
&\underline{    } 1 year - $ < $ 2 years&  &&\underline{    } 1 year - $ <
$ 2 years\\
& NA&&&NA\\
\\
Collaborative Systems:&\underline{    } $ < $ 3 months&  &&\\
&\underline{    } 3 -  $ < $ 6 months&  &&\\
&\underline{    } 6 - $ < $ 1 year&  &&\\
&\underline{    } 1 year - $ < $ 2 years&  &&\\
&\underline{    } 1 year - $ < $ 2 years&  &&\\
&NA&&&\\
\end{tabular} \\

\begin{tabular}{lrccccccccclc}\\
\hline
Overall reactions to AEN:&terrible&1 &2 &3 &4 &5 &6 &7 &8 &9 &wonderful&NA\\
&frustrating&1 &2 &3 &4 &5 &6 &7 &8 &9 &satisfying&NA\\
&dull&1 &2 &3 &4 &5 &6 &7 &8 &9 &stimulating&NA\\
&difficult&1 &2 &3 &4 &5 &6 &7 &8 &9 &easy&NA\\
&rigid&1 &2 &3 &4 &5 &6 &7 &8 &9 &flexible&NA\\
\hline
Learning to operate AEN:&difficult&1 &2 &3 &4 &5 &6 &7 &8 &9 &easy&NA\\
\hline
Getting Started&difficult&1 &2 &3 &4 &5 &6 &7 &8 &9 &easy&NA\\
\hline
Learning advanced features&difficult&1 &2 &3 &4 &5 &6 &7 &8 &9 &easy&NA\\
\hline
Time to learn to use AEN&slow&1 &2 &3 &4 &5 &6 &7 &8 &9 &fast&NA\\
\hline
Exploration of features\\
 by trial and error:&discouraging&1 &2 &3 &4 &5 &6 &7 &8 &9 &encouraging&NA\\
\hline
Exploration of features&risky&1 &2 &3 &4 &5 &6 &7 &8 &9 &safe&NA\\
\hline
Discovering new features&difficult&1 &2 &3 &4 &5 &6 &7 &8 &9 &easy&NA\\
\hline
Remembering names and\\
 use of commands:&difficult&1 &2 &3 &4 &5 &6 &7 &8 &9 &easy&NA\\
\hline
Can tasks be performed in\\
 a straight-forward manner?&never&1 &2 &3 &4 &5 &6 &7 &8 &9 &always&NA\\
\hline
Are the needs of both&never&1 &2 &3 &4 &5 &6 &7 &8 &9 &always&NA\\
 experienced and\\
 inexperienced\\
 users taken into\\
 consideration?
\end{tabular}
\newpage
Please rank the following tools:

\begin{tabular}{lrccccccccclc}\\
\hline
I used the Table of Contents&never&1 &2 &3 &4 &5 &6 &7 &8 &9 &always&NA\\
\hline
I used Unread Nodes&never&1 &2 &3 &4 &5 &6 &7 &8 &9 &always&NA\\
\hline
I used Nodes by type&never&1 &2 &3 &4 &5 &6 &7 &8 &9 &always&NA\\
\hline
I used Owned Nodes&never&1 &2 &3 &4 &5 &6 &7 &8 &9 &always&NA\\
\hline
I used the Back \& &&&&&&&&&&&&\\Navigation Menu&never&1 &2 &3 &4 &5 &6 &7 &8 &9 &always&NA\\
\hline
I used History List&never&1 &2 &3 &4 &5 &6 &7 &8 &9 &always&NA\\
\hline
I used Snoopy&never&1 &2 &3 &4 &5 &6 &7 &8 &9 &always&NA\\
\hline
I used Partyline&never&1 &2 &3 &4 &5 &6 &7 &8 &9 &always&NA\\
\hline
\\
The Table of Contents&&&&&&&&&&&&\\ is helpful&not at all&1 &2 &3 &4 &5 &6 &7 &8 &9 &very helpful&NA\\
\hline
Unread Nodes is helpful&not at all&1 &2 &3 &4 &5 &6 &7 &8 &9 &very helpful&NA\\
\hline
Nodes by type is helpful&not at all&1 &2 &3 &4 &5 &6 &7 &8 &9 &very helpful&NA\\
\hline
Owned Nodes is helpful&not at all&1 &2 &3 &4 &5 &6 &7 &8 &9 &very helpful&NA\\
\hline
The Back \& Navigation Menu&&&&&&&&&&&&\\
 is helpful&not at all&1 &2 &3 &4 &5 &6 &7 &8 &9
&very helpful&NA\\
\hline
History List is helpful&not at all&1 &2 &3 &4 &5 &6 &7 &8 &9 &very helpful&NA\\
\hline
Snoopy is helpful&not at all&1 &2 &3 &4 &5 &6 &7 &8 &9 &very helpful&NA\\
\hline
Partyline is helpful&not at all&1 &2 &3 &4 &5 &6 &7 &8 &9 &very helpful&NA\\
\hline

\end{tabular}

Please add any additional comments here:
\newpage
\addcontentsline{toc}{section}{References}


\nocite{Baecker93}
\nocite{Berners-Lee94}
\nocite{Bruce93}
\nocite{Buxton90}
\nocite{Cogn90}
\nocite{Conklin87}
\nocite{Conklin88}
\nocite{Ellis88}
\nocite{Ellis91}
\nocite{Fish88}
\nocite{Haake92}
\nocite{Halasz87}
\nocite{Marshall89}
\nocite{Neuwirth90}
\nocite{Neuwirth92}
\nocite{Smith87}
\nocite{Trigg88}

\bibliography{/group/csdl/bib/aen}
\bibliographystyle{plain}


\end{document}



