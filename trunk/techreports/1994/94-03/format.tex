%%%%%%%%%%%%%%%%%%%%%%%%%%%%%% -*- Mode: Latex -*- %%%%%%%%%%%%%%%%%%%%%%%%%%%%
%% format.tex -- 
%% RCS:            : $Id: icse94.tex,v 1.2 94/02/15 13:49:55 johnson Exp $
%% Author          : Philip Johnson
%% Created On      : Thu Feb 10 11:15:01 1994
%% Last Modified By: Philip Johnson
%% Last Modified On: Mon Jul 18 11:14:30 1994
%% Status          : Unknown
%%%%%%%%%%%%%%%%%%%%%%%%%%%%%%%%%%%%%%%%%%%%%%%%%%%%%%%%%%%%%%%%%%%%%%%%%%%%%%%
%%   Copyright (C) 1994 University of Hawaii
%%%%%%%%%%%%%%%%%%%%%%%%%%%%%%%%%%%%%%%%%%%%%%%%%%%%%%%%%%%%%%%%%%%%%%%%%%%%%%%
%% 
%% History
%% 10-Feb-1994		Philip Johnson	
%%    

%% Submission to the 4th International Conference on Software Quality

%%                    | took out the twocolumn right here.
%%                    v
\documentstyle[nftimes,twocolumn,icsq,/group/csdl/tex/lmacros]{article}

\pagestyle{empty}
\thispagestyle{empty}

\begin{document}
%\setlength{\baselineskip}{13pt}

\pagestyle{empty}
\thispagestyle{empty}

\makeieeetitle
  {SUPPORTING TECHNOLOGY TRANSFER OF FORMAL TECHNICAL REVIEW \\
   THROUGH A COMPUTER SUPPORTED COLLABORATIVE REVIEW SYSTEM}
  {Philip M. Johnson\\
   Department of Information and Computer Sciences\\
   University of Hawaii\\
   Honolulu, HI 96822\\
   (808) 956-3489\\
   {\tt johnson@hawaii.edu}}

   
   \makeieeeabstract 
   {

   Formal technical review (FTR) is an essential component of all
   modern software quality assessment, assurance, and improvement techniques,
   and is acknowledged to be the most cost-effective form of quality
   improvement when practiced effectively.  However, traditional FTR
   methods such as inspection are very difficult to adopt
   in organizations: they introduce substantial new up-front
   costs, training, overhead, and group process obstacles.  Sustained
   commitment from high-level management along with substantial
   resources is often necessary for successful technology transfer of
   FTR.
  
   Since 1991, we have been designing and evaluating a series of
   versions of a system called CSRS: an instrumented, computer-supported
   cooperative work environment for formal technical review.  The
   current version of CSRS includes an FTR method definition language,
   which allows organizations to design their own FTR method, and to
   evolve it over time. This paper describes how our approach to
   computer supported FTR can address some of the issues in technology
   transfer of FTR.

   }

\thispagestyle{empty}

\section{Introduction}

Among all the software quality improvement methods currently known, formal
technical review 
%
(FTR\foot{We define {\em formal technical review} as ``a structured
encounter where a group of technical personnel analyzes an artifact to
improve quality.  The analysis produces a structured artifact that assesses
or improves the quality of the artifact as well as the quality of the
method.'' This definition includes methods such as Fagan's code inspection
\cite{Fagan76,Fagan86}, Phased Inspections \cite{Knight93}, and FTArm
(discussed here), but excludes methods such as informal peer reviews and
walkthroughs.})
%
enjoys unique advantages.  Some studies provide evidence that FTR can be
more effective at discovering errors than testing, while others indicate
that it can discover different classes of errors than testing
\cite{Myers78,Basili86}.  In concert with other process improvements,
Fujitsu found FTR to be so effective at discovering errors that they
dropped system testing from their software development procedure
\cite{Arthur93}.  FTR forms an essential part of methods and models for
very high quality software, such as Cleanroom Software Engineering
\cite{Linger93} and the SEI Capability Maturity Model \cite{Paulk93a}.
Finally, FTR displays a unique ability to improve the quality of the
producer as well as the quality of the product by dispersing knowledge
about applications and development skills across the organization.

Given the range of advantages ascribed to FTR, and the substantial
improvements in quality and cost-reductions attributed to it by some
organizations, it is curious that formal technical review is not ubiquitous
in modern software development.  Although rigorous data on industrial use
of FTR is not publically available, responses by 70 participants to an
informal survey we conducted on FTR via USENET revealed that FTR is
practiced practiced irregularly or not at all in over 80\% of the surveyed
organizations.  Similar non-rigorous evidence for a low level of FTR
adoption in industry is discussed in \cite{Brykczynski94}.

Since 1991, we have been designing and experimentally evaluating a
computer-supported cooperative work environment for FTR called CSRS
\cite{Johnson94,Johnson93,Johnson93b}.  One product of this research was
the creation of a new, highly instrumented, asynchronous review method
called FTArm that addresses a multiplicity of problems arising in the
research on and practice of traditional FTR.  As we began discussing
technology transfer of CSRS and FTArm with industrial organizations, we
became aware of a spectrum of organizational issues surrounding the
technology transfer of FTR in general and CSRS in particular that must also
be addressed.

These issues and others motivated a recent redesign of CSRS to provide a
specialized process modelling language for FTR.  The language is intended
to allow organizations to design their own FTR method for use with CSRS, and
to support incremental evolution in the method as the organization's
needs for and use of FTR changes.

In the next section, we present some of the problems involved in successful
technology transfer and adoption of FTR.  The following section briefly
overviews the CSRS system and the FTArm method.  Following this we discuss
how the process modelling facilities of FTArm can be applied to address
some of the problems that arise in technology transfer and adoption of FTR.


\section{Issues in FTR Technology Transfer}

\subsection{The transfer process}

Recent studies of technology transfer show that it may not be a
``transfer'' at all, but rather a {\em reconstruction} by one organization
of knowledge, expertise, and technology generated by another
organization \cite{Doheny-Farina92}.  This contrasts with the conventional view of
technology transfer, in which the technology is viewed as a relatively
static object whose successful transfer induces a change in the receiving
organization without impacting upon the technology itself.  When
participants in the transfer process interpret the technology differently,
the conventional view holds that they are either misperceiving the
technology or the technology has been somehow distorted.

The conventional view appears occasionally in the literature on industrial
use of FTR methods, such as Fagan's code inspection. Adoption failures are
here interpreted as either a misperception of the meaning of the method or
a failure to implement all parts of the method.  Successful technology
transfer, from the perspective of this literature, is simply a matter of
total adherence and commitment to a single approach to formal technical
review.

Other literature embraces a more contingent and context-sensitive view of
the adoption process.  For example, a study of FTR technology transfer at
Hewlett-Packard reveals that FTR adoption goes through a series of stages
and that blind adherence to a single standardized process is a recipe for
failure, not success \cite{Grady94}.  The four stages observed at Hewlett-Packard are
described in this study in the following way:

\begin{itemizenoindent}
\item {\em Experimental.}  This stage is characterized by trial adoption
  of a not well understood technique by a few groups within the
  organization with relatively little institutional support.  Surviving
  the experimental stage of technology adoption appears to depend upon:
  (a) visionary people who can look at tools and process from another
  context and see how they can be applied locally; (b) management support
  for visionary attempts without penalty for failure; and (c) a
  supportive infrastructure, since mistakes and failures will occur and
  early success is very fragile from an organizational standpoint.

\item {\em Initial Guidelines.}  Progression out of the experimental
  phase is marked by the appearance of training classes and educational
  materials on the technique, and the creation of small-scale
  infrastructure within the organization to promote the technology.
  However, the Hewlett-Packard researchers caution that readily available training is
  a necessary but not sufficient condition for technology dispersion.
  For the technology to become further incorporated into the
  organization, effort must be made to communicate success with the
  method throughout the organization, through activities such as
  newsletters, conferences, and so forth. In addition, high-level
  management must be educated in the evolving ``best practice'' of the
  method and they must continue to display commitment and allocate
  resources to the technology

\item {\em Widespread Belief and Adoption.} This stage is characterized
  by widespread acceptance within the organization that the technology is
  useful and important to the organization's success.  However, such
  ``widespread belief'' does not translate automatically into optimal or
  even effective use of the technology, and may even sow the seeds of the
  technology transfer's destruction. 

  One potential problem at this stage is that management may become
  convinced that there is ``one best way'' and begin pushing for its
  total and exclusive adoption.  Hewlett-Packard found that their internal divisions
  resisted this, prefering a consulting approach whereby corporate
  resources were applied to understanding the specific context and
  problems of a division, and then developing an individualized strategy
  to help the group improve their current practice.

  A second potential problem is that as the use of the technology spreads
  across the organization, the aggregate cost of the technology to the
  organization becomes increasingly substantial and significant. An
  effective business case must be created to ensure that the technology
  continues to be used beyond a trial period. Otherwise management may
  decide that the technology, though promising, is not cost-effective
  when scaled to the organizational level.

\item {\em Standardization.}  While Hewlett-Packard has not yet progressed beyond the
  previous stage, their researchers suggest that there is a phase beyond
  it.  This phase appears to be characterized by total integration of the
  technology into the organization, such that questions of
  appropriateness are no longer asked---the technology has become part of
  what makes the organization what it is. The HP researchers explicitly
  note that terming this stage ``standardization'' does not imply
  adherence by the entire company to a single process, but rather that
  every project would use some form of FTR technology in an efficient,
  cost-effective manner.

\end{itemizenoindent}

\subsection{Obstacles to FTR adoption}

Many studies assert that an FTR such as Fagan's inspection is
cost-effective and improves software quality, once successfully adopted and
when practiced effectively. However, it is also clear from studies that FTR
is difficult to adopt and practice effectively.  The following obstacles to
effective FTR adoption and use is drawn from
\cite{Basili94,Brykczynski94,Russell91}:


\end{document}




