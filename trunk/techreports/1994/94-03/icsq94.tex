%%%%%%%%%%%%%%%%%%%%%%%%%%%%%% -*- Mode: Latex -*- %%%%%%%%%%%%%%%%%%%%%%%%%%%%
%% icsq94.tex -- 
%% RCS:            : $Id: icse94.tex,v 1.2 94/02/15 13:49:55 johnson Exp $
%% Author          : Philip Johnson
%% Created On      : Thu Feb 10 11:15:01 1994
%% Last Modified By: Philip Johnson
%% Last Modified On: Fri Aug  5 10:19:55 1994
%% Status          : Unknown
%%%%%%%%%%%%%%%%%%%%%%%%%%%%%%%%%%%%%%%%%%%%%%%%%%%%%%%%%%%%%%%%%%%%%%%%%%%%%%%
%%   Copyright (C) 1994 University of Hawaii
%%%%%%%%%%%%%%%%%%%%%%%%%%%%%%%%%%%%%%%%%%%%%%%%%%%%%%%%%%%%%%%%%%%%%%%%%%%%%%%
%% 
%% History
%% 10-Feb-1994		Philip Johnson	
%%    

%% Submission to the 4th International Conference on Software Quality

%%                    | took out the twocolumn right here.
%%                    v
\documentstyle[nftimes,twocolumn,icsq,/group/csdl/tex/lmacros]{article}
% Psfig/TeX 
\def\PsfigVersion{1.9}
% dvips version
%
% All psfig/tex software, documentation, and related files
% in this distribution of psfig/tex are 
% Copyright 1987, 1988, 1991 Trevor J. Darrell
%
% Permission is granted for use and non-profit distribution of psfig/tex 
% providing that this notice is clearly maintained. The right to
% distribute any portion of psfig/tex for profit or as part of any commercial
% product is specifically reserved for the author(s) of that portion.
%
% *** Feel free to make local modifications of psfig as you wish,
% *** but DO NOT post any changed or modified versions of ``psfig''
% *** directly to the net. Send them to me and I'll try to incorporate
% *** them into future versions. If you want to take the psfig code 
% *** and make a new program (subject to the copyright above), distribute it, 
% *** (and maintain it) that's fine, just don't call it psfig.
%
% Bugs and improvements to trevor@media.mit.edu.
%
% Thanks to Greg Hager (GDH) and Ned Batchelder for their contributions
% to the original version of this project.
%
% Modified by J. Daniel Smith on 9 October 1990 to accept the
% %%BoundingBox: comment with or without a space after the colon.  Stole
% file reading code from Tom Rokicki's EPSF.TEX file (see below).
%
% More modifications by J. Daniel Smith on 29 March 1991 to allow the
% the included PostScript figure to be rotated.  The amount of
% rotation is specified by the "angle=" parameter of the \psfig command.
%
% Modified by Robert Russell on June 25, 1991 to allow users to specify
% .ps filenames which don't yet exist, provided they explicitly provide
% boundingbox information via the \psfig command. Note: This will only work
% if the "file=" parameter follows all four "bb???=" parameters in the
% command. This is due to the order in which psfig interprets these params.
%
%  3 Jul 1991	JDS	check if file already read in once
%  4 Sep 1991	JDS	fixed incorrect computation of rotated
%			bounding box
% 25 Sep 1991	GVR	expanded synopsis of \psfig
% 14 Oct 1991	JDS	\fbox code from LaTeX so \psdraft works with TeX
%			changed \typeout to \ps@typeout
% 17 Oct 1991	JDS	added \psscalefirst and \psrotatefirst
%

% From: gvr@cs.brown.edu (George V. Reilly)
%
% \psdraft	draws an outline box, but doesn't include the figure
%		in the DVI file.  Useful for previewing.
%
% \psfull	includes the figure in the DVI file (default).
%
% \psscalefirst width= or height= specifies the size of the figure
% 		before rotation.
% \psrotatefirst (default) width= or height= specifies the size of the
% 		 figure after rotation.  Asymetric figures will
% 		 appear to shrink.
%
% \psfigurepath#1	sets the path to search for the figure
%
% \psfig
% usage: \psfig{file=, figure=, height=, width=,
%			bbllx=, bblly=, bburx=, bbury=,
%			rheight=, rwidth=, clip=, angle=, silent=}
%
%	"file" is the filename.  If no path name is specified and the
%		file is not found in the current directory,
%		it will be looked for in directory \psfigurepath.
%	"figure" is a synonym for "file".
%	By default, the width and height of the figure are taken from
%		the BoundingBox of the figure.
%	If "width" is specified, the figure is scaled so that it has
%		the specified width.  Its height changes proportionately.
%	If "height" is specified, the figure is scaled so that it has
%		the specified height.  Its width changes proportionately.
%	If both "width" and "height" are specified, the figure is scaled
%		anamorphically.
%	"bbllx", "bblly", "bburx", and "bbury" control the PostScript
%		BoundingBox.  If these four values are specified
%               *before* the "file" option, the PSFIG will not try to
%               open the PostScript file.
%	"rheight" and "rwidth" are the reserved height and width
%		of the figure, i.e., how big TeX actually thinks
%		the figure is.  They default to "width" and "height".
%	The "clip" option ensures that no portion of the figure will
%		appear outside its BoundingBox.  "clip=" is a switch and
%		takes no value, but the `=' must be present.
%	The "angle" option specifies the angle of rotation (degrees, ccw).
%	The "silent" option makes \psfig work silently.
%

% check to see if macros already loaded in (maybe some other file says
% "\input psfig") ...
\ifx\undefined\psfig\else\endinput\fi

%
% from a suggestion by eijkhout@csrd.uiuc.edu to allow
% loading as a style file. Changed to avoid problems
% with amstex per suggestion by jbence@math.ucla.edu

\let\LaTeXAtSign=\@
\let\@=\relax
\edef\psfigRestoreAt{\catcode`\@=\number\catcode`@\relax}
%\edef\psfigRestoreAt{\catcode`@=\number\catcode`@\relax}
\catcode`\@=11\relax
\newwrite\@unused
\def\ps@typeout#1{{\let\protect\string\immediate\write\@unused{#1}}}
\ps@typeout{psfig/tex \PsfigVersion}

%% Here's how you define your figure path.  Should be set up with null
%% default and a user useable definition.

\def\figurepath{./}
\def\psfigurepath#1{\edef\figurepath{#1}}

%
% @psdo control structure -- similar to Latex @for.
% I redefined these with different names so that psfig can
% be used with TeX as well as LaTeX, and so that it will not 
% be vunerable to future changes in LaTeX's internal
% control structure,
%
\def\@nnil{\@nil}
\def\@empty{}
\def\@psdonoop#1\@@#2#3{}
\def\@psdo#1:=#2\do#3{\edef\@psdotmp{#2}\ifx\@psdotmp\@empty \else
    \expandafter\@psdoloop#2,\@nil,\@nil\@@#1{#3}\fi}
\def\@psdoloop#1,#2,#3\@@#4#5{\def#4{#1}\ifx #4\@nnil \else
       #5\def#4{#2}\ifx #4\@nnil \else#5\@ipsdoloop #3\@@#4{#5}\fi\fi}
\def\@ipsdoloop#1,#2\@@#3#4{\def#3{#1}\ifx #3\@nnil 
       \let\@nextwhile=\@psdonoop \else
      #4\relax\let\@nextwhile=\@ipsdoloop\fi\@nextwhile#2\@@#3{#4}}
\def\@tpsdo#1:=#2\do#3{\xdef\@psdotmp{#2}\ifx\@psdotmp\@empty \else
    \@tpsdoloop#2\@nil\@nil\@@#1{#3}\fi}
\def\@tpsdoloop#1#2\@@#3#4{\def#3{#1}\ifx #3\@nnil 
       \let\@nextwhile=\@psdonoop \else
      #4\relax\let\@nextwhile=\@tpsdoloop\fi\@nextwhile#2\@@#3{#4}}
% 
% \fbox is defined in latex.tex; so if \fbox is undefined, assume that
% we are not in LaTeX.
% Perhaps this could be done better???
\ifx\undefined\fbox
% \fbox code from modified slightly from LaTeX
\newdimen\fboxrule
\newdimen\fboxsep
\newdimen\ps@tempdima
\newbox\ps@tempboxa
\fboxsep = 3pt
\fboxrule = .4pt
\long\def\fbox#1{\leavevmode\setbox\ps@tempboxa\hbox{#1}\ps@tempdima\fboxrule
    \advance\ps@tempdima \fboxsep \advance\ps@tempdima \dp\ps@tempboxa
   \hbox{\lower \ps@tempdima\hbox
  {\vbox{\hrule height \fboxrule
          \hbox{\vrule width \fboxrule \hskip\fboxsep
          \vbox{\vskip\fboxsep \box\ps@tempboxa\vskip\fboxsep}\hskip 
                 \fboxsep\vrule width \fboxrule}
                 \hrule height \fboxrule}}}}
\fi
%
%%%%%%%%%%%%%%%%%%%%%%%%%%%%%%%%%%%%%%%%%%%%%%%%%%%%%%%%%%%%%%%%%%%
% file reading stuff from epsf.tex
%   EPSF.TEX macro file:
%   Written by Tomas Rokicki of Radical Eye Software, 29 Mar 1989.
%   Revised by Don Knuth, 3 Jan 1990.
%   Revised by Tomas Rokicki to accept bounding boxes with no
%      space after the colon, 18 Jul 1990.
%   Portions modified/removed for use in PSFIG package by
%      J. Daniel Smith, 9 October 1990.
%
\newread\ps@stream
\newif\ifnot@eof       % continue looking for the bounding box?
\newif\if@noisy        % report what you're making?
\newif\if@atend        % %%BoundingBox: has (at end) specification
\newif\if@psfile       % does this look like a PostScript file?
%
% PostScript files should start with `%!'
%
{\catcode`\%=12\global\gdef\epsf@start{%!}}
\def\epsf@PS{PS}
%
\def\epsf@getbb#1{%
%
%   The first thing we need to do is to open the
%   PostScript file, if possible.
%
\openin\ps@stream=#1
\ifeof\ps@stream\ps@typeout{Error, File #1 not found}\else
%
%   Okay, we got it. Now we'll scan lines until we find one that doesn't
%   start with %. We're looking for the bounding box comment.
%
   {\not@eoftrue \chardef\other=12
    \def\do##1{\catcode`##1=\other}\dospecials \catcode`\ =10
    \loop
       \if@psfile
	  \read\ps@stream to \epsf@fileline
       \else{
	  \obeyspaces
          \read\ps@stream to \epsf@tmp\global\let\epsf@fileline\epsf@tmp}
       \fi
       \ifeof\ps@stream\not@eoffalse\else
%
%   Check the first line for `%!'.  Issue a warning message if its not
%   there, since the file might not be a PostScript file.
%
       \if@psfile\else
       \expandafter\epsf@test\epsf@fileline:. \\%
       \fi
%
%   We check to see if the first character is a % sign;
%   if so, we look further and stop only if the line begins with
%   `%%BoundingBox:' and the `(atend)' specification was not found.
%   That is, the only way to stop is when the end of file is reached,
%   or a `%%BoundingBox: llx lly urx ury' line is found.
%
          \expandafter\epsf@aux\epsf@fileline:. \\%
       \fi
   \ifnot@eof\repeat
   }\closein\ps@stream\fi}%
%
% This tests if the file we are reading looks like a PostScript file.
%
\long\def\epsf@test#1#2#3:#4\\{\def\epsf@testit{#1#2}
			\ifx\epsf@testit\epsf@start\else
\ps@typeout{Warning! File does not start with `\epsf@start'.  It may not be a PostScript file.}
			\fi
			\@psfiletrue} % don't test after 1st line
%
%   We still need to define the tricky \epsf@aux macro. This requires
%   a couple of magic constants for comparison purposes.
%
{\catcode`\%=12\global\let\epsf@percent=%\global\def\epsf@bblit{%BoundingBox}}
%
%
%   So we're ready to check for `%BoundingBox:' and to grab the
%   values if they are found.  We continue searching if `(at end)'
%   was found after the `%BoundingBox:'.
%
\long\def\epsf@aux#1#2:#3\\{\ifx#1\epsf@percent
   \def\epsf@testit{#2}\ifx\epsf@testit\epsf@bblit
	\@atendfalse
        \epsf@atend #3 . \\%
	\if@atend	
	   \if@verbose{
		\ps@typeout{psfig: found `(atend)'; continuing search}
	   }\fi
        \else
        \epsf@grab #3 . . . \\%
        \not@eoffalse
        \global\no@bbfalse
        \fi
   \fi\fi}%
%
%   Here we grab the values and stuff them in the appropriate definitions.
%
\def\epsf@grab #1 #2 #3 #4 #5\\{%
   \global\def\epsf@llx{#1}\ifx\epsf@llx\empty
      \epsf@grab #2 #3 #4 #5 .\\\else
   \global\def\epsf@lly{#2}%
   \global\def\epsf@urx{#3}\global\def\epsf@ury{#4}\fi}%
%
% Determine if the stuff following the %%BoundingBox is `(atend)'
% J. Daniel Smith.  Copied from \epsf@grab above.
%
\def\epsf@atendlit{(atend)} 
\def\epsf@atend #1 #2 #3\\{%
   \def\epsf@tmp{#1}\ifx\epsf@tmp\empty
      \epsf@atend #2 #3 .\\\else
   \ifx\epsf@tmp\epsf@atendlit\@atendtrue\fi\fi}


% End of file reading stuff from epsf.tex
%%%%%%%%%%%%%%%%%%%%%%%%%%%%%%%%%%%%%%%%%%%%%%%%%%%%%%%%%%%%%%%%%%%

%%%%%%%%%%%%%%%%%%%%%%%%%%%%%%%%%%%%%%%%%%%%%%%%%%%%%%%%%%%%%%%%%%%
% trigonometry stuff from "trig.tex"
\chardef\psletter = 11 % won't conflict with \begin{letter} now...
\chardef\other = 12

\newif \ifdebug %%% turn me on to see TeX hard at work ...
\newif\ifc@mpute %%% don't need to compute some values
\c@mputetrue % but assume that we do

\let\then = \relax
\def\r@dian{pt }
\let\r@dians = \r@dian
\let\dimensionless@nit = \r@dian
\let\dimensionless@nits = \dimensionless@nit
\def\internal@nit{sp }
\let\internal@nits = \internal@nit
\newif\ifstillc@nverging
\def \Mess@ge #1{\ifdebug \then \message {#1} \fi}

{ %%% Things that need abnormal catcodes %%%
	\catcode `\@ = \psletter
	\gdef \nodimen {\expandafter \n@dimen \the \dimen}
	\gdef \term #1 #2 #3%
	       {\edef \t@ {\the #1}%%% freeze parameter 1 (count, by value)
		\edef \t@@ {\expandafter \n@dimen \the #2\r@dian}%
				   %%% freeze parameter 2 (dimen, by value)
		\t@rm {\t@} {\t@@} {#3}%
	       }
	\gdef \t@rm #1 #2 #3%
	       {{%
		\count 0 = 0
		\dimen 0 = 1 \dimensionless@nit
		\dimen 2 = #2\relax
		\Mess@ge {Calculating term #1 of \nodimen 2}%
		\loop
		\ifnum	\count 0 < #1
		\then	\advance \count 0 by 1
			\Mess@ge {Iteration \the \count 0 \space}%
			\Multiply \dimen 0 by {\dimen 2}%
			\Mess@ge {After multiplication, term = \nodimen 0}%
			\Divide \dimen 0 by {\count 0}%
			\Mess@ge {After division, term = \nodimen 0}%
		\repeat
		\Mess@ge {Final value for term #1 of 
				\nodimen 2 \space is \nodimen 0}%
		\xdef \Term {#3 = \nodimen 0 \r@dians}%
		\aftergroup \Term
	       }}
	\catcode `\p = \other
	\catcode `\t = \other
	\gdef \n@dimen #1pt{#1} %%% throw away the ``pt''
}

\def \Divide #1by #2{\divide #1 by #2} %%% just a synonym

\def \Multiply #1by #2%%% allows division of a dimen by a dimen
       {{%%% should really freeze parameter 2 (dimen, passed by value)
	\count 0 = #1\relax
	\count 2 = #2\relax
	\count 4 = 65536
	\Mess@ge {Before scaling, count 0 = \the \count 0 \space and
			count 2 = \the \count 2}%
	\ifnum	\count 0 > 32767 %%% do our best to avoid overflow
	\then	\divide \count 0 by 4
		\divide \count 4 by 4
	\else	\ifnum	\count 0 < -32767
		\then	\divide \count 0 by 4
			\divide \count 4 by 4
		\else
		\fi
	\fi
	\ifnum	\count 2 > 32767 %%% while retaining reasonable accuracy
	\then	\divide \count 2 by 4
		\divide \count 4 by 4
	\else	\ifnum	\count 2 < -32767
		\then	\divide \count 2 by 4
			\divide \count 4 by 4
		\else
		\fi
	\fi
	\multiply \count 0 by \count 2
	\divide \count 0 by \count 4
	\xdef \product {#1 = \the \count 0 \internal@nits}%
	\aftergroup \product
       }}

\def\r@duce{\ifdim\dimen0 > 90\r@dian \then   % sin(x+90) = sin(180-x)
		\multiply\dimen0 by -1
		\advance\dimen0 by 180\r@dian
		\r@duce
	    \else \ifdim\dimen0 < -90\r@dian \then  % sin(-x) = sin(360+x)
		\advance\dimen0 by 360\r@dian
		\r@duce
		\fi
	    \fi}

\def\Sine#1%
       {{%
	\dimen 0 = #1 \r@dian
	\r@duce
	\ifdim\dimen0 = -90\r@dian \then
	   \dimen4 = -1\r@dian
	   \c@mputefalse
	\fi
	\ifdim\dimen0 = 90\r@dian \then
	   \dimen4 = 1\r@dian
	   \c@mputefalse
	\fi
	\ifdim\dimen0 = 0\r@dian \then
	   \dimen4 = 0\r@dian
	   \c@mputefalse
	\fi
%
	\ifc@mpute \then
        	% convert degrees to radians
		\divide\dimen0 by 180
		\dimen0=3.141592654\dimen0
%
		\dimen 2 = 3.1415926535897963\r@dian %%% a well-known constant
		\divide\dimen 2 by 2 %%% we only deal with -pi/2 : pi/2
		\Mess@ge {Sin: calculating Sin of \nodimen 0}%
		\count 0 = 1 %%% see power-series expansion for sine
		\dimen 2 = 1 \r@dian %%% ditto
		\dimen 4 = 0 \r@dian %%% ditto
		\loop
			\ifnum	\dimen 2 = 0 %%% then we've done
			\then	\stillc@nvergingfalse 
			\else	\stillc@nvergingtrue
			\fi
			\ifstillc@nverging %%% then calculate next term
			\then	\term {\count 0} {\dimen 0} {\dimen 2}%
				\advance \count 0 by 2
				\count 2 = \count 0
				\divide \count 2 by 2
				\ifodd	\count 2 %%% signs alternate
				\then	\advance \dimen 4 by \dimen 2
				\else	\advance \dimen 4 by -\dimen 2
				\fi
		\repeat
	\fi		
			\xdef \sine {\nodimen 4}%
       }}

% Now the Cosine can be calculated easily by calling \Sine
\def\Cosine#1{\ifx\sine\UnDefined\edef\Savesine{\relax}\else
		             \edef\Savesine{\sine}\fi
	{\dimen0=#1\r@dian\advance\dimen0 by 90\r@dian
	 \Sine{\nodimen 0}
	 \xdef\cosine{\sine}
	 \xdef\sine{\Savesine}}}	      
% end of trig stuff
%%%%%%%%%%%%%%%%%%%%%%%%%%%%%%%%%%%%%%%%%%%%%%%%%%%%%%%%%%%%%%%%%%%%

\def\psdraft{
	\def\@psdraft{0}
	%\ps@typeout{draft level now is \@psdraft \space . }
}
\def\psfull{
	\def\@psdraft{100}
	%\ps@typeout{draft level now is \@psdraft \space . }
}

\psfull

\newif\if@scalefirst
\def\psscalefirst{\@scalefirsttrue}
\def\psrotatefirst{\@scalefirstfalse}
\psrotatefirst

\newif\if@draftbox
\def\psnodraftbox{
	\@draftboxfalse
}
\def\psdraftbox{
	\@draftboxtrue
}
\@draftboxtrue

\newif\if@prologfile
\newif\if@postlogfile
\def\pssilent{
	\@noisyfalse
}
\def\psnoisy{
	\@noisytrue
}
\psnoisy
%%% These are for the option list.
%%% A specification of the form a = b maps to calling \@p@@sa{b}
\newif\if@bbllx
\newif\if@bblly
\newif\if@bburx
\newif\if@bbury
\newif\if@height
\newif\if@width
\newif\if@rheight
\newif\if@rwidth
\newif\if@angle
\newif\if@clip
\newif\if@verbose
\def\@p@@sclip#1{\@cliptrue}


\newif\if@decmpr

%%% GDH 7/26/87 -- changed so that it first looks in the local directory,
%%% then in a specified global directory for the ps file.
%%% RPR 6/25/91 -- changed so that it defaults to user-supplied name if
%%% boundingbox info is specified, assuming graphic will be created by
%%% print time.
%%% TJD 10/19/91 -- added bbfile vs. file distinction, and @decmpr flag

\def\@p@@sfigure#1{\def\@p@sfile{null}\def\@p@sbbfile{null}
	        \openin1=#1.bb
		\ifeof1\closein1
	        	\openin1=\figurepath#1.bb
			\ifeof1\closein1
			        \openin1=#1
				\ifeof1\closein1%
				       \openin1=\figurepath#1
					\ifeof1
					   \ps@typeout{Error, File #1 not found}
						\if@bbllx\if@bblly
				   		\if@bburx\if@bbury
			      				\def\@p@sfile{#1}%
			      				\def\@p@sbbfile{#1}%
							\@decmprfalse
				  	   	\fi\fi\fi\fi
					\else\closein1
				    		\def\@p@sfile{\figurepath#1}%
				    		\def\@p@sbbfile{\figurepath#1}%
						\@decmprfalse
	                       		\fi%
			 	\else\closein1%
					\def\@p@sfile{#1}
					\def\@p@sbbfile{#1}
					\@decmprfalse
			 	\fi
			\else
				\def\@p@sfile{\figurepath#1}
				\def\@p@sbbfile{\figurepath#1.bb}
				\@decmprtrue
			\fi
		\else
			\def\@p@sfile{#1}
			\def\@p@sbbfile{#1.bb}
			\@decmprtrue
		\fi}

\def\@p@@sfile#1{\@p@@sfigure{#1}}

\def\@p@@sbbllx#1{
		%\ps@typeout{bbllx is #1}
		\@bbllxtrue
		\dimen100=#1
		\edef\@p@sbbllx{\number\dimen100}
}
\def\@p@@sbblly#1{
		%\ps@typeout{bblly is #1}
		\@bbllytrue
		\dimen100=#1
		\edef\@p@sbblly{\number\dimen100}
}
\def\@p@@sbburx#1{
		%\ps@typeout{bburx is #1}
		\@bburxtrue
		\dimen100=#1
		\edef\@p@sbburx{\number\dimen100}
}
\def\@p@@sbbury#1{
		%\ps@typeout{bbury is #1}
		\@bburytrue
		\dimen100=#1
		\edef\@p@sbbury{\number\dimen100}
}
\def\@p@@sheight#1{
		\@heighttrue
		\dimen100=#1
   		\edef\@p@sheight{\number\dimen100}
		%\ps@typeout{Height is \@p@sheight}
}
\def\@p@@swidth#1{
		%\ps@typeout{Width is #1}
		\@widthtrue
		\dimen100=#1
		\edef\@p@swidth{\number\dimen100}
}
\def\@p@@srheight#1{
		%\ps@typeout{Reserved height is #1}
		\@rheighttrue
		\dimen100=#1
		\edef\@p@srheight{\number\dimen100}
}
\def\@p@@srwidth#1{
		%\ps@typeout{Reserved width is #1}
		\@rwidthtrue
		\dimen100=#1
		\edef\@p@srwidth{\number\dimen100}
}
\def\@p@@sangle#1{
		%\ps@typeout{Rotation is #1}
		\@angletrue
%		\dimen100=#1
		\edef\@p@sangle{#1} %\number\dimen100}
}
\def\@p@@ssilent#1{ 
		\@verbosefalse
}
\def\@p@@sprolog#1{\@prologfiletrue\def\@prologfileval{#1}}
\def\@p@@spostlog#1{\@postlogfiletrue\def\@postlogfileval{#1}}
\def\@cs@name#1{\csname #1\endcsname}
\def\@setparms#1=#2,{\@cs@name{@p@@s#1}{#2}}
%
% initialize the defaults (size the size of the figure)
%
\def\ps@init@parms{
		\@bbllxfalse \@bbllyfalse
		\@bburxfalse \@bburyfalse
		\@heightfalse \@widthfalse
		\@rheightfalse \@rwidthfalse
		\def\@p@sbbllx{}\def\@p@sbblly{}
		\def\@p@sbburx{}\def\@p@sbbury{}
		\def\@p@sheight{}\def\@p@swidth{}
		\def\@p@srheight{}\def\@p@srwidth{}
		\def\@p@sangle{0}
		\def\@p@sfile{} \def\@p@sbbfile{}
		\def\@p@scost{10}
		\def\@sc{}
		\@prologfilefalse
		\@postlogfilefalse
		\@clipfalse
		\if@noisy
			\@verbosetrue
		\else
			\@verbosefalse
		\fi
}
%
% Go through the options setting things up.
%
\def\parse@ps@parms#1{
	 	\@psdo\@psfiga:=#1\do
		   {\expandafter\@setparms\@psfiga,}}
%
% Compute bb height and width
%
\newif\ifno@bb
\def\bb@missing{
	\if@verbose{
		\ps@typeout{psfig: searching \@p@sbbfile \space  for bounding box}
	}\fi
	\no@bbtrue
	\epsf@getbb{\@p@sbbfile}
        \ifno@bb \else \bb@cull\epsf@llx\epsf@lly\epsf@urx\epsf@ury\fi
}	
\def\bb@cull#1#2#3#4{
	\dimen100=#1 bp\edef\@p@sbbllx{\number\dimen100}
	\dimen100=#2 bp\edef\@p@sbblly{\number\dimen100}
	\dimen100=#3 bp\edef\@p@sbburx{\number\dimen100}
	\dimen100=#4 bp\edef\@p@sbbury{\number\dimen100}
	\no@bbfalse
}
% rotate point (#1,#2) about (0,0).
% The sine and cosine of the angle are already stored in \sine and
% \cosine.  The result is placed in (\p@intvaluex, \p@intvaluey).
\newdimen\p@intvaluex
\newdimen\p@intvaluey
\def\rotate@#1#2{{\dimen0=#1 sp\dimen1=#2 sp
%            	calculate x' = x \cos\theta - y \sin\theta
		  \global\p@intvaluex=\cosine\dimen0
		  \dimen3=\sine\dimen1
		  \global\advance\p@intvaluex by -\dimen3
% 		calculate y' = x \sin\theta + y \cos\theta
		  \global\p@intvaluey=\sine\dimen0
		  \dimen3=\cosine\dimen1
		  \global\advance\p@intvaluey by \dimen3
		  }}
\def\compute@bb{
		\no@bbfalse
		\if@bbllx \else \no@bbtrue \fi
		\if@bblly \else \no@bbtrue \fi
		\if@bburx \else \no@bbtrue \fi
		\if@bbury \else \no@bbtrue \fi
		\ifno@bb \bb@missing \fi
		\ifno@bb \ps@typeout{FATAL ERROR: no bb supplied or found}
			\no-bb-error
		\fi
		%
%\ps@typeout{BB: \@p@sbbllx, \@p@sbblly, \@p@sbburx, \@p@sbbury} 
%
% store height/width of original (unrotated) bounding box
		\count203=\@p@sbburx
		\count204=\@p@sbbury
		\advance\count203 by -\@p@sbbllx
		\advance\count204 by -\@p@sbblly
		\edef\ps@bbw{\number\count203}
		\edef\ps@bbh{\number\count204}
		%\ps@typeout{ psbbh = \ps@bbh, psbbw = \ps@bbw }
		\if@angle 
			\Sine{\@p@sangle}\Cosine{\@p@sangle}
	        	{\dimen100=\maxdimen\xdef\r@p@sbbllx{\number\dimen100}
					    \xdef\r@p@sbblly{\number\dimen100}
			                    \xdef\r@p@sbburx{-\number\dimen100}
					    \xdef\r@p@sbbury{-\number\dimen100}}
%
% Need to rotate all four points and take the X-Y extremes of the new
% points as the new bounding box.
                        \def\minmaxtest{
			   \ifnum\number\p@intvaluex<\r@p@sbbllx
			      \xdef\r@p@sbbllx{\number\p@intvaluex}\fi
			   \ifnum\number\p@intvaluex>\r@p@sbburx
			      \xdef\r@p@sbburx{\number\p@intvaluex}\fi
			   \ifnum\number\p@intvaluey<\r@p@sbblly
			      \xdef\r@p@sbblly{\number\p@intvaluey}\fi
			   \ifnum\number\p@intvaluey>\r@p@sbbury
			      \xdef\r@p@sbbury{\number\p@intvaluey}\fi
			   }
%			lower left
			\rotate@{\@p@sbbllx}{\@p@sbblly}
			\minmaxtest
%			upper left
			\rotate@{\@p@sbbllx}{\@p@sbbury}
			\minmaxtest
%			lower right
			\rotate@{\@p@sbburx}{\@p@sbblly}
			\minmaxtest
%			upper right
			\rotate@{\@p@sbburx}{\@p@sbbury}
			\minmaxtest
			\edef\@p@sbbllx{\r@p@sbbllx}\edef\@p@sbblly{\r@p@sbblly}
			\edef\@p@sbburx{\r@p@sbburx}\edef\@p@sbbury{\r@p@sbbury}
%\ps@typeout{rotated BB: \r@p@sbbllx, \r@p@sbblly, \r@p@sbburx, \r@p@sbbury}
		\fi
		\count203=\@p@sbburx
		\count204=\@p@sbbury
		\advance\count203 by -\@p@sbbllx
		\advance\count204 by -\@p@sbblly
		\edef\@bbw{\number\count203}
		\edef\@bbh{\number\count204}
		%\ps@typeout{ bbh = \@bbh, bbw = \@bbw }
}
%
% \in@hundreds performs #1 * (#2 / #3) correct to the hundreds,
%	then leaves the result in @result
%
\def\in@hundreds#1#2#3{\count240=#2 \count241=#3
		     \count100=\count240	% 100 is first digit #2/#3
		     \divide\count100 by \count241
		     \count101=\count100
		     \multiply\count101 by \count241
		     \advance\count240 by -\count101
		     \multiply\count240 by 10
		     \count101=\count240	%101 is second digit of #2/#3
		     \divide\count101 by \count241
		     \count102=\count101
		     \multiply\count102 by \count241
		     \advance\count240 by -\count102
		     \multiply\count240 by 10
		     \count102=\count240	% 102 is the third digit
		     \divide\count102 by \count241
		     \count200=#1\count205=0
		     \count201=\count200
			\multiply\count201 by \count100
		 	\advance\count205 by \count201
		     \count201=\count200
			\divide\count201 by 10
			\multiply\count201 by \count101
			\advance\count205 by \count201
			%
		     \count201=\count200
			\divide\count201 by 100
			\multiply\count201 by \count102
			\advance\count205 by \count201
			%
		     \edef\@result{\number\count205}
}
\def\compute@wfromh{
		% computing : width = height * (bbw / bbh)
		\in@hundreds{\@p@sheight}{\@bbw}{\@bbh}
		%\ps@typeout{ \@p@sheight * \@bbw / \@bbh, = \@result }
		\edef\@p@swidth{\@result}
		%\ps@typeout{w from h: width is \@p@swidth}
}
\def\compute@hfromw{
		% computing : height = width * (bbh / bbw)
	        \in@hundreds{\@p@swidth}{\@bbh}{\@bbw}
		%\ps@typeout{ \@p@swidth * \@bbh / \@bbw = \@result }
		\edef\@p@sheight{\@result}
		%\ps@typeout{h from w : height is \@p@sheight}
}
\def\compute@handw{
		\if@height 
			\if@width
			\else
				\compute@wfromh
			\fi
		\else 
			\if@width
				\compute@hfromw
			\else
				\edef\@p@sheight{\@bbh}
				\edef\@p@swidth{\@bbw}
			\fi
		\fi
}
\def\compute@resv{
		\if@rheight \else \edef\@p@srheight{\@p@sheight} \fi
		\if@rwidth \else \edef\@p@srwidth{\@p@swidth} \fi
		%\ps@typeout{rheight = \@p@srheight, rwidth = \@p@srwidth}
}
%		
% Compute any missing values
\def\compute@sizes{
	\compute@bb
	\if@scalefirst\if@angle
% at this point the bounding box has been adjsuted correctly for
% rotation.  PSFIG does all of its scaling using \@bbh and \@bbw.  If
% a width= or height= was specified along with \psscalefirst, then the
% width=/height= value needs to be adjusted to match the new (rotated)
% bounding box size (specifed in \@bbw and \@bbh).
%    \ps@bbw       width=
%    -------  =  ---------- 
%    \@bbw       new width=
% so `new width=' = (width= * \@bbw) / \ps@bbw; where \ps@bbw is the
% width of the original (unrotated) bounding box.
	\if@width
	   \in@hundreds{\@p@swidth}{\@bbw}{\ps@bbw}
	   \edef\@p@swidth{\@result}
	\fi
	\if@height
	   \in@hundreds{\@p@sheight}{\@bbh}{\ps@bbh}
	   \edef\@p@sheight{\@result}
	\fi
	\fi\fi
	\compute@handw
	\compute@resv}

%
% \psfig
% usage : \psfig{file=, height=, width=, bbllx=, bblly=, bburx=, bbury=,
%			rheight=, rwidth=, clip=}
%
% "clip=" is a switch and takes no value, but the `=' must be present.
\def\psfig#1{\vbox {
	% do a zero width hard space so that a single
	% \psfig in a centering enviornment will behave nicely
	%{\setbox0=\hbox{\ }\ \hskip-\wd0}
	%
	\ps@init@parms
	\parse@ps@parms{#1}
	\compute@sizes
	%
	\ifnum\@p@scost<\@psdraft{
		%
		\special{ps::[begin] 	\@p@swidth \space \@p@sheight \space
				\@p@sbbllx \space \@p@sbblly \space
				\@p@sbburx \space \@p@sbbury \space
				startTexFig \space }
		\if@angle
			\special {ps:: \@p@sangle \space rotate \space} 
		\fi
		\if@clip{
			\if@verbose{
				\ps@typeout{(clip)}
			}\fi
			\special{ps:: doclip \space }
		}\fi
		\if@prologfile
		    \special{ps: plotfile \@prologfileval \space } \fi
		\if@decmpr{
			\if@verbose{
				\ps@typeout{psfig: including \@p@sfile.Z \space }
			}\fi
			\special{ps: plotfile "`zcat \@p@sfile.Z" \space }
		}\else{
			\if@verbose{
				\ps@typeout{psfig: including \@p@sfile \space }
			}\fi
			\special{ps: plotfile \@p@sfile \space }
		}\fi
		\if@postlogfile
		    \special{ps: plotfile \@postlogfileval \space } \fi
		\special{ps::[end] endTexFig \space }
		% Create the vbox to reserve the space for the figure.
		\vbox to \@p@srheight sp{
		% 1/92 TJD Changed from "true sp" to "sp" for magnification.
			\hbox to \@p@srwidth sp{
				\hss
			}
		\vss
		}
	}\else{
		% draft figure, just reserve the space and print the
		% path name.
		\if@draftbox{		
			% Verbose draft: print file name in box
			\hbox{\frame{\vbox to \@p@srheight sp{
			\vss
			\hbox to \@p@srwidth sp{ \hss \@p@sfile \hss }
			\vss
			}}}
		}\else{
			% Non-verbose draft
			\vbox to \@p@srheight sp{
			\vss
			\hbox to \@p@srwidth sp{\hss}
			\vss
			}
		}\fi	



	}\fi
}}
\psfigRestoreAt
\let\@=\LaTeXAtSign





\pagestyle{empty}
\thispagestyle{empty}

\sloppy
\clubpenalty = -1000
\widowpenalty = -1000
\setlength{\baselineskip}{11.5pt}
\begin{document}
\setlength{\baselineskip}{11.5pt}
\pagestyle{empty}
\thispagestyle{empty}


\makeieeetitle
  {SUPPORTING TECHNOLOGY TRANSFER OF FORMAL TECHNICAL REVIEW \\
   THROUGH A COMPUTER SUPPORTED COLLABORATIVE REVIEW SYSTEM}
  {Philip M. Johnson\\
   Department of Information and Computer Sciences\\
   University of Hawaii\\
   Honolulu, HI 96822\\
   (808) 956-3489\\
   {\tt johnson@hawaii.edu}}

   
   \makeieeeabstract 
   {

   Formal technical review (FTR) is an essential component of all
   modern software quality assessment, assurance, and improvement techniques,
   and is acknowledged to be the most cost-effective form of quality
   improvement when practiced effectively.  However, traditional FTR
   methods such as inspection are very difficult to adopt
   in organizations: they introduce substantial new up-front
   costs, training, overhead, and group process obstacles.  Sustained
   commitment from high-level management along with substantial
   resources is often necessary for successful technology transfer of
   FTR.
  
   Since 1991, we have been designing and evaluating a series of
   versions of a system called CSRS: an instrumented, computer-supported
   cooperative work environment for formal technical review.  The
   current version of CSRS includes an FTR method definition language,
   which allows organizations to design their own FTR method, and to
   evolve it over time. This paper describes how our approach to
   computer supported FTR can address some of the issues in technology
   transfer of FTR.

   }

\pagestyle{empty}
\thispagestyle{empty}

\section{Introduction}

Among all the software quality improvement methods currently known, formal
technical review 
%
(FTR\foot{We define {\em formal technical review} as ``a structured
encounter where a group of technical personnel analyzes an artifact to
improve quality.  The analysis produces a structured artifact that assesses
or improves the quality of the artifact as well as the quality of the
method.'' This definition includes methods such as Fagan's code inspection
\cite{Fagan76,Fagan86}, Phased Inspections \cite{Knight93}, and FTArm
(discussed here), but excludes methods such as informal peer reviews and
walkthroughs.})
%
enjoys unique advantages.  Some studies provide evidence that FTR can be
more effective at discovering errors than testing, while others indicate
that it can discover different classes of errors than testing
\cite{Myers78,Basili86}.  In concert with other process improvements,
Fujitsu found FTR to be so effective at discovering errors that they
dropped system testing from their software development procedure
\cite{Arthur93}.  FTR forms an essential part of methods and models for
very high quality software, such as Cleanroom Software Engineering
\cite{Linger93} and the SEI Capability Maturity Model \cite{Paulk93a}.
Finally, FTR displays a unique ability to improve the quality of the
producer as well as the quality of the product by dispersing knowledge
about applications and development skills across the organization.

Given the range of advantages ascribed to FTR, and the substantial
improvements in quality and cost-reductions attributed to it by some
organizations, it is curious that formal technical review is not ubiquitous
in modern software development.  Although rigorous data on industrial use
of FTR is not publically available, responses by 70 participants to an
informal survey we conducted on FTR via USENET revealed that FTR is
 practiced irregularly or not at all in over 80\% of the surveyed
organizations.  Similar non-rigorous evidence for a low level of FTR
adoption in industry is discussed in \cite{Brykczynski94}.

Since 1991, we have been designing and experimentally evaluating a
computer-supported cooperative work environment for FTR called CSRS
\cite{Johnson94,Johnson93,Johnson93b}.  One product of this research was
the creation of a new, highly instrumented, asynchronous review method
called FTArm that addresses a multiplicity of problems arising in the
research on and practice of traditional FTR.  As we began discussing
technology transfer of CSRS and FTArm with industrial organizations, we
became aware of a spectrum of organizational issues surrounding the
technology transfer of FTR in general and CSRS in particular that must also
be addressed.

These issues and others motivated a recent redesign of CSRS to provide a
specialized process modelling language for FTR.  The language is intended
to allow organizations to design their own FTR method for use with CSRS, and
to support incremental evolution in the method as the organization's
needs for and use of FTR changes.

In the next section, we present some of the problems involved in successful
technology transfer and adoption of FTR.  The following section briefly
overviews the CSRS system and the FTArm method.  Following this we discuss
how the process modelling facilities of FTArm can be applied to address
some of the problems that arise in technology transfer and adoption of FTR.

\section{Issues in FTR Technology Transfer}

\subsection{The transfer process}

Our model of technology transfer follows research which does not view it as
a ``transfer'' at all, but rather as a {\em reconstruction} by one
organization of knowledge, expertise, and technology generated by another
\cite{Doheny-Farina92}.  This contrasts with the conventional view of
technology transfer, in which the technology is viewed as a relatively
static object whose successful transfer induces a change in the receiving
organization without impacting upon the technology itself.  When
participants in the transfer process interpret the technology differently,
the conventional view holds that they are either misperceiving the
technology or the technology has been somehow distorted.

The conventional view appears occasionally in the literature on industrial
use of FTR methods, such as Fagan's code inspection. Adoption failures are
here interpreted as either a misperception of the meaning of the method or
a failure to implement all parts of the method.  Successful technology
transfer, from the perspective of this literature, is simply a matter of
total adherence and commitment to a single approach to formal technical
review.

Other FTR literature describes a more context-sensitive and
reconstructionist view of the adoption process.  For example, a study of
FTR technology transfer at Hewlett-Packard reveals that FTR adoption goes
through a series of stages and that blind adherence to a single
standardized process is a recipe for failure, not success \cite{Grady94}.
The four stages observed at Hewlett-Packard are described in this study in
the following way:

\begin{itemizenoindent}
\item {\em Experimental.}  This stage is characterized by trial adoption
  of a not well understood technique by a few groups within the
  organization with relatively little institutional support.  Surviving
  the experimental stage of technology adoption appears to depend upon:
  (a) visionary people who can look at tools and process from another
  context and see how they can be applied locally; (b) management support
  for visionary attempts without penalty for failure; and (c) a
  supportive infrastructure, since mistakes and failures will occur and
  early success is very fragile from an organizational standpoint.

\item {\em Initial Guidelines.}  Progression out of the experimental
  phase is marked by the appearance of training classes and educational
  materials on the technique, and the creation of small-scale
  infrastructure within the organization to promote the technology.
  However, the Hewlett-Packard researchers caution that readily available training is
  a necessary but not sufficient condition for technology dispersion.
  For the technology to become further incorporated into the
  organization, effort must be made to communicate success with the
  method throughout the organization, through activities such as
  newsletters, conferences, and so forth. In addition, high-level
  management must be educated in the evolving ``best practice'' of the
  method and they must continue to display commitment and allocate
  resources to the technology

\item {\em Widespread Belief and Adoption.} This stage is characterized
  by widespread acceptance within the organization that the technology is
  useful and important to the organization's success.  However, such
  ``widespread belief'' does not translate automatically into optimal or
  even effective use of the technology, and may even sow the seeds of the
  technology transfer's destruction. 

  One potential problem at this stage is that management may become
  convinced that there is ``one best way'' and begin pushing for its
  total and exclusive adoption.  Hewlett-Packard found that their internal divisions
  resisted this, prefering a consulting approach whereby corporate
  resources were applied to understanding the specific context and
  problems of a division, and then developing an individualized strategy
  to help the group improve their current practice.

  A second potential problem is that as the use of the technology spreads
  across the organization, the aggregate cost of the technology to the
  organization becomes increasingly substantial and significant. An
  effective business case must be created to ensure that the technology
  continues to be used beyond a trial period. Otherwise management may
  decide that the technology, though promising, is not cost-effective
  when scaled to the organizational level.

\item {\em Standardization.}  While Hewlett-Packard has not yet progressed beyond the
  previous stage, their researchers suggest that there is a phase beyond
  it.  This phase appears to be characterized by total integration of the
  technology into the organization, such that questions of
  appropriateness are no longer asked---the technology has become part of
  what makes the organization what it is. The HP researchers explicitly
  note that terming this stage ``standardization'' does not imply
  adherence by the entire company to a single process, but rather that
  every project would use some form of FTR technology in an efficient,
  cost-effective manner.

\end{itemizenoindent}

\subsection{Obstacles to FTR adoption}

Many studies assert that an FTR such as Fagan's inspection is
cost-effective and improves software quality, once successfully adopted and
when practiced effectively. However, it is also clear from studies that FTR
is difficult to adopt and practice effectively.  The following obstacles to
effective FTR adoption and use is drawn from
\cite{Basili94,Brykczynski94,Russell91}:

\begin{itemizenoindent}

\item {\em Low Technology.}  FTR typically involves a lot of
  ``meticulous, pain-staking, manual work''.  Software developers, used
  to e-mail and on-line discussions, may resist returning to hand-written
  notes and extensive meetings with high clerical and administrative
  overhead.

\item {\em Ambiguity in Data/Process Model.}  Without proper training,
  manual FTR methods are easy to misinterpret or misapply in practice.
  Successful introduction of a method requires training to impart a
  precise understanding of the process to be followed, since different
  approaches may have widely varying benefits.
  
  Ambiguity in the FTR method may also lead management to block
  introduction of a new method based upon the mistaken notion that ``we've
  tried this before and it didn't work.''

\item {\em Absolute Expense.}  The cost of developing manual FTR
  infrastructure (planning, training, developing forms) and the cost of
  performing FTR (preparation, meetings, filling out forms, data analysis)
  is substantial.  At Bell-Northern Research, one person-year of effort was
  expended for each 20 KLOC under FTR, and this introduced
  15-25\% new overhead to the development process. (These upstream costs
  were recovered downstream during testing and maintenance.) 

\item {\em Relative Expense.}  An organization which already has an
  informal review method in place may not feel that the additional benefits
  will justify the additional cost.

\item {\em Demand for proof.}  Introduction of FTR requires approval from
  management who will want to see evidence that the process is worth the
  investment.  However, management may frequently reject evidence from
  published reports as not relevent to their organization, and collecting
  statistically meaningful in-house data is impossible until inspections
  have actually been adopted. 
%  Such a catch-22 seems to require a
%  decisive management willing to act as a champion and gamble that the
%  investment will pay off.

\item {\em Developer inertia.}  Adopting FTR requires convincing
  management that the organization will benefit. However, adopting FTR
  also requires convincing developers that their professional quality of
  life will improve. FTR is often resisted by developers, who see it as 
  yet one more hurdle placed between them and successful discharge of their
  responsibilities. 

\item {\em Training.}  Successful adoption of FTR typically requires
  extensive training. For example, Bell-Northern Research developed a
  self-study video course involving examples of both effective and
  ineffective inspection meetings, the meeting process, team roles, error
  reporting, planning guidelines, and paraphrasing, along with an example
  program for practice.  Hewlett-Packard provided both initial
  training and a ``continuing education'' program to keep participants
  abreast of changes and improvements to their FTR program.

\item {\em Fire fighting.}  If a project is already having process
  problems, then the group may be resistant to the introduction of any new process
  hurdles.  Moreover, if the project is having extensive process
  problems, then FTR may not the most important way for the team to
  expend their time and resources.  The Capability Maturity Model, for
  example, only mentions peer reviews starting at Level 3, after more basic
  project management mechanisms have been put into practice. 

\item {\em Improved quality not beneficial to bottom line.}  Quality in a
  certain product may be desired but is traded off against other goals
  (such as profit, schedule, etc.)  Adoption of FTR may be resisted
  because it is viewed as improving quality at the expense of other goals
  more important to management.

\item {\em Perceived long-term inefficiency during maintenance.}  The
  cost of FTR during maintenance becomes particularly high, since a
  change to a unit requires a complete re-inspection with almost no
  savings from previous inspections.  In contrast, re-testing a changed
  unit is virtually free since it simply requires re-running tests.  In
  maintenance-heavy contexts, management may view development of
  regression test suites as more cost-effective than FTR.


\item {\em Possible ineffectiveness during maintenance.}  During
  maintenance, change to one part of the system may have a ripple effect
  that causes a fault in a different part of the system. This kind of
  fault is extremely difficult to detect using FTR, unless the
  organization is willing to allow a single change to the system to
  trigger a massive re-inspection of all potentially affected parts.
  Again, testing may be viewed as more cost-effective, since testing is
  potentially capable of detecting such interactions.

\end{itemizenoindent}

Having now identified some of the major issues in technology transfer of
FTR, we now overview CSRS, our computer supported environment for FTR.
Following this introduction, we will describe how the process modelling
language for FTR in CSRS can be used to facilitate the technology transfer
process. 

\section{CSRS: Computer Supported FTR Definition and Enactment}
\label{sec:csrs}

CSRS is a multi-user, interactive hypertext environment for performing FTR,
implemented using the Egret collaborative work environment
\cite{csdl-92-01}.  Egret has a client-server architecture, where a
database back-end server Unix process written in C++ communicates over
TCP/IP to X window client processes implemented using a customized version
of Lucid Emacs.

Egret is designed to support collaborative systems containing a mixture of
interactive ``user'' processes directly controlled by people and autonomous
``agent'' processes that provide computational services.  For example, many
CSRS methods include a mailer agent that wakes up once or twice a day,
inspects the state of review, and sends an e-mail message to participants
notifying them of any new activities for them to perform.

Just as Egret is a generic framework for collaborative group work, CSRS is
a generic framework for computer-supported FTR.  The implementation of this
generic framework involves a set of language constructs for defining a
computer-mediated FTR method.  To illustrate the capabilities of CSRS, the
next section overviews FTArm, one of the FTR methods that can be defined
using CSRS.

\subsection{The FTArm Method}

FTArm is a computer-mediated FTR method designed to leverage off the
strengths of an on-line environment to address certain problems of manual
FTR.  The FTArm process consists of seven phases where participants
interact within the roles of moderator, producer, reviewer, or
administrator.  The FTArm method is not specific to a review artifact type
or development phase.

\paragraph{Setup.} In this phase, the moderator and/or the producer
decide upon the composition of the review team and the artifacts to be
reviewed. The moderator or producer then restructures the review artifact
into a multi-node, interlinked hypertext document stored within the CSRS
database.  Regular expression-based parsing tools available in CSRS can
partially or fully automate this database entry and restructuring process.

\paragraph{Orientation.} This phase prepares the participants for the
private review phase through an overview of the review artifacts.  The
exact nature of this overview depends upon the complexity of the
review artifact and the familiarity of the reviewers with it, and can range
from a simple e-mail message to a formal, face-to-face meeting.

\paragraph {Private review.} In this phase, reviewers analyze the
review artifact nodes (termed ``source'' nodes) privately and create issue,
action and/or comment nodes.  Issue and action nodes are not publicly
available to other reviewers, though comment nodes are publicly available.
Comment nodes allow reviewers to request clarification about the
logic/algorithm of source nodes, or about the review process, and may also
contain answers to these questions by other participants.

Figure \ref{fig:issue-screen} contains a snapshot of one reviewer's
screen during the private review phase.  The function {\sf\small
t*node-schema!combine-field-IDs} is the review artifact under
analysis, as displayed in the left hand window.  A checklist of defect
classifications appears in the upper right window, while a defect
concerning this function is being documented in the lower right
window.

\begin{figure*} [t]
 {\centerline{\psfig{figure=/group/csdl/techreports/93-22/issue-private.ps}}}
\caption{{\em A CSRS screen illustrating the generation of an issue.}}
\label{fig:issue-screen}
\end{figure*}

In FTArm, reviewers must explicitly mark each source node as reviewed
when finished. While reviewers do not have access to each other's
state during private review, the moderator does.  This allows the
moderator to monitor the progress of private review.  Private review
normally terminates when all reviewers have marked all source nodes as
reviewed. In the event that no reviewer has created any issues, review
would terminate at this point.  Otherwise, public
review begins.

\paragraph{Public review.} In this phase, all nodes are now
accessable to reviewers, and all participants (including the producer)
react to the issues and actions by voting (a modified Delphi process).
Participants can also create new issue, action or comment nodes based upon
the votes or nodes of others.  Voting is used to determine the degree of
agreement within the group about the validity and implications of issues
and actions.  Public review normally concludes when all nodes have been read
by all reviewers, and when voting has stabilized on all issues.

\paragraph{Consolidation.} In this phase, the moderator analyzes the results of
public and private review, and produces a condensed written report of the
review thus far.  These consolidated reports are more
comprehensive, detailed, and accurate than typical review reports from
traditional review methods. Rather than simply a checklist of
characteristics with brief comments about the general quality of the
source, consolidation reports contain a re-organized and condensed
presentation of the analyses provided by reviewers in issue, action, and
comment nodes, thus providing contrasting opinions, the degree of
consensus, and proposals for changes.

CSRS provides the moderator with various tools to support the generation of
a nicely formatted LaTeX document containing the consolidated report.  If
the group reached consensus about all of the issues and actions during
public review, then this report presents the review outcome with respect to
artifact assessment.  A second review outcome is detailed and accurate
measurements of review outcome and process.
 
\paragraph {Group review meeting.} If the consolidated report 
identifies issues or actions that were not successfully resolved via
public and private review, the FTArm method requires a face-to-face,
group meeting as the final phase.  Here the moderator presents only
the unresolved issues or actions and summarizes the differences of
opinion.  After discussion, the group may vote to decide them, or the
moderator may unilaterally make the decision. The moderator then
updates the CSRS database, noting the decisions reached during the
group meeting and then generating a final consolidated report
representing the product of review.

\paragraph{Process Improvement Meta-Phase.}
The preceding phases provide a framework for the FTR process, but also
allow for evolution in response to various measurements automatically
provided by CSRS.  The system automatically generates a timestamped log of
the sequence of nodes visited and links traversed by participants during
review. CSRS analysis tools use this data to provide useful process
measurements, such as the number of minutes spent by each reviewer on each
source artifact, the number of issues raised per minute of review time, the
review strategy employed by participants, the level of consensus in the
review, and so forth.  This data can be used to improve such method
variables as: artifact size and complexity, review team size and
composition, private review analysis technique, review checklist
composition (if checklists are used), public review scope and duration,
individual and team effort, and scope and duration of group review.

\subsection{Process modelling in CSRS}

As the preceeding description illustrates, FTR in CSRS consists of
structured interactions between a group of people with well-defined roles.
These interactions are divided into a sequence of phases, during which they
analyze source artifacts using various analysis tools and artifacts,
resulting in the production of review artifacts.  The process is measured
and the measurement data is analyzed in various ways to provide insight
into the process and products of review.    Each of these fundamental
characteristics of an FTR method can be customized in CSRS using its
process modelling language:

\begin{itemizenoindent}

\item {\em Method definition.}  CSRS provides two language constructs
  called {\sf\small define-method} and {\sf\small define-phase} that allow method
  designers to implement a new method as a sequence of phases.  CSRS does
  not currently support iteration or conditionals in phase sequencing, in
  keeping with manual FTR techniques. (However, iterative or conditional
  activities may occur within a phase).  Phases can have entry and exit
  conditions associated with them to determine when it is possible to
  transition from one phase to the next.

  Each phase also has a distinct set of operations associated with it, as
  specified using the {\sf\small define-operation} construct.  Each operation
  can be made highly context-sensitive to the current state of review and
  the current role of the participant.

\item {\em User definition.}  CSRS allows specification of review method
  roles through the {\sf\small define-role} construct, and the actual people
  involved in a particular review through the {\sf\small define-participant}
  construct.  A given participant can play multiple roles during review.
  There is one hard-wired role called Administrator that is present by
  default in all review methods, and which is used to bootstrap the
  system.  Roles found in other methods can be defined in CSRS, such as
  moderator, producer, reviewer, and reader.  

\item {\em Artifact definition.}  CSRS provides constructs such as {\sf\small
  define-node-schema}, {\sf\small define-field-schema}, and {\sf\small
  define-link-schema} to allow construction of the type-level
  characteristics of the artifacts manipulated during review.  For
  example, a requirements review method might provide nodes or fields of
  type overview, product-functions, functional-requirements,
  external-interface, performance-constraints, hardware-limitations, and
  so forth. A C++ code review method might provide nodes or fields of
  type class-declaration, member-function, member-variable, private-part,
  public-part, protected-part, template, and so forth.

  The {\sf\small define-checklist} and {\sf\small define-checkitem} constructs allow
  specification of materials used during analysis and their properties,
  such as whether the reviewers are required to explicitly mark an item as
  satisfied by the artifact or not.

\item {\em Measurement definition.}  In the current version of CSRS,
  method designers have only binary control over the generation of the
  timestamped log file containing fine-grained raw data on the activities
  of reviewers: either fine-grained raw data is collected or it is not.
  However, designers have a great deal of control over how this data is
  analyzed, and whether the identities of reviewers are revealed in the
  analysis, whether only aggregate group statistics are generated, and so
  forth.  CSRS provides several predefined analysis tools for use by designers
  which generate spreadsheet-compatible data files on the review process
  and outcomes. 

\end{itemizenoindent}

This description covers approximately half of the language constructs, and
leaves out entirely those constructs concerned with user interface features
of the method, such as the layout of screens and menus.

While the language provides substantial support for definition of FTR
methods, it does not trivialize it. For example, the definition of FTArm in
the process modelling language requires over 100 construct invocations and
over 1000 lines of code. The language contains a mixture of declarative and
procedural specifications of behavior, and certain behaviors (such as the
agents) must be specified in terms of underlying Egret primitives.
However, this definition of FTArm represents less than 3\% of the total
size of the Egret/CSRS system, so substantial reuse of code and development
effort is achieved.


\section{Technology transfer using CSRS}

As mentioned previously, when we began discussing technology transfer of a
previous version of CSRS (that did not include a process modelling
language, but which did include FTArm), we ran into a variety of adoption
problems.  

First, managers were concerned that FTArm, while interesting, was
either too complicated a method to implement initially or too different
from their current FTR practice, and would cause significant organizational
perturbations.  

Second, developers were concerned about the detailed data to be
gathered: the idea that their activities were being so precisely monitored
caused repeated reference to the ``Big Brother'' \cite{Orwell84} nature of
the system, with its attendant possibility of management abuse.

Finally, there was a class of problems raised by both managers and
developers concerning the use of a computer-supported cooperative work
system to replace a manual process.  Such ``groupware technology transfer''
problems are significant.  The challenges of successful groupware adoption
in comparison to single-user, off-the-shelf applications has been described
as follows:
\begin{quotation}
A word processor that is immediately liked by one in five prospective
customers and disliked by the rest could be a big success.  A groupware
application to support teams of five nurses that initially appeals to only
one nurse in five is a big disaster.  \cite{Grudin94}
\end{quotation}
Interestingly, there is a great deal of overlap between the technology
transfer problems of groupware and that of FTR. An FTR method that
initially appeals to only one developer in five is also a big
disaster. 

From our discussions, we concluded that both managers and developers felt
that computer supported FTR can significantly improve their software
quality improvement practice, but that they require more control over the
FTR method enacted by the environment.  To provide this control, we
redesigned CSRS to provide the process modelling language described above.
We now illustrate how a transfer process based upon incremental evolution
in the FTR method can help support adoption of FTR in general and CSRS in
particular within an organization.  


\subsection{Experimental Phase}

The initial, experimental phase of technology transfer is particularly
fragile and prone to failure.  During this phase, the process model enacted
by CSRS must be simple, training should be minimal, and the user interface
should be simple and intuitive.  In addition, the use of a
computer-supported tool should provide both managers and developers with
significant benefits.  An immediate advantage that CSRS has over
traditional FTR methods is that it is a ``high-tech'', on-line system,
which eliminates the vast majority of clerical, manual activities formerly
associated with review.

The most appropriate initial review method to enact within CSRS during the
experimental phase always depends upon specific organizational factors, but
perhaps the most important one is the organization's prior experience with
review. 

\subsubsection{No Prior FTR Experience} 

A common situation is one in which the organization recognizes a need to
perform FTR, but has very little experience with it.  In this situation,
two technology transfers must occur: transfer of FTR into the
organizational process, and transfer of CSRS as a technology for enactment
of FTR.

Some of the obstacles to FTR adoption in organizations new to review are
fears that FTR will take too much time away from coding, that it will be
too expensive, and that developers will have difficulties adjusting to the
increased visibility of their intermediate work products.

To explore how CSRS can support FTR adoption in this situation, we designed
an FTR method called ``Hello-World'' (HW).  The HW method consists of a
single review phase in which all participants scan the review artifact and
generate anonymous comments to raise and react to issues.  Once all
reviewers have scanned the artifact, they attend a meeting to decide upon
the validity of the issues.  

The HW method provides only three automated services:
\begin{enumerate}

\item HW provides a mailer agent that runs once a night during review. It
sends a daily ``CSRS News'' e-mail message to each reviewer as long as
work remains to be done by the reviewer---either remaining review artifacts
or new comment nodes that the reviewer has not yet seen.

\item HW provides a hard-copy mechanism to provide a nicely formatted
version of the review artifact and all the comments generated about it.
This artifact also provides checkboxes for each issue to log the decision
of the participants at the group meeting about its validity.

\item HW provides coarse, anonymous statistics: the total number of minutes
users were on-line in the system, and the total number of issues raised.
\end{enumerate}


The goal of the HW method is to demonstrate to an organization new to
review that FTR can be cost-effective, that review need not be excessively
time-consuming or tedious, and that making work products visible in this
manner can be positive for both the producer and the reviewers.  By making
all reviews public immediately, it stimulates activity in the system since
there will be frequently a new comment to look at.  Allowing reviewers to
see comments in advance can reduce the chances of long or unproductive
meetings, and automatic hard-copy generation and statistics keeps clerical
overhead to a minimum.  For management, HW provides simple statistics which
can show some initial data on the ability of FTR to detect faults and the
effort required to do so.

The HW method, while technically a formal technical review method, is still
relatively unstructured, and does not allow much of the sophisticated
process control and analysis supported in a method like FTArm.  However, it
appears to be much better suited than a method like FTArm for the initial
buy-in phase of FTR adoption for organizations with no prior FTR
experience.

\subsubsection{Prior FTR Experience} 

The strategy for organizations with prior experience in FTR is quite
different.  In this case, it is important to assess the current state of
FTR within the organization and tailor a CSRS method in response. 

For example, an organization may have adopted a traditional
inspection-based process and may have experienced some success with it, but
is still in the ``experimental'' stage.  This could be result from
inconsistent application of the method resulting in inconsistent outcomes.
It could also result from a lack of resources to collect data with which to
publicize the successes of the method.  

In this case, it might be appropriate to design a CSRS-based method that
closely parallels the current method in place.  Several advantages result
from this strategy.  First, it reduces training, since developers can be
told that the CSRS FTR method is ``just like'' the one they are now using,
except for a small set of differences.  Second, the CSRS method can reduce
variation in the practice of the review method, which may lead to more
consistent outcomes.  Finally, a CSRS-based system can automatically
collect and analyze process and outcome measurements, allowing groups with
limited resources to publicize detailed and accurate data about their
experiences. Such data can help obtain management commitment to FTR.

CSRS can also apply to organizations that have tried but failed to adopt
traditional FTR methods. For example, one organization we work with
requires review of change-request documents by at least a dozen personnel
from different departments, and sometimes as many as twenty or thirty.  All
traditional formal technical review methods of which we are aware simply
view this as an ``error'': it is simply not possible to perform FTR with
more than 6-8 people.  Unfortunately for traditional FTR methods, the
reality in this organization is that the change-request document {\em must}
be reviewed by large numbers of people.  Although an ingenious, email-based
approach was developed in this organization to support FTR, we believe that
many organizations in this situation would simply abandon FTR as
inappropriate.

In such situations, CSRS may provide a means to expand the boundaries of
FTR application.  An on-line system such as CSRS can easily support reviews
involving 40 or 50 reviewers who may be geographically dispersed and unable
to attend face-to-face meetings.  Unlike email, CSRS can provide a more
structured process as well as accurate and precise measurements of process
and outcome.


\subsection{Later phases}

We expect that technology transfer of CSRS-based FTR to an organization
will progress in much the same way as the four stage model used to
characterize the Hewlett-Packard adoption process.  Progression beyond
experimental usage involves the creation of small-scale infrastructure
within the organization to further spread the technology through training.
With sufficient time, the technology could become widely adopted and
eventually become a standard part of the organization's quality improvement
practice.

A significant difference between CSRS-based FTR technology transfer and
traditional practice is the focus upon explicit evolution in FTR method as
an active part of the technology transfer process.  This contrasts to a
``big bang'' approach, in which a single, full-blown method is taught to
groups and instituted in its total, mature format.  From our prior
experiences discussing technology transfer with organizations, we believe
that successful adoption of CSRS must be performed incrementally.  The
initial method must be modest in scope, but still cost-effective.  As
the organization becomes more used to a computer-supported method, ideas
for more sophisticated services and process details will arise naturally as
the organization assesses its practice.  This evolutionary process of 
method refinement and improvement reifies the reconstructive nature of 
technology transfer using CSRS. 

\section{Conclusions}

This paper presents an approach to technology transfer of formal technical
review that is based upon an on-line, interactive environment and evolution
in the method as the nature of FTR within the organization matures over
time.

The approach can address several of the obstacles identified above.  It 
overcomes the ``low technology'' obstacle with a computer-supported
approach to eliminate much of the clerical overhead involved in traditional
FTR.  It reduces ambiguity in method application, leading to clearer
relationships between method and outcome.  It reduces several cost factors
for FTR.  In particular, it automates collection and analysis of several
common review metrics, allowing groups to quickly generate in-house data at
with little additional overhead.  By appropriate method definition, it can
help overcome developer inertia by providing them with a tool to quickly
and efficiently communicate skills and knowledge within the group. 

However, the approach is not a panacea. It still requires training,
resources, and a commitment to the process by management.  As with
other forms of FTR, it may not be appropriate for groups without basic
project management mechanisms in place, and it does not resolve
important maintenance-related issues. 

\section{Acknowledgements}

{\small 
The author gratefully acknowledges current and past members of the
Collaborative Software Development Laboratory: Danu Tjahjono, Rosemary
Andrada, Carleton Moore, Dadong Wan, and Robert Brewer for their
contributions to the development of CSRS. Support for this research was
partially provided by the National Science Foundation Research Initiation
Award CCR-9110861.
}

\bibliography{/group/csdl/bib/research-process,/group/csdl/bib/tech-transfer,/group/csdl/bib/code-inspection,/group/csdl/bib/csdl-trs}

\bibliographystyle{plain}

\end{document}




