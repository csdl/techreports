
\section{Obstacles to Groupware Adoption}

CSRS is system that supports groups of workers in accomplishing the task of
FTR, and so it is an example of a computer-supported cooperative work
environment, or ``groupware'' system.  Grudin \cite{Grudin94} notes that
successfully managing the adoption process is one of eight principle
challenges facing groupware developers. 

He proposes the following guidelines, as adapted from
\cite{Perin91,Erlich87}. 

\begin{itemize}

\item {\em Identify the group's problems and match the computer solution to it.}
  For FTR, this means identifying the specific breakdowns occuring in the FTR
  process and designing computer support intended to alleviate it. For
  example, if FTR is not being practiced because the organization believes
  it consumes too much time, an asynchronous, on-line review method that
  does not involve any group meetings may address this concern.

\item {\em Identify {\em appropriate} work processes.} For FTR, this
  implies that it is dangerous to overconstrain the review method with
  standardized, enforced procedures.  The procedures must be sufficiently
  prescriptive to lead to high quality outcomes, yet must allow adaptation
  to characteristics of the group performing review and artifact under review.

\item {\em Select appropriate pilot groups and individuals.} For FTR,
  this implies that the initial users must be receptive to the use of
  groupware technology and must be motivated to institute changes in
  their work procedures.  This in turn implies that the technology hold
  the potential for clear, measurable, and short-term improvements in
  their work process or products.

\item {\em Give the adopting group a clear understanding of the mature
  use of the application.}  This may involve a site visit, and should
  always include training that demonstrates the positive impact the FTR
  technology can have on the work day.

\item {\em Management attitude is critical to acceptance.} Particularly
  when the FTR application represents a small investment for the
  organization, management may not necessarily be explicitly supportive and
  encouraging of the new technology.  

\item {\em Prevent premature rejection.}  During the initial adoption
  period, relatively problems can potentially frustrate users and lead to
  complete rejection.  Support for active anticipation and correction of
  problems must be in place.

\end{itemize}
