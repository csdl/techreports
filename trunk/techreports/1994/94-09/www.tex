%%%%%%%%%%%%%%%%%%%%%%%%%%%%%% -*- Mode: Latex -*- %%%%%%%%%%%%%%%%%%%%%%%%%%%%
%% www.tex -- 
%% Author          : Rosemary Andrada
%% Created On      : Mon Jul 25 15:27:13 1994
%% Last Modified By: Rosemary Andrada
%% Last Modified On: Fri Dec 23 13:33:31 1994
%% Status          : Unknown
%% RCS: $Id: www.tex,v 1.3 1994/09/21 21:34:58 rosea Exp rosea $
%%%%%%%%%%%%%%%%%%%%%%%%%%%%%%%%%%%%%%%%%%%%%%%%%%%%%%%%%%%%%%%%%%%%%%%%%%%%%%%
%%   Copyright (C) 1994 University of Hawaii
%%%%%%%%%%%%%%%%%%%%%%%%%%%%%%%%%%%%%%%%%%%%%%%%%%%%%%%%%%%%%%%%%%%%%%%%%%%%%%%
%% 

\documentstyle [12pt, /group/csdl/tex/named-citations,
/group/csdl/tex/definemargins,
/group/csdl/tex/lmacros]{article}

\begin{document}

\vspace*{1in}

\begin{center}
  {\Large\bf Redefining the Web: Creating a Computer Network Community}
  \bigskip\par

  Rosemary Andrada \bigskip\par

  Collaborative Software Development Laboratory\\ Department of
  Information and Computer Sciences\\ University of Hawaii\\ Honolulu, HI
  96822\\ (808) 956-3496\\ {\tt rosea@uhics.ics.hawaii.edu} \bigskip\par

  {\bf CSDL-TR-94-09} \bigskip\par

Last Revised: \today

\end{center}

\begin{abstract}

  Organizations are formed to accomplish a goal or mission, where
  individual members do their part and make a combined effort leading
  toward this goal.  As the organization grows in size, the level of
  community inevitably deteriorates.

  This research will investigate the strengths and weaknesses of a
  computer-based approach to improving the sense of community within one
  organization, the Department of Computer Science at the University of
  Hawaii.  We will assess the current level of community by administering
  a questionnaire to members of the department.  Next, we will introduce
  a World Wide Web information system for and about the department in an
  effort to impact the level of community that exists.  We will then
  administer another questionnaire to assess the level of community
  within the department after a period of use with the information
  system.  We will analyze the results of both questionnaires and usage
  statistics logged by the system.

\end{abstract}

\newpage
\tableofcontents
\newpage

\section{Introduction}

%  What is the purpose of this research?  What is the central problem being
%  addressed?  Why is this problem important?

This research proposal is concerned with computer-based mechanisms for
improving the ``sense of community'' within an organization.  A broad
definition of the word community would be a `unified body of individuals'.
However, we view a ``sense of community'' more narrowly, in terms of the
``collective self-awareness'' of the organization.  From this definition,
it is possible to characterize the level of community by evaluating the
following measures:

\begin{enumerate}
\item{Can each person associate names with the faces of others in the
  organization?}
\item{Does everyone know each other personally?}
\item{Can people correctly identify a `resident expert' on some subject?}
\item{Is everyone aware of the different projects in the organization and
  which persons are involved in them?}
\end{enumerate}

Positive answers to these questions would indicate some level of community
within an organization.  However, I can find at least one instance of a
negative response to each question when posed to our department.

\begin{enumerate}
\item{A fellow Master's student who had been in the program for almost two
  years recently told me that he still had never seen a particular faculty
  member.}
\item{I have been with the department for almost two years and I feel that
  I know the staff and many students personally.  However, I am personally
  familiar with only a handful of the twenty faculty members.  This is
  mainly due to the fact that I've worked with them closely either as their
  teaching assistant or on a research project.}
\item{Another fellow graduate student told to me that when he was asked
  which faculty member to see regarding a particular field of computer
  science, he could not give the new student an answer.}
\item{Upon introducing ourselves on the first day of a graduate seminar
  class, a student mentioned he worked in one of the department's research
  labs.  My professor responded by suggesting that the student inform us of
  what goes on in that lab since nobody else knows.  One would at least
  expect faculty members to be aware of each other's projects.}
\end{enumerate}

% What new insight about the problem will be provided through this
% research?  Why is solving this problem interesting?

Through this research, we expect to learn the strengths and weaknesses of a
computer-based approach to improving the sense of community within one
organization, the Department of Computer Science at the University of
Hawaii.  First, we will assess the current level of community by
administering a questionnaire to members of the department.  Second, we
will introduce an information system for and about the department in an
effort to impact the level of community that exists.  Third, we will
administer another questionnaire to assess the level of community within
the department after three months of use with the information system.  We
will analyze the results of both questionnaires and usage statistics logged
by the system.

\section{The World Wide Web}

\subsection{Background}
The World Wide Web (WWW or Web, for short) is a network of inter-linked
information which started at CERN, a particle physics laboratory in Geneva,
Switzerland.  The WWW defines the components of a global information system
and how they work together.  These components consist of clients which are
used to access information servers on the network.  Clients and servers
talk to each other in a WWW protocol specification called HTTP (HyperText
Transfer Protocol).  Organizations all over the world have set up their own
information systems which they make available on this network.

A Web Server acts as the interface between a database and a client program
using the WWW protocol specification.  It is possible for a Web Server to
interface with WAIS, GOPHER and ftp servers.  However, Web Servers
primarily serve information from a local database, which is structured as a
directed graph of linked files.  There is one main file, or Home Page,
which is the entry point to information in a particular database.  Since
the Web uses HTTP, files generally need to be written in HyperText Markup
Language (HTML) format.  An HTML file consists of text, which is displayed
to the reader of an HTML document, and tags, which tell the client
programs, or browsers how to format that text.  Documents written in HTML
have the ability to link regions of text to another document. Browsers
highlight these regions to indicate that they are hypertext links.
Documents may contain links to other files in the database as well as other
files residing on other Web sites.

Traditionally, these systems are used to provide multi-media information, such
as text, graphics, video and sound.  Viewing information in these different
ways can do more than just inform; it can entertain, enlighten, spark interest
and perhaps on a small scale, bond those that put out the information with
their intended audience.

To the best of our knowledge no one has set up a Web server specifically for
the purpose of improving a sense of community among its intended audience.
Typically, a Web Server is set up to provide information and no effort is made
to assess the impact this information has on the community of users.  

\subsection{Design of Database}
Web sites typically 'serve up' information to anyone requesting it.
Sometimes the motive is to teach its users about something.  In this
research, there is a specific audience and specific information we wish to
relay to them.  Special care is needed when thinking of how to structure
the database in order to satisfy its primary requirement: improving the
sense of community in an organization.

First, the database should contain information about each person's personal
interests.  In an academic organization, research interests should be included
while commercial organizations might want to include current projects of
individuals and/or subgroups.  There should also be a section detailing
programs or services offered by the organization.  Just as families maintain
photo albums of themselves, so should an organization maintain a photoboard of
all of its members.  Information systems can offer useful information when
compiled and organized properly by a willing maintainer.  However, allowing all
members involved in an organization to make their own contribution will add a
unique flavor to the information presented.  By increasing the visibility
between the members of an organization, we also increase the sense of community
within it.

It is important to include information about people's personal interests.
Most organizations are not founded upon personal interests, but rather on a
common goal or mission, usually professional.  Making others in the same
group aware of one's hobbies is an attempt at encouraging further bonding
within a community.  This may not increase the sense of community overall
since it is unlikely that a group formed professionally will have the same
hobbies.  However, it is likely to form clusters of subgroups.  Using a Web
server, this could be implemented by having a document, linked by the main
home page, dedicated to the purpose of displaying everyone's personal
interests.  It would list many different hobbies and those people who enjoy
participating in that activity.  The list of hobbies and people would start
out small.  At first, we expect user's interests to vary greatly such that
we may get a listing of many different interests associated with only one
or two people.  As more people in the organization use the system, they
will view others' interests, realize they share it too and will go ahead
and add their names.  Some level of community would be developing as people
become more aware of other's interests and offer to share a bit of
themselves by adding their own interests to the list.

We also want to include a page detailing projects and research interests of
different individuals or groups in the organization.  In larger
organizations, people's attention tend to remain narrowly focussed on their
own projects.  We want to make people aware of other projects being
conducted to shed light on how the different parts of an organization
combine their efforts to attain their high-level goals.  Having information
about each others projects also informs us who is the resident expert on a
subject area.  Accessing this page demonstrates that the viewer has more
than a passing interest in what goes on in the organization.  People showing
interest in areas of their organization outside their own work contributes
to the sense of community.

Having a photoboard of all the members in an organization is comparable to
having a photo album of members of an extended family.  We want to create
this sense of family in the organization.  Members in an extended family do
not generally know every other member.  But keeping a record of all the
names and faces helps to maintain the family bond.  People in an
organization don't have to know every other person, but it is helpful to
have a photoboard as a reference.  It can facilitate introductions of new
members and help to avoid embarrassing moments of forgetting names.  While
it is not necessary to go out and memorize all the names and faces or to
meet each person, it is a comforting feeling to have one's picture
displayed showing that this is a place where you belong.  This is your
family at work.  

So far, much of the information suggested for inclusion could be compiled
by a third party.  This is generally the person that maintains the Web
server.  Having information provided by one source is good in that
everything can be presented uniformly and tailored around a common theme,
making the overall view of the system more coherent.  It is limited in that
it presents only one perspective of the organization.  We want to further
increase the sense of family by fostering an environment where its members
can make their own contributions to their organization's information
system.  This can be easily implemented because Web Servers allow users to
create their own Home Page.  These documents are typically maintained by the
user.

The database may be accessed anytime at a user's leisure.  They are free to
choose what information to access.  As new items are added to the database,
users need some mechanism for knowing what information has been seen before and
what is new.  New information differs between users.  We attempt to keep the
users abreast of changes in the database by having a ``what's new'' page.  This
document lists items changed in the last few days, week and month.
Unfortunately, Web servers do not automatically detect changes in the structure
of the database.  That is, it is unable to detect changes in the way documents
are linked.  Aside from parsing all HTML documents in the database, the only
way to discover changes in content is to check the date a document was last
modified.  However, this has the disadvantage that changes made to a file due
to a typographical error are indistinguishable from significant changes of
content.  In any case, until a better mechanism for detecting modifications is
developed, listing recently modified documents that may include those with
insignificant changes is better than not listing them at all.

The information content must be as self-maintaining as possible. We want to
create a collaborative system whose contents are generated and maintained
both explicitly and implicitly through the activities of the people in an
organization.  Self maintenance means both that information should be added
as a side-effect of people's normal work processes, and should be deleted
as an outcome of garbage-collection activities.  The maintainer of the Web
server, or webmaster, would explicitly keep the system up to date by adding
new information or modifying documents to reflect changes.  The database is
also implicitly maintained by users who create their own home pages.  These
documents are the sole responsibility of the users that create them.  They
can change the contents of the documents and they also decide whether or
not to keep them at all.  As an added measure of self-maintenance, it is
easy to write a program to search the file system for user-created home
pages and maintain this list on a daily basis.  As more people create home
pages, their document will be automatically added to the list.  The
expiration of these files is not a problem since a member's account and
files are purged once they leave the organization.  The photoboard can be
maintained in the same way as the list of home pages.  Users may keep a
file of their digitized picture in their directory.  The program would
search the file system for such files and maintain that list on a daily
basis.  Users can change their picture or refrain from displaying their
picture without having to consult the webmaster.  The elimination of the
``middle-man'' also expedites matters concerning how immediately the changes
will take effect.
% add blurb about personal interests in comment field?

% ??? could talk more about other self-maintaining aspects such as garbage
% collection by expiration date, number of accesses, or not recently
% accessed.

\section {The Experiment}
We will perform the following experiment to evaluate the impact of our
system on improving the sense of community in the Department of Information
and Computer Sciences at the University of Hawaii.

\subsection{Duration}
This study will last from January 11, 1994 to April 1, 1995.  The system
currently under development will be publicly released in January, 1994.
This experiment will continue for a period of about three months, at which
time we will evaluate the level of community within our department.

\subsection{Method}

% What method will be used to provide new insight into the problem?  How do
% you address this problem?

% How will questionnaire be administered and when?  How will this interact
% with server release dates?

We must first determine the level of community in the department prior to
releasing the information system.  Pre-Test questionnaires investigating
the current level of community will be sent out separately to faculty/staff
and students.  Appendix \ref{Pre-Test} contains the Pre-Test questionnaire.
The questionnaire will be e-mailed to all undergraduate and graduate
students in ICS.  However, the ICS faculty will receive it both through
e-mail as well as a hardcopy in their mailbox.  Anonymity is important.  So
all data will be kept strictly confidential.  The responses may be printed
out and turned in at a mailbox in the ICS lounge.  However, for those that
do not care to turn in a hardcopy, an e-mail reply will also be accepted.
Many people do not wish to be bothered with surveys, so we will offer an
incentive for completing the questionnaire.  People who respond by email
are automatically entered in the drawing.  Those who submit their surveys
by hardcopy may detach an entry form from the bottom and turn it in at the
same time.  The prize will be a \$30 gift certificate to a local restaurant
of the winner's choosing.  The questionnaire will be sent out a few days
before the Web server is released.

Once the server is released for public view, changes to the database will
evolve in the manner dictated by its users.  Users may find some information
included to be ineffectual and request that it be removed.  They may also find
that some useful information was omitted and request that it be added to the
database.  An area where users have more control over the evolution of the
database is in their home pages.  They may include any information about
themselves they wish to publicize to the world.  People generally give some
personal information and include links to their favorite places to visit on the
Web.

%\subsubsection{Getting People to Use the System}
An important factor in this research is encouraging as many people as
possible to use the system as well as getting them properly trained to use
it effectively.  Some of our target audience may not be interested in
participating in this research.  Nevertheless, they will be made aware of
what information exists.  The information will always be available should
they reconsider and wish to participate.  On the other hand, people may be
very interested in using our system, yet do not know how to get started.
Once they do get started, they may wish to do more with the system, but
find they don't have enough information to continue.  Members of the
department will be encouraged to regularly access the database as well as
make contributions to it.  Sessions will be provided on a regular basis on
how to use the Web client, Mosaic, for accessing database information.
Additionally, we will demonstrate the different features of the server that
facilitate contribution to the database.

After the experiment is over, we will send out post-test questionnaires to
determine the level of community at this time.  They will be conducted in
the same manner as the pre-test questionnaire.  Appendix \ref{Post-Test}
contains the post-test questionnaire.

\section{Results}
%\subsection{Data analysis}
% What are the range of outcomes from this research?  What is the
% significance of each outcome?

% What data will be collected?  What are example instances of this data?  How
% will the data be analyzed?  How will I know how well I've addressed it?

The objective of this study is to determine through experiments, if an
appropriately designed WWW server can increase the level of community in
our target organization, the Department of Information and Computer
Sciences.  Data will be collected in two ways.  The first is through the
responses we receive from the pre- and post-test questionnaires.  We will
also collect data directly from the Web server since it keeps a log of all
requests made.

% How will questionnaire data be analyzed?  What constitutes success?  Will
% any data from server statistics be collected/used/how?

\subsection{Analysis of Questionnaire Responses}
We first need to know the original state of community.  We would like to
know what specific elements of community currently exist or are missing
from the department.  There are different types of questions posed.  The
first three questions deal with raw numbers and try to establish how many
members of the department one knows.  Answers to these questions will tell
us how many other members of the department people are familiar with.
Questions 4 through 6 are fact-finding questions.  They have right and
wrong answers.  These answers will tell us how well people actually know
each other.  Question 7 informs us on how different people like to
communicate with each other.  Since we will be using the Web client,
Mosaic, we asked question 8 to see how many people will need to be
introduced to this software.  The last open-ended question will give us
feedback on what people think about the level of community within the
department.

We are looking at two main variables here.  The first is how many members
of the department are aware of the Web and actually use Web browsers to
access information across the Web.  We will only consider two categories
which is that only a few people may be using it or many people are using
it.  The other variable is the level of community.  It may be either low or
high.  Given that the Web is rather new and only gained in popularity in
September  1993, a little over a year ago, we might assume that the initial
number of Web users in the department to be only a few.  The transitory
nature of any student body in a university department allows us to assume
that the level of community would initially be low.  The students far
outnumber the possibly more cohesive subgroup of faculty and staff.  So we
begin with the assumption that few people in the department are actively
using the Web and the level of community is low.  Since there are two
variables with two possibilities each, there are four possible outcomes.

The first possibility is that the computer-based approach made no
appreciable difference to the level of community.  That is, there still
remains only a few people using the Web and the level of community is low.
It might be that not enough advertising was done to promote the use of the
Web.  So not too many more people even know about its existence.  It may
also be that people were made aware of the system but were not properly
trained to use it.  The desire to use it could be there but was not
supplemented by the ability to employ it in a useful manner.  This could be
countered by providing better or more frequent training sessions.  Lastly,
it may be that users were aware of the system and were fully knowledgeable
about how to use the system but simply did not find any utility in it.  If
this were the case, then one could argue that the database was not
appropriately designed to promote community within the department.  In any
case, infrequent use of the system will not contribute to raising the level
of community in the department.  

The second possibility is that there is still only a few number of active
users of the system, but coupled with a higher level of community.  Again,
the reasons behind having only a few users of the system can be attributed
to those as in the first case.  However, we need to discover how the level
of community increased.  One answer is that there was an outside factor and
the computer-based approach made no impact on the department's sense of
community.  A very likely answer is that only a few people submitted the
pre-test questionnaire and the same few turned in the post-test
questionnaire.  If these were the only people participating in the
research, then it would make sense that the level of community in that
smaller group had increased.  From this result, we can say that the
information system did have a direct impact on raising the sense of
community in a small group.  This is interesting, but we also know that
this can be done even without a computer mediary.

The third possibility is that more people are using the system but the
level of community is still low.  This is an interesting case in that the
fact the many people are using the system  means they have a more unified
view of the department, yet do not feel that the department has unified
sufficiently.  Since more people are utilizing the system, we know that
advertising was not a problem.  If most users were only passively using the
system, i.e. did not make home pages, add their pictures or list their
personal interests, then we could say the level of community remained low
due to a lack of active participation.  Unfortunately, this still does not
reveal to us whether the problem of raising the level of community lends
itself to a computer-based solution.

The last possibility is that we find more people using the system and an
increased sense of community among members of the department.  This is our
desired outcome.  It means we have succeeded in designing an appropriate
database for the purpose of increasing the level of community in a specific
organization.  We would have advertised well enough and trained users to
effectively use our information system.  The content of the database
represents a minimum of what is necessary to increase visibility between
department members.

Since the experiment lasts three months, we are unable to monitor the
possibility that a decline in interest in the system may directly affect
the sense of community.


\subsection{Statistics of Server Accesses}
%[Will any data from server statistics be collected/used/how?]

The Web server we are using keeps a log of all incoming requests.
Information about which documents are being requested, when, how often,
etc. is all noted in a logfile.  We use the program getstats.c written by
Kevin Hughes of Enterprise Integration Technologies (EIT) to compile these
statistics.  Unfortunately, it does not collect data on which specific user
requested what document.  However, it is able to tell us what machines and
from which domains these requests came from.  It also logs the number of
requests made by a particular machine.  This is important because it tells
us how much traffic is attributable to users outside the organization and
to users from within the organization.  The server also logs which
documents were being requested, how many times it was requested and the
last date it was requested.  Documents garnering a low number of accesses
could indicate that it is not easily found during navigation.  It may also
mean that the information is boring or not very useful.  Documents which
have not been accessed in a long while may prove to be outdated.  We hope
that documents accumulating the most number of accesses are those which are
dynamic and created by users in the department.

\section{Conclusion}
Our goal was to investigate how a computer-based information system impacts
the level of community in an organization.  We chose to use the WWW to
employ this technique.  Much thought was put into the design and content of
the database so as to increase a sense of community among a body of people.
We decided that personal interests, current work projects and a photoboard
were essential to this purpose.  Having a what's new page is necessary to
facilitate navigation through and inspection of newly added information.  A
pre-test questionnaire is to be administered to assess the original level
of community.  The information system is then released for public use.
After the three-month experiment, a post-test questionnaire is to be given
to assess the level of community after some months of using the system.  We
assumed there to be an initial state of few people being aware of the Web
and a low level of community.  A resultant state of few users and low level
of community indicates problems with getting the system started or poor
design of the system.  It does not however, discount that an appropriately
designed system could increase the level of community.  A resultant state
of few users and a high level of community might indicate that
computer-based systems can increase the level of community in a small body
of individuals.  The third possibility that there are more users but the
level of community remains low could mean the database was not
appropriately designed or that users are only passively utilizing the
system, thereby not making a significant contribution to the level of
community.  The last possibility that more users are working on the system
an the level of community has increased indicates that an appropriately
designed database can contribute to the sense of community in an
organization.

\appendix

\newpage
\section{Pre-Test questionnaire to [faculty/students]}
\label{Pre-Test}
\begin{enumerate}

\item{Which group do you work with most?}
  \begin{itemize}
  \item{Faculty}
  \item{Staff}
  \item{Grad student}
  \item{Undergrad student}
  \end{itemize}

\item{How many [ICS students/faculty members] do you feel you know
  personally?}
  \begin{itemize}
  \item{0-10}
  \item{10-20}
  \item{20-30}
  \item{over 30}
  \end{itemize}

\item{How many [ICS students/faculty members] do you think you could name,
  given their face?}
  \begin{itemize}
  \item{0-10}
  \item{10-20}
  \item{20-30}
  \item{over 30}
  \end{itemize}

\item{How many faculty, graduate students, undergraduates do you think are
  in the department?  Please provide three numeric estimates}
  \begin{itemize}
  \item{Faculty}
  \item{Grad students}
  \item{Undergraduates}
  \end{itemize}

\item{What ICS research projects are you aware of?  Please briefly list the
  projects you can think of immediately and any faculty, staff, or students
  you know who are involved in them. (If you know many people involved in a
  particular project, then simply estimate the number of involved people
  and provide that number.)}

\item{Assume you had a question about one of the following topics.  For
  each of the topics, name one or two people in the department who you
  would want to ``just stop by'' to talk with about it, or leave it blank
  if you can't think of anybody.}

  \begin{itemize}
  \item{Departmental rules}
  \item{Artificial intelligence}
  \item{Software Engineering}
  \item{Computer networks}
  \item{Hypertext and multimedia}
  \item{Cognitive Science}
  \item{Computer Programming}
  \item{Human Computer Interaction}
  \item{Computer games}
  \item{Employment opportunities}
  \end{itemize}

\item{For each of the following types of communication, indicate whether
  you use it Daily, Weekly, Monthly, or Never, to communicate with other
  people in the department.}
  \begin{itemize}
  \item{Email}
  \item{Telephone}
  \item{Fax}
  \item{Informal meetings (lunch, etc.)}
  \item{Formal meetings}
  \item{Other (please specify)}
  \end{itemize}

\item{Do you use Mosaic, or any other Web client to access the World Wide
  Web?}

\item{(Optional) Please write any other comments you have about the sense
  of community in the department below.}

\end{enumerate}

\newpage
\section{Post-Test questionnaire to [faculty/students]}
\label{Post-Test}
\begin{enumerate}

\item{Which group do you work with most?}
  \begin{itemize}
  \item{Faculty}
  \item{Staff}
  \item{Grad student}
  \item{Undergrad student}
  \end{itemize}

\item{How many [ICS students/faculty members] do you feel you know
  personally?}
  \begin{itemize}
  \item{0-10}
  \item{10-20}
  \item{20-30}
  \item{over 30}
  \end{itemize}

\item{How many [ICS students/faculty members] do you think you could name,
  given their face?}
  \begin{itemize}
  \item{0-10}
  \item{10-20}
  \item{20-30}
  \item{over 30}
  \end{itemize}

\item{How many faculty, graduate students, undergraduates do you think are
  in the department?  Please provide three numeric estimates}
  \begin{itemize}
  \item{Faculty}
  \item{Grad students}
  \item{Undergraduates}
  \end{itemize}

\item{What ICS research projects are you aware of?  Please briefly list the
  projects you can think of immediately and any faculty, staff, or students
  you know who are involved in them. (If you know many people involved in a
  particular project, then simply estimate the number of involved people
  and provide that number.)}

\item{Assume you had a question about one of the following topics.  For
  each of the topics, name one or two people in the department who you
  would want to ``just stop by'' to talk with about it, or leave it blank
  if you can't think of anybody.}

  \begin{itemize}
  \item{Departmental rules}
  \item{Artificial intelligence}
  \item{Software Engineering}
  \item{Computer networks}
  \item{Hypertext and multimedia}
  \item{Cognitive Science}
  \item{Computer Programming}
  \item{Human Computer Interaction}
  \item{Computer games}
  \item{Employment opportunities}
  \end{itemize}

\item{For each of the following types of communication, indicate whether
  you use it Daily, Weekly, Monthly, or Never, to communicate with other
  people in the department.}
  \begin{itemize}
  \item{Email}
  \item{Telephone}
  \item{Fax}
  \item{Informal meetings (lunch, etc.)}
  \item{Formal meetings}
  \item{Other (please specify)}
  \end{itemize}

\item{Do you use Mosaic or any other Web client to access the World Wide
  Web?}

\item{How often have you been accessing information from the ICS
  Department's Web site?}
  \begin{itemize}
  \item{Rarely or never}
  \item{A few times a month}
  \item{A few times a week}
  \item{Amost everyday}
  \end{itemize} 

\item{Did you create your own Home Page?}

\item{Did you find the information presented about the department useful?
  If so, in what ways?  If not, why not and how would you improve it?}

\item{(Optional) Please write any other comments you have about the sense of
  community in the department below.}

\end{enumerate}

\end{document}





% Extracted Junk

%Generate working principles for WWW database generation

%Garbage collection is as important as information addition,
%since people will not use the system once the signal-to-noise ratio
%drops too much.

%However, anyone on the internet using a Web client may access information
%from our database.  This too can be desirable since exposing our department
%to the outside world may attract prospective faculty and students, thereby
%adding to our hopeful community.
