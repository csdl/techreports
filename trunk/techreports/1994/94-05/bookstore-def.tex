%%%%%%%%%%%%%%%%%%%%%%%%%%%%%% -*- Mode: Latex -*- %%%%%%%%%%%%%%%%%%%%%%%%%%%%
%% bookstore-def.tex -- 
%% Author          : Taeho Yum-499SM94
%% Created On      : Tue Jul 12 11:18:57 1994
%% Last Modified By: Taeho Yum-499SM94
%% Last Modified On: Tue Jul 12 13:05:53 1994
%% Status          : Unknown
%% RCS: $Id$
%%%%%%%%%%%%%%%%%%%%%%%%%%%%%%%%%%%%%%%%%%%%%%%%%%%%%%%%%%%%%%%%%%%%%%%%%%%%%%%
%%   Copyright (C) 1994 University of Hawaii
%%%%%%%%%%%%%%%%%%%%%%%%%%%%%%%%%%%%%%%%%%%%%%%%%%%%%%%%%%%%%%%%%%%%%%%%%%%%%%%
%% 

\documentstyle [12pt,/group/csdl/tex/definemargins,
                /group/csdl/tex/lmacros]{report}

\begin{document}

\title{Virtual Bookstore}
\author{Philip M. Johnson\\
	Tae Ho Yum\\
	Sang-Woo Han\\
	John S. Johnson\\
  Collaborative Software Development Laboratory\\
  Department of Information and Computer Sciences\\
  University of Hawaii\\
  Honolulu, HI 96822\\}	

\date{ICS/CSDL Technical Report 94-05}


\maketitle 

\newpage
\tableofcontents
\newpage


\chapter{Introduction}

\section{Motivation}

The University of Hawaii, as with all other major universities, supports a
flourishing used book business.  This business can be roughly characterized
in the following way:

\begin{enumerate}
\item A student decides that a book that they own is no longer useful to them.
\item The student offers the book to the bookstore.
\item The bookstore determines whether or not a market exists for the book
     (i.e. that a professor will require the book the following term, and
     that course enrollment justifies "investing" in this book.)
\item The bookstore evaluates the condition of the book and makes an offer
     to the student.
\item The student accepts or rejects the offer.
\item The bookstore offers the book for sale with a sizable markup (typically 50-100%).
\item The bookstore either succeeds or fails in selling the book during the
     following semester.  If the bookstore fails to sell the used book, it
     must decide whether or not to: (a) keep it for the following semester
     and attempt to sell it again, in which case it incurs overhead for storage;
     (b) sell the book to a national used book distributor, or (c) throw
     away the book.  
\end{enumerate}

The purpose of this research project is to explore the requirements for a
``virtual bookstore'' by exploring the process alternatives and outcomes
for how such a bookstore would be organized and run.  After this initial
analysis, an instrumented, collaborative environment will be designed and
built that will allow students to buy and sell books where the current
``physical'' bookstore has been replaced by an on-line, ``virtual''
bookstore running on uhunix.  A simple electronic mail interface will be
developed to allow any current UH student with basic computer skills and a
uhunix account to successfully use the system.  More elaborate interfaces
will also be developed for skilled computer users, as well as for bookstore
administrators.  Several different process models for this virtual
bookstore will be developed and experimentally evaluated through actual use
of the system by the UH student community.  Systematic changes to the
bookstore process model will explore the impact upon fairness, efficiency,
and effectiveness of variations in information access and information
requirements for the roles of buyer and seller.

\section{Process Modelling a Virtual Bookstore}

In general, the goal of a virtual bookstore is very simple: create a more
efficient and thus lower-cost method for students to exchange used books by
eliminating the physical bookstore from the transaction.  This can benefit
both buyer and seller: for example, a selling price mid-way between the
physical bookstore's buying cost and its subsequent markup can
simultaneously give the student selling the book more money and save the
student buying the book an equivalent amount of money.  

The increased efficiency of a virtual bookstore is produced through
eliminating the overhead and risk incurred by a physical bookstore: no
salaries are paid to buy, sell, and stock books, nor is there a risk of
buying a book that cannot be sold.  In some respects, the virtual bookstore
is similar to a ``just-in-time'' inventory system.

While this goal is simple and obvious, the design of a process requires
more reflection than might first be apparent.  First, the design of the
process should satisfy, as much as possible and as often as possible (if
not guarantee), the following constraints:

\begin{itemize}
\item {\bf Fairness.}  The process should lead to the buyer paying a reasonable and
    equitable price for the goods received. The process might also try to
    guarantee that the longer a buyer waits for a book, the higher the
    probability that the buyer will be able to buy the next instance of the
    book that becomes available for sale.

\item {\bf Efficiency.} The process should occur fast: a seller wanting to sell a
    book should be able to do so as quickly as possible, and a buyer
    wanting to buy a book should be able to do so as quickly as possible.

\item {\bf Effectiveness.}  The process should lead to all sellers being able to
    sell their books if buyers are available, and all buyers being able to
    buy their books if sellers are available.
\end{itemize}

To illustrate just a few of the issues and interactions, consider the
following example process models.

\subsection{The ``Public BBS" process model}

This first process model is perhaps the simplest: sellers post a book for
sale to a public database, and buyers peruse the database and contact
sellers if a book is offered that they have an interest in.  While simple,
this process model can fail to satisfy all three essential requirements:

{\em Fairness.} Sellers can be unfair by falsely advertising a book
as being in ``excellent" condition when it is actually in poor
condition.  Buyers can be unfair by getting their friends to ``flood"
the database with false offers for the books of interest at unfairly
low prices.  This would artificially depress the perceived ``going
rate" for a book, leading to an unfairly low price for the honest
sellers.

The process also does not support fairness in the sense of
preventing ``starvation''---a buyer may never be able to buy a book,
even though instances of the desired book become available, simply
because other buyers continually ``get there first".

{\em Efficiency.} The bookstore must handle hundreds, if not thousands of
transactions, but the process model does not specify how to expire listings
or how to organize the database.  Sellers must presumably decide for
themselves how and where to post the notice, which may lead to
inconsistencies and ``lost" postings.

The process model states only that buyers can browse for books that they
want.  To maximize effectiveness, then, buyers must be continuously
checking the bookstore to see if a desired book has just been added, which
is very inefficient.

{\em Effectiveness.} As noted above, if not well organized, the
bookstore will lead to ``lost'' postings---situations in which a buyer
and seller do not connect, either because the buyer cannot find the
seller's listing (and thus the seller loses out), or because the
buyer does not check frequently enough and another buyer gets it (in
which case the buyer loses out).


\subsection{The Electronic Marketplace (Eager Transaction System)}

The system that we named the Eager Transaction system was basically a system that operated like an
electronic marketplace.  It basically served as a central place to do
trading.  In this system, a student is assured to get a better deal or
to sell their book quickly if they were eager to do a little work.
Thus the name "Eager Transaction System".

This system is actually very similar to the Public BBS process model, but we
have worked it out to better control, the market.  In this case the system just
acted as a moderator for the marketplace.  Many issues that disqualified the
Public BBS model were addressed, but there were still many faults.

{\em Fairness.} We were able to come up with a pretty fair model with some
basic restrictions.  We would penalize for false advertisments and we would
only allow a certain number of posts in the public listing.  Also the
information that is given to the system will be formatted and then parsed for
fields such as title, author, price...  This way we would have formatted
postings and the database will be able to search by field.

{\em Efficiency.} This system is more efficient than the Public BBS model, but
the users would still have to do a little work to find the better deals.  We
later decided that the students in fact do not want to do anything besides
telling the system what they want to do with a book and some criteria for the
system to choose one for them.  Basically, buying and selling college texts are
not very interesting tasks and selling or buying quickly is much more important
that getting stuck with a book or having to buy one brand new.

{\em Effectiveness.}  As with the Public BBS model, some postings can get
lost. in the system forever.  Students will probably not be interested in
checking into the system to see what they can find and doing ``work'' to buy or
sell a book.

\section{Evaluation of Requirements for a Virtual Bookstore} 

We have concluded that there will be no ``perfect" process model for a
virtual bookstore---one that simultaneously satisfies the fairness,
efficiency, and effectiveness requirements.  This is great, because it
makes the design process an interesting one: which requirements must be
sacrificed, and to what degree, in order to fulfill others?  

Although it may be possible to theoretically
sketch out the pros and cons of a particular process model, that in fact
there is no way to really understand its strengths and weaknesses except by
experimentation. 

The following tasks have been recommended to evaluate the requirements for a
virtual bookstore:

\begin{itemize}
  
\item {\em Literature Review.} It will be helpful to learn more about such
  systems.  A virtual bookstore system was recently designed for the Anderson
  Graduate School of Management \cite{bookstore-AGSM94}.  How does the
  process/data model suggested in this report fair with respect to fairness,
  effectiveness, and efficiency?

\item {\em Process Model Design.} Concurrently with literature research,
  try constructing a few process models for a virtual bookstore using Egret.
  Specify the process and data model as precisely as you can, using whatever
  notation you feel is useful.

\item {\em Process Model Evaluation.} For each model, evaluate its
  strengths and weaknesses with respect to fairness, efficiency, and
  effectiveness.  Can you discover other criteria to be used in addition to
  these three?
  
\item {\em Process Model Measurement.} For each model, design on-line
  measurements of the process that can provide an empirical basis for
  evaluating the fairness, efficiency, and effectiveness of the process.
  
\item {\em Process Model Experimentation.} Eventually, we will decide upon
  one or more models to implement using Egret and evaluate using the
  specified measures.  This experimentation will be designed and 
  evaluated according to the CSDL research process model \cite{csdl-ro-93-01}. 

\end{itemize}



\chapter{Process Model Design}

\section{Introduction (Lazy Transaction System)}

This is the working process model that we have agreed upon for now.  There
are more details to be worked out, such as the actual matching algorithm.  One
is currently in the works as a background process.

We call this the  ``Lazy Transaction System'' and it is a system that does all
the work.  A student will 
basically send in a request to buy or sell and the system will do all
of the work.  It will try to match up as many sales as possible and
keep everyone fairly satisfied.  All a student has to do is put their
faith in the system.  In this case, it is the students who are lazy rather than
the computer.

We chose this system from our experiences of being a university student as well
as knowing many others.
The more we thought about it, the more we came to conclude
that buying and selling college text books is not interesting.  It is, however, 
still a necesary chore that would be best if carried out in the most
efficient manner.  Efficiency in this case would be the greatest
number of sales for the least amount of human activity.  Who better to
give the uninteresting work than a computer?

\section{Some principles of the model}

\begin{itemize}

\item Users will send in requests to buy or sell describing exactly what
they want.

\item They will also specify ranges of prices and conditions of
books.

\item If a match is found within the user specified thresholds, then it will
be assumed that they will take the deal.
I
\item If the computer cannot make a match within user specified thresholds,
then it will try to make the best fit.

\item In any case where the computer makes a decision outside of what the
user allowed it to, the user will be allowed to bargain.

\item Time is also a factor.  Users can specify the amount of time they need
to sell the book by.

\end{itemize}

\section{The basic workings of the model}

\subsection{Condition of books}

For our model we will be considering the condition of the books.  We will
define a linear scale for the condition of books.  A linear scale should make
it easier to implement a matching algorithm.  We will clearly define the
criteria for the categories of books and hopefully most people will agree.  For
this writing we will use the letters A,B,C,D,F for the book categories.

\subsection{Data input from buyers and sellers}
\\
Seller:\\
1.	Book\\
2.	Condition\\
3.	Asking price\\
4.	Lowest accepted offer\\
5.	Time to sell book by\\
6.	Personal information\\
\\
Buyer:\\
1.	Book\\
2.	Conditions wanted\\
3.	Asking price\\
4.	Highest accepted offer\\
5.	Time to buy book by\\
6.	Personal information\\
\\

We will specify for them to be strict in entering price and condition ranges so
that they will pretty much accept if there is something in that range.  The
system will still find matches for them even if it's not in their range.
This data can be entered either by a telnet session or by a formatted email
message.

\subsection{The matching process}

We decided to have the system do the actual matchings at the end of the day or
after some other fixed time period.  For now we will assume that there will be
enough entrys by the end of the day.

\subsubsection{Best fit matches:}

Here we had an argument over the "best fit".  Should we try to obtain as many
matches as possible within user specified thresholds at the expense of breaking
up some perfect matches (where both asking prices and conditions match)?  The
argument there is what percentage of these "close" matches will actually result
in sales.  Or could we try to get as many perfect matches as possible at the expense of the
number of close matches?  The argument there is that that these perfect matches
should result in a guaranteed sale.
It's basically low risk-low yield or high risk-high yield.\\
\\
For now we will go with the most possible matches.

We will make the most matches possible and send out notices to the people
matched.  Those requests are now on hold until we get a confirmation from the
users.  All requests that have not been matched will cary over to be matched
the next day.  If both parties agree, then great, we release personal
information and let the transaction take place.  If someone does not agree,
then we put that request back as active again.  We are assuming that most
people will take offers that are withing their specified range.

If a match is not made by a certain amount of time, then the computer should go
and make a match that is close, but not within the user specified threshold.
That time period could be a week or so.  So someone using the system can expect
a match every week.  Notices will be send like the other matches.  If either
party refuses, then we can at that point let them bargain.  All possible (buyer
and seller exists) matches will be made by the end of that time period.  This
is so we don't have any requests that float in the system forever.

If a user specified a time period they need to sell the book by, then a match
will be made for the person by that time no matter what.


\section{Algorithm specifics}

Matches within user specified thresholds:

Scenario:  If a person wanted \$30-25 for a book and a person is willing to pay
\$30-25 for a book, then at what price do we notify the parties to buy/sell.
Should we let them meet half way or go to either end.  It would probably be
fair to give both of them the half way price in any overlap.

This would really be a simple model if we didn't have to deal with the
condition of the book.  If a buyer specified A-C condition and \$30-50, he
most likely does not want a C book for \$50.  The graph would look something
like this:

\begin{verbatim}
	A              xxxxxxxxx
	B      	       xxxxxxxxx
	C              xxxxxxxxx
	D
	F

	   0  10  20  30  40  50  60
\end{verbatim}

They probably want a graph that looks more like this:

\begin{verbatim}
	A                   xxxx
	B      	         xxxx
	C              xxxx
	D
	F

	   0  10  20  30  40  50  60
\end{verbatim}

Should we let them specify the price range for each book?


\chapter{Activities}
\begin{noindent}

\section{Objective and Activity Development}

\paragraph{} Formally define the research objective.  Define the method of research and the
data to be collected.

\paragraph{}Will be tracked by activities concerning files in ~csdl/objectives/94-05 and
~csdl/activities/94-05.

\paragraph{}This activity must be completed before any data can be collected.  Risk is
minimal in the stage of designing different process models for the bookstore.
Probably best if completed before any implementation of the virtual
bookstore.

\paragraph{}Tentatively will set a completion date of July 19, 1994.


\section{Literature Review}

\paragraph{}Research other similar systems.  Review report by Anderson Graduate School of
Management \cite{bookstore-AGSM94}.  Send out Usenet messages for comments on
the subject.  Review text on various business transactions.

\paragraph{}Not tracked.

\paragraph{}Required to continue with project.  Can be done concurently with any other
activity.

\paragraph{}Finished as of July 7, 1994.  Can be reopened any time to aid with project.

\section{Process Model Design}

\paragraph{}Design the process of the virtual bookstore.

\paragraph{}Will be tracked by file ~csdl/techreports/94-05/bookstore-model.*

\paragraph{}Must be completed before any further work.  Can be done concurently with
Literature Review and Process Model Evaluation.

\paragraph{}Working model exists as of July 7, 1994.

\section{Process Model Evaluation}

\paragraph{}Evaluate different process models with respect to fairness, efficiency, and
effectiveness.  Try to pick one working model to continue with project.

\paragraph{}Not tracked.

\paragraph{}Must be completed before any further work.  Can be done concurently with
Literature Review and Process Model Design

\paragraph{}Completed as of July 7, 1994.

\section{Matching Algorithm Design}

\paragraph{}This is one aspect of our current process model.  The current model centers
around this aspect and changes to the process model will most likely not affect
the existence of this aspect.  This involves the specification of a formal
algorithm to match buyers and sellers with data entered into the database.

\paragraph{}Will be tracked by file ~csdl/techreports/94-05/bookstore-algorithm.*

\paragraph{}Must be completed before any coded implementation of the virtual bookstore.
Can be done concurently with Matching Algorithm Evaluation.

\paragraph{}Projecting July 19, 1994 for a working algorithm.

\section{Matching Algorithm Evaluation.}

\paragraph{}Design on-line simulations to evaluate the matching algorithm to provide
emperical data to study the different aspects of the algorithm.

\paragraph{}Tracking method to be announced later.

\paragraph{}Must be completed before any coded implementation of the virtual bookstore.
Can be done concurently with Matching Algorithm Design.

\paragraph{}No date set as of July 7, 1994

\section{Process Model Experimentation}

\paragraph{}Implement a process model(s) using Egret and evaluate.  This experimentation
will be designed and evaluated according to the CSDL research process model
\cite{csdl-ro-93-01}.

\paragraph{}Tracking method to be announced later.

\paragraph{}Must be completed to finish research.

\paragraph{}No date set as of July 7, 1994

\bibliography{/group/csdl/bib/bookstore,/group/csdl/objectives/csdl-objectives}
\bibliographystyle{plain}

\end{document}
