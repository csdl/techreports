\section{False Positives}

False positives measures the number of wrong issues (or
invalid errors) raised by the participants.

Figure \ref{false-positives} shows the number false
positives generated by EGSM and EIAM groups. ``Issues'' is
the total number of issues raised by the group. ``Correct''
is the number of correct issues/errors identified by the
group, ``Wrong'' is the number of wrong issues. W1, W2, W3,
W4, W5 and W6 are further classification of the ``Wrong''
issues as discussed later.

The data shows that EIAM groups generated significantly more
issues than EGSM groups.  On average, EIAM groups generated
36.5 issues compared to only 14.4 issues in EGSM (All
groups).  However, many of these issues were false
positives. The amount of false positives were significantly
higher in EIAM than EGSM as shown by the column ``Wrong'' in
Figure \ref{false-positives}.  On average, EIAM groups
raised 22.0 false positives compared to only 5.3 in EGSM for
All groups.

Since the participants were not penalized for raising false
positives, it was possible that they raised issues without
evaluating their validity.  To further understand the nature
of these false positives, the false positives were
classified according to the following types:

\begin{itemize}

\item {\it Misinterpretation of logic (W1)}. 

This type of false positive was due to misinterpretation of
the logic in the code.  This includes a failure to
understand the conditional statement, the algorithm, the
input/output data, the use of primitive functions, the
specification implemented by the code, etc.  One typical
example of this false positive is as follows:

\small
\begin{verbatim}
      if (!strncmp (string1,string2,string_size)){ 
         ... 
      }
\end{verbatim}
\normalsize

The function ``strncmp'' tests whether ``string1'' is equal
to ``string2''; if so, the function returns 0 (boolean false
in C), otherwise it returns non-zero (boolean true in
C). The operator ``!'' negates the returned value of
strncmp: if string1 is equal to string2, the ``if
predicate'' evaluates to true, and thus, the statements
within \{..\} will be executed. In other words, the above
statement should be read as ``if string1 equals string2''.
Unfortunately, some participants interpreted the above
statement incorrectly as ``if not string1 equals string2''.

\item {\it Re-occurrence of the same mistakes (W2)}. 

This type of false positive was due to errors in the
previous statement.  One example is the following ``missing
else'' error:

\small
\begin{verbatim}
      if (condition)
         statement1;
      statement2;
      statement3;
      statement4; 
\end{verbatim}
\normalsize

The participants had already caught the ``missing else''
error, and yet they stated, for example, that statement3 was
wrong because statement2 was executed.  Another example is
raising this same issue again in statement4.

In the review guideline, the participants were specifically
instructed not to raise this type of issue.

\item {\it Syntactical mistakes (W3)}. 

This type of false positive was due to syntax errors in the
program, such as incorrect type assignment, missing
semicolon, etc.

In the review guidelines, the participants were told that
the code had no syntax errors, that is, it had been compiled
successfully.  Yet, some participants still raised this type
of issues.

\item {\it Stylistic mistakes (W4)}. 

This type of false positive was due to improper programming
styles, such as the use of ``for'' instead of ``while''.

In the review guidelines, the participants were instructed
not to raise stylistic issues, or to propose alternative
code; instead, they should look for programming errors.

\item {\it Misinterpretation of the specification (W5)}. 

This type of false positive was due to misinterpretation of
the given specification, usually, over generalizing the
specification.  For example, one subject stated that the
file needed to be opened before being used, although the
specification never mentioned it.

In the review guidelines, the participants were instructed
to review the code within the given specification. Anything
not provided in the specification should be considered
correct. The specification itself should always be
considered correct.


\item {\it Guessing (W6)}. 

This type of false positive was due to guessing. The
participants simply pointed out the location of the error
without describing what the actual error was.

In the review guideline, the participants were instructed to
provide the detailed explanation of the error, as well as
the location of the error.

\end{itemize}

In general, only W1 may be considered as ``true'' false
positives, the other types as ``procedural errors'', or
misinterpretation of the review guideline.

