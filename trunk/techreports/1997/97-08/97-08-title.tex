\begin{titlepage}
\vspace*{1in}
\begin{center}
   
\Large

{\bf Project LEAP:
     Lightweight, Empirical, Anti-measurement dysfunction, and Portable
     Software Developer Improvement}  

\bigskip\par
                                      \bigskip\par



\normalsize

Philip Johnson                           \medskip\par
Department of Information and Computer Sciences\\ 
University of Hawaii\\ 
Honolulu, HI 96822\\                       
(808) 956-3489\\
(808) 956-3548 (fax)\\
{\tt johnson@hawaii.edu}                 \bigskip\par

\today                                   \bigskip\par



\begin{abstract}
Project LEAP investigates the use of lightweight, empirical,
anti-measurement dysfunction, and portable approaches to software developer
improvement. A lightweight method involves a minimum of process
constraints, is relatively easy to learn, is amenable to integration with
existing methods and tools, and requires only minimal management investment
and commitment.  An empirical method supports measurements that can lead to
improvements in the software developers skill.  Measurement dysfunction
refers to the possibility of measurements being used against the
programmer, so the method must take care to collect and manipulate
measurements in a ``safe'' manner. A portable method is one that can be
applied by the developer across projects, organizations, and companies
during her career.

Project LEAP will investigate the strengths and weaknesses of this approach
to software developer improvement in a number of ways. First, it will
enhance and evaluate a LEAP-compliant toolset and method for defect entry
and analysis. Second, it will use LEAP-compliant tools to explore the
quality of empirical data collected by the Personal Software
Process. Third, it will collect data from industrial evaluation of the
toolkit and method. Fourth, it will create component-based versions of
LEAP-compliant tools for defect and time collection and analysis that can
be integrated with other software development environment
software. Finally, Project LEAP will sponsor a web site providing distance
learning materials to support education of software developers in
empirically guided software process improvement. The web site will also
support distribution and feedback of Project LEAP itself. 

\end{abstract}

\end{center}
\end{titlepage}
