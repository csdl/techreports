%%%%%%%%%%%%%%%%%%%%%%%%%%%%%% -*- Mode: Latex -*- %%%%%%%%%%%%%%%%%%%%%%%%%%%%
%% asis.tex -- 
%% Author          : Dadong Wan
%% Created On      : Tue May 12 23:47:41 1992
%% Last Modified By: Dadong Wan
%% Last Modified On: Thu May 14 14:03:44 1992
%% Status          : Unknown
%%%%%%%%%%%%%%%%%%%%%%%%%%%%%%%%%%%%%%%%%%%%%%%%%%%%%%%%%%%%%%%%%%%%%%%%%%%%%%%
%% 
%% 
%% History
%% 12-May-1992		Dadong Wan	
%%    
%% Created.
\documentstyle[/home/13/csdl/tex/aaai]{article}

\title{Supporting Collaborative Learning Through Research
Reviews\thanks{Support for this research was provided in part by the
National Science Foundation Grant CCR-9110861 and the University of Hawaii
Research Council Award R-91-867-F-728-B-270.}} 
\author{Dadong Wan\\ 
{\tt dxw@uhunix.uhcc.hawaii.edu}\\
Department of Information and Computer Sciences\\ 
University of Hawaii, Honolulu, HI 96822}

\begin{document}
\maketitle


\section{Introduction}

Research literature is not merely a formal vehicle for disseminating the
results of scientific research; it is also an important learning resource
for advanced learners (e.g., graduate students) and beginning researchers.
By closely studying the literature (instead of textbooks), the student can
gain a deeper appreciation of the content of the domain, the subtle
relationships among various research artifacts, and more importantly, the
nature of scientific knowledge construction, e.g., how to identify or
generate research problems, how to explore and evaluate alternative
solutions and claims and determine which ones are more viable, how to
engage in constructive scientific argumentation, how to build upon but not
confined by other people's work, how to effectively present research
findings, and so forth.  This level of learning typically takes place in
advanced educational forums such as graduate seminars, and manifested
mainly in the form of research reviews. Despite the increasing availability
of online databases and advanced retrieval techniques for searching the
literature, the review process remains largely a solitary activity by
individual learners with little computational support.

The purpose of this research is to investigate the structural and
process-level characteristics of research review as a collaborative
activity, and to develop computational mechanisms to facilitate such a
process. This paper briefly summarizes our ongoing efforts of developing a
software environment, called COREVIEW, for supporting collaborative
research review.  We begin with the description of the characteristics and
importance of group collaboration in research review. We then discuss the
role of the representation language in the process, and how it is realized
in COREVIEW.  We conclude with reasons why we would like to participate in
the workshop.


\section{Collaborative Research Review}

Research review is a process of critically examining available research
literature in an attempt to gain a better understanding of the subject
domain, to differentiate unique contributions of individual research, to
identify gaps within the existing knowledge, to formulate new research
problems, and to guide the development of viable solutions to those
problems. The primary purpose of research review is to help the beginning
researcher learn about the content, the underlying structure of and the
relationships among research artifacts, and the process by which they are
derived, revised and ignored, via such activities as summarization,
evaluation, integration, argumentation, and construction \cite{csdl-92-03}.
Research review, therefore, is not merely a preface to but an essential and
ongoing part of the entire research process.

Group collaboration plays a pivotal role in maximizing the learning
potential of research review. In a typical research seminar, for example,
most of learning is achieved via group-oriented activities, e.g., joint
reviews and presentations, topic-oriented discussions, brainstorming, peer
critiques, and so forth. The synergy of collaborative review comes from
both the shared concerns among the group members as well as the diversity
of their backgrounds, experiences, skills, interests, and intellectual
perspectives. The former provides a common basis as well as the motivations
necessary for group members to work together. The presence of individual
differences, on the other hand, offers a rich ground for
cross-fertilization, which may ultimately lead to better understanding of
the research literature, easier exposure of gaps and conflicting areas in
the existing knowledge, and easier generation of new insights and superior
alternative solutions.

Because of the diversity described above and the exploratory nature of
learning, research review demands both structural and procedural guidance:
the former answers the question of what should be reviewed, and latter, of
how reviews should be done. One essential characteristic of collaborative
research review is that representations at these two levels themselves are
the functions of the underlying group process, i.e., they are generated and
evolved by the group via ongoing discussions \cite{csdl-92-01}. To
characterize the dynamic interactions between the structure, process and
content of collaborative review and to provide explicit support for their
co-evolution form the basis of our research.

Computational support for collaborative research review offers considerable
potentials for enhancing the learner's abilities to learn from the
literature and from each other by removing the barriers inherent in the
non-mediated environment \cite{csdl-92-03}. The augmented system, for
example, allows structural representations and process-level guidelines to
be defined, extended, and applied in an incremental fashion. Under this
environment, the learners engage in conversation and argumentation at both
structural and content levels, create, critique and integrate artifacts as
well as feedback on them, and build group consensus on important issues. As
a side effect, the computer-supported research review allows artifacts of
collaboration to be captured and incrementally organized. This offers the
user the flexibility of ``replaying'' the process when needed.


\section{The Role of Representation}

To effectively support collaborative research review requires explicit
representation of the deep structure of the research literature and
scientific discourse. The Representational Schema of Research Artifacts, or
RESRA, is designed to be such a language \cite{csdl-92-03}. It is defined
as a set of entity and relation types; each of them is comprised of a
number of attributes. Guidelines and examples on the usages of these entity
and relation types are also provided. 

RESRA is intended for being used to represent, organize, and integrate
research and learning artifacts and activities. More specifically, it
serves the following important functions:

\begin{itemize}
  
\item an organizational tool that allows the learner to dynamically and
  incrementally integrate various types of research artifacts at a fine-grain
  level;
  
\item a mapping tool that highlights essential elements and relationships
  within as well as across research artifacts, and to help expose gaps and
  weak areas in the existing research, identify new problems, and suggest
  alternative solutions;
  
\item a communication tool, i.e., a shared ``frame of reference'' in
  group collaboration. To contrast different representations of the same
  artifact by different group members highlights the differences among
  group members, and to integrate them can lead to a deeper understanding
  of the problem domain; and
  
\item a learning tool to the beginning researcher on the conventions
  governing the written presentation of research findings.
  
\end{itemize}

It is important to realize that RESRA is inherently an open-ended language
whose structure and semantics are defined and evolved by the group.
Whatever is predefined serves merely as examples or stereotypes from which
the learner collaboratively instantiate their own domain- and/or
group-specific representations. The learners are also encouraged to engage
in structure-level conversation and argumentation, just as they do at the
content level.


\section{The COREVIEW System}

COREVIEW is a computer-augmented environment for supporting collaborative
reviews in research seminars. It provides full support for the RESRA
representational language. The implementation of COREVIEW is based on the
Egret's exploratory hypertext data model \cite{csdl-92-01}.

What differentiates COREVIEW from other hypertext-based collaborative
system, such as NoteCards \cite{Halasz87Notecards}, gIBIS \cite{Conklin88},
is the active role it assumes the user to play. Since RESRA models the deep
structure of research artifacts, using it to represent the content of the
literature is not merely a passive capturing but an active knowledge
construction process. Under this environment, the learner must consciously
learn to view the research world and to articulate themselves in those
terms. In return, COREVIEW helps the learner by providing case-based
guidance, easy interface, advanced navigation and visualization, and most
important of all, learning-enhancing computational services to compensate
for his/her efforts.


\section{Rationale for Participation}

Our research is unique in that it focuses on the learning component of
research review and explicit computational support. In particular, we treat
research review as a collaborative activity that requires advanced learners
to jointly construct and evolve a common representational language (i.e.,
RESRA) as well as guidelines and examples for using RESRA to represent and
integrate the research literature, to engage in constructive argumentation,
and to formulate new research. COREVIEW is designed to provide explicit
computational support for maximizing concomitant learning. This
learning-oriented approach toward research review forms a striking contrast
with the traditional information-based view, which emphasizes on the
efficiency, effectiveness, and timeliness of information processing,
retrieval and delivery.

We are still at an early stage of our research. Nevertheless, we believe
that our approach is promising in addressing important research issues. We
would like to use this workshop opportunity to present our research and to
receive early feedback from the workshop participants and the research
community at large. Such feedback, we hope, will help further our research.


\bibliography{asis}
\bibliographystyle{/home/3/dxw/c/tex/named-citations}

\end{document}


