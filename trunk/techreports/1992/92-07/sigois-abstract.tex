%%%%%%%%%%%%%%%%%%%%%%%%%%%%%% -*- Mode: Latex -*- %%%%%%%%%%%%%%%%%%%%%%%%%%%%
%% sigois-abstract.tex -- 
%% RCS:            : $Id$
%% Author          : Philip Johnson
%% Created On      : Mon Dec 27 10:42:16 1993
%% Last Modified By: Philip Johnson
%% Last Modified On: Mon Dec 27 11:22:28 1993
%% Status          : Unknown
%%%%%%%%%%%%%%%%%%%%%%%%%%%%%%%%%%%%%%%%%%%%%%%%%%%%%%%%%%%%%%%%%%%%%%%%%%%%%%%
%%   Copyright (C) 1993 University of Hawaii
%%%%%%%%%%%%%%%%%%%%%%%%%%%%%%%%%%%%%%%%%%%%%%%%%%%%%%%%%%%%%%%%%%%%%%%%%%%%%%%
%% 
%% History
%% 27-Dec-1993		Philip Johnson	
%%    

\documentstyle[/group/csdl/tex/ps-times,11pt,
               /group/csdl/tex/definemargins,
               /group/csdl/tex/lmacros]{article}


\input{/usr/uh/lib/tex/TeXPS/macros/psfig}

%%% make the final copy double spaced.
\definemargins{0.7in}{0.5in}{0.9in}{0.9in}{0.3in}{0.3in}
\begin{document}

\pagestyle{empty}

\begin{figure} [htbp]
 {\centerline{\psfig{figure=issue-private.ps}}}
\end{figure}

\begin{center}
{\large\bf Improving Software Quality through\\
           Computer Supported Collaborative Review}

\medskip\par
Philip M. Johnson and Danu Tjahjono\\
Department of Information and Computer Sciences\\
University of Hawaii\\
Honolulu, HI 96822\\
(808) 956-3489 (office), (808) 956-3548 (fax)\\
{\tt johnson@hawaii.edu},
{\tt dat@uhics.ics.hawaii.edu}
\end{center}

Formal technical review (FTR) is a cornerstone of software quality
assurance.  However, the labor-intensive and manual nature of review, along
with basic unresolved questions about its process and products, means that
review is typically under-utilized or inefficiently applied within the
software development process.  This paper describes CSRS, a
computer-supported cooperative work environment for software review
designed with two major goals.  First, it improves the efficiency of review
activities through a variety of computational services to reduce clerical
and adminstrative overhead.  Second, it supports empirical investigation
into improved review methods through fine-grained, high quality
instrumentation of the review process and products.  This paper presents a
typical scenario of CSRS in review, its data and process model, its
application to process maturation, its relationship to other research, and
its current status and future directions.

\end{document}



