\documentstyle [11pt, /home/3/johnson/Tex/definemargins,
                     /home/3/johnson/Tex/lmacros,
                     /home/3/johnson/Tex/named-citations]{article}  

\include{/home/3/johnson/Tex/figureholder}

\begin{document}
\begin{center}

{\Large\bf Collaborative Review and Design Rationale} 

{\large Philip Johnson\foot{Support for this research was provided in
part by the National Science Foundation Research Initiation Award
CCR-9110861 and the University of Hawaii Research Council Seed Money
Award R-91-867-F-728-B-270.}}\\
Department of Information and Computer Sciences\\ 
University of Hawaii\\ 
Honolulu, HI 96822\\
(808) 956-3489\\
{\tt johnson@uhics.ics.hawaii.edu}
\end{center}
\ls{1.2}

Representation of design rationale is difficult because design is
typically an exploratory activity.  The ultimate nature and form of a
design is emergent artifact of the process and depends in
unpredictable ways upon interactions between the domain of
application, the domain of implementation, the backgrounds of and
interactions between the designers, and so forth \cite{exploratory}.

In previous research we have investigated aspects of exploratory
programming and its evolutionary nature \cite{johnson}.  This work
convinced us of the importance of structural change in exploratory
processes.  Building upon this work, we are now refining an
implementation of a multi-user hypertext-based environment that
supports the construction of domain specific environments for various
exploratory domains.

A prominent feature of these domains is the utility of allowing the
representational structure to evolve over the course of group
activities.  This environment, called EGRET, is described in more
detail in the accompanying paper to this submission \cite{Egret}.

One of the domain specific environments under construction is for an
exploratory, collaborative approach to the review and inspection of
software designs and implementation.  Traditional approaches to the
review and implementation inspection process focus on the discovery of
errors in the particular design or implementation under inspection.
Our environment provides explicit support for additional aspects of
collaborative reviews that relate in an interesting way to
representation and support for design rationale.

Our perspective on collaborative review is that error discovery is an
important, but basically short-term benefit of the process. After all,
any additional work on the design artifact may introduce new errors,
and the benefits of error discovery are intrinsically limited to a
single project.  We believe that review activities have long-term
benefits that are neither emphasized nor supported, similar to the way
representation and support for design rationale has been neglected in
software engineeering methods and tools until recently.  Here are some
examples of long-range benefits we are designing support for and their
relationship to design rationale:
\begin{itemize}

\item {\em Support for emergent structural properties of review.}
While general guidelines and structural templates for review
activities have been published, the review (and the design process)
structure 

\end{itemize}
