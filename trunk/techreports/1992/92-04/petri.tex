\documentstyle[12pt]{article}
\addtolength{\oddsidemargin}{-.75in}
\addtolength{\evensidemargin}{-.75in}
\addtolength{\textwidth}{1.5in}
\addtolength{\topmargin}{-1.0in}
\addtolength{\textheight}{2.0in}
\setlength{\parindent}{0in}
\setlength{\parskip}{2\baselineskip}


\begin{document}

\begin{LARGE}
\mbox{}\\
\begin{center}
\mbox{}\\

{\bf A Minimalist Model\\ for Coordination}

\mbox{}\\
{\Large (To appear in
{\it Enterprise Integration Modeling:\\
Proceedings of  the First International Conference}.)}
\mbox{}\\
\mbox{}\\
\mbox{}\\
 Charles Petrie\\  MCC\\
\end{center}


\newpage
\begin{center}
{\bf Problem:}
\end{center}

\noindent
Design replay methods require too much 
\begin{itemize}
\item domain theory
\item from user
\end{itemize}
\mbox{}\\
\mbox{}\\

\begin{center}
{\bf Approach:}
\end{center}

\noindent
Identify {\it small, domain-independent} ontologies\\ with {\it
theories} (e.g., REDUX) \\that provide a lot of power for a little
information.


\newpage
\begin{center}
{\bf Services:}
\end{center}

\begin{itemize}
\item Rationales used to detect 
 \begin{itemize}
 \item {\it opportunites} and 
 \item {\it conflicts} 
 \end{itemize}
in shared work with subgoaling.
\item Support for 
 \begin{itemize}
 \item {\it backtracking} and 
 \item {\it contingencies}
 \end{itemize}
\end{itemize}
\mbox{}\\


\begin{center}
{\bf Representation:}
\end{center}

\begin{itemize}
\item Common ontology based on
  \begin{itemize}
  \item generic dependency analysis
  \item not domain theory
  \end{itemize}
\item Minimal domain semantic unification\\
(E.g., color/box $\equiv$ Farbe/Kasten)
\end{itemize}


\newpage
\begin{center}
{\bf Methods:}
\end{center}

\begin{itemize}
\item Common ontology used to structure email
\item Domain semantic deviation problems minimized
\item REDUX only small part of what's needed
  \begin{itemize}
  \item multimedia sharing
  \item IBIS-style arguments
  \item domain-specific application support
  \end{itemize}
\item Need experiments to show if even this \\minimal replay approach is
worthwhile 
\end{itemize}

\end{LARGE}
\end{document}
