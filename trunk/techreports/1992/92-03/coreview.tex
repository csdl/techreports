%%%%%%%%%%%%%%%%%%%%%%%%%%%%% -*- Mode: Latex -*- %%%%%%%%%%%%%%%%%%%%%%%%%%%%
%% coreview.tex -- 
%% Author          : Dadong Wan
%% Created On      : Sun Apr 12 18:37:29 1992
%% Last Modified By: Dadong Wan
%% Last Modified On: Thu Apr 16 14:19:34 1992
%% Status          : Unknown
%%%%%%%%%%%%%%%%%%%%%%%%%%%%%%%%%%%%%%%%%%%%%%%%%%%%%%%%%%%%%%%%%%%%%%%%%%%%%%%
\documentstyle [11pt,/home/3/dxw/c/tex/definemargins,
/home/3/dxw/c/tex/lmacros,
/home/3/dxw/c/tex/named-citations]{article}  
\input{/home/3/dxw/c/tex/psfig}
\special{header=/home/13/csdl/tex/psfig/lprep71.pro}

\definemargins{0.75in}{0.75in}{0.75in}{0.75in}{0.3in}{0.3in}
\pagestyle{empty}

\begin{document}
\ls{0.9}
\begin{center}
  {\large\bf Supporting Scientific Learning and 
  Research Review Using COREVIEW}\foot{Support for this
  research was provided in part by the National Science Foundation
  Research Initiation Award CCR-9110861 and the University of Hawaii
  Research Council Seed Money Award R-91-867-F-728-B-270}\medskip\par 
  
  \bigskip
  Dadong Wan and Philip Johnson \medskip\par
  Department of Information and Computer Sciences\\
  University of Hawaii, Honolulu, HI 96822\medskip\par
  
  {\tt dxw@uhunix.uhcc.hawaii.edu}\\
  {\tt johnson@uhunix.uhcc.hawaii.edu}\medskip\par
\end{center}

\begin{abstract}
  Scientific learning and research are becoming increasingly
  computerized. More and more such activities are mediated through
  electronic artifacts. This paper presents an artifact-based system
  called COREVIEW, to be used in the domain of research seminars. The
  emphasis of our approach is on the centrality of textualized artifacts
  in seminar activities, the relationship between different types of
  artifacts, and the dynamic interactions among them over time. Our
  system provides explicit representation of research artifacts and their
  structures, and support for the process of collaborative artifact
  generation, integration, manipulation and utilization.
\end{abstract}


\section{Introduction}

Scientific learning and research are inherently social and artifact-ladden
activities. As knowledge environments become increasingly computerized,
more and more such activities are mediated through electronic artifacts.
The trend of this embodiment of both processes and outcomes in written
artifacts, called {\it textualization}, can help improve existing
activities by making explicit the formerly implicit and uncaptured results
of verbal, face-to-face interactions. Effective computer support for
textualized group activities also enables the historical record, or process
of group interaction, to be preserved and progressively organized.
Successfully capturing, representing, relating and evaluating the process
of group interaction via its artifacts can provide many potential benefits:
a description of the rationale by which decisions and products were
created; a database of intermediate artifacts that might be reused in other
contexts and from which, a higher-order of knowledge might be constructed;
and perhaps most importantly, a means to better understand and ultimately
support the process by which scientific learning and research are
accomplished. From the learner's standpoint, the demand for relating and
integrating new research artifacts into the existing but evolving
structural context helps them better appreciate the inter-relationships
among various artifacts.

In this paper, we present COREVIEW\foot{\underline{Co}llaborative
\underline{Review}; pronounced as {\it ko-ri-vyu\/}, not {\it kor-vyu\/.}},
an artifact-based system designed to support collaborative learning and
research review in seminars. The next section describes the domain and some
existing problems within it. The following sections discuss the underlying
representation scheme, called RESRA, that models characteristics of
research artifacts, and an example use of this scheme. We conclude with
related work and future directions.


\section{Collaboration in Research Seminars}

Research seminars are an important part of science education, particularly
at the graduate level. They are most commonly employed to cover frontier
research areas of a field, or new, emergent topics that are too young to
find themselves in regular curriculum.  Seminars are also used to bridge
``gaps'' often found between different disciplines, and to meet specialized
educational and research interests of faculty and students. Unlike
traditional lecture-oriented instructions, research seminars are known for
their flexibility and diversity in content, structure and style. The format
of seminars may vary widely based on the subject matter as well as the
expectations, interests, and backgrounds of the seminar leader and
participants. Despite such variations, however, research seminars share a
set of common characteristics: the orientation toward research literature;
the emphasis on developing critical understanding and thinking skills
through intensive reading and writing, oral presentation, and classroom
discussions; and the demand for high-levels of involvement and
collaboration among seminar participants.


\subsection{Collaborative Learning in Research Seminars}

Collaborative learning plays an essential role in seminar activities. One
primary benefit of a research seminar, in fact, is to teach advanced
students and beginning researchers how to work in a collaborative fashion.
Participants can learn not only from reading literature individually or
from lectures, but also from each other through presentations,
topic-oriented discussions, brainstorming, and peer review of their
activities and artifacts. Joint reviews, presentations, and seminar
projects are ideal ways by which individuals of different backgrounds,
skills, perspectives and experiences can work together on shared tasks.

One important characteristic of collaborative learning in seminars is its
artifact orientation. Although participants may spend significant amount of
class time in face-to-face interactions, much collaborative learning takes
place through creating, digesting, evaluating, transforming,
inter-relating, and sharing various kinds of written artifacts from both
external and internal sources, e.g., research literature, original reviews,
project proposals, written feedback from others, and so forth. The
performance of individual participants is also largely evaluated on the
basis of the quality of the artifacts they produce and the process through
which they are produced.

Two essential points from the preceding discussion should be emphasized.
First, collaborative learning in research seminars takes place on two
levels: one very domain-specific level involving the particular subject
matter addressed in the seminar, e.g., machine learning or object-oriented
design; and one more domain-independent level involving how to collaborate,
how to research literature, how to present and evaluate research artifacts,
how to identify interesting problems and develop novel solutions, and so
forth. The individual artifacts produced during the seminar process may
contain a mixture of domain-specific and domain-independent information.

Second, the structure and process governing collaborative learning in a
research seminar will evolve during the course of the semester. At first,
the students will probably know little about the domain and about the
process of conducting research. By the end of the semester, students should
be much more facile on both of these levels.  This progress will be
manifested by changes in the types of artifacts produced and the process by
which they are produced.


\subsection{Barriers to Collaborative Learning} \label{sec-barriers}

The effectiveness of the traditional seminar as a collaborative learning
environment is severely circumscribed by its built-in barriers, five of
which are listed below:

\begin{itemize}
\item {\it Face-to-face barriers.} Face-to-face barriers can be either
  interpersonal or intercultural/interlingual. In a seminar setting,
  discussions can be dominated by a few ``strong personalities'' or by the
  seminar leader(s). Individual contributions to the group discussion might
  be inhibited because some participants do not feel comfortable speaking
  openly in a group, or expressing verbal disagreement with other
  participants, especially the leader. In seminars composed of people from
  different cultural and linguistic backgrounds, intercultural gaps and
  differences in language fluency might prevent the minorities from full
  participation.
  
\item {\it Same-place, same-time constraints.} In a conventional seminar,
  physical co-presence in classroom is a prerequisite for participation.
  Interactions among participants often take place only when they meet
  face-to-face. Collaborative possibilities rarely go beyond the boundaries
  of the classroom. It is also often difficult to moderate a class session
  in such a fashion that all participants are heard from and that a broad
  range of potential collaborative activities are accomplished within a
  relatively short period of time.
  
\item {\it Discontinuity across different seminar sessions.} Face-to-face
  interaction can be very effective, but its effectiveness is bounded by
  its transitory nature. Continuity between seminar sessions is difficult
  to maintain without a disciplined long-term {\it memory\/}, i.e.,
  explicit tracking of both processes and activity contents and physical
  means of connecting them together. This is particularly true when
  meetings take place on a relatively infrequent basis and participants are
  physically distributed, as often the case with most seminars.
  
\item {\it Lack of organic links between reading, writing, presentation
  and discussion, and between project planning and actual project
  activities.} Seminar activities are inherently both integrative and
  exploratory. To support either type of activity requires explicit
  representation and manipulation of links between various activity
  structures, contents, and processes. In traditional seminars, however,
  the awareness, storage and retrieval of such connections are largely left
  to individual participants. As a group, there exists no external {\it
  pool\/} of mutual artifacts which group members can contribute to and
  benefit from.
\end{itemize}

Many of these barriers might fall in the presence of explicit mechanisms
for representing and supporting artifact production, evaluation,
manipulation, integration and utilization. In the following sections, we
propose a representation scheme that allows the learner to overcome these
barriers, illustrate it with an example, and discuss how the representation
enhances collaborative learning.


\section{Representation of Seminar Artifacts}

The basic premise of the artifact-based approach is that written records
are primary objects around which seminar activities are organized. Seminar
syllabi, for example, are specifications of what types of artifacts to be
produced at what time and in what format (e.g., ``A research proposal of
10-15 pages is due the last day of class''). The research process itself
consists of transformations of artifacts and can be expressed as a set of
temporally ordered artifacts (e.g., topic identification, problem
description, literature review, draft proposal and final proposal).
Seminar activities, in essence, boil down to artifact manipulation,
including: generation, digestion, transformation, evaluation, aggregation,
inter-linking, presentation, and so forth. An artifact-based environment
must therefore organize its functionality around these activities, and
allow the learner to break down artifacts into semantically useful pieces
and to organize and present those pieces in ways that facilitate learning
and research activities.


\subsection{Categories of Artifact Activities} \label{sec:activity}

Table ~\ref{tab-act} shows the five categories of activities found in
typical research seminars. Note that the actual division between these
types may never be as clear-cut as the table shows. For example, an
argument or debate is often filled with constructive ideas and value-ladden
judgement.  The order in which the activity types appear in the table also
does not dictate the actual process steps in which they occur, even though
the sequence is typical.

One essential representation requirement is to the ability to accurately
express the semantics of artifacts generated from all these five types of
activities, and to easily relate and integrate them in some useful ways.
RESRA, described next, is such a language.

\begin{table}[ht]
  \ls{0.8}
  \begin{center}
    \begin{tabular} {|l|p{2.5in}|p{2.5in}|} \hline   
      
      {\bf Activity Type} & {\bf Description} & {\bf Example Artifacts} \\ \hline
      
      Summative & Extracting and summarizing important elements from
      an artifact with or without a predefined guideline. & List of
      research hypotheses, conclusions and future directions. \\
      \hline
      
      Evaluative & Subjective appraisal of another work without
      having to offer alternatives. & List of major strengths and
      weaknesses of a given research paper. \\ \hline
      
      Integrative & Relating, organizing, aggregating or abstracting
      existing work. & A state-of-the-art survey on a given research
      topic. \\ \hline
      
      Argumentative & Interactions among polarized point of views or
      positions on a given topic. & Recorded script of a cross-fire
      panel discussion on a controversial topic. \\ \hline
      
      Constructive & Alternative conception of a problem, solutions or
      interpretations. & IBIS, PHI, and QOP representations of design
      rationale. \\ \hline
    \end{tabular}
    \caption{Categories of Artifact Activities in Research Seminars}
    \label{tab-act}
    \ls{0.9}
  \end{center}
\end{table}


\subsection{RESRA: Representation Schema of Research Artifacts}

RESRA, i.e., REpresentation Schema of Research Artifacts, is a specialized
language used to represent, organize, and integrate research and learning
artifacts generated from both within and outside of seminars.  RESRA is
composed of two primitives: {\it entity\/} and {\it relation\/}.  The
former describes the type and the internal structure of artifacts and the
latter, inter-relationships among various {\it entities\/}. Table
~\ref{tab-er} describes the nine types entities currently defined in RESRA.
Figure ~\ref{fig-resra} shows a graphical representation of these entities
as well as the relationships among them.

\begin{table}[ht]
  \ls{0.8}
  \begin{center}
    \begin{tabular} {|l|p{2.5in}|p{2.5in}|} \hline   
      {\bf Entity Type} & {\bf Description} & {\bf Example} \\ \hline 
      
      Source & Original artifacts or their surrogates. & Fagan's original paper
      on code inspection; David's research proposal on CI. \\ \hline
      
      Topic & Short synopsis of an area of learning or research. &
      Improving software reliability through code inspection. \\
      \hline
      
      Problem & Aspect or element of a topic that requires further
      inquiry; may include background and motivation. & Current
      software verification and validation techniques are not
      adequate for uncovering software errors. \\ \hline
      
      Proposition & Statements that can be confirmed or disconfirmed
      via further investigation. & Proper use of check-points in
      code inspection can improve software productivity. \\ \hline
      
      Method & Procedures, models, or actions employed to generate evidence for
      testing a given proposition. & Fagan's formal model of code
      inspection; his experiment on inspecting OS design segments. \\ \hline
      
      Outcome & Conclusions, inferences, or generalizations made with
      regard to a particular proposition. & Code inspection
      is an effective means for reducing software errors. \\ \hline
      
      Concept & Primitive construct used as building blocks for
      theories, propositions and methods. & code inspection;
      inspection entry criteria. \\ \hline
      
      Theory & Systemic interpretation or abstraction of a problem
      situation; may be true or false. & Ausubel's theory on
      meaningful learning. \\ \hline
      
      Appraisal & General comments and evaluations on a particular
      work. & Fagan's CI model is too rigid and tedious to apply. \\
      \hline
    \end{tabular}
    \caption{Entity Types of RESRA}
    \label{tab-er}
    \ls{0.9}
  \end{center}
\end{table}

\begin{figure}[htb]
    \centerline{\psfig{figure=resra.ps,height=3.6in,clip=}}
    \caption{A Graphical Representation of RESRA}
    \label{fig-resra}
\end{figure}


RESRA offers a consistent and effective way of representing, relating and
integrating artifacts generated by all five types of seminar activities
described in Section ~\ref{sec:activity}:

\begin{itemize}
\item {\it Summative.\/} To represent summative artifacts involves simple
  creation of entities of various types. For example, to summarize an
  empirical study requires {\it problem\/}, {\it proposition\/}, {\it
  method\/} and {\it outcome\/}. The use of RESRA as a summative tool also
  helps the learner get deeper into the content of the literature than he
  might otherwise. A research paper, for instance, may contain implicit
  problems or propositions. RESRA helps expose such omissions.
  
\item {\it Evaluative.\/} Evaluating a given piece of research can be
  {\it shallow\/} or {\it deep\/}. Shallow evaluation, such as listing
  major strengths and weaknesses of a paper, does not necessarily require a
  good understanding of the entire content. It can be easily represented by
  attaching an {\it appraisal\/} to a corresponding {\it source}, {\it
  problem\/}, {\it proposition\/} or {\it outcome\/}. Deep evaluations, on
  the other hand, go beyond simple annotation via {\it appraisals\/}. It
  demands at least some level of integrative and even constructive efforts
  on the part of the learner, and the creation of other entities (e.g.,
  {\it problem\/}, {\it proposition\/}) as well as relations between them.
  
\item {\it Integrative.\/} The integration of new research knowledge with
  what the learner already knows requires the capabilities to relate,
  aggregate and generalize. With RESRA, this is achieved by adding links
  between related entities and the creation of higher-level entities.
  Writing a good literature review paper, for instance, involves more than
  just rehashing existing artifacts; it also needs a uniform framework that
  helps bring together and make sense of related artifacts.
  
\item {\it Argumentative.\/} Unlike rhetoric models (e.g., IBIS
  \cite{Conklin88Gibis}) that use generic entities such as {\it issue\/} or
  {\it position\/}, RESRA supports argumentation using entities and
  relations that characterize the structure of research artifacts. Although
  there is a close correspondence between {\it problem\/} of RESRA and {\it
  issue\/} of IBIS, {\it position\/} and {\it argument\/} do not exist in
  RESRA; instead, they are subsumed by more domain-specific entities (e.g.,
  {\it proposition\/}, {\it method\/}) and relations (e.g., {\it
  is-alternative-to\/}). In RESRA, the argument or evidence for a given
  position is placed together with its corresponding entities, such as {\it
  concept\/} and {\it outcome\/}.
  
\item {\it Constructive.\/} RESRA provides many ways to facilitate the
  construction of new knowledge. It serves as a useful thinking and
  organizational tool. Since RESRA represents artifacts in terms of
  problems, propositions, methods and outcomes, it forces the beginning
  researcher to focus more on important elements of the research literature
  instead of noncritical details. Aided with good query, navigational and
  visualization tools, RESRA can make it easier to identify new problems,
  expose gaps and weak areas of existing research, and suggest alternative
  solutions.
\end{itemize}

Most of the above description is restricted to the predefined primitives in
RESRA. It is, however, important to realize that RESRA is an open language
whose structure and semantics can evolve over time.  For example, in the
existing RESRA, evidence for supporting a given {\it problem}, {\it
proposition}, or {\it method \/} is treated as part of the entity. As time
goes on and more research addresses the same problem, the amount of
evidence can become large, and the relationships among them very complex.
At this point, evidence may need to be treated as an independent entity in
its own right. At the outset, however, such a treatment may not be
justifiable.


\subsection{An Example Use of RESRA}

Suppose that, in a software engineering seminar, David is interested in a
research project on improving software quality and reliability through code
inspection (CI). Since David is new to the field, the first thing he does
is familiarizing himself with the research literature. David reads the
classical paper on code inspection by Fagan, and represents its content
using RESRA. Figure ~\ref{fig-eg1} shows a portion of David's work. The
small circles in the diagram represent the types of RESRA entities.

By going through the RESRA exercise, David comes to know that Fagan's paper
contains three main research propositions, two of which are supported by
the reported empirical evidence, and one of which remains open (i.e.,
``data from the CI process can be used to improve software process''). In
an attempt to test his propositions, Fagan defines a formal model of code
inspection, which David puts under both {\it concept\/} and {\it method\/}.
The former seems obvious. David's reasoning on the latter is that Fagan
operationalizes the concept of CI by specifying many details of exactly
what should and should not be included as part of the process.

Despite the empirical evidence for supporting Fagan's position, David has a
few questions on the formal model. This is indicated by the {\it
appraisals\/} David has created.  One of the questions, i.e., failure to
address the role of group dynamics in the CI process, leads to a research
problem that later becomes David's thesis project.

One area which David finds RESRA particularly helpful is in identifying
implied statements in the paper. For example, Fagan does not explicitly
describe the problem he intends to address. Rather, he dives directly into
the three research propositions (see Figure ~\ref{fig-eg1}). Hence, David
has to infer the unstated problem description from the propositions, and
then represent it in the form of RESRA. Given the diversity of research
literature, such inference tasks on the part of the learner can be
substantial. On the other hand, this process also yields valuable insight
on the process of scientific research.

\begin{figure}[ht]
    \centerline{\psfig{figure=fagan.ps,height=5.7in,clip=}}
    \caption{A Partial Representation of Fagan's Paper on Code Inspection}
    \label{fig-eg1}
\end{figure}

Two important observations from this example can be made. First, RESRA is
applicable to both the review of existing literature, as illustrated above,
as well as to the construction of new research artifacts. In fact, the real
rigor of the representation may only be fully realized through seamlessly
integrating these two streams of activities. For example, the problem,
propositions and methods for David's thesis project can be naturally
resulted from the process of trying to represent, relate, and integrate
existing literature in RESRA.

The other observation is that optimal use of RESRA requires a collaborative
environment. For the sake of simplicity, the example shows only a
single-user case. RESRA allows diverse representations of the same artifact
and their subsequent integration through such activities as discussion and
presentation. These services are exactly what COREVIEW attempts to provide.


\section {COREVIEW: Supporting Collaborative Artifact Activities}

RESRA is implemented in COREVIEW, a computer-augmented environment for
facilitating collaborative seminar activities. The system is designed to
overcome the four barriers with traditional seminars described in Section
~\ref{sec-barriers}. Table ~\ref{tab-cv} summarizes how this support is
realized.

\begin{table}[ht]
  \ls{0.8}
  \begin{center}
    \begin{tabular} {|p{2.5in}|p{3.5in}|} \hline   
      {\bf Type of Barriers} & {\bf COREVIEW  Mechanisms} \\ \hline 
       
      Face-to-face barriers & Computer-supported textualization; user
      definable process model. \\ \hline
       
      Same-time, same-place constraints & Integration of synchronous
      and asynchronous modes of interaction; presentation support\\
      \hline
      
      Discontinuity across seminar sessions & Artifact integration
      using RESRA; presentation support \\ \hline
      
      Lack of organic links between different types of artifacts &
      RESRA; support for structural evolution; advanced navigational
      tools \\ \hline
    \end{tabular}
    \caption{COREVIEW Mechanisms for Overcoming Seminar Barriers}
    \label{tab-cv}
    \ls{0.9}
  \end{center}
\end{table}

COREVIEW supports both domain-specific and domain-independent collaborative
learning. Since RESRA models the deep structure of research artifacts, the
use of it to represent the content of seminar activities is no longer a
passive capturing and accumulation of artifacts but an active knowledge
construction process. Within this environment, the learner must consciously
learn to view the research world and to articulate ideas and thoughts in
the terms defined in that framework. As a result, a more integrated
domain-specific knowledge base can be generated by the seminar participants
over the course of a semester. Meanwhile, the use of RESRA to analyze and
express the content of research literature, construct proposals, and
organize presentation and discussions also helps the learner to gain a
better understanding the process of doing original research, collaborating
with peers, making presentations, and so forth. In fact, RESRA may serve as
a good meta-level guide that teaches the learner how to learn and how to
do research.


\section{Related Work}

COREVIEW represents a confluence of several streams of research. The
importance of artifacts in group settings has been explored in deliberation
of design rationale in software development \cite{Lee90}. The development
of RESRA is inspired by positive results from studies on the use of concept
maps to enhance science learning \cite{Cliburn90}. The implementation of
COREVIEW is heavily shaped by hypertext technology
\cite{Conklin88Gibis,Halasz88Reflections}, and existing collaborative
systems, i.e., NoteCards \cite{Halasz87Notecards} and gIBIS
\cite{Conklin88Gibis}. The exploratory support of COREVIEW is based on the
EGRET framework \cite{csdl-92-01}.

The potential of computer-mediated environments in facilitating
collaborative learning have long been recognized and widely researched by
both educators and researchers \cite{Hiltz88Collaborative,Butler92}.
However, most of these studies look only at loosely structured,
asynchronous messaging \cite{Butler92} and computer conferencing systems
\cite{Hiltz88Collaborative}.  Our approach, on the other hand, focuses on
much more tightly coupled settings, and attempts to bridge the gap between
synchronous and asynchronous interactions.


\section{Future Directions}

We are at an early stage of this research. Nevertheless, we believe that
COREVIEW will provide us a rich environment for exploring many promising
research issues related to the acquisition, construction and communication
of scientific knowledge among advanced learners and beginning researchers.
One potential application of COREVIEW outside of research seminars is to
support the peer refereeing and reviewing process of scientific journals
\cite{Lipetz91}. At the implementation level, we are planning to extend
COREVIEW to support Internet worldwide bulletin board system, wide area
information network (WAIS), multimedia, and so forth, and to transport it
to other platforms.


\bibliography{coreview}
\bibliographystyle{/home/3/dxw/c/tex/named-citations}

\end{document}



