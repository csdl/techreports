%%%%%%%%%%%%%%%%%%%%%%%%%%%%%% -*- Mode: Latex -*- %%%%%%%%%%%%%%%%%%%%%%%%%%%%
%% 04-14-contribution.tex -- Thesis white paper - software inspections
%% Author          : Aaron A. Kagawa
%% Created On      : Mon Sep 23 11:52:28 2004
%% Last Modified By: Aaron Kagawa
%% Last Modified On: Fri Feb  4 17:02:39 2005
%% RCS: $Id$
%%%%%%%%%%%%%%%%%%%%%%%%%%%%%%%%%%%%%%%%%%%%%%%%%%%%%%%%%%%%%%%%%%%%%%%%%%%%%%
%%   Copyright (C) 2004 Aaron A. Kagawa
%%%%%%%%%%%%%%%%%%%%%%%%%%%%%%%%%%%%%%%%%%%%%%%%%%%%%%%%%%%%%%%%%%%%%%%%%%%%%%%
%% 

\chapter{Future Directions}
\label{chapter:contribution}

%%\section{Future Directions}
This proposed research contains many limitations. Most notably the
evaluation of PRI and hackyPRI is constrained to only one specific software
project. This fact raises many issues of adoption. For example, how hard
would it be to implement PRI at another organization? How hard would it be
to calibrate PRI for another set of product and process measures? This
adoption issue can be addressed by future evaluations of PRI in other
organization settings and other software projects. However, this issue will
be left as a future direction, as I neither have the time or resources to
conduct such a thorough evaluation. However, I believe the proposed
evaluation taken in this thesis is necessary to prove or disprove that PRI
is a worthy concept to try at other organizations. In addition, future work
is needed to generalize the hackyPRI extension so that it it possible for
other organizations and/or projects. Currently, hackyPRI is tailor-made for
the CSDL organization and for the Hackystat project.

Priority Ranked Inspection was originally created for quality purposes that
span other quality improvements techniques other than software inspections.
Originally, I proposed a technique that could identify the lowest cost
approach to increase quality of a particular piece of code.  I envisioned a
Hackystat extension that could identify the right ``quality tool'' that is
needed to increase quality. For example, if the ranking showed that Unit
Tests are a problem area, then the right ``quality tool'' could be
increasing the number of Unit Tests for that particular piece of code.
However, for this proposed research I have obviously decided to focus on
one ``quality tool'', namely software inspection. I choose inspections
because the literature suggests that this process is the most effective way
to increase quality. Another future direction for this research is to
evaluate if Priority Ranked Inspection can also identify the right
``quality tool'' to use in specific situations.







