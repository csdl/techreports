%%%%%%%%%%%%%%%%%%%%%%%%%%%%%% -*- Mode: Latex -*- %%%%%%%%%%%%%%%%%%%%%%%%%%%%
%% 04-14-relatedwork.tex -- Thesis white paper - software inspections
%% Author          : Aaron A. Kagawa
%% Created On      : Mon Sep 23 11:52:28 2004
%% Last Modified By: Aaron Kagawa
%% Last Modified On: Fri Feb  4 16:37:34 2005
%% RCS: $Id$
%%%%%%%%%%%%%%%%%%%%%%%%%%%%%%%%%%%%%%%%%%%%%%%%%%%%%%%%%%%%%%%%%%%%%%%%%%%%%%
%%   Copyright (C) 2004 Aaron A. Kagawa
%%%%%%%%%%%%%%%%%%%%%%%%%%%%%%%%%%%%%%%%%%%%%%%%%%%%%%%%%%%%%%%%%%%%%%%%%%%%%%%
%% 

\chapter{Related Work}
\label{chapter:relatedwork}
This chapter presents the work related to Priority Ranked Inspection.  The
initial invention of PRI can be attributed the current traditional
inspection literature's consistent lack of information on the selection of
documents for inspection. Previous work on software inspection has focused
on the process in which inspection is conducted. Instead, this proposed
research focuses on the selection of documents for inspection.


\section{Fagan Inspection}
Michael E. Fagan invented inspections in 1976 while working at IBM.
``Inspection'', with a capital ``I'', or the term ``Fagan Inspection'' is
used when referring to his technique. Using Fagan Inspection, Bell Labs
reported 14 percent productivity increase, better tracking, early defect
detection, and more importantly the employees credited Fagan Inspection
with an ``important influence on quality and productivity'' \cite{Gilb93}.

\section{Software Inspection}
One of the most widely accepted type of inspection is ``Software
Inspection'', which was developed by Tom Gilb and Dorothy Graham in the
book of the same title. Software Inspection is based on the Fagan-style
Inspection and is generally more robust and disciplined than other
techniques.

Software Inspection is defined as a two part process, product Inspection
and process improvement. According to the Software Inspection literature,
product Inspection and process improvement cannot and should not exist
without one another.

\subsection{Project Inspection}
There are ten lengthy steps in the product Inspection portion of the
Software Inspection process. I have provided a short description of each of
the steps in the following sections.

\begin{flushleft}
  \textit{Request: Initiating the Inspection Process} \\ This
  Inspection process begins with an author's voluntary request for an
  Inspection. The request is delegated to an Inspection leader. An
  Inspection leader is a trained-and-certified employee and is generally
  not a manager. It is the leader's responsibility to organize, plan and
  conduct the inspection.
\end{flushleft}

\begin{flushleft}
  \textit{Entry: Making Sure 'Loser' Inspections don't Start} \\ The
  Inspection leader is required to check the volunteered document against
  an Entry Criteria. This criterion ensures that the document is worth
  inspecting. The leader conducts a quick look through the document to
  assess the initial quality of the document. For example, the author has
  spent an adequate amount of time working on the document, there are a
  minimal number of minor defects, etc. ``The purpose of having entry
  criteria to the Inspection process is to ensure that the time spent in
  Inspecting the product and associated documents is not wasted, but is
  well spent'' \cite{Gilb93}.
\end{flushleft}

\begin{flushleft}
  \textit{Planning: Determining the Present Inspection's Objectives and
    Tactics} \\ If, and only, if the document has successfully passed the
  entry criteria, then the Inspection leader can begin to plan the
  Inspection. This includes many managerial tasks; inviting participants,
  scheduling an Inspection meeting, gathering supportive documentation,
  establishing average optimum checking rates and suggesting areas of
  possible improvement in the document.
\end{flushleft}

\begin{flushleft}
  \textit{Kickoff Meeting: Training and Motivating the Team} \\ The purpose
  of a kickoff meeting is to ensure that Inspection process begins
  correctly. This includes dispensing required documents and explaining the
  expectations of the participants. This meeting saves time by dispensing
  the necessary information, which is needed to conduct the Inspection.
  This meeting is also an opportunity to introduce process changes in the
  Inspection process.
\end{flushleft}

\begin{flushleft}
  \textit{Individual Checking: The Search for Potential Defects} \\ The
  participants, or `` checkers'', are required to work alone to find
  potential major defects in the documents provided. These defects are
  generally identified with the aid of rules, checklists, and other
  standards of the organization. 
\end{flushleft}

\begin{flushleft}
  \textit{Logging Meeting: Log Issues Found Earlier and Check for More
    Potential Defects} \\ This meeting has three purposes: log the issues
  generated in the individual checking phase, discover more major defects,
  and identify possible ways of improving the inspection process. This
  meeting is conducted and moderated by the Inspection leader. 
\end{flushleft}

\begin{flushleft}
  \textit{Edit: Improving the Product} \\ The overall goal of Inspection
  is to remove the defects that were found. During this phase, the author
  is given a list of the issues (issues become defects if they are deemed
  as valid defects) that were identified and is required to make the
  necessary improvements to remove any defects from the document.
\end{flushleft}

\begin{flushleft}
  \textit{Follow up: Checking the Editing} \\ The purpose of this phase is
  to ensure that the Edit phase was correctly executed by the author. The
  Inspection leader must ensure that all issues are correctly classified,
  either as valid defects or invalid issues and that the author has
  corrected all known defects.
\end{flushleft}

\begin{flushleft}
  \textit{Exit: Making Sure the Product is Economic to Release} \\ The
  Inspection leader consults the exit criteria to determine if the
  inspected document contains a certain level of quality as defined by the
  exit criteria. For example, the Exit criteria can contain rules that
  specify: successful Follow Up phase completion, certain metrics about
  this particular Inspection was recorded and within limits, and that the
  number of defects are below a certain threshold.
\end{flushleft}

\begin{flushleft}
  \textit{Release: The Close of the Inspection Process} \\ This is the last 
  phase of the Inspection process. At this point, the document can be
  officially released and the Inspection process is concluded. However, if
  it is determined that there are some acceptable/unavoidable defects
  remaining in the document, then such defects must be documented.
\end{flushleft}


\subsection{Process Improvement}
Equally important to the product Inspection portion of Software Inspection
is process improvement. Process improvement is the continuous improvement
of the entire software development process. The idea is simple; Inspections
can remove defects, but process improvement can prevent defects.

In Software Inspection, process improvement can be accomplished in many
ways. A low-cost procedure could be as simple as discussing the cause of
the defects. This discussion takes place in a Process Brainstorming
Meeting. ``The purpose of the process brainstorming meeting is \textit{not}
to deal with the document and its defects. It is to deal with the
\textit{causes} of those defects'' \cite{Gilb93}.

On the other hand, process improvement can be very expensive, Process
Change Management Teams. Specialized teams can be formed to collect and
analyze the metrics that are obtained from the conducted Inspections.


\section{Inspection and Priority Ranked Inspection}
The different types of traditional inspection processes explained above are
quite different from the Priority Ranked Inspection (PRI) process that I am
proposing. The biggest difference is traditional inspection processes
provide guidelines on \textit{how to conduct inspections} and PRI provides
guidelines on \textit{what to inspect}. In fact, the PRI process does not
provide any guidance on how to inspect the documents once they are
selected. Instead, PRI is a document selection process that wraps around
any traditional inspection process.

\subsection{Lack of Discussion about Selection of Documents}
In the book, ``Software Inspection'', Tom Gilb and Dorothy Graham provide
very few paragraphs on the subject of document selection. The following is
the entire paragraph that discusses document selection.

\begin{quotation}
  \textit{The starting point for any Inspection is the request from the
    author of a document that the document be Inspected. Inspection is
    always voluntary, and authors must not be coerced into 'volunteering'
    documents against their will. \\ Authors are motivated to request
    Inspection for two reasons:
\begin{enumerate}
\item they will get help to upgrade their document before official release;
\item they must achieve exit status in order to claim that they have met a
  deadline, and that the quality of their work is really good enough.
\end{enumerate}
}
\end{quotation}

In addition, several other books address document selection for inspection
with even less words. The following is one of the few sentences about the
initiation of the inspection process in Karl Wiegers's book, ``Peer Reviews
in Software'' \cite{Wiegers02}.

\begin{quotation}
  \textit{The author initiates planning by announcing that a deliverable
    will soon be ready for inspection.  }
\end{quotation}

%% The Nasa standard for inspections totally leave this topic out.

The traditional inspection literature fails to address several key areas of
selecting a document for inspection.

\begin{enumerate}
\item What happens when an organization does not have enough resources to
  inspect every document that is ready? Inspections are expensive. It can
  consume 15 percent of the projects budget \cite{Gilb93}. What happens to
  the volunteering process when an organization can inspect one in every
  five documents? 
\item What happens when two authors volunteer two different documents at
  the same time? Which document should be selected? Selecting what to
  inspect from two documents is not difficult. However, what if there are
  twenty or a hundred different documents that are waiting to be inspected.
\item Defects can occur in documents that already ``exited'' the
  development and inspection process, can these documents be inspected? The
  current literature suggests a linear development process, which  documents
  that have been inspected and completed the development process are never
  to be inspected again.
\end{enumerate}

I strongly believe that the selection of documents for inspection is a
complicated process that warrants much more attention than the current
inspection literature provides. 

\subsection{Cost Cutting}
\begin{quotation}
  \textit{``The bottom line is that I [Tom Gilb] believe that it is more
    relevant to view Inspection as a way to control the economic aspects of
    software engineering rather than a way to get 'quality' by early defect
    removal'' \cite{Gilb99}.  }
\end{quotation}

Tom Gilb is correct. If inspections do not provide an economic benefit,
then why do them at all? However, with an estimated 10-15 percent of a
project's budget that is required to conduct successful inspections, it is
difficult for organizations with limited inspection resources to correctly
implement the suggested process. The bottom line seems to be that not all
organizations can invest 10-15 percent of their budget to inspections.

In Software Inspection, there are three primary ways to reduce the
resources; sampling, inspecting up-stream documents, and focusing on major
defects. The practice of sampling suggest that instead of inspecting an
entire document, pick one to four representative portions of the document.
The practice of inspecting up-stream documents suggests that requirement
and design documents need to be correct before programming can begin.
Focusing on major defects suggests that minor defects, such as code
comments, are irrelevant to the customer-performance of the system and
should be ignored.

In my opinion, Software Inspection and other traditional inspection
literature does not address the most obvious way to save resources, which
is minimizing the number of documents that need to be inspected. The
current literature suggests that inspections are a ``gateway'' to complete
the document's development process. This process works well for
organizations that have the resources to treat it as such. However, for
organizations with limited inspection resources, inspecting every document
is quite impossible. In contrast to traditional approaches, Priority Ranked
Inspections embraces the notion of skipping the inspection of some
documents.

\subsection{Volunteering}
Another problem associated with the selection of documents for inspection
is the notion of volunteering. Notice in Section 2.3.1 the term 'voluntary'
is emphasized in Gilb and Graham's selection process. Yet, in the very same
book, ``Software Inspections'', contains a case study of Software
Inspections used in a company where documents were required to be inspected
rather than volunteered.

\begin{enumerate}
\item ``Most who tried inspections responded with enthusiasm, but only four
  groups were continuing to do inspections - not surprisingly, those groups
  in which the manager \emph{required} them. Most groups tried a few
  inspections, then interest waned as deadlines approached. A few managers
  ignored inspections altogether, citing schedule pressures as the
  reason.''
\item ``The vice president of marketing notified his department that
  inspections were \emph{required} for approval of all mandatory documents
  produced.''
\item ``Each year, development groups are \emph{required} to inspect
  more of their pre-code documents.''
\item ``Code inspections remain optional at least until \emph{100 percent}
  of the pre-code documents are inspected.''
\end{enumerate}

















