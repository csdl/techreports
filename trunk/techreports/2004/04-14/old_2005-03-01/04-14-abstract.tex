%%%%%%%%%%%%%%%%%%%%%%%%%%%%%% -*- Mode: Latex -*- %%%%%%%%%%%%%%%%%%%%%%%%%%%%
%% 04-14-abstract.tex -- Thesis white paper - software inspections
%% Author          : Aaron A. Kagawa
%% Created On      : Mon Sep 23 11:52:28 2004
%% Last Modified By: Aaron Kagawa
%% Last Modified On: Fri Feb  4 16:07:06 2005
%% RCS: $Id$
%%%%%%%%%%%%%%%%%%%%%%%%%%%%%%%%%%%%%%%%%%%%%%%%%%%%%%%%%%%%%%%%%%%%%%%%%%%%%%
%%   Copyright (C) 2004 Aaron A. Kagawa
%%%%%%%%%%%%%%%%%%%%%%%%%%%%%%%%%%%%%%%%%%%%%%%%%%%%%%%%%%%%%%%%%%%%%%%%%%%%%%%
%% 

\begin{abstract}  % 200 words
Imagine that your project manager has budgeted 200 person-hours for the
next month to inspect newly created source code. Unfortunately, in order
to inspect all of the documents adequately, you estimate that it will
take 400 person-hours. However, your manager refuses to increase the
budgeted resources for the inspections. How do you decide which documents
to inspect and which documents to skip?

The classic definition of inspection does not provide any advice on how to
handle this situation. For example, the notion of entry criteria used in
Software Inspection \cite{Gilb93} determines when documents are ready for
inspection rather than if inspection is needed at all \cite{Ebenau94}.

%% I could talk about previous approaches here. Sampling and Up-Stream documents

This research will investigate how to prioritize inspection resources and
apply them to areas of the system that need them more. It is commonly
assumed that defects are not uniformly distributed across all documents in
a system - a relatively small subset of a system accounts for a relatively
large proportion of defects \cite{Boehm01}.  If inspection resources are
limited, then it will be more effective to identify and inspect the
defect-prone areas.

To accomplish this research, I will construct a framework based upon
automated process and product measures to distinguish documents that are
``more in need of inspection'' (MINI) from those ``less in need of
inspection'' (LINI). Some of the process and product measures include:
reported defects, unit tests, test coverage, active time, and number of
changes. Based on this framework, I hypothesize that the inspection of MINI
documents will generate more critical defects than LINI documents.

%% Each measure acts as an independent variable for determining the inspection
%% candidacy of a document and can be assigned an individual weight. 
%%Each measure affects the determination of ``more and less'' differently.
%%For example, suppose that test coverage should be weighted more than active
%%time. Therefore, weights of each measure will be calibrated based on my
%%initial guesses.  

My research will employ a very simple evaluation strategy, which includes
inspecting MINI and LINI software code and checking to see if MINI code
inspections generate more defects than LINI code inspections. There are
three milestones that measure my progress in this research. Milestone 1:
Implementation of Hackystat Extension, January 2005. Milestone 2: Completed
evaluation, March 2005. Milestone 3: Thesis submission and defense, May
2005.

\end{abstract}





