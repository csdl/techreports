%%%%%%%%%%%%%%%%%%%%%%%%%%%%%% -*- Mode: Latex -*- %%%%%%%%%%%%%%%%%%%%%%%%%%%%
%% 04-14-evaluation.tex -- Thesis white paper - software inspections
%% Author          : Aaron A. Kagawa
%% Created On      : Mon Sep 23 11:52:28 2004
%% Last Modified By: Aaron Kagawa
%% Last Modified On: Sat Nov 20 14:24:20 2004
%% RCS: $Id$
%%%%%%%%%%%%%%%%%%%%%%%%%%%%%%%%%%%%%%%%%%%%%%%%%%%%%%%%%%%%%%%%%%%%%%%%%%%%%%
%%   Copyright (C) 2004 Aaron A. Kagawa
%%%%%%%%%%%%%%%%%%%%%%%%%%%%%%%%%%%%%%%%%%%%%%%%%%%%%%%%%%%%%%%%%%%%%%%%%%%%%%%
%% 

\Section{Evaluation Methodology}
\label{sec:evaluation}
This section discusses the proposed evaluation methodology of this
research. The main thesis of this work is that Limited Resource Software
Inspection (LRSI) can distinguish documents that are in most need of
inspection from those in least need of inspection.

One way of implementing LRSI is through Hackystat, thus I will create a
Hackystat Extension called hackyLRSI. This extension will provide an
analysis, which will determine what documents are in ``most need of
inspection'' from documents that are in ``least need of inspection''. For
this specific implementation, this determination is based on a numerical
weighting system of different process and product measures. Some measures
include: reported defects, unit tests, test coverage, active time, and
number of changes. Each measure will be assigned a numerical weight and
will be individually calibrated.

It is important to note two limitations of this research. First, I am not
defining a set of attributes that represent the determination of most and
least need of inspection for all software projects. Instead, by using
hackyLRSI I will be able to go through a methodology to best calibrate the
attributes to accurately reflect the determination for the project that I
am studying. Second, LRSI is only beneficial to organizations that have
limited inspection resources. LRSI will not benefit an organization that
has the necessary resources to thoroughly inspect every document. For these
organizations it does not matter if one document is in more need of
inspection over another, since they will inspect everything.

\hspace*{1pt}

In this evaluation, I will study the implementation and inspection process
of the Hackystat System developed in the Collaborative Software Development
Laboratory (CSDL), of the University of Hawaii at Manoa. Like most
organizations, CSDL's inspection resources are limited and therefore
inspections are conducted on a weekly basis regardless of the number of
``ready'' documents. In addition, unlike most organizations who conduct
Software Inspection and have limited resources, CSDL does not conduct
sampling or inspections on up-stream documents to enhance the inspection
process as recommended by Tom Gilb \cite{Gilb93}. CSDL does not follow
these recommendations for two reasons.  First, CSDL does not have enough
resources to conduct sampling. Second, Hackystat does not contain many
up-stream requirement and design documents. Hackystat lacks these up-stream
documents primarily because CSDL develops Hackystat on its own and not for
a client.

A quick note: Hackystat does not contain documents per se and because CSDL
primarily inspects source code grouped by package, I will use the term
'packages' when referring to CSDL's use of LRSI. The term 'documents' will
still be used when referring to the general idea of inspections.

Although I am a member of CSDL and have been contributing to Hackystat, I
will minimize any possible data contamination by doing two things. First, I
will keep the results of the ``most'' and ``least'' need of inspection a
secret both during and after conducting the inspection. Second, I will not
participate in the inspections themselves.

The use of CSDL in my study indicates another limitation on this research.
The most accurate and thorough evaluation of LRSI includes making a
determination about most and least need of inspection and actually
inspecting \textit{all} the documents to test if that determination is
correct.  However, because I am using CSDL's inspection resources, which
are limited, this is not possible.

\hspace*{1pt}

Briefly talk about volunteering here. 

\hspace*{1pt}

To evaluate this thesis, I will decompose it into three claims based upon
the three intended benefits of LRSI.

\begin{enumerate}
\item LRSI can help constrain the area of volunteering.
\item LRSI can identify documents that need to be inspected that cannot be
  identified by volunteering. 
\item Documents that is deemed in most need of inspection will generate
  more critical issues than documents deemed in least need of inspection.
\end{enumerate}

The next sections will detail each of these claims and the methodologies
used in their evaluation.

The following table provides a timeline for the Evaluation of this thesis:

\begin{center}
\begin{tabular}[h] {|c|c|} \hline 
Timeline & Evaluation Activity \\ \hline
January 10, 2005 & Request developer workspace rankings \\ \hline
Januray 14, 2005 & Process developer responses and create a plan of what
will be inspected \\ \hline
January 17, 2005 & Start 4 weeks of inspection, inspecting 2 packages a
week \\ \hline
Feburary 14, 2005 & Hand pick 2 packages to inspect that was not
volunteered \\ \hline
\end{tabular}
\end{center}



\subsection{Claim 1: Constrained area of volunteering}
\label{sec:claim1}
One of the benefits of LRSI is that it constrains the area in which
developers can volunteer their code for inspection. In Software Inspection,
the area of possible inspection includes \textit{all} the documents
currently moving through the development cycle. In LRSI, this area will be
constrained to the documents that are in most need of inspection. This
smaller LRSI area seems to be advantageous for organizations that cannot
inspect every document, because it will eliminate the need and the more
importantly it limits the possibility of inspecting code that is deemed in
least need of inspection.

As an example of how LRSI benefits an organization with limited resources
consider the following fictitious scenario:

\begin{quotation}
  \textit{ The organization FooBar has enough resources available to
    conduct inspections at least once a week. Because this organization
    produces more code than is possible to inspect, they use a round-robin
    approach by allowing a different developer to volunteer a piece of code
    to inspect. This developer must pick a small portion of the code he/she
    is currently working on and this decision is primarily based solely on
    his/her subjective opinions of the code.  }
\end{quotation}

This method works well if the developer can be trusted to pick the right
code to inspect. However, developers often do not know where every critical
issue will appear. In other words, leaving this decision up to the
subjective understanding of a developer is error prone [evidence?].

LRSI provides an alternative solution to this limited resource problem.
Instead of leaving the decision of what code to inspect entirely up to the
developer, LRSI can constrain the number of possibilities by providing a
smaller area of selection. For this fictional organization, the developer
can look up what code is in most need of inspection and choose code from
this smaller list. 

\hspace*{1pt}

To evaluate this claim, I will ask the developers of Hacksytat to provide a
numerical ranking, based on their subjective feelings, of what packages
they would volunteer for inspection. The ranking will indirectly indicate
the packages that are in most and least need of inspection. With these
results I will be able to compare the developers' subjective rankings to
the LRSI most and least need of inspection determination. This evaluation
will indicate whether LRSI is really needed. For example, the findings
could indicate that developers can correctly distinguish, using their own
subjective understandings, what packages need to be inspected and packages
that do not need to be inspected.

To conduct this evaluation, I will provide each developer with a list of
Hackystat packages that they are currently working on. This will be
determined by assessing the developers' active time and commits to a
particular package. Given this listing I will ask each developer to provide
a numerical ranking of each package.

The following steps will occur in this evaluation:
\begin{enumerate}
\item Obtain the rankings of packages from each individual developer.
\item Analyze the difference between the developers' ranking against the
  LRSI most and least need of inspection determination.
\item Conduct the following inspections: 
\begin{enumerate}
\item Inspect 2 packages, where the developer and the LRSI determinations
  agree, that are in most need of inspection.
\item Inspect 2 packages, where the developer and the LRSI determinations
  agree, that are in least need of inspection.
\item Inspect 2 packages where the developer and the LRSI determinations
  disagree. The developer provides a low ranking but the LRSI claims that
  the package is in most need of inspection.
\item Inspect 2 packages where the developer and the LRSI determinations
  disagree. The developer provides a high ranking but the LRSI claims that
  the package is in least need of inspection.
\end{enumerate}
\item Analyze the results of each inspection, which includes correlating
  the number of critical issues generated with both the developer ranking
  and the LRSI determination. In addition, I may ask the developers for
  explanations of their rankings where applicable.
\item After each inspection I will adjust LRSI calibration or add new
  product and process measures as necessary.
\end{enumerate}

There are three possible results of this study. First, I may find that
developers automatically have a sense of what code is in most need of
inspection and in least need of inspection. This would indicate that LRSI
provides little added value. Second, developers provide high rankings for
both most and least need of inspection packages. Essentially, this will
indicate that sometimes the developers are correct and sometimes they are
wrong. And third, developers have no idea what code needs to be
inspected. The last two results will indicate that LRSI provides some
benefit. 

In addition, this evaluation will provide more data to refine the
calibrations of the measures that are used for the LRSI determination. For
example, if a developer rates a package very highly, LRSI finds that
package to be deemed least need of inspection, and many critical issues are
found, then this indicates that the LRSI determination is flawed.
Therefore, the LRSI determination needs to be recalibrated to include this
document. In addition to calibration, more process and product measures
could be introduced. This event, although detrimental to the previous LRSI
determination, will provide more data for calibration and the addition of
new measures that will hopefully lead to a better and more accurate LRSI
determination.



\subsection{Claim 2: LRSI is better than volunteering}
\label{sec:claim2}
Another benefit of LRSI is that it can find areas of the system that is in
most need of inspection and has not been identified using the volunteering
process. Organizations that have limited inspection resources simply cannot
volunteer and inspect every single line of code. If these organizations
blindly volunteer or pick and choose documents for inspection they could
possibly be missing some areas of the system that need inspection the most.

A real example of this benefit is the following: 

\begin{quotation}
  \textit{ Not all Hackystat packages have experts. Instead there are some
    packages that I considered to be orphans. Orphaned-packages are usually
    packages that are considerably old code or code that has been written
    by developers who has left CSDL. In addition, these packages are
    usually never inspected and are considered to be in working order.  }
\end{quotation}

This situation is quite dangerous, because as we all know a software system
evolves and outdated packages may become error prone. Therefore, it is
important to realize that not only active packages need to be inspected but
old packages can also be deemed in most need of inspection. Software
Inspection \cite{Gilb93} does not address this issue of outdated
documents. The common adage of Software Inspection is to inspect documents
as they move through the development cycle. This process tends to ignore
documents that have already finished the development cycle.

In addition, because this organization does not have the resources to
inspect every document moving through the development cycle, it is very
likely that some documents that make it through the cycle will have bugs in
it. Therefore, ensuring that even these documents are included as potential
inspection candidates is very important.

\hspace*{1pt}

To evaluate this claim, I will make several inspection recommendations of
packages deemed in most need of inspection and have not been investigated
in the previous study. Again, the idea is that developers cannot always
identify areas of the system that they think is low quality and only using
the volunteering method will likely miss some documents that are in most
need of inspection.

The following steps will occur in this evaluation:
\begin{enumerate}
\item Select a few packages that were not investigated in the previous
  study and has been deemed in most and least need of inspection. 
\item Conduct inspections on those packages. 
\item Analyze the results of each inspection, which includes correlating
  the number of critical issues generated with the LRSI determination. 
\item After each inspection I will adjust LRSI calibration or add new
  product and process measures as necessary.
\end{enumerate}

There are two possible results of this evaluation. First, the packages that
were selected were correctly categorized by LRSI. This finding will support
my claim. Second, the packages that were selected did not reflect the LRSI
determination.




\subsection{Claim 3: Most need of inspection versus least need of inspection}
\label{sec:claim3}
The last benefit of LRSI is, documents that is deemed in most need of
inspection will generated more critical issues than documents deemed in
least need of inspection.  This claim is critically important for LRSI's
success.

However, if a package is identified as least need of inspection and yields
many critical issues, then LRSI determination is flawed. I will use this
information to refine the LRSI determination. It is my hope that in the end
of the study I will have been able to successfully calibrate the LRSI
determination for the Hackystat project.

During the evaluations of the previous two claims CSDL will have conducted
at least 12 inspections.  In addition, I have and will collect information
on past and future inspections on Hackystat packages. In total, I believe I
will have data on 20 inspections along with the information on the LRSI
determinations. 

Currently, Hackystat and its extensions are comprised of 167 packages. As I
previously stated, an accurate and thorough evaluation of LRSI requires the
inspection of all packages within the LRSI determiniation. However, because
of CSDL's limited resources this is not possible. At best this will take 3
hours per inspection, totalling 501 hours of inspection. This is
unrealistic.  Therefore, my proposed evaluation will evistigate a small
percentage of the system, 20 of the 167 packages, in hopes that this
cross-section will provide adequate and acceptable results.

\hspace*{1pt}

To evaluation this claim, I will monitor the validity of the LRSI
determination, and adjusting it as necessary, throughout each inspection.
To accomplish this, I will collect specific pieces of information when
conducting inspections. The following is a specific list of the information
that is being collected:

\begin{itemize}
\item Inspection date
\item Hacksytat module, package, and inspection ID
\item LRSI determination (most need of inspection or least need of
  inspection)
\item LRSI measures and values
\item Subjective discussion of the validity of the LRSI determination before 
  the inspection
\item Number of issues generated and the categorization of these issues
  according to severity
\item Retrospective discussion after the inspection was conducted to
  indicated possible areas of improvement. 
\end{itemize}

This information will help me keep track of the progress of the inspections 
and the validity of the LRSI determination. As I previously stated, the
calibration of the LRSI determination is an ongoing and evolving
process. This information will help keep track of that evolution. 

The end goal of this information collection is to create a best practices
recommendation of the types of process and product measures and their
calibration that will provide the best LRSI results for different projects.


\subsection{Initial Results of Evaluation}
\label{sec:intialresults}
The use of LRSI to provide the determination of most and least need of
inspection has been promising. The initial implementation of the system has
proven that it is technically possible to do what I have envisioned. In
addition, I have already recommended the inspection of a package that was
in ``most need of a inspection'' and the defects and issues identified have
confirmed that the package had low quality.

Of course, I will continue to discover new attributes to define quality,
fine tune the numerical weights associated with the attributes, and
continue to recommend inspections until I believe my mechanism is ready for a
thorough evaluation.















