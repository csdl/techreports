%%%%%%%%%%%%%%%%%%%%%%%%%%%%%% -*- Mode: Latex -*- %%%%%%%%%%%%%%%%%%%%%%%%%%%%
%% 04-17.tex -- Characterizing the Software Development Process
%%              of High Performance Computing Systems
%% Author          : Michael Paulding
%% Created On      : Wed Sep 22
%% Last Modified By: 
%% Last Modified On: Mon Sep 27
%% RCS: $Id$
%%%%%%%%%%%%%%%%%%%%%%%%%%%%%%%%%%%%%%%%%%%%%%%%%%%%%%%%%%%%%%%%%%%%%%%%%%%%%%%
%%   Copyright (C) 2004 Michael Paulding
%%%%%%%%%%%%%%%%%%%%%%%%%%%%%%%%%%%%%%%%%%%%%%%%%%%%%%%%%%%%%%%%%%%%%%%%%%%%%%%
%% 

\documentclass[11pt,twocolumn]{article} 
\input{/export/home/csdl/tex/psfig/psfig}
\usepackage{/export/home/csdl/tex/icse2003/latex8}
\usepackage{times}
%% A verbatim-like environment which allows font changes
%%\usepackage{alltt}
%% New LaTeX2e graphics support
\usepackage[final]{graphicx}
% uncomment the % away on next line to produce the final camera-ready version
% and uncomment the \thispagestyle{empty} following \maketitle
\pagestyle{empty}

\begin{document}

\title{Characterizing the Software Development Process \\
of High Performance Computing Systems}


\author{\protect\begin{tabular}{ccc}
Michael Paulding \\
\end{tabular}\\
\em  Collaborative Software Development Laboratory \\
\em  Department of Information and Computer Sciences \\
\em  University of Hawai'i \\
\em  Honolulu, HI 96822 \\
\em  mpauldin@hawaii.edu}
\maketitle
\thispagestyle{empty}

\begin{abstract}

High performance computing systems are being applied to an
increasingly wide variety of domains and are also becoming radically
less expensive.  However, the Defense Advanced Research Projects
Agency (DARPA) High Productivity Computing Systems (HPCS) program
notes that dramatic increases in low-level HPCS benchmarks of
execution speed do not necessarily translate into increased
development productivity.  Essentially, the bottleneck in HPCS
applications is moving from hardware limitations to software
engineering limitations.

In addition, HPCS development adds another layer of complexity over
what is imposed by traditional software engineering.  It is often
thought that the time devoted to parallelizing and optimizing code
consumes a majority of the total development effort.  However,
empirical, public domain estimations of time allocation parallel and
sequential development time are limited or non-existent.

In this research, a paradigm is proposed to enable a developer or
project manager to understand and characterize his HPCS development.
The paradigm follows several codependent processes that provide a
broad perspective of the development.  One process of this paradigm is
to assess progress through milestone tests, each of which is a unit of
functionality of the application that, when combined together, form
the complete system.  Another process is the tracking of the degree of
parallelism in source code, or a measure of parallel constructs
(routine calls and global variables) that appear in the files over
time.  Finally, the parallel performance of the application will be
chronologically recorded.

In order to accomplish the objectives of this research, a four part
thesis is proposed:
\begin{enumerate}
\item 
The Degree of Parallelism (DOP) analysis is useful and accurate for
measuring the level of parallelism in HPCS source code.

\item
The Progress Assessment through Milestone Tests (PAMT) analysis is useful and accurate for assessing progress in the development of an HPCS application.

\item
The Parallel Performance Analysis (PPA) provides a useful and accurate means for assessing the parallel performance of an HPCS application.

\item
The three proposed analyses, DOP, PAMT, and PPA can be combined to
provide a paradigm for understanding and characterizing HPCS development.
\end{enumerate}

To complete this research in a timely fashion, I will adhere to the
following timeline: 
\begin{enumerate} 
  \item January 2005 - implementation of software system for
  characterizing HPCS development.  
  \item March 2005 Evaluation of system under academic and industry settings. 
  \item April 2005 - Submission and defense of thesis for M.S. degree.
\end{enumerate}

%This paper also describes how performance data, such as execution
%time, speedup, utilization and efficiency, can be coupled with
%milestone test data to provide a compelling picture of progress in
%HPCS projects.  With analysis of these trends, it is anticipated that
%developers and program managers will gain insight into time
%allocation, performance and deliverables of HPCS development in order
%to realize productivity bottlenecks and enhancers.

\end{abstract}


\Section{Evaluation}
\label{sec:evaluation}
In order to evaluate the claims proposed by this research, it is first
necessary to understand the proposed benefits of the research.  Once
the intended benefits have been established, the strategy for
evaluation of the benefits will be discussed in detail.

To begin with, the focus of this research, as stated in the title, is
to characterize the software development process of high performance
computing systems.  In order to accomplish the characterization, this
research proposes to automatically and unobtrusively provide analyses
that encompass HPCS software development.  These analyses include:

\begin{enumerate}
\item {\bf Degree of Parallelism (DOP) of Source Code}
\item {\bf Progress Assessment through Milestone Tests}
\item {\bf Parallel Performance Analysis}
\end{enumerate}

These analyses are independent of each other and retain loose
coupling, but each is necessary to perform a full characterization of
the HPCS software development process.

\subsection{Degree of Parallelism (DOP) of Source Code}
Determining the DOP of source code is an important component of
understanding and characterizing HPCS development.

As a refresher, the DOP for source code functions as a binary
relationship.  If a source file contains one or more parallel
constructs, then that file is assigned a DOP = 1.  A parallel
construct is defined as a routine or global variable that is used for
message passing among processors in a parallel system.  Commonly used
libraries implementing message passing are the Message Passing
Interface (MPI) and Open Message Passing (OpenMP), so specific
examples of parallel constructs involving MPI or OpenMP routines or
global variables.  Files containing at least one parallel construct are therefore termed ``parallel'' files.

In contrast, some files in an HPCS application are completely absent
of parallel constructs and are, in turn, assigned a DOP = 0.  These
files typically involve units of computation that do not require
enough resources or execution time to warrant parallelization.  For
example, computing the factorial of a single integer is likely to be
done sequentially, whereas a factorial problem containing a dataset of
thousands of integers is likely to be distributed and computed in
parallel.  In essence, a source file containing no parallel constructs
is termed a ``sequential'' file.

The metric for degree of parallelism in source code provides a
plethora of benefits and insights for the HPCS developer and project
manager.  To start with, the development of an HPCS application
typically consists of two processes: the implementation of the system
functionality (sequential part) and the optimization and
parallelization of that functionality (parallel part).  It is often
hypothesized among HPCS developers what portion of time is devoted to
parallelization, with some estimates as high as 90%, but currently
there is no concrete mechanism to measure the ratio.

In addition to providing a tangible value representing the ratio of
sequential and parallel files in a project, this metric provides added
value when coupled with the active time metric.  A developer or
project manager can use this analysis to look at historical and
in-progress data to see how much time was spent parallelizing code.
If the value they observe seems disproportionate to previous project
data or to their mental model, value is added by inspecting the
parallel files.  Consider the case in which one or more developers has
spent a large portion of the total active time on parallelization.
This set of developers (or their project manager(s)) can look at the
parallel files edited during that time interval and identify what
parallel constructs or optimization strategies caused the time drain.
Once identified, a project manager can record this data and try to
divert future teams from taking these resource draining approaches and
try new strategies to obtain better productivity.  In addition, using
the archived data, other development teams can learn about past
bottlenecks before embarking on a new development endeavour.

In order to evaluate whether this metric provides a useful and
accurate value for the degree of parallelism in source code, I will
perform the following: 

\begin{enumerate}
\item 
Incorporate the DOP tool into the daily build of HPCS developers,
sample size 5 (is this enough?)

\item 
Create Hackystat accounts for each of the developers to permit them
access to their own data.

\item 
Gather data on Hackystat over a period of 4-6 weeks, allowing
developers to view and interpret their in-process data.

\item 
Distribute a survey to the developers and their project manager(s) to
evaluate whether the DOP analysis was useful and accurate for
measuring the level of parallelism in their source code.
\end{enumerate}

\subsection{Progress Assessment through Milestone Tests (PAMT)}
The PAMT analysis provides a means for a developer or program manager
to understand the progress made during the development of an HPCS
application.

To understand this analysis, it is first necessary to provide a
concrete definition for the term milestone test.  Milestone tests are,
in the simplest form, a set of tests that fully describe the
functionality required in the HPCS application.  For example, in the
Truss PBB, the set of milestone tests would include: 
\\
\\
MT-1) The application can successfully generate a topology connecting
the attachment points on a wall to the load bearing point in free
space, using a variable number of joints.
\\
\\
MT-2) The application can successfully assign a geometry to a
previously defined topology.  Geometry assignment includes determining
the type of member connecting two joints in the truss.  In the Truss
PBB, the two options are cable or strut.
\\
\\
MT-3) Given a topology, the application must assert that the distance
between joints does not exceed the deformational constraints of steel.
In short, MT-3 will pass only if each distance between joints does not
exceed a fixed limit.
\\
\\
MT-4) Given a truss topology and geometry, the application can compute
a set of forces acting on each joint in the truss.
\\
\\
MT-5) Given a set of forces acting on a joint, the application can
verify that the net force is zero.
\\
\\
MT-6) Given the net force on each joint, the application can determine
whether the truss is in equilibrium.
\\
\\
MT-7) Given a truss topology and geometry, the application can
calculate the mass of each member.
\\
\\
MT-8) Given a truss topology and geometry, the application can
calculate the total mass of the truss, providing a potential solution
for the Truss PBB problem.
\\
\\
It is important to note that each of these milestone tests are
independent of all others, but each of the tests must be satisfied
before the application can be considered complete.  Therefore,
developers could follow many different permutations of these tests to
achieve overall completion.  However, even though the paths developers
take to implement the tasks may vary significantly, their overall
progress through the application is comparable.

For example, consider the case in which the program manager decides to
weight each of the milestone tests equally, giving each test a value
of 12.5.  Two developers implementing the application independently
can then be evaluated jointly based on the score of tests they have
implemented.  Again, both developers will be finished when their score
equals 100.  On the other hand, if a program manager decides to assign
milestones with differing weights, one developer may have implemented
more milestone tests, but have a lower score than the other developer.
Regardless, the overall progress of these developers is still simply
comparable.

In order to evaluate whether the PAMT analysis is useful and accurate
for assessing progress in a HPCS application, I will perform the
following:

\begin{enumerate}
\item 
Incorporate the sensors and tools needed to capture the milestone test
data (e.g. pass/fail) for a set of HPCS developers, sample size 5 (is
this enough?)

\item 
Create Hackystat accounts for each of the developers to permit them
access to their own data.

\item 
Gather data on Hackystat over a period of 4-6 weeks, allowing
developers to view and interpret their in-process data and total
milestone test score.

\item 
Distribute a survey to the developers and their project manager(s) to
evaluate whether the PAMT analysis was useful and accurate for
assessing progress.
\end{enumerate}

\subsection {Parallel Performance Analysis (PPA)}

Parallel performance analysis (PPA) provides critical measurements for
how well a HPCS application is performing and how well it is been
parallelized and optimized.

PPA provides 5 universal analyses for quantifying performance
characteristics of parallel software in a generic and portable manner.
Each of the 5 analyses is briefly described below:

\begin{enumerate}
\item {\bf Speedup} - 
Speedup is used as a measure to indicate the degree of speed gain in a
parallel computation.  This analysis is generally measured in the
formula S(n) = T(1)/T(n), where n is the number of processors used and
T(n) is the execution time used by n processors.  However, the speedup
analysis formula in particular has several variations depending on how
a parallel algorithm is implemented.

\item {\bf Efficiency} -
Efficiency is used as a measure to indicate the useful portion of the
work performed by n processors.  This analysis is generally measured
by the formula E(n) = S(n)/n.

\item {\bf Redundancy} -
Redundancy is used as a measure to indicate the extent of workload
increase.  Conceptually it can be though of as how many operations are
duplicated with the addition of a processor.  This analysis is
generally measured by the formula R(n) = V(n)/V(1), where V(n) is the
number of unit operations performed on n processors.

\item {\bf Utilization} -
Utilization is used as a measure to indicate the extent to which
resources are utilized during a parallel computation.  This analysis
is generally measured by the formula U(n) = R(n)E(n).

\item {\bf Quality} -
Quality is used as a measure for combining the effects of speedup,
efficiency and redundancy to assess the relative merit of a parallel
computation.  In effect, the quality analysis serves as the most
succinct and direct comparison between two parallel implementations of
the same problem.
\end{enumerate}

Parallel performance analysis provides useful, if not critical,
information to a HPCS developer or project manager.  For example, PPA
provides instant feedback for whether a change made to the system
further optimized it or slowed it down.  This information is available
in process and development teams can make adjustments as necessary
before the project deadline.

In addition, PPA couples well with the DOP and PAMT analyses.  Specifically, by combining these three analyses, a developer or project manager is enabled to understand the impact on performance when functionality is increased or decreased (e.g. more milestone tests pass or fail) and when the degree of parallelism of the source code is increased or decreased (e.g. more time is spent parallelizing or optimizing).


\begin{enumerate}
\item 
Incorporate the sensors and tools needed to capture the parallel
performance analysis data (e.g. the 5 universal performance
quantifiers) for a set of HPCS developers, sample size 5 (is this
enough?)

\item 
Create Hackystat accounts for each of the developers to permit them
access to their own data.

\item 
Gather performance data on Hackystat over a period of 4-6 weeks, allowing
developers to view and interpret their in-process data and performance values.

\item 
Distribute a survey to the developers and their project manager(s) to
evaluate whether the PPA analysis was useful and accurate for
assessing parallel performance of their HPCS application.
\end{enumerate}

%\bibliographystyle{/export/home/csdl/tex/icse2003/latex8}
%\bibliography{/export/home/csdl/bib/hpcs} 
\end{document}
