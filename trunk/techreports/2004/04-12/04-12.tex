%%%%%%%%%%%%%%%%%%%%%%%%%%% -*- Mode: Latex -*- %%%%%%%%%%%%%%%%%%%%%%%%%%%%
%% 04-12.tex -- Hackystat In Action: JPL
%% Author          : Aaron A. Kagawa
%% Created On      : Mon Sep 23 11:52:28 2002
%% Last Modified By: Aaron Kagawa
%% Last Modified On: Tue Sep  7 00:33:35 2004
%% RCS: $Id$
%%%%%%%%%%%%%%%%%%%%%%%%%%%%%%%%%%%%%%%%%%%%%%%%%%%%%%%%%%%%%%%%%%%%%%%%
%%   Copyright (C) 2002 Philip Johnson
%%%%%%%%%%%%%%%%%%%%%%%%%%%%%%%%%%%%%%%%%%%%%%%%%%%%%%%%%%%%%%%%%%%%%%%%%%

\documentclass[11pt,twocolumn]{article} 
\input{/export/home/csdl/tex/psfig/psfig}
\usepackage{/export/home/csdl/tex/icse2003/latex8}
\usepackage{times}

%% A verbatim-like environment which allows font changes
%%\usepackage{alltt}
%% New LaTeX2e graphics support

\usepackage[final]{graphicx}
% uncomment the % away on next line to produce the final camera-ready version
% and uncomment the \thispagestyle{empty} following \maketitle
\pagestyle{empty}

\begin{document}

\title{Hackystat In Action: \\ Results from a JPL installation}
\author{\protect\begin{tabular}{ccc}
Aaron A. Kagawa \\
\end{tabular}\\
\em  Collaborative Software Development Laboratory \\
\em  Department of Information and Computer Sciences \\
\em  University of Hawai'i \\
\em  Honolulu, HI 96822 \\
\em  kagawaa@hawaii.edu}
\maketitle
\thispagestyle{empty}

\begin{abstract}  % 200 words
In the summer of 2004, a Hackystat configuration was successfuly deployed at the 
Jet Propulsion Laboratory. This configuration succuessfully model the development
process of the Mission Data System. 

\end{abstract}

\Section{Introduction}
\label{sec:intro}
In the summer of 2004, a Hackystat configuration was deployed at the Jet Propulsion 
Laboratory, NASA. The goal of this configuration was to support very large software
projects such as the Mission Data System. The Mission Data System (MDS) is a system
that has hopes of introducing a common framework for all space missions. Until MDS's
conception all space missions create their own space flight, navigation, and ground
software. MDS's goal is to be able to provide a framework which all space missions
can build upon, thus reducing code duplication.

Needless to say, MDS has a very interesting application. However, our interest in
MDS was not based on its utility. Rather, on its development process. MDS's development
process has the most well defined process at JPL. It contains a sophisiticated build
process, configuraiton management, and management. MDS's development process is slightly 
based on Extreme Programming in that they follow the practice of contiuous builds by
building the system several times each day. This build process allows internal defects
to be identified and corrected very quickly. In addition, MDS uses the Harvest configuraiton
management tool, where they have created a very complicated workflow process that allows
them to track certain packages of work as it moves through the development process.

As a side affect to using the tools they have choosen, they have a very detailed record
of both the workflow and builds that occured. This provided a huge opportunity for Hackystat
to extract various metrics and analyze process and product measure in an attempt to help
increase productivity and quality.

This Hackystat In Action paper describes the design of the Hackystat system for MDS, the 
problems that we encountered, and the progress that we have made. If you are unfamilar 
with Hackystat please read __________ before continuing.

\Section {The Mission Data System}
In order 



\Section{The Hackystat Configuration}
Section 2 goes here. ...


\Section{Lessons Learned}
Our first lession learned is that


\Section{Future directions}
Future Directions


\Section{Acknowlegements}
We gratefully acknowledge support for Project Hackystat by the NSF and NASA
as part of their joint program on Highly Dependable Systems, by Sun
Microsystems as part of the DARPA High Productivity Computing Systems
program, and by IBM as part of the Eclipse Innovation Grant program.


\bibliographystyle{/export/home/csdl/tex/icse2003/latex8}
\bibliography{/export/home/csdl/techreports/04-11/04-11,/export/home/csdl/bib/csdl-trs,/export/home/csdl/bib/hackystat,/export/home/csdl/bib/psp}
\end{document}

 





















