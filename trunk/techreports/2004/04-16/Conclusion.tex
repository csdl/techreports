\chapter{Conclusion}  \label{Chapter:Conclusion}


This research introduced \textit{software project telemetry}, which is a novel, light-weight measurement approach. It includes both (1) highly automated measurement machinery for metrics collection and analysis, and (2) a methodology for in-process, empirically-guided software development process problem detection and diagnosis. In this approach, sensors collect software metrics automatically and unobtrusively. Metrics are abstracted in real time to telemetry streams, charts, and reports, which represent high-level perspectives on software development. Telemetry trends are the basis for decision-making in project management and process improvement. It overcomes many of the difficulties in existing approaches.



\section{Anticipated Contributions}

The anticipated contributions of this research are:

\begin{itemize}
	\item The concept of software project telemetry as an effective automated approach to in-process, empirically-guided software development process problem detection and diagnosis. 
  
  \item An implementation of software project telemetry which allows continuous monitoring of software project status, as well as generating and validating software process improvement hypothesis.
  
  \item The insights gained from the case studies regarding how to use software project telemetry effectively for project management and process improvement, as well as the adoption barrier of the technology.

\end{itemize}





\section{Future Directions}

There are several areas I am unable to address within the time frame of this dissertation research. They will be future directions:

\begin{itemize}
  \item \textit{Telemetry analysis user interface improvement} --- Preliminary evaluation results suggest that the usability improvement to the interface invoking telemetry analysis is desired, especially with respect to the way telemetry analysis parameter values are supplied.
  
	\item \textit{Telemetry language measurement scale type checking} --- Current telemetry language does not enforce measurement scale type checking, and meaningless mathematical operations can be applied to telemetry streams as a result. The actual usage of the telemetry language needs to be studied to determine whether it is a big problem to end users or not. If it is, then the language needs to be augmented to account for scale type difference.
	
	\item \textit{Statistic process control} --- Given a set of telemetry streams, human judgment is required to detect bad trends in the development process. Statistic process control might provide automated support and needs to be studied in the context of software project telemetry.
	
	\item \textit{Replication of case study} --- The case studies need to be replicated in different settings in order to address external validity issues.
	
	\item \textit{User base expansion} --- One of the advantage of software project telemetry is that system deployment requires very little resource. This means the technology adoption risk is quite low for a software organization. I will make user interface improvements and try to find opportunities to market the technology.
\end{itemize}














%\section{Research Summary}
%\section{Research Contributions}
%\section{Lessons Learned}
%\section{Future Directions}


%%%%%%%%%%%%%%%%%%%%%%%%%%%%%%%%%%%%%%%%%%%%%%%%%%%%%%%%%%%%%
%%
%%   Comment Start
%%
%%%%%%%%%%%%%%%%%%%%%%%%%%%%%%%%%%%%%%%%%%%%%%%%%%%%%%%%%%%%%
\begin{comment}


\section{Lessons Learned}


\subsection{Automated data collection is necessary}

Two advantages: (1) Lower data collection overhead, reduce metrics program adoption barrier. (2) Increase data accuracy. (3) Data collection in daily integration build system is important.



\subsection{Perfect solution may not be feasible}

Trying get perfect metrics may not be possible. A good example of this problem is our solution for unit test coverage computation. There are several definition of coverage:
\begin{itemize}
	\item branch level
	\item statement level
	\item method level
\end{itemize}

All have their advantages and disadvantages. Branch level coverage information indicates whether all excution path of the program has been exercised or not. It is the most accurate measure, but hard to collect at the same time.

Method level without one line method.



\section{Future Work}

\subsection{Etiquette of Data Elements}

Experience has proved numerous times that appropriate using of data is one of the most important elements to get any metrics program off the ground. If individuals see the metrics as tools to help them improve their software development processes, they are more likely to embrace the program. However, on the other hand, if they feel metrics program is used by the management to monitor their behavior, they are likely to resist it or even fake data.


Software measurement activities will be subverted if they are seen as a threat, ignored if thought incompetent or inappropriate, or discredited if they do not deliver the anticipated benefits.

There are two important issues that must be addressed:
\begin{itemize}
	\item Data privacy issue. 
	\item Provide regular feedback to developers.
\end{itemize}	
	

There should be specific rules regarding who can access what portion of data, and when data go from private to public. Some data should always be kept private to individual developers, while other data can be accessed at the project or organizational level. In particular, metrics should never be allowed to measure individual performance. Robert Grady [Grady 92] has suggested the following recommendations:

\begin{table}[tbp]
  \centering
    \begin{tabular}{|p{4.5cm}|p{4.5cm}|p{4.5cm}|} 
      \hline
      Individual & Project Team & Organization \\
      \hline
      %% row 2 column 1
      Defect rates (by individual)\newline Defect rates (by module)\newline 
      Defect rates (under development)\newline Number of compiles &
      %% row 2 column 2
      Defect rates (team)\newline Module size\newline Estimated module size\newline 
      Number of re-inspections\newline Defects per module (prerelease) &
      %% row 2 column 3
      Defect rates (by project)\newline Size (by product)\newline Effort (by project)\newline 
      Calendar times\newline Defects per module (post release)\newline 
      Effort per defect (average) \\ 
      \hline
    \end{tabular}
  \caption{Data Access Recommendations}
  \label{tab:DataAccessRecommendations}
\end{table}



\subsection{Privacy Issue}
The access rules need to be enforce at the tool level. What is Hackystat conformance level???
	 
	 
	
	
\subsection{Top down approach, or knowledge repository}

In other words, the telemetry framework is practically useless if one cannot construct suitable telemetry streams to solve problems at hand. Therefore, a methodology for a goal-driven, top-down approach to design and validate telemetry streams is necessary to supplement the implementation and make the concept of telemetry-based software project management operable.

After validation, useful streams can be put into knowledge repository. But still, it might be context sensitive (similar to software process models).
	 
	
\end{comment}
%%%%%%%%%%%%%%%%%%%%%%%%%%%%%%%%%%%%%%%%%%%%%%%%%%%%%%%%%%%%%
%%
%%   Comment Start
%%
%%%%%%%%%%%%%%%%%%%%%%%%%%%%%%%%%%%%%%%%%%%%%%%%%%%%%%%%%%%%%
