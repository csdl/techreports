\begin{abstract}

%[Problem]
Software development is slow, expensive and error prone, often resulting in products with a large number of defects which cause serious problems in usability, reliability and performance. To combat this problem, software measurement provides a systematic and empirically-guided approach to control and improve development processes and final products. Experience has shown excellent results so long as measurement programs are conscientiously implemented and followed. However, due to the high cost associated with metrics collection and difficulties in metrics decision-making, many organizations fail to benefit from measurement programs.

%[Proposed Solution]
In this dissertation, I propose a new measurement approach -- \textit{software project telemetry}. It addresses the ``metrics collection cost problem'' through highly automated measurement machinery -- sensors are used to collect metrics automatically and unobtrusively. It addresses the ``metrics decision-making problem'' through intuitive high-level visual perspectives on software development that support in-process, empirically-guided project management and process improvement. Unlike traditional metrics approaches which are primarily based on historical project databases and focused on model-based project comparison, software project telemetry emphasizes project dynamics and in-process control. It combines both the precision of traditional project management techniques and the flexibility promoted by agile community.

% The term ``telemetry'' has multiple meanings: (1) data collection is automatic and unobtrusive (2) data are displayed in time series (3) correlation analysis is a large part.

%[Claim and Evaluation]
The main claim of this dissertation is that software project telemetry provides an effective approach to (1) automated metrics collection, and (2) in-process, empirically-guided software development process problem detection and analysis. Three case studies \textcolor{red}{will be} conducted to evaluate the claim in different software development environments:

\begin{enumerate}
	\item \textbf{[completed]} A pilot case study with student users in software engineering classes to (1) test drive the software project telemetry system in preparation for the next two full-scale case studies, and (2) gather the students' opinions when the adoption of the technology is mandated by their instructor.
	
	\item \textbf{[planned]} A case study in CSDL to (1) use software project telemetry to investigate and improve its build process, and (2) evaluate the technology at the same time in CSDL (an environment typical of traditional software development with close collaboration and centralized decision-making).
	
	\item \textbf{[planned]} A case study at Ikayzo with open-source project developers (geologically-dispersed volunteer work and decentralized decision-making) to gather their opinions about software project telemetry.
\end{enumerate}
 
 
%Results
% (1) Most people favor one-button click over playing-with-data approach. 
 
 
%[Time Frame]
The time frame of this research is as follows. The implementation of the software project telemetry system is complete and deployed. I have finished the first pilot case study. I will start both the second and third case studies from October 2005, and they will last 4 - 6 months. I wish to defend my research in May or August 2006 if everything goes according to plan.


\end{abstract}

%It includes both (1) highly automated measurement machinery for metrics collection and analysis, and (2) a methodology for in-process, empirically-guided software development process problem detection and diagnosis. In this approach, sensors collect software metrics automatically and unobtrusively. Metrics are abstracted in real time to telemetry streams, charts, and reports, which represent high-level perspectives on software development. Telemetry trends are the basis for decision-making in project management and process improvement.

%The main thesis is that software telemetry provides low cost, agile alternative to software project management, which allows improved in-process decision making and impact analysis. 


