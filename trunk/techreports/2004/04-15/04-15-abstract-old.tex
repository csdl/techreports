\begin{abstract} 

Deploying best practices is a common approach for software organizations to
improve development processes. In software engineering the best practice is
``a set of guidelines and recommendations'' for software development. For
instance, the classical waterfall model divides development process into
stages requirement analysis, design, implementation, testing, integration
and maintance to direct software development.

Best practices are summarized from experiences and improved with further
investigation. Because best practices do not have solid theoretical
supports, it is difficult to determine whether they are appropriate to the
process to be improved. A software organization may hire consultant to tell
them what to do or use trial-and-error approach to find out whether a best
practice is appropriate or not. The developers in the software
organizations are either persuaded or forced to implement the best
practices.  Everett Rogers modeled the adoption trend as adoption
curve\cite{Potter:02}.  According to this model, there are five kinds of
adoptors: innovator, early adopter, early majority, late majority and
laggard. Beside adoption curve, there is the learning process.  These two
factors add the complexity to the best practice evaluation.

Instead of relying on politics, persuaveness for adoption and management
for best practice execution we instrument software development process to
continuously generate empirical data to evaluate best practice execution
and povide evidence for or against the suitability of a practice. The
technology we use is software development stream analysis (SDSA), which
studies the fine-grained process execution data. In my thesis work I will
apply it on Test-Driven Development, a very popular best practice
introduced by Extreme Programming (XP).
\end{abstract}
