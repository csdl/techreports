%%%%%%%%%%%%%%%%%%%%%%%%%%%%%% -*- Mode: Latex -*- %%%%%%%%%%%%%%%%%%%%%%%%%%%%
%% 03-04-abstract.tex -- 
%% Author          : Hongbin Kou
%% Created On      : Fri Jun  9 09:43:42 1995
%% Last Modified By: Hongbing Kou
%% Last Modified On: Tue Sep 21 09:53:22 2004
%% Status          : Unknown
%% RCS: $Id: thesis-abstract.tex,v 1.1 1998/09/19 01:24:42 jagustin Exp $
%%%%%%%%%%%%%%%%%%%%%%%%%%%%%%%%%%%%%%%%%%%%%%%%%%%%%%%%%%%%%%%%%%%%%%%%%%%%%%%
%%   Copyright (C) 1995 University of Hawaii
%%%%%%%%%%%%%%%%%%%%%%%%%%%%%%%%%%%%%%%%%%%%%%%%%%%%%%%%%%%%%%%%%%%%%%%%%%%%%%%
%% 


\begin{abstract}

``Test-Driven Development (TDD), also called Test-First Design (TFD), is a
software development practice in which test cases are incremently written
prior to code implementation\cite{George_2003}''. The rational of TDD is to
``Analyze a little, test a little, code a little and test a little,
repeat.'' This work is to recognize TDD process constructed by many 
Red/Green/Refactoring iterations and evaluate TDD by analyzing the
development activity stream. Upon successful this research will provide the
inspection support and help TDD practitioners continuously improve the development
process.
\end{abstract}

















