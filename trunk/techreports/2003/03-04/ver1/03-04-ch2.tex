%%%%%%%%%%%%%%%%%%%%%%%%%%%%%% -*- Mode: Latex -*- %%%%%%%%%%%%%%%%%%%%%%%%%%%%
%% 03-04-ch2.tex --
%% Author          : Hongbing Kou
%% Created On      : Fri Sep  5 13:50:18 1997
%% Last Modified By: 
%% Last Modified On: Thu Nov 13 17:36:02 2003
%% RCS: $Id$
%%%%%%%%%%%%%%%%%%%%%%%%%%%%%%%%%%%%%%%%%%%%%%%%%%%%%%%%%%%%%%%%%%%%%%%%%%%%%%%
%%   Copyright (C) 1998 Robert Brewer
%%%%%%%%%%%%%%%%%%%%%%%%%%%%%%%%%%%%%%%%%%%%%%%%%%%%%%%%%%%%%%%%%%%%%%%%%%%%%%%
%%

\chapter{Thesis Statement}

I am going to use Hackystat to collect the software development data and
analyze the data collected to study Test-Driven Development process. For my
thesis work I propose to do the following work.

\begin{enumerate}
\item Recognize Red/Green/Refactor(RGR) iteration with the onsite
   development activity data. Failed execution of unit test or test suite
   is the start a RGR iteration and a completed RGR iteration ends with
   green bar. Because of refactorings there may have more than one red bars
   inside a RGR cycle. We can characterize refactoring by tool-supported
   refactoring activities, for example, Eclipse supports class renaming and
   moving etc. The following activities and data are collected to support
   TDD analysis.

     \begin{enumerate}
     \item Editing activities (Including new file, delete file, edit file, rename
     file, remove file, buffer transition)
     \item Compile, recompile, build and rebuild activities
     \item Unit test invocation and test results
     \item General metrics (For example, LOC)
     \item Object metrics (Chidamber-Kermer metrics)
     \end{enumerate}

\item Evaluate software development practice with the RGR analysis
results. After we can recognize and understand RGR iteration we can use
it to evaluate how people are doing TDD in practice. We will be able to
tell whether TDD practitioners are doing TDD and where/when they commit
mistakes.

\item Testability and its effects on TDD\newline One possible big adoption
barrier of TDD is testability. It may take a big amount of time to execute
a test or is too hard to create light-weight unit tests. Is it possible to
achieve 100\% coverage? Which methods or what kinds of methods and objects
can be excluded from test set to achieve 100\% coverage.

\item Verifies the claim that TDD generates 100\% statement
coverage. (Exclude some non-testable cases like main method.) Under what
kinds of situation TDD can yield 100\% coverage.

\item Test context/proximity study \newline Unit test plays an very
  important role in modern software development practice. One reason is
  that unit test suite can be used as regression test to confirm system is
  not broken because of the maintaince changes. If there are tests failed
  some work has to be done to fix the problem to make them pass. With
  Hackystat buffer change data we can associate test class with related
  classes together.
\end{enumerate}

  











%%% Local Variables: 
%%% mode: latex
%%% TeX-master: t
%%% End: 
