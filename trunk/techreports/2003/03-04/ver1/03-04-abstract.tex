%%%%%%%%%%%%%%%%%%%%%%%%%%%%%% -*- Mode: Latex -*- %%%%%%%%%%%%%%%%%%%%%%%%%%%%
%% 03-04-abstract.tex -- 
%% Author          : Hongbin Kou
%% Created On      : Fri Jun  9 09:43:42 1995
%% Last Modified By: 
%% Last Modified On: Wed Nov 12 11:24:02 2003
%% Status          : Unknown
%% RCS: $Id: thesis-abstract.tex,v 1.1 1998/09/19 01:24:42 jagustin Exp $
%%%%%%%%%%%%%%%%%%%%%%%%%%%%%%%%%%%%%%%%%%%%%%%%%%%%%%%%%%%%%%%%%%%%%%%%%%%%%%%
%%   Copyright (C) 1995 University of Hawaii
%%%%%%%%%%%%%%%%%%%%%%%%%%%%%%%%%%%%%%%%%%%%%%%%%%%%%%%%%%%%%%%%%%%%%%%%%%%%%%%
%% 


\begin{abstract}

``Test-Driven Development (TDD), also called Test-First Design (TFD), is a
software development practice in which test cases are incremently written
prior to code implementation\cite{George_2003}''. The rational of TDD is to
``Analyze a little, test a little, code a little and test a little,
repeat.'' This work is to recognize TDD process consisting many 
Red/Green/Refactoring iterations and evaluate TDD by analyzing dynamic
metrics of artifacts created in TDD process. It will answer whether
software developers are doing Test-Driven Development by analyzing on-site
activities and whether Test-Driven Development will yield 100\% code
coverage if developers follow TDD process exactly.
\end{abstract}













