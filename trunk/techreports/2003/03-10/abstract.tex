%%%%%%%%%%%%%%%%%%%%%%%%%%%%%% -*- Mode: Latex -*- %%%%%%%%%%%%%%%%%%%%%%%%%%%%
%% abstract.tex -- 
%% RCS:            : $Id: nsf93-summary.tex,v 1.11 93/10/06 16:52:35 johnson Exp $
%% Author          : Philip Johnson
%% Created On      : Wed Aug 11 12:55:46 1993
%% Last Modified By: Philip M. Johnson
%% Last Modified On: Tue Nov 11 13:18:11 2003
%% Status          : Unknown
%%%%%%%%%%%%%%%%%%%%%%%%%%%%%%%%%%%%%%%%%%%%%%%%%%%%%%%%%%%%%%%%%%%%%%%%%%%%%%%
%%   Copyright (C) 1993 University of Hawaii
%%%%%%%%%%%%%%%%%%%%%%%%%%%%%%%%%%%%%%%%%%%%%%%%%%%%%%%%%%%%%%%%%%%%%%%%%%%%%%%
%% 
%% History
%% 11-Aug-1993          Philip Johnson  
%%    

\section{Abstract}

Highly dependable software is, by nature, predictable.  For example, one
can predict with confidence the circumstances under which the software will
work and the circumstances under which it will fail.  Empirically-based
approaches to creating predictable software are based on two assumptions:
(1) historical data can be used to develop and calibrate models that
generate empirical predictions, and (2) there exists relationships between
{\em internal} attributes of the software (i.e.  immediately measurable
process and product attributes such as size, effort, defects, complexity,
and so forth) and {\em external} attributes of the software (i.e. abstract
and/or non-immediately measurable attributes, such as ``quality'', the time
and circumstances of a specific component's failure in the field, and so
forth).  {\em Software measurement validation} is the process of
determining a predictive relationship between available internal attributes
and correspondingly useful external attributes and the conditions under
which this relationship holds.


The general objective of this research is to design, implement, and
validate software measures within a development infrastructure that
supports the development of highly dependable software systems. The
measures and infrastructure are designed to support dependable software
development in two ways: (1) They will support identification of modules at
risk for being fault-prone, enabling more efficient and effective
allocation of quality assurance resources, and (2) They will support
incremental software development through continuous monitoring,
notifications, and analyses.  Empirical assessment of these methods and
measures during use on the Mission Data System project and related testbeds
at Jet Propulsion Laboratory will advance the theory and practice of
dependable computing and software measurement validation and provide new
insight into the technological and methodological problems associated with
the current state of the art.

 







