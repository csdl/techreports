%%%%%%%%%%%%%%%%%%%%%%%%%%%%%% -*- Mode: Latex -*- %%%%%%%%%%%%%%%%%%%%%%%%%%%%
%% 03-05-ch4.tex --
%% Author          : Robert Brewer
%% Created On      : Fri Sep  5 13:50:18 1997
%% Last Modified By: Aaron Kagawa
%% Last Modified On: Tue Nov 18 18:06:26 2003
%% RCS: $Id: thesis-body.tex,v 1.4 2000/03/17 21:28:10 rbrewer Exp $
%%%%%%%%%%%%%%%%%%%%%%%%%%%%%%%%%%%%%%%%%%%%%%%%%%%%%%%%%%%%%%%%%%%%%%%%%%%%%%%
%%   Copyright (C) 1998 Robert Brewer
%%%%%%%%%%%%%%%%%%%%%%%%%%%%%%%%%%%%%%%%%%%%%%%%%%%%%%%%%%%%%%%%%%%%%%%%%%%%%%%
%%

\chapter{Experimental Design}

The experimental design of this research evaluates my thesis claims
explained in Section 1.4.  I propose to accomplish this by conducting three 
case struides in successive order. First, I will conduct a case study to
understand and validate the hackyJPLBuild representation of the MDS CCC
Harvest build process.  Second, I will conduct a statistical analysis of
historical data in hopes of identifying factors that influence problems in
the build system.  Third, I will conduct a case study to determine if the
hackyJPLBuild sysem can predict problems in the build system.

\section{Case Study 1 - Understanding and Validating}
For the hackyJPLBuild system to be successful, it must be able to
accurately represent the build process of the MDS CCC Harvest Tool.  Case
Study 1 includes understanding and validating data from the MDS CCC Harvest 
Tool.

Prior to August 2003, Dr. Philip Johnson and I created the hackyJPLBuild
system purely on assumptions we had on the data we would receive from the
MDS CCC Harvest data.  In August 2003, I recieved the shell command file
for both the Build Sensor Data and State Change Sensor Data.  These files
represent MDS work from January 2003 to August 2003 (which I will call YTD, 
Year-To-Data, from this point on).  The YTD data was then loaded into the
hackyJPLBuild system.

In this case study, I have conducted several mini studies on the data (1)
to validate my understanding of the data, (2) validate how the
hackyJPLBuild system represents this data, and (3) to validate the
correctness of the data.  I have provided the initial findings of these
mini-studies in the following sections.  Dr. Philip Johnson and I have
gathered more detailed explanation of the findings for a White Paper which
was presented for review by the MDS managers.

It should be noted that these mini studies are not based on any specific
hypotheses; rather it is a continual goal to further enhance my
understanding of the MDS CCC Harvest Tool and MDS, enhance how
hackyJPLBuild represents and analyzes the data, and to validate the actual
data in the hackyJPLBuild system.  Therefore, when interesting
opportunities arise to explore the data, I will analyze them under this
case study.

\subsection{Understanding the MDS Architectural Elements}
MDS Architectural Elements are defined as the incremental development
element.  For example, there are Implementation Task Rollup, Implementation 
Tasks, Change Package Rollups, Verification Rollups, Requirement Package,
Change Package, Internal Anomaly Report, etc.  The goal of this mini-study
is to determine how the hackyJPLBuild system can represent these elements
in the system.

I conducted a simple visual analysis of the shell command data provided by
Rich Hug's script which extracts data from the MDS CCC Harvest Tool.  I
have found that there are three types of elements; MDS Packages, Rollup
Packages, and Work Packages.  Rollup Package represents elements 
that are generated ``top-down'' through planning and requirement
specifications \cite{csdl2-02-03}.  This corresponds to the following
Architectural Elemenets: Implementation Task Rollup, Implementation Task,
Change Package Rollup, and Verification Package Rollup.  MDS Package
represents both ``top-down'' and ``bottom-up,'' where ``bottom-up'' means
that the elements are generated through defects and other problems.  This
corresponds to the following Architectural Elements: Requirement Pacakge,
Change Package, Internal Anomaly Report, Internal Modification, and
Verification Package.  MDS Package is the unifying representation of both
Rollup Packages and Work Packages.

Furthermore, there is a releationship between Rollup Packages and MDS
Packages.  Rollup Packages contain one or many Work Packages.  Work
Packages are the packages that contain the actual implementation and Rollup 
Packages are packages that contain Work Packages.

This finding has allowed me to better represent all Architectural Elements
in the hackyJPLBuild system as either Rollup Packages or Work Packages, see 
MDS Packages, Work Packages, and Rollup Packages in Section 2.2 for a more
detailed explanation of the internal representation of Architectural
Elements in the hackyJPLBuild system.


\subsection{Understanding the MDS CCC Harvest State Model}
In the hackyJPLBuild sysem, I implemented an analyses which derives the MDS 
CCC Havest State Model to understand how the different MDS Packages move
through the Harvest tool.  This analysis calculates the number of
occurrences of State Changes, in a specific time period, and for a given
MDS Package type, ie one of the Architectural Elements.  The following
figure provides an example of the MDS CCC Harvest State Model analysis.

%% Figure of Harvest State Model

Using the data provided by the previous analysis, I have created a
graphical representation of the transitions.  The following figure shows
this graphical representation.

%% CP Harvest State Model.

This figures shows several interesting results.  First, 44\% (76 of the
173) of Change Packages that leave the Dev Harvest State revert back to Dev 
at a later time.  This shows that there is a considerable amout of Rework
or unscheduled work.  Second, 100\% of Change Packages in the Integration
Test Harvest State move to the Release Harvest State without any problems.
Further implications or hypotheses of what these values indicate have not
been determined at this time.


\subsection{Validating the Raw Data}
While conducting my analysis of the MDS CCC Harvest State Model, explained
in the previous section, an interesting problem arose.  The analysis shows
a ``mysterious State'' Changes,  the State Changes were not occuring in
successive chronological order. The following figures shows this anomaly.

%% Figure of mysterious state changes.

Based on this finding, I created a Hackystat analysis, whick finds all MDS
Packages with the ``mysterious'' State Changes.  The following figure shows 
an example of this analysis.

%% Figure of Bad State Change Analsis

This analysis shows that 20\% of all Change Packages have a'' mysterious''
State Change at some point in time.  This indicates a major problem either
two things; (1) the MDS CCC Harvest Tool or (2) the script which extracts
data from the MDS CCC Harvest Tool.  In any case this problem must be
addressed.


\section{Case Study 2 - Statistical Analysis of YTD Data}
Case Study 2 includes analyzing the YTD data to identify thresholds for
factors that are associated with certain measurements of the MDS system.
Proposed factors include: number of developers, number of entanglements,
number of files in a MDS Package, and the number of build failures.
Propsed measurements include: throughput and age.

For example, I propose the creation of the ``Throughput Threshold''
analysis.  This analysis will find the average throughput value for a
certain type of a MDS Package.  For throughput valudes that are above the
average, I will calculate the threshold values by finding the averages of
the factors.  The following table shows an example of the threshold values.

Measurement: Throughput 7 Months
Thresholds: Number of Developers = 2, Number of Entanglements = 6, Number
of Files = 25, and Number of Build Failures = 4.

The previous example data shows the thresholds for several factors that
influence throughput.  For example, MDS Packages with 6 or more
entanglements will have an above average throughput measurement, MDS
Packages with 25 or more files will have an above average throughput
measurement, and so on.

If analyses can identify threshold attributes of the MDS system then,
predicting problems in the Harvest System should be relatively easy.  The
next section explains Case Study 3 - Predicting Problems.


\section{Case Study 3 - Predicting Problems}
If the previous case study can provide thresholds that indicate above
average measurements, then Case Study 3 can investigate if these
thresholds can predict future above average measurements.  This case study
will analyze real time data provided by Harvest.

The methodology for this case study is the following.  For two months the
threshold values for above average measurements will be stored in the
hackyJPLBuild system.  When a MDS Package contains a value beyond the
threshold value, the system will record the package's unique identifier and
other information about the package's current state.  At the end of two
months, I will conduct a study of the MDS Packages that were stored in the
system to analyze if the threshold values predicted about average
measurements.

If analyses can accurately predict above average measurements of the MDS
system then, preventing problems is the next logical step.  The next
section explains how hackyJPLBuild could be used to prevent problems.


\section{Preventing Problems}
The next logical step, assuming that the three previous case studies are
successful, is to apply the predictions in a prevention case study.
Preventing porblems means that we must intervene with the build and
developement process of MDS however, I feel that intervention is not an
obtainable goal for the duration of this thesis project.

When the time becomes appropriate for hackyJPLBuild to intervene with MDS,
hackyJPLBuild will have collected a significant amount of raw data and have 
created a significant amount of thresholds and predictions.  At this time,
the hackyJPLBuild system can alert developers of potential problems that
have been predicted.  Developers can then, attempt to ``bring the threshold 
value down,'' or pay special attention to the potentially problematic MDS
Package.














