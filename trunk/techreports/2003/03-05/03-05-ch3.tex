%%%%%%%%%%%%%%%%%%%%%%%%%%%%%% -*- Mode: Latex -*- %%%%%%%%%%%%%%%%%%%%%%%%%%%%
%% 03-05-ch3.tex --
%% Author          : Robert Brewer
%% Created On      : Fri Sep  5 13:50:18 1997
%% Last Modified By: Aaron Kagawa
%% Last Modified On: Wed Nov 19 19:47:29 2003
%% RCS: $Id: thesis-body.tex,v 1.4 2000/03/17 21:28:10 rbrewer Exp $
%%%%%%%%%%%%%%%%%%%%%%%%%%%%%%%%%%%%%%%%%%%%%%%%%%%%%%%%%%%%%%%%%%%%%%%%%%%%%%%
%%   Copyright (C) 1998 Robert Brewer
%%%%%%%%%%%%%%%%%%%%%%%%%%%%%%%%%%%%%%%%%%%%%%%%%%%%%%%%%%%%%%%%%%%%%%%%%%%%%%%
%%

\chapter{hackyJPLBuild System Architecture and Design}

This chapter provides a detailed explanation of the Hackystat Jet
Propulsion Laboratory Build System (hackyJPLBuild).  The initial
requirements of this system developed by Dr. Philip Johnson
\cite{csdl2-02-03}.  Using this proposal, I designed and implemented this
system.  The following sections provide explanations of the Sensor Data
Types, Abstract Data Types, and Analyses that the hackyJPLBuild system
provides.

\section{Sensor Data Types}
The hackyJPLBuild system provides two Sensor Data Types; the StateChange
and Build Sensor Data Types.  To populate the data for these SDTs a
specially designed Hackystat sensor will be attached to the MDS Harvest CM
Tool.  


\subsection{State Change Sensor Data Type}
The StateChange Sensor Data Type represents \emph{promotion} or
\emph{demotion} events of a MDS Package and aspects about the pacakge's
state at this point in time.  It contains the following fields:


\begin{description}
\item State Change Sensor Data Type Attributes:
\begin{enumerate}
\item \emph{tstamp}: A UTC timestamp (i.e. the number of milliseconds since 
9/1/1970.  Example: ``105071895265'' (which is roughly 4/18/2003
2:08pm). Every StateChange instance must have a unique tstamp, and this
tstamp should represent the time at which the promotion or demotion took
place.
\item \emph{packageId}: The unique identifier for the MDS Package that is
being promoted or demoted.  Example: ``CP-00971''.
\item \emph{startState}: The Harvest CM State this MDS Package was in just
before the promotion/demotion.  Example: ``Dev''.
\item \emph{endState}: The Harvest CM State this MDS Package is now in as a 
result of the promotion/demotion.  Example: ``Dev-Complete''.
\item \emph{newFiles}: A comma delimited list of file names that have been
added since the last state change.
\item \emph{modifiedFiles}: A comma delimited list of file names that have
been changed since the last state change.
\item \emph{deletedFiles}: A comma delimited list of file names that have
been deleted since the last state change.
\item \emph{unchangedFiles}: A comma delimited list of file names that have 
not been changed since the last state change.
\item \emph{iar}: If this MDS Package (such as a CP) associated with an
IAR, the IAR packageId is provided here.  Example: ``IAR-02068''.
\item \emph{developer}: The username of the developer responsible for this
MDS Package.  Example: ``cxing''.
\end{enumerate}
\end{description}

The following is an example of a StateChange Sensor Data Entry in XML form.

\begin{alltt}
{\small{}
<?xml version=''1.0'' encoding=''UTF-8'' ?>
<sensor>
  <entry tstamp=''97847280003'' tool=''harvest'' pacakgeId=''CP-00971''
    startState=''dev'' endState=''dev_complete'' newFiles=''newFile.cc''
    modifiedFiles=''modifiedFile.cc'' deletedFiles=''deletedFile.cc''
    unchangedFiles=''unchangedFile.cc'' iar=''IAR-02068''
    developer=''cxing'' />
</sensor>
}
\end{alltt}


\subsection{Build Sensor Data Type}
The Build Sensor Data Type represents an occurence of the build with a
given MDS Package.  It contains the following fields:

\begin{description}
\item Build Sensor Data Type Attributes:
\begin{enumerate}
\item \emph{tstamp}: A UTC timestamp representing the time at which the
build took place.  Example: ``1050710895265''.
\item \emph{packageId}: The unique identifier for the MDS Package that is
being built.  Example: ``CP-00971''.
\item \emph{testPassed}: A number indicating the total passed tests.
Example: ``25''.
\item \emph{testFailed}: A number indicating the total failed tests.
Example: ``0''.
\item \emph{failureType}: One of none, compile, link, and run-time.
\end{enumerate}
\end{description}

The following is an example of a Build Sensor Data Entry in XML form.

\begin{alltt}
{\small{}
<?xml version=''1.0'' encoding=''UTF-8'' ?>
<sensor>
  <entry tstamp=''97847280003'' tool=''harvest'' pacakgeId=''CP-00971''
    testsPassed=''10'' testsFailed=''0'' failureType=''none'' />
</sensor>
}
\end{alltt}


\subsection{Sending data to the server}
Sensor Data Types define the type of data that can be sent to a Hackystat
server.  To accomplish sending data, a Hackystat Sensor must be attached to
the development tool which we wish to monitor, in this case, Harvest.  This
sensor has been partly designed by Rich Hug of the Jet Propulsion
Laboratory, the local Harvest expert.  He developed a script which extracts
data from Harvest into a text file.  This text file can then be used as
input for the Harvest Sensor.

We have found that sending all the data from January 2003 to October 2003
to a Hackystat server has a duration of a few seconds.


\subsection{Data in the server}
Once the Sensor Data is in a Hackystat server, the Sensor Data is used as
the primary source of data, refered to as ``Raw Data''.  This raw data is
the finest grained source of data the hackyJPLBuild analyzes.  However,
analyzing raw data is computationallyn expensive becuase it is so fine
grained.  The following section discusses how I solved this problem.


\section{MDS, Work, and Rollup Packages}
The Sensor Data Types provide access to fine grained measurements of the
Harvest Tool.  However, the raw data is computationally expensive.
Therefore, a larger grained measurement was needed.  The goal of creating a
larger grained measurement is to provide aggregated, easily accessable, and
derived data that cannot be obtained from just the raw data.  To represent
this larger grained measurement the hackyJPLBuild system collects together
all Sensor Data Entries related to a specific MDS Package into an Abstract
Data Type (ADT).


There are three different types of ADTs; (1) the 
MDS Package, (2) the Work Package, and (3) the Rollup Package.  Figure 1
provides a UML diagram of the ADTs.

%% Figure of the MDS Package UML diagram goes here

The following explains the attributes of each ADT.
\begin{description}
\item MDS Package:
\begin{enumerate}
\item \emph{packageID}: The unique identifier of a MDS Package
\item \emph{startState}: The first Harvest CM State of the MDS Package.
This is determined by getting the startState of the first StateChange
Sensor Data Entry associated with this MDS Package.
\item \emph{endState}: The last Harvest CM State of the MDS Package.  This
is determined by getting the endState of the last StateChange Sensor Data
Entry associated with this MDS Package.
\item \emph{developers}: The developers that have contributed to this MDS
Package.
\end{enumerate}
\item Rollup Package:
\begin{enumerate}
\item \emph{type}: One of ITR, IT, or CPR.
\end{enumerate}
\item Work Package:
\begin{enumerate}
\item \emph{type}: One of CP, IAR, IM, or RP.
\item \emph{newFiles}: The total number of new files in this Work Package.
\item \emph{modifiedFiles}: The total number of modified files in this Work 
Package.
\item \emph{deletedFiles}: The total number of deleted files in this Work
Package.
\item \emph{iars}: The IARs associated with this Work Package.
\end{enumerate}
\end{description}


There are several things to note about the representations provided
here. (1) A MDS Package is either a Rollup Package or a Work Package.  (2)
Rollup Packages represent documentation level Pacakges.  (3) Work Packages
represent where the actual ``work'' is done.  For this reason, Work Package
are the only Package that contains attributes for files, because ``work''
cannot be done on Rollup Packages.

\section{hackyJPLBuild Analyses}

\begin{figure}[htbp]
  \centering
  \includegraphics[width=1.0\textwidth]{figs/mds.development.efficiency.eps}
  \caption{hackyJPLBuild MDS Development Efficiency Analysis Set}
  \label{fig:mds.development.efficiency}
\end{figure}




