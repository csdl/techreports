%%%%%%%%%%%%%%%%%%%%%%%%%%%%%% -*- Mode: Latex -*- %%%%%%%%%%%%%%%%%%%%%%%%%%%%
%% project-description.tex -- 
%% Author          : Robert Brewer
%% Created On      : Wed Jan 27 16:28:11 1999
%% Last Modified By: Robert Brewer
%% Last Modified On: Mon Jan 31 16:22:06 2000
%% RCS: $Id: project-description.tex,v 1.1 2000/01/31 22:12:28 rbrewer Exp rbrewer $
%%%%%%%%%%%%%%%%%%%%%%%%%%%%%%%%%%%%%%%%%%%%%%%%%%%%%%%%%%%%%%%%%%%%%%%%%%%%%%%
%%   Copyright (C) 1999 Robert Brewer
%%%%%%%%%%%%%%%%%%%%%%%%%%%%%%%%%%%%%%%%%%%%%%%%%%%%%%%%%%%%%%%%%%%%%%%%%%%%%%%
%% 

\setlength{\oddsidemargin}{0in}
\setlength{\textwidth}{7in}

\section{Project Description}
%No more than 1000 words, currently ? words

\subsection{Introduction}
%What problem are you solving?
% 283 words
Anyone who has ever had an email account has received messages containing
unwanted advertisements (commonly called spam or Unsolicited Commercial Email
[UCE]). The problem of spam has spawned fierce debate, considerable software
development (both for sending it and for filtering it), and plenty of
legislation. Despite all this, there are lots of marketers who continue to send
spam and millions of email users who want to get it out of their in-box.

Spam has become a big problem because it allows mass marketers to shift the
costs of delivering their advertisements to the recipient in a way that is not
possible with traditional media. When an advertiser sends you junk mail or
calls you on the phone it costs them a small amount of money each time they do
so. With spam the cost to the sender is miniscule, with most of the cost is
absorbed by the receiver and their Internet Service Provider (ISP). This
results in marketers being tempted to send out their advertisements to hundreds
of thousands of people knowing that even a 0.1\% response rate will make them a
profit. It also opens the door for advertisements of products like pornography
or illegal cable descramblers which many people don't wish to receive.

The problem is just as bad for the ISPs who process the mail. It's estimated
that as much as 75\% of email traffic is spam (see
\url{<http://www.emarketer.com/estats/020199_email.html>}) which means a
significant amount of an ISP's bandwidth is consumed by spam. The situation is
particularly bad here in Hawaii and around the Pacific Rim where Internet
bandwidth costs many times what it does on the US mainland.

\subsection{Existing Solutions Are Not Effective}
%Why is your solution unique?
% 274 words
The problem of spam has not been solved to date because it is a complicated
issue with economic, social, and political facets. Existing attempts to solve
the problem include:

\begin{itemize}
\item Legislation. Bills have been proposed (and sometimes passed) at the state
  and federal level. Some bills seek to legitimize spam leading to fears that
  this might `open the floodgates'. Other bills seek to criminalize it which
  some feel would be a bad precedent for free speech.
\item Email client filters. These are usually limited to clumsy user-generated
  rules which attempt to catch spam and delete it. Keeping the rules up to date
  is tedious and error prone, and each program has a different way of setting
  up filters.
\item ISP server filters. These filters usually work based on the source of the
  message so they are limited to catching spam runs which have already been
  detected and reported by others. In addition, they typically reject messages
  wholesale which takes the decision out of the hands of the recipient.
\end{itemize}

\subsection{Adaptive Personalized Spam Filters}
%What is the solution to the problem?
% 347 words
I am proposing a hybrid technical solution to the problem of spam: filtering on
both the user and ISP side. The filtering program on the user side is
predicated on differences between a spam message and a normal message. A series
of tests will be performed on each message in a users incoming mailbox . Each
of these tests will attempt to determine if the message is spam. Some of the
obvious tests that can be used are: is this message actually addressed to the
user (spam frequently is sent with the digital equivalent of ``Current
Resident'', is this message from a previous correspondent, does the message
contain telltale phrases like ``MAKE MONEY FAST''. More subtle tests will look
for things like: excessive CAPITALIZATION, excessive use of exclamation points,
lists of addresses for pyramid schemes. However, unlike most filtering systems
each test will produce a probability instead of a yes or no result. The
accuracy of the tests will be determined by feedback from the user after
examining the actual mail. The probabilities generated by each test will be
combined into a `spam probability' for each message. The spam probability can
be used in several ways: users can sort their email so that the messages that
are probably spam are shown last, or users can set a threshold probability
above which messages are automatically deleted or filed away.

This system has advantages over existing client filters. The system doesn't
have to make binary decisions about spam which allows users to set their own
tolerance for spam. It also adapts to the mail patterns of the individual user
without requiring the user to manually figure out what rules to input. However,
it leaves the final decision in the hands of the user.

%Talk about client software?: Java for platform independence, not dependent
%on a particular email client. IMAP. JavaMail.

On the server side, the system will examine each message as it is received and
assigns a spam probability. This would allow the ISP to completely reject some
messages based on initial information based on the recipient's stored profile,
thus saving bandwidth.

This system, if widely implemented, has the potential to not only greatly
reduce spam received, but it can contribute to reduced {\bf production} of
spam. When spam is filtered out of most users mailboxes, spam will cease to be
a useful marketing tool and spammers will stop spamming.

%\subsubsection{Example of MCS Use in a Classroom Setting}
%How will it actually work?

\subsection{Business Model}
%How will this make money?
%What is the revenue potential?
%What is the potential market?
% 106 words
I plan to market the client software and server software differently. I will
give away the client filtering software, allowing users to start getting rid of
spam immediately. The client software will not require their ISP to install any
software (which some ISPs are reluctant to do and even when they want to it can
take months). However, I will sell the server software to ISPs as they are more
likely to be willing to pay since they can simply account for it as a cost of
doing business. The free client software will also act as an excellent
advertisement for the server software: users will download the free software
and encourage their ISP to purchase the server side. This is similar to the
successful ``free client/costly server'' business model used by Netscape, Sun,
Microsoft and others.

%Next steps/scale up

\subsection{Project Plan}
%What is the timeline for the project?
%What milestones are there along the way to completion?
%What are the deliverables?
% 136 words
I will be working on this project after graduating in Spring 2000. I am
familiar with the tools required to write the software so I expect an initial
version of client software can be completed in a few months. At that point I
will release the client software to the world so I can get actual user
feedback. The server software will be more complicated because ISPs use a wide
variety of mail server software so work on the server software will begin after
the client is released.

\subsubsection{Deliverables}
% ? words
By the end of the grant award period, I expect to have a working version of the
client software in general public release. Depending on the demand from users,
I may also have a version of server software available. I will also have a
report describing the progress I have made on the project and relating the
lessons learned in the process.

% LocalWords:  http Kimo tex Mailinglist csdl ics hawaii edu html Kimo's htbp
% LocalWords:  mcs eps redistributable CD MAILE LocalWords videoconferencing
% LocalWords:  htb descramblers Exp spammers spamming
