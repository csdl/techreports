\chapter{Google Form for In-lab Evaluation Experiments}
\label{app:googleform}

This appendix lists the google forms that are used by the students volunteerly participated in the in-lab evaluation experiements for system admin and game designer efficiency.

\section{System admin efficiency}
There are two forms to assess the system admin efficiency.

\subsection{Makahiki Local Installation Log}

\setlength{\parindent}{0pt}
\setlength{\parskip}{3mm}

Please follow the steps outlined in this form to install Makahiki locally (including Virtualbox Linux Guest) and log the time you spent for each step.
Please choose the closest value from the list that best matches the time you spent during the installation.

Thank you!

* Required

2.1.1.1.1.2. Install Python *

Complete the "Install Python" section in Makahiki Local Installation Manual (http://makahiki.readthedocs.org/en/latest/installation-makahiki-unix.html\#install-python), record the time you spent for this section only:

Record any problem(s) you encountered when installing Python:

2.1.1.1.1.3. Install C Compiler *

Complete the "Install C Compiler" section in Makahiki Local Installation Manual(http://makahiki.readthedocs.org/en/latest/installation-makahiki-unix.html\#install-c-compiler), record the time you spent for this section only:

Record any problem(s) you encountered when installing C compiler:

2.1.1.1.1.4. Install Git *

Complete the "Install Git" section in Makahiki Local Installation Manual(http://makahiki.readthedocs.org/en/latest/installation-makahiki-unix.html\#install-git), record the time you spent for this section only:

Record any problem(s) you encountered when installing Git:

2.1.1.1.1.5. Install Pip *

Complete the "Install Pip" section in Makahiki Local Installation Manual(http://makahiki.readthedocs.org/en/latest/installation-makahiki-unix.html\#install-pip), record the time you spent for this section only:

Record any problem(s) you encountered when installing Pip:

2.1.1.1.1.6. Install Virtual Environment Wrapper *

Complete the "Install Virtual Environment Wrapper" section in Makahiki Local Installation Manual(http://makahiki.readthedocs.org/en/latest/installation-makahiki-unix.html\#install-virtual-environment-wrapper), record the time you spent for this section only:

Record the problem you encountered when installing virtual environment wrapper:

2.1.1.1.1.7. Install Python Imaging Library *

Complete the "Install Python Imaging Library" section in Makahiki Local Installation Manual (http://makahiki.readthedocs.org/en/latest/installation-makahiki-unix.html\#install-python-imaging-library), record the time you spent for this section only:

Record any problem(s) you encountered when installing Python imaging library:

2.1.1.1.1.8. Install PostgreSQL *

Complete the "Install PostgreSQL" section in Makahiki Local Installation Manual (http://makahiki.readthedocs.org/en/latest/installation-makahiki-unix.html\#install-postgresql), record the time you spent for this section only:

Record any problem(s) you encountered when installing PostgreSQL:

2.1.1.1.1.9. Install Memcache *

Complete the "Install Memcache" section in Makahiki Local Installation Manual (http://makahiki.readthedocs.org/en/latest/installation-makahiki-unix.html\#install-memcache), record the time you spent for this section only:

Record any problem(s) you encountered when installing Memcache:

2.1.1.1.1.10. Download the Makahiki source *

Complete the "Download Makahiki source" section in Makahiki Local Installation Manual (http://makahiki.readthedocs.org/en/latest/installation-makahiki-unix.html\#download-the-makahiki-source), record the time you spent for this section only:

Record the problem you encountered when download the Makahiki source:

2.1.1.1.1.11. Workon Makahiki *

Complete the "Workon Makahiki" section in Makahiki Local Installation Manual (http://makahiki.readthedocs.org/en/latest/installation-makahiki-unix.html\#workon-makahiki), record the time you spent for this section only::

Record any problem(s) you encountered when activating Makahiki virtual environment:

2.1.1.1.1.12. Install required packages *

Complete the "Install required packages" section in Makahiki Local Installation Manual (http://makahiki.readthedocs.org/en/latest/installation-makahiki-unix.html\#install-required-packages), record the time you spent for this section only:

Record any problem(s) you encountered when Installing required packages:

2.1.1.1.1.13. Setup environment variables *

Complete the "Setup environment variables" section in Makahiki Local Installation Manual (http://makahiki.readthedocs.org/en/latest/installation-makahiki-unix.html\#setup-environment-variables), record the time you spent for this section only:

Record the problem you encountered when setting up environment variables:

2.1.1.1.1.14. Initialize Makahiki *

Complete the "Initialize Makahiki" section in Makahiki Local Installation Manual (http://makahiki.readthedocs.org/en/latest/installation-makahiki-unix.html\#initialize-makahiki), record the time you spent for this section only:

Record any problem(s) you encountered when initializing Makahiki:

2.1.1.1.1.15. Start the server *

Complete the "Start the server" section in Makahiki Local Installation Manual (http://makahiki.readthedocs.org/en/latest/installation-makahiki-unix.html\#start-the-server), record the time you spent for this section only:

Record any problem you encountered when starting the server:

2.1.1.1.1.16. Verify that Makahiki is running *

Complete the "Verify that Makahiki is running" section in Makahiki Local Installation Manual (http://makahiki.readthedocs.org/en/latest/installation-makahiki-unix.html\#verify-that-makahiki-is-running), record the time you spent for this section only:

Record any problem you encountered when verifying that Makahiki is running:

Your UH email: *

\subsection{Makahiki Local Installation Log}

Please follow the steps outlined in this form to install Makahiki on Heroku and log the time you spent for each step.
Please choose the closest value from the list that best matches the time you spent during the installation.

Thank you !

* Required

2.1.1.2.1. Install Heroku *

Complete the "Install Heroku" section in Makahiki Heroku Installation Manual (http://makahiki.readthedocs.org/en/latest/installation-makahiki-heroku.html\#install-heroku), record the time you spent for this section only:

Record any problem(s) you encountered when installing Heroku:

2.1.1.2.2. Add your SSH keys to Heroku *

Complete the "Add your SSH keys to Heroku" section in Makahiki Heroku Installation Manual (http://makahiki.readthedocs.org/en/latest/installation-makahiki-heroku.html\#add-your-ssh-keys-to-heroku), record the time you spent for this section only:

Record any problem you encountered when adding your SSH keys to Heroku:

2.1.1.2.3. Verifying your Heroku account *

Complete the "Verifying your Heroku account" section in Makahiki Heroku Installation Manual (http://makahiki.readthedocs.org/en/latest/installation-makahiki-heroku.html\#verifying-your-heroku-account), record the time you spent for this section only:

Record any problem you encountered when verifying your Heroku account:

2.1.1.2.4. Setup Amazon S3 *

Complete the "Setup Amazon S3" section in Makahiki Heroku Installation Manual (http://makahiki.readthedocs.org/en/latest/installation-makahiki-heroku.html\#setup-amazon-s3), record the time you spent for this section only:

Record any problem you encountered when setting up S3:

2.1.1.2.5. Setup environment variables *

Complete the "Setup environment variables" section in Makahiki Heroku Installation Manual (http://makahiki.readthedocs.org/en/latest/installation-makahiki-heroku.html\#setup-environment-variables), record the time you spent for this section only:

Record any problem you encountered when setting up environment variables:

2.1.1.2.6. Download the Makahiki source *

Complete the "Download the Makahiki source" section in the Makahiki Heroku Installation Manual (http://makahiki.readthedocs.org/en/latest/installation-makahiki-heroku.html\#download-the-makahiki-source), record the time you spent for this section only:

Record any problem you encountered when download the Makahiki source:

2.1.1.2.7. Initialize Makahiki *

Complete the "Initialize Makahiki" section in the Makahiki Heroku Installation Manual (http://makahiki.readthedocs.org/en/latest/installation-makahiki-heroku.html\#initialize-makahiki), record the time you spent for this section only:

Record any problem you encountered when initializing Makahiki:

2.1.1.2.8. Start the server *

Complete the "Start the server" section in the Makahiki Heroku Installation Manual (http://makahiki.readthedocs.org/en/latest/installation-makahiki-heroku.html\#start-the-server), record the time you spent for this section only:

Record any problem you encountered when starting the server:

2.1.1.2.9. Verify that Makahiki is running *

Complete the "Verify Makahiki is running" section in the Makahiki Heroku Installation Manual (http://makahiki.readthedocs.org/en/latest/installation-makahiki-heroku.html\#verify-that-makahiki-is-running), record the time you spent for this section only:

Record any problem you encountered when verifying that Makahiki is running:

Your UH email: *

\section{Game designer efficiency}
There is one form to assess the game designer efficiency.

\subsection{Makahiki Configuration and Management Log}

Please follow the steps outlined in this form to configure and manage Makahiki, and log the time you spent and problems encountered for each step. Record the time you actually spent doing the tasks by choosing the closest value from the list that best matches the time you spent.
The Makahiki manual referenced below may use the local instance 127.0.0.1 as the example. For this assignment, you should use the Makahiki instance you deployed in Heroku instead of your local instance.

Thank you !

* Required

0. Update your Heroku Makahiki instance *

Read the "Updating your Makahiki instance" section in Makahiki Manual (http://makahiki.readthedocs.org/en/latest/installation-makahiki-heroku.html\#updating-your-makahiki-instance). Follow the instructions to update your Heroku instance with any changes from the Makahiki Git repository. Record the time you spent for this step only:

Record any problem(s) you encountered in this step:

1. Getting to the challenge design page *

Read the "Getting to the challenge design page" section in Makahiki Manual (http://makahiki.readthedocs.org/en/latest/challenge-design.html\#getting-to-the-challenge-design-page). Then go to the challenge design setting page of your Heroku instance. Record the time you spent for this step only:

Record any problem(s) you encountered in this step:

2. Design the global settings *

Read the "Design the global settings" section in Makahiki Manual (http://makahiki.readthedocs.org/en/latest/challenge-design-name-settings.html). In your Heroku instance, change the "Name" of the challenge and the "Logo" fields to ones of your choosing. Test that your change is in effect by checking the Logo image and label at the top of any page. Record the time you spent for this step only:

Record any problem you encountered in this step:

3. Design the teams *

Read the "Design the teams" section in Makahiki Manual (http://makahiki.readthedocs.org/en/latest/challenge-design-teams-settings.html). In your Heroku instance, add a new team called "Lehua-C" with the same group membership as the other teams in the default instance. Record the time you spent for this step only:

Record any problem you encountered in this step:

4. Set up users *

Read the "Set up users" section in Makahiki Manual (http://makahiki.readthedocs.org/en/latest/challenge-design-players-settings.html). Add two new users of your choosing to the team "Lehua-C". Make sure you assign the players to their team by going to the user's profile link. Test your changes by logging in as one of the new players, and verifying that the player is on the right team. Record the time you spent for this step only:

Record any problem you encountered in this step:

5. Specify the games to appear in your challenge *

Read the "Specify the games to appear in your challenge" section in Makahiki Manual (http://makahiki.readthedocs.org/en/latest/challenge-design-game-admin-enable-disable.html). Disable the "Water Game", and leave the other games enabled. You should see that the "Drop Down" page disappears from the top navigation bar. Record the time you spent for this step only:

Record any problem you encountered in this step:

6. Learn about how to design the resource goal games *

Read the "Design the Resource Goal Games" section in the Makahiki Manual (http://makahiki.readthedocs.org/en/latest/challenge-design-game-admin-resource-game.html). Record any questions or confusion that arises from reading this section:

6.1. Configure the Energy Goal Game for your new team *

Change the energy goal setting for the team "Lehua-C" to use manual data, and specify a time for the manual data input time. Test your changes by logging in as a player of Lehua-C, then go to "Go Low" page. You should see the calendar view of the daily energy goal game instead of the stop light visualization. Record the time you spent for this step only:

Record any problem you encountered in this step:

7. Learn about how to design Smart Grid Games *

Read the "Design the Smart Grid Game" section in the Makahiki Manual (http://makahiki.readthedocs.org/en/latest/challenge-design-game-admin-smartgrid-game.html). Record any questions or confusion that arises from reading this section:

7.0. Design on paper *

The default installation defines a Smart Grid Game (SGG) with 3 levels. For this task, design a new Level 4 that extends the existing SGG. Level 4 will have a total of four actions: 3 new actions (Activity, Event, Commitment) that you create yourself, and one old action that you choose from the existing library of actions in the default installation. Design Level 4 with a 2x2 grid layout, including 2 categories of your choice. For this step, you will only design your Level 4 on a piece of paper or a spreadsheet, as described in Makahiki Manual (http://makahiki.readthedocs.org/en/latest/challenge-design-game-admin-smartgrid-game.html\#designing-your-smart-grid-game). Specify the unlock conditions for each action to achieve some kind of unlocking sequence("path"), such as depending on the completion of other actions. Record the time you spent in this step:

Record any problem you encountered in this step:

7.1. Create a Level *

Add a new level "Level 4", with priority higher than Level 3, and some unlock condition depending on some actions from Level 2. Record the time you spent for this step only:

Record any problem you encountered in this step:

7.2 Create a new Activity action *

Create a new activity action with your own content. Make the content meaningful. Fill in the required fields. You will also specify the level (should be level 4), category (your choice), as well as the unlock condition field, which determines the action "path" of your SGG design as described in step 7.0. Record the time you spent for this step only:

Record any problem you encountered in this step:

7.3 Create a new Event action *

Create a new event action with your own content. Make the content meaningful. Fill in the required fields. You will also specify the level field (should be level 4), category field (your choice), as well as the unlock condition field, which determines the action "path" of your SGG design as described in step 7.0. Record the time you spent for this step only:

Record any problem you encountered in this step:

7.4 Create a new Commitment action *

Create a commitment action with your own content. Make the content meaningful. Fill in only the required fields. You will also specify the level field (should be level 4), category field (your choice), as well as the unlock condition field, which determines the action "path" of your SGG design as described in step 7.0. Record the time you spent for this step only:

Record any problem you encountered in this step:

7.5 Finalize the grid *

At this point, you should have created 3 new actions and put them in Level 4 of your SGG. For this step, find the final action to complete your 2x2 grid.. Go to the admin interface, find an action in the action library, and modify the level, category and unlock condition field according to your SGG design. Play-test your grid by logging in as normal player, go to the "Get Nutz" page, unlock Level 4 and all actions in Level 4. Record the time you spent for this step only:

Record any problem you encountered in this step:

8. Design the Top Score Game *

Read the "Design the Top Score Game" section in the Makahiki Manual (http://makahiki.readthedocs.org/en/latest/challenge-design-game-admin-topscore-game.html), create a new topscore prize of your choice. Test your changes by going to the "Prizes" page to see your newly created prize. Record the time you spent for this section only:

Record any problem you encountered in this step:

9. Design the Raffle Game *

Read the "Design the Raffle Game" section in the Makahiki Manual (http://makahiki.readthedocs.org/en/latest/challenge-design-game-admin-raffle-game.html). Create a new raffle prize of your choice. Test your changes by going to the "Prizes" page to see your newly created raffle prize and you can add raffle ticket to it. Record the time you spent for this section only:

Record any problem you encountered in this step:

10. Design the Badge Game Mechanics *

Read the "Design the Badge Game Mechanics" section in the Makahiki Manual (http://makahiki.readthedocs.org/en/latest/challenge-design-game-admin-badge.html). Create a new badge with an award trigger type of "smartgrid". Specify some kind of awarding condition depending on the smartgrid operations. Verify that your badge shows up in the badge catalog page and you can be awarded the new badge by doing the specified smartgrid action. Record the time you spent for this section only:

Record any problem you encountered in this step:

11. Manage Action submissions *

Read the "Manage Action submissions" section in the Makahiki Manual (http://makahiki.readthedocs.org/en/latest/execution-manage-smartgrid-game.html\#manage-action-submissions). Approve some actions submitted by you during your playtesting. Record the time you spent for this section only:

Record how many actions you approved, and record any problem you encountered in this step:

Your UH email: *
