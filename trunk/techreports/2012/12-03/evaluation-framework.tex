\chapter{SGSEAM Design}
\label{cha:framework-description}
This chapter describes the design of my proposed Serious Game Stakeholder Experience 
Assessment Method (SGSEAM). It starts with the overview of SGSEAM, followed by the discussion of assessment methodology, and the details of the proposed assessment method.

\section{Overview of SGSEAM}

The goal of SGSEAM is to identify (a) major strengths of a serious game
framework, which aids the community by indicating features of the framework to emulated, and
(b) major shortcomings of the framework, which aids the community by indicating features to avoid
and the developers of the framework by indicating the areas to improve on.

The approach that SGSEAM uses is to assess the experiences of various important stakeholders when
they interact with the serious game framework. In the full life cycle of a serious game framework
there are a great variety of potential stakeholders, including:

\begin{itemize}
\item \emph{Players}: those who participate in the game produced by the framework.
\item \emph{System admins}: those who install and maintain the technological game infrastructure.
\item \emph{Game designers}: those who design the content and game mechanics.
\item \emph{Game managers}: those who manage the game during the period of game play.
\item \emph{Developers}: those who extend, enhance and debug the game framework.
\item \emph{Researchers}: those who are conducting research using the game framework.
\item \emph{Spectators}: those who do not participate in the game play
  but are interested in the game and the results of game play. 
\item \emph{Community partners}: those who partner with the game
  organizers to help run the game (such as coordinating real-world
  events as part of the game, providing support for energy data
  collection, etc) 
\item \emph{Funding organizations}: the organizations who provide
  funding for the game or game framework.
\end{itemize}

The scope of SGSEAM is to assess serious game frameworks as software infrastructure. While
the overall success of a serious game depends on the individual success of all of these
stakeholders, SGSEAM does not address the spectator, community partner, researchers, and funding
organization stakeholders. These are important stakeholders but outside the scope of our
assessment. In the context of a serious game framework, SGSEAM focuses on players, system admins, game designers, developers and researchers.

The following sections describe the methodology used in SGSEAM, followed by the detailed
description of assessment methods for each identified stakeholder.

\section{Assessment Methodology}
Creswell \cite{creswell2003} categorizes research methods into three approaches:
quantitative, qualitative, and mixed methods, according to what knowledge claims are being made
and how knowledge is acquired. Quantitative method reflects a post-positivist paradigm where
hypotheses are specified {\em a priori} and tested by experimental design. Qualitative method
reflects a constructivist or participatory paradigm where knowledge would be acquired by
observation and open-ended design. SGSEAM employs the mixed methods approach which based on
pragmatic knowledge claims and assumption that collecting diverse types of data provides better
understanding of the research problem: assessing the strengths and shortcomings of a serious game
framework.

In SGSEAM, the concurrent triangulation strategy \cite{creswell2003} of the mixed method approach
is used.  Data collection and analysis involves both quantitative information (instrument and
analytical data recorded by the system such as website logs, interaction database, etc), as well
as qualitative information (interviews and questionnaire responses).

\section{Stakeholder Experience Assessment}

SGSEAM follows closely with the "Goal-Question-Metric" (GQM) approach \cite{caldiera1994goal} in
software engineering research. GQM defines a software  measurement model on three levels: a goal
of the measurement, a set of questions to assess the goal, and a set of metrics associated with
each question.

In SGSEAM, the assessment goals are the experiences of the identified stakeholders. For each
stakeholder, a set of questions is used to assess the strengths and shortcomings from the
stakeholder's perspective. For each question, a set of alternative assessment approaches is
proposed.

\autoref{fig:overview} provides an overview of the assessment method:

\begin{figure}[ht!]
  \centering
  \begin{tabular}{|c|c|c|}
    \hline
    \multicolumn{1}{|p{0.2\columnwidth}|}{\centering\tabhead{Stakeholder (Goal)}} &
    \multicolumn{1}{|p{0.35\columnwidth}|}{\centering\tabhead{Assessment question}} &
    \multicolumn{1}{|p{0.35\columnwidth}|}{\centering\tabhead{Assessment approaches}} \\
    \hline
    \multicolumn{1}{|p{0.2\columnwidth}|}{Players} &
    \multicolumn{1}{|p{0.35\columnwidth}|}{To what extent does the system affect players?
        To what extent does the system engage players?} &
    \multicolumn{1}{|p{0.35\columnwidth}|}{experimental study, interviews,
                engagement metrics} \\
    \hline
    \multicolumn{1}{|p{0.2\columnwidth}|}{System admins} &
    \multicolumn{1}{|p{0.35\columnwidth}|}{How easy is it to install and maintain the system?} &
    \multicolumn{1}{|p{0.35\columnwidth}|}{experimental study, interviews} \\
    \hline
    \multicolumn{1}{|p{0.2\columnwidth}|}{Game designer} &
    \multicolumn{1}{|p{0.35\columnwidth}|}{How easy is it to design a game?} &
    \multicolumn{1}{|p{0.35\columnwidth}|}{experimental study, system logs, interviews } \\
    \hline
    \multicolumn{1}{|p{0.2\columnwidth}|}{Game managers} &
    \multicolumn{1}{|p{0.35\columnwidth}|}{How easy is it to manage a game?} &
    \multicolumn{1}{|p{0.35\columnwidth}|}{experimental study, system logs, interviews} \\
    \hline
    \multicolumn{1}{|p{0.2\columnwidth}|}{Developers} &
    \multicolumn{1}{|p{0.35\columnwidth}|}{How easy is it to enhancing the system?} &
    \multicolumn{1}{|p{0.35\columnwidth}|}{experimental study interviews} \\
    \hline
  \end{tabular}
  \caption{Overview of SGSEAM}
  \label{fig:overview}
\end{figure}

There are usually multiple assessment approaches for a specific question. Different assessment
approaches will have different levels of rigor. In experimental design terms, rigor refers to
external and internal validity. The details of the individual assessment approach for each
stakeholder are descried in the following sections. The assessment approaches for a question
can be additive. The more approaches applied, the higher confidence of the assessment.

\subsection{Player Assessment}

The goal of player assessment is to determine the effectiveness of the game
framework from player's perspective. It is essential that a game produced by a serious game
framework could achieve its intended "serious" purpose. The intended purposes of serious games are
always subject specific. For example, the desired effect of a serious game for
energy education and conservation is to increases players' energy literacy and
reduces their energy consumption during (and, hopefully, after) the game. A serious game for
language learning would have a very different desired effect.

Users of SGSEAM could use domain-specific questions to assess the desired effects of their
serious game. For illustration purpose, the following two questions are used to assess a serious
game for sustainability: (a) To what extent does the game increase player's literacy in
sustainability? (b) To what extent does the game produce positive player behavior change in
sustainability?

One approach to assess the question of the effect of literacy changes is a quasi-experimental
study. A set of literacy survey questionnaires are presented to a random selection of the players
before the game (pre-test). After the game ends, the same survey (post-test) is presented to the
players who responded the pre-test survey. These two set of survey response data are compared to
understand if the game has had an impact on literacy. The extent of players' sustainability
literacy change will indicate the degree of educational effectiveness of the serious game for
sustainability.

A pre-experimental study could be used to assess the question of the effect of
sustainability behavior changes. The resource (energy and/or water) consumption data during and
after the game are recorded (post-test). They are compared to the resource consumption baseline
established before the game (pre-test).

Another approach for effectiveness assessment is to interview players about their self-reported
behavior change. The combination of resource consumption changes and self-reported behavior changes
can be combined to assess the degree of behavior effectiveness of the serious game for
sustainability.

In addition to the domain-specific goals of serious games, SGSEAM assesses a common
aspect of serious games, player engagement, to address the question of "To what extent does the
game engage players?"

Player engagement is an important measure for understanding the effectiveness of a serious game.
By investigating the degree of engagement, we can determine to what extent individuals are
participating in the game, as well as to what extent the community population is participating in
the game.

Engagement has a subtle relationship to the overall effectiveness of a serious game. It is
possible for the game to be played by only a subset of the target population, but
have an impact on those not playing by virtue of their contacts with players. Gaining
better insight into this effect is an area of active study for us. 

To obtain engagement data, SGSEAM analyzes the following measures
based upon system log data provided by the framework.

\begin{itemize}
\item participation rate
\item number of players per day
\item play time of a player per day
\item submissions of all player per day
\item social interaction of all player per day
\item website errors per day
\end{itemize}

The participation rate measures the percentage of users who used the game based on the total
eligible players. In the serious game context, it indicates the level of involvement or awareness
of the serious matters. The number of players and play time per day measure how frequently the
players interact with the game. The submissions per day measures the rate of serious game
specific activities (online or real world) that players completed, while the social interaction
per day measures the rate of social interactions happened in the game between the players. At
last, the website errors per day measures the rate of errors encountered by the players while
using the game website. In general, with the opposite of website error measurement, the higher
value these measurements are, the higher engagement level the game has.

\subsection{System Admin Assessment}

System administrators are responsible for installing and maintaining the software infrastructure
for the game. Their tasks include the framework and dependency installation, maintain the database, backups, and so forth.

One approach to assess the question of how easy it is to install and maintain the system is to use
an experimental study. A group of system admins is asked to install the system, record the time
spent and problem encountered as they complete each step. The qualitative data (i.e., the
descriptive problems reported by the participants of the study) will need to be categorized and
coded. The assessor will triangulate the reported time data and the problem categories to identify
the area of strength (less time spent) and weakness (problems and difficulties).

Another approach is a post-hoc interview. The system admin(s) are asked about their experience
after the installation. The interview includes the following questions:

\begin{itemize}
\item How much time did you require to install the system and the dependencies?
\item How much time did you require to maintain the system?
\item What problems did you encounter?
\item Did you find it difficult to admin the system? What was difficult?
\end{itemize}

After the interview data is acquired, the assessor will perform qualitative data
analysis, which involves transcribing (if the interview data is in audio format),
categorizing and coding the description of reported problems or difficulties.

The level of confidence of the above two assessment approaches varies. The experimental study
approach is more rigor because of the generality achieved from the larger population of
participants under study. The data collected during the step by step experimental study is more
accurate than the one collected in the post-hoc interview.

\subsection{Game Designer Assessment}

A game designer uses the serious game framework to design and create a serious game.
A serious game framework always provides certain tools or interfaces to game designers
with the hope that these will simplify the design of a game. Such tools might involve
configuring global settings for the game, such as how long will the game run, who are the
players, and how to design individual game elements.

SGSEAM assesses the game designer stakeholder by addressing the following two questions: (a) How
much time is required to design an instance of a serious game using the framework? and (b) How
many, and how problematic are the errors that designers encounter during the design process?

There are also three approaches for game designer assessment. One is a experimental study, where a
goup of participants is asked to use the system to perform a same set of design tasks. The time
spent and problems encountered are recorded for each tasks. The assessor will triangulate the
reported time data and the problem categories to identify the strengths and weaknesses.

A second approach is to interview the designer(s) after they had completed the design.
The following questions will be asked:
\begin{itemize}
\item How much time did you spend to complete each design task?
\item What problems did you encounter?
\item Did you find it difficult to configure? What was difficult?
\item Did you find it difficult to design a specific game? Which one, and what was difficult?
\end{itemize}

The interview data will be transcribed (if audio recording), categorized and coded to identify the
strengths and weaknesses.

A third approach is to collect the system log data related to the game designing tasks. When
available, the time spent and error encountered can be queried from the system logs. Although these
system generated data might be easier to gather in some systems, it might not provide the same
depths or insights than the other two approaches where the experiences are provided by the
participants directly. On the other hand, these system data can be supplemental to the other
approaches. They could be correlated with the data gathered from the other assessment approaches
 to increase the confident of the assessment.

\subsection{Game Manager Assessment}

A game manager uses the serious game framework to manage the serious game that the game
designers created. It is possible that a game manager is also the game designer.
Serious game frameworks normally provide certain interfaces for the managers to manage the
game. This may involve managing player submissions, monitoring the game state, entering
manual resource data, notifying winners of the game, etc.

SGSEAM assesses the game manager stakeholder with the following questions: (a) How much time is
required to manage an instance of a serious game using the framework? and (b) How many,
and how problematic are the errors that managers encounter during the design process?

Similar to the assessment of game designer experience, SGSEAM proposes three approaches. The
experimental study approach gather data from a group of participants about the time spent and
problems encountered for each task of managing the serious game. The post-hoc interview approach
gather data from the game manger(s) by asking the following questions:

\begin{itemize}
\item How much time did you spend to complete each managing task?
\item What problems did you encounter?
\item Did you find it difficult to manage? What was difficult?
\end{itemize}

\subsection{Developer Assessment}

The developer stakeholder is different from the game designer stakeholder, in that the
game designer stakeholder tailors the framework without requiring any software
development, while the Developer stakeholder enhances, corrects, and extends the system by
manipulating code. 

To investigate how easy it is to understand, extend, and debug a serious game
framework from a developer's perspective, SGSEAM assesses how much time it takes to develop an
enhancement to the game framework, and how many errors are encountered
during the process.

The experimental study assessment approach asks a group of developers to develop a same set of
enhancements to the system, and ask them to record the time spent to develop and problems
encountered.

A second assessment approach is accomplished by interviewing the developer(s) to
answer the following questions:

\begin{itemize}
\item How much time did you spend developing and debugging an
  enhancement to the game framework?
\item What problem(s) did you encounter?
\item Did you find it difficult to understand, extend and debug the
  system? What was difficult?
\end{itemize}

Similarly, the descriptive data will be categorized and coded. The time data will be correlated to the problem data to identify the areas of strength and weakness.
