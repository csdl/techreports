\chapter{Experimental Design}
\label{cha:ExperimentalDesign}

This chapter describes the design of the experiment using the competition and associated website described in \autoref{cha:system-description}. First we cover the different sources of data available for the experiment, followed by analyses performed on the data. The research questions we propose to investigate are:

\begin{itemize}
	\item \emph{To what extent was the website adopted by the participants?} Without significant adoption, it is hard to evaluate the other website-related questions.
	\item \emph{How did energy use change during the competition?} This is the standard measure for an energy competition, with the expected result being energy conservation during the competition.
	\item \emph{How did energy use change after the competition?} Understanding changes in energy use after the competition is over gives insight into whether changes during the competition were sustainable. Existing research focuses primarily on the competition itself, not examining the reasons why energy usage might rebound after the competition is over.
	%	\item \emph{How can energy literacy be assessed?} We are hypothesizing that energy literacy leads to sustainable behavior change, so assessing energy literacy is important.
	\item \emph{How effective were the tasks available via the website?} By using website log data, we can track what tasks participants undertook, and compare that to changes in their energy literacy.
	\item \emph{How appropriate were the Kukui Nut values assigned to tasks?} The Kukui Nut points assigned to tasks are intended to motivate participants to perform the tasks, but the values were assigned without any participant data.
	\item \emph{What is the relationship between energy literacy and energy usage?} We hypothesize that more energy literate participants will conserve more energy, so we examine the relationship both during the competition and afterwards.
	\item \emph{How important was floor-level near-realtime feedback?} There are good reasons to believe that floor-level near-realtime feedback will lead to increased energy conservation, but they greatly increase the competition budget and logistical complexity. Is the trade-off worth it?
\end{itemize}

\section{Administration}

As the website is intended to be the hub for competition activity, it provides an contest administration interface where participants can be added, tasks created (see \autoref{sec:competition-tasks}, and tasks verified (see \autoref{sec:task-verification}). The contest administration interface is separate from lower-level administrative tasks such as the layout of the website, database table maintenance, etc. The contest administration will be performed by the researchers, and potentially trained volunteers `deputized' to verify task completion, depending on the actual workload.

\section{External cooperation}
\label{sec:external-cooperation}

Unlike some ICS research, this project requires extensive cooperation with entities outside of the ICS department. Running the competition in student housing requires the enthusiastic cooperation of Student Housing Services, since the participants live in student housing and the meters need to be installed in the residence halls. We have met with Michael Kaptik, the director of Student Housing Services, and he appears eager to facilitate the competition (and of course Student Housing Services would benefit from any reductions in electricity use by residents). Installation of the meters themselves needs to be coordinated with Facilities, which handles electrical work on campus. We have met with David Hafner, Assistant Vice Chancellor for Campus Services who heads Facilities, and he is also very supportive of the competition plan and has indicated his willingness to facilitate the installation of the floor meters.

While the initial discussions with Student Housing and Facilities have all been positive, situations and personnel can change over time. There remains the risk that one of these entities might be unable or unwilling to cooperate, preventing the competition from taking place as planned.

If Student Housing were not supportive of holding the competition in the residence halls, it might be possible to switch to a competition between floors of some multi-story building on campus. However, this would significantly change the character of the research, and would require extensive redesign of both the competition and the website.

If Facilities was unwilling or unable to allow the installation of the floor meters in the residence halls, the research as planned could not take place. It might be possible to design an experiment that revolved solely around evaluating the effectiveness of increasing energy literacy using a redesigned website, but it would lack the critical component of evaluating the relationship between energy literacy and energy usage.

\section{Participant engagement}
\label{sec:participant-engagement}

The installation of the meters to record floor-level electricity usage is the enabling component for the energy competition between floors and residence halls. However, energy literacy and near-realtime energy feedback rely on the competition website, and particularly on participant use of the website. The vast majority of entering freshmen own computers (Michael Kaptik stated that based on past surveys student housing resident computer ownership was something like 98\%) and have used the Internet extensively. Thus there is little risk that the potential participants will not be able to use the website, but there is considerable risk that they will not bother to use the website due to lack of interest or conflicting demands on their time and attention.

We will attempt to limit this risk in several ways. 

\section{A-B testings}
\label{ab-testing}

The billboard is a non-interactive mode for the website designed to convey competition information to participants in the lobby of the residence halls in an ambient fashion. It is also expected that the billboards will remind the residents about the competition, making them more likely to participate. The billboard consists of a series of pages, which are cycled through after an appropriate delay (approximately 20 seconds). Billboard pages will display competition standings (individual, floor, and building), upcoming events, tasks recently performed by participants (in the style of the Facebook newsfeed), prize descriptions, and energy conservation tips. \autoref{fig:website-billboard} shows a mockup of the billboard display.

large computer displays in the lobby of each residence hall that loop through interesting competition information, including current competition standings, upcoming events, and recent tasks performed by participants. We expect the billboard displays to provide a continuous reminder to residents about the competition and how they might participate. Third, we plan to have posters on each floor of the residence hall to remind residents about the competition and the website. Fourth, participants will be notified about competition events via email and Facebook, with embedded links back to the website. Finally, the website makes it as easy as possible for participants to use the website by utilizing the University of \Hawaii single-sign-on system, allowing participants to log on with their UH username and password, rather than a username and password specific to the website.

\section{Real-time Game Analytics}

There are two metrics for the competition: energy consumption (EC) score and Kukui Nut (KN) score. Energy consumption is the total amount of electrical energy consumed by a floor in kWh during a round as measured by the power meters. The energy consumption is normalized by subtracting the minimum floor power multiplied by the time interval in question (see \autoref{sec:min-floor-power}). Therefore, floors are ranked in increasing order of energy consumption, with the floor with the lowest energy consumption being the winner. The floor-level energy consumption score can be aggregated spatially to obtain a score for an entire building, and also temporally to obtain a score for the entire competition across all rounds.

The parallel metric for the competition is Kukui Nut points. Kukui Nut points are assigned to individual participants for performing certain tasks via the competition website. The verification of task completion and recording of the Kukui Nut points are done entirely through the website, see \autoref{sec:website-design} for more details. Kukui Nut points can be aggregated spatially to obtain a point total for an entire floor or for an entire building, and also temporally to obtain a score for the entire competition across all rounds.