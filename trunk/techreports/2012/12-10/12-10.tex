%%%%%%%%%%%%%%%%%%%%%%%%%%%%%% -*- Mode: Latex -*- %%%%%%%%%%%%%%%%%%%%%%%%%%%%
%% 12-17.tex --  Tech report on RAs
%% Author          : Philip Johnson
%% Created On      : Mon Sep 23 11:52:28 2002
%% Last Modified By: Philip Johnson
%% Last Modified On: Mon Jun 14 12:41:23 2010
%%%%%%%%%%%%%%%%%%%%%%%%%%%%%%%%%%%%%%%%%%%%%%%%%%%%%%%%%%%%%%%%%%%%%%%%%%%%%%%
%%   Copyright (C) 2009 Philip Johnson
%%%%%%%%%%%%%%%%%%%%%%%%%%%%%%%%%%%%%%%%%%%%%%%%%%%%%%%%%%%%%%%%%%%%%%%%%%%%%%%
%% 

\documentclass[]{article}
\usepackage[final]{graphicx}
\usepackage{cite}
\usepackage{url}
\usepackage{enumitem}
\usepackage{times}
\usepackage[margin=1in]{geometry}

% uncomment the % away on next line to produce the final camera-ready version
% and uncomment the \thispagestyle{empty} following \maketitle
%\pagestyle{empty}
\begin{document}

%\onecolumn
%\setlength{\parindent}{0cm}


\title{{\bf The Role of the Resident Assistants: Lessons Learned from Two Kukui Cups}}

\author{Philip M. Johnson\\
        Collaborative Software Development Laboratory\\
        Department of Information and Computer Sciences\\
        University of Hawai`i at M\=anoa\\
        Honolulu, HI 96822\\
        johnson@hawaii.edu\\
}


\maketitle

\begin{abstract}  % 150 words
The Kukui Cup is an advanced energy challenge whose goal is to
investigate the relationships among energy literacy, sustained energy
conservation, and information technology support of behavior change. Kukui Cup
challenges have been held in the University of Hawaii Hale Aloha residence halls in both
2011 and 2012.  Resident participation in the Kukui Cup is crucial to its success, and
there is much evidence to suggest that the Resident Assistants can play an important role in
obtaining resident engagement. This report analyzes the role and impact of the Resident
Assistants in the Kukui Cup and presents insights for the 2013 Kukui Cup.
\end{abstract}

\tableofcontents
\newpage
\thispagestyle{empty}


\setlength{\parskip}{3pt plus 1pt minus 1pt} 

\section{Introduction}

Resident assistants (RAs) have many responsibilities, including administration, community
building, helping residents, creating a safe and healthy environment, promoting diversity,
assisting the transition to campus living, and supporting academic and personal growth.
Over the past two years, Resident Assistants at the University of Hawaii Hale Aloha
residence halls have supported the Kukui Cup Challenge.  In this report, I summarize our
experiences so far with the two very different approaches we took with Resident Assistants in
2011 and 2012, and what that might imply for the 2013 Kukui Cup.

\section{The 2011 Kukui Cup}

\subsection{2011 Preparations}

The inaugural Kukui Cup was three weeks long, starting in mid-October of 2011 and lasting
until the first week of November.  In preparation for the initial Kukui Cup, the following
actions with respect to RAs were taken:

\begin{itemize}[nolistsep]
\item A powerpoint presentation on the goals of the Kukui Cup were presented to the RAs a
  week or so before the start of the competition;
\item The RAs were informed that they were not eligible to participate in the prizes, but
  that ``special'' games would be provided for them.
\item The RAs were told that their participation was voluntary. 
\item A ``Game Plan'' document was provided with a complete list of all Kukui Cup activities.
\end{itemize}

\subsection{2011 Resident participation}

From an individual perspective, the 2011 Kukui Cup engaged about 418 students, although almost 80 of them were signed up in the
last days of the competition by the leading contenders for the grand prize (signing up new
players earns a player a ``referral bonus'').  So, approximately a third of the Hale Aloha
residents were engaged by the 2011 Kukui Cup.  

From a lounge perspective, there was only one lounge that obtained over 50\%
participation, and this lounge won both the prize for reducing energy the most and the
prize for collectively earning the most points.   This lounge had two RAs
who were extremely active in promoting the Kukui Cup. 

\subsection{2011 Questionnaire results}

After the competition ended, we met with the RAs to obtain their insight into how we could
improve involvement in the Kukui Cup.   The questionnaire and responses are provided in
Figures \ref{fig:ra-participation-2011} and \ref{fig:ra-suggestions-2011}.

\begin{figure*}[t]
\begin{tabular}{|l|l|}
\hline

\small  \begin{minipage}[t]{3in}
{\em Question  (1a):  If you participated (i.e. logged in and accumulated at least 50 points), then please answer the 
following:  What were the top things that motivated you to participate?}
\begin{enumerate}[noitemsep,nolistsep]
\item Energy consciousness.  I ``heart" Kukui Cup!
\item I participated because I felt it was for a good cause.
\item Top things that motivated me was the participation and getting a gift card.  When you add money to the equation you add motivation.
\item Lounge-level prizes
\item Prizes and field trips.
\item Participation reward.
\item Money and great prizes.
\item Prizes and statistical information was motivating.
\item Another thing in common with my residents, being able to do activities with them.
\item Importance of energy conservation. 
\item Get points
\item Show lounge/floor pride.
\item Prizes
\item A handful of my residents were interested and participating early on. 
\item Competition.. I wanted to be in the top... I can't help it!  Ha Ha!
\item Getting people excited to participate.  
\item Prizes
\item Being able to explain the competition to my residents and/or just to be able to know what was happening. 
\item My kids wanting to win. :)
\end{enumerate}
\end{minipage}  \normalsize

&

\small  \begin{minipage}[t]{3.25in}
{\em Question (1b): If you did not participate (i.e. never logged in or just once or twice), then please answer the 
following:  Why did you not participate, and what would be the most important change we could make next year to 
motivate RAs such as yourself to participate? }
\begin{enumerate}[noitemsep,nolistsep]
\item Was just too busy.  Give the RAs incentives before the competition starts.
\item Prizes that we were not qualified for.
\item I participated some, but never logged points.  I was just really busy at the time.  I really liked the Cup, 
though.
\item I didn't log in but I took steps to lower my energy consumption.   I think we all believed in the mission, 
perhaps we could all register together in a staff meeting or something.24. I was just so busy!   And I'm not very motivated by prizes. 
\item I went to events but did not redeem points.  I am not into earning things. 
\item I kept forgetting to log on but I participated by saving energy.
\item I was super motivated to participate but I was waaaaay caught up in school and work and homework and 
just had a lot going on.   It had nothing to do with how the Kukui Cup was run, Kukui Cup was awesome!
\item It was difficult to understand how to achieve points. 
\item I'm not sure. 
\item Pure laziness on my part.  It was a great idea and I was really interested in the event, I just got caught up 
in school work and other responsibilities.
\item I think the news that RAs could participate for prizes fell through the cracks. 
\item Too many midterms.  
\item Did not participate because school is more important.  
\item More prizes, not just gift cards.
\end{enumerate}
\end{minipage}  \normalsize
\\
\hline
\end{tabular}
\caption{2011 Questionnaire: Issues surrounding RA participation}
\label{fig:ra-participation-2011}
\end{figure*}

%%%%%%%%%%%%%%%%%%%%%%%%%%%%%%%%%%%%%%%%%%%%%%%%%%%%%%%%%%%%%%%%%%%%%%%%%%%%%%%%%%%%%%%%%%%%%%%%%%%%%%
\begin{figure*}[th!]
\begin{tabular}{|l|l|}
\hline

\small  \begin{minipage}[t]{3.75in}
{\em Question (2): This year, 400 students participated, but that means 600 students did not.  What do you think are the 
most important, realistic changes we could make to double participation for next year? }
\begin{enumerate}[noitemsep,nolistsep]
\item More advertisements {\em (repeated 9 times)}
\item Turn off all lights in building for 30 secs at night to get everyone's attention
\item Better timing for events
\item More incentives for RAs to get participation up.  Separate prize.
\item Well, RAs play a big role but the RAs also don't have motivation to join because there isn't equal opportunity. 
\item It's not part of the job so it's extra work that adds to our already busy schedule.
\item More RA incentive to get involved so residents get involved.   Maybe give the RAs some resources or have a Kukui Cup hangout place. 
\item Participation prizes.
\item ADVERTISE.  Make the event so obviously in everyone's face that they can't ignore
  it. Flyers should be  provided to the RAs to post on their floor.
\item Get the RAs motivated to encourage their residents to participate.
\item More RA incentives. 
\item Timing of events at late hours on the weekdays.
\item I would not know but I know most of my residents weren't home most days especially weekends.  So they did not participate.
\item Have more open, public, big events.
\item Too many things going on in the Kukui Cup.   Too confusing to the residents. 
\item Get more students informed, perhaps start the competition earlier.
\item Advertise it more before the challenge starts so that residents know what the challenge is about and can participate.
\item Proper presentation to all the students on how to achieve points. 
\item Find a way to get the word out faster.  Maybe ask RAs to go door-to-door, that's what I did.   Start the 
participation and incentive program for RAs in the first round.
\item Saying RAs can get programming credits will get them going, and reminders about it. 
\item Stating RAs can win the normal prizes will help.
\item Have a signup day in the courtyard with computers.
\item Get an email when you are outbid on an item. 
\item More availability for excursions.
\item More flyers, like weekly calendars for floors. 
\item Cooler prizes, more advertising, better techniques.  Better incentives for the residents, school is important, make them see why energy saving is cool.
\item Maybe motivate by providing programming credits as well. 
\item Get supervisors to exempt us from programming for the month to get more RA participation.
\item Why not give the RAs equal opportunity?  I know it's so it would be like it's not rigged buy we have nothing do with it and we have to try just as hard to get points.
\item Do it in the same way, but focus on a serious ad campaign leading up to and during the event.
\end{enumerate}
\end{minipage}  \normalsize

&

\small  \begin{minipage}[t]{2.75in}
{\em Is there anything else you feel we should know that can help us next year?  }
\begin{enumerate}[noitemsep,nolistsep]
\item Stay awesome
\item Students don't like to go out of their way.  But you did everything out of the books
  for success. So I don't know why it wasn't more successful.
\item Make the competition longer. 
\item It needs to be broadcasted more and be more competitive to get involvement. 
\item Cash prizes = more participation.
\item Consider targeting other residence halls. 
\item Free stuff gets people to participate.
\item Continue to do it!
\item Emphasize the beneficial points to the surroundings, the prizes, saving the environment, etc. 
\item Some residents don't really care for energy competition so it takes more to get them involved.   I think more 
athletic activities would help.
\item Should apply to more than just the freshman towers.  I know a lot of people in Frear and in the apartments 
\item wanted to be able to participate.
\item Keep doing this. My residents  loved it and I think it's a great asset to residential life and the broader community.
\item The RAs don't have to put in much effort, so there is no excuse.  The work to help promote should be up to us!  
\item Thank you so much for this experience!
\item Variety of prizes was good.  Prizes that people want to win will keep people motivated.
\item Programs were fun, keep up the good work.
\item Somehow integrate RA-resident activities to encourage more participation.
\item Do it in Frear.  Less resident apathy.
\item I'm not sure, you guys did so good this year.  Maybe do it in Frear?
\end{enumerate}
\end{minipage}  \normalsize
\\
\hline
\end{tabular}
\caption{2011 Questionnaire: Suggestions for change}
\label{fig:ra-suggestions-2011}
\end{figure*}

\subsection{2011 Questionnaire analysis}

Analysis of the comments revealed several common trends.  First, there appeared to be three major motivators for residents who participated:

\begin{enumerate}[noitemsep]
\item Prizes (mentioned 8 times).
\item Resident relationships (mentioned 6 times)
\item The "cause" of energy conservation (mentioned 3 times)
\end{enumerate}

Second, there appeared to be three primary obstacles identified by RAs who did not participate:

\begin{enumerate}[noitemsep]
\item Too busy/too much schoolwork. (mentioned 6 times)
\item Problems with prizes (mentioned 5 times)
\item Participated, but did not log in (mentioned 4 times)
\end{enumerate}

Finally, the following were the most popular free-response comments:

\begin{enumerate}[noitemsep]
\item Improve advertising (mentioned 14 times)
\item Provide programming credits/resources for RAs (mentioned 8 times)
\item Do not exclude RAs from regular prizes (mentioned 4 times)
\end{enumerate}

\section{The 2012 Kukui Cup}

\subsection{2011 Preparations}

During the planning of 2012, we made several significant changes to the Kukui Cup with
respect to the role of the RAs:

\begin{itemize}[noitemsep]
\item We made RAs ``regular players'' and thus eligible for all prizes. 

\item We created a second RA incentive game, in which they could obtain a
bookstore gift card if a target attendance level was reached for their events.  

\item Kukui Cup involvement was formally made a part of the RA's job responsibilities. 

\item We ramped up advertising, including: (a) promotional videos about the Kukui Cup and
micro-documentaries of the events; (b) an actively updated Facebook page, and (c) signs
posted outside all four Tower elevators with upcoming Kukui Cup events.  We posted 10 of
these ``Elevator Updates'' during the 27 days of the competition.  Beyond our own efforts
to improve advertising, we also tasked the RAs with getting the word out in their own
way. 
\end{itemize}

Shortly before the start of the 2012 Kukui Cup, we had a meeting with the RAs in which we presented the following:
\begin{itemize}[noitemsep]
\item The goals and motivation for the Kukui Cup through a powerpoint presentation.
\item The two RA incentive games they could play. One would earn them bookstore
  certificates for achieving a 50\%, 75\%, or 100\% participation from their lounge. The
  other would earn them bookstore certificates for achieving a target level of attendance
  at events. 
\item That Kukui Cup participation was mandatory, not optional.   This included planning and execution of two
  Kukui Cup events, which would count toward their program responsibilities for September.
\item We distributed the ``Game Plan'' document providing times, dates, and descriptions
  for all events during the 2012 Kukui Cup. 
\end{itemize}

In mid-August, we also distributed a spreadsheet in which RAs could sign up for events.  We initially
proposed only 2-3 RAs per event, but the Resident Directors requested an increase to 4 RAs per event. 

On September 1, two days before the competition began, I created and distributed to the
RAs a five minute YouTube video that provided a tutorial to using the Kukui Cup website
specifically geared to their needs.

During the competition, we scheduled a 30 minute meeting with the RAs associated with each
event during the week prior to the event's occurrance.  During this meeting, we went over
the event with the RAs so they could market it effectively to their students. 

\subsection{2012 In-game RA Feedback}

As noted above, RA participation in the Kukui Cup was defined as a mandatory part of their job
responsibilities in 2012.  After the first week of the challenge, we checked RA
participation and discovered that:
\begin{itemize}[noitemsep]
\item Out of the 40+ RAs, less than a dozen of them had watched the RA-specific, five
  minute YouTube video on the Kukui Cup game.
\item 23 out of the 40+ RAs had either never logged in to the game, or had logged in only
  once.
\end{itemize}

In hopes of catalyzing involvement by the RAs, we sent them an email the following week.
Although the email contained the actual RA names, they are excised in the following version:

\begin{quotation}
{\em 
\noindent Greetings, all,

\noindent As a followup to my prior email on Lounge participation, here are some
figures on RA participation:
\begin{itemize}[noitemsep]
\item 300+ points: (8 RAs, names deleted)
\item 100-299 points: (11 RAs, names deleted)
\item 25-70 points (i.e. approximately 1 login): (11 RAs, names deleted)
\item 0 points (never logged in): (12 RAs, names deleted)
\end{itemize}

I believe strongly that you cannot be an effective advocate for the
Kukui Cup if you do not participate in the game.  Students on your
floor are able to tell if you are participating or not, and it can be
disincentivizing to see that the RA doesn't care enough to play, even
worse if they are telling students to play but not doing it
themselves!

As a reminder, I've created a 6 minute video that you can watch to
learn more about how to use the system and support participation:
http://www.youtube.com/watch?v=GQsy0XB0ySw

I would like to request that you spend a few minutes at your meeting
tonight discussing Week 1.  Here are some possible discussion
questions:
\begin{enumerate}[noitemsep]
\item From those who are playing, what motivated you to play?
\item From those who aren't playing, why haven't you started?
\item What changes can we make to help promote additional participation
(by both residents and RAs in the next week)?
\end{enumerate}

It would be great if someone could send me a brief email with the
results of this discussion that I can use to help improve the
challenge.

Thanks very much to everyone who is participating and supporting the
Kukui Cup so far.

Note that it is still entirely possible for the last place lounge to
catch up to the first place lounge this week.   Starting next week,
there will be new levels unlocked, and it will be harder.   This is
your chance!
}
\end{quotation}

Our request for a discussion on items (1) - (3) was not implemented.  We did receive
feedback about the above email from
a Resident Director, which stated in part:

\begin{quotation}
{\em 
I've received some feedback this morning from RAs that they are
uncomfortable with how they have received the feedback about their participation. For
student staff it is important that certain feedback is given in a one-on-one conversation
or private email with a supervisor instead of being shared through emails to all 40+ staff
participating. [....] I don't want the format of
feedback to discourage or embarrass RAs as that could result in staff losing motivation to
participate or having negative feelings about the Kukui Cup (which none of us want). 
}
\end{quotation}

In retrospect, this reveals one of several misassumptions we had about the role of RAs in
the Kukui Cup.  We assumed that, unlike in 2011, there would be little need to ``motivate'' RAs
because it was now part of their job description.  After having given them a presentation
on the challenge, a video to watch, and a week to get engaged, we assumed that it was
perfectly reasonable to provide game-like feedback in the form of a scoreboard (given
that, as noted in the email, the participation or non-participation of the RAs was public
knowledge to their residents within the game).  What the reply from the Resident Director
made us aware of is that, operationally speaking, there is much less difference than we
had assumed between ``mandatory'' and ``voluntary'' RA involvement in the
Kukui Cup.  We will return to this issue in Section \ref{kc-2013}.

As a result of the Resident Director feedback, we sent the following email to the RAs:

\begin{quotation}
{\em
\noindent Greetings, Hale Aloha RAs,

\noindent Holly and Isaac informed me that some of you were taken aback by my
recent email, as you have been diligently promoting the Kukui Cup to
your students even though you have not been playing the game yourself.
 I certainly do not want to demoralize or disincentivize any of you,
and we in the Kukui Cup project sincerely appreciate all your efforts.
 I apologize for not having communicated clearly enough why your
efforts to promote the Kukui Cup need to be combined with at least
nominal game play on your part.  Here is some explanation:

\begin{enumerate}[noitemsep]
\item Providing specifics of the game to students.

It is one thing to say, ``The Kukui Cup is cool, you should do it."  It
is more effective to say, ``Oh, you're an English major? Did you know
you can write a poem and get points in the Kukui Cup?"  Or, ``Hey,
you're in Engineering?   You can check out an energy meter from the
CDCs and measure your refrigerator power and energy for the Kukui
Cup.".  In other words, by playing the game, you will learn about the
actual content, and can provide specific details of the game to
attract your students, which should increase the success of your
efforts.

\item ``Do as I say, not as I do" is rarely effective.

Actions speak louder than words.    By playing the game yourself, you
are showing your students that you are willing to do what you are
asking them to do, which can only help recruitment.

\item Lack of participation has a negative impact on those in your lounge who do participate.

Several features of the game (Participation Levels, Daily Energy Goal
Game, Lounge-level Top Score Game) are designed to directly reward
individuals based upon the group's involvement. By not playing the
game, you make it harder for those in your lounge to compete with
respect to these features.

\item It does not take a lot of time to play the game.

We designed the Kukui Cup activities so that the vast majority take
under 10 minutes to complete. You can play once a day for 10 minutes,
say an hour a week, and be a significant positive force for your
lounge.  We're not asking you to make the Kukui Cup a part-time job.

\item You might actually find it interesting.

We get a lot of comments from players that they find the three minute
videos and the topic of energy in Hawaii to be much more interesting
than they expected.
\end{enumerate}

}
\end{quotation}

The following table contrasts RA participation in the online game after the first week of
the competition with participation in mid-October.  

\medskip

\begin{tabular}{|p{1.5in}|p{1.5in}|p{1.5in}|}
\hline
{\bf Points} & {\bf Number of RAs \newline (as of Sept 10)} & {\bf Number of RAs \newline (as of Oct 18)} \\ \hline
300+ & 8 & 17 \\
71-299 & 11 & 11 \\
25-70 (1 login) & 11 & 7 \\
0 (never logged in) & 12 & 8 \\
\hline
\end{tabular}

The most important change is that the number of non-participating RAs (i.e. those who
either never logged in or logged in only once) has dropped from 23 to 15.  This still
seems like an unacceptably high number for a ``mandatory'' component of the RA job
description. It is not known what, if any, impact the preceeding emails had on this
change.


\subsection{2012 Resident participation}

Preliminary analysis of the data from September, 2012 indicates that the 2012 Kukui Cup
achieved almost exactly the same results as 2011.   Approximately 320 students
participated in the game, which is almost identical to the number of players from 2011
(not taking into account the last day signups).   Once again, only one lounge achieved
over 50\% participation, and just like last year, that one lounge won both the prize for
reducing energy the most and the prize for collectively earning the most points.  Like
last year, this lounge had two RAs who were extremely active in promoting the Kukui Cup.

We find it quite interesting that despite significant changes to RA participation and
significantly more advertising, we were unable to significantly increase student
participation. 

There is one indicator we are already aware of that has changed significantly and
positively in 2012.  During 2011, approximately 3,000 online activities were accomplished
by players. In 2012, that number doubled, to over 6,000 activities. In other words, 
while the total number of students playing the game did not change, those
who played the game were substantially more active in the game than last year.

\subsection{2012 Questionnaire results}

After the 2012 competition ended, we met with the RAs to obtain their insight into how we could
improve involvement in the Kukui Cup.   The process this year was more simple than last year: we handed
out index cards and asked them to write down one thing they thought went well this year
and one thing they thought should be improved.   Figure \ref{fig:2012-questionnaire-results} shows
the results. 


\begin{figure*}[th!]
\begin{tabular}{|l|l|}
\hline

\small  \begin{minipage}[t]{2.6in}
{\em What was one good thing about the Kukui Cup?}
\begin{enumerate}[noitemsep,nolistsep]
\item The structure and organization of the website and events.
\item I like the pace of the events.
\item The incentives that are given out at the events. 
\item Workshops and prizes are keeping the residents engaged.
\item Prizes, swag, energy meter
\item It's good. I believe that the residents that want to participate will.
\item Swag.
\item The prizes are a good motivator. 
\item The workshops are fun, good music and turnout. Prizes and incentives.
\item 	Lounge parties, slam poetry event.
\item 	Great incentives, great programs.
\item 	One thing that works well is the incentives, such as the t-shirts and cups.
\item 	Chillaxation and Scavenger Hunt. (on two cards)
\item 	Many events for people to go to, different times and days.
\item 	Chillaxation and slam poetry.
\item 	Bringing in guests to speak and give different points of view; going to different sites like farms and local places that support green living.
\item 	Giving away free items builds a crowd for the events. 
\item 	Types of events hosted are fun and interesting for students.
\item 	Great prizes; nice effort in getting people involved.
\item 	Prizes and incentives.
\item 	Lounge party prizes; poetry event.
\item 	Great incentives to initiate participation. (on two cards)
\item 	Prizes; food; facebook updates.
\item 	The prizes are cool
\item 	Prizes and activities
\item 	All of the Kukui events compacted into two months; online stuff is good because of easy access.
\item 	The events not in the dorms, like the wind farm and the garden on saturdays.
\item 	Free prizes for residents
\item 	Good advertising and pushing for participation; good mission -- means well -- prizes and incentives. 
\end{enumerate}
\end{minipage}  \normalsize

&

\small  \begin{minipage}[t]{4.0in}
{\em What do you think should be improved?}
\begin{enumerate}[noitemsep,nolistsep]
\item 	Better advertising. {\em mentioned six times}.
\item It was nice to be involved in the KC, but RAs felt really pressured to put KC goals above RA duties and expectations, which made us confused and conflicted.  Please don't pressure us so much.
\item People seemed to be dissuaded because it's too competitive.
\item More motivation for the residents who don't already participate.
\item Hype goes up and down; should be shorter.
\item Promotion: more effective, make it desirable.
\item More ongoing events.
\item Help with event promotion or at least fliers made with the dates of all events.
\item The website is too complicated; my residents get confused just logging on and they feel that it's not "fun" enough to spend time on even though I keep telling them how easy it is.  Also, maybe more prizes, like top prizes [beyond] the first place in each lounge.
\item Need more professional staff other than using us RAs who don't know squat about the Kukui Cup.
\item 	No initial advertisements to explain the Kukui Cup. More direction from the Kukui Cup staff to the RAs regarding what to do during events. Also involvement from the Kukui Cup staff at events. 
\item 	More incentives; more user friendly website; more appealing for freshman; not so time consuming; make it fun.
\item 	Incentives; website was confusing even for RA; better programs--less boring; staff needs to be more professional and informed.
\item 	Give us flyers to put under the doors of resident's rooms; next year clearly explain what the Kukui Cup is beforehand (goals, sustainability, competition).
\item 	Have more information about the Kukui Cup and what it is at events; more ways on how to save energy; more ways explaining why energy conservation is important;
\item 	RAs shouldn't be required to go to events.  Both events I attended were completely overstaffed and most of the RAs had nothing to do.   Advertisements are also poor: no explanation of what the Kukui Cup actually is.
\item 	Be more clear about the RA duties.  Don't require the RAs to be there unless necessary; we have busy schedules and seem to do nothing at events.
\item 	More engaging videos; prizes that related to us more; more advertising; less of a time commitment for RAs.
\item 	Clearer and more advertising; more user-friendly website; more info to RAs about events (especially the ones we signed up for).
\item 	Need to offer prizes past first places to encourage participation.
\item 	Expand prizes range; only the top win; and maybe increase advertisements for clarity.
\item 	Limited space on events; marketing; variety of events.
\item 	You have to go to almost every event to get the good stuff and most people don't have time.
\item 	Online games are a tad bit boring.  Make it more interactive/related to the residents.
\item 	Online games take too long and are boring.
\item 	Tying the electronic game into homework and energy study tips; energy meter app.
\item 	Motivate residents; my residents don't want to participate because they see no value.
\item 	Length of videos. 
\end{enumerate}
\end{minipage}  \normalsize
\\
\hline
\end{tabular}
\caption{2012 Questionnaire: Good things and suggested improvements for 2012 Kukui Cup}
\label{fig:2012-questionnaire-results}
\end{figure*}

\subsection{2012 Questionnaire analysis}

Analysis of the positive comments revealed the following common responses:

\begin{enumerate}[noitemsep]
\item Prizes/Incentives (mentioned on 18 out of 32 cards).
\item Workshops/events (mentioned on 14 out of 32 cards)
\item Other: structure of website, guest speakers, excursions, food, Facebook
\end{enumerate}

The suggestions for improvement seemed to cluster as follows:

\begin{enumerate}[noitemsep]
\item Promotion/flyers/advertising (mentioned on 16 out of 31 cards)
\item Complexity/user friendliness of website (mentioned on 4 cards)
\item Offer prizes past first place (mentioned 2 times, but appeared to be copied).
\item Make it more "fun"/less boring (mentioned 5 times)
\item Provide "professsional" staff (mentioned 2 times)
\item Clearer RA responsibilities, better direction (mentioned 3 times)
\end{enumerate}


\section{Insights for the 2013 Kukui Cup}
\label{kc-2013}

After two years of experiences with two different groups of RAs, here are our preliminary
insights:

\subsection{Effective advertising is more difficult than it appears}

Feedback from both years focused on advertising as a primary area for improvement.  Though
our advertising in 2011 was obviously deficient, we made significant new efforts to market the
Kukui Cup in 2012, including: active Facebook postings, YouTube videos, signs posted by
elevators, emails to all students, and chalking the sidewalks.  As in 2011, we attempted
to put a poster in the lobby, and just like 2011, they were vandalized almost
immediately.  

In addition to Kukui Cup staff efforts, we told the four RAs assigned to each
event that marketing the events was the most important way for them to support the
events.  We heard that RAs put up additional posters, went door-to-door, and generally
made a good faith effort to get the word out. 

Our current belief is that while it may be true that improved marketing could have an
impact on overall Kukui Cup adoption, the changes needed to significantly increase
the percentage of Hale Aloha residents playing the game are not simple or obvious.    

Part of the reason for this is because we believe that first year students are literally
barraged with new messages at the onset of their college experience; indeed, they are
barraged with new experiences.  While we view the Kukui Cup as something unique and novel
in the collegiate experience, our target audience views simply waking up in the
morning in the residence halls as a unique and novel experience.  All of the Kukui Cup
signage, postings, and videos must compete with hundreds of other equally novel stimulii every
day.  

For 2013, we recommend building upon this year's marketing materials by creating one page,
two sided pamplets for each event.  One side will always provide basic information about
the Kukui Cup (and perhaps updated challenge status).  The other side will provide
concrete details about the specific event of interest.  We need to go beyond saying, for
example, ``Kealoha, world renowned slam poet'', and instead provide a description that
explains what slam poetry is for those who are not already familiar with the term.  As has
always been the case, we must prominently feature the presence of food at the events.
This pamplet will be posted to Facebook prior to the events, and copies emailed to all RAs
to print and distribute on their floors as they see fit.


\subsection{One third of Hale Aloha residents may be the ``natural'' adoption level for
  the Kukui Cup}  

It is striking to us that despite significantly more and better marketing in 2012 than in
2011, and despite the change from voluntary to mandatory involvement of RAs, the basic
level of adoption remained unchanged.  While our advertising in 2012 could still be improved, it
was undoubtably better than in 2011.  While the engagement of RAs could be better, at
least they viewed the Kukui Cup as part of their responsibility.  Yet these improvements
did not yield even at 10\% improvement in the overall level of adoption.  

Although two years of data is not conclusive, our experience thus far provokes the 
following question:  Could 1/3 be the ``natural'' level of adoption for an activity such
as the Kukui Cup for the resident population of Hale Aloha?

Interestingly, it does appear that the changes made in 2012 did have an impact on the
engagement of those residents who played the game, as they performed almost twice as many
activities.  But we didn't, for example, get twice as many students playing.

Note that we are not claiming 1/3 as a kind of ``universal'' adoption constant.  As a
counterexample, the 2012 HPU Kukui Cup achieved over 50\% overall adoption with one
residence hall achieving 100\% participation.  We hope to do some research in an attempt
to understand better the factors that might have led to that outcome.

\subsection{Hale Aloha RAs are students too}  

There are several implications of this statement.  First, the ``natural'' involvement of
Hale Aloha RAs might be just like the population as a whole, with about 1/3 of the RAs
actively engaging with the challenge, and 2/3 participating minimally or not at all
regardless of whether participation is voluntary or mandatory.

Second, we now realize that Hale Aloha RAs as a whole do not have any special skills with
respect to engagement of their residents in the Kukui Cup.  In both years, there was only
one lounge in which the RAs achieved both significant engagement and energy reduction, and
this was independent of the presence of advertising, mandatory engagement by the RAs, etc.

Third, it is clear that from the feedback that the 2012 RAs did not feel ownership over
the Kukui Cup.  For example, the most common suggestion for improvement was ``more
advertising'', but not a single RA appeared to recognize that advertising was one of their
primary responsibilities!  They also wanted clearer definition of their responsibilities,
but we believe that their responsibilities were communicated to them in a variety of ways
at a variety of times.

\subsection{Consider running the 2013 Kukui Cup in Spring semester}  

One interesting suggestion made by an RA this year is to hold the Kukui Cup in the Spring
semester, not the Fall semester.  The rationale for this is that during Fall semester,
there are many more events competing for student's time and attention.  During Spring,
there is less happening, and so the Kukui Cup will stand out more.


\subsection{Hale Aloha staff should take over planning and execution of the Kukui Cup}  

The Kukui Cup has been planned and executed by members of the Collaborative Software
Development Laboratory (CSDL) for the past two years as part of a project sponsored by the
National Science Foundation.  That funding comes to an end in summer 2013, which will make
it difficult for CSDL to provide the same level of support for the Kukui Cup as it has in
the previous two years.  Fortunately, the project has matured to the point where other
organizations (HPU, EWC) have put on Kukui Cup challenges with only minimal support from
CSDL.

We also suspect that one of the reasons for low RA engagement is the sense that the Kukui
Cup is this externally organized project that is being imposed upon them, not a project
that they have control over.  By moving to the challenge to Spring 2013, the Hale Aloha
staff would have an entire semester to organize the challenge, which would appear to be
sufficient time. 

Finally, we believe that the staff of Hale Aloha now has enough experience with the
Challenge to make whatever changes it sees fit and create something that more closely fits
their needs.  


\end{document}
