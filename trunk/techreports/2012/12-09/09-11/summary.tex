%%%%%%%%%%%%%%%%%%%%%%%%%%%%%% -*- Mode: Latex -*- %%%%%%%%%%%%%%%%%%%%%%%%%%%%
%% summary.tex -- 
%% Author          : Philip Johnson
%% Created On      : Tue Mar 31 11:42:10 2009
%% Last Modified By: Philip Johnson
%% Last Modified On: Mon Apr 27 13:56:09 2009
%% RCS: $Id$
%%%%%%%%%%%%%%%%%%%%%%%%%%%%%%%%%%%%%%%%%%%%%%%%%%%%%%%%%%%%%%%%%%%%%%%%%%%%%%%
%%   Copyright (C) 2009 
%%%%%%%%%%%%%%%%%%%%%%%%%%%%%%%%%%%%%%%%%%%%%%%%%%%%%%%%%%%%%%%%%%%%%%%%%%%%%%%
%% 

\section*{Project Summary}
\renewcommand{\thepage} {A--\arabic{page}}

%% {\em The proposal must contain a summary of the proposed activity suitable for
%% publication, not more than one page in length. It should not be an abstract
%% of the proposal, but rather a self-contained description of the activity
%% that would result if the proposal were funded. The summary should be
%% written in the third person and include a statement of objectives and
%% methods to be employed. It must clearly address in separate statements
%% (within the one-page summary):

%% (1) the intellectual merit of the proposed activity; and

%% (2)the broader impacts resulting from the proposed activity. 

%% It should be informative to other persons working in the same or related
%% fields and, insofar as possible, understandable to a scientifically or
%% technically literate lay reader. Proposals that do not separately address
%% both merit review criteria within the one-page Project Summary will be
%% returned without review.
%% }


\noindent {\bf Overview.}  The vision of this proposal is to develop and
institutionalize a new approach to computational thinking where abstraction
and automation combine to transform the use of {\em empirical thinking} in
software development.  We call this approach ``empirical computational
thinking'', or \eCT.  The goal of this research is to explore, evaluate,
and institutionalize techniques and technologies for \eCT, building upon
research and education by ourselves and others in empirically-based
software development.

The project will pursue this vision through a set of interdependent
objectives. It will develop a Common \eCT\ Evaluation Framework that
supports understanding, comparison, and improvement of individual \eCT\
initiatives at universities. It will create a set of Canonical \eCT\
Learning Objectives, evaluate them within the University of Hawaii computer
science curriculum, and provide them as templates for use in other \eCT\
initiatives at other universities.  It will support further development and
evaluation of an \eCT\ curriculum at the University of Hawaii involving the
Software ICU and Devcathlon technologies.  It will support design and
implementation of repositories for \eCT\ curriculum materials and outcome
data to facilitate dissemination of \eCT.  It will enable
post-course evaluation of \eCT\ through the Facebook and LinkedIn social
networking technologies. Finally, it will investigate how \eCT\ can provide
a foundation for scientific and evidence-based thinking, as well as how it
can be applied outside the advanced software development curriculum.

\medskip

\noindent {\bf Intellectual Merit.}  First, this project will create and
institutionalize the notion of empirical computational thinking as a useful
component for programming courses. Second, it will create a new
community of research and practice around the unifying concept of empirical
computational thinking.  Third, it will generate two new
mechanisms for evaluating initiatives in empirical computational thinking:
the Common \eCT\ Evaluation Framework and Canonical \eCT\ Learning
Objectives. Fourth, it will lead to significantly increased use
of \eCT\ initiatives in computer science curriculum. Fifth, it 
will generate new empirical data sets regarding software development
activities in a classroom setting. Sixth, it will explore the use of \eCT\
as a foundation for scientific and evidence-based thinking.

\medskip 

\noindent{\bf Broader Impacts.}  First, this project will serve
underrepresented populations, as the University of Hawaii is an EPSCOR
state. Approximately 84\% of undergraduates at the University of Hawaii are
minorities, and the computer science students exemplify this diversity.
Second, the software engineering curriculum at the University of Hawaii is
well-regarded within the local high tech community, and many of its
graduates have gone on to leadership positions. A successful \eCT\
initiative will be transformative beyond the college and into the local
community.  Third, it supports the NSF goal of fostering
integration of research and education.  The research outcomes regarding
\eCT\ will impact directly on classroom practice. Fourth, it will create an
experience base that will facilitate the spread of \eCT\ throughout the
computer science curriculum and into other disciplines as well.


