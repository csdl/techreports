%%%%%%%%%%%%%%%%%%%%%%%%%%%%%% -*- Mode: Latex -*- %%%%%%%%%%%%%%%%%%%%%%%%%%%%
%% project.tex -- 
%% Author          : Philip Johnson
%% Created On      : Tue Mar 31 11:44:58 2009
%% Last Modified By: Philip Johnson
%% Last Modified On: Fri Apr 03 15:46:00 2009
%% RCS: $Id$
%%%%%%%%%%%%%%%%%%%%%%%%%%%%%%%%%%%%%%%%%%%%%%%%%%%%%%%%%%%%%%%%%%%%%%%%%%%%%%%
%%   Copyright (C) 2009 
%%%%%%%%%%%%%%%%%%%%%%%%%%%%%%%%%%%%%%%%%%%%%%%%%%%%%%%%%%%%%%%%%%%%%%%%%%%%%%%
%% 

\section*{Project Description}
\pagenumbering{arabic}
\renewcommand{\thepage} {C--\arabic{page}}

\subsection*{Project Vision, Goals, Objectives, and Outcomes}


{\em Describe the CT-centric vision, goals, objectives, and anticipated outcomes of the proposed project. Clearly indicate how they will contribute to realization of the three CPATH program goals: (1) contribute to the development of a globally competitive U.S. workforce with CT competencies essential to U.S. leadership in the global innovation enterprise; (2) increase the number of students developing CT competencies by infusing CT learning opportunities into undergraduate education in the core computing - computer and information science and engineering - disciplines, and in other fields of study; and, (3) demonstrate transformative CT-focused undergraduate education models that are replicable across a variety of institutions.
}

\bigskip

Jeannette Wing has written, ``Computational thinking involves solving
problems, designing systems, and understanding human behavior, by drawing
on the concepts fundamental to computer science \cite{Wing06}''.  In her
presentation ``Computational Thinking and Thinking About Computation'',
Wing refines her view of these fundamental computer science concepts as
the ``Two As'': Abstraction and Automation.  Activities
related to the first A include: choosing the right abstractions, operating
at multiple levels of abstraction, and defining relationships between
abstractions.  Activities related to the second A involve mechanizing the
first A via precise notations and models.  In essence, automation amplifies
the power of abstraction.  Computational thinking, from this perspective,
is about the correct choice of abstraction combined with the correct choice
of automation.

The vision of this proposal is to develop and institutionalize a new
approach to computational thinking where abstraction and automation combine
to transform {\em empirical thinking} in software development.

What is empirical thinking?  The term ``empirical'' is variously defined as
``derived from experiment and observation rather than theory''; ``evidence
or consequences observable by the senses''; and ``capable of being verified
or disproved by observation or experiment.''

Given this definition, empirical thinking is already commonplace in
software development.  For example, beginning programmers use empirical
thinking when they ``observe'' the output of the compiler to learn how to
write syntactically correct programs.  Beginners also tend to make
extensive use of ``experimentation'': they will run their program with
example data, compare the actual behavior to what they expect as the
behavior, then make modifications so that the observed behavior is
consistent with the expected behavior.

These examples of empirical thinking, while satisfactory for beginning
programmers, do not scale because they do not contain the two A's.
Clearly, manual observation of the compiler output or program run-time
behavior provides neither abstraction nor automation.

One would hope that as students progress into more advanced software
development courses, two things would occur: first, the complexity, size,
and personnel involved in software development projects would scale
upwards; and second, the degree of abstraction and support for automation
of their empirical thinking would increase commensurately. Unfortunately,
while advanced software development courses certainly require students to
develop significantly more sophisticated systems than their introductory
counterparts, the student's empirical thinking to a great extent remains
non-abstract and non-automated.  The principle computational support for
advanced programming classes is an integrated development environment such
as Eclipse or Visual Studio, which is a significant advance over vanilla
text editors but still provides relatively little in the way of abstraction
or automation for empirical thinking about the products and processes of
software development.

The goal of this research is to explore, evaluate, and institutionalize
tools and technologies for abstraction and automation of empirical thinking
in advanced software development courses.  As one concrete example, we have
developed a system and associated curriculum called the ``Software
Intensive Care Unit'' which automatically collects raw software development
process and product data and abstracts it into a set of ten ``vital signs''
that provide an empirical way for students to gauge the ``health'' of their
ongoing projects.  The Software ICU supports automated, abstract empirical
thinking by students about the current state and past history of both their
projects and their group processes.

To achieve this goal, we will pursue several concrete objectives. Firstr,
evaluate in courses at University of Hawaii.  Disseminate to other courses.
Create a community around the empirical thinking approach and the
technology.  Produce tailored versions for other universities.  Produce
enhancements for both professioanl settings and for more introductory
programs.

Our anticipated outcomes will directly support the goals of the CPATH program as follows:

First, we anticipate creating software development professionals with experience using abstract, automated, empirical forms of computational thinking.   

These skills will increase their productivity and competitiveness. 

As our target is advanced software development courses, this program will inject this new form of empirical computational thinking into the curriculum..


\subsection*{Intellectual Basis/Related Work}

{\em Describe the intellectual basis for the project and discuss related prior work.  Include a review of the research literature relevant to the project and provide corresponding references. }

\subsection*{Current State}

{\em Provide a current assessment of undergraduate education in the relevant participating organizations.  Describe prior pilot programs or planning activities conducted to date, if any, and their outcomes.  Where appropriate, provide institutional data to document the current environment by uploading data into the Supplementary Docs section in FastLane.}

\subsection*{Implementation Plan}

{\em Describe in detail the CT-centric activities to be undertaken to realize the project vision, goals, objectives and anticipated outcomes. 

Define, or describe how the proposing team will attempt to define, the core computing concepts, methods, technologies and tools to be integrated into promising new undergraduate education models.  Describe your plans to identify and implement effective strategies to develop and assess CT competencies in the relevant learning communities.  Identify the stakeholder cohort, e.g. K-20 administrators, faculty, teachers, students, etc., that will participate in and/or benefit from the activities. If relevant, describe how change will be effected and sustained in the participating organizations.  

Describe project milestones in the context of a project timeline and identify responsible parties and expected outcomes for each milestone.  Summarize this information in a figure that you upload into the Supplementary Docs section in FastLane. 

Describe how project outputs and outcomes will be disseminated to the relevant stakeholder groups and to the national community and if relevant, how project resources will be made available to others to adopt or adapt. Identify proactive measures to find and support adopters of promising models and/or practices. Describe plans for outreach to other groups or interested institutions that will take place during the project.  
}

\subsection*{Collaboration and Management Plan}

{\em Provide a collaboration and management plan that will guide project implementation.  Describe how the project leadership team will form, orient, manage, and reinforce relationships in the project.  Provide evidence of the commitment of the participating organizations to effect and sustain the anticipated project outcomes; letters of collaborative support should be uploaded into the Supplementary Docs section in FastLane. }

\subsection*{Evaluation Plan}

{\em Provide an evaluation plan that will inform the project progress and measure its impact.  Include a description of the instruments/metrics used to measure, document, and report on the project's progress.  Identify the evaluator who will be responsible for the evaluation component and discuss their expertise related to the evaluation as well as any other linkages to the project or organizations involved.  }







