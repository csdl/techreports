%%%%%%%%%%%%%%%%%%%%%%%%%%%%%% -*- Mode: Latex -*- %%%%%%%%%%%%%%%%%%%%%%%%%%%%
%% project.tex -- 
%% Author          : Philip Johnson
%% Created On      : Tue Mar 31 11:44:58 2009
%% Last Modified By: Philip Johnson
%% Last Modified On: Tue Mar 31 11:52:19 2009
%% RCS: $Id$
%%%%%%%%%%%%%%%%%%%%%%%%%%%%%%%%%%%%%%%%%%%%%%%%%%%%%%%%%%%%%%%%%%%%%%%%%%%%%%%
%%   Copyright (C) 2009 
%%%%%%%%%%%%%%%%%%%%%%%%%%%%%%%%%%%%%%%%%%%%%%%%%%%%%%%%%%%%%%%%%%%%%%%%%%%%%%%
%% 

\section*{Project Description}
\pagenumbering{arabic}
\renewcommand{\thepage} {C--\arabic{page}}

\subsection*{Project Vision, Goals, Objectives, and Outcomes}

Describe the CT-centric vision, goals, objectives, and anticipated outcomes of the proposed project. Clearly indicate how they will contribute to realization of the three CPATH program goals. 

The goals of the program are to:

\begin{itemize}
\item contribute to the development of a globally competitive U.S. workforce with CT competencies essential to U.S. leadership in the global innovation enterprise;

\item increase the number of students developing CT competencies by infusing CT learning opportunities into undergraduate education in the core computing - computer and information science and engineering - disciplines, and in other fields of study; and,

\item demonstrate transformative CT-focused undergraduate education models that are replicable across a variety of institutions.
\end{itemize}


\subsection*{Intellectual Basis/Related Work}

Describe the intellectual basis for the project and discuss related prior work.  Include a review of the research literature relevant to the project and provide corresponding references. 

\subsection*{Current State}

Provide a current assessment of undergraduate education in the relevant participating organizations.  Describe prior pilot programs or planning activities conducted to date, if any, and their outcomes.  Where appropriate, provide institutional data to document the current environment by uploading data into the Supplementary Docs section in FastLane.

\subsection*{Implementation Plan}

Describe in detail the CT-centric activities to be undertaken to realize the project vision, goals, objectives and anticipated outcomes. 

Define, or describe how the proposing team will attempt to define, the core computing concepts, methods, technologies and tools to be integrated into promising new undergraduate education models.  Describe your plans to identify and implement effective strategies to develop and assess CT competencies in the relevant learning communities.  Identify the stakeholder cohort, e.g. K-20 administrators, faculty, teachers, students, etc., that will participate in and/or benefit from the activities. If relevant, describe how change will be effected and sustained in the participating organizations.  

Describe project milestones in the context of a project timeline and identify responsible parties and expected outcomes for each milestone.  Summarize this information in a figure that you upload into the Supplementary Docs section in FastLane. 

Describe how project outputs and outcomes will be disseminated to the relevant stakeholder groups and to the national community and if relevant, how project resources will be made available to others to adopt or adapt. Identify proactive measures to find and support adopters of promising models and/or practices. Describe plans for outreach to other groups or interested institutions that will take place during the project.  


\subsection*{Collaboration and Management Plan}

Provide a collaboration and management plan that will guide project implementation.  Describe how the project leadership team will form, orient, manage, and reinforce relationships in the project.  Provide evidence of the commitment of the participating organizations to effect and sustain the anticipated project outcomes; letters of collaborative support should be uploaded into the Supplementary Docs section in FastLane. 

\subsection*{Evaluation Plan}

Provide an evaluation plan that will inform the project progress and measure its impact.  Include a description of the instruments/metrics used to measure, document, and report on the project's progress.  Identify the evaluator who will be responsible for the evaluation component and discuss their expertise related to the evaluation as well as any other linkages to the project or organizations involved.  







