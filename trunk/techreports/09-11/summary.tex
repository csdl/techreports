%%%%%%%%%%%%%%%%%%%%%%%%%%%%%% -*- Mode: Latex -*- %%%%%%%%%%%%%%%%%%%%%%%%%%%%
%% summary.tex -- 
%% Author          : Philip Johnson
%% Created On      : Tue Mar 31 11:42:10 2009
%% Last Modified By: Philip Johnson
%% Last Modified On: Thu Apr 23 15:02:13 2009
%% RCS: $Id$
%%%%%%%%%%%%%%%%%%%%%%%%%%%%%%%%%%%%%%%%%%%%%%%%%%%%%%%%%%%%%%%%%%%%%%%%%%%%%%%
%%   Copyright (C) 2009 
%%%%%%%%%%%%%%%%%%%%%%%%%%%%%%%%%%%%%%%%%%%%%%%%%%%%%%%%%%%%%%%%%%%%%%%%%%%%%%%
%% 

\section*{Project Summary}
\renewcommand{\thepage} {A--\arabic{page}}

%% {\em The proposal must contain a summary of the proposed activity suitable for
%% publication, not more than one page in length. It should not be an abstract
%% of the proposal, but rather a self-contained description of the activity
%% that would result if the proposal were funded. The summary should be
%% written in the third person and include a statement of objectives and
%% methods to be employed. It must clearly address in separate statements
%% (within the one-page summary):

%% (1) the intellectual merit of the proposed activity; and

%% (2)the broader impacts resulting from the proposed activity. 

%% It should be informative to other persons working in the same or related
%% fields and, insofar as possible, understandable to a scientifically or
%% technically literate lay reader. Proposals that do not separately address
%% both merit review criteria within the one-page Project Summary will be
%% returned without review.
%% }


\noindent {\bf Overview.}  The vision of this proposal is to develop and
institutionalize a new approach to computational thinking where abstraction
and automation combine to transform the use of {\em empirical thinking} in
software development.  We call this approach ``empirical computational
thinking'', or \eCT.  The goal of this research is to explore, evaluate, and institutionalize
techniques and technologies for \eCT, building upon research and education
by ourselves and others in empirically-based software development.

\medskip

\noindent {\bf Intellectual Merit.}  First, this project will create and
institutionalize the notion of empirical computational thinking as a useful
component for programming courses. Second, this project will create a new
community of research and practice around the unifying concept of empirical
computational thinking.  Third, this project will generate a new mechanism
for evaluating initiatives in empirical computational thinking: the Common
\eCT\ Evaluation Framework. Fourth, this project will lead to significantly
increased use of \eCT\ initiatives in computer science curriculum. Fifth,
this project will generate new empirical data sets regarding software
development activities in a classroom setting. Sixth, we will explore the
use of \eCT\ as a foundation for scientific and evidence-based thinking.

\medskip 

\noindent{\bf Broader Impacts.}  First, this project will serve
underrepresented populations, as the University of Hawaii is an EPSCOR
state. Approximately 84\% of undergraduates at the University of Hawaii are
minorities, and the computer science students exemplify this diversity.
Second, the software engineering curriculum at the University of Hawaii is
well-regarded within the local high tech community, and many of its
graduates have gone on to leadership positions. A successful \eCT\
initiative could thus be transformative beyond the college and into the
local community.  Third, this project supports the NSF goal of fostering
integration of research and education.  The research outcomes regarding
\eCT\ will impact directly on classroom practice.  
