\chapter{Conclusion and Future}

As Gartner put it: \cite {gartner2010}
Games often model the real world. Increasingly, real-world activities are starting to look like a game.
In the past year, gamification has emerged as a recognizable trend. Rarely does an emerging trend impact so many areas of business/society.
Given that the goal of gamification is to change human behaviors, there are many opportunities and risks.

So What? This industry has made massive study and progress with engagement. They have taken the balance between challenge and skill in flow states and fully operationalized this mechanism to keep people engaged, even glued to games. We who are in the workplace and the field of employee engagement have much to learn by studying and applying the principles and practices of gaming. We also must not get caught by gamification hysteria by keeping a focus on the limits and potential traps embedded in gamification.

It seems the major take away of reading the debates of gamification is that, before the novelty of simple gamification hadn�t worn off, and it seemed like an amazing idea that everything could be made more fun and motivational with achievements and points. Now we know it�s crucial that we make good games, rather than take the easiest bits to reproduce (points) and apply them to everything.

There was a definite feeling of, �We�re only at the start of something here, a turning point, so we better steer it well� throughout it all�

In a field rife with anecdotes but little hard data, Wharton's Werbach and New York Law School's Hunter intend to develop in-depth case studies to examine the types of business problems organizations want gamification to solve, the techniques used and the results. According to Werbach, there currently are few bridges between game design as a craft and psychological research. "That's why research is valuable -- to get beyond whether gamification is good or bad, and does it work or not."

Deterding \cite {Deterding2011mt} suggested that insight into �gamefulness� as a 
complement to �playfulness� � in terms of design goals as well as 
user behaviors and experiences  �  marks  a  valuable  and  lasting 
contribution of studying �gamified� systems.  Partly  in  reaction  to 
this, the term �gameful design� �  design  for  gameful  experiences 
�  was  also  introduced  as  a  potential  alternative  to  �gamification�.
Given the industry origins,  charged  connotations  and  debates 
Figure 1. �Gamification� between game and play, whole and parts
Figure 2. Situating �gamification� in the larger field about the practice and design of �gamification�, �gameful design� 
currently provides a new term with less baggage, and therefore a 
preferable term for academic discourse

** Important People to Follow:

Jane McGonigal

Ian Bogost

Jess Schell

....