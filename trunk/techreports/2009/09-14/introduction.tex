\chapter{Introduction}\label{chapter_introduction}
A central issue addressed in this dissertation is the possibility of recurrent behaviors discovery 
from publicly available software process artifacts. 

As recurrent behaviors are considered to be the basic building blocks of any human-driven
goal-oriented process, reflecting the development of more or less fixed ways of dealing 
with tasks \textit{based on past performance} \cite{neal2012habits} \cite{1903} -- 
the ability to discover recurrent behaviors in the context of software development equates 
to the ability to discover the evolution of characteristic mannerisms in which developers 
structure their activities, which, in turn, enables the understanding of their software 
development processes.  

I have explored an approach to this problem based on the transformation of software artifact trails 
into time series by measurements and on the subsequent application of a novel time series 
classification technique that enables characteristic patterns discovery, which, 
I hypothesize, correspond to recurrent behaviors.

This dissertation reviews the relevant work, proposes a novel interpretable time series 
classification algorithm, and presents results of its performance evaluation along with 
the results of an empirical evaluation of the proposed approach applicability to the problem 
of recurrent behaviors discovery from public software artifacts.

%\textit{The terms used throughout the thesis are defined in the Section \ref{section_terminology}. 
%Problem's background is given in Section \ref{section_background}. 
%Section \ref{section_software_process_design} discusses traditional approaches to software process 
%design and improvement contrasting them with self-organizing open-source software processes 
%and introduces public software artifacts.
%Sections \ref{section_software_trajectory} reviews a previously developed techniques for software 
%process and product improvement based on continuous measurements and introduces Software Trajectory.
%Section \ref{section_research_hypothesis} presents the research hypothesis.
%Section \ref{section_contributions} enumerates main contributions of the thesis, 
%while section \ref{section_organization} explains the thesis organization.}

\section{Basic terminology}\label{section_terminology}
\begin{defn}\label{def_process}
A \textbf{\textit{Software Process}} defines a way the software development goes. It enumerates
resources and artifacts, but most importantly, it defines a set of activities that need to be 
performed in order to design, to develop, and to maintain software systems.
\end{defn}
Examples of such activities include requirements collection and creation of UML diagrams, 
source code writing, system testing, and others. The intent behind a software process is to provide 
the control over software development effort by implementing a global strategy and by structuring
and coordinating human activities in order to achieve the goal -- to deliver a functional
software system on time and under the budget. 

%\begin{defn}\label{def_process_desc}
%A \textbf{\textit{Process Description}} is any sort of written software process description defining 
%some or all of needed resources, artifacts, actions, activities and intended process behavior.
%\end{defn}

\begin{defn}\label{def_process_model}
A \textbf{\textit{Software Process Model}} is a complete and unambiguous software process description 
that guarantees a rigorous specification ready to be executed.
\end{defn}

\begin{defn}\label{def_metric}
A \textbf{\textit{Software Metric}} is a characteristic of a software or a software process that can be 
objectively measured.
\end{defn}
Examples of software product metrics include the size of a software system measured in lines of code 
(LOC) or in functional points (FP), and the number of defects discovered in a delivered system. 
Examples of software process metrics include the velocity of a software process called ``churn'', that 
measures the amount of LOC changed per day, the response time to fix an issue, and the ``technical debt'', 
that measures deterioration of the code quality over time. 

Similarly to other sciences, measurements in Software Engineering are essential for establishing of systematic 
research. Product and process metrics are also important in software project management where they are 
used in order to derive high-level software project metrics including cost, schedule, and productivity.

\begin{defn}\label{def_artifact}
A \textbf{\textit{Software Artifact}} is one of numerous products and byproducts of a software process - 
a use-case, an UML class diagram, a change record, or a bug report. 
\end{defn}
Note, that while artifacts play an important role in software processes, where they are used to support 
software development activities and reused to document the resulting software, they are not created 
in order to enable a scientific research. Thus, since software artifacts repertoire differs among projects 
and processes, the use of software artifacts for scientific research creates an additional threat to its 
external validity, i.e. generalization.

\begin{defn}\label{def_artifact_trail}
A \textbf{\textit{Software Artifact Trail}} is an ordered by the artifact's creation time collection of 
software process artifacts.
\end{defn}
Examples of software artifact trails include an ordered by the commit time collection of a software project's 
source code change records and an ordered by the post time collection of a user's questions at StackOverflow 
website.

\newpage
%
% >> section
%
\section{Motivation. Software Crisis.}\label{section_background}

Contemporary software projects deal with the development of complex software systems and typically have 
a considerably long life cycle - well over decade.
A project's development and maintenance activities are usually carried out by geographically 
distributed teams and individuals. The development pace, the experience, and the structure of a 
development team continuously change with project progression and as developers joining and leaving. 
When combined with schedule and requirements adjustments, these create numerous difficulties 
for stakeholders, developers, and users, ultimately affecting the project success \cite{citeulike:2207657}. 

This software development complexity phenomena was identified in 1968 as ``Software crisis'' 
\cite{naur_crisis_68}, and was addressed by bringing the research and the practice of software 
development under the umbrella of Engineering in an effort to provide the control over the process 
of software development. 
Following the engineering paradigm, numerous methodologies of software design and development 
processes, known as \textit{Software Processes}, were proposed \cite{citeulike:10002165}.
Some of these were formalized into Software Process Models - industrial standards for software development 
such as CMM \cite{citeulike:9962021}, ISO \cite{iso-standard}, 
PSP \cite{citeulike:8347315}, and others \cite{citeulike:5043104}. 

In spite of this effort, industrial software development remains error-prone and more than half of all 
projects ending up failing or being very poorly executed \cite{chaos2006}.
Some of them are abandoned due to running over the budget, some are delivered with such low quality, 
or so late, that they are useless, and some, when delivered, are never used because they do not 
fulfill requirements. 

By the analysis of software project failures, it was acknowledged, that Engineering paradigm 
may not be an adequate way to control software development processes due to the large discrepancies 
between problems in Software Engineering and any other Engineering field 
\cite{citeulike:3729379} \cite{citeulike:5203446} \cite{citeulike:2207657} \cite{citeulike:12550665}.
The chief argument supporting this point of view is the drastic difference in the cost model:
while in Software Engineering there is almost no cost associated with materials and fabrication, 
these usually dominate cost in all other Engineering disciplines, but, ironically, 
Software Engineering is suffering from cost and challenges associated with 
continuous re-design of the product and its design processes - the issue that is 
hardly seen at all in other Engineering areas. 
In addition, as it was shown by the numerous studies, engineering-like models of software processes 
are typically prescriptive and rigid -- they are difficult to adapt to the particular organizational 
structure, to the project specificities, and to changing requirements \cite{citeulike:113403}. 
Thus, the degree to which an adopted process model structures software processes varies greatly 
between teams and projects and cannot guarantee the success \cite{sacchi_2001}. 
Finally, an increasing understanding and appreciation of human factors in software development 
processes over tools, technologies, and standards, suggests that the human-driven software 
process aspects are likely to be defining in the software project fate \cite{citeulike:6580825} 
\cite{citeulike:149387} \cite{1605185} \cite{citeulike:113403} \cite{citeulike:12743107}. 

However, current alternatives to Engineering-like processes that are flexible, user- and developer-centric,
and which often praised for their dynamism, flexibility, and encouragement for innovation --
such as Agile and Software craftsmanship -- are also affected by the same software process complexity. 
For example, some of the agile processes are unable to handle it by the design: 
SCRUM does not cover the whole software life-cycle \cite{Cohn_SCRUM}, 
XP does not scale for large teams \cite{Beck_XP}, 
and TDD requires an extensive expertise from developers \cite{Beck_TDD}.
In addition, the increase in flexibility is often directly linked with increase in uncertainty, which creates 
significant difficulties with project cost and effort estimation \cite{citeulike:12933080} \cite{citeulike:9928907}.
The Free/Libre/Open source software (FLOSS) projects , which typically less concern with the cost issues, 
are also affected by the complexity. 
As it was shown, most of the open-source projects never reach a ``magic'' 1.0 version 
\cite{citeulike:12480029}. Among others, the great "infant mortality rate" of open-source 
projects was related to a burnout, inability to acquire a critical mass of users, loss of leading developer(s), 
and forking \cite{richter2007critique}. 

Currently, it is widely acknowledged, that there exists no ``silver bullet'' process which 
guarantees to bring a software project to the success \cite{citeulike:1986013}. 
Processes are numerous, each has advantages and drawbacks, and each accompanied with 
success stories and failure experiences, making the process selection difficult 
and the results of its application unpredictable.
This uncertainty, and the alarming rate of software projects failures suggest, that our understanding 
of software development ``mechanics'' is limited and insufficient \cite{citeulike:12550665}. 
The enormous cost of the lost effort, measured in hundreds of billions of US dollars 
\cite{citeulike:2207657} \cite{citeulike:2207653} \cite{citeulike:2207655}, 
continues to provide motivation for further research on software process design and improvement. 

%
% >> section
%
\section{Classical approaches to software process design and improvement}\label{section_software_process_design}
Traditionally, it was assumed that the software development is performed for a profit in 
corporate, government, or military settings by people that are mostly collocated together. 
This assumption shaped early research focused on approaches for on-site ``software manufacturing'',
which were discussed for decades in Software Engineering literature. 

These classical approaches can be divided into two distinct categories. 
The first category consists of \textit{top-down techniques} which are based on proposing of a process 
that is based on a specific pattern of software development. 
For example, Waterfall Model process proposes a sequential pattern in which developers first create a 
Requirements document, then create a Design, then create an Implementation, and finally develop Tests 
\cite{citeulike:9982731}. 
Alternatively,  Test Driven Development process proposes an iterative development pattern in which
the developer must first write a test case, then write the code to implement that test case, then re-factor 
the system for maximum clarity and minimal code duplication \cite{Beck_TDD}. 

While top-down techniques follow the usual path of trials and errors, and reflect natural to 
humans creative processes of invention and experimentation -- the ``invention'' of an adequate to 
the task software process is far from trivial and its evaluation cycle is considerably expensive 
and long \cite{citeulike:5043104} \cite{citeulike:1986013}.
Moreover, it was shown that the process inventors are usually limited in their scope and tend to 
assume idealized versions of real processes, thus, they often produce ``paper lions'' - process 
models which are likely to be disruptive and unacceptable for end users, 
at least in their proposed form \cite{citeulike:9758924}.

The second category of software process design and improvement approaches consists of 
\textit{bottom-up} techniques that focus on knowledge extraction from process events log for 
its reconstruction, elicitation, validation, and enhancement \cite{citeulike:12944447}. 
Typically, this task is viewed as a two-levels problem where the process event log is aggregated and 
transformed into the chain of logical development events at first, 
and the process model is constructed at the second level \cite{citeulike:2703162} \cite{citeulike:12944447}.
Cook and Wolf, in their pioneering work on software process discovery, have shown the possibility of 
automated extraction of software process models through the mining of process event logs 
\cite{citeulike:328044} \cite{citeulike:5120757} \cite{citeulike:5128143}. 
Later work by Huo et al. shows the possibility of software process improvement through the event 
logs analysis \cite{citeulike:7691059} \cite{citeulike:7690766}. 

The bottom-up approaches, while appearing to be more systematic and potentially less challenging than invention, 
are also affected by a number of issues, among which the observability is the most significant: 
while the live project observations are technically challenging to implement due to the high cost and 
privacy concerns \cite{citeulike:12944447}, the post-process data collection, for example through interviewing, 
significantly affects the process reconstruction validity due to the frequent discrepancies between actually 
performed and reported actions \cite{citeulike:7691059}. 
Yet another significant issue is the insufficient capacity of currently available process discovery and 
representation techniques to discover and to represent models of distributed and concurrent processes 
\cite{citeulike:12944447}. 

Nevertheless, while distinct in their nature, the traditional approaches to software process design and 
improvement yield similar abstract representations of software processes which are typically expressed 
formally in a process modeling language or as flowcharts of interconnected software 
development activities \cite{citeulike:12944447} \cite{citeulike:12944456}.
While process ``inventors'' put the best of their knowledge, experience, and logical reasoning into the proposed 
sequence of activities, similarly, the process ``miners'' strive to eliminate the noise and to converge to a 
concise sequence of activities that is supported by the majority of observations. 

This particular attention of traditional approaches to the deterministic and complete model synthesis 
is often cited as limiting since it assumes idealized and streamlined environment leaving many variable 
human factors, such as a team structure and expertise, work schedule, discipline, and motivation behind 
-- the issue that has been widely recognized 
\cite{citeulike:149387} \cite{citeulike:113403} \cite{citeulike:205322} \cite{citeulike:12798652}
but still largely ignored in industrial practices mostly due to the  difficulties with human component 
benefits estimation \cite{citeulike:12798659} \cite{citeulike:12798662} \cite{csdl2-12-11}.

%
% >> section
%
\section{Free/Libre/Open Source software processes}
Currently, we see the rise of alternative to on-site Software Engineering development model -- 
people are coming together over the Internet and create software which they distribute 
openly, promoting its modification and re-distribution. 
Surprisingly, they provide a very little if any guidance on software processes, effectively allowing any 
software process to be used as long as it positively contributes to the project's goal. 
This characteristic freedom of free-software processes, while challenges traditional schools of 
Software Engineering and software process research, enables advancements in previously unexplored or 
underexplored research directions, among which is the role human factors in software development.

The free-software social movement originates from 1960s and is inspired by the philosophy of 
source code sharing and its collaborative improvement. The movement was partially formalized in 1983 
when Richard Stallman, who launched GNU Project and founded Free Software Foundation in 1985.
The commonly used term ``open-source'' was coined later, in 1998 at the very first Open Source 
Initiative (OSI) meeting \cite{osi-history}.
The free-software development community consists of self-organized individuals and teams of 
mostly non-professional programmers - amateurs, hobbyists, students, and academics. 
By using Internet, they collaborate and develop software that is distributed free of charge as source code
and is usually called Free/Libre Open Source Software (FLOSS). 

Over the years, this software development model has proven its ability to deliver increasingly complex 
and surprisingly popular software in a truly global scale - when thousands of project's contributors 
and users are scattered all over the world. A number of FLOSS projects such as Linux and its 
derivatives, Gnome, Apache HTTP Server, PostgreSQL database, and others, succeeded to develop and to 
efficiently manage distributed software processes that provide the control over the large development 
team and source code base and deliver the state of the art software whose quality is similar to or 
exceeding that of industrial projects \cite{coverity2012}. 
This fact attracted a considerable attention not only from industrial companies that seek to emulate 
successful open source software processes in traditional closed-source commercial environment 
\cite{oss_virtual_organizations} \cite{oss_balance} \cite{oss_hp} \cite{oss_4industry}, 
but from software process research community, that is fond of understanding the FLOSS success phenomena
\cite{citeulike:12550640} \cite{citeulike:5043664} \cite{citeulike:5128808} \cite{citeulike:10377366}.

%This freedom provides a thriving human-centric 
%environment for creative individuals and teams where novel software processes can be invented and tested 
%immediately.

A number of studies conducted on open source processes discovered, that they are significantly 
different from the traditional software development at many levels. 
In particular, their flexibility and the inherent capacity to adapt to changing requirements is often cited as 
the most prominent. 
For example, the specification, that is the most significant document shaping industrial software development,
is rarely considered in open-source projects. Even in the Linux kernel development, which is probably one of 
the few strictly moderated development processes, developers prise practical reasons over specifications \ref{fig:kernel}.

\begin{figure}[ht!]
   \centering
   \includegraphics[width=140mm]{figures/Linus.Kernel.ps}
   \caption{A Torvald's response in the mailing list suggesting that practical reasons, the ``real-life'', 
   should be always considered over specifications.
   Excerpt from Linux mailing list. \url{http://lkml.indiana.edu/hypermail/linux/kernel/0509.3/1441.html}}
   \label{fig:kernel}
\end{figure}

\subsection{Public software repositories}
The proliferation of open-source development continues to create publicly available software process 
artifacts on increasingly high rate changing the software process research landscape by providing 
the data covering full software development life cycle for free. 
Currently, public code hosting sites such as SourceForge, GoogleCode, and GitHub host thousands of FLOSS 
projects offering numerous software process artifacts, such as design documents, source codes, bugs and 
issue records, and developers communications.
In addition, Q\&A and social websites for developers such as StackOverflow, TopCoder, and others, becoming 
increasingly popular among software developers and users as places to discuss software issues, 
to exchange expertise, to learn new tools, and to improve skills.

Scientific community response on the public availability of software process artifacts was overwhelming 
and a number of venues was established in order to address the increased interest. 
Since 2004, the International Conference on Software Engineering (ICSE) hosts a Working Conference on 
Mining Software Repositories (MSR). The original call for papers stated MSR's purpose as 
\textit{``... to use the data stored in these software repositories to further understanding of software 
development practices ... [and enable repositories to be] used by researchers to gain empirically based 
understanding of software development, and by software practitioners to predict and plan various aspects 
of their project''} \cite{msr2004} \cite{citeulike:7853299}. 
Several other venues: International Conference on Predictive Models in Software Engineering \cite{promise12}, 
International Conference on Open Source Systems, the Workshop on Public Data about Software Development, 
and the International Workshop on Emerging Trends in FLOSS Research have also played
an important role in shaping and advancing of the new research domain.

Some of the work conducted in the domain addresses the problem of software process-related 
knowledge discovery from artifacts. Probably the most notable and relevant to 
my research is work by Jensen \& Scacchi, where they demonstrated, that the knowledge
reflecting software processes can be gathered from public systems \cite{citeulike:12550640}. 
In their later work, they had show, that it is possible to reconstruct FLOSS processes by manual mapping of 
collected process evidence to a pre-defined process meta-model \cite{citeulike:5043664} \cite{citeulike:5128808}. 
Another closely related to my research work is by Hindle et al. where they had shown that it is possible to 
discover software process patterns through partitioning of the observed activities \cite{citeulike:10377366},
and recurrent behaviors by Fourier analysis of source code change records \cite{citeulike:10377345}.

However, the mentioned above and other research work based on mining of public software process artifacts shows, 
that while public availability of software process artifacts minimizes cost of the observation and eliminates privacy concerns, 
the nature of public artifacts creates a number of new challenges which limit the scope of the research and 
significantly elevate its complexity, effectively rendering many of previously developed process research techniques 
inefficient. 
For example, the coarse granularity of public software change records hides most of the low-level development activities
such as small code edits, unit-test runs, etc., which invalidates the application of many previously developed process 
mining techniques \cite{citeulike:10377366} \cite{citeulike:2678511}.
Similarly, the artifacts duplication due to concurrent and often overlapping processes, as well as the incompleteness of 
public artifact trails prevent typically deterministic process discovery techniques from producing of consistent 
results \cite{citeulike:2678511}. Thus, novel software process analysis and discovery techniques are needed to be 
developed for public software process artifacts analysis \cite{citeulike:7853299}.

%
% >> section
%
\section{Software Trajectory Analysis}\label{section_software_trajectory}
In addition to the establishment of Engineering-like software development paradigm, the acknowledgement of 
the software crisis led to the development of similar to Engineering project management techniques based on 
software measurements.
% The software measurement is also essential in software engineering research.

\subsection{Software measurement}
The goal of software measurement is to make objective judgments about software process and product quality. 
It has been shown that an effective measurement programs help organizations understand their capacities and 
capabilities, so that they can develop achievable plans for producing and delivering of software products. 
Furthermore, a continuous measurement effort provides an effective foundation for managing process 
improvement activities, 
such as CMM \cite{citeulike:9962021}, 
PSP \cite{citeulike:8347315}, \cite{citeulike:5090131} \cite{citeulike:12929216}, 
ISO 9001 \cite{iso-standard}, and SPICE \cite{spice-standard}.

In addition to practical applications, software measurement is extensively used in the research --
it is the basis of Empirical Software Engineering research area where researchers base their conclusions on 
concrete evidence collected through experimentation and measurement of software systems and software 
processes \cite{citeulike:766768}.

\subsection{Software telemetry}\label{section_software_telemetry}
Ideally, by using measurements, a software process and product can be assessed in real-time allowing efficient 
in-process decision making.
Johnson et al. \cite{citeulike:557296}, pioneered this approach by defining software project telemetry as a 
particular style of software process and product metrics collection and analysis based on 
\textit{automated measurements over a specified time interval}. 
The authors hypothesized, that the visualization of multiple streams of collected measurements, 
which are effectively \textit{equidistant time series}, captures the project and software 
process state evolution and conveys its dynamics to the user.
They implemented an in-process software engineering measurement and analysis system called Hackystat 
\cite{citeulike:12929227} that is capable of metrics collection, processing, and telemetry streams visualization. 
The system's empirical evaluation showed, that the visual analysis of multiple telemetry streams aids 
in the in-process decision making, and it is possible to improve existing software processes by using 
the knowledge extracted by visual analysis of these streams. 
At the same time, the authors acknowledged, that it is impossible to extract a traditional analytical 
model that is capable to automate the decision making process and that machine learning application is 
desirable.

Later, Kou et al. extended Hackystat by implementing Software Development Stream Analysis Framework (SDSA) 
that is capable of partitioning of telemetry streams into sequences of development ``episodes'' by using 
pre-defined boundary conditions \cite{citeulike:6180831} \cite{citeulike:11538873}.
By designing ``operational definitions'' for test-driven development (TDD) as sets of specific rules for 
the development episodes, they showed that it is possible to characterize and to assign TDD compliance to 
individual software development episodes.
They implemented their approach in Zorro, that is a software system capable of software process measurement, 
development episodes inference, their categorization, and classification by the TDD conformance. 
As Zorro is based on pre-defined partitioning and classification rules reflecting our understanding of 
TDD processes, the authors acknowledged that the application of machine learning techniques may improve 
systems performance and advance our understanding of software processes.

%
% >> section
%
\subsection{Knowledge discovery from time series}
Both demands for machine learning methods application to the problem of software measurements analysis
identified in the previous section can be potentially met by the techniques developed in the research 
area concerned with unsupervised and semi-supervised knowledge discovery from time series.
There, time series are typically used as a proxy representing a large variety of real-life 
phenomena in a wide range of fields including, but not limited to physics, medicine, meteorology, music, 
motion capture, image recognition, signal processing, and text mining \cite{citeulike:11796594}. 
While time series usually represent observed phenomenas directly by recording their measurable 
progression in time, the pseudo time series often used for representation of various high-dimensional 
data by combining data points into ordered sequences. 
For example in spectrography data values are ordered by the component wavelengths \cite{citeulike:12550833},
in shape analysis the order is the clockwise walk direction starting from a specific point in the outline 
\cite{citeulike:12550835}, in image classification the this is the frequency of pixels sorted by color component 
values \cite{citeulike:2900542}.

Many important problems of knowledge discovery from time series reduce to the core task of finding 
characteristic, likely to be repeated, short sub-sequences that efficiently capture the studied 
phenomena specificity. In the early work these were called as 
\textit{frequent patterns} \cite{citeulike:5159615}, 
\textit{approximate periodic patterns} \cite{citeulike:1959582},
\textit{primitive shapes} \cite{citeulike:5898869}, 
\textit{class prototypes} \cite{citeulike:4406444}, 
or \textit{understandable patterns} \cite{citeulike:3978076}. 
Later, similarly to Bioinformatics, these were unified by the term \textit{motif} \cite{citeulike:3977965}.
Once discovered, time series motifs can be used for the research hypothesis generation by their association 
with known or proposed phenomenas \cite{citeulike:3977965}. 
The recent advances in finding of time series motif and in particular work based on \textit{shapelets} 
\cite{citeulike:7344347} \cite{citeulike:11957982} \cite{citeulike:12552293} and \textit{bag of patterns} 
\cite{citeulike:10525778} show the great potential of time series motif-based data mining application
to almost any phenomena that can be represented as time series.

Potentially, since software telemetry streams are time series representing the evolution of a software system 
and a software process in time, their motifs can be discovered and associated with sensible product and process characteristics.

%
% >> section
%
\subsection{Research hypothesis}\label{section_research_hypothesis}
In previous sections, I have outlined the evidence for a limited performance of traditional Engineering-like 
software development as well as the oversight by traditional approaches to software process design and improvement
of a variety of human factors that fall beyond the typical sequence of the development actions.
In contrast, I have identified a number of key differences of FLOSS software development that 
foster developer- and user-centric processes and which, if systematically studied, can potentially shed light on 
the role of human-driven aspects in software development and to improve our overall understanding of 
software processes. 
Then, I have pointed out a growing wealth of publicly available software process artifacts that enables 
systematic FLOSS processes analyses and highlighted the need for novel techniques capable of mining these datasets.
Finally, I have outlined the possibility of knowledge discovery by time series mining techniques application 
to software measurements.

All these, along with the results of previous research that has shown that it possible to discover 
recurrent behaviors on all levels of software development process hierarchy \cite{citeulike:8347315} 
in industrial \cite{citeulike:5090131} and open-source \cite{citeulike:10377345} settings, 
prompted a research hypothesis, that \textit{it is possible to discover the basic blocks of software 
processes - recurrent behaviors - from public software process artifacts}. 

\subsection{Software Trajectory Analysis (STA)}
Following the hypothesis, I have defined a Software Trajectory - an abstract representation of software
product and process evolution. As the term trajectory is used in Physiscs for an approximate path that the 
moving object draws in a physical space, or in Mathematics, where trajectory defined as the reduced in 
complexity sequence of states of a dynamic system (a Poincare' map), \textit{Software Trajectory is a curve
that only approximately describes the path drawn by an evolving software system or by an ongoing software 
process in the chosen metric space}. The analytical technique based on software trajectory construction
and its analysis I have called Software Trajectory Analysis (STA).

In a preliminary pilot study targeting the possibility of characteristic subsequences discovery from 
software telemetry streams, I have added an analytical module based on characteristic patterns mining 
to Hackystat system. This early STA implementation exploited the transformation of real-valued telemetry 
streams into short overlapping symbolic sequences with Symbolic Aggregate approXimation (SAX) \cite{sax} 
and their consequent frequency-based ranking. 
While the pilot STA implementation required the user to specify a number of non-intuitive parameters for 
SAX transform and a threshold for the pattern discrimination, some of the discovered frequent patterns 
were easily associated with characteristic recurrent software development behaviors, such as consistent 
effort or frequent testing, and the system performance was found satisfactory \cite{csdl2-10-09}.
Later, the system was improved by the addition of borrowed from Bioinformatics symbolic motif-mining and 
visualization algorithms, which not only made frequent patterns discovery subsystem more efficient, 
but aided in their comprehension through intuitive visualization. 

However, when the system was applied to time series built by measurement of public software artifacts, 
its performance significantly deteriorated affected by their coarse granularity, poor informational content,
noise, and the significant amount of the lost values. 

Addressing the identified data-mining techniques limitations, 
I have developed a novel unsupervised technique for time series classification called SAX-VSM, that enables 
discovery and ranking of class-characteristic patterns, requires no input parameters, and is rotation-invariant 
and robust to the noise and lost values \cite{sax-vsm}. 
In turn, as I shall show later, the powered by SAX-VSM algorithm STA implementation is capable to discover 
sensible characteristic subsequences from wide variety of software process artifacts.

Taking in account all of the above, the Software Trajectory Analysis is an automated systematic approach to 
recurrent behaviors discovery based on software artifacts measurements and mining. 
In contrast with previously proposed systems, that were built upon quantitative analyses of atomic development 
entities such as actions or episodes, or were relying on some pre-defined reference process models, 
STA focuses on the unsupervised discovery of naturally occurring phenomena - recurrent behaviors. 

By the design, Software Trajectory addresses a number of known issues that previously complicated and limited 
large scale studies on software processes.
First of all, Software Trajectory removes all in-process (real-time) measurement costs and privacy concerns since 
it relies on off-line measurements of public software process artifacts. 
Secondly, Software Trajectory does not depend on any prior knowledge about software processes or any model - 
unsupervised data mining techniques intended to be used in order to bootstrap knowledge by extracting of data 
summaries. 


\section{Contributions}\label{section_contributions}
My contributions include Software Trajectory Analysis (STA) approach for recurrent behaviors discovery
from software process artifact trails, the SAX-VSM algorithm for interpretable time series classification 
that powers-up STA, and their empirical evaluations: 

\begin{enumerate}

\item \textbf{Software Trajectory Analysis}

The inherent complexity and longevity of software development processes makes their studies in real time
expensive and challenging, especially at the large scale. 
In addition, the contemporary practices of highly distributed software development, that usually allow 
the significant variation in software processes, demand for new analytical techniques.

In this work I propose STA - a software process analysis technique that targets the off-line discovery 
of recurrent behaviors through the analysis of software process artifacts. 
STA consists of two steps. 
At first, it exploits software artifact measurements for the abstraction of software development 
progression as a trajectory in the chosen metrics space. 
At the second step, by application of data-mining techniques, STA finds trajectory's characteristic 
patterns which potentially correspond to recurrent behaviors and thus enable the understanding of a 
performed software processes.

\item \textbf{Interpretable time series classification with SAX-VSM}

In order to improve STA performance, I have developed a novel algorithm for interpretable time 
series classification called SAX-VSM which I present in this thesis. 
This algorithm addresses two core problems in time-series classification: 
the characteristic feature selection and the classification results interpretation. 

SAX-VSM automatically discovers and ranks time series patterns by their
class-characteristic power, 
which not only facilitates time-series classification, but provides an interpretable class
generalization.

These algorithm's strengths are essential for STA performance - they facilitate unsupervised characteristic 
patterns discovery from software trajectories and convey the understanding of performed software processes 
by association of patterns with recurrent behaviors.

\item \textbf{Empirical evaluations}

In order to assess the performance of both proposed techniques I conducted their empirical evaluation and 
present next results in this thesis:
\begin{enumerate}
 \item The experimental evaluation of SAX-VSM classification accuracy on a set of 45 classic time series 
classification problems. It shows that the proposed algorithm is competitive with, or superior
to, other 
techniques.

 \item The empirical evaluation of SAX-VSM capacity to discover class-characteristic patterns.
This study highlights advantages of the proposed algorithm over existing techniques 
emphasizing
its capacity to discover and rank short time series subsequences by their class 
characterization power and 
shows the possibility of meaningful interpretation of classification results.
 \item The results of use case-based empirical evaluation of STA capacity to discover useful recurrent 
behaviors, which include 
(i) the ``commit-fest''-related recurrent behaviors from PostgreSQL software development process; 
(ii) the software release-related recurrent behaviors from Android OS software development process;
(iii) the characteristic activity patterns of top StackOverflow contributors.
\end{enumerate}

\end{enumerate}

\section{Dissertation Outline}\label{section_organization}
The rest of this dissertation is organized as follows. 
Chapter 2 discusses relevant work from time series classification and knowledge discovery and proposes SXA-VSM algorithm.
Chapter 3 discusses related work from software process process discovery and software repository mining areas.
Chapter 4 proposes Software Trajectory analysis framework design, explains its implementation, and presents results of its empirical evaluation. 
Chapter 5 concludes and discusses several directions for future study.