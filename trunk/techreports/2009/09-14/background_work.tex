\chapter{Prior and related work}
My work is based on previous research from multiple research areas: software engineering, 
software repository mining, process mining, knowledge discovery from time series,
and information retrieval. 

By extending existing knowldge about software processes and repository mining and combining 
it with a novel technique for knowledge discovery from time series, 
I was able to develop an automated framework for recurrent behaviors discovery from publicly 
available software repositories.

In this chapter I will review previous work relevant to my research.

\section{Engineering in software development}
Since the invention of Charles Babbage’s difference engine in 1822, computers (hardware) 
have required a means of instructing them to perform a specific task. This means is known 
as a programming language. By using programming instructions, software develeopers (programmers)
write the software that is collections of computer data and instructions, which is used by 
users to accomplish specific tasks. The very first software, or more precisely a set of 
instructions implementing a method for calculating a sequence of Bernoulli numbers using 
Charles Babbage's Analytical Engine, was designed by Ada Lovelace in 1942 and predates modern 
computers. 

As time has progressed, computers have made giant leaps in processing power and become 
general-purpose computing devices. Consequently, the variety of software and 
the complexity of its development has grown ``several orders of magnitude'' \cite{naur_crisis_68}. 
This effectively created a number of organizational problems which dominated development of 
early software systems. Among these, historians are mentioning increasing scale of software projects, 
excessive ambition, difficulties in estimating project cost and length, inadequate documentation, 
skimping on design and testing \cite{mahoney_roots_1990} \cite{citeulike:12748733} 
\cite{citeulike:833903}.

All these problems were recognized at the NATO Conference on Software Engineering, held in Garmisch,
Germany in 1968. It worth noting, that before the actual conference, as was pointed by the 
conference proceedings editors, engineering paradigm was chosen without actual plan of actions - 
\textit{``the phrase ``software engineering'' was deliberately chosen as being provocative, in implying 
the need for software manufacture to be [based] on the types of theoretical foundations and practical 
disciplines[,] that are traditional in the established branches of engineering''} \cite{citeulike:12787786}.
At the conference, it was acknowledged, that the increasing complexity of programming 
work had overwhelmed the technical and managerial ability of software teams: software was shipped late,
over budget, worked inefficiently, and was unreliable. Later, while working on the proceedings, the term 
``software crisis'' was coined by the editors in order to describe the desperate state of 
software development and its increasing complexity \cite{naur_crisis_68}.

Nevertheless, the ``software engineering'' idea was provocative enough to shape the minds of researchers 
and practitioners. At large, it was accepted that the software can be successfully ``manufactured'' 
through ``engineering'' - i.e. with the application of a systematic, disciplined, quantifiable approach 
to the design, development, operation, and maintenance. 

Using this paradigm, researchers and practitioners designed a number of software development processes 
providing detailed guidelines on how to reach the goal - deliver the software - efficiently and in time. 
These processes were declared as the means for improvements in terms of quality, speed, and execution cost 
over existing practices. They were studied within academic settings and adopted by industry. 
Some of this processes were further standardized, shaping the best practices of contemporary 
software development \cite{citeulike:9962021}. 

The processes, that I am addressing here, can be found in numerous studies and described at length 
in industrial and academic literature. Too numerous to name, they can be categorized by the level 
of their application:
\begin{itemize}
 \item global organizational standards, like CMMI, ISO 9000, and SPICE. 
 \item large industrial software models, such as Waterfall, Spiral, Cleanroom, etc.
 \item smaller scale, flexible methodologies: FDD, TDD, Pair Programming, etc.
 \item general guidelines and principles aiming to improve the software process flow, 
such as ``build before commit'' and others.
 \item and finally, the processes for improving existing processes of software development 
on the organizational, team, and personal levels: TSP, PSP, Six Sigma, etc.
\end{itemize}
While some of these are products of a ``technology transfer'' - resulted by direct copying or by an 
application of existing engineering principles to software development process (like Waterfall model), 
some, and in particular agile methods, considered to be particular to software-engineering. 

\subsection{Issues with engineering-like processes}
Currently, in Software Engineering, there are numerous software process, models, methodologies, 
and coding conventions exist for any level and any stage of the software development carried out by 
any team size at any project scale. 
This amount of well-grounded and documented knowledge creates an impression, that the area of 
software development process is thoroughly explored, and that there exists a deterministic choice 
for a model and a processes for any given software project leaving almost no doubts in 
its faith - it will be completed according to the specification, in time, and under the budget.

However, it is far from being true - software projects do fail at the considerably high rate and 
many challenges exist in software product development. Many software projects continue to run over 
the budget, do not meet the schedule. Some of the seemingly proper executed projects caused property 
damage \cite{citeulike:11044022}, and a few projects caused loss of life \cite{citeulike:712058}. 
The ``Chaos Report'' from the Standish Group, that only ``35\% of software projects in 2006 can be 
categorized as successful - meaning they were completed on time, on budget and met 
user requirements'' \cite{chaos2006}. These thirty two percent of success clearly manifest that 
it is somewhat illusory statement that we are capable to understand and to control software processes.

This rather poor performance of the mainstream software engineering processes 
ignited a new wave of research in the field in recent decades. Gradually, researchers and 
practitioners alike recognized, that a direct application of classical engineering - that 
excludes a ``human component'' by governing software processes in a top-down fashion 
might be somewhat erroneous. 
Also, it was recognized that over many years, continuous formalization of the field and 
excessive attention to software process metrics may malformed not only the software development 
practices landscape, but education and a professional code of conduct. 
This recognition forced some of the professional associations to review their licensing policies 
\cite{citeulike:11045517} and some educators to change their opinions \cite{citeulike:5203446}. 

\section{Alternative software processes}
In parallel with industrial engineering-like software processes, another phenomena, 
the Free/Libre Open Source Software (FLOSS) development arose and proven its efficiency and effectiveness.
The FLOSS processes are significantly different from typical SE process on many levels:
\begin{itemize}
 \item First of all, FLOSS processes are highly geographically distributed, which significantly contradicts 
 to developers collocation usually assumed by the most of the SE processes. 
 \item Secondly, open-source processes do not explicitly define timeline, developers roles, and 
 the process itself. Typically, the project flow is rarely governed and does not require any sort of 
 documented software process. Most of FLOSS projects accept for review and embed source code from anyone.
 \item Finally, FLOSS delivers software free of charge, promoting its reuse, modification, and redistribution.
 However, mostly due to this, the maintenance and the support are rarely provided.
\end{itemize}

Nevertheless, despite to the apparent lack of the control over the software processes in FLOSS projects, they 
has proven their ability to deliver large in size software systems of high quality for the fraction of cost 
of similar industrial projects.

Most of the success is contributed to the motivation of open source developers.

Lately, third approach to software development emerged - software craftsmanship emerged 
\cite{citeulike:11058561}, \cite{citeulike:11058554}. The followers of this methodology 
are not only focused on the delivering ``well crafted'' software and continuously adding value,
but on the apprenticeship - on forming communities and engaging more people in software development.
While recommendation exists \cite{citeulike:11058784}, little known about the research 
in software craftsmanship and apprenticeship processes.

All this continues to engage thoughts and fuels the search not just for the exact definition 
of software development, but deep, fundamental for efficient processes resolving. 
Is software development an engineering discipline? Is it a craft \cite{citeulike:5203446}? 
Or is it an art \cite{citeulike:11045694}?

Answers to these questions would require extensive interdisciplinary studies to be made, 
but what is obviously clear, is that formal software development process, as an engineering 
is a creative, human activity. 
Whether software is coded 
by team where its members have a variety of skills and experience, or by a single individual,
they all driven by their believes and motivations. While usually developers agree on the use of 
particular technologies, development tools, and a development process with imposed timeline and 
a budget, the software process is - as many other human activities - highly creative and mostly 
non-recurring. Thus, the choice of methodology, technology, or tools provides only a marginal 
effect on this human-based and human-driven process. While this effect can be measured through 
the common approach by using a control environment factoring out human component, the most of 
the difference, which lies in human creativity, motivation and productivity is unknown and yet, 
mostly immeasurable.

\section{Mining software repositories}

\section{Knowledge discovery in time series}
