\chapter{Software process artifacts}
Software process discovery relies on software related artifacts collected from public software repositories. 

\section{Mining Software Repositories}
\textit{Mining software repositories} refers to the research field concerned with knowledge discovery from 
software repositories such as version control, mailing lists, bug trackers, wikis, etc. 
A comprehensive survey and a taxonomy of research questions and common techniques provided by 
Kagdi et al. \cite{citeulike:4534888}. Common research questions are: software visualization, 
code quality assessment, project evolution, change co-occurrence, which changes induce bugs or provide fixes, 
etc. The process recovery and enactment is also a  sub-field of MSR. 

Most of the work in MSR is done using publicly available open-source repositories and it was mentioned before, open-source community Much of the software and the software repositories that are mined in MSR literature is Open Source
Software (OSS), this is due to the availability of OSS. Since OSS development artifacts are typically freely
available, we can have repeated studies on the same corpus, thus allowing researchers to validate their results
and the results of others, and to compare dierent approaches easily. Some researchers have studied and
measured the general characteristics of Open Source Projects [14]. Mockus et al. [94] described the underlying
methodology of OSS projects such as Apache and Mozilla. Much of the information they used was gleaned
by mining the source control repositories of these projects. Thus much MSR research is executed on OSS.