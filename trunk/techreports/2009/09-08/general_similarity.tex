\chapter{Time-series similarity measurements}
Though not present in classical time-series analyses such as trend and seasonality identification, forecasting and others, the time-series similarity \footnote{Note that a time-series (a finite sequence of real numbers) is alternatively called ``sequence'' or ``signal'' in the literature and we might use these terms depending on the context of the reviewed paper or method.}  problem has become a cornerstone problem for recent data-mining applications. The ability to determine time-series similarity addresses many general time-series KDD problems:
\begin{itemize}
	\item identifying stocks with similar movement in price \cite{citeulike:4384535} \cite{citeulike:4031865} \cite{citeulike:4025073}
	\item finding products with similar selling pattern \cite{citeulike:4326324}
	\item identifying music with a score similar to the copyrighted one \cite{citeulike:3821484} \cite{citeulike:3815076}
	\item performing speech to text conversion \cite{citeulike:3728228}
	\item verifying signatures and performing handwriting to text conversion \cite{citeulike:3733947} \cite{citeulike:3513035}
	\item finding objects with similar movement trajectories \cite{citeulike:964832} \cite{citeulike:3728229} \cite{citeulike:3815864}
	\item detecting anomalies or events in signals \cite{citeulike:4412621} \cite{citeulike:4412617}
	\item finding developers with similar build patterns.
\end{itemize}

All of the solutions for this set of problems are based on the implementation of a time-series database enhanced with the ability to process ``time-series similarity queries'' or ``range queries''. 
%%At this point we essentially regard this problem as solved and will overlook some of the solutions in this review. 

First introduced by Goldin and Kanellakis \cite{citeulike:3815880}, the concept of the ``approximately similar time-series data'' is based on ``user-perceived similarity'' rather than direct mathematical comparison with straightforward metrics. This very important observation essentially relaxes constraints for finding similar time-series. As an example of this relaxation, authors introduce scales and shifts in order to simulate human ability in comparing time-series features. Naming scale as the $a$ coefficient and shift as $b$ they introduced a form of the ``similarity relation'' $T_{a,b}$ with properties:
\begin{align}
 & \text{1. } \forall \; X, \; X=T_{0,1}(X), \; \text{Reflexivity or identity transformation} \\
 & \text{2. } \text{if } X=T_{a,b}(Y) \; \text{then } Y=T_{\frac{1}{a},-\frac{b}{a}}(X) = T^{-1}_{a,b}(X), \; \text{Symmetry or inverse of $T_{a,b}$} \\
 & \text{3. } \text{if } X=T_{a,b}(Y) \; \text{and} \; Y=T_{c,d}(Z), \; \text{then } X=T_{zc, ad+b}(Z) = (T_{a,b} * T_{c,d})(Z), \; \text{Transitivity}
\end{align}
It is natural to quantify this ``similarity'' by assigning a cost coefficient to each of the transformations. We will return to scale and shift transformations along with smoothing and matching envelope techniques in greater detail in the section \ref{scales_and_shifts}. For now we focus on more general discussion of similarity and distance metrics properties.

Traditionally \cite{citeulike:3973409}, time-series similarity queries are divided into two categories: whole sequence matching and subsequence matching where both categories are divided by first-occurrence and all-occurrences subproblems \cite{citeulike:3815880}. The Range Query problem of finding all (sub)sequences similar to a given one within the distance $\epsilon$ is specific for this domain as well as the All-Pairs Query (``spatial joint'') of finding all pairs of (sub)sequences that are within $\epsilon$ from each other \cite{citeulike:3973409}. While whole matching requires time-series to be exactly of the same length, subsequence matching considers the shorter query time-series for which it finds the best match in the longer template time-series using a sliding-window approach and reduces the task to the whole-sequence problem in each individual comparison. In both cases the similarity query mechanism relies on metrics with a well-defined distance function quantifying the time-series similarity. 

The distance function on a set $X$ defined as:
\begin{equation}
 d: X \times X \rightarrow \mathbb{R}
\end{equation}

And if $x$, $y$ and $z$ $\in X$ the distance function $d$ required to satisfy following conditions:
\begin{align}
 & \text{1. } d(x, y) > 0, \; \text{non-negativity} \label{eq:d1} \\
 & \text{2. } d(x, y) = 0, \; \text{if and only if} \; x = y  \;  \text{identity} \\
 & \text{3. } d(x, y) = d(y, x), \; \text{symmetry} \\
 & \text{4. } d(x, z) \leq d(x, y) + d(y, z), \; \text{the triangle inequality} \label{eq:d4}
\end{align}
There are number of functions which can be used used in the time-series similarity domain and we discuss some of them: Euclidean distance in section \ref{euclidean_distance}, mentioned ``transformation rules'' in section \ref{scales_and_shifts}, Dynamic Time Warping (DTW) in section \ref{dtw} and Longest Common Subsequence (LCS) in section \ref{lcs}.
