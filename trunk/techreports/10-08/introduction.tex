\chapter{Introduction}

In this dissertation, I describe the Kukui Cup research project, which focuses on how to use information technology to educate individuals about energy and foster sustained, positive changes in energy use behavior. To investigate our approach to energy education and sustained behavior change, we held the 2011 Kukui Cup: a student housing energy competition held at the University of \Hawaii at \Manoa that combined education, game mechanics, real-time feedback, and incentives. In this chapter I explain the motivation for the research, briefly describe the system, and the results we obtained from the evaluation of the system.

The Kukui Cup project is large and could not have been accomplished by one person. Research from the project is the subject of this Ph.D. dissertation, a Computer Science Master's thesis, and two additional Ph.D. dissertation proposals. The 2011 Kukui Cup team had ten members, and the planning and execution of the competition required extensive collaboration with UH \Manoa Student Housing. Therefore, in this dissertation I use the word ``we'' to describe work or results that were obtained as part of the collaboration of the Kukui Cup team (of which I was a very active core member), and I use the word ``I'' for work that was mine alone.


\section{Motivation}

The economies of the industrialized nations of the world run on fossil fuels: oil, coal, and natural gas. The worldwide demand for energy is increasing as more countries industrialize. However, the use of fossil fuels for energy must be curtailed for a variety of reasons:

\begin{itemize}
	\item Fossil fuels are a finite resource that will eventually run out. As supplies are drained, they will become increasingly expensive due to the costs of extraction~\cite{Murray2012Climate-policy}.
	\item Many countries (such as the United States) use much more fossil fuel than they can extract from domestic sources, so they must rely on imported fuels. This leads to \emph{energy insecurity}.
	\item The extraction of fossil fuels can be very damaging to the environment, such as the practice of mountaintop removal to extract coal deposits.
	\item Burning fossil fuels results in serious environmental impacts including climate change~\cite{IPCC-synthesis-report-2007} from greenhouse gas (GHG) emissions and other air pollution.
\end{itemize}

We must develop renewable energy sources that can replace the current fossil fuel sources. However, the switch to renewable energy can be eased by reducing energy use, which I discuss in the next section.


\section{Energy Conservation}

One way fossil fuel use can be decreased is by decreasing the total amount of energy consumed. Socolow and Pacala have proposed a plan for reducing global GHG emissions to acceptable levels through the implementation of a series of `wedges', where each wedge represents a reduction of 25 billion tons of \COtwo emissions over 50 years~\cite{Socolow2008}. One of the 15 wedges they proposed is to cut electricity use in homes, offices, and stores by 25\%.

On a local level, the state of Hawaii has created the Hawaii Clean Energy Initiative, which seeks to reduce Hawaii's fossil fuel use by 70\% by 2030 through increasing the use local, renewable energy sources (for electricity and transportation fuel) to 40\% of demand and reducing demand by 30\% through efficiency and conservation~\cite{HCEI-website}.

Amory Lovins coined the term \emph{negawatt} to refer to power that has been conserved, and therefore, does not need to be generated~\cite{Kolbert2007Mr-Green}. Negawatts can be `generated' in two basic ways: by increasing the efficiency of devices that consume energy, and by changing people's behavior to reduce energy use.

Changing people's behavior with respect to energy holds significant promise in reducing energy use. Darby's survey of energy consumption research found that identical homes could differ in energy use by a factor of two or more~\cite{darby-review-2006}. Data from a military housing community on Oahu show energy usage for similar homes can differ by a factor of 4~\cite{Norton2010ZeroEnergyHomes}.

%One common way to attempt to foster behavior change is by providing information to the targeted population, often through mass media. While convenient, this approach often turns out to be ineffective~\cite{McKenzie-Mohr2009}. Two strategies that have proven to be effective are providing direct feedback on energy usage~\cite{darby-review-2006}, and a toolbox of techniques such as making public commitments and establishing social norms~\cite{McKenzie-Mohr2009}.


\section{Energy Literacy}

\emph{Energy literacy} is the understanding of energy concepts necessary to make informed decisions on energy use at both individual and societal levels. Increasing energy conservation is difficult when people do not understand energy fundamentals, and how energy is used in their homes and workplaces. At the level of public policy, there will be many decisions that will need to be made about exactly what path we choose to reduce fossil fuel use.

Some examples of energy literacy are:

\begin{itemize}
	\item Understanding the difference between power and energy (see \autoref{app:power-energy}).
	\item Knowing that a microwave uses much more power than a refrigerator, but that the refrigerator will use much more energy over time.
	\item Knowing how electricity is generated in one's community.
\end{itemize}

There is a proposed renewable energy project in Hawaii called ``Big Wind'' that would generate as much as 400\,MW from wind farms covering substantial portions of two more rural islands (Moloka`i and L\=ana`i) with excellent wind resources. The power would be transmitted via a new undersea cable to Oahu, which has the majority of the state's population, but inferior wind resources. There are advocates both for and against Big Wind, but to make an informed decision one should understand how Oahu's electricity is generated now, and the characteristics and challenges of wind energy.


\section{The 2011 Kukui Cup}

To investigate ways to foster energy conservation through behavior change and increase energy literacy, we designed and implemented the Kukui Cup. The competition is named after the kukui nut (also known as candlenut), which was burned by Native Hawaiians to provide light, making it an early form of stored energy in \Hawaii. The 2011 Kukui Cup took place in the Hale Aloha student residence halls on the University of \Hawaii at \Manoa campus.

Energy competitions have become increasingly common on college and university campuses. Buildings compete to see which one can use the least energy over some period of time, often with prizes for the winning residence hall. Unfortunately, there is some evidence that participants engage in unsustainable behaviors (such as keeping hallway lights off at night) in order to win the competition, but return to previous behaviors after the competition is over~\cite{petersen-dorm-energy-reduction}. In addition, most college competitions are intended primarily to raise awareness and conserve energy, not to conduct research on how best to achieve these goals.

The 2011 Kukui Cup took place over 3 weeks starting in October 2011 in four residence halls for first-year students on the UHM campus containing a total of approximately 1070 residents. Each residence hall is further broken into five pairs of floors, each of which share a common lounge area. These pairs of floors are referred to as \emph{lounges}, and were the team unit in the competition.

The Kukui Cup combines four important features in an effort to foster energy conservation and increase energy literacy:

\begin{itemize}
	\item Near real-time feedback on energy use in each lounge. A variety of research has shown that energy feedback can be helpful in fostering energy conservation (see \autoref{sec:energy-feedback}).
	\item Incentives in the form of prizes that can be won during the competition both through merit and through chance.
	\item Educational activities and events taking place both online and in the real world designed to increase energy literacy.
	\item A gameful design intended to engage participants and make the competition fun and worthwhile.
\end{itemize}

There were two parallel competitions during the Kukui Cup: an energy competition where lounges competed to use the least electricity, and a point competition where individuals competed to earn the most points by taking actions mediated by the competition website. To support the energy competition, we installed energy meters on each floor of each building, allowing us to record power and energy data every 15 seconds. The competition website provided each participant with a personalized data such as his or her floor's power usage in near-realtime, their floor's cumulative energy usage for the competition, and their floor's ranking in the competition.

The other major feature of the competition website is to provide participants the ability to take a variety of educational actions to earn points. The actions are designed to either increase the energy literacy of the participant, or help reduce the energy consumption of the lounge, or both. The actions are divided into three different categories: activities, commitments, and events. Activities are one-time actions that are verified through the website, such as watching a short YouTube video about energy intuition and correctly answering a question about the content. Commitments are intentions to conserve energy in the future in ways that cannot be verified through the website, such as turning off the lights when leaving a room. Events are scheduled, real world gatherings intended to increase energy literacy, such as a tour of a wind farm.

Associated with each action is a number of points. When a participant performs a action, such as determining the amount of power each device in their room consumes, they can submit information on the website verifying their completion of the action. In the case of the power audit, the information might be the list of devices in their room and the power consumption of each device. Once a website administrator verifies the information, the participant is awarded the points assigned to the completed action.


\section{Research Questions}

The research questions that I have investigated are:

\begin{itemize}
	\item \emph{To what extent was the website adopted by the participants?} Without significant adoption, it is hard to evaluate the other website-related questions.
	\item \emph{How did energy use change during the competition?} This is the standard measure for an energy competition, with the expected result being energy conservation during the competition.
	\item \emph{How did energy use change after the competition?} Understanding changes in energy use after the competition is over gives insight into whether changes during the competition were sustainable. Existing research focuses primarily on the competition itself, not examining the reasons why energy usage might rebound after the competition is over.
	\item \emph{How effective were the actions available via the website?} The actions participants undertook can be tracked using website log data and compared to changes in their energy literacy.
	\item \emph{How appropriate were the point values assigned to actions?} The points assigned to actions are intended to motivate participants to perform the actions, but the values were assigned without any participant data.
	\item \emph{What is the relationship between energy literacy and energy usage?} The hypothesis is that more energy literate participants will conserve more energy.
	\item \emph{How important was floor-level near-realtime feedback?} There are good reasons to believe that floor-level near-realtime feedback will lead to increased energy conservation, but they greatly increase the competition budget and logistical complexity. Is the trade-off worth it?
\end{itemize}


\section{Evaluation}
We have gathered four primary sources of data to examine the research questions:

\begin{itemize}
	\item power and energy data from each floor,
	\item detailed event logs from participant actions on the website,
	\item participants' performance on an energy literacy survey administered before and after the competition,
	\item and a survey on the competition as a whole administered during the final round of the competition.
\end{itemize}


\section{Results}

\fxnote{To be written after results chapter is finished}


\section{Contributions}

\fxnote{To be written after results chapter is finished}


\section{Outline}

This dissertation is organized into the following chapters:

\begin{itemize}
	\item \autoref{cha:related-work} looks at related research, including student housing energy competitions, energy feedback, and psychological techniques for fostering behavior change.
	\item \autoref{cha:system-description} describes all aspects of the 2011 Kukui Cup, including the competition, the energy data collection, and the competition web application.
	\item \autoref{cha:experimental-design} lists my research questions and explains how I have evaluated them.
	\item \autoref{cha:results} describes the results from the 2011 Kukui Cup and the evaluation of my research questions.
	\item \autoref{cha:conclusion} concludes with a list of contributions of this research and future directions.
	\item \autoref{app:power-energy} covers the definitions of power and energy, and their interrelationship. Understanding these two concepts is critical to understanding the evaluation (and an important part of energy literacy).
	\item \autoref{app:actions} lists the set of actions made available to participants through the website to improve their energy literacy.
	\item \autoref{app:energy-literacy} provides the \Hawaii-specific questionnaire designed to assess the energy literacy of participants.
	\item \autoref{app:round-3-survey} contains the questionnaire made available to participants during the final week of the competition.
\end{itemize}
