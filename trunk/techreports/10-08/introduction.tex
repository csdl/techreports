\chapter{Introduction}

The world is in the grip of an energy crisis. Fossil fuels (oil, natural gas, and coal) form the foundation of the world economy and their use is largely responsible for the industrialization and standard of living increases across the globe in the past century. However, the consumption of fossil fuels has led to a variety of problems that will have severe impacts on our environment and national economies.

There is no `silver bullet' that will solve this energy crisis, it will require a series of changes in production, transmission, and consumption of energy taking place over decades. While we will need to switch to renewable energy sources, energy conservation is also an important strategy since a reduction in energy demand makes the transition to renewable sources easier. This research examines how to motivate people to conserve electricity by changing their behavior in the context of a university dormitory energy competition.


\section{Climate Change}
\label{sec:motivation}

The primary motivation for reducing fossil fuel use is climate change. In 2007, the Intergovernmental Panel on Climate Change (IPCC) released its fourth assessment report \cite{IPCC-synthesis-report-2007}. The conclusions of this long-running analysis of studies on climate change and its effects are widely accepted as the consensus of the world's scientific community. They found that there is broad agreement that the climate is warming: air and ocean temperatures are higher, snow and ice are melting, and sea levels are rising. Further, natural systems are being affected: plant and animal ranges are moving towards the poles, and there are changes in fish and algae due to rising ocean temperatures.

The IPCC found that the warming of the climate was very likely due to anthropogenic greenhouse gas (GHG) emissions. GHG emissions from humans have increased by 70\% between 1970 and 2004. While there are a variety of GHG that impact climate change, \COtwo is the most important of the human-caused GHGs. Sea level rise in the second half of the 20th century was also very likely caused by humans, and rising sea levels have a potentially enormous impact on island communities like \Hawaii.

The IPCC found that with current climate change policies, GHG emissions are projected to continue to increase this century. Further, there is no single technology that will mitigate the problem of climate change; a range of policies and innovations is required. The report lists both energy efficiency and individual behavior modification as suggested GHG mitigation strategies.

%Other motivations for reducing or eliminating fossil fuel use are that fossil fuels are a finite resource that will eventually be exhausted. There is substantial evidence that the world has reached or will soon reach the peak of its oil production. After the peak, production will steadily decrease and production costs will increase as the remaining oil is more difficult to extract. When decreasing production is paired with skyrocketing demand, the result will be shortages and huge spikes in the price of oil. As discussed earlier, the industrialized world is highly dependent on oil, so large increases in the price of oil are expected to lead to widespread economic and social destabilization. This whole phenomenon is referred to as `peak oil'.
%Energy security
\fxnote{add brief discussion of peak oil and energy insecurity, with references}


\section{Energy Conservation}

One way fossil fuel use can be decreased is by decreasing the total amount of energy consumed. Socolow and Pacala have proposed a plan for reducing global GHG emissions to acceptable levels through the implementation of a series of `wedges', where each wedge represents a reduction of 25 billion tons of \COtwo emissions over 50 years \cite{Socolow2008}. One of the 15 wedges they proposed is to cut electricity use in homes, offices, and stores by 25\%. On a local level, the state of \Hawaii has created the \Hawaii Clean Energy Initiative, which seeks to reduce \Hawaii's fossil fuel use by 70\% by 2030 through increasing the use local energy sources (for electricity and transportation fuel) to 40\% of demand and reducing demand by 30\% through efficiency and conservation \cite{HCEI-website}.

Amory Lovins coined the term \emph{negawatt} to refer to power that has been conserved, and therefore, does not need to be generated \cite{Kolbert2007Mr-Green}. Negawatts can be `generated' in two basic ways: by increasing the efficiency of devices that consume energy, and by changing people's behavior reduce energy use.

\subsection{Energy Efficiency}

Many energy consuming devices have become more efficient over time. For example, incandescent light bulbs are increasingly being replaced with compact fluorescent bulbs that produce an equivalent amount of light but use only 20--30\% of the energy. The negawatts generated through use of more energy efficient devices have the primary advantage of not changing the functionality of the device: a more efficient refrigerator keeps food cold just as well as a less efficient one.

While energy efficiency holds significant potential for reducing energy demand, it usually involves replacement of energy-consuming devices. This involves the cost of the new device, and the environmental cost of disposing of the old device, which means that efficiency upgrades often make most economic sense when the old device needed to be replaced for other reasons such as age or wear. There are also many environments, such as offices and rental housing, where the occupants have little control over the energy-consuming devices that are used.

\subsection{Behavior Change}

Changing people's behavior with respect to energy holds significant promise in reducing energy use. Darby's survey of energy consumption research found that identical homes could differ in energy use by a factor of two or more \cite{darby-review-2006}. Data from a military housing community on Oahu show energy usage for similar homes can differ by a factor of 4 \cite{Norton2010ZeroEnergyHomes}.

One common way to attempt fostering behavior change is by providing information to the targeted population, often through mass media. While convenient, this approach often turns out to be ineffective \cite{McKenzie-Mohr2009}. Two strategies that have proven to be effective are providing direct feedback on energy usage \cite{darby-review-2006}, and a toolbox of techniques such as making public commitments and establishing social norms \cite{McKenzie-Mohr2009}. 


\section{Research Description}

This research project seeks to find ways to foster sustainable changes in behavior that lead to reduced energy usage. As discussed previously, changing behavior related to energy has the potential to be a major contribution towards reducing energy use. However, to be a significant contribution, these behavior changes must be sustainable in the long term.

\subsection{Setting}
This research is based around a dormitory energy competition at the University of \Hawaii at \Manoa. These types of competitions have become increasingly common on college and university campuses. Dorms compete to see which one can use the least energy over some period of time, often with prizes for the winning dorm. Unfortunately, there is some evidence that participants engage in unsustainable behaviors (such as keeping hallway lights off at night) in order to win the competition, but return to previous behaviors after the competition is over \cite{petersen-dorm-energy-reduction}.

\subsection{System}
The dorm energy competition will take place over 4 weeks in October 2010 in two freshman residence halls on the UHM campus. Power meters will be installed on each floor of each building and the power and energy data will be recorded every 10 to 15 seconds. Since each floor has its own meter, each floor will compete to have the lowest energy consumption during the competition.

A website is being built that will provide information about the competition to the participants. Participants will log into the website with their UH username and password, and each participant will see a personalized home page that displays data such as his or her floor's power usage in near-realtime, their floor's cumulative energy usage for the competition, and their floor's ranking in the competition. The website has been designed to take into account the research in environmental psychology about how to foster behavior change.

The other major feature of the competition website is to make a variety of tasks available to the participants. The tasks are designed to either increase the \emph{energy literacy} of the participant, or help reduce the energy consumption of the floor, or both. Energy literacy is composed of knowledge, positive attitudes, and behaviors related to energy. An example of energy knowledge would be the difference between a watt and a watt-hour, an example of a positive attitude would be ``Americans should conserve more energy'', and a positive behavior would be turning off lights when leaving a room \cite{DeWaters09c}. The tasks are divided into three different types: activities, commitments, and goals. 

Associated with each task is a number of points, called Kukui Nut points. When a participant performs a task, such as determining the amount of power each device in their room consumes, they can submit information on the website demonstrating their completion of the task. In the case of the power audit, the information might be the list of devices in their room and the power consumption of each device. Once a website administrator verifies the information, the participant is awarded the points assigned to the completed task. These website tasks create a second parallel competition to see which participants can accumulate the most points.

A variety of prizes will be provided both for the energy conservation side of the competition, and the point competition. This prize structure provides an additional motivation for the residents to participate in the competition.

\subsection{Research Questions}

The research focuses on descriptive and exploratory statistics based around research questions. The research questions that will be investigated are:

\begin{itemize}
	\item \emph{To what extent was the website adopted by the participants?} Without significant adoption, it is hard to evaluate the other website-related questions.
	\item \emph{How did energy use change during the competition?} This is the standard measure for an energy competition, with the expected result being energy conservation during the competition.
	\item \emph{How did energy use change after the competition?} Understanding changes in energy use after the competition is over gives insight into whether changes during the competition were sustainable. Existing research focuses primarily on the competition itself, not examining the reasons why energy usage might rebound after the competition is over.
	\item \emph{How effective were the tasks available via the website?} The tasks participants undertook can be tracked using website log data and compared to changes in their energy literacy.
	\item \emph{How appropriate were the Kukui Nut values assigned to tasks?} The Kukui Nut points assigned to tasks are intended to motivate participants to perform the tasks, but the values were assigned without any participant data.
	\item \emph{What is the relationship between energy literacy and energy usage?} The hypothesis is that more energy literate participants will conserve more energy.
	\item \emph{How important was floor-level near-realtime feedback?} There are good reasons to believe that floor-level near-realtime feedback will lead to increased energy conservation, but they greatly increase the competition budget and logistical complexity. Is the trade-off worth it?
\end{itemize}

\subsection{Evaluation}
There are four primary sources of data available to examine the research questions:

\begin{itemize}
	\item power and energy data from each floor,
	\item detailed event logs from participant actions on the website,
	\item participants' performance on an energy literacy survey to be administered before and after the competition,
	\item and a survey on the competition as a whole to be administered after the competition.
\end{itemize}

This rich dataset allows the examination of several relationships. The energy data alone provide insight into what effect (if any) the competition had on participants' energy usage, particularly to what degree energy use rebounds after the competition is over.

The combination of the website log data and the pre- and post-competition energy literacy scores sheds light on whether the tasks available on the website led to increased participant energy literacy, and if so, which tasks were most effective.

Finally, the combination of the energy usage data and the energy literacy scores allows look at the hypothesis that those floors that were more energy literate conserved more energy, both during and after the competition.


\section{Outline}

The proposal is organized into the following chapters:

\begin{itemize}
	\item \autoref{cha:related-work} looks at related research, including dorm energy competitions, energy feedback, and psychological techniques for fostering behavior change.
	\item \autoref{cha:system-description} describes the system we will be evaluating, which includes the dorm energy competition, and the associated website.
	\item \autoref{cha:evaluation} lists our research questions and explains our plan to evaluate them.
	\item \autoref{cha:conclusion} concludes the proposal with a list of anticipated contributions and future directions.
	\item \autoref{app:power-energy} covers the definitions of power and energy, and their interrelationship. Understanding these two concepts is critical to understanding the evaluation (and an important part of energy literacy).
	\item \autoref{app:tasks} lists the set of tasks to be made available to participants through the website to improve their energy literacy.
	\item \autoref{app:energy-literacy} provides a \Hawaii-specific survey designed to assess the energy literacy of participants.
	\item \autoref{app:qualitative-feedback} contains questionnaire to be administered to participants after the competition has ended.
\end{itemize}
