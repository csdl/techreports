%%%%%%%%%%%%%%%%%%%%%%%%%%%%%% -*- Mode: Latex -*- %%%%%%%%%%%%%%%%%%%%%%%%%%%%
%% uhtest-appendix.tex -- 
%% Author          : Robert Brewer
%% Created On      : Fri Oct  2 16:31:12 1998
%% Last Modified By: Robert Brewer
%% Last Modified On: Mon Oct  5 14:41:05 1998
%% RCS: $Id: uhtest-appendix.tex,v 1.1 1998/10/06 02:07:03 rbrewer Exp $
%%%%%%%%%%%%%%%%%%%%%%%%%%%%%%%%%%%%%%%%%%%%%%%%%%%%%%%%%%%%%%%%%%%%%%%%%%%%%%%
%%   Copyright (C) 1998 Robert Brewer
%%%%%%%%%%%%%%%%%%%%%%%%%%%%%%%%%%%%%%%%%%%%%%%%%%%%%%%%%%%%%%%%%%%%%%%%%%%%%%%
%% 

\documentclass[11pt]{article}
%%% Load some useful packages:

\usepackage{times}

% make margins smaller
\usepackage[left=2.5cm,top=2cm,right=2.5cm,bottom=2cm,nohead,nofoot]{geometry}

%% Provides customization of lists
\usepackage{enumitem}

%% Now define question list type
\newlist{question}{enumerate}{1}
\setlist[question]{resume, label=\textbf{\arabic*.}}

%% Now define multiple choice answer list type
\newlist{answer}{enumerate}{1}
\setlist[answer]{label=\alph*)}

%%% End of preamble
\begin{document}

\title{Energy Literacy Quiz B, Version 1.0.0}
\author{Robert S. Brewer}
%\date{}

\maketitle

%%% Body text goes here

\section{Quiz}

\begin{question}
	\item Power is commonly measured in units of:
\end{question}

\begin{answer}
	\item BTU
	\item joule
	\item kilowatt-hour
	\item watt
\end{answer}

\begin{question}
	\item The abbreviation "W" refers to what unit:
\end{question}

\begin{answer}
	\item watt-hour
	\item wind power
	\item wave power
	\item watt
\end{answer}

\begin{question}
	\item The watt-hour is a unit of:
\end{question}

\begin{answer}
	\item energy
	\item power
	\item distance
	\item force
\end{answer}

\begin{question}
	\item The watt-hour is abbreviated as:
\end{question}

\begin{answer}
	\item Wh
	\item wth
	\item W
	\item erg
\end{answer}

\pagebreak

\begin{question}
	\item A compact fluorescent lightbulb (CFL) used 26 Wh after running for 2 hours. How much power did the bulb consume?
\end{question}

\begin{answer}
	\item 7.5 W
	\item 13 W
	\item 26 W
	\item 52 W
\end{answer}

\begin{question}
	\item If your game console uses 200 W when turned on, how much energy would it waste if you left it on all weekend while you were away?
\end{question}

\begin{answer}
	\item 15000 Wh
	\item 100 Wh
	\item 960 kWh
	\item 9.6 kWh
\end{answer}

\begin{question}
	\item Roughly how much power does a normal compact fluorescent lightbulb (CFL) use when running?
\end{question}

\begin{answer}
	\item 20 mW
	\item 3 W
	\item 60 W
	\item 13 W
\end{answer}

\begin{question}
	\item On average, how much electrical energy does a home in Hawaii use per month?
\end{question}

\begin{answer}
	\item 37 kWh
	\item 104 kWh
	\item 390 kWh
	\item 2000 kWh
\end{answer}

\begin{question}
	\item Approximately how much energy does single standard rooftop solar panel in Hawaii generate each day?
\end{question}

\begin{answer}
	\item 100 Wh
	\item 1000 Wh
	\item 10 kWh
	\item 100 W
\end{answer}

\pagebreak

\begin{question}
	\item Burning oil is used to generate approximately what percentage of Hawaii's electricity?
\end{question}

\begin{answer}
	\item 100\%
	\item 50\%
	\item 78\%
	\item 17.5\%
\end{answer}

\begin{question}
	\item What is the approximate total annual electrical energy demand for the state of Hawaii?
\end{question}

\begin{answer}
	\item 10,500 million kWh
	\item 100 million kWh
	\item 2 million kWh
	\item 4,500 million kW
\end{answer}

\begin{question}
	\item Why is the shape of the electrical grid demand curve important?
\end{question}

\begin{answer}
	\item Less efficient power plants must be used if there are peaks in the curve 
	\item A flat curve means nobody is using any electricity
	\item The shape shows how many power plants are running
	\item The curve is lower at night if there is a lot of solar power in the grid
\end{answer}

\begin{question}
	\item What is the breakdown of the clean energy mandated by the Hawaii Clean Energy Initiative by 2030?
\end{question}

\begin{answer}
	\item 50\% from renewable sources, 10\% from conservation
	\item 30\% from solar, 30\% from wind, 10\% from waves
	\item 30\% from renewable sources, 20\% from conservation, 10\% from natural gas
	\item 30\% from energy conservation, 40\% from renewable sources
\end{answer}

\begin{question}
	\item What is the primary cause of climate change?
\end{question}

\begin{answer}
	\item Melting glaciers in Greenland
	\item Carbon dioxide released from burning fossil fuels
	\item Natural solar cycles
	\item Radioactive waste from nuclear power plants
\end{answer}

\begin{question}
	\item Approximately how much rise in sea level is predicted by the end of the century due to climate change?
\end{question}

\begin{answer}
	\item 2 inches
	\item 6 inches
	\item 1 foot
	\item 3 feet
\end{answer}

\end{document}
