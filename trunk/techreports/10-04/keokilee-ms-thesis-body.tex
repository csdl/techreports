
\chapter{Introduction}
It is becoming increasingly important to educate the public about their energy usage.  With the current state of alternative energy research, it is impossible to support the entire world on renewable energy.  In order to ensure that current energy sources are still available for generations to come, organizations from non-profits to utility providers have advocated energy conservation.  Some organizations hope to educate the up and coming generation in order to instill lasting habits that can be carried on to future generations.

One approach to educating the younger generation is to hold college dormitory energy competitions.  The goal of these competitions is to have the residents of the dorm reduce their energy usage.  Typically, the dorm that reduces their energy use the most at the end of the competition is declared the winner.  Other smaller prizes can be awarded for accomplishing certain goals, like reducing energy usage by 10 percent in a week.  The overall energy reduction is determined either by having someone read the meters or using "smart meters" that are connected to the internet and can send out data.  Universities such as Duke (Eco-lympics) and the University of Wisconsin (Energy Apocalypse) have run competitions relating to energy conservation and awareness. 

These competitions usually involve more than just reducing energy usage.  The competition organizers also include activities that relate to energy conservation and are geared toward improving energy literacy and awareness.  Examples of activities range from viewing documentaries relating to conservation to attending recycling drives.  Participation in these activities and encouraging others to do the same can also be recognized and rewarded in the context of these dorm energy competitions.

\section{The Problem}

To aid in running the competition, many of these universities used web sites to display the resident's current usage.  While it is easy to create a content management system to display mostly static data (i.e. one that is only updated when someone reads the meter), dormitory residents are more motivated by real-time feedback\cite{oberlin-feedback}.  However, the development of such a system can be a complicated and/or expensive process.  Providing real-time feedback not only requires special meters that can communicate with other devices, but also requires software that can process the data and display the relevant information to the user.  Because of this, many organizations have turned to companies like Lucid Design Group that can provide this software and hardware at a cost.

But Lucid Design Group's software only involves the visualization of energy data and does not involve energy awareness activities.  It is designed to be embedded within a website rather than a complete competition package.  Because of this, the software is unable to immediately provide user-related information.  For example, if a dorm resident wants to view their floor's energy usage, they must interact with the visualization to get the information that they need.  In the ideal case, the user would log in using their university credentials and then be able to immediately view their current standings.

As for energy activities, this type of information can be posted on a website.  However, organizers would also like to be able to track interest and participation in these activities so that users can be rewarded.  Users also could be more motivated to participate if they see others in their floor/dorm participating.  Adding in these functions go beyond what a standard content management system does.  Developing such a module for a competition would also take more time and/or resources.

\section{Makahiki}

The goal of Makahiki is to provide a complete software package for organizations that want to hold their own dorm energy competitions.  It will have the following features:

\begin{enumerate}
	\item Integration with WattDepot as a source of energy data.
	\item Support for Central Authentication Service (CAS) for enabling college students to log in with their own university account.
	\item The ability to create activities and track user participation in them.
	\item Produce visualizations of the energy data retrieved from WattDepot.
	\item Integration with social networks such as Facebook and Twitter for displaying progress and standings.
\end{enumerate}

Furthermore, Makahiki will be configurable depending on the competition organizer's needs.  These configurations include:

\begin{enumerate}
	\item Single page - A single page with news and charts.  The data for these charts can be either entered manually or retrieved from WattDepot.
	\item Tabbed interface - Includes sections for additional resources (i.e. links and videos) as well as energy data retrieved from WattDepot.
	\item Full Competition - Includes supplementary activities, commitments, and goals.  Also includes another tab for information about the competition.
\end{enumerate}

Finally, Makahiki will be open source.  This means that competition organizers can design the visual look of the application to fit their organization.  Also, advanced users can add or tweak modules in the application to fit their needs.

\section{Evaluating the Makahiki system}

The first step in evaluating the Makahiki system is to use it in our own dorm energy competition.  An initial version of Makahiki will be available for initial testing and review in the Summer of 2010.  Then, we will be holding a dorm energy competition here at the University of Hawaii at Manoa in October 2010 using both Makahiki and WattDepot.  Through the use of surveys, interviews, and/or log analysis, we hope to gain insight into how users will use the various features of Makahiki.

We also hope to provide an instance of Makahiki that looks similar to another dorm energy competition (for example, the Duke Eco-lympics).  The purpose of this is to demonstrate the configuration options and how other universities can apply Makahiki to their own competition.

\section{Thesis Claims}

During the evaluation of Makahiki, we hope to answer the following questions:

\begin{enumerate}
	\item How effective is the site in influencing the energy usage habits of users?
	\item What components of Makahiki can we improve?
	\item How well does Makahiki support the potential needs of other energy competitions?
\end{enumerate}

\section{Proposal Structure}

\autoref{relatedwork} will discuss related works, which includes other dorm energy competitions.  \autoref{makahiki} will describe the components of the Makahiki.  \autoref{evaluation} describes our proposed evaluation procedure.  Finally, \autoref{conclusion} includes the conclusion and provides a timeline for the work that is to be done to complete this project.

\chapter{Related Work}
\label{relatedwork}

\autoref{othercomps} investigates the efforts by other universities to create a dorm energy competition.

\autoref{tracking} investigates "smart meters", which are used to present energy information in a way that is easy to understand for the average person.

\section{Other Dorm Energy Competitions}
\label{othercomps}
Harvard, Yale, etc.

\section{Tracking Your Energy Usage}
\label{tracking}
TED5000, meter investigation.

\chapter{Makahiki}
\label{makahiki}

The proposed Kukui Cup System consists of two main components.  \autoref{webapp} describes the implementation of the web application.  \autoref{socialint} describes integration with popular social networks such as Facebook and Twitter.

\section{Web Application}
\label{webapp}

Investigations of CMSs.  Django/Pinax.

\subsection{WattDepot}

Development of the WattDepot system by Robert Brewer.

\subsection{Near Real-Time Updates}

Providing sub-minute updates for each floor.

\subsection{Mobile Web Application}

Formatting the website for iPhone/Android users.

\section{Social Network Integration}
\label{socialint}

Integrating with Facebook and/or Twitter.

% \section{Other Notifications}
% \label{notifications}

\chapter{Evaluation}
\label{evaluation}

Focus groups, surveys, actual competition.

\chapter{Conclusion}
\label{conclusion}

It is our hope that this system will improve the participants energy literacy in a way that makes a lasting impact on their energy use outside of the competition.

\section{Anticipated Contributions}

We want to provide a system that integrates real-time energy data along with incentives for behavioral change.

\section{Thesis Timeline}

\begin{enumerate}
\item August 2010 - An early version of the Makahiki will be available for evaluation.
\item October 2010 - The competition will take place.
\end{enumerate}