\chapter{Participant Actions}
\label{app:actions}

This appendix lists the actions available to 2011 UH Kukui Cup participants. Overall, the actions were intended to increase the energy literacy of the participants performing it, help them modify their behavior to reduce their electricity usage, or both. However, not every action met these goals. For example, some actions were included that were related to sustainability in general, and linked to energy only indirectly. Other actions were included primarily for the entertainment of participants, in keeping with the design of the challenge as an interesting and fun game to play.

The following sections list all the actions, and indicate how they would be performed, and validated by administrators. The actions are grouped into three categories: activities, commitments, and events.


\section{Activities}

See \autoref{sec:activities} for a description of what activities were in the Kukui Cup and how they were processed. \autoref{tab:activity-list} lists all the activities that were available in the 2011 UH Kukui Cup.

\begin{center}
%	\scriptsize
	\begin{longtable}{| l | r | r |}
		\caption{A list of the activities available during the challenge}\label{tab:activity-list}\\
		\hline
		Activity name & Points & Confirmation type \\ \hline \hline
		\endfirsthead
		\multicolumn{3}{c}%
{\tablename\ \thetable\ -- \textit{Continued from previous page}} \\
\hline
		Activity name & Points & Confirmation type \\ \hline \hline
		\endhead
		\hline \multicolumn{3}{r}{\textit{Continued on next page}} \\
		\endfoot
		\hline
		\endlastfoot
Watch introduction video & 20 & Q\&A \\
Watch video ``Secrets of the Kukui Cup Masters'' & 10 & Q\&A \\
Like Kukui Cup on Facebook & 5 & open-ended \\
Tweet about Kukui Cup & 5 & open-ended \\
Share Kukui Cup link on Google+ & 5 & open-ended \\
Door Art Challenge & 5--15 & image \\
Play the photo chain game & 10 & open-ended \\
Watch video about Power \& Energy & 10 & Q\&A \\
Watch video on Energy Intuition & 10 & Q\&A \\
Learn more about Power \& Energy & 15 & Q\&A \\
Learn more about Energy Intuition & 15 & Q\&A \\
Examine your lounge's energy use & 10 & open-ended \\
Watch video on how to audit your energy use & 15 & Q\&A \\
Find out how much power your stuff uses & 30 & open-ended \\
Label power hogs in your room & 15 & image \\
Watch video about Lighting & 10 & Q\&A \\
Learn more about lighting & 15 & Q\&A \\
Replace incandescent bulb with compact fluorescent (CFL) & 10 & image \\
Estimate your room's total daily energy consumption & 35 & open-ended \\
Watch Energy Generation: What's the fuss? & 10 & Q\&A \\
Watch Energy Generation: Where are we now? & 10 & Q\&A \\
Watch Energy Generation: Hawaii Clean Energy Initiative & 10 & Q\&A \\
Learn more about the fuss regarding Energy Generation & 15 & Q\&A \\
Learn more about how Hawaii generates energy now & 15 & Q\&A \\
Learn more about the Hawaii Clean Energy Initiative & 15 & Q\&A \\
Take a survey about the Kukui Cup & 40 & open-ended \\
Watch video about solar energy & 10 & Q\&A \\
Learn more about solar energy & 15 & Q\&A \\
Watch video about transportation energy use & 10 & Q\&A \\
Learn more about transportation & 15 & Q\&A \\
Configure your computer to sleep after inactivity & 20 & image \\
Watch Trash is Treasure video & 10 & Q\&A \\
Learn more about opala & 15 & Q\&A \\
Energy Geo Trek across campus & 5--45 & open-ended \\
Measure shower water flow & 15 & open-ended \\
Measure sink water flow & 15 & open-ended \\
Watch a video about climate change & 10 & Q\&A \\
Learn more about climate change & 15 & Q\&A \\
Refer a friend to the Kukui Cup & 5 & open-ended \\
Go on a No Impact date & 10 & open-ended \& image \\
Watch a video about OTEC & 10 & Q\&A \\
Write a poem on a Kukui Cup topic & 5--50 & open-ended \\
Write a letter to the editor on a Kukui Cup topic & 5--50 & open-ended \\
Make a video on a Kukui Cup topic & 5--50 & open-ended \\
Write a song about a Kukui Cup topic & 5--50 & open-ended \\
Interview someone about a Kukui Cup topic & 5--50 & open-ended \\
Make a photo blog a Kukui Cup topic & 5--50 & open-ended \\
Create Energy Window Art & 5--50 & open-ended \& image \\
Create art around a Kukui Cup topic & 5--50 & open-ended \\
Design a Kukui Cup 2012 T-shirt & 5--50 & image \\
Do something else creative on a Kukui Cup topic & 5--50 & open-ended \\
	\end{longtable}
\end{center}


\subsection{Intro video}

\textbf{Description:} If you missed it during your the first login process, watch this video explaining the competition and this website

\vspace{2ex}
\textbf{Expected benefits:} Basic understanding of the challenge


\subsection{Cup Secrets}

\textbf{Description:} Watch this video that provides some hints on how to get the most out of the Kukui Cup

\vspace{2ex}
\textbf{Expected benefits:} Deeper understanding of challenge, including commitments


\subsection{Like Cup}

\textbf{Description:} Show support for the Kukui Cup by Liking the Kukui Cup page on Facebook (it will open in a new window). Follow the link, click on the ``Like'' button, and then back on this page click the \emph{I Did This!} button. You will be prompted for your name on Facebook so we can verify your Like.

For you eager beavers that have already liked the Kukui Cup before the competition started, you can get points too. Just click \emph{I Did This!} and tell us your Facebook name.

\vspace{2ex}
\textbf{Expected benefits:} Awareness of Kukui Cup events through future news posts, promotion of Kukui Cup to friends


\subsection{Tweet link}

\textbf{Description:} If you use Twitter, tweet a link to the Kukui Cup website by pressing this button (will open in a new window) so other folks will learn about the competition:

Once you have tweeted, come back to this page and click the ``I Did This!'' button. You will be prompted for your Twitter username so we can confirm your tweet.

\vspace{2ex}
\textbf{Expected benefits:} Awareness of Kukui Cup events through future tweets, promotion of Kukui Cup to friends


\subsection{Share link}

\textbf{Description:} If you have an account on Google's new social network Google+, share the Kukui Cup website \url{http://kukuicup.manoa.hawaii.edu/} there to get the word out about the competition. Go to Google+ (will open in a new window) and click on the small paperclip icon to share a link.

Enter the text "http://kukuicup.manoa.hawaii.edu/" next to the paperclip icon and press the \emph{Add} button. Above that, type something about the Kukui Cup. Make sure this post is set to be shared with the \emph{Public}, you may have to click the "+Add more people" link to see the Public circle. If you do not share the link publicly, we will not be able to verify it. Then press the \emph{Share} button.

Once you have updated your status, go to your Google+ profile and copy the URL which will look something like "https://plus.google.com/" followed by a long number. Then come back to this page and click the "I Did This!" button. Paste in that profile URL so we can see your snazzy public post.

\vspace{2ex}
\textbf{Expected benefits:} Awareness of Kukui Cup events through future news posts, promotion of Kukui Cup to friends


\subsection{Door Art}

\textbf{Description:} Show your support for the Kukui Cup by decorating your door in an "energy conscious" fashion. You can do any of the following: include Kukui Cup logos or screen shots, invent new tag lines, show pictures related to energy, or anything else that communicates energy consciousness. When completed, take a photo of your door and click the \emph{I Did This} button to submit the picture. Based on the awesomeness of your design, you will get between 5--15 points!

If you want to show off your mad skillz, in addition to uploading your photo to us, you can publicly post your door art picture to the Kukui Cup 2011 Door Art Flickr group (will open in a new window). If you don't already have a Flickr account, you can sign up for one for free.

\vspace{2ex}
\textbf{Expected benefits:} Promotion of Kukui Cup to loungemates, have fun. [Note, this action was inspired by Nathan's description of door art~\cite[pgs. 23--27]{Nathan2005-FreshmanYear}


\subsection{Photo Chain}

\textbf{Description:} Show off the Kukui Cup and your creativity at the same time by playing the Kukui Cup Photo Chain Game. The rules are simple. You are going to add a photo of yourself to the end of the chain of photos in the Flickr Kukui Cup 2011 Photo Chain Game group. Each photo contains:

\begin{enumerate}
	\item yourself in some kind of "pose"
	\item taken at some location on the UH Manoa Campus
	\item you holding or wearing some Kukui Cup item (t-shirt, water bottle, eco-tote, or tumbler)
\end{enumerate}

To make the sequence of photos a chain, your photo must "match" the preceding photo posted to the group with respect to either the pose, the location, or the Kukui Cup item.

The caption to each photo should list the pose, the location, and the item.

Here's how you do it:

\begin{enumerate}
	\item Go to the Flickr Kukui Cup 2011 Photo Chain Game group. See what photo is at the end of the list.
	\item Take a photo of yourself in a way that extends the chain (match either the pose, location, or item)
	\item Upload your photo to your own Flickr account (creating it if necessary).
	\item Go back to the Flickr Kukui Cup 2011 Photo Chain Game group, and "join" the group. 
	\item Click "Add Photos" and select your photo. 
	\item Provide a title, such as "Photo Chain Game 12" (or whatever the next number is in the chain).
	\item Provide a caption that indicates: pose, location, item, and which of the three matches.
	\item Save.
	\item Click \emph{I Did This!} button and tell us your Flickr name so we can confirm your upload to the group and award your points.
\end{enumerate}

\vspace{2ex}
\textbf{Expected benefits:} Promotion of Kukui Cup, have fun


\subsection{Power \& Energy}

\textbf{Description:} Watch this video explaining the difference between power and energy

\vspace{2ex}
\textbf{Expected benefits:} Learn about concepts of power and energy and their interrelationship


\subsection{Energy Intuition}

\textbf{Description:} Watch this video about improving your energy intuition.

\vspace{2ex}
\textbf{Expected benefits:} Learn the energy consumption of different appliances


\subsection{Power \& Energy 2}

\textbf{Description:} While you've watched the Power \& Energy video once, this activity gives you a chance to learn more about the topic. You can rewatch the video below and explore the resource links provided. When you are done, you can get your points by answering a more difficult question. The question may require information from the video, the resource links or both.

Resource links (will open in new windows/tabs):

\begin{itemize}
	\item What is a kilowatt?
	\item What is electricity?
\end{itemize}

\vspace{2ex}
\textbf{Expected benefits:} Learn about concepts of power and energy and their interrelationship


\subsection{Energy Intuition 2}

\textbf{Description:} While you've watched the Energy Intuition video once, this activity gives you a chance to learn more about the topic. You can rewatch the video below and explore the resource links provided. When you are done, you can get your points by answering a more difficult question. The question may require information from the video, the resource links or both.

Resource links (will open in new windows/tabs):

\begin{itemize}
	\item Chart of Hawaii's oil consumption over the past five years>
	\item Watts, Volts, Amps, and Butter Heaters
\end{itemize}

\vspace{2ex}
\textbf{Expected benefits:} Learn the energy consumption of different appliances


\subsection{Check energy}

\textbf{Description:} One important aspect of the Kukui Cup is the Daily Energy Goal Game, which can be found on the Go Low page. Every lounge is assigned a daily energy goal that represents a 5\% reduction in electricity use from what the lounge was using before the competition. The Energy Goal Game shows how much energy your lounge has used so far today, and how that compares to the energy goal. The game also shows some activities that might be helpful in getting your lounge to reduce its energy use.

For this activity, you will check out the Go Low page. Read the page over, and come back to here and report three things: is your lounge above or below the energy goal, by how much, and name one activity that might help your lounge conserve energy.

\vspace{2ex}
\textbf{Expected benefits:} Awareness of lounge energy use and energy goal, reflection on how to conserve energy


\subsection{Audit Video}

\textbf{Description:} Watch this video that explains how to figure out how much energy your stuff uses.

\vspace{2ex}
\textbf{Expected benefits:} Understanding of how to conduct an energy audit of appliances using a plug load meter


\subsection{Audit Room}

\textbf{Description:} Check out a Belkin Conserve Insight meter from your tower's front desk (CDC), and then check 5 plug-in devices in your room to see how much power they use when turned on \& when turned off. Write down both the on and off values for each device as you check them because you will need that to get your points!

\vspace{2ex}
\textbf{Expected benefits:} Experience conducting an energy audit of appliances using a plug load meter, knowledge of the power consumption of appliances in player's room


\subsection{Power Hogs}

\textbf{Description:} This is a followup activity to the energy audit activity. If you haven't performed that activity, do it first and keep your notes around for this activity.

Based on the audit results, make a label for each device in your room that shows the number of watts consumed when on and off, and put it close to the power switch for those devices that have them.

Once you have made the labels, take a photo of the labels of as many devices as you can fit in one picture to be used for verification.

\vspace{2ex}
\textbf{Expected benefits:} Knowledge of the power consumption of appliances in player's room, understanding of the power of prompts in reducing energy use


\subsection{Lighting video}

\textbf{Description:} Watch this video about the ways we use energy to generate light.

\vspace{2ex}
\textbf{Expected benefits:} Knowledge of how energy is used for lighting, and how different lighting technologies compare


\subsection{Lighting video 2}

\textbf{Description:} While you've watched the Lighting video once, this activity gives you a chance to learn more about the topic. You can rewatch the video below and explore the resource links provided. When you are done, you can get your points by answering a more difficult question. The question may require information from the video, the resource links or both.

Resource links (will open in new windows/tabs):

\begin{itemize}
	\item Incandescent lighting and chocolate bunnies
\end{itemize}

\vspace{2ex}
\textbf{Expected benefits:} Knowledge of how energy is used for lighting, and how different lighting technologies compare


\subsection{CFL swap}

\textbf{Description:} Find an incandescent bulb and replace it with a CFL (compact fluorescent). If you've already replaced all the incandescent bulbs in your room with CFLs or LEDs, you might have to go hunt for one elsewhere. You should throw away the incandescent bulb, because even though it isn't burned out it is a massive energy hog and you don't want someone using it later.

For verification, please take a single photo showing both the incandescent bulb you replaced and the CFL you installed.

\vspace{2ex}
\textbf{Expected benefits:} Knowledge of how energy is used for lighting, and how different lighting technologies compare


\subsection{Room Energy}

\textbf{Description:} Ever wonder how much energy your room consumes in a day? Here's a simple procedure to figure it out.

\begin{enumerate}
	\item Check out a Belkin Conserve Insight meter from your tower's front desk (CDC)
	\item Find out how much energy your appliances use on average. For refrigerators, monitor their energy consumption for 30 minutes, then multiply by 48 to get an estimate of its total consumption for 24 hours. For microwaves, monitor its energy consumption during a 1 minute period, then multiply by the number of minutes per day that you and your roommate typically use it. Do the same for your other appliances: computer, Xbox, fans, etc.
	\item To receive credit for this activity, submit a list of the appliances in your room, your estimate of the energy each of them consume per day, and the total amount of energy your entire room uses per day (in watt-hours or kilowatt-hours). Remember to include an estimate for the overhead light if you use it. Are you surprised by this data?
\end{enumerate}

\vspace{2ex}
\textbf{Expected benefits:} Understanding of how to calculate daily energy use, reflection on personal energy use


\subsection{Energy Issues}

\textbf{Description:} Watch this video that talks about how we generate our energy in \Hawaii, and what the fuss is about.

\vspace{2ex}
\textbf{Expected benefits:} Understanding of \Hawaii's energy situation


\subsection{Energy Now}

\textbf{Description:} Watch this video that talks about how we generate our energy in Hawaii right now, and how that differs from the mainland US.

\vspace{2ex}
\textbf{Expected benefits:} Understanding of how \Hawaii generates energy now


\subsection{HCEI}

\textbf{Description:} Watch this video that talks about the Hawaii Clean Energy Initiative, a plan for increasing clean energy use in Hawaii.

\vspace{2ex}
\textbf{Expected benefits:} Understanding of what the HCEI goals mean


\subsection{Energy Issues 2}

\textbf{Description:} While you've watched the Energy Issues video once, this activity gives you a chance to learn more about the topic. You can rewatch the video below and explore the resource links provided. When you are done, you can get your points by answering a more difficult question. The question may require information from the video, the resource links or both.

Resource links (will open in new windows/tabs):

\begin{itemize}
	\item Hawaii: The State of Clean Energy
\end{itemize}

\vspace{2ex}
\textbf{Expected benefits:} Understanding of \Hawaii's energy situation


\subsection{Energy Now 2}

\textbf{Description:} While you've watched the Energy Issues video once, this activity gives you a chance to learn more about the topic. You can rewatch the video below and explore the resource links provided. When you are done, you can get your points by answering a more difficult question. The question may require information from the video, the resource links or both.

\begin{itemize}
	\item State by State Energy Prices, by the US Energy Information Office
	\item US Energy Facts, by the US Energy Information Office
\end{itemize}

\vspace{2ex}
\textbf{Expected benefits:} Understanding of how \Hawaii generates energy now


\subsection{HCEI 2}

\textbf{Description:} While you've watched the Hawaii Clean Energy Initiative video once, this activity gives you a chance to learn more about the topic. You can rewatch the video below and explore the resource links provided. When you are done, you can get your points by answering a more difficult question. The question may require information from the video, the resource links or both.

Resource links (will open in new windows/tabs):

\begin{itemize}
	\item HCEI Update: Year 2 (PDF)
	\item Island Energy Projects to move toward HCEI goals
\end{itemize}

\vspace{2ex}
\textbf{Expected benefits:} Understanding of what the HCEI goals mean


\subsection{Take Survey}

\textbf{Description:} We're in Round 3 of the Kukui Cup now, and we'd like to get some feedback from you about the competition. Click the following link to the survey, which will open in a new window. Once you are done, return here and just tell us you completed the survey, and we'll award your points.

\vspace{2ex}
\textbf{Expected benefits:} Data on participants experiences with the Kukui Cup


\subsection{Solar Energy}

\textbf{Description:} Watch this video that explains how we can get energy directly from the sun.

\vspace{2ex}
\textbf{Expected benefits:} Understanding of solar energy


\subsection{Solar Energy 2}

\textbf{Description:} While you've watched the Solar Energy video once, this activity gives you a chance to learn more about the topic. You can rewatch the video below and explore the resource links provided. When you are done, you can get your points by answering a more difficult question. The question may require information from the video, the resource links or both.

Resource links (will open in new windows/tabs):

\begin{itemize}
	\item Energy 101: Solar Power
	\item Introduction to PV
	\item How PV cells produce electricity
\end{itemize}

\vspace{2ex}
\textbf{Expected benefits:} Understanding of solar energy


\subsection{Transport Video}

\textbf{Description:} Watch this video that explains how your choice of transportation impacts your energy use.

\vspace{2ex}
\textbf{Expected benefits:} Understanding of how much energy different transportation options use, and alternatives available to participants


\subsection{Transport Video 2}

\textbf{Description:} While you've watched the Transportation video once, this activity gives you a chance to learn more about the topic. You can rewatch the video below and explore the resource links provided. When you are done, you can get your points by answering a more difficult question. The question may require information from the video, the resource links or both.

Resource links (will open in new windows/tabs):

\begin{itemize}
	\item Carbon Footprint Calculator
	\item Keep Pedalling (bicycle hip hop)
	\item How PV cells produce electricity
	\item Cycle Manoa
	\item Trains, Planes, and Automobiles: Which is the greenest way to travel long distances in the US?
\end{itemize}

\vspace{2ex}
\textbf{Expected benefits:} Understanding of how much energy different transportation options use, and alternatives available to participants


\subsection{Computer Sleep}

\textbf{Description:} Configure your computer and any external monitor to sleep after 20 minutes of inactivity (or less). You can find instructions on enabling sleep functionality at the EnergySTAR website.

Once you have changed your settings, please take a screenshot showing the new settings for use in verification.

\vspace{2ex}
\textbf{Expected benefits:} Reduced energy use from computers not actually being used


\subsection{Trash video}

\textbf{Description:} Check out this video called Trash is Treasure and maybe change your thinking about trash

\vspace{2ex}
\textbf{Expected benefits:} Understanding of where trash goes after being thrown away and ways to reduce the waste stream


\subsection{Trash Video 2}

\textbf{Description:} While you've watched the Trash is Treasure video once, this activity gives you a chance to learn more about the topic. You can rewatch the video below and explore the resource links provided. When you are done, you can get your points by answering a more difficult question. The question may require information from the video, the resource links or both.

Resource links (will open in new windows/tabs):

\begin{itemize}
	\item How the City Manages Our Waste
	\item Make A Juicer Out Of A Plastic Bottle
\end{itemize}

\vspace{2ex}
\textbf{Expected benefits:} Understanding of where trash goes after being thrown away and ways to reduce the waste stream


\subsection{Geo Trek}

\textbf{Description:} Want to go on a "geo trek"? Have a smart phone that can display Google Maps (or a friend who has one and wants to play with you)? Have an hour to walk around campus and learn about UH energy and sustainability projects? Then you're all set!

Here's the deal: you'll use your smartphone (and your own smarts) to follow a trail of clues around campus and discover eight different locations related to energy or sustainability. Each location is tagged with a special Kukui Cup sign so you'll know when you've found it. Once there, take a picture of yourself with the sign in the background for verification, and upload to Flickr. You will earn 5 Kukui Cup points for each location you find.  The sign also contains a URL that you can retrieve to obtain directions to the next location.

Ready?  Here's the first clue (you'll want to open it on your smartphone): 

When you are done, come back here, click \emph{I Did This!} and provide us with a link to your Flickr account so we can verify your trek.

\vspace{2ex}
\textbf{Expected benefits:} Understanding of different energy-related locations on campus, have fun


\subsection{Shower flow}

\textbf{Description:} Figure out how much water is used by the showers in your floor's bathroom. To do this, you will need a large container with a known volume (like a water bottle or bucket) that you can fill in the shower, and a clock of some type (like your watch or cell phone). Make sure to not get your clock wet if it isn't waterproof!

Hold the container up to the shower head, turn on the shower faucet to the level you use normally when showering, and note the time on the clock (or use a stopwatch). Watch the container fill, and note the time when it is completely full of water.

Now you know the volume of water that the shower put out, and the amount of time it was running. From these two values, you can compute the flow rate of your shower by dividing the volume by the number of seconds it took to fill. For example, if you filled a 1 gallon bucket and it took 30 seconds, then the flow rate is 2 gallons per minute.

As you will see, water comes out of shower heads pretty fast, which is why it is important to turn off the water when you aren't actively using it, like while soaping up.

\vspace{2ex}
\textbf{Expected benefits:} Experience measuring water flow, understanding of how to calculate flow rate, reflection on personal water use


\subsection{Sink Flow}

\textbf{Description:} Figure out how much water is used by the sinks in your floor's bathroom. To do this, you will need a large container with a known volume (like a water bottle or bucket) that you can fit in the sink, and a clock of some type (like your watch or cell phone). Make sure to keep your clock dry if it isn't waterproof!

Hold the container up to the faucet in the sink, turn on the water to the level you use normally when using the sink, and note the time on the clock (or use a stopwatch). Watch the container fill, and note the time when it is completely full of water.

Now you know the volume of water that the sink put out, and the amount of time it was running. From these two values, you can compute the flow rate of your sink by dividing the volume by the number of seconds it took to fill. For example, if you filled a 1 gallon bucket and it took 30 seconds, then the flow rate is 2 gallons per minute.

As you will see, water comes out of faucets pretty fast, which is why it is important to turn off the water when you aren't actively using it, like while brushing your teeth.

\vspace{2ex}
\textbf{Expected benefits:} Experience measuring water flow, understanding of how to calculate flow rate, reflection on personal water use


\subsection{Climate change}

\textbf{Description:} Watch this video talking about climate change and how it will affect \Hawaii.

\vspace{2ex}
\textbf{Expected benefits:} Better understanding of climate change and its expected impacts in \Hawaii


\subsection{Climate Change 2}

\textbf{Description:} While you've watched the Climate Change video once, this activity gives you a chance to learn more about the topic. You can rewatch the video below and explore the resource links provided. When you are done, you can get your points by answering a more difficult question. The question may require information from the video, the resource links or both.

Resource links (will open in new windows/tabs):

\begin{itemize}
	\item Hawaii's Climate Crisis Sea Level Rise is a seven minute video that includes interviews with Chip Fletcher discussing the impact of climate change, specifically rising sea levels, on Hawaii
	\item How it all ends is Greg Craven's followup to his "The Most Terrifying Video You'll Ever See", which got 7M views on YouTube.
	\item Scientific American article discusses why Americans are so ill-informed about climate change
\end{itemize}

\vspace{2ex}
\textbf{Expected benefits:} Better understanding of climate change and its expected impacts in \Hawaii


\subsection{Refer friend}

\textbf{Description:} Starting in Round 2, the Kukui Cup provides a new feature: referral bonuses. Each time you get a new user to sign up for the Kukui Cup, you can earn 10 points. All you need to do is have the new user enter your email address on the new "Referral Bonus" page during their initial login.   Both you and they will earn 10 extra points as soon as the new user earns 30 points with regular game play. There is \emph{NO LIMIT} to the number of times you can earn a referral bonus, so start signing up your friends now!

To help you get started, you can earn five extra points for one of your referrals by completing this activity. Just type in the email address of the new user you brought to the Kukui Cup in the box below. You will get your five points after the new user gets to 30 points.

Note that while this activity is just a one-shot deal, the referral bonus feature that it introduces you to can be done repeatedly.  You do not have to wait for your first referral to complete before getting new referrals.   Nor do you have to even do this activity at all in order to get referral bonuses.  So, go out there, get as many new players to sign up and enter your email as possible, and you will get 10 points each time they achieve 30 points.  (You can even coach them through their first 30 points to make sure they get a proper introduction to the game).

\vspace{2ex}
\textbf{Expected benefits:} Increased participation in Kukui Cup through referrals, have fun


\subsection{Impactless Date}

\textbf{Description:} While traditional ideas of a date include significant consumption, here's your chance to do something radical: have a good time with a friend while using the minimum amount of electrical energy possible! Use this as an opportunity to be creative: how many different things can you do with a friend while using the least possible energy? Consider your transportation: can you walk, ride a bike, or take public transportation? Consider your food: what can you eat together that is delicious but was grown, delivered, and prepared with minimal energy? Consider your entertainment: what can you do together that is interesting, novel, and fun that does not consume resources?

\vspace{2ex}
\textbf{Expected benefits:} Reflection on how to have fun with minimal energy use, have fun


\subsection{OTEC video}

\textbf{Description:} Watch this video that explains how energy can be cleanly extracted from cold ocean depths here in \Hawaii.

\vspace{2ex}
\textbf{Expected benefits:} Understanding of OTEC energy generation and potential in \Hawaii


\subsection{Write Poem}

\textbf{Description:} This activity is an opportunity to creatively explore one of the topics of the Kukui Cup in more depth. For this activity, you will write a \emph{poem} (or limerick or haiku) on the subject of one of the categories from the Smart Grid such as energy, lighting, transportation, etc. Note that this activity will require significantly more effort than normal activities, but it is also worth more points. This activity is worth a variable number of points, so the more effort and the higher quality of your poem, the more points you will receive.

\vspace{2ex}
\textbf{Expected benefits:} Creative expression and reflection on a Kukui Cup topic


\subsection{Letter to Editor}

\textbf{Description:} This activity is an opportunity to creatively explore one of the topics of the Kukui Cup in more depth. For this activity, you will write a \emph{letter to the editor} of a local publication on the subject of one of the categories from the Smart Grid such as energy, lighting, transportation, etc. Note that this activity will require significantly more effort than normal activities, but it is also worth more points. This activity is worth a variable number of points, so the more effort and the higher quality of your letter, the more points you will receive.

Some local publications you might write to:

\begin{itemize}
	\item Honolulu Weekly
	\item Star-Advertiser
	\item MidWeek
\end{itemize}

\vspace{2ex}
\textbf{Expected benefits:} Creative expression and reflection on a Kukui Cup topic, experience with community advocacy and involvement


\subsection{Make Video}

\textbf{Description:} This activity is an opportunity to creatively explore one of the topics of the Kukui Cup in more depth. For this activity, you will make a \emph{video} on the subject of one of the categories from the Smart Grid such as energy, lighting, transportation and post it on YouTube. Note that this activity will require significantly more effort than normal activities, but it is also worth more points. This activity is worth a variable number of points, so the more effort and the higher quality of your video, the more points you will receive.

Making a interesting video can be hard work. You'll find it much easier if you write out a little "screenplay" beforehand and think about what you want to communicate. YouTube has a Handbook with lots of great tips on making videos. Points will be awarded based on your video's originality, quality, and impact. For example, you talking into your webcam will get fewer points than visualizing your points with images or outdoor video. Make sure you mention the Kukui Cup either in your video or in the description field. When your video is ready, you can submit the URL for verification and scoring.

When submitting a video created by a group of players, please let us know how many people are planning to submit the video for points so we can divide the credit.

\vspace{2ex}
\textbf{Expected benefits:} Creative expression and reflection on a Kukui Cup topic


\subsection{Write Song}

\textbf{Description:} This activity is an opportunity to creatively explore one of the topics of the Kukui Cup in more depth. For this activity, you will write a \emph{song} on the subject of one of the categories from the Smart Grid such as energy, lighting, or transportation. Note that this activity will require significantly more effort than normal activities, but it is also worth more points. This activity is worth a variable number of points, so the more effort and the higher quality of your song, the more points you will receive.

For example, writing new energy-oriented lyrics to an existing song is pretty easy, and so not worth that many points. Writing an original song with lyrics and music is harder. Recording an original song is even harder, and so worth the most points.

\vspace{2ex}
\textbf{Expected benefits:} Creative expression and reflection on a Kukui Cup topic


\subsection{Interview}

\textbf{Description:} This activity is an opportunity to creatively explore one of the topics of the Kukui Cup in more depth. For this activity, you will \emph{interview} an expert on one of the subjects used as categories in the Smart Grid such as energy, lighting, or transportation. Note that this activity will require significantly more effort than normal activities, but it is also worth more points. This activity is worth a variable number of points, so the more effort and the higher quality of your interview, the more points you will receive.

Some possible interview subjects:

\begin{itemize}
	\item UH Manoa administrators (Facilities Management, Manoa Sustainability Corps)
	\item UH Manoa faculty (SOEST, HNEI, REIS)
	\item Local environmental organizations (Blue Planet Foundation, Kanu Hawaii, Surfrider Foundation)
	\item Local politicians (City Council, State Legislature, Congressional delegation)
	\item Local industry (First Wind, solar installation companies)
	\item Local government (DBEDT energy office, PUC)
\end{itemize}


You will have to find someone to interview, figure out what questions you want to ask them in advance, and then go interview them. You will probably want to record your interview so that you can refer to it later, but make sure you ask your subject if recording them is OK!

Your resulting interview can be written (like you might see in a magazine), and audio recording, or even a video. Note that making a quality audio and video interview is quite challenging, so make sure you know what you are getting into.

\vspace{2ex}
\textbf{Expected benefits:} Creative expression and reflection on a Kukui Cup topic, experience conducting an interview


\subsection{Photo Blog}

\textbf{Description:} This activity is an opportunity to creatively explore one of the topics of the Kukui Cup in more depth. For this activity, you will create a \emph{photo blog} around one of the subjects used as categories in the Smart Grid such as energy, lighting, or transportation. Note that this activity will require significantly more effort than normal activities, but it is also worth more points. This activity is worth a variable number of points, so the more effort and the higher quality of your photo blog, the more points you will receive.

A photo blog is a series of photographs on a particular subject, possibly with some explanatory text on each photo. You will need to find a place to host your photo blog, some options are: Tumblr, Flickr, or Wordpress. Remember to pick a theme for your photos related to the Kukui Cup. Here are some starter ideas:

\begin{itemize}
	\item Pictures of things using electricity across campus
	\item Pictures of energy generators (harder to find)
	\item Pictures of trash across campus (litter, recycling, dumpsters)
	\item Pictures of the different types of lighting used across campus
	\item Or something else Kukui Cup-related
\end{itemize}

\vspace{2ex}
\textbf{Expected benefits:} Creative expression and reflection on a Kukui Cup topic


\subsection{Window Art}

\textbf{Description:} Every night, the Hale Aloha residence halls create a mosaic of lit windows as students quietly study in their rooms. At least, that's what we hope you're doing in there.

It occurs to us that this mosaic of lit windows provides an opportunity for you to both be creative and inform fellow students coming back from the library about the Kukui Cup and/or energy conservation.

So, for this activity, create art on your room window that is visible from outside at night. The art should have some relationship to the theme of the Kukui Cup: use your imagination. To receive credit for this activity, take a picture of it (preferably at night) and include some text describing what your art is about. The more amazing, creative, and energy-related your window art, the more points you can earn. Since you still want to save energy, we recommend that you light up your art between 9-9:15 PM, you don't need to keep it lit all night long.

Another way to get more points for this activity is to enlist your neighbors beside you, above you, or below you to create a coordinated group window art effort. Take a photo of all of the windows, then tell us in the description which one is yours.

The more awesome your art, the closer to 50 points you'll earn.

\vspace{2ex}
\textbf{Expected benefits:} Creative expression and reflection on a Kukui Cup topic, have fun


\subsection{Make Art}

\textbf{Description:} This activity is an opportunity to creatively explore one of the topics of the Kukui Cup in more depth. For this activity, you will create a \emph{work of art} around one of the subjects used as categories in the Smart Grid such as energy, lighting, or transportation. Note that this activity will require significantly more effort than normal activities, but it is also worth more points. This activity is worth a variable number of points, so the more effort and the higher quality of your art, the more points you will receive.

The art you create for this activity should be different from the other options in this category. Here are some ideas:

\begin{itemize}
	\item painting
	\item sculpture
	\item collage
	\item sketch
\end{itemize}

You will have to capture your art in some way (scan, photograph, etc) so you can present it to us for verification and scoring.

\vspace{2ex}
\textbf{Expected benefits:} Creative expression and reflection on a Kukui Cup topic


\subsection{Design Tshirt}

\textbf{Description:} We already have awesome 2011 Kukui Cup T-shirts, but what about 2012? For this activity, you will create a design for the 2012 Kukui Cup T-shirt. Things to keep in mind:

\begin{itemize}
	\item Include our logo \& the words "Kukui Cup"
	\item Include 2012 somewhere
	\item Include our URL: kukuicup.manoa.hawaii.edu
	\item Make it stylin' and awesome
\end{itemize}

You will submit an image file with your T-shirt design for us to look at.

\vspace{2ex}
\textbf{Expected benefits:} Creative expression and reflection on a Kukui Cup topic, thinking about next year's challenge


\subsection{Wildcard}

\textbf{Description:} If you have a really creative idea for exploring one of the Kukui Cup topics that isn't covered by the other advanced activities, this is the place. We strongly recommend you contact the Kukui Cup admins using the \emph{Send Feedback} button at the top right corner of this web page and pitch your idea to us before you start on it. This will let us give you feedback on whether or not we think it is appropriate for this activity, and how many points it might be worth.

\vspace{2ex}
\textbf{Expected benefits:} Creative expression and reflection on a Kukui Cup topic


\section{Commitments}
\label{sec:commitment-list}

See \autoref{sec:commitments} for a description of what commitments were in the Kukui Cup and how they were processed. Note that commitments were participant-verified without outside intervention, so that field is not used for this category. \autoref{tab:commitment-list} shows a summary of the commitments. The unlocking pattern for commitments in the Smart Grid Game was quite simple: all commitments were unlocked after participants completed either the ``Secrets of the Kukui Cup'' or ``Power and Energy'' video activities.

\begin{table}[htbp]
	\centering
	\caption{A list of the commitments available during the challenge}
	\label{tab:commitment-list}
	\vskip 1em
	\begin{tabular}{| l | c | c |}
		\hline
		Commitment & Category & Points \tabularnewline \hline \hline
I will turn off vampire loads using a power strip & Basic Energy & 5 \\
I will turn off all appliances every night before going to sleep & Basic Energy & 5 \\
I will limit my TV use to 1 hour a day & Basic Energy & 5 \\
I will turn off the lights when leaving any room & Lights Out! & 5 \\
I will use task lighting instead of overhead lights & Lights Out! & 5 \\
I will use sunlight instead of electric lighting & Lights Out! & 5 \\
I will turn off printer when not printing & Lights Out! & 5 \\
I will do something `unplugged' every day & Lights Out! & 5 \\
I will turn off my music when leaving my room & Lights Out! & 5 \\
I will use stairs instead of elevator & Moving on & 5 \\
I won't drive alone & Moving on & 5 \\
I will take public transportation & Moving on & 5 \\
I will walk to destinations less than one mile away & Moving on & 5 \\
I will recycle all beverage containers & Opala & 5 \\
I will bring reusable bags when shopping & Opala & 5 \\
I will turn off water when brushing my teeth or shaving & Wet and Wild & 5 \\
I will turn off water when sudsing and scrubbing in shower & Wet and Wild & 5 \\
I will wash only full loads of laundry & Wet and Wild & 5 \\
I will wash my laundry in cold water & Wet and Wild & 5 \\
I will reduce my shower time by 1 minute & Wet and Wild & 5 \\
I will not eat meat & Mixed Bag & 5 \\ \hline
	\end{tabular}
\end{table}


\subsection{Turn off vampires}

\textbf{Description:} A vampire load is a device that uses power when plugged in, even when it is turned off and not doing anything. Commit to turning off any vampire loads (cell phone charger, iPod charger, game consoles, TVs) using a power strip when you are not using them, thereby saving energy. If you need a power strip, you can buy them at the UH Bookstore, or many other stores (grocery stores, drug stores, etc).

\vspace{2ex}
\textbf{Expected benefits:} Reduced electricity usage due to vampire loads, awareness of vampire loads.


\subsection{Off b4 bed}

\textbf{Description:} Commit to turning off all appliances in your room (computer, TVs, DVD/Blu-ray players, game consoles) every night before you go to sleep. Appliances use a significant amount of electricity, so turning them off when you definitely won't be using them (like when you are asleep) will save energy.

\vspace{2ex}
\textbf{Expected benefits:} Less electricity wasted on appliances that aren't being used.


\subsection{Limit TV}

\textbf{Description:} Commit to using your TV (watching shows, movies, playing games) for less than 1 hour per day. Widescreen TVs use a lot of electricity, so putting a limit on how much you use them will reduce your electricity use.

\vspace{2ex}
\textbf{Expected benefits:} Less electricity used by television.


\subsection{Turn off lights}

\textbf{Description:} Leaving lights on wastes energy for no purpose. Commit to turning off the lights when leaving any room.

\vspace{2ex}
\textbf{Expected benefits:} Reduced electricity usage due to less unneeded lighting, noticeable behavior reminder to others.


\subsection{Task lighting}

\textbf{Description:} Commit to using task lighting (like a desk lamp) instead of overhead room lights when possible. Often overhead lights provide more light than you need, or might not provide the light where you need it. Using a desk lamp will reduce your electricity use while giving you the light you need, where you need it.

\vspace{2ex}
\textbf{Expected benefits:} Reduced electricity usage due to less excess lighting.


\subsection{Use sunlight}

\textbf{Description:} Commit to using sunlight from windows or outdoors instead of turning on electric lighting. This can mean opening shades instead of turning on the lights, and/or planning your day so that tasks that require light (like reading books) are done during the day.

\vspace{2ex}
\textbf{Expected benefits:} Reduced electricity usage due to less use of electric lights.


\subsection{Printer off}

\textbf{Description:} Commit to turning off your printer when you aren't actively printing something out. This will reduce electricity use, since printers draw some power if they are turned on even when they aren't printing.

\vspace{2ex}
\textbf{Expected benefits:} Reduced electricity usage due to less standby electricity for printer.


\subsection{Pull the plug}

\textbf{Description:} Commit to turning off your computer/TV/game console and doing something that doesn't require electricity instead every day. There are many things you can do both on and off campus that don't require electricity, go find them!

\vspace{2ex}
\textbf{Expected benefits:} Reduced electricity usage, potentially increased exercise.


\subsection{Turn off music}

\textbf{Description:} Commit to turning off your music (from computer, stereo, etc) when you leave your room. You save electricity when you turn off your music when you aren't there to enjoy it.

\vspace{2ex}
\textbf{Expected benefits:} Reduced electricity usage.


\subsection{Use stairs}

\textbf{Description:} Commit to using the stairs instead of elevators during your day, whenever that is feasible. Elevators use electricity, so by using the stairs you will save some energy. Also, using the stairs is good exercise!

\vspace{2ex}
\textbf{Expected benefits:} Reduced electricity usage due to less elevator traffic, increased exercise for participant.


\subsection{Car pool}

\textbf{Description:} Commit to not driving in a car by yourself. Try riding the bus, riding a bike, walking, driving a moped, or using a vehicle with 3+ occupants instead. Transportation fuel is a major use of energy and it generates a lot of greenhouse gases, so traveling more efficiently saves energy and the planet.

\vspace{2ex}
\textbf{Expected benefits:} Reduced carbon emissions due to less single occupant car travel, reduction in traffic and parking.


\subsection{Take bus}

\textbf{Description:} Commit to taking public transportation whenever you go off campus during the commitment period. Every UH Manoa student gets a U-Pass sticker for their ID that allows unlimited free rides on the bus each semester! If for some reason you don't have your U-Pass, go to the ID counter in Campus Center.

TheBus has a great website that will help you plan trips, and even tell you when the next bus will arrive based on GPS location!

\vspace{2ex}
\textbf{Expected benefits:} Reduced carbon emissions due to less single occupant car travel, reduction in traffic and parking.


\subsection{Walk to destinations less than one mile away}

\textbf{Description:} Commit to walking to any destination less than one mile away. Walking saves energy, costs nothing, and is good exercise.

\vspace{2ex}
\textbf{Expected benefits:} Reduced gasoline usage due to car usage, increased exercise for participant.


\subsection{Recycle cans}

\textbf{Description:} Commit to recycling all (recyclable) beverage containers at one of the recycling bins on campus or around town. Making things from recycled materials generally costs less and uses less energy than making them from raw materials.

\vspace{2ex}
\textbf{Expected benefits:} Reduced carbon emissions due to recovery and eventual reuse of recyclable material, reduction in waste stream.


\subsection{Reusable bags}

\textbf{Description:} Commit to bringing and using reusable bags when shopping instead of the paper or plastic ones offered by stores. Making disposable bags requires energy, and often the bags end up in our landfills or worse yet they blow away into the ocean. Using a reusable bag saves energy and keeps trash out of our landfills.

\vspace{2ex}
\textbf{Expected benefits:} Reduced waste, reduced carbon footprint.


\subsection{Turn off sink}

\textbf{Description:} Commit to turning off water at sinks when you aren't actually using the water, such as when brushing your teeth, shaving, applying makeup, etc. Clean water is a valuable resource that shouldn't be wasted. Also pumping water from the ground into a building, heating it, and then treating the used water takes energy, so reducing the amount of water used saves energy.

\vspace{2ex}
\textbf{Expected benefits:} Reduced electricity usage due less pumping of water, reduced water use.


\subsection{Turn off shower}

\textbf{Description:} Commit to turning off water when showering except when actively rinsing off soap or shampoo. Clean water is a valuable resource that shouldn't be wasted. Also pumping water from the ground into a building, heating it, and then treating the used water takes energy, so reducing the amount of water used saves energy.

\vspace{2ex}
\textbf{Expected benefits:} Reduced electricity usage due less pumping and heating of water, reduced water use.


\subsection{Full loads of laundry}

\textbf{Description:} Commit to always washing full loads of laundry. Washing less than a full load is less efficient, leading to more electricity and water being used per piece of laundry washed.

\vspace{2ex}
\textbf{Expected benefits:} Less electricity and hot water used per item of laundry washed.


\subsection{Wash laundry in cold water}

\textbf{Description:} Commit to washing your laundry in cold water instead of warm or hot water. There are now detergents designed to be used in cold water, and it takes lots of energy to heat water up. By using cold water, you will be saving energy.

\vspace{2ex}
\textbf{Expected benefits:} Reduced electricity usage by reduction in water heating and pumping.


\subsection{Shorter showers}

\textbf{Description:} Commit to measuring the length of your shower with a watch or phone, and reducing the time by 1 minute. Clean water is a valuable resource that shouldn't be wasted. Also pumping water from the ground into a building, heating it, and then treating the used water takes energy, so reducing the amount of water used saves energy.

\vspace{2ex}
\textbf{Expected benefits:} Reduced electricity usage by reduction in water heating and pumping.


\subsection{Go meatless}

\textbf{Description:} Commit to not eating any meat (beef, pork, chicken, fish, shellfish, etc) during the commitment period. Producing meat (beef in particular) uses a great deal of energy, and produces a great deal of greenhouse gasses. A vegetarian diet uses less energy and emits less greenhouse gasses. There are many of vegetarian food options both on campus and around Honolulu, try them out!

\vspace{2ex}
\textbf{Expected benefits:} Reduced carbon footprint, potentially improved health.


\section{Events}

See \autoref{sec:events} for a description of how events were handled in the Kukui Cup and how they were processed. \autoref{tab:event-list} shows a summary of the events. The unlocking pattern for events was completely time based: events were unlocked seven days before they occurred, and remained unlocked for seven days after the event took place (to allow time for entry of attendance codes).

\begin{table}[htbp]
	\centering
	\caption[A list of the events available during the challenge]{A list of the events available during the challenge. Entries marked with an asterisk are off-campus excursions.}
	\label{tab:event-list}
	\vskip 1em
	\begin{tabular}{| l | c | c |}
		\hline
		Event name & Date/Time & Points \tabularnewline \hline \hline
Kickoff Party & 2011-10-17 18:30 & 20 \\
Play outside the cafe (1) & 2011-10-18 18:30 & 10 \\
Energy scavenger hunt & 2011-10-18 22:00 & 20 \\
Recycled fashion design & 2011-10-19 22:00 & 20 \\
Play outside the cafe (2) & 2011-10-20 18:30 & 10 \\
Flashmob design & 2011-10-20 22:00 & 20 \\
Kahuku Wind Farm\ensuremath{^*} & 2011-10-22 10:00 & 30 \\
Sustainable and Organic Farming & 2011-10-22 16:00 & 20 \\
Pedalpalooza & 2011-10-23 15:00 & 20 \\
UH Manoa Food Day & 2011-10-24 13:00 & 20 \\
Round 1 Awards Party & 2011-10-24 18:30 & 20 \\
Play outside the cafe (3) & 2011-10-25 18:30 & 10 \\
Your Sustainable Future & 2011-10-25 22:00 & 20 \\
Energy Efficient Eating & 2011-10-26 22:00 & 20 \\
Play outside the cafe (4) & 2011-10-27 18:30 & 10 \\
Movie Night & 2011-10-27 22:00 & 20 \\
Off-The-Grid Living\ensuremath{^*} & 2011-10-29 10:30 & 40 \\
Kokua Market Excursion\ensuremath{^*} & 2011-10-30 12:00 & 25 \\
Round 2 Awards Party & 2011-11-01 18:30 & 20 \\
Manoa Sustainability Corps & 2011-11-02 15:30 & 20 \\
High Energy Art and Music & 2011-11-02 22:00 & 20 \\
Energy Efficient Chillaxation & 2011-11-03 22:00 & 20 \\
First Green Friday & 2011-11-04 10:00 & 15 \\
North Shore Beach Cleanup\ensuremath{^*} & 2011-11-05 09:00 & 45 \\ \hline
	\end{tabular}
\end{table}


\subsection{Kickoff Party}

\textbf{Description:} It has begun: The Quest for the Kukui Cup 2011! If you've been wondering about the Kukui Cup banners, all will be revealed at the Kickoff Party, hosted by MC Kai and MC Cookie and featuring sick beatz by the infamous DJ Mr Nick. Get there early to score your \textbf{free} limited edition Kukui Cup 2011 t-shirt and a \textbf{secret high tech gadget} to help you in your quest for energy saving supremacy.

\vspace{2ex}
\textbf{Expected benefits:} Introducing residents to the Kukui Cup, providing t-shirts to promote the challenge, and smart strips to reduce energy usage.


\subsection{Play outside the cafe}

\textbf{Description:} On your way to dinner? Stop by the Kukui Cup table outside the Hale Aloha cafeteria to play an energy game. If you succeed, you can win a \textbf{free} prize. Even if you don't, you can get an Attendance Code and earn some points. The prizes vary from night to night, so stop by every time and try to collect them all. The table goes away when all the prizes have been given out, so get there early to maximize your chances! This is the first of four Play Outside The Cafe events.

\vspace{2ex}
\textbf{Expected benefits:} Increase energy literacy through game, promote challenge through distributed swag, and physical reminder about the challenge in a heavily trafficked place.


\subsection{Energy scavenger hunt}

\textbf{Description:} Yes, we've all scavenged for leaves and cute rocks and flowers in grade school.  But this isn't Miss Mizumoto's second grade science class: you're going after the big game now -- Kilowatts!

We'll start by teaching you how to measure power. Then, you'll divide up into teams, and get exactly 30 minutes to go back to your tower and measure the power used by appliances. Prizes will be awarded to both the team that finds the appliance that uses the least amount of power as well as well as the team that finds the appliance that uses the most amount of power.

Note: at least one member of each team needs a camera (cell phone camera OK) in order to take a picture of the appliance reading.  Free food at the end of the night? We've got you covered.

\vspace{2ex}
\textbf{Expected benefits:} Increased energy intuition, familiarity with plug-load meters


\subsection{Recycled fashion design}

\textbf{Description:} As Heidi Klum reminds us, ``In fashion, one day you're in, and the next day you're out.'' Go fashion forward by attending the Recycled Fashion Design Workshop, hosted by Project Runway Season 8 Finalist Andy South.

Assisted by UH Manoa fashion design students, you'll form small groups and use recycled materials to create a new look, while Andy provides advice and encouragement. Then a model will walk the runway to show off your creation. No matter what colors you choose, your look will be green! If you just want to watch, that's fine too.

After party snacks included, so sign up soon!

\vspace{2ex}
\textbf{Expected benefits:} understanding sustainability benefits from reused clothing, awareness of Goodwill for purchasing used clothing


\subsection{Flashmob design}

\textbf{Description:} Do you ever experience an intense, uncontrollable urge to break into song and dance in large, public places? If you've got that fever, we've got the cure: a heaping helping of the Kukui Cup Flashmob.

At this workshop, you'll start designing a clandestine energy-related song and dance skit to be busted out near the end of the Kukui Cup while consuming free munchies. Who knows, it could be the YouTube hit of November, 2011.

\vspace{2ex}
\textbf{Expected benefits:} group work, promotion of the challenge


\subsection{Kahuku Wind Farm}

\textbf{Description:} Want to see sky farmers harvesting the winds? Come with us to Kahuku to see firsthand how energy is plucked from the sky and generated for our use by First Wind's turbines.

The wind farm staff requests that everyone wear long pants and closed toe shoes.  We hope to stop in Kahuku for lunch, so bring some money.

You need to register for this free event to reserve your seat on the bus by clicking the \textbf{I want to sign up} button below.

Meet in the Hale Aloha courtyard, from there we'll get on the bus.  Don't be late!

\vspace{2ex}
\textbf{Expected benefits:} Better understanding of wind power


\subsection{Sustainable and Organic Farming}

\textbf{Description:} Your mother always told you to eat your vegetables, but did you ever consider where they came from while you forced down that last bite of rutabaga? The Sustainable Organic Farm Training (SOFT) club is a student-run organization devoted to getting at the ``roots'' of fresh produce, literally and figuratively. 

At this workshop you'll get a chance to help out at the farm, taste fresh produce, and discover out what it really means to eat natural, local, organic, sustainable produce! Yum!!

\vspace{2ex}
\textbf{Expected benefits:} Understanding of farming and its relationship to energy, introduction to the SOFT campus group


\subsection{Pedalpalooza}

\textbf{Description:} Is the Queen song ``I want to ride my bicycle'' whirring furiously through your brain while you stare down at your broken two wheeler? Have no fear, Freddy Mercury fans! Cycle Manoa is here to save your day with the Pedalpalooza workshop. If your wheels are broken, they'll teach you how to fix them for free. If your wheels are rocking, join them for a quick one hour ride around Manoa. And you can even cool down afterwards with a free bicycle-powered smoothie.

Meet at Hale Aloha Courtyard at 3pm with your wheels for a guided 1 hour ride, ending up at the Cycle Manoa HQ. If you don't have a bike but would like to attend, meet in the courtyard at 3:40 and we'll walk up to the Cycle Manoa together. At 4 PM you'll hear from the Cycle Manoa team about bicycle advocacy and bicycle repair.

\vspace{2ex}
\textbf{Expected benefits:} Understanding of benefits of bicycle transportation compared to fossil-fuel driven vehicles, introduction to Cycle Manoa group


\subsection{UH Manoa Food Day}

\textbf{Description:} Do you care about your food? Want to find out more about how to eat tasty, healthy food? Come to the UH Manoa Food Day event, which will include presentations on nutrition and food followed by Dr. Ted Radovitch a CTAHR specialist in Sustainable and Organic Farming Systems, and Dean Okimoto from Nalo Farms.

Following the presentations will be a food demonstration by Philip Shon, UH Sodexo Executive Chef who works in collaboration with Donna Ojiri, RD, General Manager of Sodexo. Taste local fresh produce, sample grass-fed Big Island beef, and local fruit beverages. By celebrating food day, helps emphasize the importance of making healthy food choices, and promote changes in food and farm policies that benefit health, the environment and well-being of us all in Hawaii.

Since this is an external event, you should sign up here but also RSVP on the UHM Food Day website (will open in a new window). When you get to the event, look for a Kukui Cup staff member (white t-shirt), and they will give you your attendance code that will get you the points for this event.

\vspace{2ex}
\textbf{Expected benefits:} Understanding of food's relationship to sustainability


\subsection{Round 1 Awards Party}

\textbf{Description:} If you have an indiscernible memory of attending an awesome awards party before, you may be experiencing some pre-deja-vu of what is soon to come: the Kukui Cup's first Round 1 Awards Party - a melange of interactive energy awareness games, an ultra-cool student DJ who goes by the oh-so-natural name of Pearl, the last of the custom limited edition Kukui Cup t-shirts, and the ever-popular smart strips.

As you are reading this, you may experience pre-withdrawals from the general awesomeness of this Party, oh, and did I mention that Awards will be being handed out as well? Yes, it may be implied in the name of this event, but along with the natural high you will inevitably feel from all the free goodies while simultaneously doing something good for the earth and jamming out to clam shell beats, you may just find yourself going home with a cool prize. 

Deja-vu or dream come true? You decide.

\vspace{2ex}
\textbf{Expected benefits:} Promotion of the challenge, distribution of incentives


\subsection{Your Sustainable Future}

\textbf{Description:} A famous British poet once wrote: ``You say you want a revolution? Well, you know, we'd all love to see the plan.''

If you're interested in helping create an energy and sustainability revolution in your classes, university and community, come plan with representatives from Blue Planet Foundation, Sustainable UH, Surfrider Foundation, Kokua Hawaii Foundation, College of Engineering, School of Architecture, Shidler College of Business, Environmental Studies, and more.

Planning the overthrow of our oil-based economy will work up an appetite, so we'll also provide snacks.

\vspace{2ex}
\textbf{Expected benefits:} Introduction to sustainability organizations, awareness of classes on sustainability topics


\subsection{Energy Efficient Eating}

\textbf{Description:} Has cafeteria food got you down in the dumps? Are you no longer amused by mystery meat? Want to get new ideas for late night munchies?

Join experts from Kokua Market in a discussion of where our food comes from in Hawaii, and inexpensive, residence hall friendly groceries. You'll sample a variety of free gourmet popcorn toppings and learn how to make your own for just pennies a serving.

We'll even stuff your goodie bag with a custom recipe book to cure those Hale Aloha hunger pangs.

\vspace{2ex}
\textbf{Expected benefits:} Understanding of food's role in sustainability, awareness of where to purchase locally-produced foods, ways to prepare food with less energy


\subsection{Movie Night}

\textbf{Description:} Watch two of the artsiest and the most hilarious shorts from the Bike Shorts Film Festival Hawaii and continue the night with the journey of a revolutionary architect in a maze of obstacles towards sustainable communities of ``Earth Ships'', completely energy autonomous off-the-grid houses built with recycled materials.

An adventure full of beautiful images and extraordinary personages. Accompanied by free popcorn and free delicious lemonade!

\vspace{2ex}
\textbf{Expected benefits:} Awareness of options for sustainable living


\subsection{Off-The-Grid Living}

\textbf{Description:} The Reppun family has been living on their farm and growing taro, coffee, honey and other food in beautiful Waihole valley for over 20 years.  Though they have the Internet, they don't any power lines. See how they live off the grid in comfort and style through hydro-electric and solar power. You'll take a bus over to the Windward side, hike into the valley to their farm, and see an amazing blend of old school and next generation Hawaii. Make sure you eat breakfast beforehand; we won't be back until after lunch.

Make sure to wear clothes and shoes that you don't mind getting wet or muddy on the farm!

Reserve your seat on the bus by clicking the \textbf{I want to sign up} button below. 

Meet in the Hale Aloha courtyard, from there we'll get on the bus.  Don't be late!

\vspace{2ex}
\textbf{Expected benefits:} Understanding of real-world renewable energy options and farming as a part of sustainable living


\subsection{Kokua Market Excursion}

\textbf{Description:} Does purchasing groceries from corporate supermarkets leave a negative taste in your mouth? Satiate your craving to support local farmers and businesses by taking a tour of the Kokua Market, the only natural foods cooperative in Hawaii.

Sample foods whilst browsing a bountiful bulk selection, innovative deli items, and a plethora of produce, and learn why Coops are so crucial to the food chain of Hawaii.  At Kokua Market, the customer reigns supreme, not profit.

Meet at Hale Aloha Courtyard and we'll walk over -- it's just five minutes away!

\vspace{2ex}
\textbf{Expected benefits:} Awareness of where to purchase locally-produced foods


\subsection{Round 2 Awards Party}

\textbf{Description:} If you went to the Kickoff and Round 1 parties (which were, obviously, awesome), you might be thinking you can skip the Round 2 party. But that would be a huge FAIL because the Round 2 party is going to totally kick it.

There will be live music by Breath of Fire, speakers from local organizations like Blue Planet Foundation and Sustainable UH, awards to Round 2 winners, and some special surprises.  The Round 2 award party is being organized by the Aloha Movement Project so you know it's gonna rock.

The party starts at 5:30pm, awards are at 6:30pm, and Breath of Fire is playing two sets at 6 and 7pm.

\vspace{2ex}
\textbf{Expected benefits:} Promotion of the challenge, distribution of incentives


\subsection{Manoa Sustainability Corps}

\textbf{Description:} Want to find out how UH's sustainability efforts tie together? Come to the monthly meeting of the UH Manoa Sustainability Corps. The Sustainability Corps is a forum for all of us to share information, ideas, data, and suggestions regarding sustainability on campus. It is also a forum to propose projects and programs that will make UH Manoa a green leader in Hawai'i and abroad.

This external event is being held in Krauss Hall, Room 012, known as the Yukiyoshi Room. When you get to the event, look for a Kukui Cup staff member (white t-shirt), and they will give you your attendance code that will get you the points for this event.

\vspace{2ex}
\textbf{Expected benefits:} Introduction to Manoa Sustainability Corps, awareness of opportunities to get involved in sustainability on campus


\subsection{High Energy Art and Music}

\textbf{Description:}

\begin{verse}
Electric blood flows through the veins of the city\\
drip, drip, drip, every drip a drop of oil\\
do I flip the switch or does the switch flip me?\\
\end{verse}

Oh. hey there. We just get carried away when we think of slam poetry. And saving electricity. If you feel the same way, join us with Kealoha, Hawaii's premier slam poet. Afterwards, you'll have a chance to lay down verse of your own with the open mic session and munch on free snacks.

\vspace{2ex}
\textbf{Expected benefits:} Understanding of how art can promote sustainability


\subsection{Energy Efficient Chillaxation}

\textbf{Description:} Stress is the cancer of emotions, and undue amounts can lead to the demise of your study habits!

At the Energy Efficient Chillaxation Workshop, discover green methods to reduce stress, such as pranayama (deep breathing techniques), massage, and yoga. We'll provide you with free herbal tea and snacks, teach you some techniques, and help you blow off steam.

But don't stress out too much about being on time!

\vspace{2ex}
\textbf{Expected benefits:} Awareness of ways to relax that don't require electricity


\subsection{First Green Friday}

\textbf{Description:} First Green Friday is a showcase is to bring together faculty, students, and staff to showcase UHM's sustainability education, research, and demonstration projects. This new event is launching for the first time this Friday! Check out groups like the Environmental Center, the Ecology Club, Surfrider and the UHM Sustainability Corps. The Kukui Cup will have a table there as well!

This event is taking place in the Sustainability Courtyard, which is between Kuykendall Hall and the Hawaii Institute for Geophysics. Check out the booths at the event, and find the Kukui Cup table to get your attendance code.

\vspace{2ex}
\textbf{Expected benefits:} Awareness of opportunities to get involved in sustainability on campus


\subsection{North Shore Beach Cleanup}

\textbf{Description:} According to Wikipedia, ``utopia'' is an ideal community or society possessing a perfect socio-politico-legal system. Hawaii, perhaps the closest thing we have to environmental perfection on earth, is regularly polluted by all the garbage washing up on our precious beaches.

Do your part to bring Oahu one step closer to utopia by attending this beach cleanup sponsored by Surfrider Foundation. Yes, you'll have to wake up early, but it's totally worth it: a free ride to the North Shore, a couple of hours making Haleiwa Beach even more beautiful than it already is, then a free lunch and prizes provided by Spy Optics!

Make sure you bring a hat, sunscreen, swim suit, and water (in a reusable water bottle, of course!)

Reserve your spot on the bus by clicking the \textbf{I want to sign up} button below. Spaces are limited.

Meet in the Hale Aloha courtyard, from there we'll get on the bus. Don't be late, this one leaves early. Note that this excursion is worth the most points of all! Good attendance by your lounge could just put you over the top!

\vspace{2ex}
\textbf{Expected benefits:} Awareness of waste stream and how it impacts Hawaii's beaches
