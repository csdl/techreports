\chapter{Participant Actions}
\label{app:actions}

This appendix lists the actions available to 2011 Kukui Cup participants. Overall, the actions were intended to increase the energy literacy of the participants performing it, help them modify their behavior to reduce their electricity usage, or both. However, not every action met these goals. For example, some actions were included that were related to sustainability in general, and linked to energy only indirectly. Other actions were included primarily for the entertainment of participants, in keeping with the design of the competition as an interesting and fun game to play.

The following sections list all the actions, and indicate how they would be performed, and validated by administrators. The actions are grouped into three categories: activities, commitments, and events.


\section{Activities}

See \autoref{sec:activities} for a description of what activities were in the Kukui Cup and how they were processed.

\begin{table}[htbp]
	\centering
		\begin{tabular}{| l | c | c | c | c |}
			\hline
			Action type & Title & Points & Round available & Depends on \tabularnewline \hline \hline
			Lehua-A & 3/30/11 \\
			Lehua-C & 3/30/11 \\
			Lehua-D & 3/30/11 \\
			Lehua-E & 3/30/11 \\
			Lehua-B & 3/31/11 \\
			Ilima-A & 9/9/11 \\
			Ilima-B & 9/9/11 \\
			Ilima-C & 9/9/11 \\
			Ilima-D & 9/9/11 \\
			Ilima-E & 9/9/11 \\
			Mokihana-A & 9/9/11 \\
			Mokihana-B & 9/9/11 \\
			Mokihana-C & 9/9/11 \\
			Mokihana-D & 9/9/11 \\
			Mokihana-E & 9/9/11 \\
			Lokelani-A & 10/6/11 \\
			Lokelani-E & 10/6/11 \\
			Lokelani-D & 10/11/11 \\
			Lokelani-B & 10/14/11 \\
			\emph{Lokelani C} & \emph{N/A} \\ \hline
		\end{tabular}
	\caption{A list of the actions available during the competition}
\label{tab:action-list}
\end{table}



\subsection{Perform room energy audit}

Description: Resident borrows a Kill-A-Watt plug load meter from their RA, then checks all plug-in appliances in their room to see what their energy consumption is when on and off.

Verification: Participant fills out form on website that contains a list of rows for each device with columns: device name, power (watts) when off, power (watts) when on, notes. Admin reviews data, checking mainly for completeness (more than 1 device?) and sanity (XBox 360s don't use 1000 W).

Reward: 10 KN

Expected benefits: Increased intuitive understanding of the watt, familiarity with vampire power, understanding of how device usage would impact energy consumption, reduced electricity usage due to turning off devices when not in use.

Psychological justifications: feedback, activity-based learning (?)

\subsection{Replace incandescent bulb with compact fluorescent (CFL)}

Description: Participant finds an incandescent bulb (perhaps from a desk lamp) and replaces it with a CFL, throwing away the incandescent bulb.

Verification: Participant takes a picture showing both the incandescent bulb and the CFL replacement and uploads it via a verification form on the website, along with a text field indicating where the replaced bulb is located. Admin briefly reviews the picture to ensure that in fact both bulbs are present.

Reward: 3 KN

Expected benefits: Reduced energy usage via CFL, awareness of energy impact of incandescent bulbs.

Psychological justifications: activity-based learning (?)

\subsection{Configure computer \& monitor to sleep after inactivity}

Description: Participant configures their computer and any external display to sleep after 20 minutes of inactivity.

Verification: Participant takes a screenshot from their computer showing sleep settings <= 20 minutes and uploads it via a verification form on the website. Admin briefly reviews the picture to ensure that the settings look correct.

Reward: 3 KN

Expected benefits: Reduced computer \& monitor energy usage, knowledge of how to set it up on other computers (friends, work, future purchases, etc).

Psychological justifications: none

\subsection{Play in EnergyPong tournament}

Description: Participant is on their floor's team in the EnergyPong tournament for their building.

Verification: Some responsible person who is not a participant (such as the speaker or an RA) records attendance and performance, which is reported to the website admins either on paper or via email.

Reward: 4 KN + 1 KN per bracket completed + 5 KN for the winning team

Expected benefits: Improved energy literacy through answering energy questions answered, floor bonding.

Psychological justifications: competition, incentives (if prizes are awarded to winning team)

\subsection{Connect to Kukui Cup on Facebook}

Description: Participant becomes a fan of the Kukui Cup Competition group on Facebook.

Verification: Participant takes a screenshot from their computer showing Facebook fan status. Admin briefly reviews the picture to ensure that the participant is a fan.

Reward: 3 KN

Expected benefits: Another avenue for communicating with students, promotion of the contest and energy literacy.

Psychological justifications: community involvement?

\subsection{Tweet about Kukui Cup}

Description: Participant sends a tweet promoting the Kukui Cup Competition with a link to the website.

Verification: Participant takes a screenshot from their computer showing the tweet in their newsfeed. Admin briefly reviews the picture to ensure that the participant tweeted.

Reward: 2 KN

Expected benefits: Promotion of the contest and energy literacy.

Psychological justifications: social networking?

\subsection{Facebook Status update about Kukui Cup}

Description: Participant updates their Facebook status promoting the Kukui Cup Competition with a link to the website.

Verification: Participant takes a screenshot from their computer showing the status in their newsfeed. Admin briefly reviews the picture to ensure that the participant updated their status.

Reward: 2 KN

Expected benefits: Promotion of the contest and energy literacy.

Psychological justifications: social networking?

\subsection{Label all plug loads in room}

Description: Followup to room energy audit. Based on the audit results, make a label for each device with the number of watts consumed when on and off, located close to the power switch for those devices that have them.

Verification: Participant takes a picture of the devices with their labels. Admin briefly reviews the picture to ensure that labels are present.

Reward: 3 KN

Expected benefits: understanding of how device usage would impact energy consumption, reduced electricity usage due to turning off devices when not in use.

Psychological justifications: prompts

\subsection{Determine carbon footprint using calculator}

Description: Participant uses a web-based carbon footprint calculator to determine their carbon footprint.

Verification: Participant enters in their computed carbon footprint into a text field. Admin briefly reviews the footprint to make sure it is sane (units include CO2 and it isn't huge or tiny).

Reward: 3 KN

Expected benefits: learning about carbon emissions, learning how carbon emissions impact the environment.

Psychological justifications: personalized data


\section{Commitments}

See \autoref{sec:commitments} for a description of what commitments were in the Kukui Cup and how they were processed. Note that commitments were participant-verified without outside intervention, so that field is not used for this category. \autoref{tab:commitment-list} shows a summary of the commitments. The unlocking pattern for commitments in the Smart Grid Game was quite simple: all commitments were unlocked after participants completed either the ``Secrets of the Kukui Cup'' or ``Power and Energy'' video activities.

\begin{table}[htbp]
	\centering
		\begin{tabular}{| l | c | c |}
			\hline
			Commitment & Category & Points \tabularnewline \hline \hline
I will turn off vampire loads using a power strip & Basic Energy & 5 \\
I will turn off all appliances every night before going to sleep & Basic Energy & 5 \\
I will limit my TV use to 1 hour a day & Basic Energy & 5 \\
I will turn off the lights when leaving any room & Lights Out! & 5 \\
I will use task lighting instead of overhead lights & Lights Out! & 5 \\
I will use sunlight instead of electric lighting & Lights Out! & 5 \\
I will turn off printer when not printing & Lights Out! & 5 \\
I will do something `unplugged' every day & Lights Out! & 5 \\
I will turn off my music when leaving my room & Lights Out! & 5 \\
I will use stairs instead of elevator & Moving on & 5 \\
I won't drive alone & Moving on & 5 \\
I will take public transportation & Moving on & 5 \\
I will walk to destinations less than one mile away & Moving on & 5 \\
I will recycle all beverage containers & Opala & 5 \\
I will bring reusable bags when shopping & Opala & 5 \\
I will turn off water when brushing my teeth or shaving & Wet and Wild & 5 \\
I will turn off water when sudsing and scrubbing in shower & Wet and Wild & 5 \\
I will wash only full loads of laundry & Wet and Wild & 5 \\
I will wash my laundry in cold water & Wet and Wild & 5 \\
I will reduce my shower time by 1 minute & Wet and Wild & 5 \\
I will not eat meat & Mixed Bag & 5 \\ \hline
		\end{tabular}
	\caption{A list of the commitments available during the competition}
\label{tab:commitment-list}
\end{table}


\subsection{Turn off vampires}

Description: A vampire load is a device that uses power when plugged in, even when it is turned off and not doing anything. Commit to turning off any vampire loads (cell phone charger, iPod charger, game consoles, TVs) using a power strip when you are not using them, thereby saving energy. If you need a power strip, you can buy them at the UH Bookstore, or many other stores (grocery stores, drug stores, etc).

Expected benefits: Reduced electricity usage due to vampire loads, awareness of vampire loads.


\subsection{Off b4 bed}

Description: Commit to turning off all appliances in your room (computer, TVs, DVD/Blu-ray players, game consoles) every night before you go to sleep. Appliances use a significant amount of electricity, so turning them off when you definitely won't be using them (like when you are asleep) will save energy.

Expected benefits: Less electricity wasted on appliances that aren't being used.


\subsection{Limit TV}

Description: Commit to using your TV (watching shows, movies, playing games) for less than 1 hour per day. Widescreen TVs use a lot of electricity, so putting a limit on how much you use them will reduce your electricity use.

Expected benefits: Less electricity used by television.


\subsection{Turn off lights}

Description: Leaving lights on wastes energy for no purpose. Commit to turning off the lights when leaving any room.

Expected benefits: Reduced electricity usage due to less unneeded lighting, noticeable behavior reminder to others.


\subsection{Task lighting}

Description: Commit to using task lighting (like a desk lamp) instead of overhead room lights when possible. Often overhead lights provide more light than you need, or might not provide the light where you need it. Using a desk lamp will reduce your electricity use while giving you the light you need, where you need it.

Expected benefits: Reduced electricity usage due to less excess lighting.


\subsection{Use sunlight}

Description: Commit to using sunlight from windows or outdoors instead of turning on electric lighting. This can mean opening shades instead of turning on the lights, and/or planning your day so that tasks that require light (like reading books) are done during the day.

Expected benefits: Reduced electricity usage due to less use of electric lights.


\subsection{Printer off}

Description: Commit to turning off your printer when you aren't actively printing something out. This will reduce electricity use, since printers draw some power if they are turned on even when they aren't printing.

Expected benefits: Reduced electricity usage due to less standby electricity for printer.


\subsection{Pull the plug}

Description: Commit to turning off your computer/TV/game console and doing something that doesn't require electricity instead every day. There are many things you can do both on and off campus that don't require electricity, go find them!

Expected benefits: Reduced electricity usage, potentially increased exercise.


\subsection{Turn off music}

Description: Commit to turning off your music (from computer, stereo, etc) when you leave your room. You save electricity when you turn off your music when you aren't there to enjoy it.

Expected benefits: Reduced electricity usage.


\subsection{Use stairs}

Description: Commit to using the stairs instead of elevators during your day, whenever that is feasible. Elevators use electricity, so by using the stairs you will save some energy. Also, using the stairs is good exercise!

Expected benefits: Reduced electricity usage due to less elevator traffic, increased exercise for participant.


\subsection{Car pool}

Description: Commit to not driving in a car by yourself. Try riding the bus, riding a bike, walking, driving a moped, or using a vehicle with 3+ occupants instead. Transportation fuel is a major use of energy and it generates a lot of greenhouse gases, so traveling more efficiently saves energy and the planet.

Expected benefits: Reduced carbon emissions due to less single occupant car travel, reduction in traffic and parking.


\subsection{Take bus}

Description: Commit to taking public transportation whenever you go off campus during the commitment period. Every UH Manoa student gets a U-Pass sticker for their ID that allows unlimited free rides on the bus each semester! If for some reason you don't have your U-Pass, go to the ID counter in Campus Center.

TheBus has a great website that will help you plan trips, and even tell you when the next bus will arrive based on GPS location!

Expected benefits: Reduced carbon emissions due to less single occupant car travel, reduction in traffic and parking.


\subsection{Walk to destinations less than one mile away}

Description: Commit to walking to any destination less than one mile away. Walking saves energy, costs nothing, and is good exercise.

Expected benefits: Reduced gasoline usage due to car usage, increased exercise for participant.


\subsection{Recycle cans}

Description: Commit to recycling all (recyclable) beverage containers at one of the recycling bins on campus or around town. Making things from recycled materials generally costs less and uses less energy than making them from raw materials.

Expected benefits: Reduced carbon emissions due to recovery and eventual reuse of recyclable material, reduction in waste stream.


\subsection{Reusable bags}

Description: Commit to bringing and using reusable bags when shopping instead of the paper or plastic ones offered by stores. Making disposable bags requires energy, and often the bags end up in our landfills or worse yet they blow away into the ocean. Using a reusable bag saves energy and keeps trash out of our landfills.

Expected benefits: Reduced waste, reduced carbon footprint.


\subsection{Turn off sink}

Description: Commit to turning off water at sinks when you aren't actually using the water, such as when brushing your teeth, shaving, applying makeup, etc. Clean water is a valuable resource that shouldn't be wasted. Also pumping water from the ground into a building, heating it, and then treating the used water takes energy, so reducing the amount of water used saves energy.

Expected benefits: Reduced electricity usage due less pumping of water, reduced water use.


\subsection{Turn off shower}

Description: Commit to turning off water when showering except when actively rinsing off soap or shampoo. Clean water is a valuable resource that shouldn't be wasted. Also pumping water from the ground into a building, heating it, and then treating the used water takes energy, so reducing the amount of water used saves energy.

Expected benefits: Reduced electricity usage due less pumping and heating of water, reduced water use.


\subsection{Full loads of laundry}

Description: Commit to always washing full loads of laundry. Washing less than a full load is less efficient, leading to more electricity and water being used per piece of laundry washed.

Expected benefits: Less electricity and hot water used per item of laundry washed.


\subsection{Wash laundry in cold water}

Description: Commit to washing your laundry in cold water instead of warm or hot water. There are now detergents designed to be used in cold water, and it takes lots of energy to heat water up. By using cold water, you will be saving energy.

Expected benefits: Reduced electricity usage by reduction in water heating and pumping.


\subsection{Shorter showers}

Description: Commit to measuring the length of your shower with a watch or phone, and reducing the time by 1 minute. Clean water is a valuable resource that shouldn't be wasted. Also pumping water from the ground into a building, heating it, and then treating the used water takes energy, so reducing the amount of water used saves energy.

Expected benefits: Reduced electricity usage by reduction in water heating and pumping.


\subsection{Go meatless}

Description: Commit to not eating any meat (beef, pork, chicken, fish, shellfish, etc) during the commitment period. Producing meat (beef in particular) uses a great deal of energy, and produces a great deal of greenhouse gasses. A vegetarian diet uses less energy and emits less greenhouse gasses. There are many of vegetarian food options both on campus and around Honolulu, try them out!

Expected benefits: Reduced carbon footprint, potentially improved health.


\section{Events}

See \autoref{sec:events} for a description of how events were handled in the Kukui Cup and how they were processed. \autoref{tab:event-list} shows a summary of the events. The unlocking pattern for events was completely time based: events were unlocked 7 days before they occurred, and remained unlocked for 7 days after the event took place (to allow time for entry of attendance codes).

\begin{table}[htbp]
	\centering
		\begin{tabular}{| l | c | c |}
			\hline
			Event name & Date/Time & Points \tabularnewline \hline \hline
Kickoff Party & 2011-10-17 18:30 & 20 \\
Play outside the cafe (1) & 2011-10-18 18:30 & 10 \\
Energy scavenger hunt & 2011-10-18 22:00 & 20 \\
Recycled fashion design & 2011-10-19 22:00 & 20 \\
Play outside the cafe (2) & 2011-10-20 18:30 & 10 \\
Flashmob design & 2011-10-20 22:00 & 20 \\
Kahuku Wind Farm\ensuremath{^*} & 2011-10-22 10:00 & 30 \\
Sustainable and Organic Farming & 2011-10-22 16:00 & 20 \\
Pedalpalooza & 2011-10-23 15:00 & 20 \\
UH Manoa Food Day & 2011-10-24 13:00 & 20 \\
Round 1 Awards Party & 2011-10-24 18:30 & 20 \\
Play outside the cafe (3) & 2011-10-25 18:30 & 10 \\
Your Sustainable Future & 2011-10-25 22:00 & 20 \\
Energy Efficient Eating & 2011-10-26 22:00 & 20 \\
Play outside the cafe (4) & 2011-10-27 18:30 & 10 \\
Movie Night & 2011-10-27 22:00 & 20 \\
Off-The-Grid Living\ensuremath{^*} & 2011-10-29 10:30 & 40 \\
Kokua Market Excursion\ensuremath{^*} & 2011-10-30 12:00 & 25 \\
Round 2 Awards Party & 2011-11-01 18:30 & 20 \\
Manoa Sustainability Corps & 2011-11-02 15:30 & 20 \\
High Energy Art and Music & 2011-11-02 22:00 & 20 \\
Energy Efficient Chillaxation & 2011-11-03 22:00 & 20 \\
First Green Friday & 2011-11-04 10:00 & 15 \\
North Shore Beach Cleanup\ensuremath{^*} & 2011-11-05 09:00 & 45 \\ \hline
		\end{tabular}
	\caption[A list of the events available during the competition]{A list of the events available during the competition. Entries marked with an asterisk are off-campus excursions.}
\label{tab:event-list}
\end{table}


\subsection{Kickoff Party}

Description: It has begun: The Quest for the Kukui Cup 2011! If you've been wondering about the Kukui Cup banners, all will be revealed at the Kickoff Party, hosted by MC Kai and MC Cookie and featuring sick beatz by the infamous DJ Mr Nick. Get there early to score your \textbf{free} limited edition Kukui Cup 2011 t-shirt and a \textbf{secret high tech gadget} to help you in your quest for energy saving supremacy.

Expected benefits: Introducing residents to the Kukui Cup, providing t-shirts to promote the competition, and smart strips to reduce energy usage.


\subsection{Play outside the cafe}

Description: On your way to dinner? Stop by the Kukui Cup table outside the Hale Aloha cafeteria to play an energy game. If you succeed, you can win a \textbf{free} prize. Even if you don't, you can get an Attendance Code and earn some points. The prizes vary from night to night, so stop by every time and try to collect them all. The table goes away when all the prizes have been given out, so get there early to maximize your chances! This is the first of four Play Outside The Cafe events.

Expected benefits: Increase energy literacy through game, promote competition through distributed swag, and physical reminder about the competition in a heavily trafficked place.


\subsection{Energy scavenger hunt}

Description: Yes, we've all scavenged for leaves and cute rocks and flowers in grade school.  But this isn't Miss Mizumoto's second grade science class: you're going after the big game now -- Kilowatts!

We'll start by teaching you how to measure power. Then, you'll divide up into teams, and get exactly 30 minutes to go back to your tower and measure the power used by appliances. Prizes will be awarded to both the team that finds the appliance that uses the least amount of power as well as well as the team that finds the appliance that uses the most amount of power.

Note: at least one member of each team needs a camera (cell phone camera OK) in order to take a picture of the appliance reading.  Free food at the end of the night? We've got you covered.

Expected benefits: Increased energy intuition, familiarity with plug-load meters


\subsection{Recycled fashion design}

Description: As Heidi Klum reminds us, ``In fashion, one day you're in, and the next day you're out.'' Go fashion forward by attending the Recycled Fashion Design Workshop, hosted by Project Runway Season 8 Finalist Andy South.

Assisted by UH Manoa fashion design students, you'll form small groups and use recycled materials to create a new look, while Andy provides advice and encouragement. Then a model will walk the runway to show off your creation. No matter what colors you choose, your look will be green! If you just want to watch, that's fine too.

After party snacks included, so sign up soon!

Expected benefits: understanding sustainability benefits from reused clothing, awareness of Goodwill for purchasing used clothing


\subsection{Flashmob design}

Description: Do you ever experience an intense, uncontrollable urge to break into song and dance in large, public places? If you've got that fever, we've got the cure: a heaping helping of the Kukui Cup Flashmob.

At this workshop, you'll start designing a clandestine energy-related song and dance skit to be busted out near the end of the Kukui Cup while consuming free munchies. Who knows, it could be the YouTube hit of November, 2011.

Expected benefits: group work, promotion of the competition


\subsection{Kahuku Wind Farm}

Description: 0Want to see sky farmers harvesting the winds? Come with us to Kahuku to see firsthand how energy is plucked from the sky and generated for our use by First Wind's turbines.

The wind farm staff requests that everyone wear long pants and closed toe shoes.  We hope to stop in Kahuku for lunch, so bring some money.

You need to register for this free event to reserve your seat on the bus by clicking the \textbf{I want to sign up} button below.

Meet in the Hale Aloha courtyard, from there we'll get on the bus.  Don't be late!

Expected benefits: Better understanding of wind power


\subsection{Sustainable and Organic Farming}

Description: Your mother always told you to eat your vegetables, but did you ever consider where they came from while you forced down that last bite of rutabaga? The Sustainable Organic Farm Training (SOFT) club is a student-run organization devoted to getting at the ``roots'' of fresh produce, literally and figuratively. 

At this workshop you'll get a chance to help out at the farm, taste fresh produce, and discover out what it really means to eat natural, local, organic, sustainable produce! Yum!!

Expected benefits: Understanding of farming and its relationship to energy, introduction to the SOFT campus group


\subsection{Pedalpalooza}

Description: Is the Queen song ``I want to ride my bicycle'' whirring furiously through your brain while you stare down at your broken two wheeler? Have no fear, Freddy Mercury fans! Cycle Manoa is here to save your day with the Pedalpalooza workshop. If your wheels are broken, they'll teach you how to fix them for free. If your wheels are rocking, join them for a quick one hour ride around Manoa. And you can even cool down afterwards with a free bicycle-powered smoothie.

Meet at Hale Aloha Courtyard at 3pm with your wheels for a guided 1 hour ride, ending up at the Cycle Manoa HQ. If you don't have a bike but would like to attend, meet in the courtyard at 3:40 and we'll walk up to the Cycle Manoa together. At 4 PM you'll hear from the Cycle Manoa team about bicycle advocacy and bicycle repair.

Expected benefits: Understanding of benefits of bicycle transportation compared to fossil-fuel driven vehicles, introduction to Cycle Manoa group


\subsection{UH Manoa Food Day}

Description: Do you care about your food? Want to find out more about how to eat tasty, healthy food? Come to the UH Manoa Food Day event, which will include presentations on nutrition and food followed by Dr. Ted Radovitch a CTAHR specialist in Sustainable and Organic Farming Systems, and Dean Okimoto from Nalo Farms.

Following the presentations will be a food demonstration by Philip Shon, UH Sodexo Executive Chef who works in collaboration with Donna Ojiri, RD, General Manager of Sodexo. Taste local fresh produce, sample grass-fed Big Island beef, and local fruit beverages. By celebrating food day, helps emphasize the importance of making healthy food choices, and promote changes in food and farm policies that benefit health, the environment and well-being of us all in Hawaii.

Since this is an external event, you should sign up here but also RSVP on the UHM Food Day website (will open in a new window). When you get to the event, look for a Kukui Cup staff member (white t-shirt), and they will give you your attendance code that will get you the points for this event.

Expected benefits: Understanding of food's relationship to sustainability


\subsection{Round 1 Awards Party}

Description: If you have an indiscernible memory of attending an awesome awards party before, you may be experiencing some pre-deja-vu of what is soon to come: the Kukui Cup's first Round 1 Awards Party - a melange of interactive energy awareness games, an ultra-cool student DJ who goes by the oh-so-natural name of Pearl, the last of the custom limited edition Kukui Cup t-shirts, and the ever-popular smart strips.

As you are reading this, you may experience pre-withdrawals from the general awesomeness of this Party, oh, and did I mention that Awards will be being handed out as well? Yes, it may be implied in the name of this event, but along with the natural high you will inevitably feel from all the free goodies while simultaneously doing something good for the earth and jamming out to clam shell beats, you may just find yourself going home with a cool prize. 

Deja-vu or dream come true? You decide.

Expected benefits: Promotion of the competition, distribution of incentives


\subsection{Your Sustainable Future}

Description: A famous British poet once wrote: ``You say you want a revolution? Well, you know, we'd all love to see the plan.''

If you're interested in helping create an energy and sustainability revolution in your classes, university and community, come plan with representatives from Blue Planet Foundation, Sustainable UH, Surfrider Foundation, Kokua Hawaii Foundation, College of Engineering, School of Architecture, Shidler College of Business, Environmental Studies, and more.

Planning the overthrow of our oil-based economy will work up an appetite, so we'll also provide snacks.

Expected benefits: Introduction to sustainability organizations, awareness of classes on sustainability topics


\subsection{Energy Efficient Eating}

Description: Has cafeteria food got you down in the dumps? Are you no longer amused by mystery meat? Want to get new ideas for late night munchies?

Join experts from Kokua Market in a discussion of where our food comes from in Hawaii, and inexpensive, residence hall friendly groceries. You'll sample a variety of free gourmet popcorn toppings and learn how to make your own for just pennies a serving.

We'll even stuff your goodie bag with a custom recipe book to cure those Hale Aloha hunger pangs.

Expected benefits: Understanding of food's role in sustainability, awareness of where to purchase locally-produced foods, ways to prepare food with less energy


\subsection{Movie Night}

Description: Watch two of the artsiest and the most hilarious shorts from the Bike Shorts Film Festival Hawaii and continue the night with the journey of a revolutionary architect in a maze of obstacles towards sustainable communities of ``Earth Ships'', completely energy autonomous off-the-grid houses built with recycled materials.

An adventure full of beautiful images and extraordinary personages. Accompanied by free popcorn and free delicious lemonade!

Expected benefits: Awareness of options for sustainable living


\subsection{Off-The-Grid Living}

Description: The Reppun family has been living on their farm and growing taro, coffee, honey and other food in beautiful Waihole valley for over 20 years.  Though they have the Internet, they don't any power lines. See how they live off the grid in comfort and style through hydro-electric and solar power. You'll take a bus over to the Windward side, hike into the valley to their farm, and see an amazing blend of old school and next generation Hawaii. Make sure you eat breakfast beforehand; we won't be back until after lunch.

Make sure to wear clothes and shoes that you don't mind getting wet or muddy on the farm!

Reserve your seat on the bus by clicking the \textbf{I want to sign up} button below. 

Meet in the Hale Aloha courtyard, from there we'll get on the bus.  Don't be late!

Expected benefits: Understanding of real-world renewable energy options and farming as a part of sustainable living


\subsection{Kokua Market Excursion}

Description: Does purchasing groceries from corporate supermarkets leave a negative taste in your mouth? Satiate your craving to support local farmers and businesses by taking a tour of the Kokua Market, the only natural foods cooperative in Hawaii.

Sample foods whilst browsing a bountiful bulk selection, innovative deli items, and a plethora of produce, and learn why Coops are so crucial to the food chain of Hawaii.  At Kokua Market, the customer reigns supreme, not profit.

Meet at Hale Aloha Courtyard and we'll walk over -- it's just five minutes away!

Expected benefits: Awareness of where to purchase locally-produced foods


\subsection{Round 2 Awards Party}

Description: If you went to the Kickoff and Round 1 parties (which were, obviously, awesome), you might be thinking you can skip the Round 2 party. But that would be a huge FAIL because the Round 2 party is going to totally kick it.

There will be live music by Breath of Fire, speakers from local organizations like Blue Planet Foundation and Sustainable UH, awards to Round 2 winners, and some special surprises.  The Round 2 award party is being organized by the Aloha Movement Project so you know it's gonna rock.

The party starts at 5:30pm, awards are at 6:30pm, and Breath of Fire is playing two sets at 6 and 7pm.

Expected benefits: Promotion of the competition, distribution of incentives


\subsection{Manoa Sustainability Corps}

Description: Want to find out how UH's sustainability efforts tie together? Come to the monthly meeting of the UH Manoa Sustainability Corps. The Sustainability Corps is a forum for all of us to share information, ideas, data, and suggestions regarding sustainability on campus. It is also a forum to propose projects and programs that will make UH Manoa a green leader in Hawai'i and abroad.

This external event is being held in Krauss Hall, Room 012, known as the Yukiyoshi Room. When you get to the event, look for a Kukui Cup staff member (white t-shirt), and they will give you your attendance code that will get you the points for this event.

Expected benefits: Introduction to Manoa Sustainability Corps, awareness of opportunities to get involved in sustainability on campus


\subsection{High Energy Art and Music}

Description:

\begin{verse}
Electric blood flows through the veins of the city\\
drip, drip, drip, every drip a drop of oil\\
do I flip the switch or does the switch flip me?\\
\end{verse}

Oh. hey there. We just get carried away when we think of slam poetry. And saving electricity. If you feel the same way, join us with Kealoha, Hawaii's premier slam poet. Afterwards, you'll have a chance to lay down verse of your own with the open mic session and munch on free snacks.

Expected benefits: Understanding of how art can promote sustainability


\subsection{Energy Efficient Chillaxation}

Description: Stress is the cancer of emotions, and undue amounts can lead to the demise of your study habits!

At the Energy Efficient Chillaxation Workshop, discover green methods to reduce stress, such as pranayama (deep breathing techniques), massage, and yoga. We'll provide you with free herbal tea and snacks, teach you some techniques, and help you blow off steam.

But don’t stress out too much about being on time!

Expected benefits: Awareness of ways to relax that don't require electricity


\subsection{First Green Friday}

Description: First Green Friday is a showcase is to bring together faculty, students, and staff to showcase UHM's sustainability education, research, and demonstration projects. This new event is launching for the first time this Friday! Check out groups like the Environmental Center, the Ecology Club, Surfrider and the UHM Sustainability Corps. The Kukui Cup will have a table there as well!

This event is taking place in the Sustainability Courtyard, which is between Kuykendall Hall and the Hawaii Institute for Geophysics. Check out the booths at the event, and find the Kukui Cup table to get your attendance code.

Expected benefits: Awareness of opportunities to get involved in sustainability on campus


\subsection{North Shore Beach Cleanup}

Description: According to Wikipedia, ``utopia'' is an ideal community or society possessing a perfect socio-politico-legal system. Hawaii, perhaps the closest thing we have to environmental perfection on earth, is regularly polluted by all the garbage washing up on our precious beaches.

Do your part to bring Oahu one step closer to utopia by attending this beach cleanup sponsored by Surfrider Foundation. Yes, you'll have to wake up early, but it's totally worth it: a free ride to the North Shore, a couple of hours making Haleiwa Beach even more beautiful than it already is, then a free lunch and prizes provided by Spy Optics!

Make sure you bring a hat, sunscreen, swim suit, and water (in a reusable water bottle, of course!)

Reserve your spot on the bus by clicking the \textbf{I want to sign up} button below. Spaces are limited.

Meet in the Hale Aloha courtyard, from there we'll get on the bus. Don't be late, this one leaves early. Note that this excursion is worth the most points of all! Good attendance by your lounge could just put you over the top!

Expected benefits: Awareness of waste stream and how it impacts Hawaii's beaches
