\begin{abstract}

The world is in the grip of a crisis in the way energy is produced and consumed. Climate change represents a huge threat to the modern way of life, particularly for island communities like \Hawaii. Many changes to our energy system will be required to resolve the crisis, and one promising part of the solution is reducing energy usage through changes in behavior. Energy usage in similar homes can differ by a factor of two to four times, demonstrating the potential contribution of behavior change to the crisis.

This research project seeks to find ways to foster sustainable changes in behavior that lead to reduced energy usage. The research will be conducted in the context of a dorm energy competition on the UH \Manoa campus in October 2010. Power meters will be installed on each floor of two freshmen residence halls. Each floor will compete to use the least energy during the 4 week competition.

A competition website will be created, where participants can log in to view near-realtime data about their floor's power usage, and also select from a variety of tasks to perform. Each task is designed to increase the participant's \emph{energy literacy} (knowledge, positive attitudes, and behaviors related to energy), and a certain number of points are assigned for the completion of each task. The points provide a parallel competition to motivate participants to perform the tasks. Prizes will be awarded to floors using the least energy, and participants obtaining the most points.

Several research questions will be investigated using the data collected, including how energy usage changed after the competition is over, whether the website tasks affected energy literacy, and whether floors that had higher energy literacy had more sustained energy conservation after the competition was complete. The research questions will be investigated using energy data from the meters, log files from the website, and an energy literacy survey administered before and after the competition.

\end{abstract}