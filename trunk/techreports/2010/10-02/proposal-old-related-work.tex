%% Intended to be included into a larger document
\chapter{Related Work}

%% Add more to introduce topics in this chapter?


\section{Related Systems}
\label{sec:related-systems}

\subsection{Related Systems Summary}

The systems here represent a diverse set of methods for monitoring and encouraging the reduction of users' environmental footprint. \autoref{tab:related-work-synthesis} summarizes the systems reviewed. The columns of the table are:

\begin{itemize}
	\item Data input: the means by which data about users' activities are input into the system. The options here are: \emph{physical sensors} (such as ammeters), a list of \emph{green actions} (like turning off lights after leaving a room), \emph{manual} input (typing in the number of miles driven), or \emph{information sensors} (grabbing data already online, like electricity usage from a utility website).
	\item Social aspect: what social functionality the system provides, if any.  The options here are: a \emph{social media plugin} (for a site such as Facebook or MySpace), a \emph{user ranking} based on some green criterion, \emph{collaboration} facilities such as a discussion forum or suggesting green actions to others, or social interaction is \emph{integral} to the system (without it the system wouldn't work).
	\item Suggestions: does the system provide suggestions to the user on how to reduce their environmental footprint?
	\item Interface: how do users interact with the system, with the primary mode of interaction listed first. The options are: a \emph{website}, a \emph{mobile phone} application (or a website tailored for mobile browsers), or some sort of \emph{custom display} (such as an energy meter that displays a home's current electrical usage).
	\item Status: the status of the system. The options here are: the system is \emph{proposed} (it only exists on paper), the system is a \emph{prototype} (it works well enough to allow some sort of evaluation, but is not fit for wider use), the system is actually working but only in a \emph{closed beta} test, or the system is open to the \emph{public}.
	\item Leverage: what way, if any, could PET leverage this system to reduce the amount of implementation effort required? The options are: \emph{data input} (providing some way to directly input data in bulk), \emph{data output} (providing a way to export data that has been collected), \emph{API} (providing programmatic means to operate the system), and \emph{source code}. Note that some of these options are not supported by any of the systems reviewed.
\end{itemize}

\begin{table}[htbp]
	\centering
		\begin{minipage}{\textwidth}
		% need small font sizes to make table fit on page
		\scriptsize
		\begin{scriptsizetabular}{| l || >{\raggedright}p{1.8cm} | >{\raggedright}p{1.7cm} | l | >{\raggedright}p{2.25cm} | p{1.1cm} | p{1.1cm} |}
			\hline
			System & Data Input & Social Aspect & Suggestions & Interface & Status & Leverage \tabularnewline \hline \hline
			
			Sutaria \& Deshmukh & physical sensor & user ranking, collaboration & yes & custom displays \& website? & proposed & N/A \tabularnewline \hline
			
			StepGreen & green actions & social media plugin & yes & website & public & none \tabularnewline \hline
			
			Virtual Polar Bear & green actions & none & no & website + Flash & prototype & none \tabularnewline \hline
	
			iamgreen & green actions & integral & yes & website & public & none \tabularnewline \hline

			Personal Kyoto & info sensor & none? & no & website & public & none? \tabularnewline \hline
%\footnote{Unable to verify without ConEdison account}
			EcoIsland & manual \& physical sensor & integral & yes & custom display, mobile phone & prototype & none \tabularnewline \hline
			
		\end{scriptsizetabular}
		\end{minipage}
	\caption{Systems related to PET}
	\label{tab:related-work-synthesis}
\end{table}
