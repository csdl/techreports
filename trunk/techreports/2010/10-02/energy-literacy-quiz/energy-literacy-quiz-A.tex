%%%%%%%%%%%%%%%%%%%%%%%%%%%%%% -*- Mode: Latex -*- %%%%%%%%%%%%%%%%%%%%%%%%%%%%
%% uhtest-appendix.tex -- 
%% Author          : Robert Brewer
%% Created On      : Fri Oct  2 16:31:12 1998
%% Last Modified By: Robert Brewer
%% Last Modified On: Mon Oct  5 14:41:05 1998
%% RCS: $Id: uhtest-appendix.tex,v 1.1 1998/10/06 02:07:03 rbrewer Exp $
%%%%%%%%%%%%%%%%%%%%%%%%%%%%%%%%%%%%%%%%%%%%%%%%%%%%%%%%%%%%%%%%%%%%%%%%%%%%%%%
%%   Copyright (C) 1998 Robert Brewer
%%%%%%%%%%%%%%%%%%%%%%%%%%%%%%%%%%%%%%%%%%%%%%%%%%%%%%%%%%%%%%%%%%%%%%%%%%%%%%%
%% 

\documentclass[11pt]{article}
%%% Load some useful packages:

\usepackage{times}

% make margins smaller
\usepackage[left=2.5cm,top=2cm,right=2.5cm,bottom=2cm,nohead,nofoot]{geometry}

%% Provides customization of lists
\usepackage{enumitem}

%% Now define question list type
\newlist{question}{enumerate}{1}
\setlist[question]{resume, label=\textbf{\arabic*.}}

%% Now define multiple choice answer list type
\newlist{answer}{enumerate}{1}
\setlist[answer]{label=\alph*)}

%%% End of preamble
\begin{document}

\title{Energy Literacy Quiz A, Version 1.0.0}
\author{Robert S. Brewer}
%\date{}

\maketitle

%%% Body text goes here

\section{Quiz}

\begin{question}
	\item The watt is a unit of:
\end{question}

\begin{answer}
	\item energy
	\item power
	\item distance
	\item force
\end{answer}

\begin{question}
	\item The watt is abbreviated as:
\end{question}

\begin{answer}
	\item wt
	\item Wh
	\item W
	\item tt
\end{answer}

\begin{question}
	\item Electrical energy is commonly measured in units of:
\end{question}

\begin{answer}
	\item BTU
	\item erg
	\item watt-hour
	\item watt
\end{answer}

\begin{question}
	\item The abbreviation "Wh" refers to what unit:
\end{question}

\begin{answer}
	\item watt
	\item wind-hour
	\item watt-hour
	\item power
\end{answer}

\pagebreak

\begin{question}
	\item A compact fluorescent lightbulb uses 13 W. If it is run for 2 hours, how much energy does it use?
\end{question}

\begin{answer}
	\item 7.5 Wh
	\item 13 Wh
	\item 26 Wh
	\item 52 Wh
\end{answer}

\begin{question}
	\item While reading your electric bill you notice that you used 72 kWh more than the previous month. You search your apartment for anything out of the ordinary and find you left a fan running in a closet all month long! Approximately how much power does the fan use?
\end{question}

\begin{answer}
	\item 100 W
	\item 10 W
	\item 300 kWh
	\item 1 kWh
\end{answer}

\begin{question}
	\item Roughly how much power does an electric oven use when turned to its highest setting?
\end{question}

\begin{answer}
	\item 100 W
	\item 500 W
	\item 1 kW
	\item 2.5 kW
\end{answer}

\begin{question}
	\item On average, how much electrical energy does a home in Hawaii use per day?
\end{question}

\begin{answer}
	\item 13 kWh
	\item 4 kWh
	\item 57 kWh
	\item 328 kWh
\end{answer}

\begin{question}
	\item What is the approximate maximum power generated from a single standard rooftop solar panel?
\end{question}

\begin{answer}
	\item 25 W
	\item 50 W
	\item 200 W
	\item 800 W
\end{answer}

\pagebreak

\begin{question}
	\item What is the source of approximately 80\% of Hawaii's electricity?
\end{question}

\begin{answer}
	\item coal
	\item wind
	\item solar
	\item oil
\end{answer}

\begin{question}
	\item What is the approximate maximum electrical power demand for the entire island of Oahu?
\end{question}

\begin{answer}
	\item 560 kW
	\item 3800 kW
	\item 11.8 GW
	\item 1.2 GW
\end{answer}

\begin{question}
	\item What is the electrical grid demand curve?
\end{question}

\begin{answer}
	\item A graph of the amount of power used on the grid over time
	\item The number of efficient appliances demanded by consumers
	\item A graph of the amount of energy used on the grid over time
	\item The amount overhead power lines can bend before breaking
\end{answer}

\begin{question}
	\item What is the goal of the Hawaii Clean Energy Initiative?
\end{question}

\begin{answer}
	\item Maintain Hawaii's energy use at current levels forever
	\item Decrease Hawaii's oil use by 20\% by 2020
	\item Get 70\% of Hawaii's energy from clean sources by 2030
	\item Get 50\% of Hawaii's energy from wind by 2050
\end{answer}

\begin{question}
	\item What are the effects of climate change?
\end{question}

\begin{answer}
	\item Global temperatures increasing by a few degrees on average
	\item Changes in seasonal rainfall patterns (droughts, floods)
	\item A significant rise in the sea level
	\item All of the above
\end{answer}

\begin{question}
	\item Approximately how much carbon dioxide is in the atmosphere now, and what level is considered safe/acceptable?
\end{question}

\begin{answer}
	\item 450 ppm, 500 ppm
	\item 387 ppm, 350 ppm
	\item 331 ppm, 350 ppm
	\item 600 ppm, 450 ppm
\end{answer}

\end{document}
