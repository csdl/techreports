\chapter{System Design}
\label{cha:system-description}

The system to be evaluated is a combination of an energy competition between residents of two freshman residence halls, and an associated competition website to be used by the residents participating in the event. The system has three goals:

\begin{itemize}
	\item Enable research into fostering sustainable environmental behavior change
	\item Improve the energy literacy of the participants
	\item Reduce the energy consumption of the residence halls
\end{itemize}

The participants compete both to reduce energy consumption in the participating residence halls, and to accumulate points by performing tasks related to energy literacy and conservation through the competition website.

This chapter describes the components of the system, and ends with a discussion of the factors that pose a risk to the successful implementation and evaluation of the system.

\section{Responsive UI Design}

We will examine the behavior of freshmen residents in student housing at the University of \Hawaii at \Manoa in the context of a energy competition. An student housing energy competition typically involves residence halls attempting to reduce their energy consumption during the competition period by the greatest amount. The competition planned here is more complicated than standard competitions so that we can obtain data on a wider variety of behavior. The working name for the competition is the Kukui Cup. The kukui nut was burned by Native Hawaiians to provide light, making it an early form of energy in \Hawaii.

\section{Cloud-based Deployment}

The core infrastructure required to enable an energy competition is electricity metering. In \Hawaii, the vast majority of energy used in buildings is electricity, so measuring direct energy use reduces to measuring electricity use. While building-level metering is common for energy competitions, for this competition we plan to have floor-level metering of electricity. Metering at the floor level has several advantages:

\begin{itemize}
	\item Finer-grained data about electricity usage
	\item Individual behavior changes more likely to be visible in data
	\item Makes the residents of a floor a natural `unit' of competition
\end{itemize}

We also require that the meters provide sub-minute sampling times, preferably 10 to 15 seconds. This is an unusual requirement for meters used outside the home. We term this requirement \emph{near-realtime} monitoring. As discussed in \autoref{sec:energy-feedback}, providing near-realtime feedback on energy use is associated with greater reductions in energy consumption. Near-realtime feedback also enables participants to empirically determine how much electricity different devices consume, and become more aware of their energy use.

The other meter requirements are provision of an open API to allow retrieval of the data, and affordable pricing.

We have evaluated several building energy meters based on these criteria, and found 4 meters that meet all the criteria. All 4 meters support the Modbus/TCP protocol \cite{modbus-website}, which allows the meters to be queried over the Internet using a standardized protocol. Final selection of the meter will be done based on feedback from UHM facilities and the results of development of software to read data from the meters.

Installation of the meters involves placing current transformers over the incoming power lines in the electrical room on each floor. The current transformers convert current flowing over lines providing each phase into a small voltage which is then measured by the electrical meter. The electrical meters under consideration all have Ethernet ports, allowing them to be connected directly into the residence hall LAN. Once connected to the UHM network, they can be queried from any location.

The other infrastructure required is to place large TV display connected to an Internet-connected computer in the lobby of each building. This display will be used as a `billboard' that cycles through information about the competition, such as the floor standings and upcoming events (see \autoref{sec:billboard-design}). The Hale Aloha towers already have flat panel displays present in the lobby, which might be suitable for billboard display. Otherwise, two displays will need to be purchased. Some manufacturers make large format displays that embed a Windows PC in the display and are designed for this type of usage (such as the Samsung 460UXN-2, which costs approximately \$2,000).

\section{Pluggable Game Components}

Using the competition metrics, we can define various awards that can be won in the competition. In the event of ties, the winner will be resolved by random selection. Since the competition consists of 4 rounds, a common pattern is to have an award for each of rounds 1 through 3, and then an award for the entire competition (all 4 rounds). To incentivize participation, each award has an associated prize. We define the following awards for individuals. Note that all individual awards relate to KN since energy data only goes down to the floor level, not individual:

\section{Configurable Game Instance}


While the impetus for the website is to support the competition, it is also intended to be provide information about the competition and residence hall energy consumption to the public (non-participants). Therefore, the website is conceived as a general portal into residence hall energy usage that will be available before and after the competition. During the competition period, the competition-specific portions of the website will be made available to participants. \autoref{fig:website-home} shows a mockup of the front page of the website, where we can see overall residence hall energy consumption.

\section{Energy Literacy Contents}

Competition participants will be able to log into the website using their UH username and password, which will lead them to a personalized home page. The website will provide the following information to participants:


\autoref{fig:website-maile} shows a mockup of the personalized home page for a participant named Maile. On the left hand side we see Maile's profile, showing her name, room number, and how many Kukui Nut points she has accumulated during the competition. The center column of the page relates to the tasks that Maile can perform to gain Kukui Nut points (\autoref{sec:competition-tasks} describes the task system in detail). The right hand side displays both power data and competition standings. The upper number is the near-realtime power usage for Maile's floor, which is colored in red as an indicator that this value is above the pre-competition baseline. The lower number is the total electrical consumption for Maile's floor in this round, which is colored in green since it is below the baseline and on target to meet the floor's goal of a 10\% reduction in energy usage for this round. The box in the lower right hand corner displays the competition standings that are most relevant to Maile.


