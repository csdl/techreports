\section{Lower Bounding}
The time-series similarity matching problem was first examined by Agrawal et al \cite{citeulike:3973409} and while employing DFT and proposing an F-index, they were relying on the use of Parseval's theorem which guaranteed that the distance between two time-series in the time domain is the same as in the frequency domain \ref{eq:dft_similarity}. By using only first $k$ coefficients of DFT and discarding the rest they were introducing false hits into F-index but ensured that there are no false-dismissals. Subsequent work by Faloutsos et al \cite{citeulike:825581} introduced Minimum Bounding Rectangles (MBRs) and generalized the F-index use for the subsequence matching. Both approaches, in order to guarantee that there is no false-dismissals for a transform $T$ and the distance measure $D_{feature}$ in the feature space, must satisfy a ``contractive property'' or lower-bounding condition:
\begin{equation}
D_{feature}(T(x),T(y)) \; \leq \; D(x,y) 
\label{eq:bounding}
\end{equation}
where $D$ is the distance and $x$ and $y$ are the time-series.