\section{Lower Bounding}
The first approach to time-series indexing by Agrawal et al \cite{citeulike:3973409} employed DFT for dimensionality reduction and F-index for similarity search. The proposed time-series approximation with only first $c$ ``significant'' DFT coefficients resulted in false hits when searching F-index, but it was shown that all false hits are false-positive and there are no false-dismissals. The author's proof of algorithm correctness based on the ``contractive property'' (lower-bounding condition) of their transform and feature-space distance choice which essentially ``underestimates'' the real distance between time-series.

The lower bounding condition is formulated as:
\begin{equation}
D_{feature}(T(X),T(Y)) \; \leq \; D_{object}(X,Y) 
\label{eq:bounding}
\end{equation}
where $D_{feature}$ is the distance between objects in feature space (Agrawal et al used the Euclidean distance), $T$ is the transform function from object space to feature space (DFT) and $D_{object}$ is the real distance between objects $X$ and $Y$ (the Euclidean distance again). 

Agrawal et al has shown that lower-bounding condition holds for DFT transform by relying on Parseval's theorem which guaranteed that the distance between two time-series in the time domain is the same as in the frequency domain (\ref{eq:dft_similarity}) and the dismissal of some DFT coefficients only decreases the distance value in the feature space. Subsequent work by Faloutsos et al \cite{citeulike:825581} formalized the lower-bounding condition requirement for the range queries processing without false-dismissals.