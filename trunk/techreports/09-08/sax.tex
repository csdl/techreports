\section{Symbolic Aggregate approXimation (SAX)}
The last method we are going to review in this writing is the current state of the art time-series representation and dimentionality reduction method called Symbolic Aggregate approXimation (SAX) which transforms original time-series data into the symbolic strings. This method, proposed by Lin et al \cite{citeulike:2821475}, turns out to be not just simple and computationally cheap in the index construction, but also extremely fast and precise in the range-query processing, moreover the use of the symbolic representation opens door to the existing CS wealth of data-structures and string-manipulation algorithms such as hashing, suffix trees, regular expression pattern matching, etc.

SAX transforms a time-series $X$ of length $n$ into the string of arbitrary length $w$ where $w << n$ typically using an alphabet $A$ of size $a >2$. The SAX technique consis of two steps, first step transforms the original time-series into PAA representation and this intermediate representation converted into the string on the second step. By relying on PAA authors managed to prove two essential for the dimensionality reduction properties: simple and efficient reduction process and lower bounding. While first was proven long before emerging of SAX, second property, the lower bounding of symbolic distance was proven by the finding that symbolic distance lower bounds the PAA distance.

Discretization of the PAA representation of the time-series in SAX implemented in a way which produces symbols corresponding to the time-series features with equal probability. The rigorous analysis of the time-series datasets used in the SAX-related work shows that normalized by the zero mean and unit of energy time-series have a Gaussian distribution. By using the Gaussian distribution properties \cite{citeulike:167581} it's easy to pick $a$ equal-sized areas under the Normal curve. The points of the cut lines slicing the the under-the-Gaussian-curve area called ``breakpoints''.
\begin{figure}[tbp]
   \centering
   \includegraphics[height=55mm]{sax_intro.eps}
   \caption{The illustration of the SAX approach taken from \cite{citeulike:2821475} depicts two pre-determined breakpoints for the three-symbols alphabet and the conversion of the time-series of length $n=128$ into PAA representation first and following mapping of the PAA coefficients into SAX symbols with $w=8$ and $a=3$ resulting in the string \textbf{baabccbc}.}
   \label{fig:sax_intro}
\end{figure}

Extending Euclidean \ref{eq:euclidean_distance} and PAA \ref{eq:paa_distnace} distances, the function returning the minimal distance between two string representations of original time series $\hat{Q}$ and $\hat{C}$ defined as
\begin{equation}
MINDIST(\hat{Q},\hat{C}) \equiv \sqrt{ \frac{n}{w} } \sqrt{ \sum_{i=1}^{w} ( dist( \hat{q}_{i}, \hat{c}_{i} ) )^{2}}
\label{eq:sax_mindist}
\end{equation} 
where the $dist$ function is implemented by using the lookup table for the particular set of the breakpoints as shown in table \ref{tbl:sax_lookup}.

\begin{table}
\begin{tabularx}{400pt}{X X X X X}
\hline
   & a   & b    & c    & d    \\
\hline
a & 0    & 0    & 0.67 & 1.34 \\
b & 0    & 0    & 0    & 0.67 \\
c & 0.67 & 0    & 0    & 0    \\
d & 1.34 & 0.67 & 0    & 0    \\
\hline
\end{tabularx}
\caption{A lookup table used by the MINDIST function for the $a=4$}
\label{tbl:sax_lookup}
\end{table}
where the singular value for each cell $(r,c)$ is computed as 
\begin{equation}
cell_{(r,c)} = 
\begin{cases} 
0, & \text{ if }\left| r-c \right| \leq 1 \\
\beta_{\max(r,c) - 1} - \beta_{\min(r,c) - 1}, & \text{ otherwise}
\end{cases}
\label{eq:cell}
\end{equation}

As shown by Li et al, the introduced SAX distance measure lower-bounds the PAA distance, i.e.
\begin{equation}
n(\bar{Q} - \bar{C})^{2} \geq n(dist(\hat{Q},\hat{C}))^2
\label{eq:}
\end{equation}

\begin{figure}[tbp]
   \centering
   \includegraphics[height=47mm]{sax_distance.eps}
   \caption{The visual representation of the two time-series $Q$ and $C$ and three distances between their representation: Euclidean distance between raw time-series (A), the distance defined for PAA coefficients (B) and the distance between two SAX representations (C).}
   \label{fig:sax_distance}
\end{figure}
