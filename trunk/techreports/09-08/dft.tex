\section{Discrete Fourier Transform (DFT)}
Agrawal et al. \cite{citeulike:3973409} proposed Direct Fourier Transform based method for the time-series indexing and similarity queries processing. Their idea essentially based on the observation of the fairly good approximation of the time-series by only few ``strong'' frequencies and on the Parseval's Theorem (aka Rayleigh's energy theorem). 

For each of the time-series $X$ of length $N$ authors proposing to extract only $k$ features, where $k<<N$ by using DFT. Thus each of the time-series is mapped in the low $k$ dimensional space and is stored in the multidimensional index using the $R^{*}$ tree approach \cite{citeulike:343069}. Authors arguing that the approach taken is characterized by the ``Completeness of the feature-extraction'' and it is ``efficiently dealing with the dimensionality curse''. 

The $n$-point DFT transform of the timeseries $X=(x_{0}, x_{1}, ... , x_{n-1})$ is defined as:
\begin{align}
& X_{f} = 1/\sqrt{(n)}\sum_{t=0}^{n-1} x_{t} \exp(-j2 \pi f t/n),\; t=0,1,...,n-1, \; j=\sqrt{(-1)} \\
& \text{and inverse transform is } \\
& x_{t} = 1/\sqrt{(n)}\sum_{f=0}^{n-1} X_{f} \exp(j2 \pi f t/n),\; t=0,1,...,n-1 
\end{align}
