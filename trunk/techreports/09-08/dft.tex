\section{Discrete Fourier Transform (DFT) based decomposition}
Agrawal et al. in \cite{citeulike:3973409} proposed a DFT-based method for the time-series indexing and similarity queries processing. Their idea is based on the observation of a fairly good approximation of the time-series by only few ``strong'' frequencies and on Parseval's Theorem (aka Rayleigh's energy theorem). 

For each of the time-series $x$ of length $N$, the authors propose to extract only the first $c$ DFT coefficients (frequencies), where $c<<N$ and $f_{c}$ is the ``cutoff frequency''. Thus each of the time-series is mapped into the low $c$ dimensional space and stored in the $F$-index, where the search over the index is implemented by using an $R^{*}$ tree \cite{citeulike:343069}. The authors argue that the approach taken is characterized by the ``completeness of the feature-extraction'' and it is ``efficiently dealing with the dimensionality curse''. 

The $n$-point DFT transform of the time-series $X=(x_{0}, x_{1}, ... , x_{n-1})$ is defined as:
\begin{align}
& X_{f} = 1/\sqrt{(n)}\sum_{t=0}^{n-1} x_{t} \exp(-j2 \pi f t/n),\; t=0,1,...,n-1, \; j=\sqrt{(-1)} \\
& \text{also the inverse transform:} \nonumber \\
& x_{t} = 1/\sqrt{(n)}\sum_{f=0}^{n-1} X_{f} \exp(j2 \pi f t/n),\; t=0,1,...,n-1 
\end{align}
and the fundamental observation of the Parseval's theorem is
\begin{equation}
\sum_{t=0}^{n-1} \left| x_{t} \right| ^{2} = \sum_{f=0}^{n-1} \left| X_{f} \right| ^{2}
\label{eq:parseval}
\end{equation}
that the energy of the sequence in the time domain equals the energy in the frequency domain.

Furthermore, the authors show that Discrete Fourier Transform inherits linearity and preservance of amplitude coefficients during the time-shifts from the Continuous Fourier Transform. Taking these properties and Parseval's theorem in account, the authors show that
\begin{equation}
\left\| x - y \right\| ^{2} \equiv \left\| X - Y \right\| ^{2}
\label{eq:dft_similarity}
\end{equation}
which implies the equivalence of the Euclidean distances between two sequences $x$ and $y$ in the time and frequency domains. 

Agrawal et al. introduces the distance function $D$ where the distance between two sequences $x$ and $y$ is the square root of the energy of the difference:
\begin{align}
D(x,y) \equiv \left( \sum_{t=0}^{n-1} \left| x_{t} - y_{t} \right| ^{2} \right) ^{1/2}
       \equiv \left( E(x-y) \right) ^{1/2} \\
\text{where } E(x) \; \text{is the energy of the sequence:} \nonumber \\
E(x) \equiv \left\| x \right\| ^{2} \equiv \sum_{t=0}^{n-1} \left| x_{t} ^{2} \right|
\label{eq:dft_distance}
\end{align}
and implements a similarity relation between two sequences by using a user defined threshold $\epsilon$ - i.e. if the distance between two sequences falls below threshold they considered to be similar:
\begin{equation}
D(x,y) \leq \epsilon \; \Rightarrow \text{$x$ similar to $y$}
\label{eq:dft_similarity_dft}
\end{equation}

Continuing the discussion, the authors compare their DFT-based method to the ``sequential scanning'' method (the only alternative available by that time). They use the same $R^{*}$ tree for the sequential indexing and implement an ``early abandoning'': as soon as the calculated Euclidean distance exceeds $\epsilon^{2}$, two sequence declared dissimilar. It was shown by comparing two approaches, that with the growth of the dataset volume and the length of sequences, the DFT-based approach outperforms the sequential search. Another interesting point shown is that, in general, $f_{c} \approx 3$ is enough to capture time-series features and build an index which provides satisfactory performance and has a contractive property, i.e. ensures no false-dismissal.
