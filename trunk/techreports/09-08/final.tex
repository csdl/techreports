\chapter{Conclusion}
There has been an increasing amount of research over the last two decades into time-series data mining, knowledge discovery and applications. It is very interesting to see how the focus of this research shifted from computationally intensive and mathematically elegant methods of spectral decomposition of time-series to extremely simple approximation methods such as PAA and SAX. In my opinion, this is due to the branching of the time-series data-mining field away from the classical time-series analysis school. While researchers started with adoption of the classical spectral analysis approach, which is almost perfect for resolving the ``secret'' of a time-series generative process, the time-series data-mining community, interested in the efficient navigation through the thousands of signals and petabytes of data, found its own extremely efficient ways to perform on-the-fly extraction and analysis of patterns. The latest application of SAX approximation to the image recognition \cite{citeulike:3175770} and monitoring insects real-time \cite{citeulike:4446167} by Keogh et al demonstrate how fast, accurate, and resource-efficient these techniques are. 