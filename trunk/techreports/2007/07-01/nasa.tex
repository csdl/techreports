%%%%%%%%%%%%%%%%%%%%%%%%%%%%%% -*- Mode: Latex -*- %%%%%%%%%%%%%%%%%%%%%%%%%%%%
%% nasa.tex -- 
%% Author          : Philip Johnson
%% Created On      : Wed Jul 27 13:31:23 1994
%% Last Modified By: 
%% Last Modified On: Mon Feb 05 10:11:03 2007
%% Status          : Unknown
%% RCS: $Id$
%%%%%%%%%%%%%%%%%%%%%%%%%%%%%%%%%%%%%%%%%%%%%%%%%%%%%%%%%%%%%%%%%%%%%%%%%%%%%%%
%%   Copyright (C) 1994 University of Hawaii
%%%%%%%%%%%%%%%%%%%%%%%%%%%%%%%%%%%%%%%%%%%%%%%%%%%%%%%%%%%%%%%%%%%%%%%%%%%%%%%
%% 

\documentclass[11pt]{article} 
\usepackage{/export/home/csdl/tex/icse2003/latex8}
\usepackage{times}
\usepackage[final]{graphicx}
% uncomment the % away on next line to produce the final camera-ready version
% and uncomment the \thispagestyle{empty} following \maketitle
%\pagestyle{empty}


\begin{document}
\title{
FINAL REPORT: \\ \protect \medskip
Supporting development of highly dependable software through
continuous, automated, in-process, and individualized software measurement validation \\ \protect \medskip
NASA-Ames Award No. NNA04CC78A
}

\medskip

\author{
Philip M. Johnson (PI)\\ 
Department of Information \& Computer Sciences\\
University of Hawaii\\
1680 East-West Rd, No. 307\\
Honolulu, HI 96822
\protect \medskip
}

\maketitle

\pagestyle{plain}

\section{Summary of Research}

The general objective of this  research project was to design,
implement, and validate software measures within a development
infrastructure that supports the development of highly dependable software
systems.  Contributions of this research project included: (a) development
of a specialized configuration of Hackystat to automatically acquire build
and workflow data from the configuration management system for the Mission
Data System (MDS) project at Jet Propulsion Laboratory; (b) development of
analyses over MDS build and workflow data to support identification of
potential bottlenecks and process validation; (c) identification of
previous unknown variation within the MDS development process; (d)
development of a generalized approach to in-process, continuous measurement
validation called Software Project Telemetry, (e) substantial enhancements
to the open source Hackystat framework, improving its generality and
usability; (f) development of undergraduate and graduate software
engineering curriculum involving the use of Hackystat for automated
software engineering metrics collection and analysis; (g) support for 3
Ph.D., 6 M.S., and 3 B.S. degree students.

The key publications resulting from this grant are listed in the ``References'' section below.

\nocite{*}

\section{Subject Inventions}

None.

\bibliography{/export/home/csdl/bib/nasa-final}
\bibliographystyle{plain}

\end{document}


