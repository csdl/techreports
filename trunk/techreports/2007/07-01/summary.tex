%%%%%%%%%%%%%%%%%%%%%%%%%%%%%% -*- Mode: Latex -*- %%%%%%%%%%%%%%%%%%%%%%%%%%%%
%% summary.tex -- 
%% RCS:            : $Id: nsf93-summary.tex,v 1.11 93/10/06 16:52:35 johnson Exp $
%% Author          : Philip Johnson
%% Created On      : Wed Aug 11 12:55:46 1993
%% Last Modified By: 
%% Last Modified On: Wed Jan 31 11:36:09 2007
%% Status          : Unknown
%%%%%%%%%%%%%%%%%%%%%%%%%%%%%%%%%%%%%%%%%%%%%%%%%%%%%%%%%%%%%%%%%%%%%%%%%%%%%%%
%%   Copyright (C) 1993 University of Hawaii
%%%%%%%%%%%%%%%%%%%%%%%%%%%%%%%%%%%%%%%%%%%%%%%%%%%%%%%%%%%%%%%%%%%%%%%%%%%%%%%
%% 
%% History
%% 11-Aug-1993          Philip Johnson  
%%    

\documentclass[11pt]{article} 
\usepackage{/export/home/csdl/tex/icse2003/latex8}
\usepackage{times}

\pagestyle{empty}

\begin{document}
\section*{Project Summary}
In this Science of Design (SoD) project, we propose to design and implement
a new testbed called ``SoDeT'' (Science of Design empirical Testbed), which
will be based upon the Hackystat Framework.  SoDeT will enable researchers
to attach software ``sensors''  which capture process and/or product data
to the tools used by participants evaluating the SoD design innovations.  The
testbed will be designed to facilitate empirical research by individual
Science of Design projects, as well as facilitate new research
opportunities in which multiple tools or techniques are combined together
and measured to see if positive or negative interactions occur.

The SoDeT project is organized into six phases: (1) Detailed
feasibility analysis, (2) Testbed kickoff, (3) Testbed implementation and
enhancement, (4) Trial adoption, (5) Testbed deployment, and (6) Testbed
findings.  The goal of this project structure is to maximize the potential
benefits and minimize the testbed adoption overhead for both individual SoD
projects and the program as a whole.

The intellectual merit of this research includes the application of novel
data gathering and analysis techniques for development of standardized,
comparable ways to operationalize the empirical measurement of design
characteristics produced by individual SoD projects. A preliminary
feasibility analysis indicates that the SoDeT approach may be appropriate
for a significant number of existing SoD projects.  SoDeT is intended to
reduce the cost to individual researchers of data collection and
analysis. Finally, the testbed will enable practical measurement in
industrial contexts as the design innovations produced by this program move
into real-world settings.

The broader impact of this research includes the development of a
sophisticated, freely available, open source software system for use by
researchers and practitioners to measure design characteristics, 
and its availability for use in educational settings.  As the University
of Hawaii is a university with 75\% minority students in an EPSCOR state,
this project will provide novel research opportunities to underrepresented
groups.

\end{document}

 







