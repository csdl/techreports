%%%%%%%%%%%%%%%%%%%%%%%%%%%%%% -*- Mode: Latex -*- %%%%%%%%%%%%%%%%%%%%%%%%%%%%
%% project.control.tex -- 
%% Author          : Philip Johnson
%% Created On      : Fri Jan 13 07:58:21 2012
%% Last Modified By: Philip Johnson
%% Last Modified On: Tue Jan 17 16:38:35 2012
%%%%%%%%%%%%%%%%%%%%%%%%%%%%%%%%%%%%%%%%%%%%%%%%%%%%%%%%%%%%%%%%%%%%%%%%%%%%%%%

\subsubsection{Control and optimization}

The primary goals of a smart, sustainable microgrid are to use electrical
energy as efficiently as possible, maximize the amount of energy coming
from renewable resources, and minimize the overall cost of energy while
retaining acceptable levels of reliability and quality. The optimization
and control module of this project will test (explore?, show?) how well
these goals can be met by scheduling electricity production and consumption
using data and forecasts from the modeling and analysis module.

Some of the important capabilities include: voltage and frequency
regulation, peak shaving and peak shifting (in order to obtain reduced
rates from the utility), and lowered overall consumption. Because cooling
constitutes most (?) of UHM's electricity load, the initial approach will
focus on cooling loads. (This also takes advantage of the fact that heat or
cold are much cheaper to store than electricity.) However, we will also
investigate the potential for optimizing other time-shiftable loads, such
as water heating, energy-intensive scientific research, and eventually
electric vehicles. (Something here about managing the backup-generator
blocks for peak shaving and ramp control?) In each case, automated
mechanisms will be supplemented by community awareness dissemination
techniques that can enable interested members of the microgrid to
participate in achieving cost savings and more efficient usage of renewable
resources. That is, we will study power control techniques at all levels of
user interaction, from total automated through automation with
user-override, to purely user-driven based on information supplied from the
optimization and control system.

The contributions of this part of the research will include development of
the software and hardware systems required to support various control and
optimization capabilities, and empirical studies that demonstrate the
extent to which those capabilities can be achieved in a real world
setting. For example, loads can be shifted by several minutes to hours with
no impact on users by adjusting the temperature band used in chillers that
provide cold water for the campus's cooling needs. However, even longer
shifts can be achieved by automatically adjusting the thermostats in
individual buildings (a process which has passive user involvement, since
users must be able to override these settings) or by actively involving
users through a ``save energy now'' message from the WattDepot. (Say more
about how this is a cutting-edge area of research, cite recent utility
critical peak pricing studies, and say that the effectiveness of these
automatic/hybrid/user-driven strategies is still not well understood, nor
have these strategies themselves been optimized, e.g., do good thermostats
exist for this purpose?)

The optimization model will seek to maximize the net benefit of load
control using a stochastic linear optimization model. This will use
estimates of the probability of setting the peak load or peak ramp in any
hour, the cost of doing so, the cost of power, and a ‚''supply curve‚'' for
load control that reflects the cost of discomfort or inconvenience for
users, exhausting their willingness to engage in load control, the cost of
operating backup generators as peak- or ramp-shaving units, or the risk of
failing to charge PHEVs and leaving them to purchase gasoline instead.

This module will also include an estimate of the amount of peak shaving or
ramp control that can be induced by sending a signal of a given strength
(e.g., shifting the temperature setpoint for chillers and/or buildings by
1°C or sending a stronger signal to users in a game-based
energy-management system). These coefficients will be dynamically updated
by occasionally comparing the behavior of the system with and without a
control signal under similar forecasted conditions.

We will also explore tradeoffs between efficiency and peak-management when
many pieces of equipment are operated at part load. For example, UHM's
smart grid and smart-chiller-distribution system will allow the campus to
run many generators or chillers at once, giving a strong peak-shaving or
ramp-shaving capability. However, this could increase power consumption,
since chillers and generators are less efficient when they are run at part
load.


