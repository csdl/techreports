%%%%%%%%%%%%%%%%%%%%%%%%%%%%%% -*- Mode: Latex -*- %%%%%%%%%%%%%%%%%%%%%%%%%%%%
%% project.control.tex -- 
%% Author          : Philip Johnson
%% Created On      : Fri Jan 13 07:58:21 2012
%% Last Modified By: Philip Johnson
%% Last Modified On: Fri Jan 13 08:00:34 2012
%%%%%%%%%%%%%%%%%%%%%%%%%%%%%%%%%%%%%%%%%%%%%%%%%%%%%%%%%%%%%%%%%%%%%%%%%%%%%%%

\subsubsection{Control and optimization}

{\em The primary goals of a smart, sustainable microgrid is to use
  electrical energy as efficiently as possible, maximize the amount of
  energy coming from renewable resources, and minimize the overall cost of
  energy while retaining acceptable levels of reliability and quality.
  This section discusses how the data provided by analytical models will be
  used to achieve those goals.  Some of the important capabilities include:
  voltage and frequency regulation, peak shaving and peak shifting (in
  order to obtain reduced rates from the utility), and lowered overall
  consumption. The initial approach will be to implement demand-response
  for the large-scale chillers in the micro-grid.  This automated mechanism
  will be supplemented by community awareness dissemination techniques that
  can enable interested members of the microgrid to participate in
  achieving cost savings and more efficient usage of renewable
  resources. 

  The contributions of this part of the research will include development of
  the software and hardware systems required to support various control and
  optimization capabilities, and empirical studies that demonstrate the
  extent to which those capabilities can be achieved in a real world
  setting. }

