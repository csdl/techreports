%%%%%%%%%%%%%%%%%%%%%%%%%%%%%% -*- Mode: Latex -*- %%%%%%%%%%%%%%%%%%%%%%%%%%%%
%% project.control.tex -- 
%% Author          : Philip Johnson
%% Created On      : Fri Jan 13 07:58:21 2012
%% Last Modified By: Philip Johnson
%% Last Modified On: Thu Jan 26 13:31:06 2012
%%%%%%%%%%%%%%%%%%%%%%%%%%%%%%%%%%%%%%%%%%%%%%%%%%%%%%%%%%%%%%%%%%%%%%%%%%%%%%%

\subsubsection{Research component: Control and optimization}
\label{sec:control}

The primary goals of a smart, sustainable microgrid are to use electrical
energy as efficiently as possible, maximize the amount of energy coming
from renewable resources, and minimize the overall cost of energy while
retaining acceptable levels of reliability and quality. The optimization
and control module of this project will study how well
these goals can be met by scheduling electricity production and consumption
using data and forecasts from the modeling and analysis module.

Some of the important capabilities of the microgrid control system will
include peak shaving/shifting, ramp-rate management, and lowered overall
consumption.  The initial approach will focus on cooling loads, which make
up most of UHM's electricity demand.  Later in the project we will develop
optimal control strategies for other time-shiftable loads, such as water
heating, energy-intensive scientific research, and eventually electric
vehicles. The control effort will also manage UHM's many individual backup
generators as a coordinated asset, making them available for peak shaving
and ramp rate management. All these efforts will improve the demand profile
presented to the electric utility and reduce the campus's electricity
bills.

This research will study power control techniques with varying degrees of user 
interaction, from direct automation through voluntary user responses to campus 
alerts. Automation is most suited for systems that don't directly affect users or
for complex systems. For example, electricity demand for chillers and water
heaters can be deferred several minutes or hours by adjusting their
temperature setpoints slightly. Automation allows consistent and
fine-grained adjustments that would not be possible with user responses
alone (e.g., simultaneously adjusting many different lights or parts of the
cooling system) \cite{Motegi07, Piette06}. Automation can also provide
faster and more frequent responses than users alone, which will be
essential for integrating renewable power both at UHM and in the larger
power system \cite{Stromback11,Callaway09}.  In contrast, many
peak-reduction programs set financial incentives a day in advance in order
to involve users more easily \cite{Boisvert04, Stromback11}, which reduces
their suitability for renewable integration. However, rich user engagement 
will also be essential to obtain the greatest possible benefit from the 
microgrid. This will require innovation in the presentation of energy 
information to users and the design of interfaces that blend automation and 
user control - work which is described in more detail in the next section.
 
The core of the microgrid optimization system will be a multi-period stochastic
linear optimization model \cite{Kall05} that chooses how to control loads and 
generators in order to minimize the total cost of providing power to the campus 
in coordination with the system grid . 
The objective (cost) function of this model will use a broad definition
of the total cost of power, including the time-varying cost of electricity, 
utility charges based on the peak load and ramp-rate for the month,
the cost of operating backup generators as peak- or ramp-shaving units, 
and the "costs" of load control itself -- inconveniencing users or 
depleting their interest in future load control measures.
During each optimization cycle, the model will consider
several alternative scenarios, each defined by specific values for future
loads, renewable power production and electricity costs, over periods ranging from 
the next few minutes to the next 24 hours, as well as forecasts of the month's peak load 
and ramps. Scenarios will be generated that cover the forecasted values 
for each of these future parameters, as well as the uncertainty range of these
forecasts. The optimizer will choose a control strategy that minimizes
the expected cost of power over all these scenarios. Voluntary user responses 
tend to be strongest for infrequent events \cite{Boisvert04}
 \cite{Faruqui10, Stromback11} and vary from user to user, so we 
will also seek to optimize the frequency with which we call for action from each user.

In addition to developing the software and hardware systems required 
for control and optimization, we will measure the effectiveness of these 
control strategies in the field. This will serve two purposes: 
First, it will provide valuable
information on the magnitude and duration of energy time-shifting 
(real or virtual storage) available 
from a wide variety of building energy systems, as well as estimates of how
this capability changes over time as user interest increases or decreases. 
Second, this information will be used to fine-tune the
optimization and control algorithms, to more precisely achieve the desired
level of load control.

We will also explore tradeoffs between efficiency and peak-management in the operation
of the campus power system. For example UHM's
smart grid and smart chiller water distribution system will allow the campus to
run many generators or chillers at once, giving a strong peak-shaving or
ramp-shaving capability. However, either of these strategies would increase
total power consumption, since chillers and generators are less efficient
when they are run at part load. Similarly, if chiller water is cooler than needed
to maintain building comfort, there is some leeway to allow temperatures to rise
(reducing electricity demand) during a peak-load event. However, overall 
efficiency can be improved by maintaining chiller water at a higher temperature
from the outset \cite{Motegi07}.

