%%%%%%%%%%%%%%%%%%%%%%%%%%%%%% -*- Mode: Latex -*- %%%%%%%%%%%%%%%%%%%%%%%%%%%%
%% project.control.tex -- 
%% Author          : Philip Johnson
%% Created On      : Fri Jan 13 07:58:21 2012
%% Last Modified By: Matthias Fripp
%% Last Modified On: Sun Jan 23 2012
%%%%%%%%%%%%%%%%%%%%%%%%%%%%%%%%%%%%%%%%%%%%%%%%%%%%%%%%%%%%%%%%%%%%%%%%%%%%%%%

\subsubsection{Control and optimization}

The primary goals of a smart, sustainable microgrid are to use electrical
energy as efficiently as possible, maximize the amount of energy coming
from renewable resources, and minimize the overall cost of energy while
retaining acceptable levels of reliability and quality. The optimization
and control module of this project will study how well
these goals can be met by scheduling electricity production and consumption
using data and forecasts from the modeling and analysis module.

Some of the important capabilities of the microgrid control system will include: 
peak shaving/shifting, ramp-rate management, lowered overall consumption
and voltage and frequency regulation (*** drop the last two, since we don't discuss 
them further? ***). The initial approach will focus on cooling loads, which make up 
most of (*** are the largest component of? ***) UHM's electricity demand. 
Later in the project we will develop optimal control strategies for other 
time-shiftable loads, such as water heating, energy-intensive scientific research, 
and eventually electric vehicles. The control effort will also manage UHM's many 
individual backup generators as a coordinated asset, making them available for 
peak shaving and ramp rate management. All these efforts will improve the demand 
profile presented to the electric utility and reduce the campus's electricity bills.

This research will study power control techniques at several levels of user 
interaction, from direct automation through voluntary user responses to campus 
alerts. Direct automation will be used for systems that run in 
the background with no direct effect on users. For example, electricity demand for 
chillers and water heaters can be deferred several minutes or hours 
by adjusting their temperature setpoints slightly. However, chillers can
defer electricity consumption even longer if users participate by raising
the setpoint of in-building thermostats or by accepting automatic adjustments 
to these thermostats. Some other loads may best be managed by community 
awareness techniques that show microgrid participants how to help achieve cost savings and more 
efficient usage of renewable resources, e.g., by reducing lighting or deferring
energy-intensive batch processes during critical power periods. 
The user-facing elements of this project will build on knowledge gained in the 
campus's successful Kukui Cup competition to reduce student energy use, 
(*** as well as collaboration with smart metering researchers in Oxford University's
Lower Carbon Futures group?***). We will 
also conduct new research on effective design of user interfaces for
shared management of the microgrid.

In addition to developing the software and hardware systems required 
for control and optimization, we will measure the effectiveness of these 
control strategies in a real world setting. This will serve two purposes: 
First, it will provide valuable
information on the magnitude and duration of energy time-shifting 
(real or virtual storage) available 
from a wide variety of building energy systems, as well as estimates of how
this capability changes over time as user interest increases or decreases. 
Second, this information will be used to fine-tune the
optimization and control algorithms, to more precisely achieve the desired
level of load control.

The microgrid optimization system will use a multi-period stochastic
linear optimization model to choose how to control loads and backup 
generators in order to minimize the total cost of providing power to the campus. 
The objective function of this model will use a broad definition
of the total cost of power, including the time-varying cost of electricity, 
utility charges based on the peak load and ramp-rate for the year,
the cost of operating backup generators as peak- or ramp-shaving units, 
and the "costs" of load control itself -- inconveniencing users or 
depleting their interest in future load control measures.
During each optimization cycle, the model will consider
several alternative scenarios, each defined by specific values for future
loads, renewable power production and electricity costs, over periods ranging from 
the next few minutes to the next 24 hours, as well as forecasts of the year's peak load 
and ramps. Scenarios will be generated that cover the forecasted values 
for each of these future parameters, as well as the uncertainty range of these
forecasts. The optimizer will choose a control strategy that minimizes
the expected cost of power over all these scenarios.

We will also explore tradeoffs between efficiency and peak-management when
many pieces of equipment are operated at part load. For example, UHM's
smart grid and smart chiller water distribution system will allow the campus to
run many generators or chillers at once, giving a strong peak-shaving or
ramp-shaving capability. However, running many chillers or generators at once 
could increase total power consumption, since these machines are less efficient
when they are run at part load.


