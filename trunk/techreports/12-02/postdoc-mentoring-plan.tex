%%%%%%%%%%%%%%%%%%%%%%%%%%%%%% -*- Mode: Latex -*- %%%%%%%%%%%%%%%%%%%%%%%%%%%%
%% postdoc-mentoring-plan.tex -- 
%% Author          : Philip Johnson
%% Created On      : Tue Mar 31 11:42:10 2009
%% Last Modified By: Philip Johnson
%% Last Modified On: Wed Jan 18 13:29:17 2012
%%%%%%%%%%%%%%%%%%%%%%%%%%%%%%%%%%%%%%%%%%%%%%%%%%%%%%%%%%%%%%%%%%%%%%%%%%%%%%%

\documentclass{proposalnsf}
\usepackage[final]{graphicx}

% NSF proposal generation template style file.
% based on latex stylefiles  written by Stefan Llewellyn Smith and
% Sarah Gille, with contributions from other collaborators.

% Fix things so that figures tend to stay away from the last page. 
\renewcommand{\topfraction}{0.85}
\renewcommand{\textfraction}{0.1}
\renewcommand{\floatpagefraction}{0.75}

% this handles hanging indents for publications
\def\rrr#1\\{\par
\medskip\hbox{\vbox{\parindent=2em\hsize=6.12in
\hangindent=4em\hangafter=1#1}}}

\def\baselinestretch{1}

\begin{document}

\section*{Postdoctoral mentoring plan}

A postdoctoral researcher will be funded on this project.  The postdoctoral
researcher’s development will be enhanced through a program of structured
mentoring activities.  The goal of the mentoring program will be to provide
the skills, knowledge and experience to prepare the postdoctoral researcher
to excel in his/her career path.  To accomplish this goal, the mentoring
plan will follow the core competencies espoused by the National
Postdoctoral Association\footnote{http://www.nationalpostdoc.org/publications/mentoring-plans/mentoring-plan/core-competencies} including: (a) Discipline-Specific Conceptual
Knowledge, (b) Professional/Research Skill Development, (c) Communication
Skills, (d) Professionalism, (e) Leadership and Management Skills, and (f)
Responsible Conduct of Research.

Our mentoring plan for this project will include:

\begin{itemize}

\item Administration of the Core Competencies Self-Assessment Checklist
  developed by the National Postdoctoral Association.  This checklist can be used to
  establist a plan for professional development of the postdoc.


\item The postdoc will participate in seminars, workshops and individual
  consultations in order to develop their skill in identifying research
  funding opportunities and write competitive proposals.

\item We will provide the postdoc with opportunities to network with
  visiting scholars related to smart grid and other energy-related areas.

\item The postdoc will be supported with travel to at least one conference
  each year, with the goal of presenting a poster
  or paper at the conference.

\item The postdoc will be invited to participate in a monthly brown bag
  lunch series in which speakers will be invited to discuss subjects
  related to career development such as how to apply for a faculty
  position, career paths outside of academia, tips for negotiating salary
  and start-up funds, how to plan and independent research agenda, etc.

\item The postdoc will participate in regular research group meetings, in
  which members will be expected to present their research, and feedback
  and coaching will be given to help all members to develop their
  communication and presentation skills.

\end{itemize}

The success of this mentoring plan will be assessed by tracking the progress of the postdoctoral fellow through her/his plan, interviews of the postdoctoral fellow to assess satisfaction with the mentoring program, and tracking of the postdoctoral fellow’s progress toward his/her career goals after finishing the postdoc.


\end{document}




