%%%%%%%%%%%%%%%%%%%%%%%%%%%%%% -*- Mode: Latex -*- %%%%%%%%%%%%%%%%%%%%%%%%%%%%
%% data-management-plan.tex -- 
%% Author          : Philip Johnson
%% Created On      : Tue Mar 31 11:42:10 2009
%% Last Modified By: Philip Johnson
%% Last Modified On: Mon Jan 30 14:38:41 2012
%%%%%%%%%%%%%%%%%%%%%%%%%%%%%%%%%%%%%%%%%%%%%%%%%%%%%%%%%%%%%%%%%%%%%%%%%%%%%%%

\documentclass{proposalnsf}
\usepackage[final]{graphicx}

% NSF proposal generation template style file.
% based on latex stylefiles  written by Stefan Llewellyn Smith and
% Sarah Gille, with contributions from other collaborators.

% Fix things so that figures tend to stay away from the last page. 
\renewcommand{\topfraction}{0.85}
\renewcommand{\textfraction}{0.1}
\renewcommand{\floatpagefraction}{0.75}

% this handles hanging indents for publications
\def\rrr#1\\{\par
\medskip\hbox{\vbox{\parindent=2em\hsize=6.12in
\hangindent=4em\hangafter=1#1}}}

\def\baselinestretch{1}

\begin{document}

\section*{Data Management Plan}

\subsubsection*{Types of data}

As part of this project, many types of data will be collected.  This
includes environmental resource data gathered from sensors including
temperature, humidity, wind direction and speed, and solar irradiance. It
also includes electrical grid data including voltage, current, power, and
reactance. Finally, it includes power and energy data from distributed
generation devices such as photovoltaic systems.

All of the above data will be experimental measurement data obtained
through physical harvesting of data from the surrounding environment and
from energy production and storage devices.  The data will be captured
using standard environmental sensors such as anemometers, power meters,
pyrometers, etc.  It will be stored in a custom-built database, as
discussed next.

\subsubsection*{Data and metadata standards}

While there are many systems that have been developed to store
environmental sensor data, such as GeoCENS, Pachube, and the Berkeley
Sensor Database, there are no commonly recognized standards for the
formatting, storage, or transmission of the data we will be collecting,
storing, and analyzing.  Because of this, we plan to store our data using a
custom, but publicly documented, set of database schemas in an
internet-accessible server running a standard open source LAMP (Linux,
Apache, MySQL, Python) stack.  We also plan to design and implement a REST
protocol for web service oriented access and manipulation of the data.

\subsubsection*{Policies for access and sharing and provisions for appropriate protection and privacy}

As noted above, we plan to design and implement a web service that will
enable external access to the data by interested researchers.  The
environmental data that we will collect, store, analyze, and publicize will
not reveal personal characteristics of users.  We will restrict real-time
access to power and/or energy consumption data, or aggregate this data as
required, in order to prevent users from being able to gain behavioral
insight from patterns of energy usage.

\subsubsection*{Policies and provisions for re-use and re-distribution}

We plan to make the data collected in this research freely available under
a Creative Commons Attribution CC BY license.  This license will let others
use, distribute, and analyze the data we collect without restriction as
long as they credit us as the original creators of the data.

We believe that the data we collect will be of interest to others
developing "smart microgrid" systems as it will provide, at a
minimum, baseline data for environmental conditions we experienced during
the course of our research.

\subsubsection*{Plans for archiving and preservation of access}

During the course of this research, we will be storing the data in a web
accessible database as discussed above.  Archiving the data past the
conclusion of the study will be done via data archiving services provided
by the University of Hawaii.  All research reports and collateral documents
created in response to the data will also be permanently archived through
the University of Hawaii technical report services.  The University of
Hawaii implements standard best practices for data storage backup and
retrieval, including off-site storage, redundant power supplies, RAID disk
storage, and so forth.



\end{document}




