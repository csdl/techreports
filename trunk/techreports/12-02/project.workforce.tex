%%%%%%%%%%%%%%%%%%%%%%%%%%%%%% -*- Mode: Latex -*- %%%%%%%%%%%%%%%%%%%%%%%%%%%%
%% project.workforce.tex -- 
%% Author          : Philip Johnson
%% Created On      : Fri Jan 13 07:58:21 2012
%% Last Modified By: Philip Johnson
%% Last Modified On: Thu Jan 26 15:31:53 2012
%%%%%%%%%%%%%%%%%%%%%%%%%%%%%%%%%%%%%%%%%%%%%%%%%%%%%%%%%%%%%%%%%%%%%%%%%%%%%%%

\subsubsection{Research component: Education and workforce development}
\label{sec:education}
  
The Hawaii Clean Energy Initiative (HECI) is a Memorandum of 
Understanding (MOU) between the state of  Hawaii and
the Department of Energy signed in 2008 that set goals for Hawaii so that by 2030
70\% of our energy will come from clean energy sources
(30\% from energy efficiency and 40\% from renewable energy sources).   
Through  the Renewable Energy and Island Sustainability (REIS) goup  we have developed 
curriculum and courses in energy and sustainability to help train this workforce that will be
needed to achieve these goals.  Much as HCEI will rely more on local energy sources the
REIS group is working to have Hawaii more reliant on locally trained experts in energy and
sustainability.

At  the UHM we are developing multidisciplinary education and research programs that span the range of jobs that will be needed to supply new educated workforce in the energy and sustainability areas.  We have funded workforce training programs at the University of Hawai�i community college level (NSF Tribal Communities University Program (TCUP) Pre-Engineering Education Collaborative (PEEC)) and at the graduate and undergraduate level at the UHM through REIS (Department of Energy (DOE) workforce training grant).   The DOE workforce training grant is
developing curriculum over a broad range of energy topics considering engineering, natural
and social science, and policy.  Student in the graduate REIS program supported by the DOE
workforce training grant take two core graduate courses in energy (one class in engineering and
the other class in social science).  The students then take other courses so that they can 
concentrate on their research which varies from studying 
nanocomposites  for fuel cells to biofuels to wave  energy to smart grid security.
  
This proposal goes one step further by focusing on an in depth understanding of microgrids,
distributed renewable energy sources, demand response, and energy efficient management.
We  will work on further development of courses in the smart grid and renewable energy
areas with a focus on systems, software, and policy issues.   We will create a variety of graduate level courses including classes on the Smart Grid and Future Electrical Energy Systems, Advanced
Software Energy Systems, and Energy Resource Assessment.  These courses will all have
engineering, economic, social,and policy aspects and provide background knowledge along
with the two core REIS graduate courses and fundamental courses in Probability, Signal Processing, Communications and Networking, Software Engineering, and Economics.  
The courses are well integrated to the four research projects and will give students a good
foundation to conduct their research.  

In addition we
are working through local funding agencies to develop a short twenty hour course on smart grids
and integration of renewable sources that will be available to UHM students and faculty and also
the external community.  This course will be broken up into five four hour segments: grid
overview,  policies and standards, tools and capabilities, communications and networking and
security, and integration of sources.

A major component of education and training is conducting research while using the Smart
Campus Energy Lab (SCEL) and working on the UHM microgrid.  Both graduate and
undergraduate students will be working on research projects in conjunction with HECO engineers
and UHM facility people.  There will be a close integration between the different research
areas and also between education in the classroom where concepts are learned, analysis where
models, algorithms, optimization, and control methods are formulated,  software and hardware
simulation studies, and the UHM campus environment 
and microgrid where analysis and simulations are confirmed.  The research projects will also educate students in team work, project management, and improve their oral and written
communication skills.

\paragraph{Integration of members of under-represented minorities}

This project will collaborate with the Native Hawaiian Science and
Engineering Mentorship Program (NHSEMP) to broaden the participation of
under-represented groups. NHSEMP is a successful program housed at the
University of Hawaii funded (in part) by the National Science Foundation
Louis Stokes Alliance for Minority Participation Program and the
U.S.~Department of Education Native Hawaiian Education Program. The program
is already very successful in attracting members of the Pacific Islander
minority into the undergraduate engineering program. Our next goal is to
achieve a smooth transition of the most talented minority students into the
best graduate programs nationwide. For example, we are already implementing
an REU exchange program with our collaborators (at MIT) in conjunction with
a joint project between MIT and University of Hawaii [NSF Grants
ECCS-0725555 and ECCS-0725649]. Throughout 2008 and 2009, MIT hosted
Hawaiian minority undergraduates from Professor Kavcic's group. Their
video-documented experiences can be viewed at 
  http://www2.hawaii.edu/\verb+~+thanhvu/Videos.html. We are also
presently in the process of selecting qualified minority undergraduates to
send to Pittsburgh, PA to participate in our joint project with Carnegie
Mellon University [NSF Grants ECCS- ECCS-1029081]. We will continue to draw
representatives of underrepresented students into our research program in
conjunction with this project. UH minority undergraduates will spend their
semesters working as researchers with at University of Hawaii and their
summer/winter breaks as interns/researchers at HECO or visiting our
research partners on the mainland.

We will also collaborate with the Society of Women Engineers (SWE) and Women inTechnology
(WIT) to recruit both undergraduate and graduate students into our program.   Faculty and
students from the REIS group have already participated in a number of outreach activities to
recruit students (especially underrepresented students into considering careers in energy).  Some
events such as ``Wow! that's engineering'' have been targeted at middle school students, other
events such as the COE open houses have been targeted at high school students, and the
Hawaiian Electric Company outreach event at Bishop Museum target  the general community.   Activities include purchasing unassembled miniature wind turbine kits and having kids and adults at these events assemble the wind turbines.  Using fans, the wind turbine blades would turn and meters would show the electricity that was generated.