%%%%%%%%%%%%%%%%%%%%%%%%%%%%%% -*- Mode: Latex -*- %%%%%%%%%%%%%%%%%%%%%%%%%%%%
%% summary.tex -- 
%% Author          : Philip Johnson
%% Created On      : Tue Mar 31 11:42:10 2009
%% Last Modified By: Philip Johnson
%% Last Modified On: Thu Jan 12 11:05:00 2012
%%%%%%%%%%%%%%%%%%%%%%%%%%%%%%%%%%%%%%%%%%%%%%%%%%%%%%%%%%%%%%%%%%%%%%%%%%%%%%%

\documentclass{proposalnsf}
\usepackage[final]{graphicx}

% NSF proposal generation template style file.
% based on latex stylefiles  written by Stefan Llewellyn Smith and
% Sarah Gille, with contributions from other collaborators.

% Fix things so that figures tend to stay away from the last page. 
\renewcommand{\topfraction}{0.85}
\renewcommand{\textfraction}{0.1}
\renewcommand{\floatpagefraction}{0.75}

% this handles hanging indents for publications
\def\rrr#1\\{\par
\medskip\hbox{\vbox{\parindent=2em\hsize=6.12in
\hangindent=4em\hangafter=1#1}}}

\def\baselinestretch{1}

\begin{document}

\section*{Project Summary}
\renewcommand{\thepage} {A--\arabic{page}}

%% {\em The proposal must contain a summary of the proposed activity suitable for
%% publication, not more than one page in length. It should not be an abstract
%% of the proposal, but rather a self-contained description of the activity
%% that would result if the proposal were funded. The summary should be
%% written in the third person and include a statement of objectives and
%% methods to be employed. It must clearly address in separate statements
%% (within the one-page summary):

%% (1) the intellectual merit of the proposed activity; and

%% (2)the broader impacts resulting from the proposed activity. 

%% It should be informative to other persons working in the same or related
%% fields and, insofar as possible, understandable to a scientifically or
%% technically literate lay reader. Proposals that do not separately address
%% both merit review criteria within the one-page Project Summary will be
%% returned without review.

% What is the intellectual merit of the proposed activity?
% How important is the proposed activity to advancing knowledge and
% understanding within its own field or across different fields? How well
% qualified is the proposer (individual or team) to conduct the project? (If
% appropriate, the reviewer will comment on the quality of the prior work.)
% To what extent does the proposed activity suggest and explore creative,
% original, or potentially transformative concepts? How well conceived and
% organized is the proposed activity? Is there sufficient access to
% resources?

% What are the broader impacts of the proposed activity?
% How well does the activity advance discovery and understanding while
% promoting teaching, training, and learning? How well does the proposed
% activity broaden the participation of underrepresented groups (e.g.,
% gender, ethnicity, disability, geographic, etc.)? To what extent will it
% enhance the infrastructure for research and education, such as facilities,
% instrumentation, networks, and partnerships? Will the results be
% disseminated broadly to enhance scientific and technological understanding?
% What may be the benefits of the proposed activity to society? 

% NSF staff also will give careful consideration to the following in making
% funding decisions:

% Integration of Research and Education
% One of the principal strategies in support of NSF's goals is to foster
% integration of research and education through the programs, projects, and
% activities it supports at academic and research institutions. These
% institutions provide abundant opportunities where individuals may
% concurrently assume responsibilities as researchers, educators, and
% students and where all can engage in joint efforts that infuse education
% with the excitement of discovery and enrich research through the diversity
% of learning perspectives.


% Integrating Diversity into NSF Programs, Projects, and Activities
% Broadening opportunities and enabling the participation of all citizens --
% women and men, underrepresented minorities, and persons with disabilities
% -- is essential to the health and vitality of science and engineering. NSF
% is committed to this principle of diversity and deems it central to the
% programs, projects, and activities it considers and supports.

\noindent The state of Hawaii is more dependent on oil than any other state in
the nation in its reliance on oil for both transportation and electricity generation.
The Hawaii Clean Energy Initiative (HCEI 2008) calls for Hawaii to
rely on 70\% of its energy to come from clean energy sources by 2030.   The University of
Hawaii is playing a major role to assist the state in achieving these goals by
conducting research, education, and workforce training in energy and sustainability.
This proposal considers  pathways to smart sustainable microgrids.   The project
considers both theoretical and practical aspects of microgrids understanding their
behavior when subject to growing distributed renewable energy sources.   Four research
projects considering getting data to modeling and analysis to control and optimization to
social and economic analysis will be integrated into a graduate and undergraduate
education program on smart grids, renewable energy, and energy efficiency.  To
test out analysis and simulation studies we will work together with UHM
facilities and Hawaiian Electric Company (HECO) on  the UHM campus microgrid testbed.

\noindent {\bf Intellectual Merit.}  This project envisions what  will be the properties of future
microgrids by considering four research projects and integrating graduate and undergraduate
education.  The project uses information technology with communications and signal processing
to study changing microgrids  as they rely more on distributed intermittent renewable energy sources and use control and optimization algorithms to reduce  energy usage while creating  a 
more intelligent, stable, and secure distribution microgrid.  To achieve this we must gather all
relevant information about the microgrid.  This involves sensing, monitoring, and building networks
to gather data to a central server and storage system.   The UHM campus microgrid will serve
as our test-bed as we gather environmental resource data, electrical grid data,  building energy
data, and collect data surveys from students, faculty, and staff.   Important issues include resource allocation (sensor placement), data fusion and mining.   A second project considers modeling and
analyzing the microgrid using probabilistic signal processing models.  A key consideration is using
distributed secure algorithms that can accurately predict  the state of the microgrid along with
detecting anomalous events.  Once we have accurate models that are constructed and verified
through simulations and the data gathering we then consider the third project which involves
decision making.   This involves using distributed control and optimization algorithms to
consider creating a secure and stable microgrid while reducing costs and energy consumption.
An important consideration to achieving these goals is to also have an informed consumer
(students, faculty, and staff) that can understand the transition to a more intelligent, secure,
and stable microgrid relying more on locally generated energy and energy efficient practices.
The fourth project will consider social, economic, and policy implications of this transition.
Students and faculty will work in a newly created smart campus energy lab (SCEL) and also
work on the UHM campus microgrid.

\noindent{\bf Broader Impacts.}  
The project will provide great educational values for our graduate  and undergraduate students as they combine what they learn in the classroom  with theoretical and practical research projects in the SCEL and on the UHM campus microgrid. The project will also actively recruit underrepresented students through connections with the Native Hawaiian Science and Engineering Mentorship Program (NHSEMP) and the Society of Women Engineers (SWE).  We will work closely with HECO and UHM facilities to model, analyze, simulate, and control portions of the electric power grid including micro-grids and renewable energy sources.

\medskip

\noindent {\bf Key Words:} energy, sensing, modeling, control. Social, economic,
and policy implications.
\end{document}




