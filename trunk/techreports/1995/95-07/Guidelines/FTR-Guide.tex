%\documentstyle [nftimes,11pt,/group/csdl/tex/definemargins,/group/csdl/tex/lmacros]{article}

%\begin{document}
%\begin{center}
%{\Large\bf Formal Technical Reviews Guidelines \\
%  (EGSM and EIAM)}\\
%\end{center}

\chapter {Formal Technical Reviews Guidelines (EGSM and EIAM)}


\section*{Objective}
The objective of the review is to find as many errors (issues) as
possible in the source code through interaction with group members
(EGSM) or by working alone (EIAM).

\section*{Review Processes}
\subsection*{Individual method (EIAM)}
\begin {itemize}
\item All participants assume the role of {\it Reviewer}.
\item After successful connection to the database, a set of source
  nodes to be reviewed will be displayed on the summary buffer.
\item You may review these nodes in any order (based on your
personal preference, judgement, and/or technique).
\item When noticing an error or anomaly in the code, 
  create an issue node. All fields in the issue node must be completed.
\item After reviewing each source node, declare the status of the node
to ``reviewed''. 
\end{itemize}

\subsection*{Group method (EGSM)}
\noindent {\bf Presenter:}
\begin{itemize}
\item The presenter also performs the task of a reviewer (searching
  for errors).
\item After successful connection, wait for the moderator's
  instruction to start the review.
\item Discuss with the group, in what order the source nodes will be
  reviewed.
\item Select the node to be review. Read loudly (paraphrasing) the
  code line by line (speak up). For example, by letting the group know
  what this particular line or block of code does.
\item Don't speak too fast. Let the group digest it before going to
  the next line or block of code.
\item Stop immediately, when noticing any problems or questions.
  Throw the questions or concerns to the group, and let the group decide
  whether they are indeed a legitimate issue. If so, create an issue
  node.
\item Also stop immediately, when someone else interrupt you by
  raising a question, concern or comment.
\item After creating an issue node, fill out the Subject, Lines, and
  Description fields. Write a brief and concise description of the
  issue. Do not write or suggest any solution. However, you may state
  the issue as, for example, ``This should have been ...''.
\item Then save the node, so that all participants can see what you
  just type.
\item Now fill out the rest of the fields by casting your vote
  (Criticality,Suggested-by, and Confidence-level).

\item Wait until all participants have casted their vote, then close
  the node.
\item Continue reading/paraphrasing the remaining of the source node.
\item When no more problem is detected, declare or set the status of
  the node to ``reviewed''.
\item Continue the above processes until all source nodes are
  reviewed.
\end{itemize}

\noindent {\bf Moderator:}
\begin{itemize}
\item The moderator also performs the task of a reviewer (searching for errors).
\item After successful connection, ensure that everyone are ready
  (i.e., have been connected successfully). Then give the instruction to
  start the review.
\item At any time, follow the presenter closely. Interrupt the
  presenter when noticing any problems or questions. Discuss the
  problems/questions with the group.
\item Moderate group discussion. Cut lengthy discussion if necessary.
\item Ensure that the participants only discuss the issue, and NOT the
  solution to the issue. When the latter occurs, cut the discussion
  immediately.
\item Also cut discussion that does not lead to a legitimate issue. For
  example, issues that seek for alternative implementation for the given
  specification, etc.
\item Do not spend too much time on a particular source node. Go back
to this node later if necessary.
\item Slow down or speed up the presenter as necessary.
\item Change the presenter if necessary.
\end{itemize}

\noindent {\bf Reviewer:}
\begin{itemize}
\item After successful connection, wait for the moderator's
instruction to start the review.
\item At any time, follow the presenter closely. Interrupt the presenter when
noticing any problems or questions. Discuss the
problems or questions with the group.
\end{itemize}


\subsection*{Other Guidelines}
\begin {itemize}
\item Your task is to identify defects or errors in the source code,
NOT to decide what to do about them.

\item Think critically. Assume that the code you are reviewing is wrong.

\item Do not stop reviewing the code once an error is detected.
You may assume that the error has been corrected when
reviewing the remaining code. For example, the array declaration is
missing, and yet the code is accessing the array. 
You should raise one issue regarding this missing array declaration,
and continue reviewing the remaining code by assuming the array
declaration is correct.

\item Do not raise issues regarding the missing or incorrect
  specification. Assume the given code is already implemented with the 
  correct specification. For example, when the specification stated
  that each instance of Driver class have a name, then do not raise issues
  asking for ``Last-Name'' or ``First-Name''.

\item Similarly, do not raise issues that require the domain
knowledge not mentioned in the specification. For
example, each instance of Driver should include driver's address and
phone number.

\item Do not raise issues regarding alternative implementation of the
  given specification/code. The goal of this review is to simply
  identifying errors or issues in the given code with the assumption that
  the specification is already correct. 
 For example, do not raise issues that state
  this code should   use pointer instead of array when the
  specification explicitly states it uses array; or internal
  representation of BigInt  should use pure character string instead of BCD
  when the specification explicitly states it used BCD representation, etc. 

\item Do not discuss or record the solution to the issue. However, you may
  state the issue as ``This should have been ....''.

\item Use separate issue nodes when recording the same issue but
occurring at different places in the code.

\item When filling out a field with selectable items, choose the one
that closely represent your opinion.

\item If you do not understand what the code does, ask the
producer. Danu or Cam will be available during the review session to
answer questions regarding the specification.

\end{itemize}

\section* {Questions ?}
Send any questions or comments to: {\bf dat@uhunix.uhcc.hawaii.edu}
 

%%\end{document}




