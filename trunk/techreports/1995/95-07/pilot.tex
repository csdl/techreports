\documentstyle [nftimes,11pt,/group/csdl/tex/named-citations,
                       /group/csdl/tex/definemargins,
                       /group/csdl/tex/lmacros]{report}

% Psfig/TeX 
\def\PsfigVersion{1.9}
% dvips version
%
% All psfig/tex software, documentation, and related files
% in this distribution of psfig/tex are 
% Copyright 1987, 1988, 1991 Trevor J. Darrell
%
% Permission is granted for use and non-profit distribution of psfig/tex 
% providing that this notice is clearly maintained. The right to
% distribute any portion of psfig/tex for profit or as part of any commercial
% product is specifically reserved for the author(s) of that portion.
%
% *** Feel free to make local modifications of psfig as you wish,
% *** but DO NOT post any changed or modified versions of ``psfig''
% *** directly to the net. Send them to me and I'll try to incorporate
% *** them into future versions. If you want to take the psfig code 
% *** and make a new program (subject to the copyright above), distribute it, 
% *** (and maintain it) that's fine, just don't call it psfig.
%
% Bugs and improvements to trevor@media.mit.edu.
%
% Thanks to Greg Hager (GDH) and Ned Batchelder for their contributions
% to the original version of this project.
%
% Modified by J. Daniel Smith on 9 October 1990 to accept the
% %%BoundingBox: comment with or without a space after the colon.  Stole
% file reading code from Tom Rokicki's EPSF.TEX file (see below).
%
% More modifications by J. Daniel Smith on 29 March 1991 to allow the
% the included PostScript figure to be rotated.  The amount of
% rotation is specified by the "angle=" parameter of the \psfig command.
%
% Modified by Robert Russell on June 25, 1991 to allow users to specify
% .ps filenames which don't yet exist, provided they explicitly provide
% boundingbox information via the \psfig command. Note: This will only work
% if the "file=" parameter follows all four "bb???=" parameters in the
% command. This is due to the order in which psfig interprets these params.
%
%  3 Jul 1991	JDS	check if file already read in once
%  4 Sep 1991	JDS	fixed incorrect computation of rotated
%			bounding box
% 25 Sep 1991	GVR	expanded synopsis of \psfig
% 14 Oct 1991	JDS	\fbox code from LaTeX so \psdraft works with TeX
%			changed \typeout to \ps@typeout
% 17 Oct 1991	JDS	added \psscalefirst and \psrotatefirst
%

% From: gvr@cs.brown.edu (George V. Reilly)
%
% \psdraft	draws an outline box, but doesn't include the figure
%		in the DVI file.  Useful for previewing.
%
% \psfull	includes the figure in the DVI file (default).
%
% \psscalefirst width= or height= specifies the size of the figure
% 		before rotation.
% \psrotatefirst (default) width= or height= specifies the size of the
% 		 figure after rotation.  Asymetric figures will
% 		 appear to shrink.
%
% \psfigurepath#1	sets the path to search for the figure
%
% \psfig
% usage: \psfig{file=, figure=, height=, width=,
%			bbllx=, bblly=, bburx=, bbury=,
%			rheight=, rwidth=, clip=, angle=, silent=}
%
%	"file" is the filename.  If no path name is specified and the
%		file is not found in the current directory,
%		it will be looked for in directory \psfigurepath.
%	"figure" is a synonym for "file".
%	By default, the width and height of the figure are taken from
%		the BoundingBox of the figure.
%	If "width" is specified, the figure is scaled so that it has
%		the specified width.  Its height changes proportionately.
%	If "height" is specified, the figure is scaled so that it has
%		the specified height.  Its width changes proportionately.
%	If both "width" and "height" are specified, the figure is scaled
%		anamorphically.
%	"bbllx", "bblly", "bburx", and "bbury" control the PostScript
%		BoundingBox.  If these four values are specified
%               *before* the "file" option, the PSFIG will not try to
%               open the PostScript file.
%	"rheight" and "rwidth" are the reserved height and width
%		of the figure, i.e., how big TeX actually thinks
%		the figure is.  They default to "width" and "height".
%	The "clip" option ensures that no portion of the figure will
%		appear outside its BoundingBox.  "clip=" is a switch and
%		takes no value, but the `=' must be present.
%	The "angle" option specifies the angle of rotation (degrees, ccw).
%	The "silent" option makes \psfig work silently.
%

% check to see if macros already loaded in (maybe some other file says
% "\input psfig") ...
\ifx\undefined\psfig\else\endinput\fi

%
% from a suggestion by eijkhout@csrd.uiuc.edu to allow
% loading as a style file. Changed to avoid problems
% with amstex per suggestion by jbence@math.ucla.edu

\let\LaTeXAtSign=\@
\let\@=\relax
\edef\psfigRestoreAt{\catcode`\@=\number\catcode`@\relax}
%\edef\psfigRestoreAt{\catcode`@=\number\catcode`@\relax}
\catcode`\@=11\relax
\newwrite\@unused
\def\ps@typeout#1{{\let\protect\string\immediate\write\@unused{#1}}}
\ps@typeout{psfig/tex \PsfigVersion}

%% Here's how you define your figure path.  Should be set up with null
%% default and a user useable definition.

\def\figurepath{./}
\def\psfigurepath#1{\edef\figurepath{#1}}

%
% @psdo control structure -- similar to Latex @for.
% I redefined these with different names so that psfig can
% be used with TeX as well as LaTeX, and so that it will not 
% be vunerable to future changes in LaTeX's internal
% control structure,
%
\def\@nnil{\@nil}
\def\@empty{}
\def\@psdonoop#1\@@#2#3{}
\def\@psdo#1:=#2\do#3{\edef\@psdotmp{#2}\ifx\@psdotmp\@empty \else
    \expandafter\@psdoloop#2,\@nil,\@nil\@@#1{#3}\fi}
\def\@psdoloop#1,#2,#3\@@#4#5{\def#4{#1}\ifx #4\@nnil \else
       #5\def#4{#2}\ifx #4\@nnil \else#5\@ipsdoloop #3\@@#4{#5}\fi\fi}
\def\@ipsdoloop#1,#2\@@#3#4{\def#3{#1}\ifx #3\@nnil 
       \let\@nextwhile=\@psdonoop \else
      #4\relax\let\@nextwhile=\@ipsdoloop\fi\@nextwhile#2\@@#3{#4}}
\def\@tpsdo#1:=#2\do#3{\xdef\@psdotmp{#2}\ifx\@psdotmp\@empty \else
    \@tpsdoloop#2\@nil\@nil\@@#1{#3}\fi}
\def\@tpsdoloop#1#2\@@#3#4{\def#3{#1}\ifx #3\@nnil 
       \let\@nextwhile=\@psdonoop \else
      #4\relax\let\@nextwhile=\@tpsdoloop\fi\@nextwhile#2\@@#3{#4}}
% 
% \fbox is defined in latex.tex; so if \fbox is undefined, assume that
% we are not in LaTeX.
% Perhaps this could be done better???
\ifx\undefined\fbox
% \fbox code from modified slightly from LaTeX
\newdimen\fboxrule
\newdimen\fboxsep
\newdimen\ps@tempdima
\newbox\ps@tempboxa
\fboxsep = 3pt
\fboxrule = .4pt
\long\def\fbox#1{\leavevmode\setbox\ps@tempboxa\hbox{#1}\ps@tempdima\fboxrule
    \advance\ps@tempdima \fboxsep \advance\ps@tempdima \dp\ps@tempboxa
   \hbox{\lower \ps@tempdima\hbox
  {\vbox{\hrule height \fboxrule
          \hbox{\vrule width \fboxrule \hskip\fboxsep
          \vbox{\vskip\fboxsep \box\ps@tempboxa\vskip\fboxsep}\hskip 
                 \fboxsep\vrule width \fboxrule}
                 \hrule height \fboxrule}}}}
\fi
%
%%%%%%%%%%%%%%%%%%%%%%%%%%%%%%%%%%%%%%%%%%%%%%%%%%%%%%%%%%%%%%%%%%%
% file reading stuff from epsf.tex
%   EPSF.TEX macro file:
%   Written by Tomas Rokicki of Radical Eye Software, 29 Mar 1989.
%   Revised by Don Knuth, 3 Jan 1990.
%   Revised by Tomas Rokicki to accept bounding boxes with no
%      space after the colon, 18 Jul 1990.
%   Portions modified/removed for use in PSFIG package by
%      J. Daniel Smith, 9 October 1990.
%
\newread\ps@stream
\newif\ifnot@eof       % continue looking for the bounding box?
\newif\if@noisy        % report what you're making?
\newif\if@atend        % %%BoundingBox: has (at end) specification
\newif\if@psfile       % does this look like a PostScript file?
%
% PostScript files should start with `%!'
%
{\catcode`\%=12\global\gdef\epsf@start{%!}}
\def\epsf@PS{PS}
%
\def\epsf@getbb#1{%
%
%   The first thing we need to do is to open the
%   PostScript file, if possible.
%
\openin\ps@stream=#1
\ifeof\ps@stream\ps@typeout{Error, File #1 not found}\else
%
%   Okay, we got it. Now we'll scan lines until we find one that doesn't
%   start with %. We're looking for the bounding box comment.
%
   {\not@eoftrue \chardef\other=12
    \def\do##1{\catcode`##1=\other}\dospecials \catcode`\ =10
    \loop
       \if@psfile
	  \read\ps@stream to \epsf@fileline
       \else{
	  \obeyspaces
          \read\ps@stream to \epsf@tmp\global\let\epsf@fileline\epsf@tmp}
       \fi
       \ifeof\ps@stream\not@eoffalse\else
%
%   Check the first line for `%!'.  Issue a warning message if its not
%   there, since the file might not be a PostScript file.
%
       \if@psfile\else
       \expandafter\epsf@test\epsf@fileline:. \\%
       \fi
%
%   We check to see if the first character is a % sign;
%   if so, we look further and stop only if the line begins with
%   `%%BoundingBox:' and the `(atend)' specification was not found.
%   That is, the only way to stop is when the end of file is reached,
%   or a `%%BoundingBox: llx lly urx ury' line is found.
%
          \expandafter\epsf@aux\epsf@fileline:. \\%
       \fi
   \ifnot@eof\repeat
   }\closein\ps@stream\fi}%
%
% This tests if the file we are reading looks like a PostScript file.
%
\long\def\epsf@test#1#2#3:#4\\{\def\epsf@testit{#1#2}
			\ifx\epsf@testit\epsf@start\else
\ps@typeout{Warning! File does not start with `\epsf@start'.  It may not be a PostScript file.}
			\fi
			\@psfiletrue} % don't test after 1st line
%
%   We still need to define the tricky \epsf@aux macro. This requires
%   a couple of magic constants for comparison purposes.
%
{\catcode`\%=12\global\let\epsf@percent=%\global\def\epsf@bblit{%BoundingBox}}
%
%
%   So we're ready to check for `%BoundingBox:' and to grab the
%   values if they are found.  We continue searching if `(at end)'
%   was found after the `%BoundingBox:'.
%
\long\def\epsf@aux#1#2:#3\\{\ifx#1\epsf@percent
   \def\epsf@testit{#2}\ifx\epsf@testit\epsf@bblit
	\@atendfalse
        \epsf@atend #3 . \\%
	\if@atend	
	   \if@verbose{
		\ps@typeout{psfig: found `(atend)'; continuing search}
	   }\fi
        \else
        \epsf@grab #3 . . . \\%
        \not@eoffalse
        \global\no@bbfalse
        \fi
   \fi\fi}%
%
%   Here we grab the values and stuff them in the appropriate definitions.
%
\def\epsf@grab #1 #2 #3 #4 #5\\{%
   \global\def\epsf@llx{#1}\ifx\epsf@llx\empty
      \epsf@grab #2 #3 #4 #5 .\\\else
   \global\def\epsf@lly{#2}%
   \global\def\epsf@urx{#3}\global\def\epsf@ury{#4}\fi}%
%
% Determine if the stuff following the %%BoundingBox is `(atend)'
% J. Daniel Smith.  Copied from \epsf@grab above.
%
\def\epsf@atendlit{(atend)} 
\def\epsf@atend #1 #2 #3\\{%
   \def\epsf@tmp{#1}\ifx\epsf@tmp\empty
      \epsf@atend #2 #3 .\\\else
   \ifx\epsf@tmp\epsf@atendlit\@atendtrue\fi\fi}


% End of file reading stuff from epsf.tex
%%%%%%%%%%%%%%%%%%%%%%%%%%%%%%%%%%%%%%%%%%%%%%%%%%%%%%%%%%%%%%%%%%%

%%%%%%%%%%%%%%%%%%%%%%%%%%%%%%%%%%%%%%%%%%%%%%%%%%%%%%%%%%%%%%%%%%%
% trigonometry stuff from "trig.tex"
\chardef\psletter = 11 % won't conflict with \begin{letter} now...
\chardef\other = 12

\newif \ifdebug %%% turn me on to see TeX hard at work ...
\newif\ifc@mpute %%% don't need to compute some values
\c@mputetrue % but assume that we do

\let\then = \relax
\def\r@dian{pt }
\let\r@dians = \r@dian
\let\dimensionless@nit = \r@dian
\let\dimensionless@nits = \dimensionless@nit
\def\internal@nit{sp }
\let\internal@nits = \internal@nit
\newif\ifstillc@nverging
\def \Mess@ge #1{\ifdebug \then \message {#1} \fi}

{ %%% Things that need abnormal catcodes %%%
	\catcode `\@ = \psletter
	\gdef \nodimen {\expandafter \n@dimen \the \dimen}
	\gdef \term #1 #2 #3%
	       {\edef \t@ {\the #1}%%% freeze parameter 1 (count, by value)
		\edef \t@@ {\expandafter \n@dimen \the #2\r@dian}%
				   %%% freeze parameter 2 (dimen, by value)
		\t@rm {\t@} {\t@@} {#3}%
	       }
	\gdef \t@rm #1 #2 #3%
	       {{%
		\count 0 = 0
		\dimen 0 = 1 \dimensionless@nit
		\dimen 2 = #2\relax
		\Mess@ge {Calculating term #1 of \nodimen 2}%
		\loop
		\ifnum	\count 0 < #1
		\then	\advance \count 0 by 1
			\Mess@ge {Iteration \the \count 0 \space}%
			\Multiply \dimen 0 by {\dimen 2}%
			\Mess@ge {After multiplication, term = \nodimen 0}%
			\Divide \dimen 0 by {\count 0}%
			\Mess@ge {After division, term = \nodimen 0}%
		\repeat
		\Mess@ge {Final value for term #1 of 
				\nodimen 2 \space is \nodimen 0}%
		\xdef \Term {#3 = \nodimen 0 \r@dians}%
		\aftergroup \Term
	       }}
	\catcode `\p = \other
	\catcode `\t = \other
	\gdef \n@dimen #1pt{#1} %%% throw away the ``pt''
}

\def \Divide #1by #2{\divide #1 by #2} %%% just a synonym

\def \Multiply #1by #2%%% allows division of a dimen by a dimen
       {{%%% should really freeze parameter 2 (dimen, passed by value)
	\count 0 = #1\relax
	\count 2 = #2\relax
	\count 4 = 65536
	\Mess@ge {Before scaling, count 0 = \the \count 0 \space and
			count 2 = \the \count 2}%
	\ifnum	\count 0 > 32767 %%% do our best to avoid overflow
	\then	\divide \count 0 by 4
		\divide \count 4 by 4
	\else	\ifnum	\count 0 < -32767
		\then	\divide \count 0 by 4
			\divide \count 4 by 4
		\else
		\fi
	\fi
	\ifnum	\count 2 > 32767 %%% while retaining reasonable accuracy
	\then	\divide \count 2 by 4
		\divide \count 4 by 4
	\else	\ifnum	\count 2 < -32767
		\then	\divide \count 2 by 4
			\divide \count 4 by 4
		\else
		\fi
	\fi
	\multiply \count 0 by \count 2
	\divide \count 0 by \count 4
	\xdef \product {#1 = \the \count 0 \internal@nits}%
	\aftergroup \product
       }}

\def\r@duce{\ifdim\dimen0 > 90\r@dian \then   % sin(x+90) = sin(180-x)
		\multiply\dimen0 by -1
		\advance\dimen0 by 180\r@dian
		\r@duce
	    \else \ifdim\dimen0 < -90\r@dian \then  % sin(-x) = sin(360+x)
		\advance\dimen0 by 360\r@dian
		\r@duce
		\fi
	    \fi}

\def\Sine#1%
       {{%
	\dimen 0 = #1 \r@dian
	\r@duce
	\ifdim\dimen0 = -90\r@dian \then
	   \dimen4 = -1\r@dian
	   \c@mputefalse
	\fi
	\ifdim\dimen0 = 90\r@dian \then
	   \dimen4 = 1\r@dian
	   \c@mputefalse
	\fi
	\ifdim\dimen0 = 0\r@dian \then
	   \dimen4 = 0\r@dian
	   \c@mputefalse
	\fi
%
	\ifc@mpute \then
        	% convert degrees to radians
		\divide\dimen0 by 180
		\dimen0=3.141592654\dimen0
%
		\dimen 2 = 3.1415926535897963\r@dian %%% a well-known constant
		\divide\dimen 2 by 2 %%% we only deal with -pi/2 : pi/2
		\Mess@ge {Sin: calculating Sin of \nodimen 0}%
		\count 0 = 1 %%% see power-series expansion for sine
		\dimen 2 = 1 \r@dian %%% ditto
		\dimen 4 = 0 \r@dian %%% ditto
		\loop
			\ifnum	\dimen 2 = 0 %%% then we've done
			\then	\stillc@nvergingfalse 
			\else	\stillc@nvergingtrue
			\fi
			\ifstillc@nverging %%% then calculate next term
			\then	\term {\count 0} {\dimen 0} {\dimen 2}%
				\advance \count 0 by 2
				\count 2 = \count 0
				\divide \count 2 by 2
				\ifodd	\count 2 %%% signs alternate
				\then	\advance \dimen 4 by \dimen 2
				\else	\advance \dimen 4 by -\dimen 2
				\fi
		\repeat
	\fi		
			\xdef \sine {\nodimen 4}%
       }}

% Now the Cosine can be calculated easily by calling \Sine
\def\Cosine#1{\ifx\sine\UnDefined\edef\Savesine{\relax}\else
		             \edef\Savesine{\sine}\fi
	{\dimen0=#1\r@dian\advance\dimen0 by 90\r@dian
	 \Sine{\nodimen 0}
	 \xdef\cosine{\sine}
	 \xdef\sine{\Savesine}}}	      
% end of trig stuff
%%%%%%%%%%%%%%%%%%%%%%%%%%%%%%%%%%%%%%%%%%%%%%%%%%%%%%%%%%%%%%%%%%%%

\def\psdraft{
	\def\@psdraft{0}
	%\ps@typeout{draft level now is \@psdraft \space . }
}
\def\psfull{
	\def\@psdraft{100}
	%\ps@typeout{draft level now is \@psdraft \space . }
}

\psfull

\newif\if@scalefirst
\def\psscalefirst{\@scalefirsttrue}
\def\psrotatefirst{\@scalefirstfalse}
\psrotatefirst

\newif\if@draftbox
\def\psnodraftbox{
	\@draftboxfalse
}
\def\psdraftbox{
	\@draftboxtrue
}
\@draftboxtrue

\newif\if@prologfile
\newif\if@postlogfile
\def\pssilent{
	\@noisyfalse
}
\def\psnoisy{
	\@noisytrue
}
\psnoisy
%%% These are for the option list.
%%% A specification of the form a = b maps to calling \@p@@sa{b}
\newif\if@bbllx
\newif\if@bblly
\newif\if@bburx
\newif\if@bbury
\newif\if@height
\newif\if@width
\newif\if@rheight
\newif\if@rwidth
\newif\if@angle
\newif\if@clip
\newif\if@verbose
\def\@p@@sclip#1{\@cliptrue}


\newif\if@decmpr

%%% GDH 7/26/87 -- changed so that it first looks in the local directory,
%%% then in a specified global directory for the ps file.
%%% RPR 6/25/91 -- changed so that it defaults to user-supplied name if
%%% boundingbox info is specified, assuming graphic will be created by
%%% print time.
%%% TJD 10/19/91 -- added bbfile vs. file distinction, and @decmpr flag

\def\@p@@sfigure#1{\def\@p@sfile{null}\def\@p@sbbfile{null}
	        \openin1=#1.bb
		\ifeof1\closein1
	        	\openin1=\figurepath#1.bb
			\ifeof1\closein1
			        \openin1=#1
				\ifeof1\closein1%
				       \openin1=\figurepath#1
					\ifeof1
					   \ps@typeout{Error, File #1 not found}
						\if@bbllx\if@bblly
				   		\if@bburx\if@bbury
			      				\def\@p@sfile{#1}%
			      				\def\@p@sbbfile{#1}%
							\@decmprfalse
				  	   	\fi\fi\fi\fi
					\else\closein1
				    		\def\@p@sfile{\figurepath#1}%
				    		\def\@p@sbbfile{\figurepath#1}%
						\@decmprfalse
	                       		\fi%
			 	\else\closein1%
					\def\@p@sfile{#1}
					\def\@p@sbbfile{#1}
					\@decmprfalse
			 	\fi
			\else
				\def\@p@sfile{\figurepath#1}
				\def\@p@sbbfile{\figurepath#1.bb}
				\@decmprtrue
			\fi
		\else
			\def\@p@sfile{#1}
			\def\@p@sbbfile{#1.bb}
			\@decmprtrue
		\fi}

\def\@p@@sfile#1{\@p@@sfigure{#1}}

\def\@p@@sbbllx#1{
		%\ps@typeout{bbllx is #1}
		\@bbllxtrue
		\dimen100=#1
		\edef\@p@sbbllx{\number\dimen100}
}
\def\@p@@sbblly#1{
		%\ps@typeout{bblly is #1}
		\@bbllytrue
		\dimen100=#1
		\edef\@p@sbblly{\number\dimen100}
}
\def\@p@@sbburx#1{
		%\ps@typeout{bburx is #1}
		\@bburxtrue
		\dimen100=#1
		\edef\@p@sbburx{\number\dimen100}
}
\def\@p@@sbbury#1{
		%\ps@typeout{bbury is #1}
		\@bburytrue
		\dimen100=#1
		\edef\@p@sbbury{\number\dimen100}
}
\def\@p@@sheight#1{
		\@heighttrue
		\dimen100=#1
   		\edef\@p@sheight{\number\dimen100}
		%\ps@typeout{Height is \@p@sheight}
}
\def\@p@@swidth#1{
		%\ps@typeout{Width is #1}
		\@widthtrue
		\dimen100=#1
		\edef\@p@swidth{\number\dimen100}
}
\def\@p@@srheight#1{
		%\ps@typeout{Reserved height is #1}
		\@rheighttrue
		\dimen100=#1
		\edef\@p@srheight{\number\dimen100}
}
\def\@p@@srwidth#1{
		%\ps@typeout{Reserved width is #1}
		\@rwidthtrue
		\dimen100=#1
		\edef\@p@srwidth{\number\dimen100}
}
\def\@p@@sangle#1{
		%\ps@typeout{Rotation is #1}
		\@angletrue
%		\dimen100=#1
		\edef\@p@sangle{#1} %\number\dimen100}
}
\def\@p@@ssilent#1{ 
		\@verbosefalse
}
\def\@p@@sprolog#1{\@prologfiletrue\def\@prologfileval{#1}}
\def\@p@@spostlog#1{\@postlogfiletrue\def\@postlogfileval{#1}}
\def\@cs@name#1{\csname #1\endcsname}
\def\@setparms#1=#2,{\@cs@name{@p@@s#1}{#2}}
%
% initialize the defaults (size the size of the figure)
%
\def\ps@init@parms{
		\@bbllxfalse \@bbllyfalse
		\@bburxfalse \@bburyfalse
		\@heightfalse \@widthfalse
		\@rheightfalse \@rwidthfalse
		\def\@p@sbbllx{}\def\@p@sbblly{}
		\def\@p@sbburx{}\def\@p@sbbury{}
		\def\@p@sheight{}\def\@p@swidth{}
		\def\@p@srheight{}\def\@p@srwidth{}
		\def\@p@sangle{0}
		\def\@p@sfile{} \def\@p@sbbfile{}
		\def\@p@scost{10}
		\def\@sc{}
		\@prologfilefalse
		\@postlogfilefalse
		\@clipfalse
		\if@noisy
			\@verbosetrue
		\else
			\@verbosefalse
		\fi
}
%
% Go through the options setting things up.
%
\def\parse@ps@parms#1{
	 	\@psdo\@psfiga:=#1\do
		   {\expandafter\@setparms\@psfiga,}}
%
% Compute bb height and width
%
\newif\ifno@bb
\def\bb@missing{
	\if@verbose{
		\ps@typeout{psfig: searching \@p@sbbfile \space  for bounding box}
	}\fi
	\no@bbtrue
	\epsf@getbb{\@p@sbbfile}
        \ifno@bb \else \bb@cull\epsf@llx\epsf@lly\epsf@urx\epsf@ury\fi
}	
\def\bb@cull#1#2#3#4{
	\dimen100=#1 bp\edef\@p@sbbllx{\number\dimen100}
	\dimen100=#2 bp\edef\@p@sbblly{\number\dimen100}
	\dimen100=#3 bp\edef\@p@sbburx{\number\dimen100}
	\dimen100=#4 bp\edef\@p@sbbury{\number\dimen100}
	\no@bbfalse
}
% rotate point (#1,#2) about (0,0).
% The sine and cosine of the angle are already stored in \sine and
% \cosine.  The result is placed in (\p@intvaluex, \p@intvaluey).
\newdimen\p@intvaluex
\newdimen\p@intvaluey
\def\rotate@#1#2{{\dimen0=#1 sp\dimen1=#2 sp
%            	calculate x' = x \cos\theta - y \sin\theta
		  \global\p@intvaluex=\cosine\dimen0
		  \dimen3=\sine\dimen1
		  \global\advance\p@intvaluex by -\dimen3
% 		calculate y' = x \sin\theta + y \cos\theta
		  \global\p@intvaluey=\sine\dimen0
		  \dimen3=\cosine\dimen1
		  \global\advance\p@intvaluey by \dimen3
		  }}
\def\compute@bb{
		\no@bbfalse
		\if@bbllx \else \no@bbtrue \fi
		\if@bblly \else \no@bbtrue \fi
		\if@bburx \else \no@bbtrue \fi
		\if@bbury \else \no@bbtrue \fi
		\ifno@bb \bb@missing \fi
		\ifno@bb \ps@typeout{FATAL ERROR: no bb supplied or found}
			\no-bb-error
		\fi
		%
%\ps@typeout{BB: \@p@sbbllx, \@p@sbblly, \@p@sbburx, \@p@sbbury} 
%
% store height/width of original (unrotated) bounding box
		\count203=\@p@sbburx
		\count204=\@p@sbbury
		\advance\count203 by -\@p@sbbllx
		\advance\count204 by -\@p@sbblly
		\edef\ps@bbw{\number\count203}
		\edef\ps@bbh{\number\count204}
		%\ps@typeout{ psbbh = \ps@bbh, psbbw = \ps@bbw }
		\if@angle 
			\Sine{\@p@sangle}\Cosine{\@p@sangle}
	        	{\dimen100=\maxdimen\xdef\r@p@sbbllx{\number\dimen100}
					    \xdef\r@p@sbblly{\number\dimen100}
			                    \xdef\r@p@sbburx{-\number\dimen100}
					    \xdef\r@p@sbbury{-\number\dimen100}}
%
% Need to rotate all four points and take the X-Y extremes of the new
% points as the new bounding box.
                        \def\minmaxtest{
			   \ifnum\number\p@intvaluex<\r@p@sbbllx
			      \xdef\r@p@sbbllx{\number\p@intvaluex}\fi
			   \ifnum\number\p@intvaluex>\r@p@sbburx
			      \xdef\r@p@sbburx{\number\p@intvaluex}\fi
			   \ifnum\number\p@intvaluey<\r@p@sbblly
			      \xdef\r@p@sbblly{\number\p@intvaluey}\fi
			   \ifnum\number\p@intvaluey>\r@p@sbbury
			      \xdef\r@p@sbbury{\number\p@intvaluey}\fi
			   }
%			lower left
			\rotate@{\@p@sbbllx}{\@p@sbblly}
			\minmaxtest
%			upper left
			\rotate@{\@p@sbbllx}{\@p@sbbury}
			\minmaxtest
%			lower right
			\rotate@{\@p@sbburx}{\@p@sbblly}
			\minmaxtest
%			upper right
			\rotate@{\@p@sbburx}{\@p@sbbury}
			\minmaxtest
			\edef\@p@sbbllx{\r@p@sbbllx}\edef\@p@sbblly{\r@p@sbblly}
			\edef\@p@sbburx{\r@p@sbburx}\edef\@p@sbbury{\r@p@sbbury}
%\ps@typeout{rotated BB: \r@p@sbbllx, \r@p@sbblly, \r@p@sbburx, \r@p@sbbury}
		\fi
		\count203=\@p@sbburx
		\count204=\@p@sbbury
		\advance\count203 by -\@p@sbbllx
		\advance\count204 by -\@p@sbblly
		\edef\@bbw{\number\count203}
		\edef\@bbh{\number\count204}
		%\ps@typeout{ bbh = \@bbh, bbw = \@bbw }
}
%
% \in@hundreds performs #1 * (#2 / #3) correct to the hundreds,
%	then leaves the result in @result
%
\def\in@hundreds#1#2#3{\count240=#2 \count241=#3
		     \count100=\count240	% 100 is first digit #2/#3
		     \divide\count100 by \count241
		     \count101=\count100
		     \multiply\count101 by \count241
		     \advance\count240 by -\count101
		     \multiply\count240 by 10
		     \count101=\count240	%101 is second digit of #2/#3
		     \divide\count101 by \count241
		     \count102=\count101
		     \multiply\count102 by \count241
		     \advance\count240 by -\count102
		     \multiply\count240 by 10
		     \count102=\count240	% 102 is the third digit
		     \divide\count102 by \count241
		     \count200=#1\count205=0
		     \count201=\count200
			\multiply\count201 by \count100
		 	\advance\count205 by \count201
		     \count201=\count200
			\divide\count201 by 10
			\multiply\count201 by \count101
			\advance\count205 by \count201
			%
		     \count201=\count200
			\divide\count201 by 100
			\multiply\count201 by \count102
			\advance\count205 by \count201
			%
		     \edef\@result{\number\count205}
}
\def\compute@wfromh{
		% computing : width = height * (bbw / bbh)
		\in@hundreds{\@p@sheight}{\@bbw}{\@bbh}
		%\ps@typeout{ \@p@sheight * \@bbw / \@bbh, = \@result }
		\edef\@p@swidth{\@result}
		%\ps@typeout{w from h: width is \@p@swidth}
}
\def\compute@hfromw{
		% computing : height = width * (bbh / bbw)
	        \in@hundreds{\@p@swidth}{\@bbh}{\@bbw}
		%\ps@typeout{ \@p@swidth * \@bbh / \@bbw = \@result }
		\edef\@p@sheight{\@result}
		%\ps@typeout{h from w : height is \@p@sheight}
}
\def\compute@handw{
		\if@height 
			\if@width
			\else
				\compute@wfromh
			\fi
		\else 
			\if@width
				\compute@hfromw
			\else
				\edef\@p@sheight{\@bbh}
				\edef\@p@swidth{\@bbw}
			\fi
		\fi
}
\def\compute@resv{
		\if@rheight \else \edef\@p@srheight{\@p@sheight} \fi
		\if@rwidth \else \edef\@p@srwidth{\@p@swidth} \fi
		%\ps@typeout{rheight = \@p@srheight, rwidth = \@p@srwidth}
}
%		
% Compute any missing values
\def\compute@sizes{
	\compute@bb
	\if@scalefirst\if@angle
% at this point the bounding box has been adjsuted correctly for
% rotation.  PSFIG does all of its scaling using \@bbh and \@bbw.  If
% a width= or height= was specified along with \psscalefirst, then the
% width=/height= value needs to be adjusted to match the new (rotated)
% bounding box size (specifed in \@bbw and \@bbh).
%    \ps@bbw       width=
%    -------  =  ---------- 
%    \@bbw       new width=
% so `new width=' = (width= * \@bbw) / \ps@bbw; where \ps@bbw is the
% width of the original (unrotated) bounding box.
	\if@width
	   \in@hundreds{\@p@swidth}{\@bbw}{\ps@bbw}
	   \edef\@p@swidth{\@result}
	\fi
	\if@height
	   \in@hundreds{\@p@sheight}{\@bbh}{\ps@bbh}
	   \edef\@p@sheight{\@result}
	\fi
	\fi\fi
	\compute@handw
	\compute@resv}

%
% \psfig
% usage : \psfig{file=, height=, width=, bbllx=, bblly=, bburx=, bbury=,
%			rheight=, rwidth=, clip=}
%
% "clip=" is a switch and takes no value, but the `=' must be present.
\def\psfig#1{\vbox {
	% do a zero width hard space so that a single
	% \psfig in a centering enviornment will behave nicely
	%{\setbox0=\hbox{\ }\ \hskip-\wd0}
	%
	\ps@init@parms
	\parse@ps@parms{#1}
	\compute@sizes
	%
	\ifnum\@p@scost<\@psdraft{
		%
		\special{ps::[begin] 	\@p@swidth \space \@p@sheight \space
				\@p@sbbllx \space \@p@sbblly \space
				\@p@sbburx \space \@p@sbbury \space
				startTexFig \space }
		\if@angle
			\special {ps:: \@p@sangle \space rotate \space} 
		\fi
		\if@clip{
			\if@verbose{
				\ps@typeout{(clip)}
			}\fi
			\special{ps:: doclip \space }
		}\fi
		\if@prologfile
		    \special{ps: plotfile \@prologfileval \space } \fi
		\if@decmpr{
			\if@verbose{
				\ps@typeout{psfig: including \@p@sfile.Z \space }
			}\fi
			\special{ps: plotfile "`zcat \@p@sfile.Z" \space }
		}\else{
			\if@verbose{
				\ps@typeout{psfig: including \@p@sfile \space }
			}\fi
			\special{ps: plotfile \@p@sfile \space }
		}\fi
		\if@postlogfile
		    \special{ps: plotfile \@postlogfileval \space } \fi
		\special{ps::[end] endTexFig \space }
		% Create the vbox to reserve the space for the figure.
		\vbox to \@p@srheight sp{
		% 1/92 TJD Changed from "true sp" to "sp" for magnification.
			\hbox to \@p@srwidth sp{
				\hss
			}
		\vss
		}
	}\else{
		% draft figure, just reserve the space and print the
		% path name.
		\if@draftbox{		
			% Verbose draft: print file name in box
			\hbox{\frame{\vbox to \@p@srheight sp{
			\vss
			\hbox to \@p@srwidth sp{ \hss \@p@sfile \hss }
			\vss
			}}}
		}\else{
			% Non-verbose draft
			\vbox to \@p@srheight sp{
			\vss
			\hbox to \@p@srwidth sp{\hss}
			\vss
			}
		}\fi	



	}\fi
}}
\psfigRestoreAt
\let\@=\LaTeXAtSign





\definemargins{1.0in}{1.0in}{1.0in}{1.0in}{0.3in}{0.3in}

\begin {document} 

\title{Comparing the cost effectiveness of Group Synchronous Review
Method and Individual Asynchronous Review Method using CSRS: \\
Results of Pilot Study} 

\author {Danu Tjahjono \\
         Department of Information and Computer Sciences \\
         University of Hawaii \\
         2565 The Mall\\
         Honolulu, HI 96822\\
         {\tt dat@uhics.ics.hawaii.edu}}

\date {ICS-TR-95-07\\ January, 1995}

\maketitle

\begin{abstract}
  This document describes a pilot experiment that compares the cost
  effectiveness of a group-based review method (EGSM) to that of an
  individual-based review method (EIAM) using CSRS. In this pilot study, no
  significant differences in review effectiveness and review cost were
  found. This document provides complete details on the procedures and
  outcomes from this pilot study, as well as the lessons learned which will
  be applied to an upcoming experimental study.
\end{abstract}

\chapter{Results of Pilot Experiment}

\section{Experimental Procedures}
The pilot experiment was conducted in the Fall 1994.
The subjects involve 24 students taking ICS-313. The subjects are
divided into 8 groups (G1..G8) of 3 participants each. 
Subjects are assigned to the groups based on their final scores in the
assignments. Specifically, we divide the subjects into 3 skill categories:
High (score 15-13), Med (score 12-10), Low (score <10). The three
participants in each group are selected randomly from the three
categories above. Furthermore, all participants assume the role of
Reviewer, one participant (in the High category) assumes the
additional role of Presenter, and another one (in the Medium category)
assumes the additional role of Moderator.

Two set of programs are selected for review. The programs are derived 
from the compiled and final versions of the student
assignments. They are constructed by first, classifying the programs
according to their common 
use of  data structures (e.g., array or pointer, char* or  built-in
String, etc), then choose the programs that contain the most
number of errors, and finally seed the selected programs with
additional errors occurring in the non selected programs. Only 1\%-5\%
of these 
errors are actually seeded. Finally, we recompile the program to
ensure no syntactical errors exist. We also ensure that no same
errors occur in both selected programs.

Each group signs up for 3 sessions corresponding to training session,
session I and session II. All three sessions take 1-2 hours. In
addition, we also have one introductory session for the
entire subjects to demonstrate  the use of CSRS in general. The entire
sessions were completed in three weeks. In session I, all the groups
review artifact I with half of the groups use EGSM, another half
use EIAM. In session II, all the groups review artifact II and
switch the review methods respectively. The actual assignments as to
which group should use which method first is based on the scheduling
of the group members. We encourage the groups to take EGSM first,
since all group members have to be present at the same time. Those
groups, who cannot make it in the first session, take EIAM instead.
For EIAM, individual members can take the review any time.
However, all members of EIAM group have to complete their individual
reviews before signing up for the group review (EGSM).
The groups assignments are shown in Table \ref{pilot-groups}.

\begin{table}[h]
  \begin{center}
    \begin{tabular} {|l|l|l|}
      \hline
      & {\bf Session I} & {\bf Session II}\\
      \hline
      & & \\
      {\bf EGSM} & G2,G1,G8,G5 & G6,G3,G4,G7 \\
      & & \\
      \hline
      & & \\
      {\bf EIAM} & G7,G6,G3,G4 & G8,G1,G5,G2 \\
      & &  \\
      \hline
     \end{tabular}
  \end{center}
  \caption{Pilot Group Assignments}
  \label{pilot-groups}
\end{table}

After completed both sessions, the groups spend 15 minutes filling out
questionnaires. 

%%The completed documentation guidelines for review participants are
%%shown in the Appendix. 


\section{Analysis of Pilot Experiment}
As mentioned earlier, one of the objective of the experiment is
to evaluate review effectiveness and review cost of synchronous review
method (EGSM) compared to asynchronous review method (EIAM).

The experimental design is based on the repeated 2x2 Latin Square. This
design allows one to evaluate the main factor (differences in review
methods) with two level of treatments (synchronous and asynchronous
methods) while blocking two major sources of variations from
review subjects and review artifacts. Each review group runs two
sessions (S1 and S2) with EGSM and EIAM review method, and source
artifact 1 (A1) and artifact 2 (A2) respectively (see Table
\ref{latin-square}) 
The result the experiment is shown in Table
\ref{total-faults}. There are 8 groups 
from ICS-313 class participated in the experiment ($G_1..G_8$). 
Each group has 3 subjects, except for Group1 and Group4 who only have
2 subjects (1 subject dropped).
Numbers in each cell represent the total number of faults in the
corresponding artifacts founds by the group (EGSM) or by the
individuals collectively (EIAM). 


\begin{table}[h]
  \begin{center}
    \begin{tabular} {|l|l|l|}
      \hline
      & {\bf A1} & {\bf A2}\\
      \hline
      & & \\
      {\bf $G_I$} & EGSM & EIAM \\
      & & \\
      \hline
      & & \\
      {\bf $G_{II}$} & EIAM & EGSM \\
      & &  \\
      \hline
     \end{tabular}
  \end{center}
  \caption{Basic 2x2 Latin Square Design}
  \label{latin-square}
\end{table}



\begin{table}[h]
  \begin{center}
    \begin{tabular} {|c|c|c|c|}
      \hline
      & & {\bf Session1} & {\bf Session2} \\
      & &  (A1: driver) & (A2: yard) \\
      \hline
      & & EGSM & EIAM \\
      \cline{2-4}  
      & G1 & 4 (2:27:46) & 10 (3:43:09)\\
      \cline{2-4}
 $G_I$ & G2 & 3 (2:01:39) & 3 (3:15:08) \\
      \cline{2-4}
      & G5 & 1 (3:48:39) & 5 (4:16:58) \\
      \cline{2-4}
      & G8 & 5 (3:36:24) & 6 (3:52:41) \\
      \hline
      & & EIAM & EGSM \\
      \cline{2-4}  
      & G3 & 6 (2:00:20) & 7 (3:28:54) \\
      \cline{2-4}
 $G_{II}$ & G4 & 4 (1:13:31) & 4 (1:38:10) \\
      \cline{2-4}
      & G6 & 6 (2:24:58) & 11 (4:24:39) \\
      \cline{2-4}
      & G7 & 6 (2:13:33) & 2 (3:05:57) \\
      \hline
     \end{tabular}
  \end{center}
  \caption{Total number of Faults and Total Review-Time}
  \label{total-faults}
\end{table}

In the following sections, we will present
the results of review effectiveness and review cost analysis.
The analysis will
follow the formulation of Latin Square Design \cite{Montgomery84}.

\subsection{Review effectiveness}

Review effectiveness is measured by the number of faults detected by
the groups. 

The statistical model for the repeated Latin Square design in general
is: 

$y_{ijkl} = \mu + \alpha_i + \tau_j + \beta_k + \rho_l + \epsilon_{ijkl}$\
for i,j,k = 1, 2, 3 (p=3); l = 1, 2, 3, 4, 5 (n=5). 

$y_{ijkl}$ is the observation in row {\sl i}, column {\sl k}, and
replicate {\sl l} for the {\sl j}th treatment.
$\mu$ is the overall mean, $\alpha_i$ is the {\sl i}th
row effect, $\tau_j$ is the {\sl j}th
treatment effect, $\beta_k$ is the
{\sl k}th column effect, $\rho_l$ is the {\sl l}th
replicate effect (i.e., variations in the row), and
$\epsilon_{ijkl}$ is the random error. 

In our case, the dependent variable measures the total number of
faults detected by the groups. The treatment effect corresponds to the
differences in review methods. The row effect corresponds to the
group variations, and the column effect corresponds to the
artifact variations.
Table \ref{effectiveness} shows the data with respect to the equation
above. 

\begin{table}[h]
\begin{center}
\begin{tabular}{|c|c|c|c|}
\hline
          & A1 (k=1)     & A2 (k=2) & $y_{i..l}$ \\
\hline                  
          & 4 (l=1,j=1)  & 10 (l=1,j=2) & 14 \\
$G_I$ (i=1) & 3 (l=2,j=1)  & 3 (l=2,j=2) & 6 \\
          & 1 (l=3,j=1)  & 5 (l=3,j=2) & 6\\
          & 5 (l=4,j=1)  & 6 (l=4,j=2) & 11\\
\hline                  
          & 6 (l=1,j=2)  & 7 (l=1,j=1) & 13\\
$G_{II}$(i=2) & 4 (l=2,j=2)  & 4 (l=2,j=1) & 8\\
           & 6 (l=3,j=2)  & 11 (l=3,j=1) & 17\\
          & 6 (l=4,j=2)  & 2 (l=4,j=1) & 8\\
\hline
$y_{..k.}$ & 35          & 48          & 83 (=$y_{....}$) \\
\hline
\end{tabular}
\end{center}
\caption{Review Effectiveness Data}
\label{effectiveness}
\end{table}


The main hypothesis we are testing is the null hypothesis $\tau_j=0$
for all j, that is, there is no difference in the effectiveness of the
review methods.
With this model, the total experimental variance can be partitioned into:

\begin{eqnarray*}
SS_{Total} & = & SS_{Treatments}+SS_{Rows}+SS_{Columns}+SS_{Replicates}+SS_{Error}
\end{eqnarray*}

The individual variance can be calculated as follow:

\begin{eqnarray*}
SS_{Treatments} & = & \sum_{j=1}^p \frac{y_{.j..}^2}{np}- \frac{y_{....}^2}{N}\\
                 & = & \frac{(13 + 24)^2 + (24 + 22)^2}{8} - \frac{83^2}{16}\\
                 & = & 435.63 - 430.56 \\
                 & = & 5.07\\
                 &   &\\
SS_{Rows} & = & \sum_{l=1}^n \sum_{i=1}^p \frac{y_{i..l}^2}{p}-\sum_{l=1}^n \frac{y_{...l}^2}{p^2}\\
          & = & \frac{14^2+6^2..+17^2+8^2}{2} - \frac{27^2+14^2+23^2+19^2}{4}\\
          & = & 487.50 - 453.75 \\
          & = & 33.75 \\
          &   &\\
SS_{Columns} & = & \sum_{k=1}^p \frac{y_{..k.}^2}{np} -  \frac{y_{....}^2}{N}\\
             & = & \frac{35^2+48^2}{8} - \frac{83^2}{16}\\
             & = & 441.12 - 430.56 \\
             & = & 10.56\\
             &   &\\
SS_{Replicates} & = & \sum_{l=1}^n \frac{y_{...l}^2}{p^2} - \frac{y_{....}^2}{N}\\
                & = & 453.75 - 430.56 \\
                & = & 23.19\\
                &   &\\
SS_{Total} & = & \sum_i \sum_j \sum_k \sum_l y_{ijkl}^2 - \frac{y_{....}^2}{N}\\
           & = & 4^2+10^2+...+6^2+2^2 - 430.56 \\
           & = & 535 - 430.56 \\
           & = & 105.56\\
           &   &\\
SS_{Error} & = & SS_{Total}-SS_{Treatments}-SS_{Rows}-SS{Columns}-SS{Replicates}\\
           & = & 105.56 - 5.07 - 33.75 - 10.56 - 23.19 \\
           & = & 32.99
\end{eqnarray*}



\begin{table}[tb]
\begin{center}
\begin{tabular}{|c|c|c|c|c|}
\hline
Source of Variation & Sum of Squares & Degrees of Freedom & Mean Square & $F_0$\\
\hline
Methods        &  5.07   & 1    & 5.07 & 1.1\\
Artifacts (col) &  10.56   & 1   & 10.56  &  \\
Groups (row)    &  33.75   & 4    & 8.4375 &  \\
Replications    &  23.19      & 3    & 7.73     &  \\
Error             & 32.99     & 7   & 4.7128 &  \\
Total             & 105.56   & 15   & 7.037      &  \\
\hline
\end{tabular}
\end{center}
\caption{Analysis of Variance of Review Effectiveness}
\label{analysis-variance1}
\end{table}


The analysis of variance is summarized in Table \ref{analysis-variance1}.
Since $F_0$ (distributed as $F_{1,7}$) is smaller than
$F_{0.05,1,7}$ = 5.59 ($\alpha$ = 0.05),
we may conclude that there is no 
significant difference between the two review methods above with
significance level 0.05. Even with significant level 0.10 
($F_{0.10,1,7}$ = 3.59),  or 0.25 ($F_{0.25,1,7}$ = 1.69), we cannot
conclude there are differences in review effectiveness between the two
methods. 

\subsection{Review Cost}
Review Cost is measured by the amount of time (i.e., review time)
required to detect a fault. Review time for a group is calculated
by taking the 
sum of review time spent by individual participants. For EGSM, this
is simply calculated by taking the presenter's review time multiplied
by the number of participants. 

The analysis is basically the same as in the previous section, except
that the dependent variable is the review cost. Table
\ref{review-cost} shows the resulting data, and Table
\ref{analysis-variance2} show the corresponding analysis of variance.


\begin{table}[h]
\begin{center}
\begin{tabular}{|c|c|c|c|}
\hline
          & A1 (k=1)     & A2 (k=2) & $y_{i..l}$ \\
\hline                  
          & 36.9 (l=1,j=1)  & 22.3 (l=1,j=2) & 59.2 \\
$G_I$ (i=1) & 40.5 (l=2,j=1)  & 65.0 (l=2,j=2) & 105.5 \\
          & 228.6 (l=3,j=1)  & 51.4 (l=3,j=2) & 280\\
          & 43.3 (l=4,j=1)  & 38.8 (l=4,j=2) & 82.1\\
\hline                  
          & 10.1 (l=1,j=2)  & 29.8 (l=1,j=1) & 39.9\\
$G_{II}$(i=2) & 18.4 (l=2,j=2)  & 24.5 (l=2,j=1) & 42.9\\
           & 24.2 (l=3,j=2)  & 24.0 (l=3,j=1) & 48.2\\
          & 22.2 (l=4,j=2)  & 93.0 (l=4,j=1) & 115.2\\
\hline
$y_{..k.}$ & 424.2          & 348.8          & 773 (=$y_{....}$) \\
\hline
\end{tabular}
\end{center}
\caption{Review Cost (Minutes/Fault)}
\label{review-cost}
\end{table}


\begin{table}[tb]
\begin{center}
\begin{tabular}{|c|c|c|c|c|}
\hline
Source of Variation & Sum of Squares & Degrees of Freedom & Mean Square & $F_0$\\
\hline
Methods        &  4495.703   & 1    & 4495.703 & 2.25\\
Artifacts (col) &  355.323   & 1   & 355.323  &  \\
Groups (row)    & 14779.525  & 4    & 3694.881 &  \\
Replications    & 7275.913    & 3    & 2425.304     &  \\
Error             & 13984.714  & 7   & 1997.8162 &  \\
Total             & 40891.178   & 15   & 2726.078      &  \\
\hline
\end{tabular}
\end{center}
\caption{Analysis of Variance of Review Cost}
\label{analysis-variance2}
\end{table}

The result of Table \ref{analysis-variance2} also shows that there are
no significant differences in review cost between the two methods at
the significant level of 0.05 and 0.10. However, it shows a
significant difference at the level of 0.25. From Table
\ref{review-cost}. it can be seen that EGSM (Group Method) is more
costly than EIAM (Individual Method).

\section{Lessons Learned}
In this section, we present some of the lessons we learned, and the
follow-up actions we will take for the main study.

\subsection{Results of data analysis}
The results of the data analysis clearly show that there are no
significant differences between group and individual review methods. 
However, we also observe that the lack of differences may be
attributed to the following factors:
\begin{itemize}
\item {\bf Insufficient comprehension of the code.}
One of the control variables in the experiment is the degree of
comprehension of the code. Unfortunately, post questionnaires show
that this variable varies widely among participants (1 to 5 on the
scale of 1-5).
One reason why we have this problem is that the experiment was
conducted two months after the participants completed their
assignments. Some participants commented that they have completely
forgotten their C++ programs. 

The main study will therefore require the experiment is conducted at
most two weeks after the students completed their assignments. 

\item {\bf Insufficient number of seeded errors.}
There are two artifacts used for the review. One has 25 known errors, the
other one has only 7 known errors.
In the first artifact, the groups only find up to 50\% of the errors, in
the second artifact some groups find almost all the errors (90\%). We
believe in the  latter case that the number of seeded errors are
insufficient. In the main study, we will seed sufficient number of
errors in both artifacts (> 20). 


\item {\bf Too many incorrect issues.}
There are too many incorrect issues raised by the participants. In most
cases, the number of incorrect issues are greater than the correct
ones. This can definitely affect review cost as more time is
needed to raise, discuss and record these issues.
One reason why this problem arises may be due to the fact that we 
encourage the participants to raise as many issues as possible 
even though they may be incorrect. We will remedy this situation by
rewarding the participants on their review effectiveness as well as
their review cost.

\end{itemize}

\subsection{Review process}
\begin{itemize}
\item {\bf The role of Moderator.}
In general, we do not know whether the moderator in this experiment
plays a significant role (i.e., whether review effectiveness is greatly 
affected by the skill of the moderator). No quantitative study on the
role of moderator has ever been conducted.

The main task of the moderator in our experiment (i.e., EGSM) is to
lead the group discussion and to prevent unproductive
discussion. In other words, the moderator has the
power to cut lengthy or improper discussion. However, we rarely
observe the moderator takes this action.

%We do not know whether, the lack of moderator action is due to the
%the smoothness of the discussion (the moderator does not think it is
%necessary to moderate/cut the discussion), the moderator is
%intimidated to do the job, or the moderator simply
%forgets to do his or her job. It may also due to the fact that we allow
%individual participants to vote 
%differently from other group members (i.e., consensus does not need to
%be reached).

The results of post questionnaires indicate that the degree of
disagreement among 
group members is slightly low (2.6 on the scale of 1-5). Domination by
some group members is low to moderate (2.8). Hostility among group
members is low (1.9 on the scale of 1-5). From this data, it seems
that there is only a slight need for the moderator to intervene.  
Our observation seems to indicate that the lack of moderator action
may be caused by the over-active role of the presenter as discussed
next. In fact, this observation is in agreement with the experiment
conducted by \cite{Brothers90}, in which they observe that the
moderator tends to loose 
influence to the Reader and Scribe. Both reader and scribe roles are
assumed by the Presenter in our experiment.
Despite lack of action, the participants rate their satisfaction
toward the moderator as moderate-high (3.7 on the scale of 1-5). 

In the main study, we will further control the role of moderator.
The principal investigator will play the role of the moderator.
This external moderator will still assume the meeting leader.
However, to prevent any bias in the experimental outcomes, he will not
involve in any decision making or group discussion.
Instead, this role will be restricted to only
administrative duties. That is, to ensure individual participants
perform their assigned roles properly. The moderator may also remind
the group when the discussions have gone too long without any
decision. This will be carried out based on some fixed amount of time
(to be determined later). When this time-out occurs, the
participants may create the issue and vote individually without having
to reach any consensus.


\item {\bf The role of Presenter.}
Presenter seems to play a significant role in this experiment
(EGSM). He or she 
controls the pace of review (through paraphrasing), and thus may
significantly affect the group performance. 
As mentioned previously, although we have a separate moderator, we
observe that the meeting leader is implicitly assumed by the
presenter. The presenter ensures that all participants have voiced
their concerns (i.e., have no more questions/comments regarding the
respective code 
under review), have voted properly on each issue, and in many times
decide whether to record the issues or not.

When asked whether the presenter was too fast in presenting the
materials, most participants did not feel so (1.8 on the scale of
1-5). In other words, the participants seem to have sufficient time to
digest the code as they are paraphrased. 
The data also shows that the groups rate their satisfaction toward the 
presenters as moderate-high (3.8).  

At this time, we do not know whether it is proper that the
presenter plays an over-active role. An attempt to ensure that the
presenter strictly performs the paraphrasing task may slow  down the
review process. At one instance during the training session, we ask the
presenter to strictly follow the moderator for initiating the
discussion after paraphrasing. We observe some discontinuities in the
discussion process. Normally, right after paraphrasing, the presenter
raises follow-up questions or comments naturally. When the moderator
takes over this task, the presenter has to wait until the moderator
allow him to speak. Thus, there are some unnecessary idle periods.
In our experiment, any participants can interrupt and raise issues
any time without waiting for the moderator to give the floor. We
believe that this process makes the discussion more active and
productive. 


\item {\bf Paraphrasing.}
We observe quite number of variations in paraphrasing skills among the 
presenters. Many presenters present the source code by letting the group
knows what the code does. Some presenters, however, talk very
little. Basically let individual members spot the issues on their own,
then discuss the  
issues once spotted. Other presenters simply read the code literally
as written (without giving much thought about what the code
does). 
The post questionnaires data indicate that presenters do not 
perform paraphrasing strictly at all time (3.3 on the scale of 1-5). 
The usefulness of paraphrasing is perceived as rather high
(3.7). Although whether it is actually inspired the participants to
find errors is perceived moderately (3.3).
One subject commented that paraphrasing was a waste of time. Some
voiced their difficulty in performing paraphrasing.

In the main study, we will ensure that the presenters get adequate
training in paraphrasing. 

\item {\bf Recording issues.}
The issue of who should record the errors in the group meeting (EGSM)
is still debatable. One may argue that the participant who finds the
errors should be required to record them since he or she knows the
exact nature of the errors. However, the rest of the group members may
have difficulty understanding the errors if the errors are not
sufficiently elaborated in the verbal discussion. Having a separate
recorder will ensure that the errors are discussed and understood
properly by all group members before being recorded. 

In our experiment (EGSM), the presenter records the issues agreed 
upon by the group. Initially, the EGSM system provides the operation
to record issues by any members of the group.
However, individual members of the
group soon do their own review (i.e., raise issues individually).
In other words, they no longer follow the presenter paraphrasing the
materials. 
This feature was then subsequently deleted from the system during
the training period. Only the presenter can record the issues raised
by the group. 
When the presenter himself/herself raising the 
issue, it is possible that the issue is not thoroughly discussed. The
data, however, show that most participants tend to agree with the
presenter (i.e., have high confidence-level on the issues first
suggested by the presenter). Therefore, we will keep this feature in
the main study.


\item {\bf Consensus.}
The degree of consensus within EGSM groups are generally high. Most
participants tend to agree with the issues raised by any members of
the group.  In EGSM, the consensus is expressed in terms of
Confidence-level: High, Medium, Low or None (for disagree). Only few
issues have 'None' confidence-level. 

Surprisingly, the consensus is also quite high for incorrect
issues. All participants tend to see the same problems once
they are identified.

As discussed previously, we will no longer encourage the participants
to raise both correct and incorrect issues.

\end{itemize}

\subsection{CSRS}
In general participants rate CSRS favorably on all aspects of the
system, which include ease of use and usefulness. The questionnaire
data also indicate that many participants believe the system can
make their review more effective and productive. 
We also observe that CSRS can effectively ensure that the
participants have performed their tasks properly. For example, 
in many occasion, the system informs
the presenter that some participants have not voted yet
although the presenter has repeatedly asked the participants to do so 
verbally. 

Two main problems identified by the participants regarding CSRS
include system instability (especially the EGSM system)
and limited viewing of source code (i.e., many participants
would like to have multiple windows displayed more than one source
nodes).  We will fix this problem in the main study.
Other comments of CSRS include mainly the misconception of what a
review system can do; many participants  
expect debugging facilities (including complete C++ documentation) be
included as part of the system.


\subsection{Group Method (EGSM) or Individual Method (EIAM)}
The results of post questionnaires indicate that most participants
prefer EGSM to EIAM.
The most commonly cited reasons include: 
one can learn from each others, and 
one can raise issues with more confidence. One participant
characterizes his issues raised with EGSM as {\it real issues} 
as supposed to {\it guessed issues} raised with EIAM.

However as we have seen, the data do not show that EGSM groups perform
better than EIAM groups. 

\newpage
\appendix
%\documentstyle [nftimes,11pt,/group/csdl/tex/definemargins,/group/csdl/tex/lmacros]{article}

%\begin{document}
%\begin{center}
%{\Large\bf Formal Technical Reviews Guidelines \\
%  (EGSM and EIAM)}\\
%\end{center}

\chapter {Formal Technical Reviews Guidelines (EGSM and EIAM)}


\section*{Objective}
The objective of the review is to find as many errors (issues) as
possible in the source code through interaction with group members
(EGSM) or by working alone (EIAM).

\section*{Review Processes}
\subsection*{Individual method (EIAM)}
\begin {itemize}
\item All participants assume the role of {\it Reviewer}.
\item After successful connection to the database, a set of source
  nodes to be reviewed will be displayed on the summary buffer.
\item You may review these nodes in any order (based on your
personal preference, judgement, and/or technique).
\item When noticing an error or anomaly in the code, 
  create an issue node. All fields in the issue node must be completed.
\item After reviewing each source node, declare the status of the node
to ``reviewed''. 
\end{itemize}

\subsection*{Group method (EGSM)}
\noindent {\bf Presenter:}
\begin{itemize}
\item The presenter also performs the task of a reviewer (searching
  for errors).
\item After successful connection, wait for the moderator's
  instruction to start the review.
\item Discuss with the group, in what order the source nodes will be
  reviewed.
\item Select the node to be review. Read loudly (paraphrasing) the
  code line by line (speak up). For example, by letting the group know
  what this particular line or block of code does.
\item Don't speak too fast. Let the group digest it before going to
  the next line or block of code.
\item Stop immediately, when noticing any problems or questions.
  Throw the questions or concerns to the group, and let the group decide
  whether they are indeed a legitimate issue. If so, create an issue
  node.
\item Also stop immediately, when someone else interrupt you by
  raising a question, concern or comment.
\item After creating an issue node, fill out the Subject, Lines, and
  Description fields. Write a brief and concise description of the
  issue. Do not write or suggest any solution. However, you may state
  the issue as, for example, ``This should have been ...''.
\item Then save the node, so that all participants can see what you
  just type.
\item Now fill out the rest of the fields by casting your vote
  (Criticality,Suggested-by, and Confidence-level).

\item Wait until all participants have casted their vote, then close
  the node.
\item Continue reading/paraphrasing the remaining of the source node.
\item When no more problem is detected, declare or set the status of
  the node to ``reviewed''.
\item Continue the above processes until all source nodes are
  reviewed.
\end{itemize}

\noindent {\bf Moderator:}
\begin{itemize}
\item The moderator also performs the task of a reviewer (searching for errors).
\item After successful connection, ensure that everyone are ready
  (i.e., have been connected successfully). Then give the instruction to
  start the review.
\item At any time, follow the presenter closely. Interrupt the
  presenter when noticing any problems or questions. Discuss the
  problems/questions with the group.
\item Moderate group discussion. Cut lengthy discussion if necessary.
\item Ensure that the participants only discuss the issue, and NOT the
  solution to the issue. When the latter occurs, cut the discussion
  immediately.
\item Also cut discussion that does not lead to a legitimate issue. For
  example, issues that seek for alternative implementation for the given
  specification, etc.
\item Do not spend too much time on a particular source node. Go back
to this node later if necessary.
\item Slow down or speed up the presenter as necessary.
\item Change the presenter if necessary.
\end{itemize}

\noindent {\bf Reviewer:}
\begin{itemize}
\item After successful connection, wait for the moderator's
instruction to start the review.
\item At any time, follow the presenter closely. Interrupt the presenter when
noticing any problems or questions. Discuss the
problems or questions with the group.
\end{itemize}


\subsection*{Other Guidelines}
\begin {itemize}
\item Your task is to identify defects or errors in the source code,
NOT to decide what to do about them.

\item Think critically. Assume that the code you are reviewing is wrong.

\item Do not stop reviewing the code once an error is detected.
You may assume that the error has been corrected when
reviewing the remaining code. For example, the array declaration is
missing, and yet the code is accessing the array. 
You should raise one issue regarding this missing array declaration,
and continue reviewing the remaining code by assuming the array
declaration is correct.

\item Do not raise issues regarding the missing or incorrect
  specification. Assume the given code is already implemented with the 
  correct specification. For example, when the specification stated
  that each instance of Driver class have a name, then do not raise issues
  asking for ``Last-Name'' or ``First-Name''.

\item Similarly, do not raise issues that require the domain
knowledge not mentioned in the specification. For
example, each instance of Driver should include driver's address and
phone number.

\item Do not raise issues regarding alternative implementation of the
  given specification/code. The goal of this review is to simply
  identifying errors or issues in the given code with the assumption that
  the specification is already correct. 
 For example, do not raise issues that state
  this code should   use pointer instead of array when the
  specification explicitly states it uses array; or internal
  representation of BigInt  should use pure character string instead of BCD
  when the specification explicitly states it used BCD representation, etc. 

\item Do not discuss or record the solution to the issue. However, you may
  state the issue as ``This should have been ....''.

\item Use separate issue nodes when recording the same issue but
occurring at different places in the code.

\item When filling out a field with selectable items, choose the one
that closely represent your opinion.

\item If you do not understand what the code does, ask the
producer. Danu or Cam will be available during the review session to
answer questions regarding the specification.

\end{itemize}

\section* {Questions ?}
Send any questions or comments to: {\bf dat@uhunix.uhcc.hawaii.edu}
 

%%\end{document}





%%% \documentstyle [nftimes,11pt,/group/csdl/tex/definemargins,/group/csdl/tex/lmacros]{article}
%%% 
%%% \begin{document}
%%% \begin{center}
%%% {\Large\bf CSRS Commands Reference}
%%% \end{center}

\chapter{CSRS Commands Reference}
\section* {Starting CSRS}
\begin{itemize}
\item To run individual review method: {\it \~csdl/bin/run-eiam}
\item To run individual review method for testing: {\it \~csdl/bin/run-eiam-test}
\item To run group review method: {\it \~csdl/bin/run-egsm}
\item To run group review method for testing: {\it \~csdl/bin/run-egsm-test}
\end{itemize}

\section* {Leaving CSRS}
\begin{itemize}
\item To exit CSRS:  {\it Menubar::CSRS::Quit}, or {\it Menubar::Session::Quit},
 or {\it C-x C-c}
\end{itemize}

\section* {Mouse Selection}
\begin{itemize}
\item To move point/cursor:  {\it Mouse::Button-1}
\item To retrieve highlighted item: {\it Mouse::Button-2}
\item To popup menu : {\it Mouse::Button-3}
\item To scroll up (half screen) : move mouse to scroll-bar and
click {\it Mouse::Button-1}
\item To scroll down (half screen) : move mouse to scroll-bar and
click {\it Mouse::Button-3}
\item To scroll smoothly : move mouse to scroll-bar and move up or
down  {\it Mouse::Button-2}
\end{itemize}

\section* {Nodes}
\begin{itemize}
\item To save current node: {\it Menubar::Button::Save}
\item To close current node: {\it Menubar::Button::Close}
\item To edit current node: {\it Menubar::Button::Lock}
\item To delete current node: {\it Menubar::CSRS::Delete node}
\end{itemize}


\section* {Fields}
\begin{itemize}
\item To fill out selectable field:
      \begin{enumerate}
        \item Move the mouse over the field label
        \item {\it Popup::choose the desired item}
      \end{enumerate}
\item To fill out regular field, type in:
      \begin{enumerate}
        \item One line text in {\bf Subject} field, or
        \item One or multiple lines text in {\bf Description} field.        
      \end{enumerate}
\end{itemize}

\section* {Summary Buffer}
\begin{itemize}
\item To select next item: {\it Menubar::Button::Next}
\item To select previous item: {\it Menubar::Button::Prev}
\item To refresh the content of buffer: {\it Menubar::Button::Refresh}
\item To close summary buffer: {\it Menubar::Button::Quit}
\end{itemize}

\section* {Source nodes}
\begin{itemize}
\item To set status to reviewed: {\it Popup::Set status to reviewed}
\item To view previous node: {\it Popup::View previous node}
\item To view class declaration: {\it Popup::View Class declaration}
\item To list unreviewed source nodes: {\it [Menu]::List unreviewed source nodes} 
\item To list all source nodes {\it [Menu]::List all source nodes}
\item To list all issues: {\it [Menu]::List all issues}
\end{itemize}

where {\it [Menu]} is : {\it Menubar::Reviewer} for EIAM, or
{\it Menubar::Public-Review} for EGSM.


\section* {Issues}
\begin{itemize}
\item To create an issue: {\it Popup:Create an issue}
\item To select lines/regions: 
   \begin{enumerate}
     \item Drag {\it Mouse::Button-1} over the region
     \item {\it Popup::Select this region}
     \item Move mouse over Lines field 
     \item {\it Popup::Select current highlighted lines}.
   \end{enumerate}

\end{itemize}

\section* {Questions ?}
Send any questions or comments to: dat@uhunix.uhcc.hawaii.edu
  

%%\end{document}



%%% \documentstyle [12pt,/group/csdl/tex/definemargins,
%%% 		     /group/csdl/tex/lmacros]{article}
%%% 
%%% \begin{document}
%%% 
%%% \begin{center}
%%%   {\large\bf 
%%% %%% CSRS Experiment Fall, 1994\\
%%%   C++ Review Project -- Fall 1994 \\
%%%   Questionnaire For Groups (EGSM)} 
%%% \end{center}

\chapter {Questionnaire For Groups (EGSM)}

\noindent {\it For each question, circle 1, 2, 3, 4 or 5 only} 

\begin{description}
%%\item [Name:]
\item [Group:]
\item [Review Role:]
\item [Source Code:]
\end{description}

\noindent
\begin {enumerate}
%%\subsection {Comprehension}
\item My understanding of the source code before the review was:
\\
Very low \hfill 1 \dotfill  2 \dotfill 3 \dotfill 4 \dotfill 5 \hfill Very high\\

\item My understanding of the source code after the review was:
\\
Very low \hfill 1 \dotfill  2 \dotfill 3 \dotfill 4 \dotfill 5 \hfill Very high\\

\item My understanding of C++ programming language was
improved after this review.
\\
Not at all true \hfill 1 \dotfill  2 \dotfill 3 \dotfill 4 \dotfill 5 \hfill Very true\\


%%\subsection* {Group strategy}

\item In general, our group found it easy to understand the logic of the
code.
\\
Not at all true \hfill 1 \dotfill  2 \dotfill 3 \dotfill 4 \dotfill 5 \hfill Very true\\

\item During the review, I follow or pay attention to the presenter.
\\
Not at all \hfill 1 \dotfill  2 \dotfill 3 \dotfill 4 \dotfill 5
\hfill At all time\\


%\item In general, my review strategy during group meeting was:
%\begin{enumerate}
%\item Simply follow the presenter closely and interrupt him or her to raise
%issues. 
%\item Other (please explain):
%\\
%\end{enumerate}


%%\subsection* {Satisfaction & Confidence}
\item For this review, I would have preferred working alone rather
than working in a group.
\\
Not at all true \hfill 1 \dotfill  2 \dotfill 3 \dotfill 4 \dotfill 5 \hfill Very true\\

\item For this review, I felt more confidence about the issues I raised when
working alone that working in a group.
\\
Not at all true \hfill 1 \dotfill  2 \dotfill 3 \dotfill 4 \dotfill 5 \hfill Very true\\

\item  In general, I felt working in this group increases my ability
in finding errors. 
\\
Not at all true \hfill 1 \dotfill  2 \dotfill 3 \dotfill 4 \dotfill 5 \hfill Very true\\


%%\subsection*{Misc}
\item The training session for group review was sufficient.
\\
Not at all true \hfill 1 \dotfill  2 \dotfill 3 \dotfill 4 \dotfill 5 \hfill Very true\\

\item There was sufficient time to work on this review.
\\
Not at all true \hfill 1 \dotfill  2 \dotfill 3 \dotfill 4 \dotfill 5 \hfill Very true\\

\item Our group was motivated to do this review project.
\\
Not at all true \hfill 1 \dotfill  2 \dotfill 3 \dotfill 4 \dotfill 5 \hfill Very true\\

\item Our group worked seriously on this review.
\\
Not at all true \hfill 1 \dotfill  2 \dotfill 3 \dotfill 4 \dotfill 5 \hfill Very true\\

\item My overall confidence in the quality of our review was
\\
Very low \hfill 1 \dotfill  2 \dotfill 3 \dotfill 4 \dotfill 5 \hfill Very high\\


%%\subsection* {Group dynamics}
\item My overall satisfaction with the discussion among my group
 members was:
\\
Very low \hfill 1 \dotfill  2 \dotfill 3 \dotfill 4 \dotfill 5 \hfill Very high\\


\item This group was too small (in number of members) for best results
in the task it was trying to do.
\\
Not at all true \hfill 1 \dotfill  2 \dotfill 3 \dotfill 4 \dotfill 5 \hfill Very true\\


\item The ideal number of people to do review like this is (1 to 10): 
\\

\item I felt comfortable working in the group on this task.
\\
Not at all true \hfill 1 \dotfill  2 \dotfill 3 \dotfill 4 \dotfill 5 \hfill Very true\\


\item There was much disagreement among the members of the group on
this task.
\\
Not at all true \hfill 1 \dotfill  2 \dotfill 3 \dotfill 4 \dotfill 5 \hfill Very true\\


\item Some people in the group dominated the discussion.
\\
Not at all true \hfill 1 \dotfill  2 \dotfill 3 \dotfill 4 \dotfill 5 \hfill Very true\\


\item My opinion was given adequate consideration by the other group members.
\\
Not at all true \hfill 1 \dotfill  2 \dotfill 3 \dotfill 4 \dotfill 5 \hfill Very true\\


\item I felt inhibited from expressing my opinion during the group discussion. 
\\
Not at all true \hfill 1 \dotfill  2 \dotfill 3 \dotfill 4 \dotfill 5 \hfill Very true\\


\item I felt I participated a great deal in the group discussion.
\\
Not at all true \hfill 1 \dotfill  2 \dotfill 3 \dotfill 4 \dotfill 5 \hfill Very true\\

\item I felt I contributed a great deal to the discovery of issues.
\\
Not at all true \hfill 1 \dotfill  2 \dotfill 3 \dotfill 4 \dotfill 5 \hfill Very true\\

\item I felt the group made a great deal of influence on my decision
about what would be an issue, the criticality of an issue, and/or confidence-level.
\\
Not at all true \hfill 1 \dotfill  2 \dotfill 3 \dotfill 4 \dotfill 5 \hfill Very true\\

\item Overall, I was satified with the group interaction
\\
Not at all true \hfill 1 \dotfill  2 \dotfill 3 \dotfill 4 \dotfill 5 \hfill Very true\\


\item I would enjoy working with members of this group again.
\\
Not at all true \hfill 1 \dotfill  2 \dotfill 3 \dotfill 4 \dotfill 5 \hfill Very true\\


\item I felt that I could express disagreement freely.
\\
Not at all true \hfill 1 \dotfill  2 \dotfill 3 \dotfill 4 \dotfill 5 \hfill Very true\\


\item There was some open hostility in the group.
\\
Not at all true \hfill 1 \dotfill  2 \dotfill 3 \dotfill 4 \dotfill 5 \hfill Very true\\


\item Before this experiment, I knew all members of the group fairly well. 
\\
Not at all true \hfill 1 \dotfill  2 \dotfill 3 \dotfill 4 \dotfill 5 \hfill Very true\\


\item I felt at ease discussing things with the group.
\\
Not at all true \hfill 1 \dotfill  2 \dotfill 3 \dotfill 4 \dotfill 5 \hfill Very true\\


\item Our group took too long to come to an agreement.
\\
Not at all true \hfill 1 \dotfill  2 \dotfill 3 \dotfill 4 \dotfill 5 \hfill Very true\\

\item I felt our group wasted too much time on unproductive
discussion.
\\
Not at all true \hfill 1 \dotfill  2 \dotfill 3 \dotfill 4 \dotfill 5 \hfill Very true\\


\item Some individual(s) in the group wanted to change things after
the group had come to a decision.
\\
Not at all true \hfill 1 \dotfill  2 \dotfill 3 \dotfill 4 \dotfill 5 \hfill Very true\\

\item I felt the presenter was too fast in presenting the code.
\\
Not at all true \hfill 1 \dotfill  2 \dotfill 3 \dotfill 4 \dotfill 5 \hfill Very true\\

\item Our group used paraphrasing technique (talking aloud by the
presenter) all the time.
\\
Not at all true \hfill 1 \dotfill  2 \dotfill 3 \dotfill 4 \dotfill 5 \hfill Very true\\

\item The paraphrasing technique inspired me in finding errors.
\\
Not at all true \hfill 1 \dotfill  2 \dotfill 3 \dotfill 4 \dotfill 5 \hfill Very true\\


\item I felt the paraphrasing technique in general was useful.
\\
Not at all true \hfill 1 \dotfill  2 \dotfill 3 \dotfill 4 \dotfill 5 \hfill Very true\\


\item Overall, I felt the presenter did a good job in presenting
the code.
\\
Not at all true \hfill 1 \dotfill  2 \dotfill 3 \dotfill 4 \dotfill 5 \hfill Very true\\


\item My group needed a stronger leader (i.e., Moderator) to keep it on track.
\\
Not at all true \hfill 1 \dotfill  2 \dotfill 3 \dotfill 4 \dotfill 5 \hfill Very true\\

\item Overall, I felt the moderator did a good job.
\\
Not at all true \hfill 1 \dotfill  2 \dotfill 3 \dotfill 4 \dotfill 5 \hfill Very true\\

\item My overall satisfaction with my group was:
\\
Very low \hfill 1 \dotfill  2 \dotfill 3 \dotfill 4 \dotfill 5 \hfill Very high\\


%%\subsection* {CSRS}
\item I would rather review this code manually using pencil and paper than
using CSRS (on-line Collaborative Software Review System).
\\
Not at all true \hfill 1 \dotfill  2 \dotfill 3 \dotfill 4 \dotfill 5 \hfill Very true\\


\item I believe CSRS made my review of this code more productive
(i.e., find lots of issues)
\\
Not at all true \hfill 1 \dotfill  2 \dotfill 3 \dotfill 4 \dotfill 5 \hfill Very true\\

\item I believe CSRS made my review of this code more effective
(i.e., find lots of ``good'' issues in a relatively short time)
\\
Not at all true \hfill 1 \dotfill  2 \dotfill 3 \dotfill 4 \dotfill 5 \hfill Very true\\

\item EGSM system (i.e., review system for group) is easy to use.
\\
Not at all true \hfill 1 \dotfill  2 \dotfill 3 \dotfill 4 \dotfill 5 \hfill Very true\\

\item EGSM system (i.e., review system for group) is useful.
\\
Not at all true \hfill 1 \dotfill  2 \dotfill 3 \dotfill 4 \dotfill 5 \hfill Very true\\


\item My overall satisfaction with EGSM system was
\\
Very low \hfill 1 \dotfill  2 \dotfill 3 \dotfill 4 \dotfill 5 \hfill Very high\\


\item My overall satisfaction with group review (EGSM) process was
\\
Very low \hfill 1 \dotfill  2 \dotfill 3 \dotfill 4 \dotfill 5 \hfill Very high\\

\item Reasons why I prefer working [alone/in a group] (please choose one
\& explain):
\\
\\

\item Problems that I had with paraphrasing technique (please
explain):
\\
\\

\item Problems that I had with the presenter (please explain):
\\
\\

\item Problems that I had with the moderator (please explain):
\\
\\

\item Problems that I had with EGSM system (please explain): 
\\
\\
\item Problems that I had with EGSM process (please explain): 
\\
\\
\item Suggestions on how to improve EGSM system (please explain):
\\
\\
\item Suggestions on how to improve EGSM process (please explain):
\\
\\

\item Other comments:

\end{enumerate}

%%\end{document}




%%% \documentstyle [12pt,/group/csdl/tex/definemargins,
%%% 		     /group/csdl/tex/lmacros]{article}
%%% 
%%% \begin{document}
%%% 
%%% \begin{center}
%%%   {\large\bf 
%%% %%  CSRS Experiment Fall, 1994\\
%%%   C++ Review Project -- Fall 1994\\
%%%   Questionnaire For Individuals (EIAM)} 
%%% \end{center}

\chapter{Questionnaire For Individuals (EIAM)} 

\begin{description}
%\item [Name:]
\item [Group:]
\item [Source Code:]
\end{description}

\noindent {\it For each question, circle 1, 2, 3, 4 or 5 only} 


\begin {enumerate}

%%\subsection {Comprehension}
\item My understanding of the  source code before the review was:
\\
Very low \hfill 1 \dotfill  2 \dotfill 3 \dotfill 4 \dotfill 5 \hfill Very high\\

\item My understanding of the source code after the review was:
\\
Very low \hfill 1 \dotfill  2 \dotfill 3 \dotfill 4 \dotfill 5 \hfill Very high\\


\item My understanding of C++ programming language was
improved after this review.
\\
Not at all true \hfill 1 \dotfill  2 \dotfill 3 \dotfill 4 \dotfill 5 \hfill Very true\\


\item In general, I found it easy to understand the logic of the code.
\\
Not at all true \hfill 1 \dotfill  2 \dotfill 3 \dotfill 4 \dotfill 5 \hfill Very true\\


%\item The reason I was able to spot this particular error (see
%attached list) was because
% \begin {enumerate}
% \item I saw a similar error or experience in the past.
% \item Because of my programming knowledge.
% \item Other (please explain):
% \end {enumerate}

%\item The reason I was not able to spot this error (see attached list)
%was because 
% \begin {enumerate}
% \item Oversight (I should have caught the error).
% \item I did not know it was an error.
% \item Other (please explain).
% \end {enumerate}


%%\subsection* {Satisfaction & Confidence}
\item For this review, I would have preferred working in a group rather
than working alone.
\\
Not at all true \hfill 1 \dotfill  2 \dotfill 3 \dotfill 4 \dotfill 5 \hfill Very true\\

\item For this review, I felt more confidence about the issues I
raised when working in a group than working alone.
\\
Not at all true \hfill 1 \dotfill  2 \dotfill 3 \dotfill 4 \dotfill 5 \hfill Very true\\


%%\subsection* {Individual strategy}
%\item In general, my review strategy was
%\begin{enumerate}
%\item Trying to understand how each module was implemented, and
%to inspect for possible errors.
%\item Mentally running test cases in each module.
%\item All the above
%\item Other (please explain):
%\\
%\\
%\end{enumerate}


%%\subsection*{Misc}
\item The training session for individual review was sufficient.
\\
Not at all true \hfill 1 \dotfill  2 \dotfill 3 \dotfill 4 \dotfill 5
\hfill Very true\\

\item There was sufficient time to work on this review.
\\
Not at all true \hfill 1 \dotfill  2 \dotfill 3 \dotfill 4 \dotfill 5 \hfill Very true\\

\item I was motivated to do this review project.
\\
Not at all true \hfill 1 \dotfill  2 \dotfill 3 \dotfill 4 \dotfill 5 \hfill Very true\\

\item I worked seriously during this review.
\\
Not at all true \hfill 1 \dotfill  2 \dotfill 3 \dotfill 4 \dotfill 5 \hfill Very true\\

\item My confidence in the overall quality of my review was:
\\
Very low \hfill 1 \dotfill  2 \dotfill 3 \dotfill 4 \dotfill 5 \hfill Very high\\


\item I felt uncomfortable doing this review.
\\
Not at all true \hfill 1 \dotfill  2 \dotfill 3 \dotfill 4 \dotfill 5 \hfill Very true\\


\item It took me a good while to decide whether a particular program
segment contained valid issues.
\\
Not at all true \hfill 1 \dotfill  2 \dotfill 3 \dotfill 4 \dotfill 5 \hfill Very true\\

\item I often wanted to change or delete issues I just created.
\\
Not at all true \hfill 1 \dotfill  2 \dotfill 3 \dotfill 4 \dotfill 5 \hfill Very true\\


%%\subsection* {CSRS}
\item I would rather review this code manually using pencil and paper than
using CSRS (on-line Collaborative Software Review System).
\\
Not at all true \hfill 1 \dotfill  2 \dotfill 3 \dotfill 4 \dotfill 5 \hfill Very true\\



\item I believe CSRS made my review of this code more productive
(i.e., find lots of issues)
\\
Not at all true \hfill 1 \dotfill  2 \dotfill 3 \dotfill 4 \dotfill 5 \hfill Very true\\

\item I believe CSRS made my review of this code more effective
(i.e., find lots of ``good'' issues in a relatively short time)
\\
Not at all true \hfill 1 \dotfill  2 \dotfill 3 \dotfill 4 \dotfill 5
\hfill Very true\\

\item Overall, I enjoyed using CSRS.
\\
Not at all true \hfill 1 \dotfill  2 \dotfill 3 \dotfill 4 \dotfill 5
\hfill Very true\\


\item EIAM system (i.e., review system for individual) is easy to use.
\\
Not at all true \hfill 1 \dotfill  2 \dotfill 3 \dotfill 4 \dotfill 5 \hfill Very true\\


\item EIAM system (i.e., review system for individual) is useful. 
\\
Not at all true \hfill 1 \dotfill  2 \dotfill 3 \dotfill 4 \dotfill 5
\hfill Very true\\

\item My overall satisfaction with EIAM system was:
\\
Very low \hfill 1 \dotfill  2 \dotfill 3 \dotfill 4 \dotfill 5 \hfill Very high\\

\item My overall satisfaction with individual review (EIAM) process
was:
\\
Very low \hfill 1 \dotfill  2 \dotfill 3 \dotfill 4 \dotfill 5 \hfill Very high\\

\item In general, my review techniques/steps were (please explain):
\\
\\
\\
\item Problems that I had with EIAM system (please explain):
\\
\\
\item Problems that I had with EIAM process (please explain):
\\
\\
\item Suggestions on how to improve EIAM system (please explain);
\\
\\
\\
\item Suggestions on how to improve EIAM process (please explain);
\\
\\
\\
\item Other comments:

\end{enumerate}

%%%\end{document}





%%% \documentstyle [12pt,/group/csdl/tex/definemargins,
%%% 		     /group/csdl/tex/lmacros]{article}
%%% 
%%% \begin{document}
%%% 
%%% \begin{center}
%%%   {\large\bf 
%%% %%% CSRS Experiment Fall, 1994\\
%%%   C++ Review Project -- Fall 1994 \\
%%%   Post Post Test Questionnaire}
%%% \end{center}
\chapter{Post Post Test Questionnaire}

\noindent {\it For each question, circle 1, 2, 3, 4 or 5 only} 

\begin {itemize}
\item I would:
    \begin{enumerate}
    \item Strongly prefer using EGSM
    \item Somewhat prefer using EGSM
    \item Equally prefer using EGSM or EIAM
    \item Somewhat prefer using EIAM
    \item Strongly prefer using EIAM
    \end{enumerate}

\item I am:
    \begin{enumerate}
      \item Much more productive using EGSM
      \item Somewhat more productive using EGSM
       \item Equally productive with EGSM or EIAM
      \item Somewhat more productive using EIAM
      \item Much more productive using EIAM
     \end{enumerate}   
\end{itemize}

%%%\end{document}


%%% \documentstyle[11pt,/group/csdl/tex/definemargins,
%%%                        /group/csdl/tex/lmacros]{article} 
%%% 
%%%           \begin{document}
%%%           \begin{center}
%%%           {\large\bf CSRS Experiment -- Source Listing \\
%%%             Source: driver }
%%% \end{center}

\chapter{Source Listing (Driver)}
\small

\section{Constant}
\subsection*{Specification:}

\subsection*{Source-code:}
\begin{verbatim}
001:const int MAXLENGTH = 256;  // Maximum length of array.
002:
003:
\end{verbatim}
\section{Driver}
\subsection*{Specification:}
A driver class. Each instance of this class represents a single
  driver. Drivers have a name, a shift (the integer 1,2,or 3),
  a pay rate and an interger number of years of experience.
\subsection*{Source-code:}
\begin{verbatim}
001:class Driver
002:
003:{
004: private:
005:  char* name;
006:  int shift;
007:  float pay_rate;
008:  int experience;
009: public:
010:  Driver();                // Constructor.
011:  ~Driver();               // Destructor.
012:  void change(char* temp_name,   //Change attributes
013:          int temp_shift, 
014:          float temp_pay_rate, 
015:          int temp_experience);
016:  void print();            // Print Driver information.
017:  };
018:
\end{verbatim}
\section{Driver::Driver()}
\subsection*{Specification:}
Constructor for Driver Class
\subsection*{Source-code:}
\begin{verbatim}
001:Driver::Driver()
002:
003:{
004:  name = new char[MAXLENGTH +1];//Allocate memory for string.
005:  if(!name)                 //If unable to allocate space, print
006:    {                               //error message.
007:      cout << "Allocation error!";
008:      exit(1);
009:    }
010:  name = "";
011:  shift = 0;
012:  pay_rate = 0.0;
013:  experience = 0;
014:}
015:
016:
\end{verbatim}
\section{Driver::\~Driver()}
\subsection*{Specification:}
Destructor for Driver class
\subsection*{Source-code:}
\begin{verbatim}
001:Driver::~Driver()
002:     
003:{
004:
005:}
006:
007:
\end{verbatim}
\section{Driver::change}
\subsection*{Specification:}
Changes the descriptions of the driver instance
\subsection*{Source-code:}
\begin{verbatim}
001:void Driver::change(char* temp_name, int temp_shift, 
002:                    float temp_pay_rate, int temp_experience)
003:
004:{
005:  name = temp_name;
006:  shift = temp_shift;
007:  pay_rate = temp_pay_rate;
008:  experience = temp_experience;
009:}
010:
011:
012:
\end{verbatim}
\section{Driver::print}
\subsection*{Specification:}
Prints the driver instance
\subsection*{Source-code:}
\begin{verbatim}
001:void Driver::print()
002:
003:{
004:  cout<<endl;
005:  cout<<"Driver Information:"<<endl;
006:  cout<<"-------------------"<<endl;
007:  cout<<"      Name : "<<name<<endl;
008:  cout<<"     Shift : "<<shift<<endl;
009:  cout<<"  Pay rate : $"<<pay_rate<<endl;
010:  cout<<"Experience : "<<experience<<" year(s)"<<endl;
011:  cout<<endl;
012:}
013:
014:
015:
016:
\end{verbatim}
\section{print\_error}
\subsection*{Specification:}
Prints error message and exits program.
\subsection*{Source-code:}
\begin{verbatim}
001:void print_error()
002:      
003:{
004:  cout<<"Allocation error!"<<endl;
005:  exit(1);
006:}
007:
008:
\end{verbatim}
\section{print\_menu}
\subsection*{Specification:}
Prints out menu of possible actions that can be perfomed
  on driver
\subsection*{Source-code:}
\begin{verbatim}
001:void print_menu()
002:  
003:{
004:  cout<<"                       MENU"<<endl;
005:  cout<<"=================================================="<<endl;
006:  cout<<"   1. Input values for an instance of driver."<<endl;
007:  cout<<"   2. Print out the instance of driver."<<endl;
008:  cout<<"   3. Quit!"<<endl;
009:  cout<<"=================================================="<<endl;
010:  cout<<"Choice: ";
011:}
012:
013:
\end{verbatim}
\section{main}
\subsection*{Specification:}
Test Driver class. Create and print description of 
   driver instance.
\subsection*{Source-code:}
\begin{verbatim}
001:int main()
002:
003:{
004:  Driver driver_info;               // driver information.
005:  char* userinput_driver;   // User input.
006:  int userinput_shift;              //      "
007:  float userinput_payrate;  //      "
008:  int userinput_years_experience; //        "
009:  char tempchar;
010:  int continue_working = 1; // Continue program = 1, Quit = 0
011:  int menu_choice;          // User's menu selection.
012:
013:  while(continue_working)
014:    {
015:      print_menu();         // Display menu.
016:      cin>>menu_choice;             // Get user's choice.
017:
018:      switch(menu_choice)
019:        {
020:        case 1:                     // Get driver information.
021:      cout<<endl;
022:      cout<<"Input the Following Driver Information:"<<endl;
023:      cout<<"---------------------------------------"<<endl;
024:      cout<<"Name of driver: ";
025:      cin.get(tempchar);
026:      userinput_driver = new char [MAXLENGTH];
027:      
028:      if(!userinput_driver) print_error();
029:      
030:      cin.getline(userinput_driver, MAXLENGTH);
031:      cout<<"Shift: ";
032:      cin >> userinput_shift;
033:      while(userinput_shift>0 && userinput_shift<4) {
034:        cout<<"Error: shift is 1, 2, or 3."<<endl;
035:        cout<<"Shift: ";
036:        cin >> userinput_shift;
037:      }
038:      cout<<"Pay Rate: ";
039:      cin>>userinput_payrate;
040:      while(userinput_payrate<0.0) {
041:        cout<<"Error: payrate is positive real number."<<endl;
042:        cout<<"Pay Rate: ";
043:        cin>>userinput_payrate;
044:      }
045:      cout<<"Years of Experience: ";
046:      cin>>userinput_years_experience;
047:      while(userinput_years_experience<0) {
048:        cout<<"Error:years experience is positive integer"<<endl;
049:        cout<<"Years of Experience: ";
050:        cin>>userinput_years_experience;
051:      }
052:      cout<<endl;
053:                                            // Create driver instance.
054:      driver_info.change(userinput_driver, 
055:                         userinput_shift,
056:                         userinput_payrate, 
057:                         userinput_years_experience);
058:      break;
059:    case 2:                 // Print driver information.
060:      driver_info.print();
061:      break;
062:    case 3:                 // Quit program.
063:      continue_working = 0;
064:      break;
065:    default:
066:      cout<<"Invalid menu choice"<<endl;
067:    } //End case
068:    } //End while
069:  delete [] userinput_driver;
070:  return 0;
071:}
072:
073:
\end{verbatim}
%%%%\end{document}

%%% \documentstyle[11pt,/group/csdl/tex/definemargins,
%%%                        /group/csdl/tex/lmacros]{article} 
%%% 
%%%           \begin{document}
%%%           \begin{center}
%%%           {\large\bf CSRS Experiment: Source Nodes}\\
%%%           \end{center}
\chapter{Source Listing (Yard)}
\small

\section{Constant}
\subsection*{Specification:}

\subsection*{Source-code:}
\begin{verbatim}
001:const int MAXLEN = 40;     // max no. of chars of string input
002:const int MIN_ROUTE = 1;
003:const int MAX_ROUTE = 100;
004:const int MIN_SHIFT = 1;         
005:const int MAX_SHIFT = 3;
006:const float MIN_PAY_RATE = 3.45;     // minimum wage
007:const float MAX_PAY_RATE = 20.0;
008:const int MIN_YEARS = 0;
009:const int MAX_YEARS = 100;
010:const int YARD_SIZE = 10;
011:
012:
\end{verbatim}
\section{driver}
\subsection*{Specification:}
A bus driver class. Each instance of this class represents a
  single driver. Drivers have a name, a shift (the integer
  1,2,or 3), a pay rate and an interger number of years of
  experience.
\subsection*{Source-code:}
\begin{verbatim}
001:class driver
002:
003:
004:{
005:  private:
006:    String name;
007:    int shift;
008:    float pay_rate;
009:    int years_exp;
010:
011:  public:
012:    driver();
013:    ~driver();
014:    void create();        // set all values of an instance of driver
015:    void print();         // print out values of a driver
016:};
017:
018:
\end{verbatim}
\section{vehicle}
\subsection*{Specification:}
Vehicle class: a base class for van and bus classes.
   Vehicles have a seating capacity and a driver.
\subsection*{Source-code:}
\begin{verbatim}
001:class vehicle
002:
003:{
004:  protected:
005:    int seating_cap;
006:    driver incharge;
007:
008:  public: 
009:    vehicle();
010:    virtual ~vehicle();
011:    void create();     // set values for a vehicle
012:    virtual void print();      // display a vehicle
013:};
014:
015:
016:
\end{verbatim}
\section{bus}
\subsection*{Specification:}
A bus class. Each instance of this class represents a
   single bus. Busses are vehicles with a route.
\subsection*{Source-code:}
\begin{verbatim}
001:class bus : public vehicle
002:
003:{
004:  protected:
005:    int route;
006:
007:  public: 
008:    bus();
009:    virtual ~bus();
010:    void create();             // set value for a bus
011:    virtual void print();      // display value of a bus
012:};
013:
014:
\end{verbatim}
\section{van}
\subsection*{Specification:}
A van class. Each instance of this class represents a single
   handivan. Handivans are vehicles with region (a string such
   as Kalihi, Manoa, etc).
\subsection*{Source-code:}
\begin{verbatim}
001:class van : public vehicle
002:
003:{
004:  protected:
005:    char* region;
006:
007:  public: 
008:    van();
009:    virtual ~van();
010:    void create();         // set value of a van
011:    virtual void print();  // display value of a van
012:};
013:
014:
015:
\end{verbatim}
\section{yard}
\subsection*{Specification:}
A yard class. Each instance of this class holds instances
   of vehicles.
\subsection*{Source-code:}
\begin{verbatim}
001:class yard
002:
003:{
004: private:
005:   vehicle* parking[YARD_SIZE];     //parking lots to hold vehicles
006:   int no_of_veh;           //no of vehicles in the yard
007:
008: public:
009:   yard();
010:   ~yard();
011:   void add(vehicle *myveh); 
012:   void remove(); 
013:   void print(); 
014:};
015:
016:
\end{verbatim}
\section{driver::driver}
\subsection*{Specification:}
Constructor for driver class
\subsection*{Source-code:}
\begin{verbatim}
001:driver::driver()
002:
003:{
004:  name = " ";
005:  shift = 0;
006:  pay_rate = 0.0;
007:  years_exp = -1;
008:}
009:
010:
\end{verbatim}
\section{driver::\~driver}
\subsection*{Specification:}
Destructor for driver class
\subsection*{Source-code:}
\begin{verbatim}
001:driver::~driver()
002:
003:{
004:  // not needed here 
005:}
006:
007:
\end{verbatim}
\section{driver::create}
\subsection*{Specification:}
Prompt user to input the descriptions of driver instance
\subsection*{Source-code:}
\begin{verbatim}
001:void driver::create()
002:
003:{
004:  name = " ";
005:  shift = 0;
006:  pay_rate = 0.0;
007:  years_exp = 0;
008:  char *userString = new char[MAXLEN + 1];
009:  char tempchar;
010:
011:// get the driver's name from user
012:  cin.get(tempchar); 
013:  while( (strlen(userString) < 1) || (strlen(userString) > MAXLEN) )
014:    {
015:      cout << "Enter driver's name - less than " << MAXLEN;
016:      cout << " characters long : ";
017:      cin.getline(userString, MAXLEN);
018:      name = userString;
019:    };
020:
021:// get the driver's shift from user
022:// loop only allow input in correct range
023:  while( (shift < MIN_SHIFT) || (shift > MAX_SHIFT) )
024:    {
025:    cout << "Enter driver shift - " << MIN_SHIFT;
026:    cout << " to " << MAX_SHIFT << " : ";
027:    cin >> shift;
028:    };
029:
030:// get the driver's pay rate from user
031:  while( (pay_rate < MIN_PAY_RATE) || (pay_rate > MAX_PAY_RATE) )
032:    {
033:    cout << "Enter driver's pay rate - " << MIN_PAY_RATE;
034:    cout << " to " << MAX_PAY_RATE << " : ";
035:    cin >> pay_rate;
036:    };
037:
038:// get the driver's years of experience from user
039:  while( (years_exp < MIN_YEARS) || (years_exp > MAX_YEARS) )
040:    {
041:    cout << "Enter driver's years of experience - " << MIN_YEARS;
042:    cout << " to " << MAX_YEARS << " : ";
043:    cin >> years_exp;
044:    };
045:}
046:
047:
\end{verbatim}
\section{driver::print}
\subsection*{Specification:}
Print driver info
\subsection*{Source-code:}
\begin{verbatim}
001:void driver::print()
002:
003:{
004:  cout << "driver info. "                                << "\n";
005:  cout << "name: "                          << name      << "\n";
006:  cout << "shift number: "                  << shift     << "\n";
007:  cout << "pay rate: "                      << pay_rate  << "\n";
008:  cout << "number of years of experience: " << years_exp << "\n";
009:}
010:
011:
\end{verbatim}
\section{vehicle::vehicle}
\subsection*{Specification:}
Constructor for vehicle class
\subsection*{Source-code:}
\begin{verbatim}
001:vehicle::vehicle()
002:
003:{
004:  seating_cap = 0;
005:}
006:
007:
\end{verbatim}
\section{vehicle::\~vehicle}
\subsection*{Specification:}
Destructor for vehicle class
\subsection*{Source-code:}
\begin{verbatim}
001:vehicle::~vehicle()
002:
003:{
004:
005:}
006:
007:
\end{verbatim}
\section{vehicle::create}
\subsection*{Specification:}
Prompt user to input the descriptions of vehicle instance
\subsection*{Source-code:}
\begin{verbatim}
001:void vehicle::create()
002:
003:{
004:// get the seating capacity from user
005:  cout << "Enter vehicle seating capacity: ";
006:  cin >> seating_cap;
007:
008:// get the driver from user
009:  incharge.create();          // incharge is a driver ADT
010:}
011:
012:
\end{verbatim}
\section{vehicle::print}
\subsection*{Specification:}
Print vehicle info
\subsection*{Source-code:}
\begin{verbatim}
001:void vehicle::print()
002:     
003:{
004:  cout << "vehicle info. " << "\n";
005:  incharge.print();             // display driver ADT info
006:  cout << "seating capacity: " << seating_cap << "\n";
007:}
008:
\end{verbatim}
\section{bus::bus}
\subsection*{Specification:}
Constructor of bus class
\subsection*{Source-code:}
\begin{verbatim}
001:bus::bus()
002:
003:{
004:  route = 0;
005:}
006:
007:
\end{verbatim}
\section{bus::\~bus}
\subsection*{Specification:}
destructor of bus class
\subsection*{Source-code:}
\begin{verbatim}
001:bus::~bus()
002:
003:{
004:
005:}
006:
007:
\end{verbatim}
\section{bus::create}
\subsection*{Specification:}
Prompt user to input the descriptions of bus instance
\subsection*{Source-code:}
\begin{verbatim}
001:void bus::create()
002:
003:{
004:  vehicle::create(); 
005:  // loop only allow input in correct range
006:  while( (route < MIN_ROUTE) || (route > MAX_ROUTE) )
007:  {
008:    cout << "Enter route number - " << MIN_ROUTE;
009:    cout << " to " << MAX_ROUTE << " : ";
010:    cin >> route;
011:  };
012:}
013:
014:
\end{verbatim}
\section{bus::print}
\subsection*{Specification:}
Print bus info
\subsection*{Source-code:}
\begin{verbatim}
001:void bus::print()
002:
003:{
004:  cout << "bus info. " << "\n";
005:  cout << "route number: " << route << "\n";
006:}
007:
008:
009:
010:
\end{verbatim}
\section{van::van}
\subsection*{Specification:}
Constructor of Handivan class
\subsection*{Source-code:}
\begin{verbatim}
001:van::van()
002:
003:{
004:  region = " ";
005:}
006:
007:
\end{verbatim}
\section{van::\~van}
\subsection*{Specification:}
Destructor of Handivan class
\subsection*{Source-code:}
\begin{verbatim}
001:van::~van()
002:
003:{
004:  
005:}
006:
007:
\end{verbatim}
\section{van::create}
\subsection*{Specification:}
Prompt user to input the descriptions of handivan instance
\subsection*{Source-code:}
\begin{verbatim}
001:void van::create()
002:
003:{
004:  char *regionString = new char[MAXLEN + 1];
005:  char tempchar;
006:
007:   vehicle::create();
008:// get the region from user
009:  cin.get(tempchar);
010:  
011:  while( (strlen(regionString) < 1) || (strlen(regionString) > MAXLEN))
012:  {
013:    cout << "Enter handivan region - less than " << MAXLEN;
014:    cout << " characters long : ";
015:    cin.getline(regionString, MAXLEN);
016:    region = regionString;
017:    
018:  };
019:
020:}
021:
022:
023:
\end{verbatim}
\section{van::print}
\subsection*{Specification:}
Print handivan info
\subsection*{Source-code:}
\begin{verbatim}
001:void van::print()
002:
003:{
004:  cout << "handivan info. " << "\n";
005:  vehicle::print();
006:  cout << "region: " << region << "\n";
007:}
008:
009:
010:
011:
\end{verbatim}
\section{yard::yard}
\subsection*{Specification:}
Constructor for yard class.
\subsection*{Source-code:}
\begin{verbatim}
001:yard::yard()
002:
003:{
004:  no_of_veh = 0; //Initializes no of vehicles to 0 
005:}
006:
007:
\end{verbatim}
\section{yard::\~yard}
\subsection*{Specification:}
Destructor for yard class.
\subsection*{Source-code:}
\begin{verbatim}
001:yard::~yard()
002:
003:{
004:  int j;
005:  for (j = 0; j < YARD_SIZE ;  j++)
006:    {
007:      delete parking[j];
008:    }
009:}
010:
011:
\end{verbatim}
\section{yard::add}
\subsection*{Specification:}
Add a vehicle in the yard instance
\subsection*{Source-code:}
\begin{verbatim}
001:void yard::add(vehicle* myveh)
002:
003:{
004:  int j = no_of_veh + 1;
005:  if(j <= YARD_SIZE)
006:    {
007:      j--;
008:      parking[j] = myveh;
009:    }
010:  else
011:    {
012:      cout << "Yard is full: vehicle not added!" << endl;
013:    }
014:}
015:
016:
017:
\end{verbatim}
\section{yard::remove}
\subsection*{Specification:}
Removes a vehicle from yard instance. Prompt the user for
  vehicle ID, which is the same as parking lot no in the
  yard, and then packs the remaining vehicles in the parking
  lots.
\subsection*{Source-code:}
\begin{verbatim}
001:void yard::remove()
002:
003:{
004:  int trash_veh = 0;    // which vehicle to get rid of?
005:
006:// get the vehicle id to remove from user
007:  while( (trash_veh < 0) || (trash_veh > no_of_veh))
008:    {
009:    cout << "Enter the vehicle id to remove -  1 to ";
010:    cout << no_of_veh << " : ";
011:    cin >> trash_veh;
012:    }
013:
014:  delete parking[trash_veh -1];    // array index starts from 0
015:  no_of_veh--;            // yard has one less vehicle now
016:  int j;
017:  for(j = trash_veh; j <= no_of_veh; j++)
018:    parking[j-1]=parking[j];     // array index starts from 0
019:
020:}
021:
022:
\end{verbatim}
\section{yard::print}
\subsection*{Specification:}
List the vehicles present in the yard instance
\subsection*{Source-code:}
\begin{verbatim}
001:void yard::print()
002:
003:{
004:  cout << "yard info. "                                  << endl;
005:  cout << "number of vehicle: "             << no_of_veh << endl;
006:  int j;
007:  for(j = 0; j < no_of_veh; j++)
008:    {
009:      cout << "vehicle id: " << j;
010:      parking[j]->print();
011:    }
012:}
013:
014:
\end{verbatim}
\section{main}
\subsection*{Specification:}
Test driver for yard. Create and test two instances of
  yard class: Kalihi and Halawa.
\subsection*{Source-code:}
\begin{verbatim}
001:int main()
002:
003:{
004:  yard kalihi;      // an instance of yard for testing
005:  yard halawa;
006:  bus *mybus;       // point to an instance of bus for holding bus vehicle
007:  van *myvan;       
008:  int choice;
009:  int continuego = 1;
010:
011:  while(continuego)
012:    {
013:      cout << "\n";
014:      cout << "------------- Menu -------------------" << endl;
015:      cout << "1 - add a bus to kalihi yard "           << endl;
016:      cout << "2 - add a handivan to kalihi yard "      << endl;
017:      cout << "3 - remove a vehicle from kalihi yard "  << endl;
018:      cout << "4 - list vehicles in kalihi yard "       << endl;
019:      cout << "--------------------------------------"  << endl;
020:      cout << "5 - add a bus to halawa yard "           << endl;
021:      cout << "6 - add a handivan to halawa yard "      << endl;
022:      cout << "7 - remove a vehicle from halawa yard "  << endl;
023:      cout << "8 - list vehicles in halawa yard "       << endl;
024:      cout << "9 - quit "                               << endl;
025:
026:      cin >> choice;            // get an input from user
027:
028:      switch(choice)
029:        {
030:        case 1:
031:          mybus->create();  //rename create
032:          kalihi.add(mybus);
033:          break;
034:        case 2:
035:          myvan->create();
036:          kalihi.add(myvan);
037:          break;
038:        case 3:
039:          kalihi.remove();
040:          break;
041:        case 4:
042:          kalihi.print();
043:          break;
044:        case 5:
045:          mybus->create();
046:          halawa.add(mybus);
047:          break;
048:        case 6:
049:          myvan->create();
050:          halawa.add(myvan);
051:          break;
052:        case 7:
053:          halawa.remove();
054:          break;
055:        case 8:
056:          halawa.print();
057:          break;
058:        case 9:
059:          continuego = 0;   // this will cause the exit of while
060:          break;
061:        default:
062:          cout << "Invalid choice." << endl;
063:        }
064:    }
065:}
066:
067:
068:
\end{verbatim}
%%%\end{document}

\chapter {Error List (ICS411 Experiment)}

\section {Pass1}

\begin{table}[hb]
\begin{center}
\begin{tabular}{|l|l|l|l|}
\hline
No & Source Code & Line & Description \\
\hline
1 & hextonum & 17 & Need to check for blanks (i.e., less than 4 digits)\\
2 &          & 25 & Return partially converted value when error occurred. \\ 
3 & Access\_Symtab & 11 & hash is uninitialized. \\ 
4 &               & 12,28,53 & Should have been SYMTABLIMIT+1.\\
5 &               & 22& Missing ! in strncmp.\\
6 &               & 54-55 & Does not exit the loop when table is full.\\
7 & Write\_Int\_File  & 27 & Missing else. \\ 
8 & P1\_Read\_Source  & 20 & Should have been $<$. \\
9 &               & 29 & Should have been \&\&. \\
10 &              & 40 & Should return when source$->$comline is true.\\
11 &              & 43 & i is uninitialized. \\ 
12 &              & 57 & Should have been 14 (not 13) \\
13 &              & 73 & source$->$operand are not initialized.\\
14 & P1\_Proc\_RESW & 30 & Should have been nwords*3.\\ 
15 & P1\_Assign\_Loc & 28 & Should have been locctr+3.\\
16 & P1\_Assign\_Sym & 12 & Extra character ! in strncmp.\\
17 &                 & 16 & Should have been STORE.\\ 
18 &                 & 17 & Should have been ==.\\ 
19 & Pass\_1 & 10  & Missing initialization of endofinput = false.\\
20 & hextonum & 22 & When first $>$ 0, the condition may not be satisfied\\
\hline
\end{tabular}
\caption{Error List for Pass1}
\end{center}
\end{table}

\section {Pass2}

\begin{table}[hb]
\begin{center}
\begin{tabular}{|l|l|l|l|}
\hline
No & Source Code & Line & Description \\
\hline
1 & dectonum & 18 & Check for 4 digits max.\\
2 & dectonum & 20 & Return partially converted value when error occurred. \\ 
3 & dectonum & 11 & i is uninitialized. \\ 
4 & P2\_Search..  &13,15 & Should have been high=mid-1 and low=mid+1.\\
5 & P2\_Search..  & 18 & Missing ! in strncmp.\\
6 & dectonum & 16 & Does not exit the loop after *converror is set to true.\\
7 & Read\_Int\_File  & 18 & Missing else. \\ 
8 & P2\_Write\_Obj  & 46 & Should have been $<$. \\
9 & P2\_Write\_Obj &32 & Should have been $||$. \\
10 &P2\_Write\_Obj &52  & Should return when objct.rectype!=ENDREC.\\
11 & P2\_Proc\_BYTE &35 & i is uninitialized. \\ 
12 & Read\_Int\_File  &31  & Should have been 5 (not 4) \\
13 & Read\_Int\_File & 19 & locctr is not read.\\
14 & P2\_Proc\_BYTE & 45 & Should have been (i-1)*2.\\ 
15 & P2\_Write\_Obj &36  & Should have been /2.\\
16 & P2\_Assemble..& 37 & Extra character ! in strncmp.\\
17 & P2\_Write\_Obj & 16 & Should have been PROGLENGTH.\\ 
18 & P2\_Proc\_BYTE & 19 & Should have been ==.\\ 
19 & P2\_Proc\_START & 19  & Missing FIRSTSTMT = false.\\
\hline
\end{tabular}
\caption{Error List for Pass2}
\end{center}
\end{table}

%%% \documentstyle[11pt,/group/csdl/tex/definemargins,
%%%                        /group/csdl/tex/lmacros]{article} 
%%% 
%%%           \begin{document}
%%%           \begin{center}
%%%           {\large\bf CSRS Experiment Results}\\
%%%           \end{center}
%%% 	  
\chapter {CSRS Experiment Results: Group1(EGSM)}
\small

\begin{description}
\item [Method:] EGSM
\item [Group:] Group1
\item [Source:] driver
\item [Participants:] ysiou (Moderator), gchen (Reviewer), yanwang (Presenter)
\end{description}
\section{Issue Lists}
\begin{enumerate}
\item {\it Issue\#202 (yanwang)}
\begin{description}
\item [Subject:] need delete pointer to delete whole charater array
\item [Criticality:] Hi:{\it gchen,ysiou,yanwang} Med:{\it } Low:{\it } None:{\it }
\item [Suggested-by:] Me:{\it gchen,yanwang,ysiou} Other-and-me:{\it } Other:{\it }
\item [Confidence-level:] Hi:{\it ysiou,yanwang,gchen} Med:{\it } Low:{\it } None:{\it }
\item [Source-node:] Driver::\~Driver()

\item [Lines:] 3-5

\item [Description:] need clear delete[] name
\end{description}
\item {\it Issue\#204 (yanwang)}
\begin{description}
\item [Subject:] can not print out a pointer
\item [Criticality:] Hi:{\it ysiou,gchen,yanwang} Med:{\it } Low:{\it } None:{\it }
\item [Suggested-by:] Me:{\it ysiou} Other-and-me:{\it gchen,yanwang} Other:{\it }
\item [Confidence-level:] Hi:{\it gchen,ysiou,yanwang} Med:{\it } Low:{\it } None:{\it }
\item [Source-node:] Driver::print

\item [Lines:] 7

\item [Description:] To print out the character array we need dereference pointer name
\end{description}
\item {\it Issue\#212 (yanwang)}
\begin{description}
\item [Subject:] pointer problem
\item [Criticality:] Hi:{\it gchen,ysiou,yanwang} Med:{\it } Low:{\it } None:{\it }
\item [Suggested-by:] Me:{\it yanwang} Other-and-me:{\it gchen,ysiou} Other:{\it }
\item [Confidence-level:] Hi:{\it gchen,ysiou,yanwang} Med:{\it } Low:{\it } None:{\it }
\item [Source-node:] Driver::Driver()

\item [Lines:] 10

\item [Description:] need to dereferent the name and then assign a value
\end{description}
\item {\it Issue\#218 (yanwang)}
\begin{description}
\item [Subject:] pointer repeat delete
\item [Criticality:] Hi:{\it gchen,yanwang,ysiou} Med:{\it } Low:{\it } None:{\it }
\item [Suggested-by:] Me:{\it } Other-and-me:{\it ysiou,gchen,yanwang} Other:{\it }
\item [Confidence-level:] Hi:{\it gchen,yanwang} Med:{\it ysiou} Low:{\it } None:{\it }
\item [Source-node:] main

\item [Lines:] 69

\item [Description:] deconstruct function should do this work!  because when we construct the
driver, name pointer has point to this array. when we deconstruct it is
destroied already!
\end{description}
\item {\it Issue\#222 (yanwang)}
\begin{description}
\item [Subject:] good programmer to easy test error
\item [Criticality:] Hi:{\it } Med:{\it gchen} Low:{\it yanwang} None:{\it ysiou}
\item [Suggested-by:] Me:{\it yanwang} Other-and-me:{\it } Other:{\it gchen,ysiou}
\item [Confidence-level:] Hi:{\it yanwang} Med:{\it gchen,ysiou} Low:{\it } None:{\it }
\item [Source-node:] Driver::change

\item [Lines:] 1-3

\item [Description:] put const to the variables
\end{description}
\item {\it Issue\#226 (yanwang)}
\begin{description}
\item [Subject:] need more space to terminate the string
\item [Criticality:] Hi:{\it gchen} Med:{\it ysiou,yanwang} Low:{\it } None:{\it }
\item [Suggested-by:] Me:{\it ysiou} Other-and-me:{\it yanwang,gchen} Other:{\it }
\item [Confidence-level:] Hi:{\it ysiou,yanwang} Med:{\it gchen} Low:{\it } None:{\it }
\item [Source-node:] main

\item [Lines:] 26

\item [Description:] we need more space at the character array to flag the termination of string
\end{description}
\end{enumerate}
\section{Review Metrics}
\begin{table}[hb]
\begin{center}
\begin{tabular}{|l|l|l|l|}
\hline
Participant & Start-time & End-time & Total Busy-time \\
\hline
ysiou & Nov 30, 1994 16:20:10 & Nov 30, 1994 17:33:53 & 1:10:43 \\
gchen & Nov 30, 1994 16:20:16 & Nov 30, 1994 17:33:57 & 1:6:25 \\
yanwang & Nov 30, 1994 16:19:58 & Nov 30, 1994 17:33:51 & 1:13:53 \\
\hline
\end{tabular}
\end{center}
\caption{Review Session}
\end{table}


\begin{table}[hb]
\begin{center}
\begin{tabular}{|l|l|l|l|}
\hline
Source & ysiou & gchen & yanwang\\
\hline
(176)Constant & 119 & 113 & 116\\
(192)main & 1395 & 803 & 1069\\
(178)Driver & 318 & 307 & 311\\
(180)Driver::Driver() & 446 & 751 & 726\\
(182)Driver::\~Driver() & 586 & 596 & 601\\
(184)Driver::change & 341 & 444 & 479\\
(186)Driver::print & 508 & 441 & 620\\
(188)print\_error & 235 & 236 & 241\\
(190)print\_menu & 146 & 151 & 148\\
\hline
\end{tabular}
\end{center}
\caption{Review Time}
\end{table}


\begin{table}[hb]
\begin{center}
\begin{tabular}{|l|l|l|}
\hline
Source node & Issue node & OK \\
\hline
(176)Constant & & \\
(192)main & \#218,\#226 (=2) & \#218,\#226\\
(178)Driver & &\\
(180)Driver::Driver() & \#212 (=1) & \#212\\
(182)Driver::\~Driver() & \#202 (=1) & \#202\\
(184)Driver::change & \#222 (=1) &\\
(186)Driver::print & \#204 (=1) &\\
(188)print\_error & &\\
(190)print\_menu & & \\
\hline
\end{tabular}
\caption{Source node v.s Issue node}
\end{center}
\end{table}

%%%\end{document}

%%% \documentstyle[11pt,/group/csdl/tex/definemargins,
%%%                        /group/csdl/tex/lmacros]{article} 
%%% 
%%%           \begin{document}
%%%           \begin{center}
%%%           {\large\bf CSRS Experiment Results}\\
%%%           \end{center}
	  
\chapter {CSRS Experiment Results: Group2(EGSM)}
\small

\begin{description}
\item [Method:] EGSM
\item [Group:] Group2
\item [Source:] driver
\item [Participants:] cokumoto (Moderator), subin (Reviewer), wagnerm (Presenter)
\end{description}
\section{Issue Lists}
\begin{enumerate}
\item {\it Issue\#200 (wagnerm)}
\begin{description}
\item [Subject:] De-allocation of memory "name"
\item [Criticality:] Hi:{\it } Med:{\it subin,wagnerm,cokumoto} Low:{\it } None:{\it }
\item [Suggested-by:] Me:{\it cokumoto} Other-and-me:{\it subin,wagnerm} Other:{\it }
\item [Confidence-level:] Hi:{\it subin,cokumoto} Med:{\it wagnerm} Low:{\it } None:{\it }
\item [Source-node:] Driver::\~Driver()

\item [Lines:] 

\item [Description:] 
\end{description}
\item {\it Issue\#202 (wagnerm)}
\begin{description}
\item [Subject:] shift input .. while loop incorrect
\item [Criticality:] Hi:{\it subin,wagnerm,cokumoto} Med:{\it } Low:{\it } None:{\it }
\item [Suggested-by:] Me:{\it cokumoto} Other-and-me:{\it subin,wagnerm} Other:{\it }
\item [Confidence-level:] Hi:{\it cokumoto,subin,wagnerm} Med:{\it } Low:{\it } None:{\it }
\item [Source-node:] main

\item [Lines:] 

\item [Description:] 
\end{description}
\item {\it Issue\#204 (wagnerm)}
\begin{description}
\item [Subject:] de-allocation of memory .. userinput\_driver
\item [Criticality:] Hi:{\it subin,wagnerm} Med:{\it cokumoto} Low:{\it } None:{\it }
\item [Suggested-by:] Me:{\it cokumoto} Other-and-me:{\it subin,wagnerm} Other:{\it }
\item [Confidence-level:] Hi:{\it subin,wagnerm,cokumoto} Med:{\it } Low:{\it } None:{\it }
\item [Source-node:] main

\item [Lines:] 

\item [Description:] 
\end{description}
\item {\it Issue\#206 (wagnerm)}
\begin{description}
\item [Subject:] allocation of memory ... twice
\item [Criticality:] Hi:{\it } Med:{\it cokumoto,subin,wagnerm} Low:{\it } None:{\it }
\item [Suggested-by:] Me:{\it subin} Other-and-me:{\it wagnerm} Other:{\it cokumoto}
\item [Confidence-level:] Hi:{\it subin} Med:{\it cokumoto,wagnerm} Low:{\it } None:{\it }
\item [Source-node:] Driver::Driver()

\item [Lines:] 

\item [Description:] 
\end{description}
\end{enumerate}
\section{Review Metrics}
\begin{table}[hb]
\begin{center}
\begin{tabular}{|l|l|l|l|}
\hline
Participant & Start-time & End-time & Total Busy-time \\
\hline
cokumoto & Nov 28, 1994 11:38:04 & Nov 28, 1994 12:23:59 & 0:39:50 \\
subin & Nov 28, 1994 11:38:01 & Nov 28, 1994 12:25:11 & 0:36:22 \\
wagnerm & Nov 28, 1994 11:37:11 & Nov 28, 1994 12:23:48 & 0:40:33 \\
\hline
\end{tabular}
\end{center}
\caption{Review Session}
\end{table}


\begin{table}[hb]
\begin{center}
\begin{tabular}{|l|l|l|l|}
\hline
Source & cokumoto & subin & wagnerm\\
\hline
(192)main & 921 & 639 & 756\\
(176)Constant & 45 & 45 & 46\\
(178)Driver & 256 & 257 & 259\\
(180)Driver::Driver() & 463 & 462 & 637\\
(182)Driver::\~Driver() & 261 & 261 & 262\\
(184)Driver::change & 133 & 134 & 136\\
(186)Driver::print & 139 & 139 & 139\\
(188)print\_error & 47 & 47 & 49\\
(190)print\_menu & 43 & 45 & 45\\
\hline
\end{tabular}
\end{center}
\caption{Review Time}
\end{table}


\begin{table}[hb]
\begin{center}
\begin{tabular}{|l|l|l|}
\hline
Source node & Issue node  & OK\\
\hline
(192)main & \#202,\#204 (=2) & \#202,\#204\\
(176)Constant & &\\
(178)Driver & &\\
(180)Driver::Driver() & \#206 (=1) &\\
(182)Driver::\~Driver() & \#200 (=1) & \#200\\
(184)Driver::change & &\\
(186)Driver::print & &\\
(188)print\_error & &\\
(190)print\_menu & &\\
\hline
\end{tabular}
\caption{Source node v.s Issue node}
\end{center}
\end{table}

%%%\end{document}

%%% \documentstyle[11pt,/group/csdl/tex/definemargins,
%%%                        /group/csdl/tex/lmacros]{article} 
%%% 
%%%           \begin{document}
%%%           \begin{center}
%%%           {\large\bf CSRS Experiment Results}\\
%%%           \end{center}
	  
\chapter {CSRS Experiment Results: Group3(EGSM)}
\small

\begin{description}
\item [Method:] EGSM
\item [Group:] Group3
\item [Source:] yard
\item [Participants:] wongcheu (Reviewer), sko (Moderator), kawak (Presenter)
\end{description}
\section{Issue Lists}
\begin{enumerate}
\item {\it Issue\#238 (kawak)}
\begin{description}
\item [Subject:] Yard-size should be no\_of\_veh
\item [Criticality:] Hi:{\it wongcheu} Med:{\it } Low:{\it } None:{\it kawak,sko}
\item [Suggested-by:] Me:{\it wongcheu} Other-and-me:{\it } Other:{\it sko,kawak}
\item [Confidence-level:] Hi:{\it wongcheu} Med:{\it } Low:{\it } None:{\it sko,kawak}
\item [Source-node:] yard::\~yard

\item [Lines:] 5-7

\item [Description:] Kevin- I say it's not an issue because this dwstructor should not have do do
anything.

Scott- no comment

Cheung- Error because the destructor might try to destroy vehicles that might
not be there.
\end{description}
\item {\it Issue\#244 (kawak)}
\begin{description}
\item [Subject:] Desstructor shouldn't do anything
\item [Criticality:] Hi:{\it } Med:{\it sko,wongcheu,kawak} Low:{\it } None:{\it }
\item [Suggested-by:] Me:{\it kawak} Other-and-me:{\it sko,wongcheu} Other:{\it }
\item [Confidence-level:] Hi:{\it kawak} Med:{\it sko,wongcheu} Low:{\it } None:{\it }
\item [Source-node:] yard::\~yard

\item [Lines:] 4-8

\item [Description:] Kevin- I think that this destructor shouldn't do anything.  If there are
vehicles in this yard, they should be destroyed bny their own destructor when
they go out of scope.  If a vehicle is referenced by more than one yard and
it is destroyerd when one of the yards go out of scope, then there wmight be
a problem when another yard tries to access a vehicle.
\end{description}
\item {\it Issue\#248 (kawak)}
\begin{description}
\item [Subject:] Should say virtual
\item [Criticality:] Hi:{\it } Med:{\it wongcheu,sko} Low:{\it } None:{\it kawak}
\item [Suggested-by:] Me:{\it wongcheu} Other-and-me:{\it sko} Other:{\it kawak}
\item [Confidence-level:] Hi:{\it } Med:{\it sko,wongcheu} Low:{\it } None:{\it kawak}
\item [Source-node:] vehicle::print

\item [Lines:] 1-2

\item [Description:] Should say virtual void to match classs declartaion
\end{description}
\item {\it Issue\#252 (kawak)}
\begin{description}
\item [Subject:] no\_of-veh not incremented
\item [Criticality:] Hi:{\it wongcheu} Med:{\it sko,kawak} Low:{\it } None:{\it }
\item [Suggested-by:] Me:{\it kawak} Other-and-me:{\it wongcheu} Other:{\it sko}
\item [Confidence-level:] Hi:{\it sko,kawak,wongcheu} Med:{\it } Low:{\it } None:{\it }
\item [Source-node:] yard::add

\item [Lines:] 

\item [Description:] no\_of\_veh is not incremented to reflect the new number of vehicles in the
yard.
\end{description}
\item {\it Issue\#254 (kawak)}
\begin{description}
\item [Subject:] Never enters loop
\item [Criticality:] Hi:{\it wongcheu} Med:{\it sko} Low:{\it kawak} None:{\it }
\item [Suggested-by:] Me:{\it sko,kawak} Other-and-me:{\it } Other:{\it wongcheu}
\item [Confidence-level:] Hi:{\it sko,kawak} Med:{\it wongcheu} Low:{\it } None:{\it }
\item [Source-node:] yard::remove

\item [Lines:] 7-8

\item [Description:] Because trash\_veh is initialized to zero, this loop which accepts user input
is never entered.
\end{description}
\item {\it Issue\#258 (kawak)}
\begin{description}
\item [Subject:] Memory never freed.
\item [Criticality:] Hi:{\it } Med:{\it wongcheu,sko,kawak} Low:{\it } None:{\it }
\item [Suggested-by:] Me:{\it kawak} Other-and-me:{\it sko,wongcheu} Other:{\it }
\item [Confidence-level:] Hi:{\it sko,wongcheu} Med:{\it kawak} Low:{\it } None:{\it }
\item [Source-node:] driver::create

\item [Lines:] 8-9

\item [Description:] This memory created is never freed.
\end{description}
\item {\it Issue\#262 (kawak)}
\begin{description}
\item [Subject:] Bad Assignment
\item [Criticality:] Hi:{\it } Med:{\it wongcheu,kawak} Low:{\it } None:{\it sko}
\item [Suggested-by:] Me:{\it } Other-and-me:{\it sko,wongcheu,kawak} Other:{\it }
\item [Confidence-level:] Hi:{\it sko,wongcheu} Med:{\it kawak} Low:{\it } None:{\it }
\item [Source-node:] van::van

\item [Lines:] 4-6

\item [Description:] Assignment won't be permanent
\end{description}
\item {\it Issue\#270 (kawak)}
\begin{description}
\item [Subject:] Memory leak
\item [Criticality:] Hi:{\it } Med:{\it wongcheu,sko,kawak} Low:{\it } None:{\it }
\item [Suggested-by:] Me:{\it } Other-and-me:{\it kawak} Other:{\it wongcheu,sko}
\item [Confidence-level:] Hi:{\it } Med:{\it wongcheu,sko,kawak} Low:{\it } None:{\it }
\item [Source-node:] van::create

\item [Lines:] 4-5

\item [Description:] Dynamically creat4ed memeory not deleted.
\end{description}
\item {\it Issue\#274 (kawak)}
\begin{description}
\item [Subject:] No Dynamic allocation
\item [Criticality:] Hi:{\it } Med:{\it sko,wongcheu,kawak} Low:{\it } None:{\it }
\item [Suggested-by:] Me:{\it wongcheu} Other-and-me:{\it kawak} Other:{\it sko}
\item [Confidence-level:] Hi:{\it kawak,wongcheu} Med:{\it sko} Low:{\it } None:{\it }
\item [Source-node:] main

\item [Lines:] 6-8

\item [Description:] No creation of new busses or vans in memery.  Points to garbage initially.
Even if it allow user to add a buss, the same memory will be overwritten,
when additional adds are done.
\end{description}
\end{enumerate}
\section{Review Metrics}
\begin{table}[hb]
\begin{center}
\begin{tabular}{|l|l|l|l|}
\hline
Participant & Start-time & End-time & Total Busy-time \\
\hline
sko & Dec 07, 1994 15:27:52 & Dec 07, 1994 16:57:42 & 0:49:37 \\
wongcheu & Dec 07, 1994 15:29:15 & Dec 07, 1994 16:56:06 & 1:17:28 \\
kawak & Dec 07, 1994 15:28:37 & Dec 07, 1994 16:57:46 & 1:9:38 \\
\hline
\end{tabular}
\end{center}
\caption{Review Session}
\end{table}


\begin{table}[hb]
\begin{center}
\begin{tabular}{|l|l|l|l|}
\hline
Source & sko & wongcheu & kawak\\
\hline
(176)Constant & 86 & 77 & 81\\
(192)driver::create & 46 & 276 & 446\\
(208)bus::create & 114 & 58 & 62\\
(224)yard::add & 0 & 315 & 21\\
(178)driver & 37 & 36 & 53\\
(194)driver::print & 0 & 13 & 13\\
(210)bus::print & 74 & 37 & 38\\
(226)yard::remove & 415 & 475 & 7\\
(180)vehicle & 165 & 159 & 173\\
(196)vehicle::vehicle & 22 & 19 & 25\\
(212)van::van & 130 & 263 & 269\\
(228)yard::print & 67 & 66 & 71\\
(182)bus & 136 & 201 & 222\\
(198)vehicle::\~vehicle & 0 & 12 & 16\\
(214)van::\~van & 0 & 67 & 69\\
(230)main & 242 & 391 & 495\\
(184)van & 32 & 77 & 79\\
(200)vehicle::create & 64 & 67 & 67\\
(216)van::create & 0 & 135 & 136\\
(186)yard & 290 & 389 & 309\\
(202)vehicle::print & 0 & 127 & 129\\
(218)van::print & 0 & 30 & 32\\
(188)driver::driver & 48 & 49 & 50\\
(204)bus::bus & 18 & 16 & 20\\
(220)yard::yard & 92 & 94 & 96\\
(190)driver::\~driver & 8 & 34 & 13\\
(206)bus::\~bus & 8 & 4 & 5\\
(222)yard::\~yard & 668 & 845 & 834\\
\hline
\end{tabular}
\end{center}
\caption{Review Time}
\end{table}


\begin{table}[hb]
\begin{center}
\begin{tabular}{|l|l|l|}
\hline
Source node & Issue node & OK\\
\hline
(176)Constant & &\\
(192)driver::create & \#258 (=1) & \#258\\
(208)bus::create & &\\
(224)yard::add & \#252 (=1) & \#252\\
(178)driver & & \\
(194)driver::print & & \\
(210)bus::print & &\\
(226)yard::remove & \#254 (=1) & \#254\\
(180)vehicle & & \\
(196)vehicle::vehicle & &\\
(212)van::van & \#262 (=1) & \\
(228)yard::print & & \\
(182)bus & & \\
(198)vehicle::\~vehicle & & \\
(214)van::\~van & &\\
(230)main & \#274 (=1) & \#274\\
(184)van & & \\
(200)vehicle::create & & \\
(216)van::create & \#270 (=1) & \#270,\#270\\
(186)yard & \\
(202)vehicle::print & \#248 (=1) &\\
(218)van::print & &\\
(188)driver::driver & & \\
(204)bus::bus & &\\
(220)yard::yard & & \\
(190)driver::\~driver & & \\
(206)bus::\~bus & & \\
(222)yard::\~yard & \#238,\#244 (=2) & \#238\\
\hline
\end{tabular}
\caption{Source node v.s Issue node}
\end{center}
\end{table}

%%%\end{document}

%%% \documentstyle[11pt,/group/csdl/tex/definemargins,
%%%                        /group/csdl/tex/lmacros]{article} 
%%% 
%%%           \begin{document}
%%%           \begin{center}
%%%           {\large\bf CSRS Experiment Results}\\
%%%           \end{center}
%%% 	  
\chapter {CSRS Experiment Results: Group4(EGSM)}
\small

\begin{description}
\item [Method:] EGSM
\item [Group:] Group4
\item [Source:] yard
\item [Participants:] hyeung (Moderator), savid (Reviewer), lnohara (Presenter)
\end{description}
\section{Issue Lists}
\begin{enumerate}
\item {\it Issue\#236 (lnohara)}
\begin{description}
\item [Subject:] logic error in loop
\item [Criticality:] Hi:{\it } Med:{\it lnohara,hyeung} Low:{\it } None:{\it }
\item [Suggested-by:] Me:{\it lnohara} Other-and-me:{\it } Other:{\it hyeung}
\item [Confidence-level:] Hi:{\it hyeung,lnohara} Med:{\it } Low:{\it } None:{\it }
\item [Source-node:] yard::remove

\item [Lines:] 7

\item [Description:] will not enter while loop
\end{description}
\item {\it Issue\#240 (lnohara)}
\begin{description}
\item [Subject:] memory leakage
\item [Criticality:] Hi:{\it } Med:{\it lnohara,hyeung} Low:{\it } None:{\it }
\item [Suggested-by:] Me:{\it hyeung} Other-and-me:{\it } Other:{\it lnohara}
\item [Confidence-level:] Hi:{\it hyeung} Med:{\it lnohara} Low:{\it } None:{\it }
\item [Source-node:] driver::create

\item [Lines:] 8

\item [Description:] command new used with no delete
\end{description}
\item {\it Issue\#244 (lnohara)}
\begin{description}
\item [Subject:] logic error. will not enter loop
\item [Criticality:] Hi:{\it } Med:{\it hyeung,lnohara} Low:{\it } None:{\it }
\item [Suggested-by:] Me:{\it } Other-and-me:{\it hyeung,lnohara} Other:{\it }
\item [Confidence-level:] Hi:{\it hyeung} Med:{\it lnohara} Low:{\it } None:{\it }
\item [Source-node:] driver::create

\item [Lines:] 39

\item [Description:] wrong initial value. years\_exp should
initially be -1 in order to enter the loop
\end{description}
\item {\it Issue\#248 (lnohara)}
\begin{description}
\item [Subject:] more memory leaking
\item [Criticality:] Hi:{\it } Med:{\it hyeung,lnohara} Low:{\it } None:{\it }
\item [Suggested-by:] Me:{\it lnohara} Other-and-me:{\it hyeung} Other:{\it }
\item [Confidence-level:] Hi:{\it hyeung,lnohara} Med:{\it } Low:{\it } None:{\it }
\item [Source-node:] van::\~van

\item [Lines:] 3-5

\item [Description:] need delete to remove char*
\end{description}
\item {\it Issue\#252 (lnohara)}
\begin{description}
\item [Subject:] not sure what this is for
\item [Criticality:] Hi:{\it } Med:{\it } Low:{\it lnohara,hyeung} None:{\it }
\item [Suggested-by:] Me:{\it lnohara} Other-and-me:{\it } Other:{\it hyeung}
\item [Confidence-level:] Hi:{\it } Med:{\it } Low:{\it lnohara,hyeung} None:{\it }
\item [Source-node:] van::print

\item [Lines:] 5 5

\item [Description:] huh?
\end{description}
\item {\it Issue\#256 (lnohara)}
\begin{description}
\item [Subject:] syntax?
\item [Criticality:] Hi:{\it } Med:{\it } Low:{\it lnohara,hyeung} None:{\it }
\item [Suggested-by:] Me:{\it hyeung} Other-and-me:{\it lnohara} Other:{\it }
\item [Confidence-level:] Hi:{\it } Med:{\it } Low:{\it lnohara,hyeung} None:{\it }
\item [Source-node:] yard::print

\item [Lines:] 10

\item [Description:] not sure if proper coding
\end{description}
\end{enumerate}
\section{Review Metrics}
\begin{table}[hb]
\begin{center}
\begin{tabular}{|l|l|l|l|}
\hline
Participant & Start-time & End-time & Total Busy-time \\
\hline
hyeung & Dec 08, 1994 15:45:17 & Dec 08, 1994 16:27:20 & 0:42:3 \\
savid & Jan 06, 1995 09:47:48 & Jan 06, 1995 09:47:48 & 0:0:0 \\
lnohara & Dec 08, 1994 15:36:12 & Dec 08, 1994 16:27:19 & 0:49:5 \\
\hline
\end{tabular}
\end{center}
\caption{Review Session}
\end{table}


\begin{table}[hb]
\begin{center}
\begin{tabular}{|l|l|l|l|}
\hline
Source & hyeung & savid & lnohara\\
\hline
(224)yard::add & 58 & 0 & 60\\
(208)bus::create & 84 & 0 & 89\\
(192)driver::create & 529 & 0 & 579\\
(176)Constant & 60 & 0 & 78\\
(226)yard::remove & 275 & 0 & 279\\
(210)bus::print & 36 & 0 & 37\\
(194)driver::print & 36 & 0 & 40\\
(178)driver & 59 & 0 & 189\\
(228)yard::print & 90 & 0 & 92\\
(212)van::van & 13 & 0 & 14\\
(196)vehicle::vehicle & 51 & 0 & 56\\
(180)vehicle & 154 & 0 & 334\\
(230)main & 152 & 0 & 157\\
(214)van::\~van & 122 & 0 & 125\\
(198)vehicle::\~vehicle & 15 & 0 & 18\\
(182)bus & 54 & 0 & 57\\
(216)van::create & 155 & 0 & 156\\
(200)vehicle::create & 47 & 0 & 49\\
(184)van & 46 & 0 & 49\\
(218)van::print & 115 & 0 & 117\\
(202)vehicle::print & 50 & 0 & 52\\
(186)yard & 71 & 0 & 74\\
(220)yard::yard & 11 & 0 & 13\\
(204)bus::bus & 14 & 0 & 17\\
(188)driver::driver & 31 & 0 & 32\\
(222)yard::\~yard & 26 & 0 & 27\\
(206)bus::\~bus & 7 & 0 & 8\\
(190)driver::\~driver & 16 & 0 & 18\\
\hline
\end{tabular}
\end{center}
\caption{Review Time}
\end{table}


\begin{table}[hb]
\begin{center}
\begin{tabular}{|l|l|l|}
\hline
Source node & Issue node & OK \\
\hline
(224)yard::add & & \\
(208)bus::create & & \\
(192)driver::create & \#240,\#244 (=2) & \#240,\#244 \\
(176)Constant & & \\
(226)yard::remove & \#236 (=1) & \#236\\
(210)bus::print & & \\
(194)driver::print & & \\
(178)driver & & \\
(228)yard::print & \#256 (=1) &  \\
(212)van::van & & \\
(196)vehicle::vehicle & & \\
(180)vehicle & & \\
(230)main & & \\
(214)van::\~van & \#248 (=1) & \#248\\
(198)vehicle::\~vehicle & & \\
(182)bus & & \\
(216)van::create & & \\
(200)vehicle::create & & \\
(184)van & & \\
(218)van::print & \#252 (=1)& \\
(202)vehicle::print & & \\
(186)yard & & \\
(220)yard::yard & & \\
(204)bus::bus & & \\
(188)driver::driver & & \\
(222)yard::\~yard & & \\
(206)bus::\~bus & & \\
(190)driver::\~driver & & \\
\hline
\end{tabular}
\caption{Source node v.s Issue node}
\end{center}
\end{table}

%%%\end{document}

%%% \documentstyle[11pt,/group/csdl/tex/definemargins,
%%%                        /group/csdl/tex/lmacros]{article} 
%%% 
%%%           \begin{document}
%%%           \begin{center}
%%%           {\large\bf CSRS Experiment Results}\\
%%%           \end{center}
%%% 	  
\chapter {CSRS Experiment Results: Group5(EGSM)}
\small

\begin{description}
\item [Method:] EGSM
\item [Group:] Group5
\item [Source:] driver
\item [Participants:] horigan (Reviewer), donthi (Moderator), casem (Presenter)
\end{description}
\section{Issue Lists}
\begin{enumerate}
\item {\it Issue\#200 (casem)}
\begin{description}
\item [Subject:] misspelled word in Specification
\item [Criticality:] Hi:{\it horigan} Med:{\it } Low:{\it casem,donthi} None:{\it }
\item [Suggested-by:] Me:{\it casem} Other-and-me:{\it horigan} Other:{\it donthi}
\item [Confidence-level:] Hi:{\it horigan} Med:{\it casem,donthi} Low:{\it } None:{\it }
\item [Source-node:] Driver

\item [Lines:] 3

\item [Description:] misspelled word interger in the specification
in node driver.
\end{description}
\item {\it Issue\#204 (casem)}
\begin{description}
\item [Subject:] Allocation error.
\item [Criticality:] Hi:{\it horigan} Med:{\it casem,donthi} Low:{\it } None:{\it }
\item [Suggested-by:] Me:{\it casem} Other-and-me:{\it donthi,horigan} Other:{\it }
\item [Confidence-level:] Hi:{\it horigan} Med:{\it donthi} Low:{\it casem} None:{\it }
\item [Source-node:] Driver::Driver()

\item [Lines:] 4

\item [Description:] Maybe an error in allocating more that enough 
memory.  There was no specification on how large that array could get.
\end{description}
\item {\it Issue\#208 (casem)}
\begin{description}
\item [Subject:] No destructor code in destructor function
\item [Criticality:] Hi:{\it donthi,casem,horigan} Med:{\it } Low:{\it } None:{\it }
\item [Suggested-by:] Me:{\it casem} Other-and-me:{\it horigan} Other:{\it donthi}
\item [Confidence-level:] Hi:{\it casem,horigan} Med:{\it } Low:{\it donthi} None:{\it }
\item [Source-node:] Driver::\~Driver()

\item [Lines:] 3-4

\item [Description:] This has no code that would deallocate memory
that was allocated in the constructor.  Serious potential for a memory leak.
\end{description}
\item {\it Issue\#212 (casem)}
\begin{description}
\item [Subject:] Name will print garbage information.
\item [Criticality:] Hi:{\it donthi} Med:{\it } Low:{\it casem} None:{\it horigan}
\item [Suggested-by:] Me:{\it donthi} Other-and-me:{\it casem} Other:{\it horigan}
\item [Confidence-level:] Hi:{\it } Med:{\it casem,donthi} Low:{\it } None:{\it horigan}
\item [Source-node:] Driver::print

\item [Lines:] 7

\item [Description:] This highlighted name will print an address
of the array name.
\end{description}
\item {\it Issue\#216 (casem)}
\begin{description}
\item [Subject:] This is not a member function and is not declared
         as part of the class.
\item [Criticality:] Hi:{\it horigan} Med:{\it } Low:{\it donthi} None:{\it casem}
\item [Suggested-by:] Me:{\it horigan} Other-and-me:{\it } Other:{\it donthi,casem}
\item [Confidence-level:] Hi:{\it horigan} Med:{\it } Low:{\it casem,donthi} None:{\it }
\item [Source-node:] print\_menu

\item [Lines:] 1

\item [Description:] In C++ style, all funtions should be in a 
class declaration and a member function of the class.
\end{description}
\item {\it Issue\#220 (casem)}
\begin{description}
\item [Subject:] Not a member function.
\item [Criticality:] Hi:{\it horigan} Med:{\it } Low:{\it casem,donthi} None:{\it }
\item [Suggested-by:] Me:{\it horigan} Other-and-me:{\it casem,donthi} Other:{\it }
\item [Confidence-level:] Hi:{\it horigan} Med:{\it casem,donthi} Low:{\it } None:{\it }
\item [Source-node:] print\_error

\item [Lines:] 1

\item [Description:] In C++ style functions should be a part of
the class declaration, and a member function.
\end{description}
\end{enumerate}
\section{Review Metrics}
\begin{table}[hb]
\begin{center}
\begin{tabular}{|l|l|l|l|}
\hline
Participant & Start-time & End-time & Total Busy-time \\
\hline
donthi & Dec 02, 1994 15:23:47 & Dec 02, 1994 16:27:06 & 1:0:6 \\
horigan & Dec 02, 1994 15:11:24 & Dec 02, 1994 16:27:04 & 1:6:9 \\
casem & Dec 02, 1994 15:10:51 & Dec 02, 1994 16:27:04 & 1:16:13 \\
\hline
\end{tabular}
\end{center}
\caption{Review Session}
\end{table}


\begin{table}[hb]
\begin{center}
\begin{tabular}{|l|l|l|l|}
\hline
Source & donthi & horigan & casem\\
\hline
(176)Constant & 20 & 249 & 251\\
(192)main & 576 & 618 & 623\\
(178)Driver & 459 & 911 & 916\\
(180)Driver::Driver() & 554 & 378 & 739\\
(182)Driver::\~Driver() & 336 & 336 & 338\\
(184)Driver::change & 391 & 230 & 446\\
(186)Driver::print & 538 & 544 & 546\\
(188)print\_error & 177 & 177 & 179\\
(190)print\_menu & 326 & 335 & 338\\
\hline
\end{tabular}
\end{center}
\caption{Review Time}
\end{table}


\begin{table}[hb]
\begin{center}
\begin{tabular}{|l|l|l|}
\hline
Source node & Issue node & OK\\
\hline
(176)Constant & & \\
(192)main & & \\
(178)Driver & \#200 (=1)& \\
(180)Driver::Driver() & \#204 (=1)& \\
(182)Driver::\~Driver() & \#208 (=1) & \#208 \\
(184)Driver::change & & \\
(186)Driver::print & \#212 (=1)& \\
(188)print\_error & \#220 (=1)& \\
(190)print\_menu & \#216 (=1)& \\
\hline
\end{tabular}
\caption{Source node v.s Issue node}
\end{center}
\end{table}

%%%\end{document}

%%% \documentstyle[11pt,/group/csdl/tex/definemargins,
%%%                        /group/csdl/tex/lmacros]{article} 
%%% 
%%%           \begin{document}
%%%           \begin{center}
%%%           {\large\bf CSRS Experiment Results}\\
%%%           \end{center}
\chapter {CSRS Experiment Results: Group6(EGSM)}	  
\small

\begin{description}
\item [Method:] EGSM
\item [Group:] Group6
\item [Source:] yard
\item [Participants:] awong (Moderator), ccheung (Reviewer), gnakamur (Presenter)
\end{description}
\section{Issue Lists}
\begin{enumerate}
\item {\it Issue\#238 (gnakamur)}
\begin{description}
\item [Subject:] Uninitialized variable userString
\item [Criticality:] Hi:{\it } Med:{\it ccheung,gnakamur} Low:{\it awong} None:{\it }
\item [Suggested-by:] Me:{\it } Other-and-me:{\it ccheung,gnakamur} Other:{\it awong}
\item [Confidence-level:] Hi:{\it gnakamur,ccheung} Med:{\it } Low:{\it awong} None:{\it }
\item [Source-node:] driver::create

\item [Lines:] 13

\item [Description:] userString is used before being initialized.
Memory was allocated but not set yet.
\end{description}
\item {\it Issue\#242 (gnakamur)}
\begin{description}
\item [Subject:] Uninitalized variable route
\item [Criticality:] Hi:{\it } Med:{\it ccheung} Low:{\it awong,gnakamur} None:{\it }
\item [Suggested-by:] Me:{\it gnakamur} Other-and-me:{\it ccheung} Other:{\it awong}
\item [Confidence-level:] Hi:{\it } Med:{\it gnakamur,ccheung,awong} Low:{\it } None:{\it }
\item [Source-node:] bus::create

\item [Lines:] 6

\item [Description:] route is used before being set to an illegal
value to guarantee that the user is prompt for the new value.
\end{description}
\item {\it Issue\#246 (gnakamur)}
\begin{description}
\item [Subject:] Information inherited from the vehicle class not printed.
\item [Criticality:] Hi:{\it } Med:{\it } Low:{\it gnakamur,awong,ccheung} None:{\it }
\item [Suggested-by:] Me:{\it } Other-and-me:{\it gnakamur,ccheung} Other:{\it awong}
\item [Confidence-level:] Hi:{\it } Med:{\it gnakamur,awong} Low:{\it ccheung} None:{\it }
\item [Source-node:] bus::print

\item [Lines:] 3-7

\item [Description:] The print function from the vehicle class is
not called to print out the info inherited from that class.
\end{description}
\item {\it Issue\#250 (gnakamur)}
\begin{description}
\item [Subject:] Unallocated memory for variable region.
\item [Criticality:] Hi:{\it gnakamur,awong,ccheung} Med:{\it } Low:{\it } None:{\it }
\item [Suggested-by:] Me:{\it } Other-and-me:{\it ccheung,gnakamur,awong} Other:{\it }
\item [Confidence-level:] Hi:{\it gnakamur,ccheung,awong} Med:{\it } Low:{\it } None:{\it }
\item [Source-node:] van::van

\item [Lines:] 4

\item [Description:] The variable region is set without allocating
memory for it.
\end{description}
\item {\it Issue\#254 (gnakamur)}
\begin{description}
\item [Subject:] Did not deallocate memory for region.
\item [Criticality:] Hi:{\it ccheung} Med:{\it gnakamur} Low:{\it awong} None:{\it }
\item [Suggested-by:] Me:{\it ccheung} Other-and-me:{\it gnakamur,awong} Other:{\it }
\item [Confidence-level:] Hi:{\it gnakamur,awong,ccheung} Med:{\it } Low:{\it } None:{\it }
\item [Source-node:] van::\~van

\item [Lines:] 3-5

\item [Description:] The memory for variable regionis not
deallocated leading to a memory leak.
\end{description}
\item {\it Issue\#258 (gnakamur)}
\begin{description}
\item [Subject:] Use of regionsString before it is initialized.
\item [Criticality:] Hi:{\it } Med:{\it gnakamur,awong,ccheung} Low:{\it } None:{\it }
\item [Suggested-by:] Me:{\it } Other-and-me:{\it gnakamur,ccheung} Other:{\it awong}
\item [Confidence-level:] Hi:{\it } Med:{\it gnakamur,ccheung} Low:{\it awong} None:{\it }
\item [Source-node:] van::create

\item [Lines:] 11

\item [Description:] regionString is not initialized leading to
unpredictable results.
\end{description}
\item {\it Issue\#262 (gnakamur)}
\begin{description}
\item [Subject:] Illegal use of char pointer assignments.
\item [Criticality:] Hi:{\it gnakamur,ccheung} Med:{\it awong} Low:{\it } None:{\it }
\item [Suggested-by:] Me:{\it awong} Other-and-me:{\it gnakamur,ccheung} Other:{\it }
\item [Confidence-level:] Hi:{\it gnakamur,ccheung} Med:{\it awong} Low:{\it } None:{\it }
\item [Source-node:] van::create

\item [Lines:] 

\item [Description:] region is a pointer to a char array and
assignments should be made with strcpy().
\end{description}
\item {\it Issue\#264 (gnakamur)}
\begin{description}
\item [Subject:] Allocation of memory for region.
\item [Criticality:] Hi:{\it gnakamur,ccheung} Med:{\it } Low:{\it awong} None:{\it }
\item [Suggested-by:] Me:{\it ccheung} Other-and-me:{\it gnakamur,awong} Other:{\it }
\item [Confidence-level:] Hi:{\it ccheung} Med:{\it gnakamur,awong} Low:{\it } None:{\it }
\item [Source-node:] van::create

\item [Lines:] 16

\item [Description:] Since region is a pointer to a char array
memory has to be allocated and deallocated for it before its use.
\end{description}
\item {\it Issue\#268 (gnakamur)}
\begin{description}
\item [Subject:] Possible use of delete on uninitialized pointer.
\item [Criticality:] Hi:{\it } Med:{\it ccheung,gnakamur} Low:{\it awong} None:{\it }
\item [Suggested-by:] Me:{\it gnakamur} Other-and-me:{\it ccheung} Other:{\it awong}
\item [Confidence-level:] Hi:{\it gnakamur} Med:{\it } Low:{\it ccheung,awong} None:{\it }
\item [Source-node:] yard::\~yard

\item [Lines:] 5

\item [Description:] All vehicles in the yard are deleted even
though there may only be a few vehicles in it.
\end{description}
\item {\it Issue\#274 (gnakamur)}
\begin{description}
\item [Subject:] Incorrect while condition.
\item [Criticality:] Hi:{\it } Med:{\it gnakamur,awong,ccheung} Low:{\it } None:{\it }
\item [Suggested-by:] Me:{\it } Other-and-me:{\it gnakamur,awong,ccheung} Other:{\it }
\item [Confidence-level:] Hi:{\it gnakamur} Med:{\it } Low:{\it awong,ccheung} None:{\it }
\item [Source-node:] yard::remove

\item [Lines:] 7

\item [Description:] should not allow vehicle number of 0 to pass.
valid numbers are from 1 to no\_of\_veh.
\end{description}
\item {\it Issue\#280 (gnakamur)}
\begin{description}
\item [Subject:] incorrect printing of vehicle id.
\item [Criticality:] Hi:{\it } Med:{\it ccheung} Low:{\it gnakamur,awong} None:{\it }
\item [Suggested-by:] Me:{\it } Other-and-me:{\it gnakamur,ccheung,awong} Other:{\it }
\item [Confidence-level:] Hi:{\it } Med:{\it gnakamur,ccheung} Low:{\it awong} None:{\it }
\item [Source-node:] yard::print

\item [Lines:] 9

\item [Description:] vehicle id starts at 1 (not 0).
\end{description}
\item {\it Issue\#286 (gnakamur)}
\begin{description}
\item [Subject:] use of mybus before initialization.
\item [Criticality:] Hi:{\it gnakamur} Med:{\it } Low:{\it awong,ccheung} None:{\it }
\item [Suggested-by:] Me:{\it gnakamur} Other-and-me:{\it ccheung} Other:{\it awong}
\item [Confidence-level:] Hi:{\it gnakamur} Med:{\it ccheung} Low:{\it awong} None:{\it }
\item [Source-node:] main

\item [Lines:] 31

\item [Description:] mybus not initialized with new operator before
its use.
\end{description}
\item {\it Issue\#290 (gnakamur)}
\begin{description}
\item [Subject:] use of myvan before initialization.
\item [Criticality:] Hi:{\it gnakamur} Med:{\it } Low:{\it awong,ccheung} None:{\it }
\item [Suggested-by:] Me:{\it } Other-and-me:{\it awong,gnakamur} Other:{\it ccheung}
\item [Confidence-level:] Hi:{\it gnakamur} Med:{\it } Low:{\it awong,ccheung} None:{\it }
\item [Source-node:] main

\item [Lines:] 35

\item [Description:] myvan was not initialized using new before its use.
\end{description}
\item {\it Issue\#294 (gnakamur)}
\begin{description}
\item [Subject:] use of mybus before initialization
\item [Criticality:] Hi:{\it gnakamur} Med:{\it } Low:{\it awong,ccheung} None:{\it }
\item [Suggested-by:] Me:{\it } Other-and-me:{\it gnakamur,ccheung} Other:{\it awong}
\item [Confidence-level:] Hi:{\it gnakamur} Med:{\it } Low:{\it awong,ccheung} None:{\it }
\item [Source-node:] main

\item [Lines:] 45

\item [Description:] new not called for mybus before its use.
\end{description}
\item {\it Issue\#298 (gnakamur)}
\begin{description}
\item [Subject:] use of myvan before initialization.
\item [Criticality:] Hi:{\it gnakamur} Med:{\it } Low:{\it awong,ccheung} None:{\it }
\item [Suggested-by:] Me:{\it } Other-and-me:{\it gnakamur,ccheung} Other:{\it awong}
\item [Confidence-level:] Hi:{\it gnakamur} Med:{\it } Low:{\it awong,ccheung} None:{\it }
\item [Source-node:] main

\item [Lines:] 49

\item [Description:] new operator not called for myvan before its use.
\end{description}
\end{enumerate}
\section{Review Metrics}
\begin{table}[hb]
\begin{center}
\begin{tabular}{|l|l|l|l|}
\hline
Participant & Start-time & End-time & Total Busy-time \\
\hline
awong & Dec 05, 1994 10:45:02 & Dec 05, 1994 12:19:43 & 1:13:33 \\
ccheung & Dec 05, 1994 10:44:32 & Dec 05, 1994 12:19:30 & 0:56:35 \\
gnakamur & Dec 05, 1994 10:44:32 & Dec 05, 1994 12:17:30 & 1:28:13 \\
\hline
\end{tabular}
\end{center}
\caption{Review Session}
\end{table}


\begin{table}[hb]
\begin{center}
\begin{tabular}{|l|l|l|l|}
\hline
Source & awong & ccheung & gnakamur\\
\hline
(176)Constant & 115 & 114 & 116\\
(192)driver::create & 205 & 207 & 443\\
(208)bus::create & 273 & 274 & 276\\
(224)yard::add & 117 & 0 & 118\\
(178)driver & 86 & 87 & 99\\
(194)driver::print & 24 & 23 & 26\\
(210)bus::print & 223 & 223 & 224\\
(226)yard::remove & 338 & 557 & 740\\
(180)vehicle & 239 & 243 & 252\\
(196)vehicle::vehicle & 20 & 22 & 23\\
(212)van::van & 118 & 118 & 119\\
(228)yard::print & 192 & 192 & 193\\
(182)bus & 103 & 104 & 114\\
(198)vehicle::\~vehicle & 8 & 8 & 10\\
(214)van::\~van & 110 & 110 & 111\\
(230)main & 442 & 452 & 454\\
(184)van & 79 & 53 & 89\\
(200)vehicle::create & 45 & 46 & 49\\
(216)van::create & 549 & 0 & 551\\
(186)yard & 115 & 22 & 109\\
(202)vehicle::print & 27 & 27 & 29\\
(218)van::print & 50 & 0 & 53\\
(188)driver::driver & 79 & 84 & 83\\
(204)bus::bus & 21 & 21 & 24\\
(220)yard::yard & 20 & 0 & 21\\
(190)driver::\~driver & 27 & 26 & 28\\
(206)bus::\~bus & 102 & 102 & 104\\
(222)yard::\~yard & 352 & 0 & 726\\
\hline
\end{tabular}
\end{center}
\caption{Review Time}
\end{table}


\begin{table}[hb]
\begin{center}
\begin{tabular}{|l|l|l|}
\hline
Source node & Issue node & OK\\
\hline
(176)Constant & & \\
(192)driver::create & \#238 (=1) & \#238 \\
(208)bus::create & \#242 (=1)& \\
(224)yard::add & & \\
(178)driver & & \\
(194)driver::print & & \\
(210)bus::print & \#246 (=1)& \#246\\
(226)yard::remove & \#274 (=1)& \#274\\
(180)vehicle & & \\
(196)vehicle::vehicle & & \\
(212)van::van & \#250 (=1)& \\
(228)yard::print & \#280 (=1)& \\
(182)bus & & \\
(198)vehicle::\~vehicle & & \\
(214)van::\~van & \#254 (=1)& \#254\\
(230)main & \#286,\#290,\#294,\#298 (=4)& \#286,\#290,\#294,\#298 \\
(184)van & & \\
(200)vehicle::create & & \\
(216)van::create & \#258,\#262,\#264 (=3)& \#258,\#264\\
(186)yard & & \\
(202)vehicle::print & & \\
(218)van::print & & \\
(188)driver::driver & & \\
(204)bus::bus & & \\
(220)yard::yard & & \\
(190)driver::\~driver & & \\
(206)bus::\~bus & & \\
(222)yard::\~yard & \#268 (=1)& \#268\\
\hline
\end{tabular}
\caption{Source node v.s Issue node}
\end{center}
\end{table}

%%%\end{document}



%%% \documentstyle[11pt,/group/csdl/tex/definemargins,
%%%                        /group/csdl/tex/lmacros]{article} 
%%% 
%%%           \begin{document}
%%%           \begin{center}
%%%           {\large\bf CSRS Experiment Results}\\
%%%           \end{center}
%%% 	  
\chapter {CSRS Experiment Results: Group7(EGSM)}
\small

\begin{description}
\item [Method:] EGSM
\item [Group:] Group7
\item [Source:] yard
\item [Participants:] sanaka (Reviewer), cklyoung (Moderator), gokimoto (Presenter)
\end{description}
\section{Issue Lists}
\begin{enumerate}
\item {\it Issue\#236 (gokimoto)}
\begin{description}
\item [Subject:] Loading a pointer into a string variable
\item [Criticality:] Hi:{\it } Med:{\it cklyoung,gokimoto} Low:{\it } None:{\it }
\item [Suggested-by:] Me:{\it gokimoto} Other-and-me:{\it } Other:{\it cklyoung}
\item [Confidence-level:] Hi:{\it } Med:{\it gokimoto} Low:{\it cklyoung} None:{\it }
\item [Source-node:] driver::create

\item [Lines:] 18

\item [Description:] 
\end{description}
\item {\it Issue\#240 (gokimoto)}
\begin{description}
\item [Subject:] Should use delete in destructor for driver
\item [Criticality:] Hi:{\it cklyoung,gokimoto} Med:{\it } Low:{\it } None:{\it }
\item [Suggested-by:] Me:{\it cklyoung} Other-and-me:{\it gokimoto} Other:{\it }
\item [Confidence-level:] Hi:{\it } Med:{\it gokimoto} Low:{\it } None:{\it }
\item [Source-node:] driver::create

\item [Lines:] 

\item [Description:] 
\end{description}
\item {\it Issue\#242 (gokimoto)}
\begin{description}
\item [Subject:] No delete in van destructor
\item [Criticality:] Hi:{\it cklyoung,gokimoto} Med:{\it } Low:{\it } None:{\it }
\item [Suggested-by:] Me:{\it cklyoung} Other-and-me:{\it gokimoto} Other:{\it }
\item [Confidence-level:] Hi:{\it } Med:{\it gokimoto,cklyoung} Low:{\it } None:{\it }
\item [Source-node:] van::create

\item [Lines:] 

\item [Description:] 
\end{description}
\item {\it Issue\#248 (gokimoto)}
\begin{description}
\item [Subject:] Format should match bus:print.
\item [Criticality:] Hi:{\it sanaka} Med:{\it cklyoung,gokimoto} Low:{\it } None:{\it }
\item [Suggested-by:] Me:{\it cklyoung} Other-and-me:{\it gokimoto} Other:{\it sanaka}
\item [Confidence-level:] Hi:{\it } Med:{\it cklyoung,gokimoto} Low:{\it sanaka} None:{\it }
\item [Source-node:] van::print

\item [Lines:] 5

\item [Description:] 
\end{description}
\item {\it Issue\#252 (gokimoto)}
\begin{description}
\item [Subject:] Memory leak.
\item [Criticality:] Hi:{\it sanaka,cklyoung,gokimoto} Med:{\it } Low:{\it } None:{\it }
\item [Suggested-by:] Me:{\it cklyoung} Other-and-me:{\it } Other:{\it sanaka,gokimoto}
\item [Confidence-level:] Hi:{\it sanaka} Med:{\it cklyoung} Low:{\it gokimoto} None:{\it }
\item [Source-node:] yard::\~yard

\item [Lines:] 

\item [Description:] Deleting pointer does not clear memory.
\end{description}
\item {\it Issue\#254 (gokimoto)}
\begin{description}
\item [Subject:] no\_of\_veh is not updated.
\item [Criticality:] Hi:{\it gokimoto,sanaka,cklyoung} Med:{\it } Low:{\it } None:{\it }
\item [Suggested-by:] Me:{\it cklyoung} Other-and-me:{\it } Other:{\it gokimoto}
\item [Confidence-level:] Hi:{\it sanaka,cklyoung} Med:{\it gokimoto} Low:{\it } None:{\it }
\item [Source-node:] yard::add

\item [Lines:] 4

\item [Description:] 
\end{description}
\item {\it Issue\#258 (gokimoto)}
\begin{description}
\item [Subject:] Allocates more than 10 spaces.
\item [Criticality:] Hi:{\it } Med:{\it sanaka} Low:{\it cklyoung,gokimoto} None:{\it }
\item [Suggested-by:] Me:{\it cklyoung} Other-and-me:{\it } Other:{\it sanaka,gokimoto}
\item [Confidence-level:] Hi:{\it cklyoung,gokimoto} Med:{\it } Low:{\it sanaka} None:{\it }
\item [Source-node:] yard

\item [Lines:] 

\item [Description:] 
\end{description}
\item {\it Issue\#262 (gokimoto)}
\begin{description}
\item [Subject:] Should decrement no\_of\_veh after for loop.
\item [Criticality:] Hi:{\it sanaka,gokimoto} Med:{\it cklyoung} Low:{\it } None:{\it }
\item [Suggested-by:] Me:{\it sanaka} Other-and-me:{\it } Other:{\it cklyoung,gokimoto}
\item [Confidence-level:] Hi:{\it sanaka} Med:{\it cklyoung,gokimoto} Low:{\it } None:{\it }
\item [Source-node:] yard::remove

\item [Lines:] 15-19

\item [Description:] 
\end{description}
\end{enumerate}
\section{Review Metrics}
\begin{table}[hb]
\begin{center}
\begin{tabular}{|l|l|l|l|}
\hline
Participant & Start-time & End-time & Total Busy-time \\
\hline
cklyoung & Dec 08, 1994 20:48:21 & Dec 08, 1994 21:15:13 & 0:22:2 \\
sanaka & Dec 08, 1994 20:29:40 & Dec 08, 1994 21:15:05 & 0:24:14 \\
gokimoto & Dec 08, 1994 19:56:13 & Dec 08, 1994 21:15:06 & 1:1:59 \\
\hline
\end{tabular}
\end{center}
\caption{Review Session}
\end{table}


\begin{table}[hb]
\begin{center}
\begin{tabular}{|l|l|l|l|}
\hline
Source & cklyoung & sanaka & gokimoto\\
\hline
(176)Constant & 0 & 8 & 156\\
(192)driver::create & 0 & 0 & 501\\
(208)bus::create & 0 & 0 & 228\\
(224)yard::add & 404 & 25 & 396\\
(178)driver & 0 & 3 & 90\\
(194)driver::print & 0 & 6 & 75\\
(210)bus::print & 0 & 219 & 61\\
(226)yard::remove & 345 & 255 & 516\\
(180)vehicle & 14 & 18 & 148\\
(196)vehicle::vehicle & 0 & 4 & 50\\
(212)van::van & 0 & 0 & 29\\
(228)yard::print & 9 & 160 & 165\\
(182)bus & 0 & 2 & 43\\
(198)vehicle::\~vehicle & 0 & 3 & 39\\
(214)van::\~van & 0 & 0 & 22\\
(230)main & 97 & 194 & 198\\
(184)van & 0 & 0 & 95\\
(200)vehicle::create & 0 & 0 & 81\\
(216)van::create & 0 & 13 & 189\\
(186)yard & 230 & 88 & 311\\
(202)vehicle::print & 0 & 43 & 33\\
(218)van::print & 2 & 137 & 32\\
(188)driver::driver & 0 & 3 & 42\\
(204)bus::bus & 0 & 3 & 50\\
(220)yard::yard & 0 & 82 & 0\\
(190)driver::\~driver & 0 & 16 & 29\\
(206)bus::\~bus & 0 & 1 & 31\\
(222)yard::\~yard & 0 & 4 & 0\\
\hline
\end{tabular}
\end{center}
\caption{Review Time}
\end{table}


\begin{table}[hb]
\begin{center}
\begin{tabular}{|l|l|l|}
\hline
Source node & Issue node & OK \\
\hline
(176)Constant & & \\
(192)driver::create & \#236,\#240 (=2)& \\
(208)bus::create & & \\
(224)yard::add & \#254 (=1)& \#254 \\
(178)driver & & \\
(194)driver::print & & \\
(210)bus::print & & \\
(226)yard::remove & \#262 (=1)& \\
(180)vehicle & & \\
(196)vehicle::vehicle & & \\
(212)van::van & & \\
(228)yard::print & & \\
(182)bus & & \\
(198)vehicle::\~vehicle & & \\
(214)van::\~van & & \\
(230)main & & \\
(184)van & & \\
(200)vehicle::create & & \\
(216)van::create & \#242 (=1)& \#242\\
(186)yard & \#258 (=1)& \\
(202)vehicle::print & & \\
(218)van::print & \#248 (=1)& \\
(188)driver::driver & & \\
(204)bus::bus & & \\
(220)yard::yard & & \\
(190)driver::\~driver & & \\
(206)bus::\~bus & & \\
(222)yard::\~yard & \#252 (=1)& \\
\hline
\end{tabular}
\caption{Source node v.s Issue node}
\end{center}
\end{table}

%%%\end{document}

%%% \documentstyle[11pt,/group/csdl/tex/definemargins,
%%%                        /group/csdl/tex/lmacros]{article} 
%%% 
%%%           \begin{document}
%%%           \begin{center}
%%%           {\large\bf CSRS Experiment Results}\\
%%%           \end{center}
%%% 	  
\chapter {CSRS Experiment Results: Group8(EGSM)}
\small

\begin{description}
\item [Method:] EGSM
\item [Group:] Group8
\item [Source:] driver
\item [Participants:] kapila (Moderator), syong (Reviewer), norio (Presenter), scheung (Reviewer)
\end{description}
\section{Issue Lists}
\begin{enumerate}
\item {\it Issue\#206 (norio)}
\begin{description}
\item [Subject:] The equal sign bothers us
\item [Criticality:] Hi:{\it scheung} Med:{\it norio,syong} Low:{\it } None:{\it kapila}
\item [Suggested-by:] Me:{\it } Other-and-me:{\it norio,scheung} Other:{\it syong,kapila}
\item [Confidence-level:] Hi:{\it } Med:{\it kapila,syong,norio,scheung} Low:{\it } None:{\it }
\item [Source-node:] Constant

\item [Lines:] 1

\item [Description:] This should be declared as:
 
  const int MAXLENGTH 256;
\end{description}
\item {\it Issue\#210 (norio)}
\begin{description}
\item [Subject:] Bad things happen when you don't use strcpy
\item [Criticality:] Hi:{\it scheung,norio,syong} Med:{\it } Low:{\it kapila} None:{\it }
\item [Suggested-by:] Me:{\it scheung} Other-and-me:{\it syong} Other:{\it norio,kapila}
\item [Confidence-level:] Hi:{\it norio,syong,scheung} Med:{\it } Low:{\it kapila} None:{\it }
\item [Source-node:] Driver::Driver()

\item [Lines:] 4

\item [Description:] This should be strcpy (name," ")
\end{description}
\item {\it Issue\#214 (norio)}
\begin{description}
\item [Subject:] Cannot use  (!name)
\item [Criticality:] Hi:{\it } Med:{\it kapila,syong,norio} Low:{\it scheung} None:{\it }
\item [Suggested-by:] Me:{\it kapila} Other-and-me:{\it norio} Other:{\it syong,scheung}
\item [Confidence-level:] Hi:{\it kapila} Med:{\it syong,scheung} Low:{\it norio} None:{\it }
\item [Source-node:] Driver::Driver()

\item [Lines:] 5

\item [Description:] Cannot use logical comparison symbol ! with the char
 pointer name to determine whether its a pointer pointing
 to an character array.
\end{description}
\item {\it Issue\#218 (norio)}
\begin{description}
\item [Subject:] Have to use strcpy for this....
\item [Criticality:] Hi:{\it syong,scheung,norio} Med:{\it } Low:{\it kapila} None:{\it }
\item [Suggested-by:] Me:{\it scheung} Other-and-me:{\it norio} Other:{\it syong,kapila}
\item [Confidence-level:] Hi:{\it scheung,kapila} Med:{\it syong,norio} Low:{\it } None:{\it }
\item [Source-node:] Driver::Driver()

\item [Lines:] 10

\item [Description:] This line should have a strcpy (name, "") if you want
 to copy an empty string to the memory location of
 character name pointer.
\end{description}
\item {\it Issue\#222 (norio)}
\begin{description}
\item [Subject:] Memory leakage
\item [Criticality:] Hi:{\it norio,syong,kapila} Med:{\it scheung} Low:{\it } None:{\it }
\item [Suggested-by:] Me:{\it norio} Other-and-me:{\it syong,kapila} Other:{\it scheung}
\item [Confidence-level:] Hi:{\it norio,syong,scheung,kapila} Med:{\it } Low:{\it } None:{\it }
\item [Source-node:] Driver::\~Driver()

\item [Lines:] 2-6

\item [Description:] Within the brackets there is suppose to be a:

 delete [] name; command to realocate the memory
  
 allocated by the pointer to the character array to name.
\end{description}
\item {\it Issue\#226 (norio)}
\begin{description}
\item [Subject:] Use srtcpy here...
\item [Criticality:] Hi:{\it syong,kapila,scheung,norio} Med:{\it } Low:{\it } None:{\it }
\item [Suggested-by:] Me:{\it } Other-and-me:{\it scheung,kapila,norio} Other:{\it syong}
\item [Confidence-level:] Hi:{\it syong,kapila,scheung,norio} Med:{\it } Low:{\it } None:{\it }
\item [Source-node:] Driver::change

\item [Lines:] 5

\item [Description:] In this instance, its suppose to be:
  
 strcpy (name, temp\_name);

 if you want to assign name a new name.
\end{description}
\item {\it Issue\#230 (norio)}
\begin{description}
\item [Subject:] This function doesn't do anything...
\item [Criticality:] Hi:{\it kapila} Med:{\it } Low:{\it norio} None:{\it syong,scheung}
\item [Suggested-by:] Me:{\it norio} Other-and-me:{\it kapila} Other:{\it syong,scheung}
\item [Confidence-level:] Hi:{\it norio,kapila} Med:{\it } Low:{\it syong,scheung} None:{\it }
\item [Source-node:] Driver::change

\item [Lines:] 1-2

\item [Description:] This function doesn't do anthing since the changes which
 are made disappears after the function is exited.

 Suppose to have no void in front of the function name.

 Suppose to have parameters passed by addresses not by
 pointers.
\end{description}
\item {\it Issue\#234 (norio)}
\begin{description}
\item [Subject:] It shouldn't be there....
\item [Criticality:] Hi:{\it scheung,norio,syong,kapila} Med:{\it } Low:{\it } None:{\it }
\item [Suggested-by:] Me:{\it scheung,kapila} Other-and-me:{\it norio,syong} Other:{\it }
\item [Confidence-level:] Hi:{\it scheung,norio,syong,kapila} Med:{\it } Low:{\it } None:{\it }
\item [Source-node:] print\_error

\item [Lines:] 5 1-8

\item [Description:] 1) If this function is suppose to return something then:
    the void should not be put in front of the function
    name, instead an int is suppose to be there.

 2) If this function is not suppose to return something,
    then the exit(1); is not suppose to be there due to
    the void.
\end{description}
\item {\it Issue\#238 (norio)}
\begin{description}
\item [Subject:] .... + 1
\item [Criticality:] Hi:{\it scheung} Med:{\it syong,norio,kapila} Low:{\it } None:{\it }
\item [Suggested-by:] Me:{\it kapila} Other-and-me:{\it } Other:{\it scheung,syong,norio}
\item [Confidence-level:] Hi:{\it scheung,norio,kapila} Med:{\it syong} Low:{\it } None:{\it }
\item [Source-node:] main

\item [Lines:] 26

\item [Description:] This should be:

 userinput\_driver = new char [MAXLENGTH + 1];
\end{description}
\item {\it Issue\#242 (norio)}
\begin{description}
\item [Subject:] Cannot use ! with this...
\item [Criticality:] Hi:{\it } Med:{\it kapila} Low:{\it syong,norio,scheung} None:{\it }
\item [Suggested-by:] Me:{\it norio} Other-and-me:{\it kapila} Other:{\it syong,scheung}
\item [Confidence-level:] Hi:{\it } Med:{\it kapila,norio} Low:{\it syong,scheung} None:{\it }
\item [Source-node:] main

\item [Lines:] 28

\item [Description:] This line uses an ! with a character pointer, which cannot
 be done.  (Cannot use a logical symbol ! to determine
 whether or not its a character pointer).

 Besides, why is this here?  Does it test just to see if
 the computer runs out of memory?
\end{description}
\item {\it Issue\#246 (norio)}
\begin{description}
\item [Subject:] Logic error...
\item [Criticality:] Hi:{\it kapila} Med:{\it syong} Low:{\it norio,scheung} None:{\it }
\item [Suggested-by:] Me:{\it scheung} Other-and-me:{\it syong,norio} Other:{\it kapila}
\item [Confidence-level:] Hi:{\it norio,kapila,scheung} Med:{\it syong} Low:{\it } None:{\it }
\item [Source-node:] main

\item [Lines:] 33

\item [Description:] This is a logic error in checking validity of the users
 inputed shift value.

 Suppose to be:

 while((userinput\_shift {\tt <} 1) || (userinput\_shift {\tt >} 3))
 \{

 \}
\end{description}
\item {\it Issue\#250 (norio)}
\begin{description}
\item [Subject:] Another Logical error...
\item [Criticality:] Hi:{\it } Med:{\it syong,kapila} Low:{\it norio,scheung} None:{\it }
\item [Suggested-by:] Me:{\it kapila} Other-and-me:{\it norio} Other:{\it syong,scheung}
\item [Confidence-level:] Hi:{\it norio,syong,kapila,scheung} Med:{\it } Low:{\it } None:{\it }
\item [Source-node:] main

\item [Lines:] 47

\item [Description:] This line doesn't capture 0 years of experience, ie:
 begining personel.

 Should be:

 while (userinput\_years\_experience ={\tt <} 0).
\end{description}
\end{enumerate}
\section{Review Metrics}
\begin{table}[hb]
\begin{center}
\begin{tabular}{|l|l|l|l|}
\hline
Participant & Start-time & End-time & Total Busy-time \\
\hline
kapila & Dec 01, 1994 11:38:34 & Dec 01, 1994 12:50:07 & 1:8:2 \\
scheung & Jan 06, 1995 11:44:28 & Jan 06, 1995 11:44:28 & 0:0:0 \\
syong & Dec 01, 1994 11:39:28 & Dec 01, 1994 12:50:14 & 1:4:12 \\
norio & Dec 01, 1994 11:37:58 & Dec 01, 1994 12:50:06 & 1:12:8 \\
\hline
\end{tabular}
\end{center}
\caption{Review Session}
\end{table}


\begin{table}[hb]
\begin{center}
\begin{tabular}{|l|l|l|l|l|}
\hline
Source & kapila & scheung & syong & norio\\
\hline
(192)print\_error & 259 & 0 & 388 & 411\\
(194)print\_menu & 196 & 0 & 45 & 42\\
(196)main & 1394 & 0 & 1243 & 1396\\
(180)Constant & 286 & 0 & 293 & 286\\
(182)Driver & 93 & 0 & 116 & 307\\
(184)Driver::Driver() & 842 & 0 & 725 & 853\\
(186)Driver::\~Driver() & 236 & 0 & 259 & 237\\
(188)Driver::change & 360 & 0 & 535 & 513\\
(190)Driver::print & 308 & 0 & 164 & 154\\
\hline
\end{tabular}
\end{center}
\caption{Review Time}
\end{table}


\begin{table}[hb]
\begin{center}
\begin{tabular}{|l|l|l|}
\hline
Source node & Issue node & OK \\
\hline
(192)print\_error & \#234 (=1)& \\
(194)print\_menu & & \\
(196)main & \#238,\#242,\#246,\#250 (=4)& \#238,\#246\\
(180)Constant & \#206 (=1)& \\
(182)Driver & & \\
(184)Driver::Driver() & \#210,\#214,\#218 (=3)& \#218\\
(186)Driver::\~Driver() & \#222 (=1)& \#222\\
(188)Driver::change & \#226,\#230 (=2)& \#226\\
(190)Driver::print & & \\
\hline
\end{tabular}
\caption{Source node v.s Issue node}
\end{center}
\end{table}

%%%\end{document}

%%% \documentstyle[11pt,/group/csdl/tex/definemargins,
%%%                        /group/csdl/tex/lmacros]{article} 
%%% 
%%%           \begin{document}
%%%           \begin{center}
%%%           {\large\bf CSRS Experiment Results}\\
%%%           \end{center}
%%%           \small 
\chapter {CSRS Experiment Results: Group1(EIAM)}
\small

\begin{description}
\item [Method:] EIAM
\item [Group:] Group1
\item [Source:] yard
\item [Participants:] ysiou (Reviewer), gchen (Reviewer), yanwang (Reviewer)
\end{description}
\section{Issue Lists}
\begin{enumerate}
\item {\it Issue\#230 (gchen)}
\begin{description}
\item [Subject:] Constructor
\item [Criticality:] Hi
\item [Confidence-level:] Hi
\item [Source-node:] driver::create

\item [Lines:] 4-7

\item [Description:] Constructor will do such job to initialize every instance.  The worst is the
each time when the create fuction is called, line 4-7 will delete the
previous instance
\end{description}
\item {\it Issue\#234 (yanwang)}
\begin{description}
\item [Subject:] wrong initialize
\item [Criticality:] Med
\item [Confidence-level:] Hi
\item [Source-node:] driver::create

\item [Lines:] 7

\item [Description:] Since in Constant, we define MIN\_YEAR = 0, if we initialize years\_exp = 0, we
can not make user input driver's years of experience. The prompt and input
are never reached!
\end{description}
\item {\it Issue\#238 (gchen)}
\begin{description}
\item [Subject:] virtual function
\item [Criticality:] Hi
\item [Confidence-level:] Hi
\item [Source-node:] vehicle::print

\item [Lines:] 1

\item [Description:] Before the function put the key word "virtual"
\end{description}
\item {\it Issue\#242 (gchen)}
\begin{description}
\item [Subject:] imcomplete print function
\item [Criticality:] Hi
\item [Confidence-level:] Hi
\item [Source-node:] bus::print

\item [Lines:] 

\item [Description:] the bus print function did not complete the
virtual function and can not print the information about the driver and the
information from the base class
\end{description}
\item {\it Issue\#244 (yanwang)}
\begin{description}
\item [Subject:] not good use char pointer
\item [Criticality:] Med
\item [Confidence-level:] Hi
\item [Source-node:] driver::create

\item [Lines:] 8

\item [Description:] In destructure, we do not delet any character array. But here, we create an
character array. Therefore, after copy usrstring to name, we need to delete
userString.
\end{description}
\item {\it Issue\#248 (gchen)}
\begin{description}
\item [Subject:] Container
\item [Criticality:] Hi
\item [Confidence-level:] Hi
\item [Source-node:] yard::add

\item [Lines:] 4

\item [Description:] this statement makes that j is always constant
and can not push the diferent instance of vehicles into the container parking
array
\end{description}
\item {\it Issue\#256 (gchen)}
\begin{description}
\item [Subject:] can not remove instance
\item [Criticality:] Hi
\item [Confidence-level:] Hi
\item [Source-node:] yard::remove

\item [Lines:] 18-19

\item [Description:] This can not delete a certain instance and
will always contain one instance
\end{description}
\item {\it Issue\#260 (gchen)}
\begin{description}
\item [Subject:] NEED MEMORY LOCATION FOR CLASS INSTANCE
\item [Criticality:] Hi
\item [Confidence-level:] Med
\item [Source-node:] main

\item [Lines:] 31

\item [Description:] need allocate memory block for each instance
of bus class.  should use new operator to assign the instance to a pointer
\end{description}
\item {\it Issue\#264 (gchen)}
\begin{description}
\item [Subject:] need new operator
\item [Criticality:] Hi
\item [Confidence-level:] Hi
\item [Source-node:] main

\item [Lines:] 35

\item [Description:] Need new operator to assign the instance to a
pointer to pass to the container.
\end{description}
\item {\it Issue\#268 (yanwang)}
\begin{description}
\item [Subject:] vehicle should not be created like this way
\item [Criticality:] Hi
\item [Confidence-level:] Med
\item [Source-node:] bus::create

\item [Lines:] 4

\item [Description:] when you want to use creat function in vheicle, you should not use this
fomular
\end{description}
\item {\it Issue\#272 (yanwang)}
\begin{description}
\item [Subject:] region is an character pointer
\item [Criticality:] Hi
\item [Confidence-level:] Hi
\item [Source-node:] van::\~van

\item [Lines:] 3-5

\item [Description:] always need delete the region character array
\end{description}
\item {\it Issue\#276 (gchen)}
\begin{description}
\item [Subject:] can not remove one instance of vehicle
\item [Criticality:] Hi
\item [Confidence-level:] Hi
\item [Source-node:] main

\item [Lines:] 39-40

\item [Description:] This function call will delete all yard instace, not one vihecle.
\end{description}
\item {\it Issue\#278 (yanwang)}
\begin{description}
\item [Subject:] badly use region pointer
\item [Criticality:] Hi
\item [Confidence-level:] Hi
\item [Source-node:] van::van

\item [Lines:] 4

\item [Description:] region is a character pointer not a string in class defination
\end{description}
\item {\it Issue\#282 (yanwang)}
\begin{description}
\item [Subject:] not good create
\item [Criticality:] Med
\item [Confidence-level:] Med
\item [Source-node:] van::create

\item [Lines:] 7

\item [Description:] need other way to create vehicle
\end{description}
\item {\it Issue\#288 (yanwang)}
\begin{description}
\item [Subject:] may not print out
\item [Criticality:] Hi
\item [Confidence-level:] Hi
\item [Source-node:] van::print

\item [Lines:] 6-7

\item [Description:] region is a pointer, not string
\end{description}
\item {\it Issue\#292 (yanwang)}
\begin{description}
\item [Subject:] need good initilize
\item [Criticality:] Med
\item [Confidence-level:] Hi
\item [Source-node:] yard::yard

\item [Lines:] 4

\item [Description:] the parking array which hold pointer should be initilized as null
\end{description}
\item {\it Issue\#296 (yanwang)}
\begin{description}
\item [Subject:] better way
\item [Criticality:] Low
\item [Confidence-level:] Hi
\item [Source-node:] yard::\~yard

\item [Lines:] 7

\item [Description:] use delete[] parking may be better
\end{description}
\item {\it Issue\#300 (yanwang)}
\begin{description}
\item [Subject:] need better way to handle
\item [Criticality:] Low
\item [Confidence-level:] Hi
\item [Source-node:] yard::add

\item [Lines:] 5-6

\item [Description:] if we use the remove in the program, it is o.k but user input trash\_veh means
nothing!
\end{description}
\item {\it Issue\#304 (gchen)}
\begin{description}
\item [Subject:] delete the array of pointers
\item [Criticality:] Hi
\item [Confidence-level:] Hi
\item [Source-node:] yard::\~yard

\item [Lines:] 4-8

\item [Description:] to delete the array of pointers need use
"delete [] parking"
\end{description}
\item {\it Issue\#308 (yanwang)}
\begin{description}
\item [Subject:] not good method
\item [Criticality:] Low
\item [Confidence-level:] Hi
\item [Source-node:] yard::remove

\item [Lines:] 14-18

\item [Description:] rearrange the array make bus id useless
\end{description}
\item {\it Issue\#312 (yanwang)}
\begin{description}
\item [Subject:] global const
\item [Criticality:] Hi
\item [Confidence-level:] Hi
\item [Source-node:] main

\item [Lines:] 2-3

\item [Description:] need involve the const declaration in Constant
\end{description}
\item {\it Issue\#316 (yanwang)}
\begin{description}
\item [Subject:] need virtual function
\item [Criticality:] Hi
\item [Confidence-level:] Hi
\item [Source-node:] yard

\item [Lines:] 13

\item [Description:] need virtual function to print out
\end{description}
\item {\it Issue\#320 (yanwang)}
\begin{description}
\item [Subject:] need declear new vehicle
\item [Criticality:] Hi
\item [Confidence-level:] Hi
\item [Source-node:] main

\item [Lines:] 35

\item [Description:] only pointer but need new instance mybus pointer means junk in its contents
\end{description}
\item {\it Issue\#326 (yanwang)}
\begin{description}
\item [Subject:] same as issue\#320
\item [Criticality:] 
\item [Confidence-level:] 
\item [Source-node:] main

\item [Lines:] 45-46

\item [Description:] 
\end{description}
\item {\it Issue\#330 (yanwang)}
\begin{description}
\item [Subject:] same as issue \#326
\item [Criticality:] Hi
\item [Confidence-level:] Hi
\item [Source-node:] main

\item [Lines:] 49

\item [Description:] 
\end{description}
\end{enumerate}
\section{Review Metrics}
\begin{table}[hb]
\begin{center}
\begin{tabular}{|l|l|l|l|}
\hline
Participant & Start-time & End-time & Total Busy-time \\
\hline
yanwang & Dec 05, 1994 16:22:52 & Dec 05, 1994 18:27:11 & 1:58:24 \\
gchen & Dec 05, 1994 16:25:31 & Dec 05, 1994 18:10:16 & 1:44:45 \\
ysiou & Jan 05, 1995 11:42:56 & Jan 05, 1995 11:42:56 & 0:0:0 \\
\hline
\end{tabular}
\end{center}
\caption{Review Session}
\end{table}


\begin{table}[hb]
\begin{center}
\begin{tabular}{|l|l|l|l|}
\hline
Source & yanwang & gchen & ysiou\\
\hline
(176)bus & 85 & 212 & 0\\
(192)vehicle::\~vehicle & 42 & 17 & 0\\
(208)van::\~van & 111 & 10 & 0\\
(224)main & 926 & 1881 & 0\\
(178)van & 135 & 67 & 0\\
(194)vehicle::create & 326 & 134 & 0\\
(210)van::create & 332 & 109 & 0\\
(180)yard & 248 & 161 & 0\\
(196)vehicle::print & 51 & 339 & 0\\
(212)van::print & 115 & 42 & 0\\
(182)driver::driver & 76 & 37 & 0\\
(198)bus::bus & 42 & 174 & 0\\
(214)yard::yard & 153 & 22 & 0\\
(184)driver::\~driver & 73 & 13 & 0\\
(200)bus::\~bus & 22 & 29 & 0\\
(216)yard::\~yard & 148 & 307 & 0\\
(170)Constant & 190 & 60 & 0\\
(186)driver::create & 1564 & 458 & 0\\
(202)bus::create & 406 & 97 & 0\\
(218)yard::add & 238 & 511 & 0\\
(172)driver & 189 & 75 & 0\\
(188)driver::print & 268 & 66 & 0\\
(204)bus::print & 63 & 260 & 0\\
(220)yard::remove & 498 & 556 & 0\\
(174)vehicle & 307 & 341 & 0\\
(190)vehicle::vehicle & 195 & 92 & 0\\
(206)van::van & 156 & 34 & 0\\
(222)yard::print & 85 & 101 & 0\\
\hline
\end{tabular}
\end{center}
\caption{Review Time}
\end{table}


\begin{table}[hb]
\begin{center}
\begin{tabular}{|l|l|l|l|l|}
\hline
Source & yanwang & gchen & ysiou & OK\\
\hline
(176)bus &  &  & & \\
(192)vehicle::\~vehicle &  &  & & \\
(208)van::\~van & \#272 (=1) &  & & \\
(224)main & \#312,\#320,\#326, & \#260,\#264, & & \#320=\#264,\#326,\\
          &  \#330 (=4)        & \#276 (=3) &  & \#330,\#260\\
(178)van &  &  & & \\
(194)vehicle::create &  &  & & \\
(210)van::create & \#282 (=1) &  & & \\
(180)yard & \#316 (=1) &  & & \\
(196)vehicle::print &  & \#238 (=1) & & \\
(212)van::print & \#288 (=1) &  & & \\
(182)driver::driver &  &  & & \\
(198)bus::bus &  &  & & \\
(214)yard::yard & \#292 (=1) &  & & \#292 \\
(184)driver::\~driver &  &  & & \\
(200)bus::\~bus &  &  & & \\
(216)yard::\~yard & \#296 (=1) & \#304 (=1) & & \\
(170)Constant &  &  & & \\
(186)driver::create & \#234,\#244 (=2) & \#230 (=1) & & \#234=\#230,\#244\\
(202)bus::create & \#268 (=1) &  & & \\
(218)yard::add & \#300 (=1) & \#248 (=1) & & \#248\\
(172)driver &  &  & & \\
(188)driver::print &  &  & & \\
(204)bus::print &  & \#242 (=1) & & \#242\\
(220)yard::remove & \#308 (=1) & \#256 (=1) & & \#256\\
(174)vehicle &  &  & & \\
(190)vehicle::vehicle &  &  & & \\
(206)van::van & \#278 (=1) &  & & \\
(222)yard::print &  &  & & \\
\hline
\end{tabular}
\caption{Source node v.s Issue node}
\end{center}
\end{table}

%%%\end{document}

%%% \documentstyle[11pt,/group/csdl/tex/definemargins,
%%%                        /group/csdl/tex/lmacros]{article} 
%%% 
%%%           \begin{document}
%%%           \begin{center}
%%%           {\large\bf CSRS Experiment Results}\\
%%%           \end{center}
%%%           \small 
\chapter {CSRS Experiment Results: Group2(EIAM)}	  
\small

\begin{description}
\item [Method:] EIAM
\item [Group:] Group2
\item [Source:] yard
\item [Participants:] cokumoto (Reviewer), subin (Reviewer), wagnerm (Reviewer)
\end{description}
\section{Issue Lists}
\begin{enumerate}
\item {\it Issue\#228 (subin)}
\begin{description}
\item [Subject:] destructor not required
\item [Criticality:] Med
\item [Confidence-level:] Hi
\item [Source-node:] driver

\item [Lines:] 13

\item [Description:] destructor is not required because
deallocatino of data types like String, int and float are automatically taken
care of.
\end{description}
\item {\it Issue\#232 (subin)}
\begin{description}
\item [Subject:] destructor not required
\item [Criticality:] Med
\item [Confidence-level:] Hi
\item [Source-node:] vehicle

\item [Lines:] 10 10

\item [Description:] destructor for vehicle class is not required
because dealloction of data types int, float, String are taken care of
automatically as in \~driver().
\end{description}
\item {\it Issue\#236 (subin)}
\begin{description}
\item [Subject:] deallocation of memory
\item [Criticality:] Hi
\item [Confidence-level:] Hi
\item [Source-node:] van::\~van

\item [Lines:] 

\item [Description:] the code for deallocation of memory is not
defined.  built in function delete[] can be used to deallocate character
array which is pointed by a pointer.
\end{description}
\item {\it Issue\#240 (subin)}
\begin{description}
\item [Subject:] why virtual?
\item [Criticality:] Hi
\item [Confidence-level:] Hi
\item [Source-node:] van

\item [Lines:] 9

\item [Description:] Van is a class.  Why is the destructor for van
class as virtual?  This is an error.
\end{description}
\item {\it Issue\#244 (subin)}
\begin{description}
\item [Subject:] incorrect initialization
\item [Criticality:] Hi
\item [Confidence-level:] Hi
\item [Source-node:] driver::driver

\item [Lines:] 7

\item [Description:] initialization of this line of code should be
0 not -1.
\end{description}
\item {\it Issue\#248 (subin)}
\begin{description}
\item [Subject:] over initialization
\item [Criticality:] Hi
\item [Confidence-level:] Hi
\item [Source-node:] driver::create

\item [Lines:] 4-7

\item [Description:] these lines are already initialized in driver
constructor, so there is no need to do so again in create function.
\end{description}
\item {\it Issue\#256 (wagnerm)}
\begin{description}
\item [Subject:] print should be virtual
\item [Criticality:] Hi
\item [Confidence-level:] Hi
\item [Source-node:] driver

\item [Lines:] 15

\item [Description:] 
\end{description}
\item {\it Issue\#260 (cokumoto)}
\begin{description}
\item [Subject:] String is not de-allocated
\item [Criticality:] Med
\item [Confidence-level:] Hi
\item [Source-node:] driver::\~driver

\item [Lines:] 4-5

\item [Description:] There is no destructor for the string name.
\end{description}
\item {\it Issue\#264 (wagnerm)}
\begin{description}
\item [Subject:] Prompting for user input should be done in main
\item [Criticality:] Hi
\item [Confidence-level:] Hi
\item [Source-node:] driver::create

\item [Lines:] 

\item [Description:] All the prompting of user input should be done
in the main program .. this allows for abstraction ,
\end{description}
\item {\it Issue\#266 (cokumoto)}
\begin{description}
\item [Subject:] No destructor for region
\item [Criticality:] Med
\item [Confidence-level:] Hi
\item [Source-node:] van::\~van

\item [Lines:] 

\item [Description:] Memory for the string region is not destroyed.
\end{description}
\item {\it Issue\#268 (wagnerm)}
\begin{description}
\item [Subject:] The vehicle should be virtual
\item [Criticality:] Hi
\item [Confidence-level:] Hi
\item [Source-node:] vehicle::print

\item [Lines:] 1-2

\item [Description:] 
\end{description}
\item {\it Issue\#272 (wagnerm)}
\begin{description}
\item [Subject:] regionString is never de-allicated .. Memory Leak !
\item [Criticality:] Hi
\item [Confidence-level:] Med
\item [Source-node:] van::create

\item [Lines:] 16

\item [Description:] 
\end{description}
\item {\it Issue\#276 (wagnerm)}
\begin{description}
\item [Subject:] Should be virtual
\item [Criticality:] Med
\item [Confidence-level:] Med
\item [Source-node:] van::print

\item [Lines:] 1-2

\item [Description:] 
\end{description}
\item {\it Issue\#280 (cokumoto)}
\begin{description}
\item [Subject:] Vehicle::print() is not called
\item [Criticality:] Med
\item [Confidence-level:] Hi
\item [Source-node:] bus::print

\item [Lines:] 

\item [Description:] Bus data is displayed, but the base class's
print function is not called.
\end{description}
\item {\it Issue\#282 (wagnerm)}
\begin{description}
\item [Subject:] To find error you must read route FIRST
\item [Criticality:] Hi
\item [Confidence-level:] Hi
\item [Source-node:] bus::create

\item [Lines:] 

\item [Description:] 
\end{description}
\end{enumerate}
\section{Review Metrics}
\begin{table}[hb]
\begin{center}
\begin{tabular}{|l|l|l|l|}
\hline
Participant & Start-time & End-time & Total Busy-time \\
\hline
wagnerm & Dec 07, 1994 10:39:52 & Dec 07, 1994 11:25:35 & 0:45:43 \\
subin & Dec 05, 1994 15:18:55 & Dec 05, 1994 16:50:33 & 1:21:23 \\
cokumoto & Dec 07, 1994 10:40:53 & Dec 07, 1994 11:57:33 & 1:8:2 \\
\hline
\end{tabular}
\end{center}
\caption{Review Session}
\end{table}


\begin{table}[hb]
\begin{center}
\begin{tabular}{|l|l|l|l|}
\hline
Source & wagnerm & subin & cokumoto\\
\hline
(224)main & 236 & 0 & 297\\
(208)van::\~van & 9 & 297 & 160\\
(192)vehicle::\~vehicle & 12 & 47 & 62\\
(176)bus & 86 & 295 & 208\\
(210)van::create & 209 & 0 & 115\\
(194)vehicle::create & 143 & 32 & 196\\
(178)van & 47 & 620 & 64\\
(212)van::print & 79 & 0 & 83\\
(196)vehicle::print & 107 & 28 & 110\\
(180)yard & 58 & 335 & 56\\
(214)yard::yard & 34 & 0 & 201\\
(198)bus::bus & 62 & 12 & 28\\
(182)driver::driver & 58 & 232 & 84\\
(216)yard::\~yard & 60 & 0 & 70\\
(200)bus::\~bus & 32 & 150 & 17\\
(184)driver::\~driver & 25 & 37 & 263\\
(218)yard::add & 207 & 213 & 163\\
(202)bus::create & 215 & 51 & 182\\
(186)driver::create & 319 & 449 & 338\\
(170)Constant & 119 & 148 & 114\\
(220)yard::remove & 122 & 381 & 140\\
(204)bus::print & 29 & 0 & 162\\
(188)driver::print & 45 & 36 & 125\\
(172)driver & 176 & 737 & 220\\
(222)yard::print & 62 & 0 & 162\\
(206)van::van & 11 & 0 & 28\\
(190)vehicle::vehicle & 43 & 47 & 120\\
(174)vehicle & 84 & 648 & 259\\
\hline
\end{tabular}
\end{center}
\caption{Review Time}
\end{table}


\begin{table}[hb]
\begin{center}
\begin{tabular}{|l|l|l|l|l|}
\hline
Source & wagnerm & subin & cokumoto & OK\\
\hline
(224)main &  &  & & \\
(208)van::\~van &  & \#236 (=1) & \#266 (=1)& \#236=\#266 \\
(192)vehicle::\~vehicle &  &  & & \\
(176)bus &  &  & & \\
(210)van::create & \#272 (=1) &  & & \\
(194)vehicle::create &  &  & & \\
(178)van &  & \#240 (=1) & & \\
(212)van::print & \#276 (=1) &  & & \\
(196)vehicle::print & \#268 (=1) &  & & \\
(180)yard &  &  & & \\
(214)yard::yard &  &  & & \\
(198)bus::bus &  &  & & \\
(182)driver::driver &  & \#244 (=1) & & \\
(216)yard::\~yard &  &  & & \\
(200)bus::\~bus &  &  & & \\
(184)driver::\~driver &  &  & \#260 (=1)& \\
(218)yard::add &  &  & & \\
(202)bus::create & \#282 (=1) &  & & \\
(186)driver::create & \#264 (=1) & \#248 (=1) & & \\
(170)Constant &  &  & & \\
(220)yard::remove &  &  & & \\
(204)bus::print &  &  & \#280 (=1)& \#280\\
(188)driver::print &  &  & & \\
(172)driver & \#256 (=1) & \#228 (=1) & & \\
(222)yard::print &  &  & & \\
(206)van::van &  &  & & \\
(190)vehicle::vehicle &  &  & & \\
(174)vehicle &  & \#232 (=1) & & \\
\hline
\end{tabular}
\caption{Source node v.s Issue node}
\end{center}
\end{table}

%%%\end{document}

%%% \documentstyle[11pt,/group/csdl/tex/definemargins,
%%%                        /group/csdl/tex/lmacros]{article} 
%%% 
%%%           \begin{document}
%%%           \begin{center}
%%%           {\large\bf CSRS Experiment Results}\\
%%%           \end{center}
%%%           \small 
%%% 	  
\chapter {CSRS Experiment Results: Group3(EIAM)}	  
\small

\begin{description}
\item [Method:] EIAM
\item [Group:] Group3
\item [Source:] driver
\item [Participants:] wongcheu (Reviewer), sko (Reviewer), kawak (Reviewer)
\end{description}
\section{Issue Lists}
\begin{enumerate}
\item {\it Issue\#194 (kawak)}
\begin{description}
\item [Subject:] no need = or int?
\item [Criticality:] Med
\item [Confidence-level:] Low
\item [Source-node:] Constant

\item [Lines:] 1

\item [Description:] I don't think you need a equal sign or type
 declaration in constants.
\end{description}
\item {\it Issue\#198 (sko)}
\begin{description}
\item [Subject:] Memory leak, no deallocation of memory.
\item [Criticality:] Hi
\item [Confidence-level:] Hi
\item [Source-node:] Driver::\~Driver()

\item [Lines:] 

\item [Description:] Since the constructor allocates memory the destructor should 
deallocate the memory upon exiting the scope.
\end{description}
\item {\it Issue\#202 (kawak)}
\begin{description}
\item [Subject:] Destructor does nothing
\item [Criticality:] Low
\item [Confidence-level:] Hi
\item [Source-node:] Driver::\~Driver()

\item [Lines:] 

\item [Description:] Descructor does nothing in program.  Should remove memory from name.
\end{description}
\item {\it Issue\#204 (wongcheu)}
\begin{description}
\item [Subject:] memory leak!
\item [Criticality:] Hi
\item [Confidence-level:] Hi
\item [Source-node:] Driver::Driver()

\item [Lines:] 10

\item [Description:] the memory just allocated to store the name has
just been lost, because this line points the pointer elsewhere removeing all
references to the memory. It should read, *name = "";.
\end{description}
\item {\it Issue\#208 (sko)}
\begin{description}
\item [Subject:] Assignment error,
\item [Criticality:] Hi
\item [Confidence-level:] Hi
\item [Source-node:] Driver::change

\item [Lines:] 5

\item [Description:] the line should be: strncpy(name, temp\_name);
\end{description}
\item {\it Issue\#212 (kawak)}
\begin{description}
\item [Subject:] Logic problems
\item [Criticality:] Low
\item [Confidence-level:] Hi
\item [Source-node:] main

\item [Lines:] 33-37

\item [Description:] Error message will display when input is within baoundaries.
\end{description}
\item {\it Issue\#216 (wongcheu)}
\begin{description}
\item [Subject:] There should be a delete!
\item [Criticality:] Hi
\item [Confidence-level:] Hi
\item [Source-node:] Driver::\~Driver()

\item [Lines:] 

\item [Description:] since there was a new in the constructor, there
should be a delete in the destructor to remove the memory.  This line should
be in there: delete [] name.
\end{description}
\item {\it Issue\#218 (kawak)}
\begin{description}
\item [Subject:] Can't do it this way ?
\item [Criticality:] Med
\item [Confidence-level:] Low
\item [Source-node:] Driver::change

\item [Lines:] 5

\item [Description:] I don't think you can just assign the name like this because the binding may
be off.  I think you have to use strcpy.
\end{description}
\item {\it Issue\#220 (wongcheu)}
\begin{description}
\item [Subject:] another memory leak!
\item [Criticality:] Hi
\item [Confidence-level:] Hi
\item [Source-node:] Driver::change

\item [Lines:] 5

\item [Description:] this line should read strcpy(name, temp\_name);
And don't forget to \#include {\tt <}string.h{\tt >}.
\end{description}
\item {\it Issue\#226 (wongcheu)}
\begin{description}
\item [Subject:] not good
\item [Criticality:] Low
\item [Confidence-level:] Hi
\item [Source-node:] Driver::change

\item [Lines:] 6-8

\item [Description:] There should be some error checking at the 
values of the arguments to make sure that they are valid.  For example, what
if temp\_experience == -2, INVALID! etc ...
\end{description}
\item {\it Issue\#236 (wongcheu)}
\begin{description}
\item [Subject:] Function never use, and might possibly be redefined elsewhere
\item [Criticality:] Low
\item [Confidence-level:] Med
\item [Source-node:] print\_error

\item [Lines:] 1-6

\item [Description:] This function is never used by the class, and is extranuouse, 
and error prone, and the client program might declare a similar function.
\end{description}
\item {\it Issue\#248 (wongcheu)}
\begin{description}
\item [Subject:] Forgetting the NULL
\item [Criticality:] Low
\item [Confidence-level:] Med
\item [Source-node:] main

\item [Lines:] 26

\item [Description:] It should be MAXLENGTH + 1, where +1 is for the 
NULL, as was the convention set back at the constructor
\end{description}
\item {\it Issue\#250 (sko)}
\begin{description}
\item [Subject:] Forgot carige return
\item [Criticality:] Med
\item [Confidence-level:] Med
\item [Source-node:] main

\item [Lines:] 26

\item [Description:] should have bee: [MAXLENGTH+1].
\end{description}
\item {\it Issue\#256 (kawak)}
\begin{description}
\item [Subject:] Notes
\item [Criticality:] Low
\item [Confidence-level:] Low
\item [Source-node:] main

\item [Lines:] 

\item [Description:] Not really errors, but I need more issues!!


You don't need all the voids for the functions.

If statement in constructor should be terminated, but it doesn't crash the
program the way it is written, and it works the same way with it.
\end{description}
\item {\it Issue\#260 (wongcheu)}
\begin{description}
\item [Subject:] Another memory leak
\item [Criticality:] Hi
\item [Confidence-level:] Hi
\item [Source-node:] main

\item [Lines:] 69

\item [Description:] This delete is only done once, outside the while
loop, while the new may be dones many times inside the loop.
\end{description}
\end{enumerate}
\section{Review Metrics}
\begin{table}[hb]
\begin{center}
\begin{tabular}{|l|l|l|l|}
\hline
Participant & Start-time & End-time & Total Busy-time \\
\hline
kawak & Nov 28, 1994 15:31:46 & Nov 28, 1994 16:12:50 & 0:41:4 \\
sko & Nov 28, 1994 15:31:36 & Nov 28, 1994 16:11:03 & 0:39:27 \\
wongcheu & Nov 28, 1994 15:31:58 & Nov 28, 1994 16:11:47 & 0:39:49 \\
\hline
\end{tabular}
\end{center}
\caption{Review Session}
\end{table}


\begin{table}[hb]
\begin{center}
\begin{tabular}{|l|l|l|l|}
\hline
Source & kawak & sko & wongcheu\\
\hline
(176)Driver::\~Driver() & 90 & 392 & 150\\
(178)Driver::change & 218 & 186 & 300\\
(180)Driver::print & 112 & 71 & 196\\
(182)print\_error & 43 & 37 & 237\\
(184)print\_menu & 54 & 47 & 113\\
(170)Constant & 175 & 50 & 100\\
(186)main & 1003 & 1083 & 640\\
(172)Driver & 236 & 282 & 264\\
(174)Driver::Driver() & 443 & 202 & 353\\
\hline
\end{tabular}
\end{center}
\caption{Review Time}
\end{table}


\begin{table}[hb]
\begin{center}
\begin{tabular}{|l|l|l|l|l|}
\hline
Source & kawak & sko & wongcheu & OK\\
\hline
(176)Driver::\~Driver() & \#202 & \#198 & \#216 & 202=198=216\\
(178)Driver::change & \#218 & \#208 & \#220,\#226 & 218=208=220\\
(180)Driver::print &  &  & & \\
(182)print\_error &  &  & \#236 (=1)& \\
(184)print\_menu &  &  & & \\
(170)Constant & \#194 (=1) &  & & \\
(186)main & \#212,\#256 & \#250 & \#248,\#260 & 212,250=248,260 \\
(172)Driver &  &  & & \\
(174)Driver::Driver() &  &  & \#204 (=1)& 204\\
\hline
\end{tabular}
\caption{Source node v.s Issue node}
\end{center}
\end{table}

%%%\end{document}

%%% \documentstyle[11pt,/group/csdl/tex/definemargins,
%%%                        /group/csdl/tex/lmacros]{article} 
%%% 
%%%           \begin{document}
%%%           \begin{center}
%%%           {\large\bf CSRS Experiment Results}\\
%%%           \end{center}
%%%           \small 
%%% 	  
\chapter {CSRS Experiment Results: Group4(EIAM)}	  
\small

\begin{description}
\item [Method:] EIAM
\item [Group:] Group4
\item [Source:] driver
\item [Participants:] hyeung (Reviewer), savid (Reviewer), lnohara (Reviewer)
\end{description}
\section{Issue Lists}
\begin{enumerate}
\item {\it Issue\#192 (lnohara)}
\begin{description}
\item [Subject:] error in declaration of constant
\item [Criticality:] Med
\item [Confidence-level:] Med
\item [Source-node:] Constant

\item [Lines:] 1

\item [Description:] it should be const MAXLENGTH = 256;
\end{description}
\item {\it Issue\#196 (hyeung)}
\begin{description}
\item [Subject:] Missing void
\item [Criticality:] Hi
\item [Confidence-level:] Low
\item [Source-node:] Driver

\item [Lines:] 10

\item [Description:] Should have included a void command in front
of Driver();
\end{description}
\item {\it Issue\#204 (lnohara)}
\begin{description}
\item [Subject:] return
\item [Criticality:] Low
\item [Confidence-level:] Low
\item [Source-node:] Driver::change

\item [Lines:] 

\item [Description:] I'm not sure if a return is required at the
end of the function.
\end{description}
\item {\it Issue\#212 (hyeung)}
\begin{description}
\item [Subject:] Wrong method of assignment
\item [Criticality:] Hi
\item [Confidence-level:] Low
\item [Source-node:] Driver::change

\item [Lines:] 5

\item [Description:] Cannot assign temp\_name to name like that.
\end{description}
\item {\it Issue\#214 (lnohara)}
\begin{description}
\item [Subject:] logic error concerning shift input
\item [Criticality:] Low
\item [Confidence-level:] Med
\item [Source-node:] main

\item [Lines:] 33

\item [Description:] The wrong range is being tested.  The program
will print an error when userinput\_shift is between 0 and 4.  The error
should be when userinput\_shift is less than 0 and greater than 4.
\end{description}
\item {\it Issue\#224 (lnohara)}
\begin{description}
\item [Subject:] destructor
\item [Criticality:] Low
\item [Confidence-level:] Med
\item [Source-node:] main

\item [Lines:] 69

\item [Description:] I would put this in the destructor function.
\end{description}
\item {\it Issue\#228 (lnohara)}
\begin{description}
\item [Subject:] no delete for previous new
\item [Criticality:] Low
\item [Confidence-level:] Med
\item [Source-node:] Driver::\~Driver()

\item [Lines:] 3-6

\item [Description:] No delete here for the new command for the
driver name.  This would cause a memory leak.
\end{description}
\item {\it Issue\#232 (hyeung)}
\begin{description}
\item [Subject:] Unneccesary line
\item [Criticality:] Low
\item [Confidence-level:] Med
\item [Source-node:] main

\item [Lines:] 25

\item [Description:] This line is not needed as the cin.getline
does what is needed.
\end{description}
\item {\it Issue\#236 (hyeung)}
\begin{description}
\item [Subject:] Logic error
\item [Criticality:] Med
\item [Confidence-level:] Hi
\item [Source-node:] main

\item [Lines:] 33

\item [Description:] Currently, whenever the user types in a valid
response, it will run the next lines, stating that there is an error.  Should
have been while(userinput\_shift{\tt <}1 || userinput\_shift{\tt >}3)
\end{description}
\item {\it Issue\#240 (hyeung)}
\begin{description}
\item [Subject:] Confusing statement
\item [Criticality:] Low
\item [Confidence-level:] Hi
\item [Source-node:] main

\item [Lines:] 41

\item [Description:] Statement is currently confusing.  Should have
been "Error: payrate should be a positive real number."
\end{description}
\item {\it Issue\#244 (hyeung)}
\begin{description}
\item [Subject:] Confusing statement
\item [Criticality:] Low
\item [Confidence-level:] Hi
\item [Source-node:] main

\item [Lines:] 48-49

\item [Description:] Statement is confusing.  Should have been
'should be' instead of 'is'.
\end{description}
\item {\it Issue\#248 (hyeung)}
\begin{description}
\item [Subject:] Missing statement
\item [Criticality:] Med
\item [Confidence-level:] Hi
\item [Source-node:] Driver::\~Driver()

\item [Lines:] 3-5

\item [Description:] Missing delete statement will cause a memory
leak.  Should had a delete statement for the new for name.
\end{description}
\end{enumerate}
\section{Review Metrics}
\begin{table}[hb]
\begin{center}
\begin{tabular}{|l|l|l|l|}
\hline
Participant & Start-time & End-time & Total Busy-time \\
\hline
lnohara & Dec 06, 1994 15:12:36 & Dec 06, 1994 15:37:19 & 0:24:43 \\
savid & Jan 06, 1995 09:45:04 & Jan 06, 1995 09:45:04 & 0:0:0 \\
hyeung & Dec 06, 1994 15:12:40 & Dec 06, 1994 16:01:28 & 0:48:48 \\
\hline
\end{tabular}
\end{center}
\caption{Review Session}
\end{table}


\begin{table}[hb]
\begin{center}
\begin{tabular}{|l|l|l|l|}
\hline
Source & lnohara & savid & hyeung\\
\hline
(176)Driver::\~Driver() & 402 & 0 & 319\\
(178)Driver::change & 176 & 0 & 204\\
(180)Driver::print & 55 & 0 & 51\\
(182)print\_error & 11 & 0 & 74\\
(184)print\_menu & 25 & 0 & 295\\
(186)main & 462 & 0 & 1292\\
(170)Constant & 179 & 0 & 33\\
(172)Driver & 87 & 0 & 443\\
(174)Driver::Driver() & 42 & 0 & 96\\
\hline
\end{tabular}
\end{center}
\caption{Review Time}
\end{table}


\begin{table}[hb]
\begin{center}
\begin{tabular}{|l|l|l|l|l|}
\hline
Source & lnohara & savid & hyeung & OK\\
\hline
(176)Driver::\~Driver() & \#228 &  & \#248 & 228=248\\
(178)Driver::change & \#204 (=1) &  & \#212 (=1)& 212\\
(180)Driver::print &  &  & & \\
(182)print\_error &  &  & & \\
(184)print\_menu &  &  & & \\
(186)main & \#214, &  & \#232,\#236,      & 214=236, \\
          & \#224 (=2) & & \#240,\#244 (=4) & 224 \\
(170)Constant & \#192 (=1) &  & & \\
(172)Driver &  &  & \#196 (=1)& \\
(174)Driver::Driver() &  &  & & \\
\hline
\end{tabular}
\caption{Source node v.s Issue node}
\end{center}
\end{table}

%%%\end{document}

%%% \documentstyle[11pt,/group/csdl/tex/definemargins,
%%%                        /group/csdl/tex/lmacros]{article} 
%%% 
%%%           \begin{document}
%%%           \begin{center}
%%%           {\large\bf CSRS Experiment Results}\\
%%%           \end{center}
%%%           \small 
\chapter {CSRS Experiment Results: Group5(EIAM)}	  
\small
	  

\begin{description}
\item [Method:] EIAM
\item [Group:] Group5
\item [Source:] yard
\item [Participants:] horigan (Reviewer), donthi (Reviewer), casem (Reviewer)
\end{description}
\section{Issue Lists}
\begin{enumerate}
\item {\it Issue\#230 (casem)}
\begin{description}
\item [Subject:] misspelled word
\item [Criticality:] Low
\item [Confidence-level:] Med
\item [Source-node:] driver

\item [Lines:] 3

\item [Description:] Just a misspelled word in specification.
\end{description}
\item {\it Issue\#238 (donthi)}
\begin{description}
\item [Subject:] in-suffcient code for destructor class for vehicle
\item [Criticality:] Med
\item [Confidence-level:] Med
\item [Source-node:] vehicle::\~vehicle

\item [Lines:] 3

\item [Description:] there is insufficient information code fror
the destructor here.  There are not even comments explaining the absence of
any statements in this section of the code.
\end{description}
\item {\it Issue\#240 (casem)}
\begin{description}
\item [Subject:] no destructor code
\item [Criticality:] Med
\item [Confidence-level:] Hi
\item [Source-node:] van::\~van

\item [Lines:] 2-6

\item [Description:] There is no destructor for the constructor.
There is allocated memory that is not going to be freed.
\end{description}
\item {\it Issue\#246 (casem)}
\begin{description}
\item [Subject:] wrong evaluation??
\item [Criticality:] Low
\item [Confidence-level:] Low
\item [Source-node:] van::create

\item [Lines:] 11

\item [Description:] wrong evaluation of the strlen.  May skip the code following the evaluation.
\end{description}
\item {\it Issue\#250 (casem)}
\begin{description}
\item [Subject:] wrong deletion of yard
\item [Criticality:] Low
\item [Confidence-level:] Low
\item [Source-node:] yard::\~yard

\item [Lines:] 7

\item [Description:] wrong way to free space of an array.
\end{description}
\item {\it Issue\#252 (donthi)}
\begin{description}
\item [Subject:] the information provided in the destructor is
insufficient.
\item [Criticality:] Med
\item [Confidence-level:] Med
\item [Source-node:] bus::\~bus

\item [Lines:] 3

\item [Description:] the destructor for the class has no
information to interpret the purpose of the function ans does not explain why
there is no need for any code.
\end{description}
\item {\it Issue\#258 (casem)}
\begin{description}
\item [Subject:] decrementing an integer.
\item [Criticality:] Med
\item [Confidence-level:] Med
\item [Source-node:] yard::add

\item [Lines:] 7

\item [Description:] This is decrementing j.  May lead to crashing 
the program.
\end{description}
\item {\it Issue\#262 (casem)}
\begin{description}
\item [Subject:] incorrect evaluation
\item [Criticality:] Med
\item [Confidence-level:] Med
\item [Source-node:] yard::remove

\item [Lines:] 7-8

\item [Description:] This evaluation will skip the while loop
because trash\_veh is bound to 0.  also trash\_veh is less than no\_of\_veh which
will also lead to skipping the body of the while loop, and no letting the
user to input the vehicle to remove.
\end{description}
\item {\it Issue\#266 (donthi)}
\begin{description}
\item [Subject:] error causing statement in the getline statement.
\item [Criticality:] Med
\item [Confidence-level:] Med
\item [Source-node:] van::create

\item [Lines:] 15

\item [Description:] the getline function has the sytax which i
don't understand.  it has the upper limit of the string length as less than
MAXLEN, function however seems to give the lenght of the expected input
string to be MAXLEN
\end{description}
\item {\it Issue\#270 (donthi)}
\begin{description}
\item [Subject:] ambiguity on the print statement
\item [Criticality:] Med
\item [Confidence-level:] Med
\item [Source-node:] van::print

\item [Lines:] 5

\item [Description:] the print statement seemst to assume that the
program will print the string "region".
\end{description}
\item {\it Issue\#274 (donthi)}
\begin{description}
\item [Subject:] ambiguity reg. how vehicle is identified
\item [Criticality:] Med
\item [Confidence-level:] Low
\item [Source-node:] yard::print

\item [Lines:] 10

\item [Description:] 
\end{description}
\item {\it Issue\#278 (casem)}
\begin{description}
\item [Subject:] negative value for years\_exp
\item [Criticality:] Low
\item [Confidence-level:] Med
\item [Source-node:] driver::driver

\item [Lines:] 7

\item [Description:] years\_exp is bounded the value of -1.
A negative value is not good for evaluating starting from 0.
\end{description}
\item {\it Issue\#284 (horigan)}
\begin{description}
\item [Subject:] 
\item [Criticality:] 
\item [Confidence-level:] 
\item [Source-node:] 

\item [Lines:] 

\item [Description:] 
\end{description}
\item {\it Issue\#286 (horigan)}
\begin{description}
\item [Subject:] 
\item [Criticality:] 
\item [Confidence-level:] 
\item [Source-node:] 

\item [Lines:] 

\item [Description:] 
\end{description}
\item {\it Issue\#288 (horigan)}
\begin{description}
\item [Subject:] class declaration
\item [Criticality:] Hi
\item [Confidence-level:] Hi
\item [Source-node:] vehicle

\item [Lines:] 6-7 6

\item [Description:] The class "driver" should include like this :
 friend class vehicle....  to allow the class "vehicle" to use the class
"driver."
\end{description}
\item {\it Issue\#292 (horigan)}
\begin{description}
\item [Subject:] 
\item [Criticality:] 
\item [Confidence-level:] 
\item [Source-node:] 

\item [Lines:] 

\item [Description:] 
\end{description}
\item {\it Issue\#294 (horigan)}
\begin{description}
\item [Subject:] 
\item [Criticality:] 
\item [Confidence-level:] 
\item [Source-node:] 

\item [Lines:] 

\item [Description:] 
\end{description}
\item {\it Issue\#296 (horigan)}
\begin{description}
\item [Subject:] 
\item [Criticality:] 
\item [Confidence-level:] 
\item [Source-node:] 

\item [Lines:] 

\item [Description:] 
\end{description}
\item {\it Issue\#300 (horigan)}
\begin{description}
\item [Subject:] 
\item [Criticality:] 
\item [Confidence-level:] 
\item [Source-node:] 

\item [Lines:] 

\item [Description:] 
\end{description}
\item {\it Issue\#302 (horigan)}
\begin{description}
\item [Subject:] 
\item [Criticality:] 
\item [Confidence-level:] 
\item [Source-node:] 

\item [Lines:] 

\item [Description:] 
\end{description}
\item {\it Issue\#304 (horigan)}
\begin{description}
\item [Subject:] 
\item [Criticality:] 
\item [Confidence-level:] 
\item [Source-node:] 

\item [Lines:] 

\item [Description:] 
\end{description}
\item {\it Issue\#306 (horigan)}
\begin{description}
\item [Subject:] 
\item [Criticality:] 
\item [Confidence-level:] 
\item [Source-node:] vehicle::vehicle

\item [Lines:] 

\item [Description:] 
\end{description}
\item {\it Issue\#308 (horigan)}
\begin{description}
\item [Subject:] 
\item [Criticality:] 
\item [Confidence-level:] 
\item [Source-node:] 

\item [Lines:] 

\item [Description:] 
\end{description}
\item {\it Issue\#310 (horigan)}
\begin{description}
\item [Subject:] the contents of print
\item [Criticality:] Med
\item [Confidence-level:] Med
\item [Source-node:] bus::print

\item [Lines:] 4-8

\item [Description:] The print fuction should include the vehicle
print function
\end{description}
\item {\it Issue\#314 (horigan)}
\begin{description}
\item [Subject:] 
\item [Criticality:] 
\item [Confidence-level:] 
\item [Source-node:] van::create

\item [Lines:] 

\item [Description:] 
\end{description}
\item {\it Issue\#316 (horigan)}
\begin{description}
\item [Subject:] 
\item [Criticality:] 
\item [Confidence-level:] 
\item [Source-node:] 

\item [Lines:] 

\item [Description:] 
\end{description}
\item {\it Issue\#318 (horigan)}
\begin{description}
\item [Subject:] 
\item [Criticality:] 
\item [Confidence-level:] 
\item [Source-node:] yard::yard

\item [Lines:] 

\item [Description:] 
\end{description}
\item {\it Issue\#320 (horigan)}
\begin{description}
\item [Subject:] deleting method is wrong/ semicolon missing
\item [Criticality:] Hi
\item [Confidence-level:] Hi
\item [Source-node:] yard::\~yard

\item [Lines:] 5-8

\item [Description:] when delete the character array, just
delete []parking is enough and after for statement the semicolon is missing.
\end{description}
\item {\it Issue\#324 (horigan)}
\begin{description}
\item [Subject:] 
\item [Criticality:] 
\item [Confidence-level:] 
\item [Source-node:] yard::add

\item [Lines:] 

\item [Description:] 
\end{description}
\item {\it Issue\#326 (horigan)}
\begin{description}
\item [Subject:] semicolons are missing
\item [Criticality:] Hi
\item [Confidence-level:] Hi
\item [Source-node:] yard::add

\item [Lines:] 9-13

\item [Description:] after if and else statement, the semicolons
should be added.
\end{description}
\item {\it Issue\#330 (horigan)}
\begin{description}
\item [Subject:] deleting method
\item [Criticality:] Low
\item [Confidence-level:] Low
\item [Source-node:] yard::remove

\item [Lines:] 14

\item [Description:] I am not sure whether it is correct or not,
but ..
\end{description}
\item {\it Issue\#334 (horigan)}
\begin{description}
\item [Subject:] semicolon missing
\item [Criticality:] Hi
\item [Confidence-level:] Hi
\item [Source-node:] yard::print

\item [Lines:] 11

\item [Description:] after for statement, a semicolon is missing.
\end{description}
\item {\it Issue\#338 (horigan)}
\begin{description}
\item [Subject:] semicolons are missing
\item [Criticality:] Hi
\item [Confidence-level:] Hi
\item [Source-node:] main

\item [Lines:] 63-64

\item [Description:] after swith and while statements, the
semicolons are missing.
\end{description}
\end{enumerate}
\section{Review Metrics}
\begin{table}[hb]
\begin{center}
\begin{tabular}{|l|l|l|l|}
\hline
Participant & Start-time & End-time & Total Busy-time \\
\hline
casem & Dec 06, 1994 15:19:03 & Dec 06, 1994 16:19:19 & 1:0:16 \\
donthi & Dec 06, 1994 15:19:48 & Dec 06, 1994 16:21:11 & 1:1:23 \\
horigan & Dec 07, 1994 15:33:32 & Dec 07, 1994 17:55:51 & 2:15:19 \\
\hline
\end{tabular}
\end{center}
\caption{Review Session}
\end{table}


\begin{table}[hb]
\begin{center}
\begin{tabular}{|l|l|l|l|}
\hline
Source & casem & donthi & horigan\\
\hline
(176)bus & 112 & 77 & 175\\
(192)vehicle::\~vehicle & 19 & 255 & 24\\
(208)van::\~van & 134 & 10 & 24\\
(224)main & 313 & 213 & 562\\
(178)van & 119 & 30 & 183\\
(194)vehicle::create & 63 & 124 & 131\\
(210)van::create & 272 & 297 & 155\\
(180)yard & 206 & 102 & 578\\
(196)vehicle::print & 51 & 39 & 81\\
(212)van::print & 17 & 123 & 65\\
(182)driver::driver & 148 & 71 & 91\\
(198)bus::bus & 18 & 37 & 28\\
(214)yard::yard & 23 & 15 & 69\\
(184)driver::\~driver & 20 & 12 & 24\\
(200)bus::\~bus & 13 & 207 & 9\\
(216)yard::\~yard & 152 & 28 & 310\\
(170)Constant & 63 & 38 & 154\\
(186)driver::create & 53 & 384 & 404\\
(202)bus::create & 625 & 64 & 114\\
(218)yard::add & 207 & 72 & 303\\
(172)driver & 251 & 75 & 260\\
(188)driver::print & 14 & 33 & 74\\
(204)bus::print & 18 & 49 & 196\\
(220)yard::remove & 253 & 191 & 518\\
(174)vehicle & 190 & 497 & 737\\
(190)vehicle::vehicle & 23 & 88 & 71\\
(206)van::van & 19 & 17 & 49\\
(222)yard::print & 86 & 103 & 148\\
\hline
\end{tabular}
\end{center}
\caption{Review Time}
\end{table}


\begin{table}[hb]
\begin{center}
\begin{tabular}{|l|l|l|l|l|}
\hline
Source & casem & donthi & horigan & OK\\
\hline
(176)bus &  &  & \#292 (=1)& \\
(192)vehicle::\~vehicle &  & \#238 (=1) & & \\
(208)van::\~van & \#240 (=1) &  & & 240\\
(224)main &  &  & \#338 (=1)& \\
(178)van &  &  & \#294 (=1)& \\
(194)vehicle::create &  &  & & \\
(210)van::create & \#246 (=1) & \#266 (=1) & \#314 (=1)& 246,266\\
(180)yard &  &  & \#296 (=1)& \\
(196)vehicle::print &  &  & \#308 (=1)& \\
(212)van::print &  & \#270 (=1) & \#316 (=1)& \\
(182)driver::driver & \#278 (=1) &  & \#300 (=1)& \\
(198)bus::bus &  &  & & \\
(214)yard::yard &  &  & \#318 (=1)& \\
(184)driver::\~driver &  &  & & \\
(200)bus::\~bus &  & \#252 (=1) & & \\
(216)yard::\~yard & \#250 (=1) &  & \#320 (=1)& \\
(170)Constant &  &  & \#284 (=1)& \\
(186)driver::create &  &  & \#302 (=1)& \\
(202)bus::create &  &  & & \\
(218)yard::add & \#258 (=1) &  & \#324,\#326 (=2)& \\
(172)driver & \#230 (=1) &  & \#286 (=1)& \\
(188)driver::print &  &  & \#304 (=1)& \\
(204)bus::print &  &  & \#310 (=1)& 310\\
(220)yard::remove & \#262 (=1) &  & \#330 (=1)& 262\\
(174)vehicle &  &  & \#288 (=1)& \\
(190)vehicle::vehicle &  &  & \#306 (=1)& \\
(206)van::van &  &  & & \\
(222)yard::print &  & \#274 (=1) & \#334 (=1)& \\
\hline
\end{tabular}
\caption{Source node v.s Issue node}
\end{center}
\end{table}

%%%\end{document}

%%% \documentstyle[11pt,/group/csdl/tex/definemargins,
%%%                        /group/csdl/tex/lmacros]{article} 
%%% 
%%%           \begin{document}
%%%           \begin{center}
%%%           {\large\bf CSRS Experiment Results}\\
%%%           \end{center}
%%%           \small 
\chapter {CSRS Experiment Results: Group6(EIAM)}	  
\small
	  

\begin{description}
\item [Method:] EIAM
\item [Group:] Group6
\item [Source:] driver
\item [Participants:] awong (Reviewer), ccheung (Reviewer), gnakamur (Reviewer)
\end{description}
\section{Issue Lists}
\begin{enumerate}
\item {\it Issue\#192 (gnakamur)}
\begin{description}
\item [Subject:] Illegal assignment of name member variable.
\item [Criticality:] Hi
\item [Confidence-level:] Hi
\item [Source-node:] Driver::Driver()

\item [Lines:] 10

\item [Description:] The statement assigns a pointer to an empty string to name.  
This replaces the pointer to the memory allocated for name and results in a
memory leak.  This may also crash the program if name is modified.
\end{description}
\item {\it Issue\#196 (ccheung)}
\begin{description}
\item [Subject:] Variable assignment error
\item [Criticality:] Med
\item [Confidence-level:] Hi
\item [Source-node:] Driver::change

\item [Lines:] 5

\item [Description:] Varible 'name' is a pointer to a char array, Just assign name = 
temp\_name would not change the variable 'name'.
\end{description}
\item {\it Issue\#198 (gnakamur)}
\begin{description}
\item [Subject:] Destructor memory leak
\item [Criticality:] Low
\item [Confidence-level:] Hi
\item [Source-node:] Driver::\~Driver()

\item [Lines:] 3-6

\item [Description:] The destructor currently does nothing, but memory was allocated 
for the name member variable in the constructor.  This memory should be
deallocated before the object is destroyed or it will result in a memory
leak.
\end{description}
\item {\it Issue\#204 (gnakamur)}
\begin{description}
\item [Subject:] Illegal assignment of name member variable
\item [Criticality:] Hi
\item [Confidence-level:] Hi
\item [Source-node:] Driver::change

\item [Lines:] 5

\item [Description:] This statement assigns the pointer to the temp\_name string to 
name.  The pointer to the memory allocated for name is lost resulting in a
memory leak, and the temp\_name variable may be deallocated elsewhere causing
ill results if the name variable is referenced.
\end{description}
\item {\it Issue\#208 (ccheung)}
\begin{description}
\item [Subject:] Variable assignment error
\item [Criticality:] Med
\item [Confidence-level:] Hi
\item [Source-node:] Driver::Driver()

\item [Lines:] 10-11

\item [Description:] Variable 'name' is a pointer to a char array.  Just assigning 
name = "" would not change the variable 'name'.
\end{description}
\item {\it Issue\#212 (ccheung)}
\begin{description}
\item [Subject:] Deallocation
\item [Criticality:] Hi
\item [Confidence-level:] Hi
\item [Source-node:] Driver::\~Driver()

\item [Lines:] 

\item [Description:] Did not explictly deallocate unused pointer space.
\end{description}
\item {\it Issue\#214 (gnakamur)}
\begin{description}
\item [Subject:] Incorrect while condition
\item [Criticality:] Hi
\item [Confidence-level:] Hi
\item [Source-node:] main

\item [Lines:] 33

\item [Description:] The while statement executes when userinput\_shift is the valid 
values of 1, 2, or 3, when it should execute when an illegal value is inputed
for userinput\_shift (eg. not 1, 2, or 3).
\end{description}
\item {\it Issue\#218 (gnakamur)}
\begin{description}
\item [Subject:] Incorrect allocation of memory for userinput\_driver
\item [Criticality:] Med
\item [Confidence-level:] Hi
\item [Source-node:] main

\item [Lines:] 26

\item [Description:] Memory is allocated for userinput\_driver within the main while 
loop, but is deallocated outside of the while loop at the end of the program.
If the user changes the driver information more than once, the result is a
memory leak.  If the user just exits the program without changing the driver,
the userinput\_driver variable is undefined when passed to the delete[]
function and the results are unpredictable.
\end{description}
\item {\it Issue\#222 (ccheung)}
\begin{description}
\item [Subject:] Input value assignment error
\item [Criticality:] Med
\item [Confidence-level:] Med
\item [Source-node:] main

\item [Lines:] 25-27

\item [Description:] 'userinput\_driver' is not getting the user input.
\end{description}
\item {\it Issue\#226 (ccheung)}
\begin{description}
\item [Subject:] Space allocation error
\item [Criticality:] Med
\item [Confidence-level:] Low
\item [Source-node:] main

\item [Lines:] 9-10

\item [Description:] 'tempchar' is a single character.  It would not hold the user 
input.
\end{description}
\item {\it Issue\#234 (awong)}
\begin{description}
\item [Subject:] memory leak
\item [Criticality:] Low
\item [Confidence-level:] Med
\item [Source-node:] Driver::\~Driver()

\item [Lines:] 1-5

\item [Description:] destructor does not provide a way to free the memory allocated 
by the name character pointer
.
\end{description}
\item {\it Issue\#238 (awong)}
\begin{description}
\item [Subject:] improper space allocated
\item [Criticality:] Med
\item [Confidence-level:] Hi
\item [Source-node:] Driver::change

\item [Lines:] 5

\item [Description:] the program does not allocated space for the name character 
pointer.
\end{description}
\item {\it Issue\#242 (awong)}
\begin{description}
\item [Subject:] improper uses of "{\tt <}" "{\tt >}" symbol in while loop
\item [Criticality:] Low
\item [Confidence-level:] Hi
\item [Source-node:] main

\item [Lines:] 33-37

\item [Description:] shift is 1,2,3 and loop does not allow user to enter these 
values to insert to shift
\end{description}
\item {\it Issue\#248 (awong)}
\begin{description}
\item [Subject:] improper assignment to pointer
\item [Criticality:] Med
\item [Confidence-level:] Low
\item [Source-node:] Driver::Driver()

\item [Lines:] 10

\item [Description:] the assignment of a "" to pointer name is not correct
\end{description}
\item {\it Issue\#254 (awong)}
\begin{description}
\item [Subject:] not enough space allocated
\item [Criticality:] Low
\item [Confidence-level:] Low
\item [Source-node:] main

\item [Lines:] 26

\item [Description:] space allocated should be maxlength + 1 to allow for the '/0' 
character just in case userinput\_driver equals maxlength.
\end{description}
\item {\it Issue\#258 (awong)}
\begin{description}
\item [Subject:] not creating instances of drivers
\item [Criticality:] Low
\item [Confidence-level:] Hi
\item [Source-node:] main

\item [Lines:] 54-57

\item [Description:] The program is not really creating new instances of drivers. 
Basically it is just changing the old instance with new values
\end{description}
\end{enumerate}
\section{Review Metrics}
\begin{table}[hb]
\begin{center}
\begin{tabular}{|l|l|l|l|}
\hline
Participant & Start-time & End-time & Total Busy-time \\
\hline
gnakamur & Nov 28, 1994 10:30:38 & Nov 28, 1994 11:26:36 & 0:47:51 \\
ccheung & Nov 28, 1994 10:30:18 & Nov 28, 1994 11:32:47 & 0:53:50 \\
awong & Nov 28, 1994 15:31:23 & Nov 28, 1994 16:14:40 & 0:43:17 \\
\hline
\end{tabular}
\end{center}
\caption{Review Session}
\end{table}


\begin{table}[hb]
\begin{center}
\begin{tabular}{|l|l|l|l|}
\hline
Source & gnakamur & ccheung & awong\\
\hline
(176)Driver::\~Driver() & 232 & 211 & 290\\
(178)Driver::change & 430 & 384 & 295\\
(180)Driver::print & 61 & 106 & 92\\
(182)print\_error & 20 & 22 & 22\\
(184)print\_menu & 45 & 58 & 76\\
(170)Constant & 39 & 168 & 74\\
(186)main & 1292 & 1515 & 1311\\
(172)Driver & 207 & 131 & 176\\
(174)Driver::Driver() & 515 & 600 & 240\\
\hline
\end{tabular}
\end{center}
\caption{Review Time}
\end{table}


\begin{table}[hb]
\begin{center}
\begin{tabular}{|l|l|l|l|l|}
\hline
Source & gnakamur & ccheung & awong & OK\\
\hline
(176)Driver::\~Driver() & \#198 & \#212 & \#234 & 198=212=234 \\
(178)Driver::change & \#204 & \#196 & \#238 & 204=196=238 \\
(180)Driver::print &  &  & & \\
(182)print\_error &  &  & & \\
(184)print\_menu &  &  & & \\
(170)Constant &  &  & & \\
(186)main & \#214,\#218 & \#222,\#226 & \#242,\#254, & 214=242,218\\
          & (=2)        &  (=2) & \#258 (=3) & 254 \\
(172)Driver &  &  & & \\
(174)Driver::Driver() & \#192& \#208 & \#248 & 192=208=248 \\
\hline
\end{tabular}
\caption{Source node v.s Issue node}
\end{center}
\end{table}

%%%\end{document}

%%% \documentstyle[11pt,/group/csdl/tex/definemargins,
%%%                        /group/csdl/tex/lmacros]{article} 
%%% 
%%%           \begin{document}
%%%           \begin{center}
%%%           {\large\bf CSRS Experiment Results}\\
%%%           \end{center}
%%%           \small 
%%% 	  
\chapter {CSRS Experiment Results: Group7(EIAM)}	  
\small

\begin{description}
\item [Method:] EIAM
\item [Group:] Group7
\item [Source:] driver
\item [Participants:] sanaka (Reviewer), cklyoung (Reviewer), gokimoto (Reviewer)
\end{description}
\section{Issue Lists}
\begin{enumerate}
\item {\it Issue\#200 (cklyoung)}
\begin{description}
\item [Subject:] while loop problem
\item [Criticality:] Med
\item [Confidence-level:] Hi
\item [Source-node:] main

\item [Lines:] 33

\item [Description:] Shift should be 1, 2, or 3.  Loop should be opposite what is 
given.  i.e., userinput\_shift {\tt <} 0 || userinput\_shift {\tt >}3
\end{description}
\item {\it Issue\#204 (sanaka)}
\begin{description}
\item [Subject:] Memory "deallocation"
\item [Criticality:] Med
\item [Confidence-level:] Hi
\item [Source-node:] Driver::\~Driver()

\item [Lines:] 3-6

\item [Description:] Must have the delete function to delete the name created in the 
constructor.
\end{description}
\item {\it Issue\#210 (cklyoung)}
\begin{description}
\item [Subject:] changing name of driver
\item [Criticality:] Med
\item [Confidence-level:] Low
\item [Source-node:] Driver::change

\item [Lines:] 5

\item [Description:] i don't think this will correctly change the name of the 
driver, but i can't remember the way it should be done
\end{description}
\item {\it Issue\#214 (sanaka)}
\begin{description}
\item [Subject:] it should use strcpy
\item [Criticality:] Low
\item [Confidence-level:] Low
\item [Source-node:] Driver::change

\item [Lines:] 5

\item [Description:] May be we should use the strcpy to copy  the two strings
\end{description}
\item {\it Issue\#218 (sanaka)}
\begin{description}
\item [Subject:] We should set the memory to MAXLENGTH+1
\item [Criticality:] Hi
\item [Confidence-level:] Med
\item [Source-node:] main

\item [Lines:] 26-30

\item [Description:] 
\end{description}
\item {\it Issue\#222 (sanaka)}
\begin{description}
\item [Subject:] Think we should read of the return here also
\item [Criticality:] Hi
\item [Confidence-level:] Med
\item [Source-node:] main

\item [Lines:] 32

\item [Description:] If we dont read the tempchar here there will be problem when 
reding this input
\end{description}
\item {\it Issue\#224 (cklyoung)}
\begin{description}
\item [Subject:] getline from tempchar
\item [Criticality:] Low
\item [Confidence-level:] Low
\item [Source-node:] main

\item [Lines:] 30

\item [Description:] is this extracting the info stored in temp char?
\end{description}
\item {\it Issue\#230 (sanaka)}
\begin{description}
\item [Subject:] This delete is outside the loop
\item [Criticality:] Med
\item [Confidence-level:] Med
\item [Source-node:] main

\item [Lines:] 69

\item [Description:] Memory leaks may occur due to the delete being outside the 
while loop
\end{description}
\item {\it Issue\#236 (gokimoto)}
\begin{description}
\item [Subject:] No need to supply arguments to change function.
\item [Criticality:] Med
\item [Confidence-level:] Med
\item [Source-node:] Driver

\item [Lines:] 12-15

\item [Description:] The change function doesn't need argument.
\end{description}
\item {\it Issue\#240 (gokimoto)}
\begin{description}
\item [Subject:] No deletion of allocated memory.
\item [Criticality:] Hi
\item [Confidence-level:] Hi
\item [Source-node:] Driver::\~Driver()

\item [Lines:] 

\item [Description:] 
\end{description}
\item {\it Issue\#242 (gokimoto)}
\begin{description}
\item [Subject:] No deletion of allocated memory.
\item [Criticality:] Hi
\item [Confidence-level:] Med
\item [Source-node:] Driver::\~Driver()

\item [Lines:] 3-5

\item [Description:] 
\end{description}
\item {\it Issue\#246 (gokimoto)}
\begin{description}
\item [Subject:] This function is hard to use
\item [Criticality:] Med
\item [Confidence-level:] Hi
\item [Source-node:] Driver::change

\item [Lines:] 1-2

\item [Description:] 
\end{description}
\item {\it Issue\#250 (gokimoto)}
\begin{description}
\item [Subject:] No spaces between {\tt <}{\tt <} and variable names.
\item [Criticality:] Hi
\item [Confidence-level:] Med
\item [Source-node:] Driver::print

\item [Lines:] 4-11

\item [Description:] 
\end{description}
\item {\it Issue\#254 (gokimoto)}
\begin{description}
\item [Subject:] Routine will never construct instance.
\item [Criticality:] Hi
\item [Confidence-level:] Low
\item [Source-node:] Driver::Driver()

\item [Lines:] 5-9

\item [Description:] 
\end{description}
\item {\it Issue\#258 (gokimoto)}
\begin{description}
\item [Subject:] I Think this will not work
\item [Criticality:] Hi
\item [Confidence-level:] Low
\item [Source-node:] main

\item [Lines:] 28

\item [Description:] The conditional seems wrong.  Maybe I'm wrong.
\end{description}
\item {\it Issue\#262 (gokimoto)}
\begin{description}
\item [Subject:] The conditional seems wrong again.
\item [Criticality:] Hi
\item [Confidence-level:] Med
\item [Source-node:] main

\item [Lines:] 33-37

\item [Description:] 
\end{description}
\item {\it Issue\#266 (gokimoto)}
\begin{description}
\item [Subject:] This delete should be in the destructor.
\item [Criticality:] Hi
\item [Confidence-level:] Med
\item [Source-node:] main

\item [Lines:] 69

\item [Description:] 
\end{description}
\end{enumerate}
\section{Review Metrics}
\begin{table}[hb]
\begin{center}
\begin{tabular}{|l|l|l|l|}
\hline
Participant & Start-time & End-time & Total Busy-time \\
\hline
gokimoto & Jan 06, 1995 11:34:54 & Jan 06, 1995 11:34:54 & 0:0:0 \\
cklyoung & Nov 28, 1994 08:43:08 & Nov 28, 1994 09:23:40 & 0:36:32 \\
sanaka & Nov 28, 1994 08:42:39 & Nov 28, 1994 09:29:40 & 0:47:1 \\
\hline
\end{tabular}
\end{center}
\caption{Review Session}
\end{table}


\begin{table}[hb]
\begin{center}
\begin{tabular}{|l|l|l|l|}
\hline
Source & gokimoto & cklyoung & sanaka\\
\hline
(176)Driver::\~Driver() & 0 & 12 & 149\\
(178)Driver::change & 0 & 239 & 125\\
(180)Driver::print & 0 & 80 & 107\\
(182)print\_error & 0 & 74 & 39\\
(184)print\_menu & 0 & 41 & 33\\
(186)main & 0 & 1437 & 1860\\
(170)Constant & 0 & 76 & 94\\
(172)Driver & 0 & 104 & 119\\
(174)Driver::Driver() & 0 & 90 & 241\\
\hline
\end{tabular}
\end{center}
\caption{Review Time}
\end{table}


\begin{table}[hb]
\begin{center}
\begin{tabular}{|l|l|l|l|l|}
\hline
Source & gokimoto & cklyoung & sanaka & OK\\
\hline
(176)Driver::\~Driver() & \#240,\#242 &  & \#204 & 242=204\\
(178)Driver::change & \#246 & \#210 & \#214 & 210=214\\
(180)Driver::print & \#250 (=1) &  & & \\
(182)print\_error &  &  & & \\
(184)print\_menu &  &  & & \\
(186)main & \#258,\#262, & \#200,\#224 & \#218,\#222,& 262=200,218 \\
          & \#266 (=3) & (=2) & \#230 (=3) & 266=230,222 \\     
(170)Constant &  &  & & \\
(172)Driver & \#236 (=1) &  & & \\
(174)Driver::Driver() & \#254 (=1) &  & & \\
\hline
\end{tabular}
\caption{Source node v.s Issue node}
\end{center}
\end{table}

%%%\end{document}

%%% \documentstyle[11pt,/group/csdl/tex/definemargins,
%%%                        /group/csdl/tex/lmacros]{article} 
%%% 
%%%           \begin{document}
%%%           \begin{center}
%%%           {\large\bf CSRS Experiment Results}\\
%%%           \end{center}
%%%           \small 
\chapter {CSRS Experiment Results: Group3(EIAM)}	  
\small
	  

\begin{description}
\item [Method:] EIAM
\item [Group:] Group8
\item [Source:] yard
\item [Participants:] norio (Reviewer), scheung (Reviewer), kapila (Reviewer)
\end{description}
\section{Issue Lists}
\begin{enumerate}
\item {\it Issue\#234 (scheung)}
\begin{description}
\item [Subject:] The destructor of a class should not be a vitrual function.
\item [Criticality:] Low
\item [Confidence-level:] Hi
\item [Source-node:] bus

\item [Lines:] 9

\item [Description:] The destructor of a class need not be a
virtual function because not another class can delete the bus class.
\end{description}
\item {\it Issue\#238 (scheung)}
\begin{description}
\item [Subject:] The destructor for a class should not be a virtual function.
\item [Criticality:] Low
\item [Confidence-level:] Hi
\item [Source-node:] vehicle

\item [Lines:] 10

\item [Description:] The class destructor is specific to a class,
no other class can delete the vehicle class.
\end{description}
\item {\it Issue\#246 (scheung)}
\begin{description}
\item [Subject:] The destructor of the class should not be a
virtual function.
\item [Criticality:] Low
\item [Confidence-level:] Hi
\item [Source-node:] van

\item [Lines:] 9

\item [Description:] No other class can delete the van class, it
should not be virtual function.
\end{description}
\item {\it Issue\#252 (scheung)}
\begin{description}
\item [Subject:] Need to use a strcpy to copy to a string
\item [Criticality:] Hi
\item [Confidence-level:] Hi
\item [Source-node:] driver::driver

\item [Lines:] 4

\item [Description:] When using a string, always need to do a
strcpy to copy a string over to the string variable.
\end{description}
\item {\it Issue\#256 (scheung)}
\begin{description}
\item [Subject:] Use strcpy to copy to a string to a string variable
\item [Criticality:] Hi
\item [Confidence-level:] Hi
\item [Source-node:] driver::create

\item [Lines:] 4

\item [Description:] Use strcpy to copy a string to a string
variable, an assign statement will not work.
\end{description}
\item {\it Issue\#260 (kapila)}
\begin{description}
\item [Subject:] logical error
\item [Criticality:] Med
\item [Confidence-level:] Hi
\item [Source-node:] driver::driver

\item [Lines:] 7

\item [Description:] number of experince should be non negative.
\end{description}
\item {\it Issue\#262 (scheung)}
\begin{description}
\item [Subject:] use strcpy to copy a string
\item [Criticality:] Hi
\item [Confidence-level:] Hi
\item [Source-node:] driver::create

\item [Lines:] 18

\item [Description:] Use a strcpy function to copy string to a
variable, an assign statment will not work.
\end{description}
\item {\it Issue\#268 (norio)}
\begin{description}
\item [Subject:] Possible logic error...
\item [Criticality:] Low
\item [Confidence-level:] Hi
\item [Source-node:] driver::driver

\item [Lines:] 7

\item [Description:] This could be a programmer logic error, such that the
 drivers years of experience initially is set at a
 negative value.

 Should be: years\_exp = 0;
\end{description}
\item {\it Issue\#276 (norio)}
\begin{description}
\item [Subject:] Isn't this taken care of....
\item [Criticality:] Med
\item [Confidence-level:] Med
\item [Source-node:] driver::create

\item [Lines:] 4-7

\item [Description:] This declaration part is not needed when creating an
 instance of a driver since the constructor initializes
 these values when an instance of a driver is created.

 Should be: Deleted.
\end{description}
\item {\it Issue\#280 (scheung)}
\begin{description}
\item [Subject:] 
\item [Criticality:] 
\item [Confidence-level:] 
\item [Source-node:] driver::create

\item [Lines:] 

\item [Description:] 
\end{description}
\item {\it Issue\#282 (scheung)}
\begin{description}
\item [Subject:] The constructor already initilize the values
\item [Criticality:] Low
\item [Confidence-level:] Hi
\item [Source-node:] driver::create

\item [Lines:] 4-7

\item [Description:] Don't need to initilize the driver variables
because the constructor initilized it already.
\end{description}
\item {\it Issue\#286 (norio)}
\begin{description}
\item [Subject:] Cannot equal a string to an array...
\item [Criticality:] Hi
\item [Confidence-level:] Med
\item [Source-node:] driver::create

\item [Lines:] 18

\item [Description:] The programmer is trying to tell the compiler that the
 drivers name is equal to the character array userString.
 This command will not copy/move the contents of the array
 into the String name.

 Should be: (strcpy name,userString);
\end{description}
\item {\it Issue\#290 (scheung)}
\begin{description}
\item [Subject:] Not documentation
\item [Criticality:] Low
\item [Confidence-level:] Hi
\item [Source-node:] vehicle::\~vehicle

\item [Lines:] 3

\item [Description:] No documentation as what this function does.
\end{description}
\item {\it Issue\#294 (kapila)}
\begin{description}
\item [Subject:] logical error
\item [Criticality:] Hi
\item [Confidence-level:] Hi
\item [Source-node:] driver::create

\item [Lines:] 39

\item [Description:] MIN\_YEARS is declared as 0. This while loop will create run time error.
\end{description}
\item {\it Issue\#296 (norio)}
\begin{description}
\item [Subject:] Logic error...
\item [Criticality:] Hi
\item [Confidence-level:] Hi
\item [Source-node:] driver::create

\item [Lines:] 23

\item [Description:] This is a logic error, this type of logic omits the values that the
programmer originally intended to be entered.

 Should be:

 while ( (shift {\tt >}= MIN\_SHIFT) \&\& (shift {\tt <}= MAX\_SHIFT) )
 \{
  .
  .
  .
 \}
\end{description}
\item {\it Issue\#302 (scheung)}
\begin{description}
\item [Subject:] No documentation
\item [Criticality:] Low
\item [Confidence-level:] Hi
\item [Source-node:] bus::\~bus

\item [Lines:] 3

\item [Description:] no documentation as to what this function will
 do.
\end{description}
\item {\it Issue\#308 (norio)}
\begin{description}
\item [Subject:] Another logic error....
\item [Criticality:] Hi
\item [Confidence-level:] Med
\item [Source-node:] driver::create

\item [Lines:] 13

\item [Description:] This is another logical error in which the comparisons are
 wrong, ie: faulty comparisons are made between 1,
 MAXLEN and userString.

 Should be:

 while ((strlen(userString) {\tt >}= 1) \&\& (strlen(userString) {\tt <}= MAXLEN))
\end{description}
\item {\it Issue\#312 (scheung)}
\begin{description}
\item [Subject:] use strcpy function to copy to a string
\item [Criticality:] Hi
\item [Confidence-level:] Hi
\item [Source-node:] van::van

\item [Lines:] 4-5

\item [Description:] Use strcpy to copy a string to a string
variable.
\end{description}
\item {\it Issue\#316 (scheung)}
\begin{description}
\item [Subject:] 
\item [Criticality:] 
\item [Confidence-level:] 
\item [Source-node:] van::\~van

\item [Lines:] 

\item [Description:] 
\end{description}
\item {\it Issue\#318 (scheung)}
\begin{description}
\item [Subject:] No documentation
\item [Criticality:] Hi
\item [Confidence-level:] Hi
\item [Source-node:] van::\~van

\item [Lines:] 3

\item [Description:] no documentation as what this function does.
\end{description}
\item {\it Issue\#322 (kapila)}
\begin{description}
\item [Subject:] wrong access
\item [Criticality:] Hi
\item [Confidence-level:] Hi
\item [Source-node:] vehicle::create

\item [Lines:] 9

\item [Description:] incharge.create is the protected in driver.
\end{description}
\item {\it Issue\#324 (scheung)}
\begin{description}
\item [Subject:] Use strcpy to copy a string
\item [Criticality:] Hi
\item [Confidence-level:] Hi
\item [Source-node:] van::create

\item [Lines:] 16-17

\item [Description:] Use strcpy to copy strings.
\end{description}
\item {\it Issue\#330 (norio)}
\begin{description}
\item [Subject:] Faulty comparisons....
\item [Criticality:] Hi
\item [Confidence-level:] Hi
\item [Source-node:] driver::create

\item [Lines:] 31

\item [Description:] This is another logic error in which the comparisons are
 wrong.

 Should be:

 while((pay\_rate {\tt >}= MIN\_PAY\_RATE) \&\& (pay\_rate {\tt <}= MAX\_PAY\_RATE))
\end{description}
\item {\it Issue\#334 (kapila)}
\begin{description}
\item [Subject:] wrong access
\item [Criticality:] Hi
\item [Confidence-level:] Hi
\item [Source-node:] vehicle::print

\item [Lines:] 5

\item [Description:] incharge is protected in driver.
\end{description}
\item {\it Issue\#338 (scheung)}
\begin{description}
\item [Subject:] Delete the parking structure
\item [Criticality:] Hi
\item [Confidence-level:] Hi
\item [Source-node:] yard::\~yard

\item [Lines:] 7

\item [Description:] This line only deletes the vehile in parking
yard.  You need to delete the parking yard itself.
\end{description}
\item {\it Issue\#342 (norio)}
\begin{description}
\item [Subject:] Comparison error...
\item [Criticality:] Hi
\item [Confidence-level:] Hi
\item [Source-node:] driver::create

\item [Lines:] 39

\item [Description:] This is a logic error in comparison...

 Should be:

 while((years\_exp {\tt >}= MIN\_YEARS) \&\& (years\_exp {\tt <}= MAX\_YEARS))
\end{description}
\item {\it Issue\#350 (kapila)}
\begin{description}
\item [Subject:] destructor
\item [Criticality:] Med
\item [Confidence-level:] Hi
\item [Source-node:] van::\~van

\item [Lines:] 3-6

\item [Description:] Need a destructor to delete char array.
\end{description}
\item {\it Issue\#354 (norio)}
\begin{description}
\item [Subject:] No bounds checking...
\item [Criticality:] Med
\item [Confidence-level:] Hi
\item [Source-node:] vehicle::create

\item [Lines:] 6-7

\item [Description:] This is not bound checked for validity, hence the user
 could enter a negative seating capacity and it would
 be alright, this would be a logical error.

 Should be: (1) bound between come number of seats
                ie: 10 (min) - 30 (max)

 while ((seating\_cap {\tt >}= min) \&\& (seating\_cap ={\tt <} max))
 \{
  .
  .
  .
 \}
\end{description}
\item {\it Issue\#358 (kapila)}
\begin{description}
\item [Subject:] destructor
\item [Criticality:] Med
\item [Confidence-level:] Hi
\item [Source-node:] yard::\~yard

\item [Lines:] 7

\item [Description:] memory leak; delete should be delete[]..
\end{description}
\item {\it Issue\#362 (scheung)}
\begin{description}
\item [Subject:] Main is not returning any values
\item [Criticality:] Med
\item [Confidence-level:] Hi
\item [Source-node:] main

\item [Lines:] 1

\item [Description:] Thus main should return a integer, but it does
not.
\end{description}
\item {\it Issue\#370 (norio)}
\begin{description}
\item [Subject:] Logic error...
\item [Criticality:] Hi
\item [Confidence-level:] Hi
\item [Source-node:] bus::create

\item [Lines:] 6

\item [Description:] This is a logic error which permits faulty values to be
 entered.

 Should be:

 while ( (route {\tt >}= MIN\_ROUTE) \&\& (route {\tt <}= MAX\_ROUTE) )
\end{description}
\item {\it Issue\#374 (scheung)}
\begin{description}
\item [Subject:] Delete the string after the function exits
\item [Criticality:] Med
\item [Confidence-level:] Hi
\item [Source-node:] driver::create

\item [Lines:] 8

\item [Description:] If you allocate the space, you need to delete
it.
\end{description}
\item {\it Issue\#380 (norio)}
\begin{description}
\item [Subject:] Missing something....?
\item [Criticality:] Hi
\item [Confidence-level:] Low
\item [Source-node:] bus::bus

\item [Lines:] 5

\item [Description:] Aren't you also suppose to call the constructor of the
 vehicle when you create an instance of a bus since the
 bus inherits from the vehicle class??

 Should be:

 .........
 \{
  .
  .
  .
 \}:vehicle::vehicle
\end{description}
\item {\it Issue\#386 (norio)}
\begin{description}
\item [Subject:] Cannot do that...
\item [Criticality:] Hi
\item [Confidence-level:] Med
\item [Source-node:] van::van

\item [Lines:] 4-5

\item [Description:] You cannot do this since its a character array, and hence
 you cannot say that the character array is equal to " ".

 Should be:
 strcpy (region, " ");

 Or another type of command which initializes the array to
 whatever default value the programmer wants then copies
 it into region using strcpy.
\end{description}
\item {\it Issue\#390 (scheung)}
\begin{description}
\item [Subject:] Bad deletion function
\item [Criticality:] Low
\item [Confidence-level:] Hi
\item [Source-node:] yard::remove

\item [Lines:] 6

\item [Description:] This a poor design of a delete function,
because the documentation says that the vehicle ID is the same as the parking
lot no.  When you delete a vehicle, it packs the remaining vehicle.  Now all
the vehicle ID changes, because of the pack.
\end{description}
\item {\it Issue\#394 (norio)}
\begin{description}
\item [Subject:] Missing something...
\item [Criticality:] Hi
\item [Confidence-level:] Med
\item [Source-node:] van::\~van

\item [Lines:] 4

\item [Description:] Since the instance of van creates an character array, it
 has to get rid of it too.  Hence this will be a memory
 leak problem.....
 
 Should be:

 delete []region;
\end{description}
\item {\it Issue\#398 (norio)}
\begin{description}
\item [Subject:] Logic comparison error...
\item [Criticality:] Hi
\item [Confidence-level:] Med
\item [Source-node:] van::create

\item [Lines:] 11

\item [Description:] This is a logic comparison error which allows the wrong
 values to be accepted as input.

 Should be:

 while ((strlen(regionString) {\tt >}= 1) \&\& (strlen(regionString {\tt <}= MAXLEN)))
 \{
  .
  .
  .
 \}
\end{description}
\item {\it Issue\#402 (norio)}
\begin{description}
\item [Subject:] Cannot do that...
\item [Criticality:] Hi
\item [Confidence-level:] Med
\item [Source-node:] van::create

\item [Lines:] 16

\item [Description:] This also cannot be done, since region is a character array, the value of
regionString cannot be copied into region using this method.

 Should be:

 strcpy (region,resiongString)
\end{description}
\item {\it Issue\#408 (norio)}
\begin{description}
\item [Subject:] Can be deleted...?
\item [Criticality:] Med
\item [Confidence-level:] Low
\item [Source-node:] van::print

\item [Lines:] 5

\item [Description:] This line can be deleted since this function inherits from
 the vehicle class and the vehicle print function was
 declared virtual.  This means that the compiler will call
 the print function automatically.

 Should be:

 Deleted.
\end{description}
\item {\it Issue\#412 (norio)}
\begin{description}
\item [Subject:] Somethings missing....
\item [Criticality:] Hi
\item [Confidence-level:] Low
\item [Source-node:] van::van

\item [Lines:] 5

\item [Description:] At the end of the \} the call to the class vehicle constructor is missing.
This should be included in this
 function since it inherits from the vehicle class.

 Should be:

 .
 .
 .
 \}vehicle::vehicle
\end{description}
\item {\it Issue\#416 (norio)}
\begin{description}
\item [Subject:] Wouldn't the last one survive...?
\item [Criticality:] Med
\item [Confidence-level:] Med
\item [Source-node:] yard::\~yard

\item [Lines:] 5

\item [Description:] This is a logical error, which leaves the last element
 in the array undeleted.  This may cause a slight memory
 leak problem.

 Should be:

 for (j = 0; j ={\tt <} YARD\_SIZE ; j++)
\end{description}
\item {\it Issue\#420 (norio)}
\begin{description}
\item [Subject:] Can you do this...?
\item [Criticality:] Hi
\item [Confidence-level:] Low
\item [Source-node:] yard::add

\item [Lines:] 8

\item [Description:] This is setting the parking array to the pointer of
 myveh. This would be an error.

 Should be:

 parking[j] = \&myveh ???
\end{description}
\item {\it Issue\#424 (norio)}
\begin{description}
\item [Subject:] Comparison error...
\item [Criticality:] Hi
\item [Confidence-level:] Med
\item [Source-node:] yard::remove

\item [Lines:] 7

\item [Description:] Logical comparison error,

 Should be:

 while ((trash\_veh {\tt >}= 0) \&\& (trash\_veh {\tt <}= no\_of\_veh))
\end{description}
\item {\it Issue\#428 (norio)}
\begin{description}
\item [Subject:] Number of veh wasn't passed in...
\item [Criticality:] Med
\item [Confidence-level:] Low
\item [Source-node:] yard::remove

\item [Lines:] 15

\item [Description:] The no\_of\_veh wasn't passed in and this function doesn't
 pass back anything so this change will be temporary until
 this function is finished, ie: the value won't be changed.

 Should be:

 (1) no\_of\_veh should be passed in..
 (2) the address location of this variable should be
     accessed, then this decrement this variable.
\end{description}
\end{enumerate}
\section{Review Metrics}
\begin{table}[hb]
\begin{center}
\begin{tabular}{|l|l|l|l|}
\hline
Participant & Start-time & End-time & Total Busy-time \\
\hline
kapila & Dec 06, 1994 11:15:04 & Dec 06, 1994 12:33:55 & 1:14:41 \\
scheung & Dec 06, 1994 11:14:29 & Dec 06, 1994 12:22:28 & 1:7:59 \\
norio & Dec 06, 1994 11:23:54 & Dec 06, 1994 12:53:55 & 1:30:1 \\
syong & Jan 06, 1995 11:42:11 & Jan 06, 1995 11:42:11 & 0:0:0 \\
\hline
\end{tabular}
\end{center}
\caption{Review Session}
\end{table}


\begin{table}[hb]
\begin{center}
\begin{tabular}{|l|l|l|l|l|}
\hline
Source & kapila & scheung & norio & syong\\
\hline
(224)yard::remove & 34 & 791 & 364 & 0\\
(208)bus::print & 21 & 15 & 53 & 0\\
(192)driver::print & 55 & 38 & 53 & 0\\
(176)driver & 202 & 66 & 126 & 0\\
(226)yard::print & 16 & 70 & 20 & 0\\
(210)van::van & 241 & 110 & 332 & 0\\
(194)vehicle::vehicle & 46 & 22 & 46 & 0\\
(178)vehicle & 436 & 229 & 185 & 0\\
(228)main & 205 & 322 & 153 & 0\\
(212)van::\~van & 103 & 86 & 114 & 0\\
(196)vehicle::\~vehicle & 198 & 76 & 22 & 0\\
(180)bus & 137 & 200 & 96 & 0\\
(214)van::create & 102 & 122 & 334 & 0\\
(198)vehicle::create & 239 & 51 & 307 & 0\\
(182)van & 134 & 139 & 69 & 0\\
(216)van::print & 42 & 17 & 172 & 0\\
(200)vehicle::print & 223 & 68 & 51 & 0\\
(184)yard & 162 & 70 & 86 & 0\\
(218)yard::yard & 21 & 80 & 36 & 0\\
(202)bus::bus & 26 & 19 & 199 & 0\\
(186)driver::driver & 203 & 110 & 182 & 0\\
(220)yard::\~yard & 104 & 175 & 210 & 0\\
(204)bus::\~bus & 19 & 94 & 22 & 0\\
(188)driver::\~driver & 192 & 6 & 25 & 0\\
(222)yard::add & 62 & 102 & 232 & 0\\
(206)bus::create & 32 & 91 & 252 & 0\\
(190)driver::create & 970 & 838 & 1514 & 0\\
(174)Constant & 186 & 35 & 101 & 0\\
\hline
\end{tabular}
\end{center}
\caption{Review Time}
\end{table}


\begin{table}[hb]
\begin{center}
\begin{tabular}{|l|l|l|l|l|}
\hline
Source & kapila & scheung & norio & OK\\
\hline
(224)yard::remove &  & \#390 (=1) & \#424,\#428 (=2) & \\
(208)bus::print &  &  &  & \\
(192)driver::print &  &  &  & \\
(176)driver &  &  &  & \\
(226)yard::print &  &  &  & \\
(210)van::van &  & \#312 (=1) & \#386,\#412 (=2) & \\
(194)vehicle::vehicle &  &  &  & \\
(178)vehicle &  & \#238 (=1) &  & \\
(228)main &  & \#362 (=1) &  & \\
(212)van::\~van & \#350 (=1) & \#316,\#318 (=2) & \#394 (=1) & 350,394 \\
(196)vehicle::\~vehicle &  & \#290 (=1) &  & \\
(180)bus &  & \#234 (=1) &  & \\
(214)van::create &  & \#324 (=1) & \#398,\#402 (=2) & \\
(198)vehicle::create & \#322 (=1) &  & \#354 (=1) & 354\\
(182)van &  & \#246 (=1) &  & \\
(216)van::print &  &  & \#408 (=1) & \\
(200)vehicle::print & \#334 (=1) &  &  & \\
(184)yard &  &  &  & \\
(218)yard::yard &  &  &  & \\
(202)bus::bus &  &  & \#380 (=1) & \\
(186)driver::driver & \#260 (=1) & \#252 (=1) & \#268 (=1) & \\
(220)yard::\~yard & \#358 (=1) & \#338 (=1) & \#416 (=1) & 416\\
(204)bus::\~bus &  & \#302 (=1) &  & \\
(188)driver::\~driver &  &  &  & \\
(222)yard::add &  &  & \#420 (=1) & \\
(206)bus::create &  &  & \#370 (=1) & \\
(190)driver::create & \#294 (=1) & \#256,\#262,\#280, & \#276,\#286,\#296, & 294,\\
                  &         & \#282,\#374 (=5) & \#308,\#330,\#342 (=6) & 374\\
(174)Constant &  &  &  & \\
\hline
\end{tabular}
\caption{Source node v.s Issue node}
\end{center}
\end{table}

%%%\end{document}


\newpage

\bibliography{/group/csdl/bib/csdl-trs,/group/csdl/bib/ftr}
\bibliographystyle{/group/csdl/tex/named-citations}

\end{document}

