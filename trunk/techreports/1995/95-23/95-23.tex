\documentstyle[nftimes,/group/csdl/tex/CHI95]{article}

\hyphenation{CO-RE-VIEW}
\begin{document}

\title{Egret: A Framework for Advanced CSCW Applications}

\author{
Philip Johnson\\
Collaborative Software Development Laboratory\\
Department of Information and Computer Sciences\\
University of Hawaii\\
Honolulu, Hawaii 96822\\
(808) 956-3489\\
johnson@hawaii.edu}

\maketitle

\abstract 

Egret is a publically available, advanced framework for construction of
computer-supported cooperative work applications.  Egret provides an
approach to multi-user, interactive application development that differs
markedly from other frameworks or infrastructures, such as Groupkit, WWW,
or Lotus Notes.  This short paper introduces Egret, its architecture,
design philosophy, selected applications, and interest groups within the
CHI community. It concludes with information on how Egret's sources,
binaries, and documentation may be obtained free of charge using the
Internet.


\paragraph{KEYWORDS:} CSCW, frameworks, hypertext, client-server, agents.

\section{INTRODUCTION}

In 1991, the Collaborative Software Development Laboratory (CSDL) at the
University of Hawaii began research on two computer-supported cooperative
work applications: a collaborative learning environment called COREVIEW and
a software review system called CO2REVIEW.  To avoid duplication of effort
for the two applications, we also began working on a generic collaborative
infrastructure called Plover.  Subsequently, COREVIEW evolved into CLARE
\cite{csdl-94-04}, CO2\-REVIEW evolved into CSRS \cite{csdl-93-17}, and
Plover evolved into Egret \cite{csdl-93-09}.  Over the past four years,
Egret has undergone continuous and intensive co-evolution with CLARE, CSRS,
and several other applications as we have learned through sometimes painful
experience about the design and performance requirements for a generic
computer-supported cooperative work framework.

In the last year, we refocused our Egret research and development
activities from support for internal application development to
enhancements necessary for external public release. These activities have
been successful, and a public distribution of Egret is now freely available
for analysis and application development at no charge.  This short paper
provides an introduction to Egret for prospective users within the CHI
community.


\section{ARCHITECTURE}

Egret implements a multi-client, multi-server, multi-agent architecture.
Egret clients and agents are implemented by a 15 KLOC extension to XEmacs,
the X-window Emacs editor. Egret servers are implemented by a 15 KLOC
system written in C++.   

Egret provides both low and high level storage and communication facilities
for the development of (primarily textual) cooperative work applications.
Data representations range from unstructured binary storage, to
schema-based, typed, structured storage records, to HTML-compatible
hypertext. Indexing and local replication mechanisms enable efficient
"relational-style" queries over the underlying network database.
Inter-process communication is implemented via TCP/IP sockets, and provides
a variety of programmatic and interactive client communication facilities.
Password mechanisms are provided to facilitate secure collaboration in
groups dispersed across the internet. Built-in instrumentation support
facilitates research and evaluation of Egret applications.


\section{DESIGN PHILOSOPHY}

No single platform is ideal for every application domain, and Egret is no
exception.  Egret's design is oriented toward application
domains which require the collection, manipulation, and propogation of 
information about the {\em state of collaboration}. Some examples of such 
state information are:

\begin{itemize}

\item Members of the group require information on the {\em current
  state} of other group members. For example, information
  concerning what the member is working on, viewing, editing, or
  responsible for now or in the future.

\item Members of the group require information about the {\em history}
  of other group members' collaborative activities. For example,
  information concerning what another member worked on recently, or
  the recent sequence of actions of another member, or the entire
  history of another member's group activities.

\item Members require information about the {\em process} of
  collaboration. What tasks are to be accomplished, in what order, and
  which ones are currently being worked upon?  What are the roles and
  responsibilities for this collaborative activity, and which group
  members fit into which roles?

\end{itemize}

Investigating the storage, representation, and retrieval mechanisms
required for collaborative state information has been a theme underlying
all of our application experiences with Egret. For example, CLARE provides
a sophisticated process and data model for collaborative learning.  CSRS
goes one step further, by providing a process and data modelling language
with which one can implement a broad spectrum of collaborative methods for
software review and inspection.  AEN investigates "virtual co-presence"
within a hypertext database, providing each user with MUD-like knowledge of
who else is logged in, where in the database they are, and what they are
doing.

Egret supports collaborative state collection and manipulation through
primitive storage types (such as system nodes, personal nodes, typed nodes,
global tables, cached nodes), primitive communication mechanisms (such as
message forms, remote evaluation, events, and single/multiple client
broadcasts), and more sophisticated features built on top of these
mechanisms (such as event logging and real-time talk facilities).


\section{APPLICATIONS}

Egret has been used to implement over a dozen collaborative systems,
ranging in size from a few hundred lines to 15 KLOC.  Each of these systems
exploits one or more aspects of the Egret Design Philosophy. For example,
CSRS is a collaborative software review system implementing a process
modelling language for definition and enactment of software quality
improvement methods.  CLARE is a system for collaborative learning based
upon a constructivist approach to knowledge acquisition. AEN is a
collaborative hypertext authoring environment that provides extensive group
process visibility called "strong collaboration."  Shemacs is a drop-in
extension to XEmacs that provides a shared, real-time concurrent Emacs
editor with character-level locking. Flashmail is a real-time messaging
facility.  URN is a collaborative USENET reader.  CSRS, AEN, Flashmail, and
Shemacs are also available with the public distribution of Egret.

\section{EGRET USER COMMUNITIES}

Egret will be of interest to several subgroups within the CHI community:

\begin{itemize}
\item {\em Reseachers and developers of collaborative applications.}

  Egret provides a robust, efficient, and well-documented multi-user
  application development environment.  For CHI researchers, Egret
  provides excellent support for measurement and analysis of
  collaborative activities.  Last but not least, Egret is free.

\item {\em Teachers and students of multi-user, interactive technology.}

  Egret can be an effective means to provide "hands-on" experience in the
  development and evaluation of collaborative, multi-user interactive
  systems.  First, Egret comes with a complete documentation, including a
  tutorial guide that incrementally introduces the facilities provided in
  Egret to support collaborative application development.  Second, Egret
  comes with example applications that illustrate many Egret principles and
  practices.  Finally, Egret comes with complete source code, enabling
  users to extend or modify the implementation to explore alternative
  paradigms for collaboration. To facilitate learning about Egret's
  architecture and design, Egret's tutorial introduction, design
  reference manual, and client-side source code have been entirely
  converted to HTML format and interlinked together.

\item {\em Designers and users of other frameworks such as Lotus Notes,
  First Class, Microsoft Exchange, ConversationBuilder, GroupKit, etc. }

  Egret provides a fully realized example of an alternative architecture
  for multi-user interactive application development. Comparative
  evaluation architectures can provide new insight into the strengths and
  weaknesses of the architectural choices made by differing CSCW
  frameworks.

\item {\em Researchers and users of Internet "softbots", and researchers
  in mixed agent-human communication.}

  Egret was explicitly designed to support both interactive users and
  autonomous agent processes.  All Egret systems include at least one
  autonomous agent process.  Previously developed applications such
  as CSRS and AEN illustrate Egret's ability to support the
  development of mixed agent-human collaborative systems with
  multiple agent processes.

\end{itemize}



\section{OBTAINING EGRET}

The public distribution of Egret is available without charge over the
Internet.  It includes the complete Egret sources, binaries (precompiled
for several unix platforms), a regression test suite, and over 300 pages of
documentation.  To obtain the Egret distribution using the WWW, point
your browser at: \newline
\small
http://www.ics.hawaii.edu/$\sim$csdl/egret/release-notes.html\#Distribution
\normalsize

This URL displays a form that you can use to obtain the instructions
for downloading the system via anonymous FTP and subsequent installation. 
If you do not have WWW access, please send e-mail to Philip Johnson
(johnson@hawaii.edu) with your name and organizational affiliation and
instructions will be sent to you via e-mail. 

\section{ACKNOWLEDGMENTS}

Egret profits from the skills and energy of many talented individuals who
have worked with CSDL through the years, including Dadong Wan, Danu
Tjahjono, Cam Moore, Robert Brewer, Rosemary Andrada, Julio Polo, Jennifer
Geis, Russ Tokuyama, Tae Ho Yum, John Johnson, Sang-Woo Han, and Jeremy
Harrison.

Research on the Egret Framework is supported in part by funding from the
National Science Foundation, the University of Hawaii Research and Training
Fund, the Pacific International Center for High Technology Research, and
Tektronix, Inc.

\bibliography{/group/csdl/bib/csdl-trs}
\bibliographystyle{plain}

\end{document}



