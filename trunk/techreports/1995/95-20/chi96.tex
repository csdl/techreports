%%%%%%%%%%%%%%%%%%%%%%%%%%%%% -*- Mode: Latex -*- %%%%%%%%%%%%%%%%%%%%%%%%%%%%
%% chi96.tex -- 
%% Author          : Rosemary Andrada
%% Created On      : Sat Oct 14 15:02:39 1995
%% Last Modified By: Rosemary Andrada
%% Last Modified On: Mon Dec 11 17:06:55 1995
%% RCS: $Id$
%%%%%%%%%%%%%%%%%%%%%%%%%%%%%%%%%%%%%%%%%%%%%%%%%%%%%%%%%%%%%%%%%%%%%%%%%%%%%%%
%%   Copyright (C) 1995 Rosemary Andrada
%%%%%%%%%%%%%%%%%%%%%%%%%%%%%%%%%%%%%%%%%%%%%%%%%%%%%%%%%%%%%%%%%%%%%%%%%%%%%%%
%% 

\documentstyle[nftimes,/group/csdl/tex/CHI95]{article}

\begin{document}

\title{The Effect of a Virtual World Wide Web Community on its Physical Counterpart}

\author{
\bf{Rosemary Andrada} \\
Collaborative Software Development Laboratory \\
2565 The Mall \\
Honolulu, Hawaii 96822 \\
(808) 845-9291 \\
rosea@hcc.Hawaii.Edu}

\maketitle

\abstract 

This paper overviews a study that assessed the strengths and weaknesses of a
computer-based approach to improving the sense of community within one
organization, the Department of Information and Computer Sciences at the
University of Hawaii.  The case study used a pretest-posttest design.  First,
several measures of the sense of community within the department were obtained
via a questionnaire.  Second, a World Wide Web information system was
introduced in an effort to affect the level of community within the department.
Third, a similar questionnaire was administered after a period of four months.
Analysis of the survey responses and system logs showed that the information
system designed to promote community had instead polarized some of its members.
In addition, the system served as a valuable diagnostic tool for discovering
what factors may help promote or inhibit community building.

\paragraph{KEYWORDS:} World Wide Web (WWW), community

\section{INTRODUCTION}
The sense of community in an organization is not a binary entity.  One cannot
say that it either exists or it does not.  As with many other difficult to
measure concepts, community exists in varying degrees.  Clark \cite{Clark73}
defines community as

\begin{quote}
  ``a sentiment which people have about themselves in relation to others and
  others in relation to themselves;...there are two essentials for the
  existence of community: a sense of significance and a sense of solidarity.
  ''
\end{quote}

More importantly, it is the perception of those involved that determines the
relative strength of a community.  Here, a ``sense of community'' is viewed as
a combination of several ingredients, including: each group member's sense of
significance, their sense of solidarity, and their collective sense of
self-awareness.  For this study, the level of community was operationalized in
terms of the following measures:

\begin{itemize}
\item{Do members feel they play an important role?}\\
\item{Do members feel a sense of belonging?}\\
\item{Can members associate names with the faces of others in the organization?}\\
\item{Do members know each other personally?}\\
\item{Can members correctly identify a `resident expert,' if there is one, on
  subjects of importance to the organization?}\\
\item{Are members aware of the different projects or issues in the
  organization?}
\end{itemize}

My thesis is that an organization's sense of community can be positively
affected through computer mediation.  An information system designed with
features that support these measures of community was the testbed for this
study.  Pre-test and post-test questionnaires were administered to the
organization to assess the sense of community.  Logfiles kept by the
information system were analyzed to evaluate how the system was actually used.

This system is similar to several others, Edunet \cite{Wolpert91}, the Well
\cite{Rheingold93}, and the Davis Community Network 
\newline ({\tt http://www.dcn.davis.ca.us/}).

Overall, department members did feel a sense of importance and belonging, and
had a good sense of collective self-awareness.  However, while the sum total of
the measures indicates some sense of community in the department, the responses
to the open-ended questions presented a community divided into subgroups of
faculty, graduate students and undergraduate students.  This distinction seemed
benign at the beginning of the study.  However, by the end of the study some
undergraduates viewed this difference as enabling one to better computing
resources and/or projects if he or she belonged to the right group.

\section{SYSTEM DESIGN}
An interactive web server was designed as the platform for this study.  The two
main requirements for a community-enhancing information system were that it
should promote user involvement and keep users well-informed.

One way to facilitate user involvement was to provide users with the ability to
contribute to the information system without any restrictions.  This was
possible by creating a personal web page or by posting one's picture up on the
department photoboard.  Users also contributed to the ``hobbies'' page.
Finally, users contributed their opinions by submitting feedback or questions
online.

Keeping users well-informed meant posting up to date information about the
current research projects, degree programs, course info, etc.  Users of the
system learned not only about the department, but also about the department
members themselves by perusing through members' home pages, seeing them on the
department photoboard and learning about their personal interests.

\section{RESULTS}
The results of this study were as follows:

\begin{itemize}
\item{Department members did not view their group as one community, but
  envisioned themselves as three communities: faculty and staff, graduate students
  and undergraduate students.}\\
\item{The faculty and staff felt importance and belonging and had a high sense
  of collective self-awareness both before and after the introduction of the
  information system.}\\
\item{Initially, the graduate and undergraduate students did not feel important
  in the department but did feel a sense of belonging.  Their collective
  self-awareness was poor.}\\
\item{At the end of the study, graduate students did not feel important but
  still felt belonging in the department.  However, their collective
  self-awareness seemed to increase.}\\
\item{At the end of the study, undergraduate students neither felt importance
  nor belonging in the department.  But their collective self-awareness seemed to
  increase.}
\end{itemize}

This study was conducted on single user population, and so further research
and/or replication is required before general conclusions can be found.
However, the study does point to several important issues to take into account
when creating virtual communities.  
%\section{CONTRIBUTIONS OF THIS STUDY}
%The results of this research gives many insights into computing systems and
%communities.  These are the contributions this research has made to those
%fields.

\begin{itemize}
\item{Physical inequities, such as equipment or laboratory resources, can
  inhibit community development.}\\
\item{The collective self-awareness of a group can be a double-edged sword.
  Information dissemination flowing in one direction can reveal inequities
  and polarize a community.}\\
\item{The design of the information system was a successful one.  Department
  members quickly adopted the system and are still using it.}\\
\item{The World Wide Web is typically promoted as a means for universal
  readership of hypertext documents.  This study articulates an alternative
  purpose for the Web.}\\ \newpage
\item{Collective self-awareness can improve through computer mediation.}
\end{itemize}

%\section{RECOMMENDATIONS FOR WEB SITE BUILDERS}
Based upon these experiences, I have a few recommendations to encourage
community building through participation in a Web site.  The first
recommendation is to look inward as well as outward.  Many Web sites focus on
an outward design.  Their goal is to increase their visibility to the world.
This is important and should not be discouraged.  However, where community is
concerned, a design addressing the needs of the members and the organization
should be emphasized.

The second recommendation is to anticipate the impact of knowledge.  Recognize
that information will be flowing in new directions.  People will learn more
about their organization.  By providing a way for users to react to this
information or becoming involved with it somehow can help alleviate alienation.

The next recommendation is to repeatedly offer training sessions.  Once users
have achieved a certain level of knowledge with the system, more advanced
training sessions should be offered to both maintain their interest as well as
allow them to better utilize the system capabilities.

The last recommendation is to assess the impact of the system on the sense of
community when it is first introduced.  This could be in the form of an online
questionnaire.  The goal here is to determine the users' reaction to the
system.  Do they like it and find it useful, or can they think of features they
would like to see incorporated?  Since the system is for their benefit, it
should certainly be tailored to their needs.

The above recommendations may help to encourage community building.  The system
itself is not as important as the process by which it is introduced.  The
members should be involved in its development and determine how it evolves.

One of the great things about the Internet and the World Wide Web is that it
breaks down physical barriers and creates a virtual community amongst its
users.  The creation of a virtual reality did not seem to eliminate all
barriers to community in the physical environment.  Was this because the
information server was not designed well enough?  Could this be a special
problem because of the dual existence of a virtual community and a physical
one?  This relationship between the physical and virtual worlds is an important
topic for future research.

\bibliography{/group/csdl/bib/www-ics,/group/csdl/bib/csdl-trs}
\bibliographystyle{plain}

\end{document}
