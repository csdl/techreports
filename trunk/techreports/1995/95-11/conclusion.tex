%%%%%%%%%%%%%%%%%%%%%%%%%%%%%% -*- Mode: Latex -*- %%%%%%%%%%%%%%%%%%%%%%%%%%%%
%% conclusion.tex -- 
%% Author          : Rosemary Andrada
%% Created On      : Sun Mar 12 18:24:26 1995
%% Last Modified By: Rosemary Andrada
%% Last Modified On: Fri Jul  7 11:32:18 1995
%% Status          : Unknown
%% RCS: $Id: conclusion.tex,v 1.3 1995/07/05 00:34:25 rosea Exp rosea $
%%%%%%%%%%%%%%%%%%%%%%%%%%%%%%%%%%%%%%%%%%%%%%%%%%%%%%%%%%%%%%%%%%%%%%%%%%%%%%%
%%   Copyright (C) 1995 University of Hawaii
%%%%%%%%%%%%%%%%%%%%%%%%%%%%%%%%%%%%%%%%%%%%%%%%%%%%%%%%%%%%%%%%%%%%%%%%%%%%%%%
%% 

\ls{1.5}
\chapter{Observations and Future Directions}
\label{chap:conclusion}

%brief restatement of thesis
%brief restatement of results
%implications
%  to community
%  future research directions
%  webmaster impact
%  web desginer impact
%  web tool designer impact

This chapter presents my observations on the results of the case study.  Later,
I discuss the contributions of this research and suggest directions for further
research.

\section{Observations}
The goal of this research was to investigate how a computer-based information
system affects the sense of community in an organization.  I defined community
through a set of measures.  Based on these measures, I designed and implemented
features on the Web information system which I thought would support community
building.  However, where the definition of community has to do with feeling a
sense of importance and belonging, there was little I believed I could do
through a computer.  On the other hand, I believed a computer-based system
could have a very positive effect on the collective self-awareness of an
organization.  I hypothesized that by providing a way for people to learn more
about their organization, they would feel more of a sense of community.

The results showed that among the faculty and staff, they already felt
important and that they belonged in this organization prior to the introduction
of the computer-based information system.  They also had a good sense of
self-awareness of the department.  Results of the post-test questionnaires
showed these characteristics still held true.  On the other hand, the graduate
and undergraduate students initially did not feel important but did feel a
sense of belonging.  They were largely unable to identify people and projects
in the department.  At the end of the study, they neither felt important nor a
sense of belonging.  However, their sense of self-awareness of the organization
appeared to increase.

This study cannot definitively say that that computer-based mechanisms affected
the sense of community within the ICS department.  However, the results do
support several interesting conjectures.

Members of the ICS community are status conscious.  They view themselves in
terms of their relative positions in the department.  Every effort was made to
keep the questionnaires anonymous to encourage honest opinions on the sense of
community in the department.  However, people repeatedly identified themselves
in the questionnaire: ``I am a lecturer,'' ``Being a grad student,'' ``I'm just
an undergraduate.''  There is nothing wrong with recognizing one's position.
These are all true statements.  It does, however, promote a division in a
community based on the idea of ``us'' and ``them.''  This division became more
prominent in the post-test responses.

I learned that not all students are created equal.  Perhaps the questionnaires
should have been addressed to graduate and undergraduate students separately.
In the pre-test questionnaire, there was only one comment indicating any
intra-student factions, ``..need more undergrad-grad interaction.  Undergrads
should be allowed to participate in research.''  However, there were numerous
comments of this type in the post-test responses.  A few were mentioned in
Chapter \ref{chap:results}.  Here are 2 more:

\ls{1}
\begin{quote}
  ``As an undergrad, I think that we'd feel a lot more welcome if we had a
  computer lab.  I mean, engineering undergrads have a lab which they don't have
  to share.  So do people in business.  Why do ICS undergrads have to wait in
  line for computers being monopolized by non-ICS students playing solitaire?
  Grad Students don't have to...''
\end{quote}
\ls{1.5}

\ls{1}
\begin{quote}
  ``There should be more opportunities for students to join in research or
  projects going on with the department.  For undergrads, there is no lab (like
  the graduate lab).''
\end{quote}
\ls{1.5}

Community among the students seemed to have polarized since the introduction of
the Web information system.  The pre-test responses indicate that, overall,
students did not feel important.  They did however, feel like they belonged in
the ICS department.  Their collective self-awareness was very low.  Although
they frequently mentioned their status as a graduate or undergraduate student,
their comments did not reveal any hostility between the two groups.  By the
time the post-test questionnaire was administered, their sense of community
seemed to worsen.  They still felt no sense of importance in the department.
They seemed divided on their sense of belonging.  Half felt it and half did
not.  There were few comments saying community was strong in their group, but
these came from graduate students.  The responses by some undergraduates as
mentioned in previous quotes took on a more bitter tone.

These comments reveal that the self-perceived community of the undergraduates
is nonexistent.  But according to the measures of community, this is not
entirely true.  The questionnaires tested their knowledge of self-awareness and
they came out well.  The server logs shows numerous occasions of accesses of
other people's home pages and other projects or course related information.
While very few admitted that the Web information system affected their sense of
community, the data shows that it has at least increased self-awareness of the
organization among the students.

If they really were more self-aware, and the definition claims this contributes
to community, why did these students not perceive it this way?  Why did they
not express their sense of community in the department as having improved?
This goes against the idea that if people know more about themselves and their
organization, they will feel more of a sense of community.  Perhaps public
disclosure of all information can do the opposite, that is, decrease the sense
of community.

This decrease in the sense of community can be seen in the undergraduates'
responses.  There were several uses of the derogatory adjective ``peon'' to
refer to themselves.  This occurred only in the post-test questionnaire
responses.  There were numerous references to the graduate students as having a
lab, workstation access, and so forth, while the undergraduates do not.  Not
having the proper resources decreases their sense of importance and relegates
them to a ``peon'' status.  Many undergraduate students indicated they ``only''
took classes and did not do much else.  Upon seeing that the faculty and
graduate students participated actively in research, the undergraduates felt
further alienation.

The graduate students' appeared to perceive themselves as having a strong sense
of community as can be seen in many of their comments.  Nearly all of them
participated in the creation of home pages whereas less than half of the
undergraduates did.  There are, of course, visible differences between the two
groups that could attribute to the cohesiveness of one group over the other.
First, graduate students are more mature.  They are likely to deal with the
inequity of resources in a different manner.  Also, there is an order of
magnitude less people in this group.  It is far easier to bring together 40
people over 240.  Lastly, having a meeting place such as the lab, offers more
opportunity for members of this group to bond.

In any case, it appears that increasing self-awareness in an organization has
the potential to increase its sense of community.  It is possible this did not
occur with the undergraduates because the information seemed to flow in one
direction.  They learned more about the projects and issues concerning the
department but had no means for affecting it or being involved.  Whereas they
were once content with their current state as students, upon learning what more
goes on in the department, they felt left out of an aspect of the department in
which they would like to participate.  This often prompts motivation in people.
However, with the undergraduates, any motivation they may have had was met with
frustrating bouts with poor resources, which had little chance of changing
immediately.  This can be an added frustration since the changes they desire
are often slow to take place.  However, undergraduates simply do not have a lot
of time.  They are the transient members of the department.  Also, unlike
graduate students, they have coursework outside of the department and cannot
dedicate all of their time to this cause.

There seemed to be some sense of community within the part of the faculty \&
staff that responded to the questionnaire.  Their personal comments regarding
``divisiveness'' and ``alternate realities'' indicated otherwise.  However,
they sized up well according to the measures of community.  The faculty and
staff felt a sense of importance and belonging and were very self-aware in both
the pre-test and post-test.  This however, does not seem to be a result of the
Web information system.  Most of the faculty said they used it although not
very frequently.

The information system introduced drew out the differences between and within
the subgroups.  Almost half of the faculty did not create home pages.  Although
the information system was set up for the department, a side effect is that by
publishing on this system, one can increase one's visibility to the world.  It
would seem that faculty members would welcome this feature as their research
can greatly benefit from it.  This information system exposed which faculty
members wanted to publish information about themselves and which ones avoided
participation.  The same is not true for the graduate and undergraduates since
there is no ``definitive list'' of all students.

Slightly more than half of the faculty \& staff participated through the
creation of home page.  Since this group was much smaller than the students, it
would seem easier to get them to participate.  Some of them said they were
interested but very busy, and deferred participation to a more convenient time
for them, like summer.  However, nearly all of the graduate students created
home pages, but only less than half of the undergraduates did.  The total
number of undergraduates includes an unknown number of part-time students.
These students were less likely to be on campus very often and possibly could
not attend training sessions.

It appears that an appropriately designed information system can not only
increase the sense of self-awareness in an organization; it can also serve as a
diagnostic tool for determining what specific factors promote or inhibit
community development.  It draws out the visible differences between members to
reveal issues important to them.  For all the good that such a system can do,
the creation of such a virtual community is not all that is necessary to
eliminate barriers to community in the physical environment.

\section{Contributions of this Research}
The results of this research offers many insights into the strengths and
weaknesses of computer based mechanisms to improving the sense of community in
an organization.  One contribution is that this research showed how inequities
between department members inhibited community development.  This was clear
between the undergraduate and graduate students.  Graduate students received
support from the department in the form of computing facilities whereas the
undergraduates did not.  This did more than prevent community development.  In
fact, it further divided the community.

Another contribution concerns the collective self-awareness of a group.  Public
disclosure of information increased communication in one direction.  This can
be detrimental to an organization if the information continues to flow one way.
There also needs to be a way for the receivers of this information to become
involved.  However, the collective self-awareness of a group can be improved
through computer mediation.

The design of the information is another contribution.  While this system
design did not improve the overall sense of community in the department, people
found it useful and quickly adopted it.  The study lasted only 4 months but
people quickly began using the system in the first month with relatively little
training.

The fourth contribution this research has made is that it has defined another
purpose for the World Wide Web.  It is no longer just a universal network of
information; it now also contributes to community building.

The final contribution is that the Web information system also served as a
diagnostic tool for factors promoting or inhibiting community development.  The
system draws out the visible differences between department members to reveal
new issues important to the group.

\section{Recommendations for Web Site Builders}
Based upon these experiences, I recommend the following to encourage community
building through participation in a Web site: look inward as well as outward,
anticipate the impact of knowledge - provide mechanisms for user involvement,
repeatedly offer training sessions and scale to skill level, publicly release
tools and assess the impact of the system on community at introduction.

The first recommendation is to look inward as well as outward.  Many Web sites
focus on an outward design.  Their goal is to increase their visibility to the
world.  This is important and should not be discouraged.  However, where
community is concerned, a design addressing the needs of the members and the
organization should be emphasized.

The second recommendation is to anticipate the impact of knowledge.  Recognize
that information will be flowing in new directions.  People will learn more
about their organization.  By providing a way for users to react to this
information or becoming involved with it somehow can help alleviate alienation.

The next recommendation is to repeatedly offer training sessions.  Once users
have achieved a certain level of knowledge with the system, more advanced
training sessions should be offered to both maintain their interest as well as
allow them to better utilize the system capabilities.

The fourth recommendation is to make a public release of all tools associated
with the system.  For example, a unix shell script, such as the one I used for
creating home pages, should be made public and advertised online so that other
users have a way to use these tools also.

The last recommendation is to assess the impact of the system on the sense of
community when it is first introduced.  This could be in the form of an online
questionnaire.  The goal here is to determine the users' reaction to the
system.  Do they like it and find it useful, or can they think of features they
would like to see incorporated?  Since the system is for their benefit, it
should certainly be tailored to their needs.

The above recommendations may help to encourage community building.  The system
itself is not as important as the process by which it is introduced.  The
members should be involved in its development and determine how it evolves.

\section{Directions for Future Research}
This case study was not designed to prove that the sense of community can be
improved through computer based mechanisms.  I explored the strengths and
weaknesses of this approach as well as how it would affect an organization.
This had the unintended result of polarizing a community.  However, I learned
that the collective self-awareness of an organization can be affected through
these means.  A direction for future research is to investigate how the system
design can be improved by also incorporating two-way communication.  Is better
communication the only response to prevent polarization of a community who
knows too much about each other?  Perhaps there are other factors?  Again, the
system can serve as a diagnostic tool in determining these factors.

Another direction for future research is to investigate how users access
information from a Web database designed for their organization.  I touched
lightly on access patterns in this study.  I looked at a small subset of users
and discovered some patterns of access.  A more controlled experiment could be
performed on a cross-section of the group to learn about how they access
organizational information.  One can discover items of relative importance and
learn what design features can be tailored to the needs of its users.

Finally, one could study the architecture of a Web information system.  This
database is unique in that its architecture rapidly changes through the
publication efforts of its users.  Is it even possible to model such a dynamic
entity?  How does the database evolve over time?  How does a change in
membership or organizational goals affect its architecture?  Is the information
cohesive such that document links point to other documents within the same
domain?

These are just a few possibilities for future research regarding the World Wide
Web.  The field is quickly evolving and more research is being done on its
social aspects as well as the technical ones.  It is important enough that the
National Science Foundation has defined its research priorities for the
Web.\footnote{See http://www.cc.gatech.edu/gvu/nsf-ws/report/Report.html} Some
of these priorities involve using the Web as a collaborative tool, studying its
architecture and performance, evaluating its usability and so forth.  In fact,
one such recommendation falls in line with this research.  They are interested
in a study of an application which makes extensive use of the Web.  The entire
process is to be carefully studied for its usability issues, its design
history, and the social and organizational changes which come about as a result
of the experiment.  I have only lightly touched on this area by investigating
use of the Web as a tool for building community.  I look forward to continuing
my research and will perhaps learn more insights on different uses of the Web.

