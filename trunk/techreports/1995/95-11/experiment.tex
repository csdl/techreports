%%%%%%%%%%%%%%%%%%%%%%%%%%%%%% -*- Mode: Latex -*- %%%%%%%%%%%%%%%%%%%%%%%%%%%%
%% experiment.tex -- 
%% Author          : Rosemary Andrada
%% Created On      : Sun Mar 12 18:24:26 1995
%% Last Modified By: Rosemary Andrada
%% Last Modified On: Fri Jul  7 11:25:45 1995
%% Status          : Unknown
%% RCS: $Id: experiment.tex,v 1.3 1995/07/05 00:33:12 rosea Exp rosea $
%%%%%%%%%%%%%%%%%%%%%%%%%%%%%%%%%%%%%%%%%%%%%%%%%%%%%%%%%%%%%%%%%%%%%%%%%%%%%%%
%%   Copyright (C) 1995 University of Hawaii
%%%%%%%%%%%%%%%%%%%%%%%%%%%%%%%%%%%%%%%%%%%%%%%%%%%%%%%%%%%%%%%%%%%%%%%%%%%%%%%
%% 

\ls{1.5}
\chapter{Experimental Design}
\label{chap:experiment}

This chapter describes the experiment performed to evaluate the effect of the
system on the sense of community in the Department of Information and Computer
Sciences at the University of Hawaii.

\section{Duration of Experiment}
This study lasted from January 10, 1995 to April 30, 1995.  The system, called
WWW-ICS, was publicly released in January, 1995.

\section{Experimental Method}

% What method will be used to provide new insight into the problem?  How do
% you address this problem?

% How will questionnaire be administered and when?  How will this interact
% with server release dates?

\subsection{Using Questionnaires to Evaluate Community}
The operational definition of community is based on feelings people have about
themselves in relation to their organization and on their knowledge of other
members and the group as a whole.  Evaluating community involves asking people
about their feelings and testing their awareness of the organization.  An
efficient way of gathering this information is by administering questionnaires.
Since I was analyzing a change in community, I evaluated it twice to get a view
before and after the introduction of the information system.

The pre-test questionnaire in Appendix \ref{sec:pre-test} asks several
questions regarding people's feelings about themselves and the organization.
Other questions test their knowledge of various elements in the organization.
Recognizing the transitory nature of a university department, I decided it was
worth examining two obvious sub-communities.  I opted to look at the students
as one community and the faculty \& staff, who enjoy a more permanent status,
as another.  So pre-test questionnaires were sent out separately to the
students and to the faculty \& staff.  The questionnaires contained essentially
the same questions except for two that tested awareness of the community as a
whole as well as the sub-communities.

The questionnaires were administered to all undergraduate and graduate students
in the department by hardcopy through their ICS classes.  Instructions for
filling them out are in Appendix \ref{sec:instructions}.  Since I was unable to
get a comprehensive list of students for email, I went to every
non-introductory ICS class to give out the questionnaire.  They were offered to
ICS majors, minors and potential majors.  Students were not allowed to turn in
multiple questionnaires.  The ICS faculty received it both through e-mail and
as a hardcopy in their mailbox.  Anonymity was important, so all data was kept
strictly confidential.  Responses were turned in to a box in the ICS lounge.
However, for those who did not care to turn in a hardcopy, an e-mail reply was
also accepted.  To generate interest and participation in this research, I
offered an incentive for completing the questionnaire.  People who responded by
email were automatically entered in a drawing for a prize.  Those who submitted
their surveys by hardcopy detached an entry form from the bottom and turned it
in at the same time.  The prize was a \$30 gift certificate to a local
restaurant of the winner's choosing and was awarded to an undergraduate
student.  The questionnaire was sent out at approximately the same time the Web
server was publicly released.  See Table \ref{tab:chronology} for the
chronology of events of this experiment.

At the end of the experiment, I sent out post-test questionnaires to determine
the level of community at that time.  The post-test version contained the same
questions as the pre-test one.  There were seven additional questions
pertaining to the information system.  Administration of the post-test
questionnaire was conducted in the same manner as the pre-test questionnaire.
Again, slightly different versions were sent out separately to the faculty \&
staff and to the students.  The faculty and staff received it through email and
by hardcopy in their mailbox.  I compiled a list of email addresses of students
from training sessions conducted throughout the semester.  The student
questionnaires were sent out both through email and I again went to all of the
ICS classes soliciting participants.  Another drawing was held for a gift
certificate and was awarded to another undergraduate student.  Appendix
\ref{sec:post-test} contains the post-test questionnaire.

\subsection{Adoption of the Information System}
I installed the Web server in August, 1994.  The community-oriented features
mentioned in Chapter \ref{chap:www} were implemented into the system.  At
first, the only users of the information system were the members of this
author's research group.  The release of the system was scheduled for January,
1995.  However, a few faculty members found out about the Web server and used
it to publish course information as well as to create home pages for
themselves.

Usage of the system took place in two forms.  One was through retrieval of
data.  To retrieve data from the information system, one needed a WWW browser
and knowledge of how to navigate the Web using the browser.  Depending on the
client application, navigation through the information system involved clicking
a mouse, pressing a keyboard button, or typing in a document's URL.  Retrieving
data from the Web is relatively simple compared to the other form of usage,
which is publishing.  Publishing documents on the information system required
knowledge of HTML and simple unix commands pertaining to file processing.

In January 1995, just at the start of the Spring Semester, I publicly released
the information system.  An important factor in this research was
participation.  Without user participation, I could not study how the system
affected the sense of community in the group.  Some students were forced to use
the system since their professors posted lecture notes and assignments online
as opposed to handing them out in class.  But all members of the department
were encouraged to become information publishers themselves.  The first step
was to advertise aggressively.  To start with, all faculty received email
informing them of this research and requesting they give full support in any
way possible.  In return, I offered my knowledge and skills to help the faculty
on a one-on-one basis to set up personal home pages and/or course-related pages
on the Web.  Flyers were posted on the department bulletin board and in the
halls advertising the existence of an ICS Web Site and training sessions on how
to get involved.  There was no comprehensive electronic mailing list of
undergraduates, so only graduate students were informed by email.  In addition
to the flyers and email messages, information on how to contribute to the ICS
Web Site were made available online to anyone visiting that site.

In the first month, introductory sessions on the World Wide Web and its
relevance to the ICS department were scheduled approximately twice a week, for
a total of 6 training sessions that month.  These sessions were open to the
faculty, staff and students of the department.  They were designed to
familiarize people with the Web and to demonstrate its simple interface and
usefulness in navigation of the Internet.  Users came away from these classes
with their very own home page.  I simplified this initial training by designing
a unix shell script that creates a simple Web page for users as shown in Figure
\ref{fig:UserPage}.  The icon near the top of the page represents an unrendered
image.  This was included by default to encourage users to include their
picture in their home page.  The training classes were designed to get people
excited about the Web.  Hopefully, they would be inspired to learn more about
it on their own and modify their home page so that it became their own
creation.

\begin{figure}[htbp]
  \centerline{\psfig{figure=UserPage.eps}} 
\caption{A screen shot of a user home page.} 
\label{fig:UserPage}
\end{figure}

I avoided talking about the case study during these classes to prevent bias in
the questionnaire responses.  I made no mention of community building and
presented the classes as an introduction to a new means for navigation of the
Internet as well as for information sharing.

%Additionally, the author was asked to give several talks on her research to
%various graduate and undergraduate level classes.

The frequency of initial training sessions died down after a month, with some
getting cancelled due to low enrollment.  There seemed to be less people
interested in learning about the Web and creating home pages.  However, many
people stated that they wanted to do more with the system, but didn't have
enough information to continue.  To accommodate this need for more information,
I began offering new twice-weekly training sessions on advanced HTML.  By this
time, about one third of the faculty put their courses online.  Students of
these professors not only got their education through the training sessions,
but also through the mandatory use of the system demanded by their teachers.
In the last month, I conducted only a few training sessions, each of which had
low enrollment.  The ICS Web server continues to run although its primary
purpose now is to provide information about the department.

\subsection{Using Logfiles for Recordkeeping}
So far, the only method of data collection mentioned is gathering questionnaire
responses.  Users' opinions are important but should not be the sole source of
data.  People can say anything they want to about community or computer
systems.  However, their actions accurately reveal how they used the system.
The server software keeps a log of all incoming requests for documents.
Through these logs, I gathered information about how many home pages were
created.  The logs also showed which members of the department participated in
publishing Web related pages.  In addition to revealing information about
publishers and what they published, they showed access patterns of the
information system.  Study of these patterns indicate the manner in which users
navigated through the database.

\section{Anticipated Outcomes}
Given that the Web is rather new, I assumed that the initial number of Web
users in the department included only a few people.  I anticipated four
possible scenarios in which the information system would affect the sense of
community in this department.

The first possibility is that the computer-based approach would make no
appreciable difference to the level of community.  That is, there would still
remain only a few people using the Web and the level of community would remain
at the status quo.  This could occur due to insufficient advertising or
training and promotion of the Web information system.  Overcoming this obstacle
involves providing better resources and/or more frequent and flexibly-scheduled
training sessions.  Alternatively, this could occur even if users were aware of
the system, were fully knowledgeable about how to use the system, but simply
did not find any utility in it.  If this were the case, then one could argue
that the database was not appropriately designed to promote community within
the department.  In any case, infrequent use of the system would not be
expected to contribute to raising the level of community in the department.

The second possibility is that there were only a few active users of the
system, but that questionnaire responses indicate an increased level of
community.  Again, the reasons behind having only a few users of the system can
be attributed to those as in the first case.  However, why did department
members feel an increased level of community?  One answer is that there was an
outside factor affecting the level of community and the computer-based approach
had no effect.  Another very likely possibility is that only a few people
submitted the pre-test questionnaire and the same few turned in the post-test
questionnaire.  If these were the only people participating in the research,
then it is possible that the information system positively affected the level
of community in that smaller subgroup of the department.  This is interesting,
but this can be done even without a computer mediary.  It is important to
discover not just whether the level of community had been raised, but also what
people thought were the causal factors contributing to its increase.

The third possibility is that more people were using the system but the level
of community did not change appreciably.  This is an interesting case because
the fact that many people were using the system indicates they have more
knowledge about the department, yet this has not led to an increased sense of
community.  Since more people were utilizing the system, advertising was not a
problem.  If most users were only passively using the system, i.e. did not make
home pages, add their pictures or list their personal interests, then it may be
that the level of community did not change due to a lack of active
participation.  This outcome does not reveal whether the task of raising the
level of community in an organization lends itself to a computer-based
solution.

The last possibility is that more people were using the system and there was an
increased sense of community among members of the department.  If it is
possible to determine that the change in community was a direct result of usage
of the information system, then more investigation is in order.  What measures
of community changed and what factors contributed to these changes?  A more
controlled experiment would be necessary to determine these answers.  The
content of the database and its features represent a minimum of what is
necessary to increase a sense of community among department members.
