%%%%%%%%%%%%%%%%%%%%%%%%%%%%%% -*- Mode: Latex -*- %%%%%%%%%%%%%%%%%%%%%%%%%%%%
%% ftarm-demo.tex -- 
%% Author          : Philip Johnson
%% Created On      : Sat Sep 10 09:36:07 1994
%% Last Modified By: Danu Tjahjono
%% Last Modified On: Fri Oct  6 15:18:36 1995
%% Status          : Unknown
%% RCS: $Id$
%%%%%%%%%%%%%%%%%%%%%%%%%%%%%%%%%%%%%%%%%%%%%%%%%%%%%%%%%%%%%%%%%%%%%%%%%%%%%%%
%%   Copyright (C) 1994 University of Hawaii
%%%%%%%%%%%%%%%%%%%%%%%%%%%%%%%%%%%%%%%%%%%%%%%%%%%%%%%%%%%%%%%%%%%%%%%%%%%%%%%
%% 

\documentstyle[nftimes,11pt,titlepage,/group/csdl/tex/lmacros,/group/csdl/tex/definemargins]{article}

\definemargins{0.80in}{0.80in}{0.80in}{0.80in}{0.3in}{0.3in}
\input{/group/csdl/tex/psfig/psfig}

\begin{document}

\title {FTArm Demonstration Guide \\
{\normalsize (Version 1.2.0)}}

\author 
{Danu Tjahjono \\
 (dat@hawaii.edu) \\
 \\
Philip Johnson \\
(johnson@hawaii.edu)\\
\\
Collaborative Software Development Laboratory\\
Department of Information and Computer Sciences\\
University of Hawaii at Manoa\\
\\
CSDL-TR-95-19}
\date \today
\maketitle
\newpage
\tableofcontents
\newpage


\section{Demo setup}
\subsection {Downloading and installation}
 Download the CSRS version 3.4.3 distribution from 
 ftp.ics.hawaii.edu/pub/csdl/csrs, and follow the
  instructions in the file INSTALLATION.  

\subsection{General environment setup}
This section provides some general environment issues.
\begin{itemize}
\item {\em Fonts.} The FTArm review method currently specifies the
following fonts:
     \begin{verbatim}
     "-misc-fixed-medium-r-normal--13-120-75-75-c-80*"
     "-misc-fixed-bold-r-normal--13-120-75-75-c-80*"
     "-adobe-times-bold-i-normal--14-140-75-75-*"
     \end{verbatim}

    These are defined in the method, so they're easy to change in general,
    but for the demo, it's probably better to simply ensure that those
    fonts are defined on your system. xlsfonts lists the available fonts. 


\item {\em Monitors.}  FTArm is currently specified assuming that you're using the
    standard Sun 17 inch monitor.  Color is preferred but not necessary.  Large
    monitors are fine, but smaller monitors will definitely present a major
    problem for the demo.  
  
\item {\em Window system.}  Life will be most simple if you can run OpenLook and
    the virtual window manager olvwm for my workstation.   Life will be 
    sufficiently simple if you can run a virtual window manager on any X
    window manager.  Life is possible without the virtual window manager,
    but then we'll probably need at least three workstations.  (See next comment).

\item {\em Workstations.}  To run the demo, we need to switch back and forth between
    an ``Administrator" client, a ``Moderator/Reviewer" client, and a
    ``Producer/Reviewer" client.  So far, we've been doing the demo with a
    virtual window manager on a single workstation, which works fine for
    us, but may be somewhat confusing to people watching the demo (they may
    get confused about which client is currently displayed on screen.)

    A better configuration might be two workstations side by side: We use
    one to manage the administrator and moderator/reviewer clients, and you
    use the other to manage the producer/reviewer client. (Or we can move
    back and forth between them.)
\end{itemize}

\newpage
\section {FTArm review phases}
In this demonstration, we perform a formal technical review process
using the first five phases of FTArm.

\subsection{Setup Phase}
In this phase, the Administrator initializes the review database.

\subsubsection{Server initialization}

\begin{enumerate}
\item Change directory to csrs-3.4.3/ftarm/projects/demo
\item Invoke {\tt ../../initialize-ftarm demo-review 10000 files}.
\item Type ``Gagent'' when prompted for Gagent password. This will: 
  \begin{itemize}
  \item Start the HBS server, and create *.hb files.
  \item Brings up Gagent process (see Figure \ref{Gagent}). 
  \end{itemize}
\begin{figure}[htb]
  {\centerline{\psfig{figure=gagent.ps}}}
  \caption{Gagent process}
  \label{Gagent}
\end{figure}

\item Wait about 30 seconds or until the prompt to enter
Administrator's password appear. 
 Then enter ``Admin'' for the password. This will
 bring up the Administrator's process to initialize the database:
  \begin{itemize}
  \item Creates schemas.
  \item Initializes participants from participants.el.
  \item Reads in source files.
  \item Creates databases.el.
  \item Starts the Orientation phase.
  \end{itemize}
\end{enumerate}

\subsection{Orientation Phase}
In this phase, the Producer presents the source artifacts to be
reviewed to the review participants. 
For this demo, we will be running two user's processes only: Producer
and Moderator. 

\subsubsection {Connect as Producer}
\begin{enumerate}
\item Invoke {\tt ../../run-ftarm} from the shell.  Wait until it is fully
brought up (see Figure \ref{xemacs}). 
\begin{figure}[htb]
  {\centerline{\psfig{figure=xemacs.ps}}}
  \caption{FTArm startup process}
  \label{xemacs}
\end{figure}

\item Invoke {\tt M-x k*session*set-connect-participant} from XEmacs
process, and enter {\it dat}. This will set the review participant to
``dat''. 
\item Pulldown the menu ``Session'', and select the menu item ``Set
role/Producer''. This will set the review role of ``dat'' to
``Producer''. 
\item Pulldown the menu ``Session'', and select the menu item
``Connect/demo-review''. This will connect to demo-review database.
\item When prompted for password, enter {\it dat}.
\item Upon successful connection, the Summary screen will display a
set of source artifacts to be presented (see Figure \ref{summary-sources}.
\begin{figure}[htb]
  {\centerline{\psfig{figure=summary-sources.ps}}}
  \caption{Summary of Source Nodes}
  \label{summary-sources}
\end{figure}

\end{enumerate}

\subsubsection{Connect as Moderator}
\begin{enumerate}
\item Switch to another virtual shell window, and invoke {\tt
run-ftarm} again. Wait until it is fully brought up. 
\item Invoke {\tt M-x k*session*set-connect-participant} from XEmacs
process, and enter {\it johnson}. This will set the review participant
to ``johnson''.
\item Pulldown the menu ``Session'', and select the menu item ``Set
role/Moderator''. This will set the review role of ``johnson'' to
``Moderator''. 
\item Pulldown the menu ``Session'', and select the menu item
``Connect/demo-review''. This will connect to demo-review database.
\item When prompted for password, enter {\it johnson}.
\item Upon successful connection, the Summary screen will display a
set of source artifacts to be presented by the producer.
\end{enumerate}


\subsubsection{Running Orientation phase}
Orientation phase is synchronous review. Any source nodes read by the
Producer will be displayed on all participants screens. However, the
Moderator has the privilege to disable/renable the presenter.

\begin{enumerate}
 \item Producer: select the first source node to be presented from summary
 screen (use mouse middle button). 
Notice that the same source node is also displayed on the Moderator's 
screen. Try to retrieve the second node as well.
 \item Producer: retrieve the third node ``Search\_Optab''.
 \item Moderator: interrupt the Producer to make a note in this source
 code (i.e., create comment node). For example, explaining that this
 procedure uses a typical  binary search  algorithm (see Figure \ref{orientation-comment})

  \begin{itemize}
   \item Move mouse to ``Search\_Optab'' node, press mouse-right button
   to popup a menu, and select the command ``Make a note''.
   \item Fill out the Subject field: Binary Search Algorithm.
   \item Fill out the Description field: This is a typical binary
   search algorithm.
   \item Notice that the other fields are read-only (non editable).
   \item Now fill out the ``Lines'' field:
     \begin{enumerate}
       \item Go to ``Search\_Optab'' node, then
         drag the mouse-left to create a region from statement high =
         MAXOPS-1 until high >= low.
       \item Without moving the mouse, popup a menu using mouse-right
       button. 
       \item Select the command ``Select this region''. Notice that
       the region now becomes permanent.
       \item Go back to comment node, move the mouse to ``Lines''
       field, popup a menu using mouse-right button, and select the
       command ``Select current highlighted lines''.
       Notice that the region is now underlined, and a star (*) appear
       on the left margin. 
       \item To toggle the underline, simply click on the 
       star using mouse-left button.
       \item Save the comment node by pressing ``Save'' button on the
       menubar. Notice that the Producer's screen now shows the
       comment node that has been filled out by the Moderator.
       \item Close the comment node by pressing ``Close'' button on
       the menubar. This should close the comment node on the
       Producer screen as well.
     \end{enumerate}
  \end{itemize}

\begin{figure}[htb]
  {\centerline{\psfig{figure=orientation-comment.ps}}}
  \caption{Creating comment node}
  \label{orientation-comment}
\end{figure}

 \item Moderator: disable ``dat'' as the presenter by
 selecting (toggling off) the  user ``dat'' from pulldown menu
 ``Moderator/Presenters../dat''. 
 \item Producer: retrieve any node, and notice that the node is not
 displayed on the Moderator's screen (i.e., the synchronization is
 off).  
\item Both Moderator and Producer disconnect from the database.
Select the command ``Disconnect'' from the pulldown menu ``CSRS''.
 \end{enumerate}


\subsubsection{End orientation phase, start private review phase}
\begin{enumerate}
\item Switch to another virtual shell window, and
invoke {\tt run-ftarm} to bring up the administrator process.
\item Type M-x k*session*set-admin-mode. This will set the user to be
the Administrator.
\item Connect to demo-review database by pulling down the menu
``Session/Connect/Demo-review''.
\item Enter {\it Admin} when prompted to enter Administrator password.
\item List all projects by pulling down the menu
``Administrator/Project/List Projects'' (see Figure \ref{project-list}).
\begin{figure}[htb]
  {\centerline{\psfig{figure=project-list.ps}}}
  \caption{Project list}
  \label{project-list}
\end{figure}

\item Middle click on the project name. This will retrieve the project
node (see Figure \ref{project-node}).
\begin{figure}[htb]
  {\centerline{\psfig{figure=project-node.ps}}}
  \caption{Project node}
  \label{project-node}
\end{figure}

\item Go to field ``Active Phase'', and middle click on the
highlighted word ``Orientation''. This will retrieve the Orientation
phase node (see Figure \ref{orientation-phase-node}).
\begin{figure}[htb]
  {\centerline{\psfig{figure=orientation-phase-node.ps}}}
  \caption{Orientation phase node}
  \label{orientation-phase-node}
\end{figure}

\item Press mouse right button, and select the command ``End this
phase'' from the popup menu.
\item Press ``Close'' button on the menubar to return to Project.
\item Middle click on ``Private-Review'' (note that Active Phase is
empty) to retrieve the private review node.
\item Press mouse right button, and select the command ``Start this
phase'' from the popup menu. This will start the private review phase.
\item Press ``Close'' button on the menubar to return to Project.
\end{enumerate}


\subsection{Private Review Phase}
In this phase, two reviewers perform their review tasks privately.

\subsubsection {Connect as Reviewer 1}
\begin{enumerate}
\item Go to johnson's XEmacs process.
\item Switch role to ``Reviewer'' using  pulldown menu
``Session/Set Role/Reviewer.
\item Reconnect to Demo-review (password: johnson).
\item Retrieve the node ``sicasm.h''.
\item Create one question node (similar to Figure \ref{orientation-comment}):
    \begin{itemize}
    \item Select the command ``Raise a question'' from popup menu in
    sicasm.h.
    \item Fill out the question node. For example, asking what the format
    of mnemonic is . Highlight the word ``mnemonic'' in the
    source node, and fill out the ``Lines'' field accordingly. 
    \item Close the question node.
    \end{itemize}
\item Popup a menu in the source node, and select the command ``Set status
to reviewed''. This indicates that I have finished reviewing this
source node.
\item Retrieve the next source node ``dectonum'' from the summary buffer.
\item Create one issue node, for example, incorrect expression in the
statement if (i $>$ first + 3), which should have been (i$>$3) (see also
Figure \ref{issue}) :   
  \begin{itemize}
    \item Mouse-right to popup a menu, and select ``Raise an issue''.
    \item Type in values for the Subject field (e.g., Incorrect expression),
    Category field (e.g., Coding incorrect), and Description field
    (e.g., i$>$first+3  should have  been i$>$3).
    \item Fill out the Criticality field by selecting item ``Hi (Fatal
    Error)'' from popup menu in the Criticality field.
    \item Fill out the Lines field which references to statement i$>$first+3
    in the source node. Follow the steps described previously.
    \item Press ``Close'' button on the menubar to close the node.
  \end {itemize}

\begin{figure}[htb]
  {\centerline{\psfig{figure=issue.ps}}}
  \caption{Issue node}
  \label{issue}
\end{figure}
   
\item Set status of the source node to ``Reviewed'' by selecting the
respective command from popup menu in source node.

\item Select pulldown menu ``CSRS/Disconnect'' to disconnect from the
database. 
\end{enumerate}

\subsubsection {Connect as Reviewer 2}
\begin{enumerate}
\item Go to dat's XEmacs process.
\item Switch to participant ``cmoore'' by invoking M-x
k*session*set-connect-participant, and enter participant name ``cmoore''.
Set the role to ``Reviewer'' as well.
\item Reconnect to Demo-review (password: cmoore).
\item Retrieve the node ``sicasm.h''.
\item Answer the question node as follows:
   \begin{itemize}
   \item Click on character star (*) using mouse-left. The word
   mnemonic should now be underlined. Then, click  on the highlighted
   word using mouse-middle to retrieve the question node. 
   \item Or click on the link label located on ``Comments'' field
   using mouse-middle. 
   \item In the question node, popup a menu and select the command
   ``Respond to this comment''. This will create a comment node on the
   upper right screen (overriding Summary screen).
   \item Fill out the response node (for example, Subject: Re:
   mnemonic format; Description:  mnemonic consists of any upper-case
   letters up to 4 characters.
   \item Close the node when done.
   \item Close the question node as well.
   \end{itemize}
   Also notice that question/response nodes are visible during private
   review phase. 
\item Go back to source node, popup a menu, and select the command 
``Set status to reviewed''.
\item Raise summary screen using pulldown menu ``Screens/summary'', then
retrieve the node ``dectonum'' (mouse-middle click).
\item Create the same issue node (i$>$first+3 should have been i$>$3).
Type in the Subject field: Extra 'first'.
Notice that the issue raised by johnson is not visible by cmoore
during private review phase.
\item Set status of the source node to ``Reviewed''
\item Retrieve the node ``Search\_Optab''.
\item First retrieve the comment node shown on the Comments field
 (Mouse middle click on the link label).
This is the note created by the Moderator during Orientation phase.
Close the node when done.
\item Go back to source node, and create another issue node (for
example, regarding extra character ! in 
statement if (!strncmp... /*FOUND VALID OPCODE*/)), and close the node
when done. 
\item  Set status of the source node to ``Reviewed''
\item Disconnect from the database.
\end{enumerate}

\subsubsection{End private review phase, start public review phase}

\begin{enumerate}
\item Switch to Administrator's XEmacs process.
\item Retrieve Private-Review node by clicking mouse-middle button
on the highlighted word ``Private-Review'' in ``Active Phase'' field.
\item Press mouse right button, and select the command ``End this
phase'' from the popup menu. Press ``yes'' on the dialog box to
override the exit conditions (i.e., johnson has not reviewed all the
source nodes).
\item Press ``Close'' button on the menubar to return to Project.
\item Mouse-middle click on ``Public-Review'' to retrieve the public review
node.  
\item Start public review phase by selecting the command ``Start this
phase'' from the popup menu. Again, press ``yes'' in the dialog box to
override the entry conditions.  
\item Press ``Close'' button on the menubar to return to Project.
\end{enumerate}


\subsection{Public Review Phase}
In this phase, each reviewer reacts to each other's issues and comments.

\subsubsection {Connect as Reviewer 1}
\begin{enumerate}
\item Switch to johnson's XEmacs process, and reconnect to
Demo-review. The summary buffer should display two unread issues
created by cmoore (see Figure \ref{summary-issues}).
\begin{figure}[htb]
  {\centerline{\psfig{figure=summary-issues.ps}}}
  \caption{Summary of Issue Nodes}
  \label{summary-issues}
\end{figure}

\item Retrieve the first unread Issue node (extra !) using
mouse-middle click. 
\item Retrieve the corresponding source node \& lines using
mouse-middle click on the Lines field.
\item Indicate your disagreement by creating a comment node:
   \begin{itemize}
   \item Move mouse to any fields below Consensus field.
   \item Popup a menu, and select ``Make a disconfirming
   argument''. This will create an empty comment node.
   \item Fill out the comment node (for example, arguing that ! is
   needed because strncmp returns 0 when equal).
   \item Close the node when done. 
   \end{itemize}
  Note that by creating this comment node, you also vote
  ``Disconfirm''. The disconfirm field in Consensus is now updated to
  1. A star (*) next to 1 indicates your vote. 
\item Close the issue node.
\item Raise the summary screen using pulldown menu ``Screens/Summary''.
\item Retrieve the second unread Issue node on the summary screen.
\item Retrieve the corresponding source node.
\item First indicate that you agreed on this issue by voting
``Confirm''
  \begin{itemize}
    \item Go to Consensus field.
    \item Press mouse-right button and  select ``Vote
    confirm''. Confirm field in Consensus is now updated to 2. 
  \end{itemize} 
\item Also notice that this issue is similar to yours.
Indicate issue similarity as follows:
\begin {itemize}
  \item Popup a menu in Issue node, and select the command ``Select this
issue''. 
  \item Go to source node and retrieve your similar issue  by following
  the link label in the  ``Issues'' field (use mouse-middle click)
  \item Go to your issue node, popup a menu and select the command
    ``Declare similarity to selected issue''. A link should now be
    inserted in the ``Related-issues'' field (see Figure
    \ref{similar-issue}).
\begin{figure}[htb]
  {\centerline{\psfig{figure=similar-issue.ps}}}
  \caption{Issue node with similar link}
  \label{similar-issue}
\end{figure}
 
  \item Close both issue nodes.
\end {itemize}
\item Disconnect from the database.
\end{enumerate}

\subsubsection {Connect as Reviewer 2}
\begin{enumerate}
\item Switch to cmoore's XEmacs process, and reconnect to
Demo-review. The summary buffer should display one unread issue.
\item Retrieve the issue node, and vote confirm.
\item Also notice that this issue is similar to yours.
Simply click on the ``similar link label'' to retrieve your issue.
\item Now, look for any comments that you have not read:
  \begin{itemize}
   \item Pulldown the menu ``Reviewer'' on the menubar.
   \item Select the command ``List unread commentary nodes''. 
      The summary buffer now shows johnson's comment concerning his
      disagreement about your other issue. 
  \end{itemize}
\item Retrieve the comment node using mouse-middle click.
\item Retrieve your issue node shown on this comment node as follows:
   \begin{itemize}
   \item Move mouse to Source-node field.
   \item Mouse middle click on the highlighted word: Issue\#
   \end{itemize}
\item After reading this comment, you agree with johnson and decide to
change your mind. Vote disconfirm accordingly. The ``Disconfirm''
field is now updated to 2.
\item Disconnect from the database.
\end{enumerate}


\subsubsection{End public review phase, start consolidation phase}

\begin{enumerate}
\item Switch to Administrator's XEmacs process.
\item Retrieve Public-Review node.
\item Press mouse right button, and select the command ``End this
phase'' from the popup menu. 
\item Press ``Close'' button on the menubar to return to Project.
\item Mouse-middle click on ``Consolidation'' to retrieve the
consolidation phase node.  
\item Start consolidation phase by selecting the command ``Start this
phase'' from the popup menu.
\item Press ``Close'' button on the menubar to return to Project.
\end{enumerate}


\subsection{Consolidation Phase}
In this phase, the Moderator consolidates all issues and comments,
decides on the final actions regarding the rework, and finally
generates the review metrics.

\subsubsection{Connect as Moderator}
\begin{enumerate}
\item Go to johnson's XEmacs process.
\item Switch role to ``Moderator'' using  pulldown menu
``Session/Set Role/Moderator.
\item Reconnect to Demo-review. Wait until the message ``User
johnson (Moderator) connected..'' appears on the minibuffer.
\item Select the command ``Consolidate all issues'' from the pulldown
menu ``Consolidation''. Wait until the message ``Done'' appears.
\item Select the command ``List all consolidated issues''
from the pulldown menu ``Consolidation''.
Notice that there are only two consolidated-issues, because two out of
three individual issues are similar (see Figure
\ref{summary-consolidation}).
\begin{figure}[htb]
  {\centerline{\psfig{figure=summary-consolidation.ps}}}
  \caption{Summary of Consolidated-issue Nodes}
  \label{summary-consolidation}
\end{figure}

\item Now, retrieve the first consolidated issue where all reviewers
disagreed (vote disconfirm). See Figure \ref{consolidated-issue}.

\begin{figure}[htb]
  {\centerline{\psfig{figure=consolidated-issue.ps}}}
  \caption{Consolidated-issue node}
  \label{consolidated-issue}
\end{figure}

\item Create consolidated-action node by selecting from the popup menu
command ``Create consolidated action''. 

\item Fill out the action node. The moderator basically rejects this
issue (Subject: Reject this issue; Action-type: Ignore,
Rework-decision: None. This is a correct implementation).
See Figure \ref{consolidated-action}.

\begin{figure}[htb]
  {\centerline{\psfig{figure=consolidated-action.ps}}}
  \caption{Consolidated-action node}
  \label{consolidated-action}
\end{figure}

\item Close the node when done.
\item Retrieve the second consolidated issue node where all reviewers
agreed (vote confirm).
\item Create consolidated-action node, fill out the node with Subject:
Incorrect expression, Action-type: Fix, and Rework-decision: rewrite the
expression using i$>$3.
\item Close all consolidated actions and issues nodes.
\item Since all issues have been resolved, we may conclude the review
process at this point. 
\item We may also want to generate the review metrics:
  \begin{itemize}
   \item Invoke the command ``Retrieve and Process'' from 
    the pulldown menu ``Consolidation/Metrics''.
   \item Invoke the command ``Analyze Private-Review time'' from
    the pulldown menu ``Consolidation/Metrics''.
    In this demo, the report shows individual efforts (in seconds)
    during  Private-Review phase (see Figure \ref{metrics}). 
\begin{figure}[htb]
  {\centerline{\psfig{figure=metrics.ps}}}
  \caption{Sample review metrics}
  \label{metrics}
\end{figure}

  \end{itemize}
\item Quit from the database.
\end{enumerate}

\section{Bringing down CSRS/FTArm}

\begin{enumerate}
\item Quit from all XEmacs users processes.
\item Quit from XEmacs Gagent process by typing Control-x Control-c.
This should also shutdown the hbs server.
\item If necessary, shutdown the hbs server using: kill -2 $<pid>$.
\end{enumerate}


\end{document}



