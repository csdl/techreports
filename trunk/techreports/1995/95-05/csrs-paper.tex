\documentstyle [12pt, /group/csdl/tex/named-citations,
/group/csdl/tex/definemargins, /group/csdl/tex/lmacros]{article}

\input{/group/csdl/tex/psfig/psfig}

\begin{document}
\begin{center}
{\Large\bf CSRS: A Collaborative Software Review System}\\
\vspace {.2in}
by\\
\vspace {.2in}
{\em Danu Tjahjono\\
(dat@uhics.ics.hawaii.edu)}\\
Nov, 1994
\vspace {.4in}
\end{center}

\section{Introduction}

Software reviews or Formal Technical Reviews (FTR) plays
an important role in software development as a means to
improve the quality of software products. Unfortunately, many
software organizations have not employed the full potential of
this technique.  One problem that hinders its adoption to an
organization is the labor intensive nature of the technique,
especially when practiced manually. Another problem is related to 
the current proliferation of FTR methods and ambiguities in their 
practices. The same review method is often practiced differently by
different organizations. Furthermore, there are many conflicting
theories as to
what factors can be attributed to the success or failure of a
particular review method. 

Our CSRS (Collaborative Software Review System) is developed to address
these problems. Instead of prescribing a specific review method, we
develop a review system with enactable review methods.
The organization can then implement their own review method
that most suitable to their organizations. Furthermore, we provide
automatic metrics collection that allows the organization to
evaluate their methods and improve them over time.

\section{Data and Process Modeling Language}

CSRS implements a generic review system by providing data and
process modeling languages.  
These languages allow an organization to develop their own review
method, or to develop various review methods and empirically
investigate them to find the most effective one under certain
circumstances. 

\subsection{Data Modeling Language}

The data modeling language describes what artifacts are manipulated by
the review method. Specifically, it defines various types of review
artifacts, their field structures, and the relationships among the
artifacts. In CSRS terminology, review artifacts are called nodes.
Each node may contain multiple fields, and each field may store
various types of object, such as 
textual string, function definitions, link definitions, or other non
textual data (e.g., audio data). Relationships among the nodes are
represented by links.   
Nodes and links are typed, that is, different review artifacts and
their relationships can be represented by different node and link types. 
A set of computational supports may also be provided for different
artifact types.

One may also define a set of attributes to node instances (with or
without a specific node type), such as node-name, creation-date,
creator, etc. A ``state'' can also be attached to a node. These
information allows one, for example, to keep track the review
status of a specific node created by a specific user during the
current or subsequent phases. 

A typical review method usually includes four top-level node
types: Source, Commentary, Checklist and Administrative.

Source nodes contain input artifacts to be reviewed. The language
system allows the user to further differentiate source node types,
such as function, macro, module, etc. 

Commentary nodes contain output artifacts generated during the
review. Similarly, one can define different commentary types,
such as, {\it issue} nodes to record defects in the source,
{\it action} nodes to record proposal in resolving the issue, 
{\it consolidated-issue} nodes to consolidate similar issue nodes, etc.

Checklist nodes contain verification aids. In manual review practices,
such as Fagan's method \cite{Fagan76}, the reviewers are often required
to consult this checklist when examining source nodes. CSRS language
system allows checklist items to be defined for a specific source
node, and/or a specific participant role.
This checklist can also be stated as {\it required} or {\it optional}. 
For the required checklist, the reviewers
must explicitly check off the item when examining the corresponding source.

Administrative nodes contain administrative information about the
review process and project related information, such as participant names
and roles, the starting and ending review date, etc.

Figure \ref{fig:async-data-model} shows an example of data model used
by FTArm (Formal, Technical, Asynchronous review
method)\cite{Johnson93,CSDL-93-17}. Figure \ref{fig:data-language}
shows an example of the data modeling language for describing
{\sl Issue} node, its field structure and its link ({\sl derives})
originated from {\sl Source} node.

\begin{figure}[htb]
  {\centerline{\psfig{figure=async-data-model.epsi}}}
  \caption{FTArm Data Model}
  \label{fig:async-data-model}
\end{figure}

\begin{figure}[tp]
  \footnotesize
  \begin{verbatim}
(l*data*define-node-schema
  :name "Issue"
  :superschema "Commentary"
  :fields (("Subject")
	   ("Category")
	   ("Criticality")
	   ("Source-node" :init-fn e*issue*set-source)
	   ("Lines")
	   ("Description")
	   ("Consensus" :init-fn (e*node*set-consensus "confirm"))
	   ("Related-issues")
	   ("Proposed-actions")
	   )
  :make-link-fn e*derives*make
  )

(l*data*define-field-schema
  :name "Subject"
  :type text
  )

(l*data*define-field-schema
  :name "Source-node"
  :type lisp
  :display-fn e*node*name
  )

(l*data*define-link-schema
 :name "derives"
 :documentation "Derives issue node from any source nodes"
 :from-nodes ("Source")
 :to-nodes ("Issue" "Consolidated-issue")
 :label "csrs-link-label"
 :at-fields  '((:from-node "Source" :to-node "Commentary" :field "Issues")
              )
 )

  \end{verbatim}
  \normalsize
  \caption{Example of CSRS data language}
  \label{fig:data-language}
\end{figure}



\subsection{Process Modeling Language}

The process modeling language describes specific steps or phases that
review participants must go through in order to complete a review activity.

A software review process normally involves a group of
people with the same or different roles reviewing software
products, and a set of predefined review phases to follow. Each review
phase involves one or more objectives to be achieved (e.g.,
understanding the product, finding defects, consolidating the
defects), a specific interaction mode (e.g., face to face group
meeting, private mode, etc), and a specific strategy or technique to
be used in order to achieve the objective (e.g., using checklist
when examining/finding defects in the source). Furthermore, each phase
may have a set of preconditions to be satisfied before entering the
phase or postconditions to satisfy in order to leave the phase.
Figure \ref{fig:async-process-model} shows an example of review
process used by FTArm \cite{Johnson93,CSDL-93-17}. 

\begin{figure}[htb]
  {\centerline{\psfig{figure=async-process-model.epsi}}}
  \caption{FTArm Process Model}
  \label{fig:async-process-model}
\end{figure}

CSRS process modeling language allows one to define these phases
descriptively. 
Specifically, constructing a review process involves first defining
review phases and their entry (pre) or exit (post) conditions, review
participants and their roles, and the
review artifacts (nodes, links, fields, and attributes) manipulated by
the participant during specific phases.  
A set of operations can also be defined for the participant 
playing a specific role during a specific phase. In general, these
operations make up the review strategy for the given phase.

CSRS provides two built-in interaction modes: asynchronous
and synchronous meeting mode. In the former, the
participants manipulate common artifacts (nodes) asynchronously.
Group discussion is done through the creation of follow-up nodes using
links. When following the discussion, the participants simply follow the
links to the appropriate nodes. Additional computational supports can
also be provided to automatically consolidate the discussion by 
traversing the proper links (for example, to display all arguments that
support a proposal mentioned in the node, the system can simply traverse
all links of type {\it confirm} originating from the node and display
the corresponding destination nodes).

Synchronous mode supports allows all participants' screens to be
synchronized with the meeting leader. Thus, when the leader displays a
node, the node will also be displayed in all participants' screens.
CSRS language system allows one to define what node types to be
synchronized, and with whom they are to be synchronized. 

The interaction mode may also involve private or group interaction. In
private mode, only the creator of the node can read the node, while in 
public mode, all participants can read the node. CSRS language system
allows one to define what artifacts can be accessed (read/write) by
whom (what role) during what phase.

CSRS implements a sequential process workflow, that is, a review process
consists of a sequential ordering of review phases.
CSRS provide both manual and automatic
activation of these phases. In the manual mode, a designated participant
(e.g., moderator) will check the status of the review periodically (or
program the system to notify him/her when the status changes) and activate
the next phase accordingly. In the automatic mode, an autonomous agent
(background process) will monitor the status of the review
periodically by checking the post-conditions of the current phase, or
pre-conditions of the next phase, and activate the next phase
accordingly when the conditions have been met.

Within a phase, the process workflow is enforced through menu
selections. The operations installed in the menu are
not only sensitive to the current phase, but also to the current role that
the participant is assuming, and the current node and field types.
This also implies that when defining new operations, one must specify
whether the operations will be applied to specific 
phases, roles, node types and/or field types. The operations can also
be defined to be automatically invoked by the
system right after the participant successfully connect to the system.
This feature defines the starting state of the process flow, that is,
guiding the participants as what to do when they first login to the
current phase. 

Figure \ref{fig:process-language} shows an example of process
language for describing the process model of FTArm (partial
description). 

\begin{figure}[tp]
  \footnotesize
  \begin{verbatim}
(l*process*define-method       ;define process model of FTR method
  :name "FTArm"                
  :phases ("Orientation" "Private" "Public" "Consolidation" "Meeting" "Conclusion")
  )

(l*process*define-phase
 :name "Private"
 :documentation "Individual reviewers examine source nodes"
 :objective "Finding issues in the source nodes."
 :roles ("Reviewer" "Producer")
 :synchronicity "Asynchronous"
 :technique "Free Review"
 :exit-condition-fn e*private*sources-reviewed-p
 :display-status-fn e*private*display-status
 :metrics-collection-p t
 )

(l*process*define-role         ;define valid role
 :name "Reviewer"
 :documentation "Participants who review source nodes."
 )

(l*process*define-access-privilege 
  :role "Reviewer"
  :phase "Private"
  :privilege ((:operation create-node 
               :access private 
               :node-schemas ("Issue" "Action" "Comment"))
              (:operation read-node 
               :access public 
               :node-schemas ("Source" "Checklist"))
              (:operation lock-node
               :acess private
               :node-schemas ("Issue" "Action" "Comment"))
              )
  )
  \end{verbatim}
  \normalsize
  \caption{Example of CSRS process language}
  \label{fig:process-language}
\end{figure}


\subsection{User Interface}
CSRS user interface is built on top of Lucid Emacs, an X-Window
version of Emacs editor that supports pull-down menu, pop-up menu, and
scroll bar. A node is displayed as an instance of Emacs
buffer, with its fields as distinguished regions in the buffer. A link
will appear as link-label in the buffer. To follow the link, the user
simply use the {\it mouse} to click on the corresponding label. The
user may also annotate a 
specific text region within a node with another node.

Different node buffers can also be displayed on different Emacs screens. 
CSRS language system allows the user to define screen real estate,
such as where to put the screen on the workstation display and
what node types should be displayed on it.

Finally, one may define a summarized view of node contents
using {\it summary-buffer}. The specific items to be displayed on the
buffer and their display format can be specified in advance.
Figure \ref{fig:csrs-screen} shows a typical CSRS screens.

\begin{figure}[htb]
  {\centerline{\psfig{figure=csrs-screen.ps}}}
  \caption{CSRS Screen}
  \label{fig:csrs-screen}
\end{figure}

\subsection{Review Metrics}
Metrics collection is always part of a well executed review activity.
CSRS collect these metrics automatically and at a
fine-grain level. There are two types of metrics collected by
the system: those associated with review artifacts and their
attributes, and those associated with review efforts (time spent by
participants).  The first type of metrics can be obtained directly
from the corresponding nodes/links generated during the review
(for example, counting the number of issue nodes generated by a
particular participant). To obtain the second type of metrics, we
instrument the primitive CSRS commands with time-stamps. When the user
invokes the command, the system will record the starting time of the
command, the artifact type upon which the command is operated on, and the
current selected screen. 
The time-stamp mechanism allows one to obtain various kinds of
review metrics that relate to user's efforts, such as, how long the
user reads particular nodes, the order of nodes traversal,
or even the order of command invocations from the beginning to
the end of the review session.

\section{System Architecture}
CSRS implements a client-server architecture. Multiple clients send
commands to a shared server. 
The server implements basic hypertext operations,
such as creating, retrieving and deleting nodes and links.
The server also implements event mechanisms
that allow different clients to synchronize their states (e.g., when one
client updates the node, the updates will be propagated to other clients
currently displaying the nodes). This  feature is used to
implement synchronous meeting mode.

The clients are implemented on top of Egret, a generic hypertext system
that provides session management to the server as well as typing
mechanism. The next layer is CSRS engine that provides internal
operations of the system, and the language subsystem that provides
enactable review methods.

One may also add special client running as a background process,
called {\it Agent}, that performs specific 
tasks by listening to specific events from the server.
Some examples of agents include GAgent (which updates internal global
tables of the system), Magent (which listens to review state
and informs the participants through mail messages accordingly), etc. 
The system architecture is shown in Figure \ref{fig:csrs-architecture}

\begin{figure}[htb]
  {\centerline{\psfig{figure=csrs-arch.epsi}}}
  \caption{CSRS Architecture}
  \label{fig:csrs-architecture}
\end{figure}

\subsection{Current Status}


Our first prototype of collaborative software review system (CSRS ver.1)
has been around for a while. It implements an asynchronous review
method \cite{Johnson93}. 
The experience we gain from this system 
(in terms of system usability and ease of use) is generally positive.
However, it also made us realize the need
for generalizing the system to implement other review methods.  

This leads to the development of CSRS version 3 that support generic
modeling capability as discussed above. This version has just recently
been completed. It is currently
used in our research group as part of software quality assurrance
method. We also plan to use the system as an experimental testbed to
investigate various Formal Technical Review factors \cite{CSDL-94-07}. 


\newpage

\bibliography{/group/csdl/bib/csdl-trs,/group/csdl/bib/ftr}
\bibliographystyle{/group/csdl/tex/named-citations}

\end{document}




