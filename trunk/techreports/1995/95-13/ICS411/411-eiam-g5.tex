%%% \documentstyle[11pt,/group/csdl/tex/definemargins,
%%%                        /group/csdl/tex/lmacros]{article} 
%%% 
%%%           \begin{document}
%%%           \begin{center}
%%%           {\large\bf CSRS Experiment Results}\\
%%%           \end{center}
%%%           \small 
\chapter {CSRS Experiment Results (ICS411): Group5 (EIAM)}
\small
	  

\begin{description}
\item [Method:] EIAM
\item [Group:] Group5
\item [Source:] Pass1
\item [Participants:] luzhang (Reviewer), kittu (Reviewer), shuang (Reviewer)
\end{description}
\section{Issue Lists}
\begin{enumerate}
\item {\it Issue\#198 (shuang)}
\begin{description}
\item [Subject:] converror is false
\item [Criticality:] Low
\item [Confidence-level:] Hi
\item [Source-node:] hextonum

\item [Lines:] 18

\item [Description:] converror will not return false after seeing a non-hex number
\end{description}
\item {\it Issue\#202 (kittu)}
\begin{description}
\item [Subject:] hash function
\item [Criticality:] Med
\item [Confidence-level:] Med
\item [Source-node:] Access\_Symtab

\item [Lines:] 17

\item [Description:] could be a mistake in hash function, in
string comparison.
\end{description}
\item {\it Issue\#206 (shuang)}
\begin{description}
\item [Subject:] logical mistake
\item [Criticality:] Hi
\item [Confidence-level:] Hi
\item [Source-node:] Access\_Symtab

\item [Lines:] 29

\item [Description:] after increment ptr, ptr never equal to hash
hence searching alway true, so there is an infinitely loop
\end{description}
\item {\it Issue\#210 (kittu)}
\begin{description}
\item [Subject:] Condition missing
\item [Criticality:] Hi
\item [Confidence-level:] Hi
\item [Source-node:] Access\_Symtab

\item [Lines:] 39

\item [Description:] When the requestcode is STORE, condition (if
requestcode == STORE) to check the requestcode is missing.
\end{description}
\item {\it Issue\#214 (shuang)}
\begin{description}
\item [Subject:] logical mistake
\item [Criticality:] Hi
\item [Confidence-level:] Hi
\item [Source-node:] Access\_Symtab

\item [Lines:] 54

\item [Description:] after increment ptr, ptr never equal to hash, hence inifinite loop
\end{description}
\item {\it Issue\#220 (kittu)}
\begin{description}
\item [Subject:] String comparison
\item [Criticality:] Hi
\item [Confidence-level:] Hi
\item [Source-node:] Access\_Symtab

\item [Lines:] 22

\item [Description:] Erroneous else if condition, of string
comparing blank8.
\end{description}
\item {\it Issue\#224 (luzhang)}
\begin{description}
\item [Subject:] Missing errorflag[8]
\item [Criticality:] Hi
\item [Confidence-level:] Hi
\item [Source-node:] P1\_Proc\_RESW

\item [Lines:] 29

\item [Description:] When error 8 happens, function still return 
newlocctr. Actually at this situation, it should be no newlocctr generated.
\end{description}
\item {\it Issue\#226 (shuang)}
\begin{description}
\item [Subject:] will create a line which might not need
\item [Criticality:] Low
\item [Confidence-level:] Hi
\item [Source-node:] Write\_Int\_File

\item [Lines:] 29-34

\item [Description:] Even if the source code is correct, the
intfile will have a line of 25 'T'
\end{description}
\item {\it Issue\#232 (kittu)}
\begin{description}
\item [Subject:] condition missing
\item [Criticality:] Low
\item [Confidence-level:] Hi
\item [Source-node:] Write\_Int\_File

\item [Lines:] 27

\item [Description:] else statment missing. This would result in
printing T in both the events where errors are found and errors not found.
\end{description}
\item {\it Issue\#236 (luzhang)}
\begin{description}
\item [Subject:] Wrong newlocctr value returned
\item [Criticality:] Hi
\item [Confidence-level:] Hi
\item [Source-node:] P1\_Assign\_Loc

\item [Lines:] 28

\item [Description:] This line gives wrong newlocctr value for 
general machine instruction.
\end{description}
\item {\it Issue\#238 (kittu)}
\begin{description}
\item [Subject:] Limiting condition could be wrong
\item [Criticality:] Low
\item [Confidence-level:] Hi
\item [Source-node:] P1\_Read\_Source

\item [Lines:] 20

\item [Description:] "for" condition ends when MAXERRORS =
25. Whereas array errorflags[i] ends when i=24 since the array starts at 0.
\end{description}
\item {\it Issue\#244 (shuang)}
\begin{description}
\item [Subject:] missing one character
\item [Criticality:] Low
\item [Confidence-level:] Hi
\item [Source-node:] P1\_Read\_Source

\item [Lines:] 57

\item [Description:] operation code includes i = 14
\end{description}
\item {\it Issue\#248 (kittu)}
\begin{description}
\item [Subject:] initialize
\item [Criticality:] Med
\item [Confidence-level:] Hi
\item [Source-node:] P1\_Read\_Source

\item [Lines:] 43-45

\item [Description:] variable 'i' should be initailized to zero
before entering the while loop.
\end{description}
\item {\it Issue\#250 (shuang)}
\begin{description}
\item [Subject:] missing code
\item [Criticality:] Hi
\item [Confidence-level:] Hi
\item [Source-node:] P1\_Read\_Source

\item [Lines:] 

\item [Description:] You miss to process operand field
\end{description}
\item {\it Issue\#254 (kittu)}
\begin{description}
\item [Subject:] value if 'i'
\item [Criticality:] Med
\item [Confidence-level:] Hi
\item [Source-node:] P1\_Read\_Source

\item [Lines:] 57-58

\item [Description:] the while loop is applicable for i=9 to i=13
i.e., five characters.  Where as source.operation is a six character
operation
\end{description}
\item {\it Issue\#258 (luzhang)}
\begin{description}
\item [Subject:] Not real compared
\item [Criticality:] Hi
\item [Confidence-level:] Hi
\item [Source-node:] P1\_Assign\_Sym

\item [Lines:] 17

\item [Description:] Here wrong compared result will encounted.
It assign TABLFULL value to symtabret.
\end{description}
\item {\it Issue\#264 (shuang)}
\begin{description}
\item [Subject:] assign wrong newlocctr value
\item [Criticality:] Low
\item [Confidence-level:] Hi
\item [Source-node:] P1\_Assign\_Loc

\item [Lines:] 28

\item [Description:] newlocctr does not equal to locctr times 3
\end{description}
\item {\it Issue\#268 (shuang)}
\begin{description}
\item [Subject:] wrong if statement
\item [Criticality:] Hi
\item [Confidence-level:] Hi
\item [Source-node:] Pass\_1

\item [Lines:] 33

\item [Description:] We don't close file if there is still
something to read
\end{description}
\item {\it Issue\#272 (kittu)}
\begin{description}
\item [Subject:] 
\item [Criticality:] 
\item [Confidence-level:] 
\item [Source-node:] P1\_Assign\_Loc

\item [Lines:] 

\item [Description:] 
\end{description}
\item {\it Issue\#276 (luzhang)}
\begin{description}
\item [Subject:] Wrong compare result here
\item [Criticality:] Hi
\item [Confidence-level:] Hi
\item [Source-node:] hextonum

\item [Lines:] 22

\item [Description:] Compare before add 3. Wrong result.
\end{description}
\item {\it Issue\#280 (shuang)}
\begin{description}
\item [Subject:] wrong if statement
\item [Criticality:] Low
\item [Confidence-level:] Hi
\item [Source-node:] P1\_Assign\_Sym

\item [Lines:] 12

\item [Description:] when source label is blank, we don't need to
access symbol table
\end{description}
\item {\it Issue\#284 (kittu)}
\begin{description}
\item [Subject:] Wrong parameter being passed
\item [Criticality:] Hi
\item [Confidence-level:] Hi
\item [Source-node:] P1\_Assign\_Sym

\item [Lines:] 16

\item [Description:] The parameter STORE should be passed instead
of SEARCH in the Access\_Symtab function.
\end{description}
\item {\it Issue\#288 (kittu)}
\begin{description}
\item [Subject:] erroneous 'if' condition
\item [Criticality:] Hi
\item [Confidence-level:] Hi
\item [Source-node:] P1\_Assign\_Sym

\item [Lines:] 17

\item [Description:] if statement should have '==' for check but not '='.
\end{description}
\item {\it Issue\#292 (shuang)}
\begin{description}
\item [Subject:] passing the wrong parameter
\item [Criticality:] Hi
\item [Confidence-level:] Hi
\item [Source-node:] P1\_Assign\_Sym

\item [Lines:] 16

\item [Description:] After we found the symbol is not in the symbol
we want to store not to search again
\end{description}
\item {\it Issue\#296 (shuang)}
\begin{description}
\item [Subject:] wrong if statement
\item [Criticality:] Hi
\item [Confidence-level:] Hi
\item [Source-node:] P1\_Assign\_Sym

\item [Lines:] 17

\item [Description:] wrong if statement, you cannot pass the compiler
\end{description}
\item {\it Issue\#298 (kittu)}
\begin{description}
\item [Subject:] erroneous if condition
\item [Criticality:] Med
\item [Confidence-level:] Med
\item [Source-node:] Pass\_1

\item [Lines:] 16

\item [Description:] if condition should be if(!source.comline).
\end{description}
\item {\it Issue\#304 (kittu)}
\begin{description}
\item [Subject:] condition not true
\item [Criticality:] Med
\item [Confidence-level:] Hi
\item [Source-node:] Pass\_1

\item [Lines:] 24-25

\item [Description:] source.labl is a 8 character string.  But the
'for' loop checks for only 6 characters
\end{description}
\item {\it Issue\#312 (luzhang)}
\begin{description}
\item [Subject:] fclose
\item [Criticality:] Med
\item [Confidence-level:] Hi
\item [Source-node:] Pass\_1

\item [Lines:] 33-35

\item [Description:] srcfile never get closed.
\end{description}
\item {\it Issue\#316 (kittu)}
\begin{description}
\item [Subject:] string length
\item [Criticality:] Med
\item [Confidence-level:] Low
\item [Source-node:] Type and var declarations

\item [Lines:] 37-38

\item [Description:] 
\end{description}
\item {\it Issue\#320 (kittu)}
\begin{description}
\item [Subject:] array length
\item [Criticality:] Med
\item [Confidence-level:] Low
\item [Source-node:] Type and var declarations

\item [Lines:] 35-41

\item [Description:] an additonal place should be added to the
string declaration to include the null character.
\end{description}
\item {\it Issue\#324 (shuang)}
\begin{description}
\item [Subject:] need to return the value of newlocctr
\item [Criticality:] Low
\item [Confidence-level:] Low
\item [Source-node:] P1\_Assign\_Loc

\item [Lines:] 11

\item [Description:] I think you need \&newlocctr inorder to 
retrun the value of newlocctr
\end{description}
\item {\it Issue\#328 (shuang)}
\begin{description}
\item [Subject:] need to return the value of newlocctr
\item [Criticality:] Low
\item [Confidence-level:] Low
\item [Source-node:] P1\_Assign\_Loc

\item [Lines:] 14

\item [Description:] I think you need to return the value of
newlocctr, add \& infront of newlocctr
\end{description}
\item {\it Issue\#332 (shuang)}
\begin{description}
\item [Subject:] need the pass the value of newlocctr
\item [Criticality:] Low
\item [Confidence-level:] Low
\item [Source-node:] P1\_Assign\_Loc

\item [Lines:] 17

\item [Description:] I quess you need to return the value of newlocctr
\end{description}
\item {\it Issue\#336 (kittu)}
\begin{description}
\item [Subject:] if condition
\item [Criticality:] Med
\item [Confidence-level:] Med
\item [Source-node:] hextonum

\item [Lines:] 22

\item [Description:] 'i' is being incremented before entering the
'if' statement. Hence the statement is errorneous since the while loop should
read 4 characters.
\end{description}
\item {\it Issue\#340 (shuang)}
\begin{description}
\item [Subject:] parameter passing
\item [Criticality:] Low
\item [Confidence-level:] Low
\item [Source-node:] P1\_Assign\_Loc

\item [Lines:] 20

\item [Description:] I think you want to pass the value of
newlocctr back
\end{description}
\item {\it Issue\#344 (kittu)}
\begin{description}
\item [Subject:] if condition
\item [Criticality:] Med
\item [Confidence-level:] Med
\item [Source-node:] hextonum

\item [Lines:] 22

\item [Description:] 'i' is incremented before entering the 'if'
condition.  Hence it would read three characters when it has to read 4
\end{description}
\item {\it Issue\#346 (shuang)}
\begin{description}
\item [Subject:] passing parameter
\item [Criticality:] Low
\item [Confidence-level:] Low
\item [Source-node:] P1\_Assign\_Loc

\item [Lines:] 23

\item [Description:] Isn't you need to pass the value of newlocctr
back by adding \& infront of newlocctr?
\end{description}
\item {\it Issue\#354 (shuang)}
\begin{description}
\item [Subject:] parameter passing
\item [Criticality:] Low
\item [Confidence-level:] Low
\item [Source-node:] P1\_Assign\_Loc

\item [Lines:] 23

\item [Description:] I quess you need to pass the value of
newlocctr by adding \& infront of newlocctr
\end{description}
\item {\it Issue\#358 (luzhang)}
\begin{description}
\item [Subject:] array out of bound
\item [Criticality:] Hi
\item [Confidence-level:] Hi
\item [Source-node:] P1\_Read\_Source

\item [Lines:] 49

\item [Description:] Case program to access an array cell which
does not exist.
\end{description}
\item {\it Issue\#362 (shuang)}
\begin{description}
\item [Subject:] initialization of i
\item [Criticality:] Low
\item [Confidence-level:] Med
\item [Source-node:] P1\_Proc\_RESW

\item [Lines:] 23

\item [Description:] You haven't initialized i before passing it to
 function dectonum
\end{description}
\item {\it Issue\#366 (luzhang)}
\begin{description}
\item [Subject:] out of array bound
\item [Criticality:] Hi
\item [Confidence-level:] Hi
\item [Source-node:] P1\_Read\_Source

\item [Lines:] 67

\item [Description:] Program acceses array cell which is out of 
array bound.
\end{description}
\end{enumerate}
\section{Review Metrics}
\begin{table}[hb]
\begin{center}
\begin{tabular}{|l|l|l|l|l|}
\hline
Participant & Start-time & End-time & Elapsed-time & Busy-time \\
\hline
shuang & May 05, 1995 11:36:10 & May 05, 1995 13:44:29 & 2:8:19 & 2:1:19 \\
kittu & May 05, 1995 11:37:34 & May 05, 1995 13:40:58 & 2:3:24 & 2:3:24 \\
luzhang & May 05, 1995 11:35:01 & May 05, 1995 13:46:23 & 2:11:22 & 1:56:52 \\
\hline
 & & Total & 6:23:5 & \\
\hline
\end{tabular}
\end{center}
\caption{Review Session}
\end{table}


\begin{table}[hb]
\begin{center}
\begin{tabular}{|l|l|l|l|}
\hline
Source & shuang & kittu & luzhang\\
\hline
(172)Type and var declarations & 4748 & 821 & 4804\\
(174)hextonum & 639 & 946 & 750\\
(176)Access\_Symtab & 1241 & 1357 & 863\\
(178)Write\_Int\_File & 346 & 488 & 2164\\
(180)P1\_Read\_Source & 1480 & 1090 & 1214\\
(182)P1\_Proc\_START & 1284 & 507 & 805\\
(184)P1\_Proc\_RESW & 1032 & 427 & 532\\
(186)P1\_Assign\_Loc & 1103 & 674 & 416\\
(188)P1\_Assign\_Sym & 1242 & 469 & 559\\
(190)Pass\_1 & 1247 & 583 & 1205\\
\hline
\end{tabular}
\end{center}
\caption{Review Time}
\end{table}

\begin{table}[hb]
\begin{center}
\begin{tabular}{|l|l|l|l|}
\hline
Source & shuang & kittu & luzhang\\
\hline
hextonum & 152.1 & 102.7 & 129.6\\
Access\_Symtab & 168.3 & 153.9 & 241.9\\
Write\_Int\_File & 374.6 & 265.6 & 59.9\\
P1\_Read\_Source & 184.9 & 251.0 & 225.4\\
P1\_Proc\_START & 84.1 & 213.0 & 134.2\\
P1\_Proc\_RESW & 108.1 & 261.4 & 209.8\\
P1\_Assign\_Loc & 94.7 & 154.9 & 251.0\\
P1\_Assign\_Sym & 78.3 & 207.2 & 173.9\\
Pass\_1 & 103.9 & 222.3 & 107.6\\
\hline
\end{tabular}
\end{center}
\caption{Paraphrasing Rate (lines/hour)}
\end{table}

\begin{table}[hb]
\begin{center}
\begin{tabular}{|l|l|l|l|l|}
\hline
Source & shuang & kittu & luzhang & OK\\
\hline
Type and var.. &  & 316,320 (=2) &  & \\
hextonum & 198 (=1) & 336,344 (=2) & 276 (=1) & \\
Access\_Symta.. & 206,214 (=2) & 202,210,220 (=3) &  & 214,220\\
Write\_Int\_Fi.. & 226 (=1) & 232 (=1) &  & 226=232\\
P1\_Read\_Sour.. & 244,250 (=2) & 238,248,254 (=3) & 358,366 (=2) & 244=254\\
                 &              &                 &        & 250,238,248 \\
P1\_Proc\_STAR.. &  &  &  & \\
P1\_Proc\_RESW & 362 (=1) &  & 224 (=1) & \\
P1\_Assign\_Lo.. & 264,324,328,332 & 272 (=1) & 236 (=1) & 264 \\
 & 340,346,354 (=7) &  &  & \\
P1\_Assign\_Sy.. & 280,292,296 (=3) & 284,288 (=2) & 258 (=1) & 280,292=284\\
                 &              &                 &        & 296=288=258 \\
Pass\_1 & 268 (=1) & 298,304 (=2) & 312 (=1) & \\
\hline
\end{tabular}
\caption{Source node v.s Issue node}
\end{center}
\end{table}

%%\end{document}
