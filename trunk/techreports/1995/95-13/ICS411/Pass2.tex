%%% \documentstyle[11pt,/group/csdl/tex/definemargins,
%%%                        /group/csdl/tex/lmacros]{article} 
%%% 
%%%           \begin{document}
%%%           \begin{center}
%%%           {\large\bf CSRS Experiment -- Source Listing \\
%%%             Source: Pass2 }
%%% \end{center}
%%% 
\chapter {CSRS Experiment (ICS411): Source Listing -- Pass2}
\small
	  

\section{Type and var declarations}
\subsection*{Specification:}

\subsection*{Source-code:}
\begin{verbatim}
001:/* 
002:THIS IS A SIMPLE ASSEMBLER FOR SIC (STANDARD VERSION) PASS 2.
003:IT READS INTERMEDIATE FILE AND GENERATES OBJECT CODE.
004:
005:THE SOURCE PROGRAM TO BE ASSEMBLED (IN PASS 1) MUST BE IN FIXED 
006:FORMAT AS FOLLOWS:
007:      BYTES  0-7  LABEL
008:               8  BLANK
009:             9-14 OPERATION CODE
010:            15-16 BLANK
011:            17-34 OPERAND
012:            35-65 COMMENT
013:
014:IF A SOURCE LINE CONTAINS "." IN THE FIRST BYTE, THE ENTIRE LINE IS
015:TREATED AS A COMMENT. 
016:*/
017:
018:#ifndef true
019:# define true    1
020:# define false   0
021:#endif
022:
023:#  define EXIT_SUCCESS  0
024:#  define EXIT_FAILURE  1
025:
026:#define QUOTE           '\''
027:
028:#define MAXERRORS       25   /* SIZE OF ARRAY OF ERROR FLAGS */
029:#define MAXOPS          25   /* SIZE OF OPCODE TABLE */
030:
031:#define BLANK6          "      "
032:#define BLANK8          "        "
033:#define BLANK18         "                  "
034:#define BLANK30         "                              "
035:#define BLANK31         "                               "
036:
037:typedef unsigned char BOOLEAN;
038:
039:typedef char CHAR4 [4];
040:typedef char CHAR6 [6];
041:typedef char CHAR8 [8];
042:typedef char CHAR18 [18];
043:typedef char CHAR30 [30];
044:typedef char CHAR31 [31];
045:typedef char CHAR50 [50];
046:typedef char CHAR66 [66];
047:
048:typedef struct SOURCETYPE {
049:  /* SOURCE LINE AND SUBFIELDS */
050:  CHAR66 line;
051:  BOOLEAN comline;
052:  CHAR8 labl;
053:  CHAR6 operation;
054:  CHAR18 operand;
055:  CHAR31 comment;
056:} SOURCETYPE;
057:
058:typedef struct OBJTYPE {
059:  /* OBJECT CODE, LENGTH, AND TYPE */
060:  enum { HEADREC, TEXTREC, ENDREC, NONE } rectype;
061:  int objlength;
062:  CHAR30 objcode;
063:} OBJTYPE;
064:
065:typedef struct REC_OPTABTYPE {
066:  CHAR6 mnemonic;
067:  int opcode;
068:} REC_OPTABTYPE;
069:
070:
071:typedef enum {
072:  SEARCH, STORE} 
073:SYMREQTYPE;
074:
075:typedef enum {
076:  FOUND, NOTFOUND, ADDED, DUPLICATE, TABLEFULL
077:}SYMRETTYPE;
078:
079:typedef enum {
080:  NORMAL, ENDFILE
081:} INTRETTYPE;
082:
083:typedef enum {
084:  VALIDOP, INVALIDOP
085:} OPRETTYPE;
086:
087:
088:/* OPCODE TABLE--ASSUME THIS TABLE IS INITIALIZED PROPERLY */
089:static REC_OPTABTYPE OPTAB[MAXOPS];
090:
091:/*ASCII CONVERSION TABLE--ASSUME IT IS INITIALIZED PROPERLY */
092:static int ASCII[256]; 
093:
094:static FILE *srcfile, *objfile, *lisfile, *intfile;
095:static int LOCCTR;
096:static CHAR6 PROGNAME;
097:static int PROGSTART;
098:
099:/* GLOBALS USED ONLY BY P2_ASSEMBLE_INST */
100:
101:static BOOLEAN FIRSTSTMT, ENDFOUND;
102:
103:/* GLOBALS USED ONLY BY P2_WRITE_OBJ */
104:
105:static int TEXTSTART, TEXTADDR, TEXTLENGTH;
106:static char TEXTARRAY[60];
107:
108:/* ERROR MESSAGES */
109:static CHAR50 ERRMSG[MAXERRORS] = {
110:  "ILLEGAL FORMAT IN LABEL FIELD                     ",/*ERRMSG[0]*/
111:  "MISSING OPERATION CODE                            ",/*ERRMSG[1]*/
112:  "ILLEGAL FORMAT IN OPERATION FIELD                 ",/*ERRMSG[2]*/
113:  "MISSING OR MISPLACED OPERAND IN START STATEMENT   ",/*ERRMSG[3]*/
114:  "ILLEGAL OPERAND IN START STATEMENT                ",/*ERRMSG[4]*/
115:  "ILLEGAL OPERAND IN BYTE STATEMENT                 ",/*ERRMSG[5]*/
116:  "ODD LENGTH HEX STRING IN BYTE STATEMENT           ",/*ERRMSG[6]*/
117:  "MISSING OR MISPLACED OPERAND IN BYTE STATEMENT    ",/*ERRMSG[7]*/
118:  "MISSING OR MISPLACED OPERAND IN RESW STATEMENT    ",/*ERRMSG[8]*/
119:  "ILLEGAL OPERAND IN RESW STATEMENT                 ",/*ERRMSG[9]*/
120:  "MISSING OR MISPLACED OPERAND IN RESB STATEMENT    ",/*ERRMSG[10]*/
121:  "ILLEGAL OPERAND IN RESB STATEMENT                 ",/*ERRMSG[11]*/
122:  "DUPLICATE LABEL DEFINITION                        ",/*ERRMSG[12]*/
123:  "TOO MANY SYMBOLS IN SOURCE PROGRAM                ",/*ERRMSG[13]*/
124:  "DUPLICATE OR MISPLACED START STATEMENT            ",/*ERRMSG[14]*/
125:  "MISSING OR MISPLACED START STATEMENT              ",/*ERRMSG[15]*/
126:  "ILLEGAL OPERAND IN WORD STATEMENT                 ",/*ERRMSG[16]*/
127:  "MISSING OR MISPLACED OPERAND IN WORD STATEMENT    ",/*ERRMSG[17]*/
128:  "MISSING OR MISPLACED OPERAND IN END STATEMENT     ",/*ERRMSG[18]*/
129:  "ILLEGAL OPERAND IN END STATEMENT                  ",/*ERRMSG[19]*/
130:  "UNDEFINED SYMBOL IN OPERAND                       ",/*ERRMSG[20]*/
131:  "STATEMENT SHOULD NOT FOLLOW END STATEMENT         ",/*ERRMSG[21]*/
132:  "ILLEGAL OPERAND FIELD                             ",/*ERRMSG[22]*/
133:  "UNRECOGNIZED OPERATION CODE                       ",/*ERRMSG[23]*/
134:  "MISSING OR MISPLACED OPERAND IN INSTRUCTION       " /*ERRMSG[24]*/
135:  };
136:
137:
138:
139:
\end{verbatim}
\section{dectonum}
\subsection*{Specification:}
THIS FUNCTIONS SCANS THE STRING STR, BEGINNING AT THE BYTE POSITION GIVEN BY
FIRST, FOR THE CHARACTER REPRESENTATION OF A DECIMAL VALUE.  THIS VALUE,
CONVERTED TO NUMERIC, IS RETURNED AS THE VALUE OF THE FUNCTION, AND LAST IS
SET TO INDICATE THE NEXT CHARACTER IN STR AFTER THE VALUE THAT WAS FOUND. IF
ANY SCANNING OR CONVERSION ERRORS ARE FOUND, CONVERROR IS SET TO TRUE. THE
MAXIMUM LENGTH VALUE TO BE SCANNED IS 4 DECIMAL DIGITS.
\subsection*{Source-code:}
\begin{verbatim}
001:int dectonum(char *str,int first, int *last, BOOLEAN *converror)
002:
003:{
004:  int n, i;
005:  BOOLEAN scanning;
006:
007:  n = 0;
008:  scanning = true;
009:  *converror = false;
010:  while (scanning) {
011:    if (str[first + i] >= '0' && str[first + i] <= '9')
012:      n = n * 10 + str[first + i] - '0';
013:    else if (str[first + i] == ' ')
014:      scanning = false;
015:    else 
016:      *converror = true;
017:    i++;
018:  }
019:  *last = first + i;
020:  return n;
021:}
022:
023:
\end{verbatim}
\section{Read\_Int\_File}
\subsection*{Specification:}
THIS FUNCTION READS INTERMEDIATE FILE INTFILE.  THE VARIABLES DESCRIBED ABOVE
ARE SET WITH THE VALUES READ FROM THE INTERMEDIATE FILE. VARIABLES THAT ARE
NOT REPRESENTED IN THE FILE (FOR EXAMPLE, SOURCE.LABL AND ERRORSFOUND FOR A
COMMENT LINE) ARE SET TO BLANK (FOR CHARACTER FIELDS) OR TO FALSE (FOR
BOOLEAN VARIABLES). IF THE END OF FILE HAS BEEN REACHED, RETURNS ENDFILE;
OTHERWISE RETURNS NORMAL.

THE INTERMEDIATE FILE CONSISTS OF THE CURRENT SOURCE PROGRAM LINE, FOLLOWED
BY A BOOLEAN VALUE (T OR F) THAT INDICATES WHETHER THIS IS A COMMENT LINE,
AND THE CURRENT LOCATION COUNTER VALUE. FOR NON-COMMENT LINES, THIS IS ALSO
FOLLOWED BY SUBFIELDS AND THE VALUE OF ERRORSFOUND (T OR F). IF AND ONLY IF
ERRORSFOUND WAS TRUE, THIS AGAIN IS FOLLOWED BY THE VALUES IN ERRORFLAGS.
\subsection*{Source-code:}
\begin{verbatim}
001:INTRETTYPE Read_Int_File (SOURCETYPE *source,
002:                      BOOLEAN *errorsfound,
003:                      BOOLEAN *errorflags)
004:
005:{
006:  int i;
007:  char ch;
008:
009:  if ((ch=getc(intfile))==EOF)
010:    return(ENDFILE);
011:
012:  ungetc(ch,intfile);
013:  for (i = 0; i <= 65; i++) 
014:    source->line[i] = getc(intfile);
015:  ch = getc(intfile);
016:  if (ch == 'T')
017:    source->comline = true;
018:  source->comline = false;
019:  if (source->comline) {
020:    memcpy(source->labl, BLANK8, sizeof(CHAR8));
021:    memcpy(source->operation, BLANK6, sizeof(CHAR6));
022:    memcpy(source->operand, BLANK18, sizeof(CHAR18));
023:    memcpy(source->comment, BLANK31, sizeof(CHAR31));
024:    *errorsfound = false;
025:    for (i = 0; i < MAXERRORS; i++)
026:      errorflags[i] = false;
027:    return(NORMAL);
028:  }
029:  for (i = 0; i <= 7; i++) 
030:    source->labl[i] = getc(intfile);
031:  for (i = 0; i <= 4; i++) 
032:    source->operation[i] = getc(intfile);
033:  for (i = 0; i <= 17; i++) 
034:    source->operand[i] = getc(intfile);
035:  for (i = 0; i <= 30; i++) 
036:    source->comment[i] = getc(intfile);
037:  ch = getc(intfile);
038:  getc(intfile);            /*discard newline*/
039:  if (ch == 'T')
040:    *errorsfound = true;
041:  else
042:    *errorsfound = false;
043:
044:  if(*errorsfound){
045:    for (i = 0; i < MAXERRORS; i++) {
046:      ch = getc(intfile);
047:      if (ch == 'T')
048:    errorflags[i] = true;
049:      else
050:    errorflags[i] = false;
051:    }
052:    getc(intfile);          /*discard newline*/
053:  }
054:  else
055:    for (i = 0; i < MAXERRORS; i++)
056:      errorflags[i] = false;
057:
058:  return (NORMAL);
059:}
060:
061:
\end{verbatim}
\section{P2\_Search\_Optab}
\subsection*{Specification:}
THIS PROCEDURE SEARCHES THE OPERATION CODE TABLE (OPTAB) FOR THE MNEMONIC
PASSED AS PARAMETER. IF THE MNEMONIC IS FOUND, RETURNCODE IS SET TO VALIDOP
AND OPCODE IS SET TO THE VALUE GIVEN IN OPTAB. OTHERWISE, RETURNCODE IS SET
TO INVALIDOP AND OPCODE IS SET TO 255.

THE ENTRIES IN OPTAB ARE ORDERED BY MNEMONIC. THIS PROCEDURE USES A BINARY
SEARCH.
\subsection*{Source-code:}
\begin{verbatim}
001:void P2_Search_Optab(char *mnemonic,
002:                 OPRETTYPE *returncode,
003:                 int *opcode)
004:
005:{
006:  int low, mid, high;
007:
008:  high = MAXOPS - 1;
009:  low = 0;
010:  do {
011:    mid = (low + high) / 2;
012:    if (strncmp(mnemonic, OPTAB[mid].mnemonic, sizeof(CHAR6)) < 0)
013:      high = mid + 1;
014:    else
015:      low = mid - 1;
016:  } while (strncmp(mnemonic, OPTAB[mid].mnemonic, sizeof(CHAR6)) &&
017:       high >= low);
018:  if (strncmp(mnemonic, OPTAB[mid].mnemonic, sizeof(CHAR6))) {
019:    /*FOUND VALID OPCODE*/
020:    *returncode = VALIDOP;
021:    *opcode = OPTAB[mid].opcode;
022:  } 
023:  else {
024:    *returncode = INVALIDOP;
025:    *opcode = 255;
026:  }
027:
028:}
029:
030:
031:
\end{verbatim}
\section{P2\_Proc\_START}
\subsection*{Specification:}
PROCESS START STATEMENT -- IF THIS IS THE FIRST SOURCE LINE, SET OBJECT TYPE
= HEADREC AND OBJECT CODE = PROGRAM NAME.
\subsection*{Source-code:}
\begin{verbatim}
001:void P2_Proc_START(SOURCETYPE source,
002:               BOOLEAN *errorsfound,
003:               BOOLEAN *errorflags,
004:               OBJTYPE *objct)
005:
006:{
007:  int i;
008:  
009:  if (!FIRSTSTMT) {
010:    *errorsfound = true;
011:    errorflags[14] = true;/*DUPLICATE OR MISPLACED START STATEMENT*/
012:  }
013:  objct->rectype = HEADREC;
014:  objct->objlength = 0;
015:  for (i = 0; i <= 5; i++)
016:    objct->objcode[i] = source.labl[i];
017:  for (i = 6; i <= 14; i++)
018:    objct->objcode[i] = ' ';
019:}
020:
021:
\end{verbatim}
\section{P2\_Proc\_BYTE}
\subsection*{Specification:}
PROCESS BYTE STATEMENT -- THE OPERAND MUST BE EITHER C'...' OR X'...'.  IF A
FORMAT ERROR IN THE OPERAND (ERRORS 5, 6, 7) WAS DETECTED PREVIOUSLY, DO NOT
ATTEMPT TO ASSEMBLE ASSUME ASCII CONVERSION TABLE HAS BEEN INITIALIZED
PROPERLY.
\subsection*{Source-code:}
\begin{verbatim}
001:void P2_Proc_BYTE (SOURCETYPE source,
002:               int LOCCTR,
003:               BOOLEAN *errorsfound,
004:               BOOLEAN *errorflags,
005:               OBJTYPE *objct)
006:
007:{
008:
009:  int i;
010:  CHAR4 temp;
011:  char hexchar;
012:  int asciival;
013:
014:  if (!errorflags[5] && !errorflags[6] && !errorflags[7]) {
015:    /* OPERAND IS C'...' -- USE THE ASCII CONVERSION TABLE TO
016:       FIND THE ASCII CODE FOR EACH CHARACTER AND PACK INTO
017:       OBJECT CODE. SET OBJECT TYPE = TEXTREC. */
018:
019:    if (source.operand[0] = 'C') {
020:      i = 1;
021:      while (source.operand[i + 1] != QUOTE) {
022:    asciival = ASCII[source.operand[i + 1]];
023:    numtohex(asciival, temp);
024:    objct->objcode[i * 2 - 2] = temp[2];
025:    objct->objcode[i * 2 - 1] = temp[3];
026:    i++;
027:      }
028:      objct->objlength = i - 1;
029:      objct->rectype = TEXTREC;
030:    } 
031:    else {
032:      /* OPERAND IS X'...' -- PACK HEX DIGITS INTO OBJECT CODE AND
033:     SET OBJECT TYPE = TEXTREC.
034:     ERRORS DETECTED: 5 */
035:      while (source.operand[i + 1] != QUOTE) {
036:    hexchar = source.operand[i + 1];
037:    if (isdigit(hexchar) || (hexchar >= 'A' && hexchar <= 'F'))
038:      objct->objcode[i - 1] = hexchar;
039:    else {
040:      *errorsfound = true;
041:      errorflags[5] = true; /* ILLEGAL OPERAND IN BYTE */
042:    }
043:    i++;
044:      }
045:      objct->objlength = i - 1;
046:      objct->rectype = TEXTREC;
047:    }/*else*/
048:  }
049:
050:}
051:
052:
053:
\end{verbatim}
\section{P2\_Assemble\_Inst}
\subsection*{Specification:}
THIS PROCEDURE GENERATES THE OBJECT CODE (IF ANY) FOR THE SOURCE STATEMENT
CURRENTLY BEING PROCESSED. THE OBJECT CODE GENERATED IS PLACED IN OBJECT
(THIS RECORD ALSO INCLUDES AN INDICATION OF THE TYPE OF OBJECT CODE AND THE
LENGTH).

THIS PROCEDURE ALSO TESTS FOR ERRORS SUCH AS A MISSING START OR END OR
STATEMENTS IMPROPERLY FOLLOWING THE END STATEMENT. IN ORDER TO DO THIS, IT
MAKES USE OF THE GLOBAL VARIABLES FIRSTSTMT AND ENDFOUND.
\subsection*{Source-code:}
\begin{verbatim}
001:void P2_Assemble_Inst(SOURCETYPE source,
002:                  int LOCCTR,
003:                  BOOLEAN *errorsfound,
004:                  BOOLEAN *errorflags,
005:                  OBJTYPE *objct)
006:
007:
008:{
009:  
010:  if (ENDFOUND) {
011:    *errorsfound = true;
012:    errorflags[21] = true;   /* STATEMENT SHOULD NOT FOLLOW END */
013:  }
014:
015:  if (!strncmp(source.operation, "START ", sizeof(CHAR6))) 
016:    P2_Proc_START(source,errorsfound,errorflags,objct);
017:  
018:  else if (!strncmp(source.operation, "WORD  ", sizeof(CHAR6))) 
019:    P2_Proc_WORD(source,LOCCTR,errorsfound,errorflags,objct);
020:
021:  else if (!strncmp(source.operation, "BYTE  ", sizeof(CHAR6)))
022:    P2_Proc_BYTE(source,LOCCTR,errorsfound,errorflags,objct);
023:
024:  else if (!strncmp(source.operation, "RESB  ", sizeof(CHAR6)) ||
025:       !strncmp(source.operation, "RESW  ", sizeof(CHAR6))) {
026:    /* NO OBJECT CODE FOR RESB OR RESW */
027:    
028:    objct->rectype = NONE;
029:  } 
030:  
031:  else if (!strncmp(source.operation, "END   ", sizeof(CHAR6))) 
032:    P2_Proc_END(source,LOCCTR,errorsfound,errorflags,objct);
033:  
034:  else 
035:    P2_Proc_INST(source,LOCCTR,errorsfound,errorflags,objct);
036:
037:  if(FIRSTSTMT && !strncmp(source.operation,"START ",sizeof(CHAR6))){
038:   /*THE FIRST SOURCE STATEMENT (EXCEPT FOR COMMENTS) MUST BE START*/
039:    *errorsfound = true;
040:    errorflags[15] = true; /*MISSING OR MISPLACED START STATEMENT*/
041:  }
042:}
043:
044:
045:
\end{verbatim}
\section{P2\_Write\_Obj}
\subsection*{Specification:}
THIS PROCEDURE PLACES THE GENERATED OBJECT CODE INTO THE OBJECT
PROGRAM. THERE ARE THREE TYPES OF OBJECT CODE TO BE HANDLED: HEADREC (FROM
START STATEMENT), ENDREC (FROM END STATEMENT), AND TEXTREC (FROM INSTRUCTIONS
AND WORD AND BYTE STATEMENTS).

TO KEEP TRACK OF THE TEXT RECORD CURRENTLY BEING CONSTRUCTED, THIS PROCEDURE
USES THE GLOBAL VARIABLES TEXTSTART, TEXTADDR, TEXTLENGTH, AND TEXTARRAY.
\subsection*{Source-code:}
\begin{verbatim}
001:void P2_Write_Obj(OBJTYPE objct,
002:              int locctr,
003:              char *progname,
004:              int proglength)
005:
006:
007:{
008:  int i;
009:  CHAR4 temp;
010:  int textbytes, forlim;
011:
012:  if (objct.rectype == HEADREC) {
013:    fprintf(objfile, "H%.6s", progname);
014:    numtohex(locctr, temp);
015:    fprintf(objfile, "00%.4s", temp);
016:    numtohex(locctr, temp);
017:    fprintf(objfile, "00%.4s\n", temp);
018:    return;
019:  }
020:
021:  /* HEADREC -- GENERATE HEADER RECORD IN OBJECT PROGRAM */
022:
023:  if (objct.rectype == TEXTREC) {
024:    /*TEXTREC -- PUT OBJECT CODE INTO A TEXT RECORD. IF THE OBJECT
025:      CODE WILL NOT FIT INTO THE CURRENT TEXT RECORD, OR IF ADDRESSES
026:      ARE NOT CONTIGUOUS, A NEW TEXT RECORD MUST BE STARTED. */
027:
028:    if (TEXTLENGTH == 0) {
029:      TEXTADDR = locctr;
030:      TEXTSTART = locctr;
031:    }
032:    if (TEXTLENGTH + objct.objlength > 60 && locctr != TEXTADDR) {
033:      putc('T', objfile);
034:      numtohex(TEXTSTART, temp);
035:      fprintf(objfile, "00%.4s", temp);
036:      textbytes = TEXTLENGTH * 2;
037:      numtohex(textbytes, temp);
038:      fprintf(objfile, "%c%c", temp[2], temp[3]);
039:      forlim = TEXTLENGTH;
040:      for (i = 0; i < forlim; i++)
041:    putc(TEXTARRAY[i], objfile);
042:      putc('\n', objfile);
043:      TEXTLENGTH = 0;
044:      TEXTSTART = locctr;
045:    }
046:    for (i = 0; i <= objct.objlength; i++)
047:      TEXTARRAY[TEXTLENGTH + i] = objct.objcode[i];
048:    TEXTLENGTH += objct.objlength;
049:    TEXTADDR = locctr + objct.objlength / 2;
050:    return;
051:  }
052:
053:  /* ENDREC -- WRITE OUT THE LAST TEXT RECORD (IF THERE IS ANYTHING
054:     IN IT) AND THEN GENERATE THE END RECORD */
055:
056:  if (TEXTLENGTH != 0) {
057:    putc('T', objfile);
058:    numtohex(TEXTSTART, temp);
059:    fprintf(objfile, "00%.4s", temp);
060:    textbytes = TEXTLENGTH / 2;
061:    numtohex(textbytes, temp);
062:    fprintf(objfile, "%c%c", temp[2], temp[3]);
063:    forlim = TEXTLENGTH;
064:    for (i = 0; i < forlim; i++)
065:      putc(TEXTARRAY[i], objfile);
066:    putc('\n', objfile);
067:  }
068:  putc('E', objfile);
069:  for (i = 0; i < objct.objlength; i++)
070:    putc(objct.objcode[i], objfile);
071:  putc('\n', objfile);
072:}
073:
074:
075:
\end{verbatim}
\section{Pass\_2}
\subsection*{Specification:}
THIS IS THE MAIN PROCEDURE FOR PASS 2. IT READS EACH LINE FROM THE
INTERMEDIATE FILE, AND CALLS P2\_ASSEMBLE\_INST AND P2\_WRITE\_OBJ FOR EACH
NON-COMMENT LINE. HOWEVER, P2\_WRITE\_OBJ IS CALLED ONLY IF GENOBJECT =
TRUE. GENOBJECT IS SET TO FALSE (TO SUPPRESS THE OBJECT PROGRAM) IF ANY
ASSEMBLY ERRORS ARE DETECTED. P2\_WRITE\_LIST IS CALLED FOR EVERY LINE
PROCESSED.
\subsection*{Source-code:}
\begin{verbatim}
001:void Pass_2()
002:
003:{
004:  SOURCETYPE source;   /* SOURCE LINE AND SUBFIELDS */
005:  BOOLEAN errorsfound; /* TRUE IF ANY ERRORS IN CURRENT STMT */
006:  BOOLEAN errorflags[MAXERRORS];  /* ERROR FLAGS FOR CURRENT STMT */
007:  OBJTYPE objct; /* OBJECT CODE FOR CURRENT STATEMENT */
008:  int proglength;
009:
010:  proglength = LOCCTR - PROGSTART;
011:  FIRSTSTMT = true;
012:  ENDFOUND = false;
013:  TEXTLENGTH = 0;
014:  rewind(intfile);
015:  rewind(lisfile);
016:  rewind(objfile);
017:
018:  while (Read_Int_File (&source, &errorsfound, errorflags)!=ENDFILE){
019:    objct.rectype = NONE;
020:    objct.objlength = 0;
021:    memcpy(objct.objcode, BLANK30, sizeof(CHAR30));
022:    if (!source.comline) {
023:      P2_Assemble_Inst(source,LOCCTR,&errorsfound,errorflags,&objct);
024:      if (!errorsfound)
025:    P2_Write_Obj(objct, LOCCTR, PROGNAME, proglength);
026:    }
027:    P2_WRITE_LIST(source,objct,errorflags);
028:  }
029:  if (intfile != NULL)
030:    fclose(intfile);
031:  intfile = NULL;
032:  if (lisfile != NULL)
033:    fclose(lisfile);
034:  lisfile = NULL;
035:  if (objfile != NULL)
036:    fclose(objfile);
037:  objfile = NULL;
038:}
039:
\end{verbatim}
%%\end{document}
