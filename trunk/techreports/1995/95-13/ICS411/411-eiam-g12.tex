%%% \documentstyle[11pt,/group/csdl/tex/definemargins,
%%%                        /group/csdl/tex/lmacros]{article} 
%%% 
%%%           \begin{document}
%%%           \begin{center}
%%%           {\large\bf CSRS Experiment Results}\\
%%%           \end{center}
%%%           \small 

\chapter {CSRS Experiment Results (ICS411): Group12 (EIAM)}
\small
	  

\begin{description}
\item [Method:] EIAM
\item [Group:] Group12
\item [Source:] Pass2
\item [Participants:] awong (Reviewer), kawak (Reviewer), scheung (Reviewer)
\end{description}
\section{Issue Lists}
\begin{enumerate}
\item {\it Issue\#196 (kawak)}
\begin{description}
\item [Subject:] var not initialized
\item [Criticality:] Med
\item [Confidence-level:] Hi
\item [Source-node:] dectonum

\item [Lines:] 11-12

\item [Description:] variable integer i is not assigned an initial value before it is used to
evaluate the array position in selected region.  This may lead to invalid
references to memory areas.
\end{description}
\item {\it Issue\#200 (scheung)}
\begin{description}
\item [Subject:] The value generate may not be numeric
\item [Criticality:] Med
\item [Confidence-level:] Hi
\item [Source-node:] dectonum

\item [Lines:] 12

\item [Description:] str[first+i] - '0' must be computed before
n * 10 is added to it.  As it stands now, n * 10 will be added to
str[first+i] first.  str[first+i] is not a true number, until it is
subtracted from '0'.
\end{description}
\item {\it Issue\#204 (scheung)}
\begin{description}
\item [Subject:] i is not initilized to a value
\item [Criticality:] Hi
\item [Confidence-level:] Hi
\item [Source-node:] dectonum

\item [Lines:] 7-8

\item [Description:] i is not initilized, we don't know what value
is i going into the while loop.  Is very possible that we are not looking at
the right location where the decimal is located.
\end{description}
\item {\it Issue\#208 (kawak)}
\begin{description}
\item [Subject:] if statement always overidden
\item [Criticality:] Low
\item [Confidence-level:] Hi
\item [Source-node:] Read\_Int\_File

\item [Lines:] 16-19

\item [Description:] whether or not the if statement is executed, the value of source-{\tt >}comline
will always be set to fals.
\end{description}
\item {\it Issue\#218 (kawak)}
\begin{description}
\item [Subject:] code neverf executed
\item [Criticality:] Low
\item [Confidence-level:] Hi
\item [Source-node:] Read\_Int\_File

\item [Lines:] 19-29

\item [Description:] block of code is never executed because source-{\tt >}comline always false at this
point.
\end{description}
\item {\it Issue\#222 (scheung)}
\begin{description}
\item [Subject:] pass pointer of asciival to numtohex
\item [Criticality:] Hi
\item [Confidence-level:] Hi
\item [Source-node:] P2\_Proc\_BYTE

\item [Lines:] 23

\item [Description:] If we don't pass a pointer to asciival,
integers such as asciival will not be updated by the function because a copy
of the value is passed to the function.
\end{description}
\item {\it Issue\#224 (awong)}
\begin{description}
\item [Subject:] Reset comline back to false even if true
\item [Criticality:] Hi
\item [Confidence-level:] Med
\item [Source-node:] Read\_Int\_File

\item [Lines:] 16-18

\item [Description:] Even if comline is true it will then be reset
to false
\end{description}
\item {\it Issue\#230 (scheung)}
\begin{description}
\item [Subject:] i is not initilized
\item [Criticality:] Hi
\item [Confidence-level:] Hi
\item [Source-node:] P2\_Proc\_BYTE

\item [Lines:] 35-36

\item [Description:] i is not initilized to a value, we don't know
where exactly we are looking at with sourc.operand[i+1].
\end{description}
\item {\it Issue\#234 (scheung)}
\begin{description}
\item [Subject:] The header record contains the progname, starting
address, and program length.
\item [Criticality:] Low
\item [Confidence-level:] Hi
\item [Source-node:] P2\_Proc\_BYTE

\item [Lines:] 

\item [Description:] 
\end{description}
\item {\it Issue\#236 (scheung)}
\begin{description}
\item [Subject:] the header has the program name, starting address,
and length of program.
\item [Criticality:] Low
\item [Confidence-level:] Hi
\item [Source-node:] P2\_Write\_Obj

\item [Lines:] 14-17

\item [Description:] This header should have the starting address
and the length of the program.  In this example, looks like the starting
address and the length of the program is the same.
\end{description}
\item {\it Issue\#240 (kawak)}
\begin{description}
\item [Subject:] always error on comment
\item [Criticality:] Low
\item [Confidence-level:] Hi
\item [Source-node:] Read\_Int\_File

\item [Lines:] 39-43

\item [Description:] Whenever the assembler encounters a comment line, there will be an error
reported by *errorsfound.
\end{description}
\item {\it Issue\#244 (scheung)}
\begin{description}
\item [Subject:] the value of TEXTADDR is wrong
\item [Criticality:] Hi
\item [Confidence-level:] Hi
\item [Source-node:] P2\_Write\_Obj

\item [Lines:] 49-50

\item [Description:] Should add locctr and objct.objlength first.
\end{description}
\item {\it Issue\#248 (awong)}
\begin{description}
\item [Subject:] repeating steps
\item [Criticality:] Low
\item [Confidence-level:] Low
\item [Source-node:] P2\_Write\_Obj

\item [Lines:] 14-17

\item [Description:] write to objfile location counter when could
be done in one step
\end{description}
\item {\it Issue\#252 (scheung)}
\begin{description}
\item [Subject:] No function declaractions.
\item [Criticality:] Hi
\item [Confidence-level:] Hi
\item [Source-node:] Type and var declarations

\item [Lines:] 

\item [Description:] In the declarations, we need to define the
functions.  You cannot use a function without declaring the function first.
\end{description}
\item {\it Issue\#254 (kawak)}
\begin{description}
\item [Subject:] if clause never executed
\item [Criticality:] Low
\item [Confidence-level:] Med
\item [Source-node:] P2\_Assemble\_Inst

\item [Lines:] 10-13

\item [Description:] this function is not called after the ENDFOUND flag is set, so this if
statement will always be false.
\end{description}
\item {\it Issue\#258 (awong)}
\begin{description}
\item [Subject:] all lines become non comment lines
\item [Criticality:] Med
\item [Confidence-level:] Med
\item [Source-node:] Read\_Int\_File

\item [Lines:] 18-28

\item [Description:] Initialize source-{\tt >}comline to false will cause
even comment lines to copying of it to the variable source
\end{description}
\item {\it Issue\#262 (scheung)}
\begin{description}
\item [Subject:] source-{\tt >}comline will always be false
\item [Criticality:] Med
\item [Confidence-level:] Hi
\item [Source-node:] Read\_Int\_File

\item [Lines:] 16-19

\item [Description:] source-{\tt >}comline will always be false, because
no matter what happen to the if statement above it, source-{\tt >}comline will be
set to false.  The correct logic should be:

	if (ch == 'T')
		source-{\tt >}comline = true;
	else
		source-{\tt >}comline = false;
\end{description}
\item {\it Issue\#266 (awong)}
\begin{description}
\item [Subject:] reversal of if then procedure
\item [Criticality:] Med
\item [Confidence-level:] Med
\item [Source-node:] P2\_Search\_Optab

\item [Lines:] 18-26

\item [Description:] the first if statement will return 0 if it is
true cause the then part of the if to fail which should be executed if the
mnemonic is found.  the two "then"," else" statements should be reversed.
\end{description}
\item {\it Issue\#270 (scheung)}
\begin{description}
\item [Subject:] FIRSTSTMT is never set to false
\item [Criticality:] Hi
\item [Confidence-level:] Med
\item [Source-node:] P2\_Proc\_START

\item [Lines:] 

\item [Description:] I could not find in any of the functions where
FIRSTSTMT is set to false.  We are going to get errors in P2\_Assemble\_Inst,
the last if statement checks to see if FIRSTSTMT is true or not.
\end{description}
\item {\it Issue\#272 (kawak)}
\begin{description}
\item [Subject:] *last set to wrong value
\item [Criticality:] Low
\item [Confidence-level:] Med
\item [Source-node:] dectonum

\item [Lines:] 19-20

\item [Description:] *last is set to first + i, after i is incremented another time by the
previous while loop.  i no longer reflects the end of the current number, but
the next space after it.
\end{description}
\item {\it Issue\#276 (awong)}
\begin{description}
\item [Subject:] i not initialized
\item [Criticality:] Med
\item [Confidence-level:] Med
\item [Source-node:] P2\_Proc\_BYTE

\item [Lines:] 31-43

\item [Description:] in this section the i variable has not been
initalized therefore will probably contain the value of 0 and cause an
incorrect evaluation of the while statement
\end{description}
\item {\it Issue\#280 (scheung)}
\begin{description}
\item [Subject:] Program name is 8 characters long
\item [Criticality:] Low
\item [Confidence-level:] Med
\item [Source-node:] P2\_Proc\_START

\item [Lines:] 15

\item [Description:] The progname is 8 characters long.
\end{description}
\item {\it Issue\#284 (scheung)}
\begin{description}
\item [Subject:] PROGNAME should be passed as a pointer
\item [Criticality:] Hi
\item [Confidence-level:] Med
\item [Source-node:] Pass\_2

\item [Lines:] 25

\item [Description:] PROGNAME should be passed by reference as a
pointer because that is what P2\_Write\_Obj is expecting.
\end{description}
\item {\it Issue\#292 (awong)}
\begin{description}
\item [Subject:] uninitialized variable
\item [Criticality:] Med
\item [Confidence-level:] Med
\item [Source-node:] dectonum

\item [Lines:] 7-22

\item [Description:] the variable i has not been initialized
\end{description}
\item {\it Issue\#296 (awong)}
\begin{description}
\item [Subject:] incorrect statements in if statement
\item [Criticality:] Med
\item [Confidence-level:] Med
\item [Source-node:] P2\_Assemble\_Inst

\item [Lines:] 37-41

\item [Description:] the strncmp function will return 0 if true and
the checking in the if statement is wrong because it negates the value of the
strncmp function and cause the then part of the if statement to execute
\end{description}
\item {\it Issue\#300 (awong)}
\begin{description}
\item [Subject:] no return statement
\item [Criticality:] Med
\item [Confidence-level:] Med
\item [Source-node:] P2\_Write\_Obj

\item [Lines:] 55-74

\item [Description:] there is no return statement in this part of
the if then statement
\end{description}
\item {\it Issue\#304 (awong)}
\begin{description}
\item [Subject:] missing else
\item [Criticality:] Med
\item [Confidence-level:] Med
\item [Source-node:] P2\_Proc\_START

\item [Lines:] 9-19

\item [Description:] the rest of the function should not execute if
the first part of the if statement is true
\end{description}
\end{enumerate}
\section{Review Metrics}
\begin{table}[hb]
\begin{center}
\begin{tabular}{|l|l|l|l|l|}
\hline
Participant & Start-time & End-time & Elapsed-time & Busy-time \\
\hline
scheung & May 05, 1995 11:42:10 & May 05, 1995 13:34:57 & 1:52:47 & 1:49:47 \\
kawak & May 05, 1995 11:41:40 & May 05, 1995 13:18:55 & 1:37:15 & 1:32:39 \\
awong & May 05, 1995 11:40:18 & May 05, 1995 13:35:23 & 1:55:5 & 1:55:5 \\
\hline
 & & Total & 5:25:7 & \\
\hline
\end{tabular}
\end{center}
\caption{Review Session}
\end{table}


\begin{table}[hb]
\begin{center}
\begin{tabular}{|l|l|l|l|}
\hline
Source & scheung & kawak & awong\\
\hline
(172)Type and var declarations & 721 & 3976 & 641\\
(174)dectonum & 764 & 467 & 418\\
(176)Read\_Int\_File & 1190 & 1119 & 1041\\
(178)P2\_Search\_Optab & 379 & 515 & 837\\
(180)P2\_Proc\_START & 500 & 98 & 594\\
(182)P2\_Proc\_BYTE & 595 & 854 & 1111\\
(184)P2\_Assemble\_Inst & 432 & 391 & 721\\
(186)P2\_Write\_Obj & 1199 & 2101 & 1142\\
(188)Pass\_2 & 784 & 883 & 397\\
\hline
\end{tabular}
\end{center}
\caption{Review Time}
\end{table}

\begin{table}[hb]
\begin{center}
\begin{tabular}{|l|l|l|l|}
\hline
Source & scheung & kawak & awong\\
\hline
dectonum & 99.0 & 161.9 & 180.9\\
Read\_Int\_File & 178.5 & 189.8 & 204.0\\
P2\_Search\_Optab & 266.0 & 195.7 & 120.4\\
P2\_Proc\_START & 136.8 & 698.0 & 115.2\\
P2\_Proc\_BYTE & 302.5 & 210.8 & 162.0\\
P2\_Assemble\_Inst & 350.0 & 386.7 & 209.7\\
P2\_Write\_Obj & 216.2 & 123.4 & 227.0\\
Pass\_2 & 174.5 & 154.9 & 344.6\\
\hline
\end{tabular}
\end{center}
\caption{Paraphrasing Rate (lines/hour)}
\end{table}


\begin{table}[hb]
\begin{center}
\begin{tabular}{|l|l|l|l|l|}
\hline
Source & scheung & kawak & awong & OK\\
\hline
Type and var.. & 252 (=1) &  &  & \\
dectonum & 200,204 (=2) & 196,272 (=2) & 292 (=1) & 204=196=292\\
Read\_Int\_Fil.. & 262 (=1) & 208,218,240 (=3) & 224,258 (=2) & 262=208=224 \\
P2\_Search\_Op.. &  &  & 266 (=1) & 266\\
P2\_Proc\_STAR.. & 270,280 (=2) &  & 304 (=1) & 270\\
P2\_Proc\_BYTE & 222,230,234 (=3) &  & 276 (=1) & 230=276\\
P2\_Assemble\_.. &  & 254 (=1) & 296 (=1) & 296\\
P2\_Write\_Obj & 236,244 (=2) &  & 248,300 (=2) & 236=248\\
Pass\_2 & 284 (=1) &  &  & \\
\hline
\end{tabular}
\caption{Source node v.s Issue node}
\end{center}
\end{table}

%%\end{document}
