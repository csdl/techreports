%%% \documentstyle[11pt,/group/csdl/tex/definemargins,
%%%                        /group/csdl/tex/lmacros]{article} 
%%% 
%%%           \begin{document}
%%%           \begin{center}
%%%           {\large\bf CSRS Experiment -- Source Listing \\
%%%             Source: Pass1 }
%%% \end{center}
%%% 
\chapter {CSRS Experiment (ICS411): Source Listing -- Pass1}
\small
	  

\section{Type and var declarations}
\subsection*{Specification:}

\subsection*{Source-code:}
\begin{verbatim}
001:/* 
002:THIS IS A SIMPLE ASSEMBLER FOR SIC (STANDARD VERSION) PASS 1. 
003:IT READS SOURCE PROGRAM INPUT AND WRITES INTO INTERMEDIATE FILE.
004:
005:THE SOURCE PROGRAM TO BE ASSEMBLED MUST BE IN FIXED FORMAT AS 
006:FOLLOWS:
007:      BYTES  0-7  LABEL
008:               8  BLANK
009:             9-14 OPERATION CODE
010:            15-16 BLANK
011:            17-34 OPERAND
012:            35-65 COMMENT
013:
014:IF A SOURCE LINE CONTAINS "." IN THE FIRST BYTE, THE ENTIRE LINE IS
015:TREATED AS A COMMENT. 
016:*/
017:
018:#ifndef true
019:# define true    1
020:# define false   0
021:#endif
022:
023:#define EXIT_SUCCESS  0
024:#define EXIT_FAILURE  1
025:#define QUOTE           '\''
026:#define MAXERRORS       25   /* SIZE OF ARRAY OF ERROR FLAGS */
027:#define MAXOPS          25   /* SIZE OF OPCODE TABLE */
028:#define SYMTABLIMIT     100   /* SIZE OF SYMBOL TABLE */
029:#define BLANK6          "      "
030:#define BLANK8          "        "
031:#define BLANK18         "                  "
032:#define BLANK31         "                               "
033:
034:typedef unsigned char BOOLEAN;
035:typedef char CHAR4 [4];
036:typedef char CHAR6 [6];
037:typedef char CHAR8 [8];
038:typedef char CHAR18 [18];
039:typedef char CHAR31 [31];
040:typedef char CHAR50 [50];
041:typedef char CHAR66 [66];
042:
043:typedef struct SOURCETYPE {
044:  /* SOURCE LINE AND SUBFIELDS */
045:  CHAR66 line;
046:  BOOLEAN comline;
047:  CHAR8 labl;
048:  CHAR6 operation;
049:  CHAR18 operand;
050:  CHAR31 comment;
051:} SOURCETYPE;
052:
053:typedef struct REC_SYMTABTYPE {
054:  CHAR8 symbol;
055:  int address;
056:} REC_SYMTABTYPE;
057:
058:
059:typedef enum {SEARCH, STORE} SYMREQTYPE;
060:
061:typedef enum {
062:  FOUND, NOTFOUND, ADDED, DUPLICATE, TABLEFULL
063:}SYMRETTYPE;
064:
065:/*SYMBOL TABLE is an array [0..SYMTABLIMIT] of REC_SYMTABTYPE*/
066:/*ASSUME THIS TABLE IS INITIALIZED PROPERLY.                 */
067:static REC_SYMTABTYPE SYMTAB[SYMTABLIMIT + 1];
068:
069:static FILE *srcfile, *objfile, *lisfile, *intfile;
070:static int LOCCTR;
071:static CHAR6 PROGNAME;
072:static int PROGSTART;
073:
074:/* Error Messages */
075:static CHAR50 ERRMSG[MAXERRORS] = {
076:  "ILLEGAL FORMAT IN LABEL FIELD                     ",/*ERRMSG[0]*/
077:  "MISSING OPERATION CODE                            ",/*ERRMSG[1]*/
078:  "ILLEGAL FORMAT IN OPERATION FIELD                 ",/*ERRMSG[2]*/
079:  "MISSING OR MISPLACED OPERAND IN START STATEMENT   ",/*ERRMSG[3]*/
080:  "ILLEGAL OPERAND IN START STATEMENT                ",/*ERRMSG[4]*/
081:  "ILLEGAL OPERAND IN BYTE STATEMENT                 ",/*ERRMSG[5]*/
082:  "ODD LENGTH HEX STRING IN BYTE STATEMENT           ",/*ERRMSG[6]*/
083:  "MISSING OR MISPLACED OPERAND IN BYTE STATEMENT    ",/*ERRMSG[7]*/
084:  "MISSING OR MISPLACED OPERAND IN RESW STATEMENT    ",/*ERRMSG[8]*/
085:  "ILLEGAL OPERAND IN RESW STATEMENT                 ",/*ERRMSG[9]*/
086:  "MISSING OR MISPLACED OPERAND IN RESB STATEMENT    ",/*ERRMSG[10]*/
087:  "ILLEGAL OPERAND IN RESB STATEMENT                 ",/*ERRMSG[11]*/
088:  "DUPLICATE LABEL DEFINITION                        ",/*ERRMSG[12]*/
089:  "TOO MANY SYMBOLS IN SOURCE PROGRAM                ",/*ERRMSG[13]*/
090:  "DUPLICATE OR MISPLACED START STATEMENT            ",/*ERRMSG[14]*/
091:  "MISSING OR MISPLACED START STATEMENT              ",/*ERRMSG[15]*/
092:  "ILLEGAL OPERAND IN WORD STATEMENT                 ",/*ERRMSG[16]*/
093:  "MISSING OR MISPLACED OPERAND IN WORD STATEMENT    ",/*ERRMSG[17]*/
094:  "MISSING OR MISPLACED OPERAND IN END STATEMENT     ",/*ERRMSG[18]*/
095:  "ILLEGAL OPERAND IN END STATEMENT                  ",/*ERRMSG[19]*/
096:  "UNDEFINED SYMBOL IN OPERAND                       ",/*ERRMSG[20]*/
097:  "STATEMENT SHOULD NOT FOLLOW END STATEMENT         ",/*ERRMSG[21]*/
098:  "ILLEGAL OPERAND FIELD                             ",/*ERRMSG[22]*/
099:  "UNRECOGNIZED OPERATION CODE                       ",/*ERRMSG[23]*/
100:  "MISSING OR MISPLACED OPERAND IN INSTRUCTION       "/*ERRMSG[24]*/
101:};
102:
103:
104:
\end{verbatim}
\section{hextonum}
\subsection*{Specification:}
THIS FUNCTION EXAMINES THE STRING STR, BEGINNING WITH THE BYTE POSITION
INDICATED BY FIRST, FOR A HEXADECIMAL VALUE. THIS VALUE, CONVERTED TO AN
INTEGER, IS RETURNED AS THE VALUE OF THE FUNCTION AND LAST IS SET TO INDICATE
THE NEXT CHARACTER IN STR AFTER THE HEX VALUE THAT WAS FOUND. IF ANY SCANNING
OR CONVERSION ERRORS ARE DETECTED, CONVERROR IS SET TO TRUE. THE MAXIMUM
LENGTH OF THE HEX VALUE TO BE SCANNED IS 4 HEX DIGITS.
\subsection*{Source-code:}
\begin{verbatim}
001:int hextonum (char *str,int first,int *last,BOOLEAN *converror)
002:
003:{
004:
005:  int n, i;
006:  BOOLEAN scanning;
007:
008:  n = 0;
009:  i = 0;
010:  scanning = true;
011:  *converror = false;
012:  while (scanning) {
013:    if (str[first + i] >= '0' && str[first + i] <= '9')
014:      n = n * 16 + str[first + i] - '0';
015:    else if (str[first + i] >= 'A' && str[first + i] <= 'F')
016:      n = n * 16 + str[first + i] - 'A' + 10;
017:    else {
018:      *converror = true;
019:      scanning = false;
020:    }
021:    i++;
022:    if (i > first + 3)
023:      scanning = false;
024:  }
025:  *last = first + i;
026:  return n;
027:}
028:
029:
030:
031:
\end{verbatim}
\section{Access\_Symtab}
\subsection*{Specification:}
THIS PROCEDURE IS USED TO ACCESS THE SYMBOL TABLE FOR THE ASSEMBLY.  IF
REQUESTCODE = SEARCH, THE SYMBOL PASSED AS A PARAMETER IS SEARCHED FOR IN THE
TABLE. IF THIS SYMBOL IS FOUND, RETURNCODE IS SET TO FOUND, AND ADDRESS IS
SET TO THE VALUE OF THE SYMBOL (FROM THE SYMBOL TABLE). IF THE SYMBOL IS NOT
FOUND IN THE TABLE, RETURNCODE IS SET TO NOTFOUND.

IF REQUESTCODE = STORE, THE SYMBOL IS ADDED TO THE TABLE, WITH VALUE GIVEN BY
ADDRESS. IF THE SYMBOL IS ADDED NORMALLY, RETURNCODE IS SET TO ADDED. IF THE
SYMBOL ALREADY EXISTS IN THE TABLE, RETURNCODE IS SET TO DUPLICATE (AND THE
SYMBOL IS NOT ADDED). IF THE TABLE IS ALREADY FULL, RETURNCODE IS SET TO
TABLEFULL.

THE SYMBOL TABLE IS ORGANIZED AS A HASH TABLE. THE HASHING FUNCTION SIMPLY
SUMS THE ORDINAL VALUES OF ALL OF THE CHARACTERS IN THE SYMBOL. COLLISIONS
ARE HANDLED BY LINEAR PROBING.
\subsection*{Source-code:}
\begin{verbatim}
001:void Access_Symtab(SYMREQTYPE requestcode,
002:               SYMRETTYPE *returncode,
003:               char *symbol,
004:               int *address)
005:
006:{
007:  BOOLEAN searching;
008:  int i, hash, ptr;
009:
010:  for (i = 0; i <= 7; i++)
011:    hash += symbol[i];
012:  hash = hash % SYMTABLIMIT;
013:  if (requestcode == SEARCH) {
014:    searching = true;
015:    ptr = hash;
016:    while (searching) {
017:      if (!strncmp(SYMTAB[ptr].symbol, symbol, sizeof(CHAR8))){
018:    *returncode = FOUND;
019:    *address = SYMTAB[ptr].address;
020:    searching = false;
021:      }
022:      else if (strncmp(SYMTAB[ptr].symbol,BLANK8,sizeof(CHAR8))){ 
023:    *returncode = NOTFOUND;
024:    *address = 0;
025:    searching = false;
026:      }
027:      else {
028:    ptr = (ptr + 1) % SYMTABLIMIT;
029:    if (ptr == hash) {
030:      *returncode = NOTFOUND;
031:      *address = 0;
032:      searching = false;
033:    }
034:      }
035:    }
036:    return;
037:  }
038:  /*requestcode == STORE*/
039:  searching = true;
040:  ptr = hash;
041:  while (searching) {
042:    if (!strncmp(SYMTAB[ptr].symbol, symbol, sizeof(CHAR8))) {
043:      *returncode = DUPLICATE;
044:      searching = false;
045:    }
046:    else if (!strncmp(SYMTAB[ptr].symbol, BLANK8, sizeof(CHAR8))) { 
047:      *returncode = ADDED;
048:      memcpy(SYMTAB[ptr].symbol, symbol, sizeof(CHAR8));
049:      SYMTAB[ptr].address = *address;
050:      searching = false;
051:    } 
052:    else {
053:      ptr = (ptr + 1) % SYMTABLIMIT;
054:      if (ptr == hash)
055:    *returncode = TABLEFULL;
056:    }
057:  } /*while*/
058:}
059:
060:
061:
\end{verbatim}
\section{Write\_Int\_File}
\subsection*{Specification:}
THIS PROCEDURE WRITES THE INTERMEDIATE FILE INTFILE.  THE CURRENT SOURCE
PROGRAM LINE IS WRITTEN, FOLLOWED BY A BOOLEAN VALUE (T OR F) THAT INDICATES
WHETHER THIS IS A COMMENT LINE, AND THE CURRENT LOCATION COUNTER VALUE. FOR
NON-COMMENT LINES, THE SUBFIELDS ARE ALSO WRITTEN OUT, FOLLOWED BY THE VALUE
OF ERRORSFOUND (T OR F). IF AND ONLY IF ERRORSFOUND WAS TRUE, THE VALUES IN
ERRORFLAGS ARE ALSO WRITTEN.
\subsection*{Source-code:}
\begin{verbatim}
001:void Write_Int_File (SOURCETYPE *source,
002:                 BOOLEAN errorsfound,
003:                 BOOLEAN *errorflags)
004:
005:{
006:  int i;
007:
008:  for (i = 0; i <= 65; i++)
009:    putc(source->line[i], intfile);
010:  if (source->comline)
011:    putc('T', intfile);
012:  else
013:    putc('F', intfile);
014:  fprintf(intfile, "%d\n", LOCCTR);
015:  if (source->comline)
016:    return;
017:  for (i = 0; i <= 7; i++)
018:    putc(source->labl[i], intfile);
019:  for (i = 0; i <= 5; i++)
020:    putc(source->operation[i], intfile);
021:  for (i = 0; i <= 17; i++)
022:    putc(source->operand[i], intfile);
023:  for (i = 0; i <= 30; i++)
024:    putc(source->comment[i], intfile);
025:  if (!errorsfound)
026:    fprintf(intfile, "F\n");
027:  fprintf(intfile, "T\n");
028:
029:  for (i = 0; i < MAXERRORS; i++) {
030:    if (errorflags[i])
031:      putc('T', intfile);
032:    else
033:      putc('F', intfile);
034:  }
035:  putc('\n', intfile);
036:}
037:
038:
039:
\end{verbatim}
\section{P1\_Read\_Source}
\subsection*{Specification:}
THIS PROCEDURE READS THE NEXT LINE FROM SRCFILE. IF THERE ARE NO MORE LINES
ON SRCFILE, ENDOFINPUT IS SET TO TRUE.

IF THE SOURCE STATEMENT CONTAINS A "." IN COLUMN 1, THEN SOURCE.COMLINE IS
SET TO TRUE. OTHERWISE, THE SUBFIELDS OF THE STATEMENT (LABL, OPERATION,
OPERAND, COMMENT) ARE SCANNED.

ERRORS THAT MAY BE DETECTED: 0, 1, 2
\subsection*{Source-code:}
\begin{verbatim}
001:void P1_Read_Source (SOURCETYPE *source,
002:                 BOOLEAN *endofinput, 
003:                 BOOLEAN *errorsfound,
004:                 BOOLEAN *errorflags)
005:
006:
007:{
008:  int i, j;
009:  char ch;
010:
011:  if ((ch=getc(srcfile))==EOF){
012:      *endofinput = true;
013:      return;
014:  }
015:
016:  ungetc(ch,srcfile);
017:
018:  for (i = 0; i <= 65; i++)
019:    source->line[i] = ' ';
020:  for (i = 0; i <= MAXERRORS; i++)
021:    errorflags[i] = false;
022:  *errorsfound = false;
023:  
024:  for(i = 0; i <= 65 && (ch=getc(srcfile))!='\n' && ch!=EOF; i++)
025:    source->line[i] = ch;
026:
027:  /*discard remaining characters till newline
028:   (implementation of readln in PASCAL) */ 
029:  while (ch!=EOF || ch != '\n')
030:    ch=getc(srcfile);
031:  
032:  if (source->line[0] == '.')
033:    source->comline = true;
034:  else
035:    source->comline = false;
036:  memcpy(source->labl, BLANK8, sizeof(CHAR8));
037:  memcpy(source->operation, BLANK6, sizeof(CHAR6));
038:  memcpy(source->operand, BLANK18, sizeof(CHAR18));
039:  memcpy(source->comment, BLANK31, sizeof(CHAR31));
040:
041:  if (source->line[0] >= 'A' && source->line[0] <= 'Z' ) { 
042:    /* THERE IS A LABEL */
043:    while (i <= 7 && (isupper(source->line[i]) ||
044:                  isdigit(source->line[i]))) {
045:      source->labl[i] = source->line[i];
046:      i++;
047:    }
048:  }
049:  for (j = i; j <= 8; j++) {
050:    if (source->line[j] != ' ') {
051:      *errorsfound = true;
052:      errorflags[0] = true;   /* ILLEGAL LABEL FIELD */
053:    }
054:  }
055:  i = 9;
056:  if (isupper(source->line[i])) {  /* THERE IS AN OPERATION CODE */
057:    while (i <= 13 && (isupper(source->line[i]) ||
058:                   isdigit(source->line[i]))) {
059:      source->operation[i - 9] = source->line[i];
060:      i++;
061:    }
062:  } else {
063:    *errorsfound = true;
064:    errorflags[1] = true;   /* MISSING OPERATION CODE */
065:  }
066:
067:  for (j = i; j <= 16; j++) {
068:    if (source->line[j] != ' ') {
069:      *errorsfound = true;
070:      errorflags[2] = true;   /* ILLEGAL OPERATION FIELD */
071:    }
072:  }
073:
074:  for (i = 35; i <= 65; i++)
075:    source->comment[i - 35] = source->line[i];
076:}  /* P1_Read_Source */
077:
078:
079:
\end{verbatim}
\section{P1\_Proc\_START}
\subsection*{Specification:}
PROCESS START STATEMENT -- CONVERT STARTING ADDRESS AND STORE IN
NEWLOCCTR. ERRORS DETECTED: 3, 4
\subsection*{Source-code:}
\begin{verbatim}
001:void P1_Proc_START (SOURCETYPE source,
002:                int *newlocctr,
003:                BOOLEAN *errorsfound,
004:                BOOLEAN *errorflags)
005:
006:{
007:  BOOLEAN converror;
008:  int i, j, temploc;
009:
010:  if (source.operand[0] == ' ') {
011:    *errorsfound = true;
012:    errorflags[3] = true;
013:    /* MISSING OR MISPLACED OPERAND IN START STATEMENT */
014:    return;
015:  }
016:  temploc = hextonum(source.operand, 0, &i, &converror);
017:  if (converror) {
018:    *errorsfound = true;
019:    errorflags[4] = true; /* ILLEGAL OPERAND IN START STATEMENT */
020:  }
021:  for (j = i; j <= 17; j++) {
022:    if (source.operand[j] != ' ') {
023:      *errorsfound = true;
024:      errorflags[4] = true;
025:      /* ILLEGAL OPERAND IN START STATEMENT */
026:    }
027:  }
028:  if (!errorflags[3] && !errorflags[4])
029:    *newlocctr = temploc;
030:}
031:
032:
\end{verbatim}
\section{P1\_Proc\_RESW}
\subsection*{Specification:}
PROCESS RESW STATEMENT -- ADD THE NUMBER OF WORDS RESERVED TO LOCCTR. ERRORS
DETECTED: 8, 9
\subsection*{Source-code:}
\begin{verbatim}
001:void P1_Proc_RESW (SOURCETYPE source,
002:               int locctr,
003:               int *newlocctr,
004:               BOOLEAN *errorsfound,
005:               BOOLEAN *errorflags)
006:     
007:
008:{
009:  BOOLEAN converror;
010:  int i, j, nwords;
011:
012:  if (source.operand[0] == ' ') {
013:    *errorsfound = true;
014:    errorflags[8] = true;
015:    /* MISSING OR MISPLACED OPERAND IN RESW STATEMENT */
016:    return;
017:  }
018:  nwords = dectonum(source.operand, 0, &i, &converror);
019:  if (converror) {
020:    *errorsfound = true;
021:    errorflags[9] = true;   /* ILLEGAL OPERAND IN RESW */
022:  }
023:  for (j = i; j <= 17; j++) {
024:    if (source.operand[j] != ' ') {
025:      *errorsfound = true;
026:      errorflags[9] = true; /* ILLEGAL OPERAND IN RESW */
027:    }
028:  }
029:  if (!errorflags[9])
030:    *newlocctr = locctr + nwords;
031:}
032:
033:
\end{verbatim}
\section{P1\_Assign\_Loc}
\subsection*{Specification:}
THIS PROCEDURE UPDATES THE LOCATION COUNTER VALUE BASED ON THE TYPE OF
STATEMENT BEING PROCESSED, PLACING THE UPDATED VALUE IN NEWLOCCTR.
\subsection*{Source-code:}
\begin{verbatim}
001:void P1_Assign_Loc(SOURCETYPE source,
002:               int locctr, 
003:               int *newlocctr,
004:               BOOLEAN *errorsfound,
005:               BOOLEAN *errorflags)
006:
007:{
008:  
009:  *newlocctr = locctr;
010:  if (!strncmp(source.operation, "START ", sizeof(CHAR6)))
011:      P1_Proc_START(source, newlocctr, errorsfound, errorflags);
012:
013:  else if (!strncmp(source.operation, "WORD  ", sizeof(CHAR6)))
014:    P1_Proc_WORD(locctr, newlocctr);
015:
016:  else if (!strncmp(source.operation, "BYTE  ", sizeof(CHAR6)))
017:    P1_Proc_BYTE(source,locctr,newlocctr,errorsfound,errorflags);
018:
019:  else if (!strncmp(source.operation, "RESW  ", sizeof(CHAR6)))
020:    P1_Proc_RESW(source,locctr,newlocctr,errorsfound,errorflags);
021:  
022:  else if (!strncmp(source.operation, "RESB  ", sizeof(CHAR6)))
023:    P1_Proc_RESB(source,locctr,newlocctr,errorsfound,errorflags);
024:  
025:  else if (!strncmp(source.operation, "END   ", sizeof(CHAR6)))
026:    ;    /* NO ACTION IN PASS 1 */
027:  else
028:    *newlocctr = locctr * 3;  /* ASSUME MACHINE INSTRUCTION */  
029:}
030:
031:
032:
\end{verbatim}
\section{P1\_Assign\_Sym}
\subsection*{Specification:}
THIS PROCEDURE ADDS THE LABEL FROM A SOURCE STATEMENT TO THE SYMBOL TABLE,
USING THE CURRENT LOCATION COUNTER VALUE AS ITS ADDRESS.  ERRORS DETECTED:
12,13
\subsection*{Source-code:}
\begin{verbatim}
001:void P1_Assign_Sym(SOURCETYPE source,
002:               int locctr,
003:               BOOLEAN *errorsfound,
004:               BOOLEAN *errorflags)
005:
006:
007:{
008:  SYMRETTYPE symtabret;
009:  int address;
010:
011:  if (!errorflags[0] && 
012:      !strncmp(source.labl, BLANK8, sizeof(CHAR8))) {
013:    Access_Symtab(SEARCH, &symtabret, source.labl, &address);
014:    if (symtabret == NOTFOUND) {
015:      address = locctr;
016:      Access_Symtab(SEARCH, &symtabret, source.labl, &address);
017:      if (symtabret = TABLEFULL) {
018:    *errorsfound = true;
019:    errorflags[13] = true;   /* SYMBOL TABLE OVERFLOW */
020:      }      
021:    }
022:    else {
023:      *errorsfound = true;
024:      errorflags[12] = true;   /* DUPLICATE LABEL */
025:    }
026:  }
027:}
028:
029:
\end{verbatim}
\section{Pass\_1}
\subsection*{Specification:}
THIS IS THE MAIN PROCEDURE FOR PASS 1. IT USES P1\_READ\_SOURCE TO READ EACH
INPUT STATEMENT (UNTIL ENDOFINPUT = TRUE). FOR NON-COMMENT LINES, IT CALLS
P1\_ASSIGN\_LOC AND P1\_ASSIGN\_SYM.  FOR ALL SOURCE LINES, IT USES
WRITE\_INT\_FILE TO WRITE THE INTERMEDIATE FILE
\subsection*{Source-code:}
\begin{verbatim}
001:void Pass_1()
002:
003:{     
004:  SOURCETYPE source;   /* SOURCE LINE AND SUBFIELDS */
005:  BOOLEAN endofinput;
006:  BOOLEAN errorflags[MAXERRORS]; /* ERROR FLAGS FOR CURRENT STMT */
007:  BOOLEAN errorsfound; /* TRUE IF ANY ERRORS IN CURRENT STMT */
008:  int i;
009:  int newlocctr;
010:
011:  rewind(srcfile);
012:  LOCCTR = 0;
013:
014:  P1_Read_Source(&source, &endofinput, &errorsfound, errorflags);
015:  while (!endofinput) {
016:    if (source.comline)
017:      newlocctr = LOCCTR;
018:    else {
019:      P1_Assign_Loc(source, LOCCTR, &newlocctr, 
020:                &errorsfound, errorflags);
021:      if (!strncmp(source.operation, "START ", sizeof(CHAR6))) {
022:    LOCCTR = newlocctr;
023:    PROGSTART = LOCCTR;
024:    for (i = 0; i <= 5; i++)
025:      PROGNAME[i] = source.labl[i];
026:      }
027:      P1_Assign_Sym (source, LOCCTR, &errorsfound, errorflags);
028:    }
029:    Write_Int_File(&source, errorsfound, errorflags);
030:    LOCCTR = newlocctr;
031:    P1_Read_Source(&source, &endofinput, &errorsfound, errorflags);
032:  }
033:  if (srcfile != NULL)
034:    fclose(srcfile);
035:  srcfile = NULL;
036:}
037:
038:
\end{verbatim}
%%\end{document}
