%%% \documentstyle[11pt,/group/csdl/tex/definemargins,
%%%                        /group/csdl/tex/lmacros]{article} 
%%% 
%%%           \begin{document}
%%%           \begin{center}
%%%           {\large\bf CSRS Experiment -- Source Listing \\
%%%             Source: Employee2 }
%%% \end{center}

\chapter {CSRS Experiment (ICS313): Source Listing -- Employee2}
\small
	  

\section{Employee}
\subsection*{Specification:}
Employee class: each instance of this class represents a single
employee. Each Employee has the following attributes: name (a string),
socSecurity (a string with format "XXX-XX-XXXX", where each X is a digit),
age (an integer between 17-55), numDependents (an integer between 0 and 10).
\subsection*{Source-code:}
\begin{verbatim}
001:class Employee
002:
003:{
004: protected:   // Protected for use in derived classes
005:  char *name;
006:  char *socSecurity;
007:  int age;
008:  int numDependents;
009: public:
010:  Employee();
011:  virtual ~Employee();
012:  int setName(char* newName);
013:  int setSocSecurity(char* newNum);
014:  int setAge(int newAge);
015:  int setNumDependents(int newNumDependents);
016:  char* getSocSecurity();
017:  virtual void print();
018:};
019:
\end{verbatim}
\section{EmployeeNode}
\subsection*{Specification:}
Structure to hold a single employee object and a pointer to the next
employee.
\subsection*{Source-code:}
\begin{verbatim}
001:class EmployeeNode
002:
003:{
004:private:
005:  Employee* data;   //pointer to an employee data
006:  EmployeeNode* next;       //link to the next employee in the list
007:  friend class Company2;//allowing Company2 to access private members
008:
009:public:
010:  EmployeeNode();           //constructor
011:  ~EmployeeNode();          //destructor
012:};
013:
014:
\end{verbatim}
\section{Company2}
\subsection*{Specification:}
Company2 class: Each instance of company2 holds instances of employees.
Instances of employee must be stored as a linked-list ordered ascendingly by
social security number.
\subsection*{Source-code:}
\begin{verbatim}
001:class Company2
002:
003:{ 
004:private:
005:  EmployeeNode* list_employee;//list of workers in the company
006:  int numEmployees;       //number of employees in the company    
007:public:
008:  Company2();
009:  ~Company2();
010:  void print(void); //print employees in the company
011:  int addEmployee(Employee* new_employee);//add new employee
012:  int deleteEmployee(char* ssn);  //delete employee from company
013:  Employee* findEmployee(char* ssn);  //find a specified employee
014:};
015:
\end{verbatim}
\section{Employee::Employee}
\subsection*{Specification:}
Constructor to initialize Employee
\subsection*{Source-code:}
\begin{verbatim}
001:Employee::Employee()
002:
003:{
004:  name = "";            //set to empty
005:  socSecurity = "";     //set to empy
006:  age = 17;             // init age
007:  numDependents = 0;    // init number of dependents
008:}
009:
010:
\end{verbatim}
\section{Employee::\~Employee}
\subsection*{Specification:}
Destuctor for Employee
\subsection*{Source-code:}
\begin{verbatim}
001:Employee::~Employee()
002:
003:{
004:  delete[] name;            // delete name array
005:  delete[] socSecurity;             // delete SSN array
006:
007:  name = 0;
008:  socSecurity = 0;
009:}
010:
011:
\end{verbatim}
\section{Employee::setName}
\subsection*{Specification:}
Set the name of the Employee to newName
  Takes a char pointer to newName.
  Returns an int 0 if name set, 1 if failed.
\subsection*{Source-code:}
\begin{verbatim}
001:int Employee::setName(char* newName) 
002:  
003:{
004:  int n = strlen(newName);
005:
006:  if (name == 0)        //no space allocated ? 
007:      name = new char [n]; //allocate the space
008:  //space is already allocated
009:  strcpy(name, newName);
010:  return 0;
011:}
012:
013:
\end{verbatim}
\section{Employee::setSocSecurity}
\subsection*{Specification:}
Set SSN for the Employee to newNum.
  Takes a char pointer to newNum.
  Returns an int 0 if SSN set 1 if failed.
  SSN must be a string with format "XXX-XX-XXXX",
  where each X is a digit.
\subsection*{Source-code:}
\begin{verbatim}
001:int Employee::setSocSecurity(char* newSSN)
002:
003:{
004:  int i = 0;
005:  int validDigits = 0; //number of valid digits
006:  
007:  if (socSecurity == 0)
008:    socSecurity = new char[12];
009:  
010:  //Check for valid SSN format
011:  while (newSSN[i] != '\0'){
012:    if (newSSN[i] >= '0' && newSSN[i] <= '9'){
013:      socSecurity[i] = newSSN[i];
014:      i++;
015:      validDigits++;
016:    }
017:    else if ((newSSN[i] == '-') && (i == 3 || i == 6))
018:      i++;
019:    else
020:      break;
021:  }
022:
023:  if (validDigits < 9)
024:      return 1; //unsuccessful
025:  else
026:      return 0; //successful
027:}
028:
029:
\end{verbatim}
\section{Employee::setAge}
\subsection*{Specification:}
Set the age of the Employee to newAge
  Takes an integer newAge.
  Returns an int 0 if age set 1 if failed.
  Age is an integer between 17 - 55.
\subsection*{Source-code:}
\begin{verbatim}
001:int Employee::setAge(int newAge)    
002:  
003:{
004:  if (newAge < 17 && newAge > 55)  // is out of the range?
005:    return 1;
006:
007:  age = newAge;          
008:  return 0;
009:}
010:
011:
\end{verbatim}
\section{Employee::setNumDependents}
\subsection*{Specification:}
Set the number of employee dependents to newNumDependents.
  Takes an integer newNumDependents.
  Returns an int 0 if the number of dependents set 1 if failed.
  The number of dependents is an integer between 0 and 10.
\subsection*{Source-code:}
\begin{verbatim}
001:int Employee::setNumDependents(int newNumDependents) 
002:  
003:{ //check range 
004:  if ((newNumDependents >= 0) && (newNumDependents <= 10)){ 
005:    numDependents = newNumDependents;       // if it is then set it!
006:    return 0;
007:  }
008:  else
009:    return 1;
010:}
011:
012:
\end{verbatim}
\section{Employee::print}
\subsection*{Specification:}
Prints the Employee. One line output per employee.
\subsection*{Source-code:}
\begin{verbatim}
001:void Employee::print()
002:
003:{
004:  cout << "Name: "<< name;
005:  cout << "  SSN: "<< socSecurity;
006:  cout << "  Age: "<< age;
007:  cout << "  #Dep: "<< numDependents;
008:  cout << "\n";
009:}
010:
011:
\end{verbatim}
\section{Employee::getSocSecurity}
\subsection*{Specification:}
Returns a pointer to the SSN of the Employee
\subsection*{Source-code:}
\begin{verbatim}
001:char* Employee::getSocSecurity()
002:
003:{
004:  return socSecurity;
005:}
006:
007:
\end{verbatim}
\section{EmployeeNode::EmployeeNode}
\subsection*{Specification:}
Constructor for Employee Node
\subsection*{Source-code:}
\begin{verbatim}
001:EmployeeNode::EmployeeNode()
002:     
003:{ 
004:  data = 0;         //initialize data field to null
005:  next = 0;         //intialize next field to null
006:}
007:
008:
\end{verbatim}
\section{EmployeeNode::\~EmployeeNode}
\subsection*{Specification:}
Destructor for Employee Node
\subsection*{Source-code:}
\begin{verbatim}
001:EmployeeNode::~EmployeeNode()
002:     
003:{
004:  //use default destructor
005:}
006:
\end{verbatim}
\section{Company2::Company2}
\subsection*{Specification:}
Constructor for Company2 class
\subsection*{Source-code:}
\begin{verbatim}
001:Company2::Company2()
002:
003:{
004:  //set list_employee to NULL pointer
005:
006:  list_employee = 0;  
007:}
008:
009:
\end{verbatim}
\section{Company2::\~Company2}
\subsection*{Specification:}
Destructor from Company2 class
\subsection*{Source-code:}
\begin{verbatim}
001:Company2::~Company2()
002:
003:{ 
004:  EmployeeNode* temp;       
005:
006:  //delete all nodes
007:  while(list_employee != 0) //if there are employees
008:    {
009:      temp = list_employee;
010:      list_employee = list_employee->next; //update first employee
011:      delete temp;     
012:    }
013:}
014:
015:
\end{verbatim}
\section{Company2::findEmployee}
\subsection*{Specification:}
Finds an Employee in the Company2.
   Takes a char pointer SSN.
   Returns a pointer to Employee if found, NULL pointer otherwise.
\subsection*{Source-code:}
\begin{verbatim}
001:Employee* Company2::findEmployee(char* ssn)
002:         
003:{   
004:  EmployeeNode* current = list_employee;
005:  int result;
006:  
007:  if (current == 0)
008:    return 0; //empty list, return Null pointer
009:
010:  result=strcmp(current->data->getSocSecurity(), ssn);
011:
012:  //traverse the list until ssn is found or the list is empty
013:  while ((result < 0) && (current != 0)){
014:    current=current->next;
015:    result=strcmp(current->data->getSocSecurity(), ssn);
016:  }
017:  if ((current==0) && (result > 0)) //key match not found
018:    return 0; //NULL pointer
019:  else // result=0
020:    return current->data ; 
021:}
022:
023:
\end{verbatim}
\section{Company2::addEmployee}
\subsection*{Specification:}
Adds a new Employee to Company2.
  Employee records are maintained in ascending order of SSN.
  Takes an Employee pointer newWorker.
  Returns an integer 0 if newWorker is successfully added,
  returns 1 otherwise.
\subsection*{Source-code:}
\begin{verbatim}
001:int Company2::addEmployee(Employee* newWorker)
002:       
003:{
004:  int compare;       //string compare result
005:  EmployeeNode* current = list_employee; //current employee node
006:  EmployeeNode* previous = 0; //before current employee node
007:  EmployeeNode* new_node = new EmployeeNode; //create new node
008:
009:  new_node->data = newWorker;  //set newWorker to this node
010:  
011:  while(current != 0){
012:    compare = strcmp (new_node->data->getSocSecurity(),
013:                  current->data->getSocSecurity());
014:    if(compare = 0){
015:      return 0; //duplicate
016:    }
017:    else if(compare < 0){
018:      previous = current;
019:      current = current->next;
020:    }
021:    else{   //found the right place, insert the node 
022:      new_node->next = current; 
023:      previous->next = new_node;
024:      numEmployees++;
025:      return 0;
026:    }
027:  }
028:  //insert at the end of the list
029:  previous->next = new_node;
030:  numEmployees++;
031:  return 0;
032:}
033:
034:
\end{verbatim}
\section{Company2::deleteEmployee}
\subsection*{Specification:}
Deletes an Employee in the Company2.
  Takes a char pointer SSN.
  Returns an integer 0 if deleted successful, 1 if failed.
\subsection*{Source-code:}
\begin{verbatim}
001:int Company2::deleteEmployee(char* ssn)
002:       
003:{
004:  EmployeeNode* current = list_employee; //current employee
005:  EmployeeNode* temp;
006:  int result;
007:
008:  if (current == 0) //empty ?
009:    return 1;   //unsuccessfull
010:
011:  while (current->next != 0){
012:    if((result=strcmp(current->next->data->getSocSecurity(),ssn))<0)
013:      current=current->next;
014:    else if (result==0){    //found, delete current->next node
015:      temp = current->next;
016:      current->next = temp->next;
017:      delete temp;
018:      return 0;                     //successful
019:    }
020:  }
021:  //if no ssn matches with any nodes in the list
022:  return 1;         //unsuccessful
023:}
024:
025:
\end{verbatim}
\section{Company2::print}
\subsection*{Specification:}
Print employess of company2
\subsection*{Source-code:}
\begin{verbatim}
001:void Company2::print(void)
002:
003:{
004:  EmployeeNode* current; //points to a current employee
005:  
006:  current = list_employee; //intializes current employee
007:
008:  if(current == 0){ //no employees in company
009:    cout <<"\nThere are no employees in the company to print.\n";
010:    return;
011:  }
012:
013:  while(current->next != 0){//if employee exists
014:    current->data->print();
015:    current = current->next;
016:  }
017:  cout << "\nThere are " << numEmployees << " employees.\n";
018:}
019:
\end{verbatim}
%%\end{document}
