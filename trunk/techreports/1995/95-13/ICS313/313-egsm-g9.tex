%\documentstyle[11pt,/group/csdl/tex/definemargins,
%                       /group/csdl/tex/lmacros]{article} 

%          \begin{document}
%          \begin{center}
%          {\large\bf CSRS Experiment Results}\\
%          \end{center}
\chapter {CSRS Experiment Results (ICS313): Group9 (EGSM)}
\small
	  

\begin{description}
\item [Method:] EGSM
\item [Group:] Group9
\item [Source:] Employee1
\item [Participants:] bacani (Presenter), dat (Moderator), toddm (Reviewer), jgeis (Reviewer)
\end{description}
\section{Issue Lists}
\begin{enumerate}
\item {\it Issue\#220 (bacani)}
\begin{description}
\item [Subject:] Array size too small
\item [Criticality:] Hi:{\it } Med:{\it toddm} Low:{\it bacani,jgeis} None:{\it }
\item [Suggested-by:] Me:{\it } Me-other:{\it bacani} Other-and-me:{\it jgeis,toddm} Other:{\it }
\item [Confidence-level:] Hi:{\it } Med:{\it toddm} Low:{\it bacani,jgeis} None:{\it }
\item [Source-node:] Employee::Employee

\item [Lines:] 4

\item [Description:] name points to an array of characters of size
1.  Names are usually more than one character
\end{description}
\item {\it Issue\#224 (bacani)}
\begin{description}
\item [Subject:] unecessarry nulling of pointers
\item [Criticality:] Hi:{\it } Med:{\it } Low:{\it jgeis,bacani,toddm} None:{\it }
\item [Suggested-by:] Me:{\it } Me-other:{\it bacani} Other-and-me:{\it jgeis,toddm} Other:{\it }
\item [Confidence-level:] Hi:{\it bacani} Med:{\it jgeis,toddm} Low:{\it } None:{\it }
\item [Source-node:] Employee::\~Employee

\item [Lines:] 7-8

\item [Description:] pointers set to null.  This is not required.
\end{description}
\item {\it Issue\#230 (bacani)}
\begin{description}
\item [Subject:] first element skipped
\item [Criticality:] Hi:{\it bacani,toddm,jgeis} Med:{\it } Low:{\it } None:{\it }
\item [Suggested-by:] Me:{\it jgeis} Me-other:{\it } Other-and-me:{\it toddm} Other:{\it bacani}
\item [Confidence-level:] Hi:{\it bacani,jgeis,toddm} Med:{\it } Low:{\it } None:{\it }
\item [Source-node:] Employee::setSocSecurity

\item [Lines:] 8

\item [Description:] In the for loop, ++i will cause the zeroth
element in the array to be skipped.
\end{description}
\item {\it Issue\#234 (bacani)}
\begin{description}
\item [Subject:] Conditions of if statement conflicting
\item [Criticality:] Hi:{\it bacani,jgeis,toddm} Med:{\it } Low:{\it } None:{\it }
\item [Suggested-by:] Me:{\it } Me-other:{\it toddm} Other-and-me:{\it bacani,jgeis} Other:{\it }
\item [Confidence-level:] Hi:{\it bacani,jgeis,toddm} Med:{\it } Low:{\it } None:{\it }
\item [Source-node:] Employee::setSocSecurity

\item [Lines:] 11

\item [Description:] A character can never be both below '0' and
above '9' so this check will never properly detect the error.
\end{description}
\item {\it Issue\#240 (bacani)}
\begin{description}
\item [Subject:] copy length possibly too long
\item [Criticality:] Hi:{\it bacani,toddm} Med:{\it jgeis} Low:{\it } None:{\it }
\item [Suggested-by:] Me:{\it bacani} Me-other:{\it } Other-and-me:{\it toddm} Other:{\it jgeis}
\item [Confidence-level:] Hi:{\it bacani} Med:{\it toddm} Low:{\it jgeis} None:{\it }
\item [Source-node:] Employee::setSocSecurity

\item [Lines:] 24

\item [Description:] n is set to the length of newNum, but newNum
is never checked to see if it's longer than 11 characters. The following copy
procedure may overwrite crucial data.
\end{description}
\item {\it Issue\#244 (bacani)}
\begin{description}
\item [Subject:] Incorrect range checking
\item [Criticality:] Hi:{\it bacani,toddm} Med:{\it jgeis} Low:{\it } None:{\it }
\item [Suggested-by:] Me:{\it } Me-other:{\it toddm} Other-and-me:{\it bacani,jgeis} Other:{\it }
\item [Confidence-level:] Hi:{\it bacani,toddm,jgeis} Med:{\it } Low:{\it } None:{\it }
\item [Source-node:] Employee::setNumDependents

\item [Lines:] 4

\item [Description:] Excludes zero as a valid value. Zero is
actually valid.
\end{description}
\item {\it Issue\#248 (bacani)}
\begin{description}
\item [Subject:] zeroth worker not initialized
\item [Criticality:] Hi:{\it } Med:{\it bacani,jgeis,toddm} Low:{\it } None:{\it }
\item [Suggested-by:] Me:{\it } Me-other:{\it jgeis} Other-and-me:{\it bacani,toddm} Other:{\it }
\item [Confidence-level:] Hi:{\it jgeis,bacani} Med:{\it toddm} Low:{\it } None:{\it }
\item [Source-node:] Company1::Company1

\item [Lines:] 6

\item [Description:] Loop terminates before reaching zeroth worker.
\end{description}
\item {\it Issue\#252 (bacani)}
\begin{description}
\item [Subject:] j points to non-existent worker
\item [Criticality:] Hi:{\it bacani,jgeis} Med:{\it toddm} Low:{\it } None:{\it }
\item [Suggested-by:] Me:{\it } Me-other:{\it jgeis} Other-and-me:{\it bacani,toddm} Other:{\it }
\item [Confidence-level:] Hi:{\it bacani,toddm,jgeis} Med:{\it } Low:{\it } None:{\it }
\item [Source-node:] Company1::addEmployee

\item [Lines:] 14

\item [Description:] For the very first worker to be added, j is
-1, and there is no Workers[-1].
\end{description}
\item {\it Issue\#256 (bacani)}
\begin{description}
\item [Subject:] j+1 points to already existent worker
\item [Criticality:] Hi:{\it bacani} Med:{\it toddm} Low:{\it jgeis} None:{\it }
\item [Suggested-by:] Me:{\it bacani} Me-other:{\it } Other-and-me:{\it } Other:{\it toddm,jgeis}
\item [Confidence-level:] Hi:{\it } Med:{\it bacani} Low:{\it toddm,jgeis} None:{\it }
\item [Source-node:] Company1::addEmployee

\item [Lines:] 35

\item [Description:] j+1 points to a worker which was previously
shifted from position j. The new worker will be overwriting a valid worker.
\end{description}
\item {\it Issue\#260 (bacani)}
\begin{description}
\item [Subject:] function never defined
\item [Criticality:] Hi:{\it } Med:{\it } Low:{\it jgeis,bacani,toddm} None:{\it }
\item [Suggested-by:] Me:{\it bacani} Me-other:{\it } Other-and-me:{\it jgeis} Other:{\it toddm}
\item [Confidence-level:] Hi:{\it } Med:{\it bacani,jgeis,toddm} Low:{\it } None:{\it }
\item [Source-node:] Company1

\item [Lines:] 7

\item [Description:] function never defined or used.
\end{description}
\end{enumerate}
\section{Review Metrics}
\begin{table}[hb]
\begin{center}
\begin{tabular}{|l|l|l|l|l|}
\hline
Participant & Start-time & End-time & Elapsed-time & Total Busy-time \\
\hline
dat & Mar 24, 1995 17:28:47 & Mar 24, 1995 18:43:58 & 1:15:11 & 0:43:26 \\
jgeis & Mar 24, 1995 16:42:30 & Mar 24, 1995 18:39:52 & 1:57:22 & 1:57:22 \\
toddm & Mar 24, 1995 16:42:37 & Mar 24, 1995 18:39:57 & 1:57:20 & 1:45:17 \\
bacani & Mar 24, 1995 16:42:28 & Mar 24, 1995 18:39:56 & 1:57:28 & 1:57:28 \\
\hline
\end{tabular}
\end{center}
\caption{Review Session}
\end{table}


\begin{table}[hb]
\begin{center}
\begin{tabular}{|l|l|l|l|l|}
\hline
Source & dat & jgeis & toddm & bacani\\
\hline
(208)Company1::findEmployee & 135 & 136 & 135 & 137\\
(192)Employee::setSocSecurity & 861 & 1381 & 1203 & 1385\\
(210)Company1::deleteEmployee & 149 & 149 & 149 & 150\\
(194)Employee::setAge & 103 & 103 & 103 & 104\\
(212)Company1::print & 156 & 157 & 157 & 157\\
(196)Employee::setNumDependents & 228 & 228 & 228 & 229\\
(180)Constant & 0 & 95 & 95 & 95\\
(198)Employee::print & 92 & 92 & 92 & 93\\
(182)Employee & 9 & 377 & 376 & 377\\
(200)Employee::getSocSecurity & 83 & 83 & 83 & 83\\
(184)Company1 & 132 & 561 & 566 & 566\\
(202)Company1::Company1 & 235 & 409 & 409 & 411\\
(186)Employee::Employee & 21 & 373 & 374 & 374\\
(204)Company1::\~Company1 & 76 & 137 & 137 & 137\\
(188)Employee::\~Employee & 14 & 518 & 519 & 519\\
(206)Company1::addEmployee & 266 & 1453 & 1269 & 1453\\
(190)Employee::setName & 0 & 707 & 348 & 708\\
\hline
\end{tabular}
\end{center}
\caption{Review Time}
\end{table}


\begin{table}[hb]
\begin{center}
\begin{tabular}{|l|l|l|}
\hline
Source node & Issue node  & OK\\
\hline
(208)Company1::findEmployee &  & \\
(192)Employee::setSocSecurity & \#230,\#234,\#240 (=3) & 234,240 \\
(210)Company1::deleteEmployee &  & \\
(194)Employee::setAge &  & \\
(212)Company1::print &  & \\
(196)Employee::setNumDependents & \#244 (=1) & 244\\
(180)Constant &  & \\
(198)Employee::print &  & \\
(182)Employee &  & \\
(200)Employee::getSocSecurity &  & \\
(184)Company1 & \#260 (=1) & \\
(202)Company1::Company1 & \#248 (=1) & 248\\
(186)Employee::Employee & \#220 (=1) & \\
(204)Company1::\~Company1 &  & \\
(188)Employee::\~Employee & \#224 (=1) & \\
(206)Company1::addEmployee & \#252,\#256 (=2) & 252\\
(190)Employee::setName &  & \\
\hline
\end{tabular}
\caption{Source node v.s Issue node}
\end{center}
\end{table}

%\end{document}
