%\documentstyle[11pt,/group/csdl/tex/definemargins,
%                       /group/csdl/tex/lmacros]{article} 

%          \begin{document}
%          \begin{center}
%          {\large\bf CSRS Experiment Results}\\
%          \end{center}

\chapter {CSRS Experiment Results (ICS313): Group3 (EGSM)}
\small

\begin{description}
\item [Method:] EGSM
\item [Group:] Group3
\item [Source:] Employee2
\item [Participants:] tkoyama (Reviewer), dat (Moderator), an (Reviewer), mscruz (Presenter)
\end{description}
\section{Issue Lists}
\begin{enumerate}
\item {\it Issue\#226 (mscruz)}
\begin{description}
\item [Subject:] Assigned value when no space allocated
\item [Criticality:] Hi:{\it mscruz,an} Med:{\it tkoyama} Low:{\it } None:{\it }
\item [Suggested-by:] Me:{\it } Me-other:{\it an} Other-and-me:{\it mscruz} Other:{\it tkoyama}
\item [Confidence-level:] Hi:{\it an,mscruz} Med:{\it tkoyama} Low:{\it } None:{\it }
\item [Source-node:] Employee::Employee

\item [Lines:] 4-5

\item [Description:] The variable should be allocated space before
a value is assigned to it.  Should have new char at least.
\end{description}
\item {\it Issue\#232 (mscruz)}
\begin{description}
\item [Subject:] Unnecessary initialization
\item [Criticality:] Hi:{\it } Med:{\it } Low:{\it an,tkoyama,mscruz} None:{\it }
\item [Suggested-by:] Me:{\it } Me-other:{\it } Other-and-me:{\it mscruz,tkoyama,an} Other:{\it }
\item [Confidence-level:] Hi:{\it mscruz,an,tkoyama} Med:{\it } Low:{\it } None:{\it }
\item [Source-node:] Employee::\~Employee

\item [Lines:] 7-8

\item [Description:] Since the variables will be destroyed anyway,
these lines will not be needed.
\end{description}
\item {\it Issue\#236 (mscruz)}
\begin{description}
\item [Subject:] No check to see if there is a newName passed in.
\item [Criticality:] Hi:{\it tkoyama} Med:{\it an,mscruz} Low:{\it } None:{\it }
\item [Suggested-by:] Me:{\it tkoyama} Me-other:{\it } Other-and-me:{\it an,mscruz} Other:{\it }
\item [Confidence-level:] Hi:{\it an,tkoyama} Med:{\it mscruz} Low:{\it } None:{\it }
\item [Source-node:] Employee::setName

\item [Lines:] 4-5

\item [Description:] newName should be first checked to see if a
name was actually passed in.
\end{description}
\item {\it Issue\#240 (mscruz)}
\begin{description}
\item [Subject:] Need to deallocate first before allocating new
\item [Criticality:] Hi:{\it an,tkoyama} Med:{\it } Low:{\it mscruz} None:{\it }
\item [Suggested-by:] Me:{\it } Me-other:{\it an} Other-and-me:{\it mscruz} Other:{\it tkoyama}
\item [Confidence-level:] Hi:{\it an,mscruz,tkoyama} Med:{\it } Low:{\it } None:{\it }
\item [Source-node:] Employee::setName

\item [Lines:] 7

\item [Description:] name has been allo
ated space initially to hold null.  It should be deallocated first before
allocating new space.
\end{description}
\item {\it Issue\#244 (mscruz)}
\begin{description}
\item [Subject:] name should be allocated n+1 space to hold null
\item [Criticality:] Hi:{\it } Med:{\it tkoyama,an} Low:{\it mscruz} None:{\it }
\item [Suggested-by:] Me:{\it } Me-other:{\it an} Other-and-me:{\it mscruz} Other:{\it tkoyama}
\item [Confidence-level:] Hi:{\it mscruz,an,tkoyama} Med:{\it } Low:{\it } None:{\it }
\item [Source-node:] Employee::setName

\item [Lines:] 7

\item [Description:] name should be allocated an extra space to
hold a null terminator.  Otherwise, garbage will follow the name.
\end{description}
\item {\it Issue\#248 (mscruz)}
\begin{description}
\item [Subject:] There seems to be no return value saying it has failied.
\item [Criticality:] Hi:{\it mscruz} Med:{\it an,tkoyama} Low:{\it } None:{\it }
\item [Suggested-by:] Me:{\it tkoyama} Me-other:{\it } Other-and-me:{\it an} Other:{\it mscruz}
\item [Confidence-level:] Hi:{\it tkoyama,an,mscruz} Med:{\it } Low:{\it } None:{\it }
\item [Source-node:] Employee::setName

\item [Lines:] 10

\item [Description:] No return value exists that tells the calling
function the operation was a failure.
\end{description}
\item {\it Issue\#252 (mscruz)}
\begin{description}
\item [Subject:] Initially allocated space not deallocated first
\item [Criticality:] Hi:{\it } Med:{\it an,tkoyama} Low:{\it mscruz} None:{\it }
\item [Suggested-by:] Me:{\it } Me-other:{\it } Other-and-me:{\it an,mscruz} Other:{\it tkoyama}
\item [Confidence-level:] Hi:{\it an,mscruz,tkoyama} Med:{\it } Low:{\it } None:{\it }
\item [Source-node:] Employee::setSocSecurity

\item [Lines:] 7-8

\item [Description:] socSecurity has been allocated space to hold
the 0.  But, it was not deallocated when new space was allocated to it.
\end{description}
\item {\it Issue\#256 (mscruz)}
\begin{description}
\item [Subject:] No function statements present
\item [Criticality:] Hi:{\it } Med:{\it an,tkoyama} Low:{\it mscruz} None:{\it }
\item [Suggested-by:] Me:{\it tkoyama} Me-other:{\it } Other-and-me:{\it an} Other:{\it mscruz}
\item [Confidence-level:] Hi:{\it mscruz,tkoyama} Med:{\it } Low:{\it an} None:{\it }
\item [Source-node:] EmployeeNode::\~EmployeeNode

\item [Lines:] 

\item [Description:] Since the list is  of pointers to Employee
objects, the destructor of this class should be responsible in deallocating
space assigned to each Employee of the list.  Since the destructor was not
declared virtual, the Employee destructor will not be automatcally invoked.
\end{description}
\item {\it Issue\#260 (mscruz)}
\begin{description}
\item [Subject:] Number of Employees not initialized to 0
\item [Criticality:] Hi:{\it mscruz,an} Med:{\it tkoyama} Low:{\it } None:{\it }
\item [Suggested-by:] Me:{\it tkoyama} Me-other:{\it } Other-and-me:{\it an} Other:{\it mscruz}
\item [Confidence-level:] Hi:{\it mscruz,an,tkoyama} Med:{\it } Low:{\it } None:{\it }
\item [Source-node:] Company2::Company2

\item [Lines:] 

\item [Description:] Since the counter numEmployees is not
automatically initialized to 0 when it is created, it must be initialized
manually by the programmer.  Should have numEmployees = 0;
\end{description}
\item {\it Issue\#262 (mscruz)}
\begin{description}
\item [Subject:] = should be ==
\item [Criticality:] Hi:{\it mscruz,an} Med:{\it tkoyama} Low:{\it } None:{\it }
\item [Suggested-by:] Me:{\it mscruz} Me-other:{\it } Other-and-me:{\it an} Other:{\it tkoyama}
\item [Confidence-level:] Hi:{\it mscruz,an,tkoyama} Med:{\it } Low:{\it } None:{\it }
\item [Source-node:] Company2::addEmployee

\item [Lines:] 14

\item [Description:] The = sign assigns a value on the right of it
to the one on the left.  The if statement tries to compare a value, not
assign one.  It should be ==.
\end{description}
\item {\it Issue\#266 (mscruz)}
\begin{description}
\item [Subject:] Wrong value returned
\item [Criticality:] Hi:{\it mscruz,an} Med:{\it tkoyama} Low:{\it } None:{\it }
\item [Suggested-by:] Me:{\it an} Me-other:{\it } Other-and-me:{\it } Other:{\it mscruz,tkoyama}
\item [Confidence-level:] Hi:{\it mscruz,an,tkoyama} Med:{\it } Low:{\it } None:{\it }
\item [Source-node:] Company2::addEmployee

\item [Lines:] 14-15

\item [Description:] According to the specs, 0 is an indication of
success.  This is a failure so should pass back a 1.
\end{description}
\item {\it Issue\#270 (mscruz)}
\begin{description}
\item [Subject:] Condition wrong
\item [Criticality:] Hi:{\it mscruz,an,tkoyama} Med:{\it } Low:{\it } None:{\it }
\item [Suggested-by:] Me:{\it tkoyama} Me-other:{\it } Other-and-me:{\it mscruz,an} Other:{\it }
\item [Confidence-level:] Hi:{\it mscruz,an,tkoyama} Med:{\it } Low:{\it } None:{\it }
\item [Source-node:] Company2::addEmployee

\item [Lines:] 18-20

\item [Description:] If the comparison is less than the current
employee being checked, than the spot where the new employee is to be
inserted has been found.  A break should have been included in this region.
\end{description}
\item {\it Issue\#274 (mscruz)}
\begin{description}
\item [Subject:] Should provide execution when compare is {\tt >} 0
\item [Criticality:] Hi:{\it tkoyama,an,mscruz} Med:{\it } Low:{\it } None:{\it }
\item [Suggested-by:] Me:{\it } Me-other:{\it } Other-and-me:{\it mscruz,an,tkoyama} Other:{\it }
\item [Confidence-level:] Hi:{\it mscruz,an,tkoyama} Med:{\it } Low:{\it } None:{\it }
\item [Source-node:] Company2::addEmployee

\item [Lines:] 21-26

\item [Description:] This part of the else-if statement should
provide execution for the condition when compare is greater than 0.
\end{description}
\item {\it Issue\#278 (mscruz)}
\begin{description}
\item [Subject:] new\_node inserted at wrong place
\item [Criticality:] Hi:{\it mscruz,an} Med:{\it tkoyama} Low:{\it } None:{\it }
\item [Suggested-by:] Me:{\it tkoyama} Me-other:{\it } Other-and-me:{\it an} Other:{\it mscruz}
\item [Confidence-level:] Hi:{\it mscruz,tkoyama,an} Med:{\it } Low:{\it } None:{\it }
\item [Source-node:] Company2::addEmployee

\item [Lines:] 29

\item [Description:] previous-{\tt >}next points to the last employee on
the list.  Therefore, when it is made to point to the new\_node, the pointer
to the last employee is lost.
\end{description}
\item {\it Issue\#282 (mscruz)}
\begin{description}
\item [Subject:] Last employee not printed
\item [Criticality:] Hi:{\it } Med:{\it an} Low:{\it tkoyama,mscruz} None:{\it }
\item [Suggested-by:] Me:{\it } Me-other:{\it mscruz} Other-and-me:{\it tkoyama,an} Other:{\it }
\item [Confidence-level:] Hi:{\it tkoyama,mscruz,an} Med:{\it } Low:{\it } None:{\it }
\item [Source-node:] Company2::print

\item [Lines:] 13

\item [Description:] Since the last employee points to a zero, it
will not satisfy the conditon of the loop.  Thus, the print function of this
employee will not be invoked.  It should be while(current != 0).
\end{description}
\end{enumerate}
\section{Review Metrics}
\begin{table}[hb]
\begin{center}
\begin{tabular}{|l|l|l|l|l|}
\hline
Participant & Start-time & End-time & Elapsed-time & Total Busy-time \\
\hline
dat & Mar 23, 1995 18:15:16 & Mar 23, 1995 21:12:03 & 2:56:47 & 1:43:51 \\
an & Mar 23, 1995 18:17:52 & Mar 23, 1995 21:04:09 & 2:46:17 & 2:14:0 \\
tkoyama & Mar 23, 1995 18:17:01 & Mar 23, 1995 21:04:08 & 2:47:7 & 1:49:33 \\
mscruz & Mar 23, 1995 18:16:59 & Mar 23, 1995 21:04:12 & 2:47:13 & 2:24:51 \\
\hline
\end{tabular}
\end{center}
\caption{Review Session}
\end{table}


\begin{table}[hb]
\begin{center}
\begin{tabular}{|l|l|l|l|l|}
\hline
Source & dat & an & tkoyama & mscruz\\
\hline
(192)Employee::setSocSecurity & 760 & 692 & 755 & 817\\
(208)Company2::\~Company2 & 204 & 204 & 435 & 205\\
(194)Employee::setAge & 95 & 79 & 85 & 86\\
(210)Company2::findEmployee & 309 & 645 & 619 & 799\\
(180)Employee & 341 & 343 & 1733 & 341\\
(196)Employee::setNumDependents & 23 & 230 & 189 & 190\\
(212)Company2::addEmployee & 1353 & 1450 & 0 & 1422\\
(182)EmployeeNode & 49 & 338 & 700 & 339\\
(198)Employee::print & 141 & 98 & 152 & 151\\
(214)Company2::deleteEmployee & 568 & 555 & 156 & 674\\
(184)Company2 & 293 & 295 & 478 & 293\\
(200)Employee::getSocSecurity & 44 & 44 & 44 & 44\\
(216)Company2::print & 217 & 543 & 86 & 549\\
(186)Employee::Employee & 365 & 379 & 365 & 365\\
(202)EmployeeNode::EmployeeNode & 167 & 169 & 167 & 168\\
(188)Employee::\~Employee & 176 & 186 & 1092 & 174\\
(204)EmployeeNode::\~EmployeeNode & 72 & 421 & 202 & 422\\
(190)Employee::setName & 611 & 908 & 779 & 1138\\
(206)Company2::Company2 & 404 & 405 & 460 & 403\\
\hline
\end{tabular}
\end{center}
\caption{Review Time}
\end{table}


\begin{table}[hb]
\begin{center}
\begin{tabular}{|l|l|l|}
\hline
Source node & Issue node  & OK\\
\hline
(192)Employee::setSocSecurity & \#252 (=1) & \\
(208)Company2::\~Company2 &  & \\
(194)Employee::setAge &  & \\
(210)Company2::findEmployee &  & \\
(180)Employee &  & \\
(196)Employee::setNumDependents &  & \\
(212)Company2::addEmployee & \#262,\#266,\#270,\#274,\#278 (=5) & 262,266,270\\
(182)EmployeeNode &  & \\
(198)Employee::print &  & \\
(214)Company2::deleteEmployee &  & \\
(184)Company2 &  & \\
(200)Employee::getSocSecurity &  & \\
(216)Company2::print & \#282 (=1) & 282\\
(186)Employee::Employee & \#226 (=1) & \\
(202)EmployeeNode::EmployeeNode &  & \\
(188)Employee::\~Employee & \#232 (=1) & \\
(204)EmployeeNode::\~EmployeeNode & \#256 (=1) & 256\\
(190)Employee::setName & \#236,\#240,\#244,\#248 (=4) & 244 \\
(206)Company2::Company2 & \#260 (=1) & 260\\
\hline
\end{tabular}
\caption{Source node v.s Issue node}
\end{center}
\end{table}

%%\end{document}
