\chapter {Error List (ICS313 Experiment)}

\section {Employee1}

\begin{table}[hb]
\begin{center}
\begin{tabular}{|l|l|l|l|l|}
\hline
No & Source Code & Line & Description & Frequency\\
\hline
1 & Employee::Employee & 4 & Memory is allocated then set to NULL (memory
leak). & 3\\
2 &                    & 5 & Uninitialized socSecurity, will print
garbage. & 1\\
3 & Employee::setName & 11 & Should have been len+1 to include Null
character. & 12\\
4 & Employee::setSocSec.. & 8 & Remaning characters after 11 are not
checked. & 5\\ 
5 &                       & 11 & Should have been $||$. & 8\\
6 &                      & 13 & Should be part of the if-then clause
(missing \{\}). & 2\\
7 &                     & 28 & Null character is not copied & 6\\
8 &                     & 32 & Should not return error, but should invoke
new and & 2\\
 &                       &   & copy newNum, or this statement can be simply
 removed. & \\
9 & Employee::setNumDep.. & 4 & Should have been $>=$ 0. & 13\\
10 & Employee::getSocSec.. & & Should not return pointer to private
member. & 2\\
11 & Company1::Company1 & 6 & Should have been i $>=$ 0. 14 & 14\\
12 & Company1::$\sim$Company1 & 4 & Employee objects not deleted (memory
leak). & 6\\
13 & Company1::addEmp.. & 11 & j can be negative. & 4\\
14 &                       & 19,23 & On duplicate SSN, algorithm doesn't
restore & 2\\
15 &                       & 23 & newWorker must be deleted (memory leak) &
0\\
16 &                       & 27 & j can be negative & 1\\
17 & Company1::findEmp.. & 5 & Loop may not terminate when array of Workers is
full & 6\\
18 & Company1::deleteE.. & 4 & Loop may not terminate when array of Workers
is full & 5\\
19 &                     & 17 & SSN must be deleted (Driver doesn't delete
SSN) & 0\\
20 &                     & 11 & Loop may not terminate when array of
Workers is full & 4\\
21 &                     & 15 & Last array is not set to Null & 3\\
22 &                     & 15 & numEmployees is not decremented & 1\\
23 & Company1::print     & 6 & Loop may not terminate when array of Workers
is full & 5\\
\hline
\end{tabular}
\caption{Error List for Employee1}
\end{center}
\end{table}

\section {Employee2}

\begin{table}[hb]
\begin{center}
\begin{tabular}{|l|l|l|l|l|}
\hline
No & Source Code & Line & Description & Frequency\\
\hline
1 & Employee::setName & 7 & Doesn't include null. Should have been n+1. & 10\\
2 &                   & 9 & Space not checked: newName can be longer than
name & 8\\
3 & Employee::setSoc.. & 11 & i will access illegal location when newSSN
$>$ 12 & 1\\ 
4 &                   & 11-21& When newSSN is invalid, old value is not
restored & 2\\
5 &                   & 18 & Doesn't copy '-' to socSecurity& 7\\
6 &                   & 23 & Should have been!=9,otherwise will accept
SSN$>$9digits & 7 \\
7 &                   & 26 & Doesn't add null character into socSecurity. &
4\\ 
8 & Employee::setAge   & 4 & Should have been $||$.  & 8 \\
9 & Employee::getSoc.. & & Should not return pointer to private member.  & 2 \\
10 &EmployeeNode::$\sim$Emp..& & Should delete employee object. & 4 \\
11 & Company2::Company2 & & numEmployees is not initialized. & 10 \\ 
12 & Company2::findEmp..&14 & Current can be NULL.  & 4 \\
13 &                    &17 & Should have been $||$ & 9 \\ 
14 &Company2::addEmp..& 7 & When duplicate, EmployeeNode needs to be
deleted. & 1 \\
15 &                  & 1  & When duplicate, newWorker needs to be
deleted. & 0 \\
16 &                  & 14 & Should have been == & 5 \\ 
17 &                  & 15 & Should return 1  & 7 \\ 
18 &                  & 17 & Should have been $>$  & 5 \\
19 &                  & 23 & When inserting at the beginning, previous is
NULL. & 2 \\ 
20 &                  & 29 & previous can be NULL. & 5 \\
21 & Company2::delete..& 11,13 & Doesn't delete first node, or current can
be NULL. & 8 \\
22 &                   & 11-20& Infinite loop when variable result always
$>$ 0 & 3 \\
23 &                   & 18 & Should decrement numEmployee before return & 1 \\
24 &                   & 22 & SSN needs to be deleted (memory leak)  & 0 \\
25 & Company2::print & 13 & Doesn't print last node & 11 \\
\hline
\end{tabular}
\caption{Error List for Employee2}
\end{center}
\end{table}
