
\chapter {Questionnaire For Groups (EGSM) -- ICS313}


\begin{enumerate}
\item My understanding of the source code before the review was: 
\\
Very low \hfill 1 \dotfill  2 \dotfill 3 \dotfill 4 \dotfill 5 \hfill Very high\\
(Score: 3.1)


\item My understanding of the source code after the review was: 
\\
Very low \hfill 1 \dotfill  2 \dotfill 3 \dotfill 4 \dotfill 5 \hfill Very high\\
(Score: 4.0)


\item My understanding of C++ programming language was improved after
this review.  
\\
Not at all true \hfill 1 \dotfill  2 \dotfill 3 \dotfill 4 \dotfill 5 \hfill Very true\\
(Score: 3.4)


\item In general, our group found it easy to understand the logic of
the code. 
\\
Not at all true \hfill 1 \dotfill  2 \dotfill 3 \dotfill 4 \dotfill 5 \hfill Very true\\
(Score: 3.1)


\item During the review, I follow or pay attention to the presenter.
\\
Not at all \hfill 1 \dotfill  2 \dotfill 3 \dotfill 4 \dotfill 5
\hfill At all time\\
(Score: 4.4)

\item For this review, I would have preferred working in a group rather 
than working alone (if I have had a choice).
\\
Not at all true \hfill 1 \dotfill  2 \dotfill 3 \dotfill 4 \dotfill 5 \hfill Very true\\
(Score: 4.1)

\item For this review, I felt more confidence about the issues I raised when 
working in a group that working alone.
\\
Not at all true \hfill 1 \dotfill  2 \dotfill 3 \dotfill 4 \dotfill 5 \hfill Very true\\
(Score: 3.9)

\item In general, I felt working in this group increases my ability
in finding errors. 
\\
Not at all true \hfill 1 \dotfill  2 \dotfill 3 \dotfill 4 \dotfill 5 \hfill Very true\\
(Score: 4.3)

\item The training session for group review was sufficient.
\\
Not at all true \hfill 1 \dotfill  2 \dotfill 3 \dotfill 4 \dotfill 5 \hfill Very true\\
(Score: 4.0)

\item There was sufficient time to work on this review.
\\
Not at all true \hfill 1 \dotfill  2 \dotfill 3 \dotfill 4 \dotfill 5 \hfill Very true\\
(Score: 3.4)

\item Our group was motivated to do this review project.
\\
Not at all true \hfill 1 \dotfill  2 \dotfill 3 \dotfill 4 \dotfill 5 \hfill Very true\\
(Score: 4.0)

\item Our group worked seriously on this review.
\\
Not at all true \hfill 1 \dotfill  2 \dotfill 3 \dotfill 4 \dotfill 5 \hfill Very true\\
(Score: 4.6)

\item My overall confidence in the quality of our review was
\\
Very low \hfill 1 \dotfill  2 \dotfill 3 \dotfill 4 \dotfill 5 \hfill Very high\\
(Score: 3.9)

\item My overall satisfaction with the discussion among my group
 members was:
\\
Very low \hfill 1 \dotfill  2 \dotfill 3 \dotfill 4 \dotfill 5 \hfill Very high\\
(Score: 4.2)

\item This group was too small (in number of members) for best results
in the task it was trying to do.
\\
Not at all true \hfill 1 \dotfill  2 \dotfill 3 \dotfill 4 \dotfill 5 \hfill Very true\\
(Score: 1.7)

\item I felt comfortable doing this review with the group.
\\
Not at all true \hfill 1 \dotfill  2 \dotfill 3 \dotfill 4 \dotfill 5 \hfill Very true\\
(Score: 4.2)

\item There was much disagreement among the members of the group on
this task.
\\
Not at all true \hfill 1 \dotfill  2 \dotfill 3 \dotfill 4 \dotfill 5 \hfill Very true\\
(Score: 2.2)

\item Some people in the group dominated the discussion.
\\
Not at all true \hfill 1 \dotfill  2 \dotfill 3 \dotfill 4 \dotfill 5 \hfill Very true\\
(Score: 2.4)

\item My opinion was given adequate consideration by the other group members.
\\
Not at all true \hfill 1 \dotfill  2 \dotfill 3 \dotfill 4 \dotfill 5 \hfill Very true\\
(Score: 4.3)


\item I felt that I could express my opinion freely during group discussion. 
\\
Not at all true \hfill 1 \dotfill  2 \dotfill 3 \dotfill 4 \dotfill 5 \hfill Very true\\
(Score: 4.4)

\item I felt that I could express disagreement freely.
\\
Not at all true \hfill 1 \dotfill  2 \dotfill 3 \dotfill 4 \dotfill 5 \hfill Very true\\
(Score: 4.1)

\item I felt that I participated a great deal in the group discussion.
\\
Not at all true \hfill 1 \dotfill  2 \dotfill 3 \dotfill 4 \dotfill 5 \hfill Very true\\
(Score: 4.2)

\item I felt our group wasted too much time on unproductive
discussion.
\\
Not at all true \hfill 1 \dotfill  2 \dotfill 3 \dotfill 4 \dotfill 5 \hfill Very true\\
(Score: 2.0)

\item I felt I contributed a great deal to the discovery of issues.
\\
Not at all true \hfill 1 \dotfill  2 \dotfill 3 \dotfill 4 \dotfill 5 \hfill Very true\\
(Score: 3.5)

\item I felt the group made a great deal of influence on my decision
about what would be an issue, the criticality of an issue, and/or confidence-level.
\\
Not at all true \hfill 1 \dotfill  2 \dotfill 3 \dotfill 4 \dotfill 5 \hfill Very true\\
(Score: 3.0)

\item Overall, I was satified with the group interaction
\\
Not at all true \hfill 1 \dotfill  2 \dotfill 3 \dotfill 4 \dotfill 5 \hfill Very true\\
(Score: 4.3)

\item I would enjoy working with members of this group again.
\\
Not at all true \hfill 1 \dotfill  2 \dotfill 3 \dotfill 4 \dotfill 5 \hfill Very true\\
(Score: 4.4)

\item There was some open hostility in the group.
\\
Not at all true \hfill 1 \dotfill  2 \dotfill 3 \dotfill 4 \dotfill 5 \hfill Very true\\
(Score: 1.6)

\item Before this experiment, I knew all members of the group fairly well. 
\\
Not at all true \hfill 1 \dotfill  2 \dotfill 3 \dotfill 4 \dotfill 5 \hfill Very true\\
(Score: 1.4)

\item Our group took too long to come to an agreement.
\\
Not at all true \hfill 1 \dotfill  2 \dotfill 3 \dotfill 4 \dotfill 5 \hfill Very true\\
(Score: 2.5)

\item Some individual(s) in the group wanted to change things after
the group had come to a decision.
\\
Not at all true \hfill 1 \dotfill  2 \dotfill 3 \dotfill 4 \dotfill 5 \hfill Very true\\
(Score: 2.2)

\item I felt the presenter was too fast in presenting the code.
\\
Not at all true \hfill 1 \dotfill  2 \dotfill 3 \dotfill 4 \dotfill 5 \hfill Very true\\
(Score: 1.9)

\item The paraphrasing technique inspired me in finding errors.
\\
Not at all true \hfill 1 \dotfill  2 \dotfill 3 \dotfill 4 \dotfill 5 \hfill Very true\\
(Score: 2.9)

\item I felt the paraphrasing technique in general was useful.
\\
Not at all true \hfill 1 \dotfill  2 \dotfill 3 \dotfill 4 \dotfill 5 \hfill Very true\\
(Score: 3.2)

\item Overall, I felt the presenter did a good job in presenting/
paraphrasing the code.
\\
Not at all true \hfill 1 \dotfill  2 \dotfill 3 \dotfill 4 \dotfill 5 \hfill Very true\\
(Score: 4.1)

\item My overall satisfaction with my group was:
\\
Very low \hfill 1 \dotfill  2 \dotfill 3 \dotfill 4 \dotfill 5 \hfill Very high\\
(Score: 4.3)

\item My overall satisfaction with group review (EGSM) process as
outlined in the FTR guideline was:
\\
Very low \hfill 1 \dotfill  2 \dotfill 3 \dotfill 4 \dotfill 5 \hfill Very high\\
(Score 4.0)

\item I believe EGSM system made my review more productive (i.e., find
lots of issues) 
\\
Not at all true \hfill 1 \dotfill  2 \dotfill 3 \dotfill 4 \dotfill 5 \hfill Very true\\
(Score: 3.9)

\item I believe EGSM system made my review more effective
(i.e., find lots of ``good'' issues in a relatively short time)
\\
Not at all true \hfill 1 \dotfill  2 \dotfill 3 \dotfill 4 \dotfill 5 \hfill Very true\\
(Score: 3.8)

\item EGSM system is easy to use.
\\
Not at all true \hfill 1 \dotfill  2 \dotfill 3 \dotfill 4 \dotfill 5 \hfill Very true\\
(Score: 4.2)


\item EGSM system is useful.
\\
Not at all true \hfill 1 \dotfill  2 \dotfill 3 \dotfill 4 \dotfill 5 \hfill Very true\\
(Score: 4.1)

\item My overall satisfaction with EGSM system was
\\
Very low \hfill 1 \dotfill  2 \dotfill 3 \dotfill 4 \dotfill 5 \hfill Very high\\
(Score: 4.1)


\item Problems that I had with paraphrasing technique (please
explain):
\begin{itemize}
\item Sometimes it just seemed so ridculous to paraphrase statements which
were obvious.
\item Need to understand C++ better before trying.  Hard to visualize what
happens in the program without seeing what it does
\item No real problems, I actually found it really useful.
\item It is difficult to find errors by reading.  Need to view limits and
processes.

\item none
\item No problems really.

\item I had problem with finding correct terms to explain some of instances
found during the discussion.

\item the paraphrasing was tedious, but I understand that it is necessary
\item If the code was not clar at the time it lead to broken sentences

\item none

\item not anything that was so terrible
\end{itemize}


\item Problems that I had with the presenter (please explain):
\begin{itemize}
\item None.  Listened to every issue, made sure he has thorough when
searching the code.
\item No problems.  He did his job well and fielded the 
 discussions quite nicely.
\item Presenting was hard to adjust to.  We would need to try some different
processes over a period of time to get it "right".  It would take time
to know what a good presenter does.
\item none, presenter was efficient.
\item I was the presenter.  I could critique myself as not paraphrasing
adequately enough.
\item N/A
\item none
\item I was the presenter so none!
\item none

\item Sometimes he forgot about me and didn't really pay attention
\end{itemize}
\begin{itemize}
\item None.  Very patient and helpful.
\item None.
\item The system is hard to process in a formal way for the first time.  The
moderator might help more by giving some direction.
\item none.
\item None.  He was very helpful.
\item N/A
\item none
\item None
\item none
\item none
\end{itemize}

\item Problems that I had with the moderator (please explain):
\begin{itemize}
\item Several times I forgot to save issues or clicked in the wrong spot and
threw everything off.
\item None.  Easy to work with, but may depend on who is in your group.
Must be able to get along, not be afraid to share.
\item time constrain
\item None.
\item none. was easy to use and intuitive.
\item None in particular.
\item the system kept freezing when in the midst of creating an issue..
\item none, it works very well and is easy to use
\item Could use a more user friendly interface
\item none
\item none I enjoyed it
\end{itemize}

\item Problems that I had with EGSM system (please explain): 
\begin{itemize}
\item It seemed to be quite tedious at times
\item Very long.
\item take time to sort out disagreements

\item None.
\item none.
\item The use of separate terminals was a large problem.  I found myself
looking at the speakers' terminals insteads.  I felt I needed
eye-contact with them.  Also, referring to a line is much easier
pointing to it on their screen rather than telling them what line to
look at.
\item None in particular.
\item I wish we had reference materials on c++ so our time wouldn't be
wasted on things we should have known

\item Good well thought out
\item none
\item none
\end{itemize}

\item Problems that I had with EGSM process (please explain): 

\begin{itemize}
\item There should be less chance of serious error for mistakes like
clicking in the wrong spot.  Also, issue-saving and summary
buffer-refreshing should be automatic.
\item Allow for bringing in reference material.  Hard to remember everything
-- having some examples to look at would make it easier.
\item Screens should be more inactive between reviewers and presenter.

\item None.
\item Line numbers might be just put in by number instead of using the
mouse.  A list of the errors on a given page could help discuss why
this is a new error - when we find more than one error in an area.
\item none.
\item Require only one terminal to be used.  
\item Given more time, our group could have been more productive.
\item None
\item none
\item Needed more time.  Maybe several shorter sessions.
\item for this to work the presenter needs to be on the  ball
\end{itemize}


\item Suggestions on how to improve EGSM system (please explain):

\begin{itemize}
\item Only paraphrase statements that there is some doubt about as sort of a
helping tool.
\item Same as above
\item None.
\item bibe.
\item Have another secondary presenter or something, just in case the
presenter interprets the a statement wrong.
\item Students must be given better access to the system. 
\item The group would work best if all involved can express themselves clearly
\item none
\end{itemize}


\item Suggestions on how to improve EGSM process (please explain):
\begin{itemize}
\item None.
\item This is a good system
\item The way people sign up for the time slots to do this project is 
not sufficient.  A possibly better way to do it would be to create 
a sign in sheet with predesignated times.  This would prevent students
from all coming in at the same time.
\item Some of the code was really difficult to comprehend even with other
people.  Should have a maximum time limit.
\item none
\item Wonderful system.  It's about time such a system was developed.  Until
now, I have never heard of a group like this.  This would surely kill
the bugs that plague a program.
\item Special thanks to the TA.
\item really enjoyed doing this stuff, 
\item None
\item I thought the technique was very good.  Although, I time spent looking
for bugs was draining.  I suggest shorter sessions so that the group
could relax.
\item 
I think this kind of programme is very useful for learning the code
written by other people.  This really helped me in better
understanding of logic and issues that we came across, which I never
thought to be of importance and usually ignored it.  I learned a lot
from this programme.  I request that all the ICS classes should have
this kind of practice.  Overall, Danu is an excellent guide and
explains well what he expects from us. Above all, this is an excellent
programme.  Much Mahalos!
\end{itemize}


\item Other comments:


\end{enumerate}


===================EIAM======================================


\chapter {Questionnaire For Individuals (EIAM) --ICS313}

\begin{enumerate}

\item  My understanding of the source code before the review was: 
\\
Very low \hfill 1 \dotfill  2 \dotfill 3 \dotfill 4 \dotfill 5 \hfill Very high\\
(Score: 3.2)

\item  My understanding of the source code after the review was: 
\\
Very low \hfill 1 \dotfill  2 \dotfill 3 \dotfill 4 \dotfill 5 \hfill Very high\\
(Score: 3.7)

\item  My understanding of C++ programming language was improved after
this review.  
\\
Not at all true \hfill 1 \dotfill  2 \dotfill 3 \dotfill 4 \dotfill 5 \hfill Very true\\
(Score: 3.4)

\item  In general, I found it easy to understand the logic of
the code. 
\\
Not at all true \hfill 1 \dotfill  2 \dotfill 3 \dotfill 4 \dotfill 5 \hfill Very true\\
(Score: 3.2)

\item  For this review, I would have preferred working alone rather 
than working in a group (if I have had a choice).
\\
Not at all true \hfill 1 \dotfill  2 \dotfill 3 \dotfill 4 \dotfill 5 \hfill Very true\\
(Score: 2.6)

\item  For this review, I felt more confidence about the issues I raised when 
working alone rather than working in a group.
\\
Not at all true \hfill 1 \dotfill  2 \dotfill 3 \dotfill 4 \dotfill 5 \hfill Very true\\
(Score: 2.3)

\item  The training session for individual review was sufficient.
\\
Not at all true \hfill 1 \dotfill  2 \dotfill 3 \dotfill 4 \dotfill 5 \hfill Very true\\
(Score: 3.8)

\item   There was sufficient time to work on this review.
\\
Not at all true \hfill 1 \dotfill  2 \dotfill 3 \dotfill 4 \dotfill 5 \hfill Very true\\
(Score: 3.9)

\item   I was motivated to do this review project.
\\
Not at all true \hfill 1 \dotfill  2 \dotfill 3 \dotfill 4 \dotfill 5 \hfill Very true\\
(Score: 4.0)

\item  My overall confidence in the quality of my review was
\\
Very low \hfill 1 \dotfill  2 \dotfill 3 \dotfill 4 \dotfill 5 \hfill Very high\\
(Score: 3.1)

\item  I felt comfortable doing this review.
\\
Not at all true \hfill 1 \dotfill  2 \dotfill 3 \dotfill 4 \dotfill 5 \hfill Very true\\
(Score: 3.5)

\item  It took me a good while to decide whether a particular program
segment contained valid issues.
\\
Not at all true \hfill 1 \dotfill  2 \dotfill 3 \dotfill 4 \dotfill 5 \hfill Very true\\
(Score: 3.7)

\item  I often wanted to change or delete issues I just created.
\\
Not at all true \hfill 1 \dotfill  2 \dotfill 3 \dotfill 4 \dotfill 5 \hfill Very true\\
(Score: 2.9)

\item  My overall satisfaction with individual review (EIAM) process as
outlined in the FTR guideline was:
\\
Very low \hfill 1 \dotfill  2 \dotfill 3 \dotfill 4 \dotfill 5 \hfill Very high\\
(Score: 3.7)

\item  I believe EIAM system made my review more productive (i.e., find
lots of issues) 
\\
Not at all true \hfill 1 \dotfill  2 \dotfill 3 \dotfill 4 \dotfill 5 \hfill Very true\\
(Score: 3.3)

\item  I believe EIAM system made my review more effective
(i.e., find lots of ``good'' issues in a relatively short time)
\\
Not at all true \hfill 1 \dotfill  2 \dotfill 3 \dotfill 4 \dotfill 5 \hfill Very true\\
(Score: 3.3)

\item  EIAM system is easy to use.
\\
Not at all true \hfill 1 \dotfill  2 \dotfill 3 \dotfill 4 \dotfill 5 \hfill Very true\\
(Score: 4.2)

\item  EIAM system is useful. 
\\
Not at all true \hfill 1 \dotfill  2 \dotfill 3 \dotfill 4 \dotfill 5 \hfill Very true\\
(Score: 4.0)

\item  My overall satisfaction with EIAM system was:
\\
Very low \hfill 1 \dotfill  2 \dotfill 3 \dotfill 4 \dotfill 5 \hfill Very high\\
(Score: 3.9)

\item  Problems that I had with EIAM system (please explain): 
\begin{itemize}
\item It is very hard to decide the code is right or wrong.  Sometimes I
have hard time to find out the thing i want.
\item to choose proper subject
\item Figuring out which buttons to press for what.
\item I couldn't verify my doubts when I had questions on the  code.  
\item the system works okay and would prove very valuable to programmers
\item I was very distracted because another student in the lab broght to my
attention a misjudged grade from class.
\item None.
\item Looking at various windows simultaneously.  There isn't much room to
fit them all.  The node and issue windows could be a bit smaller.

\item none really, again interface was easy to use
\item I liked being able to talk the issues over with others.
\item Limited way of identifying issues.
\item insufficient training

\item some times I just wanted to change the whole algorithm of the program
code.
\item I often forget to close an issue before going to the next.  Also some
odd encounting of hight lighting
\end{itemize}


\item  Problems that I had with EIAM process (please explain): 
\begin{itemize}
\item ok
\item Understanding the logic of the code

\item None.
\item not much
\item None , but the distraction
\item I found that my confidence level in raising issues was lower than the
gropu session because in the group session I could raise something
that didn't seem right to me, but I wasn't sure about, and the others
would help decide if it was correct or not.  In the individual
process, I wasn't as secure in raising issues.
\item It's a bit too restrictive.  I wish that we could confer with each
other, at least for what a particular operating system would do for
a statement.
\item none

\item Lack of confidence on some complexities.
\item Requires a lot of concentration, and if there is excess noise in the
lab, it is hard to follow the code.
\item had to keep in mind the corrections that had been made previously. In
other words, the result of debugging is not visible through the entire process.
\item None
\end{itemize}


\item  Suggestions on how to improve EIAM system (please explain):
\begin{itemize}
\item good
\item I don't know if this is possible, but having a flow chart along with
all the functions would be a great help for me.

\item Automatically refresh summary buffer after each function is reviewed.
\item provide some type of reference for better quality analysis'.

\item maybe to improve the fonts to make the words bigger,
eventhough the programmer is just allowed to create issue
the programmer should be allowed to test the code
\item None.



\item Better layout of the screen.  Not so many clicks for a function.  The
voting issues in the issue screen could use additonal options, such as
"could be better" in the criticallity options.
\item none
\item Pair screens for related functions - object file paired with its functions.
\item Provide space to test.  Hard to follow code just by examining text.
\item Non;  not at the present moment.
\item It will be nice to have automatic closure of issue when a new one is created.
\end{itemize}

\item  Suggestions on how to improve EIAM process (please explain):
\begin{itemize}
\item if possible, offer correct solution after creating the proper issue.
\item combine both the individual and group EIAM process
\item None.
\item There should be a conferring part where we could verify that we are
thinking along the same lines.
\item none
\item Make sure lab is completely quiet when setting up the session.
\end{itemize}


\item  Other comments:
\begin{itemize}
\item give some of standard subjects to be choosen
\item In general, I think that this EIAM is very useful when it comes to
source code review. One of the attribute that I like most is that you
can jump between functions rather easily.

\item To me, the source code was kind of difficult, although we already did
this assignment.  I didn't review the material very much before the review.
\item I could have been much more effective if I were not distracted, once I
found out my grade I could think of nothing else!



\item I think this is a valuable debugging tool, but it is most valuable
when used in conjunction with the group session.  I think it would
probably be most efficient if the individual session was done first so
all the members would already be familiar with the code and could
contribute more to the sessions. (This is what I personally believe,
however, I can think of arguments for doing it the other way around as well)
\item This process is too impersonal.  I could imagine myself talking to no
one and getting frustrated while at it.  Also, during the test, I
found that I always suspected there was an error, even though I'm more
than 95% sure there wasn't.  Thus, I spent too much time looking for
something that wasn't there.
\item i think this is a good idea and it should be used however i believe
there should be a stepper/debugger to help debug the code as well
as using this method! i tend to actually step through the code on
paper while working (keeping track of the variables etc)
on this so the stepper would be useful.


\item None.
\item I enjoyed the reviewing thoroghly
\item Special Thanks to the TA.
\item I do think this process may not help me very much on improving my skill of
C++, although it does offer me an excellent medium to review my code
and debug my program.  I often do the some thing on paper.  There is
no boubt a need to do programe review.  This is a good system to go.
But I may not help me directly very much.  When you are in the wood,
all you see are just trees, not the forest!  You just have to have the
skills for you to see potential problems.  A good medium of reviewing
may not make you any smarter.

Good luck and Happy hacking...

\end{itemize}


\end{enumerate}

=====================POST==================

\chapter {Final Questionnaire (ICS313) }

\begin{enumerate}

\item  I would rather review source code manually using pencil and paper than
using CSRS (on-line Collaborative Software Review System).
\\
Not at all true \hfill 1 \dotfill  2 \dotfill 3 \dotfill 4 \dotfill 5 \hfill Very true\\
(Score: 2.7)

\item  In general, CSRS is useful.
\\
Not at all true \hfill 1 \dotfill  2 \dotfill 3 \dotfill 4 \dotfill 5 \hfill Very true\\
(Score: 4.2)

\item  Overall, I enjoyed using CSRS.
\\
Not at all true \hfill 1 \dotfill  2 \dotfill 3 \dotfill 4 \dotfill 5
\hfill Very true\\
(Score: 4.0)

\item I would:
    \begin{enumerate}
   \item[(1)]Strongly prefer using EGSM
   \item[(2)]Somewhat prefer using EGSM
   \item[(3)]Equally prefer using EGSM or EIAM
   \item[(4)]Somewhat prefer using EIAM
   \item[(5)]Strongly prefer using EIAM
   \end{enumerate}
(Score: 2.2)

\item I am:
    \begin{enumerate}
     \item[(1)] Much more productive using EGSM
     \item[(2)] Somewhat more productive using EGSM
     \item[(3)] Equally productive with EGSM or EIAM
     \item[(4)] Somewhat more productive using EIAM
     \item[(5)] Much more productive using EIAM
     \end{enumerate}   
(Score: 2.0)

\item  I prefer working:
   \begin{enumerate}
     \item[(1)] Alone 
     \item[(2)] Group
   \end{enumerate} 
(Score: 1.9)

\item  Reasons for preferring working in a group:
\begin{itemize}
\item 
Working in group provides various ideas from group members.  That
helps the rest of the group members to understand the material and
innovate new ideas.  Consequently, result in better software development.
\item I have learned that it is important to have someone to look over the
mistakes that I might make during the debugging procress.
\item  
we can discuss the ? each other
\item I could find more errors when I worked with my group although we had
some disagreements on certain topics each other.
\item it provides analysis from different angles. Others may see problems
that are obvious to them but not to the programmer.
\item if no clue, group ok, if need quiet thinking, prefer
individual. Depends the situations.
\item I always prfered to work in a group, If one mistake is missed by one
it may be found by another!!!!!!!

\item In a group setting, each member can contribute their knowledge to the
checking process, and also verify the concepts or issues brought up by
other members.  There is a greater confidence that an issue is valid.
\item By working in a group, others can detect your errors, which you can
learn from since you might not have seen yourself.  Although not everyone will agree with you, you learn to
work together and share ideas peacefully.  Working alone sometimes is
hard when you don't have any idea of what's going on.  There is a
better chance that someone in your group will know the answer if you
don't.

\item my understanding of C++ is not thorough hence working in group I can
learn about the aspects of the language about which I am not confident.
\item Questionnaire

I am not very familiar with C++, so it helps to discuss in a group.
\item During EIAM, I had doubts whether an error has occurred or not.  I
could not refer to anyone in the group, so I had to determine it
myself.  That was pretty difficult.  With EGSM, I could refer to
anyone when there is something questionnable.  I enjoy the collective
reasoning we had during the session, and I can't say that I haven't
learned a thing or two from it.  In fact, I saw errors that I would
never have caught in the same amount of time.
\item I usually work alone, but I find it very helpful to work in groups.
Usually
bugs and logic errors come into focus much more quickly when in a
group.  I think that the type of group members matters a great deal,
and when the group memebers are poor, I'd rather work alone.  However,
such was not the case with this group.
\item I like writing code alone - but review without running is where I
would like some discussion.
\item I felt more confident in raising issues in the group session in
general. But, as I mentioned before, I think this is most valuable
when using both methods of evaluation.
\item group helps finds errors (or correct each other on finding errors)
faster and better.

\item In a group, we come up a lot of suggests and questions which is
helpful and useful.  I can learn things from others, and some of the
errors which I can't find out are found out by them.


\item I don't know my c++ well enough to contribute to discussions, so
working in a group is better for me to learn more.
\item Both styles have thier strong points.  However combining the 2 is better.
\item In this case, it was easier to have more people looking at the same
code.  Rather than just one knowledge base, there were three.  If one
person didn't understand, there were others around to help.  Working
alone, there was no one to ask, I have to rely on my own knowledge,
which is rather limited.

\item Still not very familiar with using this program.  Furthermore, not
good at C++ programming make the process harder as it takes more time
to understand the code first and remember the name of the valuable in
order to follow the logic of the program.  Working as a group is
better becasue the others know more about c++ and can identify more
errors than working by myself who is not familiar working with c++
yet.  

\end{itemize}

\item  Reasons for preferring working alone:
\begin{itemize}
\item spend less time argueing about whether the isssue is valid.
\item I trust my own judgement generally and I can follow my own line of
reasoning rather than relying on others to give me ideas.
\item I could get a lot more done in a given amount of time if working by
myself, but may not see the errors that I can in a group session.

\end{itemize}

\end{enumerate}

