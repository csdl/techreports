%\documentstyle[11pt,/group/csdl/tex/definemargins,
%                       /group/csdl/tex/lmacros]{article} 

%          \begin{document}
%          \begin{center}
%          {\large\bf CSRS Experiment Results}\\
%          \end{center}
%          \small 
\chapter {CSRS Experiment Results (ICS313): Group1 (EIAM)}
\small
	  

\begin{description}
\item [Method:] EIAM
\item [Group:] Group1
\item [Source:] Employee2
\item [Participants:] yhkim (Reviewer), shaodan (Reviewer), briang (Reviewer)
\end{description}
\section{Issue Lists}
\begin{enumerate}
\item {\it Issue\#212 (briang)}
\begin{description}
\item [Subject:] name is an char*
\item [Criticality:] Low
\item [Confidence-level:] Hi
\item [Source-node:] Employee::Employee

\item [Lines:] 4

\item [Description:] should be initialized.
\end{description}
\item {\it Issue\#220 (briang)}
\begin{description}
\item [Subject:] socSecurity is an char*
\item [Criticality:] Low
\item [Confidence-level:] Hi
\item [Source-node:] Employee::Employee

\item [Lines:] 5

\item [Description:] should be initialized.
\end{description}
\item {\it Issue\#224 (briang)}
\begin{description}
\item [Subject:] Always new space needed
\item [Criticality:] Med
\item [Confidence-level:] Hi
\item [Source-node:] Employee::setName

\item [Lines:] 6-7

\item [Description:] the function should check if name is zero, if
it is then allocate mem for the new name. If not zero old name array should be deleted
and new space allocated for the new array.
\end{description}
\item {\it Issue\#228 (briang)}
\begin{description}
\item [Subject:] improper copy of string
\item [Criticality:] Hi
\item [Confidence-level:] Med
\item [Source-node:] Employee::setName

\item [Lines:] 9

\item [Description:] It would be better to copy the string using
strncpy(char*, char*, number in the string)
\end{description}
\item {\it Issue\#232 (yhkim)}
\begin{description}
\item [Subject:] Same retuen value
\item [Criticality:] Med
\item [Confidence-level:] Hi
\item [Source-node:] Employee::setName

\item [Lines:] 

\item [Description:] The function always return 0 even if it failed to alocate memory space.
\end{description}
\item {\it Issue\#234 (shaodan)}
\begin{description}
\item [Subject:] initialize
\item [Criticality:] Med
\item [Confidence-level:] Med
\item [Source-node:] Employee::Employee

\item [Lines:] 4

\item [Description:] name is pointer,can't initialize like this
\end{description}
\item {\it Issue\#238 (yhkim)}
\begin{description}
\item [Subject:] Memory allocation error
\item [Criticality:] Med
\item [Confidence-level:] Hi
\item [Source-node:] Employee::setName

\item [Lines:] 

\item [Description:] When the length of the string is greater than 0, it's not going to allocate memory
for the varaible name.
\end{description}
\item {\it Issue\#240 (briang)}
\begin{description}
\item [Subject:] error set the int in a condition
\item [Criticality:] Hi
\item [Confidence-level:] Hi
\item [Source-node:] Company2::addEmployee

\item [Lines:] 14

\item [Description:] compare is being set to 0 when it should just
be checked if it is zero.
\end{description}
\item {\it Issue\#244 (shaodan)}
\begin{description}
\item [Subject:] initiallize
\item [Criticality:] Med
\item [Confidence-level:] Med
\item [Source-node:] Employee::Employee

\item [Lines:] 5

\item [Description:] wrong with initiallize
\end{description}
\item {\it Issue\#246 (yhkim)}
\begin{description}
\item [Subject:] Memory allocation problem
\item [Criticality:] Med
\item [Confidence-level:] Hi
\item [Source-node:] Employee::setSocSecurity

\item [Lines:] 

\item [Description:] If the socSecury is not Null, it might have not enough memory space.  It's better to
delete socSecury first and allocate 12 bytes right after.
\end{description}
\item {\it Issue\#256 (shaodan)}
\begin{description}
\item [Subject:] cond
\item [Criticality:] Low
\item [Confidence-level:] Low
\item [Source-node:] Employee::\~Employee

\item [Lines:] 3-4

\item [Description:] before delete it ,should test if they are 0
\end{description}
\item {\it Issue\#258 (yhkim)}
\begin{description}
\item [Subject:] wrong copy
\item [Criticality:] Med
\item [Confidence-level:] Hi
\item [Source-node:] Employee::setSocSecurity

\item [Lines:] 17

\item [Description:] This function will not copy the character `-` to socSecurity.  It will only copy the
numbers.
\end{description}
\item {\it Issue\#264 (yhkim)}
\begin{description}
\item [Subject:] Possible indefinite loop
\item [Criticality:] Med
\item [Confidence-level:] Hi
\item [Source-node:] Employee::setSocSecurity

\item [Lines:] 11

\item [Description:] If the last character is not Null, it's going to copy more characters to
socSecurity.
\end{description}
\item {\it Issue\#268 (shaodan)}
\begin{description}
\item [Subject:] space
\item [Criticality:] Med
\item [Confidence-level:] Low
\item [Source-node:] Employee::setName

\item [Lines:] 7

\item [Description:] space should be n+1 for terminal char
\end{description}
\item {\it Issue\#272 (yhkim)}
\begin{description}
\item [Subject:] Wrong range check
\item [Criticality:] Med
\item [Confidence-level:] Hi
\item [Source-node:] Employee::setAge

\item [Lines:] 4

\item [Description:] The condition is always false, so it will never return 1.

Possible correction ={\tt >} (newAge {\tt <}17 || newAge {\tt >}55)
\end{description}
\item {\it Issue\#274 (briang)}
\begin{description}
\item [Subject:] Memory Leak!!!!!!!!!!
\item [Criticality:] Hi
\item [Confidence-level:] Hi
\item [Source-node:] EmployeeNode::\~EmployeeNode

\item [Lines:] 4

\item [Description:] The destructor should delete the object that
it poits to,

I.E delete data;

else it will just delete the pointer!
\end{description}
\item {\it Issue\#278 (shaodan)}
\begin{description}
\item [Subject:] no terminal char
\item [Criticality:] Hi
\item [Confidence-level:] Med
\item [Source-node:] Employee::setName

\item [Lines:] 9

\item [Description:] after copy string,should add "\\0" to name[n]
\end{description}
\item {\it Issue\#284 (shaodan)}
\begin{description}
\item [Subject:] return
\item [Criticality:] Hi
\item [Confidence-level:] Med
\item [Source-node:] Employee::setName

\item [Lines:] 10

\item [Description:] alwasy return 0 is wrong
\end{description}
\item {\it Issue\#286 (briang)}
\begin{description}
\item [Subject:] need to init number of employ
\item [Criticality:] Med
\item [Confidence-level:] Hi
\item [Source-node:] Company2::Company2

\item [Lines:] 6

\item [Description:] the number of emp's need to be set to zero.
\end{description}
\item {\it Issue\#296 (yhkim)}
\begin{description}
\item [Subject:] No initializing
\item [Criticality:] Low
\item [Confidence-level:] Med
\item [Source-node:] Company2::Company2

\item [Lines:] 7

\item [Description:] For the company2 class, it is better to initialize the vaiable numEmployess.
\end{description}
\item {\it Issue\#300 (shaodan)}
\begin{description}
\item [Subject:] return
\item [Criticality:] Med
\item [Confidence-level:] Med
\item [Source-node:] Employee::setSocSecurity

\item [Lines:] 24

\item [Description:] forget case if char 3 or 6 is not '-'
\end{description}
\item {\it Issue\#308 (shaodan)}
\begin{description}
\item [Subject:] parethese
\item [Criticality:] Low
\item [Confidence-level:] Low
\item [Source-node:] Employee::setAge

\item [Lines:] 4

\item [Description:] wrong with parethese
\end{description}
\item {\it Issue\#312 (shaodan)}
\begin{description}
\item [Subject:] pointer
\item [Criticality:] Med
\item [Confidence-level:] Med
\item [Source-node:] Company2::Company2

\item [Lines:] 6

\item [Description:] wrong,list\_employee is 2 classes pointer argument
\end{description}
\item {\it Issue\#316 (shaodan)}
\begin{description}
\item [Subject:] delete
\item [Criticality:] Med
\item [Confidence-level:] Med
\item [Source-node:] Company2::\~Company2

\item [Lines:] 11

\item [Description:] wrong, temp is 2 classes pointer
\end{description}
\item {\it Issue\#320 (shaodan)}
\begin{description}
\item [Subject:] return
\item [Criticality:] Med
\item [Confidence-level:] Med
\item [Source-node:] Company2::addEmployee

\item [Lines:] 15

\item [Description:] wrong, no fail case
\end{description}
\item {\it Issue\#324 (shaodan)}
\begin{description}
\item [Subject:] pointer
\item [Criticality:] Med
\item [Confidence-level:] Med
\item [Source-node:] Company2::deleteEmployee

\item [Lines:] 17

\item [Description:] wrong. 2 class pointer delete not like this
\end{description}
\item {\it Issue\#328 (shaodan)}
\begin{description}
\item [Subject:] return
\item [Criticality:] Low
\item [Confidence-level:] Low
\item [Source-node:] Company2::print

\item [Lines:] 10

\item [Description:] wrong, funtion return void, no need return
\end{description}
\end{enumerate}
\section{Review Metrics}
\begin{table}[hb]
\begin{center}
\begin{tabular}{|l|l|l|l|l|}
\hline
Participant & Start-time & End-time & Elapsed-time & Total Busy-time \\
\hline
briang & Apr 07, 1995 10:28:11 & Apr 07, 1995 11:31:59 & 1:3:48 & 1:3:48 \\
shaodan & Apr 07, 1995 10:34:26 & Apr 07, 1995 12:12:17 & 1:37:51 & 1:37:51 \\
yhkim & Apr 07, 1995 10:28:20 & Apr 07, 1995 11:38:22 & 1:10:2 & 1:10:2 \\
\hline
\end{tabular}
\end{center}
\caption{Review Session}
\end{table}


\begin{table}[hb]
\begin{center}
\begin{tabular}{|l|l|l|l|}
\hline
Source & briang & shaodan & yhkim\\
\hline
(176)Employee::Employee & 221 & 729 & 64\\
(192)EmployeeNode::EmployeeNode & 65 & 65 & 35\\
(178)Employee::\~Employee & 111 & 420 & 173\\
(194)EmployeeNode::\~EmployeeNode & 169 & 75 & 52\\
(180)Employee::setName & 495 & 812 & 676\\
(196)Company2::Company2 & 343 & 308 & 156\\
(182)Employee::setSocSecurity & 466 & 531 & 843\\
(198)Company2::\~Company2 & 136 & 389 & 65\\
(184)Employee::setAge & 40 & 338 & 356\\
(200)Company2::findEmployee & 140 & 374 & 319\\
(170)Employee & 171 & 334 & 257\\
(186)Employee::setNumDependents & 24 & 100 & 77\\
(202)Company2::addEmployee & 572 & 318 & 381\\
(172)EmployeeNode & 208 & 150 & 127\\
(188)Employee::print & 38 & 59 & 106\\
(204)Company2::deleteEmployee & 117 & 185 & 120\\
(174)Company2 & 78 & 289 & 209\\
(190)Employee::getSocSecurity & 16 & 93 & 30\\
(206)Company2::print & 288 & 266 & 96\\
\hline
\end{tabular}
\end{center}
\caption{Review Time}
\end{table}


\begin{table}[hb]
\begin{center}
\begin{tabular}{|l|l|l|l|l|}
\hline
Source & briang & shaodan & yhkim & OK \\
\hline
Employee::Employee & 212,220 (=2) & 234,244 (=2) &  & \\
EmployeeNode::EmployeeNode &  &  &  & \\
Employee::\~Employee &  & 256 (=1) &  & \\
EmployeeNode::\~EmployeeNode & 274 (=1) &  &  & 274\\
Employee::setName & 224,228 (=2) & 268,278,284 (=3) & 232,238 (=2) & 268,238\\
Company2::Company2 & 286 (=1) & 312 (=1) & 296 (=1) & 286=296\\
Employee::setSocSecurity &  & 300 (=1) & 246,258,264 (=3) & 258\\
Company2::\~Company2 &  & 316 (=1) &  & \\
Employee::setAge &  & 308 (=1) & 272 (=1) & 272\\
Company2::findEmployee &  &  &  & \\
Employee &  &  &  & \\
Employee::setNumDependents &  &  &  & \\
Company2::addEmployee & 240 (=1) & 320 (=1) &  & 240,320 \\
EmployeeNode &  &  &  & \\
Employee::print &  &  &  & \\
Company2::deleteEmployee &  & 324 (=1) &  & \\
Company2 &  &  &  & \\
Employee::getSocSecurity &  &  &  & \\
Company2::print &  & 328 (=1) &  & \\
\hline
\end{tabular}
\caption{Source node v.s Issue node}
\end{center}
\end{table}

%\end{document}
