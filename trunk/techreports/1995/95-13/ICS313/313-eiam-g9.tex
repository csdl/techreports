%\documentstyle[11pt,/group/csdl/tex/definemargins,
%                       /group/csdl/tex/lmacros]{article} 

%          \begin{document}
%          \begin{center}
%          {\large\bf CSRS Experiment Results}\\
%          \end{center}
%          \small 
\chapter {CSRS Experiment Results (ICS313): Group9 (EIAM)}
\small

\begin{description}
\item [Method:] EIAM
\item [Group:] Group9
\item [Source:] Employee2
\item [Participants:] toddm (Reviewer), bacani (Reviewer), jgeis (Reviewer)
\end{description}
\section{Issue Lists}
\begin{enumerate}
\item {\it Issue\#214 (bacani)}
\begin{description}
\item [Subject:] incorrect initialization
\item [Criticality:] Hi
\item [Confidence-level:] Hi
\item [Source-node:] Employee::Employee

\item [Lines:] 4-5

\item [Description:] both name and socSecurity are pointers to
strings.  They should not be initialized as empty strings.
\end{description}
\item {\it Issue\#218 (bacani)}
\begin{description}
\item [Subject:] unnecessary code
\item [Criticality:] Low
\item [Confidence-level:] Med
\item [Source-node:] Employee::\~Employee

\item [Lines:] 7-8

\item [Description:] setting the variables to zero not required.
\end{description}
\item {\it Issue\#222 (jgeis)}
\begin{description}
\item [Subject:] unnecessary zeroing of name and socSecurity.
\item [Criticality:] Low
\item [Confidence-level:] Hi
\item [Source-node:] Employee::\~Employee

\item [Lines:] 7-8

\item [Description:] name and socSecurity are cleared by the delete
commands, no need to set to zero.
\end{description}
\item {\it Issue\#224 (bacani)}
\begin{description}
\item [Subject:] 
\item [Criticality:] 
\item [Confidence-level:] 
\item [Source-node:] Employee::setName

\item [Lines:] 

\item [Description:] 
\end{description}
\item {\it Issue\#226 (bacani)}
\begin{description}
\item [Subject:] possible overwriting of memory
\item [Criticality:] Hi
\item [Confidence-level:] Hi
\item [Source-node:] Employee::setName

\item [Lines:] 9

\item [Description:] Unless memory is properly allocated for the
correct length of name, strcpy will overwrite data.
\end{description}
\item {\it Issue\#230 (toddm)}
\begin{description}
\item [Subject:] Length of string not checked before copying
\item [Criticality:] Med
\item [Confidence-level:] Hi
\item [Source-node:] Employee::setName

\item [Lines:] 9

\item [Description:] newName copied to name without checking length
of newName
\end{description}
\item {\it Issue\#238 (toddm)}
\begin{description}
\item [Subject:] No return of failure
\item [Criticality:] Low
\item [Confidence-level:] Med
\item [Source-node:] Employee::setName

\item [Lines:] 1

\item [Description:] No error check of input string, will not
return 1 if failed because there is no "return 1;" statement.
\end{description}
\item {\it Issue\#240 (bacani)}
\begin{description}
\item [Subject:] Overwriting old data without fully checking new
\item [Criticality:] Med
\item [Confidence-level:] Hi
\item [Source-node:] Employee::setSocSecurity

\item [Lines:] 13

\item [Description:] old data will be erased regardless of
correctness of new data.  New data should be fully checked before discarding
existing data.
\end{description}
\item {\it Issue\#246 (jgeis)}
\begin{description}
\item [Subject:] cannot be both less than 17 and more than 55.
\item [Criticality:] Hi
\item [Confidence-level:] Hi
\item [Source-node:] Employee::setAge

\item [Lines:] 4

\item [Description:] The vlaue of newage cannot be less than 17 
and more than 55 simultaneously.  This statement will never be executed as
the result will always be false.
\end{description}
\item {\it Issue\#248 (bacani)}
\begin{description}
\item [Subject:] condition never true
\item [Criticality:] Hi
\item [Confidence-level:] Hi
\item [Source-node:] Employee::setAge

\item [Lines:] 4

\item [Description:] a number cannot be both below 17 and above 55.
The if statement will not properly check for the error condition.
\end{description}
\item {\it Issue\#254 (toddm)}
\begin{description}
\item [Subject:] Improper successful return
\item [Criticality:] Hi
\item [Confidence-level:] Hi
\item [Source-node:] Employee::setSocSecurity

\item [Lines:] 23

\item [Description:] There may be more validDigits than 9, which is
not correct, and there will be a return of 0.
\end{description}
\item {\it Issue\#260 (bacani)}
\begin{description}
\item [Subject:] deletion of pointer but not space
\item [Criticality:] Hi
\item [Confidence-level:] Hi
\item [Source-node:] Company2::deleteEmployee

\item [Lines:] 17-18

\item [Description:] pointer deleted, but allocated memory is still alocated.
\end{description}
\item {\it Issue\#262 (toddm)}
\begin{description}
\item [Subject:] Impossible loop
\item [Criticality:] Hi
\item [Confidence-level:] Hi
\item [Source-node:] Employee::setAge

\item [Lines:] 4

\item [Description:] If newAge = 16, it cannot also be greater than
55, so it will exit the if statement and set the age anyway.  But 16 is not a
valid newAge.  It may never detect a failed setting.
\end{description}
\item {\it Issue\#268 (bacani)}
\begin{description}
\item [Subject:] return error message without cleanup
\item [Criticality:] Hi
\item [Confidence-level:] Hi
\item [Source-node:] Company2::addEmployee

\item [Lines:] 14-15

\item [Description:] the function returns an error, but the space
for the new EmployeeNode is still allocated.
\end{description}
\item {\it Issue\#276 (jgeis)}
\begin{description}
\item [Subject:] loop may not terminate
\item [Criticality:] Med
\item [Confidence-level:] Hi
\item [Source-node:] Company2::findEmployee

\item [Lines:] 13

\item [Description:] because both conditions would have to be true
for this loop to terminate, the list could be exhausted (current=0), but the
loop would continue to execute if the ssn is not found.
\end{description}
\item {\it Issue\#280 (bacani)}
\begin{description}
\item [Subject:] Pointer points to null
\item [Criticality:] Hi
\item [Confidence-level:] Hi
\item [Source-node:] Company2::addEmployee

\item [Lines:] 6

\item [Description:] the pointer points to null.  For the very
first employee to be added, previous-{\tt >}next points to unallocated memory.
\end{description}
\item {\it Issue\#284 (jgeis)}
\begin{description}
\item [Subject:] numEmployees has not been initialized.
\item [Criticality:] Med
\item [Confidence-level:] Hi
\item [Source-node:] Company2::addEmployee

\item [Lines:] 24

\item [Description:] The value of numEmployees has not been 
previously given a value.
\end{description}
\item {\it Issue\#286 (bacani)}
\begin{description}
\item [Subject:] inappropriate check
\item [Criticality:] Med
\item [Confidence-level:] Hi
\item [Source-node:] Company2::print

\item [Lines:] 13

\item [Description:] for a company with only only one worker,
current-{\tt >}next is null, so this function will not print out this worker.
\end{description}
\item {\it Issue\#294 (toddm)}
\begin{description}
\item [Subject:] \&\& instead of ||
\item [Criticality:] Hi
\item [Confidence-level:] Low
\item [Source-node:] Company2::findEmployee

\item [Lines:] 17

\item [Description:] If the list is empty (current==0), result will
not be greater than 0 at the same time.  Statement will not return null
pointer.
\end{description}
\item {\it Issue\#298 (jgeis)}
\begin{description}
\item [Subject:] skips checking the first Employee
\item [Criticality:] Med
\item [Confidence-level:] Hi
\item [Source-node:] Company2::deleteEmployee

\item [Lines:] 12

\item [Description:] current-{\tt >}next checks the employees starting
with the second Employee.  The first is skipped.
\end{description}
\item {\it Issue\#300 (bacani)}
\begin{description}
\item [Subject:] faulty checking
\item [Criticality:] Med
\item [Confidence-level:] Hi
\item [Source-node:] Company2::findEmployee

\item [Lines:] 17

\item [Description:] it is possible that current!=0 and result{\tt >}1.
This condition is never checked for.
\end{description}
\item {\it Issue\#306 (jgeis)}
\begin{description}
\item [Subject:] last employee would be skipped
\item [Criticality:] Hi
\item [Confidence-level:] Hi
\item [Source-node:] Company2::print

\item [Lines:] 13

\item [Description:] The final employee would not be printed as
the loop does not execute when current-{\tt >}next=0.
\end{description}
\item {\it Issue\#310 (toddm)}
\begin{description}
\item [Subject:] Does not search all nodes
\item [Criticality:] Hi
\item [Confidence-level:] Med
\item [Source-node:] Company2::deleteEmployee

\item [Lines:] 11

\item [Description:] loop starts with current record, not first node.
\end{description}
\item {\it Issue\#312 (jgeis)}
\begin{description}
\item [Subject:] would not delete what temp points to.
\item [Criticality:] Med
\item [Confidence-level:] Hi
\item [Source-node:] Company2::\~Company2

\item [Lines:] 11

\item [Description:] temp is a pointer, delete temp only deletes
the pointer and not what temp actually points to.
\end{description}
\item {\it Issue\#318 (jgeis)}
\begin{description}
\item [Subject:] could have a larger SSN than allowable
\item [Criticality:] Med
\item [Confidence-level:] Hi
\item [Source-node:] Employee::setSocSecurity

\item [Lines:] 23

\item [Description:] A number could be entered as 999-99-999999.
This would go through as an acceptable value and overwrite correct numbers.
Should be if (validDigits != 9).
\end{description}
\item {\it Issue\#324 (toddm)}
\begin{description}
\item [Subject:] Wrong return value
\item [Criticality:] Hi
\item [Confidence-level:] Hi
\item [Source-node:] Company2::addEmployee

\item [Lines:] 15

\item [Description:] Returns 0 for a duplicate instance, but a
duplicate instance is not a successful add.
\end{description}
\end{enumerate}
\section{Review Metrics}
\begin{table}[hb]
\begin{center}
\begin{tabular}{|l|l|l|l|l|}
\hline
Participant & Start-time & End-time & Elapsed-time & Total Busy-time \\
\hline
jgeis & Mar 25, 1995 14:45:42 & Mar 25, 1995 16:22:11 & 1:36:29 & 1:33:29 \\
bacani & Mar 25, 1995 14:43:16 & Mar 25, 1995 16:18:07 & 1:34:51 & 1:28:52 \\
toddm & Mar 25, 1995 14:44:47 & Mar 25, 1995 16:28:28 & 1:43:41 & 1:30:58 \\
\hline
\end{tabular}
\end{center}
\caption{Review Session}
\end{table}


\begin{table}[hb]
\begin{center}
\begin{tabular}{|l|l|l|l|}
\hline
Source & jgeis & bacani & toddm\\
\hline
(176)Employee::Employee & 318 & 210 & 50\\
(192)EmployeeNode::EmployeeNode & 86 & 142 & 56\\
(178)Employee::\~Employee & 292 & 139 & 25\\
(194)EmployeeNode::\~EmployeeNode & 28 & 124 & 45\\
(180)Employee::setName & 207 & 362 & 590\\
(196)Company2::Company2 & 62 & 88 & 46\\
(182)Employee::setSocSecurity & 592 & 412 & 338\\
(198)Company2::\~Company2 & 340 & 397 & 231\\
(184)Employee::setAge & 238 & 145 & 388\\
(200)Company2::findEmployee & 1072 & 387 & 828\\
(170)Employee & 114 & 138 & 99\\
(186)Employee::setNumDependents & 107 & 88 & 73\\
(202)Company2::addEmployee & 780 & 1482 & 538\\
(172)EmployeeNode & 114 & 73 & 194\\
(188)Employee::print & 28 & 46 & 50\\
(204)Company2::deleteEmployee & 535 & 440 & 1273\\
(174)Company2 & 139 & 147 & 209\\
(190)Employee::getSocSecurity & 49 & 40 & 61\\
(206)Company2::print & 473 & 441 & 288\\
\hline
\end{tabular}
\end{center}
\caption{Review Time}
\end{table}


\begin{table}[hb]
\begin{center}
\begin{tabular}{|l|l|l|l|l|}
\hline
Source & jgeis & bacani & toddm & OK \\
\hline
Employee::Employee &  & 214 (=1) &  & \\
EmployeeNode::EmployeeNode &  &  &  & \\
Employee::\~Employee & 222 (=1) & 218 (=1) &  & \\
EmployeeNode::\~EmployeeNode &  &  &  & \\
Employee::setName &  & 224,226 (=2) & 230,238 (=2) & 226=230\\
Company2::Company2 &  &  &  & \\
Employee::setSocSecurity & 318 (=1) & 240 (=1) & 254 (=1) & 318=254,240\\
Company2::\~Company2 & 312 (=1) &  &  & \\
Employee::setAge & 246 (=1) & 248 (=1) & 262 (=1) & 246=248=262\\
Company2::findEmployee & 276 (=1) & 300 (=1) & 294 (=1) & 300=294\\
Employee &  &  &  & \\
Employee::setNumDependents &  &  &  & \\
Company2::addEmployee & 284 (=1) & 268,280 (=2) & 324 (=1) & 284,268,324\\
EmployeeNode &  &  &  & \\
Employee::print &  &  &  & \\
Company2::deleteEmployee & 298 (=1) & 260 (=1) & 310 (=1) & 298=310\\
Company2 &  &  &  & \\
Employee::getSocSecurity &  &  &  & \\
Company2::print & 306 (=1) & 286 (=1) &  & 306,286\\
\hline
\end{tabular}
\caption{Source node v.s Issue node}
\end{center}
\end{table}

%\end{document}
