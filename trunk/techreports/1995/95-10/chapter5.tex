%%%%%%%%%%%%%%%%%%%%%%%%%%%%%% -*- Mode: Latex -*- %%%%%%%%%%%%%%%%%%%%%%%%%%%%
%% chapter5.tex -- 
%% Author          : Philip Johnson
%% Created On      : Tue Mar 14 10:39:44 1995
%% Last Modified By: Philip Johnson
%% Last Modified On: Sat Sep  2 10:49:35 1995
%% Status          : Unknown
%% RCS: $Id$
%%%%%%%%%%%%%%%%%%%%%%%%%%%%%%%%%%%%%%%%%%%%%%%%%%%%%%%%%%%%%%%%%%%%%%%%%%%%%%%
%%   Copyright (C) 1995 University of Hawaii
%%%%%%%%%%%%%%%%%%%%%%%%%%%%%%%%%%%%%%%%%%%%%%%%%%%%%%%%%%%%%%%%%%%%%%%%%%%%%%%
%% 

\chapter{Toward advanced collaborative systems}

The last three chapters introduced you to the major facilities in Egret's
Server, Type, and Generic Interface subsystems.  These systems constitute
the core of Egret, and if you have successfully completed the associated
exercises, then you have gained proficiency with the core set of
Egret-based tools for building advanced collaborative systems. If
Egret was a wood shop, then you would now be a competant user of
the hammer, saw, lathe, and drill.

The preceding three chapters did not provide either a complete introduction
to all of Egret's subsystems, nor a comprehensive introduction to all of
the features in the three subsystems of interest.  Egret has three other
important subsystems: Utilities, Applets, and Metrics, and many additional
features within the previously covered systems.  However, the services
within these systems should be easily exploited now that the fundamental
paradigms for communication and coordination within Egret have been
introduced.  Continuing the analogy, this book expects that an introduction
to the slot-head screwdriver will allow you to use the Philip's head as
well, and that the manufacturer's directions supplied with the power sander
should now suffice for you to use it safely and usefully.

However, there's a long way between knowing how to operate a saw and
knowing how to use it to build a nice set of kitchen cabinets.  This book
provides you with the former, but makes only small, initial steps toward
the latter.  This is intentional.

Building a successful advanced collaborative system in Egret ultimately
requires analyzing the particular domain of group work of interest, gaining
an understanding of the current and potential roles and participants and
their current potential interactions, and then iteratively introducing
collaborative technology into the environment, each time assessing the
impact and adjusting the technology appropriately.  These skills go
far beyond the scope of this book.  With sufficient time, interest, and
energy, perhaps a companion volume to this book will eventually 
appear that addresses exactly these topics in the context of Egret. 

Until then, however, you are not left without any guidance in these 
issues.  Egret's previously developed applications---CLARE, CSRS, 
AEN, Shemacs, URN, and Flashmail---together provide a wealth of 
information by example on advanced collaborative system building using
Egret.  By studying their associated research publications, design
documents, source code, and behavior, you can leverage much from
our previous experience.  The rest is up to you to discover on your
own.  Please let us know what you find out. We're very curious.







