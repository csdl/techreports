%%%%%%%%%%%%%%%%%%%%%%%%%%%%%% -*- Mode: Latex -*- %%%%%%%%%%%%%%%%%%%%%%%%%%%%
%% chapter4.tex -- 
%% Author          : Philip Johnson
%% Created On      : Tue Mar 14 08:58:07 1995
%% Last Modified By: Philip Johnson
%% Last Modified On: Mon Sep 25 16:34:07 1995
%% Status          : Unknown
%% RCS: $Id$
%%%%%%%%%%%%%%%%%%%%%%%%%%%%%%%%%%%%%%%%%%%%%%%%%%%%%%%%%%%%%%%%%%%%%%%%%%%%%%%
%%   Copyright (C) 1995 University of Hawaii
%%%%%%%%%%%%%%%%%%%%%%%%%%%%%%%%%%%%%%%%%%%%%%%%%%%%%%%%%%%%%%%%%%%%%%%%%%%%%%%
%% 

\setcounter{chapter}{3}   %The \chapter command does a ++counter.
\chapter{The Generic Interface subsystem: Rendering type-level objects}

The Generic Interface subsystem provides the third layer of functionality
in Egret.  At the core is the Server subsystem, which implements basic
communication, storage, and coordination services.  However, the Server
subsystem provides only a very basic notion of nodes and links.  The Type
subsystem rectifies this shortcoming by providing a much more sophisticated
data model involving node, link, and field schemas and their associated
instances.  Using type-level constructs, one can define the internal
structure of nodes, partition the set of node instances in
application-specific ways, cache node contents locally, and evolve the
internal structure of nodes over time.

With the definition of the type subsystem, however, the need for user
interface facilities becomes critical.  Unlike s*node instances, which
store data as a simple string and thus create little obstacle for interface
designers, t*node instances are stored in a non-trivial implementation
format that leaves their contents difficult to interpret and impossible to
reliably edit in their raw form.

The Generic Interface subsystem provides building blocks for the generation
of XEmacs user interfaces to type-level nodes.  These building blocks---the
gi*etml, gi*nbuff, and gi*rbuff classes---were designed to support both
efficient retrieval and manipulation of type-level node contents, and
generality in the way the contents of nodes would be rendered before
presentation to the user.

The first step in learning how to use the Generic Interface subsystem is to
read the overview material in the Design Reference Manual.  After that, you
are read to being working through the material in the remainder of this
chapter.

\include{/group/csdl/techreports/95-10/Sections/p*address-interface}








