\subsection*{Application Design in Egret 3.0}
\index{Application Design Hints}

Designing an application using Egret 3.0 involves many activities, from 
data and process design to interface usability.  This section focusses on
one small part of application design---that concerned with the distribution
and representation of information within Egret. 

One part of information representation in Egret 3.0 is to decide whether or
not one or more HBS servers will be used.  If more than one HBS server is
used, then you are designing a ``distributed database'' system: something
that is definitely exciting and leading edge, but also something that we
don't have much experience with in Egret.  Let us know what you learn. In
most cases, you probably want to design an initial prototype version of
your system using a single server, and then, once you have the other kinks
out of your system, you can distribute the data.

Regardless of whether you go with a single server or multiple servers, you
will then need to decide how to represent internal state information in
your system.  This means choosing one or more snodes or snode derivatives
(stable, pnode, ptable, or gtable) within which to store your information.
Here are some heuristics to help distinguish between these alternatives:

\begin{itemize}
\item Gtables are indicated when (a) the information can be stored in
  table form, (b) the information is generally useful to all users, (c) the
  information is required frequently and/or quickly during a session, and
  (d) explicit ``freezing'' of the state of this table through locks is not
  required. (The state of the table could be {\em implicitly} frozen if
  locking is supported on all the nodes from which the gtable's information
  is derived.)
  
  The set of application end-user nodes and links is a typical example of
  information that should be stored as a gtable in most application
  domains.
  
  
\item Pnodes or ptables are indicated for storing information that, in
  aggregate, is not generally useful to all users. Instead, all users need
  to maintain their own versions of the same type of information.  Pnodes
  are not automatically cached at the local client during each session,
  while ptables are.  To create a representation of a ptable that isn't
  cached locally, use a pnode representation and manipulate it using stable
  operations.

  Metric information is a typical example of information that should be
  stored in pnodes or ptables in most application domains. 
  
  
\item Snodes and stables are indicated whenever gtables and
  pnodes/ptables are not applicable.
  
  Application-specific version control state information (which would
  necessitate strict sequential shared access) is a typical example of
  information that would be best represented as an snode or stable in
  most application domains.

\end{itemize}

Once you have decided upon the appropriate representation format for the
information, the next step is to implement it.  The next section
contains a typical implementation scenario that illustrates the essential
aspects of snode and snode derivative definition and use.  

\subsubsection*{A Pnode scenario: Metric data implementation}

This scenario illustrates how pnodes might be used to store metric data
accumulated during a session by an application.  Metric data is generated
by each user as they interact with the system, and is not normally
interesting to other users during a session, except during some later data
analysis phase. Thus, gtables would incur useless overhead.   This example
is for illustrative purposes only: the mt*metrics class provides a more
sophisticated implementation of metric collection.

Assume that the metric data is cached locally in a buffer, and must be
written out by the application at the end of each session. A pnode that
accumulates all of the metric data generated by this user is a reasonable
representation choice, since one can append the metric data to this node at
the end of each session, and thus keep the number of nodes containing
metric information down to the number of total users of the application.  
Since metric data is generated sequentially, it is reasonable to use a 
pnode, rather than ptable representation.  

Furthermore, let us assume that the application developer writes code to
generate metric information in a buffer called *metric-data* during the
session. 
Storage of this data involves the following steps:

\begin{enumerate}
\item {\em Initialization.} The first step is to create an instance of a
  pnode through the s*pnode*make form.  This form requires an db-ID which
  specifies the database server process where these pnodes (one for each
  s*user in this database) will be stored.  The natural way to do this is
  by attaching a function to the s*db@initialize-hooks variable, such
  as in the following:

  \small\begin{verbatim}
(defun app*metrics*initialize-metrics-pnode (db-ID)
  (s*pnode*make db-ID 'metric-data))

(pushnew 'app*metrics*initialize-metrics-pnode s*db@initialize-hooks)
  \end{verbatim}\normalsize
    
  The above code, when loaded, will lead to a new pnode called
  metric-data being defined when s*db*make is called on an empty database.
  
  Definition of this new pnode, in turn, means that a new s*snode will be
  automatically created for each current s*user in the database, as well as
  whenever a new s*user gets instantiated in the future. (Thus, you don't
  have to worry about the order in which you define pnodes and instantiate
  users.)
  
\item {\em Downloading.} Assume, for pedagogical purposes, that the
  application needs some bit of information from the user's metrics pnode
  whenever the user connects to an hbs, and that the function
  app*metrics*extract-connect-info returns this information when the
  current buffer is bound to the contents of the pnode.  The following
  shows how such connect-time processing would be accomplished.

  \small\begin{verbatim}
(defun app*metrics*connect-hook (db-ID user-ID)
  (s*pnode*read-with-buffer db-ID 'metric-data 
    (app*metrics*extract-connect-info)))

(pushnew 'app*metrics*connect-hook s*user@connect-hooks)
  \end{verbatim}\normalsize
  
  Note that the user-ID passed to app*metrics*connect-hook is not needed.
  By using s*pnode*with-read-buffer rather than s*pnode*read, the
  contents of the pnode do not need to be converted to a string.  The use
  of read-with-buffer and write-with-buffer operations rather than read
  and write are preferred whenever the node contents are not actually
  strings, or when only a portion of the string is required from the
  calling program.  This is because read-with-buffer and
  write-with-buffer do not create this alternative string representation.

  It is important to push this function onto the s*user@connect-hooks
  rather than onto the s*db@connect-hooks variable, since only users can
  use the ``default'' pnode operations that operate on the current client's
  pnode. (Both users and agents can use the s*pnode ``other-user''
  operations, where the user is specified explicitly.)

  
\item {\em Uploading.} Finally, at the end of the session, the new metric
  data must be saved out to each user's pnode.  This is accomplished in an
  analogous fashion:

  \small\begin{verbatim}
(defun app*metrics*disconnect-hook (db-ID user-ID)
  (let ((session-data (save-excursion
                        (set-buffer *metric-data*)
                        (buffer-string))))
    (s*pnode*append db-ID 'metric-data session-data)))

(pushnew 'app*metrics*disconnect-hook s*user@disconnect-hooks)
  \end{verbatim}\normalsize

  Here, there is no advantage to using append-with-buffer since we must
  create a string anyway to copy the contents of one buffer to another. 
\end{enumerate}


This provides a basic illustration of pnode definition and use.  

Ptables are simply pnodes which are always read and written using the
ptable operations s*ptable*get, s*ptable*set, and so forth.  They are
defined using a call to s*pnode*make. Unlike pnodes, however, there is no
need to explicitly download and upload them from the HBS as is shown in the
example above.  Ptables are automatically cached locally as part of
connect-time processing, and automatically downloaded as part of disconnect
processing.

Gtables are defined by a call to s*gtable*make, and in most cases are
created by pushing a call onto s*db*initialize-hooks, as shown in the
pnode example above.  Like ptables, gtables are automatically downloaded
during connect time, so that there is no need to define a connect-hook
function to access them.  Unlike ptables, however, the data stored in a
gtable is shared and visible to all users, and updates (through
s*gtable*set) made by one user are propogated immediately to all other
currently connected users.  Thus, there is also no need to provide a
disconnect-hook function to download gtable data. 

While it is typical to create snodes and snode derivatives at the time when
an HBS is first initialized, there is no requirement that this be so.
Snodes and their derivatives are designed to allow instantiation at any
time during the application's lifecycle.  For example, if a gtable is
created after the clients have begun using the system, then this gtable
will become available to them as soon as they reconnect to the system.
(Making the gtable available without reconnection is possible but not
currently implemented.) 

The next section illustrates message-passing: the paradigm for inter-client
communication in Egret 3.0. 

\subsubsection*{Message Passing Scenario: Watch Me Work}
\index{Message Passing}

Applications frequently need to pass information from one client to another
in real-time.  Gtables provide one facility for real-time communication,
but it only supports the communication of updates to the contents of a
shared table.  Egret provides a second facility, called message-passing, to
support much more general forms of communication between currently
connected clients.  Unlike gtables, message-passing is a real-time
facility: if a user is not currently connected when a message is sent, then
the user will never know about the message.

To illustrate the design and use of message passing, consider an application
scenario where, for some irritating reason, it is desired to print out an
informative message in the minibuffer every time a user (a) connects to the HBS,
and (b) creates a new node. This is accomplished as follows.

First, to print out the message indicating that a user has connected, the
application must define a function such as the following app*watch*connect,
and pass a call to it as a message form in the application's
domain-specific connect function:

\small\begin{verbatim}
(defun app*watch*connect (db-ID user-name entity-ID)
  "This message function prints out the name of the user that just connected."
  (message "User %s just connected.") user-name)

(defun app*db*connect ()
  "This is the domain-specific function for connecting to an HBS."

  ;; domain-specific code goes here to bind values of host, socket, and user.
  (s*db*make host socket user nil '((app*watch*connect))))

\end{verbatim}\normalsize

Thus, when a client calls app*database*connect, a call to s*db*make will
eventually result, which is passed a message form list. This list contains
a single message-form, which in turn consists of the message-function name
app*watch*connect.  The connect operation will generate a connect event,
which will be received and processed by all currently connected clients
(except for the one that just connected).  Event processing will lead to
invocation of the function app*watch*connect with the three mandatory
arguments passed to all message functions: the db-ID where the event
originated, the user-name of the client responsible for triggering the
event, and the node-ID or link-ID (if any) related to the event. Thus, when
the user ``jones'' connects, the message ``User jones just connected.''
will be displayed in all other currently connected users' minibuffers.

Next, let's assume that the application requires a message to be printed in
all other clients' minibuffers each time a user creates a node, and
furthermore, the creating user is interactively prompted for an explanatory
comment to accompany this message.  Here's how that could be accomplished:

\small\begin{verbatim}
(defun app*watch*create (db-ID user-name entity-ID comment)
  "This message function prints out the user creating a node and his comment."
  (message "Node %s created by %s, who comments: %s" entity-ID user-name comment))

(defun app*node*make () 
  "This is the domain-specific function for connecting to an HBS."
  (let ((comment-string (read-from-minibuffer "Comment: ")))
    ;; domain-specific stuff goes here.
    (t*node-schema*instantiate db-ID node-schema-ID node-name 
                               (list (list 'app*watch*create comment-string)))))
\end{verbatim}\normalsize

Thus, if the user creating a node typed in ``Hello World'' to the minibuffer
prompt, the message form {\tt ((app*watch*create "Hello World"))} would be sent
along with the create node event to all other clients. When it was received,
this function would be called with the db-ID, the user-name, the node-ID of the
newly created node, and the comment string that was provided by the user.

In reality, message-forms are usually constructed with backquote.  The
preceeding message-form could have also been constructed with: 

\small\begin{verbatim}
(` ((app*watch*create (, comment-string))))
\end{verbatim}\normalsize
