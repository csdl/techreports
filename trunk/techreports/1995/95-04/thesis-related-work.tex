%%%%%%%%%%%%%%%%%%%%%%%%%%%%%% -*- Mode: Latex -*- %%%%%%%%%%%%%%%%%%%%%%%%%%%%
%% thesis-related-work.tex -- 
%% Author          : Carleton Moore
%% Created On      : Tue Jan 10 12:03:05 1995
%% Last Modified By: Carleton Moore
%% Last Modified On: Thu Jul  6 17:36:08 1995
%% Status          : Unknown
%% RCS: $Id: thesis-related-work.tex,v 1.10 1995/07/07 03:36:34 cmoore Exp $
%%%%%%%%%%%%%%%%%%%%%%%%%%%%%%%%%%%%%%%%%%%%%%%%%%%%%%%%%%%%%%%%%%%%%%%%%%%%%%%
%%   Copyright (C) 1995 University of Hawaii
%%%%%%%%%%%%%%%%%%%%%%%%%%%%%%%%%%%%%%%%%%%%%%%%%%%%%%%%%%%%%%%%%%%%%%%%%%%%%%%
%% 

%for review purposes
%\ls{1}

\newpage

\chapter{Related Work}
\label{sec:related-work}

This chapter surveys major existing hypertext or collaborative editors,
with a comparison of AEN to these systems.  Section \ref{sec:collab-sys}
discusses different collaborative writing systems.  Section
\ref{sec:hypermedia} compares several hypermedia systems with AEN.

AEN's set of features were designed to explore and support strong
collaboration. Section \ref{sec:mechanisms} provides a detailed description
of AEN's features.  AEN provides a typed hypertext document, dynamic user
defined table of contents, real-time communications with other users of the
system, real-time knowledge of what the other users of the system are doing,
and access control for each node.  Other collaborative writing systems or
hypermedia systems provide different combinations of features.

\section{Collaborative Writing Systems}
\label{sec:collab-sys}

These systems support the process of collaborative writing.  Not all of the
systems in this section are pure collaborative writing system, but they all
support the process of collaborative writing.  I have divided the
collaborative writing systems into three categories, weakly collaborative,
additive, and synchronous.

The weakly collaborative systems are MILO \cite{Jones93} and NSF Express
\cite{Greenberg91}.  They appear to only support the combination of
different types of documents from different authors.  They do not appear to
support co-editing of other authors' work while they are still working on
it.

The additive systems are CLARE \cite{csdl-93-21}, ConversationBuilder
\cite{Kaplan92}, ForComment \cite{Ellis91,Opper88}, InterNote
\cite{Malcolm91}, gIBIS \cite{Conklin88}, and MarkUp! \cite{Allen93}.
These systems allow authors to collaboratively add to a document.  Three
different systems, CLARE, ConversationBuilder and gIBIS will be discussed
briefly.  They provide a good cross section of these types of tools.

CLARE is a computer-based collaborative learning environment, and
includes a sophisticated process and data model used to learn about
scientific text.  Users of CLARE evolve a group knowledge base by
comparing and debating their interpretations of a scientific text.
These comparisons and debates are additive.  The original statements
are not edited to reflect the new consensus.  In AEN, users edit the
original document or arguments as part of building consensus.  Unlike
AEN, CLARE does not provide synchronous communication between the
participants and there is no sense of presence in the hypertext
document.

ConversationBuilder is a collaborative work tool that supports more than
collaborative writing.  It also supports a strong, flexible process
model of obligations that the users must complete before they can move
on to another task in ConversationBuilder.  AEN was not designed to
have a strong process model, since I do not wish to inhibit any
potential styles of collaborative authoring and learning.

gIBIS is a design rationale capturing system.  The process of collecting
and storing design rationale is essentially additive in nature.  New
issues, positions and arguments are added to the existing database to keep
track of the design process.  The data stored in gIBIS is used for
reference and is not intended to be edited at a later date.  Unlike gIBIS,
AEN provides explicit facilities, such as unread nodes, access control, and
locking to support changes to existing information within the system.

The last category of collaborative writing systems, is synchronous.  The
following are examples of this category of system: Aspects
\cite{Allen93,Baecker93}, Augment \cite{Engelbart84}, CES
\cite{Ellis91,Greif87}, CoAUTHOR \cite{Bowers91}, CoEd \cite{Ellis89},
COOPerator \cite{Michels95}, EHTS: Emacs HyperText System \cite{Wiil92},
GroupWriter \cite{Malcolm91,Malcolm93}, PREP \cite{Neuwirth90,Neuwirth92},
Quilt \cite{Fish88}, SASE and SASSE \cite{Baecker93}, SEPIA \cite{Haake92},
and ShrEdit \cite{Cogn92}.  All of these systems appear to support
synchronous editing of documents by multiple authors.  In general, they do
not support different modes of access control.  Many do have a presence for
the authors of the document.  I will discuss five systems in more detail.

PREP is a collaborative writing tool that defines roles for the
collaborators.  It allows different access levels for the nodes.  It also
supports annotating the text.  It does not appear to support real time
communications or a sense of presence.

SASE and SASSE are collaborative writing systems.  SASSE is an extension
of SASE.  SASE was designed to support highly interactive synchronous
collaborative writing.  SASSE allows both synchronous and asynchronous
collaborative writing.  They both provide awareness of other authors
and their activities.  SASSE has an annotation mechanism.  Both SASE
and SASSE require external communication facilities to allow authors to
coordinate their activities.  Unlike AEN, SASE and SASSE do not appear
to have any type of access control mechanism to control to support
temporary subgroups.

SEPIA is a hypertext authoring system that supports collaborative
authoring, but does not appear to provide restricted read or write
access in order to support temporary subgroups.  SEPIA does have a
strong sense of physical presence like AEN.  Each user has a location
in the hyperdocument visible to the other users.

ShrEdit is a collaborative editor.  ShrEdit allows the users to see where the
other authors are currently editing.  Authors lock sections of text.  A
window informs the user of who else is currently editing the document.
ShrEdit does not have any access control mechanism since it works on
traditional documents.  There does not appear to be any mechanism for
real time communications or commenting of other authors' work.


\section{Hypermedia Systems}
\label{sec:hypermedia}

This section reviews several other hypermedia systems that are not directly
oriented toward collaborative writing.  These systems include: CoMEdiA
\cite{Santos92a,Santos92b,Santos93a,Santos93b}, Intermedia
\cite{Conklin87},  NoteCards \cite{Halasz87},  The
Virtual Notebook System \cite{Shipman89}, and the World Wide Web
\cite{Berners-Lee94}.

CoMEdiA is collaborative multimedia editing tool.  It provides real
time text communications, voice and video communications and allows authors to
edit text, graphics, and video.  CoMEdiA supports several roles and private
and public comments.  It does not appear to support the creation of
subgroups. 

Intermedia is one of the largest and oldest hypermedia systems
  designed to support learning.  It supports creation of hypermedia documents
  that can be commented on by the instructor or other students.  Intermedia
  appears to have a basic access control mechanism that can restrict access to
  the entire document.  Intermedia adds the concept of {\em webs}.  Webs
  control which links are visible to the reader.  This allows different users
  to have different views of the hypermedia document.  There does not appear
  to be any sense of physical presence.

NoteCards is a general hypermedia environment.  It
  provides the user with typed cards and links.  NoteCards does not
  appear to support a sense of presence or access control.  It does
  provide a graphical browser for navigation.  NoteCards has a generic
  query system to return cards that match the user-supplied
  specification.  AEN has the less generic Node List feature.  NoteCards
  allows the user to create their own type of cards.


The Virtual Notebook System is a shared on-line
  notebook designed to facilitate information acquisition, sharing and
  management in collaborative work.  Authors can concurrently access the
  same notebook, but there does not appear to be any sense of presence of
  the other users.  The view presented to the user is individually
  tailored.  There does not appear to be any way to communicate with the
  other users.  It is primarily designed for information sharing and not
  creation.  There does not appear to be any access control --- all data is
  available to the other users.

WWW is designed for global, internet-based storage
  and dissemination of hypertext, multimedia data.  Collaboration in the WWW is
  an all-or-nothing proposition: data is either available to everyone or no
  one. There is no access control, no locking, and virtually no ``state''
  maintained by the system.  These design decisions are important in allowing
  the WWW to accommodate a user population on the order of millions.  AEN, on
  the other hand, is designed to accommodate collaboration among small groups
  (less than forty).  This radical difference in scale allows AEN to
  provide many facilities not conceivable in the WWW (such as snoopy, or unread
  nodes).  However, these same facilities impose a ``ceiling'' on the maximum
  group size supported by AEN.

\section{Summary}

One key feature that distinguishes AEN from the other systems is its
specific intent to support strong collaboration and to be a tool to help
investigate collaborative construction of hypertext documents.  AEN's set of
features is guided by the desire to support and explore collaboration.




%\item HyperCard is a Hypertext system that does not support collaboration.


%\item{GROVE \cite{Ellis88}:}



%\item{ENFI \cite{Bruce93}:}


%\item{WE \cite{Smith87}:}


%Tables \ref{table:comparison1} and \ref{table:comparison2} are based upon
%Baecker, Nastos, Posner and Mawby's table\cite{Baecker93}.  I've extended
%their table to include AEN.\\
% \small
%\begin{table}
%\begin{tabular}{|c|c|c|c|c|}
%\hline
%Requirements&Aspects&GROVE&PREP&Quilt\\
%\hline
%\hline
%{\bf Individual Writing}&&&&\\
%Basic word-processing&++&--&++&--\\
%Seamlessness with other media&++&--&++&++\\
%\hline
%{\bf Collaborative Writing}&&&&\\
%Preserve identities&--&++&+&++\\
%Enhance communication&+&--&--&--\\
%{\em Enhance collaborator awareness}&&&&\\
%Focused collaboration&++&++&--&--\\
%Peripheral awareness&--&+&--&--\\
%Annotations&--&++&++&++\\
%Undo&--& &+&+\\
%Session control&+&++&--&++\\
%\hline
%{\bf Roles}&&&&\\
%Explicit roles&--&+&++&++\\
%\hline
%{\bf Activities}&&&&\\
%{\em Variety of activities}&&&&\\
%Brainstorming&--&++&++&+\\
%Researching&--&--&--&++\\
%Planning (outline)&--&++&+&+\\
%Planning (process)&--&--&+&--\\
%Writing&++&--&++&+\\
%Editing&++&--&++&+\\
%Reviewing&--&--&++&+\\
%Transitions between activities&+&--&++&++\\
%\hline
%{\bf Document Control Methods}&&&&\\
%Several access methods&--&++&++&++\\
%Separate document segments&++&--&+&++\\
%Version and change control&--&--&+&--\\
%\hline
%{\bf Writing Strategies}&&&&\\
%One or several writers&++&++&++&++\\
%Synchronous Writing&++&++&--&--\\
%Asynchronous writing&+&+&++&++\\
%\hline
%\end{tabular} \\
%{\bf Notation:}  ++ system provides good support + system can handle --
%system does not support
%\label{table:comparison1}
%\caption{Design requirements and comparison of collaborative writing
%tools.}
%\end{table}
%\normalsize
%
%
% \small
%\begin{table}
%\begin{tabular}{|c|c|c|c|c|}
%\hline
%Requirements&ShrEdit&SASE&SASSE&AEN\\
%\hline
%\hline
%{\bf Individual Writing}&&&&\\
%Basic word-processing&++&+&++&++\\
%Seamlessness with other media&+&+&++&--\\
%\hline
%{\bf Collaborative Writing}&&&&\\
%Preserve identities&++&++&++&++\\
%Enhance communication&+&--&++&+\\
%{\em Enhance collaborator awareness}&&&&\\
%Focused collaboration&+&++&++&++\\
%Peripheral awareness&--&+&++&++\\
%Annotations&--&--&+&++\\
%Undo&--&--&--&--\\
%Session control&+&--&++&++\\
%\hline
%{\bf Roles}&&&&\\
%Explicit roles&--&--&--&--\\
%\hline
%{\bf Activities}&&&&\\
%{\em Variety of activities}&&&&\\
%Brainstorming&++&+&++&+\\
%Researching&--&--&--&--\\
%Planning (outline)&--&--&++&+\\
%Planning (process)&--&--&--&--\\
%Writing&++&++&++&++\\
%Editing&++&++&++&++\\
%Reviewing&--&--&++&++\\
%Transitions between activities&+&--&++&++\\
%\hline
%{\bf Document Control Methods}&&&&\\
%Several access methods&--&--&--&++\\
%Separate document segments&--&--&--&++\\
%Version and change control&--&--&++&--\\
%\hline
%{\bf Writing Strategies}&&&&\\
%One or several writers&++&++&++&++\\
%Synchronous Writing&++&++&++&++\\
%Asynchronous writing&++&--&++&++\\
%\hline
%\end{tabular} \\
%{\bf Notation:}  ++ system provides good support + system can handle --
%system does not support
%\label{table:comparison2}
%\caption{Design requirements and comparison of collaborative writing
%tools (part 2).}
%\end{table}
%\normalsize

