%%%%%%%%%%%%%%%%%%%%%%%%%%%%%% -*- Mode: Latex -*- %%%%%%%%%%%%%%%%%%%%%%%%%%%%
%% thesis-experiment.tex -- 
%% Author          : Carleton Moore
%% Created On      : Tue Jan 10 12:01:37 1995
%% Last Modified By: Carleton Moore
%% Last Modified On: Thu Jul  6 18:23:24 1995
%% Status          : Unknown
%% RCS: $Id: thesis-experiment.tex,v 1.8 1995/07/07 03:43:21 cmoore Exp cmoore $
%%%%%%%%%%%%%%%%%%%%%%%%%%%%%%%%%%%%%%%%%%%%%%%%%%%%%%%%%%%%%%%%%%%%%%%%%%%%%%%
%%   Copyright (C) 1995 University of Hawaii
%%%%%%%%%%%%%%%%%%%%%%%%%%%%%%%%%%%%%%%%%%%%%%%%%%%%%%%%%%%%%%%%%%%%%%%%%%%%%%% 
%% 

%for review purposes
%\ls{1}
\newpage

\chapter{A Case Study of AEN}
\label{sec:evaluation}

This chapter discusses the Spring 1995 case study of AEN.  Section
\ref{sec:casestudy} presents the questions the case study is trying to
answer.  Section \ref{sec:method} discusses the method used in the case
study.  Section \ref{sec:datacollection} introduces the data that will be
collected in the study.  Section \ref{sec:dataanalysis} describes the ways
the data will be analyzed to determine if AEN supports strong
collaboration.  The last section concludes the chapter with a discussion of
some of the limitations of this study.

\section{Spring 1995 Case Study}
\label{sec:casestudy}

After implementing and using AEN during Fall, 1994, I had two major
questions: (1) Does AEN support strong collaboration? and (2) What
computational support is needed for strong collaboration?

One approach to answering these question would have been a controlled,
experimental study where the interactions, and authoring styles of a group
using AEN is compared to that of a group not using AEN.  However,
methodological and resource problems prevented that approach.  Instead, I
decided to first operationalize the definition of strong collaboration as a
set of measurable collaborative behaviors, then test to see to what degree
a group of AEN users exhibit those behaviors.  

\section{Operationalized Definition of Strong Collaboration}
\label{sec:operationalized-definition}
 
The operationalized definition of strong collaboration consists of four
classes of observable collaborative behaviors.  For each behavior, I have
defined one or more metrics.

\begin{enumerate}
\item{\em Members read each other's nodes.}  Any form of collaboration
  requires that the members of the group know what other members have
  contributed. 
  
  % Without reading the other nodes members have no way of
  %  contributing to the nodes.
  
  {\bf Metric:} {\em Readers per node (RPN)}.  This metric calculates the
  average number of readers per node for a group. For example, for a
  group of five, the lowest RPN value is one, indicating that no member
  of the group read any nodes other than those he created.  The highest
  possible RPN value for this group of five is five, indicating that every member
  read every node.
  
  
  
%\item{\em Members edit nodes that were also edited by others.}  This
%  is the most direct way that members can collaborate on a node.  One author
%  creates a node, then another author adds to it.
%  
%  {\bf Metric:} {\em Member Co-editing (MCE)}.  This metric calculates
%  the average number of members that each member co-edits with. For a
%  group of five, the lowest MCE value is one, indicating that no member
%  of the group edited nodes that any other member edited.  The highest
%  possible MCE value for this group is five, indicating that every member
%  edited every node with every other member.

\item{\em Document nodes are edited by more than one person.}  This
  component of the operationalized definition assesses the degree to
  which multiple authorship occurs.  Only document nodes are included in
  this assessment, since other node types (comment, quicky quiz, and
  quicky quiz answers) are by definition singly authored.  

  {\bf Metric:} {\em Editors per node (EPN)}.  This metric calculates the
  average number of editors per node.  The lowest EPN value possible, for
  a group of five, is one, indicating that each node was only edited by
  one member.  The highest possible EPN value for this group is five,
  indicating that every node was edited by all five group members.

\item{\em Members create feedback nodes.}  This is another way for group
  members to collaborate.  They can create feedback on the contents of a
  node.

  {\bf Metric:} {\em Feedback Node Creation (FNC)}.  This metric
  calculates the percentage of document nodes that have been commented
  on.  The lowest possible FNC value is 0\%, indicating that no document
  nodes were commented on.  The highest FNC value is 100\%, indicating that
  all document nodes were commented on.


\item{\em Members manipulate access control to publish/protect documents
  under development.}  By changing the access control members are able to
  dynamically control the process of collaboration.  Changing the access
  control from the default allows collaboration on a node.  Changing the
  access control for a node more than once indicates that the mode of
  collaboration changed for that node.

  {\bf Metric:} {\em Non-default Access Control (NAC)}.  This metric
  calculates the percentage of nodes that have had their access control
  changed from the default.  The lowest NAC value possible is 0\%, indicating
  that no nodes had their access control changed from the default.
  This also indicates that there was no interaction among the members
  since the default access control is no access for other members.  The
  highest possible NAC value is 100\%, indicating that every node has had its
  access control modified.

  {\bf Metric:} {\em Evolving Access Control (EAC)}.  This metric
  calculates the percentage of document nodes that have had their access
  control changed more than once. This metric measures the degree to
  which the groups' collaboration changed. The lowest EAC value possible
  is 0\%, indicating that no nodes had their access control changed more
  than once.  The highest possible EAC value is 100\%, indicating that
  all nodes had their access control changed more than once.

\end{enumerate}


These metrics characterize many aspects of collaborative behaviors.
Specifically, a group that exhibits uniformly high values for each of these
metrics can be said to be exhibiting ``pure'' strong collaboration.  Of
course, intermediate levels of these metrics still indicate at least a
partial presence of strong collaboration.  In order to evaluate
AEN's set of tools, tool use will be monitored and a post study survey will
ask the participants their impressions of the tools.

\section{Method}
\label{sec:method}

The basic method for this research was to administer a pre-test questionnaire,
collect metrics data generated, collect bug reports and suggestions, and
administer a post-test questionnaire.  The study lasted from March 3, 1995 to
May 5, 1995.  The data was analyzed both during and at the end of the
experiment.

The Spring 1995 ICS 413, an undergraduate senior level class, was studying
software engineering.  The class was divided into five teams.  Each team
had five members.  The teams were developing a software product to be used
by an organization outside of the university.  This involved developing and
maintaining several documents, including a requirements document, design
specification, project proposal, and project schedule.  Three of the five
teams participated in this study.  I will refer to them as A, B and C.
Teams A and C consisted of all seniors, while team B had two graduate
students and three seniors.  These three teams used AEN to develop their
requirements document.  During case study, I monitored the teams' use of
AEN to create their requirements documents.

The structure of this document was roughly defined for all teams.  The
contents of the individual team's documents were different since they were
developing different software systems.

\section{Data Collection}
\label{sec:datacollection}

During this case study three types of data were collected: metrics on AEN
use, user questionnaires, and user feedback/bug reports.  The following
sections discuss each in turn.

\subsection{Metrics}
\label{sec:metrics-description}

AEN collects and stores metrics for each user.  All metrics entries have
the same format:  [$<$metrics identifier$>$ $<$encoded time string$>$
$<$metrics type$>$ $<$node-ID$>$ $<$additional information$>$].  The syntax
for AEN's metrics is shown in Figure \ref{fig:BNF}
\small
\begin{figure}
  \begin{verbatim}
<AEN metric> ::= [mt*event "encoded Egret time string" <metrics type>
                Egret node-ID <additional information>]

<metrics type> ::= create | write | <lock type> | <toggle-access type> | 
                   <set-access type> | make-TOC | <make node list type> |
                   start-snoopy | stop-snoopy | start-partyline | 
                   stop-partyline | send

<additional information> ::= type of node | user added to access control 
                             list | new access control list | 
                             recipients of partyline message


<lock type> ::= lock | lock-to-annotate 

<toggle-access type> ::= toggle-read-access | toggle-write-access |
                         toggle-annotation-access 

<set-access type> ::= set-read-access | set-write-access |
                      set-annotation-access

<make node list type> ::= make-node-list-unread | make-node-list-type |
                          make-node-list-owner 

\end{verbatim}
\caption{Syntax of AEN's Metrics}
\label{fig:BNF}
\end{figure}
\normalsize

 An example
of a single metrics entry is the following:

\small
\begin{verbatim}
[mt*event "52@;:_" create 12 (type document)]
\end{verbatim}
\normalsize

This metric indicates that on February 16, 1995 at 11:10:47, the user
created a document node with ID 12.

Metrics are collected on the following events:

\begin{itemize}

\item{\em Node creation.}  The create node metric stores the time, the
  node-ID and the type of node created (document and comment).

\item{\em Reading the contents of a node.} The read metric stores the time,
  the node-ID and type of the node.

\item{\em Changing the contents of a node.} The write metric stores the
  time, and the node-ID of the node.

\item{\em Locking of nodes.} The locking metric stores the time, the type
  of locking action (lock to write, lock to annotate) and the node-ID.

\item{\em Changing the access privileges for a node.} The access metric
  stores the time, the node-ID and the new access privilege list.

\item{\em Creating a Table of Contents.} Creating a TOC stores the time,
  the db-ID and if it is a local TOC the node-ID of the top of the TOC.

\item{\em Creating a Node list.} A metric is generated for the different
  types of node lists (documents, comments, quicky quizzes, quicky quiz
  answers).

\item{\em Creating an Unread Nodes list.} A metric is generated for the
  different types of node lists (documents, comments, quicky quizzes, quicky
  quiz answers).

\item{\em Starting and stopping Partyline and Snoopy.} Starting these
  applications stores the time and the db-ID.

\item{\em Sending a Partyline message.} Sending a Partyline message stores
  the time of the message and the person(s) the message is sent to.  The
  contents of the message is not saved.

\end{itemize}


\subsection{Questionnaires}

Two questionnaires were developed to gain information about the users and
their evaluation of the system.  The first questionnaire, administered at
the beginning of the study, requested information about the user's
background and knowledge of hypertext and computer supported collaborative
work.  The second questionnaire, administered at the end of the study,
requested information on the user's feeling on AEN and their views on the
different tools AEN provides. Appendices \ref{app:questionnaire1} and
\ref{app:questionnaire2} contain the two questionnaires.

\subsection{Feedback/Bug reports}

A final form of data collected during the study was on-line comments on the
system produced by users.  AEN has a Feedback menu that allows the users to
send suggestions and bug reports.  These suggestions and bug reports were
used to improve AEN and as a source of ideas for future enhancements.

\section{Data Analysis}
\label{sec:dataanalysis}

The primary purpose of the case study was to find out how authors
collaborate when using AEN.  My intent was to evaluate the utility of AEN
and provide a foundation for more complete comparative studies that might be
designed and conducted.  The analysis techniques were descriptive.

The data collected consisted of metrics data, user suggestions, bug reports
and user questionnaires.  The metrics data, AEN collected, was used to
calculate the five collaborative metrics for each group.  The number of bug
reports from the Fall 1994, semester was compared to the number of bug
reports from the Spring 1995, semester to measure the change in reliability
of AEN.  The metrics on tool use and user questionnaires assessed the
frequency of use and subjective value of different tools.

\section{Limitations}
\label{sec:limitations}
There are four significant limitations to this study; the learning curve of
the participants, workstation availability, face-to-face off-line
collaboration, and learning to collaborate vs. learning to collaborate
strongly. 

The first limitation, the learning curve of the participants, was
significant since AEN is a complex system.  AEN is built upon the XEmacs
editor, which is itself complex and non-trivial to learn.  The participants were
trained to use XEmacs as well as AEN.  However, at the end of the study one
student remarked,
\ls{1.0}
\begin{quote} 
{\em I'm sorry to say, it still frustrates me.  I just can't
get a grasp of it, so I basically just wrote my documents and never did
anything else.}
\end{quote}
\ls{1.5}
The training and on-line documentation attempted to alleviate this
limitation.

The second and third limitations are closely related.  The availability of
workstations for the participants was very limited.  However, since AEN
supports asynchronous collaboration, the number of workstations needed at
any one time was reduced.  The workstations used by the participants were
located in two rooms.  This physical feature of the study environment led
to the third limitation of the study, face-to-face collaboration off-line.
Often, when participants worked synchronously, they were in the same room
and could collaborate off-line.  I had no way to collect or measure their
off-line collaboration.  I could have recorded the participants off-line
conversations, but I did not restrict the times the participants used AEN.
This made it impossible to record all of the off-line collaboration.

The last limitation, learning to collaborate vs. learning to collaborate
strongly, was significant.  None of the groups had worked together before
the study.  Thus, the participants had to learn how to collaborate with
each other before they could learn to collaborate strongly.  One student
remarked, {\em ``the only problem was that we needed to have a set of
procedures on how to use AEN.''}  Training and suggestions about how to
collaborate attempted to alleviate this limitation.


