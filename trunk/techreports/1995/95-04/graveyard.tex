intro.tex 6/7
%
%The best way do describe strong collaboration is to give an example.  Four
%authors have been given the task to develop a four act play.  In the
%classic ``weakly'' collaborative method the authors would divide up the
%play by acts and assign an act to each author.  They would then write their
%acts and at the end combine the four acts to produce the play.  Each act
%has only one author and can be directly attributed to that author.  In a
%slightly more collaborative way they could each write one act and then pass
%that act on to another author for rewrite.  In this method each act has two
%authors that have ownership of the act.  There are several benefits of
%collaborating in this manner, each author only has to worry about one or
%two acts.  They do not need or even want to know the exact details of the
%other acts as long as the other acts follow the general story line.  This
%method of collaboration is very simple and requires very little external
%support.
%
%In the strong collaboration method each author would work on each act of
%the play.  The four acts of the play could initially be given to different
%authors, but when an author sees that they can contribute to an act they
%would make the suggestion or change to the existing act.  Each author would
%work on all four acts.  There would be no way to assign ownership of an act
%to a single author since all of the authors provided input to the act.  In
%this way, all of the authors participate in the building of the entire
%play.  Strong collaboration requires each of the authors to be aware of the
%current state of each of the acts.  If an author is working on act 1 and
%this effects act 3 then the first author should be able to change act 3 or
%at least tell the author who is working on act 3 of the changes.  An author
%can have an inspiration about another act and make the changes or
%suggestions to that other act.  Strong collaboration requires more
%interaction between the authors.  There is a lot of overhead with strong
%collaboration.  It normally occurs (though not always) in non-computer
%mediated, face-to-face contexts such as a group software design session or
%an interactive classroom setting. 
%
%Characteristics of strong collaboration are collective authoring,
%collective learning, and collective ownership. In other words, during the
%process of strong collaboration, each participant is both contributing to
%the construction of the artifact, and gaining new knowledge as a direct
%result of this construction.  The constructed artifact is not simply a
%patchwork of individual contributions, but instead an incremental, emergent
%synthesis that reflects the knowledge created by the group as a whole.

%
%Based upon this description of strong collaboration we define two methods
%of authoring, concurrent and non-concurrent.  Concurrent authoring is
%defined when two, or more, authors trade editing the contents of the same
%node while they are both connected to AEN.  e.g. author A and B are
%connected to AEN.  Author A edits a node and then author B edits that same
%node before author A disconnects.  This way author A and B can see the
%changes each makes as they are being made.  Non-concurrent authoring is
%when author A and B make changes to the same node but not while they are
%both connected to AEN.


%
%Strong collaboration and Weak collaboration define a spectrum of
%collaboration.  On one end purely Weak Collaboration there is no
%interaction between the collaborators.  Each is interested in only their
%portion of the document and combine their efforts at the end of the
%collaboration.  Strong Collaboration, at the other end of the spectrum, all
%authors interact in all portions of the document and contribute to all
%portions of the document.  Pure Strong Collaboration occurs concurrently,
%all authors are aware of the entire document simultaneously.  There is no
%way to assign ownership to a portion of the document to one author.


AEN.tex 6/7

%The initial requirements of AEN were to create a virtual classroom and move
%nearly all of the lectures for a class on-line into a hypertext document
%called {\em The Annotated Egret}.  The class' subject is how to design and
%construct collaborative systems using Egret.  This class would meet both
%on-line and off-line for discussions about issues.  Most of the interaction
%in the class would occur through annotation of the lecture material with
%new hypertext links to questions, comments, and insights.  A few actual
%classroom meetings would be supplementary forums to discuss things learned
%or questioned through on-line activities \cite{csdl-94-06}.  Chapter
%\ref{sec:fall94} discusses our experiences with AEN as a virtual class
%room and it evolution.
%  The following section provides a detailed
%description of AEN's current design.

%
%The user can create multiple TOCs starting from different document nodes.
%The table of contents should help users find their way through the
%hypertext document.  The TOC provides orienting information about the
%hypertext network and prevents the ``lost-in-hyperspace'' phenomenon.
%Another mechanism AEN has to help the user navigate in the hypertext
%document is node lists.
%

%Both facilities provide services that are similar to walking into
%a room containing the other people using AEN.  There are some significant
%differences between AEN's Snoopy/Partyline and the walking into a room.
%When you walk into a room, you can only see the people in the room.  You
%have no idea how long the people have been in the room and when people left
%the room.  Snoopy gives you this information.  You can see how long ago
%people left the room and how long people have been working on their current
%task.

%In a room many conversations can go on simultaneously and you can listen to
%any of the conversations in the room.  Partyline allows you to ``listen''
%to conversations that are sent to the entire room. In a real room it is
%easy to have two conversations at the same time since we can tune out other
%noises.  Partyline does not support two ``global'' conversations well.
%Since Partyline is text-based, users have to read the messages.  When the
%user receives a message they have to see who sent it.  This does not allow
%them to totally ignore the other conversations in the room.  If they
%actually read the message they focus their attention on the other
%conversation and may loose their train of thought of their conversation.

%One special feature that Partyline does allow is to have private
%conversations.  In a real room you can tell when people are ``whispering''
%to each other.  In Partyline there is no indication that two people are
%``whispering''.

%Another special feature of Partyline is that it keeps track of
%conversations.  If a user leaves their workstation and don't log out of AEN
%the Partyline buffer records the Partyline conversations.  When they return,
%the conversations are still accessible in the Partyline buffer.  The other
%users in AEN will know that they still can ``hear'' the conversations
%since Snoopy tells them that the user is asleep.  Snoopy displays the
%asleep message if the user has not done anything in a while.  Partyline
%allows a user to be paged.  Even though the user might be asleep the
%persistence of Partyline will tell them they have been paged.  They can
%respond to the page when they return.

%\subsection{AEN's Process Model}

%All of the above mechanisms are used by AEN to support strong
%collaboration.

%learning and authoring.  The next sections discuss how these mechanisms
%are used for collaborative learning and authoring.

%\subsubsection{Collaborative Learning}

%AEN uses a constructivist model of learning.  Constructivism, often
%associated with Jean Piaget, is the theory that knowledge is constructed
%not just acquired.  The students take an active role in their learning.
%They do not sit passively as knowledge is poured into them.  They interact
%with their instructor and each other to create knowledge.  Piaget contends
%that the restructuring of prior knowledge (learning) requires challenging
%existing views and coordinating old with new knowledge\cite{Piaget77}.

%In AEN the hypertext document represents the textbook, the instructor's
%lectures, and the construction site for learning.  When students explore
%the hypertext document, they can ask questions by annotating the text.  The
%instructor or other students can interact by making annotations on the
%question or changing the original text that raised the question.  This way
%``class discussions'' can take place, and students and instructors can
%explore their views about the subjects.  By using the access control
%mechanism students and instructors can decide who can participate in these
%discussions.  This allows a rich variety to the interactions in the class.

%The Snoopy and Partyline features expand AEN's learning environment by
%providing synchronous communications.  Students can get together in AEN and
%discuss the concepts presented in the text.  They can work together on
%solving the quizzes or projects presented in AEN.  Snoopy and Partyline
%allow the instructor to hold office hours by being ``present'' in AEN at
%given times.  The students can ask the instructor questions and receive
%answers immediately.  The instructor could have a group discussion with all
%the students present in AEN. If anyone decides that they want to save the
%discussion they can create a node in the hypertext document and paste the
%conversation into it.  This way a permanent record of the discussion can be
%kept.

%The combination of hypertext document and synchronous communications allows
%the students to learn in many ways.  They can learn by exploring the text,
%asking questions, answering questions raised by their fellow students,
%building their own hypertext sections or by answering the exams and quizzes
%added by the instructor.  As the class learn, the hypertext document
%will be evolving.  It represents the restructured prior knowledge and
%the current collaborative state of learning for the class.
 

%\subsubsection{Collaborative Authoring}

%Baecker et al. introduced a taxonomy of collaborative
%writing\cite{Baecker93}.  AEN supports many of the features described in
%their taxonomy.  I will discuss each in turn.
%\begin{itemize}
%\item{\bf Roles}
%  \begin{itemize}
%  \item{\em Writer:}  AEN's user interface is XEmacs, which is an editor.
%    This allows the user to easily record their ideas and modify the existing
%    text.
%  \item{\em Consultant:}  The Partyline messaging system allows users to
%    participate without actually writing any text.  Consultants are also
%    able to make comments that are not part of the finished document.
%  \item{\em Editor:}  XEmacs and AEN's access control model allow the users
%    to edit the hypertext document.
%  \item{\em Reviewer:}  Annotations to the hypertext document allow all of
%    the users who have annotation access to comment on the hypertext
%    document.  Users can also use Partyline to talk to the author/editor of
%    the node they are reviewing.
%  \end{itemize}
%\item{\bf Activities}
%  \begin{itemize}
%  \item{\em Brainstorming:}  AEN allows authors to create nodes to store
%    their ideas.  With the right accesses other authors can build off of
%    nodes created by the other authors.
%  \item{\em Researching:}  AEN does not provide any support for external
%    research.
%  \item{\em Planning:}  AEN's backbone provides an simple mechanism to
%    create an outline structure for the hypertext document.
%  \item{\em Writing:}  AEN's interface and collaborative nature make
%    transforming ideas into text easy.
%  \item{\em Editing:}  XEmacs and AEN's access control model allow the users
%    to edit the hypertext document.
%  \item{\em Reviewing:} Annotations to the hypertext document allow all of
%    the users who have annotation access to comment on the hypertext
%    document.  Users can also use Partyline to talk to the author/editor of
%    the node they are reviewing.
%  \end{itemize}
%\item{\bf Document Control Mechanisms}
%  \begin{itemize}
%  \item{\em Centralized:}  AEN's access control mechanism can be used to
%    allow only one author to make changes to the hypertext document.  The
%    other authors would only be able to make comments or suggestions.
%  \item{\em Relay:}  AEN's access control mechanism is dynamic so the
%    access levels for the hypertext document could passed on from one author
%    to the next.
%  \item{\em Independent:}  Authors are able to use AEN's access control
%    mechanism to create their own sections of the hypertext document.  The
%    authors have complete control over the access controls for their
%    sections.
%  \item{\em Shared:}  All authors can share in the  control of the access
%    to a node in the hypertext document.  Through the locking mechanism AEN
%    ensures that the node can only be changed by one author at a time.
%  \end{itemize}
%\item{\bf Writing Strategies}
%  \begin{itemize}
%  \item{\em Single Writer:}  AEN's access control mechanism can be used to
%    allow only one author to make changes to the hypertext document.  The
%    other authors would only be able to make comments or suggestions.
%  \item{\em Scribe:}  An author can be designated to record the decisions
%    and comments made by the other authors.  This can easily be done by
%    cutting and pasting from the Partyline.
%  \item{\em Separate Writers:}  Authors are able to use AEN's access control
%    mechanism to create their own sections of the hypertext document. Each
%    of these sections are automatically part of the whole hypertext
%    document.  The authors have control of how much is included into the
%    final document.  They can also control the amount of commenting and
%    editing is done to their sections.
%  \item{\em Joint Writing:}  This is what we call strong collaboration.
%    This is the goal of AEN.  We want to support strong collaboration in
%    AEN.
%  \end{itemize}
%\end{itemize}

experiment.tex 6/11

  The next sections show
how the data was used to identify occurrences of each behavior in the
operationalized definition of strong collaboration.  For simplicity, the
group being analyzed has only three members, {\tt cmoore}, {\tt johnson},
and {\tt rosea}.

\begin{itemizenoindent}

\item{\em A majority of members create document nodes.}

  Based upon the list of users we can search the combined metrics data
  and find document node creation metrics for each user.  Figure
  \ref{fig:create} shows all three members of the group have created
  document nodes.

  \begin{figure}[htb]
    \small
    \begin{verbatim}
    User 1 (cmoore): 
[mt*event \"52H:[M\" lock-to-annotate 48 nil] 
[mt*event \"52H:[S\" create-annotation 48 (artifact-type document node-ID 60
    link-type include link-ID 9)] 
[mt*event \"52H:[S\" write 48 nil] 
[mt*event \"52H:[T\" read 60 (type document)] 
[mt*event \"52H:\\\\3\" busy 0 nil]
[mt*event \"52H:]3\" busy 0 nil] 
[mt*event \"52H:^3\" busy 0 nil] 
[mt*event \"52H:^7\" write 60 nil] 
[mt*event \"52H:^:\" unlock 60 nil]

    User 2 (johnson):
[mt*event \"52H:_a\" create 62 (type document)]
[mt*event \"52H:`3\" write 62 nil]
[mt*event \"52H:`5\" unlock 62 nil] 

    User 3 (rosea):
[mt*event \"52H;hT\" create-annotation 70 (artifact-type document node-ID
    72 link-type include link-ID 15)] 
[mt*event \"52H;hT\" write 70 nil]
[mt*event \"52H;hU\" read 72 (type document)]
[mt*event \"52H;hY\" write 72 nil]
[mt*event \"52H;h[\" unlock 72 nil]
    \end{verbatim}
    \normalsize
    \caption{Example metrics illustrating document node creation}
    \label{fig:create}
  \end{figure}


\item{\em A majority of members create feedback nodes.}

  By searching the combined metrics data we can find all the occurrences of
  comment node creation for the members.  Figure \ref{fig:feedback} shows all
  three members of the group have created feedback nodes.

\begin{figure}[htb]
  \small
  \begin{verbatim}
    User 1 (cmoore):
[mt*event \"52H:[M\" lock-to-annotate 48 nil]
[mt*event \"52H:[S\" create-annotation 48 (artifact-type comment node-ID 64
    link-type comment link-ID 11)] 
[mt*event \"52H:[S\" write 48 nil]
[mt*event \"52H:[T\" read 64 (type comment)]
[mt*event \"52H:^3\" busy 0 nil]
[mt*event \"52H:^7\" write 64 nil]
[mt*event \"52H:^:\" unlock 64 nil]

    User 2 (johnson):
[mt*event \"52H::_a\" create-annotation 54 (artifact-type comment node-ID 62
    link-type comment link-ID 11)]
[mt*event \"52H:`3\" write 62 nil]
[mt*event \"52H:`5\" unlock 62 nil] 

    User 3 (rosea):
[mt*event \"52H;hT\" create-annotation 70 (artifact-type comment node-ID 72
    link-type comment link-ID 15)]
[mt*event \"52H;hT\" write 70 nil]
[mt*event \"52H;hU\" read 72 (type comment)]
[mt*event \"52H;hY\" write 72 nil]
[mt*event \"52H;h[\" unlock 72 nil]
  \end{verbatim}
  \normalsize
  \caption{Example metrics illustrating Feedback node creation}
  \label{fig:feedback}
\end{figure}


\item{\em A majority of members read each other's nodes.}

  A query to the Unread Nodes agent will tell us which nodes have not been
  read for each user.  Given this list of unread nodes we can determine if
  the owner for each of them and see if the members read all of the other
  members nodes.

\item{\em If team members are simultaneously logged in, Partyline is used.}

  The combined metrics will show us when team members are simultaneously
  logged in.  During each of these events we can search for Partyline
  activity.  Figure \ref{example:realTime} shows two users communicating in
  real time. {\tt johnson} and {\tt cmoore} have a short conversation.

  \begin{figure}[htb]
    \small
    \begin{verbatim}
    User 1 (cmoore)
[mt*event "52A?[a" read 1428 (type document)]
[mt*event "52A?[c" lock 1428 nil]
[mt*event "52A?[B" send 0 (to ("johnson"))]
[mt*event "52A?[M" busy 0 nil]

    User 2 (johnson)
[mt*event "52A?[A" read 1688 (type document)]
[mt*event "52A?[K" send 0 (to ("cmoore"))]
[mt*event "52A?]a" lock 1688 nil]
[mt*event "52A?]3" busy 0 nil]
[mt*event "52A?]A" send 0 (to ("cmoore"))]
[mt*event "52A?]e" busy 0 nil]
[mt*event "52A?]Q" write 1688 nil]
    \end{verbatim}
    \normalsize
    \caption{Example metrics illustrating real time communications}
    \label{example:realTime}
  \end{figure}


\item{\em Each member edits nodes that were also edited by others.}

  By looking at the combined metrics we can see when members edit a node that
  another member has edited.  Figure \ref{fig:trading} shows each member
  editing node 1428.

\begin{figure}[htb]
  \small
  \begin{verbatim}
    User 1 (cmoore):
[mt*event "4B5=g3" mt*event*make 0 nil]
[mt*event "4B5=gB" read 1428 (type document)]
[mt*event "4B5=gX" lock 1428 nil]
[mt*event "4B5=h3" busy 0 nil]
[mt*event "4B5=i1" write 1428 nil]
[mt*event "4B5=i3" unlock 1428 nil]
[mt*event "4B5=i3" busy 0 nil]
[mt*event "4B5=j3" busy 0 nil]

    User 2 (johnson):
[mt*event "4BL;g3" mt*event*make 0 nil]
[mt*event "4BL;=g" read 1428 (type document)]
[mt*event "4BL=gX" lock 1428 nil]
[mt*event "4BL=h3" busy 0 nil]
[mt*event "4BL=i3" busy 0 nil]
[mt*event "4BL=j3" busy 0 nil]
[mt*event "4BL=k3" write 1428 nil]
[mt*event "4BL=l3" unlock 1428 nil]

    User 3 (rosea):
[mt*event \"52H;hU\" read 1428 (type document)]
[mt*event \"52H;hV\" lock 1428 nil]
[mt*event \"52H;hY\" write 1428 nil]
[mt*event \"52H;h[\" unlock 1428 nil]
  \end{verbatim}
  \normalsize
  \caption{Example metrics illustrating members editing the same node}
  \label{fig:trading}
\end{figure}

\item{\em All document nodes are edited by more than one person.}

  We can get a list of all of the document nodes by using the following
  command in AEN:
  \begin{verbatim}
(aen*{document}*node-IDs db-ID)
  \end{verbatim}
  By looking at all the document nodes in AEN and the combined metrics,
  we will be able to see if each document node was edited by more than
  one member of the team.

\item{\em All members find the Snoopy and Partyline features for physical
  presence to be useful.}

  The post-test questionnaire will ask the participants if they feel Snoopy
  and Partyline provided them with a sense of physical presence.  The
  questionnaire will also ask if they felt that Snoopy and Partyline were
  useful in supporting collaboration.

\item{\em Members manipulate access control to publish/protect artifacts
  under development.}

  By searching the combined metrics data we can find where each member
  manipulated the access control for a node.  Figure \ref{fig:manip} shows
  each user manipulating the access control mechanism in different ways.
  {\tt cmoore} allows the other members read and annotation access.  {\tt
  johnson} is just letting the other members read his comment.  {\tt rosea}
  is allowing all the members to read the comment.  {\tt cmoore} can create
  comments on {\tt rosea}'s comment while {\tt johnson} can edit the comment.

\begin{figure}[htb]
  \small
  \begin{verbatim}
    User 1 (cmoore):
[mt*event "4B5=gB" read 1428 (type document)]
[mt*event "4B5=kW" toggle-read-access 1428 (user t access-list (t "cmoore"))]
[mt*event "4B5=l]" toggle-annotate-access 1428 (user t access-list (t "cmoore"))]

    User 2 (johnson):
[mt*event "4BL;>]" read 2472 (type comment)]
[mt*event "4BL=gX" lock 2472 nil]
[mt*event "4BL=i1" write 2472 nil]
[mt*event "4BL=i3" unlock 2472 nil]
[mt*event "4BL=@W" toggle-read-access 2472 (user t access-list (t "johnson"))]

    User 3 (rosea):
[mt*event \"52H;hT\" create-annotation 70 (artifact-type comment node-ID 72
   link-type comment link-ID 15)]
[mt*event \"52H;hT\" write 70 nil]
[mt*event \"52H;hU\" read 72 (type comment)]
[mt*event \"52H;hY\" write 72 nil]
[mt*event \"52H;h[\" unlock 72 nil]
[mt*event "4BL=@W" toggle-read-access 72 (user t access-list (t "rosea"))]
[mt*event "4BL=@W" toggle-annotate-access 72 (user t access-list ("cmoore"
    "johnson" "rosea"))]
[mt*event "4BL=@W" toggle-write-access 72 (user t access-list ("johnson" "rosea"))]
  \end{verbatim}
  \normalsize
  \caption{Example metrics illustrating manipulating access control}
  \label{fig:manip}
\end{figure}

\item{\em All nodes underwent manipulation of access control.}

  By getting a list of all the nodes in the database and searching the
  combined metrics we can see if all of the nodes had their access control
  manipulated.

\end{itemizenoindent}


%
%Either confirmation or rejection of our hypothesis will be important to the
%understanding of strong collaboration.  If our hypotheses are
%confirmed, then it indicates what set of tools are needed to support
%strong collaboration.  If our hypotheses are rejected, then it
%contributes new support to the claim that face to face interaction is
%required for group synergy.
%


%
%\paragraph {Threats to External Validity.}
%
%Threats to external validity jeopardize the generalization of findings from
%this study to industry practice. Three such threats are apparent: (1) The
%participants may not be representative of industry professionals; (2) The
%artifacts may not be representative of industry artifacts; and (3) The
%collaboration methods may not be representative of industry methods.
%
%All three threats are real, although that does not mean the results from
%this study will be meaningless or irrelevant to industry.  It simply means
%that external validation of the results from this study will be necessary.
%
%This study is not intended to find causal relationships in strong
%collaboration, but to explore strong collaboration.  This study will
%lead to a better understanding of what strong collaboration is and some
%of the factors involved in strong collaboration.  Further research is
%needed to generalize the results of this study to industry. 



\section{Anticipated Results}

We anticipate that this study will provide useful new information about
strong collaboration, by exploring the real time communications,
physical presence, and access control aspects of strong
collaboration. In addition, the study will provide new insight into how
groups of people collaboratively author documents.





intro.tex 6/20
%
%\section{Initial Experiences}
%
%Over the past year, we have experimented with and redesigned AEN.  AEN
%places equal weight on supporting collaborative authoring of, and
%collecting data about user interaction with a hypertext document.  A third
%requirement of the system is rapid response time.  This system and our ideas
%about collaboration have undergone extensive evolution as we learned more
%about the implications of these requirements and their often conflicting
%natures.
%
%It took three months to take AEN from a simple set of requirements to the
%first prototype.  We came up with a simple hypertext authoring tool that
%allows multiple authors to edit different nodes of the same hypertext
%document at the same time.  In the Fall of 1994 we used the prototype in a
%graduate seminar class ICS 691.  The class used AEN and it's database as
%the text book and example of a collaborative system.  During the semester
%the students found bugs in AEN and made over 28 suggestions about improving
%the system.   We redesigned and made AEN more robust.
%
%We hoped that the Fall, 1994 ICS 691 class would demonstrate strong
%collaboration. However, due to problems with the reliability of AEN, they
%lost faith with AEN and did not use it for collaboration. The students
%learned that AEN could not be trusted to keep all of their work.  They
%began using alternative methods to collaborate and just uploaded the
%results to AEN for grading.
%

%
%Since we were unable to detect strong collaboration during the fall of
%1994, we decided to conduct an experimental case study during the spring of
%1995.  We conducted the case study to learn more about the computational
%requirements of strong collaboration.  
% Each group consisted of five students.  They worked with
%private companies developing software products for the companies.  The
%groups learned how to use software engineering principles to develop their
%software products.  

%\item{\em Authors concurrently edit the same document node.}  This
%  operationalization is not necessary for strong collaboration, but if it
%  occurs then this is a good indication that strong collaboration is
%  occurring.  In order to control concurrent editing the authors must
%  coordinate sharing the lock on the node.
%\item{\em Members manipulate access control to publish/protect artifacts
%  under development.}  By changing the access control members are able to
%  dynamically control the process of collaboration.

6/22 results
%Figure \ref{fig:total-final} shows the number of nodes each group owned at
%the end of the study and the number of nodes that made up their final
%requirements documents.  The difference in these two numbers can be
%attributed to two factors.  One, the group decided not to include the node
%in their final document for some reason.  Two, the node was an orphan and
%was never included in the groups' work space.  The second factor can happen
%if a member creates a node and does not link it into the group's hypertext
%network.  If the member then logs off the rest of the group has no way of
%gaining access to that node.


%\begin{figure}[htbp]
%  \begin{center}
%    \begin{tabular}{|c|c|c|}
%      \hline
%      Group&Total \# of Nodes at end& \# of Nodes in Final Document\\ \hline
%      \hline 
%      Koa&34&16\\ \hline
%      Ku Kahakai&41&14\\ \hline
%      Na Koa&25&10\\ \hline
%    \end{tabular}
%  \end{center}
%  \caption{Document nodes in final document v.s. total \# of document nodes.}
%  \label{fig:total-final}
%\end{figure}


6/26 intro



%Collaborative authoring is important since it is so popular.  Its
%popularity is due to the WWW.  The WWW only provides basic support for
%collaborative authoring.  Some of the issues that WWW does not answer are,
%access control, sychronous access, change notification,


%Many issues concerning collaborative hypertext construction are difficult
%to investigate using WWW. For example, issues such as, simultaneous
%editing, change notification, and restricted access are very difficult to
%implement, because of WWW's scale.  


6/27 results

%After looking at the RPN metric the next step to determine if strong
%collaboration has occurred is to look at the amount of collaborative
%editing that occurs.  The next two sections investigate MCE and DCE. %!!!
%                                                                     %the
%                                                                     %type
%                                                                     %of
%                                                                     %editing 
%that occurs for each node and member.  They are very similar.  The
%distinction between them is their focus, one looks at editing from the
%point of view of the member, while the other, takes the point of view of
%the nodes.
