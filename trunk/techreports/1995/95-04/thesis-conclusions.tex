%%%%%%%%%%%%%%%%%%%%%%%%%%%%%% -*- Mode: Latex -*- %%%%%%%%%%%%%%%%%%%%%%%%%%%%
%% thesis-conclusions.tex -- 
%% Author          : Carleton Moore
%% Created On      : Thu Jun 22 19:12:06 1995
%% Last Modified By: Carleton Moore
%% Last Modified On: Thu Jul  6 18:19:18 1995
%% RCS: $Id: thesis-conclusions.tex,v 1.6 1995/07/07 04:05:55 cmoore Exp cmoore $
%%%%%%%%%%%%%%%%%%%%%%%%%%%%%%%%%%%%%%%%%%%%%%%%%%%%%%%%%%%%%%%%%%%%%%%%%%%%%%%
%%   Copyright (C) 1995 Carleton Moore
%%%%%%%%%%%%%%%%%%%%%%%%%%%%%%%%%%%%%%%%%%%%%%%%%%%%%%%%%%%%%%%%%%%%%%%%%%%%%%%
%% 

%for review purposes
%\ls{1}

\chapter{AEN: Conclusions}
\label{sec:conclusions}

This chapter describes my conclusions about this research. Section
\ref{sec:con-AEN} discusses AEN.  Section \ref{sec:con-SC} presents my
conclusions about strong collaboration.  Section \ref{sec:con-research}
concludes the chapter with my conclusions about conducting research on
collaboration.

\section{AEN, a Collaborative Authoring Tool}
\label{sec:con-AEN}
A major contribution of this research is AEN itself.  This research has
shown that AEN is both a valuable research and authoring tool.  The former
is evidenced by the built-in instrumentation that allows fine-grained
process data to be collected.  The latter is highlighted by the many
quality hypertext documents that the users produced during the Fall, 1994
and Spring, 1995.

%The third contribution of this research is the data from the Fall, 1994 and
%Spring, 1994 use of AEN.   confirmed that AEN is a viable approach to
%supporting collaborative authoring of hypertext documents.  The results
%also suggest a number of interesting directions in which AEN can be
%extended and empirical investigations can be conducted.

I have four conclusions about AEN as a collaborative tool.  First, AEN
appears to be a successful collaborative authoring environment.  Second,
AEN is complex and requires significant training.  Third, AEN's lack of a
predefined process allows it to support many different processes.  Fourth,
AEN is a valuable tool in researching collaborative construction of
hypertext documents.

There are three different sets of evidence that appears indicate that AEN
is a successful collaborative authoring tool. One, in both uses of AEN ---
Fall, 1994 and Spring, 1995 --- the authors collaboratively created quality
documents.  Two, all three requirements documents from the Spring, 1995 
show high values in at least three of their collaborative metrics.  Three,
the survey results from Spring, 1995 show that the users liked
AEN.

The complexities of collaborative document creation and AEN itself
requires that users of AEN be well trained.  The training given to the
users during the Spring, 1995 case study consisted of a one hour hands on
session and any personal training the users wanted.  Only five or six users
took advantage of the personal training sessions.  The training class was
not sufficient to allow them to use all the features fully.  One user
commented {\em ``I wanted to have a something like `hot list'.''}  AEN does have
a history list that keeps track of the last 20 nodes the user visits.  This
feature is much like a hot list, but very few of the users used this
feature.

AEN is capable of supporting different styles of collaboration.  The lack
of a predefined process allows AEN to support many different group
processes.  Different groups are able to define their own processes to
collaborate and use AEN's feature to help support their processes.  The
three groups in the case study each had their own internal process for
collaboration.  Each of the groups used AEN's features to support their
process.  The lack of a predefined process forces the groups to develop
their own.  One group did not like this:
\begin{quote}
\ls{1.0}
  {\em 
  ``Overall, it was a very useful tool.  However, the only problem was
  that we needed to have a set of procedures on how to use AEN.  That
  would have increased our productivity a lot.''  
  }
\ls{1.5}
\end{quote}

AEN is a valuable tool for research on collaboration.  AEN's non-intrusive
metrics collection gives the researcher a detailed view of how individuals
collaboratively construct hypertext documents.  AEN's metrics keeps track
of each user's use of tools and activities.  Each time a user uses a tool,
AEN creates a metric (see Section \ref{sec:metrics-description} for a
discussion on the metrics.)  AEN also keeps track of the users' busy times.
This information combined with user surveys can give a researcher a
detailed view into collaboration.


AEN was designed to support strong collaboration.  The next section
presents my conclusions about strong collaboration.

\section{Strong Collaboration}
\label{sec:con-SC}

Another major contribution of this research is AEN's operationalized
definition of strong collaboration.  This definition and the associated
metrics allow the detection and measurement of collaborative behaviors.
The collaborative metrics can be used to describe collaborative
construction of hypertext document.  Two collaboratively constructed
hypertext documents can be compared using these metrics.  Being able to
compare two hypertext documents based upon their levels of collaboration
could lead to improvement in the process of creating hypertext document.

Based upon the my experiences over the last year, I recommend the following
to encourage strong collaboration: provide direct and indirect authoring
mechanisms, provide access to intermediate work products, provide
context-sensitive ``what's new,'' and provide mechanisms to allow users as
well as documents to be visible.

My first recommendation is to provide direct and indirect authoring
mechanisms.  AEN supports two methods for collaborative authoring, which I
term {\em proof-reading}(indirect) and {\em trading the lock}(direct).  In
the proof-reading method, the author creates a node and publishes it by
allowing read and annotate access.  This allows others to make comments
suggesting changes, new ideas, or just general comments on the subject.
The author can read the comments and make changes to the document or
comment on the comments.  In the trading the lock method, the author
creates a node and allows other authors to edit it by providing read and
write access.  Each author can lock the node, make a change, save the node,
and unlock it, which updates the contents of the node displayed on each
author's screen.  Strong collaboration is enhanced by providing authors
with these degrees of control over the document and styles of interaction
with others.

My second recommendation is to provide access to intermediate work
products.  One of the strongest enablers of strong collaboration is easy
accessibility to intermediate work products.  Synergy is nurtured by
permission to review another's admittedly rough, first pass at an idea,
where comments and suggestions are aimed at refining and enhancing, rather
than confirming or denying, as is frequently the case with final, polished
presentations.

My third recommendation is to provide context-sensitive ``what's new.''
When a group divides into subgroups and is actively and incrementally
building hypertext documents, context-sensitive mechanisms to automatically
inform users of what has changed are essential. Otherwise, the users will
suffer from either lack of knowledge about what is changing (if no
change-related mechanisms exist) or a low signal-to-noise ratio (if
context-free change-related mechanisms are used, in which case users are
informed of many changes that are irrelevant to them.)  In AEN, the
combination of access control, unread nodes, and Hyperstar Bulletin
provides a very nice means of selectively propagating change-related
information across groups.  The daily Hyperstar Bulletin encourages users
to log in to AEN only when necessary to see changes, and access control
allows users to control the visibility of their changes.

My last recommendation is to provide a mechanism to allow users as well as
artifacts to be visible.  In AEN, providing knowledge of who was using AEN,
where they were, and a means to communicate with them created many new
opportunities for collaboration without requiring face-to-face interaction.
  
The above recommendations should help enable strong collaboration.  In my
exploration of strong collaboration, I also made the following conclusions
about researching strong collaboration.

\section{Research into Collaboration}
\label{sec:con-research}

User confidence is vitally important to effective research on
collaboration.  If the user loses confidence in the system, they will stop
using the system to collaborate and find ways around the system to get
their work done.  I found two primary factors that impact on user
confidence; data integrity and system reliability.

First and foremost, maintaining database integrity is essential to
effective research on collaboration.  This might seem like a no-brainer,
but I learned the hard way that collaboration can be completely undermined
by lost data.  During the Fall, 1994 semester AEN continually lost data.
Even ``standard'' measures may not be enough to prevent the loss of data.
Soon after instituting a daily backup mechanism, AEN lost user data due to
the convergence of (a) a holiday weekend (Thanksgiving), (b) a filled disk
drive, and (c) a power failure.  The participants during the Fall, 1995
semester learned that they could not trust AEN with their data.  One wrote: 
\begin{quote}
\ls{1.0}
  {\em
  I have lost many hours of work from the numerous times AEN has crashed.
  It is hard to encourage group collaboration when the system is down and
  work is continually lost.  In the end, I would do all my work outside of
  AEN and upload it when it was ready.
  } 
\ls{1.5}
\end{quote}
This effectively stopped all research into collaboration in Fall, 1994.
During the Spring, 1995 case study, I moved to four backups of the system
daily.  As a result, AEN did not lose any data during the entire case study.

Second, reliability of the system is also important for user confidence.
Before the Spring, 1995 case study, I fixed the database integrity problem.
I also greatly improved system reliability.  During Fall, 1994,
AEN crashed an average of once per week.  During the Spring, 1995 use it
crashed approximately five times --- twice in the first few weeks and three
times while I was attending a workshop.  Unfortunately, the workshop was
held at a critical time in the groups' requirements document development
process.  When AEN crashed while I was away, it would often be down until
the next morning.  This lack of availability caused one user to comment,
\begin{quote}
\ls{1.0}
  {\em
  The only real problem I had with it (AEN) was that it crashed a lot and
  then I had to wait around until it was restarted.  This waiting around
  usually meant that I could not work on AEN until the next day.
  }
\ls{1.5}
\end{quote}
This comment shows a distinctly different attitude from the previous
comment.  This user was upset that he could not use AEN and had to wait
until the next day to continue to work.  The first user completely gave up
on AEN and worked outside of AEN.  By improving the reliability of AEN, I
was able to conduct the Spring, 1995 case study and see glimpses of strong
collaboration.



