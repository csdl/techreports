%%%%%%%%%%%%%%%%%%%%%%%%%%%%%% -*- Mode: Latex -*- %%%%%%%%%%%%%%%%%%%%%%%%%%%%
%% hbs-intro.tex -- 
%% Author          : Carleton Moore
%% Created On      : Fri Jan 20 17:13:59 1995
%% Last Modified By: Philip Johnson
%% Last Modified On: Thu Feb  9 12:18:55 1995
%% Status          : Unknown
%% RCS: $Id: hbs-intro.tex,v 1.3 1995/02/09 22:24:50 johnson Exp $
%%%%%%%%%%%%%%%%%%%%%%%%%%%%%%%%%%%%%%%%%%%%%%%%%%%%%%%%%%%%%%%%%%%%%%%%%%%%%%%
%%   Copyright (C) 1995 University of Hawaii
%%%%%%%%%%%%%%%%%%%%%%%%%%%%%%%%%%%%%%%%%%%%%%%%%%%%%%%%%%%%%%%%%%%%%%%%%%%%%%%
%% 

\section{Introduction}

This document describes the current design of HBS\footnote{HyperBase
Server}, an 11 KLOC Hypertext multiuser database server written in C++.  The
original implementation came from Uffe Kock Will et al \cite{Wiil90a} at the
Aalborg University Center.  We at the Collaborative Software Development
Laboratory (CSDL) rewrote the original HyperBase and added many new
features.

The goals of this document are provide the reader with an understanding of
the design of the HBS, give the reader enough information to be able to
extend the basic set of operations provided by the HBS,  and finally
to provide an
understanding of how the HBS works and the interplay of its internal parts.

Here is a suggested reading order for the HBS documentation.  First, read the HBS
Interface Specification \cite{csdl-94-14} to get an idea of what
operations HBS provides the client.  Then read this document to get a
description of the design and architecture of HBS.  This document will
refer you to the original design document ``Design and Implementation of a
HyperBase'' \cite{Wiil90a} for further detail and for a detailed discussion
of operations that are not currently used in ECS.  The ECS design
document \cite{csdl-94-13} provides detailed information about the client
system with which HBS communicates.

This document is organized as follows.  Section \ref{sec:architecture}
describes the overall architecture of HBS; Section \ref{sec:blocks}
describes each block of the architecture in detail;  Section
\ref{sec:debugging} discusses all of the built in debugging tools provided
by HBS; Section \ref{sec:extentions} explains how to extend the current set
of operations in HBS; and finally Section \ref{sec:limitations} lists the
current known bugs, limitations and issues with HBS.

