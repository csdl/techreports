%%%%%%%%%%%%%%%%%%%%%%%%%%%%%% -*- Mode: Latex -*- %%%%%%%%%%%%%%%%%%%%%%%%%%%%
%% hbs-limitations.tex -- 
%% Author          : Carleton Moore
%% Created On      : Fri Jan 20 17:40:00 1995
%% Last Modified By: Carleton Moore
%% Last Modified On: Thu Jul 20 15:04:37 1995
%% Status          : Unknown
%% RCS: $Id: hbs-limitations.tex,v 1.3 1995/07/21 01:11:07 cmoore Exp $
%%%%%%%%%%%%%%%%%%%%%%%%%%%%%%%%%%%%%%%%%%%%%%%%%%%%%%%%%%%%%%%%%%%%%%%%%%%%%%%
%%   Copyright (C) 1995 University of Hawaii
%%%%%%%%%%%%%%%%%%%%%%%%%%%%%%%%%%%%%%%%%%%%%%%%%%%%%%%%%%%%%%%%%%%%%%%%%%%%%%%
%% 

\section{Known issues, Limitations, Bugs}
\label{sec:limitations}

\subsection{Security}

The only security HBS provides is a list of authorized users and passwords.
Once an authorized user has passed the password protection, HBS provides no
further security measures.  Every client has read and write access to every
node and link in the database.  If applications need security or a way of
restricting access to nodes or links the applications will have to
implement their own restrictions using client-level mechanisms.

\subsection{Garbage collection}

There are two types of garbage collection associated with the HBS: memory
and disk space.  There is no memory garbage collection implemented in the
HBS.  Currently the HBS does not have any major memory leaks.  Developers
who extend the HBS will have to make sure that their code does not create
any memory leaks.

The second type of garbage collection is of disk space.
Currently the HBS reuses disk space, but it does not shrink the size of the
Unix files.  When an entity is deleted its disk space is added to the free
lists and when another entity is create it can use the freed disk space.
Since the HBS does not shrink the files when there are fewer entities the
file are monotonicly increasing.  This can lead to inefficiencies in disk
storage. 

\subsection{Use of standardized file system libraries}

The HBS does not use any standard file system libraries which would make
HBS much more portable.  Fortunately, all of the disk access code is in
block1 so anyone porting HBS to another file system will only have to
rewrite block1.

\subsection{Use of standardized socket/IPC libraries}

Similarly the HBS does not use standardized socket/IPC libraries.  All of
the socket code is located in block4 so changing the socket mechanisms just
requires a rewrite of block4.
