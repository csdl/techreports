%%%%%%%%%%%%%%%%%%%%%%%%%%%%%% -*- Mode: Latex -*- %%%%%%%%%%%%%%%%%%%%%%%%%%%%
%% hbs-architecture.tex -- 
%% Author          : Carleton Moore
%% Created On      : Fri Jan 20 17:15:29 1995
%% Last Modified By: Carleton Moore
%% Last Modified On: Thu Jul 20 14:37:59 1995
%% Status          : Unknown
%% RCS: $Id: hbs-architecture.tex,v 1.3 1995/07/21 01:11:55 cmoore Exp $
%%%%%%%%%%%%%%%%%%%%%%%%%%%%%%%%%%%%%%%%%%%%%%%%%%%%%%%%%%%%%%%%%%%%%%%%%%%%%%%
%%   Copyright (C) 1995 University of Hawaii
%%%%%%%%%%%%%%%%%%%%%%%%%%%%%%%%%%%%%%%%%%%%%%%%%%%%%%%%%%%%%%%%%%%%%%%%%%%%%%%
%% 

\section{HBS Architecture}
\label{sec:architecture}
\subsection{Overview}

HBS is a Hypertext server.  It's architecture consists, at the highest
level of abstraction, of four more-or-less
layered ``blocks'': file operations, basic hypertext operations, event
and lock operations, and client-server operations.  Each block will be
described below.  Figure \ref{fig:Overall} shows the way these blocks
are layered in HBS.

\begin{figure}[htb]
  \centerline{\psfig{figure=overall.eps}}
  \caption{{\bf Overall Architecture of HBS}}
  \label{fig:Overall}
\end{figure}

Block 1, the file operations layer, is responsible for the file control and
operations.  This layer contains the classes that represent the different
types of files, the file handler classes that provide a clean interface to
the higher layers, and some basic list and tree entities.

Block 2, the basic hypertext layer, is responsible for the basic operations
on nodes and links.  This layer contains all of the low level hypertext
operations provided by HBS.  It contains the operations class which
provides all of the basic hypertext operations, the data class which
represents a generic data node in the hypertext database, and the link
class which represents a generic link in the hypertext database.

Block 3, the event and lock layer, provides the locking and event mechanism
that allow multiple users to use HBS.  This layer has the eventlock
class which provides all the operations for HBS, the EventEntity and
LockEntity represent the event and lock records to keep track of lock and
event information.  

Block 4, the client server operations layer, provides all of the
multi-client connection handling routines.  There is only one class in this
layer, the hb\_client.  Each instance of this class represents one of the
ECS clients connected to HBS.  Each client is assigned three TCP/IP sockets
to communicate with HBS.  This layer handles all requests by the clients
and passes them to the lower layers.

\subsection{Canonical Operation Processing Process}

Almost every operation request is satisfied by HBS according to the
following stereotypical sequence of interactions between the various
blocks:


\begin{enumerate}
\item The client/server layer polls each client to see if they have made a
  request.  When the client has made a request the client/server level
  decides what the request was and then reads in the appropriate parameters
  for the request.  The client/server layer then passes the request on to the
  Event/Lock layer.

\item The Event/Lock layer determines if the user is able to perform the
  operations based upon the lock constraints of the database.  If the
  user is able to perform the operation the Event/Lock layer passes the
  request to the Basic Hypertext layer.

\item The Basic Hypertext layer decides what basic hypertext operations
  must be performed on links and nodes and then requests that those
  operations are performed from the node and link classes.  The node and link
  classes make requests of the File Operations layer to save or retrieve the
  information needed.

\item The File Operations layer passes up the results of the request to
  the Basic Hypertext layer which passes up the results to the Event/Lock
  layer.  If the operation was successful the Event/Lock layer sends out
  an event to all of the clients that have subscribed to events of the
  operation requested.

\item The Event/Lock layer passes the results of the operation back up
  the the Client/Server layer which passes the results back to the
  client.
\end{enumerate}
 
Figure \ref{fig:Canonical} shows the flow of requests and results in the
canonical operation processing.

\begin{figure}[htb]
  \centerline{\psfig{figure=canonical.eps}}
  \caption{{\bf Canonical Operation Processing}}
  \label{fig:Canonical}
\end{figure}

\subsection{Differences between HBS and Hyperbase}

We have made several major changes in HBS.  It no longer is compatible with
the Hyperbase.  The most significant difference is the addition of a
distinguished client, the Gagent.  We have also changed the connection
protocol to allow the HBS to ensure the client is connecting to the correct
database.  We have also extended the basic set of operations the Hyperbase
provided.  We discuss each of these changes in turn.

\subsubsection{Gagent}

Since the HBS is only used as a part of Egret \cite{csdl-92-01,csdl-93-09}
we have added in the knowledge of the Gagent.  The Gagent is treated as a
special client.  The HBS remembers which client is the Gagent and during
client connection the HBS listens to the Gagent.

\subsubsection{Client Connection}

The connection protocol has been changed to allow the Gagent to update the
global tables and ensure that the connecting client gets the correct global
state.  When a client connects to the HBS, the HBS will stop polling all
clients except the Gagent.  The other connected clients will still be able
to make requests but the HBS will temporally ignore their requests until
the connecting client tells the HBS that they are done connecting.  Once
the connecting client is finished reading in the global state they must
inform the HBS that they are done so the HBS can start servicing the other
clients. 


\subsubsection{Security}

HBS version 4.0.0 introduced security.  Now, when a client connects their
name must appear in the list of authorized users.  They must also supply a
password.  This password is checked before they are allowd to proceed.  the
security mechanism helps guarantee that only authorized users use the have
access to the HBS database.


\subsubsection{New Operations}

We have added several new operations to the HBS.  Many of these new
operations encapsulate several old operations.  Some of the operation are
new.

\begin{itemizenoindent}
\item{\em Shut Down:} brings down the HBS gracefully.
\item{\em Show Users:} shows all of the users connected to the HBS.
\item{\em Append:} appends the data to the end of the node.
\item{\em Create Node With Name:} creates a new node with the supplied name.
\item{\em Good Link:} creates a link between the two supplied nodes.
\item{\em Message:} sends a message to all connected users.
\item{\em Connected:} tells the HBS to start listening to the other users.
\item{\em Create Node with Name and Data:} creates a new node with the supplied
name and data.
\item{\em Create Node with Name, Data, and Lock:} creates a new node with
the supplied name and data.  It also locks the node.
\item{\em Abnormal Disconnect:} tells the other users that the client
abnormally disconnected.
\item{\em Delete Link:} deletes the supplied link.
\item{\em Create Link to New Node:} creates a new link with the supplied
name to the new node with the supplied name and data.
\item{\em Create Link to New Locked Node:} creates a new link with the
supplied name to the new node with the supplied name and data and locks the node.
\item{\em Write and Unlock:} writes the supplied data to the node and
unlocks the node.
\item{\em Point to Point Message:} sends the message to the supplied user.
\item{\em Change Password:} changes the user's password.
\end{itemizenoindent}


