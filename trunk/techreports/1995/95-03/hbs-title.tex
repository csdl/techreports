%%%%%%%%%%%%%%%%%%%%%%%%%%%%%% -*- Mode: Latex -*- %%%%%%%%%%%%%%%%%%%%%%%%%%%%
%% hbs-title.tex -- 
%% Author          : Carleton Moore
%% Created On      : Fri Jan 20 16:38:03 1995
%% Last Modified By: Philip Johnson
%% Last Modified On: Thu Feb  9 12:14:39 1995
%% Status          : Unknown
%% RCS: $Id: hbs-title.tex,v 1.3 1995/02/09 22:24:38 johnson Exp $
%%%%%%%%%%%%%%%%%%%%%%%%%%%%%%%%%%%%%%%%%%%%%%%%%%%%%%%%%%%%%%%%%%%%%%%%%%%%%%%
%%   Copyright (C) 1995 University of Hawaii
%%%%%%%%%%%%%%%%%%%%%%%%%%%%%%%%%%%%%%%%%%%%%%%%%%%%%%%%%%%%%%%%%%%%%%%%%%%%%%%
%% 

\title{HBS Design Document}

\author{Carleton Moore\\ 
        Collaborative Software Development Laboratory\\
        Department of Information and Computer Sciences\\ 
        2565 The Mall\\ 
        University of Hawaii\\ 
        Honolulu, Hawaii 96822\\ 
        (808) 956-3489\\
        {\tt cmoore@hawaii.edu}}

\date{ICS/CSDL-TR-95-03\\ \today}

\maketitle
\begin{abstract}

  HBS is an 11 KLOC Hypertext Multiuser Database Server written in
  C++. HBS is designed to work with ECS clients, as part of the Egret
  client-server system.  HBS is broken down into four blocks, File
  Operations, Basic Hypertext Operations, Events and Locks, and
  Client/Server Operations.  There is also a built in debugging mechanism
  and memory leak detection system.  This document describes the internal
  design of HBS.  

\end{abstract}

\newpage
\tableofcontents
\newpage