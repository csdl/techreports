%%%%%%%%%%%%%%%%%%%%%%%%%%%%%% -*- Mode: Latex -*- %%%%%%%%%%%%%%%%%%%%%%%%%%%%
%% project-background.tex -- 
%% Author          : Philip Johnson
%% Created On      : Tue Mar 31 11:44:58 2009
%% Last Modified By: Philip Johnson
%% Last Modified On: Wed Dec 16 15:29:18 2009
%% RCS: $Id$
%%%%%%%%%%%%%%%%%%%%%%%%%%%%%%%%%%%%%%%%%%%%%%%%%%%%%%%%%%%%%%%%%%%%%%%%%%%%%%%
%%   Copyright (C) 2009 
%%%%%%%%%%%%%%%%%%%%%%%%%%%%%%%%%%%%%%%%%%%%%%%%%%%%%%%%%%%%%%%%%%%%%%%%%%%%%%%
%% 

\section*{Background and significance}

% {\em The proposal must provide adequate background for reviewers to judge the novelty, uniqueness, and significance of the proposed research.}

The OPQ research project involves the development of three basic components: power quality monitoring hardware, a software service for aggregating the data, and analytics to make the data useful. 

\subsection*{Power quality monitoring hardware}

There is wide variety in the capability, cost, and user community for power quality monitoring devices.

A phasor measurement unit (PMU) captures measurements of voltage or current at a rate of 30-60 Hz, and uses GPS to ensure that the timestamps recorded between different PMUs are accurate to within 1 microsecond \cite{Zhang2007}. PMUs are typically three phase, cost approximately \$80,000. The PMU community
consists of utilities 
who install and maintain PMUs at substations or generation plants in order to assess their grid's stability.

The Wide-Area Frequency Monitoring Network (FNET) is a project by researchers at Virginia Tech based upon a GPS-synchronized single-phase ``frequency disturbance recorder'' (FDR) that can be installed at ordinary 120V outlets \cite{Zhang2010}. Currently, FNET gathers frequency data from approximately 80 FDRs installed across North America.  By monitoring changes in frequency, FNET can detect generator trips (which cause a decline in frequency) and load shedding (which cause an increase in frequency). Because the geographical location of each FDR is known, and because the timestamps are synchronized, FNET can be used to estimate both the size and location of such events. The FNET user community is members of a consortium that consists of utilities, power companies, and government groups, and supports FNET operation and research.  The cost of obtaining an FDR is not publicly available.

In summary, PMUs and FDRs are both wide-area, distributed devices for monitoring power quality at the grid-level through synchronized, time-stamped data. PMUs monitor voltage and current, while FDRs monitor frequency.  The user communities for both tend to be utilities and large power technology companies. 

Industrial manufacturing companies form a different user community for power quality monitoring. These companies are not concerned with overall grid quality but only with the quality of the power at their building.  For example, devices such as PQube \cite{pqube} connect to AC power and can collect data including voltage, frequency, THD, VARs, and harmonics. PQube data is highly accurate and each device comes with an NIST calibration certificate. A single PQube device can cost over \$5,000.   Companies such as Fluke and Tektronix sell devices for diagnosing power quality problems in industrial or laboratory settings. 

Residential consumers form a relatively unexplored user community for power quality monitoring.   One of the few commercial products for this user community is the AC Scout \cite{acscout}.  This device plugs into 120V power outlets and can monitor voltage and frequency.  Depending upon the model, the AC Scout costs between \$200 and \$400. 









\subsection*{Crowdsourced data}

Brief introduction to crowdsourced data, then move to brief overview of our application. 

include privacy issues.

open source

Why engineers need to grow a long tail;citizen engineers:

computational intelligent data analysis

\subsection*{Analytics}

What kinds of analytics are currently provided by power quality meters, and what kinds of analytics we want to try to provide.   

Can we make claims that our approach will yield superior results due to crowdsourcing? 

Penn Markham at Virginia Tech has done research on how to apply machine learning techniques (cluster analysis, classifiers) to FNET data \cite{Markham2012}. 



