\chapter{Future Directions}

Based on the related work previously summarized, building a prototype of PET seems entirely feasible. One of PET's distinguishing characteristics is the assembly of carbon emissions data from different domains of personal consumption. A basic prototype demonstrating the collection of data from different domains could collect electrical usage data from the TED 1001 electricity meter using the USB interface (see \autoref{sec:whole-home-meters}), collect air travel information from Dopplr using the web API, and collect driving information using manual data entry to a simple web application (accessible from mobile devices).

The sensor data could be stored either in Pachube or sensorbase.org to obviate the need to build the data storage infrastructure. Converting the sensor data into a carbon footprint can be done using AMEE, which appears to be tailor-made for this type of application. AMEE can store a profile for each user that includes the aggregated sensor data, easing the design of the web application that will bring it all together. Initially, the web application can simply display the contribution to the user's carbon footprint from each of the different domains, with more interesting analyses to follow.

Beyond the prototype, I am not yet sure what ``secret sauce'' will distinguish PET from other systems in this area. Combining data from different domains appears unique now, but it also seems like an obvious extension to other systems. Other ideas on what PET's research contribution could be to:

\begin{itemize}
	\item Provide a system that recommends actions to reduce the user's footprint in a more intelligent manner. Once the system has the user's sensor data, it should be able to make suggestions that are most relevant to that user's situation.
	\item Provide users with more advanced goal settings to motivate users to reduce their footprint. Personal Kyoto's Personal Kyoto Goal (see \autoref{sec:personal-kyoto}) provides a goal that is grounded in an actual treaty, which gives it more weight than some random value set by the user. Providing a way to tie goals back to the science of climate change would give them more legitimacy, and hopefully motivate users to meet them.
	\item Create a novel visualization of the carbon footprint data that makes users more aware of the information. There are many different ideas being explored in this area, it would be nice to make a contribution beyond a pie chart. Perhaps the ICU metaphor being employed by Hackystat could also be applied to carbon footprint data.
\end{itemize}

Deciding on the secret sauce and designing an experiment to test my hypotheses are the other major topics to be explored in the future.