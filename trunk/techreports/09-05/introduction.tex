\chapter{Introduction}
Climate change is one of the most serious issues confronting humankind (see \autoref{climate-change} for more details), and I would like my research to make a difference in this area. Specifically, I would like to use information technology to help people to better understand their environmental footprint, based on real data from their daily activities, and help them to evaluate how they can reduce that footprint.

While there are a variety of ways one can reduce one's carbon footprint, efficiency in energy and fuel use is a technique that can reduce consumption regardless of what sources of renewable energy turn out to be most effective in replacing fossil fuel sources. Information technology is also particularly suited to helping people to track and reduce their consumption.

To this end, I have engaged in a review of the literature surrounding this topic. My nascent topic touches on a diverse set of literature:
\begin{itemize}
	\item Climate change itself
	\item Economic issues regarding the impact of energy efficiency improvements
	\item Energy feedback systems
	\item Motivating and persuading users to change their behavior
	\item The design of environmentally persuasive systems
	\item Related systems
	\item Sensors for the collection of personal energy consumption data
	\item Calculation of carbon emissions based on sensor data
\end{itemize}

\section{The Personal Environmental Tracker}
\label{PET-description}
While a formal research proposal will be forthcoming at a later date, my current idea revolves around the concept of a Personal Environmental Tracker (PET). I envision PET as a system consisting of sensors that collect data such as home electricity or gasoline usage and send it to a database for analysis and presentation to the user. By collecting data from diverse sources, PET can help users decide what aspect of their lives they should make changes in first to maximize their reduction in environmental impact. PET's open architecture will allow other ubiquitous sustainability researchers to leverage the infrastructure for research in sensors, data analysis, or presentation of data. More details on the PET idea can be found in my position paper presented at a workshop at Ubicomp 2008 \cite{csdl2-08-01} and a presentation given to ICS 690 \cite{Brewer2008-PET-presentation}. This literature review seeks to describe and evaluate systems related to PET, and the topics listed previously.