%%%%%%%%%%%%%%%%%%%%%%%%%%%%%% -*- Mode: Latex -*- %%%%%%%%%%%%%%%%%%%%%%%%%%%%
%% summary.tex -- 
%% Author          : Philip Johnson
%% Created On      : Tue Mar 31 11:42:10 2009
%% Last Modified By: Philip Johnson
%% Last Modified On: Wed Feb  1 06:57:35 2012
%%%%%%%%%%%%%%%%%%%%%%%%%%%%%%%%%%%%%%%%%%%%%%%%%%%%%%%%%%%%%%%%%%%%%%%%%%%%%%%

\documentclass{proposalnsf}
\usepackage[final]{graphicx}

% NSF proposal generation template style file.
% based on latex stylefiles  written by Stefan Llewellyn Smith and
% Sarah Gille, with contributions from other collaborators.

% Fix things so that figures tend to stay away from the last page. 
\renewcommand{\topfraction}{0.85}
\renewcommand{\textfraction}{0.1}
\renewcommand{\floatpagefraction}{0.75}

% this handles hanging indents for publications
\def\rrr#1\\{\par
\medskip\hbox{\vbox{\parindent=2em\hsize=6.12in
\hangindent=4em\hangafter=1#1}}}

\def\baselinestretch{1}

\begin{document}

\section*{Project Summary}
\renewcommand{\thepage} {A--\arabic{page}}

%% {\em The proposal must contain a summary of the proposed activity suitable for
%% publication, not more than one page in length. It should not be an abstract
%% of the proposal, but rather a self-contained description of the activity
%% that would result if the proposal were funded. The summary should be
%% written in the third person and include a statement of objectives and
%% methods to be employed. It must clearly address in separate statements
%% (within the one-page summary):

%% (1) the intellectual merit of the proposed activity; and

%% (2)the broader impacts resulting from the proposed activity. 

%% It should be informative to other persons working in the same or related
%% fields and, insofar as possible, understandable to a scientifically or
%% technically literate lay reader. Proposals that do not separately address
%% both merit review criteria within the one-page Project Summary will be
%% returned without review.

% What is the intellectual merit of the proposed activity?
% How important is the proposed activity to advancing knowledge and
% understanding within its own field or across different fields? How well
% qualified is the proposer (individual or team) to conduct the project? (If
% appropriate, the reviewer will comment on the quality of the prior work.)
% To what extent does the proposed activity suggest and explore creative,
% original, or potentially transformative concepts? How well conceived and
% organized is the proposed activity? Is there sufficient access to
% resources?

% What are the broader impacts of the proposed activity?
% How well does the activity advance discovery and understanding while
% promoting teaching, training, and learning? How well does the proposed
% activity broaden the participation of underrepresented groups (e.g.,
% gender, ethnicity, disability, geographic, etc.)? To what extent will it
% enhance the infrastructure for research and education, such as facilities,
% instrumentation, networks, and partnerships? Will the results be
% disseminated broadly to enhance scientific and technological understanding?
% What may be the benefits of the proposed activity to society? 

% NSF staff also will give careful consideration to the following in making
% funding decisions:

% Integration of Research and Education
% One of the principal strategies in support of NSF's goals is to foster
% integration of research and education through the programs, projects, and
% activities it supports at academic and research institutions. These
% institutions provide abundant opportunities where individuals may
% concurrently assume responsibilities as researchers, educators, and
% students and where all can engage in joint efforts that infuse education
% with the excitement of discovery and enrich research through the diversity
% of learning perspectives.


% Integrating Diversity into NSF Programs, Projects, and Activities
% Broadening opportunities and enabling the participation of all citizens --
% women and men, underrepresented minorities, and persons with disabilities
% -- is essential to the health and vitality of science and engineering. NSF
% is committed to this principle of diversity and deems it central to the
% programs, projects, and activities it considers and supports.

% Edits by M Fripp 1/31/12: 
% - removed one-word summaries of research areas from first paragraph, since they're covered in the second
% - moved general descriptions of the project to the start of the second paragraph and concentrated descriptions of the research components in the second half
% - removed some repeated text from the second paragraph
% - added a summary sentence and national implications to the third paragraph

% Edits by M Fripp 2/1/12: 
% - removed references to "testbed" and "testing" from the first and second paragraphs
% - spelled out the first mention of "University of Hawaii, Manoa" and converted to "UHM" elsewhere
% - minor changes to wording in various places
% - incorporated (and edited) many of Dora Yen's changes (fairly major)

\noindent The state of Hawaii is more dependent on oil than any other state in the nation, using it for most electricity generation as well as transportation. The state-sponsored Hawaii Clean Energy Initiative calls for Hawaii to sharply reduce this dependence, obtaining 70\% of its energy from clean energy sources by 2030. The University of Hawaii is playing a major role in this effort by conducting research, education, and workforce training in energy and sustainability. The project considers both theoretical and practical aspects of response, control and status on a local, interconnected sub-system of the grid and elucidates its behavior when distributed renewable energy sources are added. The result will be a {\em smart, sustainable microgrid}. Four interlinked research projects will be integrated into a graduate and undergraduate education program on smart grids, renewable energy, and energy efficiency. 

\noindent {\bf Intellectual Merit.}  This project explores the potential of future power distribution by designing, building and evaluating a microgrid that offsets local energy usage using distributed generation (i.e. rooftop PV). This project will transform the UHM campus's electrical distribution network into a more modern, intelligent and responsive microgrid which relies on a greater share of locally generated energy and energy efficient practices. The proposal includes four research components, each of which integrates graduate and undergraduate education and engages industry partners. 
The first component assembles all relevant information about the state of the microgrid and surrounding environment. A central server and storage system will hold environmental resource data, electrical grid data and building energy data gathered in real time, as well as the results of energy-use surveys of students, faculty, and staff.  Important issues include resource allocation (sensor placement), data fusion and mining. A second component models and analyzes the microgrid using probabilistic signal processing models. A key consideration is whether distributed secure algorithms can accurately predict the state of the microgrid and detect anomalous events. Once accurate models are constructed and verified through simulations and data gathering, the third project provides decision-making capability for the microgrid. This uses distributed control and optimization algorithms to create a secure and stable microgrid which reduces costs and energy use. An important consideration in achieving these goals is helping consumers (students, faculty, and staff) to understand and assist in the optimal operation of the microgrid. The fourth project will consider social, economic, and policy implications of this transition. Students and faculty will work in a newly created smart campus energy lab (SCEL) which will house the data and networking components of the UHM campus microgrid.
 
\noindent{\bf Broader Impacts.}  By developing an operational structure analogous to the larger utility grid, the campus microgrid can be instrumented and serve as a similar platform for gathering field data on load, generation, local grid quality and usage patterning.  Using this platform, students, faculty and partners can immediately advance their knowledge on the challenges of operating, monitoring and assessing the conditions of a modern grid.  UHM graduate and undergraduate students will be able to combine what they learn in the classroom with theoretical and practical research projects in the SCEL and on the UHM campus microgrid. This understanding and practical knowledge will lead to better insight on the design and integration of new technology for the microgrid and future utility-scale smart grids.  Through collaborations with the Native Hawaiian Science and Engineering Mentorship Program (NHSEMP) and the Society of Women Engineers (SWE), the project will also target underrepresented students. This project is unique in providing industry and students with a common platform on which to assess, learn and collaborate.  Students and campus staff can gain insight on the challenges and issues of managing a grid and Hawaii's electricity industry can help focus research as well as actively learning and creatively exploring options through new simulation models.  The UHM campus and the State of Hawaii are highly motivated to adopt renewable energy alternatives.  This project will reveal how that can be done, helping show the path to a clean energy future for both Hawaii and the rest of the nation.

\medskip

\noindent {\bf Key Words:} energy, sensing, modeling, control, renewable. Social, economic,
and policy implications.
\end{document}




