%%%%%%%%%%%%%%%%%%%%%%%%%%%%%% -*- Mode: Latex -*- %%%%%%%%%%%%%%%%%%%%%%%%%%%%
%% project.social.tex -- 
%% Author          : Philip Johnson
%% Created On      : Fri Jan 13 07:58:21 2012
%% Last Modified By: Philip Johnson
%% Last Modified On: Thu Jan 26 15:30:30 2012
%%%%%%%%%%%%%%%%%%%%%%%%%%%%%%%%%%%%%%%%%%%%%%%%%%%%%%%%%%%%%%%%%%%%%%%%%%%%%%%

\subsubsection{Research component: Social, economic, privacy, security, and policy implications}
\label{sec:social}

As discussed in the introduction, one of the key properties of the Smart
Grid according to the Department of Energy is to ``enable active
participation by consumers''.  We believe that this is especially important
for our smart, sustainable microgrid, and that it would be naive to assume
it could be implemented transparently and invisibly to its users.  Many
people are used to virtually unlimited, low-cost, and reliable electrical
energy produced in an unsustainable, environmentally harmful manner and do
not yet understand why this cannot continue.

In this research component, we will investigate the social, economic,
privacy, security, and policy implications that arise during the transition
to a smart, sustainable microgrid.  We intend this research to produce
insight into how to best engage citizens in the process of weighing the
trade-offs associated with microgrid development, as well as how to best
enable them to become active participants in microgrid management. Active
participation has the potential to create efficiencies not possible through
automated techniques alone, but this will require providing users with new
forms of information along with the incentives and engagement necessary for
users to act on this information in a timely, effective, and positive
manner.  Finally, this research component will investigate what kinds of
data must be gathered in order to support broader policy decisions by local
government that can make future smart, sustainable microgrids easier to
develop and deploy.

\paragraph{Social and policy issues.}

There is a growing body of research regarding social engagement in energy
use and conservation
\cite{Hargreaves10,Stromback11,Darby06,Allcott11,Darby11,Hargreaves10,Faruqui10,
  Herter07}.  Engagement can range from passive (e.g., tolerating automated
building temperature adjustments) to active (e.g., scheduling
energy-intensive research for off-peak times).  Much of the prior research
on user behavior has focused on residential environments or other
circumstances where direct financial incentives apply.  In contrast, we
will need to actively engage users in energy conservation and management in
an environment where they have only an indirect financial stake in the
outcome.  Fortunately, we have already established a research program
called the Kukui Cup which investigates the use of real-time feedback,
incentives, education, and game mechanics to obtain sustained, positive
changes in energy behaviors among dorm residents
\cite{csdl2-11-03,csdl2-11-02}. In the Kukui Cup, direct financial
incentives also do not apply.

Our strategy for social engagement will include several elements. The first
level will be presenting users with information on concrete measures they
can take (long-term and hour-by-hour) to help the campus achieve cost
savings and greater usage of renewable resources. As an additional
incentive, we will study socially oriented ways of motivating actions that
help the campus, building on techniques we developed for the Kukui Cup. We
will also develop new interfaces such as smart phone apps to replace the
in-home display typically used in residential demand response programs. We
will design these interfaces to engage many different types of user and
help them learn successively more about the power system and their role
within it \cite{Stromback11}. This effort will also benefit from
collaboration with smart metering researchers in Oxford University's Lower
Carbon Futures group (support letter attached).

We will carefully analyze the ways in which government and university
policies act to promote or inhibit the successful development of the
microgrid.  Policy analysis must occur across a very wide spectrum. At one
end, building code policies can influence the ease of sensor installation,
use, and subsequent control capabilities.  At the other end, energy data access
policies can influence whether users can effectively participate in grid
management. 

\paragraph{Privacy and security issues.}

Enabling active participation by consumers is made difficult by the fact
that the collection, analysis, and usage of power and environmental data
within a microgrid creates significant new security and privacy
considerations.  The canonical security concern is the possibility of
unauthorized agents infiltrating the microgrid network and becoming capable
of injecting control signals into the microgrid leading to power and/or
equipment failure.  The canonical privacy concern is the possibility of
unauthorized agents infiltrating the microgrid network and using its data
to gain insight into the behaviors of campus members and organizations,
enabling them to better plan and execute robberies or other illegal
activities.

Smart grid security and privacy issues can be organized into categories
corresponding to primary grid components: the PCS system, smart meters,
power system state estimation, smart grid communication protocol, and smart
grid simulation for security analysis. 

The most common PCS system is SCADA.  Traditional PCS 
systems are designed with a defense-in-depth, firewall network protection scheme.   
Adding  more security to PCS systems is complicated due to
their real-time, low latency (sub-second), and high availability
requirements \cite{Valdes2009}. 

Smart meters provide fine-grained, near-real time information about power
use within a building or other component of the campus.  Security issues
include tampering with the smart meter readings to effectively ``steal''
power. Privacy issues include using the fine-grained data to infer the
behaviors of occupants. Berthier, Sanders, and Khurana have proposed a
comprehensive set of security tools for smart meters \cite{Berthier2010}.

Maintaining the integrity of the grid requires power system state
estimation, and yields a security risk of attackers injecting false data
into the model to create system instability or for financial gain
\cite{Xie2010}.  Allocating the processing overhead necessary to
distinguish false from real data is problematic due to the
high-availability and low-latency (sub-second) requirements for this
component. Some research has been done on how many compromised sources are
required to carry out an unobservable attack \cite{Kosut2010}.

Network communication and their associated protocols is the backbone of the
smart grid, and the security and privacy issues are diverse and dependent
upon the nature of the protocols and the types of information that are
being communicated. A particularly difficult issue is the need to interface
with legacy systems which were not designed with support for security.
Khurana et. al. have proposed a set of design guidelines for smart grid
protocols to reduce the number of vulnerabilities \cite{Khurana2010}.

Finally, smart grids cannot be taken down for testing, and so simulation
systems are used for testing instead.  While traditional grid simulation
systems are focused on availability and stability concerns, smart grids
will require these systems to support security and privacy assessment as
well. Kundar et al have begun work on a framework that provides initial
progress toward smart grid security analysis through simulation
\cite{Kundur2010}.

A contribution of this research project will be the evaluation of these
techniques in the context of microgrid design and implementation and
insights into enhanced privacy and security based upon industry standards and campus ITS requirements.

\paragraph{Economic issues.}

The final area of investigation for this research component involves the economic
implications of the microgrid.  Our principal focus in this area will be to
determine how effectively the microgrid can serve to decrease the overall
cost of energy supplied by the local utility.  Currently, the University
electrical rates for a given month are primarily a function of two
variables: the peak demand by the University during that month, and the
peak rate of increase in demand (ramp) during the month \cite{Hafner2011}.

If the microgrid design is effective, then we should be able to lower peak
demand in the following ways: (1) by integrating solar generation (which
reduced demand from the utility); (2) though automated demand response
(which, when combined with adequate prediction, should enable the system to
shut down chillers in advance of periods of high load, reducing peak
requirements), and (3) through customer-facing user interfaces, which could
inform campus members of periods of high loads and enable those with
discretionary power loads to shift them in time to periods of lighter
overall demand.
We will also be able to evaluate the ability of our prediction algorithms
to reduce the rate of increase in demand. 

By the conclusion of the project, we will be able to produce comprehensive
cost-benefit analyses of the investment required to create the sustainable,
smart microgrid and the economic benefits that accrued from its
implementation and use.  We will provide an accounting for the savings from
generation, automated demand-response, and user behavioral change.




  


