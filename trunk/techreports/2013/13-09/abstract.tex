\begin{abstract}

Sustainability education and conservation have become an international imperative due to the rising cost of energy, increasing scarcity of natural resource and irresponsible environmental practices. Over the past decade, running energy and water challenges has become a focal point for sustainability efforts at both university and industry campuses. For example, there are more than 160 college residence hall energy competitions taking place or being planned for the 2010--2011 academic year in North America~\cite{Hodge2010} to engaging students in sustainability issues.
Designers of such challenges typically have three choices for information technology: (a) build their own custom in-house solution (as was done at Oberlin College in 2006 \cite{petersen-dorm-energy-reduction}); (b) out-source to a commercial provider (as was done at the University of British Columbia in 2011); or (c) use a minimal tech solution such as a web page and manual posting of data and results (as was done at Harvard in 2012).

None of these choices are ideal: the custom in-house solution requires sophisticated design and implementation skills; out-sourcing can be financially expensive and impedes evolution; and the minimal tech solution does not fully leverage the possibilities of advanced information technology.

To provide a better alternative to these three choices, I have led an effort over the past year to design and implement an open source serious game engine for sustainability called Makahiki. Makahiki implements an extensible framework with a variety of common services for developing sustainability games including authentication; game mechanics such as leaderboards, points, and badges; a variety of built-in games and content focused in sustainability, a responsive user interface, cloud-based deployment, and the ability to customize to the needs of individual organizations.

Makahiki lowers the overhead to those who would build a custom in-house solution by providing pre-built components. It can lower the financial cost to those who would out-source by providing an open source alternative. Finally, it provides an opportunity for those who would choose a minimal tech solution to instead provide more sophisticated information technology.

To provide initial evidence regarding the ability of the Makahiki Framework to support sustainability games in different environments, we ran challenges at three organizations in Fall 2012: The University of Hawaii, Hawaii Pacific Univerisity, and the East-West Center. While these experiences provided anecdotal evidence for the usefulness of Makahiki, we realized that a more rigorous evaluation of the framework would yield better quality insight into its current quality and requirements for future enhancement.

Upon review of the literature, we found little research or experience with formal framework assessment. To address this, I have embarked on research to design an assessment mechanism for serious game frameworks, called Serious Game Stakeholder Experience Assessment Method (SGSEAM). SGSEAM is designed to provide detailed insight into the strengths and weaknesses of a serious game framework through a stakeholder perspective based approach. In my research, I applied SGSEAM to Makahiki in order to gain better insight into its strengths and weaknesses as a serious game framework. 

The contributions of my research thus includes: the Makahiki framework for serious games for sustainability; the SGSEAM assessment method, the insights into serious game framework design generated through application of SGSEAM to both Makahiki and another serious game framework, and the insights into framework assessment design in general resulting from the above. I hope this research will be of interest to researchers and practitioners across several disciplines: software engineering, game designers, and sustainability researchers.
\end{abstract}