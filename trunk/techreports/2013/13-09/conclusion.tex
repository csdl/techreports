\chapter{Conclusion}
\label{cha:conclusion}

This dissertation investigated the design, implementation, and evaluation of the serious game framework for sustainability called Makahiki and a stakeholder experience based assessment method for serious game framework. This chapter summarizes the results of the research, the contributions of the research, and possible future directions.

\section{Research Summary}
This research investigates the information technology infrastructure that can support effective and efficient development of serious games for sustainability. The research includes the development of an innovative serious game framework for sustainability that combining education and behavior change, and an assessment method accessing the effectiveness and efficiency of the IT infrastructure for serious games for sustainability with regarding the most important stakeholder's perspective.

%% TODO. more on SGSEAM

%% TODO. more on result of MAKAHIKI applications on several kukui cup

%% TODO. more on SGSEAM on makahiki result

\section{Contributions}

The contributions of this research are:

%% TODO: expand on contribution

\begin{itemize}
	\item Makahiki: open source information technology for development of serious games for sustainability.
	\item SGSEAM: an assessment method for assessing serious game framework.
	\item Evidence regarding the effectiveness and efficiency of Makahiki as a framework for development of serious games for sustainability.
	\item Evidence regarding the effectiveness and efficiency of a second system (BuildingOS) as a framework for development of serious games for sustainability.
	\item Insights into the strengths and weaknesses of the assessment method.
\end{itemize}

\section{Future Directions}

There are a variety of directions that can be pursued once this research is complete. One of them is the evaluation of the SGSEAM itself. The design of SGSEAM creates a research question of what are the strengths and weaknesses of this assessment method. 
To better answer this question, SGSEAM should be applied to another serious game development environment. BuildingOS\cite{building-dashboard} by Lucid Design Group is such a serious game framework that is suitable for SGSEAM evaluation. Our research lab had made the effort to contact Lucid Design group for the collaboration. I created the assessment plan for them which hope to minimize the effort spent from their side. But due to the workload of the Lucid design group, which is still a newly found startup company, the collaboration did not continue. A further evaluation of SGSEAM by applying to another serious game framework is still an ongoing research direction.

Other future direction of this research includes:
\begin{itemize}
	\item Evaluate the other stakeholders’ experiences

    \item Build a community to expand content and game library for Makahiki

    \item Scale and expand Makahiki to support other geographical and cultural different locations.

\end{itemize}

