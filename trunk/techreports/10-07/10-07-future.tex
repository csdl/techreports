%%%%%%%%%%%%%%%%%%%%%%%%%%%%%% -*- Mode: Latex -*- %%%%%%%%%%%%%%%%%%%%%%%%%%%%
%% 10-07-future.tex --  HICSS 44 Kukui Cup paper
%% Author          : Philip Johnson
%% Created On      : Mon Sep 23 11:52:28 2002
%% Last Modified By: Philip Johnson
%% Last Modified On: Mon Jun 14 12:55:47 2010
%%%%%%%%%%%%%%%%%%%%%%%%%%%%%%%%%%%%%%%%%%%%%%%%%%%%%%%%%%%%%%%%%%%%%%%%%%%%%%%
%%   Copyright (C) 2009 Philip Johnson
%%%%%%%%%%%%%%%%%%%%%%%%%%%%%%%%%%%%%%%%%%%%%%%%%%%%%%%%%%%%%%%%%%%%%%%%%%%%%%%
%% 

\section{Future Directions}
\label{sec:future-directions}

As noted above, we are currently planning for the inaugural Kukui Cup
competition to take place in the Spring 2011 semester.  In addition, we are
planning several other extensions to this research.

First, we are interested in adapting this technology and research approach
to the issue of residential energy consumption.  Residential energy
consumption differs in many ways from dorm energy consumption. Most
significantly, residential energy users are billed directly for the energy
use, so there is an important financial incentive for conservation.
However, the research indicates that tools such as goals and public
commitments remain important for sustained behavioral change in the residential
energy application domain. When dealing with energy data for individual homes,
ensuring the privacy of that data becomes critical because activity and
occupancy can be inferred from energy usage data. To address privacy concerns,
detailed data must only be displayed to the residents of a home and others
explicitly designated (friends, perhaps close neighbors). For all other viewers,
the energy data must be anonymized and aggregated. Most residential users would
also lack the close ties that a floor of college dormitory residents may
possess. One solution may be to create virtual teams based on existing social
networks, as specified on services like Facebook.

Second, we would like to support the use of Makahiki as a framework for
other universities who desire to implement a dorm energy competition.  We
believe that Makahiki can significantly lower the ``barrier to entry'' for
universities, and allows them to get started with a simple site if required. 

Third, we are excited by the possibility of integrating grid-level
information into WattDepot for use by Makahiki and other high level
applications.  For example, we are working with our local utility to
provide WattDepot with information on the current mix of generation sources
(i.e. coal, oil, solar, wind, etc.) and their aggregate output
(i.e. grid load).  Given this information, it would be possible for
WattDepot to provide the carbon intensity associated with the energy
consumed by end-users at any point in time.  This would make it possible
for the Kukui Cup to incentivize behaviors such as ``doing laundry when the
carbon intensity of the grid is low.''





