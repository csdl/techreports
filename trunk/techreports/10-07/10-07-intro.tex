%%%%%%%%%%%%%%%%%%%%%%%%%%%%%% -*- Mode: Latex -*- %%%%%%%%%%%%%%%%%%%%%%%%%%%%
%% 10-07-intro.tex --  HICSS 44 Kukui Cup paper
%% Author          : Philip Johnson
%% Created On      : Mon Sep 23 11:52:28 2002
%% Last Modified By: Philip Johnson
%% Last Modified On: Thu Jun 10 15:04:05 2010
%%%%%%%%%%%%%%%%%%%%%%%%%%%%%%%%%%%%%%%%%%%%%%%%%%%%%%%%%%%%%%%%%%%%%%%%%%%%%%%
%%   Copyright (C) 2009 Philip Johnson
%%%%%%%%%%%%%%%%%%%%%%%%%%%%%%%%%%%%%%%%%%%%%%%%%%%%%%%%%%%%%%%%%%%%%%%%%%%%%%%
%% 

\section{Introduction}
\label{sec:intro}

Dorm energy competitions are becoming an increasingly popular event; over
two dozen universities were listed in an online reference guide.  Dorm
energy competitions have many desirable properties: they appear to be
reliably successful at reducing energy usage during the competition, they
help foster community in the dorms, they create opportunities for
education, they enhance ecological awareness, and they are generally
perceived as fun by the residents.  To the extent that energy reductions
are achieved, they reduce the carbon footprint associated with the dorm,
and of course the electricity cost to the university. One hopes that the
lessons learned from a dorm energy competition carry over into the
student's life post-competition and even post-dormitory living.

Dorm energy competitions also provide a useful setting for research
regarding behavioral aspects of energy usage.  First, dorm residents are a
``renewable resource'': regular turn-over in occupancy means that it is
possible to embed experimental designs into the energy competition
structure and perform partial ``replications'' of the study each year.
Second, dorm residents are an interesting group to study because the
typical incentive for behavioral change, financial savings, is not present
in this group. Dorm residents normally pay a flat rate that does not change
regardless of their energy consumption.  Finally, dorm energy competitions
can easily involve hundreds of students, providing a relatively large
subject pool for analysis.

Despite this potential, very few dorm energy competitions have formed the
basis for behavioral research.  In reviewing the information available
regarding dorm energy competitions, we find that in most cases, very little
data is systematically collected, and fundamental questions regarding the
competition (such as whether consumption returned to its post-competition
level or not) go unanswered.  In some cases, this is perhaps due to the
student-led nature of the competition, where resources and expertise did
not extend to the additional level of structure and analysis required to
support a research focus.

In this paper, we report on the design and implementation of a dorm energy
competition to be held at the University of Hawaii called the Kukui Cup.
Like other dorm energy competitions, we intend this competition to foster
community, provide opportunities for education, enhance ecological
awareness, and provide fun for the residents.  Unlike other competitions,
we are designing this competition not only to address interesting research
questions regarding behavioral change with respect to energy usage, but
also to produce new technology to aid the research community, whether they
wish to carry out a dorm energy competition or energy research in some
other domain. 

We designed our approach to address three basic research questions.  First,
to what extent and in what ways does our dorm energy competition improve
the ``energy literacy'' of participating students?  Second, how effective
is our use of information technology to support behavioral change tools
including goals, commitments, and near real-time energy feedback? Third, to
what extent does our approach yield sustained changes in energy behavior,
and what factors appear to influence sustained change?

In addition to these research questions, our approach also involves the
design and implementation of two general purpose software systems.
WattDepot is a generic framework for enterprise-level energy data
collection, storage, analysis, and presentation.  It is useful not just in
the context of dorm energy competitions, but to a variety of organizations
that need to collect data from dozens to hundreds of sources and store and
analyze the results.  Makahiki is a generic framework for dorm energy
competitions.  It is useful not just for the University of Hawaii Kukui
Cup, but can be adapted to support the needs of other universities who want
information technology that can be configured to the needs of their
environment.

The remainder of this paper provides details on our approach.  Section
\ref{sec:related-work} briefly overviews the relevant literature.  Section
\ref{sec:system-design} presents the design of the WattDepot and Makahiki
systems and how they work together to support the needs of the Kukui Cup
dorm energy competition.  Section \ref{sec:experimental-design} provides details
on how the competition will facilitate investigation into our research questions. 
Section \ref{sec:future-directions} presents the current status of our work and promising future
directions. 

