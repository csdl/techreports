%%%%%%%%%%%%%%%%%%%%%%%%%%%%%% -*- Mode: Latex -*- %%%%%%%%%%%%%%%%%%%%%%%%%%%%
%% 10-07.tex --  HICSS 44 Kukui Cup paper
%% Author          : Philip Johnson
%% Created On      : Mon Sep 23 11:52:28 2002
%% Last Modified By: Philip Johnson
%% Last Modified On: Mon Jun 14 12:41:23 2010
%%%%%%%%%%%%%%%%%%%%%%%%%%%%%%%%%%%%%%%%%%%%%%%%%%%%%%%%%%%%%%%%%%%%%%%%%%%%%%%
%%   Copyright (C) 2009 Philip Johnson
%%%%%%%%%%%%%%%%%%%%%%%%%%%%%%%%%%%%%%%%%%%%%%%%%%%%%%%%%%%%%%%%%%%%%%%%%%%%%%%
%% 

%% Home page: http://www.hicss.hawaii.edu/hicss_44/authorinstruction.htm

%% Must submit to one of the minitracks:
%% http://www.hicss.hawaii.edu/hicss_44/apahome44.htm

%% It appears that this is the most relevant one:
%% http://www.hicss.hawaii.edu/hicss_44/Minitracks44/dt-infosys.pdf

%% For ``peer review mode'', do:
%%   \documentclass[conference,compsoc,peerreview]{IEEEtran}
%% and
%%   \IEEEpeerreviewmaketitle  (after the abstract).

%% Computer Society style (apparently not the one we should use?)
%\documentclass[conference,compsoc,peerreview]{IEEEtran}
%% Style used for initial submission
\documentclass[conference,peerreview]{IEEEtran}
%% CSDL Tech report version
%\documentclass[conference]{IEEEtran}
\usepackage[final]{graphicx}
\usepackage{cite}
\usepackage{url}
% uncomment the % away on next line to produce the final camera-ready version
% and uncomment the \thispagestyle{empty} following \maketitle
%\pagestyle{empty}

\begin{document}

\title{The Kukui Cup: a Dorm Energy Competition Focused on Sustainable Behavior Change and Energy Literacy}

\author{Robert S. Brewer\\
        George E. Lee \\
        Philip M. Johnson\\
\em     Collaborative Software Development Laboratory\\
        Department of Information and Computer Sciences\\
        University of Hawai`i at M\=anoa\\
        Honolulu, HI 96822\\
        rbrewer@lava.net, gelee@hawaii.edu, johnson@hawaii.edu\\
}


%\maketitle
\IEEEpeerreviewmaketitle
%\thispagestyle{empty}

\begin{abstract}  % 150 words
The Kukui Cup is an advanced dorm energy competition whose goal is to
investigate the relationships among energy literacy, sustained energy
conservation, and information technology support of behavior change. Two
general purpose open source systems have been implemented: WattDepot and
Makahiki. WattDepot provides enterprise-level collection, storage, analysis,
and visualization of energy data. Makahiki is a web application framework that
supports dorm energy competitions of varying degrees of complexity,
including a personalized homepage where participants can complete tasks
designed to increase energy literacy that can be verified by competition
administrators. The technology and approach will be evaluated in a dorm energy
competition to take place in the Spring of 2011, with hundreds of University
freshmen. The energy use of each pair of dormitory floors will be metered in
near-realtime, and the energy literacy of participants will be assessed before
and after the competition.
\end{abstract}

%%%%%%%%%%%%%%%%%%%%%%%%%%%%%% -*- Mode: Latex -*- %%%%%%%%%%%%%%%%%%%%%%%%%%%%
%% 10-07-intro.tex --  HICSS 44 Kukui Cup paper
%% Author          : Philip Johnson
%% Created On      : Mon Sep 23 11:52:28 2002
%% Last Modified By: Philip Johnson
%% Last Modified On: Thu Jun 10 15:04:05 2010
%%%%%%%%%%%%%%%%%%%%%%%%%%%%%%%%%%%%%%%%%%%%%%%%%%%%%%%%%%%%%%%%%%%%%%%%%%%%%%%
%%   Copyright (C) 2009 Philip Johnson
%%%%%%%%%%%%%%%%%%%%%%%%%%%%%%%%%%%%%%%%%%%%%%%%%%%%%%%%%%%%%%%%%%%%%%%%%%%%%%%
%% 

\section{Introduction}
\label{sec:intro}

Dorm energy competitions are becoming an increasingly popular event; over
two dozen universities were listed in an online reference guide.  Dorm
energy competitions have many desirable properties: they appear to be
reliably successful at reducing energy usage during the competition, they
help foster community in the dorms, they create opportunities for
education, they enhance ecological awareness, and they are generally
perceived as fun by the residents.  To the extent that energy reductions
are achieved, they reduce the carbon footprint associated with the dorm,
and of course the electricity cost to the university. One hopes that the
lessons learned from a dorm energy competition carry over into the
student's life post-competition and even post-dormitory living.

Dorm energy competitions also provide a useful setting for research
regarding behavioral aspects of energy usage.  First, dorm residents are a
``renewable resource'': regular turn-over in occupancy means that it is
possible to embed experimental designs into the energy competition
structure and perform partial ``replications'' of the study each year.
Second, dorm residents are an interesting group to study because the
typical incentive for behavioral change, financial savings, is not present
in this group. Dorm residents normally pay a flat rate that does not change
regardless of their energy consumption.  Finally, dorm energy competitions
can easily involve hundreds of students, providing a relatively large
subject pool for analysis.

Despite this potential, very few dorm energy competitions have formed the
basis for behavioral research.  In reviewing the information available
regarding dorm energy competitions, we find that in most cases, very little
data is systematically collected, and fundamental questions regarding the
competition (such as whether consumption returned to its post-competition
level or not) go unanswered.  In some cases, this is perhaps due to the
student-led nature of the competition, where resources and expertise did
not extend to the additional level of structure and analysis required to
support a research focus.

In this paper, we report on the design and implementation of a dorm energy
competition to be held at the University of Hawaii called the Kukui Cup.
Like other dorm energy competitions, we intend this competition to foster
community, provide opportunities for education, enhance ecological
awareness, and provide fun for the residents.  Unlike other competitions,
we are designing this competition not only to address interesting research
questions regarding behavioral change with respect to energy usage, but
also to produce new technology to aid the research community, whether they
wish to carry out a dorm energy competition or energy research in some
other domain. 

We designed our approach to address three basic research questions.  First,
to what extent and in what ways does our dorm energy competition improve
the ``energy literacy'' of participating students?  Second, how effective
is our use of information technology to support behavioral change tools
including goals, commitments, and near real-time energy feedback? Third, to
what extent does our approach yield sustained changes in energy behavior,
and what factors appear to influence sustained change?

In addition to these research questions, our approach also involves the
design and implementation of two general purpose software systems.
WattDepot is a generic framework for enterprise-level energy data
collection, storage, analysis, and presentation.  It is useful not just in
the context of dorm energy competitions, but to a variety of organizations
that need to collect data from dozens to hundreds of sources and store and
analyze the results.  Makahiki is a generic framework for dorm energy
competitions.  It is useful not just for the University of Hawaii Kukui
Cup, but can be adapted to support the needs of other universities who want
information technology that can be configured to the needs of their
environment.

The remainder of this paper provides details on our approach.  Section
\ref{sec:related-work} briefly overviews the relevant literature.  Section
\ref{sec:system-design} presents the design of the WattDepot and Makahiki
systems and how they work together to support the needs of the Kukui Cup
dorm energy competition.  Section \ref{sec:experimental-design} provides details
on how the competition will facilitate investigation into our research questions. 
Section \ref{sec:future-directions} presents the current status of our work and promising future
directions. 


%%%%%%%%%%%%%%%%%%%%%%%%%%%%%% -*- Mode: Latex -*- %%%%%%%%%%%%%%%%%%%%%%%%%%%%
%% 10-07-related.tex --  HICSS 44 Kukui Cup paper
%% Author          : Philip Johnson
%% Created On      : Mon Sep 23 11:52:28 2002
%% Last Modified By: Philip Johnson
%% Last Modified On: Thu Jun 10 15:04:49 2010
%%%%%%%%%%%%%%%%%%%%%%%%%%%%%%%%%%%%%%%%%%%%%%%%%%%%%%%%%%%%%%%%%%%%%%%%%%%%%%%
%%   Copyright (C) 2009 Philip Johnson
%%%%%%%%%%%%%%%%%%%%%%%%%%%%%%%%%%%%%%%%%%%%%%%%%%%%%%%%%%%%%%%%%%%%%%%%%%%%%%%
%% 

\section{Related Work}
\label{sec:related-work}



%%%%%%%%%%%%%%%%%%%%%%%%%%%%%% -*- Mode: Latex -*- %%%%%%%%%%%%%%%%%%%%%%%%%%%%
%% 10-07-system.tex --  HICSS 44 Kukui Cup paper
%% Author          : Philip Johnson
%% Created On      : Mon Sep 23 11:52:28 2002
%% Last Modified By: Philip Johnson
%% Last Modified On: Thu Jun 10 16:08:09 2010
%%%%%%%%%%%%%%%%%%%%%%%%%%%%%%%%%%%%%%%%%%%%%%%%%%%%%%%%%%%%%%%%%%%%%%%%%%%%%%%
%%   Copyright (C) 2009 Philip Johnson
%%%%%%%%%%%%%%%%%%%%%%%%%%%%%%%%%%%%%%%%%%%%%%%%%%%%%%%%%%%%%%%%%%%%%%%%%%%%%%%
%% 

\section{System Design}
\label{sec:system-design}

As our related work findings illustrate, current software for energy
competitions tends to be either commercial, closed systems, or special
purpose, ``on-off'' systems.  We strive in this project to create
software with an architecture that is open, extensible, and easily tailored
to the needs of different universities.  We intend the software
infrastructure from this project to provide as much of a research
contribution as our actual experimental results.

The following general requirements inform our system design.

\noindent {\em Open source.}  To maximize the potential for community
participation in development as well as use of the software, we make
all components available as open source, and utilize only freely
available third party components for development.  There are no software
costs associated with the use of our system.

\noindent {\em Platform, language, and metering infrastructure agnostic.}
We want to avoid lock-in to any particular platform, language, or metering
technology.  To avoid platform lock-in, we develop all components using
technologies such as Java, Python, Javascript, and Google Visualizations
that are available on Windows, Macintosh, and Linux platforms.  To avoid
language lock-in, the system observes a service-oriented architecture,
where components communicate with each other over HTTP via a RESTful API.
This isolates language dependencies to individual services.  For example,
the WattDepot server is written in Java, while the Makahiki web application
is written in Python.  Finally, to avoid metering technology lock-in, the
system architecture involves ``sensors'' that query any given meter using
its native protocol, then translates that into a common format for use in
the rest of the system.  Thus, adapting the system to a new meter
technology simply involves implementing the sensor for that technology.

\noindent {\em Feature subsetting.} Not all universities need or want the same 
level of sophistication in their dorm energy competitions.  In reviewing 
other sites, we found a wide spectrum of sophistication with respect to 
the kinds of information collected and the way it is presented.  Our software
is designed to support a variety of different use cases. 












%%%%%%%%%%%%%%%%%%%%%%%%%%%%%% -*- Mode: Latex -*- %%%%%%%%%%%%%%%%%%%%%%%%%%%%
%% 10-07-experiment.tex --  HICSS 44 Kukui Cup paper
%% Author          : Philip Johnson
%% Created On      : Mon Sep 23 11:52:28 2002
%% Last Modified By: Philip Johnson
%% Last Modified On: Thu Jun 10 15:05:36 2010
%%%%%%%%%%%%%%%%%%%%%%%%%%%%%%%%%%%%%%%%%%%%%%%%%%%%%%%%%%%%%%%%%%%%%%%%%%%%%%%
%%   Copyright (C) 2009 Philip Johnson
%%%%%%%%%%%%%%%%%%%%%%%%%%%%%%%%%%%%%%%%%%%%%%%%%%%%%%%%%%%%%%%%%%%%%%%%%%%%%%%
%% 

\section{Evaluation}
\label{sec:evaluation}

Our work addresses the following research questions:

\begin{enumerate}
	\item To what extent and in what ways does our dorm energy competition impact the ``energy literacy'' of participating students?
	\item How effective is our use of information technology to support behavioral change tools?
	\item To what extent does our approach yield sustained changes in energy behavior, and what factors appear to influence sustained change?
\end{enumerate}

To find the answers to these questions, we have planned a dorm energy competition for Spring 2011 semester at the University of Hawai`i at M\=anoa. The rest of this section describes the competition design, the data we plan to gather, and how we intend to analyze the data.

\subsection{Competition design}

The competition is planned to take place in multiple freshman dormitories on the M\=anoa campus. Freshmen have been targeted since they are deemed more likely to participate in dormitory events, they are a ``renewable resource'', and past research has shown that freshmen perform well in these competitions \cite{petersen-dorm-energy-reduction}. There are 10 floors per residence hall, with 26 residents per floor at full occupancy, resulting in 260 potential participants per building.

The competition will take place over three weeks in the Spring 2011 semester. Each of the first two weeks will constitute a separate round of the competition, while the results from the final week apply only to the overall competition. Structuring the competition into rounds ensures that residents that did not participate initially can start participating in a later round without undue disadvantage.

The energy usage of the participants will be measured using power meters we will install in the central electrical panels on the floors of the dorms. Due to the architectural design and electrical infrastructure of the buildings, the meters will measure the energy consumption of each pair of floors. The meters to be installed will support sampling every 15 seconds, enabling near-realtime energy feedback display. The meter data will be stored using WattDepot.

There will be two scores for the competition: energy consumption and Kukui Nut points. Energy consumption is the total amount of electrical energy consumed by a pair of floors in kWh during a round as measured by the power meters, so lower energy consumption scores are better. Kukui Nut points are awarded through the competition website (powered by Makahiki) for the completion of tasks intended to increase participants' energy literacy or reduce energy usage. Kukui Nut points are awarded to individuals through the website (though they can also be aggregated at the floor or dorm level), but the energy consumption is only recorded at the floor level. We will provide awards based on energy consumption (at floor and dormitory levels), and Kukui Nut points (at individual, floor, and dormitory levels), many with associated prizes to incentivize participation.

In addition to the information technology support of the competition, we will deploy a variety of other methods to engage residents in the competition, such as a kick-off meeting for each dorm where free T-shirts will be distributed, buttons to be distributed to all residents, signage on each floor about the competition, and closing grand prize ceremony.

\subsection{Data sources}

We plan to collect a variety of types of data from the competition. We will record both instantaneous power and cumulative energy consumed on a floor by floor basis for each residence hall, both before the competition starts and continuing for at least 6 months after the competition ends. The sampling rate will be a minimum of 1 minute outside the competition period, and a maximum of 1 minute during the competition period (with a target of 15 seconds), with both rates kept constant during the study to the degree possible.

The energy literacy of participants will be assessed at the start and end of the competition. The assessment will be through a questionnaire that is presented to participants via the contest website as an activity that can be performed for Kukui Nut points. The pre-competition questionnaire will be made available only in the first week of the competition, while the post-competition questionnaire will be made available only in the final week of the competition. Since the website-administered questionnaire is simply a task that can selected by participants, there is the potential that only those participants that feel that they are energy literate will participate in the survey, leading to bias. For this reason, in addition to administration through the website, the questionnaire will be administered in person on paper to two randomly-selected floors.

The competition website will log data about participants' actions on the site. All participant actions and events will be logged with a timestamp. Some example events are: logging into website, selecting a goal for floor participation, and submitting text to verify completion of an activity. These events can be used to create a profile of each participant.

After the competition has ended, participants that used the website will be emailed a link to a qualitative questionnaire. This questionnaire will ask for participants' assessment of the competition, the website, and energy literacy in general.

In early in the following semester (February 2011), the power data for floors will be re-examined to see whether conservation begun as part of the competition has been sustained months later. Floors with particularly high sustained conservation (compared to pre-competition average floor power), and those with low or non-conservation will be selected for an additional questionnaire, and possible face-to-face interviews to determine residents' self-assessment about why they were or were not sustaining the conservation gains made during the competition.

\subsection{Analysis}

Using the energy literacy surveys from before and after the competition, we can 
address the first research question: the impact of the competition on the energy literacy of the participants. Increased scores in post-competition energy literacy would provide an indication that the activities of the competition may increase energy literacy. We will also examine the opposite relationship, to see how the energy literacy of a pair of floors correlates to the energy consumption of those floors during the competition.

There are several ways to address the second research question: the effectiveness of our information technology to support behavioral change tools. One basic metric will be to examine the website logs to see how many residents actually participate in the competition by logging into the website, how often they log in, and how many tasks they complete. The effectiveness of the tasks in improving energy literacy will be assessed by examining the correlation between Kukui Nut points awarded per participant, and their performance on the energy literacy surveys. The relationship between a floor's energy usage and its aggregated Kukui Nut points will provide another window into the effectiveness of the information technology to support behavior change.

The third research question is to what extent does our approach yield sustained changes in energy behavior, and what factors appear to influence sustained change? Using the energy data, we can determine the energy consumption of each pair of floors before, during, and after the competition. The energy consumption after the competition ends is most important when looking for sustained change, and we will look at the relationship between energy consumption and Kukui Nut points, website use, and energy literacy.
%%%%%%%%%%%%%%%%%%%%%%%%%%%%%% -*- Mode: Latex -*- %%%%%%%%%%%%%%%%%%%%%%%%%%%%
%% 10-07-future.tex --  HICSS 44 Kukui Cup paper
%% Author          : Philip Johnson
%% Created On      : Mon Sep 23 11:52:28 2002
%% Last Modified By: Philip Johnson
%% Last Modified On: Mon Jun 14 12:55:47 2010
%%%%%%%%%%%%%%%%%%%%%%%%%%%%%%%%%%%%%%%%%%%%%%%%%%%%%%%%%%%%%%%%%%%%%%%%%%%%%%%
%%   Copyright (C) 2009 Philip Johnson
%%%%%%%%%%%%%%%%%%%%%%%%%%%%%%%%%%%%%%%%%%%%%%%%%%%%%%%%%%%%%%%%%%%%%%%%%%%%%%%
%% 

\section{Future Directions}
\label{sec:future-directions}

As noted above, we are currently planning for the inaugural Kukui Cup
competition to take place in October, 2010.  In addition, we are planning
several other extensions to this research.

First, we are interested in adapting this technology and research approach
to the issue of residential energy consumption.  Residential energy
consumption differs in many ways from dorm energy consumption. Most
significantly, residential energy users are billed directly for the energy
use, so there is an important financial incentive for conservation.
However, the research indicates that tools such as goals and public
commitments remain important in the residential energy application domain.

Second, we would like to support the use of Makahiki as a framework for
other universities who desire to implement a dorm energy competition.  We
believe that Makahiki can significantly lower the ``barrier to entry'' for
universities, and allows them to get started with a simple site if required. 

Third, we are excited by the possibility of integrating grid-level
information into WattDepot for use by Makahiki and other high level
applications.  For example, we are working with our local utility to
provide WattDepot with information on the current mix of generation sources
(i.e. coal, oil, nuclear, solar, wind, etc.) and their aggregate output
(i.e. grid load).  Given this information, it would be possible for
WattDepot to provide the carbon intensity associated with the energy
consumed by end-users at any point in time.  This would make it possible
for the Kukui Cup to incentivize behaviors such as ``doing laundry when the
carbon intensity of the grid is low.''







\section{Acknowledgments}

Financial support for this research is provided by: the Renewable Energy and
Island Sustainability (REIS) project at the University of Hawai`i at M\=anoa,
the Vice Chancellor for Campus Services at the University of Hawai`i at M\=anoa,
and the National Science Foundation (grant IIS-1017126). We would also like to
thank Shanah Trevenna of Sustainable UH for her suggestions and feedback.

\bibliographystyle{IEEEtran}
\bibliography{smartconsumer,sustainability}
\end{document}
