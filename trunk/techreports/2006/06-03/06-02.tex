%%%%%%%%%%%%%%%%%%%%%%%%%%%%%% -*- Mode: Latex -*- %%%%%%%%%%%%%%%%%%%%%%%%%%%%
%% 06-02.tex -- Microprocess paper to be submitted for SPW/ProSim 2006
%% Author          : Hongbing Kou
%% Created On      : Mon Sep 23 11:52:28 2002
%% Last Modified By: Hongbing Kou
%% Last Modified On: Mon Feb 20 23:33:07 2006
%%%%%%%%%%%%%%%%%%%%%%%%%%%%%%%%%%%%%%%%%%%%%%%%%%%%%%%%%%%%%%%%%%%%%%%%%%%%%%%
%%   Copyright (C) 2005 Hongbing Kou
%%%%%%%%%%%%%%%%%%%%%%%%%%%%%%%%%%%%%%%%%%%%%%%%%%%%%%%%%%%%%%%%%%%%%%%%%%%%%%%
%% 

\documentclass[runningheads]{llncs}
\input{psfig.sty}

%\usepackage{/export/home/csdl/tex/icse2003/latex8}
%\usepackage{times}
%\usepackage{url}
%\usepackage{comment}

%% A verbatim-like environment which allows font changes
%%\usepackage{alltt}
%% New LaTeX2e graphics support
% \usepackage[final]{graphicx}
% uncomment the % away on next line to produce the final camera-ready version
% and uncomment the \thispagestyle{empty} following \maketitle
% \pagestyle{empty}
\begin{document}

\pagestyle{headings}
\mainmatter

\title{Automated recognition of low-level process: \\
A pilot validation study of Zorro for test-driven development}

\titlerunning{Automated Recognition of Low-level Process} 

\author{
Hongbing Kou\inst{1} \and Philip M. Johnson\inst{1}}

\institute{
Collaborative Software Development Laboratory \\
Department of Information and Computer Sciences \\
University of Hawai'i \\
1680 East-West Rd. POST307,\\
Honolulu, HI 96822, USA \\
\email{\{hongbing,johnson\}@hawaii.edu}\\
\texttt{http://csdl.ics.hawaii.edu}}

\maketitle
%\thispagestyle{empty}

\begin{abstract}  % 200 words
Zorro is a system designed to automatically determine whether a developer
is complying with the Test-Driven Development (TDD) process.  Automated
recognition of TDD could benefit the software engineering community in a
variety of ways, from pedagogical aids to support the learning of
test-driven design, to support for more rigorous empirical studies on the
effectiveness of TDD in practice.  This paper presents the Zorro system and
the results of a pilot validation study, which shows that Zorro was able to
recognize test-driven design episodes correctly 89\% of the time. The
results also indicate ways to improve Zorro's classification accuracy
further, and provide evidence for the effectiveness of this approach to
low-level software process recognition.
\end{abstract}

\section{Introduction}
\label{sec:intro}

While software process research has historically focused on high-level,
long-duration phases in software development, increasing attention is now
being paid to low-level, short-duration activities as well.  While a
high-level activity such as ``requirements specification'' might take from
weeks to months to complete, a low-level activity such as ``refactor class
Foo to extract interface IFoo'' might take only seconds to complete in a
modern interactive development environment.
%%\footnotetext{Note to reviewers:
due to Springer-Verlag documentation problems, we were unable to 
format this paper correctly by the submission date. We will
correct this prior to final submission.}


The frequency and rapidity with which low-level process activities occur
creates new barriers to answering classic software process questions, such
as: what process is actually occurring (as opposed to what process is
supposed to be occurring), what is the impact of a given process on
important outcomes such as productivity and quality, and how could a given
process be improved and/or tailored to a new domain?

Fortunately, the increasing sophistication of tool support for software
development creates new ways to investigate low-level process.  By
capturing the behavior of developers as represented in their interactions
with software development tools, it may be possible to gain new insight
into what low-level processes are occurring during development and their 
impact on productivity and quality.

This paper presents recent results from our ongoing research into automated
support for recognition and analysis of low-level software processes. Our
approach leverages the Hackystat framework for automated software
engineering process and product data collection and analysis
\cite{Hackystat}, which provides infrastructure for gathering a broad
variety of developer behaviors.  On top of Hackystat, we developed a
generic, rule-based recognizer system for sensor data called ``Software
Development Stream Analysis'' (SDSA).  On top of SDSA, we developed a set
of rules and other specializations designed to recognize a specific
low-level process called Test-Driven Development (TDD) \cite{Beck:03}.  The
system resulting from this combination of Hackystat, SDSA, and TDD-specific
extensions is called ``Zorro''.

Test-driven development is an interesting low-level process to study
because substantial claims have been made for its effectiveness. For
example, TDD has been claimed to naturally generate 100\% coverage, improve
refactoring, provide useful executable documentation, produce higher code
quality, and reduce defect rates \cite{Beck:03,George:03,Maximilien:03}.
It would be a significant contribution to the software engineering
community to rigorously test these claims in controlled and/or professional
settings to better understand the conditions under which they hold, and to
further the evolution of the method itself.

Zorro can automatically monitor developer behavior and produce analyses
describing certain sequences of behaviors as test-driven development and
other sequences of behaviors as non-test-driven development.  If Zorro
recognizes TDD correctly, then we would have a powerful mechanism for
exploring how test-driven development is used in practice and its effects
on quality and productivity.  The ease with which Hackystat sensors can be
installed and the non-intrusive nature of data collection and analysis
would make possible both classroom and industrial case studies into TDD
compliance, the potential discovery of alternative processes, and the
investigation of the impact of TDD on productivity and quality.  Finally,
Zorro could be used to teach TDD by providing real-time feedback to the
developer on whether they are carrying out TDD or not.

Before we can apply Zorro to these TDD research questions, however, we must
answer two general validation questions: (1) Does the system collect the
behaviors necessary to determine TDD, and (2) Does the recognizer infer the
TDD process correctly from the collected behaviors?

In this paper, we present the design of Zorro and the results from a pilot
validation study.  To do the validation, we needed an independent source of
information about low-level developer behavior to compare to Zorro's.  For this
purpose, we designed and implemented an open source system called ``Eclipse Screen
Recorder'' (ESR), \cite{EclipseScreenRecorder}. ESR is a plug-in to the
Eclipse IDE that captures a screen image approximately once per second and
produces a quicktime movie of the developer's behaviors with respect to the
Eclipse window.

Our validation analysis compared the representation of developer behavior
captured by ESR to the representation of developer behavior inferred by
Zorro, and classified the frequency and types of differences between these
two independent representations.  We discovered that Zorro classifies
developer behavior correctly 89\% of the time, and also discovered ways we
can enhance the system in future to improve its classification accuracy further.

The contributions of this research include initial evidence that Zorro can
be an effective tool for automatic recognition of the TDD low-level
process.  Zorro also provides evidence that SDSA is a useful framework for
software process recognition.  Finally, our results reveal the importance
of validation using independent data sources as a component of the process
modelling research process, and the usefulness of mechanisms like ESR for
this purpose.

\section{Related Work}
\label{sec:related}

Osterweil has developed a view of software process research that recognizes
two complementary levels: macroprocess and microprocess \cite{Osterweil05}.
Macroprocess research is focused on the outward manifestations of
process---the time taken, costs incurred, defects generated, and so
forth. Macroprocess research traditionally correlates such outcome measures
to other project characteristics, which can suggest the impact of process
changes to these outcomes, but which suffers from the lack of any
underlying causal theory.  Bridging this gap is the province of
microprocess research, according to Osterweil, in which languages and
formal notations are used to specify process details at a sufficient level
of rigor and precision that they can be used to support causal explanation
of the outcome measures observed at the macroprocess level.  Our research
most readily fits into the ``microprocess'' level, except that instead of
producing a top-down language, our approach involves bottom-up recognition.

The Balboa research project, like Zorro, was concerned with inference of
process from low-level event streams \cite{Cook:95}. In Balboa, the event
streams were taken from the commit records of a configuration management
system, and finite state machines were created that could model the commit
stream data observed in practice. Unlike Balboa, Zorro uses instrumentation
attached to the developer's IDE, which enables access to much lower-level
events than those available through the commit records of a configuration
management system. Also, the Balboa research project was retrospective in
nature, with the researchers limited to historical project records.
Zorro's focus on active development makes additional research possible,
such as the validation studies presented in this paper.

Our research also compares in interesting ways to recent work on
understanding processes associated with open source software development
processes \cite{Jensen:05}. In this research, ``web information spaces''
are mined with the goal of discovering software process workflows via
analysis of their content, structure, update, and usage patterns. Our
approach in Zorro has both strengths and weaknesses relative to this
research.  A strength of the Zorro approach is that by attaching
instrumentation to the IDE, we can capture more detailed information
concerning developer behavior than is possible from inspection of web
information spaces. However, this can also be viewed as a weakness, in that
this instrumentation creates an adoption barrier not present when mining
already publically available information.

Another strand of related research occurs in the areas of knowledge
discovery and data mining, in which time ordered input streams are
processed to discover and classify naturally recurring patterns.  For
example, the Episode Discovery (ED) algorithm supports natural forms of
periodicity in human-generated timestamp data \cite{Heierman04}.  While
such approaches are an interesting future research area for SDSA, our
current episode discovery algorithm uses rules to decide upon episode
boundaries regardless of their frequency of occurrence.

Finally, our research relates to prior research on evaluating test-driven
design practices and their impact on productivity and quality
\cite{George:04,Muller:02,Olan:03,Edwards:04,Geras:04,Matjaz:03}.  In these
studies, researchers had limited ability to verify that the programmers who
were supposed to be using test-driven development were, in fact, using that
methodology.  Zorro, if validated, would be an important contribution to
this research community by providing a tool to ensure compliance with the
process under the experimental conditions.

\section{The design of Zorro}
\label{sec:sdsa}

The design of Zorro is highly modular and consists of three basic layers:
Hackystat, an extension to Hackystat called Software Development Stream Analysis,
and a set of rules and enhancements to SDSA to support recognition of the TDD process.

\subsection{Hackystat}

Hackystat is an open source framework for automated collection and analysis
of software engineering process and product data that we have been
developing since 2001. Hackystat supports unobtrusive data collection via
specialized ``sensors'' that are attached to development environment tools
and that send structured ``sensor data type'' instances via SOAP to a web
server for analysis via server-side Hackystat ``applications''. Over two
dozen sensors are currently available, including sensors for IDEs (Emacs,
Eclipse, Vim, VisualStudio), configuration management (CVS, Subversion),
bug tracking (Jira), testing and coverage (JUnit, CppUnit, Emma, JBlanket),
system builds and packaging (Ant), static analysis (Checkstyle, PMD,
FindBugs, LOCC, SCLC), and so forth.  Applications of the Hackystat
Framework in addition to our work on SDSA and Zorro include in-process project
management \cite{csdl2-04-11}, high performance computing
\cite{csdl2-04-22}, and software engineering education \cite{csdl2-03-12}.

\subsection{SDSA}

Software Development Stream Analysis (SDSA) is a Hackystat application that
provides a framework for organizing the various kinds of data received by
Hackystat into a form amenable for time-series analysis.  Figure
\ref{fig:Streaming} illustrates the start of this process in which the
various kinds of process and product data collected by Hackystat sensors
are filtered and merged into an abstraction called a Development Stream.

\begin{figure}[ht]
  \centerline{\psfig{figure=picture/Streaming.eps,width=120mm}}
  \caption{Development Streams}
  \label{fig:Streaming}
\end{figure} 

The next stage of SDSA processing, called Tokenizing, involves partitioning the
development stream into a sequence of ``episodes'' which should constitute
the atomic building blocks of whatever process is being recognized.  We
have developed four kinds of tokenizers for identifying episode boundaries:
the commit tokenizer uses configuration management checkins, the command
tokenizer uses a distinguished commands or command sequences, the test pass
tokenizer uses passing test invocations, and the buffer transition
tokenizer uses sequences of buffer transitions.  Figure
\ref{fig:TokenizerChain} illustrates the process of splitting up the
development stream into discrete episodes via tokenizers.

\begin{figure}[ht]
  \centering
  \psfig{figure=picture/Tokenization.eps,width=120mm}
  \caption{Tokenizing into episodes}
  \label{fig:TokenizerChain}
\end{figure} 

The final step in SDSA is to classify each episode according to the process
model of interest. In SDSA, this classification is performed using the JESS
rule based system augmented with rules to specify a particular
process. Figure \ref{fig:Classification} illustrates this process.

\begin{figure}[ht] 
  \centering
  \psfig{figure=picture/Classification.eps,width=120mm}
  \caption{Episode classification}
  \label{fig:Classification}
\end{figure} 

\subsection{SDSA specializations for TDD}

Zorro extends SDSA with rules and analyses oriented to the recognition and
classification of TDD behaviors. 
Figure \ref{fig:TDD-Microprocess} illustrates the four kinds of 
behavioral sequences associated with test-driven development. Zorro includes JESS rules
to recognize each of these four kinds of test-driven development behaviors.

\begin{figure}[ht] 
  \centering
  \psfig{figure=picture/TDD-Microprocess.eps,width=120mm}
  \caption{TDD episode description}
  \label{fig:TDD-Microprocess}
\end{figure} 

Refactoring, in which the developer alters the programs internal structure without affecting
its external behavior, is also a valid behavior during test-driven development.  Figure
\ref{fig:Refactoring-Microprocess} illustrates the four kinds of refactoring recognized by 
the Zorro rule base. 

\begin{figure}[ht] 
  \centering
  \psfig{figure=picture/Refactoring-Microprocess.eps,width=120mm}
  \caption{Refactoring episode description}
  \label{fig:Refactoring-Microprocess}
\end{figure} 

Finally, Zorro includes a user interface in the Hackystat server web
application for display of the episodes, their classification, and their
internal structure.  Figure \ref{fig:zorro-interface} illustrates the
Zorro interface. 

\begin{figure*}[ht] 
  \centering
  \psfig{figure=picture/Zorro.eps,width=120mm}
  \caption{Zorro interface}
  \label{fig:zorro-interface}
\end{figure*} 


\section{The Pilot Validation Study}
\label{sec:casestudy}

As noted above, in order to feel confident in Zorro as an appropriate tool
to investigate TDD, we must answer two basic validation questions: (1) Does
Zorro collect the behaviors necessary to determine when TDD is occurring,
and (2) Does Zorro recognize test-driven development when it is occurring?
To answer these questions, one must somehow gather an independent source of
data regarding the developer's behaviors and compare that data to what was
collected and analyzed by Zorro.

One approach to validating the system is to have an observer watching
developers as they work, and take notes as to whether they are performing
TDD or not.  We considered this but discarded it as unworkable: the use of
a human observer would be quite costly, and given the rapidity with which
TDD cycles can occur, it would be quite hard for an observer to notate all
of the TDD-related events that can occur literally within seconds of each
other. We would end up having to validate our validation technique!

Instead, we developed a plugin to Eclipse that generates a Quicktime movie
containing time-stamped screen shots of the Eclipse window at regular
intervals.  Figure \ref{fig:esr} shows the Quicktime viewer with one screen
image.  The design of ESR allows adjustment of frame rate and resolution:
the higher the frame rate and/or resolution, the larger the size of the
resulting Quicktime file. We have found that a frame rate of 1 frame per
second and a resolution of 960x640 pixels is sufficient for validation,
while producing relatively compact Quicktime files (typically 7-8 MB per
hour of screenshots).  The Quicktime movie created by ESR provides a visual
record of developer behavior that can be manually synchronized with the
Zorro analysis using the timestamps and used to answer the two validation
questions.

\begin{figure}[ht] 
  \centering
  \psfig{figure=picture/esr.eps,width=120mm}
  \caption{An ESR Quicktime file}
  \label{fig:esr}
\end{figure} 

Our pilot validation study involved the following procedure. First, we
obtained agreement from seven volunteer student subjects to participate in the
pilot study. These subjects were experienced with both Java development and
the Eclipse IDE, but not necessarily with test-driven development.  Second,
we provided them with a short description of test-driven design, and a
sample problem to implement in a test-driven design style.  The problem
was to develop a Stack abstract data type using test-driven design, and we
supplied them with an ordered list of tests to write and some sample test
methods to get them started.  Finally, they carried out the task using
Eclipse with both ESR and Zorro data collection enabled. 

To analyze the data, we created a spreadsheet in which we recorded the
results of watching the Quicktime movie and manually encoding the developer
activities that occurred.  Then, we ran the Zorro analyses, added their
results to the spreadsheet, and validated the Zorro classifications against
the video record.

\section{Results of the Pilot Study}
\label{sec:results}

\begin{figure*}[ht] 
  \centering
  \psfig{figure=picture/data-summary.eps,width=120mm}
  \caption{Summary Results}
  \label{fig:data-summary}
\end{figure*} 


Figure \ref{fig:data-summary} summarizes the results of our analyses.
Seven subjects participated, and spent between 28 and 66 minutes to
complete the task.  Zorro partitioned the overall development effort into
92 distinct episodes, out of which 86 were classified as either
Test-Driven, Refactoring, or Test-Last; the remainder were
``unclassified'', which normally corresponded to startup or shutdown
activities.

The most important result of this study is indicated by the ``Wrongly
Classified Episodes'' column, which shows the results of comparing the ESR
videos of the developer's Eclipse window to the classifications
automatically made by the Zorro recognizer.  Out of the 92 episodes under
study, 82 were validated as correctly classified, for an accuracy rate of
89\%.

The validation analysis also revealed several ways to increase the accuracy
of Zorro.  First, we discovered that our underlying Hackystat sensor
sometimes failed to record an edit to the program under development when
the ESR video showed that the developer made a ``quick change'' lasting
only a few seconds. Second, the sensor also failed to record a compilation
error when a change to the production code created a compilation error in
the non-active test code.  Finally, the current Zorro rule set sometimes
failed to partition the development stream along optimal episode
boundaries, making it problematic for the classifier to recognize the
developer's behaviors during this time period correctly. We intend to fix
these issues in the next version of the system, which should raise the
accuracy rate significantly.

It is also interesting to review the classification results apart from
their accuracy, as they provide insight into the appropriate design of
future studies. All four types of Test-Driven Development were recognized
by Zorro, although only two of the four types of Refactoring were found.
We believe that the simplicity of the software system under development in
this study may have been a factor in the limited types of refactoring, and
intend to scale up the problem complexity in future studies.

A provocative result of this study is that half the episodes (46) were
classified as test-last, even though the subjects were instructed to do
test-first development.  To some extent, this may also be an artifact of
the simplicity of the software under development. But it also reveals a
hidden ``secret'' of test-first development: sometimes, while implementing
the code to address one unit test, you can't help but implement additional
features as well. At that point, the rational behavior is to implement the
unit tests for those additional features, which effectively constitutes
test-last design.  The nature and frequency of embedded test-last within
test-first development is an interesting topic for future research.

\section{Conclusions and future directions}
\label{sec:conclusions}

The pilot study has been successful in developing an effective validation
methodology for the Zorro system, and in identifying several opportunities
for improvement to the system that should result in higher classification
accuracy in future. 

After making these improvements, our next task will be to design and carry
out a broad-scale validation study.  We intend to  expand the total
number of subjects participating in the study, and solicit both student and
professional developer participation. While we will provide a sample
problem to implement in a test-driven design approach, we also hope to
collect ``in vivo'' data from professionals who use test-driven design in
their daily work.  As before, we will collect both ESR and Zorro data from
each subject and analyze it to assess the classification accuracy of Zorro,
discover opportunities for improvement in the system, and perhaps discover
new insights into the nature of test-driven design.

If the broad-scale validation study results demonstrate that Zorro has
achieved high accuracy (95\% or better) in recognizing TDD, then we will
proceed to the next stage, which is the design of experiments to see how
developers use (or don't use) TDD in practice, the factors influencing
their decision, and the outcomes of their decisions on productivity and
quality.

Another area of future research is the application of the SDSA framework to
model other low-level software development processes.  For example, there
are a variety of best practices surrounding when a developer should commit
their changes to a configuration management repository which we could model
and assess using SDSA along with different sensors and different
classification rule sets.

\bibliographystyle{splncs}
\bibliography{/export/home/csdl/bib/zorro,/export/home/csdl/bib/tdd,/export/home/csdl/bib/csdl-trs,/export/home/csdl/bib/hackystat,/export/home/csdl/bib/psp}
\end{document}











