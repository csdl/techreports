%%%%%%%%%%%%%%%%%%%%%%%%%%%%%% -*- Mode: Latex -*- %%%%%%%%%%%%%%%%%%%%%%%%%%%%
%% 06-03.tex -- Tutorial proposal submission to XP 2006
%% Author          : Hongbing Kou
%% Created On      : Mon Sep 23 11:52:28 2002
%% Last Modified By: 
%% Last Modified On: Fri Feb 24 16:57:48 2006
%%%%%%%%%%%%%%%%%%%%%%%%%%%%%%%%%%%%%%%%%%%%%%%%%%%%%%%%%%%%%%%%%%%%%%%%%%%%%%%
%%   Copyright (C) 2005 Hongbing Kou
%%%%%%%%%%%%%%%%%%%%%%%%%%%%%%%%%%%%%%%%%%%%%%%%%%%%%%%%%%%%%%%%%%%%%%%%%%%%%%%
%% 

\documentclass[11pt,twocolumn]{article} 
\input{/export/home/csdl/tex/psfig/psfig}
\usepackage{/export/home/csdl/tex/icse2003/latex8}
\usepackage{times}
\usepackage{url}
\usepackage{comment}

%% A verbatim-like environment which allows font changes
%%\usepackage{alltt}
%% New LaTeX2e graphics support
\usepackage[final]{graphicx}
% uncomment the % away on next line to produce the final camera-ready version
% and uncomment the \thispagestyle{empty} following \maketitle
%\pagestyle{empty}
\begin{document}

\title{Introduction to the Hackystat Framework:\\
Lightweight, automated process and product metrics \\
collection and analysis for agile organizations\\ \medskip
\em Full Day Tutorial Proposal}

\author{
Philip M. Johnson\\
\em  Collaborative Software Development Laboratory \\
\em  Department of Information and Computer Sciences \\
\em  University of Hawai'i \\
\em  johnson@hawaii.edu}
\maketitle
\thispagestyle{empty}


\Section{Overview}

Hackystat is an open source framework for automated collection and analysis
of software engineering process and product data that has been under
development since 2001. Increasing usage by a variety of academic and professional 
organizations indicates that it has reached a level of maturity, stability, 
and capability that may make it of interest to organizations interested in 
the introduction of lightweight, yet sophisticated metrics collection and analysis.

Hackystat differs from other metrics collection and analysis frameworks in
one or more of the following ways.

First, Hackystat uses sensors to unobtrusively collect data from development
environment tools; there is no chronic overhead on developers to collect
product and process data. Over two dozen sensors are publically available,
including sensors for IDEs (Emacs, Eclipse, JBuilder, Vim, VisualStudio),
configuration management (CVS, Subversion), bug tracking (Jira), testing
and coverage (JUnit, CppUnit, Emma, JBlanket), system builds and packaging
(Ant), static analysis (Checkstyle, PMD, FindBugs, LOCC, SCLC), and so
forth.

Second, Hackystat is tool, environment, process, and application
agnostic. The architecture does not suppose a specific operating system
platform, a specific integrated development environment, a specific
software process, or specific application area. A Hackystat system is
configured from a set of modules that determine what tools are supported,
what data is collected, and what analyses are run on this data.

Third, Hackystat is intended to provide in-process project management
support. Many traditional software metrics approaches are based upon the
"project repository" method, in which data from prior completed projects
are used to make predictions about or support control of a current
project. In contrast, Hackystat is designed to collect data from a current,
ongoing project, and use that data as feedback into the current project.

Fourth, Hackystat provides infrastructure for empirical experimentation. For
those wishing to compare alternative approaches to development, or for
those wishing to do longitudinal studies over time, Hackystat can provide a
low-cost approach to gathering certain forms of project data.
 
Fifth, Hackystat is open source and is made available for no charge.

Finally, Hackystat was designed by developers to satisfy the measurement needs
and interests of developers.  Many traditional measurement frameworks incur
overhead on developers to produce data that made available to or of
interest only to ``management''.  While Hackystat analyses are indeed
useful for project management, the data is always accessable to, under the
control of, and designed to support the goals of developers. 

Hackystat has been applied in a variety of agile and non-agile contexts.
The Zorro Project involves the development of a sensor for Eclipse that
attempts to automatically detect when users are employing test-driven
design practices. An pilot validation study found that Zorro correctly
identified TDD episodes 89\% of the time \cite{csdl2-06-02}.  Software
Project Telemetry is an approach to metrics analysis and visualization that
provides a new style of in-process project management \cite{csdl2-04-11}.
Hackystat is being applied as a measurement technology for the DARPA High
Productivity Computing Systems program \cite{csdl2-04-22}.  Finally,
Hackystat has been used extensively for software engineering education
\cite{csdl2-03-12}.

\Section{Goals}

The primary goal of this workshop is to enable attendees to obtain a clear
understanding of the capabilities and challenges associated with
Hackystat-based software engineering measurement collection and analysis.
This understanding can help attendees decide whether or not Hackystat is
appropriate for their organization. In addition, it can provide attendees
with new perspectives on the existing measurement techniques they are using
(either voluntarily or involuntarily). 



\Section{Audience}

\Section{Agenda}
\cite{Hackystat}

Prospective TUTORIAL presenters are encouraged to submit

    a) 2-page summary of the tutorial topic (to be included in the proceedings)
    b) Draft agenda
    b) CV of the tutorial presenters
    c) 10-15 slide example set of the tutorial material


\bibliographystyle{/export/home/csdl/tex/icse2003/latex8}
\bibliography{/export/home/csdl/bib/zorro,/export/home/csdl/bib/tdd,/export/home/csdl/bib/csdl-trs,/export/home/csdl/bib/hackystat,/export/home/csdl/bib/psp}
\end{document}











