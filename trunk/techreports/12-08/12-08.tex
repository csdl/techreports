\documentclass[jou]{apa} %can be jou (for journal), man (manuscript) or doc (document)
%
%
%these next packages extend the apa class to allow for including statistical and graphic commands
\usepackage{url}   %this allows us to cite URLs in the text
\usepackage{graphicx}  %allows for graphic to float when doing jou or doc style

% 11-07 is the FDG paper.
% 10-08 is robert's thesis.
% APA style info at: http://www.ilsp.gr/homepages/protopapas/apacls.html

% We need a table that shows "Dorm Energy Competition Design", with entries to say if the
% designers revealed (a) how baseline data was collected; (b) how the ranking was
% determined; (c) what factors were taken into account in creating the ranking; (d) was
% data collected after the competition to look for rebound effect; (e) was there evidence
% of short-term, unsustainable behavior change; 

% Proposal: let the teams create an argument for why they should win (this encourages more
% introspection and potentially data gathering.)

\title{Lights Off.  Game On?  Myths and misperceptions in energy competition game design}
\author{Philip M. Johnson and Yongwen Xu and Robert S. Brewer}
\affiliation{Collaborative Software Development Laboratory \\ Information and Computer
  Sciences \\ University of Hawaii \\ Honolulu, HI USA \\ johnson@hawaii.edu}

\abstract{

  The Kukui Cup project investigates the use of ``meaningful play'' to facilitate energy
  awareness, conservation and behavioral change.  Each Kukui Cup Challenge combines real
  world and online environments in an attempt to combine information technology, game
  mechanics, educational pedagogy, and incentives in a synergistic and stimulating
  fashion.  We challenge players to: (1) acquire more sophistication about energy concepts
  and (2) experiment with new behaviors ranging from micro (such as turning off the lights
  or installing a CFL) to macro (such as taking energy-related courses, joining
  environmental groups, and political/social advocacy.)

  To inform the design of the inaugural 2011 Kukui Cup, we relied heavily on the design of
  prior collegiate energy competitions, of which there have been over 150 in the past few
  years.  Published accounts of these competitions indicate that they achieve dramatic
  reductions in energy usage (a median of 22\%) and cost savings of tens of thousands of
  dollars.  In our case, the data collected from the 2011 Kukui Cup was generally in
  agreement, with observed energy reductions of up to 16\% when using
  standard analysis techniques.  However, our analysis process caused us to look more
  closely at the methods employed to produce outcome data for energy competitions, with
  unexpected results.

  We now believe that energy competitions make significant unwarranted assumptions about
  the data they collect and the way they analyze it, which calls into question the
  accuracy of published results from this literature.  We believe a closer examination of
  these issues by the community can help improve the design not only of future energy
  challenges, but other similar forms of ``serious games for sustainability''.

  In this paper, we describe the Kukui Cup, the design myths and misperceptions it
  uncovered, and how we believe these issues should inform the design of future forms of
  ``meaningful play'' with respect to energy.

}


\acknowledgements{The Kukui Cup has been supported in part by grant IIS-1017126 from the
  National Science Foundation, and by funding from the University of Hawaii Office of
  Facilities Management.  We gratefully acknowledge the 418 players of the 2012 Kukui Cup
  and the members of the Kukui Cup team in addition to the authors who made the vision a
  reality: Kaveh Abhari, Hana Bowers, Greg Burgess, Caterina Desiato, Michelle Kat\-chuck,
  Risa Khamsi, Alex Young, and Chris Zorn.}

\shorttitle{Myths and misperceptions in energy challenge game design}
%\rightheader{Right header}
%\leftheader{Left Header}

\begin{document}
\maketitle 

\section{Introduction}  

The rising cost, increasing scarcity, and environmental impact of fossil fuels as an
energy source makes a transition to cleaner, renewable energy sources an international
imperative.  One barrier to this transition is the relatively low cost of current
energy, which makes financial incentives less effective. Another barrier is the historical
success of electrical utilities in making energy ubiquitous, reliable, and easy to access,
thus enabling widespread ignorance about basic energy principles
and trade-offs.  In Hawaii, the need for transition is especially acute, as our state
leads the nation both in the price of energy and in reliance on fossil fuels as an energy
source.

Moving away from petroleum is a technological, political, and social paradigm shift,
requiring citizens to not only think differently, but behave differently with respect to
energy policies, methods of generation, and their own consumption. Unfortunately, there is
no tradition of teaching ``energy'' as a core subject area in the United States, even
though this subject appears to be one of the most important emergent issues of the 21st
century. Anecdotal reports indicate the lack of basic energy literacy at the secondary
school level \cite{Ammons2010}.

%% It would be nice to have more citations and examples regarding the "lack of basic energy literacy."

One of the most widespread and successful approaches to raising the profile of energy use
is the collegiate dorm energy competition, which has been held on over 150 campuses
in the past few years \cite{Hodge2010}.  Published documents report that these
competitions are extremely successful.  Hodge reports median reductions in energy use of
22\% and a maximum reduction of 80\% for the competitions she studied. A case study of
Elon University reports that a seven week competition reduced energy consumption by
231,454 kWh and produced \$2,000 in electicity cost savings per week \cite{Durr2010}.

In an attempt to build upon these promising initial experiences, we began the Kukui Cup
project in 2010.  Our goal was to expand the scope from a relatively simple
``competition'' where the primary outcome measure is {\em kWh consumed} to a more
elaborate ``challenge'' in which information technology, community-based social marketing,
serious games, and educational pedagogy would be combined to support sustained change in
sustainability-related behaviors.  In addition to measuring kWh consumed, the Kukui Cup
also implements a point system intended to measure player involvement with educational
materials, workshops, excursions, and group engagement.  Through a series of experiments
held in residence halls at the University of Hawaii and elsewhere, we wanted to gain
deeper insight into how these various factors contribute to positive behavioral change.

Our first Kukui Cup challenge was held at the University of Hawaii in Fall, 2011 for the
1,000 students in the Hale Aloha residence halls. The challenge divided the students into
20 teams (called ``lounges'') of approximately 50 students each.  We started measuring
energy consumption for most teams at three weeks prior to the competition, although
several teams did not have operational energy meters until shortly before the competition
started. We used this prior data to produce baselines as a way of assessing whether and to
what extent energy reductions occurred as a result of the challenge.

Our initial analyses of the data we collected during the challenge appeared promising.
Over 400 of the students participated in the three week challenge, spending over 850 hours
in the online environment.  Student feedback regarding both real world and online aspects
of the challenge was uniformly positive.  Some teams appeared to achieve a 10-16\%
reduction in their energy usage during the challenge.  Participating students made over
1,000 commitments and earned over 80,000 points.

As we continued to look at the outcomes from the initial challenge in order to understand
how to better support sustainable behavioral change, we began to be troubled by the way
dorm energy competitions measure their outcomes, and the variety of unwarranted
assumptions underlying the results that seem to exist in the literature.  These problems
are not merely theoretical or scientific, they have a direct impact on the experience of
participants and the effectiveness of these competitions as ``meaningful play''.  For
example, the baseline calculation method used during the 2010 Campus Conservation
Nationals energy competition led some students at Oberlin College to stop participating in
the competition ``out of frustration'' \cite{Willens2010}.

Making matters worse, published reports concerning energy competitions rarely document
how, for example, baseline energy consumption is calculated, or what happens to energy
consumption after the competition ends.  Without information like this, it is hard to
assess the ``meaningfulness'' of the ``play''.  If, for example, energy consumption
returns immediately to pre-competition levels after the competition ends, then was it 
meaningful, no matter what the level of reduction was achieved during the competition?

This paper presents our findings to date about how to better understand the impact of
energy competitions and challenges as meaningful play, and how to improve this impact over
time.  We believe this process must start with a better understanding of the limitations
and inaccuracies of {\em kWh consumed} as an outcome measure, as tempting a metric as it
might be.  We argue that ``behavior'' must be interpreted and measured more broadly and
that games must be designed to promote and reward a much more diverse spectrum of behavior
change.  Rather than reward students for unsustainable changes such as unplugging vending
machines (as at Oberlin College \cite{Petersen07a}) or camping outside during the
competition (as at Carleton College \cite{Hodge2010}), future games could reward students
for sustainable changes such as enrolling in a class on energy in an upcoming semester, or
joining an environmental group that promotes renewable energy.  

The next section of this paper provides a brief overview of the Kukui Cup.   The following
section presents the ``myths and misperceptions'' that we believe to be widespread in the
design and reporting of current energy competitions. We conclude with our recommendations
for how we and others should design future energy challenges to be more effective as
meaningful play.

\section{The Kukui Cup}

A defining feature of Kukui Cup challenges is a blend of real world and virtual world
activities, all tied together through game mechanics.  In the real world, players
participate in workshops and excursions, win 
prizes, and most importantly, learn about their current lifestyle and its impact on
energy consumption.  In the online world of the Kukui Cup web application, players
earn points, achieve badges, increase their sustainability "literacy" through readings and
videos, and use social networking mechanisms to engage with friends and family about the
issues raised. The challenge is designed to make real world and online activities
complementary and synergistic.

Each Kukui Cup Challenge is typically designed with the following goals for its participants:
\begin{itemize}
\item Increase {\bf knowledge} about energy issues;
\item Gain {\bf insight} about the impact of one's current behaviors and how to change
  them for more effective energy usage;
\item Build {\bf community}, through awareness of local and national sustainability organizations and initiatives;
\item Create {\bf commitment}, from minor (turn off the lights when not in use) to major (pursue a profession related to sustainability).
\end{itemize}

To create sophisticated games based upon energy consumption, it is helpful to
collect real-time energy data from meters, store the data, perform analyses on the data, and
visualize the results. We developed WattDepot \cite{csdl2-10-05} to provide an open source, vendor-neutral
framework for energy data collection, storage, analysis, and visualization.  WattDepot is
useful not only as technology infrastructure for the Kukui Cup, but as infrastructure for
other energy-related initiative such as the Smart Grid.

Implementation of game mechanics is provided by another system we developed called
Makahiki \cite{csdl2-11-07}.  It provides an open source, component-based, extensible environment for
developing sustainability challenges such as the Kukui Cup and tailoring them to the needs
of different organizations.  One configures the Makahiki framework to produce a "challenge
instance" with a specific set of game mechanics, user interface features, and experimental
goals.  Makahiki provides sophisticated instrumentation to support evaluation of how well
the game mechanics supported the organization's goals for the challenge. 




\section{Myths and misperceptions in energy challenge game design}

The preceding section presented the design and implementation of the 2011 Kukui Cup along
with some basic outcome data in a manner consistent with the way many other energy
competitions have been presented.  Perhaps the most significant outcome from the 2011
Kukui Cup was the recognition that virtually all energy challenges, when viewed as game
designs, are fundamentally flawed: they do not create a ``fair'' game, they do not measure
what they think they are measuring, they do not measure the right things, and they do
not achieve the right outcomes. 

A fundamental property of energy challenges is competition, and a fundamental property of
competitions is that the ability to creating rankings.   The first set of myths concern 
the flaws in energy competition game design related to competition.


\subsection{Myth \#1: Percentage reduction from baseline is a valid ranking}

In general, there are two ways to create rankings in energy competitions: (1) by total
energy consumption or (2) by reduction from a baseline.

The first method is to rank the teams according to their total energy consumption (from
least to most) during the competition interval.  So, if Team A consumed 1,200 kWh during
the competition period, and Team B consumed 1,300 kWh, then Team A would be the
winner. This is the simplest method, but produces a ``fair'' competition only if every
team is equivalent with respect to the factors affecting their energy consumption.  For
example, every team should have the same number of members (because if one team has 50
participants and another has 75, then the latter team would be at an unfair disadvantage).
In addition, every team should have the same energy infrastructure (because if one team
has a well insulated building while another has a poorly insulated building, the latter
team will again be at an unfair disadvantage).  Because equality with respect to all
significant energy consumption factors is so rare, and because the unfairness of the
competition ranked this way can be so obvious, this method is relatively uncommon.

The second, more common method is to compute a ``baseline'' for each team based upon
historical usage and produce rankings for teams based upon reductions from this baseline.
For example, if Team A's baseline is 1,200 kWh during the competition period, but they
only use 1,140 kWh, then their reduction from the baseline is 60 kWh, or a 5\% reduction
in usage relative to the baseline.  If Team B's baseline is 1,300 kWh, but they use only
1,239 kWh, then their reduction from the baseline is 61 kWh, or a 4.6\% reduction in usage
relative to the baseline.  While Team B actually achieves a larger net reduction in energy
relative to the baselines (61 kWh vs. 60 kWh), all competitions we have seen will rank
Team A higher, due to a higher percentage reduction (5\% vs. 4.6\%).

The uniform use of percentage reduction from baseline for ranking exposes the first
problem with the use of baselines, because there is no {\em a priori} reason that
percentage reduction from the baseline produces a more fair ranking than absolute
reduction from the baseline.  To see why this is so, consider the following alternative
scenario where Team A lives on one floor and Team B lives on an adjacent floor.  The two
floors are similar in structure and occupancy except that Team B's floor includes a shared
laundry room.  In this case, the most fair way to produce a ranking is to calculate the
energy consumption of the shared laundry room, subtract that consumption from Team B's
data, then use absolute values to rank the floors.  Put simply, the two teams should be
considered equal once the load from the shared laundry room is factored out.

For example, let's assume the baseline for Team A is 1000 kWh/week, and that the baseline
for Team B is 1300 kWh/week (where the additional 300 kWh is due to students doing 100
loads of laundry per week at 3 kWh/load). The fairest ranking in this case is ``adjusted
absolute consumption'', where the laundry room's 300 kWh of consumption is substracted
from Team B each week and the ranking is determined from the remaining absolute 
consumption.

On the other hand, the use of percentage reduction gives Team A a significant unfair
advantage over Team B. Let's say Team A reduced their consumption by 100 kWh and Team B
reduced their consumption by 120 kWh.  Under percentage reduction, Team A would win (10\%
vs. 9\%) even though Team B reduced their consumption by 20 kWh more than Team A.  (Making
things even less fair, some members of Team A could decide to do even more laundry than
usual during the competition in order to penalize Team B.  Such ``load shifting'' of
energy consumption to other teams is a common strategy in collegiate energy competitions.)

Of course, adjusted absolute consumption does not create a fair ranking in all
circumstances. For example, if Team A has 50 players and Team B has 250 players, then some
sort of normalization to create a ``per capita'' consumption is required to provide a fair
ranking. 

Our conclusion is the following: to produce a valid ranking based upon energy consumption,
you must not only successfully measure ``normal'' energy consumption, but you must also
determine the reasons behind the differences measured, because those reasons are crucial
to creating a fair ranking.

\subsection{Myth \#2: Representative energy data can be collected immediately prior to the competition}

The preceding section assumes that ``normal'' or ``representative'' energy usage can be
determined and used as a baseline, and then discusses the problems that can arise in using
such data to create a valid ranking.  Let's now step back and consider the question: is it
possible to gather energy data immediately prior to the competition and be confident that
it is representative?

Many energy competitions create baselines using energy consumption from one to four weeks
immediately prior to the start of competition.  This approach has two significant
advantages. First, baseline data is based upon the energy usage of the players actually
engaged in the competition.  Second, baseline data is based upon the state of the
infrastructure (buildings, appliances, etc.) as it will be during the competition.

However, these advantages are offset by the significant chance that data taken immediately
prior to the competition could be non-representative--i.e. not ``normal''.  For example,
in the 2011 Kukui Cup, we had data for most (but not all) floors of the competition for up
to three weeks prior to the competition.   Figure \ref{fig:kukuicup-baseline-data-chart}
illustrates this data.

What is disconcerting about this data is the variety of trends that can be observed in
the different floors: some floors have relatively consistent energy usage, while other
floors are trending upwards in energy usage and still others are trending downwards. 

This variation means that the choice of time interval can have a significant impact
on the resulting baselines.  Figure \ref{fig:baseline-sensitivity-analysis} shows the resulting
baselines from six different scenarios: the three individual weeks prior to the
competition, the first two weeks in the three week interval, the latter two weeks in the
three week interval, or the entire three week interval.  Depending upon the choice,
variations of up to 10\% (?) in the ultimate value of the baseline can be observed.

A second problem with data immediately prior to the competition is the impact of seasonal
change.   This created a significant problem with the first Campus Conservation Nationals
event \cite{CCN}.  According to John Peterson, the ``baseline period... was, in some
cases, resulting in percentage changes for individual buildings... that were more
attributable to changes in weather and other factors than to the choices that students
were making in their dorms.''

\subsection{Myth \#3: Representative energy data can be collected from years  prior to the competition.}






\subsection{Myth \#3: Competition data can be used to estimate actual savings}

In addition to using the baseline value to produce a fair ranking among teams with
different energy factors, baselines are also commonly used to estimate how much energy the
teams would have used during the time period of the competition if the competition was not
taking place.  As a result, baseline data is generally used not only to declare the winner
(based upon the biggest percentage change under the baseline) as well as to declare energy
savings (by calculating the number of kWh that were presumably not consumed due to the
competition).

\subsection{Myth \#2: Outcomes account for confounding variables}

Explain how external factors like weather, holidays, and occupancy volatility can
artificially affect reduction results.

\subsection{Myth \#3: The good guys win}

Explain how energy hogs can more easily obtain larger percentage reductions than their
more sustainability oriented peers.

\subsection{Myth \#4: Results reflect sustainable change}

It is trivial to get a "perfect" 100\% reduction in energy; just flip the breaker.
Behaviors like unplugging vending machines are exactly similar.

\subsection{Myth \#5: Prizes are good}

Extrinsic vs. intrinsic motivation; possible that prizes could decrease intrinsic
motivation to conserve energy.

\subsection{Myth \#6: Results are comparable}

It is tempting to look at literature for guidance on how much reduction can be achieved.  In reality, results are from widely different geographic regions and infrastructures and do not provide any predictive guidance.  Would you use the weather forecast for Hawaii as a way of predicting the temperature in Duluth?

\subsection{Myth \#7: Energy challenges measure the right thing}

Focusing on in-competition energy consumption is not meaningful.  Instead, to understand
the true value of energy competitions, one should instead measure: (1) post-competition
energy consumption; changes in participant knowledge as a result of competition; changes
in participant behavior (interpreted broadly: not just micro-behaviors like turning off
the lights but macro-behaviors such as choice of major)

Like looking under the lightpost for keys.

\section{Conclusions}

Does all of this concern with accuracy and fairness really matter?  We claim it does, for
two reasons.  First, the fundamental goal of energy competitions is to increase the
literacy and sophistication of its participants.  Basing the competition on faulty data or
methods is, essentially, bad science, and we should be committed to providing players with
scientifically sound game design as a matter of educational principle.  Second, unfair
game design is tremendously demoralizing.  For example, an Oberlin student complained
about the unfair use of baselines and rankings in the Campus Conservation Nationals as
follows: ``We've turned off every [expletive] light in this building, dude, and it's not making
an [expletive] difference.''

Move away from ``competition'' and toward ``challenge''.

Change incentive and reward structure to depend less on pure kWh values, which are easy to
measure but not necessarily indicative of change, and toward broader definitions of
``behaviors''.

Change game mechanics so that short term, unsustainable changes in consumption are not
rewarded or incentivized (turn off every light in the building, etc.)   Teaching students
how to find the main circuit breaker and turn it off is not an interesting. 


\section{Reference materials}

\begin{itemize}

\item ``In the classroom, a lack of energy education'':
\url{http://www.indyweek.com/indyweek/in-the-classroom-a-lack-of-energy-education/Content?oid=1547796}.
Provides anecdotal evidence that students have very minimal energy literacy.

\item Chelsea Hodge (see dropbox folder kukucuip-2011/articles for slides).  Talks about a
  ``core component'' of a dorm energy competition being a ``short time frame'', which
  encourages students to ``go all out'', and provides as an example some Carleton College
  stduents who are camping out for a week to save energy.  This is clearly unsustainable
  behavior change. She talks about median reductions of 22\% and biggest reduction being
  80\%.  How did that number occur?

\item SLO residence hall competition.
\url{http://www.slideshare.net/AllianceToSaveEnergy/energy-and-water-residence-hall-competition}
No information on how baselines were calculated. 

\item CCN, Robert finds that 10\% reduction at stanford happened during finals week.
\url{https://groups.google.com/forum/?hl=en#!searchin/kukuicup/campus$20conservation/kukuicup/dYy6e0ORFE0/BxweJIznXRoJ}

\item More about baselines and relatively arbitrary changes.
\url{http://www.oberlinreview.org/article/dorm-energy-resource-competition-succeeds-despite/}.
A student told him, “We’ve turned off every fucking light in this building, dude, and it’s
not making a goddamn difference.” The baseline problem “killed us,” Sabo added.  Also, "Savings nationwide totaled 509,000 kilowatt hours, \$50,200, and 816,00 pounds of carbon dioxide."

Also: 
\url{https://oncampus.oberlin.edu/source/articles/2010/12/03/resource-reduction-competition-results}.
This talks about how bad choice of baselines made the competition less fair, also how
environmentally conscious students at Oberlin who have been conserving for years are
penalized because based upon percentage reduction. 

\item From Oberlin Review, ``To address the baseline problem, LDG will use “weather
  normalization,” a process that derives a baseline from five years of measurements,
  sidestepping problems of climate difference and inconsistent weather.''  But that now
  means building renovations etc. will not be controlled for.

\item From Green Cup at Berkeley, 
\url{http://tgif.berkeley.edu/index.php/grants/projects/2011-projects/28-greek-energy-savings}, 
results are always skewed towards winners, but 1/3 of the participants increased resource
usage during competition (some dramatically: 71\% increase in natural gas consumption).
Results are not looked at in depth to understand what exactly happened and why some places
increased in energy percentage-wise just as much as the top decreasers.

\item According to data compiled by Chelsea Hodges, over 150 colleges have had energy
  challenges in the 2010-2011 school year.  


\end{itemize}
\bibliography{sustainability,csdl-trs,smartconsumer,gamification,12-08}

\end{document}



