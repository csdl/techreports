\chapter{Experiment}

This chapter describes the design of the experiment using the competition and associated website described in \autoref{cha:system-description}. First we cover the different sources of data available for the experiment, followed by analyses performed on the data, and then the research questions to be investigated.

\section{Data Sources}

\subsection{Power Usage Data}
\label{sec:power-usage-data}

We will record both instantaneous power and cumulative energy consumed on a floor by floor basis for each residence hall, beginning long before the competition starts and continuing indefinitely after the competition ends. The sampling rate will be at least 1 minute outside the competition period, and less than 1 minute during the competition period (with a target of 10 seconds).

\subsection{Pre and Post-Competition Energy Literacy Questionnaires}
\label{sec:exp-literacy-questionnaire}

The energy literacy of participants will be assessed at the start and end of the competition. The assessment will be through a questionnaire that is presented to participants via the contest website as an activity that can be performed for Kukui Nut points. The pre-competition questionnaire will be made available only in the first week of the competition, while the post-competition questionnaire will be made available only in the final week of the competition. \autoref{app:energy-literacy} lists the questions that will make up the pre and post-contest questionnaires.

Since the website-administered questionnaire is simply a task that can selected by participants, there is the potential that only those participants that feel that they are energy literate will participate in the survey, leading to bias. For this reason, in addition to administration through the website, the questionnaire will be administered in person on paper to two randomly-selected floors. While the assignment of residents to a floor is not random, it is at least not self-selected. The questionnaire will be administered to the floors before the competition starts, and in the final week of the competition. The questionnaire will be removed from the activity lists shown to participants on the floors that were selected for in-person administration. However, those participants that fill out the survey on paper will receive Kukui Nut points just as if they had filled it out online.

\subsection{Website Log Data}

The contest website will extensively log data about participants' actions on the site. All participant actions and events will be logged with timestamp. A few examples of the type of events to be logged:

\begin{itemize}
\item Participant logs onto website
\item Participant selects goal for floor participation
\item Participant submits text to verify completion of an activity
\item Participant makes a selection to display energy consumption for a floor different from his/her own
\item Participant logs off of website
\end{itemize}

\subsection{Post-Competition Feedback Questionnaire}
\label{sec:post-competition-feedback}
After the competition has ended, participants that used the website will be emailed a link to a qualitative questionnaire, as part of the energy literacy post-test described in \ref{sec:exp-literacy-questionnaire}. This questionnaire will ask for participants' assessment of the competition, the website, and energy literacy in general. \autoref{app:qualitative-feedback} lists the questions to be placed in the questionnaire.

\subsection{Post-Post-Competition Sustainable Conservation Questionnaire}

In early in the following semester (February 2011), the power data for floors will be re-examined to see whether conservation begun as part of the competition has been sustained months later. Floors with particularly high sustained conservation (compared to pre-competition average floor power), and those with low or non-conservation will be selected for an additional questionnaire, and possible face-to-face interviews to determine residents' self-assessment about why they were or were not sustaining the conservation gains made during the competition.

\section{Data Analysis}

Based on the raw data collected, we can perform analyses that allow the data to be understood at a higher level of abstraction.

\subsection{Power analyses}

\subsubsection{Minimum floor power}
\label{sec:min-floor-power}
Minimum floor power is the power consumed by each floor before residents move in and with all switchable devices (such as lights) turned off. This reveals the power used by the hidden infrastructure of a floor, and may be differ between floors. The value is measured by recording the kWh consumed by each floor over a period of time (preferably days to average out any periodic consumption spikes) and divided by the length of the time interval.

\subsubsection{Pre-competition average floor power}
Pre-competition average floor power is the power consumed by each floor after residents move in, but before the competition has begun. This reveals the power use profile of the floor's residents, and will almost certainly differ between floors. The value is measured by recording the kWh consumed by each floor over a long period of time (preferably weeks to average out any periodic consumption spikes) and divided by the length of the time interval.

\subsubsection{Pre-competition total monthly floor energy}
Pre-competition total monthly floor energy is the energy consumed by each floor after residents move in, but before the competition has begun. This reveals the power use profile of the floor's residents, and will almost certainly differ between floors. The value is measured by recording the kWh consumed by each floor over a long period of time (preferably weeks to average out any periodic consumption spikes) and extrapolated to a monthly value. Thus if 15 days of data are recorded, then the pre-competition total monthly floor energy would be twice the kWh value recorded for the 15 day period.

\subsection{Participant profile}

Since the website associates activity with a particular user, we can build a profile of each user that incorporates multiple sources of data. The participant profile includes:

\begin{itemize}
\item The number of times the participant logged into or visited the website
\item The participant's scores on the pre and post-competition energy literacy questionnaire (if they filled it out), and the difference between the scores.
\item The total energy consumption during the competition for the floor they live on
\item The number of Kukui Nut points the participant accumulated during the competition
\item The set of tasks the participant completed through the website
\end{itemize}

\subsection{Floor profile}

We can also aggregate data from all the participants on a floor to generate a floor profile. The floor profile includes:

\begin{itemize}
\item The total number of Kukui Nut points accumulated by participants on the floor, along with mean and standard deviation.
\item Number of residents that logged into the website at least once
\item The mean of the floor's participants post-competition energy literacy questionnaire scores
\item The mean difference of floor participants' pre and post-competition energy literacy questionnaire scores
\item The pre-competition total monthly floor energy
\item The total energy consumption during the competition for the floor
\item The post-competition total monthly floor energy
\end{itemize}

\section{Research Questions}

Rather than use a traditional treatment-based design, we have opted to provide all participants with equal access to all the tools and information we hypothesize may be helpful via the competition website. Thus, instead of positing hypotheses to be assessed using significance testing, we focus on descriptive and exploratory statistics based around research questions.

\subsection{Competition energy usage}
\label{sec:competition-energy}

\begin{description}
\item[Research question:] How did floor and dorm energy usage during the competition differ from the energy usage before the competition?
\end{description}

This question goes to the basic premise of any energy competition: that an organized competition that includes a quantitative measure of energy usage will lead to a reduction in energy usage. Based on other student housing energy competitions (see \autoref{sec:dorm-energy-competitions}) we expect the total energy consumption for each residence hall to be reduced compared to the usage before the competition. Also, the addition of energy feedback has been shown over several studies to lead to conservation values from 5\% to 15\% (see \autoref{sec:energy-feedback}). Energy usage on an individual floor may or may not be reduced compared to the pre-competition period, as some floors may not actively participate in the competition. One indication of floor participation would be the number of residents that logged into the website at least once, as well as the average number of Kukui Nut points per resident.

\subsection{Post-competition energy usage}
\label{sec:post-competition-energy}

\begin{description}
\item[Research question:] How did energy usage during the competition differ from the energy usage after the competition?
\end{description}

While the question in \autoref{sec:competition-energy} is the usual one asked about energy competitions, this question addresses the issue of sustainability: what do participants do after the incentives have been removed? Since student housing energy competitions typically last only a fraction of a semester, if the behavior is not sustained after the competition then the positive impact both to the environment and the institution's utility bill is limited.

We hypothesize that the energy usage after the competition will be higher after the competition is over, as the incentives will have been removed and overall focus on energy by the participants will be greatly reduced. However, some habits started during the competition may persist, and the website will still provide residents with energy usage feedback, which has been shown to reduce consumption.

Further, we hypothesize floor post-competition energy literacy is inversely correlated with energy consumption in the post-competition period. This is one of the goals of energy literacy: to make students understand the reasons for being concerned about energy use, and the techniques they can use to reduce their energy usage.

\subsection{Energy literacy assessment}

\begin{description}
\item[Research question:] How can energy literacy be assessed?
\end{description}

As discussed in \autoref{sec:energy-literacy}, there are four aspects to energy literacy: knowledge, skills, attitudes, and behaviors. Knowledge and attitudes can be assessed through a written instrument such as the energy literacy questionnaire described in \autoref{sec:exp-literacy-questionnaire}, but skills and behaviors are difficult to assess in a written format. The tasks afforded by the website, however, provide a means to assess whether participants have acquired the skills and behaviors required to be energy literate. Activities such as performing an energy audit are available to participants via the website, demonstrating the acquisition of a skill. Assessment of behaviors is more challenging, but the completion of commitments available to participants via the website demonstrate at least the intent to change behavior, though without external verification.

\subsection{Improving energy literacy via website tasks}

\begin{description}
\item[Research question:] How effective are the tasks available via the website at improving energy literacy and/or reducing energy usage?
\end{description}

The design of the website (described in \autoref{sec:website-design}) and the tasks it makes available to participants are specifically intended to increase the energy literacy of those that participate in them. The effectiveness of the set of tasks is critical to determine whether the added complexity of the tasks and Kukui Nut point system is worth the effort.

One way to assess the effectiveness of the tasks is looking at the correlation between the set of tasks completed by participants, and the change in their scores between the pre and post-competition energy literacy questionnaires. We can also examine the correlation between the set of tasks completed by participants on a floor and the energy consumption of the floor through the competition.

Beyond just making tasks available to participants, the website assigns Kukui Nut point values to each task. The point values have been assigned by hand based on several factors:

\begin{itemize}
\item the expected difficulty of the task
\item the expected time required for the task
\item a guess as to how useful the task is to increasing energy literacy and/or reducing energy consumption
\item the degree to which verification is possible (i.e. commitments, which are self-verified, are worth less than activities and goals)
\end{itemize}

Using the data collected from the competition, we can gain insight into whether the point assigned values were appropriate. One way to do this is look at the correlation between participant Kukui Nut scores and their change in energy literacy scores, as well as their post-competition energy literacy score. We can also look at total Kukui Nut scores for a floor and compare that to the energy consumption of the floor.

Tuning the scores of individual tasks can be done by examining the popularity of tasks. If a task was correlated with improved energy literacy scores but was not popular, then it makes sense to increase the number of Kukui Nut points assigned to it in future competitions. Tasks that required multiple verification attempts by participants before being accepted by administrators could also be candidates for increased Kukui Nut values.

Finally, we can get information about the participants' opinion of the effectiveness of the website using the questionnaire described in \autoref{sec:post-competition-feedback}.

\subsection{Relevancy of website}

\begin{description}
\item[Research question:] How relevant/useful is the website to the participants?
\end{description}

While great effort has gone into creating the competition website that is helpful to participants, failure of participants to use the website is a significant risk (see \autoref{sec:participant-engagement}). Therefore, it is important to assess how useful the website actually was to participants.

One crude way to assess relevancy is to examine the website usage logs. The number of participants that have actually logged into the website at least once is one measure, since the website can hardly be relevant to a participant if they have never logged in. Another measure would be the number of logins/visits per participant, as a single logon by a participant during the first few days of the competition would indicate that the participant did not find the website relevant or useful. The number of tasks completed per participant will provide an indication of how engaged participants are in the interactive aspects of the website.

In addition, we can get direct data about participants' opinion of how relevant and useful the website is through the post-competition questionnaire.

\subsection{Utility of floor-level near-realtime feedback}

\begin{description}
\item[Research question:] How important is floor-level near-realtime electricity usage feedback to achieving electricity conservation?
\end{description}

Provision of floor-level near-realtime electricity usage feedback is one of the key features of this research. Providing feedback at the floor level enables competition between floors (instead of just between buildings as is commonly done), allows individual participants to see their behavior changes reflected in electricity usage (which would be swamped by the activity if measured at the building level), and provides a reason for participants to communicate and collaborate with their floormates. Near-realtime feedback allows participants to perform their own `experiments' and see how their behavior changes electricity usage.

Unfortunately, the logistics of floor-level near-realtime electricity metering provide some of the most significant challenges to the research: the cost of purchasing the meters, the time and effort required to have them installed by electricians, and the lead time required to have the meters in place before the competition can begin.

Thus it is reasonable to ask whether deploying floor-level near-realtime electricity metering is worth the effort. Since we are not undertaking a `treatment'-style experiment where some floors or buildings receive the metering and others do not, we look at indirect indications of the utility of the metering. One source of data is the popularity of tasks (based on website logs) that make use of the floor-level near-realtime metering, such as the floor goal of determining the floor's minimum power or the floor goal of reducing energy use by 10\%.

The other source of data is participant responses to questions about the usefulness of the floor-level near-realtime metering in the post-competition feedback questionnaire.

Ultimately, the decision to use floor-level near-realtime metering in future energy competitions will be a based on a cost/benefit analysis, and the answer for one institution or situation might not be appropriate for all.