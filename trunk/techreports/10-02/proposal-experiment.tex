\chapter{Experiment}


\section{Data Sources}

\subsection{Power Usage Data}
\label{sec:power-usage-data}

We will record both instantaneous power and cummulative energy consumed on a floor by floor basis, beginning long before the competition starts and continuing indefinitely after the competition ends. The sampling rate will be at least 1 minute outside the competition period, and less than 1 minute during the competition period (target of 10 seconds). We can compute the following useful values based on this data:

\begin{itemize}

\item \emph{Minimum floor power} is the power consumed by each floor before residents move in and with all switchable devices (such as lights) turned off. This reveals the power used by the hidden infrastructure of a floor, and may be differ between floors. The value is measured by recording the kWh consumed by each floor over a period of time (preferably days to average out any periodic consumption spikes) and divided by the length of the time interval.

\item \emph{Pre-competition average floor power} is the power consumed by each floor after residents move in, but before the competition has begun. This reveals the power use profile of the floor's residents, and will almost certainly differ between floors. The value is measured by recording the kWh consumed by each floor over a long period of time (preferably weeks to average out any periodic consumption spikes) and divided by the length of the time interval.

\item \emph{Pre-competition total monthly floor energy} is the energy consumed by each floor after residents move in, but before the competition has begun. This reveals the power use profile of the floor's residents, and will almost certainly differ between floors. The value is measured by recording the kWh consumed by each floor over a long period of time (preferably weeks to average out any periodic consumption spikes) and extrapolated to a monthly value. Thus if 15 days of data are recorded, then the pre-competition total monthly floor energy would be twice the kWh value recorded for the 15 day period.

\end{itemize}

\subsection{Pre and Post-Competition Energy Literacy Questionnaires}
\label{sec:exp-literacy-questionnaire}

The energy literacy of participants will be assessed at the start and end of the competition. The assessment will be through a questionnaire that is presented to participants via the contest website as an activity that can be performed for Kukui Nut points. The pre-competition questionnaire will be made available only in the first week of the competition, while the post-competition questionnaire will be made available only in the final week of the competition. \autoref{app:energy-literacy} lists the questions that will make up the pre and post-contest questionnaires.

Since the website-administered questionnaire is simply a task that can selected by participants, there is the potential that only those participants that feel that they are energy literate will participate in the survey, leading to bias. For this reason, in addition to administration through the website, the questionnaire will be administered in person on paper to two randomly-selected floors. While the assignment of residents to a floor is not random, it is at least not self-selected. The questionnaire will be administered to the floors before the competition starts, and in the final week of the competition. The questionnaire will be removed from the activity lists of participants on the selected floors. However, those participants that fill out the survey on paper will receive Kukui Nut points just as if they had filled it out online.

\subsection{Website Log Data}

The contest website will extensively log data about participants' actions on the site. All participant actions and events will be logged with timestamp.

\subsection{Post-Competition Qualitative Feedback Questionnaire}

After the competition has ended, participants that used the website will be emailed a link to a qualitative questionnaire, as part of the energy literacy post-test described in \ref{sec:exp-literacy-questionnaire}. This questionnaire will ask for participants' assessment of the competition, the website, and energy literacy in general.

\subsection{Post-Post-Competition Sustainable Conservation Questionnaire}

In early in the following semester (February 2011), the power data for floors will be re-examined to see whether conservation begun as part of the competition has been sustained months later. Floors with particularly high sustained conservation (compared to pre-competition average floor power), and those with low or non-conservation will be selected for an additional questionnaire, and possible face-to-face interviews to determine residents' self-assessment about why they were or were not sustaining the conservation gains made during the competition.

\section{Research Questions}

\subsection{Energy usage}


\begin{itemize}

\item How can student housing residents be motivated to reduce their electricity usage?

\item How sustainable are any electricity usage reductions after the competition is complete?

\item How can energy literacy be assessed?

\item How does energy literacy impact sustained energy conservation?

\item What activities are most helpful in improving participants' energy literacy?

\item How effectively can the tools of behavior modification be instantiated in a web application?

\item How important is community-level near-real-time electricity usage feedback to achieving electricity conservation?

\end{itemize}


\section{Hypotheses}

\subsection{Conservation during competition}

\emph{Averaged across all participating floors, the total energy use during the competition will be less than the pre-competition total monthly floor energy usage.} Other student housing energy competitions have demonstrated that the competition will lead to energy conservation of at least X\% (see \autoref{sec:dorm-energy-competitions}). Also, the addition of near-realtime energy feedback has been shown over several studies to lead to conservation values from 5\% to 15\%.

Testing this hypothesis is straightforward. The collected power and energy data for each floor (see \autoref{sec:power-usage-data}) will show how much energy has been used during the competition, which can be compared to the average usage recorded during the pre-competition period.

\subsection{Conservation after competition}

\emph{Averaged across all participating floors, the total energy use after the competition will be less than the pre-competition total monthly floor energy usage.} Sustained energy conservation after student housing competitions has not been throughly investigated. We hypothesize that the integration of energy literacy into the competition will lead to at least some of the energy conservation fostered during the competition being sustained afterwards.

Testing this hypothesis is straightforward. The collected power and energy data for each floor (see \autoref{sec:power-usage-data}) will show how much energy has been used after the competition, and can be broken into weekly and monthly periods. The post-competition energy usage can be compared to the average usage recorded both before and during the competition period.

\subsection{Energy literacy's effect on conservation}

\emph{A floor's average score on the post-competition energy literacy questionnaire will be inversely corrolated with the floor's energy usage during and after the competition.} We hypothesize that sustained energy conservation is fostered by increased energy literacy, therefore we would expect those floors that are most energy literate would use the least energy.

The collected power and energy data for each floor (see \autoref{sec:power-usage-data}) will show how much energy has been used after the competition, and can be broken into weekly and monthly periods. The post-competition energy usage can be compared to the average usage recorded both before and during the competition period.

\begin{itemize}

\item Participants with Kukui Nut scores will have higher and more improved post-test energy literacy scores.

\item Improving residents' energy literacy will lead to sustained electricity conservation.

\item Floors with large electricity conservation during the competition period but with low energy literacy scores post-test will have greater rebound effect than floors with higher energy literacy scores post-test.

\item Residents that complete more activities (as specified by the competition website) will improve their energy literacy more than residents that do not participate.

\item How well do Kukui Nut scores correlate with post-test energy literacy?

\end{itemize}