\chapter{Introduction}

The world is in the grip of an energy crisis. Fossil fuels (oil, natural gas, coal) form the foundation of the world economy and their use is largely responsible for the industrialization and standard of living increases across the globe in the past century. However, as we shall see, the consumption of fossil fuels has led to a variety of problems that will have severe impacts on our environment and national economies.

We must reduce (and possibly eliminate) our use of fossil fuels to reduce these impacts. There is no `silver bullet' that will solve this energy crisis, it will require a series of changes taking place over decades. One fruitful area being explored is energy conservation. In this research, I address how to motivate people to conserve electricity by changing their behavior in the context of a university dormitory.

\section{Motivation}
\label{sec:motivation}

There are three primary motivations for phasing out the use of fossil fuels in favor of other forms of energy: climate change, peak oil, and energy insecurity. I examine each one in turn.

\subsection{Climate Change}
\label{sec:climate-change}

In 2007, the Intergovernmental Panel on Climate Change (IPCC) released its fourth assessment report \cite{IPCC-synthesis-report-2007}. The conclusions of this long-running analysis of studies on climate change and its effects are widely accepted as the consensus of the world's scientific community. They found that there is broad agreement that the climate is warming: air and ocean temperatures are higher, snow and ice are melting, and sea levels are rising. Further, natural systems are being affected: plant and animal ranges are moving towards the poles, and there are changes in fish and algae due to rising ocean temperatures.

The IPCC found that the warming of the climate was very likely due to anthropogenic greenhouse gas (GHG) emissions. GHG emissions from humans have increased by 70\% between 1970 and 2004. While there are a variety of GHG that impact climate change, \COtwo is the most important of the human-caused GHGs. Sea level rise in the second half of the 20th century was also very likely caused by humans, and rising sea levels have a potentially enormous impact on island communities like Hawai`i.

With current climate change policies, GHG emissions are projected to continue to increase this century. Further, there is no single technology that will mitigate the problem of climate change; a range of policies and innovations is required. The report lists both energy efficiency and individual behavior modification as suggested GHG mitigation strategies.

When discussing gasses in the atmosphere that are linked to climate change, there are a several possible terms. Greenhouse gas (GHG) is the most general term, referring to any gas in the atmosphere that leads to a greenhouse effect, trapping thermal radiation from the sun in the Earth's atmosphere. There are several GHGs in Earth's atmosphere: water vapor (H$_2$O), carbon dioxide (\COtwo), methane (CH$_4$), nitrous oxide (N$_2$O), ozone (O$_3$), and others. Each gas has different greenhouse effects on a molecule-by-molecule basis; for example, methane has a much greater greenhouse effect than \COtwo.  However, \COtwo is found in much greater quantities in the atmosphere than methane. Water vapor is the largest component of the greenhouse effect, but its contribution is not growing rapidly as \COtwo is, and humans don't have as much control over water vapor as they do over \COtwo emissions.

For these reasons, most climate change mitigation focuses on \COtwo emissions. In this context, sometimes \COtwo is referred to as simply carbon, as in \emph{carbon footprint}. For my purposes here, the terms GHG, \COtwo, and carbon can largely be considered interchangeable except when distinctions between \COtwo and other GHGs are being discussed.

\subsection{Peak Oil}

Fossil fuels are a finite resource that the world is consuming at an ever increasing rate. There is substantial evidence that the world has reached or will soon reach the peak of its oil production. After the peak, production will steadily decrease and production costs will increase as the remaining oil is more difficult to extract. When decreasing production is paired with skyrocketing demand, the result will be shortages and huge spikes in the price of oil. As discussed earlier, the industrialized world is highly dependent on oil, so large increases in the price of oil are expected to lead to widespread economic and social destabilization. This whole phenomenon is referred to as `peak oil'. [citation needed]

The effects of peak oil would be especially profound in Hawai`i, as oil the fuel used for over 75\% of the state's electricity production [cite State factbook] and virtually all transportation. It is estimated that in 2008 Hawai`i spent \$7 billion dollars on oil imports, nearly \$800K per hour! [citation needed]

\subsection{Energy Insecurity}

Fossil fuels are distributed non-uniformly across the planet, so many industrialized nations must import fossil fuels for their energy needs. Since fossil fuels are a crucial part of industrialized economies, this leaves these countries vulnerable to price spikes, shortages, or delivery problems. For example, the United States imports petroleum from Venezuela, Saudi Arabia, and Russia, all countries that it has had turbulent relationships with in the past [cite EIA docs]. Should our relationship with one or more countries sour, it could have serious negative implications for the US economy.

In Hawai`i, the situation is even worse. Hawai`i has no local fossil fuels, so it must import 100\% of its fossil fuel needs needs by sea. This leads to a very dangerous situation, since any significant interruption in shipping traffic could lead to major energy shortages. Jeff Mikulina, the executive director of the Blue Planet Foundation put it another way, describing Hawai`i as being one supertanker away from being Amish \cite{nyt2008HawaiiMoonShot}.

\section{Physical Concepts: Power and Energy}

When discussing energy, and in particular electricity, it is important to understand what power and energy are, and how they interrelate.

\subsection{Energy}

Energy is defined as the amount of work that can be done by a force. Most of us have an intuitive notion of energy: is makes things move, it heats things up, etc. There are many units used to measure energy: joules (a very small amount of energy), BTUs, calories. When talking about electricity, the most common unit is the watt hour, abbreviated as "Wh", which is equal to 3600 joules. A watt hour is the amount of energy required to to provide 1 watt of power for one hour. Note that from a certain perspective it is somewhat peculiar to measure energy in units that include power (watt), since power is defined in terms of energy in the first place. This underlines how central the concept of power is in most of our dealings with electricity.

\subsection{Power}

Power is defined as the rate of change for energy. As with any rate, it is expressed as a quantity of energy over a unit of time. The most common unit for power (and the one used in WattDepot) is the watt, abbreviated as "W". One watt defined as one joule (a measure of energy) per second. You might be familiar with a 60 watt incandescent light bulb, which expresses how much power it uses when turned on.

\subsection{Analogy To Cars}

Power and energy are closely related, but frequently confused concepts. As an analogy, think about a car. We can talk about the speed of a car (in miles per hour, or kilometers per hour) and we can also talk about a distance driven in a car (miles or kilometers). The speedometer in the car measures the speed (distance over time), while the odometer measures the distance traveled. Speed is a rate, like power, while distance is like energy.

When we talk about speeds, we usually talk about instantaneous measurements of speed. A speed limit is the maximum instantaneous speed at which you are allowed to drive, i.e. the car's speedometer should never register a speed greater than the limit. However, when we talk about distance driven, it only makes sense to talk about a distance driven between two locations, or the distance driven over a particular time interval. There is no such thing as an instantaneous distance driven, because in at a precise instant in time, the car is not moving.

\subsection{Power vs. Energy}

Since power is the rate of change of energy, if you know how power changes over time, you can determine how much energy was consumed or produced (the area under the power curve). Similarly, if you know how much energy was used over an interval of time, you can compute the average power over that period of time (but not the instantaneous power).

In our interactions with appliances, we usually talk about their power consumption and not their energy consumption. For example, we have 60 watt light bulbs, but we wouldn't generally talk about a 60 watt hour lightbulb (unless it consumed 60 watts and then burned out!). This is because power consumption is an intrinsic characteristic of things that use electricity, while the amount of energy is used by an electrical device is determined by how long you keep it plugged in or turned on. On the other hand, energy is very important to the utility that provides your electricity, since you are billed by how much energy you have used (typically in kilowatt hours).

The two key points to remember are: power is a rate, and we always talk about energy over an interval of time.

\section{Reducing Fossil Fuel Use}

There are two principle ways that fossil fuel use can be reduced: replacing fossil fuels with other preferable energy sources, and by reducing the total amount of energy consumed. For example, the state of Hawai`i has created the Hawai`i Clean Energy Initiative, which seeks to reduce Hawai`i's fossil fuel use by 70\% by 2030 through increasing the use local energy sources (for electricity and transportation fuel) to 40\% of demand and reducing demand by 30\% through efficiency and conservation \cite{HCEI-website}.

\subsection{Renewable Energy}

As described in \autoref{sec:motivation}, there are three problems with fossil fuels: their use is negatively impacting the global climate, they are finite, and the largest consumers of fossil fuels are not necessarily the largest owners of fossil fuels. Thus any replacement energy source should be better than fossil fuels in some or all of these ways. The ideal energy source would generate minimal GHG, would be virtually inexhaustible, and widely abundant. Energy sources that meet these first two requirements are often called \emph{renewable} energy sources (or sometimes \emph{sustainable} energy sources).

Renewable energy sources can generally be broken down into two categories. Some sources are \emph{dispatchable}, meaning that as demand for energy increases or decreases, the energy produced by the source can be changed to match the demand. For example, the gates on a hydroelectric dam can be opened and closed in a matter of seconds, tuning the output power as needed. As we will see in [add ref here], dispatchability is critical in running a power grid. The primary sources of dispatchable renewable energy are:

\begin{enumerate}

\item Hydroelectric energy: the motion of water in a river as it flows downhill is captured and converted mechanically into electricity.

\item Geothermal energy: heat from a deep well (sometimes near volcanic activity) is used to heat a working fluid, which then is mechanically converted into electricity. Note that wells may eventually cool to the point that they cannot generate electricity, making geothermal energy less than fully renewable.

\item Biofuel energy: a photosynthesizing crop is grown and then processed into either ethanol (for carbohydrate-rich crops) or biodiesel (for lipid-rich crops). The resulting biofuel is then burnt as a fossil fuel would be. This qualifies as renewable since the carbon released from burning the biofuel is recaptured in the next crop through photosynthesis. However, it is critical that the inputs to the growth of the the crop be carefully accounted for, or there is the risk that the energy to grow the crops could match or exceed the energy released when burned (as is true for corn-based ethanol).

\item Tidal energy: the rise and fall of ocean tides is captured and converted mechanically into electricity.

\end{enumerate}

It is worth pointing out that the dispatchable renewable sources largely fail the third renewable criterion of wide abundance. Hydroelectric energy requires water flowing from higher ground to lower ground, something not possible in flat and low-lying areas. Geothermal energy requires `hot rock' at depths that can be drilled to economically, while tidal energy requires both coastline and large differences between high tide and low tide. Biofuel is one bright spot, since the crops can be grown in one area, and the resulting biofuel transported to a power plant elsewhere. However, the costs of transporting the biofuel would have to be taken into account in that case.

Other renewable sources are \emph{intermittent} and thereby \emph{non-dispatchable}. Some days are sunny and clear, others are cloudy, leading to big differences in power output for sources powered by the sun. Some days the wind blows strongly and steadily, others days it is calm. Thus power from intermittent sources presents a problem when integrated into a power grid: where do you get power from when the sky is cloudy and the wind is calm? The primary sources of intermittent renewable energy are:

\begin{enumerate}

\item Solar energy: the light from the sun is harnessed directly into usable energy. The most familiar form of solar energy is the photovoltaic cell that turns sunlight into electrical power. Solar thermal energy uses sunlight to heat a working medium (such as water, oil, or sodium), which then turns a turbine or other mechanism to generate electrical power. Solar hot water heaters use sunlight to heat water that would otherwise be heated using electricity or gas.

\item Wind energy: the motion of air is converted mechanically into electricity.

\item Wave energy: the wave motion of ocean water is converted mechanically into electricity.

\end{enumerate}

Solar and wind energy meet the three renewable energy criteria, and since they are widely available, they are expected to be the primary sources of renewable energy growth. Photovoltaic panels are the most viable renewable energy source at small scales such as a single house, especially in sunny locales like Hawai`i.

\subsection{Energy Conservation}

The other way that fossil fuel use can be decreased is by decreasing the amount of energy that is consumed. Reduced energy consumption also eases the transition from fossil fuels to renewable energy sources. Amory Lovins coined the term \emph{negawatt} to refer to power that has been conserved and therefore does not need to be generated \cite{Kolbert2007Mr-Green}.

Negawatts can be `generated' in two basic ways: by increasing the efficiency of devices that consume energy, and by changing people's behavior such that they use less energy.

\subsubsection{Energy Efficiency}

Many energy consuming devices have become more efficient over time. For example, incandescent light bulbs are increasingly being replaced with compact fluorescent bulbs that produce an equivalent amount of light but use only 20--30\% of the energy. The negawatts generated through use of more energy efficient devices has the primary advantage of not changing the functionality of the device: a more efficient refrigerator keeps food cold just as ably as a less efficient one.

As \autoref{sec:efficiency-rebound} shows, energy efficiency can also lead to rebound effects, where the negawatts saved through efficiency are lost to additional consumption.

\subsubsection{Behavior Change}

Changing people's behavior with respect to energy holds significant promise in reducing energy use. Darby's survey of energy consumption research found that identical homes could differ in energy use by a factor of two or more \cite{darby-review-2006}. There are three areas of focus to create and sustain behavior changes with positive environmental impact:

\begin{enumerate}

\item Feedback systems to make people aware of how their behavior affects the environment.

\item Environmental `literacy' to help people understand why behavior change is necessary, and what behavioral changes would have a positive environmental impact.

\item A toolbox of techniques to encourage, motivate, and remind people to make the behavior changes necessary for positive environmental impact.

\end{enumerate}

As Lord Kelvin is famously reputed to have said, ``If you can not measure it, you can not improve it.'' In the case of electricity usage, for many people the only feedback they receive is a monthly bill detailing the number of kilowatt-hours used over the course of the last month. Ed Lu of Google analogizes this as if there were no prices on anything at the grocery store, and shoppers were just billed at the end of the month \cite{Helft2008Googles-Energy}. Office workers or dormitory residents might never see any feedback on how much electricity they are using! Darby's review of many studies using feedback systems to facilitate energy conservation found that savings of 5--15\% could be achieved \cite{darby-review-2006}. \ref{sec:energy-feedback} explores the topic of energy feedback in depth.

