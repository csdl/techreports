%% Intended to be included into a larger document
\chapter{Related Work}

%% Add more to introduce topics in this chapter?

\section{Does Energy Efficiency Reduce Carbon Emissions?}
\label{sec:efficiency-rebound}

Many governmental plans to reduce GHG emissions involve improving energy efficiency in the home, in industry, and in transportation. While intuitively it would seem that increased energy efficiency would lead to decreased energy usage, and thereby reduced GHG emissions, surprisingly there is some evidence (both theoretical and empirical) that energy efficiency actually increases energy usage! Saunders dubbed this unintuitive notion the Khazzoom-Brookes Postulate based on conclusions reached independently by those two researchers \cite{saunders-1992}. \fxnote{Insert references to Khazzoom and Lovins papers here, after I read them.}

Using neoclassical growth theory, Saunders finds that increased energy efficiency makes energy seem cheaper, thus allowing it to be substituted for labor in production. Increased energy efficiency also increases overall economic growth, which leads to increased overall energy usage.

In discussing this effect, rebound is defined as the difference between the expected amount of energy savings from an improvement in energy efficiency, and the actual observed effect. For example, if an improvement in metal smelting technology reduces the energy required to smelt by 20\%, but the energy consumed by the metal smelting industry only goes down by 10\% then the rebound is 50\%. If the rebound is greater than 100\%, then backfire is taking place (the efficiency measure has backfired) \cite{Hanley2008Do-increases-in}. There is some debate over whether the predicted increases in energy usage will actually take place in the real world. Laitner suggests via a simple analysis that the rebound effect is small (2.4\%) \cite{Skip-Laitner:2000yg}. His equation relates future carbon emissions to current carbon emissions, increases in GDP and energy costs, and elasticities of income and energy prices to arrive at this conclusion. He goes on to a further analysis done by the Environmental Protection Agency and Lawrence Berkeley National Labs using the National Energy Modeling System showing that an ``energy-efficient/low-carbon technology path'' would suffer from a rebound effect of only 2.2\%. However, he acknowledges that consumer choices about energy usage could erode gains from efficiency, such as turning up the furnace thermostat because the cost of doing so has been effectively reduced.

The issue of consumer choices is a real one. Over the last 25 years, automobiles have been made more efficient through ``increasing the efficiency of the engine and transmission, decreasing weight, improving tires and reducing drag'' \cite{Heywood2008Fueling-Our-Future}. However, these improvements have been traded for vehicles that are larger, heavier, and faster, which has led to only modest improvements in overall fuel efficiency. This is an example of how energy efficiency may not always lead to reduced GHG emissions without motivating automobile users (and manufacturers) to buy and make fuel efficient vehicles.

Other authors find that rebound and even backfire are the likely results of economy-wide improvements in energy efficiency. The analysis of Hanley et al. finds that backfire occurs when economy-wide improvements in energy efficiency are made \cite{Hanley2008Do-increases-in}. Their theoretical analysis finds that if energy demand is relatively price-elastic (demand increases when prices are low and decreases when prices are high), then backfire will occur. Empirical evidence of rebound and backfire are hard to come by because there are indirect system-wide effects due to the increased efficiency, and these indirect effects are difficult to measure. The authors created a Computable General Equilibrium (CGE) model of Scotland that simulates the economy and environmental impact based on the inputs and outputs of the system. Using this model, almost all scenarios eventually result in backfire. They note that since non-renewable energy sources use more energy in their production than renewable sources, increased energy efficiency lowers the cost of non-renewables compared to renewables, financially favoring the use of non-renewables. Efficiency in energy production is therefore associated with a decrease in the use of energy from renewable sources. The authors also urge caution when reviewing sustainability measures such as the ratio of Gross Domestic Product (GDP) to energy usage or carbon emissions, because even if the ratio increases (less carbon per unit GDP), if the GDP as a whole increases faster, the absolute carbon emitted will increase. They suggest that backfire could be prevented by combining energy efficiency improvements with taxes on energy use or a carbon tax. Since energy efficiency effectively reduces the cost of energy, the savings could offset the cost of additional taxes, thereby blunting any impact on economic activity.

It would appear that any energy efficiency improvements will have some degree of rebound effect, thus a naive pursuit of energy efficiency without taking into account the context around the improvements could risk reducing their effectiveness, or even making them counterproductive! While many of the analyses deal at the macroeconomic level, it is not hard to think of individual scenarios where efficiency could actually increase personal usage, such buying two energy efficient refrigerators to replace one older energy-hogging refrigerator. The key to ensuring that energy efficiency improvements on the micro level lead to less GHG emissions is to combine efficiencies with changes in behavior.