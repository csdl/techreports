\chapter{Vision}

This chapter lays out the vision behind the PET research. We start with a description of the problems users face that PET is trying to address. Then we look at how existing solutions have tried to address these problems. This is followed by set of potential ``secret sauces'': novel contributions to this research area that improve upon existing approaches. We end with scenarios that show how PET would be used in different situations.

\section{Problem Statement}
\label{sec:problem-statement}

There are an increasing number of people interested in making personal changes to reduce their contribution to climate change. We focus our efforts on these people who are actively seeking to reduce their carbon footprint. The first problem these individuals face is determining what their carbon footprint is right now. Once they have determined their current footprint, they need to be able to track how that footprint increases or decreases based on the choices they make as they go about their daily activities. Finally, users need ways to decide how they might best go about reducing their carbon footprint.

\begin{enumerate}
	\item Calculate carbon footprint baseline
	\item Track carbon footprint over time
	\item Support \& motivate carbon reduction decision-making process
\end{enumerate}

\section{Existing Solutions}

As discussed in \autoref{sec:related-systems} and \autoref{sec:sensors}, there are a variety of systems and products that attempt to address these problems. We revisit those systems here with an eye towards how each addresses the three problems from \autoref{sec:problem-statement}, and what opportunities may exist for PET to improve on their capabilities.

\subsection{Calculating Carbon Footprint}

There are three techniques used by the related systems to calculate the user's carbon footprint. The first is the manual method, requiring users to enter in their consumption data by hand. mobGAS (\autoref{sec:mobgas}) and the web-based carbon calculators (\autoref{sec:carbon-calculators}) use manual data collection. To provide an accurate footprint, manual systems require users to answer a variety of questions that they will often not know the answers to. Faced with this type of situation, most users will simply guess the answer, which will likely be biased by their pre-existing beliefs about their consumption such as ``I drive too much'' or ``I'm green''.

EcoIsland (\autoref{sec:ecoisland}) used a hybrid of sensor data and manual data input. Surprisingly, they used a sensor to calculate the baseline usage, and then manual reporting to track ongoing usage (perhaps due to technical issues). Some participants said that having to report their consumption manually made them more motivated, so there may be some place for manual reporting even in a mostly-automated sensor-based system.

The third method used for calculating carbon footprints is the use of information sensors that gather data that has already been collected for other purposes. Personal Kyoto (\autoref{sec:personal-kyoto}) gathers electricity usage data from a utility company website (collected for billing and capacity planning), while Dopplr (\autoref{sec:dopplr}) uses air travel itinerary data entered to facilitate serendipitous encounters with friends and colleagues. Each system only attempts to provide a footprint based on one type of data (electricity usage, air travel), and does not provide an overall footprint. Personal Kyoto doesn't provide a carbon footprint directly, it provides a goal for reducing electricity usage that is based on the Kyoto climate change treaty. However, Personal Kyoto does provide a projection of usage based on limited data, alerting the user when the average usage is being computed from less than 12 months of data.

Notably, no system used sensors to calculate users' carbon footprint. Since sensor data collected incrementally, it would take some time before the data could be used to give an accurate footprint. It would also take multiple sensors to collect data similar to what is elicited from a web-based carbon calculator.

\subsection{Tracking Carbon Footprint}

Many of the systems examined provide a means for user to track their carbon footprint. As discussed in the previous section, mobGAS and  EcoIsland use manual data entry to track changes in users' carbon footprint. While potentially useful as a one-time baseline, users cannot be expected to regularly engage in manual data input, especially over a wide-ranging list of possible carbon emission sources.

StepGreen \autoref{sec:stepgreen} and iamgreen (\autoref{sec:iamgreen}) provide a list of ``green'' actions that users can take to reduce their carbon footprint. Users indicate which actions they have taken (including how many times they have performed some periodic actions) and those actions are converted into some sort of visible assessment. StepGreen uses action lists to estimate quantitative savings of both money and carbon, while iamgreen counts actions and displays them as leaves on a tree.

Unfortunately, green action lists have some severe limitations. Because they describe a variety of complex actions, it would be difficult to automatically sense when users had undertaken an action. In fact neither of the systems reviewed provide any automated means for data acquisition, probably for this reason. Manual data entry, especially for periodic actions (``how many times, since Sunday, have you turned off the lights after leaving a room for 10 minutes?'') will be spotty at best, since they rely on users' memory of minutiae. Faced with a list of actions to update daily or weekly, most users will choose not to participate.

Also, there are fundamental limitations on what green action lists can infer. Since they only capture information about a specific set of green actions, they can only provide estimates of \emph{savings}, rather than an overall carbon footprint. Thus a user that takes a trans-Pacific flight round trip and then recycles aluminum when they get home will see that they saved $x$ pounds of \COtwo, when in fact the air travel has obliterated the recycling savings by multiple orders of magnitude.

As discussed in the previous section, Personal Kyoto and Dopplr provide continuous tracking of carbon footprint (or electricity usage in Personal Kyoto's case) as long as new data is available from their data source.

Whole home electricity meters (\autoref{sec:whole-home-meters}) are able to track the electrical usage of a user's home over time, and some can convert the kilowatt hours used into tons of \COtwo emitted. However, these systems are typically designed to provide data on a day-to-day or month-to-month basis, rather then tracking a user's footprint over longer periods of time. Building energy displays (\autoref{sec:building-energy-displays}) often provide indications of the monitored building's carbon emissions, with multiple visualizations to make the emissions more accessible. Since the displays are usually customized for each building, they can track other sources of consumption, such as natural gas or water.

Transportation tracking systems track carbon emissions using GPS data from mobile devices. PEIR (\autoref{sec:peir}), Carbon Diem (\autoref{sec:carbon-diem}), and Ecorio (\autoref{sec:ecorio}) fall into this category. In terms of carbon footprint, all three only calculate the contribution from transportation. PEIR makes the resulting footprint available on a website, Ecorio provides the footprint only on the mobile device, and Carbon Diem's output is unknown since it is not publically available.

\subsection{Supporting And Motivating Carbon Emissions Reduction}

The reviewed systems use a variety of techniques to support users in their efforts to reduce carbon emissions, and to motivate them to change their behavior. In StepGreen and iamgreen, user's activities revolve around green action lists. As discussed in the previous section, green action lists can be used to record actions and the carbon footprint associated with those actions. When used as suggestions for improvement, green action lists also face problems. Usually the actions are presented as a unstructured list, leaving users to read through them all to see which ones are relevant and/or feasible. If the system designer wants to provide a quantitative assessment of the impact of an action, the actions have to be generated by someone with the time and inclination to investigate the impact. Often this will mean that the system designers themselves have to generate the actions, which will likely lead to the list of actions changing infrequently (such as with StepGreen). If the list of actions is ``crowdsourced'' directly from users, then it will be difficult to provide any quantitative assessment of the result of the action. Getting the actions from users also will require a means to weed out inappropriate or duplicate suggestions, though there are techniques for doing so such as voting, moderation, etc. Green action lists would be much more helpful to users if the system suggesting actively made use of the information available when providing suggestions. For example, if the user has already indicated that they do not own a car, suggesting that they roll down their car windows instead of using the air conditioner is not helpful. Instead, the system should provide pointers to mass transit and carpooling options.

While green action lists have the advantage of being easy to implement and intuitive for users, on the whole they do not appear to be a good way for long term assessment of users' environmental impact. A solution to the problems of green action assessment would be to acquire data using automated sensors (as discussed in \autoref{sec:sensors}). While the set of actions that could be detected with a small number of sensors would be greatly reduced, it is expected that the sensors could focus on the largest contributors to environmental impact. Once assessment is being done using sensor data, some of the problems of using green action lists to suggest improvements described earlier disappear. User-sourced suggestions become more viable, since their impact can be detected via sensor data.

Several systems provide integration into other social networking applications. StepGreen has a Facebook application, a MySpace profile widget, and can send Twitter messages. StepGreen's Facebook support supposedly allows sending information about one's saved emissions, but it doesn't appear to work as of this writing (February 19, 2009). PEIR supposedly has a Facebook application, which allows comparison of one's carbon emissions from transportation with one's social network. iamgreen is a Facebook application, and allows users to recommend green actions to friends. Other than iamgreen's recommendation feature, the social networking functionality of the systems seems to be first generation: simple output of data from the system to the social networking application.

mobGAS provides a global user ranking page, presumably to encourage users to compete to have the lowest emissions. While using competition to reduce usage is a popular concept, it has obvious problems when applied to carbon emissions. If users become obsessed with their rankings, they may choose to tamper with their emissions data to get a lower ranking. This seems particularly problematic in a system based on manual reporting such as mobGAS. Rankings also have the problem that for carbon emissions, lower is better, leading to a floor on carbon emissions (zero \COtwo emitted). This is different from something like a game where higher scores are usually better than lower scores.

The Virtual Polar Bear provides a unique method for motivating users to be more environmentally aware through a virtual pet. While there is significant potential in using interactive multimedia to motivate reductions in carbon emissions, one needs to be careful not to reinvent Clippy.

EcoIsland combines several motivators: characters on a virtual island with a rising ocean level, purchasing decorations for the island through lower emissions, viewing other users' islands, and green action lists. The problem with the game-like visualization is that (as the EcoIsland authors indicated) users appeared to be motivated by game issues rather than environmental issues. This can lead to measurement dysfunction, where users take actions to maximize the game play aspects, possibly to the detriment of the underlying environmental goals.

Personal Kyoto motivates users through the Kyoto treaty that the system is named after. By setting the goal for reduced personal electricity usage based on an international treaty, the system taps into desire to be a part of the larger movement to halt global warming. It also resonates particularly well with the targeted users (people in the USA) since the USA never ratified the Kyoto treaty.

The Ecorio mobile phone application supports users in reducing their transportation emissions through driving efficiency detection, providing suggestions for reducing emissions, enabling ``what if'' scenarios, and purchasing carbon offsets.

UbiGreen (\autoref{sec:ubigreen}) motivates users to increase their use of green modes of transportation like walking and biking using a tree visualization that is presented as a glanceable display on a mobile phone. The tree visualization (which is more detailed than the one provided by iamgreen) is an interesting alternative to displaying a number or a graph of carbon emissions over time. Making the display glanceable seems like a wise choice, ensuring that users don't have to log into a website to see the results of their actions.

\section{Secret Sauce Ideas}

\begin{itemize}
	\item \emph{Combine data from diverse sensors into unified carbon footprint.} Some of the systems examined get their data from sensors, and some manual systems (like web-based carbon footprint calculators) provide an overall carbon footprint, but no system examined provides a sensor-based overall carbon footprint. Collecting data using sensors reduces the overhead for users and reduces the opportunities for reporting bias, and providing an overall footprint allows users to make rational decisions on how best to reduce their footprint.

	\item \emph{Allow users to perform ``what if'' explorations on their data.} Once users are aware of their combined carbon footprint, PET would provide them the means to explore how changes in their future behavior would change their carbon footprint. For example, users can find out whether a planned transpacific trip would cause their carbon footprint to go over their personal footprint goal.
	
	\item \emph{Recommend actions to reduce the user's footprint based on their data.} Once PET has data about the user, it should be able to make suggestions that are most relevant to that user's situation. For example, someone living in a rented apartment does not usually have the option of installing a solar hot water heater. Based on the relative contributions to their footprint from different types of consumption, the system can focus first on ideas for reducing the largest segments.
	
	\item \emph{Provide users with more advanced goal settings to motivate users to reduce their footprint.} Personal Kyoto's Personal Kyoto Goal (see \autoref{sec:personal-kyoto}) provides a goal that is grounded in an actual treaty, which gives it more weight than some random value set by the user. Providing a way to tie goals back to the science of climate change would give them more legitimacy, and hopefully motivate users to meet them. An interactive version of the analysis that motivated WattzOn could be useful: users pick the level of environmental impact they find acceptable (percentage of species lost, cm of sea level rise, etc), convert that to a concentration of atmospheric \COtwo, and then convert that to a personal carbon budget.

	\item \emph{Support users' efforts to reduce their footprint through actions suggested by other users who are nearby or in similar situations.} As discussed earlier, green action lists tend to be generic so they can potentially be applicable to anyone. However, users need specific information on how to alter their behavior, something generic actions cannot provide. Users can submit strategies and tips on how to reduce one's footprint, and each item will be tagged with metadata such as the user's location, how popular the item is, etc. Items can be recommended to a user using collaborative filtering, social filtering, spatial filtering, and popularity. Since users' footprints are tracked by sensors, the actions can exist purely to support users efforts and not to assess the effectiveness of their efforts. The proof will be in the pudding.
	
	\item \emph{Create a novel visualization of the carbon footprint data that makes users more aware of the information.} There are many different ideas being explored in this area, it would be nice to make a contribution beyond a pie chart. Perhaps the ICU metaphor being employed by Hackystat could also be applied to carbon footprint data.

\end{itemize}

Figuring out how to evaluate something like a crowdsourced action recommendation system seems really difficult.

\section{Scenarios}

\subsection{Kimo's Showers}

As an example of PET usage, consider Kimo. Kimo starts using PET, with sensors measuring his energy usage, building up data about his daily activities. Kimo feels guilty about taking long showers, which he can see via the PET website leads to higher electricity usage than his neighbors. The PET system recommends two ways to reduce electricity usage: reduce the temperature on his hot water heater, and/or install a solar hot water heater. However, Kimo likes his showers hot and cannot currently afford a solar hot water heater, so he declines both those options. The PET website provides a unified view of Kimo's carbon footprint, which leads him to discover that taking the bus to work once a week instead of driving is equivalent to many hot showers. Kimo decides to take the bus to work on Fridays. After a few weeks, he decides The Bus isn't so bad, so he starts taking the bus every day.