%%%%%%%%%%%%%%%%%%%%%%%%%%%%%% -*- Mode: Latex -*- %%%%%%%%%%%%%%%%%%%%%%%%%%%%
%% uhtest.tex -- 
%% Author          : Robert Brewer
%% Created On      : Wed Sep 30 16:08:49 1998
%% Last Modified By: Robert Brewer
%% Last Modified On: Mon Oct  5 16:17:16 1998
%% RCS: $Id: uhtest.tex,v 1.2 1998/10/06 02:04:56 rbrewer Exp rbrewer $
%%%%%%%%%%%%%%%%%%%%%%%%%%%%%%%%%%%%%%%%%%%%%%%%%%%%%%%%%%%%%%%%%%%%%%%%%%%%%%%
%%   Copyright (C) 1998 Robert Brewer
%%%%%%%%%%%%%%%%%%%%%%%%%%%%%%%%%%%%%%%%%%%%%%%%%%%%%%%%%%%%%%%%%%%%%%%%%%%%%%%

%!!!!!!!!!!!!!!!!!!!!!!!!!!!!!!!!!!!!!!!!!!!!!!!!!!!!!!!!!!!!!!!!!!!!!!!!!!!!!!
%!NOTE: This example file has been prepared according to the University of
%!      Hawaii Style & Policy Manual for Theses and Dissertations dated
%!      "Revised February 1998". If you have one with a later date, you may
%!      need to make revisions to this document as well. In any event, making
%!      sure your thesis complies with Graduate Division guidelines is
%!      ultimately your responsibility. Caveat LaTeXtor. :)
%!!!!!!!!!!!!!!!!!!!!!!!!!!!!!!!!!!!!!!!!!!!!!!!!!!!!!!!!!!!!!!!!!!!!!!!!!!!!!!

%% The options are (you can only choose one from each group):
%%
%% 10pt, 11pt, 12pt: chooses the point size for the document. "11pt" is the
%%                   default.
%%
%% oneside, twoside: whether you want your document onesided or twosided. Note
%%                   that twosided is not guaranteed to work, and style
%%                   guidelines prohibit double sided printouts on final
%%                   copy. "oneside" is the default.
%%
%% draft, final: when printing drafts you can save a lot of paper by using the
%%               "draft" option. It switches to single spacing, displays overful
%%               hboxes with a black box, prints a version number on title page 
%%               and omits signature page. Of course for the final copy make
%%               sure to use the "final" option! "final" is the default.
%%
%% cm, times, palatino, newcent, bookman: switches between different font
%%                                        sets. "cm" is the Computer Modern
%%                                        font (TeX's default), the rest are
%%                                        PostScript fonts. "times" is the
%%                                        default.
%%
%% thesis, dissertation: switches between the style for a master's thesis and a 
%%                       Ph.D. dissertation. The differences are fairly minor
%%                       and limited to the front matter. "thesis" is the
%%                       default.
%%
%% actual, proposal: switches between actual document and proposal mode. In
%%                   proposal mode: the title page is simplified, the
%%                   version number is always printed, and the signature page
%%                   is omitted.
%%
%%% Load the uhthesis2e document class
\documentclass[11pt,final,times,dissertation,proposal]{uhthesis2e}
%\documentclass[11pt,draft,times,dissertation,proposal]{uhthesis2e}

%% hyperref package complains if this isn't here. Might be unneeded when
%% I switch to the new UH thesis style
\paperheight = 11in

%%% Load some useful packages:
%% New LaTeX2e graphics support
\usepackage{graphicx}
%%	using final option to force graphics to be included even in draft mode
%\usepackage[final]{graphicx}
%% Tell graphicx the default directory for all figures
\graphicspath{{figures/final/}}

%% Enable subfigure support
\usepackage{subfigure}

%% Make subsubsections numbered and included in ToC
\setcounter{secnumdepth}{3}
\setcounter{tocdepth}{3}

%% Package to linebreak URLs in a sane manner.
\usepackage{url}

%% Define a new 'smallurl' style for the package that will use a smaller font.
\makeatletter
\def\url@smallurlstyle{%
  \@ifundefined{selectfont}{\def\UrlFont{\sf}}{\def\UrlFont{\small\ttfamily}}}
\makeatother
%% Now actually use the newly defined style.
\urlstyle{smallurl}

%% Define 'tinyurl' style for even smaller URLs (such as in tables)
\makeatletter
\def\url@tinyurlstyle{%
  \@ifundefined{selectfont}{\def\UrlFont{\sf}}{\def\UrlFont{\scriptsize\ttfamily}}}
\makeatother

%% Provides additional functionality for tabular environments
\usepackage{array}

%% Set up to create an index
\usepackage{makeidx} 
\makeindex

%% Puts space after macros, unless followed by punctuation
\usepackage{xspace}

%%% Personal macros
%% Tired of typing CO2 so many times, requires xspace package
\newcommand{\COtwo}{CO\ensuremath{_2}\xspace}
%% Hawai`i with okina
\newcommand{\Hawaii}{Hawai`i\xspace}
%% Manoa with kahako
\newcommand{\Manoa}{M\=anoa\xspace}

%% Provides customization of lists
\usepackage{enumitem}

%% Now define question list type
\newlist{question}{enumerate}{1}
\setlist[question]{resume, label=\textbf{\arabic*.}}

%% Define multiple choice answer list type
\newlist{answer}{enumerate}{1}
\setlist[answer]{label=\alph*)}

%% Allows insertion of fixme notes for future work
%% Note, remove status=draft when printing final version!
\usepackage[footnote, nomargin, status=draft]{fixme}
% turned off marginclue because it generates hbox overflows for each note :(
%\usepackage[footnote, nomargin, marginclue, status=draft]{fixme}


%% Make URLs clickable
\usepackage[colorlinks, bookmarks=false]{hyperref}
%\usepackage[colorlinks, bookmarks=true, backref]{hyperref}

%% Make links to captions point to the figure, not just the caption at bottom
\usepackage[all]{hypcap}

%% Set up to create a glossary
%\usepackage[toc]{glossaries}
%\makeglossaries

%% Since I'm using the LaTeX Makefile that uses dvips, I need this
%% package to make URLs break nicely
\usepackage{breakurl}

% correct bad hyphenation here
\hyphenation{strong-ly}

%%% End of preamble
\begin{document}

%%% Declarations for Front Matter. Capitalize all of these values
%%% "normally". This allows the document class to format them properly.
%% Full title of thesis or dissertation, capitalized like a title should be.
\title{Motivating Sustained Energy Conservation and Increasing Energy Literacy In The Context of A Student Housing Energy Competition}
%% Your name, capitalized normally. Do not include any titles like Dr.
\author{Robert S. Brewer}
%% Month in which you intend to receive your degree (i.e. graduation).
%% Presumably this will be one of: May, August, or December.
\degreemonth{May}
%% Year of expected graduation.
\degreeyear{2010}
%% Type of degree to be conferred.
\degree{Doctor of Philosophy}
%% This is the chairperson of your committee. Do not use titles like Dr.
\chair{Philip M. Johnson}
%% The other members of your committee, seperated by "\\". Again, no titles,
%% and it is customary to list the outside committee member (if you have one)
%% last.
\othermembers{Foo Bar\\
Baz Biff\\
Scott Robertson\\
Anthony Kuh}
%% This is the total size of your committee, including the chairperson. The
%% signature page routine will only handle up to 6 members; if you have more
%% than that you will need to modify the document class.
\numberofmembers{5}
%% The field in which you are obtaining your degree, capitalized normally.
\field{Computer Science}
%% The version number of your document. Consistent use of this will enable you
%% to tell old drafts from new ones. Final actual documents omit this
%% automatically so you can use it without fear of submission problems at the
%% end. If you do not define this parameter, it defaults to "1.0.0".
\versionnum{0.3.0}

%%% Create the title page from all the information above. Note that the
%%% titlepage is outside the front matter.
\maketitle

\begin{frontmatter}

%%% Create the signature page (when indicated by the options)
\signaturepage

%%% Create the copyright page
%\copyrightpage

%%% Bring in the dedication page from external file
%%%%%%%%%%%%%%%%%%%%%%%%%%%%%%% -*- Mode: Latex -*- %%%%%%%%%%%%%%%%%%%%%%%%%%%%
%% uhtest-dedication.tex -- 
%% Author          : Robert Brewer
%% Created On      : Fri Oct  2 16:29:01 1998
%% Last Modified By: Robert Brewer
%% Last Modified On: Fri Oct  2 16:29:20 1998
%% RCS: $Id: uhtest-dedication.tex,v 1.1 1998/10/06 02:07:25 rbrewer Exp $
%%%%%%%%%%%%%%%%%%%%%%%%%%%%%%%%%%%%%%%%%%%%%%%%%%%%%%%%%%%%%%%%%%%%%%%%%%%%%%%
%%   Copyright (C) 1998 Robert Brewer
%%%%%%%%%%%%%%%%%%%%%%%%%%%%%%%%%%%%%%%%%%%%%%%%%%%%%%%%%%%%%%%%%%%%%%%%%%%%%%%
%% 

\begin{dedication}
\null\vfil
{\large
\begin{center}
To myself,\\\vspace{12pt}
Perry H. Disdainful,\\\vspace{12pt}
the only person worthy of my company.
\end{center}}
\vfil\null
\end{dedication}


%%% Bring in the acknowledgements section from external file
%%%%%%%%%%%%%%%%%%%%%%%%%%%%%%% -*- Mode: Latex -*- %%%%%%%%%%%%%%%%%%%%%%%%%%%%
%% uhtest-acknowledgements.tex -- 
%% Author          : Robert Brewer
%% Created On      : Fri Oct  2 16:29:43 1998
%% Last Modified By: Robert Brewer
%% Last Modified On: Fri Oct  2 16:29:52 1998
%% RCS: $Id: uhtest-acknowledgements.tex,v 1.1 1998/10/06 02:06:54 rbrewer Exp $
%%%%%%%%%%%%%%%%%%%%%%%%%%%%%%%%%%%%%%%%%%%%%%%%%%%%%%%%%%%%%%%%%%%%%%%%%%%%%%%
%%   Copyright (C) 1998 Robert Brewer
%%%%%%%%%%%%%%%%%%%%%%%%%%%%%%%%%%%%%%%%%%%%%%%%%%%%%%%%%%%%%%%%%%%%%%%%%%%%%%%
%% 

\begin{acknowledgements}
I want to ``thank'' my committee, without whose ridiculous demands, I
would have graduated so, so, very much faster.
\end{acknowledgements}


%%% Bring in the abstract section from external file
%%%%%%%%%%%%%%%%%%%%%%%%%%%%%% -*- Mode: Latex -*- %%%%%%%%%%%%%%%%%%%%%%%%%%%%
%% uhtest-abstract.tex -- 
%% Author          : Robert Brewer
%% Created On      : Fri Oct  2 16:30:18 1998
%% Last Modified By: Robert Brewer
%% Last Modified On: Fri Oct  2 16:30:25 1998
%% RCS: $Id: uhtest-abstract.tex,v 1.1 1998/10/06 02:06:30 rbrewer Exp $
%%%%%%%%%%%%%%%%%%%%%%%%%%%%%%%%%%%%%%%%%%%%%%%%%%%%%%%%%%%%%%%%%%%%%%%%%%%%%%%
%%   Copyright (C) 1998 Robert Brewer
%%%%%%%%%%%%%%%%%%%%%%%%%%%%%%%%%%%%%%%%%%%%%%%%%%%%%%%%%%%%%%%%%%%%%%%%%%%%%%%
%% 

\begin{abstract}
Abstract goes here, and will be written once the proposal is mostly done.
\end{abstract}


%%% Generate list of FiXmes, will be silent in final mode
\listoffixmes

%%% Generate table of contents
\tableofcontents

%%% Generate list of tables
\listoftables

%%% Generate list of figures
\listoffigures


\end{frontmatter}

%%% Include each chapter
\chapter{Introduction}

The world is in the grip of an energy crisis. Fossil fuels (oil, natural gas, coal) form the foundation of the world economy and their use is largely responsible for the industrialization and standard of living increases across the globe in the past century. However, as we shall see, the consumption of fossil fuels has led to a variety of problems that will have severe impacts on our environment and national economies. \fxfatal{Need rewritten introduction}

We must reduce (and possibly eliminate) our use of fossil fuels to reduce these impacts. There is no `silver bullet' that will solve this energy crisis, it will require a series of changes taking place over decades. One fruitful area being explored is energy conservation. In this research, I address how to motivate people to conserve electricity by changing their behavior in the context of a university dormitory.

\section{Motivation}
\label{sec:motivation}

There are three primary motivations for phasing out the use of fossil fuels in favor of other forms of energy: climate change, peak oil, and energy insecurity. I examine each one in turn.

\subsection{Climate Change}
\label{sec:climate-change}

In 2007, the Intergovernmental Panel on Climate Change (IPCC) released its fourth assessment report \cite{IPCC-synthesis-report-2007}. The conclusions of this long-running analysis of studies on climate change and its effects are widely accepted as the consensus of the world's scientific community. They found that there is broad agreement that the climate is warming: air and ocean temperatures are higher, snow and ice are melting, and sea levels are rising. Further, natural systems are being affected: plant and animal ranges are moving towards the poles, and there are changes in fish and algae due to rising ocean temperatures.

The IPCC found that the warming of the climate was very likely due to anthropogenic greenhouse gas (GHG) emissions. GHG emissions from humans have increased by 70\% between 1970 and 2004. While there are a variety of GHG that impact climate change, \COtwo is the most important of the human-caused GHGs. Sea level rise in the second half of the 20th century was also very likely caused by humans, and rising sea levels have a potentially enormous impact on island communities like \Hawaii.

With current climate change policies, GHG emissions are projected to continue to increase this century. Further, there is no single technology that will mitigate the problem of climate change; a range of policies and innovations is required. The report lists both energy efficiency and individual behavior modification as suggested GHG mitigation strategies.

When discussing gasses in the atmosphere that are linked to climate change, there are a several possible terms. Greenhouse gas (GHG) is the most general term, referring to any gas in the atmosphere that leads to a greenhouse effect, trapping thermal radiation from the sun in the Earth's atmosphere. There are several GHGs in Earth's atmosphere: water vapor (H$_2$O), carbon dioxide (\COtwo), methane (CH$_4$), nitrous oxide (N$_2$O), ozone (O$_3$), and others. Each gas has different greenhouse effects on a molecule-by-molecule basis; for example, methane has a much greater greenhouse effect than \COtwo.  However, \COtwo is found in much greater quantities in the atmosphere than methane. Water vapor is the largest component of the greenhouse effect, but its contribution is not growing rapidly as \COtwo is, and humans don't have as much control over water vapor as they do over \COtwo emissions.

For these reasons, most climate change mitigation focuses on \COtwo emissions. In this context, sometimes \COtwo is referred to as simply carbon, as in \emph{carbon footprint}. For my purposes here, the terms GHG, \COtwo, and carbon can largely be considered interchangeable except when distinctions between \COtwo and other GHGs are being discussed.

\subsection{Peak Oil}

Fossil fuels are a finite resource that the world is consuming at an ever increasing rate. There is substantial evidence that the world has reached or will soon reach the peak of its oil production. After the peak, production will steadily decrease and production costs will increase as the remaining oil is more difficult to extract. When decreasing production is paired with skyrocketing demand, the result will be shortages and huge spikes in the price of oil. As discussed earlier, the industrialized world is highly dependent on oil, so large increases in the price of oil are expected to lead to widespread economic and social destabilization. This whole phenomenon is referred to as `peak oil'. [citation needed]

The effects of peak oil would be especially profound in \Hawaii, as oil the fuel used for over 75\% of the state's electricity production [cite State factbook] and virtually all transportation. It is estimated that in 2008 \Hawaii spent \$7 billion dollars on oil imports, nearly \$800K per hour! [citation needed]

\subsection{Energy Insecurity}

Fossil fuels are distributed non-uniformly across the planet, so many industrialized nations must import fossil fuels for their energy needs. Since fossil fuels are a crucial part of industrialized economies, this leaves these countries vulnerable to price spikes, shortages, or delivery problems. For example, the United States imports petroleum from Venezuela, Saudi Arabia, and Russia, all countries that it has had turbulent relationships with in the past [cite EIA docs]. Should our relationship with one or more countries sour, it could have serious negative implications for the US economy.

In \Hawaii, the situation is even worse. \Hawaii has no local fossil fuels, so it must import 100\% of its fossil fuel needs needs by sea. This leads to a very dangerous situation, since any significant interruption in shipping traffic could lead to major energy shortages. Jeff Mikulina, the executive director of the Blue Planet Foundation put it another way, describing \Hawaii as being one supertanker away from being Amish \cite{nyt2008HawaiiMoonShot}.

\section{Physical Concepts: Power and Energy}

When discussing energy, and in particular electricity, it is important to understand what power and energy are, and how they interrelate.

\subsection{Energy}

Energy is defined as the amount of work that can be done by a force. Most of us have an intuitive notion of energy: is makes things move, it heats things up, etc. There are many units used to measure energy: joules (a very small amount of energy), BTUs, calories. When talking about electricity, the most common unit is the watt hour, abbreviated as "Wh", which is equal to 3600 joules. A watt hour is the amount of energy required to to provide 1 watt of power for one hour. Note that from a certain perspective it is somewhat peculiar to measure energy in units that include power (watt), since power is defined in terms of energy in the first place. This underlines how central the concept of power is in most of our dealings with electricity.

\subsection{Power}

Power is defined as the rate of change for energy. As with any rate, it is expressed as a quantity of energy over a unit of time. The most common unit for power (and the one used in WattDepot) is the watt, abbreviated as "W". One watt defined as one joule (a measure of energy) per second. You might be familiar with a 60 watt incandescent light bulb, which expresses how much power it uses when turned on.

\subsection{Analogy To Cars}

Power and energy are closely related, but frequently confused concepts. As an analogy, think about a car. We can talk about the speed of a car (in miles per hour, or kilometers per hour) and we can also talk about a distance driven in a car (miles or kilometers). The speedometer in the car measures the speed (distance over time), while the odometer measures the distance traveled. Speed is a rate, like power, while distance is like energy.

When we talk about speeds, we usually talk about instantaneous measurements of speed. A speed limit is the maximum instantaneous speed at which you are allowed to drive, i.e. the car's speedometer should never register a speed greater than the limit. However, when we talk about distance driven, it only makes sense to talk about a distance driven between two locations, or the distance driven over a particular time interval. There is no such thing as an instantaneous distance driven, because in at a precise instant in time, the car is not moving.

\subsection{Power vs. Energy}

Since power is the rate of change of energy, if you know how power changes over time, you can determine how much energy was consumed or produced (the area under the power curve). Similarly, if you know how much energy was used over an interval of time, you can compute the average power over that period of time (but not the instantaneous power).

In our interactions with appliances, we usually talk about their power consumption and not their energy consumption. For example, we have 60 watt light bulbs, but we wouldn't generally talk about a 60 watt hour lightbulb (unless it consumed 60 watts and then burned out!). This is because power consumption is an intrinsic characteristic of things that use electricity, while the amount of energy is used by an electrical device is determined by how long you keep it plugged in or turned on. On the other hand, energy is very important to the utility that provides your electricity, since you are billed by how much energy you have used (typically in kilowatt hours).

The two key points to remember are: power is a rate, and we always talk about energy over an interval of time.

\section{Reducing Fossil Fuel Use}

There are two principle ways that fossil fuel use can be reduced: replacing fossil fuels with other preferable energy sources, and by reducing the total amount of energy consumed. For example, the state of \Hawaii has created the \Hawaii Clean Energy Initiative, which seeks to reduce \Hawaii's fossil fuel use by 70\% by 2030 through increasing the use local energy sources (for electricity and transportation fuel) to 40\% of demand and reducing demand by 30\% through efficiency and conservation \cite{HCEI-website}.

\subsection{Renewable Energy}

As described in \autoref{sec:motivation}, there are three problems with fossil fuels: their use is negatively impacting the global climate, they are finite, and the largest consumers of fossil fuels are not necessarily the largest owners of fossil fuels. Thus any replacement energy source should be better than fossil fuels in some or all of these ways. The ideal energy source would generate minimal GHG, would be virtually inexhaustible, and widely abundant. Energy sources that meet these first two requirements are often called \emph{renewable} energy sources (or sometimes \emph{sustainable} energy sources).

Renewable energy sources can generally be broken down into two categories. Some sources are \emph{dispatchable}, meaning that as demand for energy increases or decreases, the energy produced by the source can be changed to match the demand. For example, the gates on a hydroelectric dam can be opened and closed in a matter of seconds, tuning the output power as needed. As we will see in [add ref here], dispatchability is critical in running a power grid. The primary sources of dispatchable renewable energy are:

\begin{enumerate}

\item Hydroelectric energy: the motion of water in a river as it flows downhill is captured and converted mechanically into electricity.

\item Geothermal energy: heat from a deep well (sometimes near volcanic activity) is used to heat a working fluid, which then is mechanically converted into electricity. Note that wells may eventually cool to the point that they cannot generate electricity, making geothermal energy less than fully renewable.

\item Biofuel energy: a photosynthesizing crop is grown and then processed into either ethanol (for carbohydrate-rich crops) or biodiesel (for lipid-rich crops). The resulting biofuel is then burnt as a fossil fuel would be. This qualifies as renewable since the carbon released from burning the biofuel is recaptured in the next crop through photosynthesis. However, it is critical that the inputs to the growth of the the crop be carefully accounted for, or there is the risk that the energy to grow the crops could match or exceed the energy released when burned (as is true for corn-based ethanol).

\item Tidal energy: the rise and fall of ocean tides is captured and converted mechanically into electricity.

\end{enumerate}

It is worth pointing out that the dispatchable renewable sources largely fail the third renewable criterion of wide abundance. Hydroelectric energy requires water flowing from higher ground to lower ground, something not possible in flat and low-lying areas. Geothermal energy requires `hot rock' at depths that can be drilled to economically, while tidal energy requires both coastline and large differences between high tide and low tide. Biofuel is one bright spot, since the crops can be grown in one area, and the resulting biofuel transported to a power plant elsewhere. However, the costs of transporting the biofuel would have to be taken into account in that case.

Other renewable sources are \emph{intermittent} and thereby \emph{non-dispatchable}. Some days are sunny and clear, others are cloudy, leading to big differences in power output for sources powered by the sun. Some days the wind blows strongly and steadily, others days it is calm. Thus power from intermittent sources presents a problem when integrated into a power grid: where do you get power from when the sky is cloudy and the wind is calm? The primary sources of intermittent renewable energy are:

\begin{enumerate}

\item Solar energy: the light from the sun is harnessed directly into usable energy. The most familiar form of solar energy is the photovoltaic cell that turns sunlight into electrical power. Solar thermal energy uses sunlight to heat a working medium (such as water, oil, or sodium), which then turns a turbine or other mechanism to generate electrical power. Solar hot water heaters use sunlight to heat water that would otherwise be heated using electricity or gas.

\item Wind energy: the motion of air is converted mechanically into electricity.

\item Wave energy: the wave motion of ocean water is converted mechanically into electricity.

\end{enumerate}

Solar and wind energy meet the three renewable energy criteria, and since they are widely available, they are expected to be the primary sources of renewable energy growth. Photovoltaic panels are the most viable renewable energy source at small scales such as a single house, especially in sunny locales like \Hawaii.

\subsection{Energy Conservation}

The other way that fossil fuel use can be decreased is by decreasing the amount of energy that is consumed. Reduced energy consumption also eases the transition from fossil fuels to renewable energy sources. Amory Lovins coined the term \emph{negawatt} to refer to power that has been conserved and therefore does not need to be generated \cite{Kolbert2007Mr-Green}.

Negawatts can be `generated' in two basic ways: by increasing the efficiency of devices that consume energy, and by changing people's behavior such that they use less energy.

\subsubsection{Energy Efficiency}

Many energy consuming devices have become more efficient over time. For example, incandescent light bulbs are increasingly being replaced with compact fluorescent bulbs that produce an equivalent amount of light but use only 20--30\% of the energy. The negawatts generated through use of more energy efficient devices has the primary advantage of not changing the functionality of the device: a more efficient refrigerator keeps food cold just as ably as a less efficient one.

% As \autoref{sec:efficiency-rebound} shows, energy efficiency can also lead to rebound effects, where the negawatts saved through efficiency are lost to additional consumption.

\subsubsection{Behavior Change}

Changing people's behavior with respect to energy holds significant promise in reducing energy use. Darby's survey of energy consumption research found that identical homes could differ in energy use by a factor of two or more \cite{darby-review-2006}. There are three areas of focus to create and sustain behavior changes with positive environmental impact:

\begin{enumerate}

\item Feedback systems to make people aware of how their behavior affects the environment.

\item Environmental `literacy' to help people understand why behavior change is necessary, and what behavioral changes would have a positive environmental impact.

\item A toolbox of techniques to encourage, motivate, and remind people to make the behavior changes necessary for positive environmental impact.

\end{enumerate}

As Lord Kelvin is famously reputed to have said, ``If you can not measure it, you can not improve it.'' In the case of electricity usage, for many people the only feedback they receive is a monthly bill detailing the number of kilowatt-hours used over the course of the last month. Ed Lu of Google analogizes this as if there were no prices on anything at the grocery store, and shoppers were just billed at the end of the month \cite{Helft2008Googles-Energy}. Office workers or dormitory residents might never see any feedback on how much electricity they are using! Darby's review of many studies using feedback systems to facilitate energy conservation found that savings of 5--15\% could be achieved \cite{darby-review-2006}. \ref{sec:energy-feedback} explores the topic of energy feedback in depth.


%% Intended to be included into a larger document
\chapter{Related Work}

This chapter lays out previously conducted work that relates to my research area. As this topic is an area of rapid innovation, some of the references exist only as web pages, not as works in peer-reviewed publications. 

\section{Climate Change}
\label{climate-change}

In 2007, the Intergovernmental Panel on Climate Change released its fourth assessment report \cite{IPCC-synthesis-report-2007}. The conclusions of this long-running analysis of studies on climate change and its effects are widely accepted as the consensus of the world's scientific community. They found that there is broad agreement that the climate is warming: air and ocean temperatures are higher, snow and ice are melting, and sea levels are rising. Further, natural systems are being affected: plant and animal ranges are moving upward, and there are changes in fish and algae due to rising ocean temperatures.

The IPCC found that the warming of the climate was very likely due to anthropogenic greenhouse gas (GHG) emissions. GHG emissions from humans have increased by 70\% between 1970 and 2004. While there are a variety of GHG that impact climate change, \COtwo is the most important of the human-caused GHGs. Sea level rise in the second half of the 20th century was also very likely caused by humans, and rising sea levels have a potentially enormous impact on island communities like Hawai`i.

With current climate change policies, GHG emissions are projected to continue to increase this century. Further, there is no single technology that will mitigate the problem of climate change; a range of policies and innovations is required. The report lists both energy efficiency and individual behavior modification as suggested GHG mitigation strategies.

\subsection{Focus on Carbon}

When discussing gasses in the atmosphere that are linked to climate change, there are a several terms. Greenhouse gas (GHG) is the most general term, referring to any gas in the atmosphere that leads to a greenhouse effect, trapping thermal radiation from the sun in the Earth's atmosphere. There are several GHGs in Earth's atmosphere: water vapor (H$_2$O), carbon dioxide (\COtwo), methane (CH$_4$), nitrous oxide (N$_2$O), ozone (O$_3$), and others. Each gas has different greenhouse effects on a molecule-by-molecule basis; for example, methane has a much greater greenhouse effect than carbon dioxide.  However, carbon dioxide (henceforth referred to as \COtwo) is found in much greater quantities in the atmosphere than methane. Water vapor is the largest component of the greenhouse effect, but it's contribution is not growing rapidly as \COtwo is and humans don't have as much control over water vapor as they do over \COtwo emissions.

For these reasons, most climate change mitigation focuses on \COtwo emissions. In this context, sometimes \COtwo is referred to as simply carbon, as in \emph{carbon footprint}. For my purposes here, the terms GHG, \COtwo, and carbon can largely be considered interchangeable except when distinctions between \COtwo and other GHGs are being discussed.

\subsection{Does Energy Efficiency Reduce Carbon Emissions?}

Many governmental plans to reduce GHG emissions involve improving energy efficiency in the home, in industry, and in transportation. While intuitively it would seem that increased energy efficiency would lead to decreased energy usage and thereby reduced GHG emissions, surprisingly there is some evidence both theoretical and empirical that energy efficiency actually increases energy usage! Saunders dubbed this unintuitive notion the Khazzoom-Brookes Postulate based on conclusions reached independently by those two researchers \cite{saunders-1992}.
%\emph{Insert references to Khazzoom and Lovins papers here, after I read them.}
Using neoclassical growth theory, Saunders finds that increased energy efficiency makes energy seem cheaper, thus allowing it to be substituted for labor in production. Increased energy efficiency also increases overall economic growth, which leads to increased overall energy usage.

In discussing this effect, rebound is defined as the difference between the expected amount of energy savings from an improvement in energy efficiency, and the actual observed effect. For example, if an improvement in metal smelting technology reduces the energy required to smelt by 20\%, but the energy consumed by the metal smelting industry only goes down by 10\% then the rebound is 50\%. If the rebound is greater than 100\%, then backfire is taking place (the efficiency measure has backfired) \cite{Hanley2008Do-increases-in}. There is some debate over whether the predicted increases will actually take place in the real world. Laitner suggests via a simple analysis that the rebound effect is small (2.4\%) \cite{Skip-Laitner:2000yg}. His equation relates future carbon emissions to current carbon emissions, increases in GDP and energy costs, and elasticities of income and energy prices to arrive at this conclusion. He goes on to a further analysis done by the Environmental Protection Agency and Lawrence Berkeley National Labs using the National Energy Modeling System showing that an ``energy-efficient/low-carbon technology path'' would suffer from a rebound effect of only 2.2\%. However, he acknowledges that consumer choices about energy usage could erode gains from efficiency, such as turning up the furnace thermostat because the cost of doing so has been effectively reduced.

The issue of consumer choices is a real one. Over the last 25 years, automobiles have been made more efficient through ``increasing the efficiency of the engine and transmission, decreasing weight, improving tires and reducing drag'' \cite{Heywood2008Fueling-Our-Future}. However, these improvements have been traded for vehicles that are larger, heavier, and faster, which has led to only modest improvements in overall fuel efficiency. This is an example of how energy efficiency may not always lead to reduced GHG emissions without motivating automobile users (and manufacturers) to buy and make fuel efficient vehicles.

Other authors find that rebound and even backfire are the likely results of economy-wide improvements in energy efficiency. The analysis of Hanley et al finds that backfire occurs when economy-wide improvements in energy efficiency are made \cite{Hanley2008Do-increases-in}. Their theoretical analysis finds that if energy demand is relatively price-elastic (demand increases when prices are low and reduces when prices are high), then backfire will occur. Empirical evidence of rebound and backfire are hard to come by because there are indirect system-wide effects due to the increased efficiency, and these indirect effects are difficult to measure. The authors created a Computable General Equilibrium (CGE) model of Scotland that simulates the economy and environmental impacts based on the inputs and outputs of the system. Using this model, almost all scenarios eventually end up in backfire. They note that since non-renewable energy sources use more energy in their production than renewable sources, increased energy efficiency actually reduces the percentage contribution of energy from renewable sources. They also urge caution when reviewing sustainability measures such as the ratio of Gross Domestic Product (GDP) to energy usage or carbon emissions, because even if the ratio increases (less carbon per unit GDP), if the GDP as a whole increases faster then the absolute carbon emitted will increase. They suggest that backfire could be prevented by combining energy efficiency improvements with taxes on energy use or a carbon tax. Since energy efficiency effectively reduces the cost of energy, the savings could offset the cost of additional taxes, thereby blunting any impact on economic activity.

It would appear that any energy efficiency improvements will have some degree of rebound effect, thus a naive pursuit of energy efficiency without taking into account the context around the improvements could risk reducing their effectiveness or even making them counterproductive! While many of the analyses deal at the macroeconomic level, it is not hard to think of individual scenarios where efficiency could actually increase personal usage, such buying two energy efficient refrigerators to replace one older energy-hogging refrigerator. Fortunately, the research plan I am pursuing is quite different: users learn about their GHG emissions (including energy usage) and based on that they decide on what actions to take, which could include increasing energy efficiency. The key to ensuring that energy efficiency improvements on the micro level lead to less GHG emissions is to combine efficiencies with changes in behavior. With the public's increased awareness of climate change, this should be a viable proposition.

\section{Energy Feedback}

One of the many ways that global GHG emissions can be reduced is through encouraging individuals to use less energy. To reduce energy use one must know how much energy one is using. Feedback systems display the consumption of a resource (such as electricity) to the user, usually in real time. Darby provides a detailed survey of studies on electricity feedback systems from the past 3 decades \cite{darby-review-2006}. The survey of 20 studies finds that, on average, the introduction of a direct (real-time) feedback system leads to reductions of energy usage ranging from 5-15\%. Feedback systems providing historical data (such as those provided with billing statements) are not as effective (0-10\% savings), but can be useful for assessing the impact of efficiency measures such as improved insulation or a more energy efficient appliance, since those savings accumulate over time.

Darby found that ``consumption in identical homes, even those designed to be low-energy dwellings, can easily differ by a factor of two or more depending on the behaviour of the inhabitants.'' This demonstrates the significant potential to curb energy usage (and thereby GHG emissions) through changes in individual's behavior.

During California's energy crisis in 2000 and 2001, Lawrence Berkeley National Laboratory created a web site that graphed data from utilities \cite{Bartholomew2008Current-Energy}. The graphs showed consumer demand for electricity (actual and forecast), and the utilities' generation capacity. Darby reports anecdotal evidence that people viewing the graphs changed their electricity usage based on the data.

There is also evidence that just the knowledge that one is being monitored can cause one to consume fewer resources. A group of researchers simulating a mission to Mars or the Moon in the Canadian Arctic for four months tracked the crew member's water usage \cite{Bamsey2008FMARS}. Water usage was monitored via automated meters during the entire mission, but during certain multi-day study periods, crew members were also required to manually log their water usage at the point of use. The authors found that water usage was 10\% less during these study periods. The reduced water usage could be due to the knowledge that the usage was being examined more closely, or perhaps the extra effort required to manually record their water usage led to crew members reducing non-essential water use (see \autoref{ecoisland} for another possible benefit to manual data collection).

While feedback can increase energy conservation, there are still cultural norms that strongly influence what behaviors are non-negotiable. Strengers performed an ethnographic study of 10 households participating in a smart metering trial to examine how their comfort and cleanliness norms affected their energy savings \cite{strengers-comfort-norms-2008}. Participants were provided with metering devices that displayed electricity and water usage, and greenhouse gas emissions in real time. The author was attempting to use feedback to change the participants societal norms for comfort and cleanliness. For example, until relatively recently, bathing weekly was the norm, but now bathing daily is considered normal behavior. Like many people, the participants did not understand connection between the consumption data and their practices. Participants tended to increase conservation by changing technology (such as using CFLs instead of incandescent light bulbs), or by minor behavioral changes like ``taking shorter showers, doing full loads of laundry''.

Strengers states that people act the way they do (in matters of cleanliness and comfort) because ``they believe society expects them to'' and because many companies and organizations have a vested interest in keeping it that way. Therefore just providing people information about their consumption is not enough, because individuals are constrained by infrastructures and social norms. She suggests increasing social interaction regarding the feedback system by making placement more prominent and encouraging discussion with household visitors, since people try to conform to the expectations of their peers.
However, it would seem that changing societal norms is one of the hardest possible means for reducing consumption. It also feeds into many of the negative stereotypes of environmentalism: stinky people living in dark, cold homes. Despite the irrationality of some of these norms, effort may be better spent focusing on areas where the effort will meet less resistance.

R\"{u}st has implemented an extreme energy feedback system called the Thighmaster \cite{Rust2008Thighmaster-web}. Inspired by the cilice (a small metal garter with inward facing spikes) worn by some members of the Catholic Opus Dei organization as part of a practice of mortification, the Thighmaster is a ``techno-garter'' that pokes the wearer with spikes when their actions are not environmentally responsible (as defined by R\"{u}st). Specifically, the Thighmaster communicates wirelessly with electricity usage sensors and a human speech sensor that monitors whether the user speaks with their plants. While more of a demonstration, the Thighmaster shows the complex emotions involved in people's reactions to climate change. It goes without saying that being pierced by spikes is unlikely to be a viable energy feedback mechanism for most users.


\section{Motivation and Persuasion}

After making users aware of their carbon footprint, PET needs to persuade them to take actions to reduce their footprint, and motivate them to continue those actions indefinitely. Young investigated the motives behind individual's environmentally responsible behaviors (ERBs) through a series of surveys \cite{Young:2000fv}. Traditionally, the motives invoked by researchers attempting to promote ERB was constrained to material incentives or disincentives and altruistic reasons. The problem with incentives is that they ``needed constant reintroduction to remain effective and they proved to be less reliable than we had hoped''. Incentives can initiate ERB, but people's behavior changes back when the incentives end, and even continuing incentives can have low reliability.

Young also describes some of the pitfalls that can be encountered in motivating ERB, such as psychological reactance, where people do the opposite of the ERB they are being asked to undertake. Those initiating behavior changes can also be negatively impacted, creating feelings of contempt for those whose behavior is changing, and self-contempt.

Self-interest generally considered cause of environmental problems, ``focusing 
solely on short-term individual or familial gain to the exclusion of long-term societal or environmental benefits'', but Young suggests that self-interest can be a solution to environmental problems. He distinguishes self-interest from selfishness, self-interest meaning each individual is responsible for getting their own needs met. Young believes that intrinsic satisfaction is a better way to motivate ERB, as people find that ``certain patterns of behavior are worth engaging in because of the personal, internal contentment that engaging in these behaviors provides.''

Based on 9 different studies of ERB across different populations and environmental focuses, the author found 3 intrinsic satisfactions:
\begin{enumerate}
	\item ``satisfaction derived from striving for behavioral competence''
	\item ``frugal, thoughtful consumption''
	\item ``participation in maintaining a community''
\end{enumerate}
Competence involves the enjoyment in completing tasks and solving problems, frugality is enjoyment from the ``careful stewardship of finite resources'', and 
participation is the enjoyment from participating in community activities such as sharing news and collaborating with others toward a shared goal.

While attitudes and norms can lead to behavior change, people also need skills and experience with the change. As Young puts it, ``without considering these variables, we make the error of assuming that once people know what they should do and why they should do it, they will automatically know how to proceed.'' In the particular case of competence as a motivator, it is important to provide people with the opportunity to utilize their competence or they will grow frustrated. He suggests that motivating through competence be done by providing an environment where information on procedures is available and new behaviors can be tried out in a supportive environment.

Darby's survey of electricity feedback programs found similar results on motivations \cite{darby-review-2006}. She found that incentives to save energy lead to savings that disappear when the incentives are removed. When trying to get people to change their behavior, she found that behavior formed over a 3 month period is more likely to persist than those behavior changes made over shorter periods. She also found that internal motivation is most important for continuing energy savings.

In a position paper, Khan and Canny suggest that the technique of social marketing would be helpful in persuading users to make environmentally beneficial changes \cite{Khan2008-social-marketing}. Social marketing is the idea of applying the principles of consumer product marketing to encourage social change. The principles they describe are: emphasis on the benefits of new behavior while minimizing the cost, consumers are strongly influenced by knowing what behaviors others are undertaking, and target audiences should be segmented with different messages to different segments. For example, in discussing the iamgreen application (see \autoref{iamgreen}) where users commit to positive environmental actions suggested by others, the authors suggest using collaborative filtering (the technique used by online merchants to suggest other products similar to the one being viewed) to suggest leaves rather than just popularity.

\section{Design of Environmentally Persuasive Systems}

Considerable research has been done on the subject of designing environmentally persuasive systems. Woodruff et al performed a qualitative study of individuals who are making a significant effort to be green, in an effort to inform future designs by documenting existing green practices and beliefs \cite{Woodruff2008-bright-green}. The participants were all involved in making their home more sustainable and energy efficient, and that was the focus of the study. The authors found that these environmentally inspired people have diverse affiliations, and traditional environmental activism isn't always central to their interests. Thirty-five homes participated in the study, with 56 people in total. The participants were mostly ``bright green environmentalists'', that is environmentalists that believe that technology can make the world more sustainable, rather than believing that technology is the root of unsustainable behavior and should be abandoned. The authors divided the participants into three groups based on their motivations: ``counterculture bio-centric activism; American frontier self-reliance and rugged independence; and trend-focused utopian optimism.'' The first group focused on stewardship of the earth, the second group was focused on frugality, do-it-yourself activities, and patriotism from getting off foreign oil. The third group was focused on trend-setting, and being ``eco-chic''.

The authors found that the participants were reflective about the positive environmental choices they made, often trying to improve their sustainability through playful analysis of the options, such as buying a product online versus buying it from a store. They found that participants eagerly assessed the performance of their homes, so that they could tune them for better energy savings. This included extensive data collection, both manually and through automation. However, after living in a house for 1.5 years, their interest in data collection had gone down, in part because their routines had been internalized. In making their homes more efficient, the participants would work on improving one area at a time, then move on to the next area. Participants also wanted to live by example and inspire others, such as by driving a hybrid car.

Based on the interviews, the authors found several implications for design. The participants tended to learn about sustainability in a depth-based manner (focusing on one area at a time) rather than in a breath-based manner. Many popular attempts to encourage environmentally responsible behavior involve short lists of relatively easy actions, which is contrary to how the participants sought information. The authors suggest that advice systems focus on the user's primary motivations in-depth rather than providing a list of easy actions. The participants found mentorship an important part of the learning process, so the authors suggest that systems match mentees with mentors that have already mastered the area of expertise being sought. The authors suggest that users be provided with ways to express their identity and share their green activities to others via social networks. Finally, based on the observation that many participants enjoyed the process of determining the most sustainable option among many choices, Woodruff et al suggest providing users with modest mental puzzles that help users explore the outcomes of different actions rather than telling them the answer outright.

Darby's review of energy feedback studies yielded some suggestions for design \cite{darby-review-2006}. She observed that historical feedback of the user's energy consumption is more effective than feedback that compared usage to others, or feedback that compared usage to normative values. However, users did report finding pie charts of typical breakdowns of home energy use helpful, even though they were averages as opposed to being drawn from the user's data. Users reported that they liked to see comparative information, but that it didn't necessarily lead to energy savings. In addition, if a user is shown comparative data that indicates that their usage is lower than their peers, it could lead to the user feeling unconcerned about increasing their consumption to match their peers.

Chetty et al performed a qualitative study of the resource management processes of 15 households in an effort to help ubiquitous computing researchers design better resource feedback systems \cite{chetty-2008}. They found that participants were unaware of real-time resource consumption for both the entire home and individual appliances. The study examined the participants usage of natural gas, electricity, and water. Thermostats seemed to be a problem for participants, leading to arguments about how they should be set and half of the homes with programmable thermostats hadn't programmed them. Some participants were in living situations where they paid a flat rate for their utilities, which led to a lack of motivation to conserve resources. Participants wanted real time information on their resource usage, on utility pricing (if there is peak load pricing), and also alerts if there is anomalous usage (such as a broken toilet using an excessive amount of water). The authors report that participants were also aware of potential privacy issues, such as being able to infer other's habits from their resource usage, and feeling shame if one is wasteful with resources.

Based on their study, Chetty et al provide some suggestions for future system designs. In the modern world, infrastructure is invisible: you don't have to know how much energy an appliance uses when you plug it in. Therefore the authors suggest visualizations ``that equate our resource usage with units of production, for example, buckets of water, bags of coal, stacks of wood, as well as a monetary amount.'' They point out that households are often made up of multiple people with different levels of interest in being green and different responsibilities (some may not have to pay the bills!), so system design will have to reflect these differences. The authors also worry about the ``green divide'' in that lower income households might not be able to afford expensive equipment. They suggest the need to make sure devices supporting resource conservation are affordable to all.


\section{Related Systems}

In this section we examine other systems that have been designed to help users become more aware of their environmental impact. For more information on PET, the system I plan to build, see \autoref{PET-description}. Some of the sensors described in \autoref{sensor-section} could also be classified as related systems, since they provide feedback to the user. However, they are described separately as their emphasis is on the sensing. Carbon calculators can also be thought of as systems that make people aware of their environmental impact, but since they are numerous and narrow in focus they are described separately in \autoref{carbon-calculators}.

The system closest to PET is the one described in the position paper by Sutaria and Deshmukh \cite{sutaria-2008}. It describes using networks of ad hoc sensors to monitor both electricity usage and miles driven by automobile, while providing real-time feedback to the user. The system described would compare the household's energy usage with others in similar situations. They also mention the possibility of integrating personal carbon trading (a sort of carbon cap-and-trade system for individuals) into the system, such as the one being investigated by the RSA CarbonLimited project \cite{carbonlimited-2007}. While the system described by Sutaria and Deshmukh is similar to PET, the system appears to be hypothetical at this point.

\subsection{StepGreen}

StepGreen is a web application designed to encourage people to undertake environmentally responsible actions \cite{step-green-website}. Mankoff et al have written about the rationale for the system and description of the design, presumably written before the site was active \cite{Mankoff2007Leveraging-Soci}. The basic idea is that half of American's consumption of energy is under their control, but people don't connect their activities with \COtwo emissions. StepGreen (also known as Footsteps, possibly an earlier name for the system) is designed to leverage online social networks to motivate personal change, by providing suggestions for improvement.

The StepGreen system is currently open to the public. Users create an account on StepGreen, and then are presented with a list of actions with positive environmental consequences (mostly reduced GHG emissions). Example actions are ``Turn off the lights when you exit the house in the morning for the day'', ``Take the stairs at work'', and ``Set your home computer to automatically hibernate/sleep after a short period of inactivity''. Each action is associated with a cost savings and reduction in \COtwo emissions, and users can get more information about the action and how the savings were calculated. For each action, users can indicate whether they are already performing that action, whether they commit to undertaking that action, or whether the action is not applicable to them. The current system does not appear to have a way to suggest to actions to be added to the list, but based on the design paper \cite{Mankoff2007Leveraging-Soci}, this might be added at some time in the future.

Once users have selected actions that they are either already performing or commit to performing, they can track them on the Reporting page. For one time actions, such as replacing an incandescent light bulb with a compact florescent bulb, users simply check off when they are completed. For recurring actions, users must indicate how many times they have performed the action since their last report in order for the system to track the activities. Based on the user's self-reporting, StepGreen calculates the amount of money saved, pounds of \COtwo saved (i.e. reduced), and missed pounds of \COtwo saved, and provides a historical graph of values.

StepGreen also provides links to social networking sites. They provide a linked Facebook application, a MySpace profile widget, and connection to Twitter. Each of these links provides a way to inform the user's social network about what actions the user is undertaking. This can serve to recruit other people to use StepGreen, provide comparisons on financial and environmental savings between peers, and encourage users to keep to their StepGreen commitments. 

StepGreen provides a useful platform for research on convincing users to change their behavior to reduce their carbon footprint. For example, a virtual polar bear was implemented to motivate users to reduce their carbon footprint (see \autoref{virtual-polar-bear}). Notes on the StepGreen research website \cite{stepgreen-research-website} indicate that there are plans to support the input of sensor data from the UbiGreen transportation sensing project that they are a part of (see \autoref{ubigreen}).

In its current state, StepGreen would be challenging to keep up to date due to the reliance on manual data input. The actions available in the system are quite wide-ranging, but focus on relatively minor changes such as not using an electric toothbrush. Without the overall view that PET intends to provide, it would be easy to focus efforts on these minor actions instead of tending to the major sources of one's carbon footprint. Due to the limitations of manual reporting, StepGreen may report missed savings that are not accurate, annoying users. For example, recycling glass is an action that is listed as having substantial carbon savings. However, if one chooses to drink water from a mug instead of purchasing a beverage and later recycling the glass container, clearly the carbon savings are greater from using the mug, but StepGreen will count the lack of recycling as missed savings. PET is designed to work from sensor data primarily, which should reduce the issue of inaccurate missed savings. PET's data focus could also allow it to improve on StepGreen's action list by suggesting actions that are directly address the user's largest sources of carbon emissions.

\subsection{Virtual Polar Bear}
\label{virtual-polar-bear}

Dillahunt et al (who are involved with the StepGreen project) have built a system providing a virtual polar bear that is affected by the user's environmental choices as a means to motivate users to reduce their carbon footprint \cite{dillahunt-virtual-polar-bear-2008}. They note that there are strong emotional bonds between humans and animals, which may help to encourage environmentally-responsible behavior. The authors performed a one week study, with subjects divided into two groups: an attachment group and a control group. The attachment group read a story about climate change impacting polar bear habitats, and were asked to name their virtual polar bear. As participants make or decline commitments to environmentally responsible actions, the ice under polar bear either grows or shrinks. The study had 20 subjects (10 for each group), all of whom were surveyed before and after to test for levels of empathy and environmental concern. The subjects in the attachment group had more fulfilled environmental commitments, which was a statistically significant difference. The attachment subjects also had a greater level of environmental concern after interacting with the polar bear. The study was short, so the authors are unsure whether effects would be sustained in a longer study. They are now working on bringing the system to a mobile platform and a polar bear application for Facebook and MySpace.

\subsection{iamgreen}
\label{iamgreen}

iamgreen is an application for the Facebook social networking platform that provides an online gathering place for environmentally conscious users \cite{iamgreen-website}. iamgreen provides all of the standard components of Facebook: a newsfeed of events from members, status updates, news articles, etc. The application provides a list of environmentally responsible statements called leaves, such as ``Most of my lightbulbs are compact fluorescents'', ``I recycle, even when it is not convenient'', and ``When I drive, it's over 40mpg baby''. Users can indicate for each statement that they engage in that behavior, they aspire to that behavior, they wish to hide the statement (removing it from the list of choices), or they want to recommend it to a friend. Users can then display the number of leaves they have committed to in their Facebook profiles. Users can also contribute new leaves that will be displayed as options to other iamgreen users.

While the leaves concept is a simple way to encourage users to make more environmentally positive choices, it suffers from some obvious deficiencies. First, leaves have mostly the same value (though apparently some actions, such as not owning a car, are worth more than one leaf). The leaf system also lacks any quantitative feedback other than the number of leaves, so the user is not provided with real insight into their environmental footprint. Like any system based on manual reporting, users have to spend time reporting any changes to their action list. Without quantitative feedback, it seems likely that many users will make some selection of leaves and then revisit them infrequently or never again.

\subsection{Personal Environmental Impact Report}

Personal Environmental Impact Report (PEIR) is a mobile phone sensing-based environmental data collector from the urban sensing group at the University of California, Los Angeles Center for Embedded Networked Sensing (CENS) \cite{peir-website, agapie-2008-seeing-our-signals}. The PEIR application runs on GPS-enabled mobile phones to record users' locations throughout the day. The data is uploaded to a central server that provides each user with a profile of their environmental data. Unlike most applications, PEIR not only tracks the user's impact on the environment through GHG emissions, it also tracks the environment's impact on the user. PEIR currently tracks the user's carbon emissions, their contribution to pollution around schools and hospitals, their exposure to fast food establishments, and their exposure to fine particulate matter in the air. The sensed data is overlaid on a map that allows users to see the location of problematic areas (primarily useful for seeing the user's exposure to particulate matter). PEIR also provides a Facebook application that shows the user's current environmental impact, and ranks the user against their friends impacts.

The only data the PEIR system directly measures is the user's location at a particular time. From the location data, activity inference is done by matching locations to freeway data from maps, and the speed from GPS location data. These are combined using a Hidden Markov Model to infer if the user is still, walking, or driving. The location and activity data is combined with GIS, weather, and environmental information such as smog levels to produce the information on the four impacts listed previously.

As of this writing (December 17, 2008) PIER is in a closed beta test, so it is not available to the public yet. It currently runs on the Nokia N80 with external GPS device or the Nokia N95 with built-in GPS, with plans to accept GPS traces from arbitrary sources (such as GPS data loggers) in the future.

PEIR is an impressive system, incorporating location sensor data and extensive analysis to provide four very different impacts. The project is currently focused on transportation-related environmental impacts because people have  choices in meeting their transportation needs, but the developers indicate that they will add sensors from other domains in the future. Currently the system only provides measurement data, and doesn't make any attempts to change users behavior. While the fast food exposure measure is cute, it isn't quite parallel to the other impacts since it is possible to not patronize fast food establishments in one's vicinity, but ones exposure to fine particles cannot be willed away.

\subsection{mobGAS}

mobGAS is mobile phone application that computes GHG emissions from a detailed, manually-entered daily diary of activities \cite{mobGAS-website}. The data entry is quite detailed, including: transport, appliance use, lighting, heating/cooling, hygiene, and food. The results can be uploaded to a central server where they can be reviewed by the user, or used to generate a ranking of all users. The system was designed for use in the European Union: only EU countries can be selected when registering an account,  so it cannot accurately calculate GHG emissions for people living in the US.

On November 25, 2008, the user ranking page\footnote{\url{http://mobgas.jrc.ec.europa.eu/mobgas/app/emissions/userrankings.po}} shows only one user ``Galloz'' who has 0.65 kg of GHG emissions. So either the ranking page is not working properly, or there is one actual user of the system, who has entered in very little data (or perhaps lives as a hermit).

The team behind mobGAS has released a ``disclosed model'' report that explains both the functionality available in the program and the model used to compute the GHG emission information \cite{Sousa-Pedrosa2008mobGAS-model}.

While attempting to capture GHG emissions from a wide range of activities is admirable, it seems unlikely that people will enter such detailed data manually on a regular basis. Even if users were so inclined, the mobile phone environment is far inferior to a desktop or laptop computer for data entry.

\subsection{Personal Kyoto}
\label{personal-kyoto}

Personal Kyoto is a web service that tracks the electricity usage of users in the New York area, and compares it to a ``Personal Kyoto Goal'' for the user \cite{Personal-Kyoto-website}. The Personal Kyoto Goal represents the limit of electricity usage that would apply to the user if the Kyoto Protocol (which the USA is not a party to) were administered on an individual basis rather than on a national basis.

The user's electricity usage is retrieved from the local utility's web site (Con Edison) using the user's account number. In addition to the monthly usage (which can vary substantially due to circumstances and the seasons), a 12 month rolling average is computed to remove the seasonal effects. The Personal Kyoto Goal is defined as 75\% of the first point of the monthly moving average when the user signed up with the web site.

Personal Kyoto is a cleverly designed system in that it uses the user's real data, but avoids manual data entry by scraping the data from the utility web site. It also gives the user a specific goal for reducing electricity use that has a real justification and ties into the environmental ``gravitas'' of the Kyoto Protocol.

\subsection{EcoIsland}
\label{ecoisland}

Takayama and Lehdonvirta have constructed a system they call EcoIsland, which attempts to ``motivate behaviour changes that reduce CO2 emissions'' using a background game-like activity, with a centrally installed display in the home \cite{takayama-2008}. Each family member has an avatar on the virtual island, and they set a family \COtwo emissions target. The family's emissions are tracked via sensors and self-reporting. If the emissions exceed the chosen target level, the water level on the island rises, and if the water level continues to rise it will eventually end the game.

Participants mobile phones have a list of suggested actions to reduce emissions, and they can self-report their actions using the phone. Participants can see the islands of other participants and they receive a periodic allowance in a virtual currency. The participants can use the virtual currency to buy decorations for their island, or they purchase carbon credits from other users. Therefore participants with low emissions can decorate their island, but those with high emissions have to spend their money on carbon credits. EcoIsland provides a metaphor for the users' emissions and makes them aware of the consequences of their actions.

At the time of the paper's writing, the sensor portion of the system was not yet implemented. The authors performed a four week pilot study of EcoIsland with 20 people in six families. During the first week, the baseline electricity usage of each participant's air conditioning system was monitored using a plug load meter (for more information on this type of meter, see \autoref{plug-load-meters}). During the second week, one participant from each household was asked to use the system, while in the third week all members were asked to use it. In the fourth week, the carbon trading system was introduced to participants. At the conclusion of the study, the participants were surveyed and 17 of 20 participants said ``they were more conscious of environmental issues after the experiment than before.'' However, users indicated that they were motivated by game issues (such as saving the sinking island and buying decorations) rather than saving the environment. Few of the participants used the carbon trading system because their targets were easy enough to achieve without trading. Air conditioner usage in participant homes showed no correlation with game outcome, but the authors believe that the short study, conducted in winter [sic!] may have affected that outcome. One interesting result is that participants noted that manual reporting contributed to their motivation, so replacing the reporting with sensors could reduce user's motivation to change.


\section{Sensors}
\label{sensor-section}

Integral to the PET concept is input from a diverse set of sensors. This section describes different types of sensors that relate to PET's mission to inform users about their environmental impact. Some of them are potentially data sources for PET.

\subsection{Electricity Usage}

Electricity usage is one of the major sources of GHG emissions for individuals, so being able to track its usage is a high priority. Electricity metering systems can be broken down into two types: plug load meters that measure the electrical load directly plugged into them, and whole home energy meters that measure the electrical usage of an entire home. Both typically provide a real-time display of electricity usage, and some sort of historical total (usually in kilowatt hours, kWh).

\subsubsection{Plug Load Meters}
\label{plug-load-meters}

The Kill-A-Watt (sold by P3 International) is an example of an inexpensive plug load meter \cite{kill-a-watt}. It is designed to be plugged into a wall outlet, and the load is then plugged into the Kill-A-Watt. An LCD display shows the current voltage, current, power, frequency, power factor, and cumulative energy used since the unit was plugged in. The Kill-A-Watt provides an easy way to determine how much electricity a particular appliance (or set of appliances if connected via a power strip) uses. The manufacturer claims the Kill-A-Watt has 0.2\% accuracy. There are several drawbacks to the Kill-A-Watt. Because of its shape, it generally obscures both of the outlets commonly found on a wall outlet in the US, preventing the second outlet from being use while measurement is taking place. The load most be plugged in via the Kill-A-Watt, so that means that the user must disconnect the load from power at least momentarily, which can be inconvenient for some loads (computers, VCRs, etc). The Kill-A-Watt also has no facility for exporting the data it collects, and if power is lost for any reason, the data collected will be lost as well.

LeBlanc attempted to address the issue of data collection with his work on recording device-level power consumption \cite{leblanc-2007}. He developed a sensor that sits between the load and the wall outlet, like the Kill-A-Watt. The sensor records electricity usage, and transmits the data wirelessly using the ZigBee protocol to a base station. Details on how to construct the wireless power monitor can be found at the author's personal website \cite{LeBlanc2008power-mon-howto}. This system solves the problem of automated data collection, but still requires the load to be unplugged before monitoring. It also faces the problem of all plug-load meters, that it can only monitor what it is connected to, so it doesn't work well for providing a comprehensive picture of electricity usage in a home.

\subsubsection{Whole Home Meters}
\label{whole-home-meters}

The Energy Detective TED Model 1001 is a whole home electricity meter from Energy, Inc \cite{the-energy-detective}. TED consists of two portions: a base unit that is connected directly to the incoming power lines at the circuit breaker box, and a display unit that connects to any power outlet in the home. The base unit uses current transformers that clamp over the incoming power cables and measure the amount of current being transmitted over them. Since the transformers clamp over the existing cables, there is no need to alter the existing wiring. The instantaneous power consumption can be computed using the current data combined with the utility voltage. This data is transmitted to the display unit through the home's electrical wiring.

Once the display unit is plugged into any outlet in the home, it receives the instant power consumption data from the base unit once a second. The power consumption data can be displayed in real-time in kW or dollars (after the user enters pricing data). It can also track historical consumption, peak usage, and project usage for the rest of the month based on historical usage. With the addition of the Footprints software package from Energy Inc, the display unit can be connected to a computer via USB to graph and record the data in a variety of formats. Energy Inc makes an API available for developers that wish to use the data directly. One developer has created an Open Source extension for the Firefox browser that displays electricity usage from TED in a toolbar inside Firefox \cite{Nick2008TED-the-Toolbar}. TED appears to be the lowest cost option for whole home electricity monitoring with computer data output.

While whole home energy meters provide only household-wide usage data, users can use the real-time display to figure out the impact of particular uses as air conditioning through trial and error experimentation. Parker et al describe a protocol for using a household-wide meter and a circuit breaker panel to localize the energy usage in a home \cite{Parker2006How-Much-Energy}. All the breakers are turned off and then turned on one at a time while recording data from the electrical meter. In 2-4 hours, users were able to generate a spreadsheet mapping the electricity usage in the home.

\subsubsection{Building Energy Displays}

Another type of electricity usage monitoring are building energy displays, which monitor electricity usage for an entire building (usually non-residential, like a school or office building) and display the usage information in some public area such as a lobby. Green TouchScreen \cite{greentouchscreen} and Building Dashboard \cite{building-dashboard} are examples of this product area. These devices aim to make building occupants aware of the overall environmental impact of the building, which is something usually invisible to the occupants. Some systems make the displays available via the web so that users can view the information from their desk as well as the lobby. The displays often provide  information beyond just electricity usage, such as water or natural gas usage, and may display the usage in units other than kWh, such as number of light bulbs lit or hours of TV watching. Beyond their potential utility in helping building occupants to reduce their energy usage, informative displays can be used to get points toward Leadership in Energy and Environmental Design (LEED) certification for a building.

\subsection{Transportation Trackers}

Personal transportation is another large segment of a user's carbon footprint, and one that is largely under their control. The sensors in this section attempt to determine the mode of transportation the carrier is using (walking, driving automobile, riding bus, riding bicycle, etc), and often record the distance travelled as well. Using the mode of transportation and the distance traveled, along with some other information such as the type of car driven, a system can provide an automated estimate of the user's carbon emissions from transportation. In fact, some of these transportation trackers are designed primarily for the purpose of estimating carbon emissions.

\subsubsection{GeoLife}

GeoLife is a transportation tracking system by Zheng et al, described in two papers \cite{zheng-learning-mode-2008, Zheng2008Understanding-mobility}. The goal of the GeoLife system is to record user's transit paths and annotate them with the mode of transportation used. These annotated paths and geotagged media can be shared with other users. This enables applications such as displaying a map that shows that users travelling from point A to point B by car take one route, but those travelling by bicycle take a different route. The GeoLife system tracks its users using GPS data recorded either from GPS-enabled mobile phones or dedicated GPS data loggers.

GeoLife uses canonical machine learning techniques to infer the transportation mode: the GPS location data is broken into segments, distinctive features are extracted from the data, the features are fed into an inference model (the authors compared the results from several options), and the resulting inferences are postprocessed. They explicitly choose to not use map data in their algorithm, since map data can be voluminous, can change based on construction, and requires a source of map data.

Location data is segmented based on two observations: when changing mode, velocity must be near zero at some point, and walking is usually the separator between modes. Thus the change points where segments begin and end are marked by walking. The authors found that raw velocity extracted from GPS data is a poor predictor of transportation mode because weather and traffic can easily impact velocity. They found three features that were predictive of transportation mode:
\begin{itemize}
	\item Heading Change Rate: cars must drive on the road, while walkers or bikers change their heading more frequently
	\item Stop Rate: busses and walkers stop more than cars, so count the number of stops per unit of distance
	\item Velocity Change Rate: change in velocity between two GPS points, as humans speed up and slow down more than cars
\end{itemize}

The authors compared three different classification algorithms with their data: 
Decision Tree, Support Vector Machine (SVM), and Bayesian Network. Using GPS data from 65 users over 10 months, manually annotated by users for ground truth, the authors found that the decision tree algorithm provided the best results. Using the GPS data and ground truth annotations, the overall inference accuracy was 76.2\%. For driving specifically, the precision was 0.861 (if the system inferred that the user was driving, that was true 86.1\% of the time) while the recall was 0.771 (if the user was driving, the system correctly inferred that 77.1\% of the time). While that is an impressive achievement for general purpose transportation inference without map data, for estimation of carbon footprint (which is not the authors focus) driving recall is the most important feature, and 77.1\% accuracy rate might be unacceptable.

\subsubsection{Carbon Diem}

Carbon Diem is a mobile phone application designed to run on GPS-enabled mobile phones \cite{Carbon-Diem-website}. Carbon Diem uses the GPS information from the phone to track what transportation methods the owner is using, and calculate a carbon footprint from that. They are initially targeting Blackberry and Nokia N-series phones, but claim to be ``platform and provider agnostic''. The web site lists AMEE (see \autoref{amee}) as a partner, so they are likely using AMEE to do the carbon footprint calculations. According to this article from the European Space Agency, the two principals have been working on the system since 2006 \cite{ESA-carbon-hero-2008}. They are trying to raise money, and focusing on the corporate market initially \cite{Fehrenbacher2008Carbon-Hero}. The application can ``tell if you drive, fly, take the train or walk'', and if they can sign a deal with a carrier or handset maker they could potentially launch to consumers in Spring 2009. According to this Guardian article, ``the software was almost 100\% accurate in working out when people were on airplanes or trains; it was between 65-75\% accurate at guessing when people travelled on buses'' \cite{Jha2008Carbon-Diem}. Until the system is publically available, it is difficult to determine whether it would be a useful sensor for PET. The key requirement is that it provide some way to export data from the phone or the presumably associated web site.

\subsubsection{Ecorio}

Ecorio is an application for Google's Android mobile phone platform \cite{Ecorio-website}. It uses GPS to detect the user's mode of transportation, and estimates carbon output from that. There is apparently support for detecting how efficiently you are driving, which is an interesting twist (though it is unclear how that could be done safely from a phone while the user is driving). Ecorio also provides suggestions to the user of ways to reduce their carbon footprint, such as links to Google Transit and carpooling information. There appears to be some ``what if'' functionality built in as well, such as how much carbon will I emit if I start taking public transit half the time. Users can also purchase carbon offsets through the application. There are plans to port the application to other platforms (the iPhone is mentioned, but that would be difficult given the restrictions on background processing). There is no indication of whether the carbon footprint data can be exported in any way, since the data and display appears to be local to the device.

\subsubsection{UbiGreen}
\label{ubigreen}

UbiGreen is a research project being undertaken by Intel Research, the University of Washington, and Carnegie Mellon University \cite{ubigreen-website}. UbiGreen uses accelerometers to determine the user's transportation mode (sensing walking, biking, or public transit), and displays the results on mobile devices. The sensor data initially came from the Intel Mobile Sensing Platform, a small belt-mountable device containing accelerometers, a processor, and Bluetooth capability for communicating with mobile phones. The mobile phone displays an image of a tree on its wallpaper: the picture displayed as a background when the phone is first accessed. Each time the sensor detects a green transportation event, the wallpaper on the mobile phone is updated by adding a leaf to the tree. This turns the mobile phone into an ambient glanceable display for the user's green transportation choices. They are working on porting the system to use the accelerometers built into the iPhone so that the system can be contained wholly on the phone.

UbiGreen shares many elements with the UbiFit system, which uses the Mobile Sensing Platform to display users' exercise activities on the wallpaper of their mobile phone \cite{Consolvo2008Flowers-or-robot}. It appears most of the technology is the same, it is just being targeted at a different set of behavior modifications. UbiFit appears to be qualitative in nature, attempting to identify and reward green transportation behaviors, but not quantify exactly how green those behaviors are.

\subsubsection{Dopplr}
\label{dopplr}
Dopplr is a web application for organizing travel plans and coordinating ad hoc meetings with friends who are also travelling \cite{dopplr-website}. Users enter in their travel plans by specifying one or more destinations and the mode of transportation used for each leg of the trip. Dopplr has an index of place names, so users can easily type in the name of each city they are visiting and Dopplr will figure out where each city is located. The system can also parse e-tickets and itineraries forwarded via email, reducing the amount of manual input required.

Dopplr is also a social network where users can share their travel plans with each other. If two friends happen to be in a particular city at the same time because their travel plans intersect, Dopplr will notify both users so they can decide if they want to meet up.

Dopplr now provides each user with a carbon footprint calculated from their trip data. The calculations are done using AMEE (see \autoref{amee}), which uses the distance of each leg and the mode of transportation to compute the number of kilograms of \COtwo emitted by each trip.

While the other sensors described in this section are measuring something about the physical world, Dopplr is more of an information sensor that leverages information that users record about themselves. In particular, the primary goal of Dopplr users is sharing travel plans with their social network, and the carbon calculation is just a convenient analysis that can be performed with the same data.

Dopplr provides an API for developers that wish to extract information about a user's travel plans programmatically. Unfortunately, at this time (December 2008) the API does not allow retrieval of the carbon data. However, the trip information is available from the API, so the calculations themselves could be done outside of Dopplr (potentially even using AMEE). Carter has created a website that calculates the carbon footprint of a user's entire Dopplr social network using this general technique \cite{offsetr-website}. Another limitation of using Dopplr for calculating air travel carbon footprints is that Dopplr generally only records the user's final destination, not the series of flights required to get to the destination. Since many long plane trips involve one or more legs, Dopplr underestimates the carbon footprint by using the (always less than or equal to) direct distance between the start and finish cities. Since Dopplr is designed around social connections, this limitation makes sense because users are unlikely to be able to meet friends between flights.

\subsection{Sensor Discussion}

While the sensors described in this section use ingenious techniques to infer the mode of transportation being used, for purposes of carbon footprint estimation, the most important data is how much the user drives (and flies, though air travel is usually infrequent enough that it could be handled by information sensors like Dopplr, see \autoref{dopplr}). Distinguishing walking from bicycling is mostly irrelevant for carbon footprint estimation since both rely on human power and not generated or stored energy. Public transportation can have a substantial carbon footprint, which could reasonably be apportioned to transit riders. However, from the perspective of personal choice, the bus or train will continue to run (and emit \COtwo) whether or not the user makes use of it, unlike a personal automobile. If the goal is to encourage users to change their behavior to reduce their carbon footprint, taking the bus to work has essentially zero emissions compared to driving to work.

Based on this simplification, the key to estimating the contribution from transportation to a user's carbon footprint is tracking the user's driving. This is a simpler problem than general transportation tracking, and even semi-manual methods could suffice. For example, each time the user fills their gas tank, they could use a simple mobile web application to record their current odometer reading and the number of gallons purchased. This data could even be recorded on paper in the vehicle and entered into a web application when the user is next in front of a web browser, eliminating the need for a mobile device.

Another issue to be aware of is the difference between efficiency and overall emissions. In the wake of gasoline price spikes and global warming fears, many drivers are keenly aware of their gas mileage. Those driving vehicles that display the instantaneous and historical gas mileage may strive to continually improve their mileage. Improved gas efficiency through skilled driving techniques is to be applauded, but the carbon footprint is the product of gas mileage with the number of miles driven. The carbon emitted through driving an efficient hybrid 50 miles daily will be much higher than a large SUV that is  driven only rarely. Efficiency is only an unalloyed good if all the driving done is non-discretionary. The unstated assumption when discussing the quest for higher gas mileages is that all driving is non-negotiable, though this is rarely the case. Tracking actual carbon emitted is a better metric to optimize, which could be spurred by a ``mileage'' diet.

\subsection{Tracking Purchases of Goods}

Another area responsible for carbon emissions are purchases that individuals make. For example, there is carbon footprint attached to a box of cereal purchased from the supermarket. Some manufacturers are moving towards carbon footprint labelling, just as there are labels listing ingredients and nutritional information. Calculating a product's carbon footprint turns out to be a complicated task, as there are both the direct emissions for the creation of the product and the indirect emissions for all the raw materials required to make it \cite{Wiedmann2007carbon-footprint}. Joseph has proposed creating a database of  environmental information about products, using the Universal Product Codes (UPC) present on most mass produced goods as index \cite{Joseph2008-personal-comm}. Mobile phones with cameras could be used to scan the UPC label on a product to instantly retrieve environmental data about the product. Dada et al have produced a demo of a carbon labelling system that uses Near Field Communication (NFC) labels that can be read wirelessly using a mobile phone \cite{dada-demo-pervasive-2008}. A printed carbon footprint label would be static, while an NFC label could produce a dynamic footprint, including information such as how the product was transported to the point of sale.

Until there is a way to determine a quantitative carbon footprint for a product, this type of data would be difficult to include in a PET-style analysis. The number of products that the average person purchases would also necessitate that the data entry be mostly automated.

%\subsection{Financial Data}
%
%\subsubsection{Wesabe?}
%
%\subsection{Manual Input}
%\subsubsection{Evernote?}


\section{Carbon Footprint}

The goal of the PET system is to make users aware of their carbon footprint and help them to reduce it, but what exactly does the term `carbon footprint' mean? If users are to reduce their footprint, it is crucial that we have a rigorous definition of the term.

Wiedmann and Minx take up this issue, noting that the term is used extensively but there is no standardized definition \cite{Wiedmann2007carbon-footprint}. Some of the questions they examine include whether carbon footprint should include other GHG like methane (which can have potent greenhouse effects), or whether non-fossil fuel emissions should be included (such as soil emissions). Another critical issue is whether the footprint should include only direct emissions, or should it also include upstream emissions? If upstream emissions are included, how are should the boundaries be set, as there is the very real risk of double counting emissions as multiple entities trace emissions upstream. Even the units of measurement for carbon footprints are not standardized: many use mass, but some have suggested pressure or area (hewing more literally to the concept of a ``footprint''). Wiedmann and Minx propose the following definition:

\begin{quote}
``The carbon footprint is a measure of the exclusive total amount of carbon dioxide emissions that is directly and indirectly caused by an activity or is accumulated over the life stages of a product.''
\end{quote}

This definition includes only \COtwo emissions, as they make up the majority of the GHG associated with climate change, it includes both direct and indirect emissions, and it standardizes on units of mass.

\subsection{Calculating Carbon Footprints}

Wiedmann and Minx describe two methods for actually calculating the carbon footprint of something: Process Analysis (PA), and Environmental Input-Output analysis (EIO) \cite{Wiedmann2007carbon-footprint}. PA is a bottom-up process used when calculating the impact of a product from creation to destruction. The primary focus of PA is direct emissions, but it can also include some second order impacts. To avoid double counting, defining the boundaries for the analysis is critical for PA. While PA works well for products, it has problems scaling up to households, industries, or governments. 

EIO works at the economic sector level, including all economic activities and environmental data, using the ``whole economic system as boundary''. EIO does not work well for micro systems such as an individual product, but it requires fewer resources to process one it has been set up. Wiedmann and Minx recommend that a hybrid of PA and EIO are used for carbon footprint calculations: using PA for the low-level portions and relying on EIO for indirect effects.

\subsection{AMEE}
\label{amee}

AMEE (Avoiding Mass Extinctions Engine) is a system designed to be ``the world's energy meter'' \cite{AMEE-website}. AMEE seeks to be a neutral platform for organizations to record energy consumption data and calculate their carbon emissions from energy use. AMEE provides web service APIs that developers can use to record energy data and calculate carbon footprints. AMEE aims to be as transparent as possible: the software behind AMEE is open source, energy data is anonymized and made available to others, and the methodologies for calculating footprints are visible for all to see and comment on. AMEE is being used by a variety of organizations, including the UK Department for Environment, Food and Rural Affairs (DEFRA), The Irish Government, The Welsh Assembly, Google, and Morgan Stanley.

AMEE would seem to be an ideal resource for PET, as it can be used both for storing energy usage data, and converting that data to a carbon footprint using standardized and peer-reviewed models.

\subsection{Web Carbon Footprint Calculators}
\label{carbon-calculators}

There are now many websites that offer to estimate a user's carbon footprint based on questionnaire data solicited from the user, such as the type of home, number of miles driven per year, and number of miles flown per year. Murray and Dey survey eleven such websites, finding a variety of differences and deficiencies \cite{Murray2007Carbon-neutral}. They found that carbon calculators rarely account for upstream emissions, which can be an important part of the emissions picture. Most of the calculators have different inputs, so they are not directly comparable, which leads to confusing results. Using a standardized set of input values (as much as could be standardized given that not all sites used the same input values), they found that footprints varied between sites. The authors suggest that carbon calculators be standardized to make them more useful to consumers. Using AMEE to perform the carbon calculations would be one way to achieve standardization, and some calculators do use AMEE as their backend.

\subsubsection{Carbon Offsets}

Carbon calculators are often intertwined with the concept of carbon offsets. Carbon offsets try to provide a means for individuals who are emitting more carbon than they would like to \emph{offset} those emissions by enabling the reduction of emissions elsewhere via a payment. The money paid to purchase the offset is used to fund emission reduction work, such as the planting of trees, the construction of renewable energy capacity, or the implementation of energy efficiency measures. To be \emph{carbon neutral} is to purchase offsets sufficient to offset all of an individual or organization's carbon emissions. Some offsets are sold by non-profit organizations, but many are sold by for-profit companies.

Murray and Dey take a skeptical view towards carbon offsets and carbon neutrality in particular \cite{Murray2007Carbon-neutral}. They point out that the concept of offsets is nothing new, and they compare it unfavorably to the selling of indulgences in the Middle Ages by the Catholic church. The authors argue that to provide real offsets, accurate carbon measurement is required, along with accounting of offsets (to ensure that the same reduction in carbon emissions is not sold multiple buyers), and verification of the offsetting activities by a third party. They investigated some organizations selling offsets to determine where the money was actually being spent on emission offsetting activities, and found that it was challenging to determine what was actually taking place as opposed to what the companies led offset buyers to believe. To deal with these problems, the authors suggest transparency in how much of the offset money actually goes to projects as opposed to how much is kept by the organization selling the offsets. They suggest that offset buyers look at the projects being supported, and only buy offsets if they would have supported those projects anyway, regardless of any carbon benefits.

Another issue with carbon offsets is the issue of inevitability: carbon is not being offset if the project recipient was planning to perform the emissions offsetting activities anyway. According to Murray and Dey, close to 50\% of Clean Development Mechanism projects (``projects controlled by the Kyoto protocol and registered with the United Nations'') checked were ``not additional to the baseline'', meaning they would have happened anyway without the offset money.

\subsection{Brief Look at Some Calculators}

To examine the results of carbon calculations from these calculators first hand, I calculated my own carbon footprint using a standardized set of data:
\begin{itemize}
	\item Home emissions: An apartment of average size in Hawai`i with an electrical bill of \$85 per month (estimate due to indirect billing through landlord)
	\item Auto emissions: 1 Toyota Prius, 2227 miles in 245 days = 3318 $\frac{mi}{yr}$ at 38 $\frac{mi}{gal}$
	\item Air travel: HNL $\rightarrow$ CMI $\rightarrow$ PDX $\rightarrow$ HNL = 17180 mi, HNL $\rightarrow$ CMI (round trip) = 8440 mi, HNL $\rightarrow$ Seoul, South Korea (round trip) = 9100 mi, total for 2008 = 34720 mi
\end{itemize}

\begin{table}[htbp]
	\begin{center}
		\caption{Sample of online carbon footprint calculators}
		\label{tab:carbon-footprint-calculators}
		% need small font sizes to make table fit on page
		\scriptsize
		\urlstyle{tinyurl}

		\begin{tabular}{| l | p{7cm} | p{5cm} |}
			\hline
			Organization & URL & Notes \\ \hline
			The Climate Trust & \url{http://www.carboncounter.org/} & Non-profit, focus on providing offsets \\ \hline
	
			Carbon Footprint Ltd & \url{http://www.carbonfootprint.com/} & UK-based business, focus on offsets \\ \hline
	
			The Nature Conservancy & \url{http://www.nature.org/initiatives/climatechange/calculator/} & Non-profit conservation org \\ \hline
	
			U.S. EPA & \url{http://www.epa.gov/climatechange/emissions/ind_calculator.html} & government agency \\ \hline

			Inconvenient Truth & \url{http://www.climatecrisis.net/takeaction/carboncalculator/} & Documentary companion site \\ \hline

			World Resources Institute & \url{http://www.safeclimate.net/calculator/} & environmental think tank \\ \hline

			Evolution Sage & \url{http://www.evolutionsage.com/calculate.html} & Hawai`i-specific calcs \\ \hline
		\end{tabular}
	\urlstyle{smallurl}
	\end{center}
\end{table}

\subsubsection{Dopplr}
Dopplr (see \autoref{dopplr}) based on out-of-state travel only (flights plus driving between cities on Pacific Northwest road trip) for 2008: 5,871 kg \COtwo.

\subsubsection{Carbon Counter}
\begin{itemize}
	\item home emissions (estimated due to lack of electrical usage data) = 3.73 metric tons \COtwo
	\item auto emissions (exact) = 0.77 metric tons \COtwo
	\item air travel emissions = 20.58 metric tons \COtwo
\end{itemize}

Note that their air travel \COtwo value is more than 3 times larger than the Dopplr value via AMEE, and that includes some long distance car trips. Carbon Counter sells offsets, so it would be in their financial interest to skew towards higher emissions. Information on their calculation methods can be found at \url{http://www.carboncounter.org/offset-your-emissions/calculations-explained.aspx}

\subsubsection{Inconvenient Truth}
\begin{itemize}
	\item Input auto emissions by make \& model of car with number of miles driven per year = 0.65 metric tons \COtwo
	\item air travel emissions by length of flight (5 extended trips [8 hrs or 5000 miles], 2 long trips [4-6 hrs or 2500 miles]) = 5.85 tons \COtwo
	\item home emissions based on average electrical bill of \$75-\$100 with 0\% of energy coming from renewable sources in Honolulu\footnote{Based on November 2008 Monthly Energy Trends report from the Hawai`i Department of Business, Economic Development, and Tourism \url{http://hawaii.gov/dbedt/info/economic/data_reports/energy-trends/}}: 2.2 metric tons \COtwo
\end{itemize}
\chapter{System Description}
\label{cha:system-description}

\section{Contest Design}

\section{Website Design}

\section{Competition Tasks}

\section{WattDepot}

\subsection{REST API}

\section{Risks}

\subsection{Meter installation}

\subsection{Student Housing cooperation}

\subsection{Participant engagement}

\chapter{Experiment}


\section{Data Sources}

\subsection{Power Usage Data}
\label{sec:power-usage-data}

We will record both instantaneous power and cummulative energy consumed on a floor by floor basis, beginning long before the competition starts and continuing indefinitely after the competition ends. The sampling rate will be at least 1 minute outside the competition period, and less than 1 minute during the competition period (target of 10 seconds). We can compute the following useful values based on this data:

\begin{itemize}

\item \emph{Minimum floor power} is the power consumed by each floor before residents move in and with all switchable devices (such as lights) turned off. This reveals the power used by the hidden infrastructure of a floor, and may be differ between floors. The value is measured by recording the kWh consumed by each floor over a period of time (preferably days to average out any periodic consumption spikes) and divided by the length of the time interval.

\item \emph{Pre-competition average floor power} is the power consumed by each floor after residents move in, but before the competition has begun. This reveals the power use profile of the floor's residents, and will almost certainly differ between floors. The value is measured by recording the kWh consumed by each floor over a long period of time (preferably weeks to average out any periodic consumption spikes) and divided by the length of the time interval.

\item \emph{Pre-competition total monthly floor energy} is the energy consumed by each floor after residents move in, but before the competition has begun. This reveals the power use profile of the floor's residents, and will almost certainly differ between floors. The value is measured by recording the kWh consumed by each floor over a long period of time (preferably weeks to average out any periodic consumption spikes) and extrapolated to a monthly value. Thus if 15 days of data are recorded, then the pre-competition total monthly floor energy would be twice the kWh value recorded for the 15 day period.

\end{itemize}

\subsection{Pre and Post-Competition Energy Literacy Questionnaires}
\label{sec:exp-literacy-questionnaire}

The energy literacy of participants will be assessed at the start and end of the competition. The assessment will be through a questionnaire that is presented to participants via the contest website as an activity that can be performed for Kukui Nut points. The pre-competition questionnaire will be made available only in the first week of the competition, while the post-competition questionnaire will be made available only in the final week of the competition. \autoref{app:energy-literacy} lists the questions that will make up the pre and post-contest questionnaires.

Since the website-administered questionnaire is simply a task that can selected by participants, there is the potential that only those participants that feel that they are energy literate will participate in the survey, leading to bias. For this reason, in addition to administration through the website, the questionnaire will be administered in person on paper to two randomly-selected floors. While the assignment of residents to a floor is not random, it is at least not self-selected. The questionnaire will be administered to the floors before the competition starts, and in the final week of the competition. The questionnaire will be removed from the activity lists of participants on the selected floors. However, those participants that fill out the survey on paper will receive Kukui Nut points just as if they had filled it out online.

\subsection{Website Log Data}

The contest website will extensively log data about participants' actions on the site. All participant actions and events will be logged with timestamp.

\subsection{Post-Competition Qualitative Feedback Questionnaire}

After the competition has ended, participants that used the website will be emailed a link to a qualitative questionnaire, as part of the energy literacy post-test described in \ref{sec:exp-literacy-questionnaire}. This questionnaire will ask for participants' assessment of the competition, the website, and energy literacy in general.

\subsection{Post-Post-Competition Sustainable Conservation Questionnaire}

In early in the following semester (February 2011), the power data for floors will be re-examined to see whether conservation begun as part of the competition has been sustained months later. Floors with particularly high sustained conservation (compared to pre-competition average floor power), and those with low or non-conservation will be selected for an additional questionnaire, and possible face-to-face interviews to determine residents' self-assessment about why they were or were not sustaining the conservation gains made during the competition.

\section{Research Questions}

\subsection{Energy usage}


\begin{itemize}

\item How can student housing residents be motivated to reduce their electricity usage?

\item How sustainable are any electricity usage reductions after the competition is complete?

\item How can energy literacy be assessed?

\item How does energy literacy impact sustained energy conservation?

\item What activities are most helpful in improving participants' energy literacy?

\item How effectively can the tools of behavior modification be instantiated in a web application?

\item How important is community-level near-real-time electricity usage feedback to achieving electricity conservation?

\end{itemize}


\section{Hypotheses}

\subsection{Conservation during competition}

\emph{Averaged across all participating floors, the total energy use during the competition will be less than the pre-competition total monthly floor energy usage.} Other student housing energy competitions have demonstrated that the competition will lead to energy conservation of at least X\% (see \autoref{sec:dorm-energy-competitions}). Also, the addition of near-realtime energy feedback has been shown over several studies to lead to conservation values from 5\% to 15\%.

Testing this hypothesis is straightforward. The collected power and energy data for each floor (see \autoref{sec:power-usage-data}) will show how much energy has been used during the competition, which can be compared to the average usage recorded during the pre-competition period.

\subsection{Conservation after competition}

\emph{Averaged across all participating floors, the total energy use after the competition will be less than the pre-competition total monthly floor energy usage.} Sustained energy conservation after student housing competitions has not been throughly investigated. We hypothesize that the integration of energy literacy into the competition will lead to at least some of the energy conservation fostered during the competition being sustained afterwards.

Testing this hypothesis is straightforward. The collected power and energy data for each floor (see \autoref{sec:power-usage-data}) will show how much energy has been used after the competition, and can be broken into weekly and monthly periods. The post-competition energy usage can be compared to the average usage recorded both before and during the competition period.

\subsection{Energy literacy's effect on conservation}

\emph{A floor's average score on the post-competition energy literacy questionnaire will be inversely corrolated with the floor's energy usage during and after the competition.} We hypothesize that sustained energy conservation is fostered by increased energy literacy, therefore we would expect those floors that are most energy literate would use the least energy.

The collected power and energy data for each floor (see \autoref{sec:power-usage-data}) will show how much energy has been used after the competition, and can be broken into weekly and monthly periods. The post-competition energy usage can be compared to the average usage recorded both before and during the competition period.

\begin{itemize}

\item Participants with Kukui Nut scores will have higher and more improved post-test energy literacy scores.

\item Improving residents' energy literacy will lead to sustained electricity conservation.

\item Floors with large electricity conservation during the competition period but with low energy literacy scores post-test will have greater rebound effect than floors with higher energy literacy scores post-test.

\item Residents that complete more activities (as specified by the competition website) will improve their energy literacy more than residents that do not participate.

\item How well do Kukui Nut scores correlate with post-test energy literacy?

\end{itemize}
\chapter{Conclusion}

\section{Anticipated contributions}

\section{Future Directions}

\section{Timeline}

\begin{itemize}
\item Spring 2010: competition design, website design, buy-in from stakeholders
\item Summer 2010: install meters in dorms
\item September 2010: competition website complete
\item October 2010: competition takes place
\item November 2010: data analysis and dissertation writing begin in earnest
\item February 2011: followup study takes place
\item May 2011: dissertation defense
\end{itemize}


%%% Switch to appendix mode
\appendix
%%% Bring in any appendices from external files
%%%%%%%%%%%%%%%%%%%%%%%%%%%%%% -*- Mode: Latex -*- %%%%%%%%%%%%%%%%%%%%%%%%%%%%
%% uhtest-appendix.tex -- 
%% Author          : Robert Brewer
%% Created On      : Fri Oct  2 16:31:12 1998
%% Last Modified By: Robert Brewer
%% Last Modified On: Mon Oct  5 14:41:05 1998
%% RCS: $Id: uhtest-appendix.tex,v 1.1 1998/10/06 02:07:03 rbrewer Exp $
%%%%%%%%%%%%%%%%%%%%%%%%%%%%%%%%%%%%%%%%%%%%%%%%%%%%%%%%%%%%%%%%%%%%%%%%%%%%%%%
%%   Copyright (C) 1998 Robert Brewer
%%%%%%%%%%%%%%%%%%%%%%%%%%%%%%%%%%%%%%%%%%%%%%%%%%%%%%%%%%%%%%%%%%%%%%%%%%%%%%%
%% 

\chapter{Participant Kukui Nut Tasks}
\label{app:tasks}

This appendix lists tasks intended to be undertaken by the competition participants. Each task should increase the energy literacy of the participants performing it, help them modify their behavior to reduce electricity usage, or both. The following lists all the possible tasks, and indicate how they would be performed, validated, and what the potential benefit would be to the person performing it. The tasks are grouped into four categories: events, activities, commitments, and goals. For more information, see \autoref{sec:competition-tasks}.


\section{Events}

One common type of task is attendance of an event. In our model, there are two ways to get credit for attending an event: activities (individual attendance), and goals (floor attendance). Since the parameters are often identical between the activity version and the goal version of an event, they are grouped together here.

For both event activities and goals, attendance is verified using non-forgeable, single-use attendance codes such as "orientation-158-B7QRX13". The codes are printed on small slips of paper that are handed out by some responsible person who is not a participant (such as the event speaker or an RA).

In the case of activities, to get credit for attending, the individual participant logs into the web site and enters in the attendance code. The website automatically awards KN points if the attendance code is valid, and it has not already been entered.

For goals, the participant that initiated the goal must log into the website after the event and indicate that the goal was met (perhaps prodding any floormates to enter their attendance codes if they haven't alread done so). The website will then award the appropriate KNs to all members of the floor (including those who did not attend). Goals must have the participation of at least half of the floor participants to be successful. If a floor achieves 100\% participation, they receive double the KN.

Relatively passive events like movies or lectures should be worth around 5 KN, while more interactive events like workshops should be worth more (perhaps 10-15 KN).

\subsection{Attend Kukui Cup orientation}

Description: Participant attends a large orientation meeting about the Kukui Cup competition.

Potential benefits: Understanding of the competition mechanics, collaboration with other floor participants on competition strategy.

Psychological justifications: ?

Activity reward: 4 KN

Goal reward: 5 KN (unlikely to be obtained, since this happens at very begining of competition)

\subsection{Attend EnergyPong tournament}

Description: Participant attends the EnergyPong tournament for their building.

Potential benefits: Improved energy literacy through hearing energy questions answered, floor bonding.

Psychological justifications: competition

Activity reward: 2 KN

Goal reward: 4 KN

\subsection{Attend a special Kukui Cup SustainableUH meeting}

Description: Participant attends a special presentation by SustainableUH team members on what SustainableUH is doing on campus.

Potential benefits: Getting involved with peers on campus, learning what challenges exist and how students are working to overcome them.

Psychological justifications: ?

Activity reward: 2 KN

Goal reward: 5 KN

\subsection{Watch the movie "Who Killed the Electric Car"}

Description: Participant watches the movie.

Potential benefits: Understanding of the possibility of de-carbonizing transportation, difficulty of changing status quo.

Activity reward: 2 KN

Goal reward: 5 KN

\subsection{Watch the movie "Enron: The Smartest Guys in the Room"}

Description: Participant watches the movie.

Potential benefits: Understanding risks and problems from utility deregulation, ethical issues.

Activity reward: 2 KN

Goal reward: 5 KN

\subsection[Watch the movie ``The End of Suburbia'']{Watch the movie ``The End of Suburbia''}

Description: Participant watches the movie.

Potential benefits: Understanding peak oil, design of communities around automotive transportation and plentiful cheap energy.

Activity reward: 2 KN

Goal reward: 5 KN

\subsection{Watch the movie "A Crude Awakening: Oil Crash"}

Description: Participant watches the movie.

Potential benefits: Understanding peak oil, consequences for society.

Activity reward: 2 KN

Goal reward: 5 KN

\subsection{Watch the movie "The Great Warming"}

Description: Participant watches the movie.

Potential benefits: Understanding climate change, consequences for society.

Activity reward: 2 KN

Goal reward: 5 KN

\subsection{Watch the movie "An Inconvenient Truth"}

Description: Participant watches the movie.

Potential benefits: Understanding climate change, consequences for society.

Activity reward: 2 KN

Goal reward: 5 KN

\subsection{Participate in a 10/10/10 work party}

Description: [http://www.350.org/ 350.org], a climate change advocacy organization is organizing a series of "work parties" to take place on October 10, 2010 (10/10/10). Participant participates in a work party in Honolulu (check website for options). Since this is off campus, might need to support alternate verification (photo and text) instead of attendance codes.

Potential benefits: Understanding climate change, consequences for society.

Activity reward: 5 KN

Goal reward: 7 KN


\section{Activities}

\subsection{Perform room energy audit}

Description: Resident borrows a Kill-A-Watt plug load meter from their RA, then checks all plug-in appliances in their room to see what their energy consumption is when on and off.

Verification: Participant fills out form on website that contains a list of rows for each device with columns: device name, power (watts) when off, power (watts) when on, notes. Admin reviews data, checking mainly for completeness (more than 1 device?) and sanity (XBox 360s don't use 1000 W).

Reward: 10 KN

Potential benefits: Increased intuitive understanding of the watt, familiarity with vampire power, understanding of how device usage would impact energy consumption, reduced electricity usage due to turning off devices when not in use.

Psychological justifications: feedback, activity-based learning (?)

\subsection{Replace incandescent bulb with compact fluorescent (CFL)}

Description: Participant finds an incandescent bulb (perhaps from a desk lamp) and replaces it with a CFL, throwing away the incandescent bulb.

Verification: Participant takes a picture showing both the incandescent bulb and the CFL replacement and uploads it via a verification form on the website, along with a text field indicating where the replaced bulb is located. Admin briefly reviews the picture to ensure that in fact both bulbs are present.

Reward: 3 KN

Potential benefits: Reduced energy usage via CFL, awareness of energy impact of incandescent bulbs.

Psychological justifications: activity-based learning (?)

\subsection{Configure computer \& monitor to sleep after inactivity}

Description: Participant configures their computer and any external display to sleep after 20 minutes of inactivity.

Verification: Participant takes a screenshot from their computer showing sleep settings <= 20 minutes and uploads it via a verification form on the website. Admin briefly reviews the picture to ensure that the settings look correct.

Reward: 3 KN

Potential benefits: Reduced computer \& monitor energy usage, knowledge of how to set it up on other computers (friends, work, future purchases, etc).

Psychological justifications: none

\subsection{Play in EnergyPong tournament}

Description: Participant is on their floor's team in the EnergyPong tournament for their building.

Verification: Some responsible person who is not a participant (such as the speaker or an RA) records attendance and performance, which is reported to the website admins either on paper or via email.

Reward: 4 KN + 1 KN per bracket completed + 5 KN for the winning team

Potential benefits: Improved energy literacy through answering energy questions answered, floor bonding.

Psychological justifications: competition, incentives (if prizes are awarded to winning team)

\subsection{Connect to Kukui Cup on Facebook}

Description: Participant becomes a fan of the Kukui Cup Competition group on Facebook.

Verification: Participant takes a screenshot from their computer showing Facebook fan status. Admin briefly reviews the picture to ensure that the participant is a fan.

Reward: 3 KN

Potential benefits: Another avenue for communicating with students, promotion of the contest and energy literacy.

Psychological justifications: community involvement?

\subsection{Tweet about Kukui Cup}

Description: Participant sends a tweet promoting the Kukui Cup Competition with a link to the website.

Verification: Participant takes a screenshot from their computer showing the tweet in their newsfeed. Admin briefly reviews the picture to ensure that the participant tweeted.

Reward: 2 KN

Potential benefits: Promotion of the contest and energy literacy.

Psychological justifications: social networking?

\subsection{Facebook Status update about Kukui Cup}

Description: Participant updates their Facebook status promoting the Kukui Cup Competition with a link to the website.

Verification: Participant takes a screenshot from their computer showing the status in their newsfeed. Admin briefly reviews the picture to ensure that the participant updated their status.

Reward: 2 KN

Potential benefits: Promotion of the contest and energy literacy.

Psychological justifications: social networking?

\subsection{Label all plug loads in room}

Description: Followup to room energy audit. Based on the audit results, make a label for each device with the number of watts consumed when on and off, located close to the power switch for those devices that have them.

Verification: Participant takes a picture of the devices with their labels. Admin briefly reviews the picture to ensure that labels are present.

Reward: 3 KN

Potential benefits: understanding of how device usage would impact energy consumption, reduced electricity usage due to turning off devices when not in use.

Psychological justifications: prompts

\subsection{Determine carbon footprint using calculator}

Description: Participant uses a web-based carbon footprint calculator to determine their carbon footprint.

Verification: Participant enters in their computed carbon footprint into a text field. Admin briefly reviews the footprint to make sure it is sane (units include CO2 and it isn't huge or tiny).

Reward: 3 KN

Potential benefits: learning about carbon emissions, learning how carbon emissions impact the environment.

Psychological justifications: personalized data


\section{Commitments}

Note that per the requirements, commitments are participant-verified without outside intervention, so that field is not used for this category.

\subsection{Turn off lights when I leave the room}

Description: The participant commits to turning off all lights whenever they are the last person to leave a room.

Reward: 2 KN

Potential benefits: Reduced electricity usage due to less unneeded lighting, highly obvious reminder of need to conserve energy.

Psychological justifications: public commitments

\subsection{Use task lighting instead of overhead lights}

Description: The participant commits to using task lighting (i.e. a desk lamp) instead of overhead room lights. Might only be appropriate if housing rooms have overhead lights.

Reward: 2 KN

Potential benefits: Reduced electricity usage due to less excess lighting.

Psychological justifications: public commitments

\subsection{Always disconnect vampire loads using a power strip}

Description: The participant commits to turning off any vampire loads (cell phone charger, iPod charger, game consoles, TVs) using a power strip when they are not using them.

Reward: 2 KN

Potential benefits: Reduced electricity usage due to vampire loads, awareness of vampire loads.

Psychological justifications: public commitments

\subsection{Turn off water when brushing teeth, shaving, etc}

Description: The participant commits to turning off any vampire loads (cell phone charger, iPod charger, game consoles, TVs) using a power strip when they are not using them.

Reward: 2 KN

Potential benefits: Reduced electricity usage due to vampire loads, awareness of vampire loads.

Psychological justifications: public commitments

\subsection{Turn off water when sudsing and scrubbing in shower}

Description: The participant commits to turning off water when showering except when actively rinsing off.

Reward: 2 KN

Potential benefits: Reduced electricity usage due to less water heating and pumping.

Psychological justifications: public commitments

\subsection{Use natural light instead of electric lighting whenever possible}

Description: The participant commits to using natural light from windows or outdoors instead of turning on electric lighting. This can mean opening shades instead of turning on the lights, and/or planning their day so that tasks that require light (like reading books, doing written homework) are done during the day.

Reward: 2 KN

Potential benefits: Reduced electricity usage due to less use of electric lights.

Psychological justifications: public commitments

\subsection{Turn off printer when not printing}

Description: The participant commits to turning off their printer when they are not actively printing out documents.

Reward: 2 KN

Potential benefits: Reduced electricity usage due to less standby electricity for printer.

Psychological justifications: public commitments

\subsection{Use stairs instead of elevator}

Description: The participant commits to using the stairs instead of elevators whenever feasible.

Reward: 2 KN

Potential benefits: Reduced electricity usage due to less elevator traffic. Increased exercise for participant.

Psychological justifications: public commitments

\subsection{Recycle all beverage containers}

Description: The participant commits recycling all (recyclable) beverage containers at an appropriate location.

Reward: 2 KN

Potential benefits: Reduced carbon emissions due to recovery and eventual reuse of recyclable material, reduction in waste stream.

Psychological justifications: public commitments

\subsection{Don't drive off-campus using a single-occupant car}

Description: The participant commits to not traveling off-campus in single-occupant car, using bus, bike, walking, or vehicle with 3+ occupants instead.

Reward: 2 KN

Potential benefits: Reduced carbon emissions due to less single occupant car travel, reduction in traffic and parking.

Psychological justifications: public commitments

\subsection{Turn off/shut down all appliances before going to sleep}

Description: The participant commits to turning off or shutting down appliances like computers, TVs, DVD players, and game consoles before going to sleep each night.

Reward: 2 KN

Potential benefits: Less electricity wasted on appliances that aren't being used.

Psychological justifications: public commitments

\subsection{Limit TV watching to 1 hour a day or less}

Description: The participant commits to watching not more than 1 hour of TV per day.

Reward: 2 KN

Potential benefits: Less electricity used by television.

Psychological justifications: public commitments

\subsection{Do only full loads of laundry}

Description: The participant commits to always doing full loads of laundry.

Reward: 2 KN

Potential benefits: Less electricity \& hot water used per piece of laundry washed.

Psychological justifications: public commitments

\subsection{Wear Kukui Cup button every day}

Description: The participant commits to wearing their Kukui Cup button every day during the commitment period.

Reward: 2 KN

Potential benefits: promotion of the contest.

Psychological justifications: public commitments

\subsection{Walk to destinations less than one mile away}

Description: The participant commits to walking to any destination less than one mile away from their residence hall.

Reward: 2 KN

Potential benefits: Reduced gasoline usage due to car usage. Increased exercise for participant.

Psychological justifications: public commitments

\subsection{Wash laundry in cold water}

Description: The participant commits to washing laundry in cold water instead of warm or hot water.

Reward: 2 KN

Potential benefits: Reduced electricity usage by reduction in water heating and pumping.

Psychological justifications: public commitments

\subsection{Reduce the shower time by 1 minute}

Description: The participant commits to measuring the length of their shower with a watch, and reducing the time by 1 minute.

Reward: 2 KN

Potential benefits: Reduced electricity usage by reduction in water heating and pumping.

Psychological justifications: public commitments

\subsection{Turn off music when leaving room}

Description: The participant commits to turning off their music (from computer, stereo, etc) when they leave the room.

Reward: 2 KN

Potential benefits: Reduced electricity usage.

Psychological justifications: public commitments

\subsection[Do something ``unplugged'' every day]{Do something ``unplugged'' every day}

Description: The participant commits to doing something that doesn't require electricity instead of watching TV, using their computer, or playing a console game.

Reward: 2 KN

Potential benefits: Reduced electricity usage, increased exercise?

Psychological justifications: public commitments

\subsection{Bring reusable bags when shopping}

Description: The participant commits to bringing and using reuseable bags when shopping instead of the paper or plastic ones offered by the store.

Reward: 2 KN

Potential benefits: Reduced waste, reduced carbon footprint.

Psychological justifications: public commitments

\subsection{Don't eat meat}

Description: The participant commits to not eating any meat (beef, pork, chicken, fish, shellfish, etc) for the commitment period.

Reward: 2 KN

Potential benefits: Reduced carbon footprint, potentially improved health.

Psychological justifications: public commitments


\section{Goals}

\subsection{Reduce our floor's energy consumption by {target} }

Description: A floor participant picks a target goal (hopefully in consultation with rest of floor) for reduction for the current period from a list of choices from 5\% to 50\% in 5\% increments.  When the goal is specified, the system uses the Average Floor Power as the value being reduced from.  The system can provide a graphic that is updated in near-real time to show whether (a) the current usage is above or below the target, and (b) whether the cumulative usage so far is above or below the target.  The graphic can also provide a count-down timer showing the time remaining to achieve this goal in days:hours:minutes.

Note that the percentage reduction is always relative to the baseline, not the prior week.  So, a floor might start out with a conservative goal of 5\%, then find that they actually achieved 16\% during the period. So, they could restart the goal for the next period, this time choosing 15\%.

Verification: Participant that picked the goal must use the web interface indicate that the goal has been met or not met, and an admin assigns points accordingly.

Reward: If the floor achieves the target reduction, then each member of the floor is awarded 1 KN per target percentage reduction. For example, if the target reduction was 5\% and the floor achieved 7\%, then each member gets 5 KNs for achieving this goal.

Potential benefits: Reduced electricity usage, group planning for how to achieve target through behavior changes.

Psychological justifications: goal setting with feedback, social norms

\subsection{Finding the minimum floor power}

Description: A floor participant picks a day and time for the floor to try to determine the minimum amount of power the floor can consume. Everyone on the floor must disconnect and unplug all loads, all lights must be turned off, etc. Then, using a laptop or mobile device, the floor's instantaneous power value is recorded from the monitors on the contest website.

Note that this goal requires near 100\% participation to be successful.

Verification: Participant that picked the goal must use the web interface indicate what power value the floor was able to record. The admin can then compare this to the [NormalizingPowerData Minimum Floor Power]. If the participants got within 10\% of the MFP, then the KN are awarded.

Reward: Each member gets 10 KNs for achieving this goal.

Potential benefits: Awareness of building infrastructure power draws, group collaboration to turn everything off, awareness of vampire loads.

Psychological justifications: ?

%%%%%%%%%%%%%%%%%%%%%%%%%%%%%% -*- Mode: Latex -*- %%%%%%%%%%%%%%%%%%%%%%%%%%%%
%% uhtest-appendix.tex -- 
%% Author          : Robert Brewer
%% Created On      : Fri Oct  2 16:31:12 1998
%% Last Modified By: Robert Brewer
%% Last Modified On: Mon Oct  5 14:41:05 1998
%% RCS: $Id: uhtest-appendix.tex,v 1.1 1998/10/06 02:07:03 rbrewer Exp $
%%%%%%%%%%%%%%%%%%%%%%%%%%%%%%%%%%%%%%%%%%%%%%%%%%%%%%%%%%%%%%%%%%%%%%%%%%%%%%%
%%   Copyright (C) 1998 Robert Brewer
%%%%%%%%%%%%%%%%%%%%%%%%%%%%%%%%%%%%%%%%%%%%%%%%%%%%%%%%%%%%%%%%%%%%%%%%%%%%%%%
%% 

\appendix
\chapter{Energy Literacy Questions}

This appendix lists the questions that assess participants' energy literacy. The questions are separated into sections based on the topic they are addressing. We provide an even number of questions for each concept being tested, so that they can be used to assess energy literacy both before and after the competition. To determine if the phrasing of the question impacts the results, half of the participants will be given the first question in pre-test, while the other half will get the second question in pre-test, and vice versa in the post test. Keywords have been attached to each question to indicate which subjects they attempt to assess. This is useful to ensure that there exists energy literacy content that addresses the concept represented by each keyword.

The questions are intended to be displayed one at a time without the ability for the participant to go back to previous questions, as later questions may imply the answer to previous questions. When administered via a web site, this is straightforward to accomplish. When administered on paper, each question could be printed on a separate sheet of paper, and participants could be directed to not turn back to previous pages.

\section{Power and Energy Concepts}

\subsection{Watt definition}

\begin{question}
	\item The watt is a unit of:
\end{question}

\begin{answer}
	\item energy
	\item power
	\item distance
	\item force
\end{answer}

Correct answer: power

Keywords: power, units

\begin{question}
	\item Power is commonly measured in units of:
\end{question}

\begin{answer}
	\item BTU
	\item joule
	\item kilowatt-hour
	\item watt
\end{answer}

Correct answer: watt

Keywords: power, units

\subsection{Watt abbreviation}

\begin{question}
	\item The watt is abbreviated as:
\end{question}

\begin{answer}
	\item wt
	\item Wh
	\item W
	\item tt
\end{answer}

Correct answer: W

Keywords: power, units

\begin{question}
	\item The abbreviation "W" refers to what unit:
\end{question}

\begin{answer}
	\item watt-hour
	\item wind power
	\item wave power
	\item watt
\end{answer}

Correct answer: watt

Keywords: power, units

\subsection{Watt-hour definition}

\begin{question}
	\item The watt-hour is a unit of:
\end{question}

\begin{answer}
	\item energy
	\item power
	\item distance
	\item force
\end{answer}

Correct answer: energy

Keywords: energy, units

\begin{question}
	\item Electrical energy is commonly measured in units of:
\end{question}

\begin{answer}
	\item BTU
	\item joule
	\item watt-hour
	\item watt
\end{answer}

Correct answer: watt-hour

Keywords: energy, units

\subsection{Watt-hour abbreviation}

\begin{question}
	\item The watt-hour is abbreviated as:
\end{question}

\begin{answer}
	\item Wh
	\item wth
	\item W
	\item erg
\end{answer}

Correct answer: Wh

Keywords: energy, units

\begin{question}
	\item The abbreviation "Wh" refers to what unit:
\end{question}

\begin{answer}
	\item watt
	\item wind-hour
	\item watt-hour
	\item power
\end{answer}

Correct answer: Wh

Keywords: energy, units

\subsection{Power/energy calculations}

\begin{question}
	\item A compact fluorescent lightbulb uses 13 W. If it is run for 2 hours, how much energy does it use?
\end{question}

\begin{answer}
	\item 7.5 Wh
	\item 13 Wh
	\item 26 Wh
	\item 52 Wh
\end{answer}

Correct answer: 26 Wh

Keywords: power, energy, unit-intuition, calculation

\begin{question}
	\item A compact fluorescent lightbulb (CFL) used 26 Wh after running for 2 hours. How much power does the bulb consume?
\end{question}

\begin{answer}
	\item 7.5 W
	\item 13 W
	\item 26 W
	\item 52 W
\end{answer}

Correct answer: 13 W

Keywords: power, energy, unit-intuition, calculation

\begin{question}
	\item If your game console uses 200 W when turned on, how much energy would it waste if you left it on all weekend while you were away?
\end{question}

\begin{answer}
	\item 15000 Wh
	\item 100 Wh
	\item 960 kWh
	\item 9.6 kWh
\end{answer}

Correct answer: 9.6 kWh

Keywords: power, energy, unit-intuition, calculation

\begin{question}
	\item While reading your electric bill you notice that you used 72 kWh more than the previous month. You search your apartment for anything out of the ordinary and find you left a fan running in a closet all month long! Approximately how much power does the fan use?
\end{question}

\begin{answer}
	\item 100 W
	\item 10 W
	\item 300 kWh
	\item 1 kWh
\end{answer}

Correct answer: 100 W

Keywords: power, energy, unit-intuition, calculation


\section{Energy Intuition}

\subsection{Consumption intuition}

\begin{question}
	\item Roughly how much power does a normal compact fluorescent lightbulb (CFL) use when running?
\end{question}

\begin{answer}
	\item 20 mW
	\item 3 W
	\item 60 W
	\item 13 W
\end{answer}

Correct answer: 13 W

Keywords: power, unit-intuition

\begin{question}
	\item Roughly how much power does an electric oven use when turned to its highest setting?
\end{question}

\begin{answer}
	\item 100 W
	\item 500 W
	\item 1 kW
	\item 2.5 kW
\end{answer}

Correct answer: 2.5 kW

Keywords: power, unit-intuition

\begin{question}
	\item On average, how much electrical energy does a home in Hawaii use per day?
\end{question}

\begin{answer}
	\item 13 kWh
	\item 4 kWh
	\item 57 kWh
	\item 328 kWh
\end{answer}

Correct answer: 13 kWh

Keywords: energy, unit-intuition, Hawaii 

\begin{question}
	\item On average, how much electrical energy does a home in Hawaii use per month?
\end{question}

\begin{answer}
	\item 37 kWh
	\item 104 kWh
	\item 390 kWh
	\item 2000 kWh
\end{answer}

Correct answer: 390 kWh

Keywords: energy, unit-intuition, Hawaii 

\subsection{Solar intuition}

\begin{question}
	\item What is the approximate maximum power generated from a single standard rooftop solar panel?
\end{question}

\begin{answer}
	\item 25 W
	\item 50 W
	\item 200 W
	\item 800 W
\end{answer}

Correct answer: 200 W

Keywords: power, unit-intuition, generation, PV

\begin{question}
	\item Approximately how much energy does single standard rooftop solar panel in Hawaii generate each day?
\end{question}

\begin{answer}
	\item 100 Wh
	\item 1000 Wh
	\item 10 kWh
	\item 100 W
\end{answer}

Correct answer: 1000 Wh

Keywords: energy, unit-intuition, generation, PV


\section{Grid knowledge}

\subsection{Generation}

\begin{question}
	\item What is the source of approximately 80\% of Hawaii's electricity?
\end{question}

\begin{answer}
	\item coal
	\item wind
	\item solar
	\item oil
\end{answer}

Correct answer: oil

Keywords: generation, utility, Hawaii

\begin{question}
	\item Burning oil is used to generate approximately what percentage of Hawaii's electricity?
\end{question}

\begin{answer}
	\item 100\%
	\item 50\%
	\item 78\%
	\item 17.5\%
\end{answer}

Correct answer: 78\%

Keywords: generation, utility, Hawaii

\subsection{Demand}

\begin{question}
	\item What is the approximate maximum electrical power demand for the entire island of Oahu?
\end{question}

\begin{answer}
	\item 560 kW
	\item 3800 kW
	\item 11.8 GW
	\item 1.2 GW
\end{answer}

Correct answer: 1.2 GW

Keywords: power, unit-intuition, generation, utility, Hawaii

[Need another question on demand, maybe demand for entire state??]

\begin{question}
	\item What is the electrical grid demand curve?
\end{question}

\begin{answer}
	\item A graph of the amount of power used on the grid over time
	\item The number of efficient appliances demanded by consumers
	\item A graph of the amount of energy used on the grid over time
	\item The amount overhead power lines can bend before breaking
\end{answer}

Correct answer: A graph of the amount of power used on the grid over time

Keywords: power, generation, utility

\begin{question}
	\item Why is the shape of the electrical grid demand curve important?
\end{question}

\begin{answer}
	\item Less efficient power plants must be used if there are peaks in the curve 
	\item A flat curve means nobody is using any electricity
	\item The shape shows how many power plants are running
	\item The curve is lower at night if there is a lot of solar power in the grid
\end{answer}

Correct answer: Less efficient power plants must be used if there are peaks in the curve

Keywords: power, generation, utility, Hawaii

\subsection{Hawaii Clean Energy Initiative}

\begin{question}
	\item What is the goal of the Hawaii Clean Energy Initiative?
\end{question}

\begin{answer}
	\item Maintain Hawaii's energy use at current levels forever
	\item Decrease Hawaii's oil use by 20\% by 2020
	\item Get 70\% of Hawaii's energy from clean sources by 2030
	\item Get 50\% of Hawaii's energy from wind by 2050
\end{answer}

Correct answer: Get 70\% of Hawaii's energy from clean sources by 2030

Keywords: energy, generation, utility, Hawaii

\begin{question}
	\item What is the breakdown of the clean energy mandated by the Hawaii Clean Energy Initiative?
\end{question}

\begin{answer}
	\item 50\% from renewable sources, 10\% from conservation
	\item 30\% from solar, 30\% from wind, 10\% from waves
	\item 30\% from renewable sources, 20\% from conservation, 10\% from natural gas
	\item 30\% from energy conservation, 40\% from renewable sources
\end{answer}

Correct answer: 30\% from energy conservation, 40\% from renewable sources

Keywords: energy, generation, conservation, utility, Hawaii

\section{Climate change}

\begin{question}
	\item What are the effects of climate change?
\end{question}

\begin{answer}
	\item Global temperatures increasing by a few degrees on average
	\item Changes in seasonal rainfall patterns (droughts, floods)
	\item A significant rise in the sea level
	\item All of the above
\end{answer}

Correct answer: All of the above

Keywords: climate change

\begin{question}
	\item What is the primary cause of climate change?
\end{question}

\begin{answer}
	\item Melting glaciers in Greenland
	\item Carbon dioxide released from burning fossil fuels
	\item Natural solar cycles
	\item Radioactive waste from nuclear power plants
\end{answer}

Correct answer: Carbon dioxide released from burning fossil fuels

Keywords: climate change

\begin{question}
	\item Approximately how much rise in sea level is predicted by the end of the century?
\end{question}

\begin{answer}
	\item 2 inches
	\item 6 inches
	\item 1 foot
	\item 3 feet
\end{answer}

Correct answer: 3 feet

Keywords: climate change

\begin{question}
	\item Approximately how much carbon dioxide is in the atmosphere now, and what level is considered safe/acceptable?
\end{question}

\begin{answer}
	\item 450 ppm, 500 ppm
	\item 387 ppm, 350 ppm
	\item 331 ppm, 350 ppm
	\item 600 ppm, 450 ppm
\end{answer}

Correct answer: 387 ppm, 350 ppm

Keywords: climate change


%%%%%%%%%%%%%%%%%%%%%%%%%%%%%% -*- Mode: Latex -*- %%%%%%%%%%%%%%%%%%%%%%%%%%%%
%% uhtest-appendix.tex -- 
%% Author          : Robert Brewer
%% Created On      : Fri Oct  2 16:31:12 1998
%% Last Modified By: Robert Brewer
%% Last Modified On: Mon Oct  5 14:41:05 1998
%% RCS: $Id: uhtest-appendix.tex,v 1.1 1998/10/06 02:07:03 rbrewer Exp $
%%%%%%%%%%%%%%%%%%%%%%%%%%%%%%%%%%%%%%%%%%%%%%%%%%%%%%%%%%%%%%%%%%%%%%%%%%%%%%%
%%   Copyright (C) 1998 Robert Brewer
%%%%%%%%%%%%%%%%%%%%%%%%%%%%%%%%%%%%%%%%%%%%%%%%%%%%%%%%%%%%%%%%%%%%%%%%%%%%%%%
%% 

\chapter{Post-Competition Qualitative Feedback Questions}
\label{app:qualitative-feedback}

This appendix lists the questions that assess participants' experiences with the competition and website. The questions are separated into sections based on the topic they are addressing.

\section{Tasks effectiveness at improving energy literacy}

\section{Relevancy of website}

\section{Opinion of floor-level near-realtime meter data}

%% Just for demo purposes, include all entries from bib file
%\nocite{*}

%%% Input file for bibliography
\bibliography{sustainability}
%% Use this for an alphabetically organized bibliography
\bibliographystyle{plain}
%% Use this for a reference order organized bibliography
%\bibliographystyle{unsrt}
%% Try using this BibTeX style that hopefully will print annotations in
%% the bibliography. This will allow me to make notes on papers in the
%% BibTeX file and have them readable in the references section until
%% I turn them into a conceptual literature review 
%\bibliographystyle{annotation}

%\printglossaries
%\addcontentsline{toc}{chapter}{Glossary}

%% Increments the page counter so the ToC entry will point at the right page
\cleardoublepage
%% Add the index to the ToC
\addcontentsline{toc}{chapter}{Index}
%% Print the actual index entries
\printindex

\end{document}
