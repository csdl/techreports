
\begin{abstract}
In software engineering, the importance of measurement is well understood, and many efficient software development metrics have been developed to help measurement. However, as the number of metrics increase, the effort required to collect data, analyze them and interpret analysis results quickly becomes overwhelming. This problem is even more critical in educational approaches regarding empirical software engineering.

The Software Intensive Care Unit is a new approach to facilitate software measurement and control with multiple software development metrics. It uses the Hackystat system to achieve automated data collection and analysis, then uses the collected analysis data to create an intensive monitoring interface for multiple ``vital signs''. A vital sign is a wrapper of a software metric with proper presentation. It consists of a historical trend and a newest state value, both of which is colored according to its ``health'' state. I hypothesize that the Software ICU is helpful to study empirical software engineering.

My research deployed and evaluated Software ICU in a senior-level software engineering course. Students' usage was logged in the system, and a survey was conducted. The results provide supporting evidence that Software ICU does help students in course project development and project team organization. In addition, the results of the study also discover some limitations of the system, including inappropriate vital sign presentation and measurement dysfunction.

\end{abstract}
