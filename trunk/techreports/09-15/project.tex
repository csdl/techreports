%%%%%%%%%%%%%%%%%%%%%%%%%%%%%% -*- Mode: Latex -*- %%%%%%%%%%%%%%%%%%%%%%%%%%%%
%% project.tex -- 
%% Author          : Philip Johnson
%% Created On      : Tue Mar 31 11:44:58 2009
%% Last Modified By: Philip Johnson
%% Last Modified On: Fri Nov  6 10:01:22 2009
%% RCS: $Id$
%%%%%%%%%%%%%%%%%%%%%%%%%%%%%%%%%%%%%%%%%%%%%%%%%%%%%%%%%%%%%%%%%%%%%%%%%%%%%%%
%%   Copyright (C) 2009 
%%%%%%%%%%%%%%%%%%%%%%%%%%%%%%%%%%%%%%%%%%%%%%%%%%%%%%%%%%%%%%%%%%%%%%%%%%%%%%%
%% 

\pagenumbering{arabic}
\renewcommand{\thepage} {C--\arabic{page}}

\renewcommand{\thesection} {C.\arabic{section}}
\setcounter{section}{0}

\section{Project Description}

\section{Introduction}

The introduction should explain what the smart grid is, a little of the recent history, and why it is important. The goal of this research is to: (a) generate some scientific data that can be used to inform policy and design decisions; (b) generate some technological infrastructure that can support future experimentation; (c) generate some methods for inquiry; (d) build a community of scientific researchers that can collaborate together.

We also want to make the point that Hawaii has some unique attributes for this kind of experimentation: it has a wealth of renewable energy sources including solar, wind, wave, and geothermal.  In addition, it is currently 90\% dependent on fossil fuels, and these fuels are very expensive, making our electrical costs the most expensive in the nation.  As a result of these factors, renewable energy has the potential to be more cost-effective in Hawaii.  Also, the grid is closed, making it easier to model and analyze. 

We also want to demonstrate alignment with the goals of the program.

``HCC research explores and improves our understanding of new
human-computer and human-human interactions, collaboration, andcompetition, developing systems that are aware of their social surroundings
and of the conceptualizations, values, preferences, abilities, special
needs, and diverse ranges of capability of the people that use them.''

Introduce thesis of research: what are the motivators of energy behavior;
what kinds of technological support facilitates long term change; are
people only motivated by cost savings; are there unintended consequences;




\section{Related work}

Microsoft's SERA (Smart Energy Reference Architecture) limits its
discussion of user-centered control to a single line: ``Customers will also
need to have capabilities for more local preferences and control.''
(p. 69)

Two technologies are being proposed for home area networks (HAN), including ZigBee and IP for Smart Objects (IPSO).
ZigBee allows for secure wireless communication between devices, where specific devices may provide specific capabilities useful for energy control and conservation.
IPSO is focused on the use of IP for connections and communication between Smart Objects.


Microsoft Hohm is a consumer-facing site.  Users fill out a profile of
their home, and if the utility is connected, they can get their energy data
automatically imported.  The site responds with recommendations on how you
can conserve energy. Does comparisons of your use with others using Hohm
and
published averages.   Goals: make it easy to enter information, and make
the recommendations very specific.  So, the recommendation to raise your
thermostat setting by 2 degrees will indicate exactly how much money you
will save and the carbon impact.


\section{Proposed contributions}

Data on consumer behaviors; open source technologies; experimental designs;
community building (?).


 










