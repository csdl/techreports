%%%%%%%%%%%%%%%%%%%%%%%%%%%%%% -*- Mode: Latex -*- %%%%%%%%%%%%%%%%%%%%%%%%%%%%
%% project.tex -- 
%% Author          : Philip Johnson
%% Created On      : Tue Mar 31 11:44:58 2009
%% Last Modified By: Philip Johnson
%% Last Modified On: Mon Nov  9 16:58:56 2009
%% RCS: $Id$
%%%%%%%%%%%%%%%%%%%%%%%%%%%%%%%%%%%%%%%%%%%%%%%%%%%%%%%%%%%%%%%%%%%%%%%%%%%%%%%
%%   Copyright (C) 2009 
%%%%%%%%%%%%%%%%%%%%%%%%%%%%%%%%%%%%%%%%%%%%%%%%%%%%%%%%%%%%%%%%%%%%%%%%%%%%%%%
%% 

\pagenumbering{arabic}
\renewcommand{\thepage} {C--\arabic{page}}

\renewcommand{\thesection} {C.\arabic{section}}
\setcounter{section}{0}

\section{Project Description}

\section{Introduction}

Development of the ``smart grid'', a modernized power infrastructure, is
one of the key technological challenges facing the United States at the
dawn of the 21st century. According to the Department of Energy, the smart
grid should: (1) Enable active participation by consumers by providing
choices and incentives to modify electricity purchasing patterns and
behavior; (2) Accommodate all generation and storage options, including
wind and solar power.  (3) Enable new products, services, and markets
through a flexible market providing cost- benefit tradeoffs to consumers
and market participants; (4) Provide reliable power that is relatively
interruption-free; (5) Optimize asset utilization and maximizes operational
efficiency; (6) Provide the ability to self-heal by anticipating and
responding to system disturbances; (7) Resist attacks on physical
infrastructure by natural disasters and attacks on cyber-structure by
malware and hackers.

In October, 2009, approximately \$3.4 billion dollars in federal stimulus
money was awarded to approximately 100 organizations in 49 states to
support smart grid development.  These awards, which were matched by \$4.7
billion dollars in private funds, are being used primarily for the
installation of smart meters, the first building block of a smart grid.
Other uses of the stimulus money include installation of a secure
communications network (Wyoming); installation of phasor measurement units
for monitoring grid stability (New York, Massachussetts, Indiana,
Louisiana, and others); and introduction of dynamic pricing (Maryland,
Arizona, California, and others).

While this stimulus package will inject over \$8 billion dollars into the
smart grid, the money is almost totally devoted to low-level infrastructure
investment.  By analogy to the Internet, it is an investment similar in
nature to upgrading copper wire to fiber optic cable, and the installation
of high performance routers and name servers.  Such infrastructure is a
necessary precursor to the kinds of information transfer and communications
mechanisms made possible by the World Wide Web.

It is clear that the smart grid will carry information about electricity as
well as electricity itself, thus forming its own kind of electrical
Internet. Interestingly, it is not at all clear what ecosystem of
higher-level services (similar to World Wide Web applications) will be
developed to communicate, analyze and interpret this low-level electrical
data.

One reason why there is so little clarity about how electrical information
can or should be communicated in the upcoming Smart Grid is due to the
nature of the traditional grid, which (from a consumer point of view) is a
classic ``black box'' technology.  For almost 100 years, consumers have
plugged appliances into the grid via electrical outlets and expected them
to ``just work''.  Put another way, the U.S. power industry has operated
for a century under the assumption that consumers should have access to a
virtually unlimited amount of high quality, stable, power.  Traditionally,
consumers have been given extremely little information about power because
they were not expected to need to care.  Previous feedback to consumers
about electrical usage came in the form of monthly bills, which even then
might be estimates if a meter reader had not inspected the meter during the
billing period.

Electricity as black box requires electricity to be cheap, reliable, and
unlimited, but those days are almost certainly over for a number of
reasons.  First, the traditional use of fossil fuels as a primary source of
power generation is not sustainable: most economists agree that ``peak
oil'' has either already occurred or will occur in the next two decades. At
that point, oil supply will decrease and prices will increase.  Second,
fossil fuels generate green house gases that contribute to climate change,
and so there is an urgent need to move to renewable energy sources such as
solar, wind, wave, and geothermal.  Integration of renewable energy
sources, unfortunately, creates significant new problems for grid
stability.  Unlike fossil-fuel based power generation which is termed
``firm'' since utility companies can generate new power from fossil fuels
at will, most renewable energy sources generate energy depending upon
generally unpredictable environmental factors.  Simply adding a solar panel
to every rooftop and tying them into the grid would actually do more harm
than good, as grid stability depends upon energy supply equaling demand on
a second to second basis, and U.S. utilities currently do not have a way to
monitor and balance a grid that incorporates such widespread, distributed
generation.

If electricity cannot any longer be a black box to consumers, then an
important question is: what kind of ``white box'' will it become? In other
words, what kind of access to energy information will the smart grid make
available to consumers? In this proposal, we put forth a set of
technological developments and associated experiments to gather scientific,
replicable data about the ways in which access to information about energy
usage impacts on consumer behavior.  Our experimental method includes both
the study of current energy monitoring technology, as well as simulations
to understand the potential impact of future smart grid information access
technologies.  The intended contributions of this research include insights
from the data that can effect policy for the smart grid, as well as
methodological and technological innovations that can be exploited by other
researchers to provide additional insight into information access on the
smart grid.




























Hawaii has unique attributes that make it interesting for smart grid
research: it has a wealth of renewable energy sources including solar,
wind, wave, and geothermal.  In addition, it is currently 90\% dependent on
fossil fuels, and these fuels are very expensive, making our electrical
costs the most expensive in the nation.  As a result of these factors,
renewable energy has the potential to be more cost-effective in Hawaii.
Also, the grid is closed, making it easier to model and analyze.


Many of the current smart grid initiatives suffer from either the ``top
down control'' or the ``consumers respond only to cost'' assumptions. 

The top down control assumption is present in initiatives that introduce
``smartness'' into the grid by enabling utilities to directly control
end-use of power, for example by turning off hot water heaters or adjusting
thermostats.   Top down control can be effective in certain situations, but
the amount of consumption management that users will accede to the
utilities is limited.  In addition, top down control creates privacy
issues.  Finally, the degree of management possible from a utility-centered
approach cannot achieve what is possible by directly involving consumers. 

The ``consumers respond only to cost'' assumption ignores the many other
motivators for energy conservation, such as environmental concerns.  

The central question to be investigated in this
research is the following: {\em What kinds of information, provided in what ways and at what
times, enables consumers to make positive, sustained changes to their
energy consumption behaviors?}


Metaphor of prius:  we can have top-down control (via minimum auto mpg
requirements), but the prius realized that the potential of its technology
could only be achieved by providing drivers with higher quality, higher
fidelity information about the ways the driver behaviors impact on mpg.  






The expected contributions of this research include: (a) outcome data
regarding consumer behavior and technology that can be used to inform
policy and design decisions; (b) technological infrastructure that can
support future experimentation; (c) replicable experimental methods for
investigating consumer behavior with the smart grid; (d) a
cross-disciplinary community of scientific researchers; 


We also want to demonstrate alignment with the goals of the program.

``HCC research explores and improves our understanding of new
human-computer and human-human interactions, collaboration, andcompetition, developing systems that are aware of their social surroundings
and of the conceptualizations, values, preferences, abilities, special
needs, and diverse ranges of capability of the people that use them.''

Introduce thesis of research: what are the motivators of energy behavior;
what kinds of technological support facilitates long term change; are
people only motivated by cost savings; are there unintended consequences;




\section{Related work}

Microsoft's SERA (Smart Energy Reference Architecture) limits its
discussion of user-centered control to a single line: ``Customers will also
need to have capabilities for more local preferences and control.''
(p. 69)

Two technologies are being proposed for home area networks (HAN), including ZigBee and IP for Smart Objects (IPSO).
ZigBee allows for secure wireless communication between devices, where specific devices may provide specific capabilities useful for energy control and conservation.
IPSO is focused on the use of IP for connections and communication between Smart Objects.


Microsoft Hohm is a consumer-facing site.  Users fill out a profile of
their home, and if the utility is connected, they can get their energy data
automatically imported.  The site responds with recommendations on how you
can conserve energy. Does comparisons of your use with others using Hohm
and
published averages.   Goals: make it easy to enter information, and make
the recommendations very specific.  So, the recommendation to raise your
thermostat setting by 2 degrees will indicate exactly how much money you
will save and the carbon impact.


\section{Proposed contributions}

Data on consumer behaviors; open source technologies; experimental designs;
community building (?).


 










