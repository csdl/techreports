%%%%%%%%%%%%%%%%%%%%%%%%%%%%%% -*- Mode: Latex -*- %%%%%%%%%%%%%%%%%%%%%%%%%%%%
%% project.tex -- 
%% Author          : Philip Johnson
%% Created On      : Tue Mar 31 11:44:58 2009
%% Last Modified By: Philip Johnson
%% Last Modified On: Mon Nov  9 07:01:39 2009
%% RCS: $Id$
%%%%%%%%%%%%%%%%%%%%%%%%%%%%%%%%%%%%%%%%%%%%%%%%%%%%%%%%%%%%%%%%%%%%%%%%%%%%%%%
%%   Copyright (C) 2009 
%%%%%%%%%%%%%%%%%%%%%%%%%%%%%%%%%%%%%%%%%%%%%%%%%%%%%%%%%%%%%%%%%%%%%%%%%%%%%%%
%% 

\pagenumbering{arabic}
\renewcommand{\thepage} {C--\arabic{page}}

\renewcommand{\thesection} {C.\arabic{section}}
\setcounter{section}{0}

\section{Project Description}

\section{Introduction}

Development of the ``smart grid'', a modernized power infrastructure, is
one of the key technological challenges facing the United States at the
dawn of the 21st century. According to the Department of Energy, the smart
grid will include seven characteristics: (1) Enables active participation
by consumers by providing choices and incentives to modify electricity
purchasing patterns and behavior; (2) Accommodates all generation and
storage options, including wind and solar power.  (3) Enables new products,
services, and markets through a flexible market providing cost- benefit
tradeoffs to consumers and market participants; (4) Provides reliable power
that is relatively interruption-free; (5) Optimizes asset utilization and
maximizes operational efficiency; (6)  Has the ability to self-heal by 
anticipating and responding to system disturbances; (7) Resists attacks on
physical infrastructure by natural disasters and attacks on cyber-
structure by malware and hackers.


Hawaii has unique attributes that make it interesting for smart grid
research: it has a wealth of renewable energy sources including solar,
wind, wave, and geothermal.  In addition, it is currently 90\% dependent on
fossil fuels, and these fuels are very expensive, making our electrical
costs the most expensive in the nation.  As a result of these factors,
renewable energy has the potential to be more cost-effective in Hawaii.
Also, the grid is closed, making it easier to model and analyze.


Many of the current smart grid initiatives suffer from either the ``top
down control'' or the ``consumers respond only to cost'' assumptions. 

The top down control assumption is present in initiatives that introduce
``smartness'' into the grid by enabling utilities to directly control
end-use of power, for example by turning off hot water heaters or adjusting
thermostats.   Top down control can be effective in certain situations, but
the amount of consumption management that users will accede to the
utilities is limited.  In addition, top down control creates privacy
issues.  Finally, the degree of management possible from a utility-centered
approach cannot achieve what is possible by directly involving consumers. 

The ``consumers respond only to cost'' assumption ignores the many other
motivators for energy conservation, such as environmental concerns.  

The central question to be investigated in this
research is the following: {\em What kinds of information, provided in what ways and at what
times, enables consumers to make positive, sustained changes to their
energy consumption behaviors?}


Metaphor of prius:  we can have top-down control (via minimum auto mpg
requirements), but the prius realized that the potential of its technology
could only be achieved by providing drivers with higher quality, higher
fidelity information about the ways the driver behaviors impact on mpg.  






The expected contributions of this research include: (a) outcome data
regarding consumer behavior and technology that can be used to inform
policy and design decisions; (b) technological infrastructure that can
support future experimentation; (c) replicable experimental methods for
investigating consumer behavior with the smart grid; (d) a
cross-disciplinary community of scientific researchers; 


We also want to demonstrate alignment with the goals of the program.

``HCC research explores and improves our understanding of new
human-computer and human-human interactions, collaboration, andcompetition, developing systems that are aware of their social surroundings
and of the conceptualizations, values, preferences, abilities, special
needs, and diverse ranges of capability of the people that use them.''

Introduce thesis of research: what are the motivators of energy behavior;
what kinds of technological support facilitates long term change; are
people only motivated by cost savings; are there unintended consequences;




\section{Related work}

Microsoft's SERA (Smart Energy Reference Architecture) limits its
discussion of user-centered control to a single line: ``Customers will also
need to have capabilities for more local preferences and control.''
(p. 69)

Two technologies are being proposed for home area networks (HAN), including ZigBee and IP for Smart Objects (IPSO).
ZigBee allows for secure wireless communication between devices, where specific devices may provide specific capabilities useful for energy control and conservation.
IPSO is focused on the use of IP for connections and communication between Smart Objects.


Microsoft Hohm is a consumer-facing site.  Users fill out a profile of
their home, and if the utility is connected, they can get their energy data
automatically imported.  The site responds with recommendations on how you
can conserve energy. Does comparisons of your use with others using Hohm
and
published averages.   Goals: make it easy to enter information, and make
the recommendations very specific.  So, the recommendation to raise your
thermostat setting by 2 degrees will indicate exactly how much money you
will save and the carbon impact.


\section{Proposed contributions}

Data on consumer behaviors; open source technologies; experimental designs;
community building (?).


 










