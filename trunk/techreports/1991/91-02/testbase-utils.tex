\chapter{Utilities Subsystem Tests}

\section {Class: u*hash}
\subsection {Operation: u*hash*make}
\subsubsection {Test-case scenario: normal, Small-hash-table-size}


This test case tests that a hash table can be created in a normal manner.
It calls the function u*hash*make  with a hash table of size (17), and expects an obarray (hash table) or object of type vector.
Afterwards, its postcondition Creates an empty vector of size 17, then it checks that the resulting table is equal to empty vector.


\noindent {\bf Result: FAILED}\
\begin {itemize}
\item 	Error-name             : unknown-return-value
\item Error-data             : nil
\item Operation and Parameter: equal ([0 0 0 0 0 0 0 0 0 0 0 0 0 0 0 0 0] [nil nil nil nil nil nil nil nil nil nil nil nil nil nil nil nil nil])
\item Expected-return-type   : object
\item Expected-return-value  : t
\item Location               : Postcondition



\end {itemize}
\subsection {Operation: u*hash*get}
\subsubsection {Test-case scenario: normal-get-key-from-hash, Hash-key-is-symbol}


This test case tests that data can be retrieved from a valid hash table given key.
Its precondition creates a hash table of size 29 and returns symbol table, then it stores the string "Data 1" in the hash table with key 'Hash-key, then it stores string "Data 2" in the hash table with the key 188.
Then it calls the function u*hash*get  with the hash key symbol 'Hash-key, and expects a return value of integer or symbol.





\noindent {\bf Result: PASSED}\
\subsubsection {Test-case scenario: normal-get-key-from-hash, Hash-key-is-integer}


This test case tests that data can be retrieved from a valid hash table given key.
Its precondition creates a hash table of size 29 and returns symbol table, then it stores the string "Data 1" in the hash table with key 'Hash-key, then it stores string "Data 2" in the hash table with the key 188.
Then it calls the function u*hash*get  with the hash key integer 188, and expects a return value of integer or symbol.





\noindent {\bf Result: PASSED}\
\subsubsection {Test-case scenario: data-not-found-in-hash-table, Non-extant-hash-key}


This test case tests that non-extant hash-key retrieves nil data.
Its precondition creates a hash table of size 29 and returns symbol table, then it stores the string "Data 1" in the hash table with key 'Hash-key, then it stores string "Data 2" in the hash table with the key 188.
Then it calls the function u*hash*get  with non-extant symbol 'Blah as the hash-key, and expects nil.





\noindent {\bf Result: PASSED}\
\subsection {Operation: u*hash*set}
\subsubsection {Test-case scenario: update-hash-data, Same-hash-key-different-data}


This test case tests that the operation u*hash*set with the same hash-key but different data will update the data.
Its precondition creates a hash table of size 29 and returns symbol table, then it stores the string "Data 1" in the hash table with key 'Hash-key.
Then it calls the function u*hash*set  with the hash key is the symbol 'Hash-key and the data is a lisp object "new-data", and expects a return value of "new-data".
Afterwards, its postcondition tests that the key 'Hash-key and corresponding data "new-data" exists in the table.


\noindent {\bf Result: FAILED}\
\begin {itemize}
\item 	Error-name             : system-abort
\item Error-data             : (void-function u*hash*existp)
\item Operation and Parameter: u*hash*existp ((quote Hash-key) table-variable)
\item Expected-return-type   : object
\item Expected-return-value  : new-data
\item Location               : Postcondition



\end {itemize}
\subsection {Operation: u*hash*rem}
\subsubsection {Test-case scenario: removes-existing-data-from-hash-table, Existing-hash-key}


This test case tests that data can be removed successfully from hash table.
Its precondition creates a hash table of size 29 and returns symbol table, then it stores the string "Data 1" in the hash table with key 'Hash-key.
Then it calls the function u*hash*rem  with existing symbol 'Hash-key, and expects t since we're removing an existing data.
Afterwards, its postcondition tests that 'Hash-key no longer exists in the hash table.


\noindent {\bf Result: FAILED}\
\begin {itemize}
\item 	Error-name             : system-abort
\item Error-data             : (void-function u*hash*existp)
\item Operation and Parameter: u*hash*existp ((quote Hash-key) table-variable)
\item Expected-return-type   : object
\item Expected-return-value  : nil
\item Location               : Postcondition



\end {itemize}
\subsubsection {Test-case scenario: removes-non-extant-data-from-hash-table, Non-extant-hash-key}


This test case tests that u*hash*rem returns nil when non-extant-key is supplied.
Its precondition creates a hash table of size 29 and returns symbol table, then it stores the string "Data 1" in the hash table with key 'Hash-key.
Then it calls the function u*hash*rem  with non-extant symbol 'Blah as the hash-key, and expects nil since no such key exist.


\noindent {\bf Result: PASSED}\

