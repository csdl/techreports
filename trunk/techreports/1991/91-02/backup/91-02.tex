%%%%%%%%%%%%%%%%%%%%%%%%%%%%%% -*- Mode: Latex -*- %%%%%%%%%%%%%%%%%%%%%%%%%%%%
%% 91-02.tex -- 
%% Author          : Philip M Johnson
%% Created On      : Thu Nov  7 09:52:47 1991
%% Last Modified By: Philip Johnson
%% Last Modified On: Tue Jun 30 12:18:53 1992
%%%%%%%%%%%%%%%%%%%%%%%%%%%%%%%%%%%%%%%%%%%%%%%%%%%%%%%%%%%%%%%%%%%%%%%%%%%%%%%

\documentstyle [/home/13/csdl/tex/definemargins,
                /home/13/csdl/tex/description,
                     /home/13/csdl/tex/named-citations,
                     /home/13/csdl/tex/psfig,
                     /home/13/csdl/tex/lmacros]{report}  
\begin{document}

\newcommand{\STAR}{\mbox{$\ast$}}  %% the public delimiter symbol

\vspace*{1in}
\begin{center}
  
{\Huge\bf Egret 2.0 Design Specification}\foot{Support for this
research was provided in part by the National Science Foundation
Research Initiation Award CCR-9110861 and the University of Hawaii
Research Council Seed Money Award R-91-867-F-728-B-270.}
  
\bigskip\par

Collaborative Software Development Laboratory\\
Department of Information and Computer Sciences\\
University of Hawaii\\
Honolulu, HI 96822\\
(808) 956-3489\\
{\tt johnson@uhics.ics.hawaii.edu}                  \medskip\par

{\bf CSDL-TR-91-02}                                 \medskip\par

Last Revised: \today                                \bigskip\par
\end{center}

\newpage
\tableofcontents         
\newpage

\ls{1.2}
\part{Design Background and Requirements}
\chapter{Overview}
\section{Introduction}

This document is one component of the architectural and functional
specification of Egret 2.0. It is a companion and introduction to the
more detailed functional specification of the system maintained in two
on-line collaborative databases, the DesignBase and the TestBase.
Recent snapshots of these on-line databases appear in the Appendix to
this document, though the on-line versions are considered the
authorative source for specification information.

Egret 2.0 is a framework for implementing exploratory collaborative
hypertext applications \cite{csdl-91-03,csdl-92-01}.  Egret provides
mechanisms for the integration of an underlying database server
application process, and an intermediate, highly expressive
exploratory type system, with a top-level editor and graphical display
applications.  In doing so, it provides a language and environment for
implementation of domain-specific collaborative systems, such as for
research review \cite{csdl-92-03,csdl-92-05}, or code inspection
\cite{csdl-92-04}.  This document contains little description of the
research concepts and motivations for Egret 2.0. For this form of
broad overview, we point the reader to any of the above references,
although \cite{csdl-92-01} and \cite{csdl-92-03} are particularly good
starting points.

The next section of this document discusses architecture-level
requirements for Egret 2.0.  The following two sections provide
important background information about our notation and view of object
orientation necessary to understanding the remainder of this document
as well as the on-line class specifications.  Following these sections
is an overview of the four major subsystems in Egret 2.0.  Appendices
contain supplemental information about the system and some of its
differences with respect to CoReView 1.2.

\section{Architectural Requirements}

This section briefly overviews some of the important requirements for 
Egret 2.0 at the architectural level.  

\subsection{Extensibility}  

Egret 2.0 has a defined architecture, including specified
responsibilities and a documented public interface for each primary
subsystem.  Each of these subsystems is internally organized as a set
of classes with public and private interfaces.  Egret 2.0 also defines
naming conventions for program entities that simplify the
identification and understanding of their behavior.

One goal of this architectural structure is to facilitate migration of
system functionality across different application processes.  For
example, certain database functions that are now implemented in Emacs
might migrate to the database server, if we find a better server
platform.  Our current choices for editor, window interface, or
graphical browser may change over time.  A second goal is to
facilitate the the implementation of new applications, such as
mechanisms for sophisticated impact analysis.  Clear specification of
components and interfaces is required to meet these goals.

\subsection{Verification} 

A design goal for Egret 2.0 is completely automated regression
testing: a suite of tests that can be run on the system to test
whether or not major functionality is present and correct after making
changes.  While setting up these tests will involve significant
additional effort, the benefits will more than offset this investment.

The design of Egret 2.0 supports this goal by providing a layered
architecture as well as a strict separation between the Interface
subsystem, where user interaction with the system occurs, and other
subsystems, where the structure of the collaborative network is
physically manipulated.  It should be possible to test most of Egret
2.0 by ``simulating'' the Interface subsystem interactions
programmatically.  The layered architecture supports incremental
integration and well as unit testing.

Regression testing is supported through the TestBase, an system for
declaratively specifying test cases and expected results.  These tests
can be run in background on a release of Egret and the results sent to
a log file.

\subsection{Information Replication and Recoverability}

A fundamental architectural feature of Egret 2.0 is local replication,
or caching, of globally maintained information.  Local caching
improves performance but requires mechanisms to maintain consistency.
Egret 2.0 addresses this problem by {\em recovery operations}. 

Recovery operations are required from all classes that create or
manipulate replicated information.  These operations ensure that that
internal data structures corrupted by hardware or software crashes can
be rebuilt.  


\subsection{Support for Exploratory Groupwork}

Support for exploratory groupwork in Egret 2.0 centers on its
mechanism for defining structure, called the Exploratory
Collaboration Type System (ECTS), and a process model called EXCON, for 
Exploration-Consolidation Cycle.

The EXCON model views exploratory collaboration as cycling between two
modes: exploration and consolidation.  During exploration,
collaborators are discovering new structural features of the domain
and reifying these features into the structure of the artifact under
construction.  During consolidation, collaborators compare and
contrast the structural features they have individually discovered
during exploration, and build consensual structures that embody the
mutually agreed upon features.

ECTS is designed to support these features of the
exploration-consolidation cycle.  In ECTS, collaborators can define
typed classes of nodes and links, and specify their internal
structure.  Once defined, classes form templates from which
collaborators create instances with common structure and behavior.

While this aspect of ECTS is completely similar to conventional
object-oriented database schema mechanisms, ECTS departs from this
model in a singular way: while classes can be used to create instances
with shared structure and behavior, instances are not constrained
to the structure and behavior as defined by their parent class.
Collaborators are free to modify instances by adding to, deleting, or
modifying the properties of its internal structure.

To support this process, ECTS also provides a structural variance tool
that indicates, for a specific node or link class, the degree to which
its instances depart from the class-level structure, and summarizes
how that variance is expressed in terms of added fields, deleted
fields, and changes to the properties of the class-level defined
fields.  This information can be aggregated over the entire database
to determine an overall measure of variance for the database.

Beyond this static snapshot of structural uniformity, the variance
tool also maintains historical information about this information.  In
order to provide a useful representation of the design history, Egret
2.0 provides the concept of layers.

\section{Notational Conventions}

\subsection{Identifier Syntax}

Function and variable names in Egret 2.0 have a standard format.
Each name must adhere to the following template:

\small\begin{verbatim}
<subsys-name><subsys-vis><class-name><class-vis><op-or-att-name>
\end{verbatim}\normalsize

\noindent where:
\begin{itemize}
  
\item {\tt <subsys-name>} is a single character that identifies the
  subsystem membership of the object (currently {\bf u} for the
  utilities subsystem, {\bf i} for the interface subsystem, {\bf t}
  for the type subsystem, and {\bf s} for the server subsystem.
  
\item {\tt <class-name>} is either the full class name or a nickname.
  
\item {\tt <subsys-vis>} and {\tt <class-vis>} are single characters
  indicating the external visibility of the object. The character
  \STAR\ indicates that the object is public, the character {\bf !}
  indicates that the object is private, and the character {\bf @}
  indicates that the object is a system administration function to be
  manipulated only by distinguished users at special times (such as
  database initialization, recovery, and so forth).  

  
\item {\tt <op-or-att-name>} is the actual name of the operation or
  attribute.
\end{itemize}

For example, the function {\bf i\STAR cmd\STAR find-node} is the
interface subsystem public operation {\sf find-node} from the class
{\sf command}, which has the nickname {\bf cmd}.  Nicknames are used
to shorten the total length of function names.  By prefixing
operations with their class name, the operation's class membership is
automatically documented, as well as allowing the same operation name
to be defined for different classes (for example, {\bf node\STAR
find-node}.)

In this document, boldface font is used when referring to a specific Egret 
operation or attribute, such as {\bf s\STAR hbserver!connect}.  Sans serif 
font is used when referring to class names or operation names in general, 
such as {\sf command} or {\sf delete}.  The typewriter font is used when
referring to Emacs or Epoch functions, such as {\tt title}, or when 
displaying sections of code that may include Egret and Emacs functions, such
as:
\small\begin{verbatim}
(foobar (n!screen*name scrn) "New Screen")
\end{verbatim}\normalsize


\section{Object Orientation in Egret 2.0}

The Egret 2.0 design is object-oriented.  However, ``object oriented''
means very different things to different people.  This section
clarifies our use of object orientation by describing the major
entities in our design: classes, objects, attributes, operations, and
collections.  It also describes how public and private interfaces for
class attributes and operations are defined\foot{The mechanisms for 
inheritance and visibility control do not attempt to extend the state
of the art: inheritance is implemented ``manually'' and does not 
provide for multiple inheritance as in CLOS, and visibility control is not as 
sophisticated as, for example, the Precise Interface Control preprocessor
for Ada. The goal here is to introduce a set of concepts for object
orientation and visibility that provide the greatest benefit for the 
lowest overhead.}.


\subsection{Classes}  

A {\em class}\/ is a collection of objects with related structure and
behavior.  {\sf Node} and {\sf link} are obvious classes in Egret 2.0.
Each class has associated with it a set of superclasses, a
set of subclasses, a set of attributes, and a set of operations.  The
design of Egret 2.0 is  neutral on the issue of single vs. 
multiple inheritance, although no occurrences of multiple inheritance
in the design currently exist.

Classes frequently have constructor and destructor operations for the objects
associated with them.  These operations are always called {\sf make}
and {\sf delete}. If these operations are not specified, then the class is an 
{\em abstract class}. Abstract classes exist to collect together related
structure that is inherited by other classes that do have constructor
and destructor operations.  For example, the {\sf button} class is an
abstract class that is inherited by the {\sf field-label} and {\sf
link-label} classes. No instances of buttons exist directly---only
field-labels and link-labels.

Classes are, in some sense, purely a design-level representation. No
explicit Emacs Lisp support for defining classes exists.  The
reification of classes in Emacs Lisp occurs purely through the use of
naming conventions for functions and variables, and voluntary
observance by programmers of the visibility constraints.  However, designing and 
programming
in an object oriented fashion will greatly ease understanding of the system architecture,
as well as supporting migration to other languages providing direct support
for these concepts.

\subsection{Objects} 

Classes are an organizational entity--what actually exists during
execution are particular instances of classes, or {\em objects}.

Each object has a unique identifier associated with it, and this
unique identifier (or the object itself) is normally the first
argument to the operations associated with a class.  This convention
implements a simple kind of message passing behavior, where the first
argument to a function call indicates the particular object that the
message (the function) is sent to.  For example, the {\bf
n\STAR node\STAR lock} operation takes a node-ID as its first argument, which
can be interpreted as having the effect of sending a message to that
particular node to lock itself.

A rough design heuristic is that if an entity does not have a
distinguishing unique ID, or if one is never needed to implement its
behavior, then the entity should probably exist as a part of another
class.  Similarly, deciding the primary home for an object operation
can sometimes be resolved by considering the most appropriate first
argument to it.

\subsection{Attributes}  

Each class instance has a set of characteristics, or {\em
attributes}\/ associated with it.  For example, each screen has a
geometry, color, name, and so forth.  These are attributes of the
class.  If a class has an attribute defined for it, then there exists
a function that returns the value of the attribute.  For example, {\bf
n!screen\STAR name} returns the {\sf name} attribute for an instance of the
class {\sf screen}.

Attributes need not be ``static''.  For example, a reasonable
attribute of the class {\sf node-buffer} is {\sf
link-under-mouse}---an attribute that returns the link ID of the link
label currently under the mouse cursor (or {\tt nil} if the mouse
cursor is not currently over any link label).

Attribute functions take a single argument, the object whose attribute
value is to be retrieved.  This argument is either the object itself
(for example, the Emacs screen object), or the unique ID of the object
(for example, a node ID).

Some attributes are {\em setable}.  This means that an operation
corresponding to the attribute is defined to change its value. This
operation is named by prefixing the attribute name with {\bf set-}.
For example, if the {\bf n!screen\STAR name} attribute is setable,
then the operation {\bf n!screen\STAR set-name} is defined to change
the value of that operation.

Not all attributes should be setable.  For example, it probably
doesn't make sense to have the {\bf node-buffer\STAR link-under-mouse}
attribute mentioned above be setable\foot{Although, you could
conceivably argue for making that attribute setable as a way of
warping the mouse cursor to a particular link label.  But we won't.}.

Attributes can be private or public.  Public attributes are indicated
notationally by the use of \STAR\ as the visibility token; {\bf !} as
the visibility token indicates a private attribute.  If an attribute is
private, then functions outside the implementation of the class should
not access or set the attribute's value.  This, of course, is not
enforced in Emacs Lisp, but is rather a way of helping programmers to 
understand what parts of a subsystem are internal details and can thus
be safely ignored.  If you find yourself as a class implementor needing 
access to a private attribute of another class, you should publicize this
observation immediately.  It may be the case that there is a public mechanism
that you are aware of and should use instead.  Alternatively, it may be
the case that the private attribute should be changed to public.

It is common to have an attribute be publically accessable but only 
privately setable.  To do this, define both the \STAR\ and {\bf !} 
forms of the attribute, but only define the set operation for the {\bf !} form.
For example, one could define two accessors for screen names, 
{\bf n!screen\STAR name} and {\bf n!screen!name}, the first public for use
by any class, and the second private to the implementation of the screen
class.  By defining the set operation for {\bf n!screen!name}, you provide
a consistent mechanism for updating the screen title internally without
making this capability public. 

\subsection{Operations}

While attributes refer to characteristics of classes, {\em operations}
refer to the behaviors of the instances of the class.  For example,
one behavior of a node is to be written out or retrieved from the
hyperbase.  These behaviors are represented by the operations
{\sf write} and {\sf retrieve}, with corresponding Emacs functions {\bf
n\STAR node\STAR write} and {\bf n\STAR node\STAR retrieve}.

Operations may take any number of arguments, but the first argument
is, by convention, an object representing the class instance.  In the
case of objects like nodes or links, the first argument is their
unique ID.  In the case of objects like buffers or screens, the first
argument may be the actual buffer or screen object itself.  Operations
must always document what form their first (as well as the other)
argument must take. The operation may take additional required or
optional arguments depending upon the nature of the behavior.


Operations can be public, indicated by the visibility token ``\STAR '',
or private, indicated by ``{\bf !}''.  Public operations are accessable to
other classes, while private operations are only accessable to the
internal implementation of the class. As noted above, this is purely
convention, and unfortunately Emacs Lisp does not enforce this
encapsulation.  To repeat, if you find yourself as a class implementor
needing access to a private operation of another class, you should
publicize this observation immediately.  It may be the case that there
is a public mechanism that you are aware of and should use instead.
Alternatively, it may be the case that the private operation should be
changed to public.

All instantiable (i.e. non-abstract) classes have a constructor and a
destructor operation associated with them.  By convention, these
operations are named {\sf make} and {\sf delete}, and are normally
public.  (It is often useful to implement two constructor operations,
for example {\bf n\STAR node\STAR make} and {\bf n!node!make}.  The first is
the public operation with publically-supplied arguments; the second is
the internal, down-and-dirty constructor that may take many more
arguments or return an internal object that should be hidden from the
external interface.

\subsection{Collections}

Operations, as defined above, operate on individual instances of the
class. There is another kind of operation, however, that operates on
the set of all objects that exist in the class. These are called {\em
collection operations}.  For example, an operation that returns a list
of all nodes of a specified type is a collection operation.  These
operations, taken together, represent many of the classical database
query and retrieval operations supported in Egret 2.0.

Collection operations are always defined relative to a class, and are
named by wrapping {\bf \{} and {\bf \}} around the class name.  Unlike
conventional class arguments, collection operations do not have a
distinguished first argument, since their name indicates which
collection the function operates on\foot{Future revisions to this design
may lead to the presence of a first argument that is passed a collection
object, consisting of some subset of objects of the named type.}.  For example, the collection
operation for finding all nodes of a specified type might be called
{\bf n\STAR \{node\}\STAR type-of}, which takes a single argument whose value is a
valid node type, and which returns a list of node IDs that are of the 
specified type.

Collection operations can be public or private, and their visibility is
indicated in the standard manner by \STAR\ or {\bf !}.

\chapter{Architecture}
\section{Overview}

The top-level architecture of Egret 2.0 currently consists of the following
major subsystems:  Utilities, Server, Type, and Interface.  Each of these
are briefly introduced below, and detailed in subsequent sections.

\begin{itemizenoindent}
  
\item {\bf Utilities.} The utilities subsystem provides functions
  that are useful across all other classes. Object classes defined in this
  module typically represent extensions of Emacs Lisp built-in constructs.
  For example, {\sf u*hook\/} and {\sf u*error\/} are built directly on top
  of the Emacs error and hook facilities. {\sf u*hash\/} extend Emacs Lisp
  to support hashtables, a data structure that is supported by the Common
  Lisp environment. 
  
\item {\bf Server.} The server subsystem is the basis upon which all
  other components of Egret 2.0 are built.  It implements classes that
  are specialized to the needs of particular database server
  applications.  For example, the server subsystem currently implements
  {\sf s*process-server}, {\sf s*node}, and {\sf s*link} classes whose
  structure and behavior are specialized for use with the hbserver
  database server application.  Future work could include, for example,
  the implementation of an alternate set of classes to work with the
  POSTGRES database server. Regardless of the specific server chosen,
  this subsystem encapsulates details of network access and low-level
  retrieval operations from the other subsystems, and provides
  operations implementing a persistant, fixed, monotyped, lockable
  record structure and other database facilities to other subsystems.
  
\item {\bf Type.} This subsystem uses the server subsystem's node
  and link definition facilities to implement a more flexible mechanism
  termed the {\em exploratory collaborative type system} (ECTS).  ECTS
  allows first-class definition of internal structure at the {\em
  instance}, as well as the class level.  
  
  In addition, the type subsystem provides a ``defeatable''
  encapsulation of the ECTS.  As discussed further below, the public
  classes of this subsystem provide other systems with a ``pure''
  logical view of the database with flexibly typed nodes and links.
  However, operating upon the server in this fashion can sometimes
  incur high overhead, and so this subsystem also allows other systems
  to exploit knowledge about the internal representation to improve
  efficiency.
  
  While the type subsystem implements properties of single classes
  or instances, the type subsystem implements properties of groups
  of classes and instances.  This includes composite objects,
  aggregation definition mechanisms such as layers and surfaces, and
  impact analysis mechanisms for ECTS that operate across groups of
  instances.  For example, these impact analysis mechanisms will
  support the migration of type level information between the class
  and instance levels.
  
  The particular semantics chosen for composite objects and
  aggregates normally depends upon to a certain extent upon the
  domain.  Therefore, some domain-specific functionality may be
  implemented within the type subsystem.

  
\item {\bf Interface.} This subsystem provides various forms of input
  and output to and from the system.  Textual and graphical interfaces
  are current targets for interface-level functionality.
  
  The interface level also contains domain-specific specializations for
  collaborative object oriented design, collaborative research,
  collaborative code inspection, and so forth.  These specializations
  are implemented in terms of calls to operations of the type subsystem
  to define domain-specific types, calls to operations of the type
  subsystem to define domain-specific aggregations, as well as
  functionality defined in this level for traversal, creation,
  modification, and communication in a domain-specific fashion.

\end{itemizenoindent}

The motivation for this top-level decomposition is to isolate those
parts of the system that appear most volatile, while still providing
high performance.  For example, the use of the hbserver application as
the underlying database server may be reconsidered in the near future
if a high quality object oriented database mechanism becomes
available. Such a database change will obviate the need for some of
the type subsystem mechanisms.  Similarly, current interface choices
and domain specializations will evolve over time.

The following sections provide more detail about each of these subsystems. 
\subsection{Utilities Subsystem}

\subsubsection{Services}

The utility subsystem provides a collection of general utilities to be used
by other EGRET modules. Similar to the server subsystem which extends the
functionality of the remote database server, the utility subsystem adds
richness to the Emacs Lisp environment. For example, Elisp does not have
built-in support for hashtable, which is available in other environments
such as the Common Lisp. The utility subsystem implements hashtable using
Elisp primitive, i.e., {\it obarray\/}.

There are also utility classes which are specifically tailored to the Egret
environment. For examples, class {\it table\/} is designed to support data
caching from the remote database server. It implements both hashtable and
alist structures, and allows the user to choose between the two based on
the type of data to be stored and the type of operations to be performed on
that data. Despite such specificity, existing operations on {\it tables\/}
can still be easily generalized and used in applications beyond Egret.


\subsubsection{Representations}

One unique feature of the utility subsystem is that its object classes are
relatively uncoupled with one another. For example, {\it error \/} is
totally orthogonal to {\it hook\/}. The main reason for this {\it
discreteness\/} is the general nature of these classes, which normally
appear first in one of the other Egret subsystems (e.g., server), and are
later generalized and moved to the utility subsystem. 

The utility subsystem currently consists of four object classes: {\it
hashtable\/}, {\it table\/}, {\it error\/}, and {\it hook\/}. The hashtable
is an efficient random-access data structure. It is implemented using
obarray and Elisp symbol primitive operations {\it intern\/} and {\it
intern-soft\/}. The rationale behind this choice is that symbol operations
are highly optimized in Elisp. At the low-level, in fact, they are
implemented using hashtable. On the other hand, the overhead of adopting
any hashing algorithm and maintaining the hashtable structure at the Elisp
level can be tremendous.

The {\it table \/} class supports both hashtable and alist, and the
implementator is given the choice of which structure to use. Alist is
hardly a viable solution for any reasonable sized table, because it
requires linear search. On the other hand, it does possess certain nice
properties which other structures (e.g., hashtable) do not have. A good
example is the pairing of {\it assoc\/} and {\it rassq\/}, which allows
both forward (i.e., key to data) and reverse matching (i.e., from data to
key).  Hashtable, in contrast, permits only one-way (i.e., key to data)
lookup. Since elements in a hashtable are stored in a random fashion, it is
also difficult to generate a sorted list of items. 

To overcome the above two problems and to provide the necessary degree of
equivalence between alist and hashtable, the utility subsystem uses
auxiliary tables in addition to hashtable. Together, they can support
functions such as generating a completion list and table lookup using both
entity ID and name, without too much sacrifice on the efficiency given by
the hashtable.

Both {\it error \/} and {\it hook \/} are extensions of built-in Elisp
constructs. Like their predecessors, error objects in Egret are represented
as symbols, and their attributes as property lists. Extension to the native
construct involve mainly additions of new operations, including tracking
error objects defined, executing a list of forms with error protection.
Similarly, extensions to the built-in hook facility include support for
ordering functions on a given hook.


\subsubsection{Interfaces}
The utility subsystem provides a number of public operations, which are
listed below. For detailed descriptions of these functions and their
arguments, refer to the online designbase:

\begin{enumerate}
\item {\bf Hashtable.} Makehash, gethash, sethash, remhash, and
  hash-existp.
  
\item {\bf Table.} Define table, which in turn defines a set of
  functions, including get, put, delete, existp, initialize,
  get-completion-list, and get-key.
  
\item {\bf Error.} Retrieve attribute values (e.g., name, error
  conditions, error message), define error object, and execute code with
  error protection.

\item {\bf Hook.} Install a function on designated hook.
\end{enumerate}



\subsection{Server Subsystem}

\subsubsection{Services}

The server subsystem interfaces to as well as extends the functionality of
the remote database server. As an interface, it acts like the server
itself, allowing transparent access to the functions of the remote server.
As an extension, the server implements services to improve the performance
via local caching of the remote data. More specifically, the server
subsystem provides the following services:

\begin{itemize}
\item  Persistent data storage, retrieval and update. 
  
\item ``Atomic'' node and link objects. The server subsystem implements
  primitive entities for data storage (nodes) and relations between them
  (links).
  
\item An event-based mechanism for two-way communication 
  between the local clients and the global database server mechanism.
  
\item Session management, e.g., connect and disconnect, and local
  structure initialization; etc.
\end{itemize}


\subsubsection{Internal Representations}

\paragraph {Domain Independence.}
A key property of the server subsystem is it makes no assumptions
about the kind of information being stored in nodes, nor the semantics
of the relationship represented by a link between two nodes.
Most of the server subsystem functionality is implemented directly within
the subsystem, with the exception of event and hook functions.

\paragraph{Event Handling.}
The server subsystem relies on the event mechanism to communicate with the
remote server, to synchronize the local and global states, and to control
concurrent access to the centralized database by multiple users. Event
handling is thus one of primary functions of the server subsystem. Its
dealings with events are, however, restricted to the mechanism level. The
semantics of event handlers are imposed and interpreted by the outer
subsystems. For example, Egret maintains a number of distinguished nodes
containing information necessary for implementing its high-level functions.
Associated with each of these distinguished nodes are a set of event
handlers supplied by the type subsystem for handling various situations
such as node update, deletion, etc. It is the server subsystem's
responsibility to install all event handlers properly and make sure they
are invoked in the right order. However, the server does not know what the
effect of these event handlers will be to the local or global state.

\paragraph{HyperBase Dependence.}
Currently, the server subsystem implements an interface to one database
server application---the HyperBase system from the University of Aalborg.
The internal structure of the server subsystem is therefore somewhat
specialized to this application.  The rationale for this decision is
that the appropriate generalization of the server subsystem to accommodate
multiple applications must arise from generalization of appropriate 
application-specific class structures.  Premature generalization of the 
class structure based upon a single application is inappropriate and 
ultimately harmful. 

The server subsystem consists of six classes: {\it node\/}, {\it link\/},
{\it server-process\/}, {\it snode\/}, {\it serror\/}, and {\it hook\/}.
The {\it node\/} and {\it link\/} classes provide a set of primitive
operations on nodes and links. {\it Server-process \/} maintains
information on individual servers, and manages sessions and synchronizes
the states of the remote and local data. {\it Snode\/} provides operations
on a set of special nodes which contain data used by various Egret modules
for internal purposes. the {\it error \/} class keeps tracks of all error
objects, and {\it hook\/}, all public hooks in the server subsystem.

The structure of the node and link classes follows directly from the fixed
(at compilation time) structure of nodes and links. For example, each field
in the server's node structure corresponds to an attribute of the node
class.  Operations on nodes include individual operations to set as well as
to lock the contents of each field.  Additional collection operations are
provided to define event handlers for the set of all nodes.

The drawback of this design is the fact that the class-level structure
of the server subsystem must change in the event that the fixed
structure of nodes, links, or events changes.  This can also be viewed
as a benefit, in that it clearly reifies within the design the obvious
dependencies that the structure of the system as a whole has upon the
atomic, internal structure of the server application.  Such dependencies
will always exist, and it is simply a question of where they are made
explicit.  At this initial stage, it appears that a clearer, more
robust, and hopefully more reliable implementation will result from this
decision.  

The server-process class provides attributes helpful in identifying the
database being connected to, the location of the data files, and so forth.
It also contains attributes providing information about the state of 
the global database, such as a list of all currently defined node-IDs.

At the server subsystem level, the contents of fields are uniformly
represented as either strings or integers. Such a decision is made for
performance reasons and to clearly delineate the responsibilities of the
server subsystem.  It is not the responsibility of this subsystem to 
process the internal contents of node DATA fields, for example. 

Retrieval in the server subsystem is accomplished purely on a
field-by-field basis; no aggregate retrieval operations for multiple
node fields are provided.  This eliminates the need to provide local
caching of node state within the server layer, and makes performance
considerations purely the responsibility of higher subsystems.  Upper
subsystems must balance the benefits of aggregating information into
the DATA field (in order to retrieve multiple items of information
with one network access) against the costs of unpacking each item from
the DATA field (which may become significant when a single item of
information in the DATA field is required on a frequent basis, such as
during a lengthy traversal of the database).  Since this cost-benefit
analysis is domain-specific, it should be resolved in a higher level,
more domain-specific subsystem.


\subsubsection{Interfaces}

The server subsystem provides a number of public operations, which are
listed below. For detailed descriptions of these functions and their
arguments, refer to the online designbase:

\begin{enumerate}
\item {\bf Node.} Create, delete, retrieve field value (i.e., name, data,
  creation date, modification date, author, modifier, font, geometry,
  incoming links, outgoing links, status), set field value, lock, and
  unlock.
  
\item {\bf Link.} Create, delete, retrieve attribute value (i.e.,
  name, source node, destination node, creation date, modification
  date, author, last modifier), and set attribute value (i.e., name,
  destination node)
  
\item {\bf Server-process.} Retrieve attribute value (i.e., name,
  description, data-path, and ip address), connect, and disconnect.

\item {\bf System node.} Define a node, with-system-node, and
  with-system-node-locked. 

\item {\bf Error.} Retrieve attribute value (i.e., name, conditions,
  message), define error object, and with-error-protected. 

\item {\bf Hook.} Install function on to a hook object.
\end{enumerate}


\subsection {Type Subsystem}

\subsubsection{Services}

The type subsystem implements a typed, flexibly structured data
model called ECTS on top of the monotyped, fixed structure data model
provided by the server subsystem. 

ECTS departs from traditional type systems by allowing first class
definition of fields at the instance as well as at the class level.
Traditionally, the set of fields associated with any instance is
determined by its type. In other words, the type specification serves
as a template for creating instances, and all instances are
constrained to exactly the structure indicated by the type
specification.  While ECTS class definition facilities do provide a
template for instance creation, instances are not subsequently
constrained to that structure. Instead, individual instances of typed
nodes can dynamically define new fields, or delete some of its
original fields.

To make ECTS work in a collaborative context, impact analysis and
other mechanisms must also be provided in order to control the
proliferation of instance-level variants.  Some of these mechanisms
are operations for the inspection and modification of individual
instances and classes.  Other operations analyze the state of the
entire database with respect to its type level structure.

To support the more mundane aspects of exploratory development, the
type subsystem is constructed to facilitate ease in renaming or
modifying other attributes of of the constituant objects (i.e. nodes,
links, fields, and schemas).  Thus, node contents always store
references to attributes of links, fields, and schemas
symbolically---by their corresponding ID.  After retrieving a node,
the local client has the responsibility of mapping the ID to the
corresponding name.  This allows changes to field or link names to be
instantly reflected in all nodes referencing them without requiring a
traversal of the entire database looking for instances and textually
changing them.  The cost is that local tables for node, schema, field,
and link instances that associates their ID with all relevent
attributes must be maintained.

\subsubsection{Representations}

\paragraph{Node Representation.}

Type nodes are represented as an aggregration of field instances.  

\paragraph{Field Representation.}

The representation for type subsystem fields within the server subsystem DATA 
field will follow along the lines discussed in our last meeting, for example:

\begin{verbatim}
(s*node*restore-field :name 'reference :contents "the contents...")
\end{verbatim}

While this is an evaluable Lisp form, it could simply be textually
parsed by non-Lisp applications.  Note that the representation of
embedded links is not a part of the type subsystem.

\subsubsection{Interface}

\paragraph{Class Node-Schema.}

The purpose of this class is to specify the organization and
operations available upon the "consensual" structures of EGRET.  The
set of instances of this class are structure-specifying entities for
which there is (more or less) consensual agreement during
collaboration.  Obvious operations upon schema instances include those
to add or delete the set of fields comprising the schema. Another
operation available upon a consensual structure is "instantiate",
which creates a new instance of a node with an internal field
structure corresponding to the consensual structure.

Conceptually, therefore, the instances of the class SCHEMA could be
plausibly viewed as "classes" from which node instances with a
particular structure are created.  However, this only a conceptual
correspondence: the *actual* class NODE is not an instance of a SCHEMA
class object.


\paragraph {Class Node-Instance.}

This purpose of this class is to specify the organization and
operations upon individual node instances appropriate to the
evolutionary concerns of EGRET. The set of node instances are both
structure-specifying (since they have locally associated field
instances) and content-holding (since they will hold the values
associated with fields.)
  
Like schemas, operations upon node instances include those to add and
subtract fields.  Similar to schema instantiation, the "clone"
operation upon a node instance results in the creation of a new node
instance with the same internal structure, but without any content.

Each node instance corresponds to an actual Hyperbase node.
(Just to be perfectly clear, the converse is not true: not every
hyperbase node corresponds to an instance of the node class.)

\paragraph {Class Field.}

Fields are the primitive structure-specifying entity of EGRET.
However, they do not exist independently, but rather only as a
component of a schema or node instance.  The same field instance could
be a component of multiple schemas and node instances, since field
instances are only structure-specifying, not content-holding.


\paragraph {Class Link-Schema.}

Analogous to Node-Schema, this class specifies the consensual link
structures. 

\paragraph{Class Link-Instance.}

Analogous to Node-Instance, this class specifies the contents and 
structural properties of individual links. 


\paragraph {Class Layer.}

Layers are the aggregation mechanism for representing and supporting
the exploration and consolidation phases of exploration collaboration.
The contents of each instance of a layer contains links to each member
node and link instance contained within it.\foot{Currently, we believe
that maintaining schema and field information outside of layers leads
to greater clarity in understanding exploratory phenomena.} Other
contents of a layer represent its relationship to prior and subsequent
defined layers.  The operations upon a layer include computations of
convergence and divergence for the layer.

Each layer instance corresponds to an actual Hyperbase node. 

Layers do *not* support functional partitions of the database, nor do
they support versioning of particular nodes, fields, or schemas.
Their sole purpose is to reflect the cyclic periods of exploration
followed by consolidation.  Thus, they tend to partition the database
into something like a temporally ordered set of "onion skins", where
each onion skin reflects a single period of exploration followed by
consolidation for a particular portion of the database. The current,
or "outermost" layer can always be referred to symbolically by the
alias "active". (Thus, no layer can be explicitly named "active")

Individual node, field, and schema instances can be members of more than
one layer.  This occurs when an instance is both actively used and 
structurally stable, so that it is relevent across multiple phases
of exploration and consolidation.

\chapter{Verification and Validation}
\section{Test Plan for Egret 2.0}

\subsection{Concepts}

Standard software engineering practice for testing involves several
interrelated components. These components are briefly defined below:

The {\em test framework} describes and implements the set of functions
and procedures that support definition of the test plan, test cases,
test data, and test log.

The {\em test environment} specifies the required state of the environment
that must exist in order to start running the test.

The {\em test plan} specifies the flow of control to be followed during 
the running of the test. In particular, it specifies the sequence of test
cases to be run, and how this sequence changes in response to positive or 
negative results of previous test cases.  The test plan also specifies the 
results of the test in terms of the format of the test log.

Each {\em test case} specifies a test of some particular behavior, 
function, or other aspect of the system.  For example, each public operation
and attribute of the system will have a corresponding test case.  Test
cases can be further broken down into sets of test data.

Each {\em test data} item specifies (a) input to the system, (b) the
state of the system during the test, and (c) the expected output.  For
example, the server subsystem test case might have a test data item
that requests connection to a particular server while the system is already
connected to a server, and which expects the connection function to return 
a specific value that indicates this particular error. 


\subsection{Requirements}

The testing of Egret 2.0 should satisfy the following requirements:

\begin{itemize}
\item {\bf Background, batch operation.}  The testing of Egret 2.0 must support
the ability for testers to invoke it with a single command and then leave it to 
run all test cases unattended.  The results of each test run should be recorded
in a log file.

\item {\bf Testing of all public operations and attributes.}  All public operations for a 
given class and subsystem must be tested.  In particular, all 
arguments to all public operations will be tested with valid values,
invalid values, and invalid types of values.

\item {\bf Independence from private operations.}  Test cases are implemented
without knowledge of the internal implementation of a class or subsystem. 

\item {\bf Verification, not diagnosis.} The primary goal of the test plan is
determine that the system either (a) passes testing, or (b) fails
testing.  Diagnosis of the problem is not a primary goal of the test plan.
In particular, determining which aspect of the implementation is in error 
is useful only when all other requirements can be simultaneously satisfied.

\item {\bf Object orientation.}  The test plan is organized in an object
oriented manner, with classes such as {\sf test-plan}, {\sf test-case}, 
and {\sf test-data}.
\end{itemize}

There are a number of additional, general comments to make about testing.
First, a test case for one public operation can implicitly presume that all other
public operations are implemented correctly.   This is mandated by the lack
of access by the test system to the private operations of the subsystem under
test, and the need to avoid re-implementing a parallel system just for testing.

The test plan should write results out to a log buffer, including the
date and time of each test case, the test case being conducted, the
specific test data, and the expected and actual results.  Some sort of
cumulative statistics about the test run should be written to the
buffer as well, including at a minimum a listing of all failing test
data items.  This log buffer should be saved to a file after every
test data item is run, in order to preserve the results of the test in
the event of the system crashing Emacs.

Some specific hints about test data item creation:

\begin{itemizenoindent}

\item Invoke each operation with invalid data at the type and the value level.
    For example, the attribute {\bf s\STAR node\STAR name} accepts a node-ID as its
    argument.  This attribute should be tested by passing {\bf s\STAR node\STAR name}
    both a string (i.e. the wrong type of argument) and an integer that
    is guaranteed not to be a valid node-ID (such as -1).  

\item Set, then test the explicit state of the system.  For example,
    invoke {\bf s\STAR node\STAR set-name} on a newly created node to change its name, then use 
    {\bf s\STAR node\STAR name} to see if the name was actually changed.

\item Set, then test the implicit state of the system.  For example,
    an implicit result of creating a node should be a change in the list of 
    node-IDs.  Disconnecting should also change the list of node-IDs.

\item Test the recovery operations.  This is difficult, since corrupting 
    the tested state normally requires access to the internal, private state.
    Instead, assume the state is correct, invoke the recovery operations,
    and see if the prior state still exists.  

\item Test concurrent and cooperative access. This kind of test requires a 
    shell script that starts up two Emacs processes, changes the user-name of
    one of them, then tests locking, event propogation, and so forth.

\end{itemizenoindent}


\subsection{Server Subsystem Testing}

Testing of the server subsystem can be broken down into the following
major categories of tests:

\begin{enumerate}

\item {\bf Connection and Disconnection.}  These tests involve checking
the ability of the system to connect and disconnect successfully from
a database application server process.   The ability to sequentially connect
to different servers will be tested.  Connecting to both new and previously
constructed databases will be tested.  The state of the server-process class
attributes and operations will be tested when both connected and disconnected
from a database server.

\item  {\bf Node operations.}  All public node operations will be tested, 
including node creation and deletion, and reading, writing, and
updating of all node fields.  

\item {\bf Link operations.}  All link operations will be tested on
representative nodes and links.   

\item {\bf Lock operations.}  The ability to lock and unlock nodes will
be tested, and the correct maintenance of relevent state information (such 
as {\bf s*node*locked-by}) will be tested.

\item {\bf Event operations.}  The ability of the server layer to 
correctly set up, disable, and order event handling will be tested.

\end{enumerate}

More specific documentation on each test case and its associated test data
will be forthcoming. 


%%% \part{Design Specification}
%%% \twocolumn \renewcommand{\horizontalline} {\rule{\textwidth}{.01in}}
%%% \ls{1.0} \small
%%% \chapter{Utilities}
\label{Utilities}

\begin{description}
\item [Name:]  Utilities

\item [Description:]
This subsystem contains generic classes and utilities for the
entire Egret environment.  Object classes and their operations
defined in this module are available for use and
specialization by all other subsystems. Due to the general
nature of this subsystem, it should always be loaded before
all other subsystems.

\item [Public-classes:]
\item {\sl u*hash}\hfill(page~\pageref{u*hash})
\item {\sl u*table}\hfill(page~\pageref{u*table})
\item {\sl u*error}\hfill(page~\pageref{u*error})
\item {\sl u*hook}\hfill(page~\pageref{u*hook})




\end{description}
\horizontalline

\section{u*hash}
\label{u*hash}

\begin{description}
\item [Name:]  u*hash

\item [Layer:]
{\sl Utilities}\hfill(page~\pageref{Utilities})

\item [Description:]

U*HASH is an implementation of the hash table data
structure in Emacs Lisp. Hash tables trade off space for
time, potentially providing faster retrieval than that
allowed by association lists, the typical alternative
Lisp table structure, which uses sequential search.

U*HASH is implemented as vectors. Each element of U*HASH
can be arbitrary Lisp object, ranging from symbols to
complex structures. Though hash tables are oriented
toward one-way mapping, i.e., from key to data,
U*HASH provides restricted reverse searching, i.e., from
data to key. It does so via maintaining an auxiliary
alist of all elements in the table.

\item [Attributes:]

\item [Operations:]
\item {\sl u*hash*make}\hfill(page~\pageref{u*hash*make})
\item {\sl u*hash*get}\hfill(page~\pageref{u*hash*get})
\item {\sl u*hash*set}\hfill(page~\pageref{u*hash*set})
\item {\sl u*hash*rem}\hfill(page~\pageref{u*hash*rem})
\item {\sl u*hash*existp}\hfill(page~\pageref{u*hash*existp})

\item [Collections:]

\item [Subclasses:]

\item [Superclasses:]

\item [Instances:]



\end{description}
\horizontalline

\subsection{u*hash*make}
\label{u*hash*make}

\begin{description}
\item [Name:]  u*hash*make

\item [Class:]
{\sl u*hash}\hfill(page~\pageref{u*hash})

\item [Parameters:]
\item {\sl size}:  integer (prime number)


\item [Return-value:] obarray of required size with all cells set
to nil.   

\item [Description:]

Creates an obarray as hashtable. SIZE, if supplied, should
be a prime number. Otherwise, defaults to default-size,
i.e., 511. 

\item [Public:]



\end{description}
\horizontalline

\subsection{u*hash*get}
\label{u*hash*get}

\begin{description}
\item [Name:]  u*hash*get

\item [Class:]
{\sl u*hash}\hfill(page~\pageref{u*hash})

\item [Parameters:]
\item {\sl hash-key}:  symbol

\item {\sl table}:  symbol


\item [Return-value:] 
data element of the table entry.

\item [Description:]
Retrieves data element from the hashtable. If optional
arg TABLE is supplied, use that table. Otherwise, use
the system default table.

\item [Public:]



\end{description}
\horizontalline

\subsection{u*hash*set}
\label{u*hash*set}

\begin{description}
\item [Name:]  u*hash*set

\item [Class:]
{\sl u*hash}\hfill(page~\pageref{u*hash})

\item [Parameters:]
\item {\sl hash-key}:  symbol

\item {\sl data}:  any Lisp object

\item {\sl table}:  symbol


\item [Return-value:] new data value

\item [Description:]
Stores DATA in table TABLE using HASH-KEY as key.  If
optional arg TABLE is not supplied, defaults to system
table. 

\item [Public:]



\end{description}
\horizontalline

\subsection{u*hash*rem}
\label{u*hash*rem}

\begin{description}
\item [Name:]  u*hash*rem

\item [Class:]
{\sl u*hash}\hfill(page~\pageref{u*hash})

\item [Parameters:]
\item {\sl hash-key}:  symbol

\item {\sl table}:  symbol


\item [Return-value:]  t if deletion is successful or
nil otherwise.

\item [Description:]

Removes data element corresponding to key HASH-KEY from
table TABLE. If TABLE is not supplied, defaults to
system table. 

\item [Public:]



\end{description}
\horizontalline

\subsection{u*hash*existp}
\label{u*hash*existp}

\begin{description}
\item [Name:]  u*hash*existp

\item [Class:]
{\sl u*hash}\hfill(page~\pageref{u*hash})

\item [Parameters:]
\item {\sl hash-key}:  symbol

\item {\sl table}:  symbol


\item [Return-value:] 
data value if found or nil otherwise.

\item [Description:]
Check the existence of the item with key HASH-KEY in
table TABLE. If TABLE is not supplied, defaults
to system table.

\item [Public:]




\end{description}
\horizontalline

\section{u*table}
\label{u*table}

\begin{description}
\item [Name:]  u*table

\item [Layer:] {\sl Utilities}\hfill(page~\pageref{Utilities})

\item [Description:]
U*TABLE is a generic class that consists of paired objects: KEY and
DATA: KEY is either string or symbol, and DATA, any lisp object.
U*TABLE is typically used to store data for quick lookup purposes. For
example, the Server subsystem uses this class extensively as cache
structures for storing selected node and link attributes so that
lookup operations can be done efficiently (i.e., no remote access.
Important U*TABLE operations include GET, PUT, DELETE, EXISTP,
INITIALIZE, and so forth.

U*TABLE is implemented using both hashtable and alist. For a given
table, one can specify whether to use hashtable or alist, depending on
the requirement, e.g., size, operation types. The default is alist. 

\item [Attributes:]

\item [Operations:]
\item {\sl u*table*define}\hfill(page~\pageref{u*table*define})

\item [Collections:]

\item [Subclasses:]

\item [Superclasses:]



\end{description}
\horizontalline

\subsection{u*table*define}
\label{u*table*define}

\begin{description}
\item [Name:]  u*table*define

\item [Class:]
{\sl u*table}\hfill(page~\pageref{u*table})

\item [Parameters:]
\item {\sl table-name}:  symbol

\item {\sl fn-prefix}:  string

\item {\sl rlookup-fn}:  slot accessor function

\item {\sl table-size}:  integer (prime number)

\item {\sl hashp}:  Boolean 

\item {\sl init-value}:  list of Lisp objects.


\item [Return-value:]

\item [Description:]
Defines a table called TABLE-NAME with requested parameter
constraints. Table operations include PUT, GET, DELETE, EXISTP,
GET-KEY, GET-COMPLETION-LIST, and INITIALIZE. They are prefixed with
FN-PREFIX.
Keyword TABLE-NAME is an optional symbol indicating the name of
table. When nil, defaults to the global table.  
Keyword FN-PREFIX is a required string used as prefix of table
operations, e.g., s!node*.
Keyword HASHP is a Boolean indicating if the table should be a
hashtable or not.  Using hashtable if t or alist otherwise.
Keyword RLOOKUP-FN is an optional arg indicating DATA slot on
which reverse lookup might be done. Currently, only one such
slot is allowed.
Keyword TABLE-SIZE is an optional prime number indicating the
size of hashtable.  Used only when HASHP is t.
Keyworkd COMPLETION-FN is an optional arg indicating DATA slot
on which completion list will be maintained. Currently, only one
such list is allowed.
Keyword INITIAL-VALUES is a set of initial values for the table.
It must be provided as an alist whose elements are in the form
of (KEY . DATA)

\item [Public:]



\end{description}
\horizontalline

\section{u*error}
\label{u*error}

\begin{description}
\item [Name:]  u*error

\item [Layer:] {\sl Utilities}\hfill(page~\pageref{Utilities})

\item [Description:]
U*ERROR is a global error class which subjects to specialization in
each subsystem. In other words, each subsystem has its own error
class, which is a subclass of U*ERROR. Inevitably, they inherits all
attributes and operations of U*ERROR. These error subclasses normally
add no new attributes and operations. Instead, they contain explicit
links to to all error instances that belong to that subclass.

U*ERROR follows the same structure as that of the Emacs Lisp error
object, i.e., each error object contains three attributes or
properties: error symbol or name, error conditions, and error message.
The attribute operations defined below are also applicable to Emacs
built-in error objects listed on pp. 535-536 of the Gnus Emacs Lisp
Reference Manual

\item [Attributes:]
\item {\sl u*error*name}\hfill(page~\pageref{u*error*name})
\item {\sl u*error*conditions}\hfill(page~\pageref{u*error*conditions})
\item {\sl u*error*message}\hfill(page~\pageref{u*error*message})

\item [Operations:]
\item {\sl u*error*define}\hfill(page~\pageref{u*error*define})
\item {\sl u*error*protected}\hfill(page~\pageref{u*error*protected})

\item [Collections:]

\item [Subclasses:]
\item {\sl s*serror}\hfill(page~\pageref{s*serror})
\item {\sl t*error}\hfill(page~\pageref{t*error})

\item [Superclasses:]




\end{description}
\horizontalline

\subsection{u*error*name}
\label{u*error*name}

\begin{description}
\item [Name:]  u*error*name

\item [Class:]
{\sl u*error}\hfill(page~\pageref{u*error})

\item [Contents:] symbol 

\item [Description:]  error name

\item [Setf-able:] no

\item [Public:]



\end{description}
\horizontalline

\subsection{u*error*conditions}
\label{u*error*conditions}

\begin{description}
\item [Name:]  u*error*conditions

\item [Class:]
{\sl u*error}\hfill(page~\pageref{u*error})

\item [Contents:] list of symbols

\item [Description:] 
Specifying a set of conditions under which error
message is to be sent to the user. Normally, this
list includes the error symbol itself adn the symbol
ERROR. Occassionally it includes additional symbols,
which are intermediate classifications, narrower than
error but broader than a single error symbol.

\item [Setf-able:]

\item [Public:]



\end{description}
\horizontalline

\subsection{u*error*message}
\label{u*error*message}

\begin{description}
\item [Name:]  u*error*message

\item [Class:]
{\sl u*error}\hfill(page~\pageref{u*error})

\item [Contents:] message to be sent to the user upon the 
occurance of the error. 

\item [Description:]

\item [Setf-able:] no

\item [Public:]



\end{description}
\horizontalline

\subsection{u*error*define}
\label{u*error*define}

\begin{description}
\item [Name:]  u*error*define

\item [Class:]
{\sl u*error}\hfill(page~\pageref{u*error})

\item [Parameters:]
\item {\sl error-msg}:  string

\item {\sl error-symbol}:  symbol


\item [Return-value:] 
macro with no interesting return value

\item [Description:]
Defines a new error symbol, ERROR-SYMBOL, with two
default condition names: value of ERROR-SYMBOL and
ERROR. Default error message is ERROR-MSG.

\item [Public:]



\end{description}
\horizontalline

\subsection{u*error*protected}
\label{u*error*protected}

\begin{description}
\item [Name:]  u*error*protected

\item [Class:]
{\sl u*error}\hfill(page~\pageref{u*error})

\item [Parameters:]
\item {\sl error-symbols}:  symbol or list of symbols

\item {\sl body}:  list of Lisp forms


\item [Return-value:] 
error object from ERROR-SYMBOLS if error occurs in the
execution of BODY or return value of last form of BODY.

\item [Description:]
executes forms in BODY and catches errors specified in
ERROR-SYMBOLS.  ERROR-SYMBOLS is a required symbol or
list of symbols that represent error conditions to be
caught.  

\item [Public:]



\end{description}
\horizontalline

\section{u*hook}
\label{u*hook}

\begin{description}
\item [Name:]  u*hook

\item [Layer:] {\sl Utilities}\hfill(page~\pageref{Utilities})

\item [Description:] 
U*HOOK is an extension of the Emacs hook mechanism whose value
is a function or a list of functions to be called under certain
pre-defined conditions. The main purpose of hooks is for
customization and modularity. The main extension is that U*HOOK
allows ordering constraints to be imposed on the functions
installed on a given hook.

Like U*ERROR, U*HOOK is a generic object class that subjects to 
specialization in various subsystems. These hook subclasses contain
links to individual instances within respective subsystems. 

\item [Attributes:]

\item [Operations:]
\item {\sl u*hook*insert}\hfill(page~\pageref{u*hook*insert})

\item [Collections:]

\item [Subclasses:]

\item [Superclasses:]



\end{description}
\horizontalline

\subsection{u*hook*insert}
\label{u*hook*insert}

\begin{description}
\item [Name:]  u*hook*insert

\item [Class:]
{\sl u*hook}\hfill(page~\pageref{u*hook})

\item [Parameters:]
\item {\sl hook-var}:  symbol

\item {\sl hook-fn}:  symbol

\item {\sl before-hook-fns}:  list of symbols


\item {\sl after-hook-fns}:  list of symbols



\item [Return-value:]
Returns t if the new event function can be
inserted into the sequence of hook functions in such
a manner as to satisfy the before and after 
constraints, or {\sl conflicting-hook-constraints} (page~\pageref{conflicting-hook-constraints})
otherwise.

\item [Description:]
Installs hook function HOOK-FN on to designated hook
variable.  Returns hook variable value if the new hook fn can
be inserted into the sequence of hook fns that satisfy
required before- and after- constraints; otherwise returns
error object CONFLICTING-HOOK-CONSTRAINTS


\item [Public:]



\end{description}
\horizontalline

\chapter{Server}
\label{Server}

\begin{description}
\item [Name:]  Server Subsystem

\item [Description:]
The SERVER subsystem is the interface between the remote
database server and other modules of the EGRET system. As
such, it provides transparent and efficient access to the
functions of the server via such facilities as caching,
events, etc. It also offers encapsulation via
comprehensive error checking and additional services
(e.g., global nodes) essential for the implementation of
advanced Egret functions.

The SERVER module consists of six classes: S*NODE, S*LINK,
S*SERVER-PROCESS, S*SYS-NODE, S*SERROR, and S*HOOK.
S*NODE and S*LINK provide a set of primitive operations on
nodes and links. S*SERVER-PROCESS manages connection
sessions and synchronizes the states of the remote and
local data. S*SYS-NODE supports a set of special nodes
which contain data used by various Egret modules for
internal purposes. S*SERROR tracks all error objects, and
S*HOOK, all public hook objects in the SERVER subsystem. 

\item [Public-classes:]
\item {\sl s*node}\hfill(page~\pageref{s*node})
\item {\sl s*link}\hfill(page~\pageref{s*link})
\item {\sl s*server-process}\hfill(page~\pageref{s*server-process})
\item {\sl s*snode}\hfill(page~\pageref{s*snode})
\item {\sl s*serror}\hfill(page~\pageref{s*serror})
\item {\sl s*event}\hfill(page~\pageref{s*event})  (actually s*event)

\item [Private-classes:]




\end{description}
\horizontalline

\section{s*node}
\label{s*node}

\begin{description}
\item [Name:]  s*node

\item [Layer:] {\sl Server}\hfill(page~\pageref{Server})

\item [Description:]
This class implements primitive operations on nodes, including
creation, deletion, field content retrieval and updating,
locking, and so forth. Most of these functions are available from
the remote server. The primary purpose of this class is to
provide a uniform and location-independent access, i.e., the
caller does not need to know whether the returned data is from
local cache or directly from the remote server.

Note that except for the two node repairing functions, which are
acessable only to the system administrator, all other operations
and attribute functions are PUBLIC. Private functions are not
documented here.

\item [Attributes:]
\item {\sl s*node*name}\hfill(page~\pageref{s*node*name})
\item {\sl s*node*data}\hfill(page~\pageref{s*node*data})
\item {\sl s*node*created-by}\hfill(page~\pageref{s*node*created-by})
\item {\sl s*node*created-date}\hfill(page~\pageref{s*node*created-date})
\item {\sl s*node*last-modified-date}\hfill(page~\pageref{s*node*last-modified-date})
\item {\sl s*node*last-modified-by}\hfill(page~\pageref{s*node*last-modified-by})
\item {\sl s*node*font}\hfill(page~\pageref{s*node*font})
\item {\sl s*node*geometry}\hfill(page~\pageref{s*node*geometry})
\item {\sl s*node*locked-by}\hfill(page~\pageref{s*node*locked-by})
\item {\sl s*node*incoming-links}\hfill(page~\pageref{s*node*incoming-links})
\item {\sl s*node*outgoing-links}\hfill(page~\pageref{s*node*outgoing-links})

\item [Operations:]
\item {\sl s*node*make}\hfill(page~\pageref{s*node*make})
\item {\sl s*node*delete}\hfill(page~\pageref{s*node*delete})
\item {\sl s*node*lock}\hfill(page~\pageref{s*node*lock})
\item {\sl s*node*unlock}\hfill(page~\pageref{s*node*unlock})
\item {\sl s*node*set-name}\hfill(page~\pageref{s*node*set-name})
\item {\sl s*node*set-data}\hfill(page~\pageref{s*node*set-data})
\item {\sl s*node*set-geometry}\hfill(page~\pageref{s*node*set-geometry})
\item {\sl s*node*set-font}\hfill(page~\pageref{s*node*set-font})

\item {\sl s*node@reset-incoming-links}\hfill(page~\pageref{s*node@reset-incoming-links})
\item {\sl s*node@reset-outgoing-links}\hfill(page~\pageref{s*node@reset-outgoing-links})

\item [Collections:]

\item [Subclasses:]

\item [Superclasses:]



\end{description}
\horizontalline

\subsection{s*node*name}
\label{s*node*name}

\begin{description}

\item [Name:]  s*node*name

\item [Class:] {\sl s*node}\hfill(page~\pageref{s*node}) 

\item [Contents:] String (40)

\item [Description:]
The name of the node. Node names are not
required to be unique at the server subsystem
level, although higher subsystems may wish to
enforce this constraint.

\item [Setf-able:] See s*node*set-name

\item [Public:]



\end{description}
\horizontalline

\subsection{s*node*data}
\label{s*node*data}

\begin{description}

\item [Name:]  s*node*data

\item [Class:] {\sl s*node}\hfill(page~\pageref{s*node})

\item [Contents:]  Unspecified 

\item [Description:]
This is the famous, variable length data
field, of which the server subsystem is
permitted to know practically nothing about.

\item [Setf-able:]


\item [Public:]



\end{description}
\horizontalline

\subsection{s*node*created-by}
\label{s*node*created-by}

\begin{description}

\item [Name:]  s*node*created-by

\item [Class:] {\sl s*node}\hfill(page~\pageref{s*node})

\item [Contents:] String (20)

\item [Description:]

The original author of the node.  This is 
automatically set by the server during 
node creation, but its value is accessable.

\item [Setf-able:]


\item [Public:]



\end{description}
\horizontalline

\subsection{s*node*created-date}
\label{s*node*created-date}

\begin{description}

\item [Name:]  s*node*created-date

\item [Class:] {\sl s*node}\hfill(page~\pageref{s*node})

\item [Contents:] String

\item [Description:]
Node creation date. Automatically set by 
the server subsystem during node creation.

\item [Setf-able:]

\item [Public:]



\end{description}
\horizontalline

\subsection{s*node*last-modified-date}
\label{s*node*last-modified-date}

\begin{description}

\item [Name:]  s*node*last-modified-date

\item [Class:] {\sl s*node}\hfill(page~\pageref{s*node})

\item [Contents:] string

\item [Description:]
Last modified date.  Automatically 
set by the server subsystem during 
node field updating.

\item [Setf-able:]


\item [Public:]



\end{description}
\horizontalline

\subsection{s*node*last-modified-by}
\label{s*node*last-modified-by}

\begin{description}

\item [Name:]  s*node*last-modified-by

\item [Class:] {\sl s*node}\hfill(page~\pageref{s*node})

\item [Contents:] String (20)

\item [Description:]
Automatically maintained by server subsystem
during node field updates.

\item [Setf-able:]

\item [Public:]



\end{description}
\horizontalline

\subsection{s*node*font}
\label{s*node*font}

\begin{description}

\item [Name:]  s*node*font

\item [Class:] {\sl s*node}\hfill(page~\pageref{s*node})

\item [Contents:] String

\item [Description:]
A font specification.  This might be checked
for validity.  This attribute is not
maintained by the server.

\item [Setf-able:]


\item [Public:]



\end{description}
\horizontalline

\subsection{s*node*geometry}
\label{s*node*geometry}

\begin{description}

\item [Name:]  s*node*geometry

\item [Class:] {\sl s*node}\hfill(page~\pageref{s*node})

\item [Contents:] string

\item [Description:]
This should be a valid geometry
specification.  It's value is not
maintained by the server.

\item [Setf-able:]


\item [Public:]



\end{description}
\horizontalline

\subsection{s*node*locked-by}
\label{s*node*locked-by}

\begin{description}

\item [Name:]  s*node*locked-by

\item [Class:] {\sl s*node}\hfill(page~\pageref{s*node})

\item [Contents:] string

\item [Description:]

This attribute is either nilt (if the node is
unlocked) or a user-name (if the node is locked by
the specified user). It can also return an error
object SHOW-LOCK-FAILS.

This value is maintained by the server
subsystem.

\item [Setf-able:]


\item [Public:]



\end{description}
\horizontalline

\subsection{s*node*incoming-links}
\label{s*node*incoming-links}

\begin{description}

\item [Name:]  s*node*incoming-links

\item [Class:] {\sl s*node}\hfill(page~\pageref{s*node})

\item [Contents:] List of link-IDs

\item [Description:]
This attribute holds a list of link-IDs
whose destination node-IDs are equal 
to the ID of this node instance.

Since this is corruptable local information,
the function s*node@reset-incoming-links
exists to rebuild this node's attribute.

\item [Setf-able:]


\item [Public:]



\end{description}
\horizontalline

\subsection{s*node*outgoing-links}
\label{s*node*outgoing-links}

\begin{description}

\item [Name:]  s*node*outgoing-links

\item [Class:] {\sl s*node}\hfill(page~\pageref{s*node})
 
\item [Contents:] A list of link-IDs

\item [Description:] 
A list of link-IDs corresponding to the
links pointing away from this node.

Since this is local, corruptable information,
the operation s*node!reset-links rebuilds
this by reference to the hbserver.

\item [Setf-able:]


\item [Public:]



\end{description}
\horizontalline

\subsection{s*node*make}
\label{s*node*make}

\begin{description}
\item [Name:]  s*node*make

\item [Class:] {\sl s*node}\hfill(page~\pageref{s*node})

\item [Parameters:]
\item {\sl node-name}:  
A valid node name. This currently means that it is a
string of less than 40 characters, and that it does
not contain leading space(s) or tabs.
  

\item [Return-value:] 
A node-ID if successful.

error object {\sl invalid-node-name} (page~\pageref{invalid-node-name}) if either
NODE-NAME contains leading space(s)/tab(s) or its
length exceeds 40.

error object {\sl create-node-fails} (page~\pageref{create-node-fails}) if node
creation hb-call fails.

\item [Description:]
This is the primitive function for obtaining
new node-IDs from the remote server.

On successful node creation, the following node
attributes are set:

s*node*created-by and s*node*last-modified-by
are set to the original author of the node;
s*node*created-date and s*node*last-modified-date
are set to the creation date; s*node*geometry and s*node*font
are set to repective user defaults, or system defaults if
user defaults are not set (see INTERFACE subsystem for more
details on user defaults)

\item [Public:]





\end{description}
\horizontalline

\subsection{s*node*delete}
\label{s*node*delete}

\begin{description}
\item [Name:]  s*node*delete

\item [Class:] {\sl s*node}\hfill(page~\pageref{s*node})

\item [Parameters:] 
\item {\sl node-ID}:  
integer (a valid hbserver node ID number)
 

\item [Return-value:]
NODE-ID if the node-ID was successfully deleted.

error object {\sl node-still-referenced} (page~\pageref{node-still-referenced}) if the
target node still contains incoming links.

error object  {\sl node-still-locked} (page~\pageref{node-still-locked}) if the target
node is still locked.

error object {\sl unknown-hb-error} (page~\pageref{unknown-hb-error}) if node
deletion hb-operation fails for reasons other
than the above.

\item [Description:]
Permanently removes node NODE-ID from remote database.

\item [Public:]


\end{description}
\horizontalline

\subsection{s*node*lock}
\label{s*node*lock}

\begin{description}
\item [Name:]  s*node*lock

\item [Class:] {\sl s*node}\hfill(page~\pageref{s*node})

\item [Parameters:] 
\item {\sl node-ID}:  
integer (a valid hbserver node ID number)


\item [Return-value:]
T if the lock was successfully obtained, nil otherwise.

\item [Description:]
Attempts to get a lock on node-ID.  Will fail if 
node-ID is already locked by another user.

\item [Public:]



\end{description}
\horizontalline

\subsection{s*node*unlock}
\label{s*node*unlock}

\begin{description}
\item [Name:]  s*node*unlock

\item [Class:] {\sl s*node}\hfill(page~\pageref{s*node})

\item [Parameters:] 
\item {\sl node-ID}:  
integer (a valid hbserver node ID number)
 

\item [Return-value:]
t if node-ID was successfully unlocked, nil otherwise.

\item [Description:]
Unlocks node-ID. The user who is trying to unlock the node
must be the same as the one who locks the node.

\item [Public:]



\end{description}
\horizontalline

\subsection{s*node*set-name}
\label{s*node*set-name}

\begin{description}
\item [Name:]  s*node*set-name

\item [Class:] {\sl s*node}\hfill(page~\pageref{s*node})

\item [Parameters:]
\item {\sl node-ID}:  
integer (a valid hbserver node ID number)

\item {\sl node-name}:  
A valid node name. This currently means that it is a
string of less than 40 characters, and that it does
not contain leading space(s) or tabs.


\item [Return-value:]
A node-ID if successful.

error object {\sl invalid-node-name} (page~\pageref{invalid-node-name}) if either
NODE-NAME contains leading space(s)/tabs or its
length exceeds 40.

Error object {\sl write-attribute-fails} (page~\pageref{write-attribute-fails}) if rename
hb-operation fails.

\item [Description:]
Resets the name of the node. Involves a write
out to the database.

\item [Public:]


\end{description}
\horizontalline

\subsection{s*node*set-data}
\label{s*node*set-data}

\begin{description}
\item [Name:]  s*node*set-data

\item [Class:] {\sl s*node}\hfill(page~\pageref{s*node})

\item [Parameters:]
\item {\sl node-ID}:  
integer (a valid hbserver node ID number)

\item {\sl node-data}:  string


\item [Return-value:]
A node-ID if successful.

Error object {\sl write-attribute-fails} (page~\pageref{write-attribute-fails}) if set-data
hb-operation fails.

\item [Description:]
Saves data to the persistent datastore on the remote
database server.


\item [Public:]



\end{description}
\horizontalline

\subsection{s*node*set-geometry}
\label{s*node*set-geometry}

\begin{description}
\item [Name:]  s*node*set-geometry

\item [Class:] {\sl s*node}\hfill(page~\pageref{s*node})

\item [Parameters:]
\item {\sl node-ID}:  
integer (a valid hbserver node ID number)

\item {\sl node-geometry}:  String
Contains a valid geometry specification.
 

\item [Return-value:]
NODE-GEOMETRY if successful.

Error object {\sl write-attribute-fails} (page~\pageref{write-attribute-fails}) if set geometry
hb-operation fails.

\item [Description:]
Resets the geometry attribute of node-ID.  
Involves a write out to the database.
Since geometry is inherently a kind of 
domain-specific idea, the server subsystem
does not checks for the validity of its
value. Instead, it treats it simply as
string. 

\item [Public:]



\end{description}
\horizontalline

\subsection{s*node*set-font}
\label{s*node*set-font}

\begin{description}
\item [Name:]  s*node*set-font

\item [Class:] {\sl s*node}\hfill(page~\pageref{s*node})

\item [Parameters:]
\item {\sl node-ID}:  
integer (a valid hbserver node ID number)

\item {\sl node-font}:  a string corresponding to a valid font name.


\item [Return-value:]
NODE-FONT if successful.

Error object {\sl write-attribute-fails} (page~\pageref{write-attribute-fails}) if set font
hb-operation fails.

\item [Description:]
Resets the font attribute.  Since this
is relatively domain-specific, the server
subsystem is not responsible for its
validity checking.

\item [Public:]



\end{description}
\horizontalline

\subsection{s*node@reset-incoming-links}
\label{s*node@reset-incoming-links}

\begin{description}
\item [Name:]  s*node@reset-incoming-links
\item [Class:]
{\sl s*node}\hfill(page~\pageref{s*node})

\item [Parameters:]
\item {\sl node-ID}:  
integer (a valid hbserver node ID number)


\item [Return-value:] 
list of link-IDs

\item [Description:]

Rebuilds the list of link-IDs that point to this node
by directly calling the remote server to traverse 
all node and link objects to determine which ones point
to this node. 
This is curruption repairing function.

\item [Public:]



\end{description}
\horizontalline

\subsection{s*node@reset-outgoing-links}
\label{s*node@reset-outgoing-links}

\begin{description}
\item [Name:]  s*node@reset-outgoing-links

\item [Class:]
{\sl s*node}\hfill(page~\pageref{s*node})

\item [Parameters:]
\item {\sl node-ID}:  
integer (a valid hbserver node ID number)
 

\item [Return-value:] 
List of link-IDs.

\item [Description:]
Rebuilds the list of link-IDs that point away from this
node by calling the remote server to traverse all nodes
and links determine which ones point away.  
This is a curruption recovery function.

\item [Public:]



\end{description}
\horizontalline

\section{s*link}
\label{s*link}

\begin{description}
\item [Name:]  s*link

\item [Layer:] {\sl Server}\hfill(page~\pageref{Server})

\item [Description:]
The S*LINK class implements a set of primitive
operations on links, including link creation, deletion,
link attribute retrieval and update, and so forth.
Most of these operations are straightforward
translation of the corresponding remote database server
functions. Because of the asymmetric handling of links
by the current server, S*LINK has to maintain a
separate structure to keep track of the source and
destination nodes of a link. Nevertheless, certain
operations, such as S*LINK*SET-SOURCE-NODE, can still
not be provided due to the built-in constraints on
links.

\item [Attributes:]
\item {\sl s*link*name}\hfill(page~\pageref{s*link*name})
\item {\sl s*link*created-by}\hfill(page~\pageref{s*link*created-by})
\item {\sl s*link*created-date}\hfill(page~\pageref{s*link*created-date})
\item {\sl s*link*last-modified-by}\hfill(page~\pageref{s*link*last-modified-by})
\item {\sl s*link*last-modified-date}\hfill(page~\pageref{s*link*last-modified-date})
\item {\sl s*link*source-node}\hfill(page~\pageref{s*link*source-node})
\item {\sl s*link*destination-node}\hfill(page~\pageref{s*link*destination-node})

\item [Operations:]
\item {\sl s*link*make}\hfill(page~\pageref{s*link*make})
\item {\sl s*link*delete}\hfill(page~\pageref{s*link*delete})
\item {\sl s*link*set-name}\hfill(page~\pageref{s*link*set-name})
\item {\sl s*link*set-destination-node}\hfill(page~\pageref{s*link*set-destination-node})

\item [Collections:]

\item [Subclasses:]

\item [Superclasses:]



\end{description}
\horizontalline

\subsection{s*link*name}
\label{s*link*name}

\begin{description}
\item [Name:]  s*link*name

\item [Class:] {\sl s*link}\hfill(page~\pageref{s*link})

\item [Contents:] string (30)

\item [Description:]
The name of the link; limited to thirty chars.

\item [Setf-able:]

\item [Public:]



\end{description}
\horizontalline

\subsection{s*link*created-by}
\label{s*link*created-by}

\begin{description}
\item [Name:]  s*link*created-by

\item [Class:] {\sl s*link}\hfill(page~\pageref{s*link})

\item [Contents:] string

\item [Description:] 
Name of the user who created this link.

\item [Setf-able:] no

\item [Public:]



\end{description}
\horizontalline

\subsection{s*link*created-date}
\label{s*link*created-date}

\begin{description}

\item [Name:]  s*link*created-date

\item [Class:] {\sl s*link}\hfill(page~\pageref{s*link})

\item [Contents:] string

\item [Description:] 
the creation date for this link

\item [Setf-able:]

\item [Public:]



\end{description}
\horizontalline

\subsection{s*link*last-modified-by}
\label{s*link*last-modified-by}

\begin{description}

\item [Name:]  s*link*last-modified-by

\item [Class:] {\sl s*link}\hfill(page~\pageref{s*link})

\item [Contents:] string

\item [Description:] 
The name of user who was last in modifying this
link.

\item [Setf-able:]

\item [Public:]



\end{description}
\horizontalline

\subsection{s*link*last-modified-date}
\label{s*link*last-modified-date}

\begin{description}

\item [Name:]  s*link*last-modified-date

\item [Class:] {\sl s*link}\hfill(page~\pageref{s*link})

\item [Contents:] string

\item [Description:] 
The date of the last modification of this link.

\item [Setf-able:]


\item [Public:]



\end{description}
\horizontalline

\subsection{s*link*source-node}
\label{s*link*source-node}

\begin{description}
\item [Name:]  s*link*source-node

\item [Class:] {\sl s*link}\hfill(page~\pageref{s*link})

\item [Contents:] node-ID

\item [Description:]
The node-ID of the source node of this link.

\item [Setf-able:]


\item [Public:]



\end{description}
\horizontalline

\subsection{s*link*destination-node}
\label{s*link*destination-node}

\begin{description}

\item [Name:]  s*link*destination-node

\item [Class:] {\sl s*link}\hfill(page~\pageref{s*link})

\item [Contents:] node-ID

\item [Description:] 
The destination node for this link.

\item [Setf-able:]

\item [Public:]



\end{description}
\horizontalline

\subsection{s*link*make}
\label{s*link*make}

\begin{description}
\item [Name:]  s*link*make

\item [Class:] {\sl s*link}\hfill(page~\pageref{s*link})

\item [Parameters:]
\item {\sl link-name}:  string (30); a valid link name

\item {\sl from-node-ID}:  node-ID

\item {\sl to-node-ID}:  node-ID




\item [Return-value:]
link-ID if successful.

error object {\sl invalid-link-name} (page~\pageref{invalid-link-name}) if LINK-NAME
contains leading space(s), or its length exceeds 30.

error object {\sl create-link-fails} (page~\pageref{create-link-fails}) if create-link
hb-calls fails. 

error object {\sl write-attribute-fails} (page~\pageref{write-attribute-fails}) if
initialization of link attributes fails.

error object {\sl update-link-info-fails} (page~\pageref{update-link-info-fails}) if update 
operations on LINK-INFO node fails.

\item [Description:]
This function validates link-name, calls remote server
to allocate a new link num; initializes
link-attributes: created by, created ate, last update
by, and last updated date; and runs s*link!make-hooks. 

\item [Public:]





\end{description}
\horizontalline

\subsection{s*link*delete}
\label{s*link*delete}

\begin{description}
\item [Name:]  s*link*delete
\item [Class:] {\sl s*link}\hfill(page~\pageref{s*link})

\item [Parameters:]
\item {\sl link-ID}:  
valid HB link ID number (integer)

\item {\sl from-node-ID}:  node-ID


\item [Return-value:]
LINK-ID if the link deletion was successful, or 
error object otherwise.

\item [Description:]
Permanently remove the link from remote database.


\item [Public:]




\end{description}
\horizontalline

\subsection{s*link*set-name}
\label{s*link*set-name}

\begin{description}
\item [Name:]  s*link*set-name

\item [Class:] {\sl s*link}\hfill(page~\pageref{s*link})

\item [Parameters:]
\item {\sl link-ID}:  
valid HB link ID number (integer)

\item {\sl link-name}:  string (30); a valid link name
 

\item [Return-value:]
LINK-NAME if successful.

error object {\sl invalid-link-name} (page~\pageref{invalid-link-name}) if LINK-NAME contains
leading space(s)/tab(s), or its length exceeds 30.

error object {\sl write-attribute-fails} (page~\pageref{write-attribute-fails}) if hb-write
operation fails.

\item [Description:]
Resets the name of the link. Note that
the link name does not necessarily
correspond to the link label.

\item [Public:]



\end{description}
\horizontalline

\subsection{s*link*set-destination-node}
\label{s*link*set-destination-node}

\begin{description}
\item [Name:]  s*link*set-destination-node

\item [Class:]
{\sl s*link}\hfill(page~\pageref{s*link})

\item [Parameters:]
\item {\sl node-ID}:  
integer (a valid hbserver node ID number)

\item {\sl link-ID}:  
valid HB link ID number (integer)


\item [Return-value:]
NODE-ID if the destination node of link LINK-ID can be
set to NODE-ID, or error object {\sl move-link-fails} (page~\pageref{move-link-fails})
otherwise. 

\item [Description:]
Reset the target node of the current link to another
node. Note that there is no reciprocal operation,
i.e., one cannot reset the source node of a link
without necessitating the creation of a new link.

\item [Public:]



\end{description}
\horizontalline

\section{s*server-process}
\label{s*server-process}

\begin{description}
\item [Name:]  s*server-process

\item [Layer:] {\sl Server}\hfill(page~\pageref{Server})

\item [Description:]
The server-process class implements connection and
communication operations and attribute functions
related  to the server process, e.g., the directory the
remote databases resides, the name of machine on which
the server is running, the description of the database,
etc. Currently, S*SERVER-PROCESS consists of only one
instance, i.e., hbserver. But the class is designed to
accommodate arbitrary number of server processes. 

\item [Attributes:]
\item {\sl s*sp*name}\hfill(page~\pageref{s*sp*name})
\item {\sl s*sp*description}\hfill(page~\pageref{s*sp*description})
\item {\sl s*sp*directory-path}\hfill(page~\pageref{s*sp*directory-path})
\item {\sl s*sp*ip}\hfill(page~\pageref{s*sp*ip})

\item [Operations:]
\item {\sl s*sp*connect}\hfill(page~\pageref{s*sp*connect})
\item {\sl s*sp*disconnect}\hfill(page~\pageref{s*sp*disconnect})

Administration Operations
\item {\sl s*sp@defserver}\hfill(page~\pageref{s*sp@defserver})

\item [Collections:]

\item [Subclasses:]

\item [Superclasses:]



\end{description}
\horizontalline

\subsection{s*sp*name}
\label{s*sp*name}

\begin{description}

\item [Name:]  s*sp*name

\item [Class:] {\sl s*server-process}\hfill(page~\pageref{s*server-process})

\item [Contents:] String 

\item [Description:]  
Name of server database. It is the same as machine name,
e.g. "uhics.ics.hawaii.edu".

\item [Setf-able:] 

\item [Public:]



\end{description}
\horizontalline

\subsection{s*sp*description}
\label{s*sp*description}

\begin{description}
\item [Name:]  s*sp*description

\item [Class:] {\sl s*server-process}\hfill(page~\pageref{s*server-process})

\item [Contents:] String

\item [Description:]

A short description of the contents or purpose of 
this database.

\item [Setf-able:]


\item [Public:]



\end{description}
\horizontalline

\subsection{s*sp*directory-path}
\label{s*sp*directory-path}

\begin{description}

\item [Name:]  s*sp*directory-path

\item [Class:] {\sl s*server-process}\hfill(page~\pageref{s*server-process})

\item [Contents:] String

\item [Description:]
The pathname to the directory where the datafiles for 
this database are kept.

\item [Setf-able:]


\item [Public:]



\end{description}
\horizontalline

\subsection{s*sp*ip}
\label{s*sp*ip}

\begin{description}
\item [Name:]  s*sp*ip

\item [Class:]
{\sl s*server-process}\hfill(page~\pageref{s*server-process})

\item [Contents:] string.

\item [Description:]
Contains full IP address, eg. "128.171.2.5"

\item [Setf-able:]

\item [Public:]



\end{description}
\horizontalline

\subsection{s*sp*connect}
\label{s*sp*connect}

\begin{description}
\item [Name:]  s*sp*connect

\item [Class:] {\sl s*server-process}\hfill(page~\pageref{s*server-process})

\item [Parameters:]
\item {\sl machine-name}:  
A string which is a valid internet machine address. 
	 

\item [Return-value:] 
t if the connection was made successfully.

error object {\sl missing-required-arg} (page~\pageref{missing-required-arg}) if MACHINE-NAME is
not supplied and public variable S*SP*CURRENT-SERVER is not
set. 

error object {\sl connection-is-on} (page~\pageref{connection-is-on}) if the user tries to 
connect while the connection is already on.

\item [Description:]
This function sets up three network stream processes:
read, write, and event; sets global variable
S*SP*CURRENT-SERVER; subscribes the initial set of
events; initializes local caches of nodes and links, and
runs S*SP!CONNECT-HOOKS.


\item [Public:]






\end{description}
\horizontalline

\subsection{s*sp*disconnect}
\label{s*sp*disconnect}

\begin{description}

\item [Name:]  s*sp*disconnect

\item [Class:] {\sl s*server-process}\hfill(page~\pageref{s*server-process})

\item [Parameters:] none

\item [Return-value:]
t if the disconnection from the server was successful.

Error object {\sl connection-is-off} (page~\pageref{connection-is-off}) if the user
attempts to disconnect while the connection is not on. 


\item [Description:]
This function deletes all three network processes (ie.,
read, write, and event) and sets global status variable
S*SP!CONNECTED to nil.

\item [Public:]




\end{description}
\horizontalline

\subsection{s*sp@defserver}
\label{s*sp@defserver}

\begin{description}
\item [Name:]  s*sp@defserver

\item [Class:]
{\sl s*server-process}\hfill(page~\pageref{s*server-process})

\item [Parameters:]
\item {\sl name}:  string, eg, "zero.ics".


\item {\sl description}:  string, e.g., "DesignBase", "Testbase". 


\item {\sl path}:  string. It must be absolute path name. 


\item {\sl ip}:  string. e.g., "128.71.4.4"



\item [Return-value:]
newly-defined server struct.

\item [Description:]
Defines a new instance of server structure.  Note that
the system administrator should predefine all servers
accessable to the user. Attempts to access an undefined
server is an error condition.

\item [Public:]



\end{description}
\horizontalline

\section{s*snode}
\label{s*snode}

\begin{description}
\item [Name:]  s*snode

\item [Layer:]
{\sl Server}\hfill(page~\pageref{Server})

\item [Description:]

The SYS-NODE is provided by the server subsystem to other EGRET modules
(including the server itself) as a uniform mechanism for handling a set
of special nodes which store data used by these modules in implementing
advanced functions. The major features of SYS-NODES in comparison with
the regular Egret nodes are:

 -- they are "hidden" from the users of the system;
 -- they are internal to the modules in which they are defined and used.
 -- they are susceptible to corruption and thus require recovery 
    operations;
 -- they are normally not deleted once being created; 
 -- operations on them are normally performed via special SYS-NODE
    specific event functions. A public function also exists to
    update the contents of the system node; 
 -- multiple subsystems can share the same system node. For example,
    unread node uses the done-node data defined by the to-do module.
 -- SYS-NODEs store and retrieve a single, variable length string,
    whose contents are interpreted by its utilizing systems; 
 -- unique IDs for system nodes (snode-ID) are not integers but a unique
    string; 
 -- concurrent access is maintained by the server subsystem. Users cannot
    explicitly lock or unlock a system node. All operations upon SYS-
    NODEs are assumed to require a brief amount of time; they are 
    reliably unlocked regardless of whether operations terminate 
    successfully or signal an error; and 
 -- definition and instantiation of a SYS-NODE are separate
    operations. Definition describes the event operations that will 
    run if the instance of the system node exists on the server. A 
    separate operation (s*snode!make) exists to actually create the entity.
\item [Attributes:]

\item [Operations:]
\item {\sl s*snode*define}\hfill(page~\pageref{s*snode*define})
\item {\sl s*snode*with-data}\hfill(page~\pageref{s*snode*with-data})
\item {\sl s*snode*with-data-locked}\hfill(page~\pageref{s*snode*with-data-locked})

\item {\sl s*snode@make}\hfill(page~\pageref{s*snode@make})

\item [Collections:]


\item [Subclasses:]


\item [Superclasses:]


\item [Instances:]



\end{description}
\horizontalline

\subsection{s*snode*define}
\label{s*snode*define}

\begin{description}
\item [Name:]  s*snode*define

\item [Class:]
{\sl s*snode}\hfill(page~\pageref{s*snode})

\item [Parameters:]
\item {\sl snode-name}:  string

\item {\sl data-event}:  symbol

\item {\sl connect}:  symbol

\item {\sl disconnect}:  symbol


\item [Return-value:]
Macro with side effect of defining an SNODE object

\item [Description:]
This macro defines a specialized system node class with a
single instance SNODE-NAME. It registers SNODE-NAME on the
global system node data structure. In addition, it defines
three operations for the newly created snode class:

\item [Public:]



\end{description}
\horizontalline

\subsection{s*snode*with-data}
\label{s*snode*with-data}

\begin{description}
\item [Name:]  s*snode*with-data

\item [Class:]
{\sl s*snode}\hfill(page~\pageref{s*snode})

\item [Parameters:]
\item {\sl snode-name}:  string

\item {\sl body}:  list of Lisp forms


\item [Return-value:] 
Return value of the last form in BODY.

\item [Description:]
Retrieves sys-node NODE-NAME and then executes BODY.
NODE-NAME must be a name of an existent node. 
BODY is a list of forms to be executed.  This macro is
evaluated for its side effects.

\item [Public:]




\end{description}
\horizontalline

\subsection{s*snode*with-data-locked}
\label{s*snode*with-data-locked}

\begin{description}
\item [Name:]  s*snode*with-data-locked

\item [Class:]
{\sl s*snode}\hfill(page~\pageref{s*snode})

\item [Parameters:]
\item {\sl snode-name}:  string

\item {\sl body}:  list of Lisp forms


\item [Return-value:]
Return value of last form in BODY or error object
{\sl lock-snode-fails} (page~\pageref{lock-snode-fails}) if s*snode-lock operation fails..

\item [Description:]
Retrieves sys-node NODE-NAME with lock and then
executes BODY.  NODE-NAME must be a name of an
existent node.  BODY is a list of forms to be
executed.  This macro is evaluated for its side
effects

\item [Public:]



\end{description}
\horizontalline

\subsection{s*snode@make}
\label{s*snode@make}

\begin{description}
\item [Name:]  s*snode@make

\item [Class:]
{\sl s*snode}\hfill(page~\pageref{s*snode})

\item [Parameters:]
\item {\sl snode-name}:  string

\item {\sl initial-value}:  string


\item [Return-value:] SNODE-NAME

\item [Description:]
Instantiates an snode, i.e., creating node in remote
database with name SNODE-NAME, and stores INITIAL-VALUE
in the node. Note this is an administrative function. If
SNODE-NAME already exists, re-initializes it to
INIT-VALUE.

\item [Public:]



\end{description}
\horizontalline

\section{s*serror}
\label{s*serror}

\begin{description}
\item [Name:]  s*serror

\item [Layer:]
{\sl Server}\hfill(page~\pageref{Server}) 

\item [Description:]
This class contains all error objects in the Server subsystem.
As a general rule, server level functions trap and report but
do not handle errors. It is the responsility of the calling
function to interpret various error objects and decide what
actions to take. Lower-level server functions use SIGNAL to
return to top-level server functions in case of error.
Top-level server functions must catch all server-level errors,
and guaranttee returning to the calling function with
approciate error objects.

\item [Attributes:] See U*ERROR

\item [Operations:] See U*ERROR

\item [Collections:]

\item [Subclasses:]

\item [Superclasses:]
\item {\sl u*error}\hfill(page~\pageref{u*error})

\item [Instances:]
\item {\sl lock-snode-fails}\hfill(page~\pageref{lock-snode-fails})
\item {\sl missing-required-arg}\hfill(page~\pageref{missing-required-arg})
\item {\sl write-attribute-fails}\hfill(page~\pageref{write-attribute-fails})
\item {\sl read-attribute-fails}\hfill(page~\pageref{read-attribute-fails})
\item {\sl get-entity-IDs-fails}\hfill(page~\pageref{get-entity-IDs-fails})
\item {\sl conflicting-hook-constraints}\hfill(page~\pageref{conflicting-hook-constraints})
\item {\sl unknown-hb-error}\hfill(page~\pageref{unknown-hb-error})
\item {\sl uninstantiated-system-node}\hfill(page~\pageref{uninstantiated-system-node})

\item {\sl parse-event-fails}\hfill(page~\pageref{parse-event-fails})
\item {\sl subscribe-event-fails}\hfill(page~\pageref{subscribe-event-fails})
\item {\sl unsubscribe-event-fails}\hfill(page~\pageref{unsubscribe-event-fails})

\item {\sl invalid-node-name}\hfill(page~\pageref{invalid-node-name})
\item {\sl show-lock-fails}\hfill(page~\pageref{show-lock-fails})
\item {\sl lock-node-fails}\hfill(page~\pageref{lock-node-fails})
\item {\sl create-node-fails}\hfill(page~\pageref{create-node-fails})
\item {\sl delete-node-fails}\hfill(page~\pageref{delete-node-fails})
\item {\sl node-still-referenced}\hfill(page~\pageref{node-still-referenced})
\item {\sl node-still-locked}\hfill(page~\pageref{node-still-locked})
\item {\sl node-not-found}\hfill(page~\pageref{node-not-found})

\item {\sl invalid-link-name}\hfill(page~\pageref{invalid-link-name})
\item {\sl move-link-fails}\hfill(page~\pageref{move-link-fails})
\item {\sl create-link-fails}\hfill(page~\pageref{create-link-fails})
\item {\sl link-not-found}\hfill(page~\pageref{link-not-found})
\item {\sl update-link-info-fails}\hfill(page~\pageref{update-link-info-fails})

\item {\sl connection-is-on}\hfill(page~\pageref{connection-is-on})
\item {\sl connection-is-off}\hfill(page~\pageref{connection-is-off})
\item {\sl server-not-found}\hfill(page~\pageref{server-not-found})









\end{description}
\horizontalline

\subsection{lock-snode-fails}
\label{lock-snode-fails}

\begin{description}
\item [Name:]  lock-snode-fails

\item [Class:]
{\sl s*serror}\hfill(page~\pageref{s*serror})

\item [Description:]
Attempt to lock a system node fails.


\end{description}
\horizontalline

\subsection{missing-required-arg}
\label{missing-required-arg}

\begin{description}
\item [Name:]  missing-required-arg


\item [Class:]
{\sl s*serror}\hfill(page~\pageref{s*serror})


\item [Description:] 
Required arg is either missing or unccaptable. 



\end{description}
\horizontalline

\subsection{write-attribute-fails}
\label{write-attribute-fails}

\begin{description}
\item [Name:]  write-attribute-fails


\item [Class:]
{\sl s*serror}\hfill(page~\pageref{s*serror})


\item [Description:] 
Data fails to be written to remote persistent store.


\end{description}
\horizontalline

\subsection{read-attribute-fails}
\label{read-attribute-fails}

\begin{description}

\item [Name:]  read-attribute-fails


\item [Class:]
{\sl s*serror}\hfill(page~\pageref{s*serror})


\item [Description:] 
Field data fails to be retrieved from remote database
server.



\end{description}
\horizontalline

\subsection{get-entity-IDs-fails}
\label{get-entity-IDs-fails}

\begin{description}
\item [Name:]  get-entity-IDs-fails

\item [Class:]
{\sl s*serror}\hfill(page~\pageref{s*serror})

\item [Description:]
Operation on retrieving the list of either all node or link
IDs from remote server fails.



\end{description}
\horizontalline

\subsection{conflicting-hook-constraints}
\label{conflicting-hook-constraints}

\begin{description}
\item [Name:]  conflicting-hook-constraints

\item [Class:]
{\sl s*serror}\hfill(page~\pageref{s*serror})

\item [Description:]
The supplied ordering constraints are conflicting with each
other and thus cannot be satisfied.

\end{description}
\horizontalline

\subsection{unknown-hb-error}
\label{unknown-hb-error}

\begin{description}

\item [Name:]  unknown-hb-error


\item [Class:]
{\sl s*serror}\hfill(page~\pageref{s*serror})


\item [Description:] 
Hyperbase operation fails due to some unknown reason(s).



\end{description}
\horizontalline

\subsection{uninstantiated-system-node}
\label{uninstantiated-system-node}

\begin{description}
\item [Name:]  uninstantiated-system-node

\item [Class:]
{\sl s*serror}\hfill(page~\pageref{s*serror})


\item [Description:] 
A SYS-NODE must be initialized before use. Fails to do so
would cause this error to be signaled.



\end{description}
\horizontalline

\subsection{parse-event-fails}
\label{parse-event-fails}

\begin{description}
\item [Name:]  parse-event-fails


\item [Class:]
{\sl s*serror}\hfill(page~\pageref{s*serror})


\item [Description:]
Incoming event string from the remote server cannot be
parsed.. 


\end{description}
\horizontalline

\subsection{subscribe-event-fails}
\label{subscribe-event-fails}

\begin{description}
\item [Name:]  subscribe-event-fails


\item [Class:]
{\sl s*serror}\hfill(page~\pageref{s*serror})


\item [Description:]
Operation for subscribing an event on remote server fails. 

\end{description}
\horizontalline

\subsection{unsubscribe-event-fails}
\label{unsubscribe-event-fails}

\begin{description}

\item [Name:]  unsubscribe-event-fails


\item [Class:]
{\sl s*serror}\hfill(page~\pageref{s*serror})


\item [Description:]
Attempt to unsubscribe an event previsouly subscribed
fails.


\end{description}
\horizontalline

\subsection{invalid-node-name}
\label{invalid-node-name}

\begin{description}

\item [Name:]  invalid-node-name


\item [Class:]
{\sl s*serror}\hfill(page~\pageref{s*serror})


\item [Description:]
Node naming violation, i.e., exceeding 40 characters or
containing leading space or tabs.


\end{description}
\horizontalline

\subsection{show-lock-fails}
\label{show-lock-fails}

\begin{description}
\item [Name:]  show-lock-fails


\item [Class:]
{\sl s*serror}\hfill(page~\pageref{s*serror})


\item [Description:]
cannot determine who has the lock to the requested node.


\end{description}
\horizontalline

\subsection{lock-node-fails}
\label{lock-node-fails}

\begin{description}

\item [Name:]  lock-node-fails


\item [Class:]
{\sl s*serror}\hfill(page~\pageref{s*serror})


\item [Description:]
cannot lock a node.


\end{description}
\horizontalline

\subsection{create-node-fails}
\label{create-node-fails}

\begin{description}
\item [Name:]  create-node-fails


\item [Class:]
{\sl s*serror}\hfill(page~\pageref{s*serror})


\item [Description:]
Attempt to create new node fails.

\end{description}
\horizontalline

\subsection{delete-node-fails}
\label{delete-node-fails}

\begin{description}
\item [Name:]  delete-node-fails


\item [Class:]
{\sl s*serror}\hfill(page~\pageref{s*serror})


\item [Description:]
Attempt to delete a node form remote persistent store fails.


\end{description}
\horizontalline

\subsection{node-still-referenced}
\label{node-still-referenced}

\begin{description}

\item [Name:]  node-still-referenced


\item [Class:]
{\sl s*serror}\hfill(page~\pageref{s*serror})


\item [Description:]
Attempts to delete a node which still has incoming links.


\end{description}
\horizontalline

\subsection{node-still-locked}
\label{node-still-locked}

\begin{description}
\item [Name:]  node-still-locked


\item [Class:]
{\sl s*serror}\hfill(page~\pageref{s*serror})


\item [Description:]
Trying to delete a node which is locked. 


\end{description}
\horizontalline

\subsection{node-not-found}
\label{node-not-found}

\begin{description}
\item [Name:]  node-not-found

\item [Class:]
{\sl s*serror}\hfill(page~\pageref{s*serror})

\item [Description:] requested node does not exist.



\end{description}
\horizontalline

\subsection{invalid-link-name}
\label{invalid-link-name}

\begin{description}
\item [Name:]  invalid-link-name


\item [Class:]
{\sl s*serror}\hfill(page~\pageref{s*serror})


\item [Description:]
Link naming violation, i.e., either the name exceeds 30
characters or contains leading space or tabs.


\end{description}
\horizontalline

\subsection{move-link-fails}
\label{move-link-fails}

\begin{description}
\item [Name:]  move-link-fails


\item [Class:]
{\sl s*serror}\hfill(page~\pageref{s*serror})

\item [Description:]
Attempt to reset the target node of a link fails.


\end{description}
\horizontalline

\subsection{create-link-fails}
\label{create-link-fails}

\begin{description}

\item [Name:]  create-link-fails


\item [Class:]
{\sl s*serror}\hfill(page~\pageref{s*serror})


\item [Description:]
Attempt to create new link fails.


\end{description}
\horizontalline

\subsection{link-not-found}
\label{link-not-found}

\begin{description}
\item [Name:]  link-not-found

\item [Class:]
{\sl s*serror}\hfill(page~\pageref{s*serror})

\item [Description:] requested link does not exist. 



\end{description}
\horizontalline

\subsection{update-link-info-fails}
\label{update-link-info-fails}

\begin{description}
\item [Name:]  update-link-info-fails

\item [Class:]
{\sl s*serror}\hfill(page~\pageref{s*serror})

\item [Description:] 
attempts to update the link-info system node fails.



\end{description}
\horizontalline

\subsection{connection-is-on}
\label{connection-is-on}

\begin{description}
\item [Name:]  connection-is-on


\item [Class:]
{\sl s*serror}\hfill(page~\pageref{s*serror})


\item [Description:]
Trying to connect to the remote server while the
current client is already connected.


\end{description}
\horizontalline

\subsection{connection-is-off}
\label{connection-is-off}

\begin{description}

\item [Name:]  connection-is-off


\item [Class:]
{\sl s*serror}\hfill(page~\pageref{s*serror})


\item [Description:]
Trying to disconnect while the client is already
disconnected from the remote server.


\end{description}
\horizontalline

\subsection{server-not-found}
\label{server-not-found}

\begin{description}
\item [Name:]  server-not-found

\item [Class:]
{\sl s*serror}\hfill(page~\pageref{s*serror})

\item [Description:] requested server is not defined.



\end{description}
\horizontalline

\section{s*event}
\label{s*event}

\begin{description}
\item [Name:]  s*event

\item [Layer:]
{\sl Server}\hfill(page~\pageref{Server})

\item [Description:]
S*EVENT represents a class of a predefined event types. It is
implemented in terms of U*HOOK, which in turn is an extension of
Emacs Lisp hook facility. S*EVENT allows arbitrary ordering
constraints to be imposed on functions to be intalled on any
given event dispatching queue.

S*EVENT is special in that all its instances are predefined; the
user of these events are not allowed to instantiated them, though
they may add functions to or delete functions from them, or
reinitialize them.  Currently, there are currently elevent (11)
event types, all of which are documentated below.  The execution
of functions on these event queues is triggered by events from
the remote database server, rather than directed by the Server
itself.

\item [Attributes:]
\item {\sl s*event*event-handlers}\hfill(page~\pageref{s*event*event-handlers})

\item [Operations:]
\item {\sl s*event*initialize}\hfill(page~\pageref{s*event*initialize})
\item {\sl s*event*add-event-handler-fn}\hfill(page~\pageref{s*event*add-event-handler-fn})
\item {\sl s*event*remove-event-handler-fn}\hfill(page~\pageref{s*event*remove-event-handler-fn})

\item [Collections:]

\item [Subclasses:]

\item [Superclasses:]

\item [Instances:]
\item {\sl s*node!newname-event-hooks}\hfill(page~\pageref{s*node!newname-event-hooks})
\item {\sl s*node!rename-event-hooks}\hfill(page~\pageref{s*node!rename-event-hooks})
\item {\sl s*node!delete-event-hooks}\hfill(page~\pageref{s*node!delete-event-hooks})
\item {\sl s*node!data-event-hooks}\hfill(page~\pageref{s*node!data-event-hooks})
\item {\sl s*node!lock-event-hooks}\hfill(page~\pageref{s*node!lock-event-hooks})
\item {\sl s*node!unlock-event-hooks}\hfill(page~\pageref{s*node!unlock-event-hooks})
\item {\sl s*node!font-event-hooks}\hfill(page~\pageref{s*node!font-event-hooks})
\item {\sl s*node!geometry-event-hooks}\hfill(page~\pageref{s*node!geometry-event-hooks})
\item {\sl s*link!newname-event-hooks}\hfill(page~\pageref{s*link!newname-event-hooks})
\item {\sl s*link!rename-event-hooks}\hfill(page~\pageref{s*link!rename-event-hooks})
\item {\sl s*link!delete-event-hooks}\hfill(page~\pageref{s*link!delete-event-hooks})












\end{description}
\horizontalline

\subsection{s*event*event-handlers}
\label{s*event*event-handlers}

\begin{description}
\item [Name:]  s*event*event-handlers

\item [Class:]
{\sl s*event}\hfill(page~\pageref{s*event})

\item [Contents:] List of functions

\item [Description:] 
An ordered list of functions to be invoked upon
receiving a given type of event.


\item [Setf-able:] no

\item [Public:]



\end{description}
\horizontalline

\subsection{s*event*initialize}
\label{s*event*initialize}

\begin{description}
\item [Name:]  s*event*initialize

\item [Class:]
{\sl s*event}\hfill(page~\pageref{s*event})

\item [Parameters:]
\item {\sl s*event-instance}:  symbol


\item [Return-value:] t

\item [Description:] Removes all current event handlers from 
S*EVENT-INSTANCE.

\item [Public:]



\end{description}
\horizontalline

\subsection{s*event*add-event-handler-fn}
\label{s*event*add-event-handler-fn}

\begin{description}
\item [Name:]  s*event*add-event-handler-fn

\item [Class:]
{\sl s*event}\hfill(page~\pageref{s*event})

\item [Parameters:]
\item {\sl s*event-instance}:  symbol

\item {\sl handler-fn-name}:  function symbol

\item {\sl before-handlers}:  functional symbol

\item {\sl after-handlers}:  function symbol


\item [Return-value:] 
List of function symbols or error object
CONFLICTING-HOOK-CONSTRAINTS. 

\item [Description:]
Inserts HANDLER-FN-NAME into S*EVENT S*EVENT-INSTANCE
so that BEFORE-HANDLERS and AFTER-HANDLERS constraintsw
are satisfied.

\item [Public:]



\end{description}
\horizontalline

\subsection{s*event*remove-event-handler-fn}
\label{s*event*remove-event-handler-fn}

\begin{description}
\item [Name:]  s*event*remove-event-handler-fn

\item [Class:]
{\sl s*event}\hfill(page~\pageref{s*event})

\item [Parameters:]
\item {\sl s*event-instance}:  symbol

\item {\sl handler-fn-name}:  function symbol


\item [Return-value:] Symbol of function being removed

\item [Description:] 
Removes function HANDLER-FN-NAME from S*EVENT-INSTANCE. 

\item [Public:]



\end{description}
\horizontalline

\subsection{s*node!newname-event-hooks}
\label{s*node!newname-event-hooks}

\begin{description}
\item [Name:]  s*node!newname-event-hooks

\item [Class:]
{\sl s*event}\hfill(page~\pageref{s*event})

\item [Description:]
Holds a list of functions executed upon receiving a
'new-node-name' event, i.e., with incoming event type 'n
name', entity type is 'node' and the target node can be
found in the local cache structure. 


\end{description}
\horizontalline

\subsection{s*node!rename-event-hooks}
\label{s*node!rename-event-hooks}

\begin{description}
\item [Name:]  s*node!rename-event-hooks

\item [Class:]
{\sl s*event}\hfill(page~\pageref{s*event})

\item [Description:]
Holds a list of functions executed upon receiving a
'rename-node-name' event, i.e., with incoming event
type 'n name', entity type is 'node' and the target
node cannot be found in the local cache structure.



\end{description}
\horizontalline

\subsection{s*node!delete-event-hooks}
\label{s*node!delete-event-hooks}

\begin{description}
\item [Name:]  s*node!delete-event-hooks

\item [Class:]
{\sl s*event}\hfill(page~\pageref{s*event})

\item [Description:]
Holds a list of functions executed upon receiving
delete-node event.  Each function is passed a node-ID.



\end{description}
\horizontalline

\subsection{s*node!data-event-hooks}
\label{s*node!data-event-hooks}

\begin{description}
\item [Name:]  s*node!data-event-hooks

\item [Class:]
{\sl s*event}\hfill(page~\pageref{s*event})

\item [Description:]
Holds a list of functions executed upon receiving a
'write data' event.



\end{description}
\horizontalline

\subsection{s*node!lock-event-hooks}
\label{s*node!lock-event-hooks}

\begin{description}

\item [Name:]  s*node!lock-event-hooks


\item [Class:]
{\sl s*event}\hfill(page~\pageref{s*event})


\item [Description:]
Holds a list of functions executed upon
receiving a 'lock' event.


\end{description}
\horizontalline

\subsection{s*node!unlock-event-hooks}
\label{s*node!unlock-event-hooks}

\begin{description}
\item [Name:]  s*node!unlock-event-hooks

\item [Class:]
{\sl s*event}\hfill(page~\pageref{s*event})

\item [Description:]
Holds a list of functions executed upon
receiving a 'unlock' event.


\end{description}
\horizontalline

\subsection{s*node!font-event-hooks}
\label{s*node!font-event-hooks}

\begin{description}
\item [Name:]  s*node!font-event-hooks

\item [Class:]
{\sl s*event}\hfill(page~\pageref{s*event})

\item [Description:]
Holds a list of functions executed upon receiving
a 'write-font' event.


\end{description}
\horizontalline

\subsection{s*node!geometry-event-hooks}
\label{s*node!geometry-event-hooks}

\begin{description}

\item [Name:]  s*node!geometry-event-hooks


\item [Class:]
{\sl s*event}\hfill(page~\pageref{s*event})


\item [Description:]
Holds a list of functions executed upon receiving a
'write-geometry' event.


\end{description}
\horizontalline

\subsection{s*link!newname-event-hooks}
\label{s*link!newname-event-hooks}

\begin{description}
\item [Name:]  s*link!newname-event-hooks

\item [Class:]
{\sl s*event}\hfill(page~\pageref{s*event})

\item [Description:]
Holds a list of functions executed upon receiving
a 'write-geometry' event.


\end{description}
\horizontalline

\subsection{s*link!rename-event-hooks}
\label{s*link!rename-event-hooks}

\begin{description}
\item [Name:]  s*link!rename-event-hooks

\item [Class:]
{\sl s*event}\hfill(page~\pageref{s*event})

\item [Description:]
Holds a list of functions executed upon receiving a
'new-link-name' event, i.e., with incoming event
type 'n name', entity type is 'link' and the target
link can be found in the local cache structure.


\end{description}
\horizontalline

\subsection{s*link!delete-event-hooks}
\label{s*link!delete-event-hooks}

\begin{description}
\item [Name:]  s*link!delete-event-hooks

\item [Class:]
{\sl s*event}\hfill(page~\pageref{s*event})

\item [Description:]
Holds a list of functions executed upon receiving
delete-link event.  Each function is passed link-ID.


\end{description}
\horizontalline

\chapter{Type}
\label{Type}

\begin{description}
\item [Name:]  Type

\item [Description:]

The Type subsystem defines a collaborative, extensible
data model on top of the fixed Server subsystem node
and link types.  

\item [Public-classes:]
\item {\sl t*node-schema}\hfill(page~\pageref{t*node-schema})
\item {\sl t*node-instance}\hfill(page~\pageref{t*node-instance})

\item {\sl t*link-schema}\hfill(page~\pageref{t*link-schema})
\item {\sl t*link-instance}\hfill(page~\pageref{t*link-instance})

\item {\sl t*field-schema}\hfill(page~\pageref{t*field-schema})

\item {\sl t*layer}\hfill(page~\pageref{t*layer})

\item {\sl t*error}\hfill(page~\pageref{t*error})
\item {\sl t*event}\hfill(page~\pageref{t*event})

\end{description}
\horizontalline

\section{t*node-schema}
\label{t*node-schema}

\begin{description}
\item [Name:]  t*node-schema

\item [Layer:] {\sl Type}\hfill(page~\pageref{Type})

\item [Description:]

Each instance of a node-schema defines the consensually 
agreed upon structural features (i.e. the set of fields)
for a set of node instances.  However, these instances
may not necessarily conform to these structural 
features. 

\item [Attributes:]
\item {\sl t*node-schema*name}\hfill(page~\pageref{t*node-schema*name})
\item {\sl t*node-schema*node-IDs}\hfill(page~\pageref{t*node-schema*node-IDs})
\item {\sl t*node-schema*field-IDs}\hfill(page~\pageref{t*node-schema*field-IDs})

\item [Operations:]
\item {\sl t*node-schema*delete}\hfill(page~\pageref{t*node-schema*delete})
\item {\sl t*node-schema*divergence}\hfill(page~\pageref{t*node-schema*divergence})
\item {\sl t*node-schema*set-name}\hfill(page~\pageref{t*node-schema*set-name})
\item {\sl t*node-schema*delete-fields}\hfill(page~\pageref{t*node-schema*delete-fields})
\item {\sl t*node-schema*add-fields}\hfill(page~\pageref{t*node-schema*add-fields})
\item {\sl t*node-schema*instantiate}\hfill(page~\pageref{t*node-schema*instantiate})
\item {\sl t*node-schema*make}\hfill(page~\pageref{t*node-schema*make})

\item [Subclasses:]


\item [Superclasses:]


\item [Instances:]




\end{description}
\horizontalline

\subsection{t*node-schema*name}
\label{t*node-schema*name}

\begin{description}
\item [Name:]  t*node-schema*name

\item [Class:] 

\item [Contents:] Symbol 

\item [Description:] The name of the node-schema

\item [Setf-able:] see t*node-schema*set-name

\item [Public:] yes.



\end{description}
\horizontalline

\subsection{t*node-schema*node-IDs}
\label{t*node-schema*node-IDs}

\begin{description}
\item [Name:]  t*node-schema*node-IDs

\item [Class:] {\sl t*node-schema}\hfill(page~\pageref{t*node-schema})

\item [Contents:] a list of node-IDs

\item [Description:]

This list of node-IDs are those that currently have
this node-schema as their associated consensual structure.

\item [Setf-able:] Updated automatically by type system. 

\item [Public:] yes



\end{description}
\horizontalline

\subsection{t*node-schema*field-IDs}
\label{t*node-schema*field-IDs}

\begin{description}
\item [Name:]  t*node-schema*field-IDs

\item [Class:] {\sl t*node-schema}\hfill(page~\pageref{t*node-schema})

\item [Contents:] a list of field-schema-IDs

\item [Description:]

Returns a list of the field-schema-IDs currently 
associated with this node-schema.

\item [Setf-able:] See t*node-schema*add-fields and 
t*node-schema*delete-fields
 
\item [Public:] yes



\end{description}
\horizontalline

\subsection{t*node-schema*delete}
\label{t*node-schema*delete}

\begin{description}
\item [Name:]  t*node-schema*delete

\item [Class:] {\sl t*node-schema}\hfill(page~\pageref{t*node-schema})

\item [Parameters:]
\item {\sl node-schema-ID}:  a server-level node-ID that corresponds to an 
instance of a type-level node-schema. (???)



\item [Return-value:]
NODE-SCHEMA-ID if successful. 

Error object {\sl invalid-node-schema-ID} (page~\pageref{invalid-node-schema-ID}) if bad 
node-schema-ID. 

Error object {\sl unknown-hb-error} (page~\pageref{unknown-hb-error}) if this call
fails for some other reason.

\item [Description:]

This operation "marks" node-schema-ID for deletion, 
thus ensuring that no future use of node-schema-ID 
will occur (once this deletion event has been propogated
to all other connected users.)  Node-schema-ID will
continue to exist in the database and its prior 
references and instances will continue to exist. 

More specifically, a deleted node-schema-ID will be
considered an invalid argument to the node-schema
operations: instantiate, add-fields, delete-fields,
set-name, and delete. The node-instance operation clone
will also be disabled for instances of this node-schema.
However, the attributes and the divergence operation will
continue to operate normally when passed this
node-schema-ID.

\item [Public:]



\end{description}
\horizontalline

\subsection{t*node-schema*divergence}
\label{t*node-schema*divergence}

\begin{description}
\item [Name:]  t*node-schema*divergence

\item [Class:] {\sl t*node-schema}\hfill(page~\pageref{t*node-schema})

\item [Parameters:]
\item {\sl node-schema-ID}:  a server-level node-ID that corresponds to an 
instance of a type-level node-schema. (???)



\item [Return-value:]
An integer corresponding to the divergence metric for 
this node-schema and its instances if successful.

Error object {\sl invalid-node-schema-ID} (page~\pageref{invalid-node-schema-ID}) if bad 
node-schema-ID.

Error object {\sl unknown-hb-error} (page~\pageref{unknown-hb-error}) if this computation
fails for some other reason.

\item [Description:]

Computes the structural divergence of the instances
of this node schema.

\item [Public:]



\end{description}
\horizontalline

\subsection{t*node-schema*set-name}
\label{t*node-schema*set-name}

\begin{description}
\item [Name:]  t*node-schema*set-name

\item [Class:] {\sl t*node-schema}\hfill(page~\pageref{t*node-schema})

\item [Parameters:]
\item {\sl node-schema-ID}:  a server-level node-ID that corresponds to an 
instance of a type-level node-schema. (???)


\item {\sl node-name}:  
A valid node name. This currently means that it is a
string of less than 40 characters, and that it does
not contain leading space(s) or tabs.


\item [Return-value:]
t if successful.

Error object {\sl invalid-node-schema-ID} (page~\pageref{invalid-node-schema-ID}) if bad 
node-schema-ID. 

Error object {\sl invalid-node-name} (page~\pageref{invalid-node-name}) if node-name is 
syntactically incorrect.

Error object {\sl unknown-hb-error} (page~\pageref{unknown-hb-error}) if this call
fails for some other reason.

\item [Description:]
Renames the node-schema-ID to node-name.  
(Note that node names need not be unique at the 
type level.)

\item [Public:]



\end{description}
\horizontalline

\subsection{t*node-schema*delete-fields}
\label{t*node-schema*delete-fields}

\begin{description}
\item [Name:]  t*node-schema*delete-fields

\item [Class:] {\sl t*node-schema}\hfill(page~\pageref{t*node-schema})

\item [Parameters:]
\item {\sl node-schema-ID}:  a server-level node-ID that corresponds to an 
instance of a type-level node-schema. (???)


\item {\sl field-schema-IDs}:  list of field-schema-ID
 

\item [Return-value:]
t if successful.
Error object {\sl invalid-node-schema-ID} (page~\pageref{invalid-node-schema-ID}) if
node-schema-ID was not a t*node-schema.

Error object {\sl invalid-field-ID} (page~\pageref{invalid-field-ID}) if field-schema-IDs 
are invalid or are not present in node-schema-ID.

Error object {\sl unknown-hb-error} (page~\pageref{unknown-hb-error}) if the call fails
for some other reason.

\item [Description:]

Removes one or more fields from node-schema-ID. 

\item [Public:]



\end{description}
\horizontalline

\subsection{t*node-schema*add-fields}
\label{t*node-schema*add-fields}

\begin{description}
\item [Name:]  t*node-schema*add-fields

\item [Class:] {\sl t*node-schema}\hfill(page~\pageref{t*node-schema})

\item [Parameters:]
\item {\sl node-schema-ID}:  a server-level node-ID that corresponds to an 
instance of a type-level node-schema. (???)


\item {\sl field-schema-IDs}:  list of field-schema-ID


\item [Return-value:] 
T if successful.

Error object {\sl invalid-node-schema-ID} (page~\pageref{invalid-node-schema-ID}) if 
node-schema-ID was not a t*node-schema.

Error object {\sl invalid-field-ID} (page~\pageref{invalid-field-ID}) if one of the 
field-schema-IDs was not a t*field-schema.

Error object {\sl unknown-hb-error} (page~\pageref{unknown-hb-error}) if the call 
fails for some other reason. 

\item [Description:]  Adds one or more field-schema-IDs to 
this node-schema.

\item [Public:]



\end{description}
\horizontalline

\subsection{t*node-schema*instantiate}
\label{t*node-schema*instantiate}

\begin{description}
\item [Name:]  t*node-schema*instantiate

\item [Class:] {\sl t*node-schema}\hfill(page~\pageref{t*node-schema})

\item [Parameters:]
\item {\sl node-schema-ID}:  a server-level node-ID that corresponds to an 
instance of a type-level node-schema. (???)



\item [Return-value:] 
node-ID if successful.

Error object {\sl invalid-node-schema-ID} (page~\pageref{invalid-node-schema-ID}) if 
node-schema-ID was not a t*node-schema.

Error object {\sl unknown-hb-error} (page~\pageref{unknown-hb-error}) if call fails
for some other reason.

\item [Description:]

Creates a new type-level node-instance based upon this
node-schema.

\item [Public:]



\end{description}
\horizontalline

\subsection{t*node-schema*make}
\label{t*node-schema*make}

\begin{description}
\item [Name:]  t*node-schema*make

\item [Class:] {\sl t*node-schema}\hfill(page~\pageref{t*node-schema})

\item [Parameters:]
\item {\sl node-name}:  
A valid node name. This currently means that it is a
string of less than 40 characters, and that it does
not contain leading space(s) or tabs.

\item {\sl field-schema-IDs}:  list of field-schema-ID


\item [Return-value:] 
node-schema-ID if successful.

Error object {\sl invalid-node-name} (page~\pageref{invalid-node-name}) if the node-name
was syntactically illegal.

Error object {\sl invalid-field-ID} (page~\pageref{invalid-field-ID}) if one or more of the
field-IDs was not a t*field-schema.

Error object {\sl unknown-hb-error} (page~\pageref{unknown-hb-error}) if the call fails
for some other reason.
 
\item [Description:]

Creates a new node-schema with field-IDs as its
structure, and returns the corresponding node-schema-ID.

\item [Public:]



\end{description}
\horizontalline

\section{t*node-instance}
\label{t*node-instance}

\begin{description}
\item [Name:]  t*node-instance

\item [Layer:] {\sl Type}\hfill(page~\pageref{Type})

\item [Description:]

Each instance of this class corresponds to an actual
content-bearing node in Egret.  Each node instance has
an associated schema, which represents the current
consensus in the group about the appropriate
field-level structure for this node.  This agreement
may or may not correspond to the actual field-level
structure of any particular node-instance.

Note that node-instance is abbreviated to "node" in
the operation and attribute names. 

\item [Attributes:]
\item {\sl t*node*schema-ID}\hfill(page~\pageref{t*node*schema-ID})
\item {\sl t*node*field-schema-IDs}\hfill(page~\pageref{t*node*field-schema-IDs})
\item {\sl t*node*incoming-link-IDs}\hfill(page~\pageref{t*node*incoming-link-IDs})
\item {\sl t*node*outgoing-link-IDs}\hfill(page~\pageref{t*node*outgoing-link-IDs})
\item {\sl t*node*layer-IDs}\hfill(page~\pageref{t*node*layer-IDs})

\item [Operations:]
\item {\sl t*node*clone}\hfill(page~\pageref{t*node*clone})
\item {\sl t*node*delete}\hfill(page~\pageref{t*node*delete})
\item {\sl t*node*set-schema-ID}\hfill(page~\pageref{t*node*set-schema-ID})
\item {\sl t*node*add-field-schema-IDs}\hfill(page~\pageref{t*node*add-field-schema-IDs})
\item {\sl t*node*delete-field-schema-IDs}\hfill(page~\pageref{t*node*delete-field-schema-IDs})
\item {\sl t*node*add-layer-ID}\hfill(page~\pageref{t*node*add-layer-ID})
\item {\sl t*node*delete-layer-ID}\hfill(page~\pageref{t*node*delete-layer-ID})
\item {\sl t*node*field-values}\hfill(page~\pageref{t*node*field-values})
\item {\sl t*node*set-field-values}\hfill(page~\pageref{t*node*set-field-values})
\item {\sl t*node*lock}\hfill(page~\pageref{t*node*lock})
\item {\sl t*node*unlock}\hfill(page~\pageref{t*node*unlock})
\item {\sl t*node*convergence}\hfill(page~\pageref{t*node*convergence})
 

\item [Subclasses:]


\item [Superclasses:]


\item [Instances:]



\end{description}
\horizontalline

\subsection{t*node*schema-ID}
\label{t*node*schema-ID}

\begin{description}
\item [Name:]  t*node*schema-ID

\item [Class:] {\sl t*node-instance}\hfill(page~\pageref{t*node-instance})

\item [Contents:] a node-schema-ID

\item [Description:]

The consensual schema for this node.

\item [Setf-able:]


\item [Public:]



\end{description}
\horizontalline

\subsection{t*node*field-schema-IDs}
\label{t*node*field-schema-IDs}

\begin{description}
\item [Name:]  t*node*field-schema-IDs

\item [Class:] {\sl t*node-instance}\hfill(page~\pageref{t*node-instance})

\item [Contents:] a list of field-schema-IDs

\item [Description:]

The internal structure of the node instance. 

\item [Setf-able:]


\item [Public:]



\end{description}
\horizontalline

\subsection{t*node*incoming-link-IDs}
\label{t*node*incoming-link-IDs}

\begin{description}
\item [Name:]  t*node*incoming-link-IDs

\item [Class:] {\sl t*node-instance}\hfill(page~\pageref{t*node-instance})

\item [Contents:] a list of link-IDs

\item [Description:]

A list of the link-IDs pointing to this node.

\item [Setf-able:]


\item [Public:]



\end{description}
\horizontalline

\subsection{t*node*outgoing-link-IDs}
\label{t*node*outgoing-link-IDs}

\begin{description}
\item [Name:]  t*node*outgoing-link-IDs

\item [Class:] {\sl t*node-instance}\hfill(page~\pageref{t*node-instance})

\item [Contents:] a list of link-IDs

\item [Description:]

A list of the link-IDs outgoing from this node.

\item [Setf-able:]


\item [Public:]



\end{description}
\horizontalline

\subsection{t*node*layer-IDs}
\label{t*node*layer-IDs}

\begin{description}
\item [Name:]  t*node*layer-IDs

\item [Class:] {\sl t*node-instance}\hfill(page~\pageref{t*node-instance})

\item [Contents:] a list of layer-IDs

\item [Description:]

The list of layers to which this node instance 
currently belongs. 

\item [Setf-able:]


\item [Public:]



\end{description}
\horizontalline

\subsection{t*node*clone}
\label{t*node*clone}

\begin{description}
\item [Name:]  t*node*clone

\item [Class:] {\sl t*node-instance}\hfill(page~\pageref{t*node-instance})

\item [Parameters:]
\item {\sl node-ID}:  
integer (a valid hbserver node ID number)


\item [Return-value:]
A new node-ID corresponding to the cloned instance 
if successful.

Error object {\sl invalid-node-ID} (page~\pageref{invalid-node-ID}) if node-ID is not
an instance of t*node-instance.

Error object {\sl unknown-hb-error} (page~\pageref{unknown-hb-error}) if the call fails
for some other reason. 

\item [Description:]

Creates and initializes a new t*node-instance with 
the node-schema-ID and field-schema-ID structure of 
node-ID. However, it does not copy the contents of 
node-ID.

\item [Public:]



\end{description}
\horizontalline

\subsection{t*node*delete}
\label{t*node*delete}

\begin{description}
\item [Name:]  t*node*delete

\item [Class:] {\sl t*node-instance}\hfill(page~\pageref{t*node-instance})

\item [Parameters:]
\item {\sl node-ID}:  
integer (a valid hbserver node ID number)


\item [Return-value:]
node-ID if successfully deleted.

Error object {\sl invalid-node-ID} (page~\pageref{invalid-node-ID}) if node-ID is 
not an instance of t*node-instance.

Error object {\sl node-still-locked} (page~\pageref{node-still-locked}) if node is locked
by some other user. 

Error object {\sl node-still-referenced} (page~\pageref{node-still-referenced}) if node
has incoming links. 

Error object {\sl unknown-hb-error} (page~\pageref{unknown-hb-error}) if call fails
for some other reason.

\item [Description:]

Deletes node-ID from the hyperbase.

\item [Public:]



\end{description}
\horizontalline

\subsection{t*node*set-schema-ID}
\label{t*node*set-schema-ID}

\begin{description}
\item [Name:]  t*node*set-schema-ID

\item [Class:] {\sl t*node-instance}\hfill(page~\pageref{t*node-instance})

\item [Parameters:]
\item {\sl node-ID}:  
integer (a valid hbserver node ID number)

\item {\sl node-schema-ID}:  a server-level node-ID that corresponds to an 
instance of a type-level node-schema. (???)



\item [Return-value:]
T if the schema associated with node-ID is updated to 
node-schema-ID.

Error object {\sl invalid-node-ID} (page~\pageref{invalid-node-ID}) if bad node-ID.

Error object {\sl invalid-node-schema-ID} (page~\pageref{invalid-node-schema-ID}) if bad
node-schema-ID.

Error object {\sl lock-node-fails} (page~\pageref{lock-node-fails}) if a lock can't be
obtained in preparation for the update. 

Error object {\sl unknown-hb-error} (page~\pageref{unknown-hb-error}) if call fails for
some other reason.

\item [Description:]

Sets the schema of node-ID to node-schema-ID. Note
that this does not change the existing structure of
node-ID at all. (It does potentially change the
convergence and divergence metric values for this
node schema and instance.)


\item [Public:]





\end{description}
\horizontalline

\subsection{t*node*add-field-schema-IDs}
\label{t*node*add-field-schema-IDs}

\begin{description}
\item [Name:]  t*node*add-field-schema-IDs

\item [Class:] {\sl t*node-instance}\hfill(page~\pageref{t*node-instance})

\item [Parameters:]
\item {\sl node-ID}:  
integer (a valid hbserver node ID number)

\item {\sl field-schema-IDs}:  list of field-schema-ID


\item [Return-value:]
T if the list of field-schema-IDs were successfully added.

Error object {\sl invalid-node-ID} (page~\pageref{invalid-node-ID}) if node-ID is not an
instance of t*node-instance.

Error object {\sl invalid-field-ID} (page~\pageref{invalid-field-ID}) if any of 
field-schema-IDs are invalid or are already members of 
node-ID.

Error object {\sl lock-node-fails} (page~\pageref{lock-node-fails}) if a lock on node-ID
cannot be obtained in preparation for the update.

Error object {\sl unknown-hb-error} (page~\pageref{unknown-hb-error}) if the call fails
for any other reason. 

\item [Description:]

Adds the list of field-schema-IDs to the structure of
node-ID.

\item [Public:]



\end{description}
\horizontalline

\subsection{t*node*delete-field-schema-IDs}
\label{t*node*delete-field-schema-IDs}

\begin{description}
\item [Name:]  t*node*delete-field-schema-IDs

\item [Class:] {\sl t*node-instance}\hfill(page~\pageref{t*node-instance})

\item [Parameters:]
\item {\sl node-ID}:  
integer (a valid hbserver node ID number)

\item {\sl field-schema-IDs}:  list of field-schema-ID


\item [Return-value:]
T if the set of field-schema-IDs (and their associated
contents, if any) were successfully deleted from node-ID.

Error object {\sl invalid-node-ID} (page~\pageref{invalid-node-ID}) if node-ID is not
a t*node-instance.

Error object {\sl invalid-field-ID} (page~\pageref{invalid-field-ID}) if any of the 
field-schema-IDs are not legal members of this
node-ID.

Error object {\sl lock-node-fails} (page~\pageref{lock-node-fails}) if node-ID cannot
be locked in preparation for the deletion. 

Error object {\sl unknown-hb-error} (page~\pageref{unknown-hb-error}) if call fails
for any other reason.

\item [Description:]

Deletes fields from an individual node instance. 

\item [Public:]



\end{description}
\horizontalline

\subsection{t*node*add-layer-ID}
\label{t*node*add-layer-ID}

\begin{description}
\item [Name:]  t*node*add-layer-ID

\item [Class:] {\sl t*node-instance}\hfill(page~\pageref{t*node-instance})

\item [Parameters:]
\item {\sl node-ID}:  
integer (a valid hbserver node ID number)

\item {\sl layer-ID}:  a unique ID for layers (possibly a node-ID?)



\item [Return-value:]
Layer-ID if successful.

Error object {\sl invalid-node-ID} (page~\pageref{invalid-node-ID}) if node-ID is not
a t*node-instance.

Error object {\sl invalid-layer-ID} (page~\pageref{invalid-layer-ID}) if layer-ID is
not the ID of a layer instance, or if it already
contains this node-ID as a member.

Error object {\sl lock-node-fails} (page~\pageref{lock-node-fails}) if a lock on
node-ID cannot be obtained in preparation for this
update.

Error object {\sl unknown-hb-error} (page~\pageref{unknown-hb-error}) if call fails
for any other reason.

\item [Description:]

Makes node-ID a member of layer-ID.

\item [Public:]



\end{description}
\horizontalline

\subsection{t*node*delete-layer-ID}
\label{t*node*delete-layer-ID}

\begin{description}
\item [Name:]  t*node*delete-layer-ID

\item [Class:] {\sl t*node-instance}\hfill(page~\pageref{t*node-instance})

\item [Parameters:]
\item {\sl node-ID}:  
integer (a valid hbserver node ID number)

\item {\sl layer-ID}:  a unique ID for layers (possibly a node-ID?)



\item [Return-value:]
The deleted layer-ID if successful.

Error object {\sl invalid-node-ID} (page~\pageref{invalid-node-ID}) if node-ID is
not a t*node-instance.

Error object {\sl invalid-layer-ID} (page~\pageref{invalid-layer-ID}) if bad layer-ID,
or if node-ID is not currently a member of layer-ID.

Error object {\sl lock-node-fails} (page~\pageref{lock-node-fails}) if a lock cannot
be obtained on node-ID (and perhaps layer-ID?).

Error object {\sl unknown-hb-error} (page~\pageref{unknown-hb-error}) if call fails
for any other reason.


\item [Description:]
Deletes this instance from the associated layer.

\item [Public:]



\end{description}
\horizontalline

\subsection{t*node*field-values}
\label{t*node*field-values}

\begin{description}
\item [Name:]  t*node*field-values

\item [Class:] {\sl t*node-instance}\hfill(page~\pageref{t*node-instance})

\item [Parameters:]
\item {\sl node-ID}:  
integer (a valid hbserver node ID number)

\item {\sl field-schema-IDs}:  list of field-schema-ID

\item {\sl buffer-instance}:  an Emacs buffer instance object


\item [Return-value:]
Returns a {\sl field-values} (page~\pageref{field-values}) object containing the 
contents of the field-schema-IDs if successful.

Error object {\sl invalid-buffer-instance} (page~\pageref{invalid-buffer-instance}) if bad 
buffer instance.

Error object {\sl invalid-node-ID} (page~\pageref{invalid-node-ID}) if node-ID is
not a t*node-instance.

Error object {\sl invalid-field-ID} (page~\pageref{invalid-field-ID}) if any of the 
field-schema-IDs are bad.

Error object {\sl unknown-hb-error} (page~\pageref{unknown-hb-error}) if call fails
for any other reason.

\item [Description:]

Obtains from the hyperbase and returns the contents
of each of the field-schema-IDs in node-ID. If the
optional buffer-instance object is supplied, the
returned value is that object with the field-values
object stored inside. Note that any prior contents
of the buffer-instance are erased.

\item [Public:]



\end{description}
\horizontalline

\subsection{t*node*set-field-values}
\label{t*node*set-field-values}

\begin{description}
\item [Name:]  t*node*set-field-values

\item [Class:] {\sl t*node-instance}\hfill(page~\pageref{t*node-instance})

\item [Parameters:]
\item {\sl node-ID}:  
integer (a valid hbserver node ID number)

\item {\sl field-values}:  a list of field-schema-IDs and their 
associated field-values.  Field-values are 
represented in the following form:

({<text string> | <link-ID>}*) 

In other words, a list of (normally alternating) text
strings and link-IDs.  Note that embedded quotation marks
in the text strings must be back-slashed to distinguish
them from the delimiting quotation marks. 

\item {\sl buffer-instance}:  an Emacs buffer instance object


\item [Return-value:]
T if node-ID is updated with field-values.

Error object {\sl invalid-node-ID} (page~\pageref{invalid-node-ID}) if node-ID
is not a t*node-instance.

Error object {\sl invalid-field-ID} (page~\pageref{invalid-field-ID}) if any of the 
field-schema-IDs specified in field-values was bad.

Error object {\sl invalid-buffer-instance} (page~\pageref{invalid-buffer-instance}) if the 
buffer-instance argument was bad.

Error object {\sl lock-node-fails} (page~\pageref{lock-node-fails}) if lock cannot 
be obtained on node-ID in preparation for the update.

Error object {\sl unknown-hb-error} (page~\pageref{unknown-hb-error}) if call fails
for some other reason.

\item [Description:]
Updates the contents of NODE-ID to the values provided
in FIELD-VALUES. The optional parameter buffer-instance,
when supplied by the caller, is an Emacs buffer-instance
containing the field-values argument.  In this case, 
the field-values parameter is not used. 

See the field-values parameter description for the
structural specification for this data. Note that this
representation will hopefully be extended and generalized
in future versions to support other (i.e. non-textual and
non-link) forms of data.

\item [Public:]



\end{description}
\horizontalline

\subsection{t*node*lock}
\label{t*node*lock}

\begin{description}
\item [Name:]  t*node*lock

\item [Class:] {\sl t*node-instance}\hfill(page~\pageref{t*node-instance})

\item [Parameters:]
\item {\sl node-ID}:  
integer (a valid hbserver node ID number)


\item [Return-value:]
T if the lock can be successfully obtained, or if
user already has a lock on node-ID.

NIL if another user has locked node-ID.

Error object {\sl invalid-node-ID} (page~\pageref{invalid-node-ID}) if node-ID is
not a t*node-instance.

Error object {\sl unknown-hb-error} (page~\pageref{unknown-hb-error}) if call fails
for some other reason.

\item [Description:]

Attempts to obtain a lock on node-ID. 

\item [Public:]



\end{description}
\horizontalline

\subsection{t*node*unlock}
\label{t*node*unlock}

\begin{description}
\item [Name:]  t*node*unlock

\item [Class:] {\sl t*node-instance}\hfill(page~\pageref{t*node-instance})

\item [Parameters:]
\item {\sl node-ID}:  
integer (a valid hbserver node ID number)


\item [Return-value:] 
T if node-ID is successfully unlocked.

Error object {\sl invalid-node-ID} (page~\pageref{invalid-node-ID}) if node-ID is
not a t*node-instance.

Error object {\sl unknown-hb-error} (page~\pageref{unknown-hb-error}) if the call
fails for some other reason.

\item [Description:]

Releases the lock (if present) on node-ID by user.

Will succeed when node-ID is already unlocked.

Will fail if node-ID is locked by another user.


\item [Public:]



\end{description}
\horizontalline

\subsection{t*node*convergence}
\label{t*node*convergence}

\begin{description}
\item [Name:]  t*node*convergence

\item [Class:] {\sl t*node-instance}\hfill(page~\pageref{t*node-instance})

\item [Parameters:]
\item {\sl node-ID}:  
integer (a valid hbserver node ID number)


\item [Return-value:] 
Integer convergence value if successful.

Error object {\sl invalid-node-ID} (page~\pageref{invalid-node-ID}) if node-ID is not a 
valid t*node-instance.

Error object {\sl unknown-hb-error} (page~\pageref{unknown-hb-error}) if the call fails
for some other reason.

\item [Description:]

Computes and returns an integer value representing 
the degree of convergence between the node
instance and its schema.

\item [Public:]



\end{description}
\horizontalline

\section{t*link-schema}
\label{t*link-schema}

\begin{description}
\item [Name:]  t*link-schema

\item [Layer:] {\sl Type}\hfill(page~\pageref{Type})

\item [Description:]

This class defines the structural properties of link
types.

\item [Attributes:]
\item {\sl t*link-schema*name}\hfill(page~\pageref{t*link-schema*name})
\item {\sl t*link-schema*to-nodes}\hfill(page~\pageref{t*link-schema*to-nodes})
\item {\sl t*link-schema*from-nodes}\hfill(page~\pageref{t*link-schema*from-nodes})
\item {\sl t*link-schema*link-IDs}\hfill(page~\pageref{t*link-schema*link-IDs})

\item [Operations:]
\item {\sl t*link-schema*make}\hfill(page~\pageref{t*link-schema*make})
\item {\sl t*link-schema*instantiate}\hfill(page~\pageref{t*link-schema*instantiate})
\item {\sl t*link-schema*add-to-node}\hfill(page~\pageref{t*link-schema*add-to-node})
\item {\sl t*link-schema*add-from-node}\hfill(page~\pageref{t*link-schema*add-from-node})
\item {\sl t*link-schema*delete-to-node}\hfill(page~\pageref{t*link-schema*delete-to-node})
\item {\sl t*link-schema*delete-from-node}\hfill(page~\pageref{t*link-schema*delete-from-node})
\item {\sl t*link-schema*set-name}\hfill(page~\pageref{t*link-schema*set-name})
\item {\sl t*link-schema*divergence}\hfill(page~\pageref{t*link-schema*divergence})
\item {\sl t*link-schema*delete}\hfill(page~\pageref{t*link-schema*delete})


\item [Subclasses:]


\item [Superclasses:]


\item [Instances:]



\end{description}
\horizontalline

\subsection{t*link-schema*name}
\label{t*link-schema*name}

\begin{description}
\item [Name:]  t*link-schema*name

\item [Class:] {\sl t*link-schema}\hfill(page~\pageref{t*link-schema})

\item [Contents:] a symbol

\item [Description:]

The name of the link-schema.

\item [Setf-able:]


\item [Public:]



\end{description}
\horizontalline

\subsection{t*link-schema*to-nodes}
\label{t*link-schema*to-nodes}

\begin{description}
\item [Name:]  t*link-schema*to-nodes

\item [Class:] {\sl t*link-schema}\hfill(page~\pageref{t*link-schema})

\item [Contents:] A link constraint expression.

\item [Description:]

An expression of the form \{T |(<node-schema-ID>*)\}.
T indicates that any node-schema-ID is valid; otherwise,
the list indicates the set of legal node-schema-IDs.

\item [Setf-able:]


\item [Public:]



\end{description}
\horizontalline

\subsection{t*link-schema*from-nodes}
\label{t*link-schema*from-nodes}

\begin{description}
\item [Name:]  t*link-schema*from-nodes

\item [Class:] {\sl t*link-schema}\hfill(page~\pageref{t*link-schema})

\item [Contents:] A link constraint expression

\item [Description:]

An expression of the form \{T |(<node-schema-ID>*)\}.
T indicates that any node-schema-ID is valid; otherwise,
the list indicates the set of legal node-schema-IDs.

\item [Setf-able:]


\item [Public:]



\end{description}
\horizontalline

\subsection{t*link-schema*link-IDs}
\label{t*link-schema*link-IDs}

\begin{description}
\item [Name:]  t*link-schema*link-IDs

\item [Class:] {\sl t*link-schema}\hfill(page~\pageref{t*link-schema})

\item [Contents:] A list of link-IDs

\item [Description:]

The current links of this type.

\item [Setf-able:]


\item [Public:]



\end{description}
\horizontalline

\subsection{t*link-schema*make}
\label{t*link-schema*make}

\begin{description}
\item [Name:]  t*link-schema*make

\item [Class:] {\sl t*link-schema}\hfill(page~\pageref{t*link-schema})

\item [Parameters:]
\item {\sl link-name}:  string (30); a valid link name

\item {\sl to-constraint-exp}:  A link constraint expression


\item {\sl from-constraint-exp}:  A link constraint expression.



\item [Return-value:]

Returns a new link-schema-ID with 
associated link-name and to and from node constraints. 

Error object {\sl invalid-constraint-exp} (page~\pageref{invalid-constraint-exp}) if the to
or from constraint expressions are illegal.

Error object {\sl unknown-hb-error} (page~\pageref{unknown-hb-error}) if call fails for
any other reason. 


\item [Description:]


\item [Public:]



\end{description}
\horizontalline

\subsection{t*link-schema*instantiate}
\label{t*link-schema*instantiate}

\begin{description}
\item [Name:]  t*link-schema*instantiate

\item [Class:] {\sl t*link-schema}\hfill(page~\pageref{t*link-schema})

\item [Parameters:]
\item {\sl link-schema-ID}:  an ID for a link schema.

\item {\sl to-node-ID}:  node-ID


\item {\sl from-node-ID}:  node-ID


\item [Return-value:]
The newly created link-ID if successful.

Error object {\sl unknown-hb-error} (page~\pageref{unknown-hb-error}) if call fails for
any other reason.

\item [Description:]

Creates a new link instance conforming to the constraints
specified for link-schema-ID. 

\item [Public:]



\end{description}
\horizontalline

\subsection{t*link-schema*add-to-node}
\label{t*link-schema*add-to-node}

\begin{description}
\item [Name:]  t*link-schema*add-to-node

\item [Class:] {\sl t*link-schema}\hfill(page~\pageref{t*link-schema})

\item [Parameters:]
\item {\sl link-schema-ID}:  an ID for a link schema.

\item {\sl node-schema-ID}:  a server-level node-ID that corresponds to an 
instance of a type-level node-schema. (???)



\item [Return-value:]
The updated to-node constraint expression if successful.

Error object {\sl invalid-node-schema-ID} (page~\pageref{invalid-node-schema-ID}) if second
arg is not a t*node-schema.

Error object {\sl invalid-link-schema-ID} (page~\pageref{invalid-link-schema-ID}) if first
arg is not a t*link-schema.

Error object {\sl unknown-hb-error} (page~\pageref{unknown-hb-error}) if call fails for 
any other reason.

\item [Description:]

Adds node-schema-ID to the list of to-node constraints,
and returns the new constraint expression.

If node-schema-ID is the symbol T, then the constraint
expression is set to T.

\item [Public:]



\end{description}
\horizontalline

\subsection{t*link-schema*add-from-node}
\label{t*link-schema*add-from-node}

\begin{description}
\item [Name:]  t*link-schema*add-from-node

\item [Class:] {\sl t*link-schema}\hfill(page~\pageref{t*link-schema})

\item [Parameters:]
\item {\sl link-schema-ID}:  an ID for a link schema.

\item {\sl node-schema-ID}:  a server-level node-ID that corresponds to an 
instance of a type-level node-schema. (???)



\item [Return-value:]
Returns the updated constraint expression if successful.

Error object {\sl invalid-link-schema-ID} (page~\pageref{invalid-link-schema-ID}) if first
arg is not a t*link-schema.

Error object {\sl invalid-node-schema-ID} (page~\pageref{invalid-node-schema-ID}) if the 
second arg is not a t*node-schema, or the distinguished
symbol T.

Error object {\sl unknown-hb-error} (page~\pageref{unknown-hb-error}) if the call fails
for any other reason. 

\item [Description:]

Adds node-schema-ID to the list of acceptable node
types for this from-node constraint expression, or 
sets the constraint expression to T if that symbol
is supplied.

\item [Public:]



\end{description}
\horizontalline

\subsection{t*link-schema*delete-to-node}
\label{t*link-schema*delete-to-node}

\begin{description}
\item [Name:]  t*link-schema*delete-to-node

\item [Class:] {\sl t*link-schema}\hfill(page~\pageref{t*link-schema})

\item [Parameters:]
\item {\sl link-schema-ID}:  an ID for a link schema.

\item {\sl node-schema-ID}:  a server-level node-ID that corresponds to an 
instance of a type-level node-schema. (???)



\item [Return-value:]
Returns the updated constraint expression if successful.

Error object {\sl invalid-link-schema-ID} (page~\pageref{invalid-link-schema-ID}) if first
arg is not a t*link-schema.

Error object {\sl invalid-node-schema-ID} (page~\pageref{invalid-node-schema-ID}) if second
arg is not a t*node-schema in the constraint 
expression for t*link-schema's to-nodes, or the 
distinguished symbol NIL. 

Error object {\sl unknown-hb-error} (page~\pageref{unknown-hb-error}) if call fails
for any other reason.

\item [Description:]

Deletes node-schema-ID from the to-node constraint
expression. If node-schema-ID is NIL, then the 
constraint expression is set to NIL.

\item [Public:]



\end{description}
\horizontalline

\subsection{t*link-schema*delete-from-node}
\label{t*link-schema*delete-from-node}

\begin{description}
\item [Name:]  t*link-schema*delete-from-node

\item [Class:] {\sl t*link-schema}\hfill(page~\pageref{t*link-schema})

\item [Parameters:]
\item {\sl link-schema-ID}:  an ID for a link schema.

\item {\sl node-schema-ID}:  a server-level node-ID that corresponds to an 
instance of a type-level node-schema. (???)



\item [Return-value:]
T if successful.

Error object {\sl invalid-link-schema-ID} (page~\pageref{invalid-link-schema-ID}) if first
arg is not a t*link-schema.

Error object {\sl invalid-node-schema-ID} (page~\pageref{invalid-node-schema-ID}) if second
argument is not a node-schema-ID currently present
in the constraint expression, or the distinguished
value NIL.

Error object {\sl unknown-hb-error} (page~\pageref{unknown-hb-error}) if call fails
for any other reason.

\item [Description:]

Deletes node-schema-ID from the constraint expression
for link from-nodes.  If NIL is supplied rather than
a node-schema-ID, then the constraint expression
is set to NIL.


\item [Public:]



\end{description}
\horizontalline

\subsection{t*link-schema*set-name}
\label{t*link-schema*set-name}

\begin{description}
\item [Name:]  t*link-schema*set-name

\item [Class:] {\sl t*link-schema}\hfill(page~\pageref{t*link-schema})

\item [Parameters:]
\item {\sl link-schema-ID}:  an ID for a link schema.

\item {\sl link-name}:  string (30); a valid link name



\item [Return-value:]
T if successful.

Error object {\sl invalid-link-name} (page~\pageref{invalid-link-name}) if illegal link-name.

Error object {\sl invalid-link-schema-ID} (page~\pageref{invalid-link-schema-ID}) if first
argument is not a t*link-schema.

Error object {\sl unknown-hb-error} (page~\pageref{unknown-hb-error}) if call fails for
any other reason.

\item [Description:]

Changes the name of link-schema-ID to link-name.

\item [Public:]



\end{description}
\horizontalline

\subsection{t*link-schema*divergence}
\label{t*link-schema*divergence}

\begin{description}
\item [Name:]  t*link-schema*divergence

\item [Class:] {\sl t*link-schema}\hfill(page~\pageref{t*link-schema})

\item [Parameters:]
\item {\sl link-schema-ID}:  an ID for a link schema.


\item [Return-value:]
An integer divergence value if successful.

Error object {\sl unknown-hb-error} (page~\pageref{unknown-hb-error}) if call fails
for any reason.

\item [Description:]

Computes the divergence metric for link-schema 
instances.

\item [Public:]



\end{description}
\horizontalline

\subsection{t*link-schema*delete}
\label{t*link-schema*delete}

\begin{description}
\item [Name:]  t*link-schema*delete

\item [Class:] {\sl t*link-schema}\hfill(page~\pageref{t*link-schema})

\item [Parameters:]
\item {\sl link-schema-ID}:  an ID for a link schema.


\item [Return-value:]
The deleted link-schema-ID if successful.

Error object {\sl invalid-link-schema-ID} (page~\pageref{invalid-link-schema-ID}) if argument
is not a link-schema-ID.

Error object {\sl unknown-hb-error} (page~\pageref{unknown-hb-error}) if call fails
for some other reason.

\item [Description:]

Deletes t*link-schema-ID.  Note that this does not
physically remove link-schema-ID from the database.
Rather, it simply prevents any new instances of 
link-schema-ID from being created.

\item [Public:]



\end{description}
\horizontalline

\section{t*link-instance}
\label{t*link-instance}

\begin{description}
\item [Name:]  t*link-instance

\item [Layer:] {\sl Type}\hfill(page~\pageref{Type})

\item [Description:]

The instances of this class define links between nodes. 

\item [Attributes:]
\item {\sl t*link*schema-ID}\hfill(page~\pageref{t*link*schema-ID})
\item {\sl t*link*to-node-ID}\hfill(page~\pageref{t*link*to-node-ID})
\item {\sl t*link*from-node-ID}\hfill(page~\pageref{t*link*from-node-ID})
\item {\sl t*link*layer-IDs}\hfill(page~\pageref{t*link*layer-IDs})

\item [Operations:]
\item {\sl t*link*clone}\hfill(page~\pageref{t*link*clone})
\item {\sl t*link*delete}\hfill(page~\pageref{t*link*delete})
\item {\sl t*link*set-schema-ID}\hfill(page~\pageref{t*link*set-schema-ID})
\item {\sl t*link*set-to-node}\hfill(page~\pageref{t*link*set-to-node})
\item {\sl t*link*add-to-constraint}\hfill(page~\pageref{t*link*add-to-constraint})
\item {\sl t*link*add-from-constraint}\hfill(page~\pageref{t*link*add-from-constraint})
\item {\sl t*link*delete-to-constraint}\hfill(page~\pageref{t*link*delete-to-constraint})
\item {\sl t*link*delete-from-constraint}\hfill(page~\pageref{t*link*delete-from-constraint})
\item {\sl t*link*add-layer-ID}\hfill(page~\pageref{t*link*add-layer-ID})
\item {\sl t*link*delete-layer-ID}\hfill(page~\pageref{t*link*delete-layer-ID})
\item {\sl t*link*convergence}\hfill(page~\pageref{t*link*convergence})


\item [Subclasses:]


\item [Superclasses:]


\item [Instances:]



\end{description}
\horizontalline

\subsection{t*link*schema-ID}
\label{t*link*schema-ID}

\begin{description}
\item [Name:]  t*link*schema-ID

\item [Class:] {\sl t*link-instance}\hfill(page~\pageref{t*link-instance})

\item [Contents:]
A link-schema-ID

\item [Description:]

The link schema from this link instance.

\item [Setf-able:]


\item [Public:]



\end{description}
\horizontalline

\subsection{t*link*to-node-ID}
\label{t*link*to-node-ID}

\begin{description}
\item [Name:]  t*link*to-node-ID

\item [Class:] {\sl t*link-instance}\hfill(page~\pageref{t*link-instance})

\item [Contents:]
A node-ID.

\item [Description:]

The node-ID to which this link instance points. 

\item [Setf-able:]


\item [Public:]



\end{description}
\horizontalline

\subsection{t*link*from-node-ID}
\label{t*link*from-node-ID}

\begin{description}
\item [Name:]  t*link*from-node-ID

\item [Class:] {\sl t*link-instance}\hfill(page~\pageref{t*link-instance})

\item [Contents:] 
A node-ID

\item [Description:]

The node-ID from which this link originates.

\item [Setf-able:]


\item [Public:]



\end{description}
\horizontalline

\subsection{t*link*layer-IDs}
\label{t*link*layer-IDs}

\begin{description}
\item [Name:]  t*link*layer-IDs

\item [Class:] {\sl t*link-instance}\hfill(page~\pageref{t*link-instance})

\item [Contents:] A list of layer-IDs

\item [Description:]

The list of layer-IDs to which this link instance belongs.

\item [Setf-able:]


\item [Public:]



\end{description}
\horizontalline

\subsection{t*link*clone}
\label{t*link*clone}

\begin{description}
\item [Name:]  t*link*clone   (?????????)

\item [Class:] {\sl t*link-instance}\hfill(page~\pageref{t*link-instance})

\item [Parameters:]
\item {\sl link-ID}:  
valid HB link ID number (integer)

\item {\sl to-node-ID}:  node-ID


\item {\sl from-node-ID}:  node-ID


\item [Return-value:]
The newly created link-ID with the same schema, 
to, and from nodes as link-ID.

\item [Description:]

This operations clones link-ID, but it is not 
clear that this operation should exist. Reactions?

\item [Public:]



\end{description}
\horizontalline

\subsection{t*link*delete}
\label{t*link*delete}

\begin{description}
\item [Name:]  t*link*delete

\item [Class:] {\sl t*link-instance}\hfill(page~\pageref{t*link-instance})

\item [Parameters:]
\item {\sl link-ID}:  
valid HB link ID number (integer)


\item [Return-value:]
The deleted link-ID if successful.

Error object {\sl invalid-link-ID} (page~\pageref{invalid-link-ID}) if not a t*link-instance.

Error object {\sl unknown-hb-error} (page~\pageref{unknown-hb-error}) if call fails
for any other reason.

\item [Description:]

Deletes link-ID.

\item [Public:]



\end{description}
\horizontalline

\subsection{t*link*set-schema-ID}
\label{t*link*set-schema-ID}

\begin{description}
\item [Name:]\item [Name:]  t*link*set-schema-ID

\item [Class:]\item [Class:] {\sl t*link-instance}\hfill(page~\pageref{t*link-instance}){\sl t*link-instance}\hfill(page~\pageref{t*link-instance})

\item [Parameters:]\item [Parameters:]
\item {\sl link-ID}:  
valid HB link ID number (integer)
\item {\sl link-ID}:  
valid HB link ID number (integer)

\item {\sl link-schema-ID}:  an ID for a link schema.
\item {\sl link-schema-ID}:  an ID for a link schema.


\item [Return-value:]\item [eturn-value:
]The\item [ updated lin]k-s\item [chema-I]D if successful.

Error object {\sl invalid-link-ID} (page~\pageref{invalid-link-ID}) if not a t*link-ID.

Error object {\sl invalid-link-schema-ID} (page~\pageref{invalid-link-schema-ID}) if not
a t*link-schema.

Error object {\sl unknown-hb-error} (page~\pageref{unknown-hb-error}) if call fails
for any other reason.

\item [Description:]

Updates the schema associated with this 

\item [Public:]



\end{description}
\horizontalline

\subsection{t*link*set-to-node}
\label{t*link*set-to-node}

\begin{description}
\item [Name:]  t*link*set-to-node

\item [Class:] {\sl t*link-instance}\hfill(page~\pageref{t*link-instance})

\item [Parameters:]
\item {\sl link-ID}:  
valid HB link ID number (integer)

\item {\sl node-ID}:  
integer (a valid hbserver node ID number)


\item [Return-value:]
Returns the updated to-node ID if successful.

Error object {\sl invalid-link-ID} (page~\pageref{invalid-link-ID}) if not a t*link.

Error object {\sl invalid-node-ID} (page~\pageref{invalid-node-ID}) if not a t*node.

Error object {\sl unknown-hb-error} (page~\pageref{unknown-hb-error}) if call fails
for any other reason.

\item [Description:]

Updates link-ID to point to node-ID.

\item [Public:]



\end{description}
\horizontalline

\subsection{t*link*add-to-constraint}
\label{t*link*add-to-constraint}

\begin{description}
\item [Name:]  t*link*add-to-constraint

\item [Class:] {\sl t*link-instance}\hfill(page~\pageref{t*link-instance})

\item [Parameters:]
\item {\sl link-ID}:  
valid HB link ID number (integer)

\item {\sl node-schema-ID}:  a server-level node-ID that corresponds to an 
instance of a type-level node-schema. (???)



\item [Return-value:]
The updated constraint expression if successful.

Error object {\sl invalid-link-ID} (page~\pageref{invalid-link-ID}) if not a t*link-instance.

Error object {\sl invalid-node-schema-ID} (page~\pageref{invalid-node-schema-ID}) if not a
t*node-schema.

Error object {\sl unknown-hb-error} (page~\pageref{unknown-hb-error}) if call fails for
any other reason.

\item [Description:]

Adds a to-node constraint. If node-schema-ID is the
distinguished symbol T, then the constraint expression
is set to T.

\item [Public:]



\end{description}
\horizontalline

\subsection{t*link*add-from-constraint}
\label{t*link*add-from-constraint}

\begin{description}
\item [Name:]  t*link*add-from-constraint

\item [Class:] {\sl t*link-instance}\hfill(page~\pageref{t*link-instance})

\item [Parameters:]
\item {\sl link-ID}:  
valid HB link ID number (integer)

\item {\sl node-schema-ID}:  a server-level node-ID that corresponds to an 
instance of a type-level node-schema. (???)



\item [Return-value:]
The updated from-node constraint expression if successful.

Error object {\sl invalid-link-ID} (page~\pageref{invalid-link-ID}) if not a t*link-instance.

Error object {\sl invalid-node-schema-ID} (page~\pageref{invalid-node-schema-ID}) if not
a t*node-schema.

Error object {\sl unknown-hb-error} (page~\pageref{unknown-hb-error}) if call fails for
any other reason.

\item [Description:]

Adds node-schema-ID to the from-constraints for this
link instance. Node-schema-ID may also be the distinguished
symbol T, which sets the constraint expression itself
to T.

\item [Public:]



\end{description}
\horizontalline

\subsection{t*link*delete-to-constraint}
\label{t*link*delete-to-constraint}

\begin{description}
\item [Name:]\item [Name:]  t*link*delete-to-constraint

\item [Class:]\item [Class:] {\sl t*link-instance}\hfill(page~\pageref{t*link-instance}){\sl t*link-instance}\hfill(page~\pageref{t*link-instance})

\item [Parameters:]\item [Parameters:]
{\sl node-schema-ID}:  a server-level node-ID that corresponds to an 
instance of a type-level node-schema. (???)

updated to node constraint expression if successful.

Error object {\sl invalid-link-ID} (page~\pageref{invalid-link-ID}) if not a t*link-ID.

Error object {\sl invalid-node-schema-ID} (page~\pageref{invalid-node-schema-ID}) if not a 
t*node-schema, and if not currently a member of the 
to-node constraints.

\item [Description:]

Deletes a to-node constraint.

\item [Public:]



\end{description}
\horizontalline

\subsection{t*link*delete-from-constraint}
\label{t*link*delete-from-constraint}

\begin{description}
\item [Name:]  t*link*delete-from-constraint

\item [Class:] {\sl t*link-instance}\hfill(page~\pageref{t*link-instance})

\item [Parameters:]
\item {\sl link-ID}:  
valid HB link ID number (integer)

\item {\sl node-schema-ID}:  a server-level node-ID that corresponds to an 
instance of a type-level node-schema. (???)



\item [Return-value:]
Returns the updated from-node constraint expression
if successful.

Error object {\sl invalid-link-ID} (page~\pageref{invalid-link-ID}) if not a t*link-instance.

Error object {\sl invalid-node-schema-ID} (page~\pageref{invalid-node-schema-ID}) if not
a t*node-schema, and not a member of the constraint
expression.

Error object {\sl unknown-hb-error} (page~\pageref{unknown-hb-error}) if call fails for
any other reason.

\item [Description:]

Deletes a from constraint from this node instance's
constraint expression.

\item [Public:]



\end{description}
\horizontalline

\subsection{t*link*add-layer-ID}
\label{t*link*add-layer-ID}

\begin{description}
\item [Name:]  t*link*add-layer-ID

\item [Class:] {\sl t*link-instance}\hfill(page~\pageref{t*link-instance})

\item [Parameters:]
\item {\sl link-ID}:  
valid HB link ID number (integer)

\item {\sl layer-ID}:  a unique ID for layers (possibly a node-ID?)



\item [Return-value:]
The updated list of layer-IDs if successful.

Error object {\sl invalid-link-ID} (page~\pageref{invalid-link-ID}) if not a legal
t*link-ID.

Error object {\sl invalid-layer-ID} (page~\pageref{invalid-layer-ID}) if not a legal
layer-ID, or if it already occurs in this link-instance.

Error object {\sl unknown-hb-error} (page~\pageref{unknown-hb-error}) if call fails
for any other reason.

\item [Description:]

Makes link-ID a member of layer-ID.

\item [Public:]



\end{description}
\horizontalline

\subsection{t*link*delete-layer-ID}
\label{t*link*delete-layer-ID}

\begin{description}
\item [Name:]  t*link*delete-layer-ID

\item [Class:] {\sl t*link-instance}\hfill(page~\pageref{t*link-instance})

\item [Parameters:]
\item {\sl link-ID}:  
valid HB link ID number (integer)

\item {\sl layer-ID}:  a unique ID for layers (possibly a node-ID?)

 

\item [Return-value:]
The updated list of layer-IDs if successful.

Error object {\sl invalid-link-ID} (page~\pageref{invalid-link-ID}) if link-ID was
not a t*link-instance.

Error object {\sl invalid-layer-ID} (page~\pageref{invalid-layer-ID}) if layer-ID was
not a t*layer, or not a current member of this link
instance's layer-IDs.

Error object {\sl unknown-hb-error} (page~\pageref{unknown-hb-error}) if call fails
for any other reason. 

\item [Description:]

Deletes a layer-ID membership from this link.

\item [Public:]



\end{description}
\horizontalline

\subsection{t*link*convergence}
\label{t*link*convergence}

\begin{description}
\item [Name:]  t*link*convergence

\item [Class:] {\sl t*link-instance}\hfill(page~\pageref{t*link-instance})

\item [Parameters:]
\item {\sl link-ID}:  
valid HB link ID number (integer)



\item [Return-value:]

An integer convergence value.

\item [Description:]

Returns the convergence between this link instance and
the set of link-schemas if successful.

Error object {\sl unknown-hb-error} (page~\pageref{unknown-hb-error}) if call fails for
any reason.

\item [Public:]



\end{description}
\horizontalline

\section{t*field-schema}
\label{t*field-schema}

\begin{description}
\item [Name:]  t*field-schema

\item [Layer:] {\sl Type}\hfill(page~\pageref{Type})

\item [Description:]

Field schemas define the internal structure of node
instance fields. While the value associated with a 
field schema could conceivably be anything (such as 
a bitmap image or some other kind of multi-media
object), this version of the type system restricts
fields to the following internal structure:

(t*field-schema*value :name '<field-name> 
                      :value <field-value>

where:

<field-name> is a valid lisp symbol indicating the name.

<field-value> is a readable lisp list with the 
following format: 
      (\{<string>|<link-ID>\}*)

In other words, the value of a field is normally an
alternating sequence of textual strings (where internal
quotation marks are backslashed) and link-IDs.
  
Note that while this representation results in a simple
implementation (just define the function
t*field-schema*value and eval the buffer that the field
set was passed in), a more efficient form would use
regular expression search to extract the contents.  

\item [Attributes:]
\item {\sl t*field-schema*name}\hfill(page~\pageref{t*field-schema*name})
\item {\sl t*field-schema*validity-fn}\hfill(page~\pageref{t*field-schema*validity-fn})


\item [Operations:]
\item {\sl t*field-schema*make}\hfill(page~\pageref{t*field-schema*make})
\item {\sl t*field-schema*delete}\hfill(page~\pageref{t*field-schema*delete})
\item {\sl t*field-schema*set-name}\hfill(page~\pageref{t*field-schema*set-name})
\item {\sl t*field-schema*set-validity-fn}\hfill(page~\pageref{t*field-schema*set-validity-fn})


\item [Subclasses:]


\item [Superclasses:]


\item [Instances:]



\end{description}
\horizontalline

\subsection{t*field-schema*name}
\label{t*field-schema*name}

\begin{description}
\item [Name:]  t*field-schema*name

\item [Class:] {\sl t*field-schema}\hfill(page~\pageref{t*field-schema})

\item [Contents:] a symbol

\item [Description:]

The name of the field. 

\item [Setf-able:] See t*field-schema*set-name.


\item [Public:]



\end{description}
\horizontalline

\subsection{t*field-schema*validity-fn}
\label{t*field-schema*validity-fn}

\begin{description}
\item [Name:]  t*field-schema*validity-fn

\item [Class:] {\sl t*field-schema}\hfill(page~\pageref{t*field-schema})

\item [Contents:] A list

\item [Description:]

This is an Emacs-Lisp function (in lambda list format) for 
assessing the validity of an object to be stored in a 
field. 

The validity function takes one argument, the field 
value, and should return T if the field value is legal
and NIL if the field value is not. 

\item [Setf-able:] See t*field-schema*set-validity-fn

\item [Public:]



\end{description}
\horizontalline

\subsection{t*field-schema*make}
\label{t*field-schema*make}

\begin{description}

\item [Name:]  t*field-schema*make


\item [Class:]
{\sl t*field-schema}\hfill(page~\pageref{t*field-schema})


\item [Parameters:]


\item [Return-value:]


\item [Description:]


\item [Public:]



\end{description}
\horizontalline

\subsection{t*field-schema*delete}
\label{t*field-schema*delete}

\begin{description}
\item [Name:]  t*field-schema*delete

\item [Class:] {\sl t*field-schema}\hfill(page~\pageref{t*field-schema})

\item [Parameters:]
\item {\sl field-schema-ID}:  an integer signifying a field-schema.


\item [Return-value:]
The deleted field-schema-ID if successful.

Error object {\sl invalid-field-ID} (page~\pageref{invalid-field-ID}) if field-schema-ID
is not a t*field-schema.

Error object {\sl unknown-hb-error} (page~\pageref{unknown-hb-error}) if call fails
for any other reason.

\item [Description:]

Deletes the field-schema.  This does not physically
remove the schema from the database, but instead
prevents any new instances of it from being made.

For the time being, node schemas and instances are
not automatically flagged as containing a deleted
field schema. 


\item [Public:]



\end{description}
\horizontalline

\subsection{t*field-schema*set-name}
\label{t*field-schema*set-name}

\begin{description}
\item [Name:]  t*field-schema*set-name

\item [Class:] {\sl t*field-schema}\hfill(page~\pageref{t*field-schema})

\item [Parameters:]
\item {\sl field-schema-ID}:  an integer signifying a field-schema.

\item {\sl node-name}:  
A valid node name. This currently means that it is a
string of less than 40 characters, and that it does
not contain leading space(s) or tabs.


\item [Return-value:]
T if successful.

Error object {\sl invalid-node-name} (page~\pageref{invalid-node-name}) if node-name
is syntactically incorrect.

Error object {\sl invalid-field-ID} (page~\pageref{invalid-field-ID}) if field-schema-ID
is not a t*field-schema.

Error object {\sl unknown-hb-error} (page~\pageref{unknown-hb-error}) if the call fails
for any other reason.

\item [Description:]

Sets the name attribute of the field-schema.

\item [Public:]



\end{description}
\horizontalline

\subsection{t*field-schema*set-validity-fn}
\label{t*field-schema*set-validity-fn}

\begin{description}
\item [Name:]  t*field-schema*set-validity-fn

\item [Class:] {\sl t*field-schema}\hfill(page~\pageref{t*field-schema})

\item [Parameters:]
\item {\sl field-schema-ID}:  an integer signifying a field-schema.

\item {\sl field-validity-fn}:  

A lambda list representing a funcallable Emacs-Lisp function for
assessing the contents of a field.

The function takes one argument, the contents of the field.
See t*field-schema for more information on the field content
structure.




\item [Return-value:] 
T if successful.

Error object {\sl invalid-field-ID} (page~\pageref{invalid-field-ID}) if field-schema-ID is not a valid t*field-schema.

Error object {\sl unknown-hb-error} (page~\pageref{unknown-hb-error}) if the call fails
for any other reason. 

\item [Description:]

Sets the validity function associated with field-schema-ID
to a new value. 


\item [Public:]



\end{description}
\horizontalline

\section{t*layer}
\label{t*layer}

\begin{description}
\item [Name:]  t*layer

\item [Layer:] {\sl Type}\hfill(page~\pageref{Type})

\item [Description:]

Defines the exploratory namespace mechanism in Egret.

\item [Attributes:]
\item {\sl t*layer*name}\hfill(page~\pageref{t*layer*name})
\item {\sl t*layer*node-IDs}\hfill(page~\pageref{t*layer*node-IDs})
\item {\sl t*layer*link-IDs}\hfill(page~\pageref{t*layer*link-IDs})

\item [Operations:]
\item {\sl t*layer*make}\hfill(page~\pageref{t*layer*make})
\item {\sl t*layer*add-node-ID}\hfill(page~\pageref{t*layer*add-node-ID})
\item {\sl t*layer*delete-node-ID}\hfill(page~\pageref{t*layer*delete-node-ID})
\item {\sl t*layer*add-link-ID}\hfill(page~\pageref{t*layer*add-link-ID})
\item {\sl t*layer*delete-link-ID}\hfill(page~\pageref{t*layer*delete-link-ID})
\item {\sl t*layer*divergence}\hfill(page~\pageref{t*layer*divergence})


\item [Subclasses:]


\item [Superclasses:]


\item [Instances:]
























\end{description}
\horizontalline

\subsection{t*layer*name}
\label{t*layer*name}

\begin{description}
\item [Name:]  t*layer*name

\item [Class:] {\sl t*layer}\hfill(page~\pageref{t*layer})

\item [Contents:]
symbol

\item [Description:]
The name of the layer.

\item [Setf-able:]


\item [Public:]



\end{description}
\horizontalline

\subsection{t*layer*node-IDs}
\label{t*layer*node-IDs}

\begin{description}
\item [Name:]  t*layer*node-IDs

\item [Class:] {\sl t*layer}\hfill(page~\pageref{t*layer})

\item [Contents:]
A list of node-IDs.

\item [Description:]

The node-IDs currently belonging to this layer.

\item [Setf-able:]


\item [Public:]



\end{description}
\horizontalline

\subsection{t*layer*link-IDs}
\label{t*layer*link-IDs}

\begin{description}
\item [Name:]  t*layer*link-IDs

\item [Class:] {\sl t*layer}\hfill(page~\pageref{t*layer})

\item [Contents:] 
A list of link-IDs

\item [Description:]

The link-IDs belonging to this layer.

\item [Setf-able:]


\item [Public:]



\end{description}
\horizontalline

\subsection{t*layer*make}
\label{t*layer*make}

\begin{description}
\item [Name:]  t*layer*make

\item [Class:] {\sl t*layer}\hfill(page~\pageref{t*layer})

\item [Parameters:]
\item {\sl layer-name}:  a string less than 40 characters, obeying
all node-name restrictions. 


\item {\sl node-IDs}:  a list of node-IDs


\item {\sl link-IDs}:  a list of link-IDs


\item [Return-value:]
A layer-ID for the newly created layer if successful.

Error object {\sl invalid-node-ID} (page~\pageref{invalid-node-ID}) if any of the node-IDs
are invalid.

Error object {\sl invalid-link-ID} (page~\pageref{invalid-link-ID}) if any of the link-IDs
are invalid.

Error object {\sl unknown-hb-error} (page~\pageref{unknown-hb-error}) if the call fails
for any other reason.

\item [Description:]

Creates and initializes a new layer. 

\item [Public:]



\end{description}
\horizontalline

\subsection{t*layer*add-node-ID}
\label{t*layer*add-node-ID}

\begin{description}
\item [Name:]  t*layer*add-node-ID

\item [Class:] {\sl t*layer}\hfill(page~\pageref{t*layer})

\item [Parameters:]
\item {\sl layer-ID}:  a unique ID for layers (possibly a node-ID?)


\item {\sl node-ID}:  
integer (a valid hbserver node ID number)


\item [Return-value:]
The updated list of node-IDs in layer-ID if successful.

Error object {\sl invalid-layer-ID} (page~\pageref{invalid-layer-ID}) if not a t*layer.

Error object {\sl invalid-node-ID} (page~\pageref{invalid-node-ID}) if not a 
t*node-instance, or if currently a member of t*layer.

\item [Description:]

Adds node-ID to layer-ID.

\item [Public:]



\end{description}
\horizontalline

\subsection{t*layer*delete-node-ID}
\label{t*layer*delete-node-ID}

\begin{description}
\item [Name:]  t*layer*delete-node-ID

\item [Class:] {\sl t*layer}\hfill(page~\pageref{t*layer})

\item [Parameters:]
\item {\sl layer-ID}:  a unique ID for layers (possibly a node-ID?)


\item {\sl node-ID}:  
integer (a valid hbserver node ID number)


\item [Return-value:]
The updated list of node-IDs in layer-ID if successful.

Error object {\sl invalid-layer-ID} (page~\pageref{invalid-layer-ID}) if not a t*layer.

Error object {\sl invalid-node-ID} (page~\pageref{invalid-node-ID}) if not a
t*node-instance, or if not in layer-ID.

Error object {\sl unknown-hb-error} (page~\pageref{unknown-hb-error}) if call fails for
any other reason.

\item [Description:]

Deletes node-ID from layer-ID.

\item [Public:]



\end{description}
\horizontalline

\subsection{t*layer*add-link-ID}
\label{t*layer*add-link-ID}

\begin{description}
\item [Name:]  t*layer*add-link-ID

\item [Class:] {\sl t*layer}\hfill(page~\pageref{t*layer})

\item [Parameters:]
\item {\sl layer-ID}:  a unique ID for layers (possibly a node-ID?)


\item {\sl link-ID}:  
valid HB link ID number (integer)


\item [Return-value:]
The updated set of link-IDs in layer-ID if successful.

Error object {\sl invalid-layer-ID} (page~\pageref{invalid-layer-ID}) if not a t*layer.

Error object {\sl invalid-link-ID} (page~\pageref{invalid-link-ID}) if not a
t*link-instance, or if link-ID is not a member of 
layer-ID.

Error object {\sl unknown-hb-error} (page~\pageref{unknown-hb-error}) if call fails
for any other reason.

\item [Description:]

Deletes link-ID from layer-ID.

\item [Public:]



\end{description}
\horizontalline

\subsection{t*layer*delete-link-ID}
\label{t*layer*delete-link-ID}

\begin{description}
\item [Name:]  t*layer*delete-link-ID

\item [Class:] {\sl t*layer}\hfill(page~\pageref{t*layer})

\item [Parameters:]
\item {\sl layer-ID}:  a unique ID for layers (possibly a node-ID?)


\item {\sl link-ID}:  
valid HB link ID number (integer)


\item [Return-value:]
The updated set of link-IDs in this layer if successful.

Error object {\sl invalid-layer-ID} (page~\pageref{invalid-layer-ID}) if not a t*layer.

Error object {\sl invalid-link-ID} (page~\pageref{invalid-link-ID}) if not a t*link,
or if not currently a member of this layer.

Error object {\sl unknown-hb-error} (page~\pageref{unknown-hb-error}) if call fails
for any other reason.

\item [Description:]

Deletes a link instance from this layer.

\item [Public:]



\end{description}
\horizontalline

\subsection{t*layer*divergence}
\label{t*layer*divergence}

\begin{description}
\item [Name:]  t*layer*divergence

\item [Class:] {\sl t*layer}\hfill(page~\pageref{t*layer})

\item [Parameters:]
\item {\sl layer-ID}:  a unique ID for layers (possibly a node-ID?)



\item [Return-value:]
The divergence value for layer-ID if successful.

Error object {\sl invalid-layer-ID} (page~\pageref{invalid-layer-ID}) if not a t*layer.

Error object {\sl unknown-hb-error} (page~\pageref{unknown-hb-error}) if call fails
for any other reason.

\item [Description:]

Computes the aggregate divergence for all of the
schemas with instances in this layer. 

\item [Public:]



\end{description}
\horizontalline

\section{t*error}
\label{t*error}

\begin{description}
\item [Name:]  t*error

\item [Layer:] {\sl Type}\hfill(page~\pageref{Type})

\item [Description:]
The instances of this class consist of the type-level
error objects. In some cases, however, the type system
may return error objects that are instances of the 
server system s*serror class, such as the error object
for an invalid node name. Thus, this class defines 
all of the errors semantically associated with the
type system facilities, but not all of the possible 
errors that the type system could return.

\item [Attributes:]

\item [Operations:]

\item [Subclasses:]

\item [Superclasses:]
\item {\sl u*error}\hfill(page~\pageref{u*error})

\item [Instances:]
\item {\sl invalid-link-ID}\hfill(page~\pageref{invalid-link-ID})
\item {\sl invalid-constraint-exp}\hfill(page~\pageref{invalid-constraint-exp})
\item {\sl invalid-link-schema-ID}\hfill(page~\pageref{invalid-link-schema-ID})
\item {\sl invalid-event-instance}\hfill(page~\pageref{invalid-event-instance})
\item {\sl invalid-layer-ID}\hfill(page~\pageref{invalid-layer-ID})
\item {\sl invalid-buffer-instance}\hfill(page~\pageref{invalid-buffer-instance})
\item {\sl invalid-node-ID}\hfill(page~\pageref{invalid-node-ID})
\item {\sl invalid-node-schema-ID}\hfill(page~\pageref{invalid-node-schema-ID})
\item {\sl invalid-field-ID}\hfill(page~\pageref{invalid-field-ID})




\end{description}
\horizontalline

\subsection{invalid-link-ID}
\label{invalid-link-ID}

\begin{description}
\item [Name:]  invalid-link-ID

\item [Class:] {\sl t*error}\hfill(page~\pageref{t*error})

\item [Description:]

Indicates the passed link instance ID was invalid.


\end{description}
\horizontalline

\subsection{invalid-constraint-exp}
\label{invalid-constraint-exp}

\begin{description}
\item [Name:]  invalid-constraint-exp

\item [Class:]
{\sl t*error}\hfill(page~\pageref{t*error})


\item [Description:]

Error object indicating the presence of an 
invalid link constraint expression.

\end{description}
\horizontalline

\subsection{invalid-link-schema-ID}
\label{invalid-link-schema-ID}

\begin{description}
\item [Name:]  invalid-link-schema-ID

\item [Class:] {\sl t*error}\hfill(page~\pageref{t*error})

\item [Description:]

Error object indicating an invalid link-schema-ID 
object.


\end{description}
\horizontalline

\subsection{invalid-event-instance}
\label{invalid-event-instance}

\begin{description}
\item [Name:]  invalid-event-instance

\item [Class:] {\sl t*error}\hfill(page~\pageref{t*error})

\item [Description:]

An error indicating that an object that wasn't a
t*event instance was passed. 


\end{description}
\horizontalline

\subsection{invalid-layer-ID}
\label{invalid-layer-ID}

\begin{description}
\item [Name:]  invalid-layer-ID

\item [Class:] {\sl t*error}\hfill(page~\pageref{t*error})

\item [Description:]

An error object indicating that the argument was
not a valid layer-ID.


\end{description}
\horizontalline

\subsection{invalid-buffer-instance}
\label{invalid-buffer-instance}

\begin{description}
\item [Name:]  invalid-buffer-instance

\item [Class:] {\sl t*error}\hfill(page~\pageref{t*error})

\item [Description:]

Error object indicating that the passed object
was not an Emacs buffer instance.


\end{description}
\horizontalline

\subsection{invalid-node-ID}
\label{invalid-node-ID}

\begin{description}
\item [Name:]  invalid-node-ID

\item [Class:] {\sl t*error}\hfill(page~\pageref{t*error})

\item [Description:]

An error object representing the invalidity of the 
passed value as a node-ID.


\end{description}
\horizontalline

\subsection{invalid-node-schema-ID}
\label{invalid-node-schema-ID}

\begin{description}
\item [Name:]  invalid-node-schema-ID

\item [Class:]

\item [Description:]

Error object indicating an invalid type-level node-schema
ID.


\end{description}
\horizontalline

\subsection{invalid-field-ID}
\label{invalid-field-ID}

\begin{description}
\item [Name:]  invalid-field-ID

\item [Class:] {\sl t*error}\hfill(page~\pageref{t*error})

\item [Description:]

The error object indicating that a bad field-schema-ID
value was detected.


\end{description}
\horizontalline

\section{t*event}
\label{t*event}

\begin{description}
\item [Name:]  t*event

\item [Layer:] {\sl Type}\hfill(page~\pageref{Type})

\item [Description:]

This class defines the set of events to be handled
by the type system.  Events are the communication
mechanism between the hyperbase and its currently
connected clients that allow changes made by one
client to be propogated to other clients. 

\item [Attributes:]
\item {\sl t*event*handlers}\hfill(page~\pageref{t*event*handlers})


\item [Operations:]
\item {\sl t*event*remove-handler}\hfill(page~\pageref{t*event*remove-handler})
\item {\sl t*event*add-handler}\hfill(page~\pageref{t*event*add-handler})
\item {\sl t*event*initialize}\hfill(page~\pageref{t*event*initialize})

\item [Subclasses:]

\item [Superclasses:]

\item [Instances:]
Due to a bug in cv-1.3, the following three links
all point to the same node,
(t*event*update-lschema-from-constraints),
even though two other event nodes exist and should
be pointed to (t*event*update-lschema-name and 
t*event*update-lschema-to-constraints). 

\item {\sl t*event*update-lschema-from-constraints}\hfill(page~\pageref{t*event*update-lschema-from-constraints})
\item {\sl t*event*update-lschema-to-constraints}\hfill(page~\pageref{t*event*update-lschema-to-constraints})
\item {\sl t*event*update-lschema-name}\hfill(page~\pageref{t*event*update-lschema-name})
\item {\sl t*event*delete-link-schema}\hfill(page~\pageref{t*event*delete-link-schema})
\item {\sl t*event*new-link-schema}\hfill(page~\pageref{t*event*new-link-schema})
\item {\sl t*event*update-fschema-val-fn}\hfill(page~\pageref{t*event*update-fschema-val-fn})
\item {\sl t*event*update-fschema-name}\hfill(page~\pageref{t*event*update-fschema-name})
\item {\sl t*event*delete-field-schema}\hfill(page~\pageref{t*event*delete-field-schema})
\item {\sl t*event*new-field-schema}\hfill(page~\pageref{t*event*new-field-schema})
\item {\sl t*event*delete-ninstance-field}\hfill(page~\pageref{t*event*delete-ninstance-field})
\item {\sl t*event*update-ninstance-field}\hfill(page~\pageref{t*event*update-ninstance-field})
\item {\sl t*event*update-ninstance-name}\hfill(page~\pageref{t*event*update-ninstance-name})
\item {\sl t*event*delete-node-instance}\hfill(page~\pageref{t*event*delete-node-instance})
\item {\sl t*event*new-node-instance}\hfill(page~\pageref{t*event*new-node-instance})
\item {\sl t*event*update-nschema-nodes}\hfill(page~\pageref{t*event*update-nschema-nodes})
\item {\sl t*event*update-nschema-fields}\hfill(page~\pageref{t*event*update-nschema-fields})
\item {\sl t*event*update-nschema-name}\hfill(page~\pageref{t*event*update-nschema-name})
\item {\sl t*event*delete-node-schema}\hfill(page~\pageref{t*event*delete-node-schema})
\item {\sl t*event*new-node-schema}\hfill(page~\pageref{t*event*new-node-schema})

























\end{description}
\horizontalline

\subsection{t*event*handlers}
\label{t*event*handlers}

\begin{description}
\item [Name:]  t*event*handlers

\item [Class:] {\sl t*event}\hfill(page~\pageref{t*event})

\item [Contents:] List of symbols

\item [Description:]

This attribute returns an ordered list of event
handler function names. 

\item [Setf-able:] See t*event*add-event-handlers and 
t*event*remove-event-handlers.


\item [Public:]



\end{description}
\horizontalline

\subsection{t*event*remove-handler}
\label{t*event*remove-handler}

\begin{description}
\item [Name:]  t*event*remove-handler

\item [Class:] {\sl t*event}\hfill(page~\pageref{t*event})

\item [Parameters:]
\item {\sl t*event-instance}:  an instance of t*event


\item {\sl handler-fn-name}:  function symbol


\item [Return-value:]
T if successful.

Error object {\sl invalid-event-instance} (page~\pageref{invalid-event-instance}) if first
arg is not a t*event-instance.

\item [Description:]

Removes handler-fn-name from t*event-instance.

\item [Public:]



\end{description}
\horizontalline

\subsection{t*event*add-handler}
\label{t*event*add-handler}

\begin{description}
\item [Name:]  t*event*add-handler

\item [Class:] {\sl t*event}\hfill(page~\pageref{t*event})

\item [Parameters:]
\item {\sl t*event-instance}:  an instance of t*event


\item {\sl handler-fn-name}:  function symbol

\item {\sl before-handlers}:  functional symbol

\item {\sl after-handlers}:  function symbol


\item [Return-value:]
T if successful.

Error object {\sl invalid-event-instance} (page~\pageref{invalid-event-instance}) if first
argument was not a legal t*event-instance.

Error object {\sl conflicting-hook-constraints} (page~\pageref{conflicting-hook-constraints}) if the 
constraints cannot be satisfied.

\item [Description:]

Adds handler-fn-name to the set of function handlers for
t*event-instance.

\item [Public:]



\end{description}
\horizontalline

\subsection{t*event*initialize}
\label{t*event*initialize}

\begin{description}
\item [Name:]  t*event*initialize

\item [Class:] {\sl t*event}\hfill(page~\pageref{t*event})

\item [Parameters:]
\item {\sl t*event-instance}:  an instance of t*event



\item [Return-value:]
T if successfully re-initialized the set of event-handlers
for event-instance to NIL.

Error object {\sl invalid-event-instance} (page~\pageref{invalid-event-instance}) if the argument
was not a t*event instance.

\item [Description:]

Resets the list of event handler functions for event-instance
to NIL.

\item [Public:]



\end{description}
\horizontalline

\subsection{t*event*update-lschema-from-constraints}
\label{t*event*update-lschema-from-constraints}

\begin{description}
\item [Name:]  t*event*update-lschema-from-constraints

\item [Class:] {\sl t*event}\hfill(page~\pageref{t*event})

\item [Description:]

Handler functions are passed a link-schema-ID and
its from node constraint expression.


\end{description}
\horizontalline

\subsection{t*event*update-lschema-to-constraints}
\label{t*event*update-lschema-to-constraints}

\begin{description}
\item [Name:]  t*event*update-lschema-to-constraints

\item [Class:] {\sl t*event}\hfill(page~\pageref{t*event})

\item [Description:]

Handler functions are passed a link-schema-ID and
its new to-node constraint expression.


\end{description}
\horizontalline

\subsection{t*event*update-lschema-name}
\label{t*event*update-lschema-name}

\begin{description}
\item [Name:]  t*event*update-lschema-name

\item [Class:] {\sl t*event}\hfill(page~\pageref{t*event})

\item [Description:]

Handler functions are passed a link-schema-ID and
its new name.


\end{description}
\horizontalline

\subsection{t*event*delete-link-schema}
\label{t*event*delete-link-schema}

\begin{description}
\item [Name:]  t*event*delete-link-schema

\item [Class:] {\sl t*event}\hfill(page~\pageref{t*event})

\item [Description:]

Handler functions are passed the link-schema-ID for
the successfully deleted link schema.

Note that schema "deletion" has a special interpretation
in Egret: the schema is not physically removed, just
marked as not available for future instantiation.


\end{description}
\horizontalline

\subsection{t*event*new-link-schema}
\label{t*event*new-link-schema}

\begin{description}
\item [Name:]  t*event*new-link-schema

\item [Class:] {\sl t*event}\hfill(page~\pageref{t*event})

\item [Description:]

Handler functions are passed a new link-schema-ID,
its name, and its to and from node constraint expressions.



\end{description}
\horizontalline

\subsection{t*event*update-fschema-val-fn}
\label{t*event*update-fschema-val-fn}

\begin{description}
\item [Name:]  t*event*update-fschema-val-fn

\item [Class:] {\sl t*event}\hfill(page~\pageref{t*event})

\item [Description:]

Handler functions are passed a field-schema-ID and
the new validity function.




\end{description}
\horizontalline

\subsection{t*event*update-fschema-name}
\label{t*event*update-fschema-name}

\begin{description}
\item [Name:]  t*event*update-fschema-name

\item [Class:] {\sl t*event}\hfill(page~\pageref{t*event})

\item [Description:]

Handler functions are passed a field-schema-ID and
its new name.


\end{description}
\horizontalline

\subsection{t*event*delete-field-schema}
\label{t*event*delete-field-schema}

\begin{description}
\item [Name:]  t*event*delete-field-schema

\item [Class:] {\sl t*event}\hfill(page~\pageref{t*event})

\item [Description:]

Handler functions are passed a field-schema-ID after
it has been successfully deleted.

Note that deletion of schemas does not result in 
removal, since their presence is required to interpret
previously created nodes. Rather, deletion indicates
that new new uses of the schema are allowed.


\end{description}
\horizontalline

\subsection{t*event*new-field-schema}
\label{t*event*new-field-schema}

\begin{description}
\item [Name:]  t*event*new-field-schema

\item [Class:] {\sl t*event}\hfill(page~\pageref{t*event})

\item [Description:]

Handler functions are passed a new field-schema-ID,
its name, and its validity function.


\end{description}
\horizontalline

\subsection{t*event*delete-ninstance-field}
\label{t*event*delete-ninstance-field}

\begin{description}
\item [Name:]  t*event*delete-ninstance-field

\item [Class:] {\sl t*event}\hfill(page~\pageref{t*event})

\item [Description:]

Handler functions are passed a node-ID and its
field-schema-ID that has just been successfully
deleted.


\end{description}
\horizontalline

\subsection{t*event*update-ninstance-field}
\label{t*event*update-ninstance-field}

\begin{description}
\item [Name:]  t*event*update-ninstance-field

\item [Class:] {\sl t*event}\hfill(page~\pageref{t*event})

\item [Description:]

Handler functions are passed a node-ID, a 
field-schema-ID, and the new field contents.

This event is also used when a new field is added
to a node.

\end{description}
\horizontalline

\subsection{t*event*update-ninstance-name}
\label{t*event*update-ninstance-name}

\begin{description}
\item [Name:]  t*event*update-ninstance-name

\item [Class:] {\sl t*event}\hfill(page~\pageref{t*event})

\item [Description:]

Handler functions are passed a node-ID and its
new name.


\end{description}
\horizontalline

\subsection{t*event*delete-node-instance}
\label{t*event*delete-node-instance}

\begin{description}
\item [Name:]  t*event*delete-node-instance

\item [Class:] {\sl t*event}\hfill(page~\pageref{t*event})

\item [Description:]

Handler functions are passed the node-ID that has just
been successfully deleted.


\end{description}
\horizontalline

\subsection{t*event*new-node-instance}
\label{t*event*new-node-instance}

\begin{description}
\item [Name:]  t*event*new-node-instance

\item [Class:] {\sl t*event}\hfill(page~\pageref{t*event})

\item [Description:]

Handler function are passed the new node-ID and
its name.


\end{description}
\horizontalline

\subsection{t*event*update-nschema-nodes}
\label{t*event*update-nschema-nodes}

\begin{description}
\item [Name:]  t*event*update-nschema-nodes

\item [Class:] {\sl t*event}\hfill(page~\pageref{t*event})

\item [Description:]

Handler functions are passed a node-schema-ID and
a list of its new node-IDs.


\end{description}
\horizontalline

\subsection{t*event*update-nschema-fields}
\label{t*event*update-nschema-fields}

\begin{description}
\item [Name:]  t*event*update-nschema-fields

\item [Class:] {\sl t*event}\hfill(page~\pageref{t*event})

\item [Description:]

Handler functions are passed a node-schema-ID and
a list of its new field-IDs.


\end{description}
\horizontalline

\subsection{t*event*update-nschema-name}
\label{t*event*update-nschema-name}

\begin{description}
\item [Name:]  t*event*update-nschema-name

\item [Class:] {\sl t*event}\hfill(page~\pageref{t*event})

\item [Description:]

Handler functions are passed a node-schema-ID and
its new name.


\end{description}
\horizontalline

\subsection{t*event*delete-node-schema}
\label{t*event*delete-node-schema}

\begin{description}
\item [Name:]  t*event*delete-node-schema

\item [Class:] {\sl t*event}\hfill(page~\pageref{t*event})

\item [Description:]

Handler functions for this instance are passed
the node-schema-ID of the just deleted node schema.


\end{description}
\horizontalline

\subsection{t*event*new-node-schema}
\label{t*event*new-node-schema}

\begin{description}
\item [Name:]  t*event*new-node-schema

\item [Class:] {\sl t*event}\hfill(page~\pageref{t*event})

\item [Description:]

Handler functions are passed the new node-schema-ID,
its name, and a list of its field-schema-IDs.




\end{description}


%%% \normalsize \onecolumn

\appendix
\chapter{Miscellaneous Addenda}
\section{Design Issues and Rationale for ECTS}
\label{app:ects}

The capabilities of ECTS to define and modify database structure
at both class and instance level raises many important and complex issues.

The first issue is the classic database schema modification impact analysis
problem.  Given a class (which can be thought of as corresponding to a database
schema) and a set of instances created from that class, what is the effect of 
making changes to the class structure?   There is a growing body of literature 
addressing the types of schema modifications that can be made without impact 
upon existing instances, as well as literature proposing mechanisms that 
dynamically and automatically restructure existing instances to bring them into
conformance with the changed schema. (Staudt-Lerner, Skarra).   

ECTS differs from this research by rejecting two of the premises upon which the
work is based: (1) that instances must necessarily be constrained to the 
structure of their parent classes' schema, and (2) that structure can only be 
defined at the class and not at the instance level.   Admittedly, these extra 
degrees of freedom can be tolerated only in exploratory domains, where their 
benefits to design experimentation outweigh their costs in reduced reliabilty 
and consistency.   The result is while this classic problem certainly exists in
ECTS, we address the problem in a different way that reflects the differences in
our domain.  This answer will be developed in more detail below.

A second issue concerns inheritance vs. copying of structural information.  In 
classic database schema data models, structural information is always inherited.
Conceptually, the structure of all instances conforms to a single structural 
description that is maintained within the schema for the parent class.  As a 
result, changes to the structure of the class' structural information must 
necessarily change the structure of all its instances. 

In ECTS, copying as well as inheritance of structural information is
needed.  Inheritance is needed for all the usual reasons: structural
specification, structural consistency, and code sharing. Copying is
also needed in order to experiment with changes to a structure without
making an impact upon previously created structures. Thus, if an
instance copies structural information from a class (or another
instance), then any changes it makes to that structural description
will be local to the instance.  Conversely, changes to the original
structural description from which the copy was made will not affect
the copying instance.

Given the ability to change structural attributes at both the class
and the instance level, and the ability to both inherit and copy
attributes, a wide variety of models for the structural relationships
between classes and instances are possible.

To facilitate this discussion, we adopt the following notation.  Let upper case A, B,
and C denote different classes, and let lower case a1, a2, a3...,  b1, b2, b3, 
and so forth denote instances of these classes.  Let p denote a 
structural property, and let p' denote a copy of property p.   Let -> indicate 
either an inheritance relationship or a copying relationship between two 
classes/instances.  If the relationship is inheritance, then p appears on both 
sides; if copying, then p' appears on the right hand side.   Finally, let A/p 
indicate that class A has property p.

The set of possible relationships can then be summarized as follows:
\small\begin{verbatim}
A/p -> B/p                (class B inherits property p from class A)
A/p -> B/p                (class B copies property p from class A)
A/p -> a1/p               (instance a1 inherits property p from class A)
A/p -> a1/p'              (instance a1 copies property p from class A)
a1/p  -> b1/p             (instance b1 inherits property p from instance a1)
a1/p  -> a2/p             (instance a2 inherits property p from instance a1)
a1/p  -> b1/p'            (instance b1 copies property p from instance a1)
a1/p  -> a2/p'            (instance a2 copies property p from instance a1)
\end{verbatim}\normalsize

The true set of possible relationships is much more complex, since it is the 
transitive closure of these relationships over finite sets of properties, 
classes, and instances.  However, this summarization provides an idea of the 
complexity of relationships between classes and instances that could be 
supported in the ECTS model.

It is our assertion, in fact, that this set of possible relationships is too 
general for exploratory purposes.  

First, allowing unrestricted possibilities for inheritance and copying of 
properties essentially eliminates the distinction between classes and instances
entirely.  Thus, it allows users to essentially set up parallel, first class 
schema systems within a single database system, a situation we explicitly wish 
to avoid.  

Second, it very easy for users to create databases with complicated mixtures of
inheritance and copying within and between classes and instances that make 
effective impact analysis and subsequent "rationalization" of the database
during consolidation extremely difficult.   

Third, supporting this generality will impose representational and computational
overhead that we would like to avoid.

Instead, ECTS supports only the following forms of copying and inheritance:

\small\begin{verbatim}
A/p -> B/p                (class B inherits property p from class A)
A/p -> a1/p               (instance a1 inherits property p from class A)
a1/p  -> b1/p'            (instance b1 copies property p from instance a1)
a1/p  -> a2/p'            (instance a2 copies property p from instance a1)
\end{verbatim}\normalsize

To motivate this decision, recall that the requirements for the ECTS is to allow
exploration, or deviation from the consensus representation embodied by the 
class schema, while still providing the means necessary to bring some or all of
the instances and classes back into consistency through modifications to both 
the class schema and the instances.

These restrictions are easily understood by the simple three-part rule: classes
only inherit; instances only inherit from their class, and instances can copy 
from any other instance.  They preserve the separate statue of classes as the 
sole manner to obtain structural sharing and economy of expression; building a 
separate hierarchy at the instance level is not economical.  Finally, it 
provides a crucial aid to experimentation by allowing users to create an 
instance of a class, then freely import interesting structural ideas from any 
other existing instance.  Once a suitable instance structure has been generated,
it can be easily ``cloned" for further experimentation in different contexts.  
When the utility of this experimental structure has been proven and agreed upon,
it can be reified within the class hierarchy itself.

ECTS also recognizes that in exploratory situations, it is not
critical for all instances of a class to be consistent with the
current schema structure, since exploratory domains generate
significant amounts of ``archivable garbage"---i.e. outmoded instances
that should be retained in order to preserve the historical evolution
of the system, but which are no longer current to the task at hand.

ECTS represents this distinction between ``active" instances and
classes and ``historical" instances and classes by the aggregation
constructs of layers and surfaces.  Collaborative exploratory
development consists of a current, top-level surface upon which new
activities takes place.  The tools for assessing variance between
classes and instances are normally confined to the structures on the
current surface.  (In standard software engineering terms, surface
corresponds most closely to a configuration.)  Surfaces are made up of
a set of layers, within which semantically cohesive sets of classes
and instances are aggregated.  (Layers correspond most closely to the
concept of a framework in object oriented terminology combined with
the concept of version from change control terminology.)


\section{Efficient Type-level Node Retrieval}
\label{app:net}

The ability to dynamically define new fields at the class or instance
level leads to an important efficiency problem.  First, remember that
server subsystem retrieves information on a field-by-field basis.
This behavior is maintained at the type subsystem, except that these
retrieval operations allow uniform transparent access to either the
fixed server node fields or the dynamic type node fields.

Unfortunately, the straightforward implementation of this design leads
to one of two situations: (1) The presentation of a node's contents by
interface mechanisms involves multiple retrieval operations for each
field in the node, each resulting in a separate type access and
unpacking operation; or (2) The presentation of a node's contents by
interface mechanisms involves a single retrieval operation on the DATA
field.  This latter approach has the advantage of retrieving an entire
group of node fields with a single network access.  However, it has
the disadvantages that the interface mechanism now is given knowledge
of the internal representation of nodes (in order to know that the
DATA field contains the required information to begin with), as well
as requiring the interface mechanism to know how to unpack the DATA
field in order to get at the desired information.

To partially overcome this disadvantage, the type subsystem not only
exports to other subsystems the basic retrieval operation for any
field defined on a particular node instance, but a set of unpacking
operations for the DATA field as well.  These unpacking operations
take a node-ID, its DATA field contents, and a field-name. They return
the field-name's contents, or a (continuable) error if the field-name
does not exist in the DATA field.  \foot{Note that the node-ID is required
since the unpacking operation might be type dependent.}  These
unpacking operations hide the actual representation of fields inside
the server subsystem DATA fields, allowing this design decision
(though not the design decision as to whether to store the field
within the DATA field or someplace else) to remain encapsulated in the
type subsystem.

The advantage of this design is that if another subsystem just needs
to know the value of a single field in a node instance, they can
simply invoke the standard type subsystem retrieval operation with
the field name and get that information without needing to know
whether the field was fixed or dynamic, without having to unpack it,
and without paying any performance penalty.  This design also allows 
prototypes to be constructed and later optimized with additional code
for DATA field handling.

On the other hand, if another subsystem needs to efficiently retrieve
the value of several fields in a node instance, it can go through the
slightly more complicated procedure of retrieving the DATA field and
manually invoking the unpacking operations to get the values.  In this
latter case, the higher subsystem needs to know somewhat more about
the internal representation (for example, it needs to know that the
fields it wants are actually in the DATA field), but it still doesn't
need to know exactly how the field contents are represented in the
DATA field.  And, most importantly, this avoids paying any performance
penalty for the information.

\section{Information Replication in Egret 2.0}

A fundamental architectural feature of Egret 2.0 is local replication,
or caching, of globally maintained information.  Local caching
improves performance but requires mechanisms to maintain consistency.
Egret 2.0 addresses this problem by {\em recovery operations}. We
first discuss the nature of information replication in Egret 2.0,
followed by the recovery operations designed to preserve their
consistency.

Global replication occurs when information in the global, shared
database is represented in more than one way.  For example, the set of
node and link instances in the database forms a network. Aspects of
this network might be redundantly represented, such as via a ``special
node'' that maintains, for each node, the set of links pointing to
that node.  Such a special node greatly speeds the computation of a
query such as ``Determine all the nodes that are linked to this
node'', which might otherwise require a complete traversal of the
entire network to answer.

Local replication occurs when information in the database is copied to
each local client during its connection to the database.  For example,
at connect time, a list of all node-IDs and their names in the
database is transmitted to the connecting client. As other
simultaneously connected clients create and delete nodes, the local
copy of this list is updated via events. Thus, if five clients are
connected to a database server, there are six copies of this information
in use: the physical copy represented by the database itself, plus the
five representations that are locally maintained by each connected
client.

It is important to note that consulting local information maintained
via the event mechanism leads to the possibility of presenting
erroneous information to a client.  As a simple example, suppose one
client queries Egret 2.0 for the total number of nodes in the database
just after another client deletes a node. If the local
client process determines the answer to this query by computing the length of the
local list of node-IDs, and if this computation is performed before the
event updating this list is received, then the query result will be
off by one.  Therefore, local information should not be used when this
event propogation delay could produce significant problems (for
example, a locally maintained list of locked nodes should not be
consulted in order to determine whether to grant a lock.)
 
If Egret 2.0 could be perfectly implemented and if its supporting
network and workstation environment could function without failure,
consistency maintenance would not be an issue.  Unfortunately, client
connections fail unpredictably, implementation errors prevent
propogation of updated information, and local and global state become
corrupted on a regular basis.  For this reason, it is the
responsibility of the implementor of any mechanism that holds either: state
information (such as an ``unread nodes'' facility); locally maintained
information (such as the ``cv-nodes'' list);  or globally maintained
information (such as the ``type database'' facility) to also implement
recovery operations.

Recovery operations repair replicated information that
has somehow gotten out of sync, or corrupted. For example, the 
recovery operation for the locally maintained list of node-IDs would
send the database server a request to re-transmit a fresh copy of this list
to the client requesting it.  The recovery operation would then replace
the corrupted list with the newly received one. 

The actual repair mechanism depends upon the nature of the replication
information, and may not entirely restore the system to its state
prior to the moment of corruption. For example, a recovery operation
for a corrupted unread nodes information might simple re-initialize
the unread nodes list for a given user (or all users) to the empty
list. 

Recovery operation names are by convention prefixed with {\bf reset-},
and whenever possible suffixed with the name of the attribute or
operation whose functioning they repair.  For example, the server
subsystem node class attribute {\bf s\STAR node\STAR incoming-links}
has a corresponding recovery operation called {\bf s\STAR
node!reset-incoming-links}.

Recovery operations are normally designated as private operations of
their class.  This is to encapsulate the decision as to whether
information will be maintained locally or globally within the class
implementation.  However, these operations are not entirely hidden,
since the knowledge of corruption may only become known within other
subsystems (most notably, the user interface subsystems).  Therefore,
the designer of a class must be responsible for publishing in the
public interface to the class these private operations.  They form
a special, distinguished class of private operations: operations
only to be used under exceptional conditions. 


\section{Implementation Changes to CoReView 1.2}
\label{app:coreview}

This section presents a potpourri of some of the open questions and challenges 
confronting us in the implementation of Egret 2.0.

\begin{itemizenoindent}

\item {\bf Design maintenance.}  The implementation of Egret 2.0 now 
contains four discrete parts:  this document, which provides an overview
of the design; the CoReView 1.2 hyperbase that actually contains the 
documentation of the individual classes, attributes, and operations; 
the testing framework and database; and the actual code implementing the system.
Keeping these descriptions consistent and up-to-date will require new
procedures.  Hopefully, this issue will be one that Danu's research will 
provide solutions for.

\item {\bf The test framework.}  While better testing is a major design
goal for Egret 2.0, work needs to be done in establishing just what a 
declarative specification of the test cases looks like, and how an 
automated database will work.

\item {\bf Instance-level field creation.}  A new behavior in Egret 2.0
is the ability to define new fields at the instance, as well as the type,
level.

\item {\bf Type-level status object for operation return values.}
Every Type level operation needs to return information about the 
success of the operation.  It may potentially need to return a range 
of error values providing details on why the operation did not succeed.
This argues for some kind of generic ``status'' object.  Error checking
must be done at the Type-level in order to allow full programmatic
access to type operations.

\item {\bf The arg-check mechanism.}  For my thesis, I implemented a 
Common Lisp macro that performed run-time checking of the argument
types to functions, signalling an error if an incorrect argument type
was provided. (This macro had a compile-time switch that allowed it to
be compiled out of the code, so that only the development version of
the system suffered from time and space overhead.) I found this
mechanism {\em very}\/ useful in debugging my thesis system. As an
additional reliability enhancement measure, I would like to arg-check
all of the functions in Egret 2.0.

\item {\bf Generalizing the link mechanism.}  I would like to propose that
we re-evaluate the current restrictive interpretation of links as
relations between two nodes.  For one thing, I find the current naming
conventions, such as ``is-a-reference-of'', to be user-hostile: I
find this notation difficult to parse and understand.  I would much
rather see such links labeled something like: ``To: Reference Node
(pmj)'' which gives me a much clearer notion of what is going on.  In
addition, this form of link also goes away from the notion of relation
and toward the notion of operation, which is a more general
interpretation of what can happen when a link-label is clicked on by
the mouse.

\item {\bf Separating node display and node retrieval.} The new design
allows a node to be retrieved from the hyperbase and put into its own buffer
without it necessarily being displayed.  The old design could only do
this within the context of a ``system node.''  

\item {\bf System nodes.}  The system node processing can be 
implemented by the Type-level operations to programmatically get at the 
contents of nodes.  In particular, system node processing should no longer
create a node implicitly if it doesn't exist (this could be an optional
switch, however.)  System nodes could be implemented in one of two 
ways: first, by requiring that the contents of all system nodes be
an evaluable lisp expression and then simply eval the string that is
returned from the HB call, or second, by putting the contents into a 
buffer but not displaying it.  In either of these situations, the notion
of a distinguished system node buffer should go away.

\item {\bf Locking.}  CoReView 1.2 has a big problem with locking nodes
programmatically but never unlocking them (due to aborting.)  There needs
to be a with-locked-node wrapper that wraps an unwind-protect form around
the lock operation to guarantee that unlocking takes place.

\item {\bf Generic Hidden Fields.}  I propose that we have a single 
``Hidden Field'' in the node contents which is evaluated in its entirety
when the node is being set up for display.  A set of language constructs
will be provided to put a lisp form into this hidden field area. There
seems to be needless duplication of code and effort in the current
implementation, when each subsystem has to set up and maintain its own
hidden fields.

\item {\bf Fixing bad modularity in CoReView 1.2.}  The current system
is structured very differently in some places.  For example, the
single function cv-user-delete-link contains code from the Interface, 
Type, and HB levels.  Be on the lookout for these situations.

\item {\bf Egret vs. CoReView Mode.}  We should design a separate
mode for Egret and CoReView in this next release.  

\item {\bf Concurrency issues.}  One concurrency problem that is not
handled in CoReView 1.2 arises in the following scenario:
\begin{enumerate}
\item cv-create-node checks cv-nodes to see if ``foo'' is an already existing name.
\item a new node event arrives that defines a new node ``foo''
\item cv-create-node calls the HB function to create node ``foo''.
\end{enumerate}
Clearly, checking cv-nodes in this way is not sufficient to see if ``foo'' already
exists.  Since the hyperbase does not require node names to be unique, we
need some kind of roll-back mechanism, or at least the ability to signal an
error if a new node event arrives with a node name that already exists.

\item {\bf Link-button and region-button bidirectional pointers.} Note that
in the new design, this bidirectionality exists between a ``region'' and
a ``link'', not between two buttons.

\item {\bf Regions as nodes.}  Future designs may lead to the current concept
of ``region'' being elevated to first class status as a ``node''.  This would
provide a natural concept of composite object not currently present in the 
design.  Future research is needed to determine the impact of this design
change.

\item {\bf Node name uniqueness.}  In Coreview 1.2, the node names are 
required to be unique, mostly in order to allow completion, but also because
many functions accepted a node name as a unique ID.  In Egret 2.0, the 
uniform use of node IDs to distinguish nodes makes it much easier to 
relax this restriction on node names.   

\end{itemizenoindent}


\bibliography{csdl-trs}
\bibliographystyle{/home/13/csdl/tex/named-citations}


\end{document}







