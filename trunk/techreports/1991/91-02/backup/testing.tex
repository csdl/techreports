\documentstyle [11pt, /home/13/csdl/tex/definemargins,           
                /home/13/csdl/tex/lmacros,                      
          /home/13/csdl/tex/functiondoc]{article} 

\begin {document}
\begin {center}
{\large\bf Egret Test System}
  
\medskip 
\today 
\end {center} 

\ls{1.2}

\section {Introduction}
The objective of the Egret test system is to detect the presence
of errors in the Egret-2.0 implementation in general, and to detect 
deviation of the implementation from the design specification specifically.

The test system is object-oriented.  It is internally organized as 
a set of classes, attributes and public operation.  It maintains
a database of test cases, test data, a set of precondition and
postcondition functions, and a set of operations to run the test system.

The test strategy includes applications of various test cases, each having
various test data, to the operations (Egret-2.0 public operations).

Each test-case  can be broken down into four parts: 
preconditions, an operation to be tested, test-data,
and postconditions.  Preconditions are sequences of operations
to set the behavior of the system prior to testing the operation, 
test-data are parameter-values passed to the operation during
the testing, and postconditions are sequences of operations 
to assess the result of the testing by examining the state of the 
system.  Multiple preconditions or postconditions  will be executed
in the order of their appearances in the test-case definition.

Since any operations used in preconditions or postconditions are
also part of the Egret system which deem to be error prone, the test
system protects itself against possible malfunctions of these operations
during execution. 
If the operations abort in the middle of execution, they will not
interfere with other test cases. 


\section {Defining test-cases}

Running a test case include defining the operation to be tested, 
its corresponding test data, preconditions, and postconditions. 
Specifically, for each test-case definition, we may have  zero or
more precondition definitions, test-data definitions,
and postcondition definitions, but only one operation
definition.  The latter describes the operation to be tested
by the test case.

When writing a test case, the following template definition must be used:

\begin {enumerate}
\item {\bf Test-case definition:}

\small\begin{verbatim}
(tf*tcase*define
     :name           ;A unique name of the test-case. 
                     ;The name can not be separated by any white spaces.
     :documentation  ;A documentation string describing the test-case.
     :operation      ;The name of the operation definition 
     :data           ;The name(s) of the data definition(s) for the test-case.
                     ;Multiple data definitions must be enclosed within 
                     ;parentheses.
     :preconditions  ;The name(s) of the precondition definitions 
     :postconditions ;The name(s) of the postcondition definitions for 
                     ;this test case.
)	
\end{verbatim}\normalsize

The documentation string in the definition must follow the style given
in the subsequent examples below.  The test system will generate
the natural language description of a test case based on these documentation
strings.

The test-case definition specifies a scenario to test the operation described
in the operation field.  

\noindent For example, 

\small\begin{verbatim}
(tf*tcase*define
 :name s*sp*connect:normal-connection
 :documentation "tests that a connection can be made to the server in a normal manner"
 :operation Normal-connection
 :data (Valid-machine-name1 Valid-machine-name2)
 :preconditions Ensure-no-connection
 :postconditions Connection-exist
 )
\end{verbatim}\normalsize

This test case tests that a connection can be made to the server in a normal
manner.  First, it runs the operation defined in the precondition 
Ensure-no-connection, then it invokes the operation to be tested defined
in  Normal-connection with parameter values defined in Valid-machine-name1
or Valid-machine-name2 depending on which test-data is being run, and
finally it executes the operation defined in the postcondition 
Connection-exist.

When running a test case, a specific test data must be included or nil if
no test-data is associated with this test case. In the above example,
both Valid-machine-name1 and Valid-machine-name2 are eligible for testing
s*sp*connect in a normal manner.  Thus, to test this particular scenario
of connection (normal connection) comprehensively, the test case
must be run twice, first with the test data Valid-machine-name1 and then 
with the test data Valid-machine-name2.

To test the operation s*sp*connect in general, other connection scenarios 
must be included.  These other scenarios are defined in  separate
test cases. We use the term {\it collection} when referring to a set of
test-cases which tests a specific operation, and the name of the collection
is the name of the operation to be tested.


\item {\bf Operation definition:}

\small\begin{verbatim}
(tf*toper*define
 :name  	; the unique name of the operation-definition.
 :documentation ; the documentation string of this definition
 :op-name	; the name of the operation/function.
 :ret-type	; the expected return type from this operation.
 :ret-value	; the expected return value from this operation.
 )
\end{verbatim}\normalsize

Operation definition describes the expected return-value of an operation
when the test case is run. There are currently four different
types of return-value: {\it object}, {\it item}, {\it predicate} and 
{\it error}.

The object indicates the operation will return a value 
specified in the return-value field of the definition. 

The item indicates the operation will return one of the values
specified in the return-value field of the definition.  Thus, the
return value field must contain a list of the expected return values.

The predicate indicates  the operation will return a specific 
type of objects regardless of the value (e.g., integer), and
testable using the predicate function specified in the return-value
field of the definition.  Thus, specifying a type predicate requires specifying
the name of the predicate function in the return-value field.

The error indicates the operation will return an error object 
with the error-name specified by the return-value field of the
definition. The error name must conform to the Egret's specification.

\noindent For example, the operation definition for the test-case above

\small\begin{verbatim}
(tf*toper*define
 :name Normal-connection
 :documentation ``expects the return-value \"zero.ics.hawaii.edu\"''
 :op-name s*sp*connect
 :ret-type object
 :ret-value "zero.ics.hawaii.edu"
 )
\end{verbatim}\normalsize

This definition specifies that the function s*sp*connect will return
``zero.ics.hawaii.edu'' when its corresponding test case is run regardless
of the test data being supplied (in this case, both Valid-machine-name1 
and Valid-machine-name2 are applicable)

\item {\bf Test-data definition}

\small\begin{verbatim}
(tf*tdata*define
 :name		; the unique name of the test-data definition 
 :documentation ; the documentation string of this definition
 :values	; the data passed to the operation defined in the test-case
 )
\end{verbatim}\normalsize

The values field contains the actual data passed (as the parameter) to the 
operation defined by the test case.

If more than one test-data are defined, they all need to conform to the
operation as well as the preconditions and the postconditions defined 
by the test-case.

\noindent For example, the test-data definitions for the above test-case

\small\begin{verbatim}
(tf*tdata*define
 :name Valid-machine-name1
 :documentation ``a valid machine name \"zero.ics.hawaii.edu\"''
 :values ("zero.ics.hawaii.edu")
 )

(tf*tdata*define
 :name Valid-machine-name2
 :documentation ``a valid machine name \"zero.ics\"''
 :values ("zero.ics")
 )
\end{verbatim}\normalsize

These definitions specify that the test-case s*sp*connect:normal-connection
will invoke s*sp*connect with the parameter value either
``zero.ics.hawaii.edu''or ``zero.ics''.

\item {\bf Precondition definition:}

\small\begin{verbatim}
(tf*tpre*define
 :name 		; the unique name of the precondition-definition
 :documentation ; the documentation string of this definition	
 :op-name 	; the operation/function name
 :param-values  ; the parameter-values passed to the op-name above
 :ret-type 	; the expected return-type of the operation/op-name above.
 :ret-value 	; the expected return-value of the op-name above.
 )
\end{verbatim}\normalsize

\noindent For example,

\small\begin{verbatim}
(tf*tpre*define
 :name Ensure-no-connection
 :documentation "ensures that no connection already exists to the server" 
 :op-name tf*no-connection
 :param-values nil
 :ret-type object
 :ret-value t
 )
\end{verbatim}\normalsize

This precondition specifies that the operation tf*no-connection will
be invoked with parameter nil and will have return-value t on successful
invocation.

\item {\bf Postcondition definition:}

\small\begin{verbatim}
(tf*tpost*define
 :name		; the unique name of the postcondition definition.
 :documentation ; the documentation string of this definition
 :op-name 	; the name of the postcondition operation.
 :param-values  ; the parameter values passed to the op-name above
 :ret-type 	; the expected return-type of the op-name above.
 :ret-value	; the expected return-value of the op-name above.
 )
\end{verbatim}\normalsize

\noindent For example,

\small\begin{verbatim}
(tf*tpost*define
 :name Connection-exist
 :documentation "ensures that a connection exists to the server"
 :op-name tf*connected-p
 :param-values nil
 :ret-type object
 :ret-value t
 )
\end{verbatim}\normalsize

This postcondition specifies that the operation tf*connected-p
will be invoked with parameter nil and will have return-value t on
successful invocation.

\end {enumerate}

Definitions (except test-case definitions) can be shared among different
test-cases.
For example, the test-data definitions of the
test-case s*sp*connect:reestablish-connection uses
the same test-data definition of the s*sp*connect:normal-connection.  

\small\begin{verbatim}
(tf*tcase*define
 :name s*sp*connect:reestablish-connection
 :documentation  "tests that a connection can be made to the server even when it already exists"
 :operation Reconnect
 :data (Valid-machine-name1 Valid-machine-name2)
 :preconditions Ensure-connection
 :postconditions Connection-exist
 )
\end{verbatim}\normalsize

This test-case uses the same test-data and postconditions definitions
of the test-case s*sp*connect:normal-connection.  
This is applicable because both test-cases test the same operation
s*sp*connect with the same parameter-values (Valid-machine-name1 and
Valid-machine-name2).

The operation definition, however, is not shareable because s*sp*connect
for reestablish-connection expects a different return value  from the one
in the s*sp*connect:normal-connection.  Specifically, the definition of
Reconnect expects return-type error and return-value connection-is-on, 
whereas the definition of Normal-connection expects return-type object
and return-value ``zero.ics.hawaii.edu''

Although, two test-cases may share the same definitions (operation, test-data,
preconditions and postconditions), the testings are completely independent.
The intermediate or final results from one test case do not interfere
with the intermediate or final results from the other test case.
The user is responsible for ensuring that this condition is met
when defining the test-cases.
 
Within the same test-case, however, intermediate results may affect other 
operations defined elsewhere. The user is allowed
to refer to the return-value of any operations within the same test-case.
For example, a precondition can have a parameter value which comes from 
the return-value of another precondition, or the value of test-data comes 
from the return-value of a precondition.  

The convention is when referring to the return value
of an operation or a function defined in the preconditions or 
postconditions, the name of the definition must be used.
Furthermore, the user is responsible for ensuring the order in which
the definitions will be invoked.

\noindent For example,

\noindent Test-case definition:

\small\begin{verbatim}
(tf*tcase*define
 :name s*link*name:valid-link-id
 :documentation "tests that the attribute link-name can be retrieved in a normal manner"
 :operation Normal-link-name
 :data Valid-link-id
 :preconditions (Ensure-connection Create-node1 Create-node2 Create-link1)
 )
\end{verbatim}\normalsize

This test-case tests that the operation defined in Normal-link-name
will behave in a normal manner when the test-data Valid-link-id is applied.
The preconditions first invoke the operation defined in
Ensure-Connection, then Create-node1, then Create-node2, and finally
Create-link1.

\noindent Test-data definition for Valid-link-id:

\small\begin{verbatim}
(tf*tdata*define
 :name Valid-link-id  
 :documentation "a valid link-id resulting from the precondition create link operation"
 :values Create-link1 
 )
\end{verbatim}\normalsize

This test-data supplies a valid link id obtained from the Create-link1
execution.  The ordering constraint is satisfied because
Create-link1 is one the preconditions in the test case; thus
it will be executed prior to running this test-data.  


\section {Features and Limitations of the test system}

Conceptually, running a test case is similar to invoking a set of LISP 
functions in the order described by the test case definition.
However, unlike a regular LISP function, the parameter values passed
to the function defined as a precondition or postcondition must
follow the following rules:
\begin {enumerate}
\item Any symbolic name declared in the {\it param-values} slot
will be passed to the operation uninterpreted except when the name
refers to the existing definition (precondition, postcondition or operation
definition) of the test case. In this case, the symbolic name
will be replaced by the actual object returned by the corresponding
definition (see similar example in the data definition for 
Valid-link-id above). When no replacement takes place,
the symbol will appear as is to the operation.
For example,
\small\begin{verbatim}
    ..
    :op-name foobar
    :param-values (foo1 foo2 foo3)
     ..
\end{verbatim}\normalsize

will be evaluated as a function call to (foobar foo1 foo2 foo3).
Thus, foo1, foo2 and foo3 are passed as variables  to the function
foobar.  To pass foo1, foo2 or foo3 as symbol, a quote must be 
included, i.e., :param-values ('foo1 'foo2 'foo3).


\item When a (list-form) appears in the param-values slot, the list
will be passed as '(list-form) to the operation, except when the
list-form is a function call.
For example,
\small\begin{verbatim}
    ..
    :op-name foobar
    :param-values (foo1 (foo2 foo3))
     ..
\end{verbatim}\normalsize

will be evaluated as (foobar foo1 '(foo2 foo3)), but


\small\begin{verbatim}
    ..
    :op-name foobar
    :param-values (foo1 (foobar2 foo3))
    ..
\end{verbatim}\normalsize

will be evaluated as (foobar foo1 (foobar2 foo3)) given that
foobar2 is a valid function.

This latter form should not be used to run  operations or functions
which have not been proven correct, such as the operations under testing; 
we must apply the operation using a test case definitions instead. 
Thus, in the example, if the function foobar2
is part of the operation to be tested, then foobar must be split into
two definitions:

\small\begin{verbatim}
    (tf*tpre*define
    :name foobar-definition
    :op-name foobar
    :param-values (foo1 foobar2-definition)
    ..
    )
\end{verbatim}\normalsize


\small\begin{verbatim}
    (tf*tpre*define
    :name foobar2-definition
    :op-name foobar2
    :param-values (foo3)
    ..
    )
\end{verbatim}\normalsize

\item Expected return value slots (:ret-value) do not accept 
Lisp function forms, except for the return value of type predicate
which accepts functions with one argument.  Thus, for example, to
test whether the return value of function foobar1 is equal to the
return value of foobar2, the user cannot use the following definition:
\small\begin{verbatim}
  (tf*toper*define
  :name foobar1-definition
  :op-name foobar1
  :param-values (foo1 foo2)
  :ret-type object
  :ret-value (foobar2)
  )
\end{verbatim}\normalsize
The user must instead define the foobar2-definition (e.g., as a precondition
function) and supply
its definition in the ret-value slot of the foobar1 definition:
\small\begin{verbatim}
  (tf*toper*define
  :name foobar1-definition
  :op-name foobar1
  :param-values (foo1 foo2)
  :ret-type object
  :ret-value foobar2-definition
  )
\end{verbatim}\normalsize


\item In some cases,  we must write our own precondition or postcondition 
functions.
For example, in the precondition {\bf Ensure-no-connection} above, 
we use the function tf*no-connection which is not part of the operations
being tested. If the implementation of this function requires or calls
any operations being tested, then the latter call must be quoted
using the following macro:

(with-error-handler (operation-name args))\

This macro will ensure any errors resulting from the execution of the
operation-name be caught appropriately, instead of aborting
the operation.  
When dealing with error condition, this user's supplied precondition
must follow the same convention  as given in the EGRET specification.

However, whenever possible it would be better simply pass the error as 
the return value of the function, and thus,  be caught by the test system 
itself. For more  information, see the implementation of tf*no-connection.
\end{enumerate}

\section {Running test-cases}

The current version of  test system requires all test-cases 
be compiled and loaded prior to running the test. The system
only performs limited syntactic and semantic checking during 
compilation of the test-case definitions.  The system will report
any errors due to incorrect use of the definition template,
or existence of multiple definitions.  Missing or incorrect
definitions, however, will not be detected until run time.

\noindent To run the test system, the user first brings up the system
to be tested, then loads the compiled version of the 
test system.

\noindent There are two ways to run the test:
\begin {enumerate}

\item {\bf Atomic function: tf*tcase*run {\it test-case-name} {\it \&optional test-data-name.}}
This function runs a specific {\it test-case-name} with a specific
{\it test-data-name}.  The latter is optional indicating no test-data
is specified for the  test-case-name.
This function returns t when the test runs successfully, or nil otherwise.

\item {\bf Interactive: tf*\{tcase\}*run.}
This command runs a collection of test-cases specified by the
user during the interactive session.  Completion table for the existing
or the user's supplied collection-name are also provided.  

At the end of the testing, the test system will enter the 
{\bf Egret-Test} mode.
In this mode, the user can analyze the result of the testing. A number
of commands are provided to browse the test cases and to interpret the
result. The help menu is available by typing {\bf ?}.

Two buffers will be created: buffer {\it Test-result} which summarizes the
the result of the test including a brief description of each test-case
and buffer {\it Test-log} which shows actual sequence of operations
executed  during the test. 
\end {enumerate}

\end {document}


