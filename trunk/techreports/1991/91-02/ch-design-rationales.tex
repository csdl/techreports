\chapter{Design Rationales}
\section{ECTS Design Rationale}
\label{ects-dr}

The capabilities of ECTS to define and modify database structure at
both class and instance level raises many important and complex
issues.

The first issue is the classic database schema modification impact
analysis problem.  Given a class (which can be thought of as
corresponding to a database schema) and a set of instances created
from that class, what is the effect of making changes to the class
structure?  There is a growing body of literature addressing the types
of schema modifications that can be made without impact upon existing
instances, as well as literature proposing mechanisms that dynamically
and automatically restructure existing instances to bring them into
conformance with the changed schema. (Staudt-Lerner, Skarra).

ECTS differs from this research by rejecting two of the premises upon
which the work is based: (1) that instances must necessarily be
constrained to the structure of their parent classes' schema, and (2)
that structure can only be defined at the class and not at the
instance level.  Admittedly, these extra degrees of freedom can be
tolerated only in exploratory domains, where their benefits to design
experimentation outweigh their costs in reduced reliabilty and
consistency.  The result is while this classic problem certainly
exists in ECTS, we address the problem in a different way that
reflects the differences in our domain.  This answer will be developed
in more detail below.

A second issue concerns inheritance vs. copying of structural
information.  In classic database schema data models, structural
information is always inherited.  Conceptually, the structure of all
instances conforms to a single structural description that is
maintained within the schema for the parent class.  As a result,
changes to the structure of the class' structural information must
necessarily change the structure of all its instances.

In ECTS, copying as well as inheritance of structural information is
needed.  Inheritance is needed for all the usual reasons: structural
specification, structural consistency, and code sharing. Copying is
also needed in order to experiment with changes to a structure without
making an impact upon previously created structures. Thus, if an
instance copies structural information from a class (or another
instance), then any changes it makes to that structural description
will be local to the instance.  Conversely, changes to the original
structural description from which the copy was made will not affect
the copying instance.

Given the ability to change structural attributes at both the class
and the instance level, and the ability to both inherit and copy
attributes, a wide variety of models for the structural relationships
between classes and instances are possible.

To facilitate this discussion, we adopt the following notation.  Let upper case A, B,
and C denote different classes, and let lower case a1, a2, a3...,  b1, b2, b3, 
and so forth denote instances of these classes.  Let p denote a 
structural property, and let p' denote a copy of property p.   Let -> indicate 
either an inheritance relationship or a copying relationship between two 
classes/instances.  If the relationship is inheritance, then p appears on both 
sides; if copying, then p' appears on the right hand side.   Finally, let A/p 
indicate that class A has property p.

The set of possible relationships are summarized in Figure \ref{inher}.

\begin{figure*}
\figureline
\small\begin{verbatim}
A/p -> B/p                (class B inherits property p from class A)
A/p -> B/p                (class B copies property p from class A)
A/p -> a1/p               (instance a1 inherits property p from class A)
A/p -> a1/p'              (instance a1 copies property p from class A)
a1/p  -> b1/p             (instance b1 inherits property p from instance a1)
a1/p  -> a2/p             (instance a2 inherits property p from instance a1)
a1/p  -> b1/p'            (instance b1 copies property p from instance a1)
a1/p  -> a2/p'            (instance a2 copies property p from instance a1)
\end{verbatim}\normalsize
\figureline
\caption{Possible inheritance relationships}
\label{inher}
\end{figure*}

The true set of possible relationships is much more complex, since it is the 
transitive closure of these relationships over finite sets of properties, 
classes, and instances.  However, this summarization provides an idea of the 
complexity of relationships between classes and instances that could be 
supported in the ECTS model.

It is our assertion, in fact, that this set of possible relationships is too 
general for exploratory purposes.  

First, allowing unrestricted possibilities for inheritance and copying of 
properties essentially eliminates the distinction between classes and instances
entirely.  Thus, it allows users to essentially set up parallel, first class 
schema systems within a single database system, a situation we explicitly wish 
to avoid.  

Second, it very easy for users to create databases with complicated mixtures of
inheritance and copying within and between classes and instances that make 
effective impact analysis and subsequent "rationalization" of the database
during consolidation extremely difficult.   

Third, supporting this generality will impose representational and computational
overhead that we would like to avoid.

Instead, ECTS supports only the forms of copying and inheritance in Figure
\ref{inher-actual}.

\begin{figure*}
\figureline
\small\begin{verbatim}
A/p -> B/p                (class B inherits property p from class A)
A/p -> a1/p               (instance a1 inherits property p from class A)
a1/p  -> b1/p'            (instance b1 copies property p from instance a1)
a1/p  -> a2/p'            (instance a2 copies property p from instance a1)
\end{verbatim}\normalsize
\figureline
\caption{Actual inheritance relationships}
\label{inher-actual}
\end{figure*}

To motivate this decision, recall that the requirements for the ECTS is to allow
exploration, or deviation from the consensus representation embodied by the 
class schema, while still providing the means necessary to bring some or all of
the instances and classes back into consistency through modifications to both 
the class schema and the instances.

These restrictions are easily understood by the simple three-part rule: classes
only inherit; instances only inherit from their class, and instances can copy 
from any other instance.  They preserve the separate statue of classes as the 
sole manner to obtain structural sharing and economy of expression; building a 
separate hierarchy at the instance level is not economical.  Finally, it 
provides a crucial aid to experimentation by allowing users to create an 
instance of a class, then freely import interesting structural ideas from any 
other existing instance.  Once a suitable instance structure has been generated,
it can be easily ``cloned" for further experimentation in different contexts.  
When the utility of this experimental structure has been proven and agreed upon,
it can be reified within the class hierarchy itself.

ECTS also recognizes that in exploratory situations, it is not
critical for all instances of a class to be consistent with the
current schema structure, since exploratory domains generate
significant amounts of ``archivable garbage"---i.e. outmoded instances
that should be retained in order to preserve the historical evolution
of the system, but which are no longer current to the task at hand.

ECTS represents this distinction between ``active" instances and
classes and ``historical" instances and classes by the aggregation
constructs of layers and surfaces.  Collaborative exploratory
development consists of a current, top-level surface upon which new
activities takes place.  The tools for assessing variance between
classes and instances are normally confined to the structures on the
current surface.  (In standard software engineering terms, surface
corresponds most closely to a configuration.)  Surfaces are made up of
a set of layers, within which semantically cohesive sets of classes
and instances are aggregated.  (Layers correspond most closely to the
concept of a framework in object oriented terminology combined with
the concept of version from change control terminology.)


\section{Efficient Type System Node Retrieval}
\label{type-dr}


The ability to dynamically define new fields at the class or instance
level leads to an important efficiency problem.  First, remember that
server subsystem retrieves information on a field-by-field basis.
This behavior is maintained at the type subsystem, except that these
retrieval operations allow uniform transparent access to either the
fixed server node fields or the dynamic type node fields.

Unfortunately, the straightforward implementation of this design leads
to one of two situations: (1) The presentation of a node's contents by
interface mechanisms involves multiple retrieval operations for each
field in the node, each resulting in a separate type access and
unpacking operation; or (2) The presentation of a node's contents by
interface mechanisms involves a single retrieval operation on the DATA
field.  This latter approach has the advantage of retrieving an entire
group of node fields with a single network access.  However, it has
the disadvantages that the interface mechanism now is given knowledge
of the internal representation of nodes (in order to know that the
DATA field contains the required information to begin with), as well
as requiring the interface mechanism to know how to unpack the DATA
field in order to get at the desired information.

To partially overcome this disadvantage, the type subsystem not only
exports to other subsystems the basic retrieval operation for any
field defined on a particular node instance, but a set of unpacking
operations for the DATA field as well.  These unpacking operations
take a node-ID, its DATA field contents, and a field-name. They return
the field-name's contents, or a (continuable) error if the field-name
does not exist in the DATA field.  \foot{Note that the node-ID is required
since the unpacking operation might be type dependent.}  These
unpacking operations hide the actual representation of fields inside
the server subsystem DATA fields, allowing this design decision
(though not the design decision as to whether to store the field
within the DATA field or someplace else) to remain encapsulated in the
type subsystem.

The advantage of this design is that if another subsystem just needs
to know the value of a single field in a node instance, they can
simply invoke the standard type subsystem retrieval operation with
the field name and get that information without needing to know
whether the field was fixed or dynamic, without having to unpack it,
and without paying any performance penalty.  This design also allows 
prototypes to be constructed and later optimized with additional code
for DATA field handling.

On the other hand, if another subsystem needs to efficiently retrieve
the value of several fields in a node instance, it can go through the
slightly more complicated procedure of retrieving the DATA field and
manually invoking the unpacking operations to get the values.  In this
latter case, the higher subsystem needs to know somewhat more about
the internal representation (for example, it needs to know that the
fields it wants are actually in the DATA field), but it still doesn't
need to know exactly how the field contents are represented in the
DATA field.  And, most importantly, this avoids paying any performance
penalty for the information.
