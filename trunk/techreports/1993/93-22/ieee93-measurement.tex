%%%%%%%%%%%%%%%%%%%%%%%%%%%%%% -*- Mode: Latex -*- %%%%%%%%%%%%%%%%%%%%%%%%%%%%
%% ieee93-measurement.tex -- 
%% RCS:            : $Id: ieee93-measurement.tex,v 1.4 93/12/27 10:43:01 johnson Exp Locker: johnson $
%% Author          : Philip Johnson
%% Created On      : Mon Dec 13 09:08:00 1993
%% Last Modified By: Philip Johnson
%% Last Modified On: Wed Nov 23 12:17:59 1994
%% Status          : Unknown
%%%%%%%%%%%%%%%%%%%%%%%%%%%%%%%%%%%%%%%%%%%%%%%%%%%%%%%%%%%%%%%%%%%%%%%%%%%%%%%
%%   Copyright (C) 1993 University of Hawaii
%%%%%%%%%%%%%%%%%%%%%%%%%%%%%%%%%%%%%%%%%%%%%%%%%%%%%%%%%%%%%%%%%%%%%%%%%%%%%%%
%% 
%% History
%% 13-Dec-1993		Philip Johnson	
%%    

\section{High Quality Measurement in CSRS}
\label{sec:metrics}

\subsection{Coarse-grained measurement}

Measurement of traditional manual review practice results in coarse-grained
statistics concerning the {\em outcome} of review, such as the number of
defects found, their breakdown with respect to the type and severity of
defect, their relationship to the size of the artifact under review, and so
forth.  Manual review practice also results in coarse-grained statistics
concerning the {\em process} of review, such as reviewer-supplied estimates
of the time spent preparing for review and participating in the meeting.

Unfortunately, high quality collection of even these coarse statistics can
be difficult and expensive.  Weller's report on 6000 manual review
experiences at Bull HN states that the preferred method involved the
transcription by clerical staff of paper-based review commentary into a
PC-based record-keeping system, even though this introduced errors and led
to the need for various ``data sanity'' checks by the technical
staff\foot{This method was preferred even over direct entry by technical
staff into a notebook PC during meetings.}.  A report on Digital's internal
educational practice reveals that even after 10 years of experience with a
software inspection curriculum, they were still unable to consistently
train technical staff to effectively collect and utilize data on the review
process \cite{Hutchings93}. As a final example, one respondent to our
informal USENET survey on FTR noted that his organization had been
collecting review data for over 10 years, but had never actually used any
of it due to the overhead of manual collection, entry, and analysis. 

By moving the review process on-line, CSRS trivially resolves some of
the traditional difficulties and expenses associated with the collection of
coarse-grained outcome statistics.  For example, Weller states
that one of the two major sources of review measurement error in his
organization is module size, presumably because this value is calculated by
hand by clerical staff.  In an on-line system such as CSRS, however, the
size of each review artifact is calculated automatically and reliably.
Similarly, outcome-oriented data on the number of defects found, their
breakdown with respect to type and severity, and the location of the source
artifact in which they appear are reported by CSRS automatically and
reliably in the Consolidated Report.  

CSRS also provides coarse-grained process statistics concerning effort
similar to those in traditional manual review.  Once again, reliable
collection of coarse-grained process statistics is problematic in manual
review methods.  For example, inadequate reviewer preparation for the group
meeting is a substantial problem in traditional review \cite{Freedman90}.
Rather than simply confess to insufficient preparation, a reviewer may
instead try to ``bluff'' their way through the meeting, exaggerating their
preparation time and then generating comments ``on the fly''.  Inadequate
preparation can fundamentally undermine the entire review process and
measurement interpretation, since it is difficult to distinguish the review
of a high quality artifact by well-prepared reviewers (that generates a
small number of minor issues) from the review of a moderate or low quality
artifact by ill-prepared reviewers (that also generates a small number of
minor issues.)  In traditional review practice, draconian responses are
suggested to detect insufficient preparation, such as having the moderator
omit one page from the review materials, and checking to see that all
reviewers request the missing page before the group meeting begins
\cite{Freedman90}.

Once again, moving the review process into an instrumented, on-line
environment can alleviate these difficulties.  CSRS keeps
track of the effort spent by each reviewer during the analysis of the
review artifact.  This information about each reviewer is also available to
the moderator (although reviewers do not have access to each other's
process statistics).  Thus, there is no ambiguity about the preparation
effort, and while reviewers may not always adequately prepare, at least
this fact will be visible and can be noted as a qualification to the review
outcome.  Additional process support, such as quality assurance tests that must be
explicitly checked off by the reviewers, can be used to increase the rigor
and confidence in the review. 


\begin{figure} 
 {\centerline{\psfig{figure=coarse.ps}}}
\caption{High quality, coarse-grained review measures.}
\label{fig:coarse-grained}
\end{figure}

Figure \ref{fig:coarse-grained} shows a table containing coarse-grained
outcome and process statistics from the Type-I review.  This table can be
generated automatically by CSRS without any clerical transcription and
without the need for data sanity checks by the technical staff. 

\subsection{Fine-grained measurement}

The fact that CSRS simultaneously increases the quality of coarse-grained
statistics about review process and outcome and decreases the overhead of
their collection is a useful contribution.  CSRS goes even farther than
this by providing high quality collection of {\em fine-grained} statistics
about review process and outcome, again without incurring review overhead.

Fine-grained outcome statistics include the number of minutes spent by each
reviewer on each hypertext node in the review artifact.  The Source-nodes
Summary window in Figure \ref{fig:initial-screen} shows one place in which
this data is available in an on-line, cumulative fashion; CSRS can also
generates spreadsheet data files containing this information.  Fine-grained
process statistics include the precise sequence of actions taken by
reviewers to perform the analysis, including how the review artifact was
initially read, how many sessions of what duration were required, when and
how issues were generated, and so forth.  Fine-grained statistics about
outcome and process provide a qualitatively different level of insight that
opens the way to new forms of inquiry into formal technical review.

The ability of CSRS to collect fine-grained measures is a result of a
combination of design decisions.  First, CSRS re-represents all review
artifacts as a hypertext document, where each node contains a single
semantic ``unit'' of the artifact at a fine grain size. As
discussed in Section \ref{sidebar:ftr-systems},  other
computer-supported review environments represent review artifacts at a
coarser grain size, such as the file or document level, or do not exploit
the use of hypertext for instrumentation. 

Second, CSRS monitors each reviewer as they use the system, and
unobtrusively generates a log file of timestamped ``CSRS events'' in
background during each session. This log file is cached locally at the
client process during the session, then written out to the database server
process at disconnect time.  CSRS provides a variety of tools for
post-processing of this raw instrumentation data.

Given that each node appears in its own window, CSRS can detect the focus
of reviewer attention by generating events to record display and dismissal
of each CSRS node during the review session.  CSRS also records the
occurrence of interesting reviewer actions, such as voting or marking a
node as reviewed, that do not result in window activity.

\begin{figure} 
 {\centerline{\psfig{figure=event-log.ps}}}
 \caption{An example event log from the first minute of one review session.}
\label{fig:event-log}
\end{figure}

Figure \ref{fig:event-log} shows a small portion of the timestamped event
log for the Type-I review. (Over 4300 timestamps were generated during
this review.)  It is easy to see that certain measures, such as the
elapsed time a node is reviewed, could be calculated by simply subtracting
the time that a node is closed from the time the node is read from the
server database process.

Although this approach appears reasonable at first glance, our initial
experiences (in reviews prior to Type-I) quickly demonstrated otherwise.
In a multi-tasking, multi-windowing system such as Unix and X-windows,
reviewers may interrupt their CSRS activities to read their e-mail or
perform other computer-related functions.  Perhaps the phone rings, or a
colleague interrupts with unrelated business.  When dedicated workstations
are involved, a reviewer might leave the CSRS system up and running over
the weekend in the middle of review, as we discovered to our chagrin when
we measured an apparent effort of over 50 hours by one reviewer for a small
5 line function!

To account for each of these types of idle time, we augmented the ``top
down'' timestamping with a timer-based, ``bottom-up'' timestamp process
that wakes up once per minute, determines whether there has been any
low-level editing activity (such as a keypress or mouse movement) within
CSRS during the past minute, and if so, generates a ``busy'' timestamp in
the event log. 

This combination of top-down and bottom-up timestamping has proven to be very
effective: it provides us with a record of the ``interesting'' user
activities from a measurement perspective, as well as providing an accurate
reflection of the time spent by the reviewer actually working with CSRS,
with a precision of plus or minus one minute. 

We have investigated one potential flaw in this approach: what about a
reviewer who simply ``stares at the screen'' for many minutes at a time?
CSRS would represent such behavior as idle time, even though the reviewer
was hard at work.  Fortunately, empirical analysis of our timestamp logs
indicates that such a behavior rarely, if ever, happens, at least during
our laboratory studies.  We investigated this by calculating the
inter-event interval for top-down events and found that timestamps are
typically generated less than 30 seconds apart over 80\% of the time, less
than a minute apart over 90\% of the time, and less than three minutes
apart over 98\% of the time.

Once again, the hypertext structure of the review artifact plays an
essential role in high quality data collection. Since the artifact is
presented to the reviewer in individual small nodes, it appears to be very
rare that a user will spend many minutes on a single node without {\em any}
keyboard activity (including scrolling the screen, moving the mouse to
``keep your place'', etc.)

\section{Applications of High Quality Measurement}
\label{sec:applications}

\subsection{Improving coarse-grained measurement applications}

High quality, fine grain measures of FTR provided by CSRS
improve traditional empirical inquiries into FTR-based software
quality improvement, as well as support entirely new forms of empirical
investigation.

FTR measures are commonly used to provide insight into review productivity,
such as the number of errors found per unit size of an artifact, and the
cost to detect them.  Researchers suggest that such a cost-effectiveness
metric can be used as a driver for changes to the FTR process, such as
increasing or decreasing the number of reviewers, shortening or lengthening
the review time, changing the specific analysis technique used for review,
and so forth.

While feasible in principle, such empirically-based process improvement is
very difficult to practice with traditional, manual methods.  First, as
noted above, collection and analysis of FTR data is very expensive and
requires significant quality assurance\foot{In other words, the quality
assurance data requires quality assurance :-)}.  Thus, an organization must
be willing to commit new resources over and above the basic resources for
FTR.  In effect, management is betting that this additional investment will
result in data that can be used to improve the quality and efficiency
enough to offset this investment.  This may explain why only very large
corporations with a strong commitment to process improvement have published
findings on FTR measurements, and why our informal USENET survey finds such
low percentages of FTR data usage.

Second, in order to provide a scientifically sound basis for process
change, the organization must be willing to undertake FTR under controlled
experimental conditions in order to minimize the effect of extraneous
procedural variables on the outcome data. In addition, the organization
must replicate the FTRs enough to yield statistically significant
conclusions that warrant changes to the FTR process.  Once again, only very
large corporations have the time and sufficient numbers of FTRs required to
perform such empirically-justified process improvement.  FTR data published
by different organizations is virtually incomparable, since these studies
do not specify the technique or control its application sufficiently to
support experimental replication within another organization.

CSRS addresses both of these issues.  First, collection of FTR data in CSRS
is automated, and CSRS provides analysis tools to automate the generation
of many popular measures of cost-effectiveness from the raw data.  Thus, an
organization needs commit only the resources for formal technical review;
no additional resources are required for measurement collection and initial
analysis.

Second, CSRS intrinsically controls for many of the extraneous variables
that can weaken the validity and comparability of manual FTR measurements.
Since the review process is on-line, and since CSRS can enforce a
particular process and analysis technique, review experiment designers can
both specify a particular style of review, and analyze the data to see if
it was actually followed.  As a result, CSRS data is much more amenable to
cross-organizational comparison than any manual technique.  This means that
smaller organizations with less resources can both contribute to and profit
from the experiences of others with CSRS.  (As CSRS technology is
transferred into industry, we intend to set up an on-line public database
of CSRS metric data to support such inter-organizational cooperation.)

\subsection{Fine-grained measurement applications}

While CSRS can provide significant improvement to traditional
coarse-grained measurements of FTR, perhaps the most exciting aspect of its
design is that it enables entirely new forms of inquiry into review
processes through fine-grained instrumentation. In this section, we will
overview two ongoing research studies that exploit this capability.


\subsubsection{Complexity/Effort Relationships}

The overall effort required for formal technical review is difficult to
predict at other than very general levels, such as ``15\% of the entire
software development effort''.  This information is useful from a
historical perspective, but provides little help to allocating specific
resources to specific review artifacts.  The need for better estimators of
review effort has long been identified as an important open problem in
review practice \cite{Fagan86}.

The fine-grained instrumentation in CSRS provides a novel approach to this
problem.  Since both the review artifact and the precise time and effort
spent upon each component of it by each reviewer are captured by the
system, it may be possible to correlate features of the review artifact to
effort expended and generate a predictive measure of effort.  A
sufficiently precise correlation would allow an organization to predict the
resources required for review activities based upon the complexity computed
for the code, or conversely, to fit the set of artifacts to the resources
available for review.

We are currently studying this possibility for the review of code 
artifacts.  In this study, we are investigating the relationship of 
review effort to three size-based measures of software complexity: Halstead's 
volume measure \cite{Halstead77}, McCabe's cyclomatic
number \cite{McCabe76}, and lines of code.  

\begin{figure}
 {\centerline{\psfig{figure=complexity.ps}}}
\caption{A scatter plot of review effort vs. one size-based complexity
measure: lines of code.}
\label{fig:scatter}
\end{figure}

Figure \ref{fig:scatter} illustrates some of the data we have collected so
far.  There is no statistically significant {\em linear} correlation between effort and
any of the three measures of complexity for this data set, although, with
some imagination, a general positive correlation might be seen.  Thus far,
lines of code is the most predictive of the three complexity
measures. The most important result of these initial analyses has been a
refinement of our experimental hypotheses.  We will continue to add new
review experiments to our data set to see if, given an increased sample
size, a statistically significant trend emerges.  In addition, these
analyses have suggested to us that other factors, such as reviewer
experience, may be an important influence on review effort.  We will be
investigating these alternative hypotheses as well, which may result in a
multi-factored predictive function of review effort, similar in spirit to
the COCOMO model.

\subsubsection{Protocol Analysis of Reviewer Behavior}

If formal technical review methods are to reach their full potential, much
more needs to be known about the details of both individual and group FTR
behavior.  Past research on FTR has determined that the effectiveness of
FTR will vary tremendously depending upon the review technique employed and
the quality of the group process.  Yet little is known about the precise
``styles'' of review adopted by different people, the structure and process
of consensus building during review, and the effect of these behaviors upon
the review outcome.  Precise measurement of the process of formal technical
review is the first step toward understanding, and ultimately, improvement.

The design of CSRS provides unique insight into the fine-grained process of
individual and group review.  By breaking down a review artifact into
its constituent components, representing it as an interlinked hypertext
network, and recording the patterns of traversal and actions of the
reviewers, CSRS provides a qualitatively different level of 
process representation than is possible in other manual or automated
approaches to review.

This representation does, however, require automated support tools, since
many thousands of individual timestamped review events are recorded during
a typical review experience.  We have implemented a visualization tool for
review called Timeplot to support our initial inquiries in this area.
Timeplot processes the review event logs, and automatically generates a
LaTeX document illustrating the contents of each CSRS screen during each
minute of review.  Timeplot also reveals patterns of idle time during 
a review session.  Figure \ref{fig:timeplot} shows the process 
of one review session generated by Timeplot for the exemplary review 
experience. 
 

\begin{figure}
\noindent{\large\bf Session 376: Sep 29, 1993 17:10:03}
\small
\begin{verbatim}
Phase: Private

Real  Active   Source                     Review2       Other Events
Time  Time     Screen                     Screen

17:10 00:00    -                                            
  :     :      |t*nschema                               Reviewed t*nschema. 
  :     :      |t*def-set-attribute                         
17:11 00:01    -                                            
  :     :      |t*def-add-attribute-vals                Reviewed t*def-add-attribute-vals. 
  :     :      |t*def-set-attribute                         
17:12 00:02    -                                            
  :     :      |t*def-delete-schema                     Reviewed t*def-set-attribute. 
  :     :      |                                            
17:13 00:03    |                                            
  :     :      |                                            
  :     :      |                                            
17:14 00:04    |                                            
  :     :      |                                            
  :     :      |                                            
17:15 00:05    |                                            
  :     :      |                                            
  :     :      |                                            
17:16 00:06    |                                            
  :     :      |                                            
  :     :      |                                            
17:17 00:07    |                                            
  :     :      |                                            
  :     :      |                                            
17:18 00:08    |                          -             
  :     :      |                          |Comment#374  Confirmed Comment#374. 
  :     :      |                          |             
17:19 00:09    |                          |             
  :     :      |                          |             
  :     :      |                          |             
17:20 00:10    |                          |             
  :     :      |                          |             
  :     :      |                          |             
17:21 00:11    -                          -             
\end{verbatim}
\normalsize
\caption{An excerpt from an example Timeplot graph from one review
session during the Type-I review. This session actually lasted for 
22 minutes, but only the first 10 minutes are shown here.}
\label{fig:timeplot}
\end{figure}

We are currently engaged in analyzing this fine-grained behavior from a
variety of perspectives.  We are working on a grammar-based approach to
extracting patterns from the sequence of reviewer actions, with the goal of
identifying typical behaviors across reviewers, and to identify
higher-level activities occurring during the review process.  Such work
builds upon recent research in protocol analysis \cite{Smith91} and
statistical techniques \cite{Gottman90}.
















 





