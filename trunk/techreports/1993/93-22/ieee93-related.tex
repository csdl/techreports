%%%%%%%%%%%%%%%%%%%%%%%%%%%%%% -*- Mode: Latex -*- %%%%%%%%%%%%%%%%%%%%%%%%%%%%
%% ieee93-related.tex -- 
%% Author          : Philip Johnson
%% Created On      : Wed Nov 23 09:02:37 1994
%% Last Modified By: Philip Johnson
%% Last Modified On: Wed Nov 23 12:24:40 1994
%% Status          : Unknown
%% RCS: $Id$
%%%%%%%%%%%%%%%%%%%%%%%%%%%%%%%%%%%%%%%%%%%%%%%%%%%%%%%%%%%%%%%%%%%%%%%%%%%%%%%
%%   Copyright (C) 1994 University of Hawaii
%%%%%%%%%%%%%%%%%%%%%%%%%%%%%%%%%%%%%%%%%%%%%%%%%%%%%%%%%%%%%%%%%%%%%%%%%%%%%%%
%% 

\section{Related Work}
\label{sec:related}

Research related to CSRS falls into two general categories: research on 
formal technical review methods, and research on computer-supported FTR.

\subsection{Formal Technical Review Methods}
\label{sidebar:ftr-methods}

\paragraph{Code Inspection.}

The seminal formal technical review method is Fagan's code inspection
\cite{Fagan86}.  It begins with a presentation by the producer
about the material to be reviewed.  Next, reviewers prepare for the review
meeting by informally analyzing the review artifact.  The central and most
precisely prescribed activity is the inspection meeting, which is a
face-to-face group examination and issue consolidation activity.  The
reader paraphrases the source materials, statement by statement, and the
participants interrupt with questions that may eventually lead to discovery
of errors.  Errors are recorded and fed back into future review sessions.
Fagan's code inspection method presumes {\em no} automated support
technology.

\paragraph{Software Inspection.}

Gilb's Inspection method \cite{Gilb93} is a direct descendent of Fagan's
Code Inspection. However, Gilb introduces a substantial number of
refinements and elaborations to Code Inspection, and generalizes it for all
artifacts produced during software development.  Also similar to Fagan,
Gilb's method is entirely manual in nature.

\paragraph {Software Review.}

Humphrey's software review \cite{Humphrey90} begins with group
comprehension meeting where participants review background materials
presented by the producer. Next, reviewers analyze the review artifacts
individually, using a error checklist to guide their analysis.  The
producer then correlates and consolidates the errors found by individual
reviewers.  The final phase is a group meeting where the producer presents
the error list item by item, and the participants discuss the significance
of the errors and decide upon an action.

\paragraph{Active Design Review.}

Active Design Review \cite{Parnas85} is conducted in four phases.  First,
the producer presents a brief overview of the module under review.  Next,
reviewers analyze the module individually.  Active Design Review specifies
a novel method for this individual review, where each reviewer must provide
answers to a set of questionnaires designed by the producer, thus
guaranteeing comprehension.  Furthermore, individual reviewers are utilized
as specialists, and review only sections of the artifact.  Next, the
producer meets with the reviewers individually to discuss the concerns
raised in the questionnaire. Finally, the producer creates a consolidated
report.

\paragraph{Cleanroom Inspection.}

Cleanroom Inspection is a component of Cleanroom software development
\cite{Dyer92a}.  In general, Cleanroom Inspection consists of individual
review, followed by consolidation by the review leader.  The novel aspect
of Cleanroom Inspection is that analysis of the review artifact involves
step-wise verification of a formal specification against the actual source
code.

\paragraph{Phased Inspection.}

Phased Inspection \cite{Knight93} involves two kinds of review
phases---single inspector and multiple inspector.  Single inspector phases
involve mostly simple clerical procedures to establish basic structural or
syntactic properties of the review artifact, and that require only minimal
comprehension of the artifact.  After all single inspector phases complete,
the multiple inspector phases begin.  These phases involve active
comprehension and group-based analysis of the review artifact, and may
involve both standard checklists (such as in Fagan Inspection) and active
checklists (such as in Active Design Review).  The final phase is
reconciliation, in which the the inspectors compare their findings in a
group meeting using Delphi-like technique.  Phased Inspection, like CSRS,
is predicated upon computer support, and its system, called INSPEQ, is
discussed briefly in the next section.


\subsection{Computer-supported FTR systems}
\label{sidebar:ftr-systems}


\paragraph{ICICLE.}

ICICLE \cite{Brothers90} is an X window-based system that was designed to
support Fagan's Code Inspection on C language software.  ICICLE allows
reviewers to perform the code inspection meeting on-line, providing various
tools to help control the paraphrasing process and generation of issues.
The ICICLE experience reveals some of the difficulties encountered when
trying to ``faithfully reproduce'' an intrinsically manual process in a
computer-mediated environment.  ICICLE did not include instrumentation
support for other than simple statistics on the total number of defects
discovered and their severity.  ICICLE was oriented only to C code, and
represented artifacts at the granularity of files.

\paragraph {INSPEQ.}

INSPEQ \cite{Knight93} is an X window-based support environment for the
Phased Inspection review method.  It provides a multi-window based
environment to support display of source code, checklists, standards, and
comments.  INSPEQ represents source artifacts as files, and provides some
coarse-grained instrumentation to record elapsed time and the numbers of
issues raised.

\paragraph{Scrutiny.}

Scrutiny \cite{Gintell93} is an X window-based system developed at Bull HN
Information Systems originally to support inspection-at-a-distance,
following Fagan's method.  Unlike other on-line face-to-face systems such
as ICICLE, Scrutiny can be applied to a wide range of documents, and is
built on top of ConversationBuilder, a general purpose, collaborative
support environment \cite{Kaplan92}.  Scrutiny inherits the hypertext
capabilities of ConversationBuilder, but has no representation of time,
thus severely limiting its instrumentation capabilities.


\paragraph {CSI.}

CSI \cite{Mashayekhi93} is a system developed at the University of
Minnesota to allow inspection-at-a-distance using Humphrey's method.  CSI
uses an X window-based environment called Suite that provides hypertext
capabilities, and supports both asynchronous and synchronous meeting
capabilities.  CSI does support certain coarse-grained time-based measures.

