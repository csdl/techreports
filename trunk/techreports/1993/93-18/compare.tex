%%%%%%%%%%%%%%%%%%%%%%%%%%%%%% -*- Mode: Latex -*- %%%%%%%%%%%%%%%%%%%%%%%%%%%%
%% compare.tex -- 
%% RCS:            : $Id$
%% Author          : Carleton Moore
%% Created On      : Fri Sep 24 10:28:13 1993
%% Last Modified By: Carleton Moore
%% Last Modified On: Tue Jul 12 09:09:54 1994
%% Status          : Unknown
%%%%%%%%%%%%%%%%%%%%%%%%%%%%%%%%%%%%%%%%%%%%%%%%%%%%%%%%%%%%%%%%%%%%%%%%%%%%%%%
%%   Copyright (C) 1993 University of Hawaii
%%%%%%%%%%%%%%%%%%%%%%%%%%%%%%%%%%%%%%%%%%%%%%%%%%%%%%%%%%%%%%%%%%%%%%%%%%%%%%%
%% 
%% History
%% 24-Sep-1993		Carleton Moore	
%%    

\documentstyle [12pt,/group/csdl/tex/definemargins]{article}      % Unix
% \documentstyle [12pt,definemargins]{article}            % Macintosh
\input{/home/3/dxw/c/tex/psfig}
\begin{document}

\title{Hyperbase Server Performance comparisons} \author{Cam Moore}

\maketitle

I tested the three versions of the hyperbase server.  I connected to each
of the servers separately and called their functions.  I recorded the time
it took to execute each function 200 times.  For each function I repeated
this procedure five times and averaged the times.  The following table
contains the results of my timing:

\small

             \begin{figure}[htpb]
             \begin{center}
             \begin{tabular} {|l|c|c|c|} \hline
             \multicolumn{4}{|c|}{{\bf Timing Results}} \\  \hline
{\em Function} & {\em Beta 0.2} & {Cam} & {Rose}\\ \hline
Create Node & 42 & 43 & 42  \\
\hline
Write & 43 & 42 & 41 \\
\hline
Read & 41 & 47 & 44 \\
\hline
Event & 40 & 47 & 48 \\
\hline
UnEvent & 40 & 41 & 46 \\
\hline
Show Event & 84 & 86 & 84 \\
\hline
Lock & 41 & 42 & 50 \\
\hline
UnLock & 40 & 40 & 45 \\
\hline
Show Lock & 40 & 43 & 48 \\
\hline
Browse & 163 & 170 & 161\\
\hline
\end{tabular}
             \end{center}
             \caption{Average Times. }
             \end{figure}
             \normalsize
The times are very close.  It appears that there is no real difference in
performance between the three servers.  I beleive the differences in the
times are due to different loads on the machine while the test was being run.




\end{document}

