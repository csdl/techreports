\begin{titlepage}
\vspace*{1in}
\begin{center}
   
\Large

{\bf An Instrumented Approach to} \medskip\par
{\bf Improving Software Quality}   \medskip\par
{\bf through Formal Technical Review}  \bigskip\par
                                         \bigskip\par

\normalsize

Philip Johnson                           \medskip\par
Department of Information and Computer Sciences\\ 
University of Hawaii\\ 
Honolulu, HI 96822\\                       
(808) 956-3489\\
(808) 956-3548 (fax)\\
{\tt johnson@hawaii.edu}                 \bigskip\par

\today                                   \bigskip\par

{\bf Submission category:} research      \bigskip\par

{\bf Keywords:}  software quality assurance, formal technical review,
computer-supported cooperative work.      \bigskip\par


\begin{abstract}
  
  Formal technical review (FTR) is an essential component of all software
  quality assessment, assurance, and improvement techniques.  However, FTR
  exists in a variety of forms, and the process of effectively choosing and
  adopting an FTR technique appropriate to a particular organization is not
  well-understood.  Furthermore, current FTR practice is essentially manual
  in nature, leading to significant expense, clerical overhead, group process
  obstacles, and research methodology problems.
  
  This paper presents results from our research to date on CSRS, an
  instrumented, computer-supported cooperative work environment for
  formal technical review.  CSRS addresses problems in the practice of
  FTR by providing computer support for both process and products of FTR.
  CSRS also addresses problems in the research on FTR through
  fine-grained instrumentation that enables collection of novel forms
  of high quality data about the FTR method currently implemented in CSRS.
  This paper describes CSRS, its instrumentation support, and selected
  findings from its use to explore issues in formal technical review.
  
\end{abstract}


\end{center}
\end{titlepage}
