%%%%%%%%%%%%%%%%%%%%%%%%%%%%%% -*- Mode: Latex -*- %%%%%%%%%%%%%%%%%%%%%%%%%%%%
%% icse94-intro.tex -- 
%% RCS:            : $Id: icse94-intro.tex,v 1.4 94/02/14 15:39:04 johnson Exp Locker: johnson $
%% Author          : Philip Johnson
%% Created On      : Thu Aug 12 11:14:18 1993
%% Last Modified By: Philip Johnson
%% Last Modified On: Tue Feb 15 08:00:52 1994
%% Status          : Unknown
%%%%%%%%%%%%%%%%%%%%%%%%%%%%%%%%%%%%%%%%%%%%%%%%%%%%%%%%%%%%%%%%%%%%%%%%%%%%%%%
%%   Copyright (C) 1993 University of Hawaii
%%%%%%%%%%%%%%%%%%%%%%%%%%%%%%%%%%%%%%%%%%%%%%%%%%%%%%%%%%%%%%%%%%%%%%%%%%%%%%%
%% 
%% History
%% 12-Aug-1993		Philip Johnson	
%%    

\section{Introduction}

Assessment and improvement of software quality is increasingly recognized
as a fundamental problem, if not {\em the}\/ fundamental problem
confronting software engineering in the 1990's.  Low quality has always
figured prominently in explanations for software mishaps, from the Mariner
I probe destruction in 1962, to AT\&T's 4EES switching circuit failure in
1992.  More recently, however, low software quality has also been
implicated in competitive failure on a corporate scale \cite{Arthur93}, as
well as in loss of life on a human scale \cite{Leveson93}.

Research on tools and techniques to improve software quality shows
that formal technical review (FTR) provides unique and important
benefits.  Some studies provide evidence that FTR can be more
effective at discovering errors than testing, while others indicate
that it can discover different classes of errors than testing
\cite{Myers78,Basili86}.  In concert with other process improvements,
Fujitsu found FTR to be so effective at discovering errors that they
dropped system testing from their software development procedure
\cite{Arthur93}.  FTR forms an essential part of methods and models
for very high quality software, such as Cleanroom Software Engineering
and the SEI Capability Maturity Model.
Finally, FTR displays a unique ability to improve the quality of the 
producer as well as the quality of the product.


FTR always involves the bringing together of a group of technical personnel
to analyze an artifact of the software development process, typically with
the goal of discovering errors or anomolies, and always results in a
structured document specifying the outcome of review.  Beyond this general
similarity, specific approaches to FTR exhibit wide variations in process
and products, from Fagan Code Inspection \cite{Fagan76}, to Active
Design Reviews \cite{Parnas85}, to Phased Inspections \cite{Knight91}.

Despite its importance and potential, the state of both FTR practice and
research suffers from problems that hinder its adoption and effective use
within organizations.  First, most FTR methods are manual, prone to
breakdown, and highly labor-intensive, consuming a great deal of expensive
human technical resources. For example, a recent study documents that a
single code inspection of a 20 KLOC software system consumes one
person-year of effort by skilled technical staff \cite{Russell91}.  Second,
high-quality empirical data about the process and products of FTR is
difficult to obtain and comparatively evaluate.  Only Fagan code inspection
enjoys a relatively broad range of published data about its use and
effectiveness.  The lack of such research data makes it difficult to
compare different methods, improve the process, or match a method to a
particular organizational culture and application domain.

For the past two years, we have been experimenting with a
computer-supported cooperative work environment called CSRS (for
Collaborative Software Review System), coupled with a method called FTArm
(for Formal, Technical, Asynchronous review method). This system and method
are designed to address problems in both the practice of and research on
FTR.  Laboratory studies demonstrate that a highly instrumented,
collaborative environment that puts most processes and products of FTR
on-line can lead to increased user satisfaction, enhanced capture of
significant knowledge about software, useful new measures of FTR processes
and products, and finally, higher quality software.

The primary goal of this paper is to inform the software engineering
research and development community on how computer-mediated FTR can not
only address certain problems associated with manual approaches, but can
also provide high quality and low cost data useful for improving the
process and products of FTR.  We believe our experiences with CSRS and
FTArm provide useful insights to the designers of current and future formal
technical review systems.







 


