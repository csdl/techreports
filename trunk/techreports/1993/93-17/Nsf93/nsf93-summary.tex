%%%%%%%%%%%%%%%%%%%%%%%%%%%%%% -*- Mode: Latex -*- %%%%%%%%%%%%%%%%%%%%%%%%%%%%
%% nsf93-summary.tex -- 
%% RCS:            : $Id: nsf93-summary.tex,v 1.9 93/09/01 16:17:27 johnson Exp $
%% Author          : Philip Johnson
%% Created On      : Wed Aug 11 12:55:46 1993
%% Last Modified By: Philip Johnson
%% Last Modified On: Wed Sep  1 16:16:22 1993
%% Status          : Unknown
%%%%%%%%%%%%%%%%%%%%%%%%%%%%%%%%%%%%%%%%%%%%%%%%%%%%%%%%%%%%%%%%%%%%%%%%%%%%%%%
%%   Copyright (C) 1993 University of Hawaii
%%%%%%%%%%%%%%%%%%%%%%%%%%%%%%%%%%%%%%%%%%%%%%%%%%%%%%%%%%%%%%%%%%%%%%%%%%%%%%%
%% 
%% History
%% 11-Aug-1993		Philip Johnson	
%%    

\section{Project Summary}

Formal technical review (FTR) is an essential component of all software
quality assessment, assurance, and improvement techniques.  However, FTR
exists in a variety of forms, and the process of effectively choosing and
adopting an FTR technique appropriate to a particular organization is not
well-understood.  Furthermore, current FTR practice is essentially manual
in nature, leading to significant expense, clerical overhead, group process
obstacles, and research methodology problems.

This research proposal describes a four year project to address these
problems in the practice of and research upon formal technical review.
First, it will support continued development of a computer-supported
cooperative work environment for FTR called CSRS (for Collaborative
Software Review System) that provides automated support for many time
consuming aspects of FTR, and avoids many group process problems inherent
in traditional, manual review.  Second, through instrumentation support in
CSRS, rich forms of novel empirical data will be captured that can provide
helpful new insight into the process and products of formal technical
review. Finally, public release of the instrumented environment will lead
to technology transfer benefits and the creation of an on-line database of
empirical data about a broad range of review process and products.  The
significance of this project is in its dual provision of advanced
technology to improve software quality, combined with instrumentation to
facilitate generation of important review measures and wide distribution
of this data.

