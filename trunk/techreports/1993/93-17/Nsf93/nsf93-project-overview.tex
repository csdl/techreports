%%%%%%%%%%%%%%%%%%%%%%%%%%%%%% -*- Mode: Latex -*- %%%%%%%%%%%%%%%%%%%%%%%%%%%%
%% nsf93-project-overview.tex -- 
%% RCS:            : $Id: nsf93-project-overview.tex,v 1.8 93/09/01 16:18:56 johnson Exp $
%% Author          : Philip Johnson
%% Created On      : Thu Aug 12 11:14:18 1993
%% Last Modified By: Philip Johnson
%% Last Modified On: Wed Sep  1 16:18:49 1993
%% Status          : Unknown
%%%%%%%%%%%%%%%%%%%%%%%%%%%%%%%%%%%%%%%%%%%%%%%%%%%%%%%%%%%%%%%%%%%%%%%%%%%%%%%
%%   Copyright (C) 1993 University of Hawaii
%%%%%%%%%%%%%%%%%%%%%%%%%%%%%%%%%%%%%%%%%%%%%%%%%%%%%%%%%%%%%%%%%%%%%%%%%%%%%%%
%% 
%% History
%% 12-Aug-1993		Philip Johnson	
%%    

\subsection{Overview}

Assessment and improvement of software quality is increasingly recognized
as a fundamental problem, if not {\em the}\/ fundamental problem
confronting software engineering in the 1990's.  Low quality has always
figured prominently in explanations for software mishaps, from the Mariner
I probe destruction in 1962, to AT\&T's 4EES switching circuit failure in
1992.  More recently, however, low software quality has also been
implicated in competitive failure on a corporate scale \cite{Arthur93}, as
well as in loss of life on a human scale \cite{Leveson93}.

Research on tools and techniques to improve software quality has shown
formal technical review (FTR) to provide unique and important benefits.
Some studies provide evidence that FTR is both more effective at catching
errors than testing, and that it catches different kinds of errors than
testing \cite{Myers78,Basili86}.  In concert with other process
improvements, Fujitsu found FTR to be so effective at removing defects that
they dropped system testing from their software development procedure
\cite{Arthur93}.  FTR forms an essential part of methods that produce very
high quality software, such as Cleanroom Software Engineering
\cite{Linger93}.  FTR plays a substantial role in the SEI Capability
Maturity Model \cite{Paulk93a}, with involvement in the following key
practices: Software Quality Assurance (Level 2), Peer Reviews (Level 3),
Software Quality Management (Level 4) and Defect Prevention (Level 5).
Finally, FTR has the potential to improve the quality of the {\em producer}\/
as well as the {\em product}.


FTR always involves the bringing together of a group of technical personnel
to analyze an artifact of the software development process, typically with
the goal of discovering errors or anomolies, and always results in a
structured document specifying the outcome of review.  Beyond this general
similarity, specific approaches to FTR exhibit wide variations in process
and products, from Fagan Code Inspection \cite{Fagan76,Fagan86}, to Active
Design Reviews \cite{Parnas85}, to Phased Inspections \cite{Knight91}.

Despite its importance and potential, the state of both FTR practice and
research suffers from problems that hinder its adoption and effective use
within organizations.  First, most FTR methods are manual, prone to
breakdown, and highly labor-intensive, consuming a great deal of expensive
human technical resources. For example, a recent study documents that a
single code inspection of a 20 KLOC software system consumes an entire
man-year of effort by skilled technical staff \cite{Russell91}.  Second,
high-quality empirical data about the process and products of FTR is
difficult to obtain and comparatively evaluate.  Only Fagan code inspection
enjoys a relatively broad range of published data about its use and
effectiveness.  The lack of such research data makes it difficult to
compare different methods, improve the process, or match a method to a
particular organizational culture and application domain.

For the past two years, we have been experimenting with a computer
supported cooperative work environment that is designed to address problems
in both the practice of and research on formal technical review.  This
system, called CSRS\footnote{Collaborative Software Review System}
\cite{csdl-92-07,csdl-93-07}, has matured beyond the proof-of-concept stage
and is now in regular use for FTR within our organization.  CSRS
demonstrates that a highly instrumented, collaborative environment that
puts most processes and products of FTR on-line is possible and can lead to
increased user satisfaction, enhanced capture of significant knowledge
about software, useful new measures of FTR processes and products, and
finally, higher quality software.

This proposal describes a four year project designed to build upon our 
prior work on computer-supported FTR using CSRS.  In this proposal, we seek
funding for four major activities:

\begin{itemizenoindent}
\item {\em Continuing development of automated, instrumented support for
  FTR.} This project involves continuing development of CSRS, with
  enhancements targetted toward increasing its portability across application
  and organizational domains, utility for research experimentation, and ease
  of external installation and adoption.
  
\item {\em Empirical studies of FTR.} This project will include several
  experimental studies designed to provide novel, high quality empirical
  data concerning the nature and effectiveness of FTR in various
  organizational, process, and product contexts.
  
\item {\em Public release of CSRS.} From its inception, CSRS has used
  only public domain platforms such as GNU C++ and Emacs Lisp in order to
  facilitate eventual public distribution.  This project will involve
  enhancement and quality assurance activities necessary to release CSRS
  for public distribution, along with documentation and training materials.
  
\item {\em Development of an on-line CSRS Metric Database.} The motivation
  for public release of CSRS is not only to provide the software development
  community with a useful tool for FTR, although this is a substantial
  benefit. In addition, public release is intended to lead to automated
  collection and on-line publication of empirical data about the process and
  products of FTR across a spectrum of organizations.  This fine-grained data
  about FTR process and products will be maintained in an on-line database
  server that will provide the software engineering research and development
  community access to non-proprietary, empirical data about FTR.

\end{itemizenoindent}

The first-order significance of this project will be the creation,
evaluation, and public distribution of an environment that lowers the cost
and improves the quality of software review.  The second-order significance
of this project will be the instantiation of a new paradigm for research on
software review, whereby organizations will have a means to share
information and improve their review practice through contribution and
access to a database of review measurements based upon a single, uniform
environment.  Our goal is a two-order multiplier on the research
investment: not only will the results (i.e. CSRS) improve a broad array of
organizations, but the results (i.e. the Metric Database) will improve CSRS (as
well as provide experiences that improve other research efforts in software
review.)

The remainder of this Project Description is organized as follows.
Sections \ref{sec:external-research} and \ref{sec:csrs} provide relevent
background information on formal technical review and on our previous
research accomplishments.  Section \ref{sec:proposed-research} describes
the proposed research, its significance and risks, and its project plan.






 


