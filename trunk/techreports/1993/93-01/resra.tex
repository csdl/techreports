\section{RESRA: the Thematic Representation}
\label{sec:resra}

\subsection{The Structure of Learning Artifacts}

The structure of learning materials or, in our terminology, {\it
artifacts\/}, such as journal articles and research reports, may be viewed
at various levels. At the top level, the student sees an article as
consisting of a title and a number of sections, subsections, figures,
tables, et al. This structural distinction is based on the visible
attribute, and is {\it presentational\/} in nature.  Often, it has little
bearing on the content of the underlying artifact.  Looking down one level,
however, one may note that the same artifacts also use such standard
headings as ``abstract'', ``introduction'', ``experimental design'',
``outcome'', ``related work'', ``conclusions'', and so forth. These labels
tell the learner the type of content immediately followed. They are,
however, still primarily organizational, and therefore belong to the {\it
presentational structure\/}, or PS.

Argumentation plays a critical role in both science and learning processes
\cite{Cross90}. {\it Rhetorical structures\/} or RS, such as the one
proposed by \cite{Toulmin84}, provide a useful means of understanding
research artifacts, especially, relationships between artifacts. For
instance, the contents of a series of articles published in the ``technical
communication'' columns of a professional journal are capturable using
rhetorical models. Unlike PS, which focuses on the {\it surface
structure\/} of learning artifacts, RS represents the {\it deep
structure\/}, which cannot be derived without first understanding the
artifact content. Some research is devoted solely to the RS-based learning
(e.g., \cite{Cavalli-Sforza92}).

The third type of structure is the {\it thematic structure\/}, or TS. As
the name implies, TS characterizes the theme or essential elements of
learning materials and relationships between them. Similar to RS, TS is
content-oriented. It, however, goes beyond RS in that it models both the
discursive and the domain structure, both intra- and inter-artifact
relationships.  For example, TS can include such primitives as ``concept'',
``claim'', which, when instantiated into the field of software engineering,
might include ``software complexity'' (concept) and ``Object-oriented
design offers an effective solution to software complexity'' (claim). These
features of TS make it a valuable basis of knowledge representation, which
explains why our representational scheme, i.e., RESRA, is based on it.


\subsection{Knowledge Representation and Metalearning}

Knowledge representation (KR) is often associated with automated reasoning
and machine-based intelligence. In this context, however, we are interested
in using KR to facilitate and augment human learning. Since, according to
\cite{Swaminathan90}, KR models are either epistemological or ontological,
KR-centered learning is metalearning, that is, it helps the student deciding 
what kind of knowledge to use as well as how to structure it.

Our thematic representation (i.e., RESRA, see Section ~\ref{sec:resra1})
falls to the category of content theories. As observed by Swaminathan,
content theories are usually incomplete, vague and, as a result, do not
usually lead to unique mappings from the text under study into the
primitives proposed by the theory \cite{Swaminathan90}. Though these
problems have been a constant source of criticism against certain KR
schemes, such as Schank's theory of conceptual dependency (CD)
\cite{Schank75}, they are desirable features in the current context,
because the representation is no longer primarily for machine reasoning,
but rather a heuristic basis for human learning. The meta and heuristic
values of the representation are manifested in the following four ways:

\begin{itemize}

\item a mapping tool that highlights essential elements and relationships
within as well as across learning artifacts;

\item an organizational tool that allows the learner to dynamically and
incrementally integrate various types of learning artifacts at a fine-grain
level;
  
\item a communication tool, i.e., a shared ``frame of reference'' in group
collaboration: contrasting different representations of the same artifact
by different group members can highlight the differences among group members,
while integrating them can lead to a fuller understanding of the subject
domain; and
  
\item a learning tool for the student on the conventions
governing the written presentation of learning and research results.
\end{itemize}


\subsection{Five Levels of Collaborative Learning}
\label{sec:activity}

Learning in a group setting may take many different forms, including joint
projects, writing, reading, discussion, and so forth. Our observation of
these activities has led to these five common themes: summarization,
evaluation, integration, argumentation, and construction, as described in
Table ~\ref{tab:act}). The actual division between these types, however,
may never be as clear-cut as what the table suggests. For example, an
argument is often filled with constructive ideas and value-laden judgement.
In addition, the order in which the activity types appear in the table also
does not dictate the actual process steps in which they occur, though the
sequence is typical. One implication of this categorization is that a
representation needs to be able to accurately express the semantics of
artifacts generated from all these five types of activities, and to relate
and integrate them in some useful fashion. RESRA, which will be described
next, is such a language.

\begin{table}[ht]
  \ls{0.8}
  \begin{center}
    \begin{tabular} {|l|p{2.25in}|p{2.25in}|} \hline   
      
      {\bf Activity Type} & {\bf Description} & {\bf Example Artifacts} \\
\hline
      
      Summarization & Extracting, condensing, and relating important
elements from an artifact. & List of hypotheses and findings from a
research paper. \\ \hline
      
      Evaluation & Subjective appraisal of a given piece of work. &
Criticisms on the flaws of an experimental design. \\ \hline
      
      Integration & Relating, aggregating or abstracting previously
scattered themes. & A state-of-the-art survey on a given topic.
      \\ \hline
      
      Argumentation & Interactions among polarized points of view with
regards to a given topic. & Recorded script of a panel discussion
on information privacy. \\ \hline
      
      Construction & New proposals, formulations, or interpretations of new
or existing problems or solutions. & RESRA as a new way of supporting
metalearning. \\ \hline
    \end{tabular}
    \caption{Five Level of Collaborative Learning}
    \label{tab:act}
    \ls{0.9}
  \end{center}
\end{table}


\subsection{RESRA}
\label{sec:resra1}

RESRA, i.e., REpresentational Schema of Research Artifacts, is a
specialized language for representing the thematic structure of research
and learning artifacts generated from both in and outside of classrooms. In
essence, RESRA is composed of two primitives: {\it entity\/} and {\it
relation\/}. The former describes the property and the structure of
artifacts and the latter, relationships between entities. Table
~\ref{tab:er} lists the set of entities currently defined in RESRA.  Figure
~\ref{fig:resra} shows graphically the relationships between those
entities. 

{\small
\begin{table}[hbt]
  \ls{0.8}
  \begin{center}
    \begin{tabular} {|l|p{2.25in}|p{2.25in}|} \hline   
      {\bf Entity Type} & {\bf Description} & {\bf Example} \\ \hline
      
      Source (SO) & Identifiable written object, either object itself or
a pointer it, i.e., reference. & An article by Ashton; the notes from
Kyle's talk. \\ \hline
      
      Problem (PR) & A phenomenon, event, or process whose understanding
requires further inquiry; & Metalearning is not adequately supported by
existing computer-based tools. \\ \hline
      
      Claim (CL) & A position or statement about a given problem situation.
& CLARE can help the student learn how to learn. \\ \hline
      
      Evidence (EV) & Data gathered for the purpose of supporting or
objecting to a given claim. & The result of our experiments has shown that
CLARE users generate better quality research reviews than that of
non-users. \\ \hline
      
      Method (ME) & Procedures, models, or actions used for generating
evidence for a particular claim. & Three-week experiment involving six
groups (group size = 3), three of use CLARE, and the other three do not. \\
\hline
      
      Concept (CO) & Primitive construct used as building blocks for
problem statements, theories, claims, and methods. & metalearning;
knowledge representation.  \\ \hline
      
      Theory (TH) & A systemic formulation about a particular problem
domain, derivable through deductive or inductive procedures. & Ausubel's
theory of meaningful learning. \\ \hline
      
      Thing (TI) & A natural or man-made entity that is under study. &
Atom, NoteCards.  \\ \hline
      
      Critique (CR) & Comments on a given claim, evidence, method,
source, et al. & CLARE claim will much be strengthened by including
example usage experience of the system. \\ \hline
      
      Question (QU) & Aspects of a claim, theory, concept, etc., about
which the learner is still in doubt. & How does CLARE differ from such
systems as NoteCards? \\ \hline
      
      Suggestion (SU) & Ideas, recommendations, or feedbacks on how to
improve an existing problem statement, claim, method, et al.  & \\ \hline
    \end{tabular}
    \caption{Primitive Types of RESRA}
    \label{tab:er}
    \ls{1.0}
  \end{center}
\end{table}
}

\begin{figure}[htb]
  \fbox{\centerline{\psfig{figure=Figures/resra.eps,height=3.5in}}}
  \caption{A Graphical Representation of RESRA}
  \label{fig:resra}
\end{figure}


RESRA is designed to support the five levels of learning described in Table
~\ref{tab:act}, even though Figure ~\ref{fig:resra} does not show such
correspondence. Below are some illustrations.

\paragraph{Summarization}

Summarative primitives are the basis of RESRA representation. Normally, for
a well-defined artifact type such as surveys, conceptual papers, empirical
reports, a RESRA ``template'', which consists of a set of entities and
relations, may be defined. For example, an empirical study is expected to
contain instances of such primitives as {\it problem\/}, {\it claim\/},
{\it method\/}, and {\it evidence\/}, while in a conceptual paper, {\it
concept\/}, {\it theory\/}, and {\it claim\/}, and in a survey, {\it
source\/}, and {\it claim\/}. These templates provide useful thematic
heuristics which orient the learner's attention and lead to unusual
discoveries such as uncovering implicit {\it problem\/}(s) or {\it
claim\/}(s) in a research paper.

\paragraph{Evaluation}

One thrust of using RESRA as a evaluation tool is the fine-granularity
it entails: instead of merely listing major strengths and weaknesses of a
given work at the artifact level, {\it critique\/}, {\it question\/}, and
{\it suggestion\/} are directed at microscopic structures like {\it
claim\/}, {\it method\/}, {\it evidence\/}, and so forth, and relationships
among them. This ``deep-level'' evaluation requires the learner to have a
good grasp of the artifact under concern, and keeps him/her in a constructive
mode by asking questions and offering suggestions.

\paragraph{Integration}

As a key component of collaborative and meaningful learning, integration
involves relating, linking, and consolidating RESRA instances created by
different individuals and about different artifacts. As illustrated in
Section ~\ref{sec:resra-example}, it normally takes place after individual
learners complete their summarization and evaluation. In other
words, a repertoire of RESRA instances must pre-exist prior to integration.

Integration often implies generalization and abstraction. For example, a
good survey paper is not merely a rehashing of existing artifacts; instead,
it needs a coherent framework to help bring together and make sense of
related artifacts. Although RESRA does not define any aggregate, CLARE is
equipped with its own aggregation mechanisms (see ~\ref{sec:features}).

\paragraph{Argumentation}

RESRA subsumes two commonly used rhetoric models, i.e., IBIS
\cite{Conklin88Gibis} and \cite{Toulmin84}. Argumentation in RESRA involves
making alternative claims, defending existing claims using evidence, or
posting questions on both. It is commonly employed as a means to resolve
representational differences among individual learners so that integration
might be achieved. However, RESRA can also be used to capture the
rhetorical process exhibited in research literature by linking together
through appropriate relation types related artifacts, e.g., a chain of
articles that centered on a provocative primary work.

\paragraph{Construction}

RESRA is fundamentally a knowledge construction tool. Because of its
thematic nature, RESRA can help expose gaps in existing knowledge by juxtaposing
contending/related claims and different learners' perspectives, and by
highlighting essential elements and relationships among various artifacts.
It also encourages the learner to ask questions which in turn forms a basis
for further inquiry. RESRA may also be considered as a sophisticated idea
structuring tool for individuals and groups alike. It allows one to
``index'' ideas as they appear, and eventually leads to a systematic
formulation, and perhaps, a new artifact. 

In sum, RESRA offers a sound representational basis for supporting all five
levels of learning. Though the description thus far is largely confined to
the predefined RESRA primitives, it is important to realize that one
principal feature of RESRA is its open-endedness: the users are not only
allowed but encouraged to extend and adapt the initial set of primitives,
and engage discussions about proper representational structures for the
given domain and group settings. The value of RESRA resides in two
qualities: heuristic and definitional, but more in the former than in the
latter.  Therefore, a computational environment that claims to support
RESRA, e.g., CLARE, ought to provide services that exploit this heuristic
nature.


\subsection{An example use of RESRA}
\label{sec:resra-example}

Figure ~\ref{fig:cb1} and Figure ~\ref{fig:cb2} are two example RESRA
representations (condensed for the sake of space) from two individuals
about the same research report \cite{Kaplan92}, both of which include
summarative and evaluative instances. Though {\bf R1} and {\bf R2} look
alike in some ways, they differ significantly in others. Those differences,
as shown in Table ~\ref{tab:difference}, reflect different viewpoints,
focuses, and levels of understanding of the artifact by the two students.

\begin{figure}[htb]
  \fbox{\centerline{\psfig{figure=Figures/cb1.eps,height=3.5in}}}
  \caption{RESRA Representation of [Kaplan92] by Student A}
  \label{fig:cb1}
\end{figure}

\begin{figure}[htb]
  \fbox{\centerline{\psfig{figure=Figures/cb2.eps,height=3.5in}}}
  \caption{RESRA Representation of [Kaplan92] by Student B}
  \label{fig:cb2}
\end{figure}

\begin{table}[ht]
  \ls{0.8}
  \begin{center}
    \begin{tabular} {|p{2.75in}|p{2.75in}|} \hline   
      {\bf R1 } & {\bf R2}\\ \hline 4 claims, including 2 sub-claims. & 1
claim. \\ \hline
      
      Winograd/Flores' language/action theory is a claim. & W/F theory is a
concept. \\ \hline
      
      The example is evidence. & The example is a thing. \\ \hline
      
      0 suggestion. & 1 suggestion for future direction. \\ \hline
      
      2 critiques on claims, 1 on example, and 1 on concept. & 1 critique
on claim, 1 on concept, and 3 on things. \\ \hline
      
      4 concepts, two of which are not in R2. & 4 concepts, two of which
are not in R1. \\ \hline
    \end{tabular}
    \caption{Key Differences Between Two Representations}
    \label{tab:difference}
    \ls{0.9}
  \end{center}
\end{table}

Suppose that the two students who created the above representations are
given opportunity to see each other's work. In light of obvious
differences, they are expected to ask each other many questions. For
example, student {\bf A} may ask student {\bf B} why the example CB usage
scenario in the paper is treated as {\it thing\/} instead of {\it
evidence\/} supporting the authors' claims. Similarly, student {\bf B} may
ask student {\bf A} how he has come up with four {\it claims\/} instead of
just one. Such an exchange helps the students better understand
each other's perspectives with regard to the artifact and allows them to
reach a consensual view of the artifact itself. It also leads to the
creation of additional RESRA instances which expose the reasoning behind
the initial representation. Figure ~\ref{fig:cbx} shows merely one of many
possible reconciliated representations. Note that such a figure may be the
evolutionary result of several intermediate RESRA representations.

\begin{figure}[htb]
  \fbox{\centerline{\psfig{figure=Figures/cbx.eps,height=3.5in}}}
  \caption{An integrated RESRA Representation of [Kaplan92]}
  \label{fig:cbx}
\end{figure}


