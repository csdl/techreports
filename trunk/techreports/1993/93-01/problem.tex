\section{Introduction}
\label{sec:introduction}

While collaborative learning is common and widely encouraged in college
classrooms, the level of computer-based support has been quite limited.
Many existing learning support environments, such as computer-assisted
instruction (CAI) and intelligent tutoring systems (ITS), are designed to
facilitate individual learning.  Computer-mediated communication (CMC),
namely, electronic mail and bulletin-board systems (e.g. EIES), is perhaps
the most commonly used collaborative learning tool
\cite{HILTZ88Collaborative}. They have been found quite effective in
creating ``virtual classrooms'' by overcoming the ``same-time, same-place''
requirement of the face-to-face learning.  However, such systems do not
provide services specialized for specific learning tasks and processes.
Other more sophisticated tools, e.g.  Intermedia \cite{Yankelovich88},
attempt to integrate multiple applications (e.g., word processing, drawing,
mail) into a single environment. In addition, they also provide hypermedia
capabilities which allow easy linking and integration of various types of
objects (e.g., text, graphics, video) into any application. Despite their
functional versatility, these tools are still fairly general-purpose, and
their usefulness is confined to mere online authoring, browsing,
annotation, or information sharing, all of which have long been available
as separate systems.  What these systems fail to provide is task-specific
support, such as helping the learner structure a research paper, prepare a
presentation, or engage in a focused discussion.  What is also missing is
the ability to represent and highlight individual differences and
similarities and the ability to leverage on them.

This research concerns the representation, or lack of same, that underlines
existing collaborative learning systems. We claim that the lack of an
expressive and usable representation for organizing the subject content of
learning, integrating various classroom activities, (e.g., reading,
writing, discussion, presentation) and comparing and contrasting various
viewpoints from individual learners accounts for most of the above
mentioned problems.  First, we propose a representational scheme, called
RESRA, which characterizes the thematic structure of learning and research
materials or artifacts.  We develop a computer-based tool, called CLARE,
that supports the use of RESRA for various learning tasks, e.g., writing
research reviews, engaging in online discussions, drafting research
proposals.  We also design two experiments which will allow us to
empirically evaluate the effectiveness of CLARE and test our research
claims.

This proposal is organized as following: Section ~\ref{sec:problem}
describes the problem, context, and our research claims. Section
~\ref{sec:resra} elaborates the representational framework, i.e., RESRA.
Section ~\ref{sec:clare} depicts the main design features and the
architecture of CLARE. Section ~\ref{sec:evaluation} outlines the two
experiments we plan to conduct to evaluate CLARE. Section
~\ref{sec:related-work} relates the current research to a broader context
of existing work.  And finally, Section ~\ref{sec:plan} enlists the plan of
actions for the current project.

\section{The Problem: Context and Claims}
\label{sec:problem}

\subsection{Theoretical Background}
\label{sec:motivation}

An essential consideration of any successful computer-based support
environment is the melding of appropriate theory with innovative approaches
to tool development. Our work is driven by three distinct and yet related
learning theories: {\it metalearning\/}, {\it meaningful learning\/} and
{\it constructivist pedagogy\/}. Together, these theories underscore the
inadequacies of existing tools and provide directions for new systems.

\paragraph{Metalearning}

The notion of {\it metalearning\/}, as defined in \cite{Novak84}, concerns
the understanding of the nature and structure of knowledge and learning
itself. The other side of the coin is {\it content learning\/}, or
understanding of the content of a specific topic, such as how human brains
work. In a typical classroom setting, the two types of learning are often
interwined. Participants in research seminars, for instance, are expected
to both understand the particular subject matter addressed in the seminar,
e.g., artificial intelligence, and to learn how to collaborate, how to
research literature, how to present and evaluate research artifacts, how to
identify interesting problems and develop novel solutions, and so forth.
The significance of this distinction are twofold: first, as described in
Section ~\ref{sec:problem1}, although content learning tools are improving,
the support for metalearning is not, not to mention the support for the
integration of the two. Second, metalearning has become increasingly
important in today's world in part because of the accelerating rate at
which knowledge is produced and disseminated. Students will find the
subject content they learn in school to quickly become obsolete. In
contrast, the metaknowledge they acquire will enable them in the long run
to adapt and cope with the changing state of human knowledge.

\paragraph{Meaningful Learning}

The fundamental assumption of meaningful learning theory, also known as
assimilation theory of cognitive learning, is that the single most
important factor influencing learning is what the learner already knows,
and that learning is evidenced by a change in the meaning of experience
rather than a change in behavior, as held by behavioral psychologists
\cite{Ausubel63,Ausubel78}. The key question is how to help students to
reflect on their experience and to construct new meanings.  Novak and Gowin
\cite{Novak84} propose two metacognitive strategies: concept maps and Vee
diagrams, both of which are tools intended to represent changes in the
knowledge structure of students over time and help them ``learn how to
learn''.  Concept mapping is widely accepted in the educational community.
Numerous studies have shown its effectiveness in facilitating meaningful
learning \cite{Cliburn90,Novak90,Roth92}. The Vee diagram, however, is less
widely known and used.

\paragraph{Constructivist Pedagogy}

Constructivism holds that knowledge in general and scientific knowledge in
particular is socially constructed \cite{Berger66,Knorr-Cetina81}. Such
knowledge, instead of being the same for individuals, is
``taken-to-be-shared'' \cite{Roth92} with communities of knowers. To become
a member of such a community, students need to engage in collaborative
interactions and undergo learning situations which allow them to be
enculturated into the discourse practice of a field. In order to form
classroom communities which function like those of scientists and
researchers, for example, students need to be given the opportunity to
engage in authentic practices of scientists and researchers.

\subsection{Problem Characterization}
\label{sec:problem1}

In an earlier phase of this research, we identified four problems in
collaborative learning that could be addressed through computerized support
\cite{csdl-92-03}:

\begin{itemize}
\item {\it Face-to-face barriers.} Face-to-face barriers can be either
  interpersonal or intercultural/interlingual. In a seminar setting, for
  example, discussions can be dominated by a few ``strong personalities''
  or by the seminar leader(s). Individual contributions to the group
  discussion might be inhibited because some participants do not feel
  comfortable speaking openly in a group, or expressing verbal disagreement
  with other participants, especially the leader. In seminars composed of
  people from different cultural and linguistic backgrounds, intercultural
  gaps and differences in language fluency might prevent the minorities
  from full participation.
  
\item {\it Same-place, same-time constraints.} In a conventional seminar,
  again, physical co-presence in classroom is a prerequisite for
  participation.  Interactions among participants often take place only
  when they meet face-to-face. Collaborative possibilities rarely go beyond
  the boundaries of the classroom. It is also often difficult to moderate a
  class session in such a fashion that all participants are heard from and
  that a broad range of potential collaborative activities are accomplished
  within a relatively short period of time.
  
\item {\it Discontinuity across different seminar sessions.} Face-to-face
  interaction can be very effective, but its effectiveness is bounded by
  its transitory nature. Continuity between seminar sessions, for instance,
  is difficult to maintain without a disciplined long-term {\it memory\/},
  i.e., explicit tracking of both processes and activity contents and
  physical means of connecting them together. This is particularly true
  when meetings take place on a relatively infrequent basis and
  participants are physically distributed, as often the case with most
  seminars.
  
\item {\it Lack of organic links between reading, writing, presentation
  and discussion.} Seminar activities are inherently both integrative and
  exploratory. To support either type of activity requires explicit
  representation and manipulation of links between various activity
  structures, contents, and processes. In traditional seminars, however,
  the awareness, storage and retrieval of such connections are largely left
  to individual participants.  As a group, there exists no external {\it
  pool\/} of mutual artifacts which group members can contribute to and
  benefit from.
\end{itemize}

Perhaps in part due to their significance, most of the above problems have
been addressed in existing learning tools. For example, computer-mediated
communication is used to increase student participation outside physical
classrooms \cite{HILTZ88Collaborative}. Similarly, hypermedia systems, such
as Intermedia \cite{Yankelovich88} and NoteCards \cite{Halasz87NOTECARDS},
are found quite effective for both authoring, browsing and presenting
shared information. What is missing in these tools, however, is mechanisms
necessary to facilitate metalearning, to help the student extract meanings
from research papers, books, presentations, and discussions, and to enable
the student to do both in a collaborative fashion. Intermedia, for
instance, allows multiple users to concurrently create and follow links in
the same web, but provides little hint on what those links and webs mean to
the user.

Concept mapping is perhaps among the few attempts to provide explicit
support for meta- and meaningful learning \cite{Novak84} in classrooms.
Systems such as SemNet \cite{Fisher90} are based on semantic network
theory, a model of human memory and knowledge representation first proposed
by \cite{Quilian67}. In fact, the term {\it semantic network\/} and {\it
concept map\/} are used interexchangeably in SemNet. What differentiates
the two seems that semantic networks are constructed by trained knowledge
engineers for machine reasoning, while concept maps are built and used by
human learners. This difference implies that concept maps are much simpler
than semantic networks. It also reveals two potential pitfalls of concept
mapping as a metalearning tool:

\begin{itemize}
\item {\it Atomic structure:\/} In concept maps, all knowledge must
  ultimately be reduced to concepts and links among them.  This, though
  easily achievable for introductory textbooks, is far from adequate in
  advanced learning (e.g., graduate seminars), which often requires analysis
  and synthesis using high-level constructs, e.g., claims, problems.
  
\item {\it Free form of expression:\/} Concept map, like the designer's
  sketch pad, gives the learner maximum freedom in deciding what to draw
  and how to draw it. The representation does not dictate nor provides any
  heuristic guidelines on how it should be used. This flexibility, which
  makes concept maps extremely expressive, also adds little structure
  useful as the basis of computation, and which human learners may rely on
  to help them make sense of the map. The latter is especially significant
  in a collaborative sense, for this arbitrary nature means that it is
  difficult to compare, contrast, and integrate concept maps generated by
  different individual leaners.
\end{itemize}

Few existing systems support collaborative construction of concept maps.
In their study, for example, \cite{Roth92} have to rely on movable paper
clips instead of a computerized system. Hence, the basic problem still
remains:

\begin{quotation}
  {\it Learning theories suggest that metaknowledge, metalearning, and
  collaboration are essential to meaningful learning. However, existing
  computer-based environments are largely for supporting content learning,
  communication, and information sharing. Even metacognitive tools such as
  concept maps fail to provide adequate structural heuristics for both
  computation and augmenting human learning.}
\end{quotation}

The above problem is what this research is intended to address. The
following sections will provide detailed description of our approach.


\subsection{Thesis}
\label{sec:claims}

The fundamental premise of our approach is that a well conceived
semi-structured representation, coupled with a set of properly chosen
computational services, can provide a sound basis for facilitating
collaborative meta- and meaningful learning. The key phrases here are
``well-conceived'' and ``properly chosen''. By that we do not mean that we
hold the ``holy grail'' or ``silver bullet'', if such a thing indeed
exists. Our notion of a ``well-conceived'' representation is one that
provides the learner with maximum heuristic values in the following three
areas:

\begin{itemize}
\item content learning: i.e., helping the learner organize and make sense
  of specific learning materials;
  
\item exploration of metaknowledge: i.e., providing a basis on which new
  representational primitives might be identified; and
  
\item ongoing and incremental interplay between the above two.
\end{itemize}

Our notion of ``properly chosen'' services are the ones which support this
interplay between representational/meta-level exploration and content
learning.  Because of this structural indeterminacy, we believe that examples are
essential in directing proper use of a representation. In our system, we
strive to provide examples at various levels of abstraction and allow them
to be upgraded on an ongoing basis.

The central thesis of this research is that CLARE represents a novel and
useful approach to collaborative learning: CLARE supported learning will be
more effective than that of the traditional mode of learning, i.e.,
face-to-face and pencil-and-paper based. It will also show significant
improvement over other existing computer-based tools, such as concept
mapping, or hypermedia. The focus of this research is on the former.
Specifically, the two experiments described in Section
~\ref{sec:evaluation} are designed to empirically test this claim. In the
near future, we plan to conduct experiments which compare CLARE with other
tools of the same class, e.g., concept mapping.

\subsection{Contributions}
\label{sec:contributions}

The research contributions of this work can be viewed at three levels:

\begin{enumerate}
\item Conceptually, RESRA represents a new approach to collaborative
  learning that is based on constructivist pedagogy and the
  Ausubel-Novak-Gowin theory of meaningful learning. It overcomes the
  structural weakness of concept maps. On the other hand, RESRA constructs
  are not intended to restrict the user's expressiveness and how they learn
  but rather, serve as a heuristic basis for computation and for evolving a
  representation appropriate for meaning extraction and knowledge
  construction in specific group and domain settings.
  
\item Technically, CLARE implements a set of services for facilitating
  content and metalearning based on RESRA. The five levels of abstraction it
  embodies, i.e., RESRA primitives, domain, template, and example libraries,
  and specific instances, creates room for both computation and learner
  exploration. The explicit treatment of perspectives and the provision of a
  multiway comparator illustrate collaborative services beyond simple access
  control, information sharing, and online annotation.
  
\item Empirically, our evaluation experiments will provide some primary
  data on how students react, use, and think about our approach. They will
  also allow us to assess the effect of CLARE on the student's performance
  of selected learning tasks. The outcome of these experiments should shed
  important light on proper mechanisms for supporting collaborative
  metalearning, and provide a basis for further investigation.
\end{enumerate}

\subsection{Limitations}
\label{sec:limitations}

Collaborative learning is a complex activity to study and support. In this
research, we do not attempt to address every aspect of the subject, nor to
create a universal system that has all neat features of existing learning
support tools. Instead, we have chosen to focus on the role of
representation in collaborative learning and on developing pertinent
computational mechanisms to facilitate the use of such a representation in
various learning activities.  Below are four major limitations of the current
research:

\begin{enumerate}
\item CLARE is not an AI system. It does not provide automatic inference,
  user modeling, or natural language understanding capabilities. RESRA, the
  underlying knowledge representation scheme, is intended for helping
  human learners evaluate and construct knowledge rather than for improving
  machine reasoning, despite that many useful computations in CLARE are
  centered on RESRA.
  
\item CLARE's artifact-based approach renders its emphasis on the
  structural characteristics and relationships of collaborative learning.
  It presumes the presence of certain learning activities, e.g.,
  evaluation, argumentation. However, it does not prescribe any process
  model, i.e., how those activities should be combined.  Instead, it
  insists that choosing a proper learning model is the responsibility of
  the course or experiment designer rather than the system designer.
  
\item CLARE helps ameliorate certain problems related to face-to-face
  collaboration (e.g., ``dominant personality'') to the extent that it is a
  computer-mediated environment. However, it cannot entirely overcome the
  interpersonal and intercultural conflicts inherent in collaboration. When
  used in a face-to-face setting, it is possible that the effectiveness of
  CLARE as an augmented learning tool be overshadowed by some interpersonal
  or intercultural factors. CLARE provides no means to separate one from the
  other.
  
\item At the system level, CLARE does not have certain advanced
  functionalities found in other existing learning support environments.
  Examples include multimedia, version control, and fancy graphical
  interface. This is in part due to the pilot nature of our system: we plan
  to incrementally incorporate above functions as we have opportunity to
  empirically validate the usefulness of the CLARE's core functionalities.
\end{enumerate}











