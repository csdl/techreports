\documentstyle [/group/csdl/tex/ps-times,11pt,
                /group/csdl/tex/definemargins,
                /group/csdl/tex/lmacros]{article}
\begin{document}

\vspace*{1in}
\pagestyle{empty}

\begin{center}
  
{\Large\bf Collaborative Classification and Evaluation of Usenet}

        \bigskip
                     
         Robert S. Brewer and Philip M. Johnson\foot{Primary contact person.} \bigskip

         Department of Information \& Computer Sciences\\
         University of Hawaii\\
         Honolulu, HI 96822\\
         Tel: (808) 956-6920, (808) 956-3489\\
         E-mail: {\tt rbrewer@uhics.ics.hawaii.edu}, {\tt johnson@hawaii.edu}
         
         \bigskip

 {\bf Submission category:} Paper      \bigskip\par         

         \bigskip
\end{center}

\section* {ABSTRACT}

Usenet is an example of the potential and problems of the nascent National
Information Infrastructure. While Usenet makes an enormous amount of useful
information available to its users, the daily data overwhelms any user who
tries to read more than a fraction of it. This paper presents a
collaboration-oriented approach to information classification and evaluation
for very large, dynamic database structures such as Usenet. Our approach is
implemented in a system called URN, a multi-user, collaborative, hypertextual
Usenet reader.  We show that this collaborative method, coupled with an
adaptive interface, radically improves the overall relevance level of
information presented to a user.

\end{document}
