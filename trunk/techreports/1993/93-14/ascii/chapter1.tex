%%%%%%%%%%%%%%%%%%%%%%%%%%%%%% -*- Mode: Latex -*- %%%%%%%%%%%%%%%%%%%%%%%%%%%%
%% chapter1.tex -- 
%% RCS:            : $Id: chapter1.tex,v 1.11 94/04/07 20:46:10 dxw Exp Locker: dxw $
%% Author          : Dadong Wan
%% Created On      : Tue May 11 23:43:46 1993
%% Last Modified By: Dadong Wan
%% Last Modified On: Fri Apr  8 02:04:50 1994
%% Status          : Unknown
%%%%%%%%%%%%%%%%%%%%%%%%%%%%%%%%%%%%%%%%%%%%%%%%%%%%%%%%%%%%%%%%%%%%%%%%%%%%%%%
%%   Copyright (C) 1993 University of Hawaii
%% 
%% History
%% 11-May-1993		Dadong Wan	
%%    created
%%% \documentstyle [12pt,/group/csdl/tex/definemargins,
%%% /group/csdl/tex/lmacros]{report} % Psfig/TeX 
\def\PsfigVersion{1.9}
% dvips version
%
% All psfig/tex software, documentation, and related files
% in this distribution of psfig/tex are 
% Copyright 1987, 1988, 1991 Trevor J. Darrell
%
% Permission is granted for use and non-profit distribution of psfig/tex 
% providing that this notice is clearly maintained. The right to
% distribute any portion of psfig/tex for profit or as part of any commercial
% product is specifically reserved for the author(s) of that portion.
%
% *** Feel free to make local modifications of psfig as you wish,
% *** but DO NOT post any changed or modified versions of ``psfig''
% *** directly to the net. Send them to me and I'll try to incorporate
% *** them into future versions. If you want to take the psfig code 
% *** and make a new program (subject to the copyright above), distribute it, 
% *** (and maintain it) that's fine, just don't call it psfig.
%
% Bugs and improvements to trevor@media.mit.edu.
%
% Thanks to Greg Hager (GDH) and Ned Batchelder for their contributions
% to the original version of this project.
%
% Modified by J. Daniel Smith on 9 October 1990 to accept the
% %%BoundingBox: comment with or without a space after the colon.  Stole
% file reading code from Tom Rokicki's EPSF.TEX file (see below).
%
% More modifications by J. Daniel Smith on 29 March 1991 to allow the
% the included PostScript figure to be rotated.  The amount of
% rotation is specified by the "angle=" parameter of the \psfig command.
%
% Modified by Robert Russell on June 25, 1991 to allow users to specify
% .ps filenames which don't yet exist, provided they explicitly provide
% boundingbox information via the \psfig command. Note: This will only work
% if the "file=" parameter follows all four "bb???=" parameters in the
% command. This is due to the order in which psfig interprets these params.
%
%  3 Jul 1991	JDS	check if file already read in once
%  4 Sep 1991	JDS	fixed incorrect computation of rotated
%			bounding box
% 25 Sep 1991	GVR	expanded synopsis of \psfig
% 14 Oct 1991	JDS	\fbox code from LaTeX so \psdraft works with TeX
%			changed \typeout to \ps@typeout
% 17 Oct 1991	JDS	added \psscalefirst and \psrotatefirst
%

% From: gvr@cs.brown.edu (George V. Reilly)
%
% \psdraft	draws an outline box, but doesn't include the figure
%		in the DVI file.  Useful for previewing.
%
% \psfull	includes the figure in the DVI file (default).
%
% \psscalefirst width= or height= specifies the size of the figure
% 		before rotation.
% \psrotatefirst (default) width= or height= specifies the size of the
% 		 figure after rotation.  Asymetric figures will
% 		 appear to shrink.
%
% \psfigurepath#1	sets the path to search for the figure
%
% \psfig
% usage: \psfig{file=, figure=, height=, width=,
%			bbllx=, bblly=, bburx=, bbury=,
%			rheight=, rwidth=, clip=, angle=, silent=}
%
%	"file" is the filename.  If no path name is specified and the
%		file is not found in the current directory,
%		it will be looked for in directory \psfigurepath.
%	"figure" is a synonym for "file".
%	By default, the width and height of the figure are taken from
%		the BoundingBox of the figure.
%	If "width" is specified, the figure is scaled so that it has
%		the specified width.  Its height changes proportionately.
%	If "height" is specified, the figure is scaled so that it has
%		the specified height.  Its width changes proportionately.
%	If both "width" and "height" are specified, the figure is scaled
%		anamorphically.
%	"bbllx", "bblly", "bburx", and "bbury" control the PostScript
%		BoundingBox.  If these four values are specified
%               *before* the "file" option, the PSFIG will not try to
%               open the PostScript file.
%	"rheight" and "rwidth" are the reserved height and width
%		of the figure, i.e., how big TeX actually thinks
%		the figure is.  They default to "width" and "height".
%	The "clip" option ensures that no portion of the figure will
%		appear outside its BoundingBox.  "clip=" is a switch and
%		takes no value, but the `=' must be present.
%	The "angle" option specifies the angle of rotation (degrees, ccw).
%	The "silent" option makes \psfig work silently.
%

% check to see if macros already loaded in (maybe some other file says
% "\input psfig") ...
\ifx\undefined\psfig\else\endinput\fi

%
% from a suggestion by eijkhout@csrd.uiuc.edu to allow
% loading as a style file. Changed to avoid problems
% with amstex per suggestion by jbence@math.ucla.edu

\let\LaTeXAtSign=\@
\let\@=\relax
\edef\psfigRestoreAt{\catcode`\@=\number\catcode`@\relax}
%\edef\psfigRestoreAt{\catcode`@=\number\catcode`@\relax}
\catcode`\@=11\relax
\newwrite\@unused
\def\ps@typeout#1{{\let\protect\string\immediate\write\@unused{#1}}}
\ps@typeout{psfig/tex \PsfigVersion}

%% Here's how you define your figure path.  Should be set up with null
%% default and a user useable definition.

\def\figurepath{./}
\def\psfigurepath#1{\edef\figurepath{#1}}

%
% @psdo control structure -- similar to Latex @for.
% I redefined these with different names so that psfig can
% be used with TeX as well as LaTeX, and so that it will not 
% be vunerable to future changes in LaTeX's internal
% control structure,
%
\def\@nnil{\@nil}
\def\@empty{}
\def\@psdonoop#1\@@#2#3{}
\def\@psdo#1:=#2\do#3{\edef\@psdotmp{#2}\ifx\@psdotmp\@empty \else
    \expandafter\@psdoloop#2,\@nil,\@nil\@@#1{#3}\fi}
\def\@psdoloop#1,#2,#3\@@#4#5{\def#4{#1}\ifx #4\@nnil \else
       #5\def#4{#2}\ifx #4\@nnil \else#5\@ipsdoloop #3\@@#4{#5}\fi\fi}
\def\@ipsdoloop#1,#2\@@#3#4{\def#3{#1}\ifx #3\@nnil 
       \let\@nextwhile=\@psdonoop \else
      #4\relax\let\@nextwhile=\@ipsdoloop\fi\@nextwhile#2\@@#3{#4}}
\def\@tpsdo#1:=#2\do#3{\xdef\@psdotmp{#2}\ifx\@psdotmp\@empty \else
    \@tpsdoloop#2\@nil\@nil\@@#1{#3}\fi}
\def\@tpsdoloop#1#2\@@#3#4{\def#3{#1}\ifx #3\@nnil 
       \let\@nextwhile=\@psdonoop \else
      #4\relax\let\@nextwhile=\@tpsdoloop\fi\@nextwhile#2\@@#3{#4}}
% 
% \fbox is defined in latex.tex; so if \fbox is undefined, assume that
% we are not in LaTeX.
% Perhaps this could be done better???
\ifx\undefined\fbox
% \fbox code from modified slightly from LaTeX
\newdimen\fboxrule
\newdimen\fboxsep
\newdimen\ps@tempdima
\newbox\ps@tempboxa
\fboxsep = 3pt
\fboxrule = .4pt
\long\def\fbox#1{\leavevmode\setbox\ps@tempboxa\hbox{#1}\ps@tempdima\fboxrule
    \advance\ps@tempdima \fboxsep \advance\ps@tempdima \dp\ps@tempboxa
   \hbox{\lower \ps@tempdima\hbox
  {\vbox{\hrule height \fboxrule
          \hbox{\vrule width \fboxrule \hskip\fboxsep
          \vbox{\vskip\fboxsep \box\ps@tempboxa\vskip\fboxsep}\hskip 
                 \fboxsep\vrule width \fboxrule}
                 \hrule height \fboxrule}}}}
\fi
%
%%%%%%%%%%%%%%%%%%%%%%%%%%%%%%%%%%%%%%%%%%%%%%%%%%%%%%%%%%%%%%%%%%%
% file reading stuff from epsf.tex
%   EPSF.TEX macro file:
%   Written by Tomas Rokicki of Radical Eye Software, 29 Mar 1989.
%   Revised by Don Knuth, 3 Jan 1990.
%   Revised by Tomas Rokicki to accept bounding boxes with no
%      space after the colon, 18 Jul 1990.
%   Portions modified/removed for use in PSFIG package by
%      J. Daniel Smith, 9 October 1990.
%
\newread\ps@stream
\newif\ifnot@eof       % continue looking for the bounding box?
\newif\if@noisy        % report what you're making?
\newif\if@atend        % %%BoundingBox: has (at end) specification
\newif\if@psfile       % does this look like a PostScript file?
%
% PostScript files should start with `%!'
%
{\catcode`\%=12\global\gdef\epsf@start{%!}}
\def\epsf@PS{PS}
%
\def\epsf@getbb#1{%
%
%   The first thing we need to do is to open the
%   PostScript file, if possible.
%
\openin\ps@stream=#1
\ifeof\ps@stream\ps@typeout{Error, File #1 not found}\else
%
%   Okay, we got it. Now we'll scan lines until we find one that doesn't
%   start with %. We're looking for the bounding box comment.
%
   {\not@eoftrue \chardef\other=12
    \def\do##1{\catcode`##1=\other}\dospecials \catcode`\ =10
    \loop
       \if@psfile
	  \read\ps@stream to \epsf@fileline
       \else{
	  \obeyspaces
          \read\ps@stream to \epsf@tmp\global\let\epsf@fileline\epsf@tmp}
       \fi
       \ifeof\ps@stream\not@eoffalse\else
%
%   Check the first line for `%!'.  Issue a warning message if its not
%   there, since the file might not be a PostScript file.
%
       \if@psfile\else
       \expandafter\epsf@test\epsf@fileline:. \\%
       \fi
%
%   We check to see if the first character is a % sign;
%   if so, we look further and stop only if the line begins with
%   `%%BoundingBox:' and the `(atend)' specification was not found.
%   That is, the only way to stop is when the end of file is reached,
%   or a `%%BoundingBox: llx lly urx ury' line is found.
%
          \expandafter\epsf@aux\epsf@fileline:. \\%
       \fi
   \ifnot@eof\repeat
   }\closein\ps@stream\fi}%
%
% This tests if the file we are reading looks like a PostScript file.
%
\long\def\epsf@test#1#2#3:#4\\{\def\epsf@testit{#1#2}
			\ifx\epsf@testit\epsf@start\else
\ps@typeout{Warning! File does not start with `\epsf@start'.  It may not be a PostScript file.}
			\fi
			\@psfiletrue} % don't test after 1st line
%
%   We still need to define the tricky \epsf@aux macro. This requires
%   a couple of magic constants for comparison purposes.
%
{\catcode`\%=12\global\let\epsf@percent=%\global\def\epsf@bblit{%BoundingBox}}
%
%
%   So we're ready to check for `%BoundingBox:' and to grab the
%   values if they are found.  We continue searching if `(at end)'
%   was found after the `%BoundingBox:'.
%
\long\def\epsf@aux#1#2:#3\\{\ifx#1\epsf@percent
   \def\epsf@testit{#2}\ifx\epsf@testit\epsf@bblit
	\@atendfalse
        \epsf@atend #3 . \\%
	\if@atend	
	   \if@verbose{
		\ps@typeout{psfig: found `(atend)'; continuing search}
	   }\fi
        \else
        \epsf@grab #3 . . . \\%
        \not@eoffalse
        \global\no@bbfalse
        \fi
   \fi\fi}%
%
%   Here we grab the values and stuff them in the appropriate definitions.
%
\def\epsf@grab #1 #2 #3 #4 #5\\{%
   \global\def\epsf@llx{#1}\ifx\epsf@llx\empty
      \epsf@grab #2 #3 #4 #5 .\\\else
   \global\def\epsf@lly{#2}%
   \global\def\epsf@urx{#3}\global\def\epsf@ury{#4}\fi}%
%
% Determine if the stuff following the %%BoundingBox is `(atend)'
% J. Daniel Smith.  Copied from \epsf@grab above.
%
\def\epsf@atendlit{(atend)} 
\def\epsf@atend #1 #2 #3\\{%
   \def\epsf@tmp{#1}\ifx\epsf@tmp\empty
      \epsf@atend #2 #3 .\\\else
   \ifx\epsf@tmp\epsf@atendlit\@atendtrue\fi\fi}


% End of file reading stuff from epsf.tex
%%%%%%%%%%%%%%%%%%%%%%%%%%%%%%%%%%%%%%%%%%%%%%%%%%%%%%%%%%%%%%%%%%%

%%%%%%%%%%%%%%%%%%%%%%%%%%%%%%%%%%%%%%%%%%%%%%%%%%%%%%%%%%%%%%%%%%%
% trigonometry stuff from "trig.tex"
\chardef\psletter = 11 % won't conflict with \begin{letter} now...
\chardef\other = 12

\newif \ifdebug %%% turn me on to see TeX hard at work ...
\newif\ifc@mpute %%% don't need to compute some values
\c@mputetrue % but assume that we do

\let\then = \relax
\def\r@dian{pt }
\let\r@dians = \r@dian
\let\dimensionless@nit = \r@dian
\let\dimensionless@nits = \dimensionless@nit
\def\internal@nit{sp }
\let\internal@nits = \internal@nit
\newif\ifstillc@nverging
\def \Mess@ge #1{\ifdebug \then \message {#1} \fi}

{ %%% Things that need abnormal catcodes %%%
	\catcode `\@ = \psletter
	\gdef \nodimen {\expandafter \n@dimen \the \dimen}
	\gdef \term #1 #2 #3%
	       {\edef \t@ {\the #1}%%% freeze parameter 1 (count, by value)
		\edef \t@@ {\expandafter \n@dimen \the #2\r@dian}%
				   %%% freeze parameter 2 (dimen, by value)
		\t@rm {\t@} {\t@@} {#3}%
	       }
	\gdef \t@rm #1 #2 #3%
	       {{%
		\count 0 = 0
		\dimen 0 = 1 \dimensionless@nit
		\dimen 2 = #2\relax
		\Mess@ge {Calculating term #1 of \nodimen 2}%
		\loop
		\ifnum	\count 0 < #1
		\then	\advance \count 0 by 1
			\Mess@ge {Iteration \the \count 0 \space}%
			\Multiply \dimen 0 by {\dimen 2}%
			\Mess@ge {After multiplication, term = \nodimen 0}%
			\Divide \dimen 0 by {\count 0}%
			\Mess@ge {After division, term = \nodimen 0}%
		\repeat
		\Mess@ge {Final value for term #1 of 
				\nodimen 2 \space is \nodimen 0}%
		\xdef \Term {#3 = \nodimen 0 \r@dians}%
		\aftergroup \Term
	       }}
	\catcode `\p = \other
	\catcode `\t = \other
	\gdef \n@dimen #1pt{#1} %%% throw away the ``pt''
}

\def \Divide #1by #2{\divide #1 by #2} %%% just a synonym

\def \Multiply #1by #2%%% allows division of a dimen by a dimen
       {{%%% should really freeze parameter 2 (dimen, passed by value)
	\count 0 = #1\relax
	\count 2 = #2\relax
	\count 4 = 65536
	\Mess@ge {Before scaling, count 0 = \the \count 0 \space and
			count 2 = \the \count 2}%
	\ifnum	\count 0 > 32767 %%% do our best to avoid overflow
	\then	\divide \count 0 by 4
		\divide \count 4 by 4
	\else	\ifnum	\count 0 < -32767
		\then	\divide \count 0 by 4
			\divide \count 4 by 4
		\else
		\fi
	\fi
	\ifnum	\count 2 > 32767 %%% while retaining reasonable accuracy
	\then	\divide \count 2 by 4
		\divide \count 4 by 4
	\else	\ifnum	\count 2 < -32767
		\then	\divide \count 2 by 4
			\divide \count 4 by 4
		\else
		\fi
	\fi
	\multiply \count 0 by \count 2
	\divide \count 0 by \count 4
	\xdef \product {#1 = \the \count 0 \internal@nits}%
	\aftergroup \product
       }}

\def\r@duce{\ifdim\dimen0 > 90\r@dian \then   % sin(x+90) = sin(180-x)
		\multiply\dimen0 by -1
		\advance\dimen0 by 180\r@dian
		\r@duce
	    \else \ifdim\dimen0 < -90\r@dian \then  % sin(-x) = sin(360+x)
		\advance\dimen0 by 360\r@dian
		\r@duce
		\fi
	    \fi}

\def\Sine#1%
       {{%
	\dimen 0 = #1 \r@dian
	\r@duce
	\ifdim\dimen0 = -90\r@dian \then
	   \dimen4 = -1\r@dian
	   \c@mputefalse
	\fi
	\ifdim\dimen0 = 90\r@dian \then
	   \dimen4 = 1\r@dian
	   \c@mputefalse
	\fi
	\ifdim\dimen0 = 0\r@dian \then
	   \dimen4 = 0\r@dian
	   \c@mputefalse
	\fi
%
	\ifc@mpute \then
        	% convert degrees to radians
		\divide\dimen0 by 180
		\dimen0=3.141592654\dimen0
%
		\dimen 2 = 3.1415926535897963\r@dian %%% a well-known constant
		\divide\dimen 2 by 2 %%% we only deal with -pi/2 : pi/2
		\Mess@ge {Sin: calculating Sin of \nodimen 0}%
		\count 0 = 1 %%% see power-series expansion for sine
		\dimen 2 = 1 \r@dian %%% ditto
		\dimen 4 = 0 \r@dian %%% ditto
		\loop
			\ifnum	\dimen 2 = 0 %%% then we've done
			\then	\stillc@nvergingfalse 
			\else	\stillc@nvergingtrue
			\fi
			\ifstillc@nverging %%% then calculate next term
			\then	\term {\count 0} {\dimen 0} {\dimen 2}%
				\advance \count 0 by 2
				\count 2 = \count 0
				\divide \count 2 by 2
				\ifodd	\count 2 %%% signs alternate
				\then	\advance \dimen 4 by \dimen 2
				\else	\advance \dimen 4 by -\dimen 2
				\fi
		\repeat
	\fi		
			\xdef \sine {\nodimen 4}%
       }}

% Now the Cosine can be calculated easily by calling \Sine
\def\Cosine#1{\ifx\sine\UnDefined\edef\Savesine{\relax}\else
		             \edef\Savesine{\sine}\fi
	{\dimen0=#1\r@dian\advance\dimen0 by 90\r@dian
	 \Sine{\nodimen 0}
	 \xdef\cosine{\sine}
	 \xdef\sine{\Savesine}}}	      
% end of trig stuff
%%%%%%%%%%%%%%%%%%%%%%%%%%%%%%%%%%%%%%%%%%%%%%%%%%%%%%%%%%%%%%%%%%%%

\def\psdraft{
	\def\@psdraft{0}
	%\ps@typeout{draft level now is \@psdraft \space . }
}
\def\psfull{
	\def\@psdraft{100}
	%\ps@typeout{draft level now is \@psdraft \space . }
}

\psfull

\newif\if@scalefirst
\def\psscalefirst{\@scalefirsttrue}
\def\psrotatefirst{\@scalefirstfalse}
\psrotatefirst

\newif\if@draftbox
\def\psnodraftbox{
	\@draftboxfalse
}
\def\psdraftbox{
	\@draftboxtrue
}
\@draftboxtrue

\newif\if@prologfile
\newif\if@postlogfile
\def\pssilent{
	\@noisyfalse
}
\def\psnoisy{
	\@noisytrue
}
\psnoisy
%%% These are for the option list.
%%% A specification of the form a = b maps to calling \@p@@sa{b}
\newif\if@bbllx
\newif\if@bblly
\newif\if@bburx
\newif\if@bbury
\newif\if@height
\newif\if@width
\newif\if@rheight
\newif\if@rwidth
\newif\if@angle
\newif\if@clip
\newif\if@verbose
\def\@p@@sclip#1{\@cliptrue}


\newif\if@decmpr

%%% GDH 7/26/87 -- changed so that it first looks in the local directory,
%%% then in a specified global directory for the ps file.
%%% RPR 6/25/91 -- changed so that it defaults to user-supplied name if
%%% boundingbox info is specified, assuming graphic will be created by
%%% print time.
%%% TJD 10/19/91 -- added bbfile vs. file distinction, and @decmpr flag

\def\@p@@sfigure#1{\def\@p@sfile{null}\def\@p@sbbfile{null}
	        \openin1=#1.bb
		\ifeof1\closein1
	        	\openin1=\figurepath#1.bb
			\ifeof1\closein1
			        \openin1=#1
				\ifeof1\closein1%
				       \openin1=\figurepath#1
					\ifeof1
					   \ps@typeout{Error, File #1 not found}
						\if@bbllx\if@bblly
				   		\if@bburx\if@bbury
			      				\def\@p@sfile{#1}%
			      				\def\@p@sbbfile{#1}%
							\@decmprfalse
				  	   	\fi\fi\fi\fi
					\else\closein1
				    		\def\@p@sfile{\figurepath#1}%
				    		\def\@p@sbbfile{\figurepath#1}%
						\@decmprfalse
	                       		\fi%
			 	\else\closein1%
					\def\@p@sfile{#1}
					\def\@p@sbbfile{#1}
					\@decmprfalse
			 	\fi
			\else
				\def\@p@sfile{\figurepath#1}
				\def\@p@sbbfile{\figurepath#1.bb}
				\@decmprtrue
			\fi
		\else
			\def\@p@sfile{#1}
			\def\@p@sbbfile{#1.bb}
			\@decmprtrue
		\fi}

\def\@p@@sfile#1{\@p@@sfigure{#1}}

\def\@p@@sbbllx#1{
		%\ps@typeout{bbllx is #1}
		\@bbllxtrue
		\dimen100=#1
		\edef\@p@sbbllx{\number\dimen100}
}
\def\@p@@sbblly#1{
		%\ps@typeout{bblly is #1}
		\@bbllytrue
		\dimen100=#1
		\edef\@p@sbblly{\number\dimen100}
}
\def\@p@@sbburx#1{
		%\ps@typeout{bburx is #1}
		\@bburxtrue
		\dimen100=#1
		\edef\@p@sbburx{\number\dimen100}
}
\def\@p@@sbbury#1{
		%\ps@typeout{bbury is #1}
		\@bburytrue
		\dimen100=#1
		\edef\@p@sbbury{\number\dimen100}
}
\def\@p@@sheight#1{
		\@heighttrue
		\dimen100=#1
   		\edef\@p@sheight{\number\dimen100}
		%\ps@typeout{Height is \@p@sheight}
}
\def\@p@@swidth#1{
		%\ps@typeout{Width is #1}
		\@widthtrue
		\dimen100=#1
		\edef\@p@swidth{\number\dimen100}
}
\def\@p@@srheight#1{
		%\ps@typeout{Reserved height is #1}
		\@rheighttrue
		\dimen100=#1
		\edef\@p@srheight{\number\dimen100}
}
\def\@p@@srwidth#1{
		%\ps@typeout{Reserved width is #1}
		\@rwidthtrue
		\dimen100=#1
		\edef\@p@srwidth{\number\dimen100}
}
\def\@p@@sangle#1{
		%\ps@typeout{Rotation is #1}
		\@angletrue
%		\dimen100=#1
		\edef\@p@sangle{#1} %\number\dimen100}
}
\def\@p@@ssilent#1{ 
		\@verbosefalse
}
\def\@p@@sprolog#1{\@prologfiletrue\def\@prologfileval{#1}}
\def\@p@@spostlog#1{\@postlogfiletrue\def\@postlogfileval{#1}}
\def\@cs@name#1{\csname #1\endcsname}
\def\@setparms#1=#2,{\@cs@name{@p@@s#1}{#2}}
%
% initialize the defaults (size the size of the figure)
%
\def\ps@init@parms{
		\@bbllxfalse \@bbllyfalse
		\@bburxfalse \@bburyfalse
		\@heightfalse \@widthfalse
		\@rheightfalse \@rwidthfalse
		\def\@p@sbbllx{}\def\@p@sbblly{}
		\def\@p@sbburx{}\def\@p@sbbury{}
		\def\@p@sheight{}\def\@p@swidth{}
		\def\@p@srheight{}\def\@p@srwidth{}
		\def\@p@sangle{0}
		\def\@p@sfile{} \def\@p@sbbfile{}
		\def\@p@scost{10}
		\def\@sc{}
		\@prologfilefalse
		\@postlogfilefalse
		\@clipfalse
		\if@noisy
			\@verbosetrue
		\else
			\@verbosefalse
		\fi
}
%
% Go through the options setting things up.
%
\def\parse@ps@parms#1{
	 	\@psdo\@psfiga:=#1\do
		   {\expandafter\@setparms\@psfiga,}}
%
% Compute bb height and width
%
\newif\ifno@bb
\def\bb@missing{
	\if@verbose{
		\ps@typeout{psfig: searching \@p@sbbfile \space  for bounding box}
	}\fi
	\no@bbtrue
	\epsf@getbb{\@p@sbbfile}
        \ifno@bb \else \bb@cull\epsf@llx\epsf@lly\epsf@urx\epsf@ury\fi
}	
\def\bb@cull#1#2#3#4{
	\dimen100=#1 bp\edef\@p@sbbllx{\number\dimen100}
	\dimen100=#2 bp\edef\@p@sbblly{\number\dimen100}
	\dimen100=#3 bp\edef\@p@sbburx{\number\dimen100}
	\dimen100=#4 bp\edef\@p@sbbury{\number\dimen100}
	\no@bbfalse
}
% rotate point (#1,#2) about (0,0).
% The sine and cosine of the angle are already stored in \sine and
% \cosine.  The result is placed in (\p@intvaluex, \p@intvaluey).
\newdimen\p@intvaluex
\newdimen\p@intvaluey
\def\rotate@#1#2{{\dimen0=#1 sp\dimen1=#2 sp
%            	calculate x' = x \cos\theta - y \sin\theta
		  \global\p@intvaluex=\cosine\dimen0
		  \dimen3=\sine\dimen1
		  \global\advance\p@intvaluex by -\dimen3
% 		calculate y' = x \sin\theta + y \cos\theta
		  \global\p@intvaluey=\sine\dimen0
		  \dimen3=\cosine\dimen1
		  \global\advance\p@intvaluey by \dimen3
		  }}
\def\compute@bb{
		\no@bbfalse
		\if@bbllx \else \no@bbtrue \fi
		\if@bblly \else \no@bbtrue \fi
		\if@bburx \else \no@bbtrue \fi
		\if@bbury \else \no@bbtrue \fi
		\ifno@bb \bb@missing \fi
		\ifno@bb \ps@typeout{FATAL ERROR: no bb supplied or found}
			\no-bb-error
		\fi
		%
%\ps@typeout{BB: \@p@sbbllx, \@p@sbblly, \@p@sbburx, \@p@sbbury} 
%
% store height/width of original (unrotated) bounding box
		\count203=\@p@sbburx
		\count204=\@p@sbbury
		\advance\count203 by -\@p@sbbllx
		\advance\count204 by -\@p@sbblly
		\edef\ps@bbw{\number\count203}
		\edef\ps@bbh{\number\count204}
		%\ps@typeout{ psbbh = \ps@bbh, psbbw = \ps@bbw }
		\if@angle 
			\Sine{\@p@sangle}\Cosine{\@p@sangle}
	        	{\dimen100=\maxdimen\xdef\r@p@sbbllx{\number\dimen100}
					    \xdef\r@p@sbblly{\number\dimen100}
			                    \xdef\r@p@sbburx{-\number\dimen100}
					    \xdef\r@p@sbbury{-\number\dimen100}}
%
% Need to rotate all four points and take the X-Y extremes of the new
% points as the new bounding box.
                        \def\minmaxtest{
			   \ifnum\number\p@intvaluex<\r@p@sbbllx
			      \xdef\r@p@sbbllx{\number\p@intvaluex}\fi
			   \ifnum\number\p@intvaluex>\r@p@sbburx
			      \xdef\r@p@sbburx{\number\p@intvaluex}\fi
			   \ifnum\number\p@intvaluey<\r@p@sbblly
			      \xdef\r@p@sbblly{\number\p@intvaluey}\fi
			   \ifnum\number\p@intvaluey>\r@p@sbbury
			      \xdef\r@p@sbbury{\number\p@intvaluey}\fi
			   }
%			lower left
			\rotate@{\@p@sbbllx}{\@p@sbblly}
			\minmaxtest
%			upper left
			\rotate@{\@p@sbbllx}{\@p@sbbury}
			\minmaxtest
%			lower right
			\rotate@{\@p@sbburx}{\@p@sbblly}
			\minmaxtest
%			upper right
			\rotate@{\@p@sbburx}{\@p@sbbury}
			\minmaxtest
			\edef\@p@sbbllx{\r@p@sbbllx}\edef\@p@sbblly{\r@p@sbblly}
			\edef\@p@sbburx{\r@p@sbburx}\edef\@p@sbbury{\r@p@sbbury}
%\ps@typeout{rotated BB: \r@p@sbbllx, \r@p@sbblly, \r@p@sbburx, \r@p@sbbury}
		\fi
		\count203=\@p@sbburx
		\count204=\@p@sbbury
		\advance\count203 by -\@p@sbbllx
		\advance\count204 by -\@p@sbblly
		\edef\@bbw{\number\count203}
		\edef\@bbh{\number\count204}
		%\ps@typeout{ bbh = \@bbh, bbw = \@bbw }
}
%
% \in@hundreds performs #1 * (#2 / #3) correct to the hundreds,
%	then leaves the result in @result
%
\def\in@hundreds#1#2#3{\count240=#2 \count241=#3
		     \count100=\count240	% 100 is first digit #2/#3
		     \divide\count100 by \count241
		     \count101=\count100
		     \multiply\count101 by \count241
		     \advance\count240 by -\count101
		     \multiply\count240 by 10
		     \count101=\count240	%101 is second digit of #2/#3
		     \divide\count101 by \count241
		     \count102=\count101
		     \multiply\count102 by \count241
		     \advance\count240 by -\count102
		     \multiply\count240 by 10
		     \count102=\count240	% 102 is the third digit
		     \divide\count102 by \count241
		     \count200=#1\count205=0
		     \count201=\count200
			\multiply\count201 by \count100
		 	\advance\count205 by \count201
		     \count201=\count200
			\divide\count201 by 10
			\multiply\count201 by \count101
			\advance\count205 by \count201
			%
		     \count201=\count200
			\divide\count201 by 100
			\multiply\count201 by \count102
			\advance\count205 by \count201
			%
		     \edef\@result{\number\count205}
}
\def\compute@wfromh{
		% computing : width = height * (bbw / bbh)
		\in@hundreds{\@p@sheight}{\@bbw}{\@bbh}
		%\ps@typeout{ \@p@sheight * \@bbw / \@bbh, = \@result }
		\edef\@p@swidth{\@result}
		%\ps@typeout{w from h: width is \@p@swidth}
}
\def\compute@hfromw{
		% computing : height = width * (bbh / bbw)
	        \in@hundreds{\@p@swidth}{\@bbh}{\@bbw}
		%\ps@typeout{ \@p@swidth * \@bbh / \@bbw = \@result }
		\edef\@p@sheight{\@result}
		%\ps@typeout{h from w : height is \@p@sheight}
}
\def\compute@handw{
		\if@height 
			\if@width
			\else
				\compute@wfromh
			\fi
		\else 
			\if@width
				\compute@hfromw
			\else
				\edef\@p@sheight{\@bbh}
				\edef\@p@swidth{\@bbw}
			\fi
		\fi
}
\def\compute@resv{
		\if@rheight \else \edef\@p@srheight{\@p@sheight} \fi
		\if@rwidth \else \edef\@p@srwidth{\@p@swidth} \fi
		%\ps@typeout{rheight = \@p@srheight, rwidth = \@p@srwidth}
}
%		
% Compute any missing values
\def\compute@sizes{
	\compute@bb
	\if@scalefirst\if@angle
% at this point the bounding box has been adjsuted correctly for
% rotation.  PSFIG does all of its scaling using \@bbh and \@bbw.  If
% a width= or height= was specified along with \psscalefirst, then the
% width=/height= value needs to be adjusted to match the new (rotated)
% bounding box size (specifed in \@bbw and \@bbh).
%    \ps@bbw       width=
%    -------  =  ---------- 
%    \@bbw       new width=
% so `new width=' = (width= * \@bbw) / \ps@bbw; where \ps@bbw is the
% width of the original (unrotated) bounding box.
	\if@width
	   \in@hundreds{\@p@swidth}{\@bbw}{\ps@bbw}
	   \edef\@p@swidth{\@result}
	\fi
	\if@height
	   \in@hundreds{\@p@sheight}{\@bbh}{\ps@bbh}
	   \edef\@p@sheight{\@result}
	\fi
	\fi\fi
	\compute@handw
	\compute@resv}

%
% \psfig
% usage : \psfig{file=, height=, width=, bbllx=, bblly=, bburx=, bbury=,
%			rheight=, rwidth=, clip=}
%
% "clip=" is a switch and takes no value, but the `=' must be present.
\def\psfig#1{\vbox {
	% do a zero width hard space so that a single
	% \psfig in a centering enviornment will behave nicely
	%{\setbox0=\hbox{\ }\ \hskip-\wd0}
	%
	\ps@init@parms
	\parse@ps@parms{#1}
	\compute@sizes
	%
	\ifnum\@p@scost<\@psdraft{
		%
		\special{ps::[begin] 	\@p@swidth \space \@p@sheight \space
				\@p@sbbllx \space \@p@sbblly \space
				\@p@sbburx \space \@p@sbbury \space
				startTexFig \space }
		\if@angle
			\special {ps:: \@p@sangle \space rotate \space} 
		\fi
		\if@clip{
			\if@verbose{
				\ps@typeout{(clip)}
			}\fi
			\special{ps:: doclip \space }
		}\fi
		\if@prologfile
		    \special{ps: plotfile \@prologfileval \space } \fi
		\if@decmpr{
			\if@verbose{
				\ps@typeout{psfig: including \@p@sfile.Z \space }
			}\fi
			\special{ps: plotfile "`zcat \@p@sfile.Z" \space }
		}\else{
			\if@verbose{
				\ps@typeout{psfig: including \@p@sfile \space }
			}\fi
			\special{ps: plotfile \@p@sfile \space }
		}\fi
		\if@postlogfile
		    \special{ps: plotfile \@postlogfileval \space } \fi
		\special{ps::[end] endTexFig \space }
		% Create the vbox to reserve the space for the figure.
		\vbox to \@p@srheight sp{
		% 1/92 TJD Changed from "true sp" to "sp" for magnification.
			\hbox to \@p@srwidth sp{
				\hss
			}
		\vss
		}
	}\else{
		% draft figure, just reserve the space and print the
		% path name.
		\if@draftbox{		
			% Verbose draft: print file name in box
			\hbox{\frame{\vbox to \@p@srheight sp{
			\vss
			\hbox to \@p@srwidth sp{ \hss \@p@sfile \hss }
			\vss
			}}}
		}\else{
			% Non-verbose draft
			\vbox to \@p@srheight sp{
			\vss
			\hbox to \@p@srwidth sp{\hss}
			\vss
			}
		}\fi	



	}\fi
}}
\psfigRestoreAt
\let\@=\LaTeXAtSign




%%% \special{header=/group/csdl/tex/psfig/lprep71.pro}
%%% \begin{document}
\setcounter{chapter}{0}
\chapter{Overview}
\label{sec:overview}

This dissertation presents a new approach to collaborative learning that is
based on the thematic structure of scientific text\footnote{Throughout this
dissertation, the term {\it scientific text\/} is used interchangeably with
{\it scientific artifacts\/}, {\it research literature\/}, and {\it
learning artifacts\/}; they refer to the written record of human
knowledge. Examples of scientific text include discussion papers, journal
articles, conference papers, monographs. Literary text, such as poetry,
short stories, and novels, is excluded unless explicitly noted otherwise.}
and the theory of learning as collaborative knowledge-building. It also
describes a software system called CLARE that embodies such a conceptual
approach. Furthermore, it discusses the experience from sixteen usage
sessions of CLARE by six groups of students from two different
university-level classes. This usage indicates that CLARE is a useful
environment to support meaningful learning in collaborative settings.

In general, this research addresses the question, ``what can the computer
do for a group of learners beyond helping them share information?''  This
question can be rephrased more specifically as, ``how can the higher-level
knowledge --- the knowledge about (1) deep structures of individual
scientific artifacts, (2) inter-relationships between artifacts, and (3)
processes of knowledge-construction --- be used to facilitate meaningful
learning in group settings?'' Alternatively, ``what kinds of structural,
process-level, and computational support are required to help learners
reconstruct and evaluate the thematic features of scientific artifacts,
compare different interpretations of those features, deliberate reasonings
behind those interpretations, and integrate different points of view on the
artifact?''

CLARE responds to the above questions with the following features:

\begin{itemize}
\item A multi-user, distributed environment for supporting artifact-based
  knowledge-building;
  
\item A two-phase, five-step process model for helping learners organize
  their collaborative learning activities;
  
\item A thematically-oriented representation language that serves as the
  principal integrative mechanism between learners and various types of
  learning activities;
  
\item A shared, evolving knowledge base that captures not only the final
  product but also the process that leads to that product; and 
  
\item An unobtrusive instrumentation mechanism that collects fine-grained
  data about the process of collaboration and learning.
\end{itemize}

CLARE is built upon a representation language called RESRA, which serves
the following roles in collaborative knowledge-construction:

\begin{itemize}
\item A meta-cognitive tool that highlights essential thematic features and
  the relationships between these features;
  
\item An organizational tool that allows the learner to dynamically and
  incrementally integrate at a fine-grained level various types of
  research artifacts;
  
\item A shared frame of reference that facilitates communication and
  argumentation among learners; and
  
\item A tool for studying the norms governing the written communication of
  scientific knowledge.
\end{itemize}

This chapter provides an overview of the dissertation. It begins with a
hypothetical usage scenario that illustrates the basic process and main
features of collaborative learning under the CLARE environment. The
subsequent section describes the main motivations behind the current
work. Next, it introduces the main thesis, which is followed by the major
research contributions made by this work. The scope and limitations are
highlighted next. The chapter concludes with an overview of the
organization for the reminder of this dissertation.


\section{Collaborative learning using CLARE}
\label{sec:example}

\subsection{A usage scenario}

Scott, Chris, Mary, and Todd\footnote{The names of these four users are
fictitious. However, this usage scenario itself is only {\it somewhat\/}
hypothetical: it is based on the result from a manual experiment on RESRA
--- the CLARE's representation language --- during the early phase of this
research. The experiment was not done under CLARE, because the system was
not yet completed at that time.} are in a research seminar on
computer-supported cooperative work (CSCW). Unlike traditional seminars in
which interactions among participants take place largely in a face-to-face
setting, this seminar is featured with a structured, distributed online
environment called CLARE. Rather than receiving a stack of printed papers
to read, students are asked to analyze and discuss selected research
literature online. The example paper used in the current scenario is
``Supporting collaborative software development with ConversationBuilder''
by Kaplan, et al \cite{Kaplan92}. The full-text of the paper is available
online in CLARE in a hypertext format: each node corresponds to a {\it
semantic unit\/} or a section of the paper. Individual nodes are connected
via links derived directly from the structure of the artifact. Hence,
students may navigate the entire document by simply following these links.

The study session is organized into two phases: {\it exploration\/} and
{\it consolidation\/.} During the exploration phase, individual students
summarize in terms of predefined representation primitives, such as
\fbox{{\sf problem}} and \fbox{{\sf concept}}, what they view as important
features of the paper. They also evaluate those features and the
relationships between them by making critiques, raising questions, and
offering suggestions. During the consolidation phase, they compare the
result of their individual summarizations and evaluations to discern
ambiguities, differences, and similarities. When ambiguities and/or
differences occur, they may question, critique, propose, or clarify by
providing new information. In response to other learners' questions,
critiques, or suggestions, one may defend, deliberate, or amend one's own
positions. Toward the end, learners go back to the online artifacts they
have created and connect together similar and related positions and
explanations to produce a more coherent body of knowledge.

\begin{figure}[htb]
  \centerline{\psfig{figure=Figures/kaplan1.eps,width=5.5in}}
  \caption{An example of Scott's view during the exploration phase}
  \label{fig:summarize}
\end{figure}

\begin{figure}[htb]
   \fbox{\centerline{\psfig{figure=Figures/cb1.eps,height=3.5in}}}
  \caption{Scott's representation of [KTBB92] (condensed)}
  \label{fig:cb1}
\end{figure}

\begin{figure}[htb]
  \fbox{\centerline{\psfig{figure=Figures/cb2.eps,height=3.5in}}}
  \caption{Chris' representation of [KTBB92] (condensed)}
  \label{fig:cb2}
\end{figure}

\begin{figure}[htb]
  \centerline{\psfig{figure=Figures/kaplan2.eps,width=5.5in}}
  \caption{A comparative view of \fbox{{\sf problem}} instances by Scott
  Chris, Mary, and Todd}
  \label{fig:compare}
\end{figure}

When Scott logged on to CLARE for the first time, he navigated through the
entire document by following the built-in links to get a general sense of
what the paper is about. He then zoomed-in to individual nodes and examined
more closely their contents. When he encountered a paragraph or a text
region which he thought was important, he highlighted it by dragging the
mouse over it, and selected {\sf Create node\/} from the popup menu. CLARE
shows a submenu which lists all predefined node types, for example,
\fbox{{\sf concept\/}}, \fbox{{\sf claim\/}}. Selecting any of these
entries leads to the creation of a node of the corresponding type.  For
example, the left window in Figure \ref{fig:summarize} shows a highlighted
region as describing the \fbox{{\sf problem}} Scott considered that the
author attempts to address in that paper.  Hence, he created a \fbox{{\sf
problem\/}} node for this region, which is shown in the lower right
window. The {\it italicized\/} text under the field {\bf
Summarizations}\footnote{{\bf Summarizations} here is a field label which means
that all links underneath it are generated during summarization.  The other
three link fields, namely, {\bf Evaluations}, {\bf Deliberations\/}, and
{\bf Integrations\/} have the similar semantics.} is the link to the region
from which the node is derived. The main content field, namely, {\bf
Description}, contains Scott's description of what he thinks the authors
attempt to convey. Note that the content of this field is not a verbatim
copy of the original authors' statements. Rather, it articulates Scott's
understanding of the selected text region and, to some extent, of the paper
as a whole.

By following the same procedure, Scott created a \fbox{{\sf claim\/}} node
based on a different part of the paper, which states that
ConversationBuilder offers a viable solution to the problem of active and
flexible support. Because there seems to exist a direct relationship
between the two nodes, Scott wanted to connect them together. Hence, he
selected {\sf Enter LINK Mode\/} from the {\sf Summarize\/} menu, which
creates two windows in the lower half of the screen. The two windows show
the two nodes he has created thus far.  Scott moves the mouse to the
\fbox{{\sf claim\/}} window, select {\it Responds-to PROBLEM} from the
context-sensitive popup menu, followed by pressing the mouse on the
\fbox{{\sf problem\/}} window: a bidirectional link of the type {\it
responds-to\/} was added between the two nodes. This link indicates that
the claim is made with respect to to the \fbox{{\sf problem}} or,
alternatively, the \fbox{{\sf problem}} is responded by that \fbox{{\sf
claim}}. Following this action, Scott exited from the link mode, and went
on to create more nodes.

From reading the user guide \cite{csdl-93-15}, Scott learned that CLARE
possesses some knowledge about the structure of the current paper. Hence,
he wanted to know whether CLARE could provide him reasonable advice about
what to look for next. He selected {\sf Consult for what's missing} from
the option {\sf Template Guide\/} of the {\sf Summarize\/} menu. CLARE pops
up a new window, which shows the following information:

\begin{itemize}
\item The current paper is a {\it concept paper\/}\footnote{CLARE
  characterizes the structural features of scientific text based on the
  artifact types.  The {\it concept paper\/} is one of five such
  predefined artifact types. Other types include {\it experience
  papers\/}, {\it empirical papers\/}, {\it essays\/}, and {\it survey
  papers\/}. See Section \ref{sec:crf} for a detailed description of these
  artifact types.};
  
\item Scott has thus far created one tuple (a pair of nodes connected by
  a link of a predefined type), namely, \fbox{{\sf claim}} \( \stackrel{
  responds-to}{\longrightarrow} \) \fbox{{\sf problem}}, and no orphan
  nodes, i.e., nodes which are not connected to other nodes; and
  
\item Based on the structural knowledge the system knows about the
  current artifact, CLARE suggests Scott to consider creating tuple of
  the following types:

  \begin{itemize}
  \item \( \fbox{{\sf evidence}}
    \stackrel{supports}{\longrightarrow} \fbox{{\sf claim}} \)

  \item \fbox{{\sf claim}} \( \stackrel{defines}{\longrightarrow} \)
    \fbox{{\sf concept}}
    
  \item \fbox{{\sf claim}} \( \stackrel{defines}{\longrightarrow} \)
    \fbox{{\sf method}}
  \end{itemize}
\end{itemize}

Scott took to heart the above suggestions from CLARE; he went back to the
source nodes one more time. Unlike previous passes in which he read to
understand detailed contents, this time John was looking for specific
themes, such as \fbox{{\sf evidence}}, \fbox{{\sf concept}}, and \fbox{{\sf
method}}. For example, he did discover an \fbox{{\sf evidence}} from
another section of the paper, namely, the example use of
ConversationBuilder, which {\it supports\/} the \fbox{{\sf claim}} he
identified earlier. He also found a few important \fbox{{\sf concept}}
instances, such as ``action space,'' ``protocol.''  At the same time, he
evaluated the paper content by creating a number of \fbox{{\sf critique}}
nodes. However, he was unable to find any \fbox{{\sf method}} instance.
Figure \ref{fig:cb1} shows an abbreviated, graphical depiction of Scott's
representation of \cite{Kaplan92}\footnote{CLARE currently does not have a
graphical browser. The graphical depiction here is paraphrased,
abbreviated, and hand-drawn from the actual data.}.

While Scott was busy with his exploration of the paper, Chris, Mary, and
Todd were also in and out of CLARE engaging in similar types of activities.
During this phase, however, they could not see what each other were doing,
that is, what text regions were highlighted, or what nodes and links were
created. Rather, all four of them were independently deriving their own
representations of the paper and their views on it. Figure \ref{fig:cb2},
for example, shows an abbreviated depiction of Chris' representation of the
same paper. Note that Figure \ref{fig:cb1} and \ref{fig:cb2} are quite
different not only in term of the type of nodes and links, the origins of
these nodes but, perhaps most important of all, the contents of those
nodes. These differences and their implications will become evident in the
consolidation phase.

The consolidation phase was activated two days after the session began.
This gave all four learners adequate time to complete the exploration
phase. The primary goals of consolidation are:

\begin{itemize}
\item To expose the differences and ambiguities of individual learners'
  interpretations and evaluations of the selected artifact;
  
\item To deliberate, resolve, and augment these differences and
  ambiguities; and
  
\item To link together similar and related views held by different
  learners.
\end{itemize}

Figure \ref{fig:compare} shows a typical user view of CLARE during the
consolidation phase. The left window presents primitive-based comparison of
node instances created by Scott, Chris, Mary, and Todd. In the current
example, the window shows the comparison of \fbox{{\sf problem}} instances
--- four independent interpretations of what the original authors attempt
to address in the paper. By displaying them next to each other, it is
relatively easy to discern similarities, differences, and ambiguities. For
example, although each learner has a \fbox{{\sf problem}} node, their
content is quite distinct: Scott's and Mary's characterizations are very
similar; both explicitly mention the conflicting requirements between the
situated nature of collaborative work that calls for flexible support, and
the demand for active support. Todd's statement, however, calls solely for
flexible support. The latter is not intended by the original authors, for
they explicitly state that providing {\it either\/} flexible {\it or\/}
active support is not difficult; the difficulty only arises when both have
to be satisfied. Chris' \fbox{{\sf problem}} introduces the issue of
support for change, which is not touched on in the paper.

While CLARE's comparison mode helps uncover differences and ambiguities,
its argumentation feature supports deliberation and resolution of those
differences and ambiguities. In the above example, for instance, although
Chris' view does not reflect that of the original authors, he did introduce
some new themes to the scene, most noticeably, ``EGRET'' and ``process
maturation.'' Since both Chris and Scott were involved in the design of
EGRET system and interested in the process maturation, they could elaborate
these themes by creating a separate \fbox{{\sf thing}} node for EGRET, and
a \fbox{{\sf concept}} for process maturation. Similarly, Scott might
request Todd to explain why he thought that the sole support for
flexibility is a \fbox{{\sf problem}} by creating a \fbox{{\sf question}}
node.

Finally, similar and related nodes can be integrated by selecting
appropriate options from the {\sf Integrate\/} menu. For example, since
Scott's and Mary's \fbox{{\sf problem}} nodes are similar, a {\it
is-similar-to\/} link could be added between those two nodes by selecting
the option {\sf Declare two nodes as SIMILAR} from the {\sf Integrate}
menu. In addition, Mary believed that Scott's representation was more
articulated than hers, even though they both captured the same
information. Hence, she decided to endorse Scott's view by selecting {\sf
Endorse current node} from the {\sf Integrate} menu.

One main result of the above comparative, argumentative, and integrative
activities is a deeper understanding of the content of the selected paper
by Chris, Scott, Mary, and Todd. In addition, this process also leads to a
knowledge base that captures various interpretations and evaluations of the
paper and interactions by these four learners.


\subsection{Discussion}

The above example illustrates a typical usage scenario of CLARE: it
highlights a collaborative learning process that is guided by a
well-defined representation language and a process protocol, and supported
by a computer-based environment. Unlike traditional learning which takes
place in laboratories or classrooms, the current process is supported by a
virtual environment. This implies that Mary and Scott, for example, might
be geographically and temporally distributed, but can still compare their
interpretations of the selected paper and discuss their differences and
similarities using CLARE. Perhaps more importantly, the example scenario
represents a new type of learning called {\it collaborative learning from
scientific text}. Figure \ref{fig:learning-community} shows major
components of this learning and they are related together to support
collaborative knowledge construction. The boxes that are connected to the
outmost circles indicate that learning in this context begins with
scientific text (e.g., \cite{Kaplan92}), as opposed to scientific
experiments or lectures. The process consists four steps: {\it
summarization \& evaluation\/}, {\it comparison\/}, {\it argumentation\/},
and {\it integration\/}. The direction of the large shaded arrows indicates
that, as the process moves toward the center, the amount of interactions
among the group members increases and, concurrently, the group knowledge
converges. The ultimate result is a dynamic group knowledge base, which
integrates various interpretations, evaluations, deliberations, and
extensions of the subject content of the selected artifacts by a group of
learners.

%%% Each small circle in the figure represents a {\it summarative\/} node, such
%%% as \fbox{{\sf claim}} or \fbox{{\sf concept}}; each small triangle
%%% represents an {\it evaluative\/} node, e.g., \fbox{{\sf critique}},
%%% \fbox{{\sf suggestion}}. The {\it argumentative\/} nodes, which are created
%%% during the {\it argumentation\/} layer, are represented by small
%%% rectangles.  Integration, the innermost layer or the final step, does not
%%% require the creation of new node instances; it simply {\it elevates\/}
%%% existing node instances into that layer through endorsing.

\begin{figure}[htb]
  \fbox{{\centerline{\psfig{figure=Figures/learning-community.eps,height=3.5in}}}}
  \caption{Collaborative knowledge-building in CLARE}
  \label{fig:learning-community}
\end{figure}

The outermost layer represents {\it summarization \& evaluation\/},
corresponding to the CLARE's {\it exploration\/} phase. Its primary purpose
is bootstrapping --- to reconstruct the thematic structure of the knowledge
embedded in the selected artifact. Summarization is very similar to reverse
engineering in software development, which attempts to recover the design
information embedded in the software source code.  Evaluation, on the other
hand, brings the learners' perspectives to bear with the learning context
by allowing them to explicitly state their views on the content of the
artifact. As shown in Figure \ref{fig:learning-community} by the isolated
clusters of nodes around each learner, the exploration phase is private;
each learner independently derives his own representation and
assessment. As a result, a learner cannot either be swayed by, nor
free-ride off the work of others. The result from this step --- the
summarative and evaluative representations by individual learners --- forms a
basis for the second phase.

Because of the difference in backgrounds, interests, and intellectual
perspectives of the learners involved, the representations from the
previous step are likely to be different, as evidenced from the example.
The {\it comparison\/} mode provides a convenient means of uncovering
differences in the interpretation and evaluation, and ambiguities in the
presentation.  This mode forms the baseline for subsequent two steps --- {\it
argumentation } and {\it integration\/} --- in which learners deliberate,
extend, and integrate their interpretative and evaluative knowledge.

Collaborative knowledge-building in CLARE bears many similarities with
knowledge-building in the scientific community (see Section
\ref{sec:research artifacts}). On this ground, CLARE deviates from many
other existing learning systems in which collaborative learning is largely
manifested as information sharing (see Section \ref{sec:cscl-systems} for a
review of such systems). The next section explains the main motivation
behind the current approach.


\section{Motivation}
\label{sec:motivation}

This research is motivated by two main trends: one is technological and the
other is theoretical. The former is the predominant emphasis on {\it
access} by existing collaborative learning systems. The latter is the
increasing recognition of the importance of meta-knowledge in human
learning. While the technological force propels the need for computational
support beyond information sharing, the theoretical development forms a
conceptual underpinning for the current approach.


\subsection{Technological biases in current learning support systems}

The two most widely used types of collaborative learning environments are
{\it virtual classrooms\/} and {\it hypermedia systems\/}. The former
encompasses a wide range of computer-mediated communication technologies,
including electronic mail, computer conferencing, and bulletin-board
systems. The latter promises to deliver integrated learning environments
that link together a wide range of applications and distributed data, such
as text, graphics, video, etc. Despite their seemingly differences, virtual
classroom and hypermedia systems share essentially the same focus, namely,
support for information sharing. They both aim at overcoming the
geographical, temporal, and media constraints of traditional face-to-face
interactions and printed media by allowing the learner access to the {\it
right information\/} in the {\it right media \/} or {\it presentational
format\/}, or access to the {\it right people\/} at the {\it right time\/}.
For example, computer-mediated communication systems, such as EIES
\cite{Hiltz88}, have been successfully used to overcome the {\it same-time,
same-place\/} constraints of traditional classrooms, and to increase
student participation outside physical classrooms.  Similarly, hypermedia
systems, such as Intermedia \cite{Yankelovich88} and NoteCards
\cite{Halasz87Notecards}, have been found effective for browsing and
navigating large shared information space.  However, both virtual classroom
and hypermedia systems suffer from some major problems: {\it information
overload\/} in virtual classrooms and {\it lost-in-the-hyperspace\/} in
hypermedia environments. At a deeper level, these problems are rooted in
the same cause: the lack of explicit, fine-grained characterization and
representation of the thematic structures of learning artifacts. In virtual
classrooms, for example, online discussions typically take place within
various {\it interest groups.\/} Such division are generally
coarse-grained. The relationships between these groups are also {\it
implied\/} rather than {\it explicitly\/} specified. In hypermedia systems,
the power of linking and the emphasis on non-linearity often lead to the
excessive use of such features. The net result, similar to the over-use of
the {\it goto\/} statement in computer programs, is a network of nodes with
spaghetti-like structures whose semantics are difficult, if not impossible,
to understand. As a result, the potential for {\it deep-level\/}
collaboration among a group of learners is severely limited.

At a more fundamental level, the above problems with virtual classroom and
hypermedia systems are the manifestation of a techno-centric approach to
learning support embedded in these systems. They indicate that insufficient
attention has been paid to the underlying theories of human learning. The
next section presents one of such theories --- {\it assimilation theory of
cognitive learning\/} --- and its technological implications.


\subsection{Cognitive learning theory and concept mapping}

While technologists continue to improve the functionality and the interface
of software and hardware tools, theorists are breaking new grounds in
understanding human learning. One of the major developments in educational
psychology is the theory of {\it meaningful learning,\/} also known as the
{\it assimilation theory of cognitive learning\/}. The main premise of this
theory is that the most important factor influencing human learning is the
learner's prior knowledge; that human learning is evidenced by a change in
the meaning of experience (as opposed to a change in behavior); and that
the key role of the educator is to help students reflect on their
experience and to continuously construct new meanings
\cite{Ausubel63,Ausubel78}. To facilitate this process, Novak and Gowin
\cite{Novak84} --- two of the main proponents of this theory --- have
developed two meta-cognitive strategies: {\it concept maps\/} and {\it Vee
diagrams\/}. Both are the tools for representing changes in the knowledge
structure of students over time, and for helping them {\it learn how to
learn\/} (see Section \ref{sec:kr-schemes} for details on concept maps and
Vee diagrams).

Concept mapping represents the first true attempt to provide explicit
support for meta-learning \cite{Novak84}. It has enjoyed a wide acceptance
in the educational community. Numerous studies have shown its effectiveness
in facilitating student learning in science
\cite{Cliburn90,Novak90,Roth92}. Despite the strong empirical evidence
supporting its usefulness, however, concept mapping as a collaborative
learning tool suffers from three main problems:

\begin{itemize}
\item {\it Non-hierarchical structures:\/} In concept maps, all knowledge
  features must ultimately be reduced to concepts and links between them.
  This, though achievable for introductory, well-understood knowledge, is
  inadequate in advanced learning context, which often requires analyses
  and syntheses to be done using higher-level constructs, e.g., {\it
  claim\/}, {\it problem\/}. The lack of abstraction capabilities
  severely limits the usefulness of concept mapping for advanced learners
  such as graduate students.
  
\item {\it Free form of expression:\/} Concept maps, like the designer's
  sketch pad, give the learner full freedom in deciding what to draw and
  how to draw it. The representation does not dictate nor provide any
  structural heuristics on how it should be used. While this flexibility
  makes concept maps extremely expressive, it also adds little structure
  on which useful manipulations can be applied, and which human learners
  may use to help them decipher the map. The latter is especially
  significant in collaborative settings, for this arbitrariness implies
  that it is difficult to compare, contrast, and integrate concept maps
  generated by different learners.
  
\item {\it Individual learning tool:\/} Concept mapping has hitherto
  primarily been used in facilitating individual learning. Few existing
  systems support collaborative construction of concept maps. In their
  study of concept mapping in a group setting, for example, \cite{Roth92}
  have to rely on movable paper clips instead of a computer-supported
  environment.
\end{itemize}

The above problems with concept mapping indicate the need for extending the
current strategy and for designing alternative strategies to support human
learning. Moreover, the general lack of technological support for concept
mapping has also prevented the realization of full potentials of such
approaches. The current research is intended to address both of these
problems.


\subsection{Toward a theory-based collaborative learning support environment}

This dissertation attempts to bridge the gap between the recent development
in the theories of human learning and the current state of technological
support for such activities. It does so by adopting the assimilation theory
of cognitive learning as its conceptual basis, and by providing a new type
of computer-based learning environment that focuses on collaborative
learning as knowledge-building. It addresses the above problems with
concept-mapping --- the theorists' solution to the problem of facilitating
meaningful learning --- by proposing a new representation and a computational
environment that supports the use and manipulation of this representation.
The next section outlines the main thesis underlying this research.


\section{Research thesis}
\label{sec:thesis}

The basic premise of the current research is that collaborative learning is
not simply sharing of information among learners but rather collaborative
knowledge construction similar to that taking place among researchers in
the scientific community. One important form of collaborative learning is
organized on scientific text, which attempts to bridge the gap between the
knowledge-building in the scientific community and the knowledge-building
in classrooms by systematically analyzing and discussing research
literature. Scientific text is an important source of the most current,
evolving knowledge. More importantly, It is one of the primary sources for
learning about scientific knowledge-building itself\footnote{Another way of
learning the scientific knowledge-building is through direct participation,
e.g., apprenticeship. The two are complementary rather than mutually
exclusive.}. The structure of scientific text often reveals patterns
underlying scientific discourse and the norms governing formal presentation
of research findings.

The central claim of this research is that CLARE provides a viable
computer-augmented environment for collaborative learning from scientific
text. First, CLARE does not follow the technology-driven paradigm. Rather,
it is grounded in the assimilation theory of cognitive learning --- a
well-established learning theory, and constructionism --- a widely adopted
pedagogy. Second, CLARE incorporates a unique learning model that
integrates such key activities as summarization, evaluation, comparison,
argumentation, and integration.  As illustrated in Figure
\ref{fig:learning-community}, these activities form the basic building
blocks of an artifact-based knowledge construction process. Third, CLARE is
built on a thematically-based representation language which (1) draws on
the basic principles of knowledge representation from AI, and (2) overcomes
the three drawbacks of concept mapping --- one of the main meta-cognitive
tools proposed as part of the cognitive learning theory.

One of the main characteristics of this current research is its emphasis on
the higher-level structure of knowledge or {\it meta-knowledge\/}, and the
use of such knowledge as the {\it glue\/} that links together:

\begin{itemize}
\item Knowledge-building in the scientific community and
  knowledge-building in classroom settings;
  
\item Exploration and consolidation phases in CLARE; and
  
\item Different interpretations and points of view from individual
  learners.
\end{itemize}

The proposed representation language exemplifies what such high-order
knowledge structure is; it specifies what content-level themes that the
learner should attend to but does not dictate how such themes are to be
found or used. It should be noted that the process of identifying such
high-order constructs is an important form of meta-learning which CLARE
aims to explicitly support.


\section{Research contributions}
\label{sec:contributions}

This research addresses an important and yet hitherto untapped area of
research --- the explicit representation and the use of high-order
knowledge as a means to facilitate collaborative learning. It addresses a
broader issue of knowledge representation in the context of human learning.
Despite that knowledge representation is at the central stage of AI
research, the requirements for supporting human learning, especially, human
collaborative learning, are quite different from that for AI systems. Along
this direction, the current research raises a number of significant
questions. For example, what is the exact role of the representation in
human collaborative learning?  What are characteristics of the
representation language appropriate for human learning? What types of
computational augmentation are required in such a context? Though this
research may not lead to definite answers to these and many other related
questions, it does represent the first step toward the ultimate
understanding of such important issues.

Conceptually, CLARE demonstrates a computer-supported environment that is
based on constructionist pedagogy and the assimilation theory of cognitive
learning. Despite that, pedagogically, the view of learning as
knowledge-building is well-established, the technological support thus far
has not yet extended beyond information sharing. This research highlights
the connections between knowledge-building in the scientific community and
knowledge-building in classroom settings by defining a new type of
learning, called {\it collaborative learning from scientific text\/}. This
learning is centered on the knowledge --- both content and meta-level --- 
embedded in scientific text. CLARE provides explicit process and
representation support for such learning.

Representationally, RESRA --- the CLARE's underlying representation
language --- applies the principles of knowledge representation to the
thematic features of scientific text. It overcomes the structural weakness
of concept maps by providing a small initial set of primitive types and
canonical forms, which also serve as useful basis for exploring
higher-order structure of human knowledge. The key contribution of RESRA is
its potentials in integrating different points of view held by individual
learners.

Technically, CLARE is both a research and a learning tool. As such, it
provides an extendible computational environment that balances usability
and the research-level support. The former is highlighted by the five-step
process model that helps structure various learning activities, and by the
comparison mode, which enables learners to make fine-grained comparisons of
their own representations with those of others. The latter is evidenced by
the built-in instrumentation mechanism that allows fine-grained process
data to be gathered unobtrusively.

Empirically, the data from sixteen usage sessions of CLARE by six different
groups of students from two separate classes confirmed that CLARE is a
viable approach to supporting collaborative learning from scientific text.
The result also suggests a number of interesting directions in which CLARE
can be extended and further empirical investigations can be conducted.


\section{Scope and limitations}
\label{sec:limitations}

Collaborative learning is a complex activity to study and support. This
research does not attempt to address all important aspects of the subject,
nor create an all-purpose system that provides merely a collection of neat
features. Instead, it focuses on the role of representation in
collaborative learning, and on providing services which augment the human
use and manipulation of this representation. Below are three major
limitations of the current research:

\begin{itemize}
\item CLARE helps ameliorate certain problems related to face-to-face
  collaboration (e.g., the {\it dominant personality\/} effect) to the
  extent that it is a computer-mediated environment. However, it does not
  attempt to overcome the interpersonal and inter-cultural conflicts
  inherent in typical group settings. When used in a face-to-face
  setting, it is possible that the effectiveness of CLARE as an augmented
  learning tool be overshadowed by some interpersonal or inter-cultural
  factors. CLARE provides no means to separate the two.
  
\item At the system level, CLARE does not have certain advanced
  functionalities found in other learning support environments. Examples
  include multi-media, version control, and fancy graphical interface. This
  is in part due to the pilot nature of the current implementation; new
  features will be added as the first-hand experience with the system
  increases and, consequently, the underlying process is better understood.
  
\item Despite the importance of allowing learners to define their own
  representation primitives, this capability is currently unavailable to
  the user.
\end{itemize}


\section{Organization of this dissertation}
\label{sec:organization}

The remainder of this dissertation is organized as follows. Chapter 2
describes the conceptual framework for the current research. First, it
redefines {\it knowledge representation\/} in the context of human
learning, and relates this concept to the theory of {\it meaningful
learning\/}. Next, it discusses the role of scientific text in the
knowledge-building process in both research and classroom settings. The
following two sections introduce the representation language (RESRA), and
the SECAI model of collaborative learning. The chapter concludes with a
discussion about the role of representation in the proposed framework.

Chapter 3 describes RESRA --- the proposed representation language based on
the thematic structure of scientific text. It begins by identifying four
major requirements for the representation. Then, it defines several key
concepts underlying RESRA. The following two sections elaborate the two
primary constructs of the representation: {\it primitive\/} and {\it
canonical forms (CRFs)}. Examples are given to illustrate their semantics
and usages. The final section of the chapter briefly discusses the
extendibility of RESRA.

Chapter 4 presents the design and implementation of CLARE. It begins with
the three types of requirements for the system: {\it conceptual\/}, {\it
data\/}, and {\it usability\/}. Next, it discusses two main design
considerations of CLARE: layered + object-oriented design and the decision
on services over interface. The system architecture is described next,
which is followed by the depiction of the interface features. A road-map of
the system functionality is also provided. The chapter concludes with a
brief history and status report.

Chapter 5 describes the experiments designed for evaluating CLARE. It first
revisits research problem. Then, it describes the ten hypotheses that guide
the evaluation experiments. Next, it identifies and relates the three types
of empirical data for each hypothesis: {\it outcome\/}, {\it process\/},
and {\it assessment}, the procedures for collecting such data, and methods
for analyzing them. Finally, it describes the two sets of experiments to be
conducted, including the task, subjects, procedures, and the execution
plan.

Chapter 6 presents findings on CLARE from the evaluation experiments. It
begins with an overview of the experiment and its findings. The actual
result presentation is organized into three parts. Section
\ref{sec:c6-hypothesis} describes the findings with respect to each of the
ten hypotheses identified in Chapter 5. Section \ref{sec:rep-issues}
discusses main issues that arose from the use of the RESRA representation,
including a list of common representation errors abstracted from the usage
data. Section \ref{sec:strategies} identifies several usage strategies
employed by the learners during summarization.  Section \ref{sec:case}
presents a detailed analysis of one group session using CLARE. The purpose
of this last section is to bring together all previous discussions with a
single case example, and to compare this example with the hypothetical
usage scenario described in Chapter 1. This chapter concludes with a
discussion on the major results from the CLARE evaluation.

Chapter 7 reviews prior work pertinent to the current research. It is
organized into four sections. Section \ref{sec:theory} reviews the
theoretical work on which CLARE is based.  More specifically, it covers
constructionism and the assimilation theory of cognitive learning. Section
\ref{sec:representation} describes schema theory and related knowledge
representation languages.  Section \ref{sec:cscl-systems} surveys major
existing collaborative learning systems and empirical findings on them.
This chapter concludes with a summary of the relationships between CLARE
and the work being reviewed.

Chapter 8 concludes this dissertation. It begins with a review of the basic
problem this research has attempted to address and the approach it has
adopted. Next, it describes the major contributions this research has made
in furthering the understanding of collaborative learning in
computer-augmented environments. The final section of the chapter discuss
several promising directions in which the current research might be
extended.

%%% \newpage
%%% \singlespace
%%% \bibliography{../bib/clare}
%%% \bibliographystyle{alpha}
%%% 
%%% 
%%% \end{document}
%%% 



