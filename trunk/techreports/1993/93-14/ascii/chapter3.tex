%%% \documentstyle [12pt,/group/csdl/tex/definemargins,
%%% /group/csdl/tex/lmacros]{report}
%%% % Psfig/TeX 
\def\PsfigVersion{1.9}
% dvips version
%
% All psfig/tex software, documentation, and related files
% in this distribution of psfig/tex are 
% Copyright 1987, 1988, 1991 Trevor J. Darrell
%
% Permission is granted for use and non-profit distribution of psfig/tex 
% providing that this notice is clearly maintained. The right to
% distribute any portion of psfig/tex for profit or as part of any commercial
% product is specifically reserved for the author(s) of that portion.
%
% *** Feel free to make local modifications of psfig as you wish,
% *** but DO NOT post any changed or modified versions of ``psfig''
% *** directly to the net. Send them to me and I'll try to incorporate
% *** them into future versions. If you want to take the psfig code 
% *** and make a new program (subject to the copyright above), distribute it, 
% *** (and maintain it) that's fine, just don't call it psfig.
%
% Bugs and improvements to trevor@media.mit.edu.
%
% Thanks to Greg Hager (GDH) and Ned Batchelder for their contributions
% to the original version of this project.
%
% Modified by J. Daniel Smith on 9 October 1990 to accept the
% %%BoundingBox: comment with or without a space after the colon.  Stole
% file reading code from Tom Rokicki's EPSF.TEX file (see below).
%
% More modifications by J. Daniel Smith on 29 March 1991 to allow the
% the included PostScript figure to be rotated.  The amount of
% rotation is specified by the "angle=" parameter of the \psfig command.
%
% Modified by Robert Russell on June 25, 1991 to allow users to specify
% .ps filenames which don't yet exist, provided they explicitly provide
% boundingbox information via the \psfig command. Note: This will only work
% if the "file=" parameter follows all four "bb???=" parameters in the
% command. This is due to the order in which psfig interprets these params.
%
%  3 Jul 1991	JDS	check if file already read in once
%  4 Sep 1991	JDS	fixed incorrect computation of rotated
%			bounding box
% 25 Sep 1991	GVR	expanded synopsis of \psfig
% 14 Oct 1991	JDS	\fbox code from LaTeX so \psdraft works with TeX
%			changed \typeout to \ps@typeout
% 17 Oct 1991	JDS	added \psscalefirst and \psrotatefirst
%

% From: gvr@cs.brown.edu (George V. Reilly)
%
% \psdraft	draws an outline box, but doesn't include the figure
%		in the DVI file.  Useful for previewing.
%
% \psfull	includes the figure in the DVI file (default).
%
% \psscalefirst width= or height= specifies the size of the figure
% 		before rotation.
% \psrotatefirst (default) width= or height= specifies the size of the
% 		 figure after rotation.  Asymetric figures will
% 		 appear to shrink.
%
% \psfigurepath#1	sets the path to search for the figure
%
% \psfig
% usage: \psfig{file=, figure=, height=, width=,
%			bbllx=, bblly=, bburx=, bbury=,
%			rheight=, rwidth=, clip=, angle=, silent=}
%
%	"file" is the filename.  If no path name is specified and the
%		file is not found in the current directory,
%		it will be looked for in directory \psfigurepath.
%	"figure" is a synonym for "file".
%	By default, the width and height of the figure are taken from
%		the BoundingBox of the figure.
%	If "width" is specified, the figure is scaled so that it has
%		the specified width.  Its height changes proportionately.
%	If "height" is specified, the figure is scaled so that it has
%		the specified height.  Its width changes proportionately.
%	If both "width" and "height" are specified, the figure is scaled
%		anamorphically.
%	"bbllx", "bblly", "bburx", and "bbury" control the PostScript
%		BoundingBox.  If these four values are specified
%               *before* the "file" option, the PSFIG will not try to
%               open the PostScript file.
%	"rheight" and "rwidth" are the reserved height and width
%		of the figure, i.e., how big TeX actually thinks
%		the figure is.  They default to "width" and "height".
%	The "clip" option ensures that no portion of the figure will
%		appear outside its BoundingBox.  "clip=" is a switch and
%		takes no value, but the `=' must be present.
%	The "angle" option specifies the angle of rotation (degrees, ccw).
%	The "silent" option makes \psfig work silently.
%

% check to see if macros already loaded in (maybe some other file says
% "\input psfig") ...
\ifx\undefined\psfig\else\endinput\fi

%
% from a suggestion by eijkhout@csrd.uiuc.edu to allow
% loading as a style file. Changed to avoid problems
% with amstex per suggestion by jbence@math.ucla.edu

\let\LaTeXAtSign=\@
\let\@=\relax
\edef\psfigRestoreAt{\catcode`\@=\number\catcode`@\relax}
%\edef\psfigRestoreAt{\catcode`@=\number\catcode`@\relax}
\catcode`\@=11\relax
\newwrite\@unused
\def\ps@typeout#1{{\let\protect\string\immediate\write\@unused{#1}}}
\ps@typeout{psfig/tex \PsfigVersion}

%% Here's how you define your figure path.  Should be set up with null
%% default and a user useable definition.

\def\figurepath{./}
\def\psfigurepath#1{\edef\figurepath{#1}}

%
% @psdo control structure -- similar to Latex @for.
% I redefined these with different names so that psfig can
% be used with TeX as well as LaTeX, and so that it will not 
% be vunerable to future changes in LaTeX's internal
% control structure,
%
\def\@nnil{\@nil}
\def\@empty{}
\def\@psdonoop#1\@@#2#3{}
\def\@psdo#1:=#2\do#3{\edef\@psdotmp{#2}\ifx\@psdotmp\@empty \else
    \expandafter\@psdoloop#2,\@nil,\@nil\@@#1{#3}\fi}
\def\@psdoloop#1,#2,#3\@@#4#5{\def#4{#1}\ifx #4\@nnil \else
       #5\def#4{#2}\ifx #4\@nnil \else#5\@ipsdoloop #3\@@#4{#5}\fi\fi}
\def\@ipsdoloop#1,#2\@@#3#4{\def#3{#1}\ifx #3\@nnil 
       \let\@nextwhile=\@psdonoop \else
      #4\relax\let\@nextwhile=\@ipsdoloop\fi\@nextwhile#2\@@#3{#4}}
\def\@tpsdo#1:=#2\do#3{\xdef\@psdotmp{#2}\ifx\@psdotmp\@empty \else
    \@tpsdoloop#2\@nil\@nil\@@#1{#3}\fi}
\def\@tpsdoloop#1#2\@@#3#4{\def#3{#1}\ifx #3\@nnil 
       \let\@nextwhile=\@psdonoop \else
      #4\relax\let\@nextwhile=\@tpsdoloop\fi\@nextwhile#2\@@#3{#4}}
% 
% \fbox is defined in latex.tex; so if \fbox is undefined, assume that
% we are not in LaTeX.
% Perhaps this could be done better???
\ifx\undefined\fbox
% \fbox code from modified slightly from LaTeX
\newdimen\fboxrule
\newdimen\fboxsep
\newdimen\ps@tempdima
\newbox\ps@tempboxa
\fboxsep = 3pt
\fboxrule = .4pt
\long\def\fbox#1{\leavevmode\setbox\ps@tempboxa\hbox{#1}\ps@tempdima\fboxrule
    \advance\ps@tempdima \fboxsep \advance\ps@tempdima \dp\ps@tempboxa
   \hbox{\lower \ps@tempdima\hbox
  {\vbox{\hrule height \fboxrule
          \hbox{\vrule width \fboxrule \hskip\fboxsep
          \vbox{\vskip\fboxsep \box\ps@tempboxa\vskip\fboxsep}\hskip 
                 \fboxsep\vrule width \fboxrule}
                 \hrule height \fboxrule}}}}
\fi
%
%%%%%%%%%%%%%%%%%%%%%%%%%%%%%%%%%%%%%%%%%%%%%%%%%%%%%%%%%%%%%%%%%%%
% file reading stuff from epsf.tex
%   EPSF.TEX macro file:
%   Written by Tomas Rokicki of Radical Eye Software, 29 Mar 1989.
%   Revised by Don Knuth, 3 Jan 1990.
%   Revised by Tomas Rokicki to accept bounding boxes with no
%      space after the colon, 18 Jul 1990.
%   Portions modified/removed for use in PSFIG package by
%      J. Daniel Smith, 9 October 1990.
%
\newread\ps@stream
\newif\ifnot@eof       % continue looking for the bounding box?
\newif\if@noisy        % report what you're making?
\newif\if@atend        % %%BoundingBox: has (at end) specification
\newif\if@psfile       % does this look like a PostScript file?
%
% PostScript files should start with `%!'
%
{\catcode`\%=12\global\gdef\epsf@start{%!}}
\def\epsf@PS{PS}
%
\def\epsf@getbb#1{%
%
%   The first thing we need to do is to open the
%   PostScript file, if possible.
%
\openin\ps@stream=#1
\ifeof\ps@stream\ps@typeout{Error, File #1 not found}\else
%
%   Okay, we got it. Now we'll scan lines until we find one that doesn't
%   start with %. We're looking for the bounding box comment.
%
   {\not@eoftrue \chardef\other=12
    \def\do##1{\catcode`##1=\other}\dospecials \catcode`\ =10
    \loop
       \if@psfile
	  \read\ps@stream to \epsf@fileline
       \else{
	  \obeyspaces
          \read\ps@stream to \epsf@tmp\global\let\epsf@fileline\epsf@tmp}
       \fi
       \ifeof\ps@stream\not@eoffalse\else
%
%   Check the first line for `%!'.  Issue a warning message if its not
%   there, since the file might not be a PostScript file.
%
       \if@psfile\else
       \expandafter\epsf@test\epsf@fileline:. \\%
       \fi
%
%   We check to see if the first character is a % sign;
%   if so, we look further and stop only if the line begins with
%   `%%BoundingBox:' and the `(atend)' specification was not found.
%   That is, the only way to stop is when the end of file is reached,
%   or a `%%BoundingBox: llx lly urx ury' line is found.
%
          \expandafter\epsf@aux\epsf@fileline:. \\%
       \fi
   \ifnot@eof\repeat
   }\closein\ps@stream\fi}%
%
% This tests if the file we are reading looks like a PostScript file.
%
\long\def\epsf@test#1#2#3:#4\\{\def\epsf@testit{#1#2}
			\ifx\epsf@testit\epsf@start\else
\ps@typeout{Warning! File does not start with `\epsf@start'.  It may not be a PostScript file.}
			\fi
			\@psfiletrue} % don't test after 1st line
%
%   We still need to define the tricky \epsf@aux macro. This requires
%   a couple of magic constants for comparison purposes.
%
{\catcode`\%=12\global\let\epsf@percent=%\global\def\epsf@bblit{%BoundingBox}}
%
%
%   So we're ready to check for `%BoundingBox:' and to grab the
%   values if they are found.  We continue searching if `(at end)'
%   was found after the `%BoundingBox:'.
%
\long\def\epsf@aux#1#2:#3\\{\ifx#1\epsf@percent
   \def\epsf@testit{#2}\ifx\epsf@testit\epsf@bblit
	\@atendfalse
        \epsf@atend #3 . \\%
	\if@atend	
	   \if@verbose{
		\ps@typeout{psfig: found `(atend)'; continuing search}
	   }\fi
        \else
        \epsf@grab #3 . . . \\%
        \not@eoffalse
        \global\no@bbfalse
        \fi
   \fi\fi}%
%
%   Here we grab the values and stuff them in the appropriate definitions.
%
\def\epsf@grab #1 #2 #3 #4 #5\\{%
   \global\def\epsf@llx{#1}\ifx\epsf@llx\empty
      \epsf@grab #2 #3 #4 #5 .\\\else
   \global\def\epsf@lly{#2}%
   \global\def\epsf@urx{#3}\global\def\epsf@ury{#4}\fi}%
%
% Determine if the stuff following the %%BoundingBox is `(atend)'
% J. Daniel Smith.  Copied from \epsf@grab above.
%
\def\epsf@atendlit{(atend)} 
\def\epsf@atend #1 #2 #3\\{%
   \def\epsf@tmp{#1}\ifx\epsf@tmp\empty
      \epsf@atend #2 #3 .\\\else
   \ifx\epsf@tmp\epsf@atendlit\@atendtrue\fi\fi}


% End of file reading stuff from epsf.tex
%%%%%%%%%%%%%%%%%%%%%%%%%%%%%%%%%%%%%%%%%%%%%%%%%%%%%%%%%%%%%%%%%%%

%%%%%%%%%%%%%%%%%%%%%%%%%%%%%%%%%%%%%%%%%%%%%%%%%%%%%%%%%%%%%%%%%%%
% trigonometry stuff from "trig.tex"
\chardef\psletter = 11 % won't conflict with \begin{letter} now...
\chardef\other = 12

\newif \ifdebug %%% turn me on to see TeX hard at work ...
\newif\ifc@mpute %%% don't need to compute some values
\c@mputetrue % but assume that we do

\let\then = \relax
\def\r@dian{pt }
\let\r@dians = \r@dian
\let\dimensionless@nit = \r@dian
\let\dimensionless@nits = \dimensionless@nit
\def\internal@nit{sp }
\let\internal@nits = \internal@nit
\newif\ifstillc@nverging
\def \Mess@ge #1{\ifdebug \then \message {#1} \fi}

{ %%% Things that need abnormal catcodes %%%
	\catcode `\@ = \psletter
	\gdef \nodimen {\expandafter \n@dimen \the \dimen}
	\gdef \term #1 #2 #3%
	       {\edef \t@ {\the #1}%%% freeze parameter 1 (count, by value)
		\edef \t@@ {\expandafter \n@dimen \the #2\r@dian}%
				   %%% freeze parameter 2 (dimen, by value)
		\t@rm {\t@} {\t@@} {#3}%
	       }
	\gdef \t@rm #1 #2 #3%
	       {{%
		\count 0 = 0
		\dimen 0 = 1 \dimensionless@nit
		\dimen 2 = #2\relax
		\Mess@ge {Calculating term #1 of \nodimen 2}%
		\loop
		\ifnum	\count 0 < #1
		\then	\advance \count 0 by 1
			\Mess@ge {Iteration \the \count 0 \space}%
			\Multiply \dimen 0 by {\dimen 2}%
			\Mess@ge {After multiplication, term = \nodimen 0}%
			\Divide \dimen 0 by {\count 0}%
			\Mess@ge {After division, term = \nodimen 0}%
		\repeat
		\Mess@ge {Final value for term #1 of 
				\nodimen 2 \space is \nodimen 0}%
		\xdef \Term {#3 = \nodimen 0 \r@dians}%
		\aftergroup \Term
	       }}
	\catcode `\p = \other
	\catcode `\t = \other
	\gdef \n@dimen #1pt{#1} %%% throw away the ``pt''
}

\def \Divide #1by #2{\divide #1 by #2} %%% just a synonym

\def \Multiply #1by #2%%% allows division of a dimen by a dimen
       {{%%% should really freeze parameter 2 (dimen, passed by value)
	\count 0 = #1\relax
	\count 2 = #2\relax
	\count 4 = 65536
	\Mess@ge {Before scaling, count 0 = \the \count 0 \space and
			count 2 = \the \count 2}%
	\ifnum	\count 0 > 32767 %%% do our best to avoid overflow
	\then	\divide \count 0 by 4
		\divide \count 4 by 4
	\else	\ifnum	\count 0 < -32767
		\then	\divide \count 0 by 4
			\divide \count 4 by 4
		\else
		\fi
	\fi
	\ifnum	\count 2 > 32767 %%% while retaining reasonable accuracy
	\then	\divide \count 2 by 4
		\divide \count 4 by 4
	\else	\ifnum	\count 2 < -32767
		\then	\divide \count 2 by 4
			\divide \count 4 by 4
		\else
		\fi
	\fi
	\multiply \count 0 by \count 2
	\divide \count 0 by \count 4
	\xdef \product {#1 = \the \count 0 \internal@nits}%
	\aftergroup \product
       }}

\def\r@duce{\ifdim\dimen0 > 90\r@dian \then   % sin(x+90) = sin(180-x)
		\multiply\dimen0 by -1
		\advance\dimen0 by 180\r@dian
		\r@duce
	    \else \ifdim\dimen0 < -90\r@dian \then  % sin(-x) = sin(360+x)
		\advance\dimen0 by 360\r@dian
		\r@duce
		\fi
	    \fi}

\def\Sine#1%
       {{%
	\dimen 0 = #1 \r@dian
	\r@duce
	\ifdim\dimen0 = -90\r@dian \then
	   \dimen4 = -1\r@dian
	   \c@mputefalse
	\fi
	\ifdim\dimen0 = 90\r@dian \then
	   \dimen4 = 1\r@dian
	   \c@mputefalse
	\fi
	\ifdim\dimen0 = 0\r@dian \then
	   \dimen4 = 0\r@dian
	   \c@mputefalse
	\fi
%
	\ifc@mpute \then
        	% convert degrees to radians
		\divide\dimen0 by 180
		\dimen0=3.141592654\dimen0
%
		\dimen 2 = 3.1415926535897963\r@dian %%% a well-known constant
		\divide\dimen 2 by 2 %%% we only deal with -pi/2 : pi/2
		\Mess@ge {Sin: calculating Sin of \nodimen 0}%
		\count 0 = 1 %%% see power-series expansion for sine
		\dimen 2 = 1 \r@dian %%% ditto
		\dimen 4 = 0 \r@dian %%% ditto
		\loop
			\ifnum	\dimen 2 = 0 %%% then we've done
			\then	\stillc@nvergingfalse 
			\else	\stillc@nvergingtrue
			\fi
			\ifstillc@nverging %%% then calculate next term
			\then	\term {\count 0} {\dimen 0} {\dimen 2}%
				\advance \count 0 by 2
				\count 2 = \count 0
				\divide \count 2 by 2
				\ifodd	\count 2 %%% signs alternate
				\then	\advance \dimen 4 by \dimen 2
				\else	\advance \dimen 4 by -\dimen 2
				\fi
		\repeat
	\fi		
			\xdef \sine {\nodimen 4}%
       }}

% Now the Cosine can be calculated easily by calling \Sine
\def\Cosine#1{\ifx\sine\UnDefined\edef\Savesine{\relax}\else
		             \edef\Savesine{\sine}\fi
	{\dimen0=#1\r@dian\advance\dimen0 by 90\r@dian
	 \Sine{\nodimen 0}
	 \xdef\cosine{\sine}
	 \xdef\sine{\Savesine}}}	      
% end of trig stuff
%%%%%%%%%%%%%%%%%%%%%%%%%%%%%%%%%%%%%%%%%%%%%%%%%%%%%%%%%%%%%%%%%%%%

\def\psdraft{
	\def\@psdraft{0}
	%\ps@typeout{draft level now is \@psdraft \space . }
}
\def\psfull{
	\def\@psdraft{100}
	%\ps@typeout{draft level now is \@psdraft \space . }
}

\psfull

\newif\if@scalefirst
\def\psscalefirst{\@scalefirsttrue}
\def\psrotatefirst{\@scalefirstfalse}
\psrotatefirst

\newif\if@draftbox
\def\psnodraftbox{
	\@draftboxfalse
}
\def\psdraftbox{
	\@draftboxtrue
}
\@draftboxtrue

\newif\if@prologfile
\newif\if@postlogfile
\def\pssilent{
	\@noisyfalse
}
\def\psnoisy{
	\@noisytrue
}
\psnoisy
%%% These are for the option list.
%%% A specification of the form a = b maps to calling \@p@@sa{b}
\newif\if@bbllx
\newif\if@bblly
\newif\if@bburx
\newif\if@bbury
\newif\if@height
\newif\if@width
\newif\if@rheight
\newif\if@rwidth
\newif\if@angle
\newif\if@clip
\newif\if@verbose
\def\@p@@sclip#1{\@cliptrue}


\newif\if@decmpr

%%% GDH 7/26/87 -- changed so that it first looks in the local directory,
%%% then in a specified global directory for the ps file.
%%% RPR 6/25/91 -- changed so that it defaults to user-supplied name if
%%% boundingbox info is specified, assuming graphic will be created by
%%% print time.
%%% TJD 10/19/91 -- added bbfile vs. file distinction, and @decmpr flag

\def\@p@@sfigure#1{\def\@p@sfile{null}\def\@p@sbbfile{null}
	        \openin1=#1.bb
		\ifeof1\closein1
	        	\openin1=\figurepath#1.bb
			\ifeof1\closein1
			        \openin1=#1
				\ifeof1\closein1%
				       \openin1=\figurepath#1
					\ifeof1
					   \ps@typeout{Error, File #1 not found}
						\if@bbllx\if@bblly
				   		\if@bburx\if@bbury
			      				\def\@p@sfile{#1}%
			      				\def\@p@sbbfile{#1}%
							\@decmprfalse
				  	   	\fi\fi\fi\fi
					\else\closein1
				    		\def\@p@sfile{\figurepath#1}%
				    		\def\@p@sbbfile{\figurepath#1}%
						\@decmprfalse
	                       		\fi%
			 	\else\closein1%
					\def\@p@sfile{#1}
					\def\@p@sbbfile{#1}
					\@decmprfalse
			 	\fi
			\else
				\def\@p@sfile{\figurepath#1}
				\def\@p@sbbfile{\figurepath#1.bb}
				\@decmprtrue
			\fi
		\else
			\def\@p@sfile{#1}
			\def\@p@sbbfile{#1.bb}
			\@decmprtrue
		\fi}

\def\@p@@sfile#1{\@p@@sfigure{#1}}

\def\@p@@sbbllx#1{
		%\ps@typeout{bbllx is #1}
		\@bbllxtrue
		\dimen100=#1
		\edef\@p@sbbllx{\number\dimen100}
}
\def\@p@@sbblly#1{
		%\ps@typeout{bblly is #1}
		\@bbllytrue
		\dimen100=#1
		\edef\@p@sbblly{\number\dimen100}
}
\def\@p@@sbburx#1{
		%\ps@typeout{bburx is #1}
		\@bburxtrue
		\dimen100=#1
		\edef\@p@sbburx{\number\dimen100}
}
\def\@p@@sbbury#1{
		%\ps@typeout{bbury is #1}
		\@bburytrue
		\dimen100=#1
		\edef\@p@sbbury{\number\dimen100}
}
\def\@p@@sheight#1{
		\@heighttrue
		\dimen100=#1
   		\edef\@p@sheight{\number\dimen100}
		%\ps@typeout{Height is \@p@sheight}
}
\def\@p@@swidth#1{
		%\ps@typeout{Width is #1}
		\@widthtrue
		\dimen100=#1
		\edef\@p@swidth{\number\dimen100}
}
\def\@p@@srheight#1{
		%\ps@typeout{Reserved height is #1}
		\@rheighttrue
		\dimen100=#1
		\edef\@p@srheight{\number\dimen100}
}
\def\@p@@srwidth#1{
		%\ps@typeout{Reserved width is #1}
		\@rwidthtrue
		\dimen100=#1
		\edef\@p@srwidth{\number\dimen100}
}
\def\@p@@sangle#1{
		%\ps@typeout{Rotation is #1}
		\@angletrue
%		\dimen100=#1
		\edef\@p@sangle{#1} %\number\dimen100}
}
\def\@p@@ssilent#1{ 
		\@verbosefalse
}
\def\@p@@sprolog#1{\@prologfiletrue\def\@prologfileval{#1}}
\def\@p@@spostlog#1{\@postlogfiletrue\def\@postlogfileval{#1}}
\def\@cs@name#1{\csname #1\endcsname}
\def\@setparms#1=#2,{\@cs@name{@p@@s#1}{#2}}
%
% initialize the defaults (size the size of the figure)
%
\def\ps@init@parms{
		\@bbllxfalse \@bbllyfalse
		\@bburxfalse \@bburyfalse
		\@heightfalse \@widthfalse
		\@rheightfalse \@rwidthfalse
		\def\@p@sbbllx{}\def\@p@sbblly{}
		\def\@p@sbburx{}\def\@p@sbbury{}
		\def\@p@sheight{}\def\@p@swidth{}
		\def\@p@srheight{}\def\@p@srwidth{}
		\def\@p@sangle{0}
		\def\@p@sfile{} \def\@p@sbbfile{}
		\def\@p@scost{10}
		\def\@sc{}
		\@prologfilefalse
		\@postlogfilefalse
		\@clipfalse
		\if@noisy
			\@verbosetrue
		\else
			\@verbosefalse
		\fi
}
%
% Go through the options setting things up.
%
\def\parse@ps@parms#1{
	 	\@psdo\@psfiga:=#1\do
		   {\expandafter\@setparms\@psfiga,}}
%
% Compute bb height and width
%
\newif\ifno@bb
\def\bb@missing{
	\if@verbose{
		\ps@typeout{psfig: searching \@p@sbbfile \space  for bounding box}
	}\fi
	\no@bbtrue
	\epsf@getbb{\@p@sbbfile}
        \ifno@bb \else \bb@cull\epsf@llx\epsf@lly\epsf@urx\epsf@ury\fi
}	
\def\bb@cull#1#2#3#4{
	\dimen100=#1 bp\edef\@p@sbbllx{\number\dimen100}
	\dimen100=#2 bp\edef\@p@sbblly{\number\dimen100}
	\dimen100=#3 bp\edef\@p@sbburx{\number\dimen100}
	\dimen100=#4 bp\edef\@p@sbbury{\number\dimen100}
	\no@bbfalse
}
% rotate point (#1,#2) about (0,0).
% The sine and cosine of the angle are already stored in \sine and
% \cosine.  The result is placed in (\p@intvaluex, \p@intvaluey).
\newdimen\p@intvaluex
\newdimen\p@intvaluey
\def\rotate@#1#2{{\dimen0=#1 sp\dimen1=#2 sp
%            	calculate x' = x \cos\theta - y \sin\theta
		  \global\p@intvaluex=\cosine\dimen0
		  \dimen3=\sine\dimen1
		  \global\advance\p@intvaluex by -\dimen3
% 		calculate y' = x \sin\theta + y \cos\theta
		  \global\p@intvaluey=\sine\dimen0
		  \dimen3=\cosine\dimen1
		  \global\advance\p@intvaluey by \dimen3
		  }}
\def\compute@bb{
		\no@bbfalse
		\if@bbllx \else \no@bbtrue \fi
		\if@bblly \else \no@bbtrue \fi
		\if@bburx \else \no@bbtrue \fi
		\if@bbury \else \no@bbtrue \fi
		\ifno@bb \bb@missing \fi
		\ifno@bb \ps@typeout{FATAL ERROR: no bb supplied or found}
			\no-bb-error
		\fi
		%
%\ps@typeout{BB: \@p@sbbllx, \@p@sbblly, \@p@sbburx, \@p@sbbury} 
%
% store height/width of original (unrotated) bounding box
		\count203=\@p@sbburx
		\count204=\@p@sbbury
		\advance\count203 by -\@p@sbbllx
		\advance\count204 by -\@p@sbblly
		\edef\ps@bbw{\number\count203}
		\edef\ps@bbh{\number\count204}
		%\ps@typeout{ psbbh = \ps@bbh, psbbw = \ps@bbw }
		\if@angle 
			\Sine{\@p@sangle}\Cosine{\@p@sangle}
	        	{\dimen100=\maxdimen\xdef\r@p@sbbllx{\number\dimen100}
					    \xdef\r@p@sbblly{\number\dimen100}
			                    \xdef\r@p@sbburx{-\number\dimen100}
					    \xdef\r@p@sbbury{-\number\dimen100}}
%
% Need to rotate all four points and take the X-Y extremes of the new
% points as the new bounding box.
                        \def\minmaxtest{
			   \ifnum\number\p@intvaluex<\r@p@sbbllx
			      \xdef\r@p@sbbllx{\number\p@intvaluex}\fi
			   \ifnum\number\p@intvaluex>\r@p@sbburx
			      \xdef\r@p@sbburx{\number\p@intvaluex}\fi
			   \ifnum\number\p@intvaluey<\r@p@sbblly
			      \xdef\r@p@sbblly{\number\p@intvaluey}\fi
			   \ifnum\number\p@intvaluey>\r@p@sbbury
			      \xdef\r@p@sbbury{\number\p@intvaluey}\fi
			   }
%			lower left
			\rotate@{\@p@sbbllx}{\@p@sbblly}
			\minmaxtest
%			upper left
			\rotate@{\@p@sbbllx}{\@p@sbbury}
			\minmaxtest
%			lower right
			\rotate@{\@p@sbburx}{\@p@sbblly}
			\minmaxtest
%			upper right
			\rotate@{\@p@sbburx}{\@p@sbbury}
			\minmaxtest
			\edef\@p@sbbllx{\r@p@sbbllx}\edef\@p@sbblly{\r@p@sbblly}
			\edef\@p@sbburx{\r@p@sbburx}\edef\@p@sbbury{\r@p@sbbury}
%\ps@typeout{rotated BB: \r@p@sbbllx, \r@p@sbblly, \r@p@sbburx, \r@p@sbbury}
		\fi
		\count203=\@p@sbburx
		\count204=\@p@sbbury
		\advance\count203 by -\@p@sbbllx
		\advance\count204 by -\@p@sbblly
		\edef\@bbw{\number\count203}
		\edef\@bbh{\number\count204}
		%\ps@typeout{ bbh = \@bbh, bbw = \@bbw }
}
%
% \in@hundreds performs #1 * (#2 / #3) correct to the hundreds,
%	then leaves the result in @result
%
\def\in@hundreds#1#2#3{\count240=#2 \count241=#3
		     \count100=\count240	% 100 is first digit #2/#3
		     \divide\count100 by \count241
		     \count101=\count100
		     \multiply\count101 by \count241
		     \advance\count240 by -\count101
		     \multiply\count240 by 10
		     \count101=\count240	%101 is second digit of #2/#3
		     \divide\count101 by \count241
		     \count102=\count101
		     \multiply\count102 by \count241
		     \advance\count240 by -\count102
		     \multiply\count240 by 10
		     \count102=\count240	% 102 is the third digit
		     \divide\count102 by \count241
		     \count200=#1\count205=0
		     \count201=\count200
			\multiply\count201 by \count100
		 	\advance\count205 by \count201
		     \count201=\count200
			\divide\count201 by 10
			\multiply\count201 by \count101
			\advance\count205 by \count201
			%
		     \count201=\count200
			\divide\count201 by 100
			\multiply\count201 by \count102
			\advance\count205 by \count201
			%
		     \edef\@result{\number\count205}
}
\def\compute@wfromh{
		% computing : width = height * (bbw / bbh)
		\in@hundreds{\@p@sheight}{\@bbw}{\@bbh}
		%\ps@typeout{ \@p@sheight * \@bbw / \@bbh, = \@result }
		\edef\@p@swidth{\@result}
		%\ps@typeout{w from h: width is \@p@swidth}
}
\def\compute@hfromw{
		% computing : height = width * (bbh / bbw)
	        \in@hundreds{\@p@swidth}{\@bbh}{\@bbw}
		%\ps@typeout{ \@p@swidth * \@bbh / \@bbw = \@result }
		\edef\@p@sheight{\@result}
		%\ps@typeout{h from w : height is \@p@sheight}
}
\def\compute@handw{
		\if@height 
			\if@width
			\else
				\compute@wfromh
			\fi
		\else 
			\if@width
				\compute@hfromw
			\else
				\edef\@p@sheight{\@bbh}
				\edef\@p@swidth{\@bbw}
			\fi
		\fi
}
\def\compute@resv{
		\if@rheight \else \edef\@p@srheight{\@p@sheight} \fi
		\if@rwidth \else \edef\@p@srwidth{\@p@swidth} \fi
		%\ps@typeout{rheight = \@p@srheight, rwidth = \@p@srwidth}
}
%		
% Compute any missing values
\def\compute@sizes{
	\compute@bb
	\if@scalefirst\if@angle
% at this point the bounding box has been adjsuted correctly for
% rotation.  PSFIG does all of its scaling using \@bbh and \@bbw.  If
% a width= or height= was specified along with \psscalefirst, then the
% width=/height= value needs to be adjusted to match the new (rotated)
% bounding box size (specifed in \@bbw and \@bbh).
%    \ps@bbw       width=
%    -------  =  ---------- 
%    \@bbw       new width=
% so `new width=' = (width= * \@bbw) / \ps@bbw; where \ps@bbw is the
% width of the original (unrotated) bounding box.
	\if@width
	   \in@hundreds{\@p@swidth}{\@bbw}{\ps@bbw}
	   \edef\@p@swidth{\@result}
	\fi
	\if@height
	   \in@hundreds{\@p@sheight}{\@bbh}{\ps@bbh}
	   \edef\@p@sheight{\@result}
	\fi
	\fi\fi
	\compute@handw
	\compute@resv}

%
% \psfig
% usage : \psfig{file=, height=, width=, bbllx=, bblly=, bburx=, bbury=,
%			rheight=, rwidth=, clip=}
%
% "clip=" is a switch and takes no value, but the `=' must be present.
\def\psfig#1{\vbox {
	% do a zero width hard space so that a single
	% \psfig in a centering enviornment will behave nicely
	%{\setbox0=\hbox{\ }\ \hskip-\wd0}
	%
	\ps@init@parms
	\parse@ps@parms{#1}
	\compute@sizes
	%
	\ifnum\@p@scost<\@psdraft{
		%
		\special{ps::[begin] 	\@p@swidth \space \@p@sheight \space
				\@p@sbbllx \space \@p@sbblly \space
				\@p@sbburx \space \@p@sbbury \space
				startTexFig \space }
		\if@angle
			\special {ps:: \@p@sangle \space rotate \space} 
		\fi
		\if@clip{
			\if@verbose{
				\ps@typeout{(clip)}
			}\fi
			\special{ps:: doclip \space }
		}\fi
		\if@prologfile
		    \special{ps: plotfile \@prologfileval \space } \fi
		\if@decmpr{
			\if@verbose{
				\ps@typeout{psfig: including \@p@sfile.Z \space }
			}\fi
			\special{ps: plotfile "`zcat \@p@sfile.Z" \space }
		}\else{
			\if@verbose{
				\ps@typeout{psfig: including \@p@sfile \space }
			}\fi
			\special{ps: plotfile \@p@sfile \space }
		}\fi
		\if@postlogfile
		    \special{ps: plotfile \@postlogfileval \space } \fi
		\special{ps::[end] endTexFig \space }
		% Create the vbox to reserve the space for the figure.
		\vbox to \@p@srheight sp{
		% 1/92 TJD Changed from "true sp" to "sp" for magnification.
			\hbox to \@p@srwidth sp{
				\hss
			}
		\vss
		}
	}\else{
		% draft figure, just reserve the space and print the
		% path name.
		\if@draftbox{		
			% Verbose draft: print file name in box
			\hbox{\frame{\vbox to \@p@srheight sp{
			\vss
			\hbox to \@p@srwidth sp{ \hss \@p@sfile \hss }
			\vss
			}}}
		}\else{
			% Non-verbose draft
			\vbox to \@p@srheight sp{
			\vss
			\hbox to \@p@srwidth sp{\hss}
			\vss
			}
		}\fi	



	}\fi
}}
\psfigRestoreAt
\let\@=\LaTeXAtSign




%%% \special{header=/group/csdl/tex/psfig/lprep71.pro}
%%% \begin{document}

\setcounter{chapter}{2}
\chapter{RESRA: the Representation Language}
\label{sec:resra}

The previous chapter discusses the motivation and the conceptual framework
for a representation-based approach to collaborative learning. It also
introduces RESRA --- a special-purpose language designed for representing
the thematic features of scientific text, and individual learners' points
of view. This chapter describes RESRA in detail. It begins by identifying
four major requirements for the language. Next, it defines several key
concepts underlying RESRA. The subsequent two sections elaborate the two
main constructs of the representation: {\it primitive\/} and {\it canonical
forms}, respectively. Examples are given to illustrate their semantics. The
chapter concludes by discussing the extendibility of RESRA.

\section{Representational requirements}
\label{sec:c3-design}

RESRA falls into the genre of the semi-structured representation.  Unlike
traditional knowledge representation schemes (e.g., predicate logic, frame)
which aim at formalizing knowledge for machine reasoning, the primary
purpose of the semi-structured representation is to aid human reasoning and
communication. RESRA, in particular, is designed to help human learners
extract meanings from research artifacts, and facilitate interactions among
learners. Below are the main requirements for the language:

\begin{itemize}
\item {\it RESRA should represent the essential feature of a research
  artifact.\/} The term {\it feature\/} encompasses individual {\it
  thematic components\/} of an artifact as well as the relationships
  between those components. Although the focus of RESRA is on the {\it
  essential feature,\/} the {\it essentiality\/} of a thematic component
  is by no means absolute: a learner can conceivably take a minor point
  made by the author and treat it as something major, because it happens
  to be of interest to him.  Hence, RESRA needs to be able to express a
  full range of thematic features.
  
  The key to the expressiveness of RESRA lies in the granularity of its
  primitives: if the grain size is too coarse, it sacrifices the
  expressiveness of the representation, i.e., less differentiating; on
  the other hand, if the grain size is too fine, the primitives are
  likely to be incomplete, and more difficult to learn and use. The
  design of RESRA needs to strike a delicate balance between the two.
  
\item {\it RESRA is capable of representing various learners' views on
  the content of an artifact.\/} Similar to the previous requirement, the
  most important consideration here is the right grain-size, i.e.,
  different levels of views, such as evaluative views, constructive views,
  et al.
  
\item {\it RESRA mirrors the structure of the CLARE learning model.\/}
  Specifically, RESRA primitives should be divided into four groups: {\it
  summarization\/}, {\it evaluation\/}, {\it argumentation\/}, and {\it
  integration\/}; each supports the corresponding component in the SECAI
  learning model.
  
\item {\it RESRA is usable by human users.\/} One important criterion
  of usability is {\it parsimony\/}, i.e., the initial set of primitives
  is small enough that their semantics can be mastered by human learners
  in a reasonable period of time. To support a large primitives set or to
  allow learners to create their own primitives, RESRA needs to provide
  mechanisms for domain specialization.
\end{itemize}

Though it is conceivable that RESRA be used in a paper-and-pencil mode, the
representation presupposes a computer-based support environment. One main
benefit such an environment provides is the capability to make the
primitive selection context sensitive and, as a result, increases the ease
of using the language.


\section{Basic concepts}
\label{sec:concepts}

RESRA, or the REpresentational Schema of Research Artifacts, is a language
that combines aspects of two widely used knowledge representation schemes:
semantic networks and frames. At core, RESRA consists of two constructs:
{\it primitives\/} and {\it canonical forms\/}. The former is the atomic
building-block of RESRA. It is composed of two genres: {\it node\/} and
{\it link\/}. A node is used to represent the essential thematic feature of
a learning artifact and the learner's points of view. A link is for
describing the inter-relationship between any two nodes. For example,

\begin{quotation}
 \( \fbox{{\sf theory}} \stackrel{suggests}{\longrightarrow} \fbox{{\sf
 claim}} \)
\end{quotation}

where {\sf theory\/} and {\sf claim\/} are node primitives; {\it
suggests\/} is a link primitive; the expression as a whole is called a {\sl
tuple\/}.

A RESRA node typically consists of a number of {\it fields\/}; each
describes one aspect of the targeted theme. For example, the node primitive
\fbox{{\sf claim\/}} consists of such fields as ``name,'' ``type of
claim,'' ``description,'' and so on. RESRA links are often referred to as
tuples, for example, \( \fbox{{\sf evidence}}
\stackrel{suggests}{\longrightarrow} \fbox{{\sf problem}} \).  The process
of creating an {\it instance\/} of a node or link primitive is called {\it
instantiation\/}. Node and link primitives are described in the next
section.

The {\it Canonical RESRA Form\/}, or CRF for short, is a collection of
inter-related tuples that, together, describe an exemplary thematic
structure of a particular type of artifact, such as a concept or empirical
paper. While the primitive is useful for revealing the fine-grained
components of an artifact, CRF is for highlighting the artifact-level
features. The two are complementary. CRFs are described in detail in
Section \ref{sec:crf}.


\section{RESRA primitives}
\label{sec:resra-primitives}

RESRA primitives are divided into four groups: {\it summarative\/}, {\it
evaluative\/}, {\it argumentative\/}, and {\it integrative\/}, which mirror
the four types of primary activities in the SECAI model. The following
section describes each category in turn. A graphical depiction is also
provided.


\subsection{Summarative primitives}
\label{sec:summarative primitives}

Summarative node and link primitives are intended for {\it summarizing\/}
the content of a learning artifact, a process similar to reconstructing
design information from program source code. CLARE provides 9 summarative
node primitives and 14 link primitives, which are described below. Figure
\ref{fig:sum-resra} is the corresponding graphical depiction.

\begin{figure}[htb]
 \fbox{\centerline{\psfig{figure=Figures/sum-resra.eps,height=4.0in}}}
  \caption{RESRA summarative primitives at a glance}
  \label{fig:sum-resra}
\end{figure}


\subsubsection{Summarative node primitives}

\paragraph{}

\noindent\fbox{{\sf Problem}}:\hspace{.2in}A phenomenon, event, or process
whose behavior cannot be fully explained based on the current state of
knowledge and, hence, requires further inquiry. An example \fbox{{\sf
problem \/}} is: ``Despite rapid improvement of development environments
and testing techniques, software systems still contain bugs.''  Note that a
\fbox{{\sf problem\/}} is not a {\it question\/}, though it is often
triggered by a question. A problem is normally be accompanied by a detailed
description.

\paragraph{}

\noindent\fbox{{\sf Claim}}:\hspace{.2in}An assertion about a given problem
situation that can be either supported or refuted. For example, ``Cleanroom
engineering provides a viable solution in producing zero defect software.''
One important characteristic of a \fbox{{\sf claim\/}} is {\it
falsifiability\/}, i.e., a claim can be falsified through evidence and
logical reasoning.

Depending upon the level of evidentiality, claims can be divided into four
types: {\it conjecture\/}, {\it hypothesis\/}, {\it fact\/}, and {\it
axiom\/}:

\begin{itemize}
\item {\it Conjecture\/}: an omen-like claim with little, if any at all,
  supporting evidence. Commonly, it is a result of ``scientific
  imagination,'' and seen in highly theoretic domains, such as mathematics.
  
\item {\it Hypothesis\/}: a claim that is derived from an existing theory
  or based upon available empirical/experiential evidence. Hypotheses
  formulation and testing constitutes a substantial portion of scientific
  activities. Hypotheses, when well supported by empirical evidence, are 
  considered as scientific ``facts.''
  
\item {\it Fact\/}: a claim that is well supported by available evidence,
  and thus considered as true based on the current state of knowledge.
  
\item {\it Axiom\/}: a claim that is generally accepted as true, and thus
  may be used as a basis for inference or argument.
\end{itemize}

The most common form of claims in the scientific discourse is
{\sf hypothesis\/}. A claim is always made in relation to a particular
problem, regardless whether it is explicitly stated.

\paragraph{}

\noindent\fbox{{\sf Evidence}}:\hspace{.2in}Data gathered for the purpose
of supporting or refuting a claim. For example, ``The use of cleanroom
techniques has yielded a 10-fold reduction of defects in the project
Alpha.''  Evidence can be either qualitative or quantitative. It is also
always stated with respect to a given claim, and used to support or counter
the same claim.

\paragraph{}

\noindent\fbox{{\sf Theory}}:\hspace{.2in}A systematic formulation about a
particular phenomenon or problem. An example of \fbox{{\sf theory}} is
Halstead's theory of software complexity, i.e., software science. A theory
normally consists of a set of coherent, inter-related claims about the
targeted problem domain. These claims may be derived deductively from other
theories, or inductively from well-established facts or hypotheses. An
important characteristic of the theory is its {\it predictability\/}; a
theory can be used to predict the outcome of a particular phenomena.

\paragraph{}

\noindent\fbox{{\sf Concept}}:\hspace{.2in}A primitive construct used in
formulating theory, claim, or method. Examples of \fbox{{\sf concept\/}}
are: ``meta-learning,'' ``knowledge representation,'' ``example-based
learning.'' The main feature of the concept is atomicity: concepts are the
basic building block of human knowledge.

\paragraph{}

\noindent\fbox{{\sf Method}}:\hspace{.2in}Procedures or techniques used for
generating evidence for a particular claim. Examples of \fbox{{\sf
method\/}} are: Delphi study, nominal grouping technique, Waterfall model
of system development. A method itself can be the target of an inquiry.

\paragraph{}

\noindent\fbox{{\sf Source}}:\hspace{.2in}An identifiable written artifact,
either artifact itself or the pointer to it, i.e., surrogate or reference.
Examples of \fbox{{\sf source\/}} are:, ``War and Peace,'' ``Slides from
John's presentation,'' and so on.  A source is a {\it medium\/} which
contains other conceptual constructs, such as \fbox{{\sf problem\/}},
\fbox{{\sf theory\/}}, \fbox{{\sf concept\/}}. However, \fbox{{\sf
source\/}} it in itself is not a conceptual construct.

\paragraph{}

\noindent\fbox{{\sf Thing}}:\hspace{.2in}An object, event, or process
(natural or man-made) that is the source of a problem or target of an
inquiry. Examples are ``Unix,'' ``Macintosh,'' ``CLARE.''

\paragraph{}

\noindent\fbox{{\sf Other}}:\hspace{.2in}An open-ended node primitive that
can be used to represent anything that falls outside of the above node
primitives.


\subsubsection{Summarative link primitives}

\paragraph{Addresses:}

This link specifies a ``carrier'' relationship between an artifact
(\fbox{{\sf source\/}}) and a conceptual theme (\fbox{{\sf problem\/}}). It
consists of one tuple: \( \fbox{{\sf source}}
\stackrel{addresses}{\longrightarrow} \fbox{{\sf problem}} \). For example,

\small{
\begin{itemize}
  
\item {\sf Mill92}: ``Certifying the Correctness of Software'' by Harlen
  Mill.
  
\item {\sf Inadequacy of unit testing}: Unit testing and debugging cannot
  uncover and remove all important errors in complex software system.
  
\item Hence, \hspace{.1in}\fbox{\sf Mill92} \(
  \stackrel{addresses}{\longrightarrow} \) \fbox{\sf Inadequacy of unit
  testing}.
\end{itemize}
}

\paragraph{Responds-to:}

This link describes a knowledge-level ``stimulus/response'' relationship
between two RESRA nodes: the ``stimulus'' is the uncertain, perplexing state
of knowledge about a particular phenomena, while the ``response'' is a
position or assertion that aims at clarifying or resolving that situation.
The tuple is expressed as \( \fbox{{\sf
claim}}\stackrel{responds-to}{\longrightarrow} \fbox{{\sf problem}} \),  For
example,

\small
\begin{itemize}
\item {\sf Inadequacy of unit testing}: Unit testing and debugging cannot
  uncover and remove all important errors in complex software system.
  
\item {\sf Cleanroom reduces defects}: Zero defect software is an
  achievable goal by using rigorous development and formal verification
  techniques from cleanroom engineering.
  
\item Hence, \hspace{.1in}\fbox{{\sf Inadequacy of unit testing}} \(
  \stackrel{responds-to}{\longrightarrow} \) \fbox{{\sf Cleanroom reduces
  defects}}.
\end{itemize}
\normalsize


\paragraph{Suggests:}

This link represents a ``triggering'' relationship, i.e., the presence of
one RESRA instance leads to the recognition or awareness of the other. It
consists of four tuples:

\begin{itemize}
\item \fbox{{\sf claim}} \( \stackrel{suggests}{\longrightarrow} \)
  \fbox{{\sf problem}}
  
\item \fbox{{\sf theory}} \( \stackrel{suggests}{\longrightarrow} \)
  \fbox{{\sf problem}}
  
\item \fbox{{\sf theory}} \( \stackrel{suggests}{\longrightarrow} \)
  \fbox{{\sf claim}}
  
\item \fbox{{\sf evidence}} \( \stackrel{suggests}{\longrightarrow} \)
  \fbox{{\sf problem}}
\end{itemize}

See Section \ref{sec:argumentative primitives} for additional tuples. Below
is an example of {\it suggests\/}:

\small
\begin{itemize}
\item {\sf Cleanroom engineering minimizes defects}: Zero defect
  software is an achievable goal by using rigorous development and formal
  verification techniques from cleanroom engineering.
  
\item {\sf difficulty in achieving zero defect UI:\/} A defect user
  interface is not only difficult to use but also difficult to verify using
  formal techniques.  Hence, cleanroom engineering may not be the solution.
  
\item \fbox{{\sf Cleanroom engineering minimizes defects}}
\( \stackrel{suggests}{\longrightarrow}\)
\fbox{{\sf difficulty in achieving zero defect UI:\/}}
\end{itemize}
\normalsize

\paragraph{Presupposes:}

This link depicts a logical dependency relationship between two claims,
i.e., \fbox{{\sf claim\(_{i} \)}} \(
\stackrel{presupposes}{\longrightarrow} \) \fbox{{\sf claim\(_{j} \)}}. For
example,

\small
\begin{itemize}
\item {\sf Programming as theory-building:\/} Programming is
  ``theory-building'' (Peter Naur).
  
\item {\sf Reconstruction of a programming theory:} Re-establishing the
  theory of a program merely from the documentation is strictly impossible.
  
\item Hence, \hspace{.01in}\( \fbox{{\sf Reconstruction of a programming
  theory\/}} \stackrel{presupposes}{\longrightarrow} \fbox{{\sf Programming
  as theory-building}} \).
\end{itemize}
\normalsize


\paragraph{Is-alternative-to:}

This link describes a {\it competing\/} relationship between two RESRA
nodes of the same type. It consists of four tuples:

\begin{itemize}
\item \fbox{{\sf Problem\(_{i} \)}}
\( \stackrel{is-alternative-to}{\longrightarrow}  \)
\fbox{{\sf problem\(_{j} \)}}

\item \fbox{{\sf Claim\(_{i} \)}}
  \( \stackrel{is-alternative-to}{\longrightarrow} \)
\fbox{{\sf claim\(_{j} \)}}

\item \fbox{{\sf Method\(_{i} \)}}
  \( \stackrel{is-alternative-to}{\longrightarrow} \)
\fbox{{\sf method\(_{j} \)}}

\item \fbox{{\sf Theory\(_{i} \)}}
  \( \stackrel{is-alternative-to}{\longrightarrow} \)
\fbox{{\sf theory\(_{j} \)}}
\end{itemize}

\paragraph{}Below is an example of the alternative claim:

\small
\begin{itemize}
\item {\sf unrecoverable theory}: Re-establishing the theory of a program
  merely from the documentation is strictly impossible.
  
\item {\sf documentationist}: Improved methods of documentation
  are able to communicate everything necessary for the maintenance and
  modification of a program.
  
\item Hence, \hspace{r
example,

\small
\begin{itemize}
\item {\sf Suggestion:} I would like to see cleanroom engineering used in
  some non-conventional domains, such as groupware
  
\item {\sf Claim:} Zero defect software is an achievable goal by using
  rigorous development and formal verification techniques from cleanroom
  engineering.

\item Hence, \fbox{{\sf Suggestion}} \(
  \stackrel{augments}{\longrightarrow} \) \fbox{{\sf claim\/}}
\end{itemize}
\normalsize

\subsection{Argumentative primitives}
\label{sec:argumentative primitives}

RESRA argumentative primitives are used for deliberating alternative
positions and points of view on a particular problem or artifact. One main
feature of CLARE is that it treats artifact-mediated argumentation among a
group of researchers, e.g., debates taking place through a series of
exchanges of artifacts, in the same way as the argumentation taking place
within a classroom among a group of students. It uses RESRA for both
purposes. The argumentative primitives are the aggregation of summarative
and evaluative ones described in Sections \ref{sec:summarative primitives}
and \ref{sec:evaluative primitives}, plus the following tuples:

\begin{itemize}
   \item \fbox{{\sf question}} \( \stackrel{suggests}{\longrightarrow} \)
  \fbox{{\sf problem}}
  
 \item \fbox{{\sf question}} \( \stackrel{suggests}{\longrightarrow} \)
  \fbox{{\sf claim}}

 \item \fbox{{\sf question}} \( \stackrel{suggests}{\longrightarrow} \)
     \fbox{{\sf concept}}

   \item \fbox{{\sf question}} \( \stackrel{suggests}{\longrightarrow} \)
  \fbox{{\sf method}}

\item \fbox{{\sf question}} \( \stackrel{suggests}{\longrightarrow} \)
  \fbox{{\sf theory}}
\end{itemize}


\paragraph{}RESRA argumentative primitives are depicted in Figure
\ref{fig:arg-resra}. The solid lines represent newly-added links.

\begin{figure}[htb]
 \fbox{\centerline{\psfig{figure=Figures/arg-resra.eps,height=4.5in}}}
  \caption{RESRA argumentative primitives at a glance}
  \label{fig:arg-resra}
\end{figure}


\subsection{Integrative primitives}
\label{sec:integrative primitives}

Integrative primitives are used for consolidating and inter-relating RESRA
instances created by different learners. The focus of integration is on the
relationship among existing nodes. Hence, it does not lead to the creation
of new nodes, except for annotations, which document the reasons behind
endorsing a particular relationship. RESRA has four integrative link
primitives, which are described below.

\paragraph{Is-similar-to:}

This link declares the two RESRA node instances share the same view and
thus may be consolidated into one. Automatic merging of two nodes are not
always possible or desirable, since it is rare that two nodes are totally
congruent. The {\it is-similar-to\/} relationship is restricted to the
RESRA nodes of the same type. In other words, one cannot declare that a
\fbox{{\sf concept\/}} {\it is-similar-to\/} a \fbox{{\sf claim}}, or vice
versa.  However, one can declare one \fbox{{\sf concept\/}} as {\it
is-similar-to\/} another \fbox{{\sf concept\/}}. From one of the
previous examples, one may declare:

\indent \fbox{{\sf Box structured design}} \(
  \stackrel{is-similar-to}{\longrightarrow} \) \fbox{{\sf OO design}}

\paragraph{Share-same-perspective:}
  
A perspective represents a consistent way of viewing a problem or
phenomena. Learners can share the same perspective, even though the
detailed node and link instances they create are different. For example,
important views on a given software project might include:

\begin{itemize}
\item {\it Customer's/user's perspective:\/} Ensuring their needs for a new
  system being met;
  
\item {\it Management perspective:\/} Focusing on project management
  issues, including cost, deadline, employee morales, etc.;
  
\item {\it Designer's perspective:\/} Ensuring that the requirement
  specification is realized through optimal algorithms and high reliability;
  and
  
\item {\it Programmer's perspective:\/} Ensuring that the design is
  implemented consistently, efficiently, and readably (through proper
  documentation).
\end{itemize}

A RESRA node instance on an experience report of a software project is
likely to fall into one of the above perspectives. By aggregating RESRA
nodes by perspectives, it introduces necessary structures that help make
the group outcome more understandable.


\paragraph{Contains:}

This link represents a part-whole relationship between two RESRA nodes.
Typically, the two connected nodes should be the same type, though it is
not alway necessary. For example, CLARE is a {\sf thing\/}, while {\sf
RESRA\/} is a {\sf concept\/}. Nevertheless, one can declare \fbox{{\sf
CLARE}} \( \stackrel{contains}{\longrightarrow} \) \fbox{{\sf RESRA}}

\paragraph{Is-related-to:}

This link is open-ended, and can be used to express any type of
relationship that falls outside the above categories. As a result, it
should be used only when no more specific type is applicable.



\section{Canonical RESRA Forms (CRFs)}
\label{sec:crf}

\subsection{Overview}

The Canonical RESRA Form (CRF) is a {\it template\/} of RESRA node and link
primitives used to represent the typical thematic structure of scientific
text.  It represents commonly accepted formats by which the practitioners
in a given domain formally communicate and share knowledge. CRFs may be
viewed in terms of Kuhnian notion of the {\it paradigm\/}) of the
scientific practice. As the newcomer to a field, the student may find it
helpful to learn about the major CRFs used in that domain, and to be able
to recognize them in their various disguised forms, apply them in
evaluating existing artifacts and in constructing their own artifacts, and
use them to help organize domain knowledge and collaborate with their
peers.

Because of the unique nature of the problems in each subject domain and the
concomitant methodological differences, CRFs often vary from one field to
another.  For example, software engineering, which is a relatively young
field, exhibits a variety of relatively ill-defined templates.  In
contrast, psychology, perhaps one of the most well-developed social science
disciplines, contains a much smaller number of less varied, more formalized
templates. This document provides a set of templates derived from the
domains of software engineering, in particular, software quality
assurance (SQA). It is important to note that, like RESRA, CRFs presented
below are exemplary rather than exhaustive. The learners are encouraged to
use these examples as a basis for creating their own domain-specific CRFs.


\subsection{Five example CRFs}

Each of the following CRFs consists of a short description, a graphical
representation, and an example to illustrate how the CRF might be applied
to an artifact.


\subsubsection{Concept papers}

\paragraph{Description:}

A {\it concept paper\/} presents a new conceptualization of a problem, or a
new method, technique, or approach to addressing an existing problem.  In a
concept paper, the author might back up his claims by some type of
evidence, such as contrived examples, experiential or empirical data, and
so on. However, the main contribution of such a paper does not lie in the
strength of its supporting data but rather the demonstrated novelty and
potentials of the proposed technique, concept, etc.

Figure \ref{fig:concept-crf} shows the CRF for concept papers. At the core
of a concept paper lies the {\it claim\/} about how the new {\it method\/}
and {\it concept\/} described in the paper help solve the {\it problem\/}
conceived by the author. Even though some {\it evidence\/} might be
provided to support the claims, the major portion of the paper is typically
devoted to the definition of new concepts, and the elaboration of the
method and the relationship between the two. The claim might be broken down
into sub-claims that are treated separately.

\begin{figure}[htb]
 \fbox{\centerline{\psfig{figure=Figures/concept-crf.eps}}}
  \caption{A CRF for concept papers}
  \label{fig:concept-crf}
\end{figure}


\paragraph{An example:}

A good example of a concept paper is \cite{csdl-92-07}, whose RESRA node
representation is listed below. Figurr
example,

\small
\begin{itemize}
\item {\sf Suggestion:} I would like to see cleanroom engineering used in
  some non-conventional domains, such as groupware
  
\item {\sf Claim:} Zero defect software is an achievable goal by using
  rigorous development and formal verification techniques from cleanroom
  engineering.

\item Hence, \fbox{{\sf Suggestion}} \(
  \stackrel{augments}{\longrightarrow} \) \fbox{{\sf claim\/}}
\end{itemize}
\normalsize

\subsection{Argumentative primitives}
\label{sec:argumentative primitives}

RESRA argumentative primitives are used for deliberating alternative
positions and points of view on a particular problem or artifact. One main
feature of CLARE is that it treats artifact-mediated argumentation among a
group of researchers, e.g., debates taking place through a series of
exchanges of artifacts, in the same way as the argumentation taking place
within a classroom among a group of students. It uses RESRA for both
purposes. The argumentative primitives are the aggregation of summarative
and evaluative ones described in Sections \ref{sec:summarative primitives}
and \ref{sec:evaluative primitives}, plus the following tuples:

\begin{itemize}
   \item \fbox{{\sf question}} \( \stackrel{suggests}{\longrightarrow} \)
  \fbox{{\sf problem}}
  
 \item \fbox{{\sf question}} \( \stackrel{suggests}{\longrightarrow} \)
  \fbox{{\sf claim}}

 \item \fbox{{\sf question}} \( \stackrel{suggests}{\longrightarrow} \)
     \fbox{{\sf concept}}

   \item \fbox{{\sf question}} \( \stackrel{suggests}{\longrightarrow} \)
  \fbox{{\sf method}}

\item \fbox{{\sf question}} \( \stackrel{suggests}{\longrightarrow} \)
  \fbox{{\sf theory}}
\end{itemize}


\paragraph{}RESRA argumentative primitives are depicted in Figure
\ref{fig:arg-resra}. The solid lines represent newly-added links.

\begin{figure}[htb]
 \fbox{\centerline{\psfig{figure=Figures/arg-resra.eps,height=4.5in}}}
  \caption{RESRA argumentative primitives at a glance}
  \label{fig:arg-resra}
\end{figure}


\subsection{Integrative primitives}
\label{sec:integrative primitives}

Integrative primitives are used for consolidating and inter-relating RESRA
instances created by different learners. The focus of integration is on the
relationship among existing nodes. Hence, it does not lead to the creation
of new nodes, except for annotations, which document the reasons behind
endorsing a particular relationship. RESRA has four integrative link
primitives, which are described below.

\paragraph{Is-similar-to:}

This link declares the two RESRA node instances share the same view and
thus may be consolidated into one. Automatic merging of two nodes are not
always possible or desirable, since it is rare that two nodes are totally
congruent. The {\it is-similar-to\/} relationship is restricted to the
RESRA nodes of the same type. In other words, one cannot declare that a
\fbox{{\sf concept\/}} {\it is-similar-to\/} a \fbox{{\sf claim}}, or vice
versa.  However, one can declare one \fbox{{\sf concept\/}} as {\it
is-similar-to\/} another \fbox{{\sf concept\/}}. From one of the
previous examples, one may declare:

\indent \fbox{{\sf Box structured design}} \(
  \stackrel{is-similar-to}{\longrightarrow} \) \fbox{{\sf OO design}}

\paragraph{Share-same-perspective:}
  
A perspective represents a consistent way of viewing a problem or
phenomena. Learners can share the same perspective, even though the
detailed node and link instances they create are different. For example,
important views on a given software project might include:

\begin{itemize}
\item {\it Customer's/user's perspective:\/} Ensuring their needs for a new
  system being met;
  
\item {\it Management perspective:\/} Focusing on project management
  issues, including cost, deadline, employee morales, etc.;
  
\item {\it Designer's perspective:\/} Ensuring that the requirement
  specification is realized through optimal algorithms and high reliability;
  and
  
\item {\it Programmer's perspective:\/} Ensuring that the design is
  implemented consistently, efficiently, and readably (through proper
  documentation).
\end{itemize}

A RESRA node instance on an experience report of a software project is
likely to fall into one of the above perspectives. By aggregating RESRA
nodes by perspectives, it introduces necessary structures that help make
the group outcome more understandable.


\paragraph{Contains:}

This link represents a part-whole relationship between two RESRA nodes.
Typically, the two connected nodes should be the same type, though it is
not alway necessary. For example, CLARE is a {\sf thing\/}, while {\sf
RESRA\/} is a {\sf concept\/}. Nevertheless, one can declare \fbox{{\sf
CLARE}} \( \stackrel{contains}{\longrightarrow} \) \fbox{{\sf RESRA}}

\paragraph{Is-related-to:}

This link is open-ended, and can be used to express any type of
relationship that falls outside the above categories. As a result, it
should be used only when no more specific type is applicable.



\section{Canonical RESRA Forms (CRFs)}
\label{sec:crf}

\subsection{Overview}

The Canonical RESRA Form (CRF) is a {\it template\/} of RESRA node and link
primitives used to represent the typical thematic structure of scientific
text.  It represents commonly accepted formats by which the practitioners
in a given domain formally communicate and share knowledge. CRFs may be
viewed in terms of Kuhnian notion of the {\it paradigm\/}) of the
scientific practice. As the newcomer to a field, the student may find it
helpful to learn about the major CRFs used in that domain, and to be able
to recognize them in their various disguised forms, apply them in
evaluating existing artifacts and in constructing their own artifacts, and
use them to help organize domain knowledge and collaborate with their
peers.

Because of the unique nature of the problems in each subject domain and the
concomitant methodological differences, CRFs often vary from one field to
another.  For example, software engineering, which is a relatively young
field, exhibits a variety of relatively ill-defined templates.  In
contrast, psychology, perhaps one of the most well-developed social science
disciplines, contains a much smaller number of less varied, more formalized
templates. This document provides a set of templates derived from the
domains of software engineering, in particular, software quality
assurance (SQA). It is important to note that, like RESRA, CRFs presented
below are exemplary rather than exhaustive. The learners are encouraged to
use these examples as a basis for creating their own domain-specific CRFs.


\subsection{Five example CRFs}

Each of the following CRFs consists of a short description, a graphical
representation, and an example to illustrate how the CRF might be applied
to an artifact.


\subsubsection{Concept papers}

\paragraph{Description:}

A {\it concept paper\/} presents a new conceptualization of a problem, or a
new method, technique, or approach to addressing an existing problem.  In a
concept paper, the author might back up his claims by some type of
evidence, such as contrived examples, experiential or empirical data, and
so on. However, the main contribution of such a paper does not lie in the
strength of its supporting data but rather the demonstrated novelty and
potentials of the proposed technique, concept, etc.

Figure \ref{fig:concept-crf} shows the CRF for concept papers. At the core
of a concept paper lies the {\it claim\/} about how the new {\it method\/}
and {\it concept\/} described in the paper help solve the {\it problem\/}
conceived by the author. Even though some {\it evidence\/} might be
provided to support the claims, the major portion of the paper is typically
devoted to the definition of new concepts, and the elaboration of the
method and the relationship between the two. The claim might be broken down
into sub-claims that are treated separately.

\begin{figure}[htb]
 \fbox{\centerline{\psfig{figure=Figures/concept-crf.eps}}}
  \caption{A CRF for concept papers}
  \label{fig:concept-crf}
\end{figure}


\paragraph{An example:}

A good example of a concept paper is \cite{csdl-92-07}, whose RESRA node
representation is listed below. Figure \ref{fig:johnson} provides a
complete RESRA representation.  Note that the nbuff review example is used
to illustrate various aspects of the method, although it might also be
considered as the evidence for supporting the authors' claims.  The main
contribution of the paper, however, does not lie in the supporting
evidence, which is quite weak, but rather in the way in which the problem
is presented and the way in which the method is related to the problem.

\small
\begin{quotation}
  \noindent {\sf Problem}: There exist three main road-blocks to fully
  effective formal technical review: labor-intensive nature of review,
  incompatibility with incremental development methods, and lack of
  support by existing tools for adapting review methods to specific
  organizational contexts.
  
  \noindent {\sf Claim}: CSRS provides an effective means to overcome the
  above three barriers and to realize full potentials of FTR.
  
  \noindent {\sf Method}: The CSRS data and process models.

  \noindent {\sf Evidence}: Review of NBUFF using CSRS.
\end{quotation}
\normalsize

\begin{figure}[htb]
 \fbox{\centerline{\psfig{figure=Figures/johnson.eps}}}
  \caption{A RESRA representation of [JT92]} 
  \label{fig:johnson}
\end{figure}


\subsubsection{Experience Papers}

\paragraph{Description:}

An {\it experience paper\/} is a factual account of the experience of an
individual, group, or organization with a new method, technology,
instrument, theory, etc. An experience paper does not have the same level
of rigor as an empirical study or novelty as a concept paper. The important
contribution of an experience paper lies in the fact that it is from the
real world. Since any model, theory, claim, or technique requires some
level of generalization, it needs change when applied back to a specific
real world situation. The experience paper is often used to report such
experience.

Figure \ref{fig:experience-crf} shows a CRF for experience papers. Such a
paper typically begins with a problem confronted by an individual, group,
or organization. Based upon previous work or the experience from other
people, a claim is made that a given method/technique should be used to
solve the problem. The method is then applied and the experience is
described. Like empirical studies, the outcome from an experience paper may
be either positive or negative; both are equally valuable. In the case of
failure, what is important is the discussion of the possible cause of
failure and lessons learned.

\begin{figure}[htb]
 \fbox{\centerline{\psfig{figure=Figures/experience-crf.eps}}}
  \caption{A CRF for experience papers} 
  \label{fig:experience-crf}
\end{figure}


\paragraph{An example:}

A good example of the experience paper is \cite{Bush90}, which discusses
the experience of introducing formal software review to JPL. The RESRA node
instances for this paper are listed below. The relationships between those
nodes are depicted in Figure \ref{fig:bush}.

\small
\begin{quotation}
  \noindent {\sf Problem}: Software has become a rising part of the JPL
  work hours (from 50\% of the mid-1980's to estimated 80\% by the year
  0\% 2000) and, as a result, so is the importance of being able to produce
  error-free systems in a cost-effective manner.
  
  \noindent \( {\sf Claim_{1}} \): Fagan reports that the use of the
  Fagan's Inspections can correctly find up to 95\% of all defects before
  entering the test phase.
  
  \noindent \( {\sf Claim_{2}} \): Fagan's method is the most
  cost-effective defect-detection technique appropriate to the JPL.
  
  \noindent {\sf Method}: JPL adopted the 7-step Fagan's Inspection method
  with checklist of tailored questions. Extensive training were provided to
  both managers (in the value of inspections) and developers (to get most out
  of inspections).

  \noindent {\sf Evidence}: Within the period of 21 months, 300 inspections
  have been conducted; 10 projects have adopted the method as part of their
  procedures. The number of defects per inspection were 4 major defects and
  12 minor ones. The average cost for finding, fixing, and verifying a
  defect is between \$9 and \$12 , compared to about \$10, 000 to find and
  fix the same defect in the later life cycle.
\end{quotation}
\normalsize

\begin{figure}[htb]
\fbox{\centerline{\psfig{figure=Figures/bush.eps}}}
  \caption{A RESRA representation of [Bush, 90]}
  \label{fig:bush}
\end{figure}


\subsubsection{Empirical papers}

\paragraph{Description:}

An {\it empirical paper} reports the result of a hypothesis-driven,
systematic investigation of particular problem domain. It involves
experimentation, i.e., hypothesis formulation, experimental design,
execution, data analysis, and drawing conclusions. Unlike the concept
paper, which introduces new methods or new formulation of a problem,
empirical studies often attempt to demonstrate whether an existing method
or understanding of a problem is indeed well-grounded. Hence, the empirical
study starts at where the concept paper leaves off. Empirical studies
constitute a major percentage of artifacts in established disciplines.
This is not surprising given the fact that a mature field typically have
methods and problems defined.

Figure \ref{fig:empirical-crf} shows CRF for empirical papers. The
beginning and end point of an empirical study is the claim, or hypothesis,
which is either derived from a theory in a field or based on the claims
made in previous work. The key to an empirical study is the lower triangle,
i.e., claim, method, and evidence. The methodology plays a vital role in
this process because it directly affects the validity and generalizability
of the data and the outcomes. The resulting data may or may not support the
initial hypothesis, however. This is quite different from the concept paper
in which only positive data is reported.

\begin{figure}[htb]
 \fbox{\centerline{\psfig{figure=Figures/empirical-crf.eps}}}
  \caption{A CRF for empirical papers}
  \label{fig:empirical-crf}
\end{figure}


\paragraph{An example:}

\cite{Curtis79} is a typical empirical paper, which reports the result of
the third experiment attempting to relate the complexity metrics developed
by Halstead and McCabe to the difficulty programmers experience in
understanding and modifying programs. The RESRA node instances are shown
below, and the relationships between them are depicted in Figure
\ref{fig:curtis}. Note that the RESRA representation of this paper matches
only with the lower triangle in Figure \ref{fig:empirical-crf}.

\small
\begin{quotation}
  \noindent {\sf Problem}: Software complexity metrics are abound.
  However, studies thus far are still inconclusive about which metric is
  the best predictor of programmer performance.
  
  \noindent {\sf Claim\/}: Software complexity metrics that count
  operators, operands, and elementary control flows are better predictors
  of the difficulty programmers experience in working with software than a
  simple count of LOC.
  
  \noindent {\sf Method}: Fifty-four professional programmers from six
  different locations participated in the experiment that involved three
  different programs, each with three different versions of control flow,
  and they were presented in three different length varying from 25 to 225
  lines of code. To control for individual difference in performance, a
  within-subject \( 3^4 \) factorial design was employed.
  
  \noindent {\sf Evidence}: The result indicated that, at the subroutine
  level, all three complexity metrics (i.e., Halstead's E, McCabe's v(G),
  and LOC) predicted the performance equally well, accounting for 40-45\%
  of the variance in performance scores. At the program level, however,
  Halstead's E accounted for over twice as much variance in performance as
  the LOC (56\% vs. 27\% respectively) while the variance accounted for by
  McCabe's v(G) fell between these values (42\%). 
\end{quotation}
\normalsize

\begin{figure}[htb]
 \fbox{\centerline{\psfig{figure=Figures/curtis.eps}}}
  \caption{An example RESRA representation of [CSM79]}
  \label{fig:curtis}
\end{figure}


\subsubsection{Essays}

\paragraph{Description:}

The {\it essay} is a loosely-defined artifact, sometimes called the
``opinion paper,'' which contains observations and elucidation of a
particular problem or issue. Often, essays are written by experts in the
field; their opinions and points of view represent insights distilled from
their years of experience. Though methodologically they are not as
{\it scientific\/} and rigorous, essays are an important part of the
intellectual landscape in many disciplines, especially in applied ones such
as software engineering, in which expertise plays a just as important, if
not more important, role as the absolute truth.

Essays vary widely in styles. At the thematic level, however, their
structures are quite singular, generalizable into some standard forms.
Figure \ref{fig:expository} shows one RESRA template of the essay paper.
The center of such an artifact is often the expert's opinions or claims.
The problem at which the claim is targeted may or may not be explicitly
described. Antithetical claims are typically identified and countered by
the same evidence that supports the author's particular claim.  Evidence in
an essay is typically anecdotal, drawing heavily from the author's
experience and observations, though it is not uncommon to see secondary
evidence from other authors in the field.  It is also uncommon that essays
introduce new methods or concepts.

\begin{figure}[htb]
 \fbox{\centerline{\psfig{figure=Figures/expository-crf.eps}}}
  \caption{A CRF for research essays}
  \label{fig:expository}
\end{figure}

\paragraph{An example :}

\cite{Knuth92} is a good example of essays, whose RESRA representation is
listed below. Figure \ref{fig:knuth} shows the relationships among the
various nodes. Note that this paper as whole does not neatly fits to the
template depicted in Figure \ref{fig:expository}, since it contains two
groups of claims: primary (i.e., {\sf claim-1\/} and {\sf evidence-1\/} and
secondary (i.e., the rest)). However, the latter matches with the template.
Such deviations are not uncommon in essays, for their structural forms are
much less restrictive compared to other artifact types.

\small
\begin{quotation}
  \noindent \( {\sf Claim_{1}} \): Learning through trial-and-errors can
  be enhanced by maintaining a record of the mistakes one has made.
  
  \noindent\( {\sf Evidence_{1}} \): The author's finds his error log of the
  \TeX\ system, which contains 867 entries, so instructive that he publishes
  it so that other people can benefit from his experience.

  \noindent\( {\sf Claim_{2}\/} \): Programming is ``theory-building'' (from Peter Naur).
  
  \noindent\( {\sf Claim_{3}} \): Re-establishing the theory of a program merely from
  the documentation is strictly impossible;
  
  \noindent\( {\sf Claim_{4}} \): Improved methods of documentation are able to
  communicate everything necessary for the maintenance and modification of
  a program.
  
  \noindent\( {\sf Evidence_{2}} \): Hundreds of \TeX\/ users around the
  world have demonstrated they understand the ``theory'' of the \TeX\/
  program through making special-purpose extensions to the existing code
  and by offering highly appropriate advice to users, even though they
  have only read its documentation.
\end{quotation}
\normalsize

\begin{figure}[htb]
 \fbox{\centerline{\psfig{figure=Figures/knuth.eps,width=4.5in}}}
  \caption{A RESRA representation of [Knuth92]}
  \label{fig:knuth}
\end{figure}


\subsubsection{Survey papers}

\paragraph{Description:}

A {\it survey\/} is an overview of existing work in a particular topical
area or subject domain. It highlights what has been accomplished thus far,
what major problems have been encountered, and where future efforts should
be devoted to. A survey represents a {\it synthesis\/} of previous work.
The main contribution of a survey paper is that it summarizes the current
state of a given topic, and provides a single-point entry for the
beginners of the field.

Figure \ref{fig:survey-crf} shows a CRF for survey papers. The focal
point of a survey is the problem under concern. It brings together all
related work, some of which may involve different claims, while others
share the same claim but provide different evidence. At the end, the author
normally attempts to summarize the above by offering his own concluding
claims, in particular, about the direction in which the domain is heading.

\begin{figure}[htb]
  \fbox{\centerline{\psfig{figure=Figures/survey-crf.eps}}}
  \caption{A CRF for survey papers}
  \label{fig:survey-crf}
\end{figure}


\paragraph{An example:}

An example survey paper is \cite{Tichy92}. The RESRA node
instances for the paper are listed below, and the relationships between
those nodes are depicted in Figure \ref{fig:tichy}. Note that this paper is
not very representative in that it covers an incremental but nevertheless
singular view of the selected topic, i.e., the evolving concept of
the ``programming-in-the-large,'' instead of multiple, competing views of
the same topic, as one expects from a typical survey.

\small
\begin{quotation}
  \noindent {\sf Problem}: While programming-in-the-small is a well
  established discipline, programming-in-the-large, albeit no less
  important, is a much less developed subject area.
  
  \noindent \( {\sf Source_{1}} \): Parmas, David. ``Designing software for
  ease of extension and contraction.'' {\it CACM\/}, 15(2):1053-8, Dec'72.

  \noindent \( {\sf Source_{2}} \): DeRemer, Frank and Hans, Kron.
  ``Programming-in-the-large versus programming-in-the-small.'' {\it
  TSE\/}, SE-2(2):80-6, Jun'76.
  
  \noindent \( {\sf Source_{3}} \): Tichy, Walter. ``Tools for software
  configuration management.'' {\it In Proc. of Int. Conference on Software
  Version and Configuration Control\/}, Jan'88.
  
  \noindent\( {\sf Claim_{1}\/} \): Parmas states that details that are
  likely to change during the evolution of a system should be made into
  separate modules, while the interface of those modules should be made
  insensitive to changes in these details (i.e., ``information hiding.'')
  
  \noindent\( {\sf Claim_{2}} \): DeRemer and Kron claims that
  structuring a large collection of modules to form a ``system'' is an
  essentially distinct intellectual activity from that of constructing
  the individual modules (i.e., programming-in-the-small) and,
  correspondingly, distinct languages should be used for the two
  activities.
  
  \noindent\( {\sf Claim_{3}} \): Software configuration management (SCM)
  improves programming-in-the-large by providing automated tool for such
  tasks as version control, configuration selection, and system building.
  
  \noindent {\sf Evidence}: Richkind's {\sf SCCS\/} and Tichy's {\sf
  RCS\/} for automated revision control; Feldman's {\sf make\/} for
  maintaining, updating, and generating related programs.
  
  \noindent\( {\sf Claim_{4}} \): The three promising measures for
  further improve quality and productivity in programming-in-the-large
  are: better education of software engineers; better tools; and avoiding
  programming or code reuse.
\end{quotation}
\normalsize

\begin{figure}[htb]
 \fbox{\centerline{\psfig{figure=Figures/tichy.eps}}}
  \caption{A RESRA representation of [Tichy, 92]}
  \label{fig:tichy}
\end{figure}


\section{Extending RESRA}
\label{sec:extensions}

RESRA is designed to be an open-ended language. Regardless of the size of
the initial primitive and template set, RESRA is incomplete with respect to
the universe of artifacts it is intended to represent. The goal is not,
however, to provide a comprehensive, encyclopedic set of meta constructs
for human knowledge, if such attempt is plausible.  Instead, constructing
such a {\it thorough\/} set of RESRA constructs is an ongoing activity that
can be most profitably done by the learners themselves, rather than by a
third party. The primary objective of this research is to provide a
representation whose predefined constructs are:

\begin{itemize}
\item Coherent, self-contained, and powerful enough that can be used to
  adequately represent both the thematic feature of the learning artifact
  and different learners' points of view; and
  
\item A basis from which any group of learners may collaboratively derive
  or adapt their own domain-specific or even group-specific
  representations.  From this perspective, RESRA is always {\it
  illustrative\/}, which provides the learner with a sense of what to do
  next.
\end{itemize}

Learning to extend RESRA by creating new primitive types and canonical
forms, or adapting existing ones is a significant, albeit not always easy,
meta-level activity which CLARE aims at promoting. In fact, exploring the
structural evolution of a representation is one of our original primary
research interests.  The underlying platform, i.e., EGRET, is designed to
support such type-level changes \cite{csdl-91-03}.


%%% \newpage
%%% \singlespace
%%% \bibliography{../bib/clare,../bib/csdl-trs}
%%% \bibliographystyle{alpha}
%%% 
%%% 
%%% \end{document}






