\batchmode
%\documentstyle [nftimes,cscw] {article}






\def\PsfigVersion{1.9}

\makeatother
\newenvironment{tex2html_wrap}{}{}
\newwrite\lthtmlwrite
\def\lthtmltypeout#1{{\let\protect\string\immediate\write\lthtmlwrite{#1}}}%
\newbox\sizebox
\begin{document}
\pagestyle{empty}
\stepcounter{section}
\stepcounter{section}
\stepcounter{subsection}
{\newpage
\clearpage
\samepage \( \stackrel{responds-to}{\longrightarrow}
\)
}

{\newpage
\clearpage
\samepage \(\stackrel{supports}{\longrightarrow}\)
}

{\newpage
\clearpage
\samepage \begin{figure}[htb]
 \centerline{\psfig{figure=Figures/experience-crf.eps,width=3.0in}}
   
  \label{fig:experience-crf}
\end{figure}
}

{\newpage
\clearpage
\samepage \begin{figure*}% latex2html id marker 54
[htb]
  \begin{center}
    

    \begin{tabular} {|l|p{2.5in}|p{2.5in}|} \hline   
      {\bf Node Type} & {\bf Description} & {\bf Example} \\  \hline \hline
      
      Problem & A phenomenon, event, or process whose understanding
      requires further inquiry. & Meta-learning is not adequately
      supported by existing tools. \\  \hline
      
      Claim & A position or proposition about a given problem
      situation.  & Cleanroom engineering provides a viable solution
      in producing zero defect software. \\  \hline
      
      Evidence & Data gathered for the purpose of supporting or
      refuting a given claim. & The use of cleanroom techniques led
      to a 5-fold reduction of defects in project Alpha. \\ 
      \hline 

      Theory & A systemic formulation about a particular problem
      domain, derivable through deductive or inductive procedures. &
      Ausubel's theory of meaningful learning. \\  \hline
      
      Method & Procedures or techniques used to generate evidence for
      a particular claim. & Delphi study; nominal grouping technique;
      waterfall software development model. \\  \hline
      
      Concept & A primitive construct used in formulating theory,
      claim, or method. & Meta-learning; Knowledge representation.
      \\  \hline
      
      Thing & A natural or man-made object that is under study.  &
      Rock; Intermedia.  \\  \hline
      
      Source & An identifiable written artifact, either artifact
      itself or a reference to it. & An
      article by Ashton; the notes from Kyle's talk. \\  \hline \hline
      
      Critique & Critical remarks or comments about a particular
      claim, evidence, method, source, et al., or relationships
      between them. & Applications of cleanroom
      engineering appear limited to domains with well-defined requirements.
      \\  \hline
      
      Question & Aspects of a claim, theory, concept, etc., about
      which the learner is still in doubt. & How does box-structured
      design differ from object-oriented design? \\  \hline
      
      Suggestion & Ideas, recommendations, or feedback on how to
      improve an existing problem statement, claim, method, et al.  &
      I would like to see cleanroom engineering used in some
      non-conventional domains, such as groupware. \\  \hline
    \end{tabular}
    \captiontwocolumns{A synopsis of RESRA node primitives.}
    \label{tab:resra}
      \end{center}
\end{figure*}
}

{\newpage
\clearpage
\samepage \begin{figure*}% latex2html id marker 67
[htb]
 \centerline{\psfig{figure=Figures/sum-resra.eps,height=4.0in}}
  \captiontwocolumns{A graphical illustration of RESRA summarative node and link
  primitives and the relationships between them.}
  \label{fig:sum-resra}
\end{figure*}
}

{\newpage
\clearpage
\samepage \begin{figure*}% latex2html id marker 82
[htb]
 \centerline{\psfig{figure=Figures/fagan.eps,width=5.0in}}
 \captiontwocolumns{An expert's RESRA representation of Fagan's paper on code inspection.}
  \label{fig:fagan}
\end{figure*}
}

\stepcounter{subsection}
{\newpage
\clearpage
\samepage \begin{figure*}% latex2html id marker 116
[htb]
  \centerline{\psfig{figure=Figures/secai.eps,width=5.0in}}
  \captiontwocolumns{The SECAI process model for collaborative learning from scientific text.}
  \label{fig:secai}
\end{figure*}
}

\stepcounter{subsection}
{\newpage
\clearpage
\samepage \begin{figure*}% latex2html id marker 136
[htb]
  \centerline{\psfig{figure=Figures/explore.eps,width=6.0in}}
  \captiontwocolumns{A user view of CLARE during the exploration phase.  The left
  hand window contains a portion of the artifact under study.  The lower
  right window contains an evidence node created by the learner.  The upper
  right window summarizes what the learner has created during exploration
  thus far.}
  \label{fig:explore}
\end{figure*}
}

{\newpage
\clearpage
\samepage \begin{figure*}% latex2html id marker 155
[htb]
  \centerline{\psfig{figure=Figures/consolidate.eps,width=6.0in}}
  \captiontwocolumns{A user view of CLARE during the consolidation phase.  The upper
  left window contains a comparative summary of the problems identified
  by each learner in the scientific artifact.  One of the actual problem
  node instances is displayed in the lower right hand window.  The lower
  left window displays the portion of the scientific text from which this
  problem was derived. The upper right window contains a summary
  of the activities of each learner.}
  \label{fig:consolidate}
\end{figure*}
}

\stepcounter{section}
\stepcounter{subsection}
{\newpage
\clearpage
\samepage \begin{figure}[htb]
  \begin{center}
  \begin{tabular} {|c|r|r|r|r|} \hline   
    Exp. & Logins & Time (hrs)& Nodes & Size (Kb) \\ 
    \hline \hline {\bf 1a.}  & 120 & 82.85 & 472 & 90.02 \\  \hline {\bf
    1b.}  & 115 & 67.90 & 513 & 107.97 \\  \hline {\bf 1c.}  & 84 &
    53.68 & 440 & 105.16 \\  \hline {\bf 2a.}  & 85 & 54.42 & 162 &
    39.42 \\  \hline {\bf 2b.}  & 53 & 37.55 & 207 & 49.67 \\  \hline
    \hline {\bf Total} & 457 & 296.40 & 1794 & 392.24 \\  \hline
   \end{tabular}
  \end{center}
    
   \label{tab:summary-stat}
\end{figure}
}

\stepcounter{subsection}
\stepcounter{subsubsection}
\stepcounter{subsubsection}
{\newpage
\clearpage
\samepage \begin{figure*}% latex2html id marker 207
[hbtp]
 \centerline{\psfig{figure=Figures/rep-all.eps,height=3.9in}}
 \captiontwocolumns{An example collaborative representation network by four CLARE users} 
  \label{fig:arg-example}
\end{figure*}
}

\stepcounter{subsection}
\stepcounter{section}
\stepcounter{section}
\stepcounter{section}
\stepcounter{section}

\end{document}
