% file name:  my.tex
%
\documentstyle[times,cscw,epsf]{article}

\begin{document}

\title{CSCW'94 Conference Papers Guidelines}
\author{{\authorfont John Doe}  \\ 
Blocks Corporation \\
3333 Meadow Rd. \\
Suite 4200 \\
Plato Alama, TX 92564, USA \\
Tel: 1-301-403-4333 \\
E-mail: doe@blocks.com 
\and
{\authorfont Marie Dupont} \\
BLA \\
43 rue des Jardins \\
99001, Menthe la Jolie, France \\
Tel: 33-1-63-34-34-55 \\
E-mail: Dupont@bla.bla.fr
}
\date{}
\maketitle

\begin{abstract}
(would be placed here)	\\
All body text in Times Roman 10 point.

NOTE: the directions in the Call for Participation call for 12 point
Times font.  Please use 10 point Times Roman font.
\end{abstract}
% the following command \copyrightspace has to be put here.
\copyrightspace
\keywords Guides, instructions, etc.

\section{INTRODUCTION}
These instructions are for formatting papers for CSCW'94.  We are
striving to give the book a single, high quality appearance.
To do this, authors must follow some simple guidelines.

In essence, we ask you to make your paper look exactly like this document.
You should match the type style, type size, line spacing, indentation,
and layout format as closely as you can.  In fact, if you received this
document online, you can use it as template.

Use an A4 or 8.5"x11" sheet of paper. Center the image on the page.
The whole image of your text should absolutely fit in a 17.8 cm x 24
cm box (7" by 9 1/4"). Your submitted material will be
photographed 1-to-1 (no reduction) for printing. We have included
here recommendations to help you match this sample with the
facilities you might have, such as Script, Microsoft Word on
Macintosh or PC, MacWrite, or an impact printer.

\section{TITLE AND AUTHORS}
The title, author's names and affiliations run across the full width of
the page.  We also recommend phone number and e-mail address, if
available.  (See the top of this page for two names with different addresses.
If only one address needed, center all text in the page) \\
$\bullet$ Title area: 1 column, 5.9 cm (2 1/3") length, 17.8 cm (7") width. \\
$\bullet$ Title: bold 18 point Helvetica - mixed cases \\
$\bullet$ Names: italic 12 point times roman \\
$\bullet$ Addresses, Tel, E-Mail:  plain 12 point times roman

Do not say ``first contact'' in the papers or summaries, only on the
cover sheet.

\section{ABSTRACT}
Each paper should begin with an abstract, followed by a set of
keywords, both placed in the left column of the first page under the
left half of the title.  
Abstracts should be no more than 100 words in length.

\section{FIRST PAGE COPYRIGHT NOTICE}
Remember to leave at least 2.5 cm (1") of blank space at the bottom
of  the  left column  of  the first  page, as on this page. You must
leave this space for the copyright notice on ALL submissions
intended for publication in the proceedings or companion (i.e. papers
as well as the one or two page summaries).  Please note that all the
authors of all accepted papers and summaries will have to sign a
copyright release form.  Those forms will be sent with the acceptance
letters and need to be returned rapidly.  We encourage the contact
persons of each submission to keep track of their co-authors'
locations because they will be responsible for rapidly collecting the
signatures.

\section{NUMBER OF PAGES}
The maximum length for papers is 12 pages.

\section{TWO COLUMNS}
After the title use a double-column format as shown here. Column
width is 8.5 cm, with 0.8 cm between columns (for a total image
width still equal to 17.8 cm).  Total text length should remain
between 23.2 and 24 cm (9 1/4") Right margins should be justified,
not ragged.   Separate each paragraph by a blank line (and do not
indent them)  Hyphenation is at your own discretion.  The two
columns of the last page should be of equal length.

\section{SECTIONS}
The title of a section should be in Helvetica 9 point bold in all
capitals.  Notice that the sections are not numbered!

\subsection{Subsections}
The title of subsections should be in Helvetica 9 point bold with only
the initial letters capitalized. (NOTE: Words like "the" and "a" are not
capitalized.)

\subsubsection{Subsubsections.} 
The heading for subsubsections should be in
Helvetica 9 point italics with initial letters capitalized. (Note: Words
like "the" and "a" are not capitalized.)  Here the heading is not
followed by a return.

\section{TYPESET TEXT}
Papers should be prepared on a typesetter or word processor.  Please
do not use your favorite obscure font.  We want to produce a book
that looks like a book, not like many dissimilar papers thrown
together.  Please use a 10 point Times Roman font, or other Roman
font with serifs, as close as possible in appearance to Times Roman in
which these guidelines have been set.  The target is to have a 10
point type set on an 11 point line, as you see here.  Do not use a
sans-serif font (e.g., {\helveticafont Helvetica}) 
except for emphasis, headings and the
title.  The Press 10 point font available to users of Script is a most
acceptable substitute for Times Roman.  If actual Times Roman is not
available, users might try font Computer Modern Roman.  Macintosh
users should use the font named Times.

If you do not have a laser printer, try to borrow one, rent one, or
make friends with somebody who has one.  In some cases you might
be able to bring a disk to a business that will print your document
for you.   If you really cannot use a laser printer send your
submission on the best alternative printer you have.  As a very last
resort, if typesetting facilities are not available, papers can be
typewritten on a typewriter.  In this case, the text must be prepared
on larger pages and then reduced 25\%.

\section{FIGURES}
Figures must be inserted at the appropriate point in your text (Figure~1).
Figures can extend over the two columns up to 17.8 cm (7") if
necessary.  Black and white photographs (not Polaroid prints) may be
mounted on the camera-ready paper with glue or double-sided tape.
(Please note that even clear tape mounted over figures or text will
cause a noticeable smudge; attach figures only from behind.)

\begin{figure}[h]
        \vspace{1.5cm}
        \caption{an example of figure caption.
It is set in Helvetica 9 point, with a small 0.5 cm indentation.}
\end{figure}
For better quality you can have stats or screened velox prints that
are about 150 lines per inch prepared by your local printing service.
Of course you can do it yourself with a good scanner.  Remember that
some readers may never see anything other than a poor photocopy
of your paper, so make sure that the figure will still be readable (try
to see how it looks after recopying it a couple of times).

\section{COLOR IMAGES (PLATES)}
Color images (plates) must be placed on separate pages at the end of
your submission. They are grouped on separate pages so that in the
proceedings they may be collected into a single color section.
Consider carefully whether you really prefer color: black and white
images, which you prepare and place as they will appear within the
rest of your submission, may be as meaningful and nice as color
images separated from your text.

Color plate captions are to be the author(s)' last names followed by a
comma (,) and a plate number (e.g., Jones, Smith, Color Plate 1).
Captions of color plates are to be used when referring to them in the
text (e.g. See Smith, Color Plate 4).

Slides and transparencies are the preferred methods of submitting
color material for camera ready materials (on the other hand the
copies for review should have paper prints). We will not be able to
return any of these materials. Please do not submit color xerographic
copies of photographs.

\section{REFERENCES AND CITATIONS}
Use the standard CACM format for references, i.e., a numbered list at
the end of the article, ordered alphabetically by first author, and
referenced by number in brackets [2]. (See the examples of citations
at the end of this section or other examples in the April 93 issue of
the Communications of the ACM, page 135). References should be
materials accessible to the public (i.e., articles in standard journals
and open conference proceedings.)  Internal technical reports should
be avoided unless easily accessible (i.e. you can give the address to
obtain it).  Private communications should be acknowledged, not
referenced.

\section{HEADERS, FOOTERS AND PAGE NUMBERING}
Do not use headers, footers or footnotes. Page numbers, footers and
headers will be added when the Conference Proceedings is
assembled. Papers submitted to the paper chairs for review should
have page numbers (to help the review process).
But do NOT put numbers on the two page summaries.

\section{DISTRIBUTION}
The Conference Proceedings will be  distributed to all conference
participants and will be sold by ACM. It will also be distributed to all
members of SIGCHI and SIGOIS, as well as a large number of
libraries.

\section{OTHER CONSIDERATIONS}
\subsection{No Private Material}
Presentations may not contain any proprietary or confidential
material. Please clear all materials before submitting or presenting
them.  Submission of pictures of identifiable people should be done
only with the understanding that responsibility for collection of
appropriate permissions rests with the submitter not CSCW'94.

\subsection{Equation}
Displayed equations should be centered, with optional equation
numbers right-justified to the right margin of the column.

\subsection{Spelling}
Spelling may follow any dialect of English (e.g., British, Canadian, or
American) but please use it consistently.

%%%% you have to find a proper place to put \pagebreak to make 
%%%% the two columns of the last page equal length.
\subsection{Language}
The language of CSCW'94 is English for an international audience.
Avoid puns and  slang. English will not be the mother tongue of
many readers (and reviewers).  Today, part of striking the right tone
is also handling gender-linked terms sensitively.  Avoid gender
specific expressions when unnecessary.  \pagebreak But also try to avoid the
awkward he/she or her/his.  Consider the plural or replace third
person singular possessives with articles (e.g. "a" rather than "his").
Occasionally  passive forms can be used.

\section{ACKNOWLEDGMENTS}
These instructions are adapted from the "Standard CHI'94 Conference
Format: Preparing Submissions for Papers and Summaries."

\begin{thebibliography}{1}

\bibitem{cole}
Cole,~William~G.
\newblock Understanding Bayesian Reasoning via Graphical Displays.
\newblock In {\em Proceedings of CHI'89 Human Factors in Computing Systems\/}
  (April 30-May 4, Austin, TX), ACM/SIGCHI, N.Y., 1989, pp.~381--386.

\bibitem{gary}
Garey,~M.R.
\newblock Optimal binary identification procedures.
\newblock {\em SIAM J. Math.} 23,2 (Feb. 1972), 173--186.

\bibitem{garey}
Garey,~M.R. and Johnson,~D.S.
\newblock {\em Computers and Intractability: A Guide to the Theory of
  NP-Completeness}.
\newblock Freeman, San Francisco, California, 1979.

\end{thebibliography}

\end{document}

