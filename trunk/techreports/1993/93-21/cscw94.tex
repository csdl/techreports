%%%%%%%%%%%%%%%%%%%%%%%%%%%%%% -*- Mode: Latex -*- %%%%%%%%%%%%%%%%%%%%%%%%%%%%
%% cscw94.tex -- 
%% RCS:            : $Id: cscw94.tex,v 1.24 1994/06/17 21:30:44 dxw Exp $
%% Author          : Dadong Wan
%% Created On      : Wed Dec  1 16:25:27 1993
%% Last Modified By: Dadong Wan
%% Last Modified On: Fri Jun 17 11:30:22 1994
%% Status          : Unknown
%%%%%%%%%%%%%%%%%%%%%%%%%%%%%%%%%%%%%%%%%%%%%%%%%%%%%%%%%%%%%%%%%%%%%%%%%%%%%%%
%%   Copyright (C) 1993 University of Hawaii
%%%%%%%%%%%%%%%%%%%%%%%%%%%%%%%%%%%%%%%%%%%%%%%%%%%%%%%%%%%%%%%%%%%%%%%%%%%%%%%
%% 
%% History
%% 1-Dec-1993		Dadong Wan	
%%    created
\documentstyle [nftimes,cscw] {article}
% Psfig/TeX 
\def\PsfigVersion{1.9}
% dvips version
%
% All psfig/tex software, documentation, and related files
% in this distribution of psfig/tex are 
% Copyright 1987, 1988, 1991 Trevor J. Darrell
%
% Permission is granted for use and non-profit distribution of psfig/tex 
% providing that this notice is clearly maintained. The right to
% distribute any portion of psfig/tex for profit or as part of any commercial
% product is specifically reserved for the author(s) of that portion.
%
% *** Feel free to make local modifications of psfig as you wish,
% *** but DO NOT post any changed or modified versions of ``psfig''
% *** directly to the net. Send them to me and I'll try to incorporate
% *** them into future versions. If you want to take the psfig code 
% *** and make a new program (subject to the copyright above), distribute it, 
% *** (and maintain it) that's fine, just don't call it psfig.
%
% Bugs and improvements to trevor@media.mit.edu.
%
% Thanks to Greg Hager (GDH) and Ned Batchelder for their contributions
% to the original version of this project.
%
% Modified by J. Daniel Smith on 9 October 1990 to accept the
% %%BoundingBox: comment with or without a space after the colon.  Stole
% file reading code from Tom Rokicki's EPSF.TEX file (see below).
%
% More modifications by J. Daniel Smith on 29 March 1991 to allow the
% the included PostScript figure to be rotated.  The amount of
% rotation is specified by the "angle=" parameter of the \psfig command.
%
% Modified by Robert Russell on June 25, 1991 to allow users to specify
% .ps filenames which don't yet exist, provided they explicitly provide
% boundingbox information via the \psfig command. Note: This will only work
% if the "file=" parameter follows all four "bb???=" parameters in the
% command. This is due to the order in which psfig interprets these params.
%
%  3 Jul 1991	JDS	check if file already read in once
%  4 Sep 1991	JDS	fixed incorrect computation of rotated
%			bounding box
% 25 Sep 1991	GVR	expanded synopsis of \psfig
% 14 Oct 1991	JDS	\fbox code from LaTeX so \psdraft works with TeX
%			changed \typeout to \ps@typeout
% 17 Oct 1991	JDS	added \psscalefirst and \psrotatefirst
%

% From: gvr@cs.brown.edu (George V. Reilly)
%
% \psdraft	draws an outline box, but doesn't include the figure
%		in the DVI file.  Useful for previewing.
%
% \psfull	includes the figure in the DVI file (default).
%
% \psscalefirst width= or height= specifies the size of the figure
% 		before rotation.
% \psrotatefirst (default) width= or height= specifies the size of the
% 		 figure after rotation.  Asymetric figures will
% 		 appear to shrink.
%
% \psfigurepath#1	sets the path to search for the figure
%
% \psfig
% usage: \psfig{file=, figure=, height=, width=,
%			bbllx=, bblly=, bburx=, bbury=,
%			rheight=, rwidth=, clip=, angle=, silent=}
%
%	"file" is the filename.  If no path name is specified and the
%		file is not found in the current directory,
%		it will be looked for in directory \psfigurepath.
%	"figure" is a synonym for "file".
%	By default, the width and height of the figure are taken from
%		the BoundingBox of the figure.
%	If "width" is specified, the figure is scaled so that it has
%		the specified width.  Its height changes proportionately.
%	If "height" is specified, the figure is scaled so that it has
%		the specified height.  Its width changes proportionately.
%	If both "width" and "height" are specified, the figure is scaled
%		anamorphically.
%	"bbllx", "bblly", "bburx", and "bbury" control the PostScript
%		BoundingBox.  If these four values are specified
%               *before* the "file" option, the PSFIG will not try to
%               open the PostScript file.
%	"rheight" and "rwidth" are the reserved height and width
%		of the figure, i.e., how big TeX actually thinks
%		the figure is.  They default to "width" and "height".
%	The "clip" option ensures that no portion of the figure will
%		appear outside its BoundingBox.  "clip=" is a switch and
%		takes no value, but the `=' must be present.
%	The "angle" option specifies the angle of rotation (degrees, ccw).
%	The "silent" option makes \psfig work silently.
%

% check to see if macros already loaded in (maybe some other file says
% "\input psfig") ...
\ifx\undefined\psfig\else\endinput\fi

%
% from a suggestion by eijkhout@csrd.uiuc.edu to allow
% loading as a style file. Changed to avoid problems
% with amstex per suggestion by jbence@math.ucla.edu

\let\LaTeXAtSign=\@
\let\@=\relax
\edef\psfigRestoreAt{\catcode`\@=\number\catcode`@\relax}
%\edef\psfigRestoreAt{\catcode`@=\number\catcode`@\relax}
\catcode`\@=11\relax
\newwrite\@unused
\def\ps@typeout#1{{\let\protect\string\immediate\write\@unused{#1}}}
\ps@typeout{psfig/tex \PsfigVersion}

%% Here's how you define your figure path.  Should be set up with null
%% default and a user useable definition.

\def\figurepath{./}
\def\psfigurepath#1{\edef\figurepath{#1}}

%
% @psdo control structure -- similar to Latex @for.
% I redefined these with different names so that psfig can
% be used with TeX as well as LaTeX, and so that it will not 
% be vunerable to future changes in LaTeX's internal
% control structure,
%
\def\@nnil{\@nil}
\def\@empty{}
\def\@psdonoop#1\@@#2#3{}
\def\@psdo#1:=#2\do#3{\edef\@psdotmp{#2}\ifx\@psdotmp\@empty \else
    \expandafter\@psdoloop#2,\@nil,\@nil\@@#1{#3}\fi}
\def\@psdoloop#1,#2,#3\@@#4#5{\def#4{#1}\ifx #4\@nnil \else
       #5\def#4{#2}\ifx #4\@nnil \else#5\@ipsdoloop #3\@@#4{#5}\fi\fi}
\def\@ipsdoloop#1,#2\@@#3#4{\def#3{#1}\ifx #3\@nnil 
       \let\@nextwhile=\@psdonoop \else
      #4\relax\let\@nextwhile=\@ipsdoloop\fi\@nextwhile#2\@@#3{#4}}
\def\@tpsdo#1:=#2\do#3{\xdef\@psdotmp{#2}\ifx\@psdotmp\@empty \else
    \@tpsdoloop#2\@nil\@nil\@@#1{#3}\fi}
\def\@tpsdoloop#1#2\@@#3#4{\def#3{#1}\ifx #3\@nnil 
       \let\@nextwhile=\@psdonoop \else
      #4\relax\let\@nextwhile=\@tpsdoloop\fi\@nextwhile#2\@@#3{#4}}
% 
% \fbox is defined in latex.tex; so if \fbox is undefined, assume that
% we are not in LaTeX.
% Perhaps this could be done better???
\ifx\undefined\fbox
% \fbox code from modified slightly from LaTeX
\newdimen\fboxrule
\newdimen\fboxsep
\newdimen\ps@tempdima
\newbox\ps@tempboxa
\fboxsep = 3pt
\fboxrule = .4pt
\long\def\fbox#1{\leavevmode\setbox\ps@tempboxa\hbox{#1}\ps@tempdima\fboxrule
    \advance\ps@tempdima \fboxsep \advance\ps@tempdima \dp\ps@tempboxa
   \hbox{\lower \ps@tempdima\hbox
  {\vbox{\hrule height \fboxrule
          \hbox{\vrule width \fboxrule \hskip\fboxsep
          \vbox{\vskip\fboxsep \box\ps@tempboxa\vskip\fboxsep}\hskip 
                 \fboxsep\vrule width \fboxrule}
                 \hrule height \fboxrule}}}}
\fi
%
%%%%%%%%%%%%%%%%%%%%%%%%%%%%%%%%%%%%%%%%%%%%%%%%%%%%%%%%%%%%%%%%%%%
% file reading stuff from epsf.tex
%   EPSF.TEX macro file:
%   Written by Tomas Rokicki of Radical Eye Software, 29 Mar 1989.
%   Revised by Don Knuth, 3 Jan 1990.
%   Revised by Tomas Rokicki to accept bounding boxes with no
%      space after the colon, 18 Jul 1990.
%   Portions modified/removed for use in PSFIG package by
%      J. Daniel Smith, 9 October 1990.
%
\newread\ps@stream
\newif\ifnot@eof       % continue looking for the bounding box?
\newif\if@noisy        % report what you're making?
\newif\if@atend        % %%BoundingBox: has (at end) specification
\newif\if@psfile       % does this look like a PostScript file?
%
% PostScript files should start with `%!'
%
{\catcode`\%=12\global\gdef\epsf@start{%!}}
\def\epsf@PS{PS}
%
\def\epsf@getbb#1{%
%
%   The first thing we need to do is to open the
%   PostScript file, if possible.
%
\openin\ps@stream=#1
\ifeof\ps@stream\ps@typeout{Error, File #1 not found}\else
%
%   Okay, we got it. Now we'll scan lines until we find one that doesn't
%   start with %. We're looking for the bounding box comment.
%
   {\not@eoftrue \chardef\other=12
    \def\do##1{\catcode`##1=\other}\dospecials \catcode`\ =10
    \loop
       \if@psfile
	  \read\ps@stream to \epsf@fileline
       \else{
	  \obeyspaces
          \read\ps@stream to \epsf@tmp\global\let\epsf@fileline\epsf@tmp}
       \fi
       \ifeof\ps@stream\not@eoffalse\else
%
%   Check the first line for `%!'.  Issue a warning message if its not
%   there, since the file might not be a PostScript file.
%
       \if@psfile\else
       \expandafter\epsf@test\epsf@fileline:. \\%
       \fi
%
%   We check to see if the first character is a % sign;
%   if so, we look further and stop only if the line begins with
%   `%%BoundingBox:' and the `(atend)' specification was not found.
%   That is, the only way to stop is when the end of file is reached,
%   or a `%%BoundingBox: llx lly urx ury' line is found.
%
          \expandafter\epsf@aux\epsf@fileline:. \\%
       \fi
   \ifnot@eof\repeat
   }\closein\ps@stream\fi}%
%
% This tests if the file we are reading looks like a PostScript file.
%
\long\def\epsf@test#1#2#3:#4\\{\def\epsf@testit{#1#2}
			\ifx\epsf@testit\epsf@start\else
\ps@typeout{Warning! File does not start with `\epsf@start'.  It may not be a PostScript file.}
			\fi
			\@psfiletrue} % don't test after 1st line
%
%   We still need to define the tricky \epsf@aux macro. This requires
%   a couple of magic constants for comparison purposes.
%
{\catcode`\%=12\global\let\epsf@percent=%\global\def\epsf@bblit{%BoundingBox}}
%
%
%   So we're ready to check for `%BoundingBox:' and to grab the
%   values if they are found.  We continue searching if `(at end)'
%   was found after the `%BoundingBox:'.
%
\long\def\epsf@aux#1#2:#3\\{\ifx#1\epsf@percent
   \def\epsf@testit{#2}\ifx\epsf@testit\epsf@bblit
	\@atendfalse
        \epsf@atend #3 . \\%
	\if@atend	
	   \if@verbose{
		\ps@typeout{psfig: found `(atend)'; continuing search}
	   }\fi
        \else
        \epsf@grab #3 . . . \\%
        \not@eoffalse
        \global\no@bbfalse
        \fi
   \fi\fi}%
%
%   Here we grab the values and stuff them in the appropriate definitions.
%
\def\epsf@grab #1 #2 #3 #4 #5\\{%
   \global\def\epsf@llx{#1}\ifx\epsf@llx\empty
      \epsf@grab #2 #3 #4 #5 .\\\else
   \global\def\epsf@lly{#2}%
   \global\def\epsf@urx{#3}\global\def\epsf@ury{#4}\fi}%
%
% Determine if the stuff following the %%BoundingBox is `(atend)'
% J. Daniel Smith.  Copied from \epsf@grab above.
%
\def\epsf@atendlit{(atend)} 
\def\epsf@atend #1 #2 #3\\{%
   \def\epsf@tmp{#1}\ifx\epsf@tmp\empty
      \epsf@atend #2 #3 .\\\else
   \ifx\epsf@tmp\epsf@atendlit\@atendtrue\fi\fi}


% End of file reading stuff from epsf.tex
%%%%%%%%%%%%%%%%%%%%%%%%%%%%%%%%%%%%%%%%%%%%%%%%%%%%%%%%%%%%%%%%%%%

%%%%%%%%%%%%%%%%%%%%%%%%%%%%%%%%%%%%%%%%%%%%%%%%%%%%%%%%%%%%%%%%%%%
% trigonometry stuff from "trig.tex"
\chardef\psletter = 11 % won't conflict with \begin{letter} now...
\chardef\other = 12

\newif \ifdebug %%% turn me on to see TeX hard at work ...
\newif\ifc@mpute %%% don't need to compute some values
\c@mputetrue % but assume that we do

\let\then = \relax
\def\r@dian{pt }
\let\r@dians = \r@dian
\let\dimensionless@nit = \r@dian
\let\dimensionless@nits = \dimensionless@nit
\def\internal@nit{sp }
\let\internal@nits = \internal@nit
\newif\ifstillc@nverging
\def \Mess@ge #1{\ifdebug \then \message {#1} \fi}

{ %%% Things that need abnormal catcodes %%%
	\catcode `\@ = \psletter
	\gdef \nodimen {\expandafter \n@dimen \the \dimen}
	\gdef \term #1 #2 #3%
	       {\edef \t@ {\the #1}%%% freeze parameter 1 (count, by value)
		\edef \t@@ {\expandafter \n@dimen \the #2\r@dian}%
				   %%% freeze parameter 2 (dimen, by value)
		\t@rm {\t@} {\t@@} {#3}%
	       }
	\gdef \t@rm #1 #2 #3%
	       {{%
		\count 0 = 0
		\dimen 0 = 1 \dimensionless@nit
		\dimen 2 = #2\relax
		\Mess@ge {Calculating term #1 of \nodimen 2}%
		\loop
		\ifnum	\count 0 < #1
		\then	\advance \count 0 by 1
			\Mess@ge {Iteration \the \count 0 \space}%
			\Multiply \dimen 0 by {\dimen 2}%
			\Mess@ge {After multiplication, term = \nodimen 0}%
			\Divide \dimen 0 by {\count 0}%
			\Mess@ge {After division, term = \nodimen 0}%
		\repeat
		\Mess@ge {Final value for term #1 of 
				\nodimen 2 \space is \nodimen 0}%
		\xdef \Term {#3 = \nodimen 0 \r@dians}%
		\aftergroup \Term
	       }}
	\catcode `\p = \other
	\catcode `\t = \other
	\gdef \n@dimen #1pt{#1} %%% throw away the ``pt''
}

\def \Divide #1by #2{\divide #1 by #2} %%% just a synonym

\def \Multiply #1by #2%%% allows division of a dimen by a dimen
       {{%%% should really freeze parameter 2 (dimen, passed by value)
	\count 0 = #1\relax
	\count 2 = #2\relax
	\count 4 = 65536
	\Mess@ge {Before scaling, count 0 = \the \count 0 \space and
			count 2 = \the \count 2}%
	\ifnum	\count 0 > 32767 %%% do our best to avoid overflow
	\then	\divide \count 0 by 4
		\divide \count 4 by 4
	\else	\ifnum	\count 0 < -32767
		\then	\divide \count 0 by 4
			\divide \count 4 by 4
		\else
		\fi
	\fi
	\ifnum	\count 2 > 32767 %%% while retaining reasonable accuracy
	\then	\divide \count 2 by 4
		\divide \count 4 by 4
	\else	\ifnum	\count 2 < -32767
		\then	\divide \count 2 by 4
			\divide \count 4 by 4
		\else
		\fi
	\fi
	\multiply \count 0 by \count 2
	\divide \count 0 by \count 4
	\xdef \product {#1 = \the \count 0 \internal@nits}%
	\aftergroup \product
       }}

\def\r@duce{\ifdim\dimen0 > 90\r@dian \then   % sin(x+90) = sin(180-x)
		\multiply\dimen0 by -1
		\advance\dimen0 by 180\r@dian
		\r@duce
	    \else \ifdim\dimen0 < -90\r@dian \then  % sin(-x) = sin(360+x)
		\advance\dimen0 by 360\r@dian
		\r@duce
		\fi
	    \fi}

\def\Sine#1%
       {{%
	\dimen 0 = #1 \r@dian
	\r@duce
	\ifdim\dimen0 = -90\r@dian \then
	   \dimen4 = -1\r@dian
	   \c@mputefalse
	\fi
	\ifdim\dimen0 = 90\r@dian \then
	   \dimen4 = 1\r@dian
	   \c@mputefalse
	\fi
	\ifdim\dimen0 = 0\r@dian \then
	   \dimen4 = 0\r@dian
	   \c@mputefalse
	\fi
%
	\ifc@mpute \then
        	% convert degrees to radians
		\divide\dimen0 by 180
		\dimen0=3.141592654\dimen0
%
		\dimen 2 = 3.1415926535897963\r@dian %%% a well-known constant
		\divide\dimen 2 by 2 %%% we only deal with -pi/2 : pi/2
		\Mess@ge {Sin: calculating Sin of \nodimen 0}%
		\count 0 = 1 %%% see power-series expansion for sine
		\dimen 2 = 1 \r@dian %%% ditto
		\dimen 4 = 0 \r@dian %%% ditto
		\loop
			\ifnum	\dimen 2 = 0 %%% then we've done
			\then	\stillc@nvergingfalse 
			\else	\stillc@nvergingtrue
			\fi
			\ifstillc@nverging %%% then calculate next term
			\then	\term {\count 0} {\dimen 0} {\dimen 2}%
				\advance \count 0 by 2
				\count 2 = \count 0
				\divide \count 2 by 2
				\ifodd	\count 2 %%% signs alternate
				\then	\advance \dimen 4 by \dimen 2
				\else	\advance \dimen 4 by -\dimen 2
				\fi
		\repeat
	\fi		
			\xdef \sine {\nodimen 4}%
       }}

% Now the Cosine can be calculated easily by calling \Sine
\def\Cosine#1{\ifx\sine\UnDefined\edef\Savesine{\relax}\else
		             \edef\Savesine{\sine}\fi
	{\dimen0=#1\r@dian\advance\dimen0 by 90\r@dian
	 \Sine{\nodimen 0}
	 \xdef\cosine{\sine}
	 \xdef\sine{\Savesine}}}	      
% end of trig stuff
%%%%%%%%%%%%%%%%%%%%%%%%%%%%%%%%%%%%%%%%%%%%%%%%%%%%%%%%%%%%%%%%%%%%

\def\psdraft{
	\def\@psdraft{0}
	%\ps@typeout{draft level now is \@psdraft \space . }
}
\def\psfull{
	\def\@psdraft{100}
	%\ps@typeout{draft level now is \@psdraft \space . }
}

\psfull

\newif\if@scalefirst
\def\psscalefirst{\@scalefirsttrue}
\def\psrotatefirst{\@scalefirstfalse}
\psrotatefirst

\newif\if@draftbox
\def\psnodraftbox{
	\@draftboxfalse
}
\def\psdraftbox{
	\@draftboxtrue
}
\@draftboxtrue

\newif\if@prologfile
\newif\if@postlogfile
\def\pssilent{
	\@noisyfalse
}
\def\psnoisy{
	\@noisytrue
}
\psnoisy
%%% These are for the option list.
%%% A specification of the form a = b maps to calling \@p@@sa{b}
\newif\if@bbllx
\newif\if@bblly
\newif\if@bburx
\newif\if@bbury
\newif\if@height
\newif\if@width
\newif\if@rheight
\newif\if@rwidth
\newif\if@angle
\newif\if@clip
\newif\if@verbose
\def\@p@@sclip#1{\@cliptrue}


\newif\if@decmpr

%%% GDH 7/26/87 -- changed so that it first looks in the local directory,
%%% then in a specified global directory for the ps file.
%%% RPR 6/25/91 -- changed so that it defaults to user-supplied name if
%%% boundingbox info is specified, assuming graphic will be created by
%%% print time.
%%% TJD 10/19/91 -- added bbfile vs. file distinction, and @decmpr flag

\def\@p@@sfigure#1{\def\@p@sfile{null}\def\@p@sbbfile{null}
	        \openin1=#1.bb
		\ifeof1\closein1
	        	\openin1=\figurepath#1.bb
			\ifeof1\closein1
			        \openin1=#1
				\ifeof1\closein1%
				       \openin1=\figurepath#1
					\ifeof1
					   \ps@typeout{Error, File #1 not found}
						\if@bbllx\if@bblly
				   		\if@bburx\if@bbury
			      				\def\@p@sfile{#1}%
			      				\def\@p@sbbfile{#1}%
							\@decmprfalse
				  	   	\fi\fi\fi\fi
					\else\closein1
				    		\def\@p@sfile{\figurepath#1}%
				    		\def\@p@sbbfile{\figurepath#1}%
						\@decmprfalse
	                       		\fi%
			 	\else\closein1%
					\def\@p@sfile{#1}
					\def\@p@sbbfile{#1}
					\@decmprfalse
			 	\fi
			\else
				\def\@p@sfile{\figurepath#1}
				\def\@p@sbbfile{\figurepath#1.bb}
				\@decmprtrue
			\fi
		\else
			\def\@p@sfile{#1}
			\def\@p@sbbfile{#1.bb}
			\@decmprtrue
		\fi}

\def\@p@@sfile#1{\@p@@sfigure{#1}}

\def\@p@@sbbllx#1{
		%\ps@typeout{bbllx is #1}
		\@bbllxtrue
		\dimen100=#1
		\edef\@p@sbbllx{\number\dimen100}
}
\def\@p@@sbblly#1{
		%\ps@typeout{bblly is #1}
		\@bbllytrue
		\dimen100=#1
		\edef\@p@sbblly{\number\dimen100}
}
\def\@p@@sbburx#1{
		%\ps@typeout{bburx is #1}
		\@bburxtrue
		\dimen100=#1
		\edef\@p@sbburx{\number\dimen100}
}
\def\@p@@sbbury#1{
		%\ps@typeout{bbury is #1}
		\@bburytrue
		\dimen100=#1
		\edef\@p@sbbury{\number\dimen100}
}
\def\@p@@sheight#1{
		\@heighttrue
		\dimen100=#1
   		\edef\@p@sheight{\number\dimen100}
		%\ps@typeout{Height is \@p@sheight}
}
\def\@p@@swidth#1{
		%\ps@typeout{Width is #1}
		\@widthtrue
		\dimen100=#1
		\edef\@p@swidth{\number\dimen100}
}
\def\@p@@srheight#1{
		%\ps@typeout{Reserved height is #1}
		\@rheighttrue
		\dimen100=#1
		\edef\@p@srheight{\number\dimen100}
}
\def\@p@@srwidth#1{
		%\ps@typeout{Reserved width is #1}
		\@rwidthtrue
		\dimen100=#1
		\edef\@p@srwidth{\number\dimen100}
}
\def\@p@@sangle#1{
		%\ps@typeout{Rotation is #1}
		\@angletrue
%		\dimen100=#1
		\edef\@p@sangle{#1} %\number\dimen100}
}
\def\@p@@ssilent#1{ 
		\@verbosefalse
}
\def\@p@@sprolog#1{\@prologfiletrue\def\@prologfileval{#1}}
\def\@p@@spostlog#1{\@postlogfiletrue\def\@postlogfileval{#1}}
\def\@cs@name#1{\csname #1\endcsname}
\def\@setparms#1=#2,{\@cs@name{@p@@s#1}{#2}}
%
% initialize the defaults (size the size of the figure)
%
\def\ps@init@parms{
		\@bbllxfalse \@bbllyfalse
		\@bburxfalse \@bburyfalse
		\@heightfalse \@widthfalse
		\@rheightfalse \@rwidthfalse
		\def\@p@sbbllx{}\def\@p@sbblly{}
		\def\@p@sbburx{}\def\@p@sbbury{}
		\def\@p@sheight{}\def\@p@swidth{}
		\def\@p@srheight{}\def\@p@srwidth{}
		\def\@p@sangle{0}
		\def\@p@sfile{} \def\@p@sbbfile{}
		\def\@p@scost{10}
		\def\@sc{}
		\@prologfilefalse
		\@postlogfilefalse
		\@clipfalse
		\if@noisy
			\@verbosetrue
		\else
			\@verbosefalse
		\fi
}
%
% Go through the options setting things up.
%
\def\parse@ps@parms#1{
	 	\@psdo\@psfiga:=#1\do
		   {\expandafter\@setparms\@psfiga,}}
%
% Compute bb height and width
%
\newif\ifno@bb
\def\bb@missing{
	\if@verbose{
		\ps@typeout{psfig: searching \@p@sbbfile \space  for bounding box}
	}\fi
	\no@bbtrue
	\epsf@getbb{\@p@sbbfile}
        \ifno@bb \else \bb@cull\epsf@llx\epsf@lly\epsf@urx\epsf@ury\fi
}	
\def\bb@cull#1#2#3#4{
	\dimen100=#1 bp\edef\@p@sbbllx{\number\dimen100}
	\dimen100=#2 bp\edef\@p@sbblly{\number\dimen100}
	\dimen100=#3 bp\edef\@p@sbburx{\number\dimen100}
	\dimen100=#4 bp\edef\@p@sbbury{\number\dimen100}
	\no@bbfalse
}
% rotate point (#1,#2) about (0,0).
% The sine and cosine of the angle are already stored in \sine and
% \cosine.  The result is placed in (\p@intvaluex, \p@intvaluey).
\newdimen\p@intvaluex
\newdimen\p@intvaluey
\def\rotate@#1#2{{\dimen0=#1 sp\dimen1=#2 sp
%            	calculate x' = x \cos\theta - y \sin\theta
		  \global\p@intvaluex=\cosine\dimen0
		  \dimen3=\sine\dimen1
		  \global\advance\p@intvaluex by -\dimen3
% 		calculate y' = x \sin\theta + y \cos\theta
		  \global\p@intvaluey=\sine\dimen0
		  \dimen3=\cosine\dimen1
		  \global\advance\p@intvaluey by \dimen3
		  }}
\def\compute@bb{
		\no@bbfalse
		\if@bbllx \else \no@bbtrue \fi
		\if@bblly \else \no@bbtrue \fi
		\if@bburx \else \no@bbtrue \fi
		\if@bbury \else \no@bbtrue \fi
		\ifno@bb \bb@missing \fi
		\ifno@bb \ps@typeout{FATAL ERROR: no bb supplied or found}
			\no-bb-error
		\fi
		%
%\ps@typeout{BB: \@p@sbbllx, \@p@sbblly, \@p@sbburx, \@p@sbbury} 
%
% store height/width of original (unrotated) bounding box
		\count203=\@p@sbburx
		\count204=\@p@sbbury
		\advance\count203 by -\@p@sbbllx
		\advance\count204 by -\@p@sbblly
		\edef\ps@bbw{\number\count203}
		\edef\ps@bbh{\number\count204}
		%\ps@typeout{ psbbh = \ps@bbh, psbbw = \ps@bbw }
		\if@angle 
			\Sine{\@p@sangle}\Cosine{\@p@sangle}
	        	{\dimen100=\maxdimen\xdef\r@p@sbbllx{\number\dimen100}
					    \xdef\r@p@sbblly{\number\dimen100}
			                    \xdef\r@p@sbburx{-\number\dimen100}
					    \xdef\r@p@sbbury{-\number\dimen100}}
%
% Need to rotate all four points and take the X-Y extremes of the new
% points as the new bounding box.
                        \def\minmaxtest{
			   \ifnum\number\p@intvaluex<\r@p@sbbllx
			      \xdef\r@p@sbbllx{\number\p@intvaluex}\fi
			   \ifnum\number\p@intvaluex>\r@p@sbburx
			      \xdef\r@p@sbburx{\number\p@intvaluex}\fi
			   \ifnum\number\p@intvaluey<\r@p@sbblly
			      \xdef\r@p@sbblly{\number\p@intvaluey}\fi
			   \ifnum\number\p@intvaluey>\r@p@sbbury
			      \xdef\r@p@sbbury{\number\p@intvaluey}\fi
			   }
%			lower left
			\rotate@{\@p@sbbllx}{\@p@sbblly}
			\minmaxtest
%			upper left
			\rotate@{\@p@sbbllx}{\@p@sbbury}
			\minmaxtest
%			lower right
			\rotate@{\@p@sbburx}{\@p@sbblly}
			\minmaxtest
%			upper right
			\rotate@{\@p@sbburx}{\@p@sbbury}
			\minmaxtest
			\edef\@p@sbbllx{\r@p@sbbllx}\edef\@p@sbblly{\r@p@sbblly}
			\edef\@p@sbburx{\r@p@sbburx}\edef\@p@sbbury{\r@p@sbbury}
%\ps@typeout{rotated BB: \r@p@sbbllx, \r@p@sbblly, \r@p@sbburx, \r@p@sbbury}
		\fi
		\count203=\@p@sbburx
		\count204=\@p@sbbury
		\advance\count203 by -\@p@sbbllx
		\advance\count204 by -\@p@sbblly
		\edef\@bbw{\number\count203}
		\edef\@bbh{\number\count204}
		%\ps@typeout{ bbh = \@bbh, bbw = \@bbw }
}
%
% \in@hundreds performs #1 * (#2 / #3) correct to the hundreds,
%	then leaves the result in @result
%
\def\in@hundreds#1#2#3{\count240=#2 \count241=#3
		     \count100=\count240	% 100 is first digit #2/#3
		     \divide\count100 by \count241
		     \count101=\count100
		     \multiply\count101 by \count241
		     \advance\count240 by -\count101
		     \multiply\count240 by 10
		     \count101=\count240	%101 is second digit of #2/#3
		     \divide\count101 by \count241
		     \count102=\count101
		     \multiply\count102 by \count241
		     \advance\count240 by -\count102
		     \multiply\count240 by 10
		     \count102=\count240	% 102 is the third digit
		     \divide\count102 by \count241
		     \count200=#1\count205=0
		     \count201=\count200
			\multiply\count201 by \count100
		 	\advance\count205 by \count201
		     \count201=\count200
			\divide\count201 by 10
			\multiply\count201 by \count101
			\advance\count205 by \count201
			%
		     \count201=\count200
			\divide\count201 by 100
			\multiply\count201 by \count102
			\advance\count205 by \count201
			%
		     \edef\@result{\number\count205}
}
\def\compute@wfromh{
		% computing : width = height * (bbw / bbh)
		\in@hundreds{\@p@sheight}{\@bbw}{\@bbh}
		%\ps@typeout{ \@p@sheight * \@bbw / \@bbh, = \@result }
		\edef\@p@swidth{\@result}
		%\ps@typeout{w from h: width is \@p@swidth}
}
\def\compute@hfromw{
		% computing : height = width * (bbh / bbw)
	        \in@hundreds{\@p@swidth}{\@bbh}{\@bbw}
		%\ps@typeout{ \@p@swidth * \@bbh / \@bbw = \@result }
		\edef\@p@sheight{\@result}
		%\ps@typeout{h from w : height is \@p@sheight}
}
\def\compute@handw{
		\if@height 
			\if@width
			\else
				\compute@wfromh
			\fi
		\else 
			\if@width
				\compute@hfromw
			\else
				\edef\@p@sheight{\@bbh}
				\edef\@p@swidth{\@bbw}
			\fi
		\fi
}
\def\compute@resv{
		\if@rheight \else \edef\@p@srheight{\@p@sheight} \fi
		\if@rwidth \else \edef\@p@srwidth{\@p@swidth} \fi
		%\ps@typeout{rheight = \@p@srheight, rwidth = \@p@srwidth}
}
%		
% Compute any missing values
\def\compute@sizes{
	\compute@bb
	\if@scalefirst\if@angle
% at this point the bounding box has been adjsuted correctly for
% rotation.  PSFIG does all of its scaling using \@bbh and \@bbw.  If
% a width= or height= was specified along with \psscalefirst, then the
% width=/height= value needs to be adjusted to match the new (rotated)
% bounding box size (specifed in \@bbw and \@bbh).
%    \ps@bbw       width=
%    -------  =  ---------- 
%    \@bbw       new width=
% so `new width=' = (width= * \@bbw) / \ps@bbw; where \ps@bbw is the
% width of the original (unrotated) bounding box.
	\if@width
	   \in@hundreds{\@p@swidth}{\@bbw}{\ps@bbw}
	   \edef\@p@swidth{\@result}
	\fi
	\if@height
	   \in@hundreds{\@p@sheight}{\@bbh}{\ps@bbh}
	   \edef\@p@sheight{\@result}
	\fi
	\fi\fi
	\compute@handw
	\compute@resv}

%
% \psfig
% usage : \psfig{file=, height=, width=, bbllx=, bblly=, bburx=, bbury=,
%			rheight=, rwidth=, clip=}
%
% "clip=" is a switch and takes no value, but the `=' must be present.
\def\psfig#1{\vbox {
	% do a zero width hard space so that a single
	% \psfig in a centering enviornment will behave nicely
	%{\setbox0=\hbox{\ }\ \hskip-\wd0}
	%
	\ps@init@parms
	\parse@ps@parms{#1}
	\compute@sizes
	%
	\ifnum\@p@scost<\@psdraft{
		%
		\special{ps::[begin] 	\@p@swidth \space \@p@sheight \space
				\@p@sbbllx \space \@p@sbblly \space
				\@p@sbburx \space \@p@sbbury \space
				startTexFig \space }
		\if@angle
			\special {ps:: \@p@sangle \space rotate \space} 
		\fi
		\if@clip{
			\if@verbose{
				\ps@typeout{(clip)}
			}\fi
			\special{ps:: doclip \space }
		}\fi
		\if@prologfile
		    \special{ps: plotfile \@prologfileval \space } \fi
		\if@decmpr{
			\if@verbose{
				\ps@typeout{psfig: including \@p@sfile.Z \space }
			}\fi
			\special{ps: plotfile "`zcat \@p@sfile.Z" \space }
		}\else{
			\if@verbose{
				\ps@typeout{psfig: including \@p@sfile \space }
			}\fi
			\special{ps: plotfile \@p@sfile \space }
		}\fi
		\if@postlogfile
		    \special{ps: plotfile \@postlogfileval \space } \fi
		\special{ps::[end] endTexFig \space }
		% Create the vbox to reserve the space for the figure.
		\vbox to \@p@srheight sp{
		% 1/92 TJD Changed from "true sp" to "sp" for magnification.
			\hbox to \@p@srwidth sp{
				\hss
			}
		\vss
		}
	}\else{
		% draft figure, just reserve the space and print the
		% path name.
		\if@draftbox{		
			% Verbose draft: print file name in box
			\hbox{\frame{\vbox to \@p@srheight sp{
			\vss
			\hbox to \@p@srwidth sp{ \hss \@p@sfile \hss }
			\vss
			}}}
		}\else{
			% Non-verbose draft
			\vbox to \@p@srheight sp{
			\vss
			\hbox to \@p@srwidth sp{\hss}
			\vss
			}
		}\fi	



	}\fi
}}
\psfigRestoreAt
\let\@=\LaTeXAtSign



                
\begin{document}

\title {Computer Supported Collaborative Learning Using\\
        CLARE: the Approach and Experimental Findings}
                     
\author{{\authorfont Dadong Wan} \\
Center for Information Technology \& Management\\
Walter A. Haas School of Business\\
University of California\\
Berkeley, CA 94720, USA\\
Tel: (510) 642-4617\\
E-mail: dxw@haas.berkeley.edu
\and
{\authorfont Philip M. Johnson} \\
Department of Information and Computer Sciences\\
University of Hawaii\\
Honolulu, HI 96822, USA\\
Tel: (808) 956-3489\\
E-mail: johnson@hawaii.edu
}
\date{}
\maketitle
                     
\begin{abstract}
  Current collaborative learning systems focus on maximizing shared
  information. However, ``meaningful learning'' is not simply information
  sharing but, more importantly, knowledge construction.  CLARE is a
  computer-supported learning environment that facilitates meaningful
  learning through collaborative knowledge construction. CLARE provides a
  semi-formal representation language called RESRA and an explicit
  process model called SECAI. Experimental evaluation through 300 hours
  of classroom usage indicates that CLARE does support meaningful
  learning, and that a major bottleneck to computer-mediated knowledge
  construction is summarization.  Lessons learned through the design and
  evaluation of CLARE provide new insights into both collaborative
  learning systems and collaborative learning theories.
\end{abstract}
\copyrightspace

\keywords Computer supported collaborative learning; collaborative work; knowledge
representation; knowledge construction; meaningful learning.


\section{INTRODUCTION}
\label{sec:introduction}

Current computer-supported collaborative learning systems focus on
improving shared access to information, people, and media. One example is
virtual classroom systems, which range from plain e-mail, computer
conferencing, and bulletin-board systems, to more specialized systems such
as EIES \cite{Hiltz88}.  Virtual classrooms allow learners to interact with
their peers and instructors, as well as access online information in a
manner independent of time and geographic location. They augment
traditional classroom learning by removing the requirement for physical
co-presence and by improving access to information and people.

A second example is hypermedia systems, such as Intermedia
\cite{Yankelovich88} and Mosaic \cite{Andreessen93}. These systems
typically provide distributed mechanisms for structuring large information
spaces. They also provide mechanisms for presentation and integration of
various media formats, including text, graphics, voice, and video.
Hypermedia systems improve access by removing the constraint of text-based
interaction, broadening the scope of sharable information, and reducing the
effort required to make information sharable.

Although virtual classrooms and hypermedia systems are successful in
improving information access, they do not typically offer explicit
mechanisms to help learners better interpret and assimilate the
information, the context surrounding its creation and use, and the
perspectives on it of the author or other learners. Simply improving
information access without supporting learning leads directly to the
problems of ``information overload'' and ``lost-in-hyperspace''.  In the
coming age of the Information Superhighway, it will be essential to
explicitly support learning as well as access.  Software systems must
provide users with structural and process-level support on how to
comprehend new information, how to relate new information to what they
already know, and how to identify, compare, and integrate different
interpretations of the same information.  In other words, how to
meaningfully learn in an environment of vastly improved information access.

This paper presents our approach to providing computational support for
meaningful learning through a process of collaborative knowledge
construction.  The next section presents three major components of our
approach: a representation language for meaningful learning called RESRA, a
process model to guide the use of RESRA called SECAI, and a computational
environment designed to support this language and process called CLARE.
The following section describes our evaluation of this approach through
about 300 hours of classroom usage. Analysis of the results indicates that
the approach is effective in facilitating meaningful learning. It also
reveals the strategies of knowledge construction used by learners, and
several significant obstacles to meaningful learning within this paradigm.
The following section briefly connects this research to related work in
learning theory, cognitive psychology, and computer-supported cooperative
work. The final section revisits the essential contributions of this
research and outlines promising future research directions.


\section{OVERVIEW OF CLARE}
\label{sec:clare-approach}

\subsection{RESRA}
\label{sec:resra}

RESRA, which stands for ``REpresentational Schema of Research Artifacts,''
is a semi-structured knowledge representation language designed
specifically to facilitate collaborative learning from scientific text,
such as research papers. It has the following design goals:

\begin{itemize}
\item An organizational tool that allows incremental, fine-grained
  representation and integration of scientific artifacts;
  
\item A mapping tool that highlights essential thematic features and
  relationships within and across scientific text, and that helps
  expose gaps and ambiguities in existing knowledge;
  
\item A communication tool and a shared frame of reference that
  highlights similarities and differences between learners'
  points of view; and
  
\item A tool for learning about the norms and conventions governing formal
  communication of scientific knowledge.
\end{itemize}

To achieve these goals, RESRA defines three types of conceptual constructs:
{\it node primitives\/}, {\it link primitives}, and {\it canonical
forms}. Node primitives represent discrete thematic features of the
artifact, for example, {\it claim\/}, {\it concepts}, and {\it
theory\/}. They also explicitly represent the learner's points of view in
terms of {\it critiques\/}, {\it questions\/} and {\it suggestions\/}.

Link primitives describe relationships between thematic features
represented by node primitives. For example, in a research paper, a {\it
claim\/} is typically made with respect to a particular {\it problem\/} and
must be supported by some {\it evidence\/}. In RESRA, these relationships
are expressed as ``{\it claim\/} \( \stackrel{responds-to}{\longrightarrow}
\) {\it problem''\/} and ``{\it evidence\/}
\(\stackrel{supports}{\longrightarrow}\) {\it claim\/},'' where {\it
responds-to\/} and {\it supports\/} are link primitives.
   
The canonical form characterizes typical artifact-level thematic structures
as a directed graph of RESRA node and link primitives. Research artifacts
are classified into various types, such as {\it concept}, {\it empirical},
and {\it survey\/} papers. For each of these artifact types, a
stereotypical RESRA structural graph can be identified. For example, in
software engineering, one important type of research artifact is an
``experience paper''. Such artifacts report the experience of an
organization with a software package or strategy, including the problem it
attempted to solve or alleviate, the initial justification for adopting
that software or strategy, and how the actual outcome compared to the
expected result. In RESRA, a canonical form for experience reports could be
expressed as shown in Figure \ref{fig:experience-crf}.

\begin{figure}[htb]
 \centerline{\psfig{figure=Figures/experience-crf.eps,width=3.0in}}
  \caption{A RESRA canonical form for ``experience paper''} 
  \label{fig:experience-crf}
\end{figure}

Learners in software engineering may use the above structural model as a
``template'' to guide their interpretation and evaluation of all experience
papers in that domain, and their efforts in constructing their own
experience papers so that they may also conform to the same structure. In a
group setting, such structural knowledge can be used as a shared framework
for learners to engage in discussions about the content of related
artifacts.

RESRA is based upon three premises: (1) Human knowledge can be represented
in term of a small number of node and link primitives; (2) The use of these
primitives to characterize scientific artifacts and subsequent group
activities are a meaningful learning process; and (3) Different learners
are likely to generate different representations of the same artifact; by
comparing these representations, one can discern the similarities and
differences in points of view held by individual learners. 

Figure \ref{tab:resra} provides a synopsis of all RESRA node primitives,
while Figure \ref{fig:sum-resra} graphically depicts relationships
between representative RESRA node and link primitives.

\begin{figure*}[htb]
  \begin{center}
    

    \begin{tabular} {|l|p{2.5in}|p{2.5in}|} \hline   
      {\bf Node Type} & {\bf Description} & {\bf Example} \\ \hline \hline
      
      Problem & A phenomenon, event, or process whose understanding
      requires further inquiry. & Meta-learning is not adequately
      supported by existing tools. \\ \hline
      
      Claim & A position or proposition about a given problem
      situation.  & Cleanroom engineering provides a viable solution
      in producing zero defect software. \\ \hline
      
      Evidence & Data gathered for the purpose of supporting or
      refuting a given claim. & The use of cleanroom techniques led
      to a 5-fold reduction of defects in project Alpha. \\
      \hline 

      Theory & A systemic formulation about a particular problem
      domain, derivable through deductive or inductive procedures. &
      Ausubel's theory of meaningful learning. \\ \hline
      
      Method & Procedures or techniques used to generate evidence for
      a particular claim. & Delphi study; nominal grouping technique;
      waterfall software development model. \\ \hline
      
      Concept & A primitive construct used in formulating theory,
      claim, or method. & Meta-learning; Knowledge representation.
      \\ \hline
      
      Thing & A natural or man-made object that is under study.  &
      Rock; Intermedia.  \\ \hline
      
      Source & An identifiable written artifact, either artifact
      itself or a reference to it. & An
      article by Ashton; the notes from Kyle's talk. \\ \hline \hline
      
      Critique & Critical remarks or comments about a particular
      claim, evidence, method, source, et al., or relationships
      between them. & Applications of cleanroom
      engineering appear limited to domains with well-defined requirements.
      \\ \hline
      
      Question & Aspects of a claim, theory, concept, etc., about
      which the learner is still in doubt. & How does box-structured
      design differ from object-oriented design? \\ \hline
      
      Suggestion & Ideas, recommendations, or feedback on how to
      improve an existing problem statement, claim, method, et al.  &
      I would like to see cleanroom engineering used in some
      non-conventional domains, such as groupware. \\ \hline
    \end{tabular}
    \captiontwocolumns{A synopsis of RESRA node primitives.}
    \label{tab:resra}
      \end{center}
\end{figure*}

\begin{figure*}[htb]
 \centerline{\psfig{figure=Figures/sum-resra.eps,height=4.0in}}
  \captiontwocolumns{A graphical illustration of RESRA summarative node and link
  primitives and the relationships between them.}
  \label{fig:sum-resra}
\end{figure*}

Figure \ref{fig:fagan} shows an example use of RESRA that summarizes a
seminal paper on software code inspection \cite{Fagan76}.  The {\em source}
node in the upper left corner provides a reference to the artifact under
study. The paper addresses one {\em problem}, against which three {\em
claim}s are made. To support these {\em claim}s, the author introduces
three {\em concept}s and two {\em method}s. The latter are used to generate
{\em evidence}, which in turn supports two out of the three {\em claim}s.

\begin{figure*}[htb]
 \centerline{\psfig{figure=Figures/fagan.eps,width=5.0in}}
 \captiontwocolumns{An expert's RESRA representation of Fagan's paper on code inspection.}
  \label{fig:fagan}
\end{figure*}

Several interesting observations can be made about Figure \ref{fig:fagan}.
First, the representation captures what is important in the research paper:
it is not simply an outline of the paper but rather a map of its knowledge
structure that reflects the learner's mental model about the author's
intent. Second, as discussed later in this paper, different learners will
derive quite different representations of this same artifact. Figure
\ref{fig:fagan} is merely one of many possible representations. By
comparing and contrasting these distinct representations, one can gain a
better understanding of not only what this artifact is really about but
perhaps more interestingly, how different learners interpret the same
content. Third, also as described later, constructing these representations
is non-trivial and requires significant learner effort, since RESRA node
primitives, such as the {\em problem} node in Figure \ref{fig:fagan}, may
be only implied rather than explicitly stated by the author. Thus, the
learner must infer these objects from the context of the paper, and relate
them to other RESRA objects. Deriving such a representation is a meaningful
learning experience because learners must ask
themselves many deep-level questions: what is the claim(s) being
made? With respect to what problem? Is a given theme a {\it claim\/} or
{\it theory\/}? How are those themes related? Are there any ``orphan'' or
unconnected themes?  Answers to these and other related questions reveal
what the artifact really means to individual learners.


\subsection{SECAI}
\label{sec:secai}

SECAI, which stands for ``Summarization, Evaluation, Comparison,
Argumentation, and Integration,'' defines an explicit process model for
collaborative learning from scientific text. Figure \ref{fig:secai} shows
how these activities are related together to support collaborative
knowledge construction. The world outside the concentric circles consists
of various types of scientific artifacts, which constitute the raw material of
learning.  Metaphorically, collaborative learning with SECAI pulls learners
from an external, isolated, and individual position inward toward an
internal, integrated, and collaborative perspective on the artifact.

The first phase of SECAI is called {\it exploration}, which consists of two
activities: {\it summarization\/} and {\it evaluation\/}. During this
phase, learners derive a personal representation of the artifact and an
evaluation of its content, both expressed in terms of RESRA. This phase is
performed {\it privately} -- learners are not allowed to see what other
learners are doing or have done.  This policy prevents learners from being
distracted by each other's work or from free-riding off the work of others.
The result of exploration is a set of representations similar to that shown
in Figure \ref{fig:fagan} (except for the addition of evaluation nodes).

The second phase of SECAI is called {\it consolidation}, which consists of
three activities: {\it comparison}, {\it argumentation}, and {\it
integration}. During comparison, learners evaluate the similarities and
differences between their representation and those of other learners.
Comparison is done at three levels: the artifact level, the link primitive
level, and the node primitive level.  At the artifact level, learners
compare their classifications of the artifact type, such as whether the
artifact appears to be a {\it concept paper\/} or an {\it experience
report\/}. In addition, they also compare their representations with
respect to the canonical form selected for the artifact to see how their
representations deviate from the standard one. At the link primitive level,
learners compare their derived relationships, such as each learner's
interpretation of the supporting evidence for a claim. Finally, at the node
primitive level, learners compare their instantiations of nodes and the
artifact content referenced by them.

Comparison activities provide a basis for {\it argumentation\/}. For
example, suppose that John compares his representation to Jane's, and
determines that his representation contains a {\em problem} node that is
apparently missing from Jane's representation.  He might then generate a
{\em critique} node concerning Jane's representation, noting that it is
missing an important problem raised in the artifact.  Jane might respond by
agreeing that her representation omitted an important problem.
Alternatively, Jane might respond that one of her {\em problem} nodes in
fact subsumes the problem noted by John.  Another potential response might
be to disagree with John's interpretation of the artifact content as a
problem: that it was actually a method or claim of the research, as
described in Jane's representation. This process of comparison and
argumentation leads to an improved understanding of the meaning of the
artifact. Perhaps as importantly, it reveals other learners' perspectives
on the artifact.

The final step in the consolidation phase is {\it integration}, where
learners create explicit links between their individual representations to
improve their collective coherence and consistency. Going back to our
hypothetical learners, if Jane realized that John had correctly identified
a problem missing from her representation, she could integrate her
representation by linking John's {\em problem} node into her representation
in the appropriate places.  Alternatively, if Jane believed that one of her
{\em problem} nodes subsumed the problem identified by John, she could
integrate his representation by creating an {\em is-part-of} link between
his and her nodes.

The large shaded arrows in Figure \ref{fig:secai} indicate the direction in
which SECAI ``pushes'' the group process: as learners proceed through the
various activities, the level of collaboration among learners increases
and, concurrently, a group knowledge base emerges.  This dynamic knowledge
base articulates both areas of consensus and areas of disagreements among a
group of learners as they summarize, evaluate, compare, deliberate, and
integrate their individual perspectives on the learning artifact.

\begin{figure*}[htb]
  \centerline{\psfig{figure=Figures/secai.eps,width=5.0in}}
  \captiontwocolumns{The SECAI process model for collaborative learning from scientific text.}
  \label{fig:secai}
\end{figure*}


\subsection{CLARE}
\label{sec:clare}

CLARE, which stands for ``Collaborative Learning And Research
Environment,'' is a distributed learning environment that supports SECAI
and RESRA.  It is a client-server system, running in a Unix/X-windows
environment, and is built on top of the EGRET framework for exploratory
collaboration \cite{csdl-92-01,csdl-93-09}.

Use of CLARE begins by converting the scientific text to be studied into
CLARE's internal hypertext format. Typically, the full text of the document
is split up into smaller chunks called {\it source nodes\/}, each one
corresponding to a physical section or subsection of the document. The
current implementation of CLARE does not provide support for graphical
images, so tables and figures are shown in the hypertext network as logical
references and supplied to the learners as hardcopy documents.

Figure \ref{fig:explore} shows a snapshot of the CLARE user interface
during the exploration phase. The left window shows a source node
corresponding to one section of the artifact under study.  The node is
connected to other source nodes via the {\it Up\/}, {\it Next\/}, and {\it
Prev\/} links displayed on the first line of that node.  Learners navigate
through the scientific artifact under study by following these links.

To summarize a paragraph or any arbitrary block of text, the learner first
highlights the text by dragging the mouse over it, and then selects the
corresponding node type from the {\sf Summarize\/} menu. CLARE creates a
new node of the chosen type (e.g., {\em problem}, {\em evidence}) with
default field template, and displays it in the lower right window. An
explicit link is also automatically added between the selected text in the
left window and the newly created node. The learner then provides
annotative comments about the summarization in the {\bf Description} field.

The above process is repeated until the learners believe they have fully
summarized the document. (Summary nodes may also be created without
reference to any text in the artifact.)  Evaluation nodes are created
analogously, although both source nodes and summary nodes may serve as
targets of evaluation.  The learner adds RESRA link primitives between two
eligible nodes by choosing {\sf Link Mode\/} from the {\sf Summarize\/}
menu. The upper right window in Figure \ref{fig:explore} shows what the
current learner has created so far for the current artifact.

\begin{figure*}[htb]
  \centerline{\psfig{figure=Figures/explore.eps,width=6.0in}}
  \captiontwocolumns{A user view of CLARE during the exploration phase.  The left
  hand window contains a portion of the artifact under study.  The lower
  right window contains an evidence node created by the learner.  The upper
  right window summarizes what the learner has created during exploration
  thus far.}
  \label{fig:explore}
\end{figure*}

Figure \ref{fig:consolidate} shows a snapshot of the CLARE user interface
during the consolidation phase. The upper left window displays a
comparative view of the {\em problem\/} instances created by three learners
(Peter, Cam, and Rose) during exploration. It highlights the differences
and similarities between these learners with respect to their views on the
problem the original authors attempt to address. Similar comparisons can
also be made for the other RESRA primitives, for example, {\it claim}, {\it
evidence}.  To see the actual node instance corresponding to a given entry,
the learner mouse clicks on the bold italicized text.  If a learner then
wants to know the precise place in the artifact from which the
summarization node was derived, he can follow the link in the {\bf
Summarization} field.

To raise a question or make a critique on the {\em problem} node, one
selects the corresponding RESRA node type from the {\sf Argument\/} menu in
the lower right window. This creates a new node instance of that type and
links it to the {\em problem} node.  When the creator of the {\em problem}
node sees this new link, he can explain or defend his position using a
similar procedure. He might also declare his {\em problem} as {\sl
similar-to\/} another node using the {\sf Integrate\/} menu.

The upper-right window in Figure \ref{fig:consolidate} summarizes the state
of all learner's work. For example, Peter has created ten {\em claim}
nodes, while Cam has seven, and Rose has six. Clicking on an entry
generates a detailed listing for the corresponding user, from which
individual node instances for any learner can be retrieved.

\begin{figure*}[htb]
  \centerline{\psfig{figure=Figures/consolidate.eps,width=6.0in}}
  \captiontwocolumns{A user view of CLARE during the consolidation phase.  The upper
  left window contains a comparative summary of the problems identified
  by each learner in the scientific artifact.  One of the actual problem
  node instances is displayed in the lower right hand window.  The lower
  left window displays the portion of the scientific text from which this
  problem was derived. The upper right window contains a summary
  of the activities of each learner.}
  \label{fig:consolidate}
\end{figure*}

In addition to these computational services, CLARE also includes an
instrumentation mechanism that unobtrusively gathers fine-grained process
data. Each time the user performs a semantically interesting action, such
as creating a new node or link, a timestamp representing this event and the
time at which it occurred is recorded. The instrumentation also detects
periods of idle-time for correcting elapsed-time calculations.  This
instrumentation provides vital data for answering such process-level
questions as: what is the sequence in which learners visit nodes, and does
this navigation strategy differ among learners? How much time do learners
spend on each portion of the document? How much time do learners spend on
each activity of SECAI?  How are node creation and the link creation
related procedurally?  Answering these and other questions was the goal of 
a set of experiments we performed using CLARE, discussed next. 


\section{EXPERIMENTS AND FINDINGS}
\label{sec:findings}

\subsection{Experiments}

Two sets of experiments were conducted to evaluate the effectiveness of
CLARE as an environment for meaningful learning through collaborative
knowledge construction.  For this evaluation, three types of data were
collected:

\begin{itemize}
\item {\it Assessment:\/} Gathered through a questionnaire administered
  after each experiment session;
  
\item {\it Outcome:\/} Online CLARE database created during each session;
  and
  
\item {\it Process: \/} Gathered automatically through the built-in
  instrumentation mechanism. 
\end{itemize}

The subjects were 24 computer science students who were enrolled in two
software engineering courses in the Department of Information and Computer
Science at the University of Hawaii in Fall, 1993. The task was
collaborative analysis and deliberation of research papers in software
engineering using CLARE. The first experiment involved 16 upper-level
undergraduate students, who were evenly divided into four groups. The
experiment was repeated three times with three different research papers.
The second experiment involved 8 graduate students, who were evenly divided
into two groups. This experiment was repeated twice with two different
research papers. All experimental sessions lasted approximately one week. A
few sessions lasted two to three days longer due to interruptions from
other class activities.

The experiments were conducted between September and October, 1993. The
subjects collectively accumulated about 300 hours of usage time, and
created about 1,800 nodes with a total text size of nearly 400 kilobytes. A
total of over 80,000 timestamps were gathered during these sessions. Figure
\ref{tab:summary-stat} provides a short summary of the experimental data.

\begin{figure}[htb]
  \begin{center}
  \begin{tabular} {|c|r|r|r|r|} \hline   
    Exp. & Logins & Time (hrs)& Nodes & Size (Kb) \\
    \hline \hline {\bf 1a.}  & 120 & 82.85 & 472 & 90.02 \\ \hline {\bf
    1b.}  & 115 & 67.90 & 513 & 107.97 \\ \hline {\bf 1c.}  & 84 &
    53.68 & 440 & 105.16 \\ \hline {\bf 2a.}  & 85 & 54.42 & 162 &
    39.42 \\ \hline {\bf 2b.}  & 53 & 37.55 & 207 & 49.67 \\ \hline
    \hline {\bf Total} & 457 & 296.40 & 1794 & 392.24 \\ \hline
   \end{tabular}
  \end{center}
    \caption{Selected CLARE statistics.}
   \label{tab:summary-stat}
\end{figure}


\subsection{Results}
\label{sec:results}

\subsubsection{Viability/usability of CLARE}

The post-session survey responses from the subjects show that CLARE is a
novel and useful collaborative learning tool: approximately 70\% of
learners indicated CLARE helped them understand the content of research
papers in a way not possible before, and nearly 80\% of learners indicated
that CLARE helped them understand their peers' perspectives in a way not
possible before. Approximately 84\% of the learners found that RESRA
provides a useful means for characterizing the important content of
research papers, and 90\% of the learners agreed that RESRA helped expose
different points of view on the same artifact.

Responses from the post-session survey also assessed the usefulness of
individual components of our approach. The RESRA node primitives and the
SECAI learning model were ranked the highest, assessed by 82\% and 70\% of
the users, respectively, as ``very'' or ``extremely'' useful. The least
useful features were the online examples, assessed by 25\% of the users as
``not'' useful. RESRA canonical forms, the comparison mode, and link
primitives were received mixed reactions from the user.

In addition to the empirical data, subjective responses to the approach
were revealing.  The following response from a subject shows that, in at
least one instance, CLARE succeeded in fostering meaningful learning:

\begin{quotation}
\noindent  {\it ``... I don't quite know how to use
  it [CLARE] very well yet, but it
  really helped me get more out 
  of the artifact we read.  Without
  CLARE I would have just read the
  artifact and not really studied it
  or learned about the subject. CLARE
  made me look at the artifact from
  another point of view.  That point of
  view was what is the author trying to
  tell me and how is the author
  trying to tell me that information ...
  Before I used CLARE I just read the
  artifacts.  Now using CLARE I look
  for the meaning of the artifact and
  learn more about the subject...''}
\end{quotation}


\subsubsection{Issues in CLARE-based collaborative learning}

Detailed analysis of the outcome and process data revealed a number of
interesting issues regarding collaborative learning using CLARE.  These
issues are discussed fully in \cite{csdl-93-14}, here we present four of
the most significant: mis-interpretations of RESRA, failures in
summarization, summarization strategies, and collaboration in CLARE.

First, RESRA was interpreted in a wide variety of ways among the learners.
Despite the presence of hands-on training, detailed user documentation, and
online examples, many subjects still seemed to fail to grasp the semantics
of RESRA primitives, as evidenced by a substantial number of times in which
RESRA nodes and links were used incorrectly.  For example, though {\em
theory\/} is defined as ``a systematic formulation about a particular
problem domain...'', the following use of the primitive clearly does not
satisfy this definition:

\begin{verbatim}
   TYPE: theory
   SUBJ: No single development improves
         the situation
   DESC: No single development aids in
         improving the software problem,
         at least not with respect to
         productivity, reliability or
         simplicity.
\end{verbatim}

Other typical errors in using RESRA include evidence nodes
containing no evidence, suggestions containing no proposals from the
learner, claims that are ``neutral,'' evidence stated as claims,
explanations or predictions identified as theories, problems treated as
learner's disagreements with the author's claims instead of what the author
attempts to address, and so forth. 

Second, although subjects spent about 66\% of their time on summarization,
they frequently failed to adequately summarize the artifact by correctly
identifying major themes and relationships and filtering out the minor
ones.  For example, Figure \ref{fig:fagan} provides an expert summary of
\cite{Fagan76} using 11 nodes. The 16 subjects analyzing this artifact
generated an average of 19 nodes. Given this relatively large number, one
would expect that all major themes would be covered, as well as a few minor
ones.  However, analysis of representations reveals that: (a) none of the
16 learners correctly identified the problem; (b) only seven learners
correctly identified one or two of the three major claims; (c) only ten
learners had the evidence right; and (d) only six learners had one of the
two methods right. On the other hand, many minor themes of the paper were
found in the learners' representations.

Third, the process data also reveals a set of stereotypical strategies used
by learners in summarizing the content of an artifact. The strategies are
characterized by the sequencing of summarative node and link instance
creation:

\begin{itemize}
\item {\it Nodes only:} Create summarative nodes only. No attempt is made
  to connect them together using RESRA link primitives.
  
\item {\it Nodes for an entire document, then links:} First create
  summarative nodes for the entire artifact, and then link them together.
  
\item {\it Nodes, then links, but one section at a time:} Create
  summarative nodes for a single source node and/or its adjacent source
  nodes, and then create links between them; repeat the same process until
  all source node are summarized.
  
\item {\sl (2) first, then (3):} A combination of the first and second
  strategy. First create summarative nodes for the entire artifact,
  followed by a wave of link creation. Next, selectively create additional
  summarative nodes, immediately followed by the link creation.
\end{itemize}

Excluding the 36\% of learners/sessions who adopted the first
strategy, i.e., creating no summarative links, there was no noticeable
correlation found between the strategy used and the quality of
summarization.

Finally, in the SECAI model, explicit collaboration among learners takes
place in the form of comparing their representations, deliberating
reasonings behind them, and ultimately, integrating them into a coherent
whole. Figure \ref{fig:arg-example} shows an example collaborative
representation network generated by four first-time CLARE users. Of a total
of 92 nodes in the network, 34 were created during the argumentation phase.
Most of these nodes (32 of 34) are {\it evaluative \/} in nature, which in
turn can be categorized into two groups: pointing out the correct use of
RESRA primitives and identifying ambiguities/inaccuracies in other
learners' representations. In ``critique642,'' for instance, Mary points
out that, in ``claim528,'' Scott has totally mis-interpreted the original
authors' meaning. To assess the accuracy of Scott's representation, the
process data shows that Mary in fact verifies the node content with the
source from which Scott's node was derived.

Figure \ref{fig:arg-example} also shows the presence of {\it
constructive\/} (as opposed to {\it evaluative\/}) argumentation, in which
learners do not merely critique or question each other's positions but
engage in active knowledge-building by formulating new problems, proposing
alternative claims, supplying additional evidence, and so on. In
``evidence662,'' for example, Chris counters Mary's claim (``claim522'')
with new evidence.

Another noticeable feature about Figure \ref{fig:arg-example} is the
absence of integration activities, which turns out to be quite typical
across CLARE sessions. A few learners elected to add explicit integrative
links between their representations or vote for best representations. As a
result, the group knowledge base consists of substantial amount of
redundancy and inconsistency.

\begin{figure*}[hbtp]
 \centerline{\psfig{figure=Figures/rep-all.eps,height=3.9in}}
 \captiontwocolumns{An example collaborative representation network by four CLARE users} 
  \label{fig:arg-example}
\end{figure*}


\subsection{Discussion}
\label{sec:discussion}

The results provide a strong indication that RESRA, SECAI, and CLARE
together provide a useful means for meaningful learning through
collaborative knowledge construction.  Users indicated that the RESRA
representation and the SECAI process model provided effective and novel
support for representing the meaning of an artifact and comparing their
interpretation to others. The results also indicated that users were not
quite satisfied with the user interface of the CLARE prototype. Analysis
of the experimental data provided new insight into the strategies used by
subjects as they constructed their representations.

However, the results also yielded a surprising paradox: though users rated
RESRA quite highly, they also made a substantial number of errors in its
application.  This raises the following question: is RESRA still useful for
its intended purpose, despite individual variations in its interpretation
and deviations from its intended usage?

The answer to this question appears to be a qualified yes, at least in the
current context of use. It is possible, particularly at the node primitive
level, for a learner to use a primitive incorrectly but usefully, simply
because node and link primitives force the learner to analyze the artifact
in terms of high level conceptual abstractions.  Analyzing the artifact in
this way provides a useful means to discover structural and content issues
that might not be perceived otherwise. In addition, incorrect use of
primitives also create collaborative opportunities, since the mistake might
be identified by another learner.  In many cases, the creation,
identification, discussion, and correction of representation errors can
facilitate meaningful learning both at the content level (by constructing
an awareness of the true nature of the artifact under study) and at a
meta-level (by constructing an awareness of the true nature of the
representational language.)

However, incorrect use of RESRA node primitives can also raise a major
barrier to effective use of the representation, since it can prevent
effective use of RESRA link primitives.  For example, an {\em evidence\/}
either {\it supports\/} or {\it counters} a {\em claim}.  At the node
primitive level, it may not be of great consequence to misrepresent a {\em
claim\/} as a {\it theory\/}.  However, this mistake has significant
consequences at the link primitive level, since the link primitives for
claims will not be available.  Thus, incorrect node primitive choices
can inhibit the creation of links between nodes, a characteristic found in
several subjects' representations.

During the design of RESRA, SECAI, and CLARE, we expected that the
summarization activity would be rather straightforward and that most
collaborative interaction would occur during argumentation and integration.
The results surprised us: most of the subjects' effort and collaborative
activity centered on summarization, and the summarization activity was not
at all straightforward: subjects made substantial errors both in choice of
RESRA primitives and in choice of the artifact content to summarize.  We
now realize that this seemingly straightforward process of reconstructing a
scientific paper within a simple representation language poses an awesome
challenge for many learners.  The ultimate success of a collaborative
learning environment has everything to do with how well it meets this
initial challenge.  We expect this problem will arise no matter what the
nature of the representational language. Rather than hope for a ``silver
bullet'' knowledge representation language, collaborative learning system
designers should instead ensure that process-level mechanisms exist to
overcome these breakdowns when they inevitably occur.

Finally, we want to point out the exploratory nature of the evaluation
results discussed thus far: our primary goal was to assess the the
viability and usability of the CLARE approach and to provide evidence and
insights about what learners might do when confronted with such a novel
learning environment. Through this study, we hope to provide a rich ground
on which more rigorous experimentation and field studies on
computer-supported collaborative learning can be formulated and performed.


\section{RELATED WORK}
\label{sec:related work}

CLARE is grounded in two theoretical tenets: social constructionism
\cite{Berger66,Knorr-Cetina81} and the assimilation theory of cognitive
learning \cite{Ausubel63,Novak84}. The former affirms the social nature of
learning and the imperative of engaging learners in collaborative knowledge
construction, as opposed to merely information sharing. It provides a
philosophical foundation for the learning activities that CLARE
supports. The latter is centered on the concept of {\it meaningful
learning\/}, which defines learning as an ongoing process of relating new
knowledge to what the learner already knows. Meaningful learning emphasizes
explicit use of meta-knowledge to enhance human learning. Toward this end,
two meta-cognitive strategies have been proposed: {\it concept mapping\/}
and {\it Vee diagramming\/} \cite{Gowin81,Novak84}. While the effectiveness
of concept mapping is well supported empirically
\cite{Cliburn90,Novak90,Roth92}, its inadequacy as a collaborative learning
tool and its lack of computerized support have directly prompted the
current research.

RESRA is related to {\it schema theory\/} in cognitive psychology, which
contends that human minds store and retrieve knowledge about the external
world in terms of abstract chunks called schemas \cite{Stillings87} and
that the schema plays in an essential role in the selection, abstraction,
interpretation, and integration of information \cite{Alba83}.  RESRA is
similar to some knowledge representation research in AI, particularly RA
\cite{Swaminathan90}, which proposes an episodic representation for
research literature.

A number of semi-structured representation schemes are found in the
literature, for example, IBIS \cite{Kunz70,Conklin88}, DRL\cite{Lee91What},
Toulmin's rhetorical model \cite{Toulmin58,Cavalli-Sforza92}.  RESRA
differs from those schemes in that, among other things, it is fully
integrated with an explicit process model (i.e., SECAI) that defines how
the scheme is to be used.

CLARE falls into a special type of computer-based learning environments
called {\it collaborative knowledge construction tools\/}, as contrasted
with information sharing tools, such as EIES \cite{Hiltz88}, Intermedia
\cite{Yankelovich88}, and Mosaic \cite{Andreessen93}. CLARE is similar to
CSILE (Computer-Supported Intentional Learning Environment)
\cite{Scardamadia93,Scardamadia92} in that both systems reify a social
constructionist paradigm and provide an environment conducive to
collaborative knowledge construction. However, CLARE's representation
scheme provides a meta-cognitive framework for collaborative learning,
while CSILE's four thinking types (``I know,'' ``high-level questions,''
``plan,'' and ``problem'') only allow learners to categorize their
intentions. In addition, CLARE provides an explicit process model to
control the application of the representation and the process of
collaboration.


\section{CONCLUSIONS AND FUTURE DIRECTIONS}
\label{sec:conclusions}

This paper presents findings on a computer-based approach for supporting
learning as a process of collaborative knowledge-building.  The system,
CLARE, differs from other learning systems in three important ways. First,
it provides a semi-structured knowledge representation language that serves
as a shared meta-cognitive framework to facilitate communication and
collaboration among learners. Second, it defines an explicit process model
of collaborative learning. Third, it implements fine-grained
instrumentation mechanisms to gather detailed process data concerning the
behavior of its users.

Analysis of experimental data confirms that CLARE is a novel environment
that fosters meaningful learning. It shows that RESRA and SECAI provide
useful structural and process-level guidance on how to collaboratively
construct knowledge. In addition, analysis also reveals a number of issues
for further research.

First, computer-supported collaborative learning is still a quite recent
phenomenon for which no coherent theoretical frameworks yet exist. Many
current learning theories, such as the ones on which CLARE is based, do not
explicitly address such essential issues as how people develop shared
mental models of the same artifact or problem situation, how to deal with
differing terminologies for the same construct, and so on. As evidenced
from the CLARE experimental result, these problems are intrinsic and also
essential to collaborative knowledge-building, and cannot be answered by
simply extrapolating proposals from individual-based learning
theories. Hence, new and better theoretical explanations and guidance are
called for.

Second, collaborative learning is not confined to classroom settings or
scientific artifacts. Rather, it is part of every work situation in which
artifact-based collaboration is required, such as software development,
business proposal development, and so forth. Our long-term goal is to
develop computer-supported environments that foster collaborative learning
across task domains through tailorable structural and process-level
support. RESRA and SECAI represent the first step in this direction.

Finally, we intend to apply CLARE's approach to collaborative learning as a
means to assess the quality of the research/learning artifact and to help
authors improve the quality of the artifacts they create. We hypothesize
that a good scientific artifact contains a clearly-articulated knowledge
structure and thus is easier to summarize using CLARE than a bad artifact.
As a simple test of this hypothesis, we invite you---our readers---to
contemplate a RESRA representation of this paper, and send us your
assessments and discoveries.


\section{Acknowledgments}
\label{sec:acks}

The authors gratefully acknowledge the assistance of the other members of
the Collaborative Software Development Laboratory in all phases of this
research: Danu Tjahjono, Robert Brewer, Cam Moore, and Rosemary Andrada.
Support for this research was partially provided by the National Science
Foundation Research Initiation Award CCR-9110861.

\begin{thebibliography}{1}

\bibitem{Alba83}
J.W. Alba and L.~Hasher.
\newblock Is memory schematic?
\newblock {\em Psychological Bulletin}, 93:203--31, 1983.

\bibitem{Andreessen93}
Marc Andreessen.
\newblock {NCSA} {M}osaic technical summary.
\newblock Available via anonymous ftp from
  "ftp.ncsa.uiuc.edu:/Web/mosaic-papers/mosaic.ps.Z".

\bibitem{Ausubel63}
David~P. Ausubel.
\newblock {\em The psychology of meaningful verbal learning}.
\newblock Grune \& Stratton, 1963.

\bibitem{Berger66}
Peter Berger and Thomas Luckman.
\newblock {\em The Social Construction of Reality: a Treatise in the Sociology
  of Knowledge}.
\newblock Doubleday, 1966.

\bibitem{Cavalli-Sforza92}
Violetta Cavalli-Sforza, Gareth Gabrys, Alan M.Lesgold, and Arlene Weiner.
\newblock Engaging students in scientific activity and scientific controversy.
\newblock In Kishore Swaminathan, editor, {\em Proceedings of 1992 AAAI
  Workshop on Communicating Scientific and Technical Knowledge}, pages 99--106,
  San Jose, 1992.

\bibitem{Cliburn90}
Joseph~W. {Cliburn, Jr.}
\newblock Concept maps to promote meaningful learning.
\newblock {\em Journal of College Science Teaching}, pages 212--217, February
  1990.

\bibitem{Conklin88}
J.~Conklin and L.~Begeman.
\newblock g{IBIS}: A hypertext tool for exploratory policy discussion.
\newblock {\em ACM Transactions on Office Information Systems}, 6(4):303--331,
  October 1988.

\bibitem{Fagan76}
Michael~E. Fagan.
\newblock Design and code inspections to reduce errors in program development.
\newblock {\em IBM System Journal}, 15(3):182--211, 1976.

\bibitem{Gowin81}
D.B. Gowin.
\newblock {\em Educating}.
\newblock Cornell University Press, 1981.

\bibitem{Hiltz88}
S.P. Hiltz.
\newblock Collaborative learning in a virtual classroom: Highlights of
  findings.
\newblock In {\em Proceedings of the ACM Conference on Computer-Supported
  Cooperative Work}, pages 282--290, 1988.

\bibitem{csdl-92-01}
Philip~M. Johnson.
\newblock Supporting exploratory {CSCW} with the {EGRET} framework.
\newblock In {\em Proceedings of the 1992 Conference on Computer Supported
  Cooperative Work}, November 1992.

\bibitem{csdl-93-09}
Philip~M. Johnson.
\newblock Experiences with {EGRET}: An exploratory group work environment.
\newblock {\em Collaborative Computing}, 1(1), 1994.

\bibitem{Knorr-Cetina81}
K.D. Knorr-Centina.
\newblock {\em The manufacture of knowledge: an essay on the constructivist and
  contextural nature of science}.
\newblock Pergamon Press, Oxford, 1981.

\bibitem{Kunz70}
W.~Kunz and H.~Rittel.
\newblock Issues as elements of information systems.
\newblock Technical report, Institute of Urban and Regional Development,
  University of California at Berkeley, 1977.

\bibitem{Lee91What}
Jintae Lee and Kum-Yew Lai.
\newblock What's in design rationale?
\newblock {\em Human-Computer Interaction}, 6(3,4):251--280, 1991.

\bibitem{Novak90}
J.D. Novak.
\newblock Concept mapping: a useful tool for science education.
\newblock {\em Journal of Research in Science Teaching}, 27(10):937--49, dec
  1990.

\bibitem{Novak84}
Joseph~D. Novak and D.~Bob Gowin.
\newblock {\em Learning how to learn}.
\newblock Cambridge University Press, 1984.

\bibitem{Roth92}
Wolff-Michael Roth and Anita Roychoudhury.
\newblock The social construction of scientific concepts or the concept map as
  conscription device and tool for social thinking in high school science.
\newblock {\em Science Education}, 76(5):531--557, 1992.

\bibitem{Scardamadia92}
Marlene Scardamadia, C.~Bereiter, C.~Brett, P.J. Burtis, C.~Calhoun, and N.S.
  Lea.
\newblock Educational applications of a networked communal database.
\newblock {\em Interactive Learning Environments}, 2(1):45--71, 1992.

\bibitem{Scardamadia93}
Marlene Scardamalia and Carl Bereiter.
\newblock Technologies for knowledge-building discourse.
\newblock {\em Communications of the ACM}, 36(5):37--41, May 1993.

\bibitem{Stillings87}
Neil Stillings, Mark Feinstein, Jay Garfield, E.L. Rissland, D.A. Rosenbaum,
  S.E. Weisler, and L.Baker-Ward.
\newblock {\em Cognitive science: an introduction}.
\newblock MIT Press, 1987.

\bibitem{Swaminathan90}
Kishore Swaminathan.
\newblock {\em {RA}: a memory organization to model the evolution of scientific
  knowledge}.
\newblock PhD thesis, University of Massachusetts at Amherst, 1990.
\newblock also COINS Technical Report 90-80.

\bibitem{Toulmin58}
S.~Toulmin.
\newblock {\em The use of argument}.
\newblock Cambridge University Press, 1958.

\bibitem{csdl-93-14}
Dadong Wan.
\newblock {\em {CLARE}: A Computer-Supported Collaborative Learning Environment
  Based on the Thematic Structure of Scientific Text}.
\newblock PhD thesis, University of Hawaii, Department of Information and
  Computer Sciences, 1994.

\bibitem{Yankelovich88}
Nicole Yankelovich, Bernard~J. Haan, Norman K.~Y. Meyrowitz, and Steven~M.
  Drucker.
\newblock Intermedia: the concept and the construction of a seamless
  information environment.
\newblock {\em Computer}, January 1988.

\end{thebibliography}

%\bibliography{/group/csdl/techreports/93-14/bib/clare,/group/csdl/techreports/93-14/bib/csdl-trs,/group/csdl/techreports/93-14/bib/cscl-systems}
%\bibliographystyle{plain}

\end{document}







