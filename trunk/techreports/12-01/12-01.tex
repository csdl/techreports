% 12-01.tex
% FDG 2012 conference doctoral consortium
% WORKS WITH V3.2SP OF ACM_PROC_ARTICLE-SP.CLS
% January 2012
% Author: Yongwen Xu
%
% ----------------------------------------------------------------------------------------------------------------
% This .tex file (and associated .cls V3.2SP) *DOES NOT* produce:
%       1) The Permission Statement
%       2) The Conference (location) Info information
%       3) The Copyright Line with ACM data
%       4) Page numbering
% ---------------------------------------------------------------------------------------------------------------
% It is an example which *does* use the .bib file (from which the .bbl file
% is produced).
% REMEMBER HOWEVER: After having produced the .bbl file,
% and prior to final submission,
% you need to 'insert'  your .bbl file into your source .tex file so as to provide
% ONE 'self-contained' source file.
%

\documentclass{acm_proc_article-sp}

%% Package to linebreak URLs in a sane manner.
\usepackage{url}

\begin{document}

\title{Designing A ``Serious Game'' Engine for Sustainability}

%\numberofauthors{1} 

\author{
\smallskip
Yongwen Xu\\ 
       \affaddr{Information and Computer Sciences}\\
       \affaddr{University of Hawai`i at M\=anoa}\\
       \affaddr{Honolulu, HI 96822}\\
       \email{yxu@hawaii.edu}
}

\maketitle
\begin{abstract}
Sustainability education has become an international imperative due to the rising cost of energy, increasing scarcity of natural resource and irresponsible environmental practices. My research seeks to investigate the approach of building a customizable ``serious game'' engine called Makahiki that can be easily adapted to the need of different organizations to create engaging games to educate newer generations on renewable energy, energy conservation and sustainability in general. This work is motivated by the encouraging result of the inaugural resident hall energy competition at University of Hawaii in Fall 2011, ``The Quest to Kukui Cup''. Makahiki intends to provide a production quality, pluggable component-based game engine and an experimental test bed for game related research in the context of sustainability.
  
\end{abstract}

% A category with the (minimum) three required fields
\category{L.5.1}{Game-based Learning}{Gaming}

\terms{Human Factors, Games, Education, Motivation}

\keywords{Serious Games, Education, Gamification, Sustainability}% NOT required for Proceedings

\section{Introduction}
% Almost every research document begins with a section that frames the research and motivates the problem being studied. It describes some domain, indicates a problem in general terms, and explains why the problem is worth solving. Questions a CHI reader should be able to answer after reading the motivation section are:
% * What is the general area being addressed?
% * Is this relevant to CHI?
% * What is the motivation for studying a particular problem?
% * What makes it worth the effort?
% * Is it a 'real' problem in everyday life, and/or is it a 'theoretical' problem that is worth solving?
% * Would anyone care if I solved this?

Sustainability education and conservation has become an international imperative due to the rising cost of energy, increasing scarcity of natural resource and irresponsible environmental practices. On the other hand, education of sustainability has been almost completely absent from the traditional educational system. There is no teaching of "sustainability" as a core subject area  in public schools in the United States. The sustainability problem has proven difficult to solve. Despite many attempts by modern environmental movements, little progress has been made such as in the area of climate change. Any behavior change requires citizens to first obtain a wide range of knowledge including energy generation, water consumption and recycling practices. 

The wide spread of game techniques has rendered game a unique media in many areas beyonds entertainment. Serious game has been explored to provide alternative to education and other areas. Gamification, the concept of applying game techniques to non-game context \cite{Deterding2011mt}, was included in Gartner Group's Hype cycle list for emerging technologies for 2011. 

This paper proposes the design of Makahiki, a customizable ``serious game'' engine to be easily adapted by different organizations to create engaging games to educate newer generations on renewable energy, energy conservation and sustainability.  The goal of the Makahiki is to provide a production quality game engine and an experimental test bed for game related research in education and behavior change in sustainability.

The initial version of Makahiki was used to create an energy challenge game called ``The Quest for the Kukui Cup'' \cite{csdl2-11-02} for approximately 1,000 first year students living in residence halls at the University of Hawaii in Fall 2011. During the three weeks of the competition, over 400 of the eligible students played the game, for a total of 850 game play hours in the online play and real world events, such as workshop on energy audit, excursion to a wind farm, and energy-related activities on campus. The challenge was so well received that University of Hawaii had decided to make the challenge an "annual tradition" for every first year students. Plans are under way to organize new challenges in 2012 for University of Hawaii and another university in Hawaii with new contents and possible new games in water conservation and recycling. Makahiki game engine will serve as the technology enabler for these challenges and hope to impact a new generation of future society leaders on sustainability awareness and behavior.

\section{Background and Motivation}
% Provide a miniature literature review to give the reader enough background to (a) gain sufficient knowledge about what others have done, (b) know how your work will build upon this prior work, and (c) be assured that you have sufficient knowledge of the relevant literature. You should highlight only the key literature here; a full review is not required. Questions a CHI reader should be able to answer after reading this section are:
%
% * Did the author provide enough background to help me know what others have done in this area, as well as what discipline(s) have considered this area?
% * Does the author have sufficient knowledge of the relevant literature necessary to do the proposed work?
% * How does the author’s proposed work fit within and extend what has been done before?

Energy awareness and feedback have been proved to be effective for reducing energy consumption. \cite{darby-review-2006}. Energy competitions or challenges are introduced to college dormitories and residential homes as ways to educate and incentivize energy reduction. Petersen et al. describe their experiences deploying a real-time feedback
system in an Oberlin College dorm energy competition in 2005\cite{petersen-dorm-energy-reduction}. They found a 32\% reduction in electricity use across all dormitories. The Campus Conservation National, is another nationwide energy and water use reduction competition on college campuses \cite{competetoreduce}. Over 170 colleges and universities across the U.S. have signed up to participate in 2012. These dorm energy related competitions are based on the software system called ``The Building Dashboard'', developed by Lucid Design Group \cite{building-dashboard}. The Building Dashboard enables viewing, comparing and sharing building energy and water use information on the web in compelling visual interface, but the cost of the system creates the barrier for wider adoptions. It mainly focus on providing energy information as a passive media. There is little interaction between participants and the system.

Games on the other hand, have been shown with great potential as successful interactive media that provide engaging interfaces in various serious contexts \cite{mcgonigal2011reality}. Priebatsch attempts to build a game layer on top of the world with his location-based service startup \cite{Priebatsch2010ted}. Reeves et al. described the design of Power House, an energy game that connects home smart meters to an online multiple player game with the goal to improve home energy behavior \cite{Reeves2011powerhouse}.  ROI Research and Recyclebank launched the Green Your Home Challenge as a case study of employing gamification techniques online to encourage residential green behavioral changes offline \cite{gamingforgood}. 

After the inaugural Kukui Cup game, We have been approached with requests to adapt the system to three other sites in the Fall of 2012, each of which involves different requirements for the system. The current implementation of Makahiki is not sufficient to easily support all these sites. These new requirements motivate us to re-design Makahiki as a pluggable, customizable component based framework that will support all of these versions of challenges.  The resulting framework will provide a flexible, yet powerful game engine for implementing ``serious games'' in sustainability.

\section{Research Goals}

% Provide a very concise statement of your thesis or problem statement. This should be the highest-level problem or goal you plan to address and is sometimes posed as a hypothesis, proposition or conjecture. This is often followed by a small list of specific problems and sub-problems that need to be solved if you are going to satisfy your hypothesis or thesis. Problems should be stated unambiguously. The importance of the problem should be mentioned if it hasn't already been done so in the prior sections. Of course, the problem must be worthy of a PhD thesis. Questions a CHI reader should be able to answer after reading this section are:
% * Did the author succinctly identify the thesis, problem or set of problems being addressed?
% * Is this problem worthy of a CHI PhD thesis?

There are two research goals in Makahiki: (a) creating a production quality customizable ``serious game'' engine in sustainability that can be easily adapted to the needs of different organization, and (b) providing an experimental test bed for serious game research on the effectiveness of different game mechanics in the context of sustainability.

The challenges of creating a customizable game engine are:  (a) create a new instance of Makahiki by selecting the games they want the system to support, and (b) extend Makahiki by writing new game components, and (c) the above two can be easily implemented by non-technical organizations with minimal technical support.

In order to provide an experimental test bed for game research, Makahiki intends to support A-B testing, where different game mechanics could be configured using the game engine to create "treatments" to different user groups. 

\section{System Description}

One significant barrier to adoption of the initial Makahiki system is that it is difficult to install, requires complex hardware, backups, failover, and system administration sophistication. These barriers are absent with the Lucid Designs solution since it is sold as (an expensive) cloud-based service. On the other hand, the significant barrier to adoption of Lucid's is that it's expensive, in addition to its limited flexibility and the inherently non-game approach.

Firstly, the new design of Makahiki is to provide an "ecosystem" of interrelated games in a flexible configuration. The set of pluggable game components (``Gamelet'') includes mechanic gamelets such as Leaderboard, Quest, Badge, WallPost, MyAchievement, and Prize, and mini-game gamelets such as Activity game, Daily energy goal game, Raffle game, and Water game. The gamelets are interrelated with each other. For example, players earn points by participating in the Activities gamelets to learn about sustainability literacy; the Daily energy goal gamelet to conserve energy; the Water gamelet to conserve water. The points can be ``spend'' through the participation in the Raffle gamelet. The Quest mechanic gamelet provides guidance to engage players to earn more points and badges. Makahiki also support the extension of adding new gamelet by providing the gamelet plugin interface. A Game instance can be instantiated by Makahiki with a specific configuration file that specify the game settings, gamelets they want to include and how to layout them in the site.

Secondly, Makahiki will implement the responsive web design that provides a consistent game experience across different platforms from computer screens to tablets, and to mobile phones. The  initial evaluation of 2011 Kukui Cup shows that the mobile experience is crucial for user adoption and interaction in the context of energy game. Designing one consistent web interface instead of two, one for computer and one for mobile, will significantly lower the development cost yet still provide consistent access from the convenience of mobile devices. 

Thirdly, Makahiki will include real-time game analytics that could be useful to provides quantitative instrumentation that enables us to track when, where, and for how long each player accessed each page of the site and the interaction with each game components (gamelets).  Unlike generic web server logs, this feature could track per-player game-specific behaviors. For example, instrumentations could enable us to determine how players allocated and deallocated tickets to the Raffle game, or whether they watched, paused, or skipped over the video portion of a Activity game.

Fourthly, Makahiki includes default set of general sustainability related educational contents that provide a pedagogical organized materials to improve sustainability literacy. Some examples are: educational video about renewable energy, tips and instructions on conservation of energy and water, and various commitments to sustainability practices. In addition to the general contents that serve the basis of a game for sustainability education, Makahiki provides an easy to use admin interface to add any organizational specific contents to the game.

Lastly, Makahiki is designed to deploy directly in the cloud-based Platform As A Service (PAAS), such as Heroku. Our goal is to provide the easiest adoption approach that organizations will not need to provide their own technical resources to manage their instances, instead, the tailed instances will be available in the cloud for them as they need it. To use the the cloud-based instances, an organization just need to configure the game settings, customize the contents, and be ready to launch the game.

\section{Evaulations}
% While the previous section details the problem you are addressing, your job here is to translate this into research goals and corresponding methods. Each goal should briefly indicate how you are going to solve the problem, i.e., the research method(s) you will use. Goals should be operational; i.e., if you later claim to achieve your goal, you should be able to match your solution against the goal statement. Then describe what contributions you expect to make if you satisfy these goals. Note: some authors may prefer to combine problem statements, goals, methods  and contributions into a single section.
%
% We cannot overstate how important it is to have clear goals. When problems, goals, methods and contributions are not clearly stated, readers will be unable to evaluate your solutions. Questions a CHI reader should be able to answer after reading this section are:
% * What are the specific goals being pursued? 
% * Do these goals actually help solve some or all of the stated problem(s)?
% * Has the author stated how s/he will achieve this goal (i.e., the method)?
% * Are the goals actionable, i.e., will we know when a goal is actually attained?

To achieve the research goal of building a customizable game engine, we plan to use Makahiki to create several sustainability game instances for organizations with different requirements. One instance is the recurring 2012 Kukui Cup energy competition in University of Hawaii. A second instance is an energy challenge for the international student resident hall also in University of Hawaii. In this instance, the energy meters will not be available and thus the real-time energy monitoring component will not be useful. In order to compete in energy consumption, Makahiki need to support the manual energy data input. A third instance wants to create a challenge about water use in addition to energy, which requires the extending of a new game in Makahiki. A four instance is a middle school with significantly different needs for educational content. We will evaluate how well the Makahiki engine can be adapted to support all four instances of sustainability games.

In addition to the extensive instrumentation of the gamelet usage, we plan to survey and interview both the participants and the organizers of the games to evaluate how well users perceive the sustainability games produced by Makahiki engine.

To evaluate goal of game related research test bed capability of Makahiki, we plan to introduce ``treatment'' in some of the game instances described above. We will devise some game mechanic hypotheses that are interesting, and see if the test bed could be used to adequately test the hypotheses. For example, to evaluate the effectiveness of the badge mechanics, we could create a ``A-B testing'' case that the badge gamelet is configured to be only available to one of the two groups of players. The game analytics data will be used to analyze how effective the badges are to motivate player's participation.

\section{Expected Contributions}

% Use this section to connect your research approach back to the problem statement. This should be a short section that conveys what you anticipate as results or outcomes from your dissertation project and how it will contribute to the HCI research community.

We believe that sustainability education and conservation behavior change require many efforts from multiple disciplines. The contribution of Makahiki is to minimize the barriers of implementing a serious game in sustainability by providing an open source, easily customizable, cloud-based production quality game engine to the vast organizations.

The analytics and A-B testing capability of Makahiki also makes it not only a game engine, but also an experimental testbed for understanding the impact of game techniques in the context of sustainability.

\section{Research Status}

% Applicants to the Doctoral Consortium should have begun their research, but should not have completed it in its entirety. You should briefly state where you currently are in your university’s PhD program. We understand that different universities may organize their programs quite differently, so feel free to give some background if this will help you to be clear. Remember that we are seeking candidates who have an approved dissertation research topic and are carrying out their work, but who have enough work ahead of them that they can benefit from the exchanges and discussions that will take place at the Consortium. Some points you may want to include are:
% * What kind of academic program you are in
% * How you primarily identify yourself (e.g., computer scientist, ethnographer, social scientist, etc.)
% * How many years you have been in that program, and how many years you anticipate you have left before graduating 
% * Whether you have completed your candidacy / research proposal stage (if required at your university)
% * A brief summary of the state of your research (e.g., what you have done vs. what you  have left to do)
% * What you hope to gain from attending the Doctoral Consortium

% Clearly state what you have done and what you have left to do. Summarize the most important findings thus far, and make it clear how these findings match and inform your original problems and goals. Include a short argument as to why these findings are important. Include references and brief descriptions to key publications (if any) arising from your thesis work. State how much of your actual thesis document is written, and what form it is in (e.g., outline, rough draft, etc.)
%
% In addition to describing your current status (as of submission), please also include a paragraph about your future plans, for example the research activities remaining, and how much time you expect these to take, and what sorts of assistance you hope to obtain through your participation in the Consortium.

I am currently in the second year of the PhD portion in the Computer Science program of University of Hawaii. I have a Master degree in Computer science with a thesis on a new approach of database query optimization. I had finished all coursework and passed the written PhD qualify examination.  I plan to defend my dissertation proposal in May 2012.

I had co-developed the initial version of the Makahiki engine and participated in the experiment of 2011 Kukui Cup and the data analysis afterward. I  am currently designing and implementing the new architecture of Makahiki described in this paper. Using the newly re-designed engine, we plan to deploy a second resident hall energy challenge game in Fall 2012 as part of the student resident life annual program in University of Hawaii, and provide support to the deployment in Hawaii Pacific University and other organizations in Fall 2012. We also plan to create control groups within the competition games to collect data about different game mechanics and their relationship to sustainability education and conservation. 

If my experiment proceeds according to plan in Fall 2012, I will evaluate how well the Makahiki engine can be adapted to different organization's need and the effectiveness of game research test bed from the data we collect from the experiment. I hope to finish my dissertation in the Summer of 2013. 

Attending the consortium experience will give me opportunity to incorporate feedbacks and suggestions provided by the field experts into my current research and upcoming experiment. Any improvement to the Makahiki engine will benefit the continuous energy education and conservation effort in University of Hawaii and sustainability efforts in various organizations.

\bibliographystyle{abbrv}
\bibliography{csdl-trs,gamification,sustainability}  

\balancecolumns

\end{document}
