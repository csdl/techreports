\appendix
\chapter{Onboarding Evaluation Script}
\label{appendix:script}

(Before they come in, we need their UH username and their name)

\subsection{Pre-Test}

(Web browser should be open to Google or something ``neutral'')

My name is $<$name$>$ and I'm here to guide you through this session.

We asked you to come here to try using a web site that we are working on so that it works as intended.  This session should take about an hour.

The first thing I want to make clear right away is that we are testing the web site, not you.  You can't do anything wrong here.  In fact, this is probably the one place today where you don't have to worry about making mistakes.

Because of that, I want to ask you to be as honest as possible.  We are doing this to find ways to improve the site, so we need to hear your reactions, no matter how brutally honest they may be.

As you go through the site, I'm going to ask you as much as possible to think out loud.  Say exactly what's going through your mind, what you're trying to do, and what you see.

If you have any questions as we go along, just ask them.  I may not be able to answer right away because we're interested in what people do when they don't have someone to guide them through.  When the session is done, I will try to answer any questions that may have come up.  If you need to take a break at any time, please let me know.

The computer you are using has a screen and voice recording program.  With your permission, we're going to record your actions on the screen and our conversation.  These recordings will only be used to help us figure out how to improve the site, and it won't be seen by anyone except the people working on this project.  Don't worry about the camera on the computer.  We are not videotaping our session.

(Introduce other observers, if there are any)

If you're okay with this, I'm going to ask you to sign a permission form.  It just says that we have your permission to record your voice and actions and that it will only be seen by people working on the project.  We can provide you with a copy of this form if you'd like.

(Give permission form and pen)
(Start ScreenFlow)

Do you have any questions?

Great!  Before we look at the site, I want to provide you with a little context. In October, we plan on having a competition in the Hale Aloha dorms called the Kukui Cup.  We plan to get the word out by having events in the dorms and by putting up posters.  The URL for the website you are about to use will be displayed around campus.  That's all I'll tell you for now.

(If the user is not a freshman in the dorms, we want them to pretend that they are)

Okay, now we are going to look at the web site.  Starting from the landing page, I want you to do as much as possible on the site for a few minutes.  Please remember to talk out loud and say what's on your mind.  I'll be here to give you a few reminders.

(click bookmark for the landing page)

\subsection{Post-Test}

Good.  Before you go, I'd like to ask you a few questions about your background and the website.

What do you think about social games on Facebook?  Do you play them?  Why or why not?

How many of your friends (online and off) play games online?  What kind of games do they play?

What did you think about the Kukui Cup website?  What about the background?  Did you find it childish or cartoony?

Is this something you think you or your friends would like to participate in?  Why?

What issues did you have while using the web site?

What can we do to improve the web site?

(Ask for questions from observers)

Do you have any questions for me?

\chapter{Post-Beta Survey Results}
\label{appendix:beta-survey}

Q1: What aspects did you like most about the Kukui Cup beta test?

\begin{itemize}
  \item Repeating the Kukui Cup experience in other locations (such as public schools).
  \item Competitive aspect
  \item Points, being in competition with other teams and exhorting teammates to try harder if they were lagging behind.
  \item Liked videos, game stuff, trying for the prizes, and the competition part was fun, especially the real-time feedback and standings.
  \item Educational aspect
  \item Learning about energy
  \item Enjoyed learning about energy and conserving energy
  \item Always new content
  \item Quests and unlocking new levels
  \item Liked site interactivity
  \item Opportunity to see what is going on at UH
  \item Kukui Cup will be helpful to students
  \item Intensity (time period and level of engagement required)
  \item Smart Grid board
  \item Gamification worked well
  \item Video length of 2-3 minutes was good.
\end{itemize}

Q2: What did you like least about the Kukui Cup beta experience?

\begin{itemize}
  \item Problems accessing visualizations in the canopy
  \item Worry that college students may not find our activities sufficiently stimulating compared to other things available to them, but competition could help that
  \item Information might be too advanced for average incoming freshman not interested in sustainability (based on showing some stuff to a friend)
  \item Mobile interface, should be more like desktop website
  \item commitments. Too easy to cheat (and user reports signing up just to get the points)
  \item Having to wait for commitments to end to collect points
  \item Not enough questions to answer, could do all of it in one sitting, but shied away from the activities that would take too long so perhaps they weren't worth enough points?
  \item More transparency on points awarded for advanced activities
  \item Did not care for some of the survey questions, particularly CNS (note: these are a set of pre-survey questions that were given out for another thesis).
  \item Beta test seemed to short
  \item Initially thought having 1st round mostly during weekend was bad, but ended up having more free time during weekend than during the week.
  \item Went through most activities in round 1. Might be hard to maintain their interest unless there is new info and challenges as time goes on.
  \item Worried about amount of manual work required by administrators, might want to switch to automated scheme.
  \item Page navigation took getting used to.
  \item Would have liked reminders: allocate raffle tickets, end of rounds, event.
\end{itemize}

Q3: What suggestions (big or small) do you have for improving the Kukui Cup for our October launch at UH?

\begin{itemize}
  \item Additional ways for players to get more points, and possibly increasing the point value of activities in later rounds so late starters could catch up.
  \item Perform a 1 week trial run to generate buzz. Give out prizes, but then reset scores to zero.
  \item Push competitive aspect of Kukui Cup as much as possible. Possibly have comical ``SportsCenter'' type daily wrap up show to highlight certain efforts.
  \item Allow participants to play the game a little more before getting let into the canopy. Higher point minimum for canopy access. 
  \item Scoreboards rotate too fast
  \item More prominence for Secrets of the Kukui Cup video
  \item Canopy was boring compared to SGG activities
  \item Some videos were too long and/or boring, leading to just trying to answer the question without watching the video and just skimming if the answer was unknown
  \item Fix IE problem, many students use IE and UH computer labs default to IE
  \item Teams play a big role in motivation, suggests connecting to team leaders [will these be the RAs? we shall see...]
  \item Add reward component to Canopy to continue motivation
  \item Give out points or badge for useful feedback/bug reports
  \item Only award prizes after survey is complete [won't work since we are decoupling the survey from the competition]
  \item What happened to the post test on energy literacy?
  \item More (optional) email reminders
\end{itemize}

Q4: What did you find to be your primary motivators and de-motivators for participating in the beta test?

\begin{itemize}
  \item Demotivator: lack of activities available in round 2 due to completing most of them (except advanced activities) in round 1.
  \item Motivator: being able to actually affect energy use (during the October competition)
  \item Demotivator: lack of real energy data for competition
  \item Motivator: point total. Would have also been highly motivated by real energy data if that had been available. Suggests awarding more points for energy conservation compared to activities since energy conservation will be harder.
  \item Motivators: points, beating other team members, prizes
  \item Motivators: points
  \item Motivator: points, not wanting to be lowest scoring member of team
  \item Motivator: prizes
  \item Motivator: prizes, but after winning a prize completely stopped playing (though partly due to time constraints)
  \item Demotivators: commitments are confusing, mobile interface
  \item Motivator: videos help inform, thereby motivating activity
  \item Motivator: interested in this area, but didn't know much going in
  \item Motivator: initially ``Spouse PhD badge/prize'', but later the engaging interface and competition kept up motivation
\end{itemize}

Q5: Any other comments you'd like to share with us?

\begin{itemize}
  \item Content is wonderful, but can be improved over time.
  \item Trash is Treasure video narrator accent made video hard to follow.
  \item Repeating content to allow it to sink in is beneficial.
  \item Interested to hear if the 3 week competition can have any long lasting impacts on energy conservation
  \item Suggestion to pursue financial motivations: floor pays more or less money based on energy use, with running total and projected cost available to all floor residents.
  \item Surprised by how much they wanted to play
  \item Just due to beta test, being more conscious about energy use
  \item Video content should be more ``modern''
  \item Starting competition over a weekend is hard [shouldn't be a problem in October]
  \item Wish for questions to be multiple choice or T/F so feedback would be instant.
  \item Waiting for answer verification not fun, and leads to uncertainty of whether points were counted or not
  \item Good luck!
  \item I was so impressed!
  \item Great job on the site! It's very well done, especially some of the visual design aesthetics. I don't need to call it a ``purely functional'' site or anything.
\end{itemize}