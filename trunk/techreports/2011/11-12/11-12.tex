\documentclass[11pt,final]{article}
%\documentclass[11pt,draft]{article}

%%% Load some useful packages:
%% "New" LaTeX2e graphics support.
\usepackage{graphicx}
%%	using final option to force graphics to be included even in draft mode
%\usepackage[final]{graphicx}
%% Tell graphicx the default directory for all figures
%\graphicspath{{figures/final/}}

%% Enable subfigure support
\usepackage{subfigure}

%% Make subsubsections numbered and included in ToC
\setcounter{secnumdepth}{3}
\setcounter{tocdepth}{3}

%% Package to linebreak URLs in a sane manner.
\usepackage{url}

%% Define a new 'smallurl' style for the package that will use a smaller font.
\makeatletter
\def\url@smallurlstyle{%
  \@ifundefined{selectfont}{\def\UrlFont{\sf}}{\def\UrlFont{\small\ttfamily}}}
\makeatother
%% Now actually use the newly defined style.
\urlstyle{smallurl}

%% Define 'tinyurl' style for even smaller URLs (such as in tables)
\makeatletter
\def\url@tinyurlstyle{%
  \@ifundefined{selectfont}{\def\UrlFont{\sf}}{\def\UrlFont{\scriptsize\ttfamily}}}
\makeatother

%% Provides additional functionality for tabular environments
\usepackage{array}

%% Provides additional functionality for enumeration environments
\usepackage{enumitem}

%% Puts space after macros, unless followed by punctuation
\usepackage{xspace}

%%% Personal macros
%% Tired of typing CO2 so many times, requires xspace package
\newcommand{\COtwo}{CO\ensuremath{_2}\xspace}
%% Formatting W, Wh, kW, kWh properly as units
\newcommand{\W}{\,W\xspace}
\newcommand{\Wh}{\,Wh\xspace}
\newcommand{\kW}{\,kW\xspace}
\newcommand{\kWh}{\,kWh\xspace}

%% Make margins less ridiculous
\usepackage{fullpage}

%% Allows insertion of fixme notes for future work
\usepackage[footnote, nomargin]{fixme}

%%%% Turned off for tech report, should be turned on for research portfolio
%% Turn on double spacing
%\usepackage{setspace}
%\doublespacing

%% Make URLs clickable
%\usepackage[colorlinks, bookmarks=false]{hyperref}
\usepackage[colorlinks, bookmarks=true]{hyperref}

%% Since I'm using the LaTeX Makefile that uses dvips, I need this
%% package to make URLs break nicely
\usepackage{breakurl}

%%% End of preamble
\begin{document}

\title{Results from Energy Audit of Hale Aloha}
\author{Robert S. Brewer \\
Collaborative Software Development Lab \\
Department of Information and Computer Sciences \\
University of Hawai`i \\
Honolulu, HI \\
rbrewer@lava.net \\
\\
CSDL Technical Report 11--12 \\
Version 1.1 \\
\url{http://csdl.ics.hawaii.edu/techreports/11-12/11-12.pdf}
}
\date{January 2012}

%%% Create the title page from all the information above. Note that the
%%% titlepage is outside the front matter.
\maketitle

\section{Summary}

Here are the key findings of the Hale Aloha energy audit conducted jointly between Housing and the Kukui Cup team:

\begin{itemize}
\item It appears that there are no large, pervasive ``hidden'' electrical loads on the floors. We did see some differences in power use after trying to unplug all loads, but most of this is likely due to the scattered residents present during the audit. Before the 2012 Kukui Cup (possibly summer 2012), we should go back and run an audit of each tower when no residents are present.
\item Refrigerators appear to be largest source of electricity use in the towers. Fridges are on round the clock, and with an average of 1 to $1.25$ per room, they add up. We should get more accurate usage data from a sample of fridges in the towers using plug load meters, and figure out how address the fridge issue in the 2012 Kukui Cup.
\item The meter in the Lokelani C lounge was producing bad data because the CTs were installed on the incoming main feed lines before it split off to the panel. This was corrected on 1/23/2012. All of the bad data was purged from the WattDepot database, and accurate data is now being collected.
\end{itemize}

\section{Introduction}

From December 19 to December 22, 2011, a joint team from UHM Student Housing and the Kukui Cup project conducted an energy audit of the four Hale Aloha towers, one tower per day. Housing had the following goals:

\begin{enumerate}
\item Making a count of the types of appliances residents have in their rooms.
\item Unplugging unused \& unneeded appliances for residents who were away for winter break.
\item Noting any violations of rules, such as attaching things to fire sprinkler heads, or having an unapproved air conditioner.
\end{enumerate}

The Kukui Cup team was also interested in the results of item 1, and our other goals were:

\begin{enumerate}[resume]
\item Determine if there are any hidden loads that are not under the control of residents.
\item Figure out why Lokelani C's energy use is so much higher than other lounges.
\end{enumerate}

\subsection{Auditing process}

We used the following auditing process:

\begin{itemize}
\item One team examines each room on the first floor of the lounge. They record the appliances present in the room on the worksheet created by Mike Weakley (this will probably be an underestimate since some residents took things home for break, or moved out ``contraband'' appliances).
\item For unoccupied rooms, all appliances are unplugged. If fridges are found with perishable items, or other devices that shouldn't be left unplugged (like fish tanks), a note is made to replug them after the audit is complete.
\item For occupied rooms, the resident is asked to unplug everything until the audit is complete.
\item Once we believe everything has been unplugged, we examine the power readings on the two meters that monitor each lounge:
	\begin{itemize}
	\item The meter in the IDF/telco room only handles the new plugs installed in each room from the recent renovation. So, once everything is unplugged, then the meter should display only the power used by the meter itself. If it doesn't, then we missed a plug load somewhere, or the panel controls some unexpected hidden infrastructure. We can also take this opportunity to measure how much power the meter itself consumes by turning off all the breakers except for the one for the meter itself.
	\item The lounge panel controls room plugs, lights, and other things. We will systematically turn off each breaker to see what it actually controls, and how much power it is using (at least at that moment) by watching the meter display.
	\end{itemize}
\item Once the meter readings are complete, all devices that need continuous power are plugged back in.
\end{itemize}

To speed up the process, two teams worked in parallel to audit the two floors that make up each lounge.

\section{Results}
\label{sec:results}

\subsection{Panel audits}

After each room was examined for devices to unplug, the overall power usage from each meter was recorded on a per-panel basis. \autoref{tab:power-per-panel} shows the power data that was collected.

\begin{table}[htbp]
	\centering
	\scriptsize
		\begin{tabular}{| l | c | c |}
			\hline
			Lounge & Panel & Power after unplugging (W) \tabularnewline \hline \hline
			Mokihana D & telco & 0 \tabularnewline \hline
			Lokelani D & telco & 3 \tabularnewline \hline
			Lehua B & telco & 4 \tabularnewline \hline
			Lokelani C & telco & 4 \tabularnewline \hline
			Lehua D & telco & 4.5 \tabularnewline \hline
			Lehua A & telco & 4.7 \tabularnewline \hline
			Lehua E & telco & 4.8 \tabularnewline \hline
			Ilima B & telco & 4.8 \tabularnewline \hline
			Lehua C & telco & 5.4 \tabularnewline \hline
			Lokelani E & telco & 5.7 \tabularnewline \hline
			Lokelani A & telco & 6 \tabularnewline \hline
			Ilima A & telco & 6.8 \tabularnewline \hline
			Mokihana A & telco & 6.8 \tabularnewline \hline
			Ilima D & telco & 7 \tabularnewline \hline
			Lokelani B & telco & 7 \tabularnewline \hline
			Mokihana B & telco & 7 \tabularnewline \hline
			Mokihana C & telco & 7 \tabularnewline \hline
			Ilima C & telco & 7.5 \tabularnewline \hline
			Mokihana E & telco &7.7 \tabularnewline \hline
			Ilima E & telco & 8 \tabularnewline \hline \hline
			Lehua B & lounge & 65 \tabularnewline \hline
			Lehua E & lounge & 78 \tabularnewline \hline
			Mokihana A & lounge & 106 \tabularnewline \hline
			Lokelani D & lounge & 129 \tabularnewline \hline
			Lehua D & lounge & 130 \tabularnewline \hline
			Ilima B & lounge & 133 \tabularnewline \hline \hline
			Ilima A & lounge & 183 \tabularnewline \hline
			Lokelani A & lounge & 195 \tabularnewline \hline
			Mokihana D & lounge & 225 \tabularnewline \hline
			Ilima C & lounge & 227 \tabularnewline \hline
			Ilima E & lounge & 229 \tabularnewline \hline
			Ilima D & lounge & 230 \tabularnewline \hline
			Lehua C & lounge & 243 \tabularnewline \hline
			Mokihana C & lounge & 252 \tabularnewline \hline
			Mokihana B & lounge & 253 \tabularnewline \hline
			Lokelani B & lounge & 287 \tabularnewline \hline
			Mokihana E & lounge & 335 \tabularnewline \hline
			Lehua A & lounge & 682 \tabularnewline \hline
			Lokelani E & lounge & 694 \tabularnewline \hline \hline
			\emph{Lokelani C} & \emph{lounge} & \emph{1834} \tabularnewline \hline
		\end{tabular}
	\caption{Power use per panel after unplugging in increasing order}
\label{tab:power-per-panel}
\end{table}

From \autoref{tab:power-per-panel}, we can see that for all the telco panels, we were able to get the power down to the roughly 5\W that our meters appear to consume. This confirms that the new panels in the telecom/IDF rooms only provide power to plug loads in resident rooms. This is not a surprise, but it is good to be sure of it.

The lounge panels are more complicated, since in addition to the resident room plugs and overhead lights, they control other loads. As discussed in \autoref{sec:lokelani-c}, the data from Lokelani C (last row in the table) should be ignored. \autoref{tab:power-per-panel} shows that the lounge power can go as low as 65\W, or as high as 694\W.

Unfortunately, it was not possible in this audit to track down all the loads from the lounge panels for a few reasons. Most lounges had one or more residents present, and we know based on readings from the telco panels that they did not always unplug all of their devices. They also were free to make use of the bathrooms during the audit, which contain lights and plugs. The inserts on the lounge panels that list what each breaker controls were mostly out of date or missing, making it difficult to tell whether the load on a breaker was from a resident room or something else. With the residents present and the time constraints of this audit, it was not possible to track down unexplained power from most breakers on the lounge panels.

The lounges from Lehua B to Ilima B in \autoref{tab:power-per-panel} show loads primarily from the networking equipment and the Shark meter, with a few scattered smaller loads (20--30\W). The rest of the lounges have increasing numbers of loads beyond the networking equipment, some quite sizeable (up to 200\W). It seems quite likely that some of these loads were residents that didn't unplug their devices, plugged them back in before the audit was complete, or devices that were missed during a sweep of the room. However, on the basis of this audit, we cannot rule out that there might be loads totaling as much as 640\W in a couple lounges. Also, since the audit was performed at a single time, if there are non-resident loads that are time based, they could have been missed by the audit.

To truly determine the base load for each lounge and any differences between lounges, we should conduct another energy audit before the 2012 Kukui Cup. This energy audit should take place when there are no residents and no maintenance work going on in each tower. Hopefully this could take place in the summer of 2012. Long-time employees like Alvin may also be able to shed light on where some loads are coming from.

\subsection{Refrigerators}

Based on the appliance spreadsheet produced by Michael Weakley, Lehua, Ilima, and Lokelani have an average of about 1.25 fridges per room, while Mokihana has an average of 1 fridge per room. Based on this, it seems likely that a significant portion of the base load in Hale Aloha comes from fridges. I noticed one fridge had a yellow EnergyGuide label that said it used 300\kWh per year, which is about 34\W of power permanently on. \autoref{tab:fridge-distribution} shows the distribution of fridges across the four towers.

\begin{table}[htbp]
	\centering
		\begin{tabular}{| c || c | c | c | c | c | c |}
			\hline
			Tower & \# fridges & avg.\ fridges/room & 0 fridges & 1 fridge & 2 fridges & 3 fridges \tabularnewline \hline \hline
			
			Mokihana & 149 & 1.06 & 19 & 93 & 28 & 0  \tabularnewline \hline
			
			Lehua & 176 & 1.26 & 7 & 90 & 43 & 0 \tabularnewline \hline
			
			Ilima & 176 & 1.26 & 7 & 91 & 41 & 1 \tabularnewline \hline

			Lokelani & 177 & 1.26 & 12 & 79 & 49 & 0 \tabularnewline \hline

		\end{tabular}
	\caption{Fridge distribution by tower}
\label{tab:fridge-distribution}
\end{table}

In the spring 2012 semester we should get more in-depth data on actual energy use by a sample of fridges using plug load meters over an extended period of time (a week or a month). This will give us a better idea of how much energy is being consumed by fridges.

If fridges do turn out to be a major consumer of energy, we should think of creative ways of reducing that energy load, either as part of the 2012 Kukui Cup or even earlier. For example, if only 1 fridge was allowed per room, 161 fridges would be removed, though that could lead to the purchase of larger fridges. We could also spec out the most energy efficient fridge available and steer residents towards that when they are outfitting their room. Other ideas: keep more food that doesn't require refrigeration? Another great survey topic would be to figure out what people keep in their fridge. For example, if people use them primarily to keep their drinks cold, there might be other ways to chill drinks as needed, rather than 24$\times$7.

\subsection{Lokelani C}
\label{sec:lokelani-c}
The power data produced by the meter in the Lokelani C lounge appears highly incorrect. We knew that this meter was reporting almost double the power use of the other lounges, but thought it could be due to either abnormally high plug loads, or some sort of hidden infrastructure. However, even when all breakers on the panel were in the off position, the meter showed about 1800\W of power in use (fluctuating from 1769 to 1834 watts)! On 1/23/2012 an electrician was brought in to troubleshoot the problem. The lounge panels have main feeds that come up from the basement, and each of these lines splits in two to power two lounge panels. The CTs on the Lokelani C panel had been incorrectly installed on the incoming power lines \emph{before} they split off to the panel, instead of after. So the meter was functioning properly, it was just metering both the Lokelani C panel load and the load from Lokelani D. This explains the data we observed from Lokelani C: usage roughly double other lounge panels.

The electrician moved the CTs to after the split for the panel, and now the Lokelani C panel reports power usage in line with other lounges. All of the inaccurate data has been scrubbed from the WattDepot repository, and new data is being recorded now. Theoretically, the correct historical energy data for Lokelani C could be recovered by subtracting the values recorded for Lokelani D from the inaccurate Lokelani C values. However, in the interests of accuracy, no data has been recreated in this way.

\subsection{Network equipment}

The networking equipment (router and switch, and occasionally power over Ethernet injectors) located in the telecom/IDF room seems to use different amounts of power in different towers. For example, the equipment in Lehua D's telecom room was recorded as 40\W, while Ilima E uses 180\W.

For the audits of Lokelani and Mokihana we used a Belkin Conserve Insight plug load meter to measure the power usage from each piece of network equipment in the telecom room, and the result was always within 10\% of what we measured using the circuit breaker method. This provides some assurance that our meter installations are providing fairly accurate data.

\subsection{Other Observations}

Lehua and Ilima have 2 outlet boxes per wall in rooms, while Lokelani and Mokihana have 3 outlet boxes per wall. It might be worth examining if having more outlets makes residents more likely to plug things in and thereby use more electricity. It is also unusual that this difference spans phase 1 (Lehua and Lokelani) and phase 2 (Ilima and Mokihana), unlike all other differences in construction that I'm aware of.

During the systematic powering off of each breaker, each breaker --- even those with no load connected --- seems to contribute a small amount of power (less than 1\W). This does not impact the results, but useful to know for future auditors.

The Shark meters seem to draw about 5\W, though that number is approximate because the meters probably aren't very accurate at low power levels. The 40 meters installed in Hale Aloha use about 200 W continuously, roughly the equivalent of an Xbox, or the amount generated by a single solar panel.

During the systematic powering off of each breaker on the telco panels, when roughly 60\% of the breakers are powered off, observed load goes to 0\W. This is true even though the breakers being powered off had no load connected (everything had been unplugged).

After the room devices were plugged back in, the load is often higher than when we started. This is probably in part because all the reconnected mini-fridges have gone above their thermostat setting, and all the compressors are running at once.

Four unapproved air conditioners were found during the audit, and there might have been more which were removed in anticipation of the audit. Since having an unapproved AC can result in a large fine, those using unapproved AC units might decide to leave them on 24$\times$7 since they will be fined the same amount if they get caught regardless of how much they used it.

\section{Acknowledgements}

Mahalo to George Lee and Yongwen Xu from the Kukui Cup team for working on the audit, and Michael Weakley, Roland Castillo and all the student workers from Housing who made this possible.

%% Just for demo purposes, include all entries from bib file
%\nocite{*}

%%% Input file for bibliography
%\bibliography{sustainability}
%% Use this for an alphabetically organized bibliography
%\bibliographystyle{plain}
%% Use this for a reference order organized bibliography
%\bibliographystyle{unsrt}
%% Try using this BibTeX style that hopefully will print annotations in
%% the bibliography. This will allow me to make notes on papers in the
%% BibTeX file and have them readable in the references section until
%% I turn them into a conceptual literature review 
%\bibliographystyle{annotation}

\end{document}