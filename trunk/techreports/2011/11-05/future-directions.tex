\chapter{Conclusion and Future Directions}

\section {Conclusion}

A key findings from the Gartner report "Gamification Primer: Life becomes a game" \cite {gartner2010} is that games often model the real world, gamification has emerged as a recognizable trend and impact so many areas of business/society whereas exists many opportunities and risks. They recommend "Only organizations with a high risk tolerance should attempt to broadly exploit this trend today; organizations with a lower risk tolerance should watch this trend develop and/or begin small pilot applications."

There was a definite feeling of infancy of gamification, be it the definition of gamification or the effectiveness of gamification, there are debates from different areas of business.  Most of gamification thought leaders agree that the current state of gamification is mainly focus on extrinsic rewards such as points, badges and leaderboards,  and this novelty of simple gamification will have its effectiveness in user engagement before the novelty worn off. Many also see the bigger potentials of sustainable gamification with deeper researches in the intrinsic rewards from good game designs. Sebastian Deterding even introduce the term "gameful design" (design  for  gameful  experiences)  as  a  potential  alternative  to  "gamification". \cite {Deterding2011mt}. He argues that,  "given the industry origins and the debates about the practice and design of gamification, 'gameful design' currently provides a new term with less baggage, and therefore a preferable term for academic discourse".

Be it "gamification" or "gameful design", the debate and the above literature surveys warrant broader academic research in this interdisciplinary area that bridges HCI and game studies and other fields to study a wide ranges of gamified applications. The major take away of reading the debates of gamification is that, this is a field rife with anecdotes but little hard data. \cite {Wharton2011}.  "That's why research is valuable -- to get beyond whether gamification is good or bad, and does it work or not." 

One approach to provide empirical research into the gamified application is to collect the application's game related metrics and analyze the effectiveness of the game mechanics applied in the application. The current use of the social game metrics surveyed above will provide a good starting points of analytics of the gamified system.

\section {Future Directions}

The current state of the gamification is focus on the relatively superficial game mechanics, such as point, level, leader board and badges. More and more researchers and commercial service providers are looking more in-depth approach to achieve engagement of whatever industries the gamification is applied on. The followings are a few directions and efforts in furthering the effectiveness of gamification:

1. Social interaction.  With the social games are transforming so many non-gamers into casual gamers in a massively engaging way, the studies of social interaction in game will inevitably benefit the progress in gamification application.

2. Mobility. Mobile devices' ubiquitousness is one of the main reason that the mobile games are invading people's every minute in everyday life. This unique engaging factor should also be gamification's research topic. 

3. Analytics. although most of the commercial services provide some kinds of engagement metrics and behavior analytics, it is still an new area that need broader, deeper researches and experiments to find out what works and how it works. 


Because gamification is relatively new field, the development of new thoughts and new areas of gamified application will emerge and change rapidly. To closely follow the future development in this field, a growing list of gamification thought leaders and their biographies is provided in the appendix as the future readings and researches.
