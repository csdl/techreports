%%%%%%%%%%%%%%%%%%%%%%%%%%%%%% -*- Mode: Latex -*- %%%%%%%%%%%%%%%%%%%%%%%%%%%%
%% abstract.tex -- 
%% Author          : Joseph Dane
%% Created On      : Fri Oct  8 21:04:34 1999
%% Last Modified By: Joe Dane
%% Last Modified On: Wed Oct 20 12:15:34 1999
%% RCS: $Id$
%%%%%%%%%%%%%%%%%%%%%%%%%%%%%%%%%%%%%%%%%%%%%%%%%%%%%%%%%%%%%%%%%%%%%%%%%%%%%%%
%% Copyright (c) 1999 Joseph Dane
%%%%%%%%%%%%%%%%%%%%%%%%%%%%%%%%%%%%%%%%%%%%%%%%%%%%%%%%%%%%%%%%%%%%%%%%%%%%%%%
%% 

\begin{abstract}

  Effective program size measurement is difficult to accomplish.  Factors
  such as program implementation language, programmer experience and
  application domain influence the effectiveness of particular size metrics
  to such a degree that it is unlikely that any single size metric will be
  appropriate for all applications. This thesis introduces a tool, LOCC,
  which provides a generic architecture and interface to the production and
  use of different size metrics.  Developers can use the size metrics
  distributed with LOCC or can design their own metrics, which can be
  easily incorporated into LOCC.  LOCC pays particular attention to the
  problem of supporting incremental development, where a work product is
  not created all at once but rather through a sequence of small changes
  applied to previously developed programs.  LOCC requires that developers
  of new size metrics support this approach by providing a means of
  comparing two versions of a program.  LOCC's effectiveness was evaluated
  by using it to count over 50,000 lines of Java code, by soliciting
  responses to a questionnaire sent to users, and by personal reflection on
  the process of using and extending it.  The evaluation revealed that
  users of LOCC found that it assisted them in their development process,
  although there were some improvements which could be made.


\end{abstract}
