\documentstyle[psfig, twocolumn]{article} % default is 10pt
style/epsf.sty
\setlength{\textheight}{9.45in}
\setlength{\oddsidemargin}{-0.5in}
\setlength{\evensidemargin}{-0.5in}
\setlength{\topmargin}{-0.49in}
\setlength{\footskip}{0.5in}
\def \columnsep{0.2in}
\def \textwidth{7.45in}
\begin{document}
\title{\vspace{-0.75in} Project LEAP:
     Lightweight, Empirical, Anti-measurement dysfunction, and Portable
     Software Developer Improvement}

\author{Philip M. Johnson \\
Collaborative Software Development Laboratory \\
Department of Information and Computer Sciences \\
University of Hawaii \\
Honolulu, HI 96822 \\
johnson@hawaii.edu \\
}

\bibliographystyle{alpha}

\date{}

\maketitle

Project LEAP investigates the use of lightweight, empirical,
anti-measurement dysfunction, and portable approaches to software developer
improvement. A lightweight method involves a minimum of process
constraints, is relatively easy to learn, is amenable to integration with
existing methods and tools, and requires only minimal management investment
and commitment.  An empirical method supports measurements that can lead to
improvements in software developer skill.  Measurement dysfunction
refers to the possibility of measurements being used against the
programmer, so the method must take care to collect and manipulate
measurements in a ``safe'' manner. A portable method is one that can be
applied by the developer across projects, organizations, and companies
during her career.

Project LEAP has thus far produced the publically available Leap toolkit
(see http://csdl.ics.hawaii.edu and follow the links to ``Leap'').  This
toolkit is in active use by software engineering students and professional
software developers, who use it to collect, analyze, and archive their
software engineering development data.  Project LEAP has also produced
LOCC, a modular, extensible, grammar-based tool for measuring work product
size, which is similarly available at our web-site. 

Current research involves an experimental assessment of the usability of 
the Leap toolkit, as well as an investigation into the relative merits
of 14 different approaches to size estimation.  Our future plans include
incorporation of software agents for automated data collection, 
comprehensive training materials for distance education, and a public
repository for Leap data definitions, extensions, and integrations with
various development environments. 


\end{document}

