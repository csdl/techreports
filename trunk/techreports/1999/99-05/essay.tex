%%%%%%%%%%%%%%%%%%%%%%%%%%%%%% -*- Mode: Latex -*- %%%%%%%%%%%%%%%%%%%%%%%%%%%%
%% essay.tex -- 
%% Author          : Robert Brewer
%% Created On      : Wed Jan 27 17:17:21 1999
%% Last Modified By: Robert Brewer
%% Last Modified On: Mon Feb  1 14:15:56 1999
%% RCS: $Id: essay.tex,v 1.2 1999/02/02 04:01:40 rbrewer Exp $
%%%%%%%%%%%%%%%%%%%%%%%%%%%%%%%%%%%%%%%%%%%%%%%%%%%%%%%%%%%%%%%%%%%%%%%%%%%%%%%
%%   Copyright (C) 1999 Robert Brewer
%%%%%%%%%%%%%%%%%%%%%%%%%%%%%%%%%%%%%%%%%%%%%%%%%%%%%%%%%%%%%%%%%%%%%%%%%%%%%%%
%% 

\section{Essay Question}
%No more than 700 words, currently exactly 700 words (not counting heading and quote)

\begin{quote}
  {\em As an Aspect Technology Fund grant recipient, how would you contribute
    to the field of technology and promote the spirit of entrepreneurship?}
\end{quote}

As grant recipient, I will be extending a technology which I am developing in
the realm of academia. I am hopeful that this kind of project can serve as a
model for other students and faculty to commercialize their research projects.
My project will also demonstrate that the open source approach is particularly
appropriate for university software entrepreneurs due to factors like the
reduced start up costs required.

Technology transfer from UH to industry is an important way to improve the
economy here in Hawaii. I personally feel that the state's economic doldrums
are in great part due to our excessive reliance on tourism. As the Asian
economic crisis has shown, Hawaii cannot afford to rely so heavily on tourism.
Furthermore, excessive tourism clearly damages Hawaii's fragile environment,
which hurts all of us.

The information technology industry doesn't suffer from these problems.
Customers don't really care where a program or web page was written. For
information technology, Hawaii's geographical isolation is irrelevant.
Information technology has virtually no environmental impact compared to other
industries, which is another advantage.

One of the most common ways to make money in information technology is to
develop software and sell it to customers (with support being an afterthought).
While this can be a very successful model, it does have inherent risks. One of
the biggest risks is the development of the software product. Many companies
have gone bankrupt by underestimating the time and effort required to create a
product at the level of quality that customers expect. After the product is
complete, you have to expend large sums marketing it and distributing it to
customers.  Next, you hope that the customers like your product enough to pay
for it.  Finally, even if you are successful you cannot rest on your laurels:
you must immediately start working on upgrades or new products to insure your
revenue stream.

The open source licensing model for software (which I am using for MCS)
sidesteps many of these problems. In many cases the initial version of the open
source product isn't of ``commercial quality''. Developing this version
requires less effort than its traditional counterpart as it may not do
everything or be free of faults, yet it will still be usable. This is the
moment of truth: is the product good enough that people will download it and
use it?  If so, we immediately start tapping into the benefits of open source.
Since it is free, many people will be willing to try it out. Since the source
code is available, some akamai users who find bugs will decide to fix them and
contribute the fixes back to the project.  Other users will want MCS to do
things that it doesn't do yet; they may decide to go ahead and make the
enhancements themselves instead of waiting for someone else. Since everything
is shared freely, MCS taps into a vast resource of programming expertise at no
cost!

While the open source model helps in the areas of development, marketing, and
distribution, it does so at the expense of the traditional revenue stream:
software sales. The solution is to generate revenue by selling service and
support. The advantage here is that service and support provide a continuing
revenue stream, unlike the traditional model. Because this revenue starts soon
after the release of the product, the startup costs for such a company are
lower than for a traditional company. The continuing revenues also allow an
open source company to fund their next product internally instead of requiring
additional capital infusions. More information about the open source movement
and its success stories can be found at \url{<http://www.opensource.org/>}.

This grant will assist me in encouraging others to take advantage of this
exciting business model and other entrepreneurial opportunities. Many academic
projects fall by the wayside when the research is complete, yet companies like
Netscape and Inktomi have shown that commercialization of research is possible.
I also want to show people both here in Hawaii and throughout the world that
building a company on the open source model is feasible and a great way to give
back to the global Internet community. As a successful researcher in the
Collaborative Software Development Lab here at UHM and as one of the founders
of the local Internet service provider LavaNet, I feel I have the ability to
make all this happen.
% LocalWords:  http opensource LocalWords tex www org
