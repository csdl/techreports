%%%%%%%%%%%%%%%%%%%%%%%%%%%%%% -*- Mode: Latex -*- %%%%%%%%%%%%%%%%%%%%%%%%%%%%
%% 09-02.tex --     ESEM 2009 Submission
%% Author          : Philip Johnson
%% Created On      : Wed Jan 07 14:06:37 2009
%% Last Modified By: Philip Johnson
%% Last Modified On: Thu Jan 08 13:26:15 2009
%% RCS: $Id$
%%%%%%%%%%%%%%%%%%%%%%%%%%%%%%%%%%%%%%%%%%%%%%%%%%%%%%%%%%%%%%%%%%%%%%%%%%%%%%%
%%   Copyright (C) 2009 
%%%%%%%%%%%%%%%%%%%%%%%%%%%%%%%%%%%%%%%%%%%%%%%%%%%%%%%%%%%%%%%%%%%%%%%%%%%%%%%
%% 
\documentclass{acm_proc_article-sp}

%
\def\sharedaffiliation{%
\end{tabular}
\begin{tabular}{c}}
%
\begin{document}

\title{Experiences with the Software Intensive Care Unit}

\numberofauthors{2} 

\author{
  \alignauthor Philip Johnson\\
  \email{johnson@hawaii.edu}
%
  \alignauthor Shaoxuan Zhang \\
  \email{sz@hawaii.edu}
%
  \sharedaffiliation
  \affaddr{Collaborative Software Development Laboratory}\\
  \affaddr{Department of Information and Computer Sciences}\\
  \affaddr{University of Hawaii}
}

\date{01 March 2008}

\maketitle
\begin{abstract}
One goal of the Hackystat Framework is to facilitate the teaching of
software metrics in classroom settings.  To that end, we have conducted
classroom evaluations in 2003, 2006, and 2008.  This paper reports in
detail on our most recent approach to teaching software metrics in the
classroom by way of an approach called the ``Software ICU''.  In this
approach, students learn about ten empirical project ``vital signs'' and
use the Hackystat Framework to put their students projects into a virtual
``intensive care unit'' where these vital signs can be assessed and
monitored.  We conducted a questionnaire-based evaluation that provides
insight into the strengths and weaknesses of this approach, how it compares
to previous approaches using the Hackystat Framework, and promising future
directions.
\end{abstract}

\category{D.2.8}{Software Engineering}{Metrics}[complexity measures, performance measures, software quality measures]

\section{Introduction}

We introduce the motivation for the Software ICU here. 

\section {Related Work}

We've worked on this before \cite{csdl2-07-02,csdl2-03-12,csdl2-03-13}.

\section{The Software ICU}

An overview of how it works.  

\section{Evaluation}

How we evaluated it.

\section{Conclusions and future directions}

Where we will go next.

\section{Acknowledgments}

We will acknowledge some folks here.

\bibliographystyle{abbrv}
\bibliography{csdl-trs}  
\end{document}

