%%%%%%%%%%%%%%%%%%%%%%%%%%%%%% -*- Mode: Latex -*- %%%%%%%%%%%%%%%%%%%%%%%%%%%%
%% thesis-appendix.tex -- 
%% Author          : Robert Brewer
%% Created On      : Tue Jan 10 12:04:50 1995
%% Last Modified By: 
%% Last Modified On: Thu Apr 03 10:55:08 2003
%% Status          : Unknown
%% RCS: $Id: thesis-appendix.tex,v 1.3 2000/03/17 21:28:36 rbrewer Exp $
%%%%%%%%%%%%%%%%%%%%%%%%%%%%%%%%%%%%%%%%%%%%%%%%%%%%%%%%%%%%%%%%%%%%%%%%%%%%%%%
%%   Copyright (C) 1998 Robert Brewer
%%%%%%%%%%%%%%%%%%%%%%%%%%%%%%%%%%%%%%%%%%%%%%%%%%%%%%%%%%%%%%%%%%%%%%%%%%%%%%%
%% 

\appendix
\chapter {Extreme Coverage Evaluation Questionnaires}


\begin{figure}[htbp]
  \centering
  \includegraphics[width=1.0\textwidth]{figs/fig.questionnaire.pre.eps}
  \caption{Pre-Use Questionnaire}
  \label{fig:questionnaire.pre}
\end{figure}

\begin{figure}[htbp]
  \centering
  \includegraphics[width=1.0\textwidth]{figs/fig.questionnaire.post.eps}
  \caption{Post-Use Questionnaire}
  \label{fig:questionnaire.post}
\end{figure}

\chapter {Questionnaire Data}

Note: The fourth question does not apply to the Pre-Use Questionnaire, so
answers are marked with ``n/a''.  Answers to open-ended questions (6-8) are
denoted as ``(Pre-Use)'' for answers to the Pre-Use Questionnaire and
``(Post-Use)'' for answers to the Post-Use Questionnaire.  Questions which
were not answered are marked with ``(no answer)''.  Answers to the
open-ended questions are presented as close to the original answers and
possible.

\pagebreak

\section{Student A}

\begin{tabular}{lcc} \\
 & {\bf Pre-Use} & {\bf Post-Use} \\ \hline
1. Unit tests are very important for creating & Strongly agree & Strongly agree \\
   correctly functioning software. \\ \hline

2. Designing unit tests to support correctly & Strongly agree & Agree \\
   functioning software is hard\\ \hline

3. My current set of unit tests does a good & Agree & No opinion \\
   job of ensuring that my software functions \\
   correctly.\\ \hline

4. JBlanket helps me to write unit tests that & n/a & Agree \\
   ensure the correct functioning of my software.\\ \hline

5. To the nearest 25\%, what \% of the methods & 75\% & 100\% \\
   in your software are currently invoked by \\
   your unit tests?\\ \hline
\\
\end{tabular}

\begin{tabular}{l}
  \begin{minipage}[b]{.85\linewidth}
    6. Please briefly describe one or two of the most significant problems
       you've encountered while designing unit tests.  (Do not include the
       problem of learning how to use unit testing facilities such as JUnit
       or HttpUnit.)\\
    \\
    (Pre-Use)
    \begin{itemize}
      \item Updating unit test has a major overhead even for the smallest change
            in the code.
      \item Thinking for all possible combination of case that
            a unit test class should have.
    \end{itemize}

    (Post-Use)
    \begin{itemize}
      \item Methods w/ 0 lines of code is asked to be covered.
      \item Updating test. \\
    \end{itemize}
  \end{minipage}
  \\
  \begin{minipage}[b]{.85\linewidth}
    7. Briefly describe how access to JBlanket has influenced the way your
    write unit tests.\\
    \\
    (Post-Use) Makes me feel safer to know I'm at 100\%.\\
  \end{minipage}
  \\
  \begin{minipage}[b]{.85\linewidth}
    8. What would you suggest we do to improve the usefulness of JBlanket?\\
    \\
    (Post-Use) (no answer)
  \end{minipage}
\end{tabular}

\pagebreak

\section{Student B}

\begin{tabular}{lcc} \\
 & {\bf Pre-Use} & {\bf Post-Use} \\ \hline
1. Unit tests are very important for creating & Agree & Strongly agree \\
   correctly functioning software. \\ \hline

2. Designing unit tests to support correctly & Agree & No opinion \\
   functioning software is hard\\ \hline

3. My current set of unit tests does a good & Agree & Strongly agree \\
   job of ensuring that my software functions \\
   correctly.\\ \hline

4. JBlanket helps me to write unit tests that & n/a & Strongly agree \\
   ensure the correct functioning of my software.\\ \hline

5. To the nearest 25\%, what \% of the methods & 75\% & 100\% \\
   in your software are currently invoked by \\
   your unit tests?\\ \hline
\\
\end{tabular}

\begin{tabular}{l}
  \begin{minipage}[b]{.85\linewidth}
    6. Please briefly describe one or two of the most significant problems
       you've encountered while designing unit tests.  (Do not include the
       problem of learning how to use unit testing facilities such as JUnit
       or HttpUnit.)\\
    \\
    (Pre-Use)
    \begin{itemize}
      \item Translating html tags to work with unit tests.\\
            Example: $<$br$>$ currently messes up my unit tests.
    \end{itemize}

    (Post-Use) (no answer)\\
  \end{minipage}
  \\
  \begin{minipage}[b]{.85\linewidth}
    7. Briefly describe how access to JBlanket has influenced the way your
    write unit tests.\\
    \\
    (Post-Use) Least amount of code with the most amount of coverage.\\
  \end{minipage}
  \\
  \begin{minipage}[b]{.85\linewidth}
    8. What would you suggest we do to improve the usefulness of JBlanket?\\
    \\
    (Post-Use) Option for unit testing single line methods. (toggle on and off)
  \end{minipage}
\end{tabular}

\pagebreak

\section{Student C}

\begin{tabular}{lcc} \\
 & {\bf Pre-Use} & {\bf Post-Use} \\ \hline
1. Unit tests are very important for creating & Agree & Agree \\
   correctly functioning software. \\ \hline

2. Designing unit tests to support correctly & Strongly agree & Strongly agree \\
   functioning software is hard\\ \hline

3. My current set of unit tests does a good & No opinion & Agree \\
   job of ensuring that my software functions \\
   correctly.\\ \hline

4. JBlanket helps me to write unit tests that & n/a & No opinion \\
   ensure the correct functioning of my software.\\ \hline

5. To the nearest 25\%, what \% of the methods & 25\% & 100\% \\
   in your software are currently invoked by \\
   your unit tests?\\ \hline
\\
\end{tabular}

\begin{tabular}{l}
  \begin{minipage}[b]{.85\linewidth}
    6. Please briefly describe one or two of the most significant problems
       you've encountered while designing unit tests.  (Do not include the
       problem of learning how to use unit testing facilities such as JUnit
       or HttpUnit.)\\
    \\
    (Pre-Use)
    \begin{itemize}
      \item The main problem now is that having to finish all code before
            doing unit tests.  I know can write without it being completely
            done, but the part we're working on is a major part of the
            project.
      \item Other than that, having to remove stuff, logging in through
            HttpUnit is a big hassle which makes the method really long and
            might make it go over 200 lines.
    \end{itemize}

    (Post-Use)
    \begin{itemize}
      \item Null pointers.
      \item Having them work one time, then fail (without changing
            anything), then work again (no change again).\\
    \end{itemize}
  \end{minipage}
  \\
  \begin{minipage}[b]{.85\linewidth}
    7. Briefly describe how access to JBlanket has influenced the way your
    write unit tests.\\
    \\
    (Post-Use) Write it more often to get 100\%.\\
  \end{minipage}
  \\
  \begin{minipage}[b]{.85\linewidth}
    8. What would you suggest we do to improve the usefulness of JBlanket?\\
    \\
    (Post-Use) Not round off \%.  If maybe can go faster.
  \end{minipage}
\end{tabular}

\pagebreak

\section{Student D}

\begin{tabular}{lcc} \\
 & {\bf Pre-Use} & {\bf Post-Use} \\ \hline
1. Unit tests are very important for creating & Strongly agree & Strongly agree \\
   correctly functioning software. \\ \hline

2. Designing unit tests to support correctly & No opinion & Agree \\
   functioning software is hard\\ \hline

3. My current set of unit tests does a good & Agree & Agree \\
   job of ensuring that my software functions \\
   correctly.\\ \hline

4. JBlanket helps me to write unit tests that & n/a & Disagree \\
   ensure the correct functioning of my software.\\ \hline

5. To the nearest 25\%, what \% of the methods & 50\% & 100\% \\
   in your software are currently invoked by \\
   your unit tests?\\ \hline
\\
\end{tabular}

\begin{tabular}{l}
  \begin{minipage}[b]{.85\linewidth}
    6. Please briefly describe one or two of the most significant problems
       you've encountered while designing unit tests.  (Do not include the
       problem of learning how to use unit testing facilities such as JUnit
       or HttpUnit.)\\
    \\
    (Pre-Use)
    \begin{itemize}
      \item Sometimes it's hard to try to cover a certain method.  For
            instance, if I try to make a testcase for SysInfo.java in the
            CREST, (maybe in the Hackystat too) I have no idea to write
            getRelease, and getBuildTime.
    \end{itemize}

    (Post-Use)
    \begin{itemize}
      \item hard to test void return type and the method related to file
            manipulation.\\
    \end{itemize}
  \end{minipage}
  \\
  \begin{minipage}[b]{.85\linewidth}
    7. Briefly describe how access to JBlanket has influenced the way your
    write unit tests.\\
    \\
    (Post-Use) (no answer) \\
  \end{minipage}
  \\
  \begin{minipage}[b]{.85\linewidth}
    8. What would you suggest we do to improve the usefulness of JBlanket?\\
    \\
    (Post-Use) default plugins for ANT.
  \end{minipage}
\end{tabular}

\pagebreak

\section{Student E}

\begin{tabular}{lcc} \\
 & {\bf Pre-Use} & {\bf Post-Use} \\ \hline
1. Unit tests are very important for creating & Strongly agree & Strongly agree \\
   correctly functioning software. \\ \hline

2. Designing unit tests to support correctly & No opinion & Strongly agree \\
   functioning software is hard\\ \hline

3. My current set of unit tests does a good & Strongly disagree & No opinion \\
   job of ensuring that my software functions \\
   correctly.\\ \hline

4. JBlanket helps me to write unit tests that & n/a & Agree \\
   ensure the correct functioning of my software.\\ \hline

5. To the nearest 25\%, what \% of the methods & 25\% & 100\% \\
   in your software are currently invoked by \\
   your unit tests?\\ \hline
\\
\end{tabular}

\begin{tabular}{l}
  \begin{minipage}[b]{.85\linewidth}
    6. Please briefly describe one or two of the most significant problems
       you've encountered while designing unit tests.  (Do not include the
       problem of learning how to use unit testing facilities such as JUnit
       or HttpUnit.)\\
    \\
    (Pre-Use)
    \begin{itemize}
      \item Determine if you actually covered all links and form fill ins
            for both valid and invalid instances.
    \end{itemize}

    (Post-Use)
    \begin{itemize}
      \item Order of JUNIT test runs would output different results (would
      get errors running one bat file while it ran perfect for the other).\\
    \end{itemize}
  \end{minipage}
  \\
  \begin{minipage}[b]{.85\linewidth}
    7. Briefly describe how access to JBlanket has influenced the way your
    write unit tests.\\
    \\
    (Post-Use) JBlanket is excellent!  It help me pinpoint packages which
    have inadequate.  However once it was covered I gave very little
    thought to conditional and branch coverage.\\
  \end{minipage}
  \\
  \begin{minipage}[b]{.85\linewidth}
    8. What would you suggest we do to improve the usefulness of JBlanket?\\
    \\
    (Post-Use) Make it run faster.  Provide some way to include test
    coverage for conditional and branch testing.
  \end{minipage}
\end{tabular}

\pagebreak

\section{Student F}

\begin{tabular}{lcc} \\
 & {\bf Pre-Use} & {\bf Post-Use} \\ \hline
1. Unit tests are very important for creating & Agree & Strongly agree \\
   correctly functioning software. \\ \hline

2. Designing unit tests to support correctly & Agree & Agree \\
   functioning software is hard\\ \hline

3. My current set of unit tests does a good & Agree & Strongly agree \\
   job of ensuring that my software functions \\
   correctly.\\ \hline

4. JBlanket helps me to write unit tests that & n/a & Strongly agree \\
   ensure the correct functioning of my software.\\ \hline

5. To the nearest 25\%, what \% of the methods & 25\% & 100\% \\
   in your software are currently invoked by \\
   your unit tests?\\ \hline
\\
\end{tabular}

\begin{tabular}{l}
  \begin{minipage}[b]{.85\linewidth}
    6. Please briefly describe one or two of the most significant problems
       you've encountered while designing unit tests.  (Do not include the
       problem of learning how to use unit testing facilities such as JUnit
       or HttpUnit.)\\
    \\
    (Pre-Use)
    \begin{itemize}
      \item Don't know where to start.
      \item Bombs whenever make changes to packages.
    \end{itemize}

    (Post-Use)
    \begin{itemize}
      \item Testing pages which require linking to other pages. \\
    \end{itemize}
  \end{minipage}
  \\
  \begin{minipage}[b]{.85\linewidth}
    7. Briefly describe how access to JBlanket has influenced the way your
    write unit tests.\\
    \\
    (Post-Use) Able to write unit test quicker.  Know what I still need to
    test. \\
  \end{minipage}
  \\
  \begin{minipage}[b]{.85\linewidth}
    8. What would you suggest we do to improve the usefulness of JBlanket?\\
    \\
    (Post-Use) (no answer).
  \end{minipage}
\end{tabular}

\pagebreak

\section{Student G}

\begin{tabular}{lcc} \\
 & {\bf Pre-Use} & {\bf Post-Use} \\ \hline
1. Unit tests are very important for creating & No opinion & Strongly agree \\
   correctly functioning software. \\ \hline

2. Designing unit tests to support correctly & Disagree & Agree \\
   functioning software is hard\\ \hline

3. My current set of unit tests does a good & Disagree & Agree \\
   job of ensuring that my software functions \\
   correctly.\\ \hline

4. JBlanket helps me to write unit tests that & n/a & Strongly agree \\
   ensure the correct functioning of my software.\\ \hline

5. To the nearest 25\%, what \% of the methods & 25\% & 100\% \\
   in your software are currently invoked by \\
   your unit tests?\\ \hline
\\
\end{tabular}

\begin{tabular}{l}
  \begin{minipage}[b]{.85\linewidth}
    6. Please briefly describe one or two of the most significant problems
       you've encountered while designing unit tests.  (Do not include the
       problem of learning how to use unit testing facilities such as JUnit
       or HttpUnit.)\\
    \\
    (Pre-Use)
    \begin{itemize}
      \item The main problem is finding time to write them because there
            isn't enough time to even write the classes.
      \item Another problem is forgetting to change the test when the
            classes are change.
    \end{itemize}

    (Post-Use)
    \begin{itemize}
      \item When one of the HttpUnit tests errors out it sometimes causes
            the rest of the tests to fail as well.\\
    \end{itemize}
  \end{minipage}
  \\
  \begin{minipage}[b]{.85\linewidth}
    7. Briefly describe how access to JBlanket has influenced the way your
    write unit tests.\\
    \\
    (Post-Use) Instead of going for quantity I try for the quality of the tests.\\
  \end{minipage}
  \\
  \begin{minipage}[b]{.85\linewidth}
    8. What would you suggest we do to improve the usefulness of JBlanket?\\
    \\
    (Post-Use) Make it faster.
  \end{minipage}
\end{tabular}

\pagebreak

\section{Student H}

\begin{tabular}{lcc} \\
 & {\bf Pre-Use} & {\bf Post-Use} \\ \hline
1. Unit tests are very important for creating & Agree & Agree \\
   correctly functioning software. \\ \hline

2. Designing unit tests to support correctly & Disagree & Disagree \\
   functioning software is hard\\ \hline

3. My current set of unit tests does a good & Strongly disagree & No opinion \\
   job of ensuring that my software functions \\
   correctly.\\ \hline

4. JBlanket helps me to write unit tests that & n/a & Agree \\
   ensure the correct functioning of my software.\\ \hline

5. To the nearest 25\%, what \% of the methods & 25\% & 100\% \\
   in your software are currently invoked by \\
   your unit tests?\\ \hline
\\
\end{tabular}

\begin{tabular}{l}
  \begin{minipage}[b]{.85\linewidth}
    6. Please briefly describe one or two of the most significant problems
       you've encountered while designing unit tests.  (Do not include the
       problem of learning how to use unit testing facilities such as JUnit
       or HttpUnit.)\\
    \\
    (Pre-Use)
    \begin{itemize}
      \item Making unit tests are troublesome because it takes too much
            time and also does not build upon the program your working on.
      \item The unit tests are also inconvienient when you can test what
            you do a lot faster manually.
    \end{itemize}

    (Post-Use)
    \begin{itemize}
      \item Doesn't test Javascript functionality.\\
    \end{itemize}
  \end{minipage}
  \\
  \begin{minipage}[b]{.85\linewidth}
    7. Briefly describe how access to JBlanket has influenced the way your
    write unit tests.\\
    \\
    (Post-Use) I write more unit tests to test more parts of the system.\\
  \end{minipage}
  \\
  \begin{minipage}[b]{.85\linewidth}
    8. What would you suggest we do to improve the usefulness of JBlanket?\\
    \\
    (Post-Use) Have JBlanket see if every line of code is invoked.
  \end{minipage}
\end{tabular}

\pagebreak

\section{Student I}

\begin{tabular}{lcc} \\
 & {\bf Pre-Use} & {\bf Post-Use} \\ \hline
1. Unit tests are very important for creating & Agree & Agree \\
   correctly functioning software. \\ \hline

2. Designing unit tests to support correctly & Strongly agree & Agree \\
   functioning software is hard\\ \hline

3. My current set of unit tests does a good & No opinion & Agree \\
   job of ensuring that my software functions \\
   correctly.\\ \hline

4. JBlanket helps me to write unit tests that & n/a & Agree \\
   ensure the correct functioning of my software.\\ \hline

5. To the nearest 25\%, what \% of the methods & 50\% & 100\% \\
   in your software are currently invoked by \\
   your unit tests?\\ \hline
\\
\end{tabular}

\begin{tabular}{l}
  \begin{minipage}[b]{.85\linewidth}
    6. Please briefly describe one or two of the most significant problems
       you've encountered while designing unit tests.  (Do not include the
       problem of learning how to use unit testing facilities such as JUnit
       or HttpUnit.)\\
    \\
    (Pre-Use)
    \begin{itemize}
      \item Because the system changes constantly updating the unit tests
            becomes very tedious and cumbersome.
      \item Because of dual VM's in unit testing software and the jdk,
            sometimes it's hard to test also some unit testing doesn't seem
            to work properly sometimes even though it's set up correctly (I
            think).
    \end{itemize}

    (Post-Use)
    \begin{itemize}
      \item Sometimes the test doesn't go to the right page so it's hard to
            because an error occurs that may have no correlation to the
            testing being done.\\
    \end{itemize}
  \end{minipage}
  \\
  \begin{minipage}[b]{.85\linewidth}
    7. Briefly describe how access to JBlanket has influenced the way your
    write unit tests.\\
    \\
    (Post-Use) Definitely think more about unit tests covering more
               ``area'' of source code.  More functionality tests rather
               than unit tests.\\
  \end{minipage}
  \\
\end{tabular}

\pagebreak

\begin{tabular}{l}
  \begin{minipage}[b]{.85\linewidth}
    8. What would you suggest we do to improve the usefulness of JBlanket?\\
    \\
    (Post-Use)
    \begin{itemize}
      \item I think it's best to just include all methods regardless of
            length in tests.
      \item Maybe also have a trace capability to show when methods were
            called.
    \end{itemize}
  \end{minipage}
\end{tabular}

\pagebreak

\section{Student J}

\begin{tabular}{lcc} \\
 & {\bf Pre-Use} & {\bf Post-Use} \\ \hline
1. Unit tests are very important for creating & Agree & Strongly agree \\
   correctly functioning software. \\ \hline

2. Designing unit tests to support correctly & Agree & Agree \\
   functioning software is hard\\ \hline

3. My current set of unit tests does a good & No opinion & Agree \\
   job of ensuring that my software functions \\
   correctly.\\ \hline

4. JBlanket helps me to write unit tests that & n/a & Strongly agree \\
   ensure the correct functioning of my software.\\ \hline

5. To the nearest 25\%, what \% of the methods & 100\% & 100\% \\
   in your software are currently invoked by \\
   your unit tests?\\ \hline
\\
\end{tabular}

\begin{tabular}{l}
  \begin{minipage}[b]{.85\linewidth}
    6. Please briefly describe one or two of the most significant problems
       you've encountered while designing unit tests.  (Do not include the
       problem of learning how to use unit testing facilities such as JUnit
       or HttpUnit.)\\
    \\
    (Pre-Use)
    \begin{itemize}
      \item There was a problem earlier in semester when we made a junit
            test, but it kept failing.  When we took the same exact code
            out and put it in a test class, it worked fine.  Other than
            that never really had problems.
    \end{itemize}

    (Post-Use)
    \begin{itemize}
      \item None really, we found out what the problem was earlier in the
            semester and it wasn't a junit test problem.\\
    \end{itemize}
  \end{minipage}
  \\
  \begin{minipage}[b]{.85\linewidth}
    7. Briefly describe how access to JBlanket has influenced the way your
    write unit tests.\\
    \\
    (Post-Use) I don't think we tested out every little detail since we
    were just really looking to get the system to 100\%.\\
  \end{minipage}
  \\
  \begin{minipage}[b]{.85\linewidth}
    8. What would you suggest we do to improve the usefulness of JBlanket?\\
    \\
    (Post-Use) None.
  \end{minipage}
\end{tabular}

\pagebreak

\section{Student K}

\begin{tabular}{lcc} \\
 & {\bf Pre-Use} & {\bf Post-Use} \\ \hline
1. Unit tests are very important for creating & Agree & Strongly agree \\
   correctly functioning software. \\ \hline

2. Designing unit tests to support correctly & Agree & Agree \\
   functioning software is hard\\ \hline

3. My current set of unit tests does a good & Disagree & Disagree \\
   job of ensuring that my software functions \\
   correctly.\\ \hline

4. JBlanket helps me to write unit tests that & n/a & Disagree \\
   ensure the correct functioning of my software.\\ \hline

5. To the nearest 25\%, what \% of the methods & 25\% & 100\% \\
   in your software are currently invoked by \\
   your unit tests?\\ \hline
\\
\end{tabular}

\begin{tabular}{l}
  \begin{minipage}[b]{.85\linewidth}
    6. Please briefly describe one or two of the most significant problems
       you've encountered while designing unit tests.  (Do not include the
       problem of learning how to use unit testing facilities such as JUnit
       or HttpUnit.)\\
    \\
    (Pre-Use)
    \begin{itemize}
      \item Getting myself to do them.  Anytime I finish writing some code,
            I just physically test it w/o writing unit test immediately.
            There needs to be some outside inspiration for me to write unit
            test your I just won't do it.
    \end{itemize}

    (Post-Use)
    \begin{itemize}
      \item To create test that test all cases of position functionally.
      \item To test if the display looks right.\\
    \end{itemize}
  \end{minipage}
  \\
  \begin{minipage}[b]{.85\linewidth}
    7. Briefly describe how access to JBlanket has influenced the way your
    write unit tests.\\
    \\
    (Post-Use) I wrote less test cases that covered more code.  Instead of
               write a whole bunch of testcases for each method, I just
               call the ``super'' method that calls all the little small
               ones.\\
  \end{minipage}
  \\
  \begin{minipage}[b]{.85\linewidth}
    8. What would you suggest we do to improve the usefulness of JBlanket?\\
    \\
    (Post-Use) Clarity of jblanket output.  What does \_\_\_ out 100\% mean
               for everything?  Some were clear, some weren't.
  \end{minipage}
\end{tabular}

\pagebreak

\section{Student L}

\begin{tabular}{lcc} \\
 & {\bf Pre-Use} & {\bf Post-Use} \\ \hline
1. Unit tests are very important for creating & Agree & Agree \\
   correctly functioning software. \\ \hline

2. Designing unit tests to support correctly & Agree & Strongly agree \\
   functioning software is hard\\ \hline

3. My current set of unit tests does a good & Disagree & Agree \\
   job of ensuring that my software functions \\
   correctly.\\ \hline

4. JBlanket helps me to write unit tests that & n/a & Agree \\
   ensure the correct functioning of my software.\\ \hline

5. To the nearest 25\%, what \% of the methods & 50\% & 100\% \\
   in your software are currently invoked by \\
   your unit tests?\\ \hline
\\
\end{tabular}

\begin{tabular}{l}
  \begin{minipage}[b]{.85\linewidth}
    6. Please briefly describe one or two of the most significant problems
       you've encountered while designing unit tests.  (Do not include the
       problem of learning how to use unit testing facilities such as JUnit
       or HttpUnit.)\\
    \\
    (Pre-Use)
    \begin{itemize}
      \item Most of the time I concentrate on implementing functionality
            as opposed to testing.  I view functionality as being more
            important.  So, when it comes to the point of deciding to
            implement functionality or test, I choose functionality.
      \item Also, the use of inspection provides me with a ``good enough''
            view of correct functionality.
      \item However, I do try to do tests and I like tests.
    \end{itemize}

    (Post-Use)
    \begin{itemize}
      \item Reaching 100\% method coverage does not mean that the software
           is fault free.  If you make that assumption you are worse off
           then not having 100\%.\\
    \end{itemize}
  \end{minipage}
  \\
  \begin{minipage}[b]{.85\linewidth}
    7. Briefly describe how access to JBlanket has influenced the way your
    write unit tests.\\
    \\
    (Post-Use) I was able to ensure that my methods were being called.  If
               a method was reported as not tested I tested it.\\
  \end{minipage}
  \\
  \begin{minipage}[b]{.85\linewidth}
    8. What would you suggest we do to improve the usefulness of JBlanket?\\
    \\
    (Post-Use) Would it be hard to do statement coverage?
  \end{minipage}
\end{tabular}

\pagebreak

\section{Student M}

\begin{tabular}{lcc} \\
 & {\bf Pre-Use} & {\bf Post-Use} \\ \hline
1. Unit tests are very important for creating & Agree & Strongly agree \\
   correctly functioning software. \\ \hline

2. Designing unit tests to support correctly & Agree & Agree \\
   functioning software is hard\\ \hline

3. My current set of unit tests does a good & Agree & Agree \\
   job of ensuring that my software functions \\
   correctly.\\ \hline

4. JBlanket helps me to write unit tests that & n/a & Agree \\
   ensure the correct functioning of my software.\\ \hline

5. To the nearest 25\%, what \% of the methods & 50\% & 100\% \\
   in your software are currently invoked by \\
   your unit tests?\\ \hline
\\
\end{tabular}

\begin{tabular}{l}
  \begin{minipage}[b]{.85\linewidth}
    6. Please briefly describe one or two of the most significant problems
       you've encountered while designing unit tests.  (Do not include the
       problem of learning how to use unit testing facilities such as JUnit
       or HttpUnit.)\\
    \\
    (Pre-Use)
    \begin{itemize}
      \item (no answer)
    \end{itemize}

    (Post-Use)
    \begin{itemize}
      \item Trying to figure out how to thourougly test the system.\\
    \end{itemize}
  \end{minipage}
  \\
  \begin{minipage}[b]{.85\linewidth}
    7. Briefly describe how access to JBlanket has influenced the way your
    write unit tests.\\
    \\
    (Post-Use) It has influenced me to write more unit tests.\\
  \end{minipage}
  \\
  \begin{minipage}[b]{.85\linewidth}
    8. What would you suggest we do to improve the usefulness of JBlanket?\\
    \\
    (Post-Use) Increase the speed if possible.  Also maybe integrate it
               into an IDE e.g. JBuilder.
  \end{minipage}
\end{tabular}

\chapter{JBlanket data}

The coverage data gathered during the evaluation period is presented as
both graphs and tables.

The following metrics were graphed and sorted by service:
\begin{itemize}
  \item extreme coverage
  \item total LOC
  \item test LOC
  \item total one-line methods
  \item total multi-line methods
  \item total tested multi-line methods
\end{itemize}

\clearpage

% start display of each service separately

% Service -- FAQ
\section {FAQ}

% line 1
\begin{figure}[htbp]
  \begin{minipage}[htbp]{.45\linewidth}
    \begin{center}
      \includegraphics[width=\linewidth]{figs/fig.coverage.faq.eps}
      \caption{Extreme coverage - FAQ}
      \label{fig:coverage.faq}
    \end{center}
  \end{minipage}
\hfill
  \begin{minipage}[htbp]{.45\linewidth}
    \begin{center}
      \includegraphics[width=\linewidth]{figs/fig.loc.total.faq.eps}
      \caption{Total LOC - FAQ}
      \label{fig:loc.total.faq}
    \end{center}
  \end{minipage}
\end{figure}

%line 2
\begin{figure}[htbp]
  \begin{minipage}[htbp]{.45\linewidth}
    \begin{center}
      \includegraphics[width=\linewidth]{figs/fig.coverage.oneline.faq.eps}
      \caption{Total one-line methods - FAQ}
      \label{fig:coverage.oneline.test.faq}
    \end{center}
  \end{minipage}
\hfill
  \begin{minipage}[htbp]{.45\linewidth}
    \begin{center}
      \includegraphics[width=\linewidth]{figs/fig.loc.test.faq.eps}
      \caption{Test LOC - FAQ}
      \label{fig:loc.test.faq}
    \end{center}
  \end{minipage}
\end{figure}

\clearpage

% Service -- Login

\section {Login}

% line 1
\begin{figure}[htbp]
  \begin{minipage}[htbp]{.45\linewidth}
    \begin{center}
      \includegraphics[width=\linewidth]{figs/fig.coverage.login.eps}
      \caption{Extreme coverage - Login}
      \label{fig:coverage.login}
    \end{center}
  \end{minipage}
\hfill
  \begin{minipage}[htbp]{.45\linewidth}
    \begin{center}
      \includegraphics[width=\linewidth]{figs/fig.loc.total.login.eps}
      \caption{Total LOC - Login}
      \label{fig:loc.total.login}
    \end{center}
  \end{minipage}
\end{figure}

%line 2
\begin{figure}[htbp]
  \begin{minipage}[htbp]{.45\linewidth}
    \begin{center}
      \includegraphics[width=\linewidth]{figs/fig.coverage.oneline.login.eps}
      \caption{Total one-line methods - Login}
      \label{fig:coverage.oneline.test.login}
    \end{center}
  \end{minipage}
\hfill
  \begin{minipage}[htbp]{.45\linewidth}
    \begin{center}
      \includegraphics[width=\linewidth]{figs/fig.loc.test.login.eps}
      \caption{Test LOC - Login}
      \label{fig:loc.test.login}
    \end{center}
  \end{minipage}
\end{figure}

\clearpage

% Service -- Newsbulletin

\section {Newsbulletin}

% line 1
\begin{figure}[htbp]
  \begin{minipage}[htbp]{.45\linewidth}
    \begin{center}
      \includegraphics[width=\linewidth]{figs/fig.coverage.news.eps}
      \caption{Extreme coverage - Newsbulletin}
      \label{fig:coverage.news}
    \end{center}
  \end{minipage}
\hfill
  \begin{minipage}[htbp]{.45\linewidth}
    \begin{center}
      \includegraphics[width=\linewidth]{figs/fig.loc.total.news.eps}
      \caption{Total LOC - Newbulletin}
      \label{fig:loc.total.news}
    \end{center}
  \end{minipage}
\end{figure}

%line 2
\begin{figure}[htbp]
  \begin{minipage}[htbp]{.45\linewidth}
    \begin{center}
      \includegraphics[width=\linewidth]{figs/fig.coverage.oneline.news.eps}
      \caption{Total one-line methods - Newsbulletin}
      \label{fig:coverage.oneline.test.news}
    \end{center}
  \end{minipage}
\hfill
  \begin{minipage}[htbp]{.45\linewidth}
    \begin{center}
      \includegraphics[width=\linewidth]{figs/fig.loc.test.news.eps}
      \caption{Test LOC - Newsbulletin}
      \label{fig:loc.test.news}
    \end{center}
  \end{minipage}
\end{figure}

\clearpage

% Service -- Poll

\section {Poll}

% line 1
\begin{figure}[htbp]
  \begin{minipage}[htbp]{.45\linewidth}
    \begin{center}
      \includegraphics[width=\linewidth]{figs/fig.coverage.poll.eps}
      \caption{Extreme coverage - Poll}
      \label{fig:coverage.poll}
    \end{center}
  \end{minipage}
\hfill
  \begin{minipage}[htbp]{.45\linewidth}
    \begin{center}
      \includegraphics[width=\linewidth]{figs/fig.loc.total.poll.eps}
      \caption{Total LOC - Poll}
      \label{fig:loc.total.poll}
    \end{center}
  \end{minipage}
\end{figure}

%line 2
\begin{figure}[htbp]
  \begin{minipage}[htbp]{.45\linewidth}
    \begin{center}
      \includegraphics[width=\linewidth]{figs/fig.coverage.oneline.poll.eps}
      \caption{Total one-line methods - Poll}
      \label{fig:coverage.oneline.test.poll}
    \end{center}
  \end{minipage}
\hfill
  \begin{minipage}[htbp]{.45\linewidth}
    \begin{center}
      \includegraphics[width=\linewidth]{figs/fig.loc.test.poll.eps}
      \caption{Test LOC - Poll}
      \label{fig:loc.test.poll}
    \end{center}
  \end{minipage}
\end{figure}

\clearpage

% Service -- Resume

\section {Resume}

% line 1
\begin{figure}[htbp]
  \begin{minipage}[htbp]{.45\linewidth}
    \begin{center}
      \includegraphics[width=\linewidth]{figs/fig.coverage.resume.eps}
      \caption{Extreme coverage - Resume}
      \label{fig:coverage.resume}
    \end{center}
  \end{minipage}
\hfill
  \begin{minipage}[htbp]{.45\linewidth}
    \begin{center}
      \includegraphics[width=\linewidth]{figs/fig.loc.total.resume.eps}
      \caption{Total LOC - Resume}
      \label{fig:loc.total.resume}
    \end{center}
  \end{minipage}
\end{figure}

%line 2
\begin{figure}[htbp]
  \begin{minipage}[htbp]{.45\linewidth}
    \begin{center}
      \includegraphics[width=\linewidth]{figs/fig.coverage.oneline.resume.eps}
      \caption{Total one-line methods - Resume}
      \label{fig:coverage.oneline.test.resume}
    \end{center}
  \end{minipage}
\hfill
  \begin{minipage}[htbp]{.45\linewidth}
    \begin{center}
      \includegraphics[width=\linewidth]{figs/fig.loc.test.resume.eps}
      \caption{Test LOC - Resume}
      \label{fig:loc.test.resume}
    \end{center}
  \end{minipage}
\end{figure}

\clearpage

% Service -- Techreports

\section {Techreports}

% line 1
\begin{figure}[htbp]
  \begin{minipage}[htbp]{.45\linewidth}
    \begin{center}
      \includegraphics[width=\linewidth]{figs/fig.coverage.tech.eps}
      \caption{Extreme coverage - Techreports}
      \label{fig:coverage.tech}
    \end{center}
  \end{minipage}
\hfill
  \begin{minipage}[htbp]{.45\linewidth}
    \begin{center}
      \includegraphics[width=\linewidth]{figs/fig.loc.total.tech.eps}
      \caption{Total LOC - Techreports}
      \label{fig:loc.total.tech}
    \end{center}
  \end{minipage}
\end{figure}

%line 2
\begin{figure}[htbp]
  \begin{minipage}[htbp]{.45\linewidth}
    \begin{center}
      \includegraphics[width=\linewidth]{figs/fig.coverage.oneline.tech.eps}
      \caption{Total one-line methods - Techreports}
      \label{fig:coverage.oneline.test.tech}
    \end{center}
  \end{minipage}
\hfill
  \begin{minipage}[htbp]{.45\linewidth}
    \begin{center}
      \includegraphics[width=\linewidth]{figs/fig.loc.test.tech.eps}
      \caption{Test LOC - Techreports}
      \label{fig:loc.test.tech}
    \end{center}
  \end{minipage}
\end{figure}

\clearpage

% Service -- Textbooks

\section {Textbooks}

% line 1
\begin{figure}[htbp]
  \begin{minipage}[htbp]{.45\linewidth}
    \begin{center}
      \includegraphics[width=\linewidth]{figs/fig.coverage.text.eps}
      \caption{Extreme coverage - Textbooks}
      \label{fig:coverage.text}
    \end{center}
  \end{minipage}
\hfill
  \begin{minipage}[htbp]{.45\linewidth}
    \begin{center}
      \includegraphics[width=\linewidth]{figs/fig.loc.total.text.eps}
      \caption{Total LOC - Textbooks}
      \label{fig:loc.total.text}
    \end{center}
  \end{minipage}
\end{figure}

%line 2
\begin{figure}[htbp]
  \begin{minipage}[htbp]{.45\linewidth}
    \begin{center}
      \includegraphics[width=\linewidth]{figs/fig.coverage.oneline.text.eps}
      \caption{Total one-line methods - Textbooks}
      \label{fig:coverage.oneline.test.text}
    \end{center}
  \end{minipage}
\hfill
  \begin{minipage}[htbp]{.45\linewidth}
    \begin{center}
      \includegraphics[width=\linewidth]{figs/fig.loc.test.text.eps}
      \caption{Test LOC - Textbooks}
      \label{fig:loc.test.text}
    \end{center}
  \end{minipage}
\end{figure}

\clearpage

% Service -- Tutor

\section {Tutor}

% line 1
\begin{figure}[htbp]
  \begin{minipage}[htbp]{.45\linewidth}
    \begin{center}
      \includegraphics[width=\linewidth]{figs/fig.coverage.tutor.eps}
      \caption{Extreme coverage - Tutor}
      \label{fig:coverage.tutor}
    \end{center}
  \end{minipage}
\hfill
  \begin{minipage}[htbp]{.45\linewidth}
    \begin{center}
      \includegraphics[width=\linewidth]{figs/fig.loc.total.tutor.eps}
      \caption{Total LOC - Tutor}
      \label{fig:loc.total.tutor}
    \end{center}
  \end{minipage}
\end{figure}

%line 2
\begin{figure}[htbp]
  \begin{minipage}[htbp]{.45\linewidth}
    \begin{center}
      \includegraphics[width=\linewidth]{figs/fig.coverage.oneline.tutor.eps}
      \caption{Total one-line methods - Tutor}
      \label{fig:coverage.oneline.test.tutor}
    \end{center}
  \end{minipage}
\hfill
  \begin{minipage}[htbp]{.45\linewidth}
    \begin{center}
      \includegraphics[width=\linewidth]{figs/fig.loc.test.tutor.eps}
      \caption{Test LOC - Tutor}
      \label{fig:loc.test.tutor}
    \end{center}
  \end{minipage}
\end{figure}

\clearpage

% end display of each service separately

\begin{table}[htbp]
  \begin{center}
    \caption{CREST results}
    \includegraphics[width=1.0\linewidth]{figs/table.crest.results.3a.eps}
    \label{table:crest.results}
  \end{center}
\end{table}

\begin{table}[htbp]
  \begin{center}
    \caption{CREST results, con't}
    \includegraphics[width=1.0\linewidth]{figs/table.crest.results.3b.eps}
  \end{center}
\end{table}

\begin{table}[htbp]
  \begin{center}
    \caption{Change in metrics of CREST services}
    \includegraphics[width=0.95\linewidth]{figs/table.change.4a.eps}
    \label{table:crest.change}
  \end{center}
\end{table}

\begin{table}[htbp]
  \begin{center}
    \caption{Change in metrics of CREST services, con't}
    \includegraphics[width=0.95\linewidth]{figs/table.change.4b.eps}
  \end{center}
\end{table}
