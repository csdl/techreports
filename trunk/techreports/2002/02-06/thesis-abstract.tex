%%%%%%%%%%%%%%%%%%%%%%%%%%%%%% -*- Mode: Latex -*- %%%%%%%%%%%%%%%%%%%%%%%%%%%%
%% thesis-abstract.tex -- 
%% Author          : Joy M. Agustin
%% Created On      : Fri Jun  9 09:43:42 1995
%% Last Modified By: 
%% Last Modified On: Thu Apr 03 10:37:50 2003
%% Status          : Unknown
%% RCS: $Id: thesis-abstract.tex,v 1.1 1998/09/19 01:24:42 jagustin Exp $
%%%%%%%%%%%%%%%%%%%%%%%%%%%%%%%%%%%%%%%%%%%%%%%%%%%%%%%%%%%%%%%%%%%%%%%%%%%%%%%
%%   Copyright (C) 1995 University of Hawaii
%%%%%%%%%%%%%%%%%%%%%%%%%%%%%%%%%%%%%%%%%%%%%%%%%%%%%%%%%%%%%%%%%%%%%%%%%%%%%%%
%% 

%for review purposes
%\ls{1}

\begin{abstract}
Unit testing is an important part of software testing that aids in the
discovery of bugs sooner in the software development process.  Extreme
Programming (XP), and its Test First Design technique, relies so heavily upon
unit tests that the first code implemented is made up entirely of
test cases.  Furthermore, XP considers a feature to be completely coded
only when all of its test cases pass.  However, passing all test cases does
not necessarily mean the test cases are good.

Extreme Coverage (XC) is a new approach that helps to assess and improve the
quality of software by enhancing unit testing.  It extends the XP requirement
that all test cases must pass with the requirement that all defect-prone
testable methods must be invoked by the tests.  Furthermore, a set of flexible
rules are applied to XC to make it as attractive and light-weight as unit
testing is in XP.  One example rule is to exclude all methods containing one
line of code from analysis.  I designed and implemented a new tool, called
JBlanket, that automates the XC measurement process similar to the way that
JUnit automates unit testing.  JBlanket produces HTML reports similar to JUnit
reports which inform the user about which methods need to be tested next.

In this research, I explore the feasibility of JBlanket, the amount of effort
needed to reach and maintain XC, and the impact that knowledge of XC has on
system implementation through deployment and evaluation in an academic
environment.  Results show that most students find JBlanket to be a useful tool
in developing their test cases, and that knowledge of XC did influence the
manner in which students implemented their systems. However,  more studies are
needed to conclude precisely how much effort is needed to reach and maintain
XC.  

This research lays the foundation for future research directions.  One
direction involves increasing its flexibility and value by expanding and
refining the rules of XC.  Another direction involves tracking XC behavior to
find out when it is and is not applicable.
\end{abstract}
