\documentclass[11pt]{article}

%%% Load some useful packages:
%% "New" LaTeX2e graphics support.
\usepackage{graphicx}
%%	using final option to force graphics to be included even in draft mode
%\usepackage[final]{graphicx}
%% Tell graphicx the default directory for all figures
\graphicspath{{figures/final/}}

%% Enable subfigure support
\usepackage{subfigure}

%% Make subsubsections numbered and included in ToC
\setcounter{secnumdepth}{3}
\setcounter{tocdepth}{3}

%% Package to linebreak URLs in a sane manner.
\usepackage{url}

%% Define a new 'smallurl' style for the package that will use a smaller font.
\makeatletter
\def\url@smallurlstyle{%
  \@ifundefined{selectfont}{\def\UrlFont{\sf}}{\def\UrlFont{\small\ttfamily}}}
\makeatother
%% Now actually use the newly defined style.
\urlstyle{smallurl}

%% Define 'tinyurl' style for even smaller URLs (such as in tables)
\makeatletter
\def\url@tinyurlstyle{%
  \@ifundefined{selectfont}{\def\UrlFont{\sf}}{\def\UrlFont{\scriptsize\ttfamily}}}
\makeatother

%% Provides additional functionality for tabular environments
\usepackage{array}

%% Puts space after macros, unless followed by punctuation
\usepackage{xspace}

%%% Personal macros
%% Tired of typing CO2 so many times, requires xspace package
\newcommand{\COtwo}{CO\ensuremath{_2}\xspace}

%% Make margins less ridiculous
\usepackage{fullpage}

%% Allows insertion of fixme notes for future work
\usepackage[footnote, nomargin]{fixme}

%%%% Turned off for tech report, should be turned on for research portfolio
%% Turn on double spacing
%\usepackage{setspace}
%\doublespacing

%% Make URLs clickable
%\usepackage[colorlinks, bookmarks=false]{hyperref}
\usepackage[colorlinks, bookmarks=true]{hyperref}

%% Since I'm using the LaTeX Makefile that uses dvips, I need this
%% package to make URLs break nicely
\usepackage{breakurl}

%%% End of preamble
\begin{document}

\title{Proposal for Electricity Conservation Experiments in Saunders Hall}
\author{Robert S. Brewer \\
Collaborative Software Development Lab \\
Department of Information and Computer Sciences \\
University of Hawai`i \\
Honolulu, HI \\
rbrewer@hawaii.edu \\
\\
CSDL Technical Report 09-12 \\
\url{http://csdl.ics.hawaii.edu/techreports/09-12/09-12.pdf}
}

%%% Create the title page from all the information above. Note that the
%%% titlepage is outside the front matter.
\maketitle

%% Philip suggests it needs a ToC
%\tableofcontents

%\begin{abstract}
%Insert abstract here.
%\end{abstract}

\section{Introduction}

The University of Hawai`i at M\=anoa has set the goals of reducing its electricity usage by 30\% by 2012 and 50\% by 2015 (based on a 2003 benchmark) \cite{Moreno2006UHM-energy-goals, 2007UHM-HECO-pr}. A variety of tactics will be required to meet these aggressive goals. One promising technique is to encourage the occupants of buildings to reduce their electricity usage. The first step in reducing electricity usage is determining how much is being used.

Electricity usage for Saunders Hall is now instrumented using the Obvius AcquiSuite \cite{ObviusAcquiSuite}. The total building electricity usage is tracked, as well as the individual usage of five floors of Saunders ($2^{nd}$, $3^{rd}$, $4^{th}$, $5^{th}$, and $6^{th}$ floors). The metering has turned Saunders into a living laboratory where experiments can be run on how to encourage occupants to reduce their electricity usage. The per-floor metering is particularly useful because it will allow intra-building comparisons: different techniques for encouraging electricity conservation can be introduced on different floors, and their relative effectiveness compared. This research seeks to determine the effectiveness of different techniques (an active area of research worldwide), and can guide us towards the best ways to reduce electricity usage throughout the campus.

There are a variety of possible interventions that may encourage occupants to reduce their electricity usage. To assess the relative effectiveness of the interventions, we plan a series of experiments in Saunders. However, the participants of each experiment will be the occupants of Saunders, rather than a set of participants recruited anew for each experiment. We expect two negative consequences of the continuity of the subjects: reduced subject interest/enthusiasm, and diminishing conservation returns.

The envisioned series of experiments will take place over at least one semester, likely multiple semesters. While we anticipate interest in the project from Saunders occupants, after the first couple of rounds, the participants may just ``tune out'' the interventions, or they may become actively annoyed by the intrusion of surveys and the like.

The other expected problem is that there is only so much discretionary electricity usage taking place in Saunders, since we do not expect occupants to make changes that significantly impact their ability to teach, learn, or otherwise do their job (work in complete darkness, permanently unplug all computers, etc). Thus, the early experimental rounds may be sufficiently effective to have virtually eliminated discretionary electricity usage. If there is no remaining ``fat'' to be trimmed, then the effectiveness of interventions in later rounds cannot be accurately assessed.

For these two reasons, the order of the interventions must be chosen carefully to prioritize the ones that are likely to make the biggest research contributions.

We examine the envisioned interventions below, as well as the literature relevant to each option. We conclude with a sketch of an experimental design for investigating electricity conservation at Saunders.

\section{Electricity Usage Feedback}
\label{sec:feedback}

Many studies have demonstrated that providing real-time feedback on resource consumption to the occupants of a building will lead them to reduce their consumption by 5--15\% \cite{darby-review-2006}. Unlike homeowners who at least receive monthly bills, occupants of campus buildings are generally unaware of the electricity used by the building, and since they are not responsible for paying for their electricity usage they have no direct economic incentive for conservation.

Therefore, we assume all of the experiments will include some means for the building occupants to determine both building's current electrical power consumption, and historical data. We examine three standard techniques for providing feedback: websites, public displays, and individual displays.

\subsection{Website}
\label{sec:website}

Setting up a web site that provides electricity usage data is a common means of providing feedback. An energy conservation competition between dormitories at Oberlin College used a website as the primary means to inform students of their electricity usage \cite{petersen-dorm-energy-reduction}. Google.org (Google's non-profit arm) is developing a system called PowerMeter to allow users to upload their power usage data (either from utility smart meters or user-installed whole home energy meters) to Google for display and annotation \cite{Google-PowerMeter}. During California's energy crisis in 2000 and 2001, Lawrence Berkeley National Laboratory created a web site that graphed data from utility organizations \cite{Bartholomew2008Current-Energy}. The graphs showed consumer demand for electricity (actual and forecast), and the utilities' generation capacity. Darby reports anecdotal evidence that people viewing the graphs changed their electricity usage based on the data \cite{darby-review-2006}.

In a building environment, most occupants will have access to a web browser, making website feedback an attractive least common denominator for feedback. Usage of the website can be tracked fairly accurately using log file information, providing a good estimate of how many occupants actually viewed the feedback. In the Oberlin dorm conservation competition study, 46\% of residents looked at the feedback website \cite{petersen-dorm-energy-reduction}.

Websites are also relatively inexpensive to deploy, and often require no additional hardware purchases (assuming server capacity is already available). However, websites require the user to take action to see the feedback, which is a major drawback. Most users would likely require some motivation in order to seek out the information on a website. There are other options for displaying feedback data on personal computers, including widgets on personalized home pages (such as iGoogle), desktop widgets, and screen savers (though screen savers are now widely considered a waste of electricity given the standby power capabilities of monitors). Electricity usage data could also be ``pushed'' to occupants who desire it using instant messaging or Twitter. Interested occupants could subscribe to alerts that would be sent when electricity usage exceeds user-definable thresholds. Pushing data to occupants could provide them with a level of awareness that isn't contingent on user action, but choosing thresholds that provide useful feedback without overwhelming the user could be challenging.

\subsection{Public display}

Public displays of electricity usage usually consist of a large monitor attached to a computer displaying information comparable to that available on a website, sized for viewing at a distance. The information on display will often cycle through multiple screens, so that anyone viewing it can see more than one set of information without user input. Some LEED certified buildings include kiosks at the entry point to the building that can display electricity usage. Two such commercially available interactive displays are GreenTouchScreen and Building Dashboard \cite{greentouchscreen, building-dashboard}. For our purposes, we focus on the non-interactive aspects of public displays. Public displays need not be limited to the entry point of a building, there can be multiple displays distributed throughout a building. Displays are installed in high-traffic areas of a building to ensure the maximum number of occupants see the feedback.

The primary advantage of public displays are that they lead to occupants seeing feedback in the course of their day, without any explicit action required on their part. The primary disadvantage of displays are that they require hardware purchases and the displays run the risk of theft or vandalism. Tracking usage of public displays is also considerably more difficult than tracking usage of a website. Combining a video camera with a public display would provide a record of when passersby viewed the display, but would require either extensive manual analysis of the video or use of sophisticated computer vision algorithms to track the gaze of passersby.

\subsection{Individual or portable displays}

Individual displays are feedback displays brought down to the level of one per room or suite of rooms. Some are designed to be mounted on a wall, others placed on the tabletop. Providing individual displays brings the feedback even closer to the user, combining the desktop convenience of a feedback website with the always visible nature of a public display. However, having displays in every room of a building would be expensive, and potentially a significant source of electricity usage. It's also unclear how much additional value an individual feedback display provides. In an apartment setting, using a device like the TED 1001 \cite{the-energy-detective}, occupants can see how their actions impact the apartment's electricity usage. In a large building, actions in an office (such as turning off the lights) are unlikely to be visible in a display of the overall building's usage, thus removing one powerful feature of having a nearby feedback display.

\section{Collaborative website}
\label{sec:collab-website}

As discussed in \autoref{sec:feedback}, the conservation effect of providing consumption feedback to end-users has been well-established. Naturally, the actual conservation occurs not due to the feedback, but because of actions taken by the occupants: changes in behavior or replacement of equipment. Occupants may seek information on how to reduce their consumption. Woodruff et al. performed a qualitative study of individuals who are making a significant effort to be green, in an effort to inform the design of future environmentally persuasive 
systems \cite{Woodruff2008-bright-green}. They found that participants often sought out information on sustainability topics they were interested in, and found that mentorship was an important part of their learning process. This suggests that one route to increase the impact of feedback is to facilitate collaboration between building occupants.

Many of of the studies in \autoref{sec:feedback} were performed in residential settings, where the number of occupants that need to coordinate to reduce usage is small. In the residential context, ad hoc collaboration (likely face to face) is sufficient to result in gains in conservation.

In contrast, a large building has many occupants, and to affect significant conservation many of them will need to change their behavior. In this setting, ad hoc communication methods may prove inadequate. We suggest that a collaborative website would be a useful means of facilitating collaboration between occupants of a building or a particular floor of a building. Such a website would provide a threaded discussion area, as well as wiki pages to allow participants to make particularly important information available to others.

\section{Competition}
\label{sec:competition}

Competition is used in many contexts to encourage the participants to maximize their efforts toward a goal. Competitions between buildings on which building can reduce their electricity usage seems like an obvious way to facilitate conservation. Petersen et al. studied contests between dormitories at Oberlin College \cite{petersen-dorm-energy-reduction}, comparing high resolution feedback in two dorms (a website updated every 20 seconds) with low resolution feedback in the other dorms (weekly poster updates). They found that the dorms reduced electricity usage by 32\% overall, with the best dorm reducing by 56\%. As expected, the dorms receiving high resolution feedback reduced their electricity usage the most. While the electricity usage was impressively reduced during the competition, the residents indicated that some of the strategies they employed were not sustainable, such as ``unplugging vending machines and turning off hallway lights at night''. Any competition must be careful to ensure that the electricity reductions measured are sustainable, or risk overestimating the impact of the competition. The authors did track electricity usage for two weeks after the competition had ended, and found that electricity usage had gone down even further, but the weather had gotten warmer with more sunlight so it is not clear whether conservation was continuing at the same level or not.

The two dorms receiving high resolution feedback each had three floors, and usage on all three floors was tracked separately, but the website only displayed the usage from two of the floors (leaving the last as a control). The authors found no change in usage between floors that had their per-floor usage displayed and those that did not. However, there was no competition between the floors only between dormitories.

A confounding factor of conservation competitions is the introduction of feedback alongside the competition itself. For example, in the Oberlin study, it is not clear what proportion of the reduction of electricity usage can be ascribed to the contest versus the introduction of feedback to the residents. In the study they discuss the potential benefits of rolling out high resolution feedback to all the dorms, but don't discuss the benefits of further contests. Moreover, contests are difficult to sustain over long periods of time. Overall, the benefit of contests independent of feedback is unclear, but they provide a convenient ``hook'' to get occupants attention initially.

\subsection{Incentives}
\label{sec:incentives}

Many contests involve incentives for winning the competition. In the Oberlin dorm study, the residents of the winning dorms were invited to an ice cream party \cite{petersen-dorm-energy-reduction}. Financial incentives are another option, and can be financed through the savings from energy conservation. Unfortunately, Darby's survey found that while incentives can change behavior, the changes are likely to disappear when incentives are removed \cite{darby-review-2006}. De Young's work on the motives for environmentally responsible behavior also found incentives had to be continually reintroduced to be effective \cite{Young:2000fv}. He suggests that it is more effective to motivate people using intrinsic motivations rather than extrinsic ones like financial incentives. In the Oberlin dorm study, the researchers note that only about 10\% of the winning dorm residents attended the ice cream parties, and hypothesized that perhaps the incentive was not important to the residents.

While contest incentives might have questionable effectiveness, there are other ways to create a financial motive for conservation. Individuals who reduce their home's electricity usage are rewarded with lower utility bills for as long as they maintain the reductions. This is not an incentive that has to be reintroduced, but a natural outcome of the conservation. If departmental budgeting were changed such that departments were responsible for their electricity usage, this would create a lasting incentive to reduce electricity usage.

\subsection{Extended competition}

Another modification of the competition intervention is increasing the length of the competition period. In the Oberlin study, the competition only lasted two weeks, which is quite short \cite{petersen-dorm-energy-reduction}. There were presumably practical reasons for keeping the contest short, since most of the dorms electricity usage had to be tracked manually, and seasonal weather changes present confounding factors to an extended competition. Darby notes as a rule of thumb that changes sustained over three months are more likely to be maintained \cite{darby-review-2006}. She also noted that in a study in Oslo of informative billing over three years, at the end of the study the participants had internalized the behavior changes and could not recall them without being prompted by researchers.

%\section{Aggressive communications}
%
%One option presented in the ``Sustainability Through Energy Conservation''  grant application is the aggressive communications.

\section{Experimental Design}

Based on the preceding review of the literature, we recommend the following experimental design to maximize the novelty of the results. To test multiple different interventions, the experiment will be conducted in a series of phases or rounds. Given that the schedule of the building occupants is greatly determined by the academic calendar, each phase of the experiment should fit into a single semester. For the Fall 2009 semester, we propose three rounds, with each round being four weeks long. Classes start at UH Manoa on August 24, so we propose starting on the following Monday, August 31 to allow the participant's schedules to settle down after the first week of classes. The three rounds would therefore be August 31 to September 27, September 28 to October 25, and October 26 to November 22. The first and last periods contain a single UH holiday, but the second period does not. A fourth round would have to run over Thanksgiving and would run over finals week (less building occupancy), so it is not recommended.

In addition to the metric of how much electricity is conserved during the experiment, it would be useful to survey the participants about their feelings regarding the effectiveness of the different interventions. The artifacts from participation in the collaborative website will be another data source that can be analyzed to determine the effectiveness of the website or other aspects of the experiment. The content of the survey is not discussed here.

%What is the rationale for this ordering?
%
%How long is each round? 
%
%What do you mean by "nominal incentive"? 
%
%What happens at the end of each round? (i.e. are results posted)? 
%
%I think what I'm wanting here is more of an experimental design.  Instead of an enumerated list of four items, expand it into four subsections whose titles are "Nominal incentive", "Collaborative Website", etc. Then talk about what will happen in each of these rounds.
%Foo

\subsection{Competition with nominal incentive}

Despite the challenges with competitions discussed in \autoref{sec:competition}, competitions provide a useful hook to recruit participants and get them interested in the idea of conservation. For the first round we recommend a competition between floors 2-6, with each floor competing to reduce their power consumption relative to a baseline established from historical meter data. Participants will be able to access a website that displays electricity usage data to track the performance of each floor.

\begin{table}[htbp]
	\centering
		\begin{tabular}{| l || c | c | c | c |}
			\hline
			Floor \# & Basic website & Competition & Collaborative website & Public displays \tabularnewline \hline \hline
			
			2 & X & X &  & \tabularnewline \hline
			
			3 & X & X &  & \tabularnewline \hline

			4 & X & X &  & \tabularnewline \hline

			5 & X & X &  & \tabularnewline \hline
			
			6 & X & X &  & \tabularnewline \hline
		\end{tabular}
	\caption{Round 1 interventions per floor}
	\label{tab:round1-per-floor}
\end{table}

At the end of the competition, results will be posted on the website and on a flyer posted in the lobby of Saunders and across from the elevators on each of the participating floors. The floor that has reduced its electricity usage by the greatest percentage compared to a baseline will win a party with food provided (perhaps pizza or ice cream, based on participant input). The idea behind the incentive is to provide some tangible benefit to winning the competition, but nothing extravagant or financial.

We recommend starting with this intervention because the introduction of any feedback system to Saunders is likely to increase conservation. Therefore it makes sense to start the experiment with an intervention that has been studied before, rather than a more novel intervention, to establish a baseline for consumption in the presence of feedback. Starting with the other feedback interventions could make it difficult to determine whether it was the particular type of feedback, or just the addition of feedback that lead to the change in consumption. For this reason, we recommend that the competition between floors and access to the website be continued during the second and third rounds so that the only change will be the addition of new interventions.

\subsection{Add collaborative website}

The second round of the experiment involves adding a per-floor collaborative website of the type described in \autoref{sec:collab-website}. We propose providing these websites to floors 2, 4, and 6, leaving floors 3 and 5 as control floors continuing in the competition as in the first round. The collaborative website would supersede the usage website for the experimental floors, while the control floors would continue to use the usage website.

\begin{table}[htbp]
	\centering
		\begin{tabular}{| l || c | c | c | c |}
			\hline
			Floor \# & Basic website & Competition & Collaborative website & Public displays \tabularnewline \hline \hline
			
			2 &   & X & X & \tabularnewline \hline
			
			3 & X & X &  & \tabularnewline \hline

			4 &   & X & X & \tabularnewline \hline

			5 & X & X &  & \tabularnewline \hline
			
			6 &   & X & X & \tabularnewline \hline
		\end{tabular}
	\caption{Round 2 interventions per floor}
	\label{tab:round2-per-floor}
\end{table}

We propose scheduling the collaborative website as the second round because it is the most novel of the interventions proposed, so scheduling it early reduces the chances that all reasonable conservation actions have been taken or that participants will be burnt out. The collaborative website also dovetails well with the public displays recommended for round three: the displays can show recent activity on the website as a way of increasing awareness about the web community. Scheduling the collaborative website second also gives the community time to continue to build going into the third round.

\subsection{Add per-floor public feedback displays}

The third round of the competition involves adding public feedback displays to some floors. We proposed providing these displays to floors 4, 5, and 6. This leaves floor 3 to continue as a control, and provides floor 5 with a display but no collaborative website.

\begin{table}[htbp]
	\centering
		\begin{tabular}{| l || c | c | c | c |}
			\hline
			Floor \# & Basic website & Competition & Collaborative website & Public displays \tabularnewline \hline \hline
			
			2 &   & X & X & \tabularnewline \hline
			
			3 & X & X &  & \tabularnewline \hline

			4 &   & X & X & X \tabularnewline \hline

			5 & X & X &  & X \tabularnewline \hline
			
			6 &   & X & X & X \tabularnewline \hline
		\end{tabular}
	\caption{Round 3 interventions per floor}
	\label{tab:round3-per-floor}
\end{table}

The proposed scheduling per floor will provide a mix of different combinations. Floor 2 continues with the collaborative website, which will provide data on whether there is additional benefit from a second round with the website. Floor 5 will provide data on just adding a public display without the benefit of a collaborative website, while floors 4 and 6 will examine whether the public display provides an additional benefit on top of the collaborative website.

%\subsection{Financial Incentive}
%
%Remove public displays and website, add financial incentive?

\subsection{Post-competition observation}

After the third round is complete, we will disable the collaborative website and remove the public displays, but retain the basic usage website. We will continue to track electricity usage in the post competition period to see whether consumption increases in the absence of an active experimental program. The post competition period could last from November 23 till December 6. This includes Thanksgiving, but not finals week or the last week of classes. After the observation is complete, we could consider restoring the collaborative website or public displays if there is strong interest from the building occupants.

\begin{table}[htbp]
	\centering
		\begin{tabular}{| l || c | c | c | c |}
			\hline
			Floor \# & Basic website & Competition & Collaborative website & Public displays \tabularnewline \hline \hline
			
			2 & X &   &   & \tabularnewline \hline
			
			3 & X &   &  & \tabularnewline \hline

			4 & X &   &   &   \tabularnewline \hline

			5 & X &   &  &   \tabularnewline \hline
			
			6 & X &   &   &   \tabularnewline \hline
		\end{tabular}
	\caption{Round 3 interventions per floor}
	\label{tab:post-competition-per-floor}
\end{table}

\section{Conclusion}

In conclusion, there are a variety of possible interventions that could be attempted in Saunders Hall to motivate occupants to reduce their electricity usage. We have presented a sketch of an experimental design designed to explore novel research areas in feedback and provide a roadmap towards reducing electricity usage on the M\=anoa campus. In future semesters, the other interventions could be performed, such as extended competitions and other types of incentives. We welcome feedback on the schedule presented here, and other interventions that could be performed on Saunders.


%% Just for demo purposes, include all entries from bib file
%\nocite{*}

%%% Input file for bibliography
\bibliography{sustainability}
%% Use this for an alphabetically organized bibliography
\bibliographystyle{plain}
%% Use this for a reference order organized bibliography
%\bibliographystyle{unsrt}
%% Try using this BibTeX style that hopefully will print annotations in
%% the bibliography. This will allow me to make notes on papers in the
%% BibTeX file and have them readable in the references section until
%% I turn them into a conceptual literature review 
%\bibliographystyle{annotation}

\end{document}