\chapter{Energy Literacy Questionnaire}
\label{app:energy-literacy}

This appendix details the contents of the questionnaire that was administered to assess subjects energy literacy, group identification, and connectedness to nature. Each section briefly relates the source and goal of that segment of the questionnaire, and then lists the actual items presented to subjects.

When subjects filled out the questionnaire via the SurveyGizmo website, the questions were broken into pages. Each page provided subjects the ability to move forward to the next page in the questionniare, but not back to previous pages. In the energy knowledge section of the questionnaire, this allowed later questions to include of information that might provide the answer to previous questions (such as what unit electrical power is measured in). The pages of the survey were:

\begin{enumerate}
	\item Informed consent via email address
	\item Energy attitudes and behavior
	\item Energy knowledge 1 (questions 1--5)
	\item Energy knowledge 2 (questions 6--9)
	\item Energy knowledge 3 (questions 10--13)
	\item Group identification and connectedness to nature
	\item Open feedback on questionnaire
	\item Thank you page
\end{enumerate}

Most items on the questionnaire were \emph{required}, meaning that subjects could not move to the next page of the questionniare without submitting an answer. However, each required item included the choice ``Choose not to answer'' for those subjects that did not want to answer the item. The one exception is the entry of the email address on the informed consent page, which was required with no option to skip. Due to way the knowledge ranking questions (questions 5a--5c and 7a--7e) were presented in SurveyGizmo, these questions did not have a ``Choose not to answer'' option, so they were not marked required.

\section{Energy Attitudes}
\label{sec:attitude-items}

%% Cite DeWaters paper

\section{Energy Behaviors}
\label{sec:behavior-items}

%% Commitments from competition


\section{Energy Knowledge}
\label{sec:knowledge-items}

These factual questions assess energy knowledge. As discussed at the beginning of this appendix, the knowledge questions were separated into three pages. When presented to subjects, the order of questions within the page was randomized, as was the order of the multiple choice answers. I have assigned keywords to each question to indicate which subjects they attempt to assess.

Each page was prefaced with the following instructions:

``Please answer the following questions to the best of your ability, without consulting any books or the Internet. We are interested in what you know right now.''

\subsection{Knowledge Page 1}

\noindent
1. Electrical power is commonly measured in units of:

\begin{answer}
	\item volts (V)
	\item watt-hours (Wh)
	\item joule (J)
	\item watts (W)
	\item British Thermal Units (BTU)
	\item Choose not to answer
\end{answer}

Correct answer: watt

Keywords: power, units

\vspace{5 mm}
\noindent
2. What is the primary cause of current climate changes?

\begin{answer}
	\item Carbon dioxide released from burning fossil fuels
	\item There is no cause, climate change isn't real
	\item Natural solar cycles
	\item Radioactive waste from nuclear power plants
	\item Melting glaciers in Greenland
	\item Choose not to answer
\end{answer}

Correct answer: Carbon dioxide released from burning fossil fuels

Keywords: climate change

\vspace{5 mm}
\noindent
3. Electrical energy is commonly measured in units of

\begin{answer}
	\item erg
	\item ampere (A)
	\item British Thermal Units (BTU)
	\item watt-hours (Wh)
	\item watts (W)
	\item Choose not to answer
\end{answer}

Correct answer: watt-hour

Keywords: energy, units

\vspace{5 mm}
\noindent
4. What is the breakdown of the clean energy mandated by 2030 by the Hawaii Clean Energy Initiative?

\begin{answer}
	\item 20\% from renewable sources, 80\% from energy conservation
	\item 30\% from energy conservation, 40\% from renewable sources
	\item 50\% from renewable sources, 10\% from conservation
	\item 30\% from solar, 30\% from wind, 10\% from waves
	\item 30\% from renewable sources, 20\% from conservation, 10\% from natural gas
	\item Choose not to answer
\end{answer}

Correct answer: 30\% from energy conservation, 40\% from renewable sources

Keywords: energy, generation, conservation, utility, \Hawaii

\vspace{5 mm}
\noindent
5a--5c. Order these types of light sources from lowest to highest power usage, assuming they provide the same amount of light:

\begin{answer}
	\item incandescent bulb
	\item compact fluorescent lightbulb (CFL)
	\item light-emitting diode (LED)
\end{answer}

Correct answer: c, b, a

Keywords: lighting, energy intuition


\subsection{Knowledge Page 2}

\noindent
6. Approximately how much carbon dioxide (CO2) is in the atmosphere now, and what level is considered the safe upper limit to avoid the worst effects of climate change?

\begin{answer}
	\item 450 ppm CO2 in atmosphere now, 500 ppm CO2 safe upper limit
	\item 331 ppm CO2 in atmosphere now, 350 ppm CO2 safe upper limit
	\item 393 ppm CO2 in atmosphere now, 350 ppm CO2 safe upper limit
	\item 600 ppm CO2 in atmosphere now, 450 ppm CO2 safe upper limit
	\item 100 ppm CO2 in atmosphere now, 50 ppm CO2 safe upper limit
	\item Choose not to answer
\end{answer}

Correct answer: 393 ppm, 350 ppm

Keywords: climate change

\vspace{5 mm}
\noindent
7a--7e. Order these appliances from lowest to highest power usage:

\begin{answer}
	\item desk lamp with compact fluorescent lightbulb (CFL)
	\item mobile phone charger (while charging)
	\item plasma TV
	\item microwave
	\item laptop
\end{answer}

Correct answer: b, a, e, c, d

Keywords: energy intuition

\vspace{5 mm}
\noindent
8. On average, how much electrical energy does a home in Hawaii use per day?

\begin{answer}
	\item 400 W
	\item 20 kWh
	\item 87 kWh
	\item 328 kWh
	\item 4 kWh
	\item Choose not to answer
\end{answer}

Correct answer: b

Keywords: energy intuition, \Hawaii

\vspace{5 mm}
\noindent
9. What is the approximate maximum power generated from a single standard rooftop solar panel?

\begin{answer}
	\item 25 W
	\item 800 W
	\item 50 W
	\item 10 kW
	\item 200 W
	\item Choose not to answer
\end{answer}

Correct answer: 200 W

Keywords: power, energy intuition, generation, PV


\subsection{Knowledge Page 3}

\noindent
10. What are the expected long-term effects of current climate changes?

\begin{answer}
	\item A significant rise in the sea level
	\item Global temperatures increasing by a few degrees on average
	\item Increasing sea water acidity
	\item Changes in seasonal rainfall patterns (droughts, floods)
	\item All of the above
	\item Choose not to answer
\end{answer}

Correct answer: All of the above

Keywords: climate change

\vspace{5 mm}
\noindent
11. What is currently the source of approximately 80\% of Hawaii's electricity?

\begin{answer}
	\item oil
	\item wind
	\item natural gas
	\item coal
	\item solar
	\item Choose not to answer
\end{answer}

Correct answer: oil

Keywords: generation, utility, \Hawaii

\vspace{5 mm}
\noindent
12. A compact fluorescent lightbulb (CFL) uses 13 W. If it is run for 2 hours, how much energy does it use?

\begin{answer}
	\item 13 Wh
	\item 7.5 Wh
	\item 26 Wh
	\item 130 Wh
	\item 52 Wh
	\item Choose not to answer
\end{answer}

Correct answer: 26 Wh

Keywords: power, energy, calculation

\vspace{5 mm}
\noindent
13. If your game console uses 200 W when turned on, how much energy would it waste if you left it on all weekend while you were away?

\begin{answer}
	\item 15000 Wh
	\item 100 Wh
	\item 960 kWh
	\item 9.6 kWh
\end{answer}

Correct answer: 9.6 kWh

Keywords: power, energy, calculation


\section{Group Identification}
\label{group-id-items}

%% Cite Arrow-Carrini scale

\section{Connectedness To Nature}
\label{cns-items}

%% Cite CNS scale (plus new paper)
