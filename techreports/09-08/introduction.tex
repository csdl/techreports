\chapter{Introduction}
\section{Preamble}
This literature review provides a handy guide for the data mining toolkit I am developing extending the open-source framework for automated collection of the software development process data named Hackystat. Hackystat gathers data about human activity (developers behavior) and various code metrics indicating size, structure and the quality of the code. After transformation and aggregation of the raw sensor data Hackystat stores it in the temporal database providing users with infrastructure for the retrieval and visualization of the telemetry data. It is possible not only to select a desired data granularity but also to refine telemetry streams by specifying individual or groups of developers, set of projects as well as individual metrics and their derivatives. While all of this provides rich environment for the analysis and interpretation of the software development process the comparison of telemetry streams is still performed by ``eyeballing'' and it's automatization is a highly desired feature. Once processing of the range queries will be implemented it will allow extension of the analysis toolkit with various KDD methods aiming but not limited to the indexing, clustering and development pattern discovery.

\section{Introduction into time-series}
The time-series is a data type represented by a sequence of data points sampled at successive times and usually found as the sequence of real or integer numbers with or without attached timestamps. The time-series data is arising naturally from observations and reflecting an evolution of some subject or a development of some phenomena in time. Since the time-series is the only way to store a valuable and often non-reproducible information, it makes time-series data ubiquitous and important not only in every scientific field but also in the everyday life. For example, it is very common to see visualized time-series representing financial information about stocks and currency fluctuations, weather changes or social trends in newspapers or TV programs. Medical observations, such as the blood pressure, heart beat rate or a body temperature changes is another example of time-series commonly seen in life. According to Tufte \cite{citeulike:1454223} ``The time-series plot is the most frequently used form of graphic design. With one dimension marching along to the regular rhythm of seconds, minutes, hours, days, weeks, months, years, or millennia, the natural ordering of the time scale gives this design a strength and efficiency of interpretation found in no other graphic arrangement.'' The figure \ref{fig:10century} from the Tufte book depicts the oldest known example of a time-series plot showing the planetary orbits inclinations and is dated by tenth century.
\begin{figure}[tbp]
   \centering
   \includegraphics[height=70mm]{10century.eps}
   %%{seriesheatmap}
   \caption{Ancient time-series plot showing the planetary orbits inclinations.}
   \label{fig:10century}
\end{figure} 

It is generally assumed that the consecutive measured values in the time-series are sampled at equally spaced time intervals (which is not always true and if required, the resampling and interpolation used). This specific ordering of sample values in time often bear the valuable information about the dependency of successive observation on earlier ones and is a key feature distinguishing the time-series data from usually independent statistical samples. 
This explicit recognition of ordering makes the time-series analysis techniques somehow different from statistical analyses \cite{citeulike:3989988}. It should be noted, that while many time-series analysis methods treat time-series as the simple sequences of real numbers discarding the actual time-stamps, the time-ordering is always considered and it is assumed that unpronounced time interval are equal between the sampled values.

The time-series analysis methods in general combine various statistical and pattern recognition techniques while using information presented in the time-ordering of samples. Historically time-series analysis is divided by two major fields of study: first is the explorative and descriptive analysis and the second field is the forecasting. 
While the descriptive analyzes focused on the understanding of the time-series generating processes itself by finding trends, periodicity and some hidden features within the time-series \cite{citeulike:2206845}, the predictive analyses aiming forecasting the future of the time-series generating process based on the information found by conducting descriptive analyses \cite{citeulike:3449765}. 
Both, descriptive and predictive analyses, based on the identification of some pattern in the time-series which could be formally described and correspond to the time-series generating process nature. Once this pattern is found and compiled into the formal model during the explorative analysis it is used for the extrapolation of the time-series into future. Such a model development and forecasting could be traced from the babylonian astrologist predicting celestial phenomenas to the contemporary stochastic and deterministic models.

Taking in account the fact that the most of the data collected automatically by sensing and monitoring are time series, it is hard to overemphasize the importance and value of the time-series data and analyses. Many of public and private time-series databases were created for tracking and analysis of various information and contemporary trends in the cost of the data storage, increased bandwidth and progress in information science induced tremendous growth in the time series data volume and variety. There are public databases which tracking financial indexes and climate change (Figure \ref{fig:onlineDB}), astronomical observations \cite{citeulike:4373331} and medical information \cite{citeulike:4373332}. This tremendous growth in the data volume and availability during the last decade of the XX century demanded for new approaches in the KDD (Knowledge Discovery and Data mining) applications capable of handling of very large volumes of temporal data.
\begin{figure}[tbp]
   \centering
   \includegraphics[height=190mm]{onlineDB.eps}
   %%{seriesheatmap}
   \caption{Examples of time-series databases. Upper screenshot is the FRED (Federal Reserve Economic Data) database of 20,056 U.S. economic time series, lower is the part of the NOAA Climate time-series database.}
   \label{fig:onlineDB}
\end{figure} 
