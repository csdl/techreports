\documentclass[11pt,oneside]{report}

%%% Load some useful packages:
%% "New" LaTeX2e graphics support.
\usepackage{graphicx}
%%	using final option to force graphics to be included even in draft mode
%\usepackage[final]{graphicx}
%% Tell graphicx the default directory for all figures
\graphicspath{{figures/}}

%% Enable subfigure support
\usepackage{subfigure}

%% Make subsubsections numbered and included in ToC
\setcounter{secnumdepth}{3}
\setcounter{tocdepth}{3}

%% Package to linebreak URLs in a sane manner.
\usepackage{url}

%% Define a new 'smallurl' style for the package that will use a smaller font.
\makeatletter
\def\url@smallurlstyle{%
  \@ifundefined{selectfont}{\def\UrlFont{\sf}}{\def\UrlFont{\small\ttfamily}}}
\makeatother
%% Now actually use the newly defined style.
\urlstyle{smallurl}

%% Define 'tinyurl' style for even smaller URLs (such as in tables)
\makeatletter
\def\url@tinyurlstyle{%
  \@ifundefined{selectfont}{\def\UrlFont{\sf}}{\def\UrlFont{\scriptsize\ttfamily}}}
\makeatother

%% Provides additional functionality for tabular environments
\usepackage{array}

%% Puts space after macros, unless followed by punctuation
\usepackage{xspace}

%% Make margins less ridiculous
\usepackage{fullpage}

%% Allows insertion of fixme notes for future work
\usepackage[footnote, nomargin]{fixme}

%%%% Turned off for tech report, should be turned on for research portfolio
%% Turn on double spacing
%\usepackage{setspace}
%\doublespacing

%% Make URLs clickable
%\usepackage[colorlinks, bookmarks=false]{hyperref}
\usepackage[colorlinks, bookmarks=true]{hyperref}

%% Since I'm using the LaTeX Makefile that uses dvips, I need this
%% package to make URLs break nicely
\usepackage{breakurl}

\usepackage{amsmath,amsfonts}
\numberwithin{equation}{subsection}
%%\usepackage{nonfloat}
\usepackage{bbm}
\usepackage{setspace}
\onehalfspacing
\usepackage{tabularx}

%%% End of preamble
\begin{document}

\title{Literature Review on Time Series Indexing}
\author{Pavel V. Senin \\
Collaborative Software Development Lab \\
Department of Information and Computer Sciences \\
University of Hawaii \\
Honolulu, HI \\
senin@hawaii.edu \\
\\
CSDL Technical Report 09-08 \\
\url{http://csdl.ics.hawaii.edu/techreports/09-08/09-08.pdf}
}
\date{April 2009\\[3pt]
Copyright \copyright\ Pavel V. Senin 2009}

%%% Create the title page from all the information above. Note that the
%%% titlepage is outside the front matter.
\maketitle

%% Philip suggests it needs a ToC
\tableofcontents

\begin{abstract}
Similarity search in time-series databases has become an active research area in the past decade due to the tremendous growth of the amount of temporal data collected and publicly available. The complexity of this similarity problem lies in the high dimensionality of the temporal data making convenient methods inappropriate. The most promising approaches involve dimensionality reduction and indexing techniques which are the subject of this review. After starting with general introduction to the time-series and classical time-series analyses we will discuss in detail time-series normalization techniques and relevant distance metrics. fWe conclude with a review of the dimensionality-reduction and indexing methods proposed to date.
\end{abstract}

%% Start with introduction
\chapter{Introduction}
Delivering high quality software products within the budget and in time is the main goal and the most 
challenging task of Software Engineering. Years of scientific research in this area resulted in a 
number of software processes providing detailed guidelines on how to reach 
the goal efficiently. These processes manifested themselfs as the means for improvements in terms 
of quality, speed and cost over existing practices. Many were implemented and tested within academic 
and industrial settings and proved proposed superiority. Some of these processes were successfully 
adopted and standardized in industry shaping the best practices of contemporary software development 
\cite{citeulike:9962021}. Moreover, there are plethora of processes for improving existing processes 
of software development on the team \cite{citeulike:9962027} and personal 
levels \cite{citeulike:9962022}.

The processes I am mentioning here are the well-known large formal models such as Waterfall and Spiral, 
as well as more flexible iterative agile approaches like XP, SCRUM or FDD. These are also sets of 
rules and recommendations which can be applied to certain stages of the software processes 
such as Test Driven Development or Pair Programming; there are general guidelines helping 
to improve the correctness of a product and standards, like CMMI or ISO 9000; guidlines for testing 
and measurements, code syntax rules and formatting styles, code comments 
recomendations \cite{citeulike:900855}. 

From the first sight, taking all this in account,  one would guess that 
the area of software processes is thoroughly explored and there are clear choices of processes 
and models for the one in charge making decision... But it is not true - despite many choices 
one can make, no one can foretell what is the ``best'' process to choose for certain constraints.
What managers are left with are the equal alternatives and vague promises. 
This deficiency in knowledge is the main coause of the ``software crisis'' phenomena point is supported by the fact that according to ``Chaos Report'' from the Standish 
Group (Rubinstein) \cite{SDTimes} only ``35\% of software projects in 2006 can be categorized as successful - meaning 
they were completed on time, on budget and met user requirements''. 
These thirty five percent of success clearly saying that it is somewhat difficult to make 
a statement that we are fully understand and able to control software processes. 
Moreover, over years, while this idea of a software process formalizations shaped the 
programming practices, which once thought to be a creative human activity accessible by amateurs 
and hobbyists \cite{citeulike:9958822} into a serious engineering discipline, bounded 
by requirements for education, standardized processes, rules, certifications, and strict 
financial requirements from stakeholders the opposite idea was born - the idea of 
software development as a craft. Interesting that such a duality of views can be found 
in the work of a single person \cite{citeulike:5203446}.

Clearly, there is a great room for research and improvement of our understanding of software processes.
This exploratory study is yet another attempt at the understanding. In my research work I am 
exploring techniques aiming the understanding of small processes which are 
rather the reflection of personal behaviors or habits of software development rather than a 
formalized constructs. Also, I would like to emphasize, that in this work I will not 
address the need and means of the process synthesis, its quality assessment, productiveness
or any topics related to the software product itself; I would rather focus on the specific issue - 
uncovering an existence and studying the programming habits. 

This thesis presents a methodology for finding recurrent behaviors through the 
analysis of the variety of software process artifacts left after performing a 
software process. I have called this methodology ``Software Trajectory'' and it consists 
of four distinct steps. Each of these steps has a specific goal and compromising variety of 
means to reach it. 
At first software process artifacts are identified and collected. 
At second, they are cleaned, organized and classified. 
On the third step particular research questions are formulated and data are organized and indexed. 
And finally, a set of KDD techniques is applied in order to undercover recurrent behaviors which 
could potentially shed a light on the performed process details. 

My personal motivation for performing this work is coming from the recognition of the 
importance of the software in our lives and the severity of issues with its development. 
Through my everyday experiences with software development and use I have stumble upon 
a number of issues which made me realize that mentioned ``software crisis'' phenomena is very real.
As a user in industrial and academic settings I often find myself facing software failures 
which create numerous difficulties for reaching production or research goals. As a developer, 
in an attempt to be productive and in order to deliver a better software I have studied and 
explored a number of formal processes, however, sometime I found myself seeing a very little of 
rationale behind their application, and moreover, in this exploration, when facing the process
application failing to help I was unable to comprehend what exactly went wrong and what need 
to be changed. All of these experiences made me studying software process research and exploring
novel approaches to software process recovery on my own in order to understand software process better.

\section{Research area overview}
As mentioned above, in this thesis I am focusing on a very narrow subject - exploring approaches
for uncovering of recurrent behaviors or ``programming habits'' out of software process artifacts.
Before narrowing further 


Software is usually coded by teams. Members of these teams are agreed and bound to use 
a particular technologies and development tools, they also agree on following well defined 
development process which is constrained by a timeline and budget. These are necessary 
constraints to keep work organized, however there is a great freedom in what they actually 
do in every single moment of time in order to progress towards lines of code which eventually 
will result in software. For example one developer may follow test first process while
another writes tests at last.  This freedom of choice in ordering of development activities 
while being much appreciated by talented and creative individuals creates an impression 
of chaotic and unordered activities for random observers, newbies and people in 
charge - so there we have all the attempts of imposing an order 
(or control) on all of the development activities. Metrics and models of processes






%% similarity measures
\chapter{Time-series similarity measurements.}
Previously not explored in the classical time-series analyses such as trend and seasonality identification, forecasting, etc. the time-series similarity problem become the cornerstone problem in recently emerged data-mining applications. The ability to determine time-series similarity allows to resolve many general time-series KDD problems:
\begin{itemize}
	\item identifying stocks with similar movement in price \cite{citeulike:4295242} \cite{citeulike:4031865} \cite{citeulike:4025073}
	\item finding products with similar selling pattern \cite{citeulike:4326324}
	\item identifying music with the score similar to the copyrighted one \cite{citeulike:3821484} \cite{citeulike:3815076}
	\item performing speech to text conversion \cite{citeulike:3728228}
	\item verifying signatures and performing handwriting to text conversion \cite{citeulike:3733947} \cite{citeulike:3513035}
	\item finding objects with similar movemet trajectories \cite{citeulike:964832} \cite{citeulike:3728229} \cite{citeulike:3815864}
	\item finding developers with similar build patterns.
\end{itemize}

All of the solutions for stipulated problems based on the implementation of the time-series database enhanced by ability to process ``time-series similarity queries'' and at this point we essentially regard this problem as solved and will just overlook most of the solutions in this review.  Traditionally \cite{citeulike:3973409}, the similarity queries divided by two categories: whole matching and subsequence matching. The very detailed explanation of both problems is given in \cite{citeulike:3815880}: Goldin and Kanellakis starting with the exact-match problem and moving towards the similarity relation definition through the approximate match definition introducing the ``match up to similarity'' technique based on the ``user-percieved similarity''. The scales and shifts introduced by athors form a ``similarity relation'' $T_{a,b}$ having the next key properties:
\begin{align}
 & \mbox{1. } \forall X, \; X=T_{0,1}(X), \; \mbox{Reflexivity or identity transformation} \\
 & \mbox{2. } \mbox{if } X=T_{a,b}(Y) \; \mbox{then } Y=T{1/a,-b/a}(X) = T^{-1}_{a,b}(X), \; \mbox{Symmetry or inverse of $T_{a,b}$} \\
 & \mbox{3. } \mbox{if } X=T_{a,b}(Y) and Y=T{c,d}(Z), \; \mbox{then } X=T_{zc, ad+b}(Z) = (T_{a,b} * T_{c,d})(Z), \; \mbox{Transitivity}
\end{align}

the While whole matching requires time-series to be exactly the same length, the subsequence matching consider the smaller query time-series for which it finds the best match in the larger template time-series. In both cases the similarity query mechanism relies on some metrics with well defined distance function which used to quantify time-series similarity. 

The distance function on a set $X$ defined as:
\begin{equation}
 d: X \times Y \rightarrow \mathbb{R}
\end{equation}

And if $x$, $y$ and $z$ $\in X$ the distance function $d$ required to satisfy follwing conditions:
\begin{align}
 & \mbox{1. } d(x, y) > 0, \; \mbox{non-negativity} \\
 & \mbox{2. } d(x, y) = 0, \; \mbox{if and only if} \; x = y  \;  \mbox{identity} \\
 & \mbox{3. } d(x, y) = d(y, x), \; \mbox{symmetry} \\
 & \mbox{4. } d(x, z) \leq d(x, y) + d(y, z), \; \mbox{triangle inequality}
\end{align}
While a distance function is required to satisfy to all of these, we should note, that the Dynamic Time Warping (DTW) distance which is extremely popular among speech, writing and sign language recognition algorithms \cite{citeulike:3496861} \cite{citeulike:3744226} \cite{citeulike:3733947} \cite{citeulike:3789964} fails to satisfy the triangular inequality \cite{citeulike:4343286} \cite{citeulike:4343933}.
\section{Euclidean distance}
Given two time-series $X=(x_{1},x_{2}...x_{n})$ and $Y=(y_{1},y_{2}...y_{m})$ where $n=m$ the Euclidean distance is defined as
\begin{equation}
 D_{euclidean}(X,Y) = \sqrt{ \sum_{i=1}^{n}  (x_{i} - y_{i})^2 }
 \label{eq:euclidean_distance}
\end{equation}
While being easiest to calculate and satisfying to all of the distance requirements the Euclidean distance, when applying to the raw time-series data, seems to bee too rigid to capture similarity between time series which follow the very same pattern in time but somehow different in scale. In \cite{citeulike:4107287} authors outlining that Euclidean distance is ``..impractical in several applications, particularly for multimedia applications, where shrinking and stretching of the data are very typical'' and providing an example of two time-series representing the ``Happy birthday'' tune which were found distinct by the direct application of the Euclidean distance, but after normalization and scaling appeared to be identical as shown at the Figure \ref{fig:happybirthday}.
\begin{figure}[tbp]
   \centering
   \includegraphics[height=90mm]{happy_birthday.eps}
   \caption{a) Figure taken from \cite{citeulike:4107287} depicts raw pitch contour extracted from a sung query represents a query sequence Q, and a MIDI pitch contour of �Happy Birthday� song represents a candidate sequence C. b) A rescaled query sequence Q with scaling factor = 1.25. c) Both sequences after mean normalization at the query�s length. The shaded region shows their Euclidean distance.}
   \label{fig:happybirthday}
\end{figure} 

The first transformation applied by authors, normalization, discussed in the next section followed by the transformation rules. The Euclidean distance has a linear complexity.
\section{Time series normalization}

Normalization is a type of mathematical transformation of time series from one value domain into another value domain with a purpose to obtain a specific set of statistical features such as a limit of values, a certain variance, standard deviation or average for the transformed (normalized) time series. The operation usually takes one or two steps and transforms each of the elements $x_{i} \in X$ of input sequence into the element $x_{i}^{'} \in X^{'}$. There are different types of normalization known:
\begin{itemize}
	\item Normalization into an Interval \cite{citeulike:4295248} \cite{citeulike:2753031}
  \item Normalization to Sum 1
  \item Normalization to Euclidean Norm 1
  \item Normalization to Zero Mean
  \item Normalization to Zero Mean and Unit Standard Deviation \cite{citeulike:3815880}
\end{itemize}

\subsection{Normalization into an Interval}
This type of normalization procedure ensures that all elements of an input vector are scaled proportionally into an output vector with predefined upper and lower limits.
Let's $L_{min}$ to be a desired lower limit, $L_{max}$ to be a upper limit and let's $x_{max} = \max \left\{ x_{i}, x_{i} \in X \right\}$ and $x_{min} = \min \left\{ x_{i}, x_{i} \in X \right\}$. Than
\[
x_{i}^{'} = \frac{ (x_{i}-x_{min}) (L_{max} - L_{min}) }{ x_{max} - x_{min} } + L_{min	}, \: i \in \mathbb{N}
\]
is a normalization procedure.

\subsection{Normalization to Sum 1 and Normalization to Euclidean Norm 1}
While normalization to Sum 1 ensures that elements of $x^{'}$ sum up to 1:
\[
1 = \sum_{i=1}^{N} x_{i}
\]
by normalization procedure:
\[
x_{i}^{'} = \frac{ x_{i} }{ \sum_{i=1}^{N} x_{i} }
\]
the Normalization to Euclidean Norm 1 transforms the input vector values proportionally into an output vector with a Euclidean norm of 1:
\[
1 = \sum_{i=1}^{N} x_{i}^{2}
\]
by transformation procedure:
\[
x_{i}^{'} = \frac{ x_{i} }{ \sum_{i=1}^{N} x_{i}^2 }
\]

\subsection{Normalization to the zero mean}
This method ensures that the mean of the normalized vector will be approximately $0$. The mean of the vector $X$ of length $N$ calculated as:
\[
\mu_{X} = \frac{1}{N}\sum_{i=1}^{N}x_{i}
\]
and the normalization procedure is:
\[
x_{i}^{'} = x_{i} - \mu, \: i \in \mathbb{N}
\]

\subsection{Normalization to Zero Mean and Unit Standard Deviation} \label{sect:normalization}
This type of normalization taken from the \cite{citeulike:3815880} and ensures that all elements of the input vector are transformed into the output vector which mean is approximately $0$ while the standard deviation (and variance) are in a range close to $1$.
This procedure uses mean $\mu$ and standard deviation which calculated as 
\[
\sigma = \sqrt{ \frac{ \sum_{i=1}^{N} (x_{i} - \mu)^{2} }{ N - 1 } }
\]
or equivalently
\[
\sigma = \sqrt{ \frac{
                  N \left( \sum_{i=1}^{N} x_{i}^{2}  \right) - 
                  \left( \sum_{i=1}^{N} x_{i} \right) ^{2}
                }{
                  N(N-1)
                }  
          }
\]
The normalization itself is 
\begin{equation}
x_{i}^{'} = \frac{x_{i} - \mu}{\sigma}, \: i \in \mathbb{N}
\end{equation}
and yields the vector $x_{i}^{'}$ such as $\mu \approx 0$ and $\sigma \approx 1$.

According to the most of the recent work \cite{citeulike:3815880} \cite{citeulike:2821475} \cite{citeulike:3978002} this type of time-series normalization is the best known transformation of the raw time-series which preserves original time-series features. Nevertheless, the article by Lin et al. \cite{citeulike:2821475} explains reasons and solutions for some cases when the zero mean and unit standard deviation normalization fails. For example if signal is constant at most of the time span with minor noise at the short interval, this normalization will overamplify the noise to the maximal amplitude.
\section{Transformation rules (smoothening, scales and shifts).} \label{scales_and_shifts}
As we have seen, the Euclidean distance is efficient and easily computable but it doesn't capture similar time-series in the presence of noise, scales or shifts. Much effort have been put in to overcome this issue. As mentioned before, in \cite{citeulike:3815880} authors introduced scale and the shift shape-perceiving transformations which are applied before measuring the similarity with Euclidean distance. Their proposed definition of similar time-series is based on the transformation $T_{a,b}$ where $a$ is the scaling coefficient and $b$ is the shift coefficient. This ``similarity transformation $T_{a,b}$'' is a mapping of each of the $x_{i} \in X$ into $x_{i}\acute{} = a*x_{i}+b$ and if this transformation can be found for two time-series $X$ and $Y$ such that $X=T_{a,b}(Y)$ time-series $X$ and $Y$ are considered to be similar.

Agrawal et al. \cite{citeulike:3816327} proposed more general method of aligning of time-series by allowing any arbitrary segment of the query time-series to be scaled and stretched by any suitable amount and adding the ability of deletion of any non-matching segment (note that this work also seems to be the one of the first proposing LCS application to time-series similarity, see \ref{lcs}). Also, they introduced an $\epsilon$ envelope around the template sequence which aimed to deal with noise on the query time-series. The figure  \ref{fig:transform} shows principles of such a method walking from the raw time series through discussed transformations.

\begin{figure}[tbp]
   \centering
   \includegraphics[height=50mm]{transform.eps}
   %%{seriesheatmap}
   \caption{Illustration of scaling, smoothening and shifting: the leftmost plot depicts the raw time-series, plot at the middle shows the query time-series after scaling and smoothening and the plot at the right adds shifts to transform while showing $\epsilon$ alignment envelope.}
   \label{fig:transform}
\end{figure} 

Since the Euclidean distance is calculated only after discussed transformations applied, it is unable to capture the cost and complexity of the these transformations and does not reflect the true similarity. If the cost assigned to the each of the transformations performed \cite{citeulike:3731711}, than the integral of all effective costs will measure the true similarity between time-series.

The complexity of scales and shift transformation is $O(NM)$ where $N$ and $M$ are the query and template sequences.


\section{Dynamic Time Warping (DTW)}
\section{Longest Common Subsequence (LCS) }
Another widely used metrics in the time-series similarity is the Longest Common Subsequence or LCS. LCS could be seen as the successor of the application of the Euclidean distance, DTW or scales and shifts \cite{citeulike:3816327} since it might be essentially based on any of these. 

The idea of LCS is explained in the classical Computer Science manuscript \cite{citeulike:180287} for the string matches and perfectly described in \cite{citeulike:4367061} for the real values and feature sequences. While Yazdani and Ozsoyoglu designed their algorithm specifically for image sequences their explain it as an algorithm of matching real-valued sequences and specifically pointed application to time-series and time-series features as Fourier coefficients etc. The Figure \ref{fig:lcs} taken from the article and explains the approach taken: each of the images is approximated by the set of the specific to the domain real-values, and if $R$ and $S$ match than $FR$ and $FS$ match anlagously, if $FR$ and $FS$ are similar than $R$ and $S$ are similar, but we will extend the (image) decomposition approach in the next chapter, here we will focus on the LCS algorithm.

\begin{figure}[tbp]
   \centering
   \includegraphics[height=40mm]{lcs.eps}
   %%{seriesheatmap}
   \caption{This figure is taken from \cite{citeulike:4367061} and while it designed to explain LCS application to the sequences of images it easily explains any of LCS and feature decomposition based approach to time-series similarity.}
   \label{fig:lcs}
\end{figure} 



\chapter{Dimensionality reduction, indexing and bounding.}
As we seen with application of LCS to images in \cite{citeulike:4367061} Yazdani and Ozsoyoglu charachterized each of the images by the set of Fourier coefficients. By employing such a solution they managed to resolve the ``dimensionality curse'' in the time series-similarity problem which essentially decreases the performance of any similarity algorithm to the one of the sequential search with the growth of dimensionality \cite{citeulike:4384496} \cite{citeulike:2843857} \cite{citeulike:4384489} \cite{citeulike:343069}.

The problem addressed in this chapter is the approximation (lossy compression) of a time series data in a way which allows to maintain a fast random access, comparison and indexing of the time-series with minimal errors. There are many approximation proposed in the research literature and the Figure \ref{fig:approximations} from \cite{citeulike:2821475} illustrates their hierarchy. All of the methods essentially at the first step transform given set of time series into some low-dimensional representation and index them in order to utilize data-mining, indexing and clustering algorithms later. Key questions we would ask in this review of the reduction methods are the sensitivity and selectivity of algorithms, their complexity and performance.

\begin{figure}[tbp]
   \centering
   \includegraphics[height=45mm]{ts_representations.eps}
   %%{seriesheatmap}
   \caption{The figure from \cite{citeulike:2821475} illustrating a hierarchy of all of the flavors of time series representation (decomposition).}
   \label{fig:approximations}
\end{figure} 

\section{Lower Bounding}
The time-series similarity matching problem was first examined by Agrawal et al \cite{citeulike:3973409} and while employing DFT and proposing F-index they were relying on the use of the Parseval's theorem which guaranteed that the distance between frequiencies in the time domaine is the same as in the frequency domain \ref{eq:dft_similarity}. By using only first $f$ coefficients of DFT they were introducing false hits into F-Index but ensured that there are no false-dismissals. Following work by Faloutsos et al \cite{citeulike:825581} introduced Minimum Bounding Rectangles (MBRs) and generalized the F-index use for the subsequence matching. Both approaches in order to guarantee that there is no fals-dismissals for a transform $T$ the distance measure in the feature space must must satisfy a ``contractive property'' or lower-bounding condition:
\begin{equation}
D_{feature}(T(x),T(y)) \; \leq \; D(x,y) 
\label{eq:bounding}
\end{equation}
where $D$ is the distance and $x$ and $y$ are the time-series.
\section{Discrete Fourier Transform (DFT)}
Agrawal et al. in \cite{citeulike:3973409} proposed a DFT-based method for the time-series indexing and similarity queries processing. Their idea based on the observation of the fairly good approximation of the time-series by only few ``strong'' frequencies and on the Parseval's Theorem (aka Rayleigh's energy theorem). 

For each of the time-series $x$ of length $N$ authors proposing to extract only $c$ first DFT coefficients (frequencies), where $c<<N$ and $f_{c}$ is the ``cutoff frequency''. Thus each of the time-series is mapped into the low $c$ dimensional space and stored in the $F$-index, whether the search over the index implemented by using the $R^{*}$ tree \cite{citeulike:343069}. Authors arguing that the approach taken is characterized by the ``completeness of the feature-extraction'' and it is ``efficiently dealing with the dimensionality curse''. 

The $n$-point DFT transform of the time-series $X=(x_{0}, x_{1}, ... , x_{n-1})$ is defined as:
\begin{align}
& X_{f} = 1/\sqrt{(n)}\sum_{t=0}^{n-1} x_{t} \exp(-j2 \pi f t/n),\; t=0,1,...,n-1, \; j=\sqrt{(-1)} \\
& \text{also the inverse transform:} \nonumber \\
& x_{t} = 1/\sqrt{(n)}\sum_{f=0}^{n-1} X_{f} \exp(j2 \pi f t/n),\; t=0,1,...,n-1 
\end{align}
and the fundamental observation of the Paseval's theorem is
\begin{equation}
\sum_{t=0}^{n-1} \left| x_{t} \right| ^{2} = \sum_{f=0}^{n-1} \left| X_{f} \right| ^{2}
\label{eq:parseval}
\end{equation}
that the energy of the sequence in the time domain equals the energy in the frequency domain.

Narrowing further, authors show that Discrete Fourier Transform inherits linearity and preservance of amplitude coefficients during the time-shifts from the Continuous Fourier Transform. Taking this properties and Parseval's theorem in account authors stating that
\begin{equation}
\left\| x - y \right\| ^{2} \equiv \left\| X - Y \right\| ^{2}
\label{eq:dft_similarity}
\end{equation}
which implies the equivalence of the Euclidean distances between two sequences $x$ and $y$ in the time and frequency domains. 

Agrawal et al. introducing the distance function $D$ where distance between two sequences $x$ and $y$ is the square root of the energy of the difference:
\begin{align}
D(x,y) \equiv \left( \sum_{t=0}^{n-1} \left| x_{t} - y_{t} \right| ^{2} \right) ^{1/2}
       \equiv \left( E(x-y) \right) ^{1/2} \\
\text{where } E(x) \; \text{is the energy of the sequence:} \nonumber \\
E(x) \equiv \left\| x \right\| ^{2} \equiv \sum_{t=0}^{n-1} \left| x_{t} ^{2} \right|
\label{eq:dft_distance}
\end{align}
and implementing the similarity relation between two sequences by using a user defined threshold $\epsilon$ - i.e. if distance between two sequences falls below threshold they are similar:
\begin{equation}
D(x,y) \leq \epsilon \; \Rightarrow \text{$x$ similar to $y$}
\label{eq:dft_similarity_dft}
\end{equation}

Continuing the discussion authors comparing their DFT-based method to the ``sequential scanning'' method. They used the same $R^{*}$ tree for the sequential indexing and implemented ``early abandoning'' as soon as the calculated Euclidean distance exceeds $\epsilon^{2}$ declaring two sequence dissimilar. It was shown by comparing two approaches that with the growth of the dataset and the length of the sequences the DFT-based approach outperforms the sequential search. Another interesting point shown is that in general $f_{c} \approx 3$ is enough to capture time-series features and build index which provides satisfactory performance and has a contractive property, i.e. ensures no false-dismissal.

\section{Singular Value decomposition (SVD)}
The Singular Value Decomposition for ad-hoc query support in the time-series databases was proposed by Korn et. al. in \cite{citeulike:4373332}. They started with a case study of the typical warehouse dataset (AT\&T) of sale patterns representing the data as a matrix $X$ of size $N \times M$ where rows correspond to distinct customers and columns to the time intervals. One of the observations is that the amount of customers is much larger than the length of the time-series: $N>>M$. Another observation is that the size of such a matrix could easily reach the unmanageable in order to process ``random access'' queries size. The proposed solution for this problem is to design a ``compression'' algorithm which by trading in some of the precision will allow to process significant amount of the data with random access queries. 

While being aware of some convenient compression techniques such as LZ and Huffman, as well as the spectral decomposition (DFT, DCT and DWT) and clustering techniques, Korn et. al. are arguing that their Singular Vale Decomposition approach is the only solution which does not suffer from the CPU-time complexity and performs spectral decomposition in the optimal manner.

The SVD decomposition method is based on the theorem stating that the real-valued matrix $X$ can be expressed as 
\begin{equation}
X = U \times \Lambda \times V^{t}
\label{eq:svd_transform}
\end{equation}
where $U$ is a column-orthonormal $N \times r$ matrix ($r$ is rank of the matrix $X$), $\Lambda$ is a diagonal $r \times r$ matrix of eigenvalues $\lambda_{i}$ of $X$ and $V$ is a column-orthonormal $M \times r$ matrix. Which essentially is the spectral decomposition:
\begin{equation}
X = \lambda_{1} u_{1} \times v_{1}^{t} + \lambda_{2} u_{2} \times v_{2}^{t} + \ldots + \lambda_{r} u_{r} \times v_{r}^{t}
\label{eq:svd_spectrum}
\end{equation}
of the matrix $X$ where $u_{i}$ and $v_{i}$ are column vectors of the $U$ and $V$ and $\lambda_{i}$ is a diagonal elements of $\Lambda$.

Once $X$ is decomposed (two passes over the matrix) into the form of spectrum \ref{eq:svd_spectrum}, authors suggesting to truncate the sum in equation to the first $k$ terms, $k \leq r \leq M$:
\begin{equation}
\hat{X} = \sum_{i=1}^{k} \lambda_{i} u_{i} \times v_{i}
\label{eq:svd}
\end{equation}
where $k$ depends on the space restriction. Note, that these $k$ terms also known as the ``principal components'' in the PCA analysis. 

The compression ratio after applying the SVD transform is 
\begin{equation}
s = \frac{N*k + k + k + k*M}{N*M} \approx \frac{k}{M}
\label{eq:svd_compression}
\end{equation}
since $N >> M \geq k$. The reconstruction of the any cell of the original matrix $\hat{X}$ takes only $O(k)$ time:
\begin{equation}
\hat{x}_{i,j}  = \sum_{m=1}^{k} \lambda_{m} * u_{i,m} * v_{j,m},\; i=1,...,N;\; j=1,...,M
\label{eq:svd_reconstruct}
\end{equation}

Since the SVD approximation could yield considerable errors when restoring some of the original data, authors suggesting to keep correcting information in the form of tuples $(row,column,delta)$ for the values which approximated poorly (outliers) calling this method as ``SVD with Details'' or SVDD. 

\begin{figure}[tbp]
   \centering
   \includegraphics[height=55mm]{svd.eps}
   %%{seriesheatmap}
   \caption{The Reconstruction Error (RMSPE) versus disk storage space (\%) for clustering, DCT, SVD and SVDD.}
   \label{fig:svd_benchmark}
\end{figure} 

As per performance of both SVD and SVDD methods, they were compared with DCT (discrete cosine transform) and the S-PLUS embedded clustering method and found to be significantly superior \ref{fig:svd_benchmark}, especially SVDD with the growth of the space allowed to use for corrective information.
\section{Discrete Wavelet Transform (DWT)}
\section{Piecewise Approximation of time-series (PLA, PCA, PAA, APCA)}
Following the first wave of time-series similarity methods based on the spectral decomposition such as DFT, DCT, SVD, CHEB\footnote{Chebyshev polynomials based decomposition methods were omitted in this review due to the space constraints and could be found in \cite{citeulike:2753031} and overall very similar to APCA in performance}
and Haar wavelets, another approach has become popular among the time-series data-mining community: piecewise approximation. Faloutsos et al \cite{citeulike:4344279} proposed Piecewise Flat Approximation, Morinaka et al \cite{citeulike:4295248} and Chen et al \cite{citeulike:4165220} proposed Piecewise Linear Approximation (PLA), Yi \& Faloutsos \cite{citeulike:2946589} and Keogh et al \cite{citeulike:3000416} followed with Piecewise Aggregate Approximation (PAA), Chakrabarti et al \cite{citeulike:1736140} proposed Adaptive Piecewise Linear Approximation (APLA). All this work has shown that surprisingly simple piecewise-based approximation methods outperform previous spectral decomposition based techniques by being easy to compute and index while satisfying the contractive property condition (\ref{eq:bounding}).

We will review the PAA method which approximates the time-series $X$ of length $n$ into vector $\bar{X} = ( \bar{x}_{1}, ..., \bar{x}_{M} )$ of any arbitrary length $M \leq n$ where each of $\bar{x_{i}}$ is calculated by following the next formula:
\begin{equation}
\bar{x}_{i} = \frac{M}{n} \sum_{j=n/M(i-1)+1}^{(n/M)i} x_{j}
\label{eq:paa}
\end{equation}

This simply means that in order to reduce the dimensionality from $n$ to $M$, we first divide the original time-series into $M$ equally sized frames and secondly compute the mean values for each frame. The sequence assembled from the mean values is the PAA transform of the original time-series. It was shown by Keogh et al that the complexity of the PAA transform can be reduced from $O(NM)$ (\ref{eq:paa}) to $O(Mm)$ where $m$ is the number of sliding windows (frames). The satisfaction of the transform to bounding condition in order to guarantee no false dismissals was also shown by Yi \& Faloutsos and Keogh et al by introducing the distance:
\begin{equation}
D_{PAA}(\bar{X}, \bar{Y}) \equiv \sqrt{\frac{n}{M}} \sqrt{ \sum_{i=1}^{M} 
\left(  \bar{x}_{i} - \bar{y}_{i} \right)}
\label{eq:paa_distnace}
\end{equation}
and showing that $D_{PAA}(\bar{X}, \bar{Y}) \leq D(X,Y)$.

Concluding the piecewise based time-series approximation methods review we should note that PAA is very similar to Haar-wavelet based approach \cite{citeulike:4384535} as shown at Figure \ref{fig:paa_comparison}. Another nice feature of the PAA is the ability to process range queries with sequences of unequal to index size dimension, as was shown by Keogh et al in \cite{citeulike:3000416}.
\begin{figure}[tbp]
   \centering
   \includegraphics[height=95mm]{paa_comparison.eps}
   %%{seriesheatmap}
   \caption{The combination of figures from \cite{citeulike:3000416} depicts different approaches for the time-series approximation (decomposition): a) the time-series spectral approximation (Fourier); b) SVD-based approximation; c) Haar-wavelet based approximation; d) Piecewise Aggregate Approximation where transformed values shown as ``box'' basis functions.}
   \label{fig:paa_comparison}
\end{figure} 

Overall, the PAA based approach to the time-series similarity problem was found to be very competitive in precision to the spectral-decomposition based methods while outperforming all competitors in the speed of index building. In terms of the index performance, PAA index has a constant time of insertions and deletion whether in case of SVD, for example, we have to rebuild the whole matrix. Also, Keogh et al has shown that PAA has ability to handle weighted Euclidean distance metrics, allowing implementation of more sophisticated querying techniques like ``relevance feedback'' \cite{citeulike:4406444}.

\section{Symbolic Aggregate approXimation (SAX)}
The last approach for the time-series similarity problem we are reviewing in this writing is the current state of the art time-series representation and dimensionality reduction method called Symbolic Aggregate approXimation (SAX) which transforms original time-series data into symbolic strings. This method, proposed by Lin et al \cite{citeulike:2821475}, turns out to be not only extremely simple and computationally cheap, but also fast and precise in the range-query processing. Moreover, the use of the symbolic representation opens door to the existing wealth of data-structures and string-manipulation algorithms in computer science such as hashing, suffix trees, regular expression pattern matching, etc.

SAX transforms a time-series $X$ of length $n$ into the string of arbitrary length $\omega$ where typically $\omega << n$, using an alphabet $A$ of size $ a \geq 2$. The SAX algorithm consist of two steps: at the first step it transforms the original time-series into PAA representation and this intermediate representation than converted into the string during the second step. Use of the PAA at the first step brings an advantage of simple and efficient dimensionality reduction while providing the important lower bounding property. Second step, actual conversion of PAA coefficients into letters also computationally efficient and the lower bounding of symbolic distance was proven by Lin et al.

Discretization of the PAA representation of the time-series into SAX implemented in a way which produces symbols corresponding to the time-series features with equal probability. The rigorous analysis of the time-series datasets available for authors shows that normalized by the zero mean and unit of energy time-series follow the Normal distribution law. By using the Gaussian distribution properties \cite{citeulike:167581} it's easy to pick $a$ equal-sized areas under the Normal curve. The points of the cut lines slicing the the under-the-Gaussian-curve area called ``breakpoints''.
\begin{figure}[tbp]
   \centering
   \includegraphics[height=45mm]{sax_intro.eps}
   \caption{The illustration of the SAX approach taken from \cite{citeulike:2821475} depicts two pre-determined breakpoints for the three-symbols alphabet and the conversion of the time-series of length $n=128$ into PAA representation first and following mapping of the PAA coefficients into SAX symbols with $w=8$ and $a=3$ resulting in the string \textbf{baabccbc}.}
   \label{fig:sax_intro}
\end{figure}

Extending Euclidean \ref{eq:euclidean_distance} and PAA \ref{eq:paa_distnace} distances, the function returning the minimal distance between two string representations of original time series $\hat{Q}$ and $\hat{C}$ defined as
\begin{equation}
MINDIST(\hat{Q},\hat{C}) \equiv \sqrt{ \frac{n}{w} } \sqrt{ \sum_{i=1}^{w} ( dist( \hat{q}_{i}, \hat{c}_{i} ) )^{2}}
\label{eq:sax_mindist}
\end{equation} 
where the $dist$ function is implemented by using the lookup table for the particular set of the breakpoints as shown in table \ref{tbl:sax_lookup} where the singular value for each cell $(r,c)$ is computed as 
\begin{equation}
cell_{(r,c)} = 
\begin{cases} 
0, \text{ if }\left| r-c \right| \leq 1 \\
\beta_{\max(r,c) - 1} - \beta_{\min(r,c) - 1}, \text{ otherwise}
\end{cases}
\label{eq:cell}
\end{equation}
\begin{table}
\begin{tabularx}{400pt}{X X X X X}
\hline
   & a   & b    & c    & d    \\
\hline
a & 0    & 0    & 0.67 & 1.34 \\
b & 0    & 0    & 0    & 0.67 \\
c & 0.67 & 0    & 0    & 0    \\
d & 1.34 & 0.67 & 0    & 0    \\
\hline
\end{tabularx}
\caption{A lookup table used by the MINDIST function for the $a=4$}
\label{tbl:sax_lookup}
\end{table}
\begin{figure}[tbp]
   \centering
   \includegraphics[height=47mm]{sax_distance.eps}
   \caption{The visual representation of the two time-series $Q$ and $C$ and three distances between their representation: Euclidean distance between raw time-series (A), the distance defined for PAA coefficients (B) and the distance between two SAX representations (C).}
   \label{fig:sax_distance}
\end{figure}


As shown by Li et al, the introduced SAX distance measure lower-bounds the PAA distance, i.e.
\begin{equation}
\sum_{i=1}^{n} (q_{i} - c_{i})^{2} \geq n(\bar{Q} - \bar{C})^{2} \geq n(dist(\hat{Q},\hat{C}))^2
\label{eq:sax_bounding}
\end{equation}
	
\chapter{Conclusion}
There has been an increasing amount of research over the last two decades into time-series data mining, knowledge discovery and applications. It is very interesting to see how the focus of this research shifted from computationally intensive and mathematically elegant methods of spectral decomposition of time-series to extremely simple approximation methods such as PAA and SAX. In my opinion, this is due to the branching of the time-series data-mining field away from the classical time-series analysis school. While researchers started with adoption of the classical spectral analysis approach, which is almost perfect for resolving the ``secret'' of a time-series generative process, the time-series data-mining community, interested in the efficient navigation through the thousands of signals and petabytes of data, found its own extremely efficient ways to perform on-the-fly extraction and analysis of patterns. The latest application of SAX approximation to the image recognition \cite{citeulike:3175770} and monitoring insects real-time \cite{citeulike:4446167} by Keogh et al demonstrate how fast, accurate, and resource-efficient these techniques are. 


%%% Input file for bibliography
\bibliography{seninp}
%% Use this for an alphabetically organized bibliography
\bibliographystyle{plain}
%% Use this for a reference order organized bibliography
%\bibliographystyle{unsrt}
%% Try using this BibTeX style that hopefully will print annotations in
%% the bibliography. This will allow me to make notes on papers in the
%% BibTeX file and have them readable in the references section until
%% I turn them into a conceptual literature review 
%\bibliographystyle{annotation}

\end{document}