\chapter{Time-series similarity measurements.}
Previously not explored in the classical time-series analyses such as trend and seasonality identification, forecasting, etc. the time-series similarity problem become the cornerstone problem in recently emerged data-mining applications. The ability to determine time-series similarity allows to resolve many general time-series KDD problems:
\begin{itemize}
	\item identifying stocks with similar movement in price \cite{citeulike:4295242} \cite{citeulike:4031865} \cite{citeulike:4025073}
	\item finding products with similar selling pattern \cite{citeulike:4326324}
	\item identifying music with the score similar to the copyrighted one \cite{citeulike:3821484} \cite{citeulike:3815076}
	\item performing speech to text conversion \cite{citeulike:3728228}
	\item verifying signatures and performing handwriting to text conversion \cite{citeulike:3733947} \cite{citeulike:3513035}
	\item finding objects with similar movemet trajectories \cite{citeulike:964832} \cite{citeulike:3728229} \cite{citeulike:3815864}
	\item finding developers with similar build patterns.
\end{itemize}

All of the solutions for stipulated problems based on the implementation of the time-series database enhanced by ability to process ``time-series similarity queries'', at this point we essentially regard this problem as solved and will just overlook most of the solutions.  Traditionally \cite{citeulike:3973409}, the similarity queries divided by two categories: whole matching and subsequence matching. While whole matching requires time-series to be exactly the same length, the subsequence matching consider the smaller query time-series for which it finds the best match in the larger template time-series. In both cases the similarity query mechanism relies on some metrics with well defined distance function which used to quantify time-series similarity. 

The distance function on a set $X$ defined as:
\begin{equation}
 d: X \times Y \rightarrow \mathbb{R}
\end{equation}

And if $x$, $y$ and $z$ $\in X$ the distance function $d$ required to satisfy follwing conditions:
\begin{align}
 & \mbox{1. } d(x, y) > 0, \; \mbox{non-negativity} \\
 & \mbox{2. } d(x, y) = 0, \; \mbox{if and only if} \; x = y  \;  \mbox{identity} \\
 & \mbox{3. } d(x, y) = d(y, x), \; \mbox{symmetry} \\
 & \mbox{4. } d(x, z) \leq d(x, y) + d(y, z), \; \mbox{triangle inequality}
\end{align}
While a distance function is required to satisfy to all of these, we should note, that the Dynamic Time Warping (DTW) distance which is extremely popular among speech, writing and sign language recognition algorithms \cite{citeulike:3496861} \cite{citeulike:3744226} \cite{citeulike:3733947} \cite{citeulike:3789964} fails to satisfy the triangular inequality \cite{citeulike:4343286} \cite{citeulike:4343933}.