\chapter{Conclusion}
There has been an increasing amount of research over the last two decades concerning time-series data mining, knowledge discovery and applications. It is very interesting to see how the focus of this research shifted from computationally intensive and mathematically elegant methods of spectral decomposition of time-series to extremely simple approximation methods such as PAA and SAX. In my opinion, this observed phenomena is due to the original branching of the time-series data-mining field from classical time-series analysis school. While started with adoption of the classical spectral analysis approach, which is almost perfect for resolving a ``secret'' of a time-series generative process, time-series data-mining community, interested in the efficient navigation through the thousands of signals and petabytes of data, found extremely efficient ways to perform on-the-fly extraction and analysis of patterns. The latest application of SAX approximation to the image and pattern reconition \cite{citeulike:3175770} and monitoring insects real-time \cite{citeulike:4446167} by Keogh et al demonstrate how fast, accurate and resource-efficient this technique is. 