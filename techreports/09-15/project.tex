%%%%%%%%%%%%%%%%%%%%%%%%%%%%%% -*- Mode: Latex -*- %%%%%%%%%%%%%%%%%%%%%%%%%%%%
%% project.tex -- 
%% Author          : Philip Johnson
%% Created On      : Tue Mar 31 11:44:58 2009
%% Last Modified By: Philip Johnson
%% Last Modified On: Sun Nov 22 11:58:11 2009
%% RCS: $Id$
%%%%%%%%%%%%%%%%%%%%%%%%%%%%%%%%%%%%%%%%%%%%%%%%%%%%%%%%%%%%%%%%%%%%%%%%%%%%%%%
%%   Copyright (C) 2009 
%%%%%%%%%%%%%%%%%%%%%%%%%%%%%%%%%%%%%%%%%%%%%%%%%%%%%%%%%%%%%%%%%%%%%%%%%%%%%%%
%% 

\pagenumbering{arabic}
\renewcommand{\thepage} {C--\arabic{page}}

\renewcommand{\thesection} {C.\arabic{section}}
\setcounter{section}{0}

\section{Project Description}

\section{Introduction}

Development of the ``smart grid'', a modernized power infrastructure, is
one of the key technological challenges facing the United States at the
dawn of the 21st century. According to the Department of Energy \cite{doe}, the smart
grid should: (1) Enable active participation by consumers by providing
choices and incentives to modify electricity purchasing patterns and
behavior; (2) Accommodate all generation and storage options, including
wind and solar power.  (3) Enable new products, services, and markets
through a flexible market providing cost-benefit tradeoffs to consumers
and market participants; (4) Provide reliable power that is relatively
interruption-free; (5) Optimize asset utilization and maximizes operational
efficiency; (6) Provide the ability to self-heal by anticipating and
responding to system disturbances; (7) Resist attacks on physical
infrastructure by natural disasters and attacks on cyber-structure by
malware and hackers.

In October, 2009, approximately \$3.4 billion dollars in federal stimulus
money was awarded to approximately 100 organizations in 49 states to
support smart grid development.  These awards, which were matched by \$4.7
billion dollars in private funds, are being used primarily for the
installation of smart meters, an initial enabling technology for the smart grid.
Other uses of the stimulus money include installation of a secure
communications network (Wyoming); installation of phasor measurement units
for monitoring grid stability (New York, Massachussetts, Indiana,
Louisiana, and others); and introduction of dynamic pricing (Maryland,
Arizona, California, and others).

While this stimulus package will inject over \$8 billion dollars into the
smart grid, the investment is almost totally focused on low-level
infrastructure.  By analogy to the Internet, it is similar to upgrading a
copper wire network to fiber optic cable, along with installation of high
performance routers and name servers.  Such infrastructure is a necessary
requirement for high-level Internet services such as the World Wide Web,
but does relatively little to determine the nature of those services. 

The smart grid will carry information about electricity as well as
electricity itself, thus forming its own kind of electrical Internet. At
the current time, it is not clear what ecosystem of higher-level services
will be developed to communicate, analyze and interpret this low-level
electrical data.

One reason why there is little clarity about information processing in the
Smart Grid is due to the nature of the current grid, which (from a consumer
point of view) is a classic ``black box'' technology.  For almost 100
years, consumers have plugged appliances into the grid via electrical
outlets and expected them to ``just work''.  Put another way, the
U.S. power industry has operated for a century under the assumption that
consumers should have access to a virtually unlimited amount of high
quality, stable, power.  Traditionally, consumers have been given extremely
little information about power beyond a monthly bill because neither
consumers nor utilities believed it was needed.

Electricity as black box requires electricity to be cheap, reliable, and
unlimited, but those days appear to be drawing to a close for a number of
reasons.  First, the traditional use of fossil fuels as a primary source of
power generation is not sustainable: most economists agree that ``peak
oil'' has either already occurred or will occur in the next two decades. At
that point, oil supply will decrease and prices will increase.  Second,
fossil fuels generate green house gases that contribute to climate change,
and so there is an urgent need to move to renewable energy sources such as
solar, wind, wave, and geothermal.  Third, reliance on fossil fuels creates
a variety of political problems.

Integration of renewable energy sources, unfortunately, creates significant
new problems related to storage and grid stability.  Unlike fossil-fuel
based power generation (called ``firm'' since utility companies can
generate new power from fossil fuels at will), most renewable energy
sources generate energy depending upon generally unpredictable
environmental factors.  As a result, simply adding a solar panel to every
rooftop in a community and tying them into the grid would actually do more
harm than good.  This is because grid stability depends upon energy supply
equaling demand on a second to second basis, and U.S. utilities currently
do not have a way to monitor and balance a grid that incorporates
widespread, distributed generation without distributed storage. 

If electricity cannot any longer be a black box to consumers, then an
important question is: what kind of ``white box'' will it become? In other
words, what kind of access to energy information should the smart grid make
available to consumers? At the highest level, this is the research question
that we pursue in this proposal.

In the following pages, we present a set of technological developments and
associated experiments to gather scientific, replicable data about the ways
in which access to information about energy usage impacts on consumer
behavior.  Our experimental method includes both the study of current
energy monitoring technology, as well as simulations to understand the
potential impact of future smart grid information access technologies.  The
contributions of this research will include: (1) case study data and
interpretations intended to provide new insight into the Smart Grid; (2)
methods for inquiry into Smart Grid technologies and their impact on
consumers; and (3) open source technologies to facilitate scientific
experimentation on the Smart Grid.

We believe it is vitally important to fund scientific study of the way in
which consumers respond to various forms of energy information, because
current data about smart grids and consumers is almost totally anecdotal in
nature, and market forces may not necessarily yield the optimal outcome by
themselves and in the absence of scientific guidance.  This is because
market forces in the power sector are not always in alignment with consumer
interests. As one example, most utility companies are publically owned
companies whose profits are directly tied to how much energy they sell.  As
a result, they have an economic disincentive for energy conservation.

The design of Smart Grid technologies tends to reflect one of the following
three views of energy consumers:

{\em (1) Consumers are too lazy or uninterested to change their 
behaviors based upon energy information.}

This view is reflected in smart meter designs that communicate only with
utilities and do not provide data back to consumers about their own
consumption. It is also reflected in the design of demand response systems
that gives utilities control over energy consumption as well as
generation, by enabling utilities to turn off water heaters or change
thermostat settings in homes and offices. 

There are at least three problems with this view. First, utilities can only
control consumption to a certain degree through demand response, so the
impact of such systems is limited.  Second, there is evidence that people
do not wish utilities to be able to unilaterally make such decisions.
Third, some worry about the privacy implications of allowing the utilities
to gain access to household energy consumption; for example, that data
would be gathered that the utility could then sell for the purpose of
targeted advertising.

{\em (2) Consumers respond only to cost.}

This view is reflected in dynamic pricing mechanisms, where the consumer is
charged different rates at different times of day depending upon the cost
to the utility of providing energy at that time.

Cost is, indeed, a powerful incentive for behavioral change, but dynamic
pricing schemes are intended to support ``peak shaving'', or curtailment in
the maximum amount of power the utility must be prepared to provide to the
community.

{\em (3) Consumers are motivated by ecological considerations.}

There are a growing number of sites which help consumers to understand the
ecological impact of their energy-related behavior.  For example, the
Ecotricity site has a ``UK Grid Live'' page which displays a stoplight in
one of three states: red, yellow, or green, depending upon the carbon
intensity of the grid.  

In our research, we take the view that any or all of these situations could
be the case, even within a single individual over time. Our goal is to
learn about the factors that lead to one or more of these decisions. We
will develop technology that enables us to explore an integrated view.













\section{Random Notes}

\subsection{Design}

We want to investigate: timeliness of feedback (instantaneous (similar to
TED), delayed by 15 minutes (similar to powermeter), delayed by one day or
more (similar to electrical company).  type of feedback (device specific
(similar to kill a watt), house-level (similar to TED), community level
(nothing similar), grid level (similar to Ecotricity etc.).  form of
feedback (webpage, twitter, facebook, IM, text message)  incentives (none,  These are the
independent variables.  

We want to get at the following:  user preferences; which do users
gravitate toward when given a choice? behavioral impacts (can we observe
changes in energy-related behavior?  What were the implicit incentives for
the behavioral change? (cost savings, aspirational needs (beat others;
improve skill); ecological concerns; for the kids/country).  

Our study design must also take into account demographic issues: whether
the study subjects live in detached houses vs. multi-unit apartments;
whether they are owners or renters; whether utilities are shared or
individual; whether their data is soley based on their behavior or
aggregated with others; whether they have the economic means to effect
change (socio-economic status). 

Broadly, we are looking at what form next generation interfaces should
take, combined with what kinds of energy-related information should be
provided.  We are not looking at UI-specific issues, but rather the kinds
of information provided by the user interface. 

\subsection{Guthridge}

"In terms of energy efficiency and conservation, just installing a smart
meter isn't going to have much effect," says Greg Guthridge, a smart-grid
consultant with Accenture.  "You'd be led to believe that you put in
real-time pricing and a nice Web site and immediately get a 15\% reduction
in energy use," says Accenture's Guthridge. "But reality is quite
different," and big hurdles remain in getting customers, most of whom are
accustomed to pay little attention to their power bills, to micromanage
their energy use and accept energy market price risk.  About half of
U.S. homes are on equalized billing programs that even out power bills
month to month and are the antithesis of real-time pricing programs. "These
customers have zero ability to connect their usage to real-time pricing,"
says Guthridge. \cite{Stone09}.

\subsection{Hawaii}

Hawaii has unique attributes that make it interesting for smart grid
research: it has a wealth of renewable energy sources including solar,
wind, wave, and geothermal.  In addition, it is currently 90\% dependent on
fossil fuels, and these fuels are very expensive, making our electrical
costs the most expensive in the nation.  As a result of these factors,
renewable energy has the potential to be more cost-effective in Hawaii.
Also, the grid is closed, making it easier to model and analyze.



The central question to be investigated in this
research is the following: {\em What kinds of information, provided in what ways and at what
times, enables consumers to make positive, sustained changes to their
energy consumption behaviors?}

\subsection{Prius}

Metaphor of prius:  we can have top-down control (via minimum auto mpg
requirements), but the prius realized that the potential of its technology
could only be achieved by providing drivers with higher quality, higher
fidelity information about the ways the driver behaviors impact on mpg.

\subsection {BECC notes}

From Moezzi: the vexed relationship between increased device efficiency and
absolute levels of consumption (Herring 1998; Levett 1998; Lutzenhiser
1993, 2002b; Moezzi & Diamond 2005; Sanne 2002; Shove 2003a, 2003b; Shui &
Dowlatabadi 2005; Wilhite 2007; Wilhite et al. 2000).  These protests do
not offer a replacement for the device-centered energy.

Demand-response and load-shape issues are especially interesting from a
behavioral point-of-view, but are not the main focus of energy efficiency
potential studies.

The McKinsey study found that U.S. energy productivity defined as the ratio
of real gross domestic product (GDP) to unit of energy was the lowest
energy productivity among developed countries.  (McKinsey Global Institute
2007a, 2007b).

These terms follow the energy efficiency field's dominant theoretical model
of how energy is used and saved: energy consumption is modeled as being
determined by physical characteristics and humans, as economic agents,
making energy-relevant purchases based on cost-effectiveness in order to
fulfill service needs, which are defined exogenously. Lutzenhiser (1993,
2009) refers to this treatment broadly as the physical-technical-economic
model of energy consumption and energy consumption change, or PTEM. Social
scientists have long pointed out how much real life deviates from the PTEM
(e.g., Ehrhardt-Martinez et al. 2008; Lutzenhiser 1993, 2009; Shove 1998,
2003b). Nearly everybody, not just social scientists, agrees that there are
deviations, but the questions are:

Table 2: The Variety of Realms Influencing Energy Use shows a
categorization of the kinds of behaviors that consumers could enact in
order to affect their energy usage.  This could become an instrument for
use in assessing the impact of the user interface. 


\subsection{Expected Contributions}

The expected contributions of this research include: (a) outcome data
regarding consumer behavior and technology that can be used to inform
policy and design decisions; (b) technological infrastructure that can
support future experimentation; (c) replicable experimental methods for
investigating consumer behavior with the smart grid; (d) a
cross-disciplinary community of scientific researchers; 


\subsection{Goals of NSF program}

We also want to demonstrate alignment with the goals of the program.

``HCC research explores and improves our understanding of new
human-computer and human-human interactions, collaboration, andcompetition, developing systems that are aware of their social surroundings
and of the conceptualizations, values, preferences, abilities, special
needs, and diverse ranges of capability of the people that use them.''


are there unintended consequences;




\section{Related work}

Microsoft's SERA (Smart Energy Reference Architecture) limits its
discussion of user-centered control to a single line: ``Customers will also
need to have capabilities for more local preferences and control.''
(p. 69)

Two technologies are being proposed for home area networks (HAN), including ZigBee and IP for Smart Objects (IPSO).
ZigBee allows for secure wireless communication between devices, where specific devices may provide specific capabilities useful for energy control and conservation.
IPSO is focused on the use of IP for connections and communication between Smart Objects.


Microsoft Hohm is a consumer-facing site.  Users fill out a profile of
their home, and if the utility is connected, they can get their energy data
automatically imported.  The site responds with recommendations on how you
can conserve energy. Does comparisons of your use with others using Hohm
and
published averages.   Goals: make it easy to enter information, and make
the recommendations very specific.  So, the recommendation to raise your
thermostat setting by 2 degrees will indicate exactly how much money you
will save and the carbon impact.


\section{Proposed contributions}

Data on consumer behaviors; open source technologies; experimental designs;
community building (?).


 
\subsubsection{Smart grid behavioral assumptions}

The design of Smart Grid technologies tends to reflect one of the following
three views of energy consumers:

{\em (1) Consumers are too lazy or uninterested to change their 
behaviors based upon energy information.}

This view is reflected in smart meter designs that communicate only with
utilities and do not provide data back to consumers about their own
consumption. It is also reflected in the design of demand response systems
that gives utilities control over energy consumption as well as
generation, by enabling utilities to turn off water heaters or change
thermostat settings in homes and offices. 

There are at least three problems with this view. First, utilities can only
control consumption to a certain degree through demand response, so the
impact of such systems is limited.  Second, there is evidence that people
do not wish utilities to be able to unilaterally make such decisions.
Third, some worry about the privacy implications of allowing the utilities
to gain access to household energy consumption; for example, that data
would be gathered that the utility could then sell for the purpose of
targeted advertising.

{\em (2) Consumers respond only to cost.}

This view is reflected in dynamic pricing mechanisms, where the consumer is
charged different rates at different times of day depending upon the cost
to the utility of providing energy at that time.

Cost is, indeed, a powerful incentive for behavioral change, but dynamic
pricing schemes are intended to support ``peak shaving'', or curtailment in
the maximum amount of power the utility must be prepared to provide to the
community.

{\em (3) Consumers are motivated by ecological considerations.}

There are a growing number of sites which help consumers to understand the
ecological impact of their energy-related behavior.  For example, the
Ecotricity site has a ``UK Grid Live'' page which displays a stoplight in
one of three states: red, yellow, or green, depending upon the carbon
intensity of the grid.  

In our research, we take the view that any or all of these situations could
be the case, even within a single individual over time. Our goal is to
learn about the factors that lead to one or more of these decisions. We
will develop technology that enables us to explore an integrated view.









