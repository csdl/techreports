%%%%%%%%%%%%%%%%%%%%%%%%%%%%%% -*- Mode: Latex -*- %%%%%%%%%%%%%%%%%%%%%%%%%%%%
%% project-techreport.tex -- 
%% Author          : Philip Johnson
%% Created On      : Tue Mar 31 11:44:58 2009
%% Last Modified By: Philip Johnson
%% Last Modified On: Sun Nov 22 11:24:35 2009
%% RCS: $Id$
%%%%%%%%%%%%%%%%%%%%%%%%%%%%%%%%%%%%%%%%%%%%%%%%%%%%%%%%%%%%%%%%%%%%%%%%%%%%%%%
%%   Copyright (C) 2009 
%%%%%%%%%%%%%%%%%%%%%%%%%%%%%%%%%%%%%%%%%%%%%%%%%%%%%%%%%%%%%%%%%%%%%%%%%%%%%%%
%% 

\section{Introduction}

Development of the ``smart grid'', a modernized power infrastructure, is
one of the key technological challenges facing the United States at the
dawn of the 21st century. According to the Department of Energy, the smart
grid should: (1) Enable active participation by consumers by providing
choices and incentives to modify electricity purchasing patterns and
behavior; (2) Accommodate all generation and storage options, including
wind and solar power.  (3) Enable new products, services, and markets
through a flexible market providing cost-benefit tradeoffs to consumers and
market participants; (4) Provide reliable power that is relatively
interruption-free; (5) Optimize asset utilization and maximize operational
efficiency; (6) Provide the ability to self-heal by anticipating and
responding to system disturbances; (7) Resist attacks on physical
infrastructure by natural disasters and attacks on cyber-structure by
malware and hackers \cite{NETL:GridCharacteristics}.

In October, 2009, the government awarded approximately \$3.4 billion in federal stimulus
money to approximately 100 organizations in 49 states to
support smart grid development.  These awards, matched by \$4.7
billion in private funds, will be used primarily for the
installation of smart meters, an initial enabling technology for the smart grid.
Other uses of the stimulus money include installation of a secure
communications network (Wyoming); installation of phasor measurement units
for monitoring grid stability (New York, Massachussetts, Indiana,
Louisiana, and others); and introduction of dynamic pricing (Maryland,
Arizona, California, and others).

While substantial, this investment is almost totally focused on low-level
infrastructure.  By analogy to the Internet, it is similar to upgrading a
copper wire network to fiber optic cable, along with installation of high
performance routers and name servers.  Such infrastructure is a necessary
requirement for high-level Internet services such as the World Wide Web,
but does relatively little to determine the nature of those services.

The smart grid will produce and consume information about electricity as
well as electricity itself, thus forming a specialized kind of Internet. At
the current time, it is not clear what ecosystem of higher-level services
should be developed to best communicate, analyze and interpret this
low-level electrical data to consumers.  (By ``consumers'', we mean all of
the various customers of utilities: both individuals and businesses
alike. )

One reason why there is little clarity about consumer-facing information
services in the Smart Grid is the nature of the current grid, which (from a
consumer point of view) is a classic ``black box'' technology.  For almost
100 years, utility customers have plugged appliances into the grid via
electrical outlets and expected them to ``just work''.  Put another way,
the U.S. power industry has operated for a century under the assumption
that consumers should have reliable access to a virtually unlimited amount
of high quality power.  Traditionally, consumers have been given extremely
little information about power beyond a monthly bill because neither
consumers nor utilities believed further information was necessary or
appropriate.

Electricity as black box requires electricity to be cheap, reliable, and
unlimited, but those days appear to be drawing to a close for a number of
reasons.  First, the traditional use of fossil fuels as a primary source of
power generation is not sustainable: most economists agree that ``peak
oil'' has either already occurred or will occur in the next two decades. At
that point, oil supply will decrease and prices will increase.  Second,
fossil fuels generate green house gases that contribute to climate change,
and so there is an urgent need to move to renewable energy sources such as
solar, wind, wave, and geothermal.  Third, reliance on fossil fuels creates
a variety of political problems.

Integration of renewable energy sources, unfortunately, creates significant
new problems related to storage and grid stability.  Unlike fossil-fuel
based power generation (called ``firm'' since utility companies can
generate new power from fossil fuels at will), most renewable energy
sources generate energy depending upon generally unpredictable
environmental factors.  As a result, simply adding a solar panel to every
rooftop and tying them into the grid would actually do more harm than good
at present.  This is because grid stability depends upon energy supply
equaling demand on a second to second basis, and utilities generally do not
have a way to monitor and balance a grid that incorporates high levels of
widespread, distributed generation. Thus, renewable energy sources are not
a magic bullet that can replace fossil fuels while enabling the US
electrical infrastructure to remain a black box to consumers.

Even if electricity could remain a black box to consumers, there is
compelling evidence suggesting that it should not.  According to the July 2009
report ``Unlocking energy efficiency in the U.S. Economy'' \cite{Granade09}, there is the
potential to reduce annual non-transportation energy consumption by roughly
23 percent by 2020, eliminating more than \$1.2 trillion in waste.  Such a
reduction in energy use would result in abatement of 1.1 gigatons of
greenhouse gas emissions annually. ``Unlocking'' this untapped potential
requires, in part, improved access by consumers to energy information,
along with changes in behavior based on this information that produce the
desired efficiencies.

If electricity cannot and should not remain a black box to consumers, then a
central question is: what kind of ``white box'' should it become? This is,
in essence, the question to be investigated in this proposal.  More specifically, {\em
  What kinds of information, provided in what ways and at what times,
  enables consumers to make positive, sustained changes to their energy
  consumption behaviors?}

In this research, we propose to develop technologies and experimental
methods that will enable us to gather scientific, replicable data about the
ways in which access to information about energy usage impacts on consumer
behavior.  Our technologies will build upon current energy user interface
successes and failures; leverage the emerging smart grid standards;
incorporate open source development techniques; and respect findings from
behavioral research. We will develop two experiments based upon our
technologies with the intent that they be easily replicated:: the first
based upon a university campus dorm competition and the second based upon a
community energy challenge.

The anticipated contributions of this research will include: (1) outcome
data from our two case studies that provide new insight into requirements
for Smart Grid information services; (2) replicable methods for inquiry
into Smart Grid technologies and their impact on consumers; and (3) open
source technologies to facilitate scientific experimentation on the Smart
Grid.

The remainder of this proposal is organized as follows.  Section
\ref{sec:related-work} overviews research and technologies that serve as a
basis for our approach.  Section \ref{sec:methodology} presents our proposed
experimental approach.  Section \ref{sec:merit} discusses the intellectual
merit and broader impact of this research. 



\section{Related work}
\label{sec:related-work}

\subsection{Energy efficiency and behavior}

\subsubsection{Smart grid behavioral assumptions}

The design of Smart Grid technologies tends to reflect one of the following
three views of energy consumers:

{\em (1) Consumers are too lazy or uninterested to change their 
behaviors based upon energy information.}

This view is reflected in smart meter designs that communicate only with
utilities and do not provide data back to consumers about their own
consumption. It is also reflected in the design of demand response systems
that gives utilities control over energy consumption as well as
generation, by enabling utilities to turn off water heaters or change
thermostat settings in homes and offices. 

There are at least three problems with this view. First, utilities can only
control consumption to a certain degree through demand response, so the
impact of such systems is limited.  Second, there is evidence that people
do not wish utilities to be able to unilaterally make such decisions.
Third, some worry about the privacy implications of allowing the utilities
to gain access to household energy consumption; for example, that data
would be gathered that the utility could then sell for the purpose of
targeted advertising.

{\em (2) Consumers respond only to cost.}

This view is reflected in dynamic pricing mechanisms, where the consumer is
charged different rates at different times of day depending upon the cost
to the utility of providing energy at that time.

Cost is, indeed, a powerful incentive for behavioral change, but dynamic
pricing schemes are intended to support ``peak shaving'', or curtailment in
the maximum amount of power the utility must be prepared to provide to the
community.

{\em (3) Consumers are motivated by ecological considerations.}

There are a growing number of sites which help consumers to understand the
ecological impact of their energy-related behavior.  For example, the
Ecotricity site has a ``UK Grid Live'' page which displays a stoplight in
one of three states: red, yellow, or green, depending upon the carbon
intensity of the grid.  

In our research, we take the view that any or all of these situations could
be the case, even within a single individual over time. Our goal is to
learn about the factors that lead to one or more of these decisions. We
will develop technology that enables us to explore an integrated view.


\subsection{Energy data interfaces}

\subsection{Energy data repositories}

\subsection{Energy data design issues}

\section{Methodology}
\label{sec:methodology}

\section{Intellectual merit and broader impacts}
\label{sec:merit}








 










