% CHI Extended Abstracts template.
% Tested with XeTeX on Mac OS X (Get it from http://tug.org/mactex)
% The latest version is available at <http://manas.tungare.name/software/latex/>
% 
% Filename: chi-ext.cls
% 
% CHANGELOG:
%   2010-08-09   Manas Tungare      Updated copyright info for CHI 2011
%   2009-12-04   Stephen Voida      Updated copyright info for CHI 2010
%   2009-11-17   Manas Tungare      Refactored the title & author sections.
%   2008-11-25   Manas Tungare      Initial create.
% 
% LICENSE:
%   Public domain: You are free to do whatever you want with this template.
%   If you improve this in any way, please drop me a note <manas@tungare.name>,
%   so I can share the updates with everyone.
%   
%   PLEASE RECONSIDER BEFORE FORKING THIS TEMPLATE; there are already
%   several versions of the chiproceedings template for no good reason.
%   DO NOT REDISTRIBUTE THIS FILE UNDER A DIFFERENT FILENAME unless you
%   have a very good reason to change its name.


%%%%%%%%%%%%%%%%%%%%%%%%%%%%%%%%%%%%%%%%%%%%%%%%%%%%%%%%%%%%%%%%%%%%%%%%%%%%
%%%% Ended up not using this LaTeX document, switched to Word template  %%%%
%%%% for better fidelity to doctoral consortium template.               %%%%
%%%%%%%%%%%%%%%%%%%%%%%%%%%%%%%%%%%%%%%%%%%%%%%%%%%%%%%%%%%%%%%%%%%%%%%%%%%%


\documentclass{chi-ext}

\title{The Kukui Cup: Fostering Sustained Energy Behavior Change and Energy Literacy In A Dorm Energy Competition}

\author{
  \textbf{Robert S. Brewer}\\
  Collaborative Software Development Laboratory\\
  Department of Information and Computer Sciences\\
  University of Hawai`i at M\=anoa\\
  Honolulu, HI 96822\\
  rbrewer@lava.net\\
}

\keywords{Sustainability, Energy, Behavior Change}

% See http://www.acm.org/class/how_to_use.html for help using the ACM
% Classification system.
\acmclassification{K.4.m Computers and Society: Miscellaneous, H.5.2 Information Interfaces and Presentation: User interfaces – Evaluation/methodology}

\copyrightinfo{
  Copyright is held by the author/owner(s). \\
  \emph{CHI 2011}, May 7--12, 2011, Vancouver, BC, Canada. \\
  ACM  978-1-4503-0268-5/11/05. \\
}

% Repeat author names (minus affiliations and addresses) and title here 
% for PDF metadata.
\hypersetup{
  pdfauthor={Robert S. Brewer},
  pdfkeywords={Sustainability, Energy, Behavior Change},
  pdfsubject={Computer Human Interaction},
  pdftitle={The Kukui Cup: Fostering Sustained Energy Behavior Change and Energy Literacy In A Dorm Energy Competition},
}

\begin{document}
\maketitle

\begin{multicols}{2}
  
\makeauthors

\section{Abstract}

Energy conservation through behavior change will play a critical role in addressing the world's energy crisis. My research seeks to investigate the relationships among energy literacy, sustained energy conservation, and information technology support of behavior change through an advanced dorm energy competition to take place in Spring 2011. The energy use of each pair of dormitory floors will be metered in near-realtime, and a competition website will make participants aware of their energy usage and provide them with educational tasks designed to increase their energy literacy. The participants interaction with the website will be logged and their energy literacy will be assessed both before and after the competition.

\section{Keywords}
\makeatletter \@keywords \makeatother

\makecopyright

\section{ACM Classification Keywords}
\makeatletter \@acmclassification \makeatother

%------------------------------------------------------------------------

\section{General Terms}

Human Factors, Experimentation, Measurement

\section{Introduction}

% Almost every research document begins with a section that frames the research and motivates the problem being studied. It describes some domain, indicates a problem in general terms, and explains why the problem is worth solving. Questions a CHI reader should be able to answer after reading the motivation section are:
% * What is the general area being addressed?
% * Is this relevant to CHI?
% * What is the motivation for studying a particular problem?
% * What makes it worth the effort?
% * Is it a 'real' problem in everyday life, and/or is it a 'theoretical' problem that is worth solving?
% * Would anyone care if I solved this?

The world is in the grip of an energy crisis. Fossil fuels (oil, natural gas, and coal) form the foundation of the world economy. However, the consumption of fossil fuels has led to a variety of problems that will have severe impacts on our environment, and one of the most serious impacts is climate change. In 2007, the Intergovernmental Panel on Climate Change (IPCC) released its fourth assessment report \cite{IPCC-synthesis-report-2007}. They found that there is broad agreement that the climate is warming: air and ocean temperatures are higher, snow and ice are melting, and sea levels are rising.

One way fossil fuel use can be decreased is by decreasing the total amount of energy consumed, and changing people's behavior with respect to energy holds significant promise in reducing energy use. Darby's survey of energy consumption research found that identical homes could differ in energy use by a factor of two or more \cite{darby-review-2006}.

Changing people's behavior is difficult, and to achieve meaningful energy conservation, behavior changes must be sustained. Two strategies that have proven to be effective are providing direct feedback on energy usage \cite{darby-review-2006}, and a toolbox of techniques such as making public commitments and establishing social norms \cite{McKenzie-Mohr2009}. My research evaluates these techniques in the context of a dorm energy competition, where each floor of the competing dorms attempts to use the least energy during a 3 week period. A competition website will provide near-realtime energy usage data, standings, and a variety of educational tasks that participants can perform to increase their \emph{energy literacy}. The design of the competition and website provides fertile ground for exploring non-traditional methods for both energy conservation and education, since dorm residents do not have the traditional incentives in those areas: financial savings, and grades/course credit respectively.

\section{Background and Related Work}

% Provide a miniature literature review to give the reader enough background to (a) gain sufficient knowledge about what others have done, (b) know how your work will build upon this prior work, and (c) be assured that you have sufficient knowledge of the relevant literature. You should highlight only the key literature here; a full review is not required. Questions a CHI reader should be able to answer after reading this section are:
%
% * Did the author provide enough background to help me know what others have done in this area, as well as what discipline(s) have considered this area?
% * Does the author have sufficient knowledge of the relevant literature necessary to do the proposed work?
% * How does the author’s proposed work fit within and extend what has been done before?

Energy competitions on college campuses have taken place on many campuses over the last several years. Petersen et al.\ described their experiences deploying a realtime feedback system in an Oberlin College dorm energy competition in 2005 \cite{petersen-dorm-energy-reduction}. They found a 32\% reduction in electricity use across all dormitories, with freshmen dorms and those dorms receiving real-time feedback reducing usage the most. Participants reported making unsustainable changes such as keeping hallway lights off during the competition. Overall, there has been little analysis on energy usage after competitions finish, or how positive behavior changes could be sustained.

To reduce their energy use, people must know how much energy they are using. Darby provides a detailed survey of studies on electricity feedback systems from the past 3 decades \cite{darby-review-2006}. The survey of 20 studies found that, on average, the introduction of a direct (real-time) feedback system leads to reductions of energy usage ranging from 5-15\%. Another survey of energy feedback focused on utility pilot programs conducted by Faruqui et al., found that customers that actively used the display averaged a 7\% reduction in energy usage \cite{Faruqui09}. A variety of products are available for recording and visualizing energy use, ranging from large buildings (Building Dashboard, GreenTouchScreen), to single residences (The Energy Detective, Google PowerMeter).

StepGreen is a social web application designed to encourage people to undertake environmentally responsible actions. Mankoff et al.\ have written about the rationale for the system and description of the design  \cite{Mankoff2007Leveraging-Soci}. StepGreen is designed to leverage online social networks to motivate personal change, by providing suggestions for improvement. Users pick positive environmental actions that they have performed from a list of options. Based on users' self-reports, StepGreen calculates the amount of money saved, and pounds of CO$_2$ saved. Grevet et al.\ studied social visualizations in StepGreen with a dorm competition at Wellesley College, and found that the list of actions was not well suited to their lifestyle \cite{Grevet10}.

A variety of methods have been employed in an attempt to get people to change their behavior to be environmentally sustainable; McKenzie-Mohr provides a good summary of the area in his online book \cite{McKenzie-Mohr2009}. Simply providing information about sustainable behavior tends to not lead to behavior change. Techniques that have been shown to work are obtaining public commitments, setting goals, and influencing social norms \cite{Pallak80, Becker78, Schultz2007SocialNorms}. 

\emph{Energy literacy} is the understanding of energy concepts as they relate both on the individual level and on the national/global level. Solving the world energy crisis will require everyone to understand how energy is generated and consumed, so that they can make more informed choices in their lives and as informed citizens involved in their communities. DeWaters and Powers of Clarkson University have been working on an energy literacy survey instrument for middle and high school students \cite{DeWaters09}. They define energy literacy as consisting of three components: knowledge, attitudes, and behaviors.

\section{Problem Statement}

% Provide a very concise statement of your thesis or problem statement. This should be the highest-level problem or goal you plan to address and is sometimes posed as a hypothesis, proposition or conjecture. This is often followed by a small list of specific problems and sub-problems that need to be solved if you are going to satisfy your hypothesis or thesis. Problems should be stated unambiguously. The importance of the problem should be mentioned if it hasn't already been done so in the prior sections. Of course, the problem must be worthy of a PhD thesis. Questions a CHI reader should be able to answer after reading this section are:
% * Did the author succinctly identify the thesis, problem or set of problems being addressed?
% * Is this problem worthy of a CHI PhD thesis?

My research seeks to determine what information presented in what ways can foster sustainable changes in behavior that lead to reduced energy usage. Addressing this problem requires the ability to track energy usage, determine what behavior is taking place, and how energy use and behavior change over a sustained period of time.

\section{Competition}

To gain insight into my research problem, I will conduct an experiment as part of a dorm energy competition over 3 weeks in the Spring 2011 semester on the UH M\=anoa campus. The competition currently targets 3 dorms occupied by approximately 780 freshmen. To track energy usage during the competition, power meters will be installed on each pair of floors in each building and the power and energy data will be recorded every 10 to 15 seconds. Since each floor has its own meter, each floor will compete to have the lowest energy consumption during the competition. I have developed a system, WattDepot, to provide the means for collecting, storing, and visualizing the energy data \cite{csdl2-10-05}.

I have designed a website that will provide information about the competition to the participants. Participants will see a personalized home page that displays data such as his or her floor's power usage in near-realtime, and their floor's ranking in the competition. The website has been designed to take into account the research in environmental psychology about how to foster behavior change.

The other major feature of the competition website is to make a variety of educational tasks available to the participants. The tasks are organized in clusters around particular topics such as energy, climate change, and transportation. The tasks consist of commitments, activities, and events.

Associated with each task is a number of points. When a participant performs a task, such as determining the amount of power each device in their room consumes, they can submit information on the website demonstrating their completion of the task. In the case of the power audit, the information might be the list of devices in their room and the power consumption of each device. Once a website administrator verifies the information, the participant is awarded the points assigned to the completed task. These website tasks create a second parallel competition to see which participants can accumulate the most points.

A variety of prizes will be provided both for the energy conservation side of the competition, and the point competition. This prize structure provides an additional motivation for the residents to participate in the competition.

\section{Evaluation}

% While the previous section details the problem you are addressing, your job here is to translate this into research goals and corresponding methods. Each goal should briefly indicate how you are going to solve the problem, i.e., the research method(s) you will use. Goals should be operational; i.e., if you later claim to achieve your goal, you should be able to match your solution against the goal statement. Then describe what contributions you expect to make if you satisfy these goals. Note: some authors may prefer to combine problem statements, goals, methods  and contributions into a single section.
%
% We cannot overstate how important it is to have clear goals. When problems, goals, methods and contributions are not clearly stated, readers will be unable to evaluate your solutions. Questions a CHI reader should be able to answer after reading this section are:
% * What are the specific goals being pursued? 
% * Do these goals actually help solve some or all of the stated problem(s)?
% * Has the author stated how s/he will achieve this goal (i.e., the method)?
% * Are the goals actionable, i.e., will we know when a goal is actually attained?

There are four primary sources of data available to examine the research questions: power and energy data from each floor, detailed event logs from participant actions on the website, participants' performance on an energy literacy survey to be administered before and after the competition, and a survey on the competition as a whole to be administered after the competition.

This rich dataset allows the examination of several relationships. The energy data alone provide insight into what effect (if any) the competition had on participants' energy usage, particularly to what degree energy use rebounds after the competition is over. The combination of the website log data and the pre- and post-competition energy literacy scores sheds light on whether the tasks available on the website led to increased participant energy literacy, and if so, which tasks were most effective. Finally, the combination of the energy usage data and the energy literacy scores allows look at the hypothesis that those floors that were more energy literate conserved more energy, both during and after the competition.

\section{Expected Contributions}

% Use this section to connect your research approach back to the problem statement. This should be a short section that conveys what you anticipate as results or outcomes from your dissertation project and how it will contribute to the HCI research community.

The anticipated contributions of this research are: insight into the effectiveness of psychological techniques for fostering behavior change as embodied in a website, understanding of the effectiveness of educational tasks presented through the website in increasing energy literacy, and the utility of energy visualizations designed to foster energy conservation.

\section{Research Status}

% Applicants to the Doctoral Consortium should have begun their research, but should not have completed it in its entirety. You should briefly state where you currently are in your university’s PhD program. We understand that different universities may organize their programs quite differently, so feel free to give some background if this will help you to be clear. Remember that we are seeking candidates who have an approved dissertation research topic and are carrying out their work, but who have enough work ahead of them that they can benefit from the exchanges and discussions that will take place at the Consortium. Some points you may want to include are:
% * What kind of academic program you are in
% * How you primarily identify yourself (e.g., computer scientist, ethnographer, social scientist, etc.)
% * How many years you have been in that program, and how many years you anticipate you have left before graduating 
% * Whether you have completed your candidacy / research proposal stage (if required at your university)
% * A brief summary of the state of your research (e.g., what you have done vs. what you  have left to do)
% * What you hope to gain from attending the Doctoral Consortium

% Clearly state what you have done and what you have left to do. Summarize the most important findings thus far, and make it clear how these findings match and inform your original problems and goals. Include a short argument as to why these findings are important. Include references and brief descriptions to key publications (if any) arising from your thesis work. State how much of your actual thesis document is written, and what form it is in (e.g., outline, rough draft, etc.)
%
% In addition to describing your current status (as of submission), please also include a paragraph about your future plans, for example the research activities remaining, and how much time you expect these to take, and what sorts of assistance you hope to obtain through your participation in the Consortium.

I am currently a PhD candidate in the Computer Science program. In our program, this means that I have a MS, have finished all coursework, passed a written qualifying examination, assembled a dissertation committee, and defended my dissertation proposal in a public oral examination before my committee. I have been in the PhD program for 3 years.

At this time many parts of the competition have been planned including: the tentative schedule of events, an initial set of clustered educational tasks, draft pre and post-competition energy literacy questionnaires. Much of the competition website design is complete, but implementation is ongoing (the implementation is being performed by a fellow graduate student as part of his MS thesis). The power meters have been ordered, but not received or installed. The experiment is currently expected to take place in February 2011, but there are external factors (meter installation) that could delay the experiment until later in the Spring, or possibly even into Fall 2011.

My dissertation is currently in outline form, though significant portions (such as parts of the literature review) will be taken from my proposal. The research has generated two publications to date: one describing the WattDepot energy data system \cite{csdl2-10-05}, and another describing the design of the competition and website \cite{csdl2-10-07}.

If my experiment proceeds according to plan in February 2011, I hope to finish my dissertation sometime in the Summer of 2011. I fully realize that this puts the doctoral consortium closer to the completion of my dissertation than might be the case for other participants, and my experiment will already have taken place by the time of the conference. However, I will be gathering a large and diverse set of data from the experiment, and I feel the consortium experience will provide new perspectives on the data and on additional analyses that would be beneficial. My advisor has funding to run additional versions of the experiment in the Fall of 2011 and 2012, so feedback from the consortium will also help us to improve the future experimental iterations.

\bibliography{csdl-trs,smartconsumer,sustainability}
\bibliographystyle{abbrv}

\end{multicols}

\end{document}