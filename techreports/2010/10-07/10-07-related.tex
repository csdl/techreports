%%%%%%%%%%%%%%%%%%%%%%%%%%%%%% -*- Mode: Latex -*- %%%%%%%%%%%%%%%%%%%%%%%%%%%%
%% 10-07-related.tex --  HICSS 44 Kukui Cup paper
%% Author          : Philip Johnson
%% Created On      : Mon Sep 23 11:52:28 2002
%% Last Modified By: Philip Johnson
%% Last Modified On: Thu Jun 10 15:04:49 2010
%%%%%%%%%%%%%%%%%%%%%%%%%%%%%%%%%%%%%%%%%%%%%%%%%%%%%%%%%%%%%%%%%%%%%%%%%%%%%%%
%%   Copyright (C) 2009 Philip Johnson
%%%%%%%%%%%%%%%%%%%%%%%%%%%%%%%%%%%%%%%%%%%%%%%%%%%%%%%%%%%%%%%%%%%%%%%%%%%%%%%
%% 

\section{Related Work}
\label{sec:related-work}

Our research draws on work from multiple areas. First, we discuss other dorm energy competitions, then we cover energy feedback research. Next, we examine related technological systems, and relate our work to psychological work on behavior change. Finally, we examine the concept of energy literacy.

\subsection{Dormitory energy competitions}

Energy competitions on college campuses involve residence halls competing to see which building can use the least energy over a period of time. The competitions tap into both the residents competitive urges, and their interest in environmental issues. However, unlike a home environment, the residents do not financially benefit from any reduction in electricity use resulting from their behavior changes, since residence hall fees are flat-rate and do not change based on energy usage. This leads to residents being completely unaware of their energy usage, since they lack even a monthly bill as feedback. Dillahunt et al.\ describe similar a similar situation with their investigation of energy usage in low income communities, where individuals may not be billed directly for electricity and may not have the means to upgrade appliances \cite{Dillahunt2009-low-income}. Despite these differences, Dillahunt et al.\ found that the residents of low-income housing were still motivated to save energy and came up with diverse energy-saving solutions, which may suggest that dorm residents can be similarly motivated.

The most basic type of energy competition website displays energy data which is updated manually on a periodic basis (such as weekly). The Wellesley College Green Cup \cite{wellesley-green-cup} is an example of this type of competition. 

Other schools have more complicated and interactive competition websites, such as the early adopter Oberlin College. Petersen et al.\ describe their experiences deploying a realtime feedback system in an Oberlin College dorm energy competition in 2005 \cite{petersen-dorm-energy-reduction}. 22 dormitories were in competition over a 2 week period, with 2 dorms having feedback updates every 20 seconds, and the other 20 getting updates every week. The realtime dorms also recorded electricity usage for each of the three floors, but only displayed the data from two of the floors, leaving the third as a control. Web pages were used to provide feedback to students, since they all had computers and Internet access in their rooms. They found a 32\% reduction in electricity use across all dormitories, with the 2 realtime feedback dorms reducing usage the most. Freshman dorms were among the highest electricity reducers, while upperclassman dormitory reductions were quite low (average 2\% reduction). During a 2 week post-competition period, the average electricity usage was similar to consumption levels during the competition. However, the weather was warmer and there was more sunlight during the post-competition period, so it is unclear if the reduction was competition-related. In a post-competition survey, respondents indicated that some behaviors, such as turning off hallway lights at night and unplugging vending machines were not sustainable outside the competition period.

While dorm energy competitions are being conducted with regularity, often the emphasis appears to be on the event and not on research on the effects of the competition. In particular, there has been little analysis on energy usage after the competition is over, or how positive behavior changes could be sustained.

\subsection{Energy feedback}

As Lord Kelvin is famously reputed to have said, ``If you can not measure it, you can not improve it.'' In the case of electricity usage, for many people the only feedback they receive is a monthly bill detailing the number of kilowatt-hours used over the course of the last month. Ed Lu of Google analogizes this as if there were no prices on anything at the grocery store, and shoppers were just billed at the end of the month \cite{Helft2008Googles-Energy}. Office workers or dormitory residents might never see any feedback on how much electricity they are using!

To reduce energy use, people must know how much energy they are using. Feedback systems display the consumption of a resource (such as electricity) to the user, usually in real time. Darby provides a detailed survey of studies on electricity feedback systems from the past 3 decades \cite{darby-review-2006}. The survey of 20 studies finds that, on average, the introduction of a direct (real-time) feedback system leads to reductions of energy usage ranging from 5-15\%. Feedback systems providing historical data (such as those provided with billing statements) are not as effective (0-10\% reductions), but can be useful for assessing the impact of efficiency measures such as improved insulation or a more energy efficient appliance, since those savings accumulate over time.

Darby found that ``consumption in identical homes, even those designed to be low-energy dwellings, can easily differ by a factor of two or more depending on the behaviour of the inhabitants.'' This finding demonstrates the significant potential to curb energy usage through changes in individual's behavior.

Another survey of energy feedback was conducted Faruqui et al., looking at 12 utility pilot programs that installed in-home displays with near-realtime feedback \cite{Faruqui09}. They found that customers that actively used the display averaged a 7\% reduction in energy usage, while those pilot programs that included pre-paid electrical services reduced their energy usage by 14\%. The sustainability of the energy reduction is unclear based on the pilot studies since they were of limited length. The authors believe it is unknown whether the residents of homes with displays will acclimate to the display and cease to use it to reduce their energy usage.

Providing energy feedback is a critical foundation for any attempt to reduce energy consumption, and the feedback itself will likely curb energy usage somewhat. However, Darby points out that while feedback is critical for energy conservation behaviors, feedback alone is not always enough \cite{darby-2000-making-it-obvious}. Other factors that lead to higher rates of energy conservation include contact with an advisor when needed, training and social infrastructure.

\subsection{Related systems}

In this section we examine other systems designed to help users make environmentally-positive behavior changes.

StepGreen is a social web application designed to encourage people to undertake environmentally responsible actions \cite{step-green-website}. Mankoff et al.\ have written about the rationale for the system and description of the design  \cite{Mankoff2007Leveraging-Soci}. StepGreen is designed to leverage online social networks to motivate personal change, by providing suggestions for improvement. Users create an account on StepGreen, and then are presented with a list of actions with positive environmental consequences such as ``Turn off the lights when you exit the house in the morning for the day''. Each action is associated with its cost savings and reduction in greenhouse gas emissions. For each action, users can indicate whether they are already performing that action, whether they commit to undertaking that action, or whether the action is not applicable to them. For recurring actions, users must indicate how many times they have performed the action since their last report in order for the system to track the activities. Based on the user's self-reporting, StepGreen calculates the amount of money saved, pounds of CO$_2$ saved (i.e., reduced), and missed pounds of CO$_2$ saved, and provides a historical graph of these values.

In its current state, it is challenging for StepGreen users to keep up to date due to the reliance on manual data input. Grevet et al.\ studied social visualizations in StepGreen with a dorm competition at Wellesley College, and found that the list of actions was not well suited to their lifestyle \cite{Grevet10}.

The Building Dashboard \cite{building-dashboard} and Green TouchScreen \cite{greentouchscreen} are systems that aim to make building occupants aware of the overall environmental impact of their building. While these systems are feature rich, they are relatively expensive and closed commercial systems, making integration with other software difficult.

At the scale of a single residence, there are systems like the TED 5000 \cite{the-energy-detective} that provide both the metering hardware and closely-integrated software for storing and displaying energy data. These systems provide a good solution for a single residence, but are not designed for wider-scale data collection.

Pachube is a hosted service that provides a rich API for storing, retrieving, and visualizing sensor data in an effort to build the `internet of things' \cite{pachube-site}. Pachube has created an active community around a concept similar to our WattDepot server (see Section \ref{sec:wattdepot-server}), with a variety of software libraries, input tools, and applications. However, Pachube is a commercial hosted service that enforces limits on the rate at which data can be stored and retrieved (50 requests every 3 minutes as of this writing), which may be less than some applications require. In addition, WattDepot is focused on energy data, so the WattDepot REST API provides energy-specific methods that make it easier for clients to focus on their application domain.

Google PowerMeter is a web application developed to make smart meter data available to the end users living in smart metered homes \cite{Google-PowerMeter}. Google partners with utilities that have rolled out smart meters, and collects the power data from the utility. PowerMeter also works with the TED 5000 home energy meter that can be installed by end-users without interaction with the utility. The data is recorded at 15 minute intervals, and presented in a variety of graphs that show daily usage and home base load levels. The primary interface for PowerMeter is a web gadget that is installed on the user's iGoogle home page. PowerMeter allows users to share their data with others, and has added an API to allow users to get access to their raw data. Google PowerMeter focuses on single-family homes, and the energy visualization design reflects that focus. Home-oriented visualizations and the lack of real-time energy data make PowerMeter inappropriate for a dormitory deployment at this time.

\subsection{Fostering sustainable behavior}

A variety of methods have been employed in an attempt to get people to change their behavior to be environmentally sustainable; McKenzie-Mohr provides a good summary of the area in his online book \cite{McKenzie-Mohr2009}. Simply providing information about sustainable behavior tends to not lead to behavior change. For example, Geller performed an investigation of the impact of three hour workshops on energy conservation that included a survey before and after the workshop \cite{Geller81}. The results of the survey indicated that the workshop had increased the energy literacy of the attendees and they indicated a willingness to implement energy conservation in their homes. However, followup visits with a selected group of 40 of the attendees found that very few had actually taken action (insulating their water heater or installing low-flow showerheads that had been given out during the workshops).

Techniques that have been shown to work are obtaining commitments, setting goals, and influencing social norms. Asking an individual to make a commitment has been shown to be an effective tool in changing behavior. In particular, an initial small, innocuous commitment can lead later to a larger commitment. For example, Freedman and Fraser conducted experiments in which subjects were asked to perform a small task (such as signing a petition to keep California beautiful) and then later asked to perform a more onerous task (such as placing a large billboard on their lawn that said ``Keep California Beautiful'') \cite{Freedman66}. They found that subjects that committed to the small task were much more likely to agree to the second task. The authors call this the ``foot-in-the-door'' technique. One of the reasons this technique is believed to work is the desire by individuals for self-consistency.

Making commitments public can increase their effectiveness. Pallak et al.\ studied residents that were asked to make a commitment to conserve electricity and natural gas \cite{Pallak80}. Some homes were asked to make a private commitment, while others were asked if their commitment could be publicized, though they were never actually published. Those that made commitments that they thought were public conserved more energy than the private committers, even one year later and after they were told that their names were not actually going to be publicized.

Goals can be thought of as commitments that can be objectively measured, which makes for a good pairing with feedback. Becker investigated goal setting along with feedback of home electricity use \cite{Becker78}. Half of the subjects were given a goal of reducing electricity use by 20\% during the summer, the other half were given a goal of 2\%. The subjects given the higher goal conserved between 13\%--15\%, while the group with the smaller goal did no better than a control group. Houwelingen and van Raaij investigated use of natural gas in homes and compared daily feedback with monthly feedback and self reporting, with all groups having a conservation goal of 10\% \cite{Houwelingen89}. The group with daily feedback reduced their energy use by 12.3\%, and some reduction continued in the year after the feedback device was removed from their home.

Social norms are one way in which people's behavior is influenced by the behavior of others. Cialdini et al.\ make the distinction between descriptive norms (the way things are) and injunctive norms (the way things ought to be) \cite{Cialdini90}. In a series of experiments on littering, they found that subjects were significantly influenced by observing the behavior of others. For example, subjects that viewed someone else littering were more likely to litter a handbill that had been placed on their car. Also, subjects that viewed someone else littering into a clean environment were less likely to litter than those that observed littering into an environment that already contained a lot of litter.

One problem with descriptive norms is that they can lead to `boomerang effects' where the norm has the effect of decreasing the desired behavior. Schultz et al.\ investigated this issue in the context of home energy conservation \cite{Schultz2007SocialNorms}. 290 homes were divided into two groups: one that would receive a written descriptive norm regarding their energy usage, and one that would receive the descriptive norm plus an injunctive norm. The descriptive norm showed subjects whether they were above or below the average energy usage in their neighborhood. The injunctive norm was simply a frowning or smiling emoticon based on whether the subject home was using more or less than the average consumption respectively. They found that homes that only received the descriptive norm led to energy conservation in homes above the average, but led to increased energy usage in homes below the average (the boomerang effect). However, those homes that also received the injunctive emoticon did not have a boomerang effect. Clearly injunctive norms are an important addition to any attempt to use comparative data to foster energy conservation.

\subsection{Energy literacy}

\emph{Energy literacy} is the understanding of energy concepts as they relate both on the individual level and on the national/global level. Solving the world energy crisis will require everyone to understand how energy is generated and consumed, so that they can make more informed choices in their lives and as informed citizens involved in their communities.

Defining and assessing energy literacy are therefore key to any attempt to improve energy literacy. DeWaters and Powers of Clarkson University have been working on an energy literacy survey instrument for middle and high school students \cite{DeWaters09c, DeWaters09}. They define energy literacy as consisting of three components: knowledge, attitudes, and behaviors. An example of energy knowledge would be understanding that the kilowatt-hour is the basic measure of electrical energy. Energy attitudes refers to concepts like needing to make more use of renewable energy in our power grid. Energy behaviors refer to specific things that can be done to reduce energy use, such as turning off lights when leaving a room.

Earlier work on assessing energy literacy includes a survey of attitude, knowledge, and intentions by Geller \cite{Geller81} given to participants at energy conservation workshops in the wake of the 1970s energy crisis.
