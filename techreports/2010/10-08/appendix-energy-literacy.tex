\chapter{Energy Literacy Questionnaire}
\label{app:energy-literacy}

This appendix details the contents of the questionnaire that was administered to assess subjects energy literacy, group identification, and connectedness to nature. Each section briefly relates the source and goal of that segment of the questionnaire, and then lists the actual items presented to subjects.

When subjects filled out the questionnaire via the SurveyGizmo~\cite{surveygizmo} website, the questions were broken into pages. Each page provided subjects the ability to move forward to the next page in the questionnaire, but not back to previous pages. In the energy knowledge section of the questionnaire, this inability to backtrack allowed later questions to include of information that might provide the answer to previous questions (such as what unit electrical power is measured in). The pages of the survey were:

\begin{enumerate}
	\item Informed consent via email address,
	\item Energy attitudes and behavior,
	\item Energy knowledge 1 (questions 1--5),
	\item Energy knowledge 2 (questions 6--9),
	\item Energy knowledge 3 (questions 10--13),
	\item Group identification and connectedness to nature,
	\item Open feedback on questionnaire, and
	\item Thank you page.
\end{enumerate}

Most items on the questionnaire were \emph{required}, meaning that subjects could not move to the next page of the questionnaire without submitting an answer. However, each required item included the choice ``Choose not to answer'' for those subjects that did not want to answer the item. The one exception is the entry of the email address on the informed consent page, which was required with no option to skip. Due to way the knowledge ranking questions (questions 5a--5c and 7a--7e) were presented in SurveyGizmo, these questions did not have a ``Choose not to answer'' option, so they were not marked required.

\section{Energy Attitudes}
\label{sec:attitude-items}

The energy attitudes section of the questionnaire was based on the affective subscale of the energy literacy questionnaire developed by DeWaters and Powers~\cite{DeWaters2011}. There are 18 statements in the attitudes section, and subjects were asked to respond to each one using the following five-point Likert-style scale:

\begin{enumerate}
	\item Strongly agree
	\item Agree
	\item Neutral
	\item Disagree
	\item Strongly disagree
	\item Choose not to answer
\end{enumerate}

Those statements marked with (\textbf{R}) were reverse scored so that their scores would match the direction of the rest of the statements. I made two changes from the DeWaters and Powers affective scale. The wording of statement 11 was changed from ``using'' to ``generating'', clarifying that there is no problem using renewable energy. The other change was the addition of statement 18, which was part of the behavior scale for DeWaters and Powers but matched the attitude questions here better than the behavior items.

The statements were prefaced with the following instructions: ``Please indicate how you feel about each statement below. There are no right or wrong answers.''

The statements in the attitude section were:

\begin{enumerate}
	\item Energy education should be an important part of every school's curriculum.
	\item I would do more to save energy if I knew how.
	\item Saving energy is important.
	\item The way I personally use energy does not really make a difference to the energy problems that face our nation. (\textbf{R})
	\item I don't need to worry about turning the lights or computers off in the residence halls, because the school pays for the electricity. (\textbf{R})
	\item Americans should conserve more energy.
	\item We don't have to worry about conserving energy, because new technologies will be developed to solve the energy problems for future generations. (\textbf{R})
	\item All electrical appliances should have a label that shows the resources used in making them, their energy requirements, and operating costs.
	\item The government should have stronger restrictions about the gas mileage of new cars.
	\item We should make more of our electricity from renewable resources.
	\item America should develop more ways of generating renewable energy, even if it means that energy will cost more.
	\item Efforts to develop renewable energy technologies are more important than efforts to find and develop new sources of fossil fuels.
	\item Laws protecting the natural environment should be made less strict in order to allow more energy to be produced. (\textbf{R})
	\item More wind farms should be built to generate electricity, even if the wind farms are located in scenic valleys, farmlands, and wildlife areas.
	\item More oil fields should be developed as they are discovered, even if they are located in areas protected by environmental laws. (\textbf{R})
	\item I believe that I can contribute to solving the energy problems by making appropriate energy-related choices and actions.
	\item I believe that I can contribute to solving energy problems by working with others.
	\item Many of my everyday decisions are affected by my thoughts on energy use.
\end{enumerate}



\section{Energy Behaviors}
\label{sec:behavior-items}

The energy behaviors section of the questionnaire was inspired by the behavioral subscale of the energy literacy questionnaire developed by DeWaters and Powers~\cite{DeWaters2011}. There are 17 statements in the behaviors section, and subjects were asked to respond to each one using the following five-point Likert-style scale from DeWaters/Powers:

\begin{enumerate}
	\item Always or almost always
	\item Quite frequently
	\item Sometimes
	\item Not very often
	\item Never or hardly ever
	\item Not applicable
	\item Choose not to answer
\end{enumerate}

The choice of ``not applicable'' was added to allow subjects to respond to statements that might not apply to them, such as driving a car if they do not own a car. Those statements marked with (\textbf{R}) were reverse scored so that their scores would match the direction of the rest of the statements.

The statements in DeWaters and Powers behavior subscale were tailored for middle school and high school students in New York State, which unfortunately made many of the statements inappropriate for college students in \Hawaii. For example, two questions from the DeWaters/Powers behavior subscale are: ``My family turns the heat down at night to save energy.'' and ``I walk or bike to go short distances, instead of asking for a ride in the car.''. Instead of the DeWaters/Powers statements, I used statements derived from the commitments that participants could make as part of the challenge (see \autoref{sec:commitments}). The commitments were already tailored to college students in \Hawaii living in student housing. 

The statements were prefaced with the following instructions: ``For the following statements, please select the choice that best describes your behavior. Please be honest, there are no right or wrong answers.''

The statements in the behavior section were:

\begin{enumerate}
	\item I turn off all appliances (TV, computer, game console, etc) every night before going to sleep.
	\item I leave my computer and/or monitor on, even when they are not being used. (\textbf{R})
	\item I turn off vampire loads (like cell phone chargers) using a power strip.
	\item I leave the lights on when I leave a room. (\textbf{R})
	\item I use task lighting (like desk lamps) rather than overhead lighting.
	\item I use sunlight rather than electric lighting whenever possible.
	\item I take the stairs rather than the elevator whenever feasible.
	\item I drive alone (no passengers). (\textbf{R})
	\item I walk, bike, or roll to go short distances, instead of driving.
	\item I use public transportation.
	\item I recycle my cans and bottles.
	\item I bring reusable bags when shopping.
	\item I eat meat. (\textbf{R})
	\item I turn off water when brushing my teeth, shaving, etc.
	\item I turn off water in the shower when soaping and scrubbing.
	\item I wash only full loads of laundry.
	\item I wash my laundry in warm or hot water. (\textbf{R})
\end{enumerate}


\section{Energy Knowledge}
\label{sec:knowledge-items}

These factual questions assess energy knowledge. As discussed at the beginning of this appendix, the knowledge questions were separated into three pages. When presented to subjects, the order of questions within the page was randomized, as was the order of the multiple choice answers. I have assigned keywords to each question to indicate which subjects they attempt to assess.

Each page was prefaced with the following instructions:

``Please answer the following questions to the best of your ability, without consulting any books or the Internet. We are interested in what you know right now.''

\subsection{Knowledge Page 1}

\noindent
1. Electrical power is commonly measured in units of:

\begin{answer}
	\item volts (V)
	\item watt-hours (Wh)
	\item joule (J)
	\item watts (W)
	\item British Thermal Units (BTU)
	\item Choose not to answer
\end{answer}

Correct answer: watt

Keywords: power, units

\vspace{5 mm}
\noindent
2. What is the primary cause of current climate changes?

\begin{answer}
	\item Carbon dioxide released from burning fossil fuels
	\item There is no cause, climate change isn't real
	\item Natural solar cycles
	\item Radioactive waste from nuclear power plants
	\item Melting glaciers in Greenland
	\item Choose not to answer
\end{answer}

Correct answer: Carbon dioxide released from burning fossil fuels

Keywords: climate change

\vspace{5 mm}
\noindent
3. Electrical energy is commonly measured in units of

\begin{answer}
	\item erg
	\item ampere (A)
	\item British Thermal Units (BTU)
	\item watt-hours (Wh)
	\item watts (W)
	\item Choose not to answer
\end{answer}

Correct answer: watt-hour

Keywords: energy, units

\vspace{5 mm}
\noindent
4. What is the breakdown of the clean energy mandated by 2030 by the Hawaii Clean Energy Initiative?

\begin{answer}
	\item 20\% from renewable sources, 80\% from energy conservation
	\item 30\% from energy conservation, 40\% from renewable sources
	\item 50\% from renewable sources, 10\% from conservation
	\item 30\% from solar, 30\% from wind, 10\% from waves
	\item 30\% from renewable sources, 20\% from conservation, 10\% from natural gas
	\item Choose not to answer
\end{answer}

Correct answer: 30\% from energy conservation, 40\% from renewable sources

Keywords: HCEI

\vspace{5 mm}
\noindent
5a--5c. Order these types of light sources from lowest to highest power usage, assuming they provide the same amount of light:

\begin{answer}
	\item incandescent bulb
	\item compact fluorescent lightbulb (CFL)
	\item light-emitting diode (LED)
\end{answer}

Correct answer: c, b, a

Keywords: lighting, energy intuition


\subsection{Knowledge Page 2}

\noindent
6. Approximately how much carbon dioxide (CO2) is in the atmosphere now, and what level is considered the safe upper limit to avoid the worst effects of climate change?

\begin{answer}
	\item 450 ppm CO2 in atmosphere now, 500 ppm CO2 safe upper limit
	\item 331 ppm CO2 in atmosphere now, 350 ppm CO2 safe upper limit
	\item 393 ppm CO2 in atmosphere now, 350 ppm CO2 safe upper limit
	\item 600 ppm CO2 in atmosphere now, 450 ppm CO2 safe upper limit
	\item 100 ppm CO2 in atmosphere now, 50 ppm CO2 safe upper limit
	\item Choose not to answer
\end{answer}

Correct answer: 393 ppm, 350 ppm

Keywords: climate change

\vspace{5 mm}
\noindent
7a--7e. Order these appliances from lowest to highest power usage:

\begin{answer}
	\item desk lamp with compact fluorescent lightbulb (CFL)
	\item mobile phone charger (while charging)
	\item plasma TV
	\item microwave
	\item laptop
\end{answer}

Correct answer: b, a, e, c, d

Keywords: energy intuition

\vspace{5 mm}
\noindent
8. On average, how much electrical energy does a home in Hawaii use per day?

\begin{answer}
	\item 400 W
	\item 20 kWh
	\item 87 kWh
	\item 328 kWh
	\item 4 kWh
	\item Choose not to answer
\end{answer}

Correct answer: b

Keywords: energy intuition, \Hawaii

\vspace{5 mm}
\noindent
9. What is the approximate maximum power generated from a single standard rooftop solar panel?

\begin{answer}
	\item 25 W
	\item 800 W
	\item 50 W
	\item 10 kW
	\item 200 W
	\item Choose not to answer
\end{answer}

Correct answer: 200 W

Keywords: power, energy intuition, generation, PV


\subsection{Knowledge Page 3}

\noindent
10. What are the expected long-term effects of current climate changes?

\begin{answer}
	\item A significant rise in the sea level
	\item Global temperatures increasing by a few degrees on average
	\item Increasing sea water acidity
	\item Changes in seasonal rainfall patterns (droughts, floods)
	\item All of the above
	\item Choose not to answer
\end{answer}

Correct answer: All of the above

Keywords: climate change

\vspace{5 mm}
\noindent
11. What is currently the source of approximately 80\% of Hawaii's electricity?

\begin{answer}
	\item oil
	\item wind
	\item natural gas
	\item coal
	\item solar
	\item Choose not to answer
\end{answer}

Correct answer: oil

Keywords: generation, utility, \Hawaii

\vspace{5 mm}
\noindent
12. A compact fluorescent lightbulb (CFL) uses 13 W. If it is run for 2 hours, how much energy does it use?

\begin{answer}
	\item 13 Wh
	\item 7.5 Wh
	\item 26 Wh
	\item 130 Wh
	\item 52 Wh
	\item Choose not to answer
\end{answer}

Correct answer: 26 Wh

Keywords: power, energy, calculation

\vspace{5 mm}
\noindent
13. If your game console uses 200 W when turned on, how much energy would it waste if you left it on all weekend while you were away?

\begin{answer}
	\item 15000 Wh
	\item 100 Wh
	\item 960 kWh
	\item 9.6 kWh
\end{answer}

Correct answer: 9.6 kWh

Keywords: power, energy, calculation


\section{Group Identification}
\label{group-id-items}

I used the Arrow-Carini Group Identification Scale 2.0~\cite{Henry1999} for the group identification section of the questionnaire. It consists of 12 statements in three subscales: affective, behavioral, and cognitive. Subjects were asked to respond to each one using the following seven-point Likert-style scale:

\begin{enumerate}
	\item Strongly disagree
	\item Moderately disagree
	\item Slightly disagree
	\item Neutral
	\item Slightly agree
	\item Moderately agree
	\item Strongly agree
	\item Choose not to answer
\end{enumerate}

Those statements marked with (\textbf{R}) were reverse scored so that their scores would match the direction of the rest of the statements. The statements were prefaced with the following instructions: ``Please answer each of these questions in terms of the way you generally feel about your lounge. There are no right or wrong answers. Using the following scale, simply state as honestly and candidly as you can what you are presently experiencing.''

The statements in the group identification section were:

\begin{enumerate}
	\item I would prefer to be in a different lounge. (\textbf{R})
	\item In this lounge, members don't have to rely on one another. (\textbf{R})
	\item I think of this lounge as part of who I am.
	\item Members of this lounge like one another.
	\item All members need to contribute to achieve the lounge's goals.
	\item I see myself as quite different from other members of the lounge. (\textbf{R})
	\item I enjoy interacting with the members of this lounge.
	\item This lounge accomplishes things that no single member could achieve.
	\item I don't think of this lounge as part of who I am. (\textbf{R})
	\item I don't like many of the other people in this lounge. (\textbf{R})
	\item In this lounge, members do not need to cooperate to complete group tasks. (\textbf{R})
	\item I see myself as quite similar to other members of the lounge.
\end{enumerate}


\section{Connectedness To Nature}
\label{cns-items}

This section of the questionnaire used the Connectedness to Nature Scale (CNS) developed by Mayer and Frantz~\cite{MayerFrantz2004}. It consists of 14 statements. Subjects were asked to respond to each one using the following five-point Likert-style scale:

\begin{enumerate}
	\item Strongly disagree
	\item Disagree
	\item Neutral
	\item Agree
	\item Strongly agree
	\item Choose not to answer
\end{enumerate}

Those statements marked with (\textbf{R}) were reverse scored so that their scores would match the direction of the rest of the statements. The statements were prefaced with the following instructions: ``Please answer each of these questions in terms of the way you generally feel. There are no right or wrong answers. Using the following scale, simply state as honestly and candidly as you can what you are presently experiencing.''

The statements in the group identification section were:

\begin{enumerate}
	\item I often feel a sense of oneness with the natural world around me.
	\item I think of the natural world as a community to which I belong.
	\item I recognize and appreciate the intelligence of other living organisms.
	\item I often feel disconnected from nature. (\textbf{R})
	\item When I think of my life, I imagine myself to be part of a larger cyclical process of living.
	\item I often feel a kinship with animals and plants.
	\item I feel as though I belong to the Earth as equally as it belongs to me.
	\item I have a deep understanding of how my actions affect the natural world.
	\item I often feel part of the web of life.
	\item I feel that all inhabitants of Earth, human, and nonhuman, share a common 'life force'.
	\item Like a tree can be part of a forest, I feel embedded within the broader natural world.
	\item When I think of my place on Earth, I consider myself to be a top member of a hierarchy that exists in nature. (\textbf{R})
	\item I often feel like I am only a small part of the natural world around me, and that I am no more important than the grass on the ground or the birds in the trees.
	\item My personal welfare is independent of the welfare of the natural world. (\textbf{R})
\end{enumerate}
