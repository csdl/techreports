\chapter{Participant Kukui Nut Tasks}
\label{app:tasks}

This appendix lists tasks intended to be undertaken by the competition participants. Each task should increase the energy literacy of the participants performing it, help them modify their behavior to reduce electricity usage, or both. The following lists all the possible tasks, and indicate how they would be performed, validated, and what the potential benefit would be to the person performing it. The tasks are grouped into four categories: events, activities, commitments, and goals. For more information, see \autoref{sec:competition-tasks}.

We expect that additional tasks will be developed between the time of this proposal and the actual competition start date.


\section{Events}

One common type of task is attendance of an event. In our model, there are two ways to get credit for attending an event: activities (individual attendance), and goals (floor attendance). Since the parameters are often identical between the activity version and the goal version of an event, they are grouped together here.

For both event activities and goals, attendance is verified using non-forgeable, single-use attendance codes such as "orientation-158-B7QRX13". The codes are printed on small slips of paper that are handed out by some responsible person who is not a participant (such as the event speaker or an RA).

In the case of activities, to get credit for attending, the individual participant logs into the web site and enters in the attendance code. The website automatically awards KN points if the attendance code is valid, and it has not already been entered.

For goals, the participant that initiated the goal must log into the website after the event and indicate that the goal was met (perhaps prodding any floormates to enter their attendance codes if they haven't already done so). The website will then award the appropriate KNs to all members of the floor (including those who did not attend). Goals must have the participation of at least half of the floor participants to be successful. If a floor achieves 100\% participation, they receive double the KN.

Relatively passive events like movies or lectures should be worth around 5 KN, while more interactive events like workshops should be worth more (perhaps 10-15 KN).

\subsection{Attend Kukui Cup orientation}

Description: Participant attends a large orientation meeting about the Kukui Cup competition.

Potential benefits: Understanding of the competition mechanics, collaboration with other floor participants on competition strategy.

Psychological justifications: ?

Activity reward: 4 KN

Goal reward: 5 KN (unlikely to be obtained, since this happens at very beginning of competition)

\subsection{Attend EnergyPong tournament}

Description: Participant attends the EnergyPong tournament for their building.

Potential benefits: Improved energy literacy through hearing energy questions answered, floor bonding.

Psychological justifications: competition

Activity reward: 2 KN

Goal reward: 4 KN

\subsection{Attend a special Kukui Cup SustainableUH meeting}

Description: Participant attends a special presentation by SustainableUH team members on what SustainableUH is doing on campus.

Potential benefits: Getting involved with peers on campus, learning what challenges exist and how students are working to overcome them.

Psychological justifications: ?

Activity reward: 2 KN

Goal reward: 5 KN

\subsection{Watch the movie "Who Killed the Electric Car"}

Description: Participant watches the movie.

Potential benefits: Understanding of the possibility of de-carbonizing transportation, difficulty of changing status quo.

Activity reward: 2 KN

Goal reward: 5 KN

\subsection{Watch the movie "Enron: The Smartest Guys in the Room"}

Description: Participant watches the movie.

Potential benefits: Understanding risks and problems from utility deregulation, ethical issues.

Activity reward: 2 KN

Goal reward: 5 KN

\subsection[Watch the movie ``The End of Suburbia'']{Watch the movie ``The End of Suburbia''}

Description: Participant watches the movie.

Potential benefits: Understanding peak oil, design of communities around automotive transportation and plentiful cheap energy.

Activity reward: 2 KN

Goal reward: 5 KN

\subsection{Watch the movie "A Crude Awakening: Oil Crash"}

Description: Participant watches the movie.

Potential benefits: Understanding peak oil, consequences for society.

Activity reward: 2 KN

Goal reward: 5 KN

\subsection{Watch the movie "The Great Warming"}

Description: Participant watches the movie.

Potential benefits: Understanding climate change, consequences for society.

Activity reward: 2 KN

Goal reward: 5 KN

\subsection{Watch the movie "An Inconvenient Truth"}

Description: Participant watches the movie.

Potential benefits: Understanding climate change, consequences for society.

Activity reward: 2 KN

Goal reward: 5 KN

\subsection{Participate in a 10/10/10 work party}

Description: [http://www.350.org/ 350.org], a climate change advocacy organization is organizing a series of "work parties" to take place on October 10, 2010 (10/10/10). Participant participates in a work party in Honolulu (check website for options). Since this is off campus, might need to support alternate verification (photo and text) instead of attendance codes.

Potential benefits: Understanding climate change, consequences for society.

Activity reward: 5 KN

Goal reward: 7 KN


\section{Activities}

\subsection{Perform room energy audit}

Description: Resident borrows a Kill-A-Watt plug load meter from their RA, then checks all plug-in appliances in their room to see what their energy consumption is when on and off.

Verification: Participant fills out form on website that contains a list of rows for each device with columns: device name, power (watts) when off, power (watts) when on, notes. Admin reviews data, checking mainly for completeness (more than 1 device?) and sanity (XBox 360s don't use 1000 W).

Reward: 10 KN

Potential benefits: Increased intuitive understanding of the watt, familiarity with vampire power, understanding of how device usage would impact energy consumption, reduced electricity usage due to turning off devices when not in use.

Psychological justifications: feedback, activity-based learning (?)

\subsection{Replace incandescent bulb with compact fluorescent (CFL)}

Description: Participant finds an incandescent bulb (perhaps from a desk lamp) and replaces it with a CFL, throwing away the incandescent bulb.

Verification: Participant takes a picture showing both the incandescent bulb and the CFL replacement and uploads it via a verification form on the website, along with a text field indicating where the replaced bulb is located. Admin briefly reviews the picture to ensure that in fact both bulbs are present.

Reward: 3 KN

Potential benefits: Reduced energy usage via CFL, awareness of energy impact of incandescent bulbs.

Psychological justifications: activity-based learning (?)

\subsection{Configure computer \& monitor to sleep after inactivity}

Description: Participant configures their computer and any external display to sleep after 20 minutes of inactivity.

Verification: Participant takes a screenshot from their computer showing sleep settings <= 20 minutes and uploads it via a verification form on the website. Admin briefly reviews the picture to ensure that the settings look correct.

Reward: 3 KN

Potential benefits: Reduced computer \& monitor energy usage, knowledge of how to set it up on other computers (friends, work, future purchases, etc).

Psychological justifications: none

\subsection{Play in EnergyPong tournament}

Description: Participant is on their floor's team in the EnergyPong tournament for their building.

Verification: Some responsible person who is not a participant (such as the speaker or an RA) records attendance and performance, which is reported to the website admins either on paper or via email.

Reward: 4 KN + 1 KN per bracket completed + 5 KN for the winning team

Potential benefits: Improved energy literacy through answering energy questions answered, floor bonding.

Psychological justifications: competition, incentives (if prizes are awarded to winning team)

\subsection{Connect to Kukui Cup on Facebook}

Description: Participant becomes a fan of the Kukui Cup Competition group on Facebook.

Verification: Participant takes a screenshot from their computer showing Facebook fan status. Admin briefly reviews the picture to ensure that the participant is a fan.

Reward: 3 KN

Potential benefits: Another avenue for communicating with students, promotion of the contest and energy literacy.

Psychological justifications: community involvement?

\subsection{Tweet about Kukui Cup}

Description: Participant sends a tweet promoting the Kukui Cup Competition with a link to the website.

Verification: Participant takes a screenshot from their computer showing the tweet in their newsfeed. Admin briefly reviews the picture to ensure that the participant tweeted.

Reward: 2 KN

Potential benefits: Promotion of the contest and energy literacy.

Psychological justifications: social networking?

\subsection{Facebook Status update about Kukui Cup}

Description: Participant updates their Facebook status promoting the Kukui Cup Competition with a link to the website.

Verification: Participant takes a screenshot from their computer showing the status in their newsfeed. Admin briefly reviews the picture to ensure that the participant updated their status.

Reward: 2 KN

Potential benefits: Promotion of the contest and energy literacy.

Psychological justifications: social networking?

\subsection{Label all plug loads in room}

Description: Followup to room energy audit. Based on the audit results, make a label for each device with the number of watts consumed when on and off, located close to the power switch for those devices that have them.

Verification: Participant takes a picture of the devices with their labels. Admin briefly reviews the picture to ensure that labels are present.

Reward: 3 KN

Potential benefits: understanding of how device usage would impact energy consumption, reduced electricity usage due to turning off devices when not in use.

Psychological justifications: prompts

\subsection{Determine carbon footprint using calculator}

Description: Participant uses a web-based carbon footprint calculator to determine their carbon footprint.

Verification: Participant enters in their computed carbon footprint into a text field. Admin briefly reviews the footprint to make sure it is sane (units include CO2 and it isn't huge or tiny).

Reward: 3 KN

Potential benefits: learning about carbon emissions, learning how carbon emissions impact the environment.

Psychological justifications: personalized data


\section{Commitments}

Note that per the requirements, commitments are participant-verified without outside intervention, so that field is not used for this category.

\subsection{Turn off lights when I leave the room}

Description: The participant commits to turning off all lights whenever they are the last person to leave a room.

Reward: 2 KN

Potential benefits: Reduced electricity usage due to less unneeded lighting, highly obvious reminder of need to conserve energy.

Psychological justifications: public commitments

\subsection{Use task lighting instead of overhead lights}

Description: The participant commits to using task lighting (i.e. a desk lamp) instead of overhead room lights. Might only be appropriate if housing rooms have overhead lights.

Reward: 2 KN

Potential benefits: Reduced electricity usage due to less excess lighting.

Psychological justifications: public commitments

\subsection{Always disconnect vampire loads using a power strip}

Description: The participant commits to turning off any vampire loads (cell phone charger, iPod charger, game consoles, TVs) using a power strip when they are not using them.

Reward: 2 KN

Potential benefits: Reduced electricity usage due to vampire loads, awareness of vampire loads.

Psychological justifications: public commitments

\subsection{Turn off water when brushing teeth, shaving, etc}

Description: The participant commits to turning off any vampire loads (cell phone charger, iPod charger, game consoles, TVs) using a power strip when they are not using them.

Reward: 2 KN

Potential benefits: Reduced electricity usage due to vampire loads, awareness of vampire loads.

Psychological justifications: public commitments

\subsection{Turn off water when sudsing and scrubbing in shower}

Description: The participant commits to turning off water when showering except when actively rinsing off.

Reward: 2 KN

Potential benefits: Reduced electricity usage due to less water heating and pumping.

Psychological justifications: public commitments

\subsection{Use natural light instead of electric lighting whenever possible}

Description: The participant commits to using natural light from windows or outdoors instead of turning on electric lighting. This can mean opening shades instead of turning on the lights, and/or planning their day so that tasks that require light (like reading books, doing written homework) are done during the day.

Reward: 2 KN

Potential benefits: Reduced electricity usage due to less use of electric lights.

Psychological justifications: public commitments

\subsection{Turn off printer when not printing}

Description: The participant commits to turning off their printer when they are not actively printing out documents.

Reward: 2 KN

Potential benefits: Reduced electricity usage due to less standby electricity for printer.

Psychological justifications: public commitments

\subsection{Use stairs instead of elevator}

Description: The participant commits to using the stairs instead of elevators whenever feasible.

Reward: 2 KN

Potential benefits: Reduced electricity usage due to less elevator traffic. Increased exercise for participant.

Psychological justifications: public commitments

\subsection{Recycle all beverage containers}

Description: The participant commits recycling all (recyclable) beverage containers at an appropriate location.

Reward: 2 KN

Potential benefits: Reduced carbon emissions due to recovery and eventual reuse of recyclable material, reduction in waste stream.

Psychological justifications: public commitments

\subsection{Don't drive off-campus using a single-occupant car}

Description: The participant commits to not traveling off-campus in single-occupant car, using bus, bike, walking, or vehicle with 3+ occupants instead.

Reward: 2 KN

Potential benefits: Reduced carbon emissions due to less single occupant car travel, reduction in traffic and parking.

Psychological justifications: public commitments

\subsection{Turn off/shut down all appliances before going to sleep}

Description: The participant commits to turning off or shutting down appliances like computers, TVs, DVD players, and game consoles before going to sleep each night.

Reward: 2 KN

Potential benefits: Less electricity wasted on appliances that aren't being used.

Psychological justifications: public commitments

\subsection{Limit TV watching to 1 hour a day or less}

Description: The participant commits to watching not more than 1 hour of TV per day.

Reward: 2 KN

Potential benefits: Less electricity used by television.

Psychological justifications: public commitments

\subsection{Do only full loads of laundry}

Description: The participant commits to always doing full loads of laundry.

Reward: 2 KN

Potential benefits: Less electricity \& hot water used per piece of laundry washed.

Psychological justifications: public commitments

\subsection{Wear Kukui Cup button every day}

Description: The participant commits to wearing their Kukui Cup button every day during the commitment period.

Reward: 2 KN

Potential benefits: promotion of the contest.

Psychological justifications: public commitments

\subsection{Walk to destinations less than one mile away}

Description: The participant commits to walking to any destination less than one mile away from their residence hall.

Reward: 2 KN

Potential benefits: Reduced gasoline usage due to car usage. Increased exercise for participant.

Psychological justifications: public commitments

\subsection{Wash laundry in cold water}

Description: The participant commits to washing laundry in cold water instead of warm or hot water.

Reward: 2 KN

Potential benefits: Reduced electricity usage by reduction in water heating and pumping.

Psychological justifications: public commitments

\subsection{Reduce the shower time by 1 minute}

Description: The participant commits to measuring the length of their shower with a watch, and reducing the time by 1 minute.

Reward: 2 KN

Potential benefits: Reduced electricity usage by reduction in water heating and pumping.

Psychological justifications: public commitments

\subsection{Turn off music when leaving room}

Description: The participant commits to turning off their music (from computer, stereo, etc) when they leave the room.

Reward: 2 KN

Potential benefits: Reduced electricity usage.

Psychological justifications: public commitments

\subsection[Do something ``unplugged'' every day]{Do something ``unplugged'' every day}

Description: The participant commits to doing something that doesn't require electricity instead of watching TV, using their computer, or playing a console game.

Reward: 2 KN

Potential benefits: Reduced electricity usage, increased exercise?

Psychological justifications: public commitments

\subsection{Bring reusable bags when shopping}

Description: The participant commits to bringing and using reusable bags when shopping instead of the paper or plastic ones offered by the store.

Reward: 2 KN

Potential benefits: Reduced waste, reduced carbon footprint.

Psychological justifications: public commitments

\subsection{Don't eat meat}

Description: The participant commits to not eating any meat (beef, pork, chicken, fish, shellfish, etc) for the commitment period.

Reward: 2 KN

Potential benefits: Reduced carbon footprint, potentially improved health.

Psychological justifications: public commitments


\section{Goals}
\label{sec:goal-list}

\subsection{Reduce our floor's energy consumption by {target} }
\label{sec:goal-reduce-energy}

Description: A floor participant picks a target goal (hopefully in consultation with rest of floor) for reduction for the current period from a list of choices from 5\% to 50\% in 5\% increments.  When the goal is specified, the system uses the Average Floor Power as the value being reduced from.  The system can provide a graphic that is updated in near-real time to show whether (a) the current usage is above or below the target, and (b) whether the cumulative usage so far is above or below the target.  The graphic can also provide a count-down timer showing the time remaining to achieve this goal in days:hours:minutes.

Note that the percentage reduction is always relative to the baseline, not the prior week.  So, a floor might start out with a conservative goal of 5\%, then find that they actually achieved 16\% during the period. So, they could restart the goal for the next period, this time choosing 15\%.

Verification: Participant that picked the goal must use the web interface indicate that the goal has been met or not met, and an admin assigns points accordingly.

Reward: If the floor achieves the target reduction, then each member of the floor is awarded 1 KN per target percentage reduction. For example, if the target reduction was 5\% and the floor achieved 7\%, then each member gets 5 KNs for achieving this goal.

Potential benefits: Reduced electricity usage, group planning for how to achieve target through behavior changes.

Psychological justifications: goal setting with feedback, social norms

\subsection{Finding the minimum floor power}
\label{sec:goal-minimum-energy}

Description: A floor participant picks a day and time for the floor to try to determine the minimum amount of power the floor can consume. Everyone on the floor must disconnect and unplug all loads, all lights must be turned off, etc. Then, using a laptop or mobile device, the floor's instantaneous power value is recorded from the monitors on the contest website.

Note that this goal requires near 100\% participation to be successful.

Verification: Participant that picked the goal must use the web interface indicate what power value the floor was able to record. The admin can then compare this to the Minimum Floor Power determined before residents moved in. If the participants got within 10\% of the MFP, then the KN are awarded.

Reward: Each member gets 10 KNs for achieving this goal.

Potential benefits: Awareness of building infrastructure power draws, group collaboration to turn everything off, awareness of vampire loads.

Psychological justifications: ?
