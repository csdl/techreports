\begin{abstract}
%% No longer than 350 words for ProQuest publication

We designed the Kukui Cup challenge to foster energy conservation and increase energy literacy. Based on a review of the literature, the challenge combined a variety of elements into an overall game experience, including: real-time energy feedback, goals, commitments, competition, and prizes.

We designed a software system called Makahiki to provide the online portion of the Kukui Cup challenge. Energy use was monitored by smart meters installed on each floor of the Hale Aloha residence halls on the University of \Hawaii at \Manoa campus.

In October 2011, we ran the UH Kukui Cup challenge for the over 1000 residents of the Hale Aloha towers. To evaluate the Kukui Cup challenge, I conducted three experiments: challenge participation, energy literacy, and energy use.

Many residents participated in the challenge, as measured by points earned and actions completed through the challenge website. I measured the energy literacy of a random sample of Hale Aloha residents using an online energy literacy questionnaire administered before and after the challenge. I found that challenge participants' energy knowledge increased significantly compared to non-challenge participants. Positive self-reported energy behaviors increased after the challenge for both challenge participants and non-participants, leading to the possibility of passive participation by the non-challenge participants.

I found that energy use varied substantially between and within lounges over time. Variations in energy use over time complicated the selection of a baseline of energy use to compare the levels during and after the challenge. The best team reduced its energy use during the challenge by 16\%. However, team energy conservation did not appear to correlate to participation in the challenge, and there was no evidence of sustained energy conservation after the challenge. The problems inherent in assessing energy conservation using a baseline call into question this common practice.

My research has generated several contributions, including: a demonstration of increased energy literacy as a result of the challenge, the discovery of fundamental problems with the use of baselines for assessing energy competitions, the creation of two open source software systems, and the creation of an energy literacy assessment instrument.

\end{abstract}