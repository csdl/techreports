%%%%%%%%%%%%%%%%%%%%%%%%%%%%%% -*- Mode: Latex -*- %%%%%%%%%%%%%%%%%%%%%%%%%%%%
%% project.tex -- 
%% Author          : Philip Johnson
%% Created On      : Tue Mar 31 11:44:58 2009
%% Last Modified By: Philip Johnson
%% Last Modified On: Mon Apr 06 15:00:58 2009
%% RCS: $Id$
%%%%%%%%%%%%%%%%%%%%%%%%%%%%%%%%%%%%%%%%%%%%%%%%%%%%%%%%%%%%%%%%%%%%%%%%%%%%%%%
%%   Copyright (C) 2009 
%%%%%%%%%%%%%%%%%%%%%%%%%%%%%%%%%%%%%%%%%%%%%%%%%%%%%%%%%%%%%%%%%%%%%%%%%%%%%%%
%% 

\section*{Project Description}
\pagenumbering{arabic}
\renewcommand{\thepage} {C--\arabic{page}}

\subsection*{Project Vision, Goals, Objectives, and Outcomes}


{\em Describe the CT-centric vision, goals, objectives, and anticipated outcomes of the proposed project. Clearly indicate how they will contribute to realization of the three CPATH program goals: (1) contribute to the development of a globally competitive U.S. workforce with CT competencies essential to U.S. leadership in the global innovation enterprise; (2) increase the number of students developing CT competencies by infusing CT learning opportunities into undergraduate education in the core computing - computer and information science and engineering - disciplines, and in other fields of study; and, (3) demonstrate transformative CT-focused undergraduate education models that are replicable across a variety of institutions.
}

\bigskip

Jeannette Wing has written, ``Computational thinking involves solving
problems, designing systems, and understanding human behavior, by drawing
on the concepts fundamental to computer science'' \citep{Wing06}.  In her
presentation ``Computational Thinking and Thinking About Computation'',
Wing refines her view of these fundamental computer science concepts in terms of 
the ``Two As'': Abstraction and Automation.  Activities
related to the first ``A'' include: choosing the right abstractions, operating
at multiple levels of abstraction, and defining relationships between
abstractions.  Activities related to the second ``A'' involve mechanizing the
first A via precise notations and models.  In essence, automation amplifies
the power of abstraction.  Computational thinking, from this perspective,
involves the correct choice of abstraction combined with the correct choice
of automation.

The vision of this proposal is to develop and institutionalize a new
approach to computational thinking where abstraction and automation combine
to transform {\em empirical thinking} in software development.

What is empirical thinking?  The term ``empirical'' is variously defined as
``derived from experiment and observation rather than theory''; ``evidence
or consequences observable by the senses''; and ``capable of being verified
or disproved by observation or experiment.''

Given these definitions, it is clear that some sort of empirical thinking
is already commonplace in software development.  For example, beginning
programmers use empirical thinking when they ``observe'' the output of the
compiler to learn how to write syntactically correct programs.  Beginners
also tend to make extensive use of ``experimentation'': they execute their
program with example data, compare the actual behavior to what they expect,
then make modifications until the observed behavior matches their
expectations.

These forms of empirical thinking, while satisfactory for beginning
programmers, do not scale well because they lack both abstraction and
automation. Thus, they fail to constitute the kind of computational
thinking of interest to the CPATH program.

One would hope that as students progress into more advanced software
development courses, the curriculum would scale in at least two
ways. First, the complexity, size, and personnel involved in software
development projects would scale upwards.  Second, the level of abstraction
and automation of their empirical thinking would scale
commensurately. Unfortunately, while advanced software development courses
certainly require students to develop significantly more sophisticated
systems than their introductory counterparts, the pedagogy around empirical
thinking remains mostly non-abstract and non-automated.  The principle
computational support for advanced programming classes is an integrated
development environment such as Eclipse or Visual Studio. While this is a
significant advance over vanilla text editors, such IDEs provide relatively
little in the way of abstraction or automation for empirical thinking about
the products and processes of software development.

The goal of this research is to explore, evaluate, and institutionalize
tools and technologies for abstraction and automation of empirical thinking
in software development.  For example, we have developed a
system and associated curriculum called the ``Software Intensive Care
Unit'' \citep{csdl2-09-02}.  In this approach, sensors attached to
development tools automatically collect raw software development process
and product data and abstract it into a set of ten ``vital signs'' that
provide an empirical way for students to gauge the ``health'' of their
ongoing projects.  The Software ICU supports automated, abstract empirical
thinking by students about the current state and past history of both their
projects and their group processes.

To achieve this goal, we will pursue three concrete objectives, organized
according to the three years of this grant.  

First, we will perform a study during the 2009-2010 academic year at the
University of Hawaii in which we will validate and extend the findings from
our initial case study of the Software ICU during 2008.  We will also
evaluate another technology for advanced empirical thinking called
Devcathlon, which is currently under development and scheduled for initial
release by the time this project begins.

Second, we will partner with other academic institutions and departments
during the 2010-2011 academic year to gain insight into the issues that
occur when integrating empirical thinking into other advanced software
development courses.  In addition, we will begin exploring the issues
involved in adapting these initial materials both downward (into more
elementary curriculum); upward (into post-scholastic, professsional
settings); and outward (into related disciplines such as engineering or
information technology).

Third, we will use the data gathered and the collaborations formed during
the first two years to form an open source consortium to further spread the
use of abstract, automated empirical thinking.  This consortium will itself
be transformative in that it will combine open source technologies (such as
the Software ICU and Devcathlon) with open source curriculum materials
(which facilitate the introduction of the materials) with open source data
(results from the application of these technologies and pedagogies, which
can be used to guide future evolution of these approach).

Our vision, goal, and objectives are designed to produce outcomes that
directly support the goals of the CPATH program.  First, establishing
automated, abstract empirical thinking as an integral component of software
development courses can significantly improve the competitiveness of our
software engineering workforce by giving them facility with a powerful tool
for computational thinking.  Second, our approach begins by infusing a new,
empirical form of computational thinking into the advanced software
development curriculum and then propogates it downward, upward, and
outward.  Third, we will collect data and experiences on the effectiveness
of this approach across a variety of institutions and pedagogical models.

\subsection*{Intellectual Basis/Related Work}

{\em Describe the intellectual basis for the project and discuss related prior work.  Include a review of the research literature relevant to the project and provide corresponding references. }

There are two major strands of research and educational practice related to our approach: empirical thinking in software development, advanced project-based software development courses. 

\subsection*{Research from the empirical software engineering community}

As noted above, empirical thinking occurs naturally in software development
activities.  There is also a well-established research community focussing
on the promulgation and advancement of empirical techniques in software
development.  Organizations such as the International Software Engineering
Research Network (ISERN), journals such as Empirical Software Engineering,
conferences such as the International Symposium on Empirical Software
Engineering and Measurement, and workshops such as Empirical Studies of
Programmers all provide potential forums for the role of empirical thinking
in software development.

Unfortunately, review of these forums indicates that the role of empirical
thinking in the software development curriculum receives relatively little
attention.  For example, in the 328 articles published in Empirical
Software Engineering from 2002 to 2008, we found only one article that
focused explicitly on the use of empirical techniques as part of software
development pedagogy \cite{Pfahl03}.  In the other articles, we found that
while students are frequently employed in empirical research, the goal of
the participation is to support the testing of a research hypothesis
unrelated to classroom pedagogy.  For example, \cite{Babar08} used students
to support a controlled study of the differences between distributed and
face-to-face meetings for software architecture evaluation.
\cite{Carver06} used students as subjects to test an approach for helping
novice programmers learn software inspection techniques more quickly.
\cite{Host00} performed a study in which students were used to determine if
the data collected from students differs from data collected from
professionals.  Of course, our purpose is not to criticize the use of
students as subjects for empirical research, but rather to point out that,
at least in the case of the Journal of Empirical Software Engineering,
there is little attention on teaching students empirical techniques.

Of course, it may be that this journal is not the appropriate venue for
such efforts.  Unfortunately, review of other forums yields similar
results.  The Empirical Studies of Programmers workshop series focused on
the use of empirical technique to understand programmer behavior, rather
than teaching students how to use empirical thinking to affect their own
behavior.  The Empirical Software Engineering and Measurement conference
series has provided a forum for a variety of applications of empirical
techniques, including those for testing and analysis, coordination and
communication, estimation, modeling and architecture, inspections, defect
classification, and fault-prone module prediction.  The only paper we could
find from the ESEM conference series that focused explicitly on the
introduction of empirical techniques into the classroom setting was our own
\citep{csdl2-03-12}.

\subsection*{Research in the software engineering education community}

The previous section demonstrates that the empirical software engineering community does
not focus on the pedagogy of empirical thinking.  What if we instead look at the 
software engineering educational community: do they focus on empirical techniques? 
The IEEE Conference on Software Engineering Education and Training is the premier annual
conference in this area.  












\subsection*{Current State}

{\em Provide a current assessment of undergraduate education in the relevant participating organizations.  Describe prior pilot programs or planning activities conducted to date, if any, and their outcomes.  Where appropriate, provide institutional data to document the current environment by uploading data into the Supplementary Docs section in FastLane.}

\subsection*{Implementation Plan}

{\em Describe in detail the CT-centric activities to be undertaken to realize the project vision, goals, objectives and anticipated outcomes. 

Define, or describe how the proposing team will attempt to define, the core computing concepts, methods, technologies and tools to be integrated into promising new undergraduate education models.  Describe your plans to identify and implement effective strategies to develop and assess CT competencies in the relevant learning communities.  Identify the stakeholder cohort, e.g. K-20 administrators, faculty, teachers, students, etc., that will participate in and/or benefit from the activities. If relevant, describe how change will be effected and sustained in the participating organizations.  

Describe project milestones in the context of a project timeline and identify responsible parties and expected outcomes for each milestone.  Summarize this information in a figure that you upload into the Supplementary Docs section in FastLane. 

Describe how project outputs and outcomes will be disseminated to the relevant stakeholder groups and to the national community and if relevant, how project resources will be made available to others to adopt or adapt. Identify proactive measures to find and support adopters of promising models and/or practices. Describe plans for outreach to other groups or interested institutions that will take place during the project.  
}

\subsection*{Collaboration and Management Plan}

{\em Provide a collaboration and management plan that will guide project implementation.  Describe how the project leadership team will form, orient, manage, and reinforce relationships in the project.  Provide evidence of the commitment of the participating organizations to effect and sustain the anticipated project outcomes; letters of collaborative support should be uploaded into the Supplementary Docs section in FastLane. }

\subsection*{Evaluation Plan}

{\em Provide an evaluation plan that will inform the project progress and measure its impact.  Include a description of the instruments/metrics used to measure, document, and report on the project's progress.  Identify the evaluator who will be responsible for the evaluation component and discuss their expertise related to the evaluation as well as any other linkages to the project or organizations involved.  }



%% Letters of support
%%  Referentia
%%  Blanca Polo (community college level)
%%  Gail Kaiser 
%%  Vic Basili/UMD
%%  Pekka
%%  Hakan
%%  Laurie Williams
%%  Barry Boehm
%%  Dan Port (ITM) 
%%  Brian Pentland 
%%  The friend of Brian who visited apache foundation, etc. 
%%  Claes
%%  Morisio

%%  Contents of letter: importance of empirical thinking in software engineering; soundness of this research approach; potential interest in collaboration; 


