%%%%%%%%%%%%%%%%%%%%%%%%%%%%%% -*- Mode: Latex -*- %%%%%%%%%%%%%%%%%%%%%%%%%%%%
%% paper.tex -- 
%% Author          : Robert Brewer
%% Created On      : Sun Nov  7 16:04:20 1999
%% Last Modified By: Robert Brewer
%% Last Modified On: Tue Mar  7 10:54:01 2000
%% RCS: $Id: paper.tex,v 1.2 2000/01/25 01:16:30 rbrewer Exp rbrewer $
%%%%%%%%%%%%%%%%%%%%%%%%%%%%%%%%%%%%%%%%%%%%%%%%%%%%%%%%%%%%%%%%%%%%%%%%%%%%%%%
%%   Copyright (C) 1999 Robert Brewer
%%%%%%%%%%%%%%%%%%%%%%%%%%%%%%%%%%%%%%%%%%%%%%%%%%%%%%%%%%%%%%%%%%%%%%%%%%%%%%%
%% 

\documentstyle[twocolumn,icse2000,times]{article}

% Needed for PS figures in old style LaTeX
% Psfig/TeX 
\def\PsfigVersion{1.9}
% dvips version
%
% All psfig/tex software, documentation, and related files
% in this distribution of psfig/tex are 
% Copyright 1987, 1988, 1991 Trevor J. Darrell
%
% Permission is granted for use and non-profit distribution of psfig/tex 
% providing that this notice is clearly maintained. The right to
% distribute any portion of psfig/tex for profit or as part of any commercial
% product is specifically reserved for the author(s) of that portion.
%
% *** Feel free to make local modifications of psfig as you wish,
% *** but DO NOT post any changed or modified versions of ``psfig''
% *** directly to the net. Send them to me and I'll try to incorporate
% *** them into future versions. If you want to take the psfig code 
% *** and make a new program (subject to the copyright above), distribute it, 
% *** (and maintain it) that's fine, just don't call it psfig.
%
% Bugs and improvements to trevor@media.mit.edu.
%
% Thanks to Greg Hager (GDH) and Ned Batchelder for their contributions
% to the original version of this project.
%
% Modified by J. Daniel Smith on 9 October 1990 to accept the
% %%BoundingBox: comment with or without a space after the colon.  Stole
% file reading code from Tom Rokicki's EPSF.TEX file (see below).
%
% More modifications by J. Daniel Smith on 29 March 1991 to allow the
% the included PostScript figure to be rotated.  The amount of
% rotation is specified by the "angle=" parameter of the \psfig command.
%
% Modified by Robert Russell on June 25, 1991 to allow users to specify
% .ps filenames which don't yet exist, provided they explicitly provide
% boundingbox information via the \psfig command. Note: This will only work
% if the "file=" parameter follows all four "bb???=" parameters in the
% command. This is due to the order in which psfig interprets these params.
%
%  3 Jul 1991	JDS	check if file already read in once
%  4 Sep 1991	JDS	fixed incorrect computation of rotated
%			bounding box
% 25 Sep 1991	GVR	expanded synopsis of \psfig
% 14 Oct 1991	JDS	\fbox code from LaTeX so \psdraft works with TeX
%			changed \typeout to \ps@typeout
% 17 Oct 1991	JDS	added \psscalefirst and \psrotatefirst
%

% From: gvr@cs.brown.edu (George V. Reilly)
%
% \psdraft	draws an outline box, but doesn't include the figure
%		in the DVI file.  Useful for previewing.
%
% \psfull	includes the figure in the DVI file (default).
%
% \psscalefirst width= or height= specifies the size of the figure
% 		before rotation.
% \psrotatefirst (default) width= or height= specifies the size of the
% 		 figure after rotation.  Asymetric figures will
% 		 appear to shrink.
%
% \psfigurepath#1	sets the path to search for the figure
%
% \psfig
% usage: \psfig{file=, figure=, height=, width=,
%			bbllx=, bblly=, bburx=, bbury=,
%			rheight=, rwidth=, clip=, angle=, silent=}
%
%	"file" is the filename.  If no path name is specified and the
%		file is not found in the current directory,
%		it will be looked for in directory \psfigurepath.
%	"figure" is a synonym for "file".
%	By default, the width and height of the figure are taken from
%		the BoundingBox of the figure.
%	If "width" is specified, the figure is scaled so that it has
%		the specified width.  Its height changes proportionately.
%	If "height" is specified, the figure is scaled so that it has
%		the specified height.  Its width changes proportionately.
%	If both "width" and "height" are specified, the figure is scaled
%		anamorphically.
%	"bbllx", "bblly", "bburx", and "bbury" control the PostScript
%		BoundingBox.  If these four values are specified
%               *before* the "file" option, the PSFIG will not try to
%               open the PostScript file.
%	"rheight" and "rwidth" are the reserved height and width
%		of the figure, i.e., how big TeX actually thinks
%		the figure is.  They default to "width" and "height".
%	The "clip" option ensures that no portion of the figure will
%		appear outside its BoundingBox.  "clip=" is a switch and
%		takes no value, but the `=' must be present.
%	The "angle" option specifies the angle of rotation (degrees, ccw).
%	The "silent" option makes \psfig work silently.
%

% check to see if macros already loaded in (maybe some other file says
% "\input psfig") ...
\ifx\undefined\psfig\else\endinput\fi

%
% from a suggestion by eijkhout@csrd.uiuc.edu to allow
% loading as a style file. Changed to avoid problems
% with amstex per suggestion by jbence@math.ucla.edu

\let\LaTeXAtSign=\@
\let\@=\relax
\edef\psfigRestoreAt{\catcode`\@=\number\catcode`@\relax}
%\edef\psfigRestoreAt{\catcode`@=\number\catcode`@\relax}
\catcode`\@=11\relax
\newwrite\@unused
\def\ps@typeout#1{{\let\protect\string\immediate\write\@unused{#1}}}
\ps@typeout{psfig/tex \PsfigVersion}

%% Here's how you define your figure path.  Should be set up with null
%% default and a user useable definition.

\def\figurepath{./}
\def\psfigurepath#1{\edef\figurepath{#1}}

%
% @psdo control structure -- similar to Latex @for.
% I redefined these with different names so that psfig can
% be used with TeX as well as LaTeX, and so that it will not 
% be vunerable to future changes in LaTeX's internal
% control structure,
%
\def\@nnil{\@nil}
\def\@empty{}
\def\@psdonoop#1\@@#2#3{}
\def\@psdo#1:=#2\do#3{\edef\@psdotmp{#2}\ifx\@psdotmp\@empty \else
    \expandafter\@psdoloop#2,\@nil,\@nil\@@#1{#3}\fi}
\def\@psdoloop#1,#2,#3\@@#4#5{\def#4{#1}\ifx #4\@nnil \else
       #5\def#4{#2}\ifx #4\@nnil \else#5\@ipsdoloop #3\@@#4{#5}\fi\fi}
\def\@ipsdoloop#1,#2\@@#3#4{\def#3{#1}\ifx #3\@nnil 
       \let\@nextwhile=\@psdonoop \else
      #4\relax\let\@nextwhile=\@ipsdoloop\fi\@nextwhile#2\@@#3{#4}}
\def\@tpsdo#1:=#2\do#3{\xdef\@psdotmp{#2}\ifx\@psdotmp\@empty \else
    \@tpsdoloop#2\@nil\@nil\@@#1{#3}\fi}
\def\@tpsdoloop#1#2\@@#3#4{\def#3{#1}\ifx #3\@nnil 
       \let\@nextwhile=\@psdonoop \else
      #4\relax\let\@nextwhile=\@tpsdoloop\fi\@nextwhile#2\@@#3{#4}}
% 
% \fbox is defined in latex.tex; so if \fbox is undefined, assume that
% we are not in LaTeX.
% Perhaps this could be done better???
\ifx\undefined\fbox
% \fbox code from modified slightly from LaTeX
\newdimen\fboxrule
\newdimen\fboxsep
\newdimen\ps@tempdima
\newbox\ps@tempboxa
\fboxsep = 3pt
\fboxrule = .4pt
\long\def\fbox#1{\leavevmode\setbox\ps@tempboxa\hbox{#1}\ps@tempdima\fboxrule
    \advance\ps@tempdima \fboxsep \advance\ps@tempdima \dp\ps@tempboxa
   \hbox{\lower \ps@tempdima\hbox
  {\vbox{\hrule height \fboxrule
          \hbox{\vrule width \fboxrule \hskip\fboxsep
          \vbox{\vskip\fboxsep \box\ps@tempboxa\vskip\fboxsep}\hskip 
                 \fboxsep\vrule width \fboxrule}
                 \hrule height \fboxrule}}}}
\fi
%
%%%%%%%%%%%%%%%%%%%%%%%%%%%%%%%%%%%%%%%%%%%%%%%%%%%%%%%%%%%%%%%%%%%
% file reading stuff from epsf.tex
%   EPSF.TEX macro file:
%   Written by Tomas Rokicki of Radical Eye Software, 29 Mar 1989.
%   Revised by Don Knuth, 3 Jan 1990.
%   Revised by Tomas Rokicki to accept bounding boxes with no
%      space after the colon, 18 Jul 1990.
%   Portions modified/removed for use in PSFIG package by
%      J. Daniel Smith, 9 October 1990.
%
\newread\ps@stream
\newif\ifnot@eof       % continue looking for the bounding box?
\newif\if@noisy        % report what you're making?
\newif\if@atend        % %%BoundingBox: has (at end) specification
\newif\if@psfile       % does this look like a PostScript file?
%
% PostScript files should start with `%!'
%
{\catcode`\%=12\global\gdef\epsf@start{%!}}
\def\epsf@PS{PS}
%
\def\epsf@getbb#1{%
%
%   The first thing we need to do is to open the
%   PostScript file, if possible.
%
\openin\ps@stream=#1
\ifeof\ps@stream\ps@typeout{Error, File #1 not found}\else
%
%   Okay, we got it. Now we'll scan lines until we find one that doesn't
%   start with %. We're looking for the bounding box comment.
%
   {\not@eoftrue \chardef\other=12
    \def\do##1{\catcode`##1=\other}\dospecials \catcode`\ =10
    \loop
       \if@psfile
	  \read\ps@stream to \epsf@fileline
       \else{
	  \obeyspaces
          \read\ps@stream to \epsf@tmp\global\let\epsf@fileline\epsf@tmp}
       \fi
       \ifeof\ps@stream\not@eoffalse\else
%
%   Check the first line for `%!'.  Issue a warning message if its not
%   there, since the file might not be a PostScript file.
%
       \if@psfile\else
       \expandafter\epsf@test\epsf@fileline:. \\%
       \fi
%
%   We check to see if the first character is a % sign;
%   if so, we look further and stop only if the line begins with
%   `%%BoundingBox:' and the `(atend)' specification was not found.
%   That is, the only way to stop is when the end of file is reached,
%   or a `%%BoundingBox: llx lly urx ury' line is found.
%
          \expandafter\epsf@aux\epsf@fileline:. \\%
       \fi
   \ifnot@eof\repeat
   }\closein\ps@stream\fi}%
%
% This tests if the file we are reading looks like a PostScript file.
%
\long\def\epsf@test#1#2#3:#4\\{\def\epsf@testit{#1#2}
			\ifx\epsf@testit\epsf@start\else
\ps@typeout{Warning! File does not start with `\epsf@start'.  It may not be a PostScript file.}
			\fi
			\@psfiletrue} % don't test after 1st line
%
%   We still need to define the tricky \epsf@aux macro. This requires
%   a couple of magic constants for comparison purposes.
%
{\catcode`\%=12\global\let\epsf@percent=%\global\def\epsf@bblit{%BoundingBox}}
%
%
%   So we're ready to check for `%BoundingBox:' and to grab the
%   values if they are found.  We continue searching if `(at end)'
%   was found after the `%BoundingBox:'.
%
\long\def\epsf@aux#1#2:#3\\{\ifx#1\epsf@percent
   \def\epsf@testit{#2}\ifx\epsf@testit\epsf@bblit
	\@atendfalse
        \epsf@atend #3 . \\%
	\if@atend	
	   \if@verbose{
		\ps@typeout{psfig: found `(atend)'; continuing search}
	   }\fi
        \else
        \epsf@grab #3 . . . \\%
        \not@eoffalse
        \global\no@bbfalse
        \fi
   \fi\fi}%
%
%   Here we grab the values and stuff them in the appropriate definitions.
%
\def\epsf@grab #1 #2 #3 #4 #5\\{%
   \global\def\epsf@llx{#1}\ifx\epsf@llx\empty
      \epsf@grab #2 #3 #4 #5 .\\\else
   \global\def\epsf@lly{#2}%
   \global\def\epsf@urx{#3}\global\def\epsf@ury{#4}\fi}%
%
% Determine if the stuff following the %%BoundingBox is `(atend)'
% J. Daniel Smith.  Copied from \epsf@grab above.
%
\def\epsf@atendlit{(atend)} 
\def\epsf@atend #1 #2 #3\\{%
   \def\epsf@tmp{#1}\ifx\epsf@tmp\empty
      \epsf@atend #2 #3 .\\\else
   \ifx\epsf@tmp\epsf@atendlit\@atendtrue\fi\fi}


% End of file reading stuff from epsf.tex
%%%%%%%%%%%%%%%%%%%%%%%%%%%%%%%%%%%%%%%%%%%%%%%%%%%%%%%%%%%%%%%%%%%

%%%%%%%%%%%%%%%%%%%%%%%%%%%%%%%%%%%%%%%%%%%%%%%%%%%%%%%%%%%%%%%%%%%
% trigonometry stuff from "trig.tex"
\chardef\psletter = 11 % won't conflict with \begin{letter} now...
\chardef\other = 12

\newif \ifdebug %%% turn me on to see TeX hard at work ...
\newif\ifc@mpute %%% don't need to compute some values
\c@mputetrue % but assume that we do

\let\then = \relax
\def\r@dian{pt }
\let\r@dians = \r@dian
\let\dimensionless@nit = \r@dian
\let\dimensionless@nits = \dimensionless@nit
\def\internal@nit{sp }
\let\internal@nits = \internal@nit
\newif\ifstillc@nverging
\def \Mess@ge #1{\ifdebug \then \message {#1} \fi}

{ %%% Things that need abnormal catcodes %%%
	\catcode `\@ = \psletter
	\gdef \nodimen {\expandafter \n@dimen \the \dimen}
	\gdef \term #1 #2 #3%
	       {\edef \t@ {\the #1}%%% freeze parameter 1 (count, by value)
		\edef \t@@ {\expandafter \n@dimen \the #2\r@dian}%
				   %%% freeze parameter 2 (dimen, by value)
		\t@rm {\t@} {\t@@} {#3}%
	       }
	\gdef \t@rm #1 #2 #3%
	       {{%
		\count 0 = 0
		\dimen 0 = 1 \dimensionless@nit
		\dimen 2 = #2\relax
		\Mess@ge {Calculating term #1 of \nodimen 2}%
		\loop
		\ifnum	\count 0 < #1
		\then	\advance \count 0 by 1
			\Mess@ge {Iteration \the \count 0 \space}%
			\Multiply \dimen 0 by {\dimen 2}%
			\Mess@ge {After multiplication, term = \nodimen 0}%
			\Divide \dimen 0 by {\count 0}%
			\Mess@ge {After division, term = \nodimen 0}%
		\repeat
		\Mess@ge {Final value for term #1 of 
				\nodimen 2 \space is \nodimen 0}%
		\xdef \Term {#3 = \nodimen 0 \r@dians}%
		\aftergroup \Term
	       }}
	\catcode `\p = \other
	\catcode `\t = \other
	\gdef \n@dimen #1pt{#1} %%% throw away the ``pt''
}

\def \Divide #1by #2{\divide #1 by #2} %%% just a synonym

\def \Multiply #1by #2%%% allows division of a dimen by a dimen
       {{%%% should really freeze parameter 2 (dimen, passed by value)
	\count 0 = #1\relax
	\count 2 = #2\relax
	\count 4 = 65536
	\Mess@ge {Before scaling, count 0 = \the \count 0 \space and
			count 2 = \the \count 2}%
	\ifnum	\count 0 > 32767 %%% do our best to avoid overflow
	\then	\divide \count 0 by 4
		\divide \count 4 by 4
	\else	\ifnum	\count 0 < -32767
		\then	\divide \count 0 by 4
			\divide \count 4 by 4
		\else
		\fi
	\fi
	\ifnum	\count 2 > 32767 %%% while retaining reasonable accuracy
	\then	\divide \count 2 by 4
		\divide \count 4 by 4
	\else	\ifnum	\count 2 < -32767
		\then	\divide \count 2 by 4
			\divide \count 4 by 4
		\else
		\fi
	\fi
	\multiply \count 0 by \count 2
	\divide \count 0 by \count 4
	\xdef \product {#1 = \the \count 0 \internal@nits}%
	\aftergroup \product
       }}

\def\r@duce{\ifdim\dimen0 > 90\r@dian \then   % sin(x+90) = sin(180-x)
		\multiply\dimen0 by -1
		\advance\dimen0 by 180\r@dian
		\r@duce
	    \else \ifdim\dimen0 < -90\r@dian \then  % sin(-x) = sin(360+x)
		\advance\dimen0 by 360\r@dian
		\r@duce
		\fi
	    \fi}

\def\Sine#1%
       {{%
	\dimen 0 = #1 \r@dian
	\r@duce
	\ifdim\dimen0 = -90\r@dian \then
	   \dimen4 = -1\r@dian
	   \c@mputefalse
	\fi
	\ifdim\dimen0 = 90\r@dian \then
	   \dimen4 = 1\r@dian
	   \c@mputefalse
	\fi
	\ifdim\dimen0 = 0\r@dian \then
	   \dimen4 = 0\r@dian
	   \c@mputefalse
	\fi
%
	\ifc@mpute \then
        	% convert degrees to radians
		\divide\dimen0 by 180
		\dimen0=3.141592654\dimen0
%
		\dimen 2 = 3.1415926535897963\r@dian %%% a well-known constant
		\divide\dimen 2 by 2 %%% we only deal with -pi/2 : pi/2
		\Mess@ge {Sin: calculating Sin of \nodimen 0}%
		\count 0 = 1 %%% see power-series expansion for sine
		\dimen 2 = 1 \r@dian %%% ditto
		\dimen 4 = 0 \r@dian %%% ditto
		\loop
			\ifnum	\dimen 2 = 0 %%% then we've done
			\then	\stillc@nvergingfalse 
			\else	\stillc@nvergingtrue
			\fi
			\ifstillc@nverging %%% then calculate next term
			\then	\term {\count 0} {\dimen 0} {\dimen 2}%
				\advance \count 0 by 2
				\count 2 = \count 0
				\divide \count 2 by 2
				\ifodd	\count 2 %%% signs alternate
				\then	\advance \dimen 4 by \dimen 2
				\else	\advance \dimen 4 by -\dimen 2
				\fi
		\repeat
	\fi		
			\xdef \sine {\nodimen 4}%
       }}

% Now the Cosine can be calculated easily by calling \Sine
\def\Cosine#1{\ifx\sine\UnDefined\edef\Savesine{\relax}\else
		             \edef\Savesine{\sine}\fi
	{\dimen0=#1\r@dian\advance\dimen0 by 90\r@dian
	 \Sine{\nodimen 0}
	 \xdef\cosine{\sine}
	 \xdef\sine{\Savesine}}}	      
% end of trig stuff
%%%%%%%%%%%%%%%%%%%%%%%%%%%%%%%%%%%%%%%%%%%%%%%%%%%%%%%%%%%%%%%%%%%%

\def\psdraft{
	\def\@psdraft{0}
	%\ps@typeout{draft level now is \@psdraft \space . }
}
\def\psfull{
	\def\@psdraft{100}
	%\ps@typeout{draft level now is \@psdraft \space . }
}

\psfull

\newif\if@scalefirst
\def\psscalefirst{\@scalefirsttrue}
\def\psrotatefirst{\@scalefirstfalse}
\psrotatefirst

\newif\if@draftbox
\def\psnodraftbox{
	\@draftboxfalse
}
\def\psdraftbox{
	\@draftboxtrue
}
\@draftboxtrue

\newif\if@prologfile
\newif\if@postlogfile
\def\pssilent{
	\@noisyfalse
}
\def\psnoisy{
	\@noisytrue
}
\psnoisy
%%% These are for the option list.
%%% A specification of the form a = b maps to calling \@p@@sa{b}
\newif\if@bbllx
\newif\if@bblly
\newif\if@bburx
\newif\if@bbury
\newif\if@height
\newif\if@width
\newif\if@rheight
\newif\if@rwidth
\newif\if@angle
\newif\if@clip
\newif\if@verbose
\def\@p@@sclip#1{\@cliptrue}


\newif\if@decmpr

%%% GDH 7/26/87 -- changed so that it first looks in the local directory,
%%% then in a specified global directory for the ps file.
%%% RPR 6/25/91 -- changed so that it defaults to user-supplied name if
%%% boundingbox info is specified, assuming graphic will be created by
%%% print time.
%%% TJD 10/19/91 -- added bbfile vs. file distinction, and @decmpr flag

\def\@p@@sfigure#1{\def\@p@sfile{null}\def\@p@sbbfile{null}
	        \openin1=#1.bb
		\ifeof1\closein1
	        	\openin1=\figurepath#1.bb
			\ifeof1\closein1
			        \openin1=#1
				\ifeof1\closein1%
				       \openin1=\figurepath#1
					\ifeof1
					   \ps@typeout{Error, File #1 not found}
						\if@bbllx\if@bblly
				   		\if@bburx\if@bbury
			      				\def\@p@sfile{#1}%
			      				\def\@p@sbbfile{#1}%
							\@decmprfalse
				  	   	\fi\fi\fi\fi
					\else\closein1
				    		\def\@p@sfile{\figurepath#1}%
				    		\def\@p@sbbfile{\figurepath#1}%
						\@decmprfalse
	                       		\fi%
			 	\else\closein1%
					\def\@p@sfile{#1}
					\def\@p@sbbfile{#1}
					\@decmprfalse
			 	\fi
			\else
				\def\@p@sfile{\figurepath#1}
				\def\@p@sbbfile{\figurepath#1.bb}
				\@decmprtrue
			\fi
		\else
			\def\@p@sfile{#1}
			\def\@p@sbbfile{#1.bb}
			\@decmprtrue
		\fi}

\def\@p@@sfile#1{\@p@@sfigure{#1}}

\def\@p@@sbbllx#1{
		%\ps@typeout{bbllx is #1}
		\@bbllxtrue
		\dimen100=#1
		\edef\@p@sbbllx{\number\dimen100}
}
\def\@p@@sbblly#1{
		%\ps@typeout{bblly is #1}
		\@bbllytrue
		\dimen100=#1
		\edef\@p@sbblly{\number\dimen100}
}
\def\@p@@sbburx#1{
		%\ps@typeout{bburx is #1}
		\@bburxtrue
		\dimen100=#1
		\edef\@p@sbburx{\number\dimen100}
}
\def\@p@@sbbury#1{
		%\ps@typeout{bbury is #1}
		\@bburytrue
		\dimen100=#1
		\edef\@p@sbbury{\number\dimen100}
}
\def\@p@@sheight#1{
		\@heighttrue
		\dimen100=#1
   		\edef\@p@sheight{\number\dimen100}
		%\ps@typeout{Height is \@p@sheight}
}
\def\@p@@swidth#1{
		%\ps@typeout{Width is #1}
		\@widthtrue
		\dimen100=#1
		\edef\@p@swidth{\number\dimen100}
}
\def\@p@@srheight#1{
		%\ps@typeout{Reserved height is #1}
		\@rheighttrue
		\dimen100=#1
		\edef\@p@srheight{\number\dimen100}
}
\def\@p@@srwidth#1{
		%\ps@typeout{Reserved width is #1}
		\@rwidthtrue
		\dimen100=#1
		\edef\@p@srwidth{\number\dimen100}
}
\def\@p@@sangle#1{
		%\ps@typeout{Rotation is #1}
		\@angletrue
%		\dimen100=#1
		\edef\@p@sangle{#1} %\number\dimen100}
}
\def\@p@@ssilent#1{ 
		\@verbosefalse
}
\def\@p@@sprolog#1{\@prologfiletrue\def\@prologfileval{#1}}
\def\@p@@spostlog#1{\@postlogfiletrue\def\@postlogfileval{#1}}
\def\@cs@name#1{\csname #1\endcsname}
\def\@setparms#1=#2,{\@cs@name{@p@@s#1}{#2}}
%
% initialize the defaults (size the size of the figure)
%
\def\ps@init@parms{
		\@bbllxfalse \@bbllyfalse
		\@bburxfalse \@bburyfalse
		\@heightfalse \@widthfalse
		\@rheightfalse \@rwidthfalse
		\def\@p@sbbllx{}\def\@p@sbblly{}
		\def\@p@sbburx{}\def\@p@sbbury{}
		\def\@p@sheight{}\def\@p@swidth{}
		\def\@p@srheight{}\def\@p@srwidth{}
		\def\@p@sangle{0}
		\def\@p@sfile{} \def\@p@sbbfile{}
		\def\@p@scost{10}
		\def\@sc{}
		\@prologfilefalse
		\@postlogfilefalse
		\@clipfalse
		\if@noisy
			\@verbosetrue
		\else
			\@verbosefalse
		\fi
}
%
% Go through the options setting things up.
%
\def\parse@ps@parms#1{
	 	\@psdo\@psfiga:=#1\do
		   {\expandafter\@setparms\@psfiga,}}
%
% Compute bb height and width
%
\newif\ifno@bb
\def\bb@missing{
	\if@verbose{
		\ps@typeout{psfig: searching \@p@sbbfile \space  for bounding box}
	}\fi
	\no@bbtrue
	\epsf@getbb{\@p@sbbfile}
        \ifno@bb \else \bb@cull\epsf@llx\epsf@lly\epsf@urx\epsf@ury\fi
}	
\def\bb@cull#1#2#3#4{
	\dimen100=#1 bp\edef\@p@sbbllx{\number\dimen100}
	\dimen100=#2 bp\edef\@p@sbblly{\number\dimen100}
	\dimen100=#3 bp\edef\@p@sbburx{\number\dimen100}
	\dimen100=#4 bp\edef\@p@sbbury{\number\dimen100}
	\no@bbfalse
}
% rotate point (#1,#2) about (0,0).
% The sine and cosine of the angle are already stored in \sine and
% \cosine.  The result is placed in (\p@intvaluex, \p@intvaluey).
\newdimen\p@intvaluex
\newdimen\p@intvaluey
\def\rotate@#1#2{{\dimen0=#1 sp\dimen1=#2 sp
%            	calculate x' = x \cos\theta - y \sin\theta
		  \global\p@intvaluex=\cosine\dimen0
		  \dimen3=\sine\dimen1
		  \global\advance\p@intvaluex by -\dimen3
% 		calculate y' = x \sin\theta + y \cos\theta
		  \global\p@intvaluey=\sine\dimen0
		  \dimen3=\cosine\dimen1
		  \global\advance\p@intvaluey by \dimen3
		  }}
\def\compute@bb{
		\no@bbfalse
		\if@bbllx \else \no@bbtrue \fi
		\if@bblly \else \no@bbtrue \fi
		\if@bburx \else \no@bbtrue \fi
		\if@bbury \else \no@bbtrue \fi
		\ifno@bb \bb@missing \fi
		\ifno@bb \ps@typeout{FATAL ERROR: no bb supplied or found}
			\no-bb-error
		\fi
		%
%\ps@typeout{BB: \@p@sbbllx, \@p@sbblly, \@p@sbburx, \@p@sbbury} 
%
% store height/width of original (unrotated) bounding box
		\count203=\@p@sbburx
		\count204=\@p@sbbury
		\advance\count203 by -\@p@sbbllx
		\advance\count204 by -\@p@sbblly
		\edef\ps@bbw{\number\count203}
		\edef\ps@bbh{\number\count204}
		%\ps@typeout{ psbbh = \ps@bbh, psbbw = \ps@bbw }
		\if@angle 
			\Sine{\@p@sangle}\Cosine{\@p@sangle}
	        	{\dimen100=\maxdimen\xdef\r@p@sbbllx{\number\dimen100}
					    \xdef\r@p@sbblly{\number\dimen100}
			                    \xdef\r@p@sbburx{-\number\dimen100}
					    \xdef\r@p@sbbury{-\number\dimen100}}
%
% Need to rotate all four points and take the X-Y extremes of the new
% points as the new bounding box.
                        \def\minmaxtest{
			   \ifnum\number\p@intvaluex<\r@p@sbbllx
			      \xdef\r@p@sbbllx{\number\p@intvaluex}\fi
			   \ifnum\number\p@intvaluex>\r@p@sbburx
			      \xdef\r@p@sbburx{\number\p@intvaluex}\fi
			   \ifnum\number\p@intvaluey<\r@p@sbblly
			      \xdef\r@p@sbblly{\number\p@intvaluey}\fi
			   \ifnum\number\p@intvaluey>\r@p@sbbury
			      \xdef\r@p@sbbury{\number\p@intvaluey}\fi
			   }
%			lower left
			\rotate@{\@p@sbbllx}{\@p@sbblly}
			\minmaxtest
%			upper left
			\rotate@{\@p@sbbllx}{\@p@sbbury}
			\minmaxtest
%			lower right
			\rotate@{\@p@sbburx}{\@p@sbblly}
			\minmaxtest
%			upper right
			\rotate@{\@p@sbburx}{\@p@sbbury}
			\minmaxtest
			\edef\@p@sbbllx{\r@p@sbbllx}\edef\@p@sbblly{\r@p@sbblly}
			\edef\@p@sbburx{\r@p@sbburx}\edef\@p@sbbury{\r@p@sbbury}
%\ps@typeout{rotated BB: \r@p@sbbllx, \r@p@sbblly, \r@p@sbburx, \r@p@sbbury}
		\fi
		\count203=\@p@sbburx
		\count204=\@p@sbbury
		\advance\count203 by -\@p@sbbllx
		\advance\count204 by -\@p@sbblly
		\edef\@bbw{\number\count203}
		\edef\@bbh{\number\count204}
		%\ps@typeout{ bbh = \@bbh, bbw = \@bbw }
}
%
% \in@hundreds performs #1 * (#2 / #3) correct to the hundreds,
%	then leaves the result in @result
%
\def\in@hundreds#1#2#3{\count240=#2 \count241=#3
		     \count100=\count240	% 100 is first digit #2/#3
		     \divide\count100 by \count241
		     \count101=\count100
		     \multiply\count101 by \count241
		     \advance\count240 by -\count101
		     \multiply\count240 by 10
		     \count101=\count240	%101 is second digit of #2/#3
		     \divide\count101 by \count241
		     \count102=\count101
		     \multiply\count102 by \count241
		     \advance\count240 by -\count102
		     \multiply\count240 by 10
		     \count102=\count240	% 102 is the third digit
		     \divide\count102 by \count241
		     \count200=#1\count205=0
		     \count201=\count200
			\multiply\count201 by \count100
		 	\advance\count205 by \count201
		     \count201=\count200
			\divide\count201 by 10
			\multiply\count201 by \count101
			\advance\count205 by \count201
			%
		     \count201=\count200
			\divide\count201 by 100
			\multiply\count201 by \count102
			\advance\count205 by \count201
			%
		     \edef\@result{\number\count205}
}
\def\compute@wfromh{
		% computing : width = height * (bbw / bbh)
		\in@hundreds{\@p@sheight}{\@bbw}{\@bbh}
		%\ps@typeout{ \@p@sheight * \@bbw / \@bbh, = \@result }
		\edef\@p@swidth{\@result}
		%\ps@typeout{w from h: width is \@p@swidth}
}
\def\compute@hfromw{
		% computing : height = width * (bbh / bbw)
	        \in@hundreds{\@p@swidth}{\@bbh}{\@bbw}
		%\ps@typeout{ \@p@swidth * \@bbh / \@bbw = \@result }
		\edef\@p@sheight{\@result}
		%\ps@typeout{h from w : height is \@p@sheight}
}
\def\compute@handw{
		\if@height 
			\if@width
			\else
				\compute@wfromh
			\fi
		\else 
			\if@width
				\compute@hfromw
			\else
				\edef\@p@sheight{\@bbh}
				\edef\@p@swidth{\@bbw}
			\fi
		\fi
}
\def\compute@resv{
		\if@rheight \else \edef\@p@srheight{\@p@sheight} \fi
		\if@rwidth \else \edef\@p@srwidth{\@p@swidth} \fi
		%\ps@typeout{rheight = \@p@srheight, rwidth = \@p@srwidth}
}
%		
% Compute any missing values
\def\compute@sizes{
	\compute@bb
	\if@scalefirst\if@angle
% at this point the bounding box has been adjsuted correctly for
% rotation.  PSFIG does all of its scaling using \@bbh and \@bbw.  If
% a width= or height= was specified along with \psscalefirst, then the
% width=/height= value needs to be adjusted to match the new (rotated)
% bounding box size (specifed in \@bbw and \@bbh).
%    \ps@bbw       width=
%    -------  =  ---------- 
%    \@bbw       new width=
% so `new width=' = (width= * \@bbw) / \ps@bbw; where \ps@bbw is the
% width of the original (unrotated) bounding box.
	\if@width
	   \in@hundreds{\@p@swidth}{\@bbw}{\ps@bbw}
	   \edef\@p@swidth{\@result}
	\fi
	\if@height
	   \in@hundreds{\@p@sheight}{\@bbh}{\ps@bbh}
	   \edef\@p@sheight{\@result}
	\fi
	\fi\fi
	\compute@handw
	\compute@resv}

%
% \psfig
% usage : \psfig{file=, height=, width=, bbllx=, bblly=, bburx=, bbury=,
%			rheight=, rwidth=, clip=}
%
% "clip=" is a switch and takes no value, but the `=' must be present.
\def\psfig#1{\vbox {
	% do a zero width hard space so that a single
	% \psfig in a centering enviornment will behave nicely
	%{\setbox0=\hbox{\ }\ \hskip-\wd0}
	%
	\ps@init@parms
	\parse@ps@parms{#1}
	\compute@sizes
	%
	\ifnum\@p@scost<\@psdraft{
		%
		\special{ps::[begin] 	\@p@swidth \space \@p@sheight \space
				\@p@sbbllx \space \@p@sbblly \space
				\@p@sbburx \space \@p@sbbury \space
				startTexFig \space }
		\if@angle
			\special {ps:: \@p@sangle \space rotate \space} 
		\fi
		\if@clip{
			\if@verbose{
				\ps@typeout{(clip)}
			}\fi
			\special{ps:: doclip \space }
		}\fi
		\if@prologfile
		    \special{ps: plotfile \@prologfileval \space } \fi
		\if@decmpr{
			\if@verbose{
				\ps@typeout{psfig: including \@p@sfile.Z \space }
			}\fi
			\special{ps: plotfile "`zcat \@p@sfile.Z" \space }
		}\else{
			\if@verbose{
				\ps@typeout{psfig: including \@p@sfile \space }
			}\fi
			\special{ps: plotfile \@p@sfile \space }
		}\fi
		\if@postlogfile
		    \special{ps: plotfile \@postlogfileval \space } \fi
		\special{ps::[end] endTexFig \space }
		% Create the vbox to reserve the space for the figure.
		\vbox to \@p@srheight sp{
		% 1/92 TJD Changed from "true sp" to "sp" for magnification.
			\hbox to \@p@srwidth sp{
				\hss
			}
		\vss
		}
	}\else{
		% draft figure, just reserve the space and print the
		% path name.
		\if@draftbox{		
			% Verbose draft: print file name in box
			\hbox{\frame{\vbox to \@p@srheight sp{
			\vss
			\hbox to \@p@srwidth sp{ \hss \@p@sfile \hss }
			\vss
			}}}
		}\else{
			% Non-verbose draft
			\vbox to \@p@srheight sp{
			\vss
			\hbox to \@p@srwidth sp{\hss}
			\vss
			}
		}\fi	



	}\fi
}}
\psfigRestoreAt
\let\@=\LaTeXAtSign





% Changes: 
%  Page limit: 10 pages
%  Abstract limit: 200 words 
%  Contact information is now gall@infosys.tuwien.ac.at
%  URL for author kit changed: needs to be filled in.
%  All other font and sizing information is the same:
%    check whether it still applies.

\begin{document}

\title{Improving Problem-Oriented\\Mailing List Archives with MCS}

\author{
        \hspace*{-2ex}
        \parbox{4.0in} {\begin{center}
        {\authornamefont Robert S. Brewer}\\ 
        Collaborative Software Development Laboratory\\
        Department of Information \& Computer Sciences\\
        University of Hawaii, Manoa\\
        Honolulu, Hawaii 96822  USA\\
        (808) 956-6920\\
        rbrewer@lava.net
        \end{center} }
}

\maketitle

% If you want to print drafts of the paper with a draft 
% notice in the copyright space, comment out the \copyrightspace
% line below and include the \submitspace line below instead.

\copyrightspace
%\submitspace{Draft of paper submitted to ICSE 2000.}

% Page number on the initial page can be omitted in both the review and
% final submission (and should be removed in the final submission).  The
% line below does that.

\thispagestyle{empty}  % suppresses page number on first page

% In the review submission, page numbers should appear (they can be omitted
%  from the first page).  The pagestyle command below puts them in.
% In the final submission of accepted papers, page numbers should be
%  omitted; remove or comment out the pagestyle line below to omit them. 

%\pagestyle{plain}

% Use \section* instead of \section to suppress numbering for
% the abstract, acknowledgments, and references.


\section*{ABSTRACT}
Developers often use electronic mailing lists when seeking assistance with a
particular software application. The archives of these mailing lists provide a
rich repository of problem-solving knowledge. Developers seeking a quick answer
to a problem find these archives inconvenient, because they lack efficient
searching mechanisms, and retain the structure of the original conversational
threads which are rarely relevant to the knowledge seeker.

We present a system called MCS which improves mailing list archives through a
process called {\em condensation}. Condensation involves several tasks:
extracting only messages of longer-term relevance, adding metadata to those
messages to improve searching, and potentially editing the content of the
messages when appropriate to clarify. The condensation process is performed by
a human editor (assisted by a tool), rather than by an artificial intelligence
(AI) system.

We describe the design and implementation of MCS, and compare it to related
systems. We also present our experiences condensing a 1428 message mailing list
archive to an archive containing only 177 messages (an 88\% reduction). The
condensation required only 1.5 minutes of editor effort per message. The
condensed archive was adopted by the users of the mailing list.

\subsection{Keywords}
Knowledge condensation, mailing lists, archives, collective memory

\section{INTRODUCTION}
Modern software development is a complicated task. Over the course of a project
a developer may use: a design tool, an editor, a compiler, a debugger, a
regression test system, and a packaging tool. The developer may need to use
third party libraries or application programming interfaces (APIs) from a
variety of sources. On top of all these software development related tools,
developers may be responsible for the installation and maintenance of their
computing environment: operating system, hardware drivers, word processing,
electronic mail, etc. In this kind of complicated technical environment,
problems and questions inevitably arise.

%In the case of commercial products, developers can seek out technical support
%from the publisher. However, commercial support is not always the best venue
%for a variety of reasons: the cost may be high, the publisher may have a vested
%interest in not providing certain information (like defect reports), or the
%level of support desired might simply be unavailable from the publisher. In the
%case of open source products \cite{open-source-website}, often there is no
%central source for support.

Electronic mailing lists have become a common means for users to exchange
information and help each other to solve problems. They can be administered by
the producer of the product, but they are often run by a user of the product
who wants to create a community for the product's users. These mailing lists
often become an essential information source for the product, providing
up-to-the-minute information and wise advice from experienced users.

As useful as mailing lists are, they have problems that limit their usefulness.
A popular mailing list can have tens or hundreds of new messages daily, but
keeping up with that level of traffic is prohibitive for most subscribers.
While there is a lot of valuable knowledge available, it can be buried among a
seemingly endless stream of beginner questions, off-topic discussions, and
sometimes unsolicited advertising. The amount of traffic leads many subscribers
to delete or file away messages from the list without reading them, simply due
to time constraints. When a subscriber has a question, they frequently send it
to the list blindly, without knowing whether the answer was just posted
recently. This further adds to the information glut.

Luckily for developers, there is another way to find solutions to problems: the
mailing list archives. Most mailing lists maintain an archive of all the
messages sent to the list, and usually provide some searching capability. With
the rise of the browser, most archives are made available via a web page with a
search form, such as the Sun archive of the ``jserv-interest'' list for the
discussion of Sun's Java Web Server
(http://archives.java.sun.com/\linebreak[0]archives/\linebreak[0]jserv-interest.html).
These searchable archives provide a way for developers with problems to see if
their problem has already been discussed, and possibly even solved.

Unfortunately, mailing list archives are poorly equipped to support
problem-solving queries. All the irrelevant information that has been sent to
the list is immortalized in the archive, making it difficult to find the useful
information. Searchable archives also face the problem that any particular
query may return an enormous number of hits. For example, a developer looking
for help on how to redirect a web client to different web page with the Java
Web Server might do a search for ``redirect'' on the jserv-interest mailing
list archive. As of this writing, that search returns 175 articles, many of
which are irrelevant to the developer's goal. The search hits are displayed in
chronological order, and this arbitrary ordering doesn't help the developers
find the solutions they are looking for. Another problem with conventional
archives is that a particular question may have been asked and answered many
times with varying levels of accuracy and clarity. A developer might find a
message proposing a solution only to miss the follow up message which explains
how that solution is flawed.

We have developed a method for improving the archives of these kinds of
problem-oriented mailing lists which we call {\em condensation}. Condensation
involves several tasks: extracting only the messages of longer-term relevance,
adding metadata to those messages to improve searching and browsing, and even
editing the content of the messages when appropriate to clarify or provide
context. The condensation process is performed by a human editor (assisted by a
tool), rather than an AI system.

\section{CONDENSATION AS A SOLUTION}
The goal of condensation is to take the voluminous data stream generated by a
mailing list and extract the information which would be useful to future users
of the archive.  As an analogy, newspapers provide a daily report on current
events but are limited by short deadlines, a broad subscriber base, and other
considerations. These considerations prevent them from analyzing which events
are accurate or relevant over the long term. A story published one day might be
amended or retracted the next, depending on how events unfold. However, a book
describing world events will tend to have a longer deadline which permits more
reflection and analysis: a hoax which might occupy weeks of headlines in a
newspaper will probably be little more than a footnote in a book (unless the
book is about newspaper hoaxes). The book can also have an index to enable
readers to jump directly to the information they are interested in. It is this
refinement of information that we refer to as condensation.

There are a variety of ways that the information could be condensed, depending
on the intended use of the archive. Our goal is to provide a searchable archive
of information that allows developers to quickly find solutions to specific
problems. Since the goal is to help developers find solutions as efficiently as
possible, there is little point in preserving the conversational nature of the
mailing list data stream. Users with problems are looking for solutions, not
conversation. Condensation requires omitting unimportant, contradictory, or
inaccurate messages, removing unimportant portions of messages, inserting new
text into messages when required for clarification, and adding new messages to
the database from scratch when that is the best way to explain something. For
this narrow focus on problem solving, we can categorize each message as either
a problem or a solution. Each message is annotated with keywords which are
actually relevant to the subject of the message. The result of this process is
a condensed archive, an archive which does not suffer from the problems of an
unabridged searchable one.  Since only truly useful information will be put
into the archive, the amount of data to be searched is smaller which improves
the odds of a search being accurate. Developers also benefit by having a set of
standardized keywords which insure that messages using different terms but
discussing the same topic will be retrieved by a single search.

\section{MCS: A SYSTEM FOR CONDENSATION}
\label{sec:mcs-system}
To demonstrate the improvements possible through condensation, we have
constructed a new software system for condensing mailing list archives. We have
named this system (for lack of imagination) the Mailinglist Condensation
System or MCS (http://csdl.ics.hawaii.edu/\linebreak[0]Research/MCS/MCS.html).
MCS has two main parts: one which is dedicated to taking the raw material from
the mailing list and condensing it, and another which stores the condensed
messages and allows developers to access them.

One way to perform the condensation would be to implement an AI system that
reads the messages and then decides what information to keep, what to throw
away, and what keywords to assign to each. In order to perform this task
adequately, the system would need superb natural language processing
capabilities and an in-depth knowledge of the mailing list domain. Such a
system is currently at or beyond the state of the art, and would at any rate
require a substantial investment of resources to complete.

A practical alternative to an AI system is the employment of human editors for
condensation, along with extensive tool support to lower editing overhead to an
acceptable level. Humans are quite good at examining textual information and
determining what is useful and what is not, while computers are good at queries
across structured data \cite{Brooks:1996:CST}. This alternative also exploits
the presence of mailing list gurus: subscribers who read all messages sent to
the list and who are domain experts. Therefore, in MCS, humans do the editing
using the MCS editing program which makes the process as efficient as possible.
Only the editors need to use the editing portion; the interface of the end-user
portion is simpler and geared towards ease of use. If AI systems become
available which can replace some or all of the manual effort required by the
editor, they can be added to MCS.

\subsection{Requirements}
MCS was designed to help users of problem-solving mailing lists by improving
the usability of the list archives. Making archives more useful not only helps
the archive users, it also helps to improve the quality of the mailing list
itself, because people are less likely to re-request information which is
easily available via the archive. To achieve this goal, the user community must
adopt MCS in preference to the many existing systems for generating and
maintaining searchable mailing list archives. To encourage users to adopt the
system, the design of MCS takes into account two issues: an explicit domain
focus, and the existing list community.

Most mailing list archive search engines are designed to work with any mailing
list. Because they must work with any mailing list, conventional search engines
are limited to keyword searches and simple search results presentation. The
idea of MCS is the exact opposite: mailing list archives can be enhanced by
tailoring the search engine to a particular mailing list domain. By embracing
the details of a particular kind of mailing list MCS provides greater utility
and efficiency for archive users.

Because MCS receives its input from a mailing list, it is crucial that MCS be
designed with the social structure of a user-supported mailing list in mind.
Specifically, the mailing list and its community should not be adversely
affected by MCS. Any attempt to impose restrictions on how people read or
participate in the list (like requiring users to use special software or
compose messages in a certain format) would be met with blistering criticism.
MCS must stand apart from the mailing list itself, limited to using messages
from the list on an as-is basis. MCS also takes into account the needs of the
user community by having very low requirements accessing the archive. The
archive is accessed using a web browser which is presumably standard equipment
for most mailing list participants. Furthermore, the web pages themselves are
simple; they contain no images, no Java applets, and no JavaScript to ensure
that users can use older browsers to access the archive. The omission and
editing of messages is central to MCS, but those actions can reasonably arouse
suspicion among list members as to the fairness of the editing. To assuage
these fears and to assure context, MCS provides a link from each edited message
to the original message maintained in a separate unabridged archive.

\subsection{Functionality}
In addition to condensation MCS provides several novel features which
facilitate user searching. All examples and screenshots in this section refer
to a condensed archive of the ``jcvs'' mailing list
(http://www.gjt.org/\linebreak[0]servlets/\linebreak[0]MailingLists/\linebreak[0]ListInfo.html/\linebreak[0]jcvs).
The jcvs mailing list exists for the discussion of the jCVS system
(http://www.trustice.com/\linebreak[0]java/\linebreak[0]jcvs/\linebreak[0]index.shtml),
which is a Java client for the Concurrent Versions System (CVS)
(http://www.sourcegear.com/\linebreak[0]CVS). See Section \ref{sec:evaluation}
for more information about the archive.

\subsubsection{Keywords}
Messages in MCS are assigned keywords by the editor. The keywords are chosen
sparingly such that there are only a few for each message, instead of indexing
all the words in each message. These keywords are organized into a hierarchy of
categories by the editor. Each keyword and category can be annotated by a
description and an URL (Uniform Resource Locator) when appropriate. The
maintenance of the keyword hierarchy is a major part of the editor's task.
While maintaining keywords is time consuming, having the keywords organized in
this way provides advantages to the archive user. A frequent problem when using
conventional archives is figuring out what keyword has been used for a
particular concept. For example, ``freeze'', ``hang'', and ``lock-up'' are all
words which describe the same concept, but a user might have to try searching
for all three in order to retrieve all the problems related to that concept.
With the keyword hierarchy, the synonym problem is all but eliminated. Having a
relatively small number of keywords arranged in a tree also allows users to
browse through the keywords and learn what kinds of topics are contained in the
archive. Keywords can be browsed using a system similar to the one used at the
Yahoo web portal (http://www.yahoo.com/).

%\begin{figure*}[htbp]
%  {\centerline{\psfig{figure=figs/keyword-editor.eps}}}
%  \caption{The tool used to edit the keyword hierarchy}
%  \label{fig:keyword-editor}
%\end{figure*}

The grouping of keywords into categories enables another useful option for
users. MCS allows users to perform a {\em 2D search} by performing simultaneous
searches for pairs of keywords. The user selects two categories which contain
keywords, and then initiates the 2D search. MCS performs the cross-product of
the two categories, and for each tuple of keywords it performs an AND search of
the database. The result is a table which shows the co-incidence of the
keywords in the two categories. Figure \ref{fig:2d-search} shows the results of
a 2D search with the categories of ``Java Concepts'' versus ``jCVS Versions''.
The archive this search was performed on has a limited amount of data, but the
results provide some insight as to which concepts have proven problematic with
which software versions. Note that just because the JavaHelp row has no matches
doesn't mean that no messages related to JavaHelp are in the database. It just
means that no JavaHelp-related messages refer to a particular version of jCVS,
probably because the version of jCVS was not relevant to the problem or
solution.

\begin{figure*}[htbp]
  {\centerline{\psfig{figure=figs/2d-search.eps,height=4in}}}
  \caption{A 2D search of Java Concepts vs. jCVS Versions}
  \label{fig:2d-search}
\end{figure*}

\subsubsection{Message Types}
MCS has a very simple schema for the messages it stores. Each message has a
type which currently can is either ``problem'' or ``solution''. This typing has
a profound affect on MCS. When a developer has a problem, they can search the
archive to see if they can find a message describing a similar problem. Once
they find a relevant problem in the archive, they can immediately see what
solutions have been proposed for that problem. Figure \ref{fig:search-results}
shows the results of a search for the keyword ``Swing''. Since MCS understands
that problems and solutions are associated with one another, it can group
together the search results so developers can see related problems and
solutions at a glance. The typing of messages also makes the editor's job
easier because they only have to worry about messages which are problems or
solutions, all other kinds of messages can be discarded.

\begin{figure*}[htbp]
  {\centerline{\psfig{figure=figs/search-results.eps,height=4in}}}
  \caption{Results from a search for keyword ``Swing''}
  \label{fig:search-results}
\end{figure*}

\subsubsection{Searching by Symptom}
MCS also provides symptom-based searching. It is common for a developer to be
aware of the symptoms of their problem, but unaware as to what the cause might
be. While editing messages, the editor may notice that a problem message
contains within it a textual pattern which is symptomatic of the problem. The
symptom is usually an error message of some sort. The editor can convert the
symptom into a regular expression, thereby stripping out all the irrelevant
parts of the symptom. This regular expression symptom is stored with the
message as metadata. Later, when a developer encounters a similar problem, they
can copy and paste the error message directly into a text field the symptom
search mode on an MCS web page and initiate a search. MCS will then attempt to
match the given text against all the symptom expressions in the archive,
displaying the results to the user.

For example, suppose the editor noticed this error text in a message being
edited:

{\scriptsize
\begin{verbatim}
java.lang.NoClassDefFoundError:
 javax/swing/DefaultBoundedRangeModel
   at com.ice.jcvsii.JCVS.instanceMain(Compiled Code)
   at com.ice.jcvsii.JCVS.main(Compiled Code)
\end{verbatim}}

From his or her knowledge of the domain, the editor knows that this is
symptomatic of using the wrong version of the Java Swing class library. So in
this case, the relevant portion of this error in regular expression form would
be:

{\scriptsize
\begin{verbatim}
java\.lang\.NoClassDefFoundError: javax/swing/.*
\end{verbatim}}

This symptom gets at the core of the error, because the only two important
parts are the type of the error and initial prefix of the mismatched package.
If a developer later pasted in the following different error message, MCS can
still match it to the correct problem:

{\scriptsize
\begin{verbatim}
java.lang.NoClassDefFoundError:
 javax/swing/text/JTextArea
   at com.ice.jcvsii.JCVS.instanceMain(JCVS.java:81)
   at com.ice.jcvsii.JCVS.main(JCVS.java:63)
\end{verbatim}}

\section{IMPLEMENTATION}
\label{sec:implementation}
MCS consists of two subsystems: the server side that stores the archive and
provides a World-Wide Web interface to archive users, and the editing tool that
allows editors to submit edited messages to the archive. MCS has been
implemented entirely in Java which means both the server and editor can be run
on a wide variety of platforms. The implementation of MCS consists of 7387
non-comment lines of Java code, 343 methods, and 55 classes.

The server side consists of several Java servlets and support classes. Servlets
are a way to extend the functionality of server, particularly web servers. The
MCS servlets conform to version 2.0 of the Sun Servlet API, which means that
they can be used with any of the many web servers which support servlets. The
web server we used was Sun's Java Web Server. The servlets are responsible for:
storing the condensed messages in a simple flat file database, accepting user
queries, presenting search results and messages to users. There are also
servlets which interact with the editing tool to allow updates to the database.

The editor side makes use of the fact that the data source for condensation is
email. The messages to be condensed are stored in normal mailbox folders on an
email server. The Java email client ICEMail
(http://www.trustice.com/\linebreak[0]java/icemail/) has been extended for use
as the MCS editing tool. Editors use ICEMail to contact the email server using
IMAP (Internet Message Access Protocol), and use the standard ICEMail interface
to read and delete messages from the list. When the editor encounters a message
which needs to be condensed for inclusion in the archive, he or she accesses
the MCS extensions to ICEMail which allow the editing and annotation to take
place in a separate window. When the condensation of the message is complete,
the editor can upload it to the MCS database with the press of a button.

Users of the MCS-condensed archive access it using their web browser. They
go to a particular URL, and the MCS servlets dynamically produce the HTML which
is rendered by the user's browser. The servlets produce only standard HTML to
enable almost any browser to use the archive.

\section{RELATED WORK}
There are a variety of systems and research related to maintaining and
searching collective memory. Here, we examine several such systems and compare
them to MCS. Some of these systems are somewhat informal (like moderated
mailing lists and FAQ files), and some are formal research projects. The
informal systems are based on the author's knowledge of those systems and
generally do not have references because they evolved from common Internet
practices.

\subsection{Moderated Mailing Lists}
Some mailing lists address the signal to noise problem by having a moderator or
a group of moderators. All submissions to the list are forwarded to the
moderator(s) who read the messages and decide whether or not to distribute them
to the list. On most lists, the moderator(s) do not edit the messages
submitted. They just choose whether or not to distribute the message. Also, to
allay fears of censorship on the part of the subscribers, usually the criteria
used to decide whether to distribute a message are rather liberal, e.g., the
message is related to the topic of the mailing list and not an advertisement
\cite{pedersen2-96}.

While moderation can be useful for maintaining a high signal to noise ratio, it
suffers from several problems addressed in the design of MCS. Moderation
requires a substantial commitment on the part of the moderator(s) to review
submissions in a timely manner. Failure to do so halts all traffic on the
mailing list and annoys subscribers who have come to expect the short
turnaround time that digital media can provide. Moderators also tend to face
continual concerns from subscribers as to whether they are moderating in a fair
and consistent manner. Since MCS does not affect the list distribution itself
at all, most concerns about censorship should be eliminated. MCS provides a
link from each edited message to the original unabridged message so users can
easily see what was edited out or changed in any particular message.

\subsection{Frequently-Asked Question Files}
Most frequently-asked question (FAQ) documents attempt to provide a similar
service to MCS: a condensed version of important and useful information that
came from a mailing list or newsgroup. There are several important differences
between the two systems. FAQ files are usually maintained without specific tool
support so they require extensive effort on the part of the maintainer to
create and update. FAQ files are generally created with the intention of easy
distribution either as plain text or HTML. Because of this requirement, FAQ
files are mostly limited in size to a few hundred kilobytes and they are
laid-out to be easy for humans to read. Since FAQs cannot be of arbitrary size
and complexity, they must omit useful information.

MCS does not have these limitations. Since the system is not intended to be
distributed by FTP or by posting to a mailing list or newsgroup, it can be as
large as necessary. A sophisticated query system is an integral part of MCS, so
it is not necessary that the underlying data be structured in an easily
understandable human format. Because MCS lacks these two restrictions, it need
not limit the archives it creates to merely frequently-asked topics, it can
contain any information that would be useful regardless of how broad its
appeal.

\subsection{FAQ {\scshape Finder}}
FAQ {\scshape Finder} allows users to quickly find answers to questions by
searching a database made up of FAQ documents posted to Usenet \cite{Burke97}.
The user enters his or her question into the system in natural language. First
the system uses standard information retrieval techniques to determine which
FAQs in the database are most likely to contain the answer to the question. It
presents the top five FAQs to the user, who can select the most likely
candidate.  Then the system uses a combination of lexical and semantic
similarity checks between the asked question and the question-answer pairs in
the FAQ file. It then presents the five most likely pairs for user
consideration.  A live version of the system can be found on the web
(http://faqfinder.ics.uci.edu:8001/).

While FAQ {\scshape Finder} is an interesting system, it is attempting to solve
a different problem than MCS. FAQ {\scshape Finder} assumes that there exists a
large number of FAQ files which are already organized in question-answer
format, and from those files it attempts to help users find the answer to their
questions.  The designers of FAQ {\scshape Finder} explicitly chose not to
implement any domain-specific knowledge into their system because their
intended dataset is a large number of unrelated FAQ files. MCS attempts to
create a FAQ-like body of knowledge from a mailing list, and then present the
condensed information in useful, possibly domain-specific ways. In this way MCS
attempts to solve the problem of getting the information into an FAQ-like
state, which is already presupposed in FAQ {\scshape Finder}. It might be
possible to create a ``stub'' FAQ which FAQ {\scshape Finder} could index, and
if the user's question is a good match, FAQ {\scshape Finder} would just send
the user to the MCS-created archive.

\subsection{Answer Garden and Answer Garden 2}
\label{sec:answer-garden}
The Answer Garden system is designed to provide an organically growing database
of answers to questions by end-users \cite{Ackerman90}. Users interact with the
system by answering a series of diagnostic multiple-choice questions which lead
them through the tree of answers already in the system. If users find that
their questions are not answered in the database, they can enter their
questions into the system and it will be forwarded to an appropriate expert via
email. When the expert answers, the result is sent back to the original
question-poser and also inserted into the tree for future retrieval.

Answer Garden's goal in life is to answer questions. Like MCS, it uses human
input to decide what questions and answers should be in the database. However,
Answer Garden is really only suited to the task of answering questions. A user
who just wants to browse information either has to answer the diagnostic
questions or guess where on the tree the information might be located. It also
requires a group of experts to be responsible for answering the questions posed
by users. MCS does not require users to use any special software to continue
participation in the mailing list, while Answer Garden assumes that all users
will use the Answer Garden tool when they have a question. In addition, MCS
provides the symptom search method which allows a user to use an error message
to find the solution to a problem immediately.  Answer Garden requires users to
answer a series of diagnostic questions, with no way to short circuit the
process.

Answer Garden 2 is a refinement of the Answer Garden system. It improves on
Answer Garden by adding a system of gradual escalation for questions input into
the system (thereby providing more context to the person answering the
question), and a subsystem for collaboratively ``refining'' the information in
the database \cite{cscw96*97}. All of this is built on a set of versatile and
configurable components which allow the system to be tuned for a particular
environment.

Although Answer Garden 2 appears much closer to MCS in its goals, the two
systems appear to differ in implementation and user interaction. Answer Garden
2 does not implement features such as 2D or symptom search. In addition, MCS
has been field tested as discussed in Section \ref{sec:evaluation}.

%This system appears to implement many of the features required for MCS.  The
%system which inputs data into the system (CafeCK) provides a mechanism for
%capturing mailing list messages, and the ``refining'' system called Co-Refinery
%allows collecting, culling, organizing, and distilling information. The
%Co-Refinery system seems particularly close to MCS's requirements.
%Unfortunately, Answer Garden 2 is not available for public distribution
%[Ackerman, personal communication], so the actual implementation was not
%available for use as a foundation for MCS. In addition, there does not appear
%to have been an evaluation of the Answer Garden 2 system in the field, so it is
%difficult to obtain further insight into the differences between MCS and Answer
%Garden 2.

\subsection{Faq-O-Matic}
Faq-O-Matic was created to solve some of the same problems MCS addresses: the
difficulty in finding answers in mailing list archives, and the substantial
effort required to maintain an FAQ
(http://www.dartmouth.edu/$\sim$jonh/\linebreak[0]ff-serve/cache/1.html).
Faq-O-Matic addresses these issues by creating a dynamic WWW-based FAQ which
and member of a user community can contribute to. Any user can browse through
the web pages and make additions as necessary. This provides an easy way to
maintain an FAQ since any member of the community can volunteer to help.
However, there is no centralized authority in charge of the FAQ, so pieces of
potentially incorrect or mutually conflicting information can be posted.
Furthermore, new additions have to be written from scratch by contributors,
unlike MCS.

\subsection{Vector-Based Text Searching and Summarization}
The standard technique for search and retrieval from large text databases is
the vector method. Each document or segment is decomposed into a vector of
terms which is assigned a weight which is proportional to the frequency of the
term on the document, but inversely proportional to the frequency of the term
in the whole collection of documents. Using this technique, searches can be
conducted by comparing the vector of the query, to the vector of each document
in the collection, and a summary of a document can be generated by selecting
segments of the document which are computed to be most relevant
\cite{salton-96}. However, this work is not directly applicable to MCS. The
domain of MCS is relatively short email message, unlike the domain of automatic
text summarization which typically deals with larger documents like news
stories or encyclopedia entries. The vector-based approach is problematic for
MCS, because the descriptions of problems and solutions are so small that there
is insufficient data for such statistical methods. The summarization techniques
are not relevant since they work on the paragraph level. Summarizing an
encyclopedia entry by selecting paragraphs makes sense, but for a 200 word
message it makes little sense.

\subsection{Other Work}
W. B. Frakes and B. A. Nejmeh developed a system for software reuse based on
searchable archives of annotated source code \cite{frakes-87}. This research is
similar to MCS in the use of human annotation of the documents in the database.
However, it appears that the metadata in the CATALOG system was created by the
author of the source code module, while in MCS the metadata is added by the
editor after the fact. Rub\'{e}n Prieto-D\'{\i}az has also done work on
software reuse search systems using a system called {\em faceted
  classification} \cite{diaz-91}. Faceted classification involves creating a
set of categories and a controlled vocabulary of terms for each category. Each
document is then assigned one term from each category. This type of
classification scheme could be used in MCS to ease the burden of maintaining
the keyword hierarchy, but it would make it difficult to perform meaningful 2D
searches since each category would contain many terms.

The LaSSIE system attacks the problem of disseminating architectural knowledge
about a large software system to all developers working on the system
\cite{lassie-91}. In LaSSIE, a ``reverse knowledge engineer'' manually creates
entries in a knowledge base which permits natural language queries and semantic
retrieval. MCS is like LaSSIE in that it requires a manual process to generate
the metadata required for retrieval. While LaSSIE has the semantic retrieval
mechanism, it is unclear whether that system could be applied to the far less
structured domain of mailing list content.

The MediaDoc system also addresses the problem of explaining software systems
to users \cite{1998:ase:erdem}. MediaDoc employs a complicated model of the
user, including the user's goals and plans. While this approach may be able to
provide better responses to user queries, it will incur even more editor
overhead than already present in MCS.

The Knowledge Depot system provides a group memory repository, and a project
awareness system to a workgroup using standard email distribution lists
\cite{knowledge-depot}. Messages are categorized in the repository by the
workgroup members using keywords in the subject lines of the messages. The
project awareness subsystem allows users to be sent periodic summaries of
recent activity in categories they are interested in. While having users
perform the categorization may work in a small workgroup, it would not be
feasible in MCS, because it requires additional effort on the part of all users
and assumes that all users are equally qualified to perform the categorization.

\section{EVALUATION}
\label{sec:evaluation}
To evaluate the research hypothesis that MCS is an improvement over existing
archives, we designed a case study of MCS. The case study involved creating a
condensed archive of a mailing list, releasing the archive to the list
subscribers, and collecting qualitative and quantitative data on the users of
the archive. The goal of the case study was to answer three research questions:
can messages be condensed in a reasonable amount of time, will subscribers of
the mailing list adopt the condensed archive, and will the archive users prefer
the condensed archive to an existing archive?

\subsection{Target Mailing List}
As mentioned in Section \ref{sec:mcs-system}, we selected the jcvs mailing list
as the target for the case study. As a trial run, we also condensed the
``icemail'' mailing list which exists for the discussion of the ICEMail program
mentioned in Section \ref{sec:implementation}. This mailing list had many fewer
subscribers, and while the condensed archive was announced to the subscribers,
no data was collected on their usage of the archive. Both the jcvs condensed
archive (http://csdl.ics.hawaii.edu:8100/) and the icemail condensed archive
(http://csdl.ics.hawaii.edu:8090/) are online and publicly accessible.

\subsection{Study Implementation}
\label{sec:study-implementation}
The case study was implemented in several stages:

\begin{enumerate}
\item The trial run condensation of the icemail list was performed.
\item The MCS software was revised from the lessons learned in the trial run.
\item The jcvs archive was condensed over several weeks.
\item The jcvs archive was announced to the mailing list on January 24, 2000.
\item Users were able to use the condensed archive at their leisure.
\item On February 10, 2000, an online questionnaire was made available via the
  top page of the MCS archive. An announcement was made to the mailing list
  informing users of the questionnaire's existence and encouraging them to fill
  it out.
\item On February 23, 2000, I ceased collecting data from the questionnaire,
  thus ending the case study.
\end{enumerate}

\subsection{Results}

\subsubsection{Editor Overhead}
Even if condensed archives are deemed by users to be more useful than
conventional archives, the system will not be adopted if the condensation
process requires too much effort by the editor. The editing task must be
feasible to perform, and it must not require too much time spent per message.
We assessed this measure by collecting time data from the editing tool as the
archive was condensed.

As we condensed the two archives, we recorded how much time was spent editing.
We recorded time spent reading messages, condensing them, and any external
reading required to condense the messages. We did not record time spent fixing
any critical defects in the MCS software, as they were discovered as that time
is not relevant to determining the expected time required to condense future
archives. Table \ref{tab:editing-time} shows the time data for both archives.

\begin{table*}[htbp]
  \begin{center}
    \begin{tabular} {|r|c|c|c|} \hline
      {\bf Mailing List} & {\bf Messages Examined} & {\bf Condensing Time} &
      {\bf Average Time Per Message}\\ \hline\hline
      icemail & 166 & 481 & 2.90\\ \hline
      jcvs & 1428 & 2165 & 1.52\\ \hline
    \end{tabular}
    \caption{Editing time results for two condensed archives (all times in minutes)}
    \label{tab:editing-time}
  \end{center}
\end{table*}

As you can see, it took substantially less time per message examined when
condensing the jcvs archive compared to the icemail archive. We attribute this
to two factors: tool improvement, and editor improvement. The editing tool had
a variety of quirks and defects when the first archive was condensed, so the
condensation required substantial manual effort. After we condensed the first
archive, we made many improvements to the editing tool which increased the
speed with which messages could be condensed. In addition, we learned more
about how to condense from the experience of condensing the first archive. The
increased knowledge allowed us to spend less time thinking about those issues
when condensing the second archive. With more practice and enhancements to the
editing tool, it should be entirely possible to bring the amount of time spent
per message to one minute or lower. For a medium to low traffic mailing lists,
this seems like an entirely acceptable amount of time to spend editing,
particularly since this includes the time required to read the email for the
first time, which an editor would presumably be doing anyway.

Table \ref{tab:archive-stats} shows a summary of the contents of the archives.
As you can see, the archives contain only a fraction of the messages examined
(23\% for icemail, 12\% for jcvs). This is to be expected because one of the
goals of MCS is removing unnecessary messages. The jcvs list had a smaller
percentage retained than the icemail list, presumably due to the heavier
traffic of the jcvs list. For both lists, the number of keywords is fairly
close to the number of archived messages. Because most messages contain
multiple keywords, this indicates that many messages used the same keywords,
otherwise the number of keywords would be larger than the number of archived
messages. Symptoms were also fairly common on both lists: 21\% of icemail
problems had symptoms, 30\% of jcvs problems had symptoms. The relatively high
incidence of problems with symptoms indicates that the symptom search can be a
useful search technique.

\begin{table*}[htbp]
  \begin{center}
    \begin{tabular} {|r|c|c|c|c|c|} \hline
      {\bf Mailing List} & {\bf Messages Examined} &
      {\bf Messages Archived} & {\bf Problems} &
      {\bf Keywords} & {\bf Symptoms}\\
      \hline\hline
      icemail & 166 & 39 & 19 & 40 & 4\\ \hline
      jcvs & 1428 & 177 & 82 & 120 & 26\\ \hline
    \end{tabular}
    \caption{Statistics on the composition of two condensed archives}
    \label{tab:archive-stats}
  \end{center}
\end{table*}

\subsubsection{Adoption}
We define adoption as a significant fraction of the subscribers of the targeted
mailing list using the condensed archive either in addition to or instead of
the traditional archives. We have used the number of list subscribers as an
estimate of the number of potential condensed archive users. The adoption
percentage is then the number of condensed archive users divided by the number
of list subscribers, expressed as a percentage. To decide what adoption
percentage would be indicative of success, we consulted Everett's work on the
the diffusion of innovations \cite{diffusion-innovations}. He divides adopters
into five categories based on the rate at which they adopt innovations. The two
categories containing the most rapid adopters are the {\em Innovators}
(consisting of 2.5\% of the population), and {\em Early Adopters} (consisting
of 13.5\% of the population). We decided to target both these categories, so
our target adoption percentage is the sum of the category sizes: 16\%.

To measure the adoption percentage, we needed two pieces of information: the
number of list subscribers and the number of users of the condensed archive.
The list maintainer provided the number of list subscribers from the
subscription list. An estimate for the number of users of the condensed archive
was obtained by analysis of web server log data from the condensed archive.
Note that the adoption percentage, as we have defined it, is an imperfect
measure since we cannot positively determine whether the users of the condensed
jcvs archive are actually subscribers of the mailing list.

The Java Web Server, like most web servers, records a log of all HTTP
\cite{rfc-http} requests made to it. Each log entry contains the request made,
the IP address of the requester, and a timestamp. At this level, the data
provides mere hit count information which is a poor indicator of the number of
actual users of the system. There are a variety of ways to track users more
closely, but they generally require the user to either register and log on or
accept {\it cookies}, which many people consider intrusive. Since the major
goal for MCS is adoption, annoying users is to be avoided at all costs.

Using only the raw request data, there are two ways to estimate the number of
users of the archive: unique IP address counting, and organizational analysis.
The first method involves simply counting the number of unique IP addresses
which have made requests. This technique, however, has problems because of
dynamic IP addressing and the use of public access computers (such as in a
University lab). In the case of dynamic IP addressing, a user may access the
archive from the same computer but over the course of a day that computer's IP
address might change which would cause this user to be counted more than once.
In the case of a public access computer, multiple people may use the same
computer to access the archive. Since the computer only has one IP address, the
multiple users will only be counted once.  Dynamic IP addressing is expected to
be more prevalent than shared computers among jcvs subscribers, so we expect
the unique IP address count to be an overestimate of the adoption rate.

The other method for estimating the number of users of the archive is
organizational analysis. Organizational analysis attempts to collate the number
of distinct organizations that issued requests to the web server. While
requests are recorded in the log by IP address, the Domain Name System (DNS)
can be used to map the IP address into a domain name. Domain names can be more
useful than raw IP addresses as they indicate what organization an IP address
belongs to. Of course this method has its own problems: multiple users at the
same organization are only counted once, and some IP addresses cannot be
resolved to a domain name. However these problems should make the size
organization list an underestimate of the number of users of the archive.

Using a program called {\it Analog} (http://www.analog.cx/), we analyzed the
log file. Since the web logs were used to assess adoption of the MCS archive,
we removed any requests which originated from any of the researchers'
workstations. According to the analysis by Analog, the web server received
requests from 99 distinct IP addresses during the case study. As mentioned
earlier, this value is almost certainly an overestimate of the actual users
since some users probably accessed the archive from different computers.
However, this value is probably an upper bound on the number of users of the
archive.

Analog also generated what it calls an organization report which uses the
organizational analysis technique. There were 70 entries in the organizational
report. Some of the entries are not actual users such as the googlebot.com
entry which is presumably a spider which collects data for the Google search
engine (http://www.google.com/). The last nine entries have only one request
which indicates that they didn't really do anything meaningful with the
archive. On the other hand, however, there were 176 requests from IP addresses
which could not be mapped into domain names which would presumably raise the
organization count if they could have been resolved. The organization list also
counts multiple users coming from the same organization as one, which could
cause an underestimation of the number of users. On balance, the value of 70 is
a better and more conservative estimate of the number of actual users than the
99 distinct IP addresses value.

Given this estimate of the number of users of the archive, we can now estimate
the percentage of list subscribers that used the archive. The list had
approximately 401 subscribers at the start of the case study. Using the figure
70 as the estimate of the number of archive users, we find that this accounts
for roughly 17\% of the list membership. This exceeds the 16\% goal which we
set as the measure of whether or not the list subscribers had adopted the
condensed archive.

\subsubsection{Preference}
Assessment of the users' preference of the condensed archive over traditional
archives was determined in a qualitative way through a user survey administered
using a web form on the archive site. The survey included the question: ``Since
the new problem-solving archive has been available, do you find yourself using
it instead of the old archive?''. All the questions were multiple choice except
for two open-ended feedback questions.

A total of six questionnaires were filled out. We classified the six
questionnaires returned into three different groups: those who had used neither
the old archive nor MCS, those who had only used MCS, and those who had used
both the old archive and the MCS archive. Each group had two questionnaire
results which fit the characteristics. Due to the small number of
questionnaires returned, we limit our analysis to qualitative trends that we
noticed in the data.

The two users who did use both archives reported that: they always found what
they were looking for in the MCS archive, they were completely satisfied with
the MCS archive, and that they were willing to volunteer as editors. This makes
some sense: in order to fully appreciate MCS you need to have used traditional
mailing list archives. The willingness of respondents to consider volunteering
to be editors is encouraging, and provides some hope that the burden of editing
could be spread out among multiple editors. All four users that had used the
MCS archives rated the archive as satisfactory or completely satisfactory.

\section{FUTURE WORK}
\label{sec:future-work}
%MCS is only just now emerging from its shell, leaving a wealth of opportunities
%for further evaluation and extension.

\subsection{Scalability Improvements}
MCS was designed as a research system, and as such, many decisions were made in
favor of speed and ease of implementation. However, if MCS grows to serve large
archives, work will be required on its scalability. MCS uses a flat-file
storage structured rather than a backend database, so some slow down will occur
when there are many users and many condensed messages. MCS also assumes that
there is only a single editor for an archive, which obviously does not scale
well to high-traffic lists where editors will want to share the daily workload.

\subsection{Adoption by Other Mailing Lists}
Convincing other mailing lists to use the software for their archives would be
the final stage in moving the software out into general use. This adoption
process may be more difficult because it requires the mailing list's community
to embrace the system and it also requires recruitment of one or more editors
from the mailing list.

%\subsection{Editor Recruitment}
%The editor(s) obviously play(s) a crucial role in the operation of MCS. Without
%continual updating, the database becomes of only historical interest. For
%foreseeable future, the author will be acting as the sole editor. In order to
%ensure the continued survival of the condensed archives, it will be necessary
%to recruit other editors. If the archive is useful enough and the editing tool
%is easy to use, it should be possible to get volunteers from the list to step
%forward as editors.

%\subsection{Open Source Distribution}
%As part of the growing Open Source movement (http://www.opensource.org/), we
%plan to release the MCS source code to the public under the GNU General Public
%License.

\subsection{Expansion into Technical Support Market}
Mailing lists are used extensively both internally and externally by those who
provide technical support. In this kind of environment, MCS could be used to do
a sort of ``data mining'' on old email archives, turning them into valuable
knowledge bases which can in turn reduce support costs. With the potential of
lowered costs, it would make sense for corporations to support editors either
within their company or even external editors.

\section*{ACKNOWLEDGMENTS}
This research would not have been possible without the help of Philip Johnson,
and all the members of the CSDL research group. Your encouragement is very much
appreciated. Thanks also to Yuka Nagashima for her steadfast support
throughout. This research is supported by the National Science Foundation under
Contract Number CCR-98-04010.

\bibliography{/export/home/csdl/bib/mcs-oldtex}
\bibliographystyle{abbrv}
\end{document}

% LocalWords:  tex Sep Jul bsdi Nexial EWS htbp supportnet eps fPIC mcs Ap FEPs
% LocalWords:  symptomlookup portmaster microsurveys Microsurvey microsurvey co
% LocalWords:  Pedersen pagestyle API metadata APIs jserv ICEMail JavaScript vs
% LocalWords:  Servlet Mailinglist jcvs JavaHelp CafeCK Faq Matic Frakes Prieto
% LocalWords:  Diaz LaSSIE TREC REtrieval MediaDoc Johnson's icemail Everett's
% LocalWords:  com uk collab lanl gov googlebot Google jonh Nejmeh az workgroup
% LocalWords:  CCR html MailingLists ListInfo java shtml ff
