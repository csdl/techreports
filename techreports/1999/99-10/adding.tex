%%%%%%%%%%%%%%%%%%%%%%%%%%%%%% -*- Mode: Latex -*- %%%%%%%%%%%%%%%%%%%%%%%%%%%%
%% adding.tex -- 
%% Author          : Joe Dane
%% Created On      : Tue Oct  5 15:35:04 1999
%% Last Modified By: Joe Dane
%% Last Modified On: Tue Nov  9 09:31:12 1999
%% RCS: $Id$
%%%%%%%%%%%%%%%%%%%%%%%%%%%%%%%%%%%%%%%%%%%%%%%%%%%%%%%%%%%%%%%%%%%%%%%%%%%%%%%
%%   Copyright (C) 1999 Joe Dane
%%%%%%%%%%%%%%%%%%%%%%%%%%%%%%%%%%%%%%%%%%%%%%%%%%%%%%%%%%%%%%%%%%%%%%%%%%%%%%%
%% 

\chapter{Adding a size metric to LOCC}
\label{chap:add}

LOCC was designed with one goal clearly in mind: to make it as simple
as possible to add size metrics to the tool.  It was clear early in
the development of LOCC that it would be foolish to think that one single
size measure would be sufficient for all circumstances.  Differences
in development languages alone would most likely require a
modification of the size metric being used.  If the tool was to have
any chance of being used, it would have to be flexible enough to
support multiple size metrics.

Once a developer has decided on a new metric, his efforts should be
focused on implementing that metric and not on the details of
interaction with the user of the metric.  It should be noted that
these details are not as trivial as they may seem.  It is common in
computer science research to find products developed with the goal of
illustrating some fine point of the research in question, but which take
little care with actual usability issues.  One may design the perfect
program size metric, but making it simple to use is usually left ``as
an exercise for the reader''.  LOCC alleviates this problem by taking 
care of the boring details of the interfaces to the user
and file system.

\section{Deciding on a metric}

The first step is naturally to figure out what is to be counted and how.
This should be the most difficult step, and if it is, then LOCC can be
considered a success.  The only limitations imposed on the size metric by
LOCC are that it be implemented as one or more Java classes and that the
classes fulfill the size counting interfaces defined by LOCC.  These
interfaces are so general that they will not constrain the designer of new
size metrics.

To illustrate the process of adding a new size metric to LOCC, I will
present a simple size metric, one which counts the number of semicolons in
a Java source file.  This is not quite as useless as it might sound: since
semicolons are used to separate statements in Java, the number of
semicolons should be close to the number of statements.  Two functions
(methods, actually) must be defined: one for measuring total size and one
for measuring size difference.  In this case, total size will be the total
number of semicolons in a source file.  Size difference will be defined
simply as the difference in the numbers of semicolons in the respective
source files.

\section{Write the Java code}

%%%%%%%%%%%%%%%%%%%%%%%%%%%%%%% -*- Mode: Latex -*- %%%%%%%%%%%%%%%%%%%%%%%%%%%%
%% semi-fig.tex -- 
%% Author          : Joe Dane
%% Created On      : Thu Oct 28 10:32:13 1999
%% Last Modified By: Joe Dane
%% Last Modified On: Thu Oct 28 10:43:54 1999
%% RCS: $Id$
%%%%%%%%%%%%%%%%%%%%%%%%%%%%%%%%%%%%%%%%%%%%%%%%%%%%%%%%%%%%%%%%%%%%%%%%%%%%%%%
%%   Copyright (C) 1999 Joe Dane
%%%%%%%%%%%%%%%%%%%%%%%%%%%%%%%%%%%%%%%%%%%%%%%%%%%%%%%%%%%%%%%%%%%%%%%%%%%%%%%
%% 


\begin{figure}
  \centering
  \caption{A simple size metric}
  \label{fig:locc-ext}
\end{figure}

\begin{figure}
  \centering
  \input{JavaSemiCount.java}
  \caption{A simple size metric}
  \label{fig:locc-ext}
\end{figure}

The Java code which implements this count is shown in Figure
\ref{fig:locc-ext}. In most cases the size metric designer would
probably spread the implementation out over several classes.  To keep
things compact here, I have compressed everything into a single
top-level class and an enclosed inner class.  The top level class,
{\tt JavaSemiCount} will be loaded by the {\tt LOCCClassLoader} and
incorporated into LOCC.  The size metric will be constructed by a call
to its zero-argument constructor, and LOCC will ask the metric for a
description of itself by calling the {\tt getName()}, {\tt
  getCLIArg()}, and {\tt getOutputFormats()} methods.  


This size metric provides a single output format, represented by the
inner class {\tt PlainOutputFormat}.  Like the {\tt SizeMeasure} of which it 
is a part, {\tt PlainOutputFormat} provides methods which allow LOCC
to ask about the format name, description, and command line argument
switch.  {\tt PlainOutputFormat} additionally implements the methods of 
both the {\tt TotalPrinter} and {\tt DiffPrinter} interfaces, so when
it is asked to supply references to the objects which will ultimately 
execute the count, it simply returns references to itself.  LOCC
stores these references and calls {\tt printTotal} and {\tt printDiff} 
on them after opening streams on the input and output files.


\section{Modify the LOCC preferences file}

\begin{figure}
  \centering
  \input{locc.properties}
  \caption{An LOCC properties file}
  \label{fig:locc-prefs}
\end{figure}

After writing and compiling the Java code, the next step is to direct
LOCC to the code.  This is done by modifying a preferences file.  LOCC 
uses both a global and a user-level preferences file, so the
modifications can be done at either level.  The preferences file must
look something like Figure \ref{fig:locc-prefs}.  

The format of the file is a standard Java preferences file.  Comments
can begin with either the ``{\tt !}'' or the ``{\tt \#}'' characters
and extend to the end of line.  Blank lines are ignored.  Non-comment
lines must be formatted either as {\em key : value\/} or {\em key = value\/}.

The ``sizes'' line is a list of short names of size metrics.  The names are
arbitrary, and are only used within the preferences file.  For each
size metric named on this line, three additional lines must appear.  
The keys for these lines are generated from the name of the size
metric with the strings ``dir'', ``class'' and ``package''
appended. The ``dir'' value specifies the directory in the file system
in which the classes reside.  The ``class'' value gives the name of
the main Java class which implements the {\tt SizeMeasure} interface.  
The class must exist in a file in the directory
specified in the ``dir'' value.  The use of jar or zip files is not
supported.  The ``package'' value indicates the Java package in which
the size code resides.  Classes from sub-packages of this package are
loaded in a manner similar to the system class loader, from
directories beneath the top level directory.

\section{Use the metric}

After the preferences file has been modified, LOCC will be ready to
use the new metric.  Simply starting LOCC in the usual way will cause
the new classes to be loaded and incorporated into LOCC.  The entire
process, from the design to the use of the size metric took 20
minutes.  This particular size metric is, admittedly, rather simple.
The important point is that almost all of that 20 minutes was used in
actually thinking about how the size metric should work, and very
little effort was wasted in thinking about extraneous issues.
