%%%%%%%%%%%%%%%%%%%%%%%%%%%%%% -*- Mode: Latex -*- %%%%%%%%%%%%%%%%%%%%%%%%%%%%
%% eval.tex -- 
%% Author          : Joe Dane
%% Created On      : Tue Oct  5 16:18:53 1999
%% Last Modified By: Joe Dane
%% Last Modified On: Wed Oct 20 14:50:59 1999
%% RCS: $Id$
%%%%%%%%%%%%%%%%%%%%%%%%%%%%%%%%%%%%%%%%%%%%%%%%%%%%%%%%%%%%%%%%%%%%%%%%%%%%%%%
%%   Copyright (C) 1999 Joe Dane
%%%%%%%%%%%%%%%%%%%%%%%%%%%%%%%%%%%%%%%%%%%%%%%%%%%%%%%%%%%%%%%%%%%%%%%%%%%%%%%
%% 


\chapter{Evaluation}
\label{chap:eval}

How well has LOCC lived up to the claims presented earlier in this
thesis?  There have been both successes and failures, but on balance
LOCC seems to have lived up to its claims.  LOCC has been evaluated
using three methods. First, LOCC has been run on a large Java code base to
confirm that it can calculate size data for a wide variety of source code.
Selected output from LOCC's
default {\sc javaline} metric has been checked against manually
counted size data and has been found to be accurate.  Second,
current users of LOCC have been surveyed.  Their answers, while
qualitative, have indicated that LOCC is useful.
Finally, I have tried to objectively evaluate the process of adding
and extending size measures within LOCC.  The responses from the user
survey provided information as to what could be changed in the {\sc
  javaline} size metric, and I have attempted to evaluate the ease
with which these changes were made.

\section{Baseline functionality}

LOCC has been used to measure the sizes of several systems developed
at the CSDL, including the AWT library from Sun (99 classes, 1468 methods,
13080 lines), the Leap toolkit (308 classes, 2328 methods, 49082 lines) and
LOCC itself (76 classes, 555 methods, 8116 lines).  It has been
used successfully in a graduate-level software engineering course.

A subset of the output from these uses of LOCC has been compared to
manually counted size data.  In all cases, the output from LOCC
has been identical to the hand-counted cases.

\section{User survey}

All known users of LOCC were given a brief survey and asked for both
their general impression of LOCC and specific problems they
encountered while using it.  I received five responses to the survey.
The responses were mostly positive, with all respondents indicating that
they had found LOCC useful.  While the general attitude was favorable,
there were some suggestions as to how LOCC could be improved.

The most commonly mentioned problem was with the {\sc javaline} size
metric.  The metric, as it was originally defined, made an effort to
do a ``structural difference'' when comparing two versions of a
program.  That is, {\sc javaline} would try to find exactly how many
lines had been added to each package, class, and method in a file.  It 
did this by trying to match old and new versions of the package,
class, or method, by using the algorithm described in Section
\ref{sect:whats-diff}.  It was noted in that section that if the developer
made a simple change to the source code that interfered with {\sc
  javaline\/}'s matching algorithm, such as renaming a class or adding 
a parameter to a method, then the class or method would be counted as
new code.  At the time of {\sc javaline\/}'s initial design, it was
not know how much this would affect people.

After reviewing the responses of the user survey, it was clear that
many people had encountered this problem.  The students in the
software engineering class especially, who were developing projects on 
their own and had little reason to maintain interface consistency
across versions, had problems with this.  The changes enacted to
alleviate the problem are described below, in Section \ref{eval:add}.

Several people mentioned features they wished existed in the user
interface.  One feature mentioned more than once was the ability to
recursively measure files in a directory and its subdirectories.  This 
feature is planned for the next revision of LOCC.

Despite the problems with {\sc javaline}, most respondents felt that
LOCC has been successful, and had significantly helped them in
collection of their process data.

\section{Evaluation of LOCC extension}
\label{eval:add}

Probably the most interesting aspect of LOCC is how it allows, even
encourages, experimentation with size metrics.  As described in
Chapter \ref{chap:add}, adding a new size measure to LOCC is only as
difficult as designing and implementing the metric itself.  For a
simple metric, this may be a matter of a few hours.  

The situation with the {\sc javaline} metric is itself interesting.
The problems involved in the syntax tree node matching algorithm were
apparent, and yet it still seemed that a diff that was aware of the
structure of the language shouldn't be thrown away.  The Leap toolkit, 
for one thing, prefers that size measures, total and difference, be
expressed in this way.  Also, it can be worthwhile after working on
modifying a program for some time to take a look at where the changes
have been made.  A diff which can distinguish changes made to one
method from changes made to another can provide this.

\singlespace
\begin{figure}
  \centering
  \begin{verbatim}
public class aClass {
    int i, j;
    public String methodOne() {
        return "methodOne";
    }
    public String methodTwo() {
        return "methodTwo";
    }
}
\end{verbatim}
  \caption{Old program version}
  \label{old-version2}
\end{figure}

\begin{figure}
  \centering
  \begin{verbatim}
public class aClass {
    int i, j;
    public String methodOne(int k) {
        return "methodOne";
    }
    public String methodTwo() {
        return "methodTwo";
    }
}
\end{verbatim}
  \caption{New program version}
  \label{new-version2}
\end{figure}
\doublespace

Therefore, the {\sc javaline} metric was split in two.  The original
metric would be modified to try to be a bit smarter in its matching.
The new metric, called {\sc simple-javaline}, was identical to the
original in total size counting, but for size difference counting
considered only file level differences.  That is, a diff in {\sc
  simple-javaline} compares two files, without regard for program
structure, returning the number of new or changed lines in the new
file.

{\sc javaline} was improved by adding some logic to its matching
algorithm.  Matching would begin as in the original metric.  After a
failure to find a match, a special search for a matching node would be
undertaken.  This search proceeded by sequentially examining each of a 
number of candidate nodes.  The candidate nodes were those at the same
level in the old syntax tree as the new node being matched.


For example, consider the program versions in Figures
\ref{old-version2} and \ref{new-version2}.  Assume that the metric has 
proceeded to the point where it is counting the contents of {\tt
  aClass}, and it is looking for a match in the old syntax tree for
the method with signature {\tt String methodOne(int)}.  After failing
to find a match, it will examine every child node of the old {\tt
  aClass} node.  For each of these nodes, a textual comparison will be 
made with the new {\tt methodOne} method.  If we find a node in the
old tree which ``looks like'' the new method node, then we consider
the two nodes a match.  If more than one node in the old tree looks
like the new node, we take the old node which looks most like the new
one as the match.  We say a node looks like another if the ratio of
new and changed lines to total lines is less than 0.25.  In other
words, if fewer than 1/4 of the lines in a method are new or changed,
we should be able to find a match.

Much of the code to implement the new and improved {\sc javaline} was
taken without change from the original version.\footnote{Only 12 lines were 
  added.  If we counted deleted lines, the count would have been higher.}
In particular, the entire parser was reused.  The process of
making the changes and testing the new metric took 120 minutes.  The
construction of the {\sc simple-javaline} metric took 30 minutes.  The 
fact that these changes could be made and incorporated into a working
product so quickly is the most convincing evidence that LOCC has
succeeded in its goals.






