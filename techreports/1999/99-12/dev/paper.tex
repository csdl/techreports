%%%%%%%%%%%%%%%%%%%%%%%%%%%%%% -*- Mode: Latex -*- %%%%%%%%%%%%%%%%%%%%%%%%%%%%
%% paper.tex -- 
%% Author          : Carleton Moore
%% Created On      : Thu Nov  4 09:58:31 1999
%% Last Modified By: Carleton Moore
%% Last Modified On: Thu Nov 11 10:29:38 1999
%% RCS: $Id$
%%%%%%%%%%%%%%%%%%%%%%%%%%%%%%%%%%%%%%%%%%%%%%%%%%%%%%%%%%%%%%%%%%%%%%%%%%%%%%%
%%   Copyright (C) 1999 Carleton Moore
%%%%%%%%%%%%%%%%%%%%%%%%%%%%%%%%%%%%%%%%%%%%%%%%%%%%%%%%%%%%%%%%%%%%%%%%%%%%%%%
%% 

\documentstyle[twocolumn,icse2000]{article}
% Changes: 
%  Page limit: 10 pages
%  Abstract limit: 200 words 
%  Contact information is now gall@infosys.tuwien.ac.at
%  URL for author kit changed: needs to be filled in.
%  All other font and sizing information is the same:
%    check whether it still applies.
% Psfig/TeX 
\def\PsfigVersion{1.9}
% dvips version
%
% All psfig/tex software, documentation, and related files
% in this distribution of psfig/tex are 
% Copyright 1987, 1988, 1991 Trevor J. Darrell
%
% Permission is granted for use and non-profit distribution of psfig/tex 
% providing that this notice is clearly maintained. The right to
% distribute any portion of psfig/tex for profit or as part of any commercial
% product is specifically reserved for the author(s) of that portion.
%
% *** Feel free to make local modifications of psfig as you wish,
% *** but DO NOT post any changed or modified versions of ``psfig''
% *** directly to the net. Send them to me and I'll try to incorporate
% *** them into future versions. If you want to take the psfig code 
% *** and make a new program (subject to the copyright above), distribute it, 
% *** (and maintain it) that's fine, just don't call it psfig.
%
% Bugs and improvements to trevor@media.mit.edu.
%
% Thanks to Greg Hager (GDH) and Ned Batchelder for their contributions
% to the original version of this project.
%
% Modified by J. Daniel Smith on 9 October 1990 to accept the
% %%BoundingBox: comment with or without a space after the colon.  Stole
% file reading code from Tom Rokicki's EPSF.TEX file (see below).
%
% More modifications by J. Daniel Smith on 29 March 1991 to allow the
% the included PostScript figure to be rotated.  The amount of
% rotation is specified by the "angle=" parameter of the \psfig command.
%
% Modified by Robert Russell on June 25, 1991 to allow users to specify
% .ps filenames which don't yet exist, provided they explicitly provide
% boundingbox information via the \psfig command. Note: This will only work
% if the "file=" parameter follows all four "bb???=" parameters in the
% command. This is due to the order in which psfig interprets these params.
%
%  3 Jul 1991	JDS	check if file already read in once
%  4 Sep 1991	JDS	fixed incorrect computation of rotated
%			bounding box
% 25 Sep 1991	GVR	expanded synopsis of \psfig
% 14 Oct 1991	JDS	\fbox code from LaTeX so \psdraft works with TeX
%			changed \typeout to \ps@typeout
% 17 Oct 1991	JDS	added \psscalefirst and \psrotatefirst
%

% From: gvr@cs.brown.edu (George V. Reilly)
%
% \psdraft	draws an outline box, but doesn't include the figure
%		in the DVI file.  Useful for previewing.
%
% \psfull	includes the figure in the DVI file (default).
%
% \psscalefirst width= or height= specifies the size of the figure
% 		before rotation.
% \psrotatefirst (default) width= or height= specifies the size of the
% 		 figure after rotation.  Asymetric figures will
% 		 appear to shrink.
%
% \psfigurepath#1	sets the path to search for the figure
%
% \psfig
% usage: \psfig{file=, figure=, height=, width=,
%			bbllx=, bblly=, bburx=, bbury=,
%			rheight=, rwidth=, clip=, angle=, silent=}
%
%	"file" is the filename.  If no path name is specified and the
%		file is not found in the current directory,
%		it will be looked for in directory \psfigurepath.
%	"figure" is a synonym for "file".
%	By default, the width and height of the figure are taken from
%		the BoundingBox of the figure.
%	If "width" is specified, the figure is scaled so that it has
%		the specified width.  Its height changes proportionately.
%	If "height" is specified, the figure is scaled so that it has
%		the specified height.  Its width changes proportionately.
%	If both "width" and "height" are specified, the figure is scaled
%		anamorphically.
%	"bbllx", "bblly", "bburx", and "bbury" control the PostScript
%		BoundingBox.  If these four values are specified
%               *before* the "file" option, the PSFIG will not try to
%               open the PostScript file.
%	"rheight" and "rwidth" are the reserved height and width
%		of the figure, i.e., how big TeX actually thinks
%		the figure is.  They default to "width" and "height".
%	The "clip" option ensures that no portion of the figure will
%		appear outside its BoundingBox.  "clip=" is a switch and
%		takes no value, but the `=' must be present.
%	The "angle" option specifies the angle of rotation (degrees, ccw).
%	The "silent" option makes \psfig work silently.
%

% check to see if macros already loaded in (maybe some other file says
% "\input psfig") ...
\ifx\undefined\psfig\else\endinput\fi

%
% from a suggestion by eijkhout@csrd.uiuc.edu to allow
% loading as a style file. Changed to avoid problems
% with amstex per suggestion by jbence@math.ucla.edu

\let\LaTeXAtSign=\@
\let\@=\relax
\edef\psfigRestoreAt{\catcode`\@=\number\catcode`@\relax}
%\edef\psfigRestoreAt{\catcode`@=\number\catcode`@\relax}
\catcode`\@=11\relax
\newwrite\@unused
\def\ps@typeout#1{{\let\protect\string\immediate\write\@unused{#1}}}
\ps@typeout{psfig/tex \PsfigVersion}

%% Here's how you define your figure path.  Should be set up with null
%% default and a user useable definition.

\def\figurepath{./}
\def\psfigurepath#1{\edef\figurepath{#1}}

%
% @psdo control structure -- similar to Latex @for.
% I redefined these with different names so that psfig can
% be used with TeX as well as LaTeX, and so that it will not 
% be vunerable to future changes in LaTeX's internal
% control structure,
%
\def\@nnil{\@nil}
\def\@empty{}
\def\@psdonoop#1\@@#2#3{}
\def\@psdo#1:=#2\do#3{\edef\@psdotmp{#2}\ifx\@psdotmp\@empty \else
    \expandafter\@psdoloop#2,\@nil,\@nil\@@#1{#3}\fi}
\def\@psdoloop#1,#2,#3\@@#4#5{\def#4{#1}\ifx #4\@nnil \else
       #5\def#4{#2}\ifx #4\@nnil \else#5\@ipsdoloop #3\@@#4{#5}\fi\fi}
\def\@ipsdoloop#1,#2\@@#3#4{\def#3{#1}\ifx #3\@nnil 
       \let\@nextwhile=\@psdonoop \else
      #4\relax\let\@nextwhile=\@ipsdoloop\fi\@nextwhile#2\@@#3{#4}}
\def\@tpsdo#1:=#2\do#3{\xdef\@psdotmp{#2}\ifx\@psdotmp\@empty \else
    \@tpsdoloop#2\@nil\@nil\@@#1{#3}\fi}
\def\@tpsdoloop#1#2\@@#3#4{\def#3{#1}\ifx #3\@nnil 
       \let\@nextwhile=\@psdonoop \else
      #4\relax\let\@nextwhile=\@tpsdoloop\fi\@nextwhile#2\@@#3{#4}}
% 
% \fbox is defined in latex.tex; so if \fbox is undefined, assume that
% we are not in LaTeX.
% Perhaps this could be done better???
\ifx\undefined\fbox
% \fbox code from modified slightly from LaTeX
\newdimen\fboxrule
\newdimen\fboxsep
\newdimen\ps@tempdima
\newbox\ps@tempboxa
\fboxsep = 3pt
\fboxrule = .4pt
\long\def\fbox#1{\leavevmode\setbox\ps@tempboxa\hbox{#1}\ps@tempdima\fboxrule
    \advance\ps@tempdima \fboxsep \advance\ps@tempdima \dp\ps@tempboxa
   \hbox{\lower \ps@tempdima\hbox
  {\vbox{\hrule height \fboxrule
          \hbox{\vrule width \fboxrule \hskip\fboxsep
          \vbox{\vskip\fboxsep \box\ps@tempboxa\vskip\fboxsep}\hskip 
                 \fboxsep\vrule width \fboxrule}
                 \hrule height \fboxrule}}}}
\fi
%
%%%%%%%%%%%%%%%%%%%%%%%%%%%%%%%%%%%%%%%%%%%%%%%%%%%%%%%%%%%%%%%%%%%
% file reading stuff from epsf.tex
%   EPSF.TEX macro file:
%   Written by Tomas Rokicki of Radical Eye Software, 29 Mar 1989.
%   Revised by Don Knuth, 3 Jan 1990.
%   Revised by Tomas Rokicki to accept bounding boxes with no
%      space after the colon, 18 Jul 1990.
%   Portions modified/removed for use in PSFIG package by
%      J. Daniel Smith, 9 October 1990.
%
\newread\ps@stream
\newif\ifnot@eof       % continue looking for the bounding box?
\newif\if@noisy        % report what you're making?
\newif\if@atend        % %%BoundingBox: has (at end) specification
\newif\if@psfile       % does this look like a PostScript file?
%
% PostScript files should start with `%!'
%
{\catcode`\%=12\global\gdef\epsf@start{%!}}
\def\epsf@PS{PS}
%
\def\epsf@getbb#1{%
%
%   The first thing we need to do is to open the
%   PostScript file, if possible.
%
\openin\ps@stream=#1
\ifeof\ps@stream\ps@typeout{Error, File #1 not found}\else
%
%   Okay, we got it. Now we'll scan lines until we find one that doesn't
%   start with %. We're looking for the bounding box comment.
%
   {\not@eoftrue \chardef\other=12
    \def\do##1{\catcode`##1=\other}\dospecials \catcode`\ =10
    \loop
       \if@psfile
	  \read\ps@stream to \epsf@fileline
       \else{
	  \obeyspaces
          \read\ps@stream to \epsf@tmp\global\let\epsf@fileline\epsf@tmp}
       \fi
       \ifeof\ps@stream\not@eoffalse\else
%
%   Check the first line for `%!'.  Issue a warning message if its not
%   there, since the file might not be a PostScript file.
%
       \if@psfile\else
       \expandafter\epsf@test\epsf@fileline:. \\%
       \fi
%
%   We check to see if the first character is a % sign;
%   if so, we look further and stop only if the line begins with
%   `%%BoundingBox:' and the `(atend)' specification was not found.
%   That is, the only way to stop is when the end of file is reached,
%   or a `%%BoundingBox: llx lly urx ury' line is found.
%
          \expandafter\epsf@aux\epsf@fileline:. \\%
       \fi
   \ifnot@eof\repeat
   }\closein\ps@stream\fi}%
%
% This tests if the file we are reading looks like a PostScript file.
%
\long\def\epsf@test#1#2#3:#4\\{\def\epsf@testit{#1#2}
			\ifx\epsf@testit\epsf@start\else
\ps@typeout{Warning! File does not start with `\epsf@start'.  It may not be a PostScript file.}
			\fi
			\@psfiletrue} % don't test after 1st line
%
%   We still need to define the tricky \epsf@aux macro. This requires
%   a couple of magic constants for comparison purposes.
%
{\catcode`\%=12\global\let\epsf@percent=%\global\def\epsf@bblit{%BoundingBox}}
%
%
%   So we're ready to check for `%BoundingBox:' and to grab the
%   values if they are found.  We continue searching if `(at end)'
%   was found after the `%BoundingBox:'.
%
\long\def\epsf@aux#1#2:#3\\{\ifx#1\epsf@percent
   \def\epsf@testit{#2}\ifx\epsf@testit\epsf@bblit
	\@atendfalse
        \epsf@atend #3 . \\%
	\if@atend	
	   \if@verbose{
		\ps@typeout{psfig: found `(atend)'; continuing search}
	   }\fi
        \else
        \epsf@grab #3 . . . \\%
        \not@eoffalse
        \global\no@bbfalse
        \fi
   \fi\fi}%
%
%   Here we grab the values and stuff them in the appropriate definitions.
%
\def\epsf@grab #1 #2 #3 #4 #5\\{%
   \global\def\epsf@llx{#1}\ifx\epsf@llx\empty
      \epsf@grab #2 #3 #4 #5 .\\\else
   \global\def\epsf@lly{#2}%
   \global\def\epsf@urx{#3}\global\def\epsf@ury{#4}\fi}%
%
% Determine if the stuff following the %%BoundingBox is `(atend)'
% J. Daniel Smith.  Copied from \epsf@grab above.
%
\def\epsf@atendlit{(atend)} 
\def\epsf@atend #1 #2 #3\\{%
   \def\epsf@tmp{#1}\ifx\epsf@tmp\empty
      \epsf@atend #2 #3 .\\\else
   \ifx\epsf@tmp\epsf@atendlit\@atendtrue\fi\fi}


% End of file reading stuff from epsf.tex
%%%%%%%%%%%%%%%%%%%%%%%%%%%%%%%%%%%%%%%%%%%%%%%%%%%%%%%%%%%%%%%%%%%

%%%%%%%%%%%%%%%%%%%%%%%%%%%%%%%%%%%%%%%%%%%%%%%%%%%%%%%%%%%%%%%%%%%
% trigonometry stuff from "trig.tex"
\chardef\psletter = 11 % won't conflict with \begin{letter} now...
\chardef\other = 12

\newif \ifdebug %%% turn me on to see TeX hard at work ...
\newif\ifc@mpute %%% don't need to compute some values
\c@mputetrue % but assume that we do

\let\then = \relax
\def\r@dian{pt }
\let\r@dians = \r@dian
\let\dimensionless@nit = \r@dian
\let\dimensionless@nits = \dimensionless@nit
\def\internal@nit{sp }
\let\internal@nits = \internal@nit
\newif\ifstillc@nverging
\def \Mess@ge #1{\ifdebug \then \message {#1} \fi}

{ %%% Things that need abnormal catcodes %%%
	\catcode `\@ = \psletter
	\gdef \nodimen {\expandafter \n@dimen \the \dimen}
	\gdef \term #1 #2 #3%
	       {\edef \t@ {\the #1}%%% freeze parameter 1 (count, by value)
		\edef \t@@ {\expandafter \n@dimen \the #2\r@dian}%
				   %%% freeze parameter 2 (dimen, by value)
		\t@rm {\t@} {\t@@} {#3}%
	       }
	\gdef \t@rm #1 #2 #3%
	       {{%
		\count 0 = 0
		\dimen 0 = 1 \dimensionless@nit
		\dimen 2 = #2\relax
		\Mess@ge {Calculating term #1 of \nodimen 2}%
		\loop
		\ifnum	\count 0 < #1
		\then	\advance \count 0 by 1
			\Mess@ge {Iteration \the \count 0 \space}%
			\Multiply \dimen 0 by {\dimen 2}%
			\Mess@ge {After multiplication, term = \nodimen 0}%
			\Divide \dimen 0 by {\count 0}%
			\Mess@ge {After division, term = \nodimen 0}%
		\repeat
		\Mess@ge {Final value for term #1 of 
				\nodimen 2 \space is \nodimen 0}%
		\xdef \Term {#3 = \nodimen 0 \r@dians}%
		\aftergroup \Term
	       }}
	\catcode `\p = \other
	\catcode `\t = \other
	\gdef \n@dimen #1pt{#1} %%% throw away the ``pt''
}

\def \Divide #1by #2{\divide #1 by #2} %%% just a synonym

\def \Multiply #1by #2%%% allows division of a dimen by a dimen
       {{%%% should really freeze parameter 2 (dimen, passed by value)
	\count 0 = #1\relax
	\count 2 = #2\relax
	\count 4 = 65536
	\Mess@ge {Before scaling, count 0 = \the \count 0 \space and
			count 2 = \the \count 2}%
	\ifnum	\count 0 > 32767 %%% do our best to avoid overflow
	\then	\divide \count 0 by 4
		\divide \count 4 by 4
	\else	\ifnum	\count 0 < -32767
		\then	\divide \count 0 by 4
			\divide \count 4 by 4
		\else
		\fi
	\fi
	\ifnum	\count 2 > 32767 %%% while retaining reasonable accuracy
	\then	\divide \count 2 by 4
		\divide \count 4 by 4
	\else	\ifnum	\count 2 < -32767
		\then	\divide \count 2 by 4
			\divide \count 4 by 4
		\else
		\fi
	\fi
	\multiply \count 0 by \count 2
	\divide \count 0 by \count 4
	\xdef \product {#1 = \the \count 0 \internal@nits}%
	\aftergroup \product
       }}

\def\r@duce{\ifdim\dimen0 > 90\r@dian \then   % sin(x+90) = sin(180-x)
		\multiply\dimen0 by -1
		\advance\dimen0 by 180\r@dian
		\r@duce
	    \else \ifdim\dimen0 < -90\r@dian \then  % sin(-x) = sin(360+x)
		\advance\dimen0 by 360\r@dian
		\r@duce
		\fi
	    \fi}

\def\Sine#1%
       {{%
	\dimen 0 = #1 \r@dian
	\r@duce
	\ifdim\dimen0 = -90\r@dian \then
	   \dimen4 = -1\r@dian
	   \c@mputefalse
	\fi
	\ifdim\dimen0 = 90\r@dian \then
	   \dimen4 = 1\r@dian
	   \c@mputefalse
	\fi
	\ifdim\dimen0 = 0\r@dian \then
	   \dimen4 = 0\r@dian
	   \c@mputefalse
	\fi
%
	\ifc@mpute \then
        	% convert degrees to radians
		\divide\dimen0 by 180
		\dimen0=3.141592654\dimen0
%
		\dimen 2 = 3.1415926535897963\r@dian %%% a well-known constant
		\divide\dimen 2 by 2 %%% we only deal with -pi/2 : pi/2
		\Mess@ge {Sin: calculating Sin of \nodimen 0}%
		\count 0 = 1 %%% see power-series expansion for sine
		\dimen 2 = 1 \r@dian %%% ditto
		\dimen 4 = 0 \r@dian %%% ditto
		\loop
			\ifnum	\dimen 2 = 0 %%% then we've done
			\then	\stillc@nvergingfalse 
			\else	\stillc@nvergingtrue
			\fi
			\ifstillc@nverging %%% then calculate next term
			\then	\term {\count 0} {\dimen 0} {\dimen 2}%
				\advance \count 0 by 2
				\count 2 = \count 0
				\divide \count 2 by 2
				\ifodd	\count 2 %%% signs alternate
				\then	\advance \dimen 4 by \dimen 2
				\else	\advance \dimen 4 by -\dimen 2
				\fi
		\repeat
	\fi		
			\xdef \sine {\nodimen 4}%
       }}

% Now the Cosine can be calculated easily by calling \Sine
\def\Cosine#1{\ifx\sine\UnDefined\edef\Savesine{\relax}\else
		             \edef\Savesine{\sine}\fi
	{\dimen0=#1\r@dian\advance\dimen0 by 90\r@dian
	 \Sine{\nodimen 0}
	 \xdef\cosine{\sine}
	 \xdef\sine{\Savesine}}}	      
% end of trig stuff
%%%%%%%%%%%%%%%%%%%%%%%%%%%%%%%%%%%%%%%%%%%%%%%%%%%%%%%%%%%%%%%%%%%%

\def\psdraft{
	\def\@psdraft{0}
	%\ps@typeout{draft level now is \@psdraft \space . }
}
\def\psfull{
	\def\@psdraft{100}
	%\ps@typeout{draft level now is \@psdraft \space . }
}

\psfull

\newif\if@scalefirst
\def\psscalefirst{\@scalefirsttrue}
\def\psrotatefirst{\@scalefirstfalse}
\psrotatefirst

\newif\if@draftbox
\def\psnodraftbox{
	\@draftboxfalse
}
\def\psdraftbox{
	\@draftboxtrue
}
\@draftboxtrue

\newif\if@prologfile
\newif\if@postlogfile
\def\pssilent{
	\@noisyfalse
}
\def\psnoisy{
	\@noisytrue
}
\psnoisy
%%% These are for the option list.
%%% A specification of the form a = b maps to calling \@p@@sa{b}
\newif\if@bbllx
\newif\if@bblly
\newif\if@bburx
\newif\if@bbury
\newif\if@height
\newif\if@width
\newif\if@rheight
\newif\if@rwidth
\newif\if@angle
\newif\if@clip
\newif\if@verbose
\def\@p@@sclip#1{\@cliptrue}


\newif\if@decmpr

%%% GDH 7/26/87 -- changed so that it first looks in the local directory,
%%% then in a specified global directory for the ps file.
%%% RPR 6/25/91 -- changed so that it defaults to user-supplied name if
%%% boundingbox info is specified, assuming graphic will be created by
%%% print time.
%%% TJD 10/19/91 -- added bbfile vs. file distinction, and @decmpr flag

\def\@p@@sfigure#1{\def\@p@sfile{null}\def\@p@sbbfile{null}
	        \openin1=#1.bb
		\ifeof1\closein1
	        	\openin1=\figurepath#1.bb
			\ifeof1\closein1
			        \openin1=#1
				\ifeof1\closein1%
				       \openin1=\figurepath#1
					\ifeof1
					   \ps@typeout{Error, File #1 not found}
						\if@bbllx\if@bblly
				   		\if@bburx\if@bbury
			      				\def\@p@sfile{#1}%
			      				\def\@p@sbbfile{#1}%
							\@decmprfalse
				  	   	\fi\fi\fi\fi
					\else\closein1
				    		\def\@p@sfile{\figurepath#1}%
				    		\def\@p@sbbfile{\figurepath#1}%
						\@decmprfalse
	                       		\fi%
			 	\else\closein1%
					\def\@p@sfile{#1}
					\def\@p@sbbfile{#1}
					\@decmprfalse
			 	\fi
			\else
				\def\@p@sfile{\figurepath#1}
				\def\@p@sbbfile{\figurepath#1.bb}
				\@decmprtrue
			\fi
		\else
			\def\@p@sfile{#1}
			\def\@p@sbbfile{#1.bb}
			\@decmprtrue
		\fi}

\def\@p@@sfile#1{\@p@@sfigure{#1}}

\def\@p@@sbbllx#1{
		%\ps@typeout{bbllx is #1}
		\@bbllxtrue
		\dimen100=#1
		\edef\@p@sbbllx{\number\dimen100}
}
\def\@p@@sbblly#1{
		%\ps@typeout{bblly is #1}
		\@bbllytrue
		\dimen100=#1
		\edef\@p@sbblly{\number\dimen100}
}
\def\@p@@sbburx#1{
		%\ps@typeout{bburx is #1}
		\@bburxtrue
		\dimen100=#1
		\edef\@p@sbburx{\number\dimen100}
}
\def\@p@@sbbury#1{
		%\ps@typeout{bbury is #1}
		\@bburytrue
		\dimen100=#1
		\edef\@p@sbbury{\number\dimen100}
}
\def\@p@@sheight#1{
		\@heighttrue
		\dimen100=#1
   		\edef\@p@sheight{\number\dimen100}
		%\ps@typeout{Height is \@p@sheight}
}
\def\@p@@swidth#1{
		%\ps@typeout{Width is #1}
		\@widthtrue
		\dimen100=#1
		\edef\@p@swidth{\number\dimen100}
}
\def\@p@@srheight#1{
		%\ps@typeout{Reserved height is #1}
		\@rheighttrue
		\dimen100=#1
		\edef\@p@srheight{\number\dimen100}
}
\def\@p@@srwidth#1{
		%\ps@typeout{Reserved width is #1}
		\@rwidthtrue
		\dimen100=#1
		\edef\@p@srwidth{\number\dimen100}
}
\def\@p@@sangle#1{
		%\ps@typeout{Rotation is #1}
		\@angletrue
%		\dimen100=#1
		\edef\@p@sangle{#1} %\number\dimen100}
}
\def\@p@@ssilent#1{ 
		\@verbosefalse
}
\def\@p@@sprolog#1{\@prologfiletrue\def\@prologfileval{#1}}
\def\@p@@spostlog#1{\@postlogfiletrue\def\@postlogfileval{#1}}
\def\@cs@name#1{\csname #1\endcsname}
\def\@setparms#1=#2,{\@cs@name{@p@@s#1}{#2}}
%
% initialize the defaults (size the size of the figure)
%
\def\ps@init@parms{
		\@bbllxfalse \@bbllyfalse
		\@bburxfalse \@bburyfalse
		\@heightfalse \@widthfalse
		\@rheightfalse \@rwidthfalse
		\def\@p@sbbllx{}\def\@p@sbblly{}
		\def\@p@sbburx{}\def\@p@sbbury{}
		\def\@p@sheight{}\def\@p@swidth{}
		\def\@p@srheight{}\def\@p@srwidth{}
		\def\@p@sangle{0}
		\def\@p@sfile{} \def\@p@sbbfile{}
		\def\@p@scost{10}
		\def\@sc{}
		\@prologfilefalse
		\@postlogfilefalse
		\@clipfalse
		\if@noisy
			\@verbosetrue
		\else
			\@verbosefalse
		\fi
}
%
% Go through the options setting things up.
%
\def\parse@ps@parms#1{
	 	\@psdo\@psfiga:=#1\do
		   {\expandafter\@setparms\@psfiga,}}
%
% Compute bb height and width
%
\newif\ifno@bb
\def\bb@missing{
	\if@verbose{
		\ps@typeout{psfig: searching \@p@sbbfile \space  for bounding box}
	}\fi
	\no@bbtrue
	\epsf@getbb{\@p@sbbfile}
        \ifno@bb \else \bb@cull\epsf@llx\epsf@lly\epsf@urx\epsf@ury\fi
}	
\def\bb@cull#1#2#3#4{
	\dimen100=#1 bp\edef\@p@sbbllx{\number\dimen100}
	\dimen100=#2 bp\edef\@p@sbblly{\number\dimen100}
	\dimen100=#3 bp\edef\@p@sbburx{\number\dimen100}
	\dimen100=#4 bp\edef\@p@sbbury{\number\dimen100}
	\no@bbfalse
}
% rotate point (#1,#2) about (0,0).
% The sine and cosine of the angle are already stored in \sine and
% \cosine.  The result is placed in (\p@intvaluex, \p@intvaluey).
\newdimen\p@intvaluex
\newdimen\p@intvaluey
\def\rotate@#1#2{{\dimen0=#1 sp\dimen1=#2 sp
%            	calculate x' = x \cos\theta - y \sin\theta
		  \global\p@intvaluex=\cosine\dimen0
		  \dimen3=\sine\dimen1
		  \global\advance\p@intvaluex by -\dimen3
% 		calculate y' = x \sin\theta + y \cos\theta
		  \global\p@intvaluey=\sine\dimen0
		  \dimen3=\cosine\dimen1
		  \global\advance\p@intvaluey by \dimen3
		  }}
\def\compute@bb{
		\no@bbfalse
		\if@bbllx \else \no@bbtrue \fi
		\if@bblly \else \no@bbtrue \fi
		\if@bburx \else \no@bbtrue \fi
		\if@bbury \else \no@bbtrue \fi
		\ifno@bb \bb@missing \fi
		\ifno@bb \ps@typeout{FATAL ERROR: no bb supplied or found}
			\no-bb-error
		\fi
		%
%\ps@typeout{BB: \@p@sbbllx, \@p@sbblly, \@p@sbburx, \@p@sbbury} 
%
% store height/width of original (unrotated) bounding box
		\count203=\@p@sbburx
		\count204=\@p@sbbury
		\advance\count203 by -\@p@sbbllx
		\advance\count204 by -\@p@sbblly
		\edef\ps@bbw{\number\count203}
		\edef\ps@bbh{\number\count204}
		%\ps@typeout{ psbbh = \ps@bbh, psbbw = \ps@bbw }
		\if@angle 
			\Sine{\@p@sangle}\Cosine{\@p@sangle}
	        	{\dimen100=\maxdimen\xdef\r@p@sbbllx{\number\dimen100}
					    \xdef\r@p@sbblly{\number\dimen100}
			                    \xdef\r@p@sbburx{-\number\dimen100}
					    \xdef\r@p@sbbury{-\number\dimen100}}
%
% Need to rotate all four points and take the X-Y extremes of the new
% points as the new bounding box.
                        \def\minmaxtest{
			   \ifnum\number\p@intvaluex<\r@p@sbbllx
			      \xdef\r@p@sbbllx{\number\p@intvaluex}\fi
			   \ifnum\number\p@intvaluex>\r@p@sbburx
			      \xdef\r@p@sbburx{\number\p@intvaluex}\fi
			   \ifnum\number\p@intvaluey<\r@p@sbblly
			      \xdef\r@p@sbblly{\number\p@intvaluey}\fi
			   \ifnum\number\p@intvaluey>\r@p@sbbury
			      \xdef\r@p@sbbury{\number\p@intvaluey}\fi
			   }
%			lower left
			\rotate@{\@p@sbbllx}{\@p@sbblly}
			\minmaxtest
%			upper left
			\rotate@{\@p@sbbllx}{\@p@sbbury}
			\minmaxtest
%			lower right
			\rotate@{\@p@sbburx}{\@p@sbblly}
			\minmaxtest
%			upper right
			\rotate@{\@p@sbburx}{\@p@sbbury}
			\minmaxtest
			\edef\@p@sbbllx{\r@p@sbbllx}\edef\@p@sbblly{\r@p@sbblly}
			\edef\@p@sbburx{\r@p@sbburx}\edef\@p@sbbury{\r@p@sbbury}
%\ps@typeout{rotated BB: \r@p@sbbllx, \r@p@sbblly, \r@p@sbburx, \r@p@sbbury}
		\fi
		\count203=\@p@sbburx
		\count204=\@p@sbbury
		\advance\count203 by -\@p@sbbllx
		\advance\count204 by -\@p@sbblly
		\edef\@bbw{\number\count203}
		\edef\@bbh{\number\count204}
		%\ps@typeout{ bbh = \@bbh, bbw = \@bbw }
}
%
% \in@hundreds performs #1 * (#2 / #3) correct to the hundreds,
%	then leaves the result in @result
%
\def\in@hundreds#1#2#3{\count240=#2 \count241=#3
		     \count100=\count240	% 100 is first digit #2/#3
		     \divide\count100 by \count241
		     \count101=\count100
		     \multiply\count101 by \count241
		     \advance\count240 by -\count101
		     \multiply\count240 by 10
		     \count101=\count240	%101 is second digit of #2/#3
		     \divide\count101 by \count241
		     \count102=\count101
		     \multiply\count102 by \count241
		     \advance\count240 by -\count102
		     \multiply\count240 by 10
		     \count102=\count240	% 102 is the third digit
		     \divide\count102 by \count241
		     \count200=#1\count205=0
		     \count201=\count200
			\multiply\count201 by \count100
		 	\advance\count205 by \count201
		     \count201=\count200
			\divide\count201 by 10
			\multiply\count201 by \count101
			\advance\count205 by \count201
			%
		     \count201=\count200
			\divide\count201 by 100
			\multiply\count201 by \count102
			\advance\count205 by \count201
			%
		     \edef\@result{\number\count205}
}
\def\compute@wfromh{
		% computing : width = height * (bbw / bbh)
		\in@hundreds{\@p@sheight}{\@bbw}{\@bbh}
		%\ps@typeout{ \@p@sheight * \@bbw / \@bbh, = \@result }
		\edef\@p@swidth{\@result}
		%\ps@typeout{w from h: width is \@p@swidth}
}
\def\compute@hfromw{
		% computing : height = width * (bbh / bbw)
	        \in@hundreds{\@p@swidth}{\@bbh}{\@bbw}
		%\ps@typeout{ \@p@swidth * \@bbh / \@bbw = \@result }
		\edef\@p@sheight{\@result}
		%\ps@typeout{h from w : height is \@p@sheight}
}
\def\compute@handw{
		\if@height 
			\if@width
			\else
				\compute@wfromh
			\fi
		\else 
			\if@width
				\compute@hfromw
			\else
				\edef\@p@sheight{\@bbh}
				\edef\@p@swidth{\@bbw}
			\fi
		\fi
}
\def\compute@resv{
		\if@rheight \else \edef\@p@srheight{\@p@sheight} \fi
		\if@rwidth \else \edef\@p@srwidth{\@p@swidth} \fi
		%\ps@typeout{rheight = \@p@srheight, rwidth = \@p@srwidth}
}
%		
% Compute any missing values
\def\compute@sizes{
	\compute@bb
	\if@scalefirst\if@angle
% at this point the bounding box has been adjsuted correctly for
% rotation.  PSFIG does all of its scaling using \@bbh and \@bbw.  If
% a width= or height= was specified along with \psscalefirst, then the
% width=/height= value needs to be adjusted to match the new (rotated)
% bounding box size (specifed in \@bbw and \@bbh).
%    \ps@bbw       width=
%    -------  =  ---------- 
%    \@bbw       new width=
% so `new width=' = (width= * \@bbw) / \ps@bbw; where \ps@bbw is the
% width of the original (unrotated) bounding box.
	\if@width
	   \in@hundreds{\@p@swidth}{\@bbw}{\ps@bbw}
	   \edef\@p@swidth{\@result}
	\fi
	\if@height
	   \in@hundreds{\@p@sheight}{\@bbh}{\ps@bbh}
	   \edef\@p@sheight{\@result}
	\fi
	\fi\fi
	\compute@handw
	\compute@resv}

%
% \psfig
% usage : \psfig{file=, height=, width=, bbllx=, bblly=, bburx=, bbury=,
%			rheight=, rwidth=, clip=}
%
% "clip=" is a switch and takes no value, but the `=' must be present.
\def\psfig#1{\vbox {
	% do a zero width hard space so that a single
	% \psfig in a centering enviornment will behave nicely
	%{\setbox0=\hbox{\ }\ \hskip-\wd0}
	%
	\ps@init@parms
	\parse@ps@parms{#1}
	\compute@sizes
	%
	\ifnum\@p@scost<\@psdraft{
		%
		\special{ps::[begin] 	\@p@swidth \space \@p@sheight \space
				\@p@sbbllx \space \@p@sbblly \space
				\@p@sbburx \space \@p@sbbury \space
				startTexFig \space }
		\if@angle
			\special {ps:: \@p@sangle \space rotate \space} 
		\fi
		\if@clip{
			\if@verbose{
				\ps@typeout{(clip)}
			}\fi
			\special{ps:: doclip \space }
		}\fi
		\if@prologfile
		    \special{ps: plotfile \@prologfileval \space } \fi
		\if@decmpr{
			\if@verbose{
				\ps@typeout{psfig: including \@p@sfile.Z \space }
			}\fi
			\special{ps: plotfile "`zcat \@p@sfile.Z" \space }
		}\else{
			\if@verbose{
				\ps@typeout{psfig: including \@p@sfile \space }
			}\fi
			\special{ps: plotfile \@p@sfile \space }
		}\fi
		\if@postlogfile
		    \special{ps: plotfile \@postlogfileval \space } \fi
		\special{ps::[end] endTexFig \space }
		% Create the vbox to reserve the space for the figure.
		\vbox to \@p@srheight sp{
		% 1/92 TJD Changed from "true sp" to "sp" for magnification.
			\hbox to \@p@srwidth sp{
				\hss
			}
		\vss
		}
	}\else{
		% draft figure, just reserve the space and print the
		% path name.
		\if@draftbox{		
			% Verbose draft: print file name in box
			\hbox{\frame{\vbox to \@p@srheight sp{
			\vss
			\hbox to \@p@srwidth sp{ \hss \@p@sfile \hss }
			\vss
			}}}
		}\else{
			% Non-verbose draft
			\vbox to \@p@srheight sp{
			\vss
			\hbox to \@p@srwidth sp{\hss}
			\vss
			}
		}\fi	



	}\fi
}}
\psfigRestoreAt
\let\@=\LaTeXAtSign





\begin{document}

\title{Teaching Software Engineering skills with the Leap Toolkit}

\author{
        \hspace*{-2ex}
        \parbox{4.3in} {\begin{center}
        {\authornamefont Carleton A. Moore}\\ 
        Collaborative Software Development Laboratory\\
        Information \& Computer Sciences Department\\
        University of Hawaii, Manoa\\
        Honolulu, Hawaii 96822  USA \\
        (808) 956-6920\\
        cmoore@hawaii.edu
        \end{center} }}
\maketitle
%
% If you want to print drafts of the paper with a draft 
% notice in the copyright space, comment out the \copyrightspace
% line above and include the \submitspace line below instead.
%
%\copyrightspace
\submitspace{Draft of paper submitted to ICSE 2000.}



% Page number on the initial page can be omitted in both the review and
% final submission (and should be removed in the final submission).  The
% line below does that.

\thispagestyle{empty}  % suppresses page number on first page

% In the review submission, page numbers should appear (they can be omitted
%  from the first page).  The pagestyle command below puts them in.
% In the final submission of accepted papers, page numbers should be
%  omitted; remove or comment out the pagestyle line below to omit them. 

\pagestyle{plain}

% Use \section* instead of \section to suppress numbering for
% the abstract, acknowledgments, and references.


\section*{ABSTRACT}
The Personal Software Process (PSP) teaches software developers many valuable
software engineering techniques.  Developers learn how to develop high quality
software efficiently and how to accurately estimate the amount of effort it will
take. To accomplish this the PSP forces the developer to follow a very strict
development model, to manually record time, to defect and size data, and analyze
their data.  The PSP appears successful at improving developer performance during
the training, yet there are questions concerning long-term adoption rates and
the accuracy of PSP data.

This paper presents our experiences using the Leap toolkit, an automated tool
to support personal developer improvement.  The Leap toolkit incorporates ideas
from the PSP and group review.  It relaxes some of the constraints in the PSP
and reduces process overhead.  We are using the Leap toolkit in an advanced
software engineering course at the University of Hawaii, Manoa.

\subsection{Keywords}
Software Developer Education, Process Improvement, Measurement, Personal Software Process

\section{INTRODUCTION}

Every software engineer occasionally wishes they got home from work sooner and spent
less of their weekends at work.  Software developers work very hard and very
long, yet software is often delivered late, over budget, and full of
defects. How can software engineers learn to produce high quality software more 
efficiently? 

Software developers and managers have addressed the issue of software quality
and development issues by focusing on the work product and the development
organization.  Work product solutions include practices such as testing and
reviews.  Testing and additionally reviews help improve the work product and
reviews attempt to reduce the cost of development by finding and fixing the
defect earlier in the development process.

Organization solutions focus on the software development organization and
processes.  There are hundreds of organization level solutions.  Two widely
followed organization-mode solutions are the Capability Maturity
Model\cite{Paulk95} and ISO 9000\cite{ISO9000}.  Both of these methods focus on
the development organization and processes.

Traditional software engineering techniques such as requirements
specifications, modularity, data abstraction, coupling and cohesion, PERT and
GANTT charts, version control, and so forth are ``best practice'' techniques
that provide developers tools for solving specific software engineering
problems. Software developers should know how to use these ``best practices''
on their development problems.  Not all practices are appropriate to every
situation, however using the correct practice can save a project.

When a developer joins a software development organization they must learn the
particular development processes for the organization.  The developer has to
learn what testing and review methods are used in the organization.  They must
learn how the organization manages the software development process.


In 1995, Watts Humphrey introduced the Personal Software Process in his book
{\em A Discipline for Software Engineering}\cite{Humphrey95}, a software
development process and improvement process for individual software developers.
The PSP incorporates many of the above best practices into a single method for
software developer improvement.

After using the PSP for over two years we developed the Leap toolkit to
overcome restrictions that we noticed in the PSP.  The Leap toolkit relaxes
many of the constraints in the PSP, and includes extensive automated support in 
order to simplify training and adoption of individual, empirically-based
developer improvement.

\section{Personal Software Process (PSP)}

The PSP teaches software developers techniques intended to support high-quality
software development and how to improve estimate the amount of effort required
to produce software.  These two goals drive the entire PSP process.

\subsection{Learning software development skills with the PSP}

To teach developers how to use the PSP, Humphrey defines seven PSP processes
(0, 0.1, 1.0, 1.1, 2.0, 2.1, 3.0). Each process has detailed scripts telling
the user exactly how to perform the process. Figure \ref{fig:psp-levels} shows
the seven levels.  Each level builds upon the previous levels and introduce new
software engineering concepts.  Exercises at the end of each chapter ask the
reader to use the knowledge from the chapter to improve their development
skills. The book teaches powerful development techniques: design and code
reviews, size and time estimation methods, and design templates. These
techniques help the developer produce high quality products efficiently.  As
developers go through the book they develop 10 small software projects using
the different PSP levels.

\tiny
\begin{center}
  \begin{figure}[htb]
    \setlength{\unitlength}{1.5cm}
    \begin{picture}(5,5)
      \put(0,0.4){\parbox[b]{1.0cm}{Baseline Personal Process}}
      \put(0.75,0.25){\framebox(1.25,0.75){\parbox[b]{1.5cm}{{\bf PSP0} Time
            \& Defect recording}}}
      \put(2.0,0.35){\framebox(1.25,0.75){\parbox[b]{1.5cm}{{\bf PSP0.1} Coding
            Standard, Size Measurement}}}
      \thicklines
      \put(1.5,1.25){\oval(1.0,1.0)[tl]}
      \put(1.5,1.75){\vector(1,0){0.2}}
      \thinlines
      \put(0.5,1.4){\parbox[b]{1.0cm}{Personal Planning Process}}
      \put(1.75,1.25){\framebox(1.25,0.75){\parbox[b]{1.5cm}{{\bf PSP1} Size
            \& Time estimation}}}
      \put(3.0,1.35){\framebox(1.25,0.75){\parbox[b]{1.5cm}{{\bf PSP1.1} Task
      \& Schedule Planning}}}
      \thicklines
      \put(2.5,2.25){\oval(1.0,1.0)[tl]}
      \put(2.25,2.75){\vector(1,0){0.2}}
      \thinlines
      \put(1.0,2.4){\parbox[b]{1.0cm}{Personal Quality Management}}
      \put(2.75,2.25){\framebox(1.25,0.75){\parbox[b]{1.5cm}{{\bf PSP2} Code \&
      Design Reviews}}}
      \put(4.0,2.35){\framebox(1.25,0.75){\parbox[b]{1.5cm}{{\bf PSP2.1} Design 
      Templates}}}
      \thicklines
      \put(3.5,3.25){\oval(1.0,1.0)[tl]}
      \put(3.5,3.75){\vector(1,0){0.2}}
      \thinlines
      \put(1.5,3.4){\parbox[b]{1.0cm}{Cyclic Personal Process}}
      \put(3.75,3.25){\framebox(1.25,0.75){\parbox[b]{1.5cm}{{\bf PSP3} Cyclic Development}}}
    \end{picture}
    \caption{PSP levels}
    \label{fig:psp-levels}
  \end{figure}
\end{center}
\normalsize

In the Baseline Personal Process level the developer learns how to record their
development process. They record all defects that they find in their source
code, all the time it takes them to develop the source code and the number of
lines of code in their source.  From PSP 0.1 on the developer must use a very
strict waterfall method of development.  100\% of the code must be written
before the developer can do their first compilation. This development method
teaches the developer how to manage their development process.  The next level
of the PSP teaches the developer how to plan their projects.

In the Personal Planning Process level the developer learns how to estimate and plan
future projects.  The developer starts the planning phase by estimating the
size of the next project.  The PSP uses a Proxy Based Estimation (PROBE)
technique to simplify size estimation.  To estimate the number of LOC for the
next project, the developer develops an initial design and counts the number of
methods.  Methods are the proxy for lines of code.  The developer estimates the
size of each method.  Based upon historical project size data the average LOC
for five different method sizes is calculated.  The developer then counts up
the number of methods of each size category and multiplies by the average LOC
per method for that size category.  The developer sums up all the LOC and gets
the total estimated LOC for the project.  Once the size of the project is
estimated the developer can estimate the amount of time the project will take.
The PSP uses a complicated time estimation technique.  The PSP time estimation
technique requires the developer to calculate two linear regressions.  The
first regression is based upon their planned size for each project vs the
actual time spent developing the project.  The second regression is between the
actual size of each project and the actual time of taken for the project.
Depending upon the correlation between the two size data sets, planned or
actual size, and the actual time taken, the developer uses linear regression or
historical average to calculate the effort for the next project.  Based upon
the estimated time and past direct hour calculation the developer can schedule
the project.  The next level of the PSP teaches the developer how to increase
the quality of their source code.

In the Personal Quality Management level the developer learns how to
efficiently find and remove defects from their code by using reviews.  They
also learn how to use design templates to help improve their design skills.
Prior to the introduction of design and code reviews most of the defects are
removed during the compile and test phases.  Adding design and code review
efficiently catches many defect before the compile phase.  At this level in the 
PSP developers can plan their projects and efficiently remove defects.  The
next level teaches developer how to build larger software products.

In the Cyclic Development Process level developers learn how to break large
project into smaller chunks.  The PSP teaches developer how to do high-level
design to find the cycles of development.  Each cycle of the large project is
considered a sub-project and the developer can use the PSP to manage them.  To
determine if the PSP is a powerful tool for improving an individual's software
engineering skills, many researchers have evaluated the PSP.

\subsection{Evaluations of the PSP}

In a 1996 article, Watts Humphrey reported the results of 104 engineers taking
the PSP course\cite{Humphrey96}. He states that the two goals of PSP were met.
First, reported defects fell from an average of 116.4 defects per thousand
lines of code (KLOC) for assignment 1 to 48.9 defects per KLOC for assignment
10. Second, the estimation accuracy of the students increased.  For assignment
1 32.7\% of the engineers' estimates were within 20\% of their actual times. By 
assignment 10 49.0\% of the engineer's estimates were within 20\% of their
actual times.

In 1996, Sherdil and Madhavji studied human-oriented improvement in the
Software Process\cite{Sherdil96}. They used PSP as a basis for their studies.
They found that subjects reduced their defect by 13\% after project 6, when
code reviews are introduced. They also found that their subjects reduced their
size estimation error by more than 7\% than expected.

Hayes and Over conducted an extensive study, with 298 engineers, of the
PSP\cite{Hayes97}.  The results of the study were impressive. Over the projects
completed, the median improvement in size estimation was a factor of 2.5.  This
means that 50\% of the engineers reduced their size estimation error by a
factor of 2.5.  The median improvement in time estimation was 1.75.  The median 
reduction in overall defect density was by a factor of 1.5.  The engineers
substantially reduced the percentage of defects surviving to later stages of
development. 

Anne Disney did her masters thesis on data quality issues in the
PSP\cite{Disney98}.  She found that in her sample of students who learned the
PSP, the errors in their data were significant.  These errors lead to incorrect
insights into the students development practices.  For example, in several
cases the students' incorrect data indicated that they were over estimating
their yield when in fact they were underestimating their yield.

Most of the errors that Disney found in the the students' PSP data were caused
by the manual nature of the PSP.  66\% of the errors were either calculation
errors or transfer errors.  The students had difficulty correctly manually
calculating some of the values required and transferring the values between the
many forms used in the PSP.  This research and our experiences with using the
PSP raised some important issues with the PSP.

\subsection{Issues with the PSP}

After using the PSP for over two years, we noticed several issues with the PSP:
\begin{itemize}
\item{The PSP is designed for only for software development. It hardwires
    estimation, size measurement, and development processes. Since the PSP is
    designed around a set of hardcopy forms, the processes in PSP are
    intimately tied to the hardcopy forms.  Modifying the forms or the
    processes are sufficiently difficult that the Humphrey strongly advises
    against doing so, at least until after finishing the course.
    
    The entire PSP process is focused on one type of work product, source code.
    If the software developer does not produce source code they cannot use the
    PSP. They must also modify most of the forms if they do not have a
    ``compile'' phase in their development process.  Some of the key analyses
    calculate the number of defects before and after ``compile''.  Without a
    ``compile'' phase these analyses are meaningless.}
\item{The manual nature of the PSP introduces large amounts of overhead for the
    developer.  It also reduces the benefits to the developer since it is
    difficult to analyze all the data.  For each project the developer records
    their time, size and defects on log forms.  At the end of the project they
    produce a postmortem report that summarizes the project.  In higher levels
    of the PSP the developer must use over 10 different forms.  Producing a
    report of the most costly defect type is a very time consuming process
    since the developer must pour through all of their defect logs and collate
    the data manually.}
\item{The PSP collects all defects from the first project.  This project
    introduces additional overhead to the developer's development process.
    They have to follow a new development process and record their effort.  The
    addition of recording all their defects changes their development process
    significantly.  Manually recording each defect on the defect recording log
    interrupts their thought processes.}
\item{The PSP has no group support. The PSP is a personal process, however some
    of the most valuable insights about your development process may come from
    other developers that can see issues that you cannot.  Coaches observe
    athletes perform and provide insight the athlete cannot get themselves,
    similarly in software development other observers can provide insights that
    the developer cannot see themselves.}
\end{itemize}

These issues motivated us to begin designing an automated, empirically based,
personal process improvement tool.  Our goal is to reduce the collection and
analysis overhead for the developer.  This should improve the benefits to the
developer and the long term adoption of empirically based process improvement.
To pursue this work, we initiated Project LEAP, {\footnotesize {\tt
    <http://csdl.ics.hawaii.edu/Research/LEAP/LEAP.html>}}, and began
developing the Leap Toolkit.

\section{Leap}



\subsection{Design Criteria}

We hypothesized that improved support for software developer improvement would
be obtained by attempting to satisfy four major design criteria: a light-weight
process, empirical measurement, minimization of measurement dysfunction, and
portability within and across organizations.

\subsubsection{Criteria \#1: Light-Weight}

The first principle is that any tool or process used in software developer
improvement should be light weight.  This means that the tool or process should
not impose too much overhead on the developer.  Data collection should be easy
to perform and should not add significant effort to the process.  The processes
that are used should not impose a burden on the developer. We do not want the
developer to worry about the improvement effort while they are doing the
development.  They should be worrying about the development.  Analyses and
other work should also require as little effort by the developer as possible.
The benefit of using the improvement processes should outweigh the cost of to
the developer.

This principle implies that any improvement process must be automated as much
as possible.  A manual process requires too much overhead by the developer.
The overhead of recording information by hand and manually doing the analyses
will out weigh the benefits of the process.  The PSP suffers from the problem
of high overhead for data entry and analysis.  It also suffers from high
process overhead.

\subsubsection{Criteria \#2: Empirical}

We believe in empirical data collection--that improvements should be based
upon the developer's experiences. Leap supports the observe-evaluate-modify
improvement cycle.  Each modification is then tested by
further observation to see if the change is actually an improvement or just a
false start.  By using looking at their development empirically the developer
is able to judge for themselves what is best.

\subsubsection{Criteria \#3: Anti-measurement Dysfunction}

Based upon our experiences with industrial software development and Richard
Austin's book {\em Measuring and Managing Performance in
  Organizations}\cite{Austin96}, we believe that any process improvement method
should deal with the issue of measurement dysfunction. The empirical data
collected could be misused.  This issue is important since the development
process is very interesting to people other than the developer.  If there is
measurement dysfunction then the data collected and analyses will not reflect
reality.  Any insights gained from this data and analyses will be faulty and
may cause more problems than they solve.

\subsubsection{Criteria \#4: Portable}

Software developers often change jobs and the tool support for their
development improvement should be portable.  They should be able to take their
data and the tool support with them when they change organizations or jobs. A
tool that supports developer improvement that cannot follow the developer as
they move is not going to help those developers very much.


Based upon the four design principles we developed the Leap Toolkit, a Java
application for software developer improvement.  The Leap Toolkit combines the
data recording and analysis ideas of the PSP with group review.  The developer
can record time, size, and defect data for their project just like in the PSP.
They can also ask their co-workers to review their work product using the Leap
Toolkit and share any defects the reviewers find.  The Leap Toolkit also
supports checklists and patterns. 


\subsection{Benefits of Leap}

The Leap Toolkit provides many benefits and has allowed us to explore
different aspects of personal software process improvement.  Some of the
benefits are fewer constraints, lower overhead and support for group review. 

\subsubsection{Relaxes many constraints in the PSP}
The Leap toolkit allows the developer to define their own size definitions,
development processes, defect types, defect severities and document types. 

By allowing the developer to define their own hierarchical size definition the
Leap toolkit can support different size measurements for the same work product.
For example, a developer might define a object oriented size definition for
Java that consists of packages that contain classes that contain methods that
have lines of code. They might also define a size definition that just counts
the number of expressions in the source code.  Both of these size measures
could be used for the same source code.  The Leap Toolkit allows the developer
to compare their projects in either of these size definitions.

The Leap toolkit allows the developer to define their own processes.  They can
use the PSP's software development process or use their own process. The
developer can define processes for non-software development activities.  For
example, one writing process includes the following phases: Brain storming,
Planning, Outlining, Writing and Editing. 

Leap also relaxes the constraint that time, size and defects must always be
recorded.  In our Advanced Software Engineering classes we wait until the
students have mastered time and size recording before we introduce defect
recording.  This reduces the cognitive overhead on the students.  They become
accustomed to collecting data about their development process without overly
disrupting their process.  Teaching the developer how to collect the data that
they are interested in gives them more control over their process improvement
efforts.

\subsubsection{Low overhead}

\begin{figure}[htb]
  \centerline{\psfig{figure=io-word.epsi,width=3.3in}} 
 \caption{Time Recording Tool (Io) recording time for this paper. This screen
 shot illustrates an interactive time. During this session, 70 minutes of
 direct time has elapsed with almost 12 minutes of interrupt time.}
\label{fig:io}
\end{figure}

Since the Leap Toolkit is an automated tool the amount of overhead for
recording size, time and defect is greatly reduced.  The developer does not
have to take their hands off the keyboard and mouse to enter their data.  Time
recording is simplified by a single line time entry tool. Figure \ref{fig:io}
shows the time entry tool Io recording time for this paper.

The Leap Toolkit also has a similar tool for defect collection.  This tool
simplifies the recording of defects and their fix times.  Figure \ref{fig:mano} 
shows the defect recording tool Mano.

\begin{figure*}[htb]
  \centerline{\psfig{figure=mano-word.epsi}} 
 \caption{Defect Recording Tool (Mano) recording a simple syntax error found in 
 a draft of this paper.}
\label{fig:mano}
\end{figure*}

As a part of his Master's thesis Joseph Dane developed LOCC\cite{Dane99} a
grammar based size counting tool that produces size data that the Leap Toolkit
can use.  Currently, LOCC supports counting Java, C++ and plain ASCII text
files. Using LOCC greatly simplifies the problem of determining the actual size
of the current project.

All of these tools for data collection have reduced the developer's overhead so
much that, in many cases, it now takes more effort to fake your data than it
takes to collect accurate data.  In her study of students' PSP data, Anne
Disney found some suspicious data that possibly indicated that the students
were making up their data. The Leap toolkit solves this problem--at least for
situations where the data is faked due to time constraints.

The Leap Toolkit also allows the developer to rapidly analyze their data.  In
the manual PSP, comparing different time estimation techniques is very time
consuming.  Leap allows the developer to rapidly compare different time
estimation models such as: linear regression, historical average, exponential
regression, and power regression.  The Leap Toolkit also allows the developer
to quickly compare their planning values to their actual values. Figure
\ref{fig:timeest} shows the Leap time estimation tool.
\begin{center}
  \begin{figure*}[htb]
    \centerline{\psfig{figure=timeest.ps,width=7in,angle=270}} 
    \caption{Time Estimation Tool with the author's Java development
    data. Showing the different time trend line options.}
    \label{fig:timeest}
  \end{figure*}
\end{center}
\subsubsection{Support group review}

To provide additional insight into a developer's process, the Leap Toolkit
supports group review.  Users can email their Leap data to each other.  This
allows reviewers to use the Leap Toolkit to record and share defect data.  The
Leap toolkit can load all the defects found by the different reviewers and then 
filter, and sort the data.  The insights provided by other reviewers might be
even more valuable than the defect detected by the developer.

The Leap Toolkits flexibility even allows reviewers to record data about their
review process and improve their review techniques.  We can use the toolkit to
teach developers different review methods and compare the results of these
different methods. 


\section{Strengths: Leap vs. PSP}

The PSP's strong scripts tell the developer exactly what they have to do and
exactly what data they must collect.  There is very little ambiguity when using 
the PSP.  The user just follows the scripts and does the analyses called for.
The developer does not have to create their own goals, questions, and metrics:
they are built into the PSP.

Leap builds upon many of the strengths of PSP.  Automated tool support for data
collection and analysis greatly reduces the user's overhead.  By relaxing many
of the constraints, Leap gains valuable flexibility. Developers can use the
Leap toolkit to improve their development skills in a wide range of activities
not possible with the PSP.  Developers are able to experiment with different
development techniques to find the one best suited to them.

\section{Experiences with Leap}
We started developing Leap in the summer of 1997. Our first release, 1.7.0, of
the system, in November 1997, only supported recording and analyzing defect
data.  Since then we have made 25 public releases of the Leap toolkit.
Currently, the Leap toolkit, version 5.8.2, consists of over 41,000 lines of
Java code, over 2,000 methods, and over 275 classes in 13 packages.  It is
available for down-load from {\footnotesize {\tt
<http://csdl.ics.hawaii.edu/Tools/LEAP/LEAP.html>}}.  

We have used the Leap toolkit to help teach advanced software engineering at
the University of Hawaii.  The Leap toolkit is currently being used by Dr.
Philip Johnson in ICS 613: Advanced Software Engineering {\footnotesize {\tt
    <http://csdl.ics.hawaii.edu/$\sim$johnson/613f99/>}}.  Over the past 2 and
a half years we have collected data on over 350 projects.  The Leap toolkit has
been used by several people working in industry not affiliated with the
Collaborative Software Development Laboratory.

We are currently conducting an evaluation of the Leap toolkit by surveying and
interviewing the students in ICS 613.  We are also using the Leap toolkit to
investigate the accuracy of different time estimation methods based upon
historical size and time data.  The initial results of our evaluation will be
available in January, 2000.


%%% Input file for bibliography
\bibliography{/export/home/csdl/bib/psp,/export/home/csdl/bib/leap,/export/home/csdl/bib/csdl-trs}
%% Use this for an alphabetically organized bibliography
%\bibliographystyle{plain}
\bibliographystyle{abbrv}
\onecolumn

\end{document}

