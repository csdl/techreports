\documentstyle[twocolumn,times,/group/csdl/tex/icse99]{article}
\input{/group/csdl/tex/psfig/psfig}

% Changes: 
%  References to 97 are now to 99
%  Page limit is now 10 pages instead of 11
%  Abstract limit is now 200 words instead of 100
%  Contact information is now icse99@cs.cmu.edu and 412-268-5056.
%  Acknowledgements have been changed to note ICSE 97 contribution.
%  URL for author kit changed: needs to be filled in.
%  All other font and sizing information is the same:
%    check whether it still applies.

\begin{document}

\title{Leap: A ``Personal Information Environment'' \\ for Software Engineers}

\author{
        %\hspace*{-2ex}
        \parbox{4.3in} 
        {\begin{center}
        {\authornamefont Philip M. Johnson}\\ 
        Collaborative Software Development Laboratory\\
        Department of Information and Computer Sciences\\
        University of Hawaii\\
        Honolulu, HI 96822 USA \\
        johnson@hawaii.edu
        \end{center} }
}



\maketitle
\copyrightspace
%
% If you want to print drafts of the paper with a draft 
% notice in the copyright space, comment out the \copyrightspace
% line above and include the \submitspace line below instead.
%
% \submitspace{Draft of paper submitted to ICSE 99.}


% Page number on the initial page can be omitted in both the review and
% final submission (and should be removed in the final submission).  The
% line below does that.

\thispagestyle{empty}  % suppresses page number on first page

% In the review submission, page numbers should appear (they can be omitted
%  from the first page).  The pagestyle command below puts them in.
% In the final submission of accepted papers, page numbers should be
%  omitted; remove or comment out the pagestyle line below to omit them. 

% \pagestyle{plain}

% Use \section* instead of \section to suppress numbering for
% the abstract, acknowledgements, and references.

\section*{ABSTRACT}

The Leap toolkit is designed to provide Lightweight, Empirical,
Anti-measurement dysfunction, and Portable approaches to
software developer improvement. Using Leap, software engineers gather and
analyze personal data concerning time, size, defects, patterns, and
checklists. They create and maintain definitions describing their software
development procedures, work products, and project attributes, including
document types, defect types, severities, phases, and size definitions.
Leap also supports asynchronous software review and facilitates integration
of this group-based data with individually collected data.  The Leap
toolkit provides a ``reference model'' for a personal information
environment to support skill acquisition and improvement for software
engineers.

\subsection{Keywords}
Software developer improvement, metrics

\section{INTRODUCTION}

The demands of software development on ``Internet Time" include shortened
time to market, reduced development budgets, and faster release cycles.
The pace of technical and economic innovation in Internet Time industries
tends to result in increased organizational volatility, including frequent
restructuring and high employee turn-over.  These combined pressures can
wreak havoc with traditional, top-down process improvement initiatives,
which typically require: (a) sustained commitment from top-level management
for years at a time; (b) ``champions" who remain within the organization
with stable responsibilities; and (c) a stable developer, product, and
market focus so that any process improvement opportunities identified
during one product or development cycle remain relevant during the next.
Finally, top-down process improvement initiatives tend to incur significant
financial and administrative costs to implement and administer the program, report its
findings, and justify its continued existence.


{\em Empirical} top-down process improvement initiatives must combat an
additional problem: measurement dysfunction.  Research by Robert Austin on
software development organizations identifies measurement dysfunction as a
significant obstacle to process improvement \cite{Austin96}.  Measurement
dysfunction refers to a situation in which the act of measurement affects
the organization in a counter-productive fashion.  Such dysfunction occurs
because many process measures have two potential applications: (1) to
provide {\em information} to the organization and (2) to support {\em
  performance evaluation} of individuals and groups.  Since it is
impossible for an organization to guarantee that a measure, once collected
by the organization, will never be used for performance evaluation, process
measures may be skewed to reflect what the organization (or process
improvement team) wants or needs to hear, rather than what is actually
occurring in the organization.


Despite these concerns, traditional top-down process improvement
initiatives remain an important and valuable component of a high quality
software development organization.  However, it is also possible to pursue
a ``bottom up", developer-centered approach that addresses many of these
concerns. In a bottom-up approach, the focus is on providing individual
developers with the insights necessary to acquire and improve their
technical skills.  Management buy-in and support becomes secondary to the
developers' self-interest in their own professional development.
Management reports on the progress and success of the individual's skill
acquisition efforts are no longer required and in fact counterproductive,
since preserving the privacy of personal measurements and insights is
crucial to preventing measurement dysfunction.  Finally, the tendency of
modern software developers to frequently change organizations can undermine
their commitment to top-down process improvement initiatives, while a
bottom-up approach represents a ``portable" activity that the developer can
maintain across jobs and organizations.


For two years, we have pursued a research initiative regarding bottom-up
technical skill acquisition and improvement called Project Leap.  We hope
through this research initiative to uncover some of the principles
underlying successful bottom-up process improvement.  Project Leap
leverages our prior research experiences in formal technical review
\cite{Johnson98} and the Personal Software Process \cite{Disney98}.  Based
upon these experiences, we conjecture that approaches to bottom-up process
improvement are made more effective by obeying the four ``Leap" design
constraints:

\begin{itemize}
  
\item {\bf L}ightweight. Bottom-up methods should be light\-weight.  In
  other words, they must involve a minimum of process constraints, be easy
  to learn, be easy to integrate with existing methods and tools, and require
  minimal investment and commitment from management. If a bottom-up method
  imposes new overhead on a developer, then that effort should yield a
  short-term, positive return-on-investment to that same developer.
  
\item {\bf E}mpirical. Bottom-up methods should have a quantitative, as
  well as qualitative dimension.  A lightweight orientation cannot be
  gained at the expense of high quality collection and analysis of data.
  Developer improvement should be observable over time through measurements
  of effort, defects, size, and time, in combination with improvements in checklists, 
  patterns, and so forth.
  
\item {\bf A}nti-measurement dysfunction.   Measurement, while an
  integral part of effective bottom-up methods, should be carefully
  designed to minimize dysfunction.  Yet the most simple solution to
  dysfunction---making all data totally private---is incompatible with the
  benefits to the organization of sharing certain kinds of data and process
  improvements. A goal of Project LEAP is to find a suitable balance between
  public and private measurement data.
  
\item {\bf P}ortable. 
  Useful developer improvement support should not be tied to a particular
  organization such that the developer must ``give up'' the data and tools
  when they leave the organization. Rather, this support should be akin to a
  developer's address book; a kind of ``personal information environment''
  for their software engineering skill set that they can take with them
  across projects and companies.

\end{itemize}

These four criteria, when composed together, create additional
requirements. For example, we believe that extensive automation is required
for any method that is simultaneously lightweight, empirical, and
anti-measurement dysfunction.  On the other hand, automation clearly does
not guarantee lightweight processes or meaningful empirical evidence of
improvement. As an example, one criticism of our CSRS automated review
system \cite{Johnson94} was that its extensive measurement system would
lead to dysfunctional behavior in an industrial setting.


Our efforts in Project Leap have produced a toolkit which has been in
public release for approximately one year, and in active classroom and
research use for approximately six months. The Leap toolkit is also under
small scale evaluation at two of our industrial affiliate sites, and we
intend to pursue broader industrial evaluation over the coming year. 
Indeed, our motivation for this research demo presentation is to 
introduce the toolkit to a broad audience and solicit increased 
involvement in its evaluation from the research and industry communities. 


The Leap toolkit is implemented in 100\% Java and runs on most platforms.
The most recent release is available for download from our research group
home page at http://csdl.ics.hawaii.edu/.


\section{The Leap method}

The tools in the Leap toolkit are designed to support the following general
paths of personal and review-based data collection and analysis as
illustrated in Figure \ref{cycle}.

\begin{figure} [tbp]
    {\centerline{\psfig{figure=cycle2.eps}}}
    \caption{\label{cycle} Paths of data collection
    and analysis in Leap. Several potential entry and exit
    points exist.}
\end{figure}

  
In Leap, there are two ``central'' activities: gathering primary data and
performing Leap analyses. These can be augmented by secondary activities of
refining definitions, checklists, and patterns. Finally, these central
and/or secondary activities can be directed toward individual skill
acquisition and improvement or group review of work products. The following 
paragraphs provide a bit more detail:

\begin{itemize}
  
\item {\em Generate or refine goals for technical skill acquisition and
    improvement.} Example goals could include improved estimation of size
  or time, improved skill at upstream design, increased direct hours on
  major tasks, decreased incidence of certain classes of defects, etc.

\item {\em Generate or refine definitions of projects, defect types,
    document types, etc.} Leap definitions provide two benefits: (1) they
  reduce the overhead of data collection (by providing pull-down or pop-up
  menu support for definition usage), and (2) they aid in analysis (by
  ensuring common terminology between projects).
  
\item {\em Collect primary data on size, time, and defects.} The basic Leap
  toolkit has been augmented with specialized tool support for in-process
  time data collection, in-process defect collection, and with a tool for
  hierarchical, grammar-based size calculation.
  
\item {\em Obtain additional defect data via group review.} Leap builds in
  support for asynchronous review and dissemination of review artifacts via
  the web and/or email.
  
\item {\em Perform analyses on primary data.} Leap builds in dozens of
  analyses accessible through pull-down menus, including charts and tables
  providing project summary statistics on time, size, and defects; defect
  types, rates (size/time, defects/time), densities (defects/time,
  defects/size), trends, and planning/estimation tool support.
  
\item {\em Evaluate progress toward goals.} Leap analyses provide software
  engineers with quantitative data that they can use to determine if they
  are making progress toward their goals. Sample goals include
  targets for direct hours applied to given projects, reduction in the
  frequency and/or expense of certain types of design or implementation
  defects, improvement in review efficiency and/or effectiveness, and
  improvement in the accuracy of planned size and time values.  Figure
  \ref{leap} illustrates one example analysis chart for planning.
  
\item {\em Generate checklists and patterns. } To support defect prevention
  and encode ``best practice", Leap enables developers to generate checklist
  items and simple process pattern information.

\end{itemize}


Leap is similar to other formulations for individual and group process
improvement in software engineering, such as Goal-Question-Metric (GQM)
\cite{Basili84} and the Personal Software Process (PSP) \cite{Humphrey95}.
It is the attempt to satisfy the Leap constraints in a bottom-up context
that produces differences between the Leap toolkit and many other
approaches.

First, Leap does not record authorship or other identification; all data
collected and manipulated in Leap is unattributed.  In a personal
environment, authorship is unnecessary, and lack of attribution is a small
but important step toward decreased measurement dysfunction.

Second, while Leap provides various definition mechanisms to enable
developers to describe what kind of procedures they used to develop a work
product or perform a review, Leap makes no attempt to enforce or assess
compliance with a particular procedure. Indeed, Leap recognizes that useful
definitions must be ``bootstrapped" over time through the use of the tool.

Third, Leap enables developers complete control over what kinds of Leap
information is shared with others. While developers typically are happy to
share certain kinds of insights, such as checklists and patterns, we have
found that time data is especially susceptible to measurement dysfunction.
Thus, for example, Leap makes it easy for developers to keep track of time
they devote to a review activity, but provide to others only the set of
defects they uncovered during review.

Fourth, Leap provides an integration mechanism for both personal software
engineering data and data generated through the process of group review.

Leap is similar in many ways to the Personal Software Process (PSP).  The
essential differences between Leap and the PSP are as follows.  First, the
PSP views automated support as helpful but optional.  In contrast, Leap
views automation as essential to reducing the overhead of process data
collection and analysis to an acceptable level. Our prior research also
indicates that automation may be necessary (though not sufficient) for
collection of accurate personal process data \cite{Disney98}.  Second, the
PSP involves an essentially ``heavyweight" orientation toward process
definition and adherence: in the PSP, one is instructed to follow quite
rigidly defined process scripts which sometimes involve practices quite
unfamiliar to most developers (such as to completely code all system
definitions before compiling for the first time).  Leap allows a more
``lightweight" orientation, in which one can begin collecting and analyzing
data without a great deal of process definition, adding such definitions
incrementally when deemed useful.  Third, unlike PSP, Leap integrates
support for asynchronous review as an essential service in the toolkit.
Fourth, the PSP requires you to collect data on your defects---what you do
wrong. In addition to defects, Leap also helps you to collect data on your
patterns---what you do right.

\begin{figure*} [t]
    {\centerline{\psfig{figure=planword.eps}}}
    \caption{\label{leap} 
    A sample analysis in Leap allowing the developer to use historical
    data to estimate the time required for a new project. The tool
    produces estimates by applying either linear regression or
    average/min/max analyses to historical data. The size metric used 
    is user-definable. 
   }
\end{figure*}

\section*{STATUS}

Leap has been in internal and public release for approximately one year,
and we are now seeking greater external use in academic and industrial
settings. We provide three paths for extension of the platform. First,
the Leap data file specification is a simple, restricted HTML format,
which facilitates interoperation of Leap with other software 
engineering tools at the file level. Second, Leap includes an
``extensions'' mechanism, which allows developers to write Java 
code that can be dynamically linked to the Leap at invocation time and 
allow third-party Java tools to extend Leap with new menus and 
applications. Finally, we intend to make an open source version of 
the Leap tool kit available for direct modification, experimentation, and
enhancement by the software engineering community.



\section*{ACKNOWLEDGMENTS}

I gratefully acknowledge my colleagues in the Collaborative Software
Development Laboratory (Cam Moore, Robert Brewer, Jennifer Geis, Joe Dane,
Jay Corbett, Anne Disney, and Russ Tokuyama).  This research was sponsored
in part by grants CCR-9403475 and CCR-9804010 from the National Science
Foundation.

\bibliographystyle{plain}
\bibliography{/group/csdl/bib/psp,/group/csdl/bib/ftr} 

\end{document}
