%%%%%%%%%%%%%%%%%%%%%%%%%%%%%% -*- Mode: Latex -*- %%%%%%%%%%%%%%%%%%%%%%%%%%%%
%% 10-07-system.tex --  HICSS 44 Kukui Cup paper
%% Author          : Philip Johnson
%% Created On      : Mon Sep 23 11:52:28 2002
%% Last Modified By: Philip Johnson
%% Last Modified On: Thu Jun 10 16:08:09 2010
%%%%%%%%%%%%%%%%%%%%%%%%%%%%%%%%%%%%%%%%%%%%%%%%%%%%%%%%%%%%%%%%%%%%%%%%%%%%%%%
%%   Copyright (C) 2009 Philip Johnson
%%%%%%%%%%%%%%%%%%%%%%%%%%%%%%%%%%%%%%%%%%%%%%%%%%%%%%%%%%%%%%%%%%%%%%%%%%%%%%%
%% 

\section{System Design}
\label{sec:system-design}

As our related work findings illustrate, current software for energy
competitions tends to be either commercial, closed systems, or special
purpose, ``on-off'' systems.  We strive in this project to create
software with an architecture that is open, extensible, and easily tailored
to the needs of different universities.  We intend the software
infrastructure from this project to provide as much of a research
contribution as our actual experimental results.

The following general requirements inform our system design.

\noindent {\em Open source.}  To maximize the potential for community
participation in development as well as use of the software, we make
all components available as open source, and utilize only freely
available third party components for development.  There are no software
costs associated with the use of our system.

\noindent {\em Platform, language, and metering infrastructure agnostic.}
We want to avoid lock-in to any particular platform, language, or metering
technology.  To avoid platform lock-in, we develop all components using
technologies such as Java, Python, Javascript, and Google Visualizations
that are available on Windows, Macintosh, and Linux platforms.  To avoid
language lock-in, the system observes a service-oriented architecture,
where components communicate with each other over HTTP via a RESTful API.
This isolates language dependencies to individual services.  For example,
the WattDepot server is written in Java, while the Makahiki web application
is written in Python.  Finally, to avoid metering technology lock-in, the
system architecture involves ``sensors'' that query any given meter using
its native protocol, then translates that into a common format for use in
the rest of the system.  Thus, adapting the system to a new meter
technology simply involves implementing the sensor for that technology.

\noindent {\em Feature subsetting.} Not all universities need or want the same 
level of sophistication in their dorm energy competitions.  In reviewing 
other sites, we found a wide spectrum of sophistication with respect to 
the kinds of information collected and the way it is presented.  Our software
is designed to support a variety of different use cases. 











