%%%%%%%%%%%%%%%%%%%%%%%%%%%%%% -*- Mode: Latex -*- %%%%%%%%%%%%%%%%%%%%%%%%%%%%
%% 10-07-experiment.tex --  HICSS 44 Kukui Cup paper
%% Author          : Philip Johnson
%% Created On      : Mon Sep 23 11:52:28 2002
%% Last Modified By: Philip Johnson
%% Last Modified On: Thu Jun 10 15:05:36 2010
%%%%%%%%%%%%%%%%%%%%%%%%%%%%%%%%%%%%%%%%%%%%%%%%%%%%%%%%%%%%%%%%%%%%%%%%%%%%%%%
%%   Copyright (C) 2009 Philip Johnson
%%%%%%%%%%%%%%%%%%%%%%%%%%%%%%%%%%%%%%%%%%%%%%%%%%%%%%%%%%%%%%%%%%%%%%%%%%%%%%%
%% 

\section{Evaluation}
\label{sec:evaluation}

Our work addresses the following research questions:

\begin{enumerate}
	\item To what extent and in what ways does our dorm energy competition impact the ``energy literacy'' of participating students?
	\item How effective is our use of information technology to support behavioral change tools?
	\item To what extent does our approach yield sustained changes in energy behavior, and what factors appear to influence sustained change?
\end{enumerate}

To find the answers to these questions, we have planned a dorm energy competition for Fall 2010 semester at the University of Hawai`i at M\=anoa. The rest of this section describes the competition design, the data we plan to gather, and how we intend to analyze the data.

\subsection{Competition design}

The competition is planned to take place in multiple freshman dormitories on the M\=anoa campus. Freshmen have been targeted since they are deemed more likely to participate in dormitory events, they are a ``renewable resource'', and past research has shown that freshmen perform well in these competitions \cite{petersen-dorm-energy-reduction}. There are 10 floors per residence hall, with 26 residents per floor at full occupancy, resulting in 260 potential participants per building.

The competition will take place over three weeks in the Fall 2010 semester. Each of the first two weeks will constitute a separate round of the competition, while the results from the final week apply only to the overall competition. Structuring the competition into rounds ensures that residents that did not participate initially can start participating in a later round without undue disadvantage.

The energy usage of the participants will be measured using power meters we will install in the central electrical panels on the floors of the dorms. Due to the architectural design and electrical infrastructure of the buildings, the meters will measure the energy consumption of each pair of floors. The meters to be installed will support sampling every 15 seconds, enabling near-realtime energy feedback display. The meter data will be stored using WattDepot.

There will be two scores for the competition: energy consumption and Kukui Nut points. Energy consumption is the total amount of electrical energy consumed by a pair of floors in kWh during a round as measured by the power meters, so lower energy consumption scores are better. Kukui Nut points are awarded through the competition website (powered by Makahiki) for the completion of tasks intended to increase participants' energy literacy or reduce energy usage. Kukui Nut points are awarded to individuals through the website (though they can also be aggregated at the floor or dorm level), but the energy consumption is only recorded at the floor level. We will provide awards based on energy consumption (at floor and dormitory levels), and Kukui Nut points (at individual, floor, and dormitory levels), many with associated prizes to incentivize participation.

In addition to the information technology support of the competition, we will deploy a variety of other methods to engage residents in the competition, such as a kick-off meeting for each dorm where free T-shirts will be distributed, buttons to be distributed to all residents, signage on each floor about the competition, and closing grand prize ceremony.

\subsection{Data sources}

We plan to collect a variety of types of data from the competition. We will record both instantaneous power and cumulative energy consumed on a floor by floor basis for each residence hall, both before the competition starts and continuing for at least 6 months after the competition ends. The sampling rate will be a minimum of 1 minute outside the competition period, and a maximum of 1 minute during the competition period (with a target of 15 seconds), with both rates kept constant during the study to the degree possible.

The energy literacy of participants will be assessed at the start and end of the competition. The assessment will be through a questionnaire that is presented to participants via the contest website as an activity that can be performed for Kukui Nut points. The pre-competition questionnaire will be made available only in the first week of the competition, while the post-competition questionnaire will be made available only in the final week of the competition. Since the website-administered questionnaire is simply a task that can selected by participants, there is the potential that only those participants that feel that they are energy literate will participate in the survey, leading to bias. For this reason, in addition to administration through the website, the questionnaire will be administered in person on paper to two randomly-selected floors.

The competition website will log data about participants' actions on the site. All participant actions and events will be logged with a timestamp. Some example events are: logging into website, selecting a goal for floor participation, and submitting text to verify completion of an activity. These events can be used to create a profile of each participant.

After the competition has ended, participants that used the website will be emailed a link to a qualitative questionnaire. This questionnaire will ask for participants' assessment of the competition, the website, and energy literacy in general.

In early in the following semester (February 2011), the power data for floors will be re-examined to see whether conservation begun as part of the competition has been sustained months later. Floors with particularly high sustained conservation (compared to pre-competition average floor power), and those with low or non-conservation will be selected for an additional questionnaire, and possible face-to-face interviews to determine residents' self-assessment about why they were or were not sustaining the conservation gains made during the competition.

\subsection{Analysis}

Using the energy literacy surveys from before and after the competition, we can 
address the first research question: the impact of the competition on the energy literacy of the participants. Increased scores in post-competition energy literacy would provide an indication that the activities of the competition may increase energy literacy. We will also examine the opposite relationship, to see how the energy literacy of a pair of floors correlates to the energy consumption of those floors during the competition.

There are several ways to address the second research question: the effectiveness of our information technology to support behavioral change tools. One basic metric will be to examine the website logs to see how many residents actually participate in the competition by logging into the website, how often they log in, and how many tasks they complete. The effectiveness of the tasks in improving energy literacy will be assessed by examining the correlation between Kukui Nut points awarded per participant, and their performance on the energy literacy surveys. The relationship between a floor's energy usage and its aggregated Kukui Nut points will provide another window into the effectiveness of the information technology to support behavior change.

The third research question is to what extent does our approach yield sustained changes in energy behavior, and what factors appear to influence sustained change? Using the energy data, we can determine the energy consumption of each pair of floors before, during, and after the competition. The energy consumption after the competition ends is most important when looking for sustained change, and we will look at the relationship between energy consumption and Kukui Nut points, website use, and energy literacy.