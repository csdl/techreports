%%%%%%%%%%%%%%%%%%%%%%%%%%%%%% -*- Mode: Latex -*- %%%%%%%%%%%%%%%%%%%%%%%%%%%%
%% 10-07-related.tex --  HICSS 44 Kukui Cup paper
%% Author          : Philip Johnson
%% Created On      : Mon Sep 23 11:52:28 2002
%% Last Modified By: Philip Johnson
%% Last Modified On: Thu Jun 10 15:04:49 2010
%%%%%%%%%%%%%%%%%%%%%%%%%%%%%%%%%%%%%%%%%%%%%%%%%%%%%%%%%%%%%%%%%%%%%%%%%%%%%%%
%%   Copyright (C) 2009 Philip Johnson
%%%%%%%%%%%%%%%%%%%%%%%%%%%%%%%%%%%%%%%%%%%%%%%%%%%%%%%%%%%%%%%%%%%%%%%%%%%%%%%
%% 

\section{Related Work}
\label{sec:related-work}

Our research draws on work from multiple areas. First, we discuss other dorm energy competitions, then we cover energy feedback research. Next, we examine related technological systems, and finally we relate our work to psychological work on behavior change.

\subsection{Dormitory Energy Competitions}

Energy competitions on college campuses involve residence halls competing to see which building can use the least energy over a period of time. The competitions tap into both the residents competitive urges, and their interest in environmental issues. However, unlike a home environment, the residents do not financially benefit from any reduction in electricity use resulting from their behavior changes, since residence hall fees are flat-rate and do not change based on energy usage. This leads to residents being completely unaware of their energy usage, since they lack even a monthly bill as feedback.

The most basic type of energy competition website displays energy data which is updated manually on a periodic basis (such as weekly). The Wellesley College Green Cup \cite{wellesley-green-cup} is an example of this type of competition. 

Other schools have more complicated and interactive competition websites, such as the early adopter Oberlin College. Petersen et al.\ describe their experiences deploying a realtime feedback system in an Oberlin College dorm energy competition in 2005 \cite{petersen-dorm-energy-reduction}. 22 dormitories were in competition over a 2 week period, with 2 dorms having feedback updates every 20 seconds, and the other 20 getting updates every week. The realtime dorms also recorded electricity usage for each of the three floors, but only displayed the data from two of the floors, leaving the third as a control. Web pages were used to provide feedback to students, since they all have computers and Internet access in their rooms. They found a 32\% reduction in electricity use across all dormitories, with the 2 realtime feedback dorms reducing usage the most. Freshman dorms were among the highest electricity reducers, while upperclassman dormitories were quite low (average 2\% reduction). During a 2 week post-competition period, the average electricity usage was similar to consumption levels during the competition. However, the weather was warmer and there was more sunlight during the post-competition period, so it is unclear if the reduction was competition-related. In a post-competition survey, respondents indicated that some behaviors, such as turning off hallway lights at night and unplugging vending machines were not sustainable outside the competition period.

While dorm energy competitions are being conducted with regularity, beyond the work at Oberlin, the emphasis appears to be on the event and not on research on the effects of the competition. In particular, there has been analysis on energy usage after the competition is over, or how postive behavior changes could be sustained.

\subsection{Energy Feedback}

As Lord Kelvin is famously reputed to have said, ``If you can not measure it, you can not improve it.'' In the case of electricity usage, for many people the only feedback they receive is a monthly bill detailing the number of kilowatt-hours used over the course of the last month. Ed Lu of Google analogizes this as if there were no prices on anything at the grocery store, and shoppers were just billed at the end of the month \cite{Helft2008Googles-Energy}. Office workers or dormitory residents might never see any feedback on how much electricity they are using!

To reduce energy use, people must know how much energy they are using. Feedback systems display the consumption of a resource (such as electricity) to the user, usually in real time. Darby provides a detailed survey of studies on electricity feedback systems from the past 3 decades \cite{darby-review-2006}. The survey of 20 studies finds that, on average, the introduction of a direct (real-time) feedback system leads to reductions of energy usage ranging from 5-15\%. Feedback systems providing historical data (such as those provided with billing statements) are not as effective (0-10\% reductions), but can be useful for assessing the impact of efficiency measures such as improved insulation or a more energy efficient appliance, since those savings accumulate over time.

Darby found that ``consumption in identical homes, even those designed to be low-energy dwellings, can easily differ by a factor of two or more depending on the behaviour of the inhabitants.'' This finding demonstrates the significant potential to curb energy usage through changes in individual's behavior.

Another survey of energy feedback was conducted Faruqui et al., looking at 12 utility pilot programs that installed in-home displays with near-realtime feedback \cite{Faruqui09}. They found that customers that actively used the display averaged a 7\% reduction in energy usage, while those pilot programs that included pre-paid electrical services reduced their energy usage by 14\%. The sustainability of the energy reduction is unclear based on the pilot studies since they were of limited length. The authors believe it is unknown whether the residents of homes with displays will acclimate to the display and cease to use it to reduce their energy usage.

Providing energy feedback is a critical foundation for any attempt to reduce energy consumption, and the feedback itself will likely curb energy usage somewhat. However, Darby points out that while feedback is critical for energy conservation behaviors, feedback alone is not always enough \cite{darby-2000-making-it-obvious}. Other factors that lead to higher rates of energy conservation include contact with an advisor when needed, and training and social infrastructure.

\subsection{Related Systems}

In this section we examine other systems that have been designed to help users become more aware of their environmental impact, or make environmentally-positive behavior changes.


