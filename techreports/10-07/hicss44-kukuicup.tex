%%%%%%%%%%%%%%%%%%%%%%%%%%%%%% -*- Mode: Latex -*- %%%%%%%%%%%%%%%%%%%%%%%%%%%%
%% hicss44-kukuicup.tex --  HICSS 44 Kukui Cup paper
%% Author          : Philip Johnson
%% Created On      : Mon Sep 23 11:52:28 2002
%% Last Modified By: Philip Johnson
%% Last Modified On: Thu May 27 11:11:30 2010
%%%%%%%%%%%%%%%%%%%%%%%%%%%%%%%%%%%%%%%%%%%%%%%%%%%%%%%%%%%%%%%%%%%%%%%%%%%%%%%
%%   Copyright (C) 2009 Philip Johnson
%%%%%%%%%%%%%%%%%%%%%%%%%%%%%%%%%%%%%%%%%%%%%%%%%%%%%%%%%%%%%%%%%%%%%%%%%%%%%%%
%% 

%% Home page: http://www.hicss.hawaii.edu/hicss_44/authorinstruction.htm

%% Must submit to one of the minitracks:
%% http://www.hicss.hawaii.edu/hicss_44/apahome44.htm

%% It appears that this is the most relevant one:
%% http://www.hicss.hawaii.edu/hicss_44/Minitracks44/dt-infosys.pdf

%% For ``peer review mode'', do:
%%   \documentclass[conference,compsoc,peerreview]{IEEEtran}
%% and
%%   \IEEEpeerreviewmaketitle  (after the abstract).

%\documentclass[conference,compsoc,peerreview]{IEEEtran}
\documentclass[conference,compsoc,peerreview]{IEEEtran}
\usepackage[final]{graphicx}
\usepackage{cite}
\usepackage{url}
% uncomment the % away on next line to produce the final camera-ready version
% and uncomment the \thispagestyle{empty} following \maketitle
%\pagestyle{empty}

\begin{document}

\title{The Kukui Cup: A dorm energy competition focused on sustainable behavior change and energy literacy}

\author{Robert S. Brewer\\
        Philip M. Johnson\\
\em     Collaborative Software Development Laboratory\\
        Department of Information and Computer Sciences\\
        University of Hawai`i at M\=anoa\\
        Honolulu, HI 96822\\
        rbrewer@lava.net, johnson@hawaii.edu\\
}


%\maketitle
\IEEEpeerreviewmaketitle
%\thispagestyle{empty}

\begin{abstract}  % 150 words
Abstract goes here.
\end{abstract}


\section{Introduction}
\label{sec:intro}

Dorm energy competitions are becoming an increasingly popular event; over
two dozen universities were listed in an online reference guide
\cite{competitions}.  Dorm energy competitions have many desirable
properties: they appear to be reliably successful at reducing energy usage
during the competition, they help foster community in the dorms, they
create opportunities for education, they enhance ecological awareness, and
they are generally perceived as fun by the residents.  To the extent that
energy reductions are achieved, they reduce the carbon footprint associated
with the dorm, and of course the electricity cost to the university.

Dorm energy competitions also provide a useful setting for research on
behavioral change.   First, dorm residents are a ``renewable resource'':
regular turn-over in occupancy means that it is possible to embed
experimental designs into the energy competition structure and 



\cite{GooglePowerMeter}

\section{Related Work}

\section{System Design}

\section{Evaluation Plan}

\section{Future Directions}


\section{Acknowledgments}

Financial support for this research was provided by the Renewable Energy and
Island Sustainability (REIS) project at the University of Hawai`i at M\=anoa.


\bibliographystyle{IEEEtran}
\bibliography{smartconsumer}
\end{document}
