\section{Research Problem Statement and Research Questions}
Software is coded by humans. Whether in team or individually, humans perform relevant 
daily activities in order to reach the goal - deliver the software. Understanding of these
human activities spanning throgh the lifecycle of the software, in connection with personal 
and team's motivations, environment settings and constraints essentially enables one to
comprehend the software process. It is worth noting, that roughly \todo[inline]{put here something 
about two components - the human-driven, non-recurrent and creative activity - 
the behavioral component - and the process and the technology/toolkit component which
provides a measurable marginal effect}

\todo[inline]{Here, put stuff about observing the process and artifacts availability}

While it is shown by the large body of previous research, that it is technically possible 
to factor out the impact of the technology and the process, the impact of the behavioral 
component is yet to be studied. Hence, my primary research question is this:
\begin{myindentpar}{0.07\linewidth}
 \textbf{Can we discover recurrent behaviors in software processes from project
  repository artifacts?}
\end{myindentpar}
 
As mentioned above, in this thesis I am focusing on a very narrow subject - exploring approaches
for uncovering of recurrent behaviors or ``programming habits'' out of software process artifacts.



\section{Research area overview}