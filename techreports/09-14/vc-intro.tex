\chapter{Software Process Recovery}
Software development process was always being under focus of various stakeholders due to the number of reasons ranging from standards compliance to business and security intelligence. When no live obseravtions are made on the performers, due to the timing, cost, or privacy issues, recovering software process from artifacts could be complex and expensive process. Researchers have suggested a possibility of software process recovery by interviewing of developers and managers and by analysis of process artifacts: such as printed documents - designs, use-cases, software inspections or electronic artifact trails: version, bug and issue control systems and mailing lists. This research resulted in many published work: 
\begin{itemize}
\item{Cook \& Wolf in \cite{citeulike:328044} discuss an event-based framework for process discovery based on grammar inference and finite state machines. The authors directly applied their framework to Software Configuration Management (SCM) logs demonstrating satisfactory results.}
\item{Jensen \& Scacchi \cite{citeulike:5043664} describe an interesting framework built upon mapping between process artifacts and process entities into an universal generic meta-model. Application of their human-involved technique leveraged a pre-existing domain knowledge for the effective pruning and iterative process revision resulted in ``workflows discovery''.}
\item{German in \cite{citeulike:421438} performed a manual mining of the GNOME process artifacts: documentations, CVS logs, and one hundred and four mailing-lists in order to describe the development processes from GSD (Global Software Development) point of view.
}
\item{Ripoche tried a more automatic approach in \cite{citeulike:9112798} by developing a generic model for process-based explanation of bugs persistence using state diagrams and probabilistic choices.}
\end{itemize}
In my study I am relying on all of the previous works in a demonstrated ability of recovering of process from the artifacts trails, in this chapter I will introduce a novel technique of pattern mining form the sofwtare process artifacts trails. 
the importance of the correct mapping between process artifacts and process entities as well as a demonstration of iterative, human-involved technique of process revision which is emphasizing importance of pre-existing domain knowledge in the effective pruning of the search space.

\section{SCM, Software configuration management system}

\section{Software process recovery from SCM system}
As Ball et al. \cite{citeulike:9004378} and Zimmermann \& Wei�gerber \cite{citeulike:5058462} point out - all of the contemporary version control systems provide considerably large amount of auxiliary information about software change. In particular, version control system, when coupled with a mailing lists and (or) bug and issue tracking system is capable of providing information \textit{who} changed \textit{what} and \textit{why}. Which seems to be a fair amount of information needed for one's opinion about the change. It is possible to get an overall understanding of the change necessity through the analysis of bug and issue reports. Version control itself provides quantitative data about files and LOC and changed, added or deleted. The analysis of code snapshots (versions) allows to quantify the change in terms of various software metrics like complexity, cohesion etc. When considered in time all this data provides a solid background for a software evolution research.

However, what is very difficult to know from any contemporary SCM system is that a software process behind the changes. Nevertheless many research in the field of MSR was done in order to shed a light on the software process itself. \cite{citeulike:9007622} There are only traces of such information present in version control transactions. In order to recover some insights about the performed software process information behind a software change statistics behind the change can show us some behavioral patterns blanks in the single transactions can be restored by statistics
outliers effect can be diminished by statistics
