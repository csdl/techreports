%\subsection{Mining Software Repositories}\label{mackground.msr.summary}
%%\subsection{Mining Software Repositories}\label{mackground.msr.summary}
%%\subsection{Mining Software Repositories}\label{mackground.msr.summary}
%%\subsection{Mining Software Repositories}\label{mackground.msr.summary}
%\input{background_msr_summary}
According to Kagdi et al. \cite{citeulike:4534888} the term \textit{mining
software repositories (MSR)} ``... has been coined to describe a broad class of
investigations into the examination of software repositories.'' The ``software
repositories'' here refer to various sources containing artifacts produced by
software process. Examples of such sources are version-control systems (CVS,
SVN, etc.), requirements/change/bug control systems (Bugzilla, Trac etc.),
mailing lists archives and social networks. These repositories have different
purposes but they support a single goal - a software change which is the single
unit of the software evolution. 

In the literature, \textit{software change} defined as an addition, deletion or
modification of any software artifact such as requirement, design document, test
case, function in the source code, etc. Typically, software change is realized
as the source code modification; and while version control system keeps track of
actual source code changes, other repositories track various artifacts (called
\textit{metadata}) about these changes: a description of a rationale behind a
change, tracking number assigned to a change, assignment to a particular
developer, communications among developers about a change, etc.

Researchers mine this wealth of data from repositories in order to extract
relevant information and discover relationships about a particular evolutionary
characteristic. For example, one may be interested in the growth of a system
during each change, or reuse of components from version to version. Later in this
thesis I will review MSR research field in detail highlighting relevant to my
research work and comparing my results with existing MSR findings.
According to Kagdi et al. \cite{citeulike:4534888} the term \textit{mining
software repositories (MSR)} ``... has been coined to describe a broad class of
investigations into the examination of software repositories.'' The ``software
repositories'' here refer to various sources containing artifacts produced by
software process. Examples of such sources are version-control systems (CVS,
SVN, etc.), requirements/change/bug control systems (Bugzilla, Trac etc.),
mailing lists archives and social networks. These repositories have different
purposes but they support a single goal - a software change which is the single
unit of the software evolution. 

In the literature, \textit{software change} defined as an addition, deletion or
modification of any software artifact such as requirement, design document, test
case, function in the source code, etc. Typically, software change is realized
as the source code modification; and while version control system keeps track of
actual source code changes, other repositories track various artifacts (called
\textit{metadata}) about these changes: a description of a rationale behind a
change, tracking number assigned to a change, assignment to a particular
developer, communications among developers about a change, etc.

Researchers mine this wealth of data from repositories in order to extract
relevant information and discover relationships about a particular evolutionary
characteristic. For example, one may be interested in the growth of a system
during each change, or reuse of components from version to version. Later in this
thesis I will review MSR research field in detail highlighting relevant to my
research work and comparing my results with existing MSR findings.
According to Kagdi et al. \cite{citeulike:4534888} the term \textit{mining
software repositories (MSR)} ``... has been coined to describe a broad class of
investigations into the examination of software repositories.'' The ``software
repositories'' here refer to various sources containing artifacts produced by
software process. Examples of such sources are version-control systems (CVS,
SVN, etc.), requirements/change/bug control systems (Bugzilla, Trac etc.),
mailing lists archives and social networks. These repositories have different
purposes but they support a single goal - a software change which is the single
unit of the software evolution. 

In the literature, \textit{software change} defined as an addition, deletion or
modification of any software artifact such as requirement, design document, test
case, function in the source code, etc. Typically, software change is realized
as the source code modification; and while version control system keeps track of
actual source code changes, other repositories track various artifacts (called
\textit{metadata}) about these changes: a description of a rationale behind a
change, tracking number assigned to a change, assignment to a particular
developer, communications among developers about a change, etc.

Researchers mine this wealth of data from repositories in order to extract
relevant information and discover relationships about a particular evolutionary
characteristic. For example, one may be interested in the growth of a system
during each change, or reuse of components from version to version. Later in this
thesis I will review MSR research field in detail highlighting relevant to my
research work and comparing my results with existing MSR findings.
According to Kagdi et al. \cite{citeulike:4534888} the term \textit{mining
software repositories (MSR)} ``... has been coined to describe a broad class of
investigations into the examination of software repositories.'' The ``software
repositories'' here refer to various sources containing artifacts produced by
software process. Examples of such sources are version-control systems (CVS,
SVN, etc.), requirements/change/bug control systems (Bugzilla, Trac etc.),
mailing lists archives and social networks. These repositories have different
purposes but they support a single goal - a software change which is the single
unit of the software evolution. 

In the literature, \textit{software change} defined as an addition, deletion or
modification of any software artifact such as requirement, design document, test
case, function in the source code, etc. Typically, software change is realized
as the source code modification; and while version control system keeps track of
actual source code changes, other repositories track various artifacts (called
\textit{metadata}) about these changes: a description of a rationale behind a
change, tracking number assigned to a change, assignment to a particular
developer, communications among developers about a change, etc.

Researchers mine this wealth of data from repositories in order to extract
relevant information and discover relationships about a particular evolutionary
characteristic. For example, one may be interested in the growth of a system
during each change, or reuse of components from version to version. Later in this
thesis I will review MSR research field in detail highlighting relevant to my
research work and comparing my results with existing MSR findings.