\chapter{Software Process}
There are different approaches for software development which were designed in order to 
support the creation of software systems. One of the aspects of all is the use of 
software process model. This model provides developers with guidance about their tasks 
and the activities that should be undertaken during development. It is advocated that the 
use of established and well structured model is crucial for the complex projects to 
orchestrate collaborative effort of multiple teams working on the software product. 

These models (or methodology) can be characterized by the series of distinct phases. 
Each phase is executed in order to provide a part of the software system or revise 
an existing part. Examples of such phases are the requirements collection, user manual 
or the functional module of the software system. While the methodology is applied for 
such a process it is not only prescribes an ordering of carried activities, but 
provides a framework for estimation of resources, definition of milestones, managing 
time and effort and monitoring the progress. 

\section{Waterfall model}
Many variation of the original Royce's (1970) waterfall model exist. \todo{citation needed} 
This is the probably the oldest of formal software process models and it is very popular 
for large and complex systems. It is an example of a plan-driven process - one must 
plan and schedule activities before execution. The waterfall 
model forms the basis of many standards in the industry. It is a refinement of a 
stagewise model developed in 50th (boehm?).

There is successive and distinct progression of stages in waterfall model; each state 
is well defined and forms the basis for the successive one. In addition to specific 
deliverable, each state producing a document which describes what exactly occurred 
during the stage, providing certain visibility to the processes, and facilitating 
audit. Each state has milestones and deliverables explicitly predefined providing 
a binding contract. 

Five stages of waterfall model reflect the fundamental development activities:
\begin{enumerate}
 \item Requirements analysis and definition. The feasibility study is conducted 
and the requirements collected. The system functionality clarified and
documented in details - the requirements specification document - delivered  
and will serve as the system specification. It is assumed that customer's 
expectations are articulated and will not change much throughout the development 
process.
 \item The system design. Within this phase requirements for hardware and 
software components defined by the establishing an overall system architecture.
All system modules and their interactions are designed in accordance with 
collected requirements.
 \item Implementation and system testing. The actual code produced, individual 
modules are tested individually. This phase embeds all quality control checks.
 \item Integration and system testing. System assembled and once complete, it
is tested and evaluated in actual conditions (alpha tested).
 \item Delivery and Maintenance. System is tested by actual customer(s) 
(beta testing). Identified errors are corrected and if found satisfiable, 
software system is handed to customer (delivered) and enters the maintenance 
phase. 
\end{enumerate}



The Institute of Electrical and Electronics Engineers defines software engineering as 
“the application of a systematic, disciplined, quantifiable approach to development, 
operation, and maintenance of software; that is, the application of engineering software”
