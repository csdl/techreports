
\chapter{Introduction}

In large companies, it is possible to have dozens or even hundreds of software projects underway at any given point in time. This kind of scale produces new challenges, as well as new opportunities for these companies. Both challenges and opportunities result from the need of the company to successfully understand and exploit their ownership of a portfolio of software projects. 

Single project management is easy to achieve from various way. Many software metrics, such as coverage, complexity and file size, provide significant information about a software project. But it is reported that, almost all metrics have deficiency when judge a state of a project with only one or two of them. Therefore, people usually trend to look into more metrics to understand a project. However, while the number of projects increases as well as the number of metrics increases, the data to read inflates rapidly and become overwhelming.

In order to address this issue and achieve efficient governance over large number of projects, we introduce an analysis tool called Software Intensive Care Unit. It is from the idea of intensive care unit in hospitals where patients are placed with sensors from sophisticated machines and they are consistently monitoring the patients' various vital sign such as pulse, breathe, etc. Similarly, Software Intensive Care Unit(SICU) provides various software metrics, each of which reveals one aspect of the project. These metrics are the vital signs of a software project. With all these vital signs together, users can have a fast but complete view of the projects, leading to easier understanding of projects and make better decisions and practice. Though one might argue that software projects' performance cannot be simply judged by a set of metrics, the case is the same in medicine. A doctor will not simply diagnose a patient by those vital signs, but it is undoubtedly that the medical intensive care unit helps the diagnosis a lot, so it is the software intensive care unit.

However, to build up the SICU is a great challenge. Both what data to show and how to show them are essential to successfully set up a SICU that offer enough insight into the most interesting aspect of the projects without too much data overwhelming. Moreover, interesting things are various from different situation. Some setting may be common over projects and organization, but we don't believe there is a golden rule for all projects. So system should give users the capability to customize as their need.

\section{Research Questions}

\chapter{Related Work}
\section{Software Project Metric Measurement}
Agile development has been become a prevailing development method and many metric measurements have been developed to help manage project process and product quality.

\section{Multi-Project Management and Project Portfolio Management}
Effective multi-project management is now a major challenge in many companies. Many studies on Project Portfolio try to address it in different ways. Risk and cost are major factor to be considered in business model. It is reasonable and effective to manage projects in order to maximize the profit of the company.

However, when considering project's success rate, the numbers are usually given by managers or even just taken randomly. But this factor is actually predictable and measurable via software metric measurements. Study of project portfolio with software metrics together to acquire better management over multi-projects are far from plenty. 

\chapter{Software Intensive Care Unit}
In order utilize the well developed software engineering measures to achieve good management of software development projects, we introduce a system called Software Intensive Care Unit consists of a number of software development metrics and use these metrics to generate a number of vital signs that state represent "health" state of software development projects.

\section{Vital Signs}
Vital signs of software projects are measured by various software development process or product metrics. Each of these metric reveal an aspect of the state of the software project.
\begin{description}
\item[Coverage] 
Coverage is a measure used in software testing. It describes the degree to which the source code of a program has been tested. It measures the quality of the tests, which is an essential part of program quality insurance.
\item[Cyclomatic complexity] 
Cyclomatic complexity, was developed by Thomas J. McCabe and is used to measure the complexity of a program. It directly measures the number of linearly independent paths through a program's source code.\cite{mccabe:complexity} Higher cyclomatic complexity means more distinct control paths in a program module, thus it is more difficult to fully test them and the untested paths may have deficiency inside. Therefore, program modules are preferred to have lower complexity.
\item[Coupling] 
Coupling, or say dependency, is the degree to which each program module relies on each one of the other modules.\cite{wiki:coupling} It is actually a measure of the complexity of the whole system's module reference tree. Whenever one module is modified, there will be a chance that the changes may cause bugs in one of modules that relies on this one. Therefore, higher coupling value implies higher risk of introducing bugs when making changes, thus more difficult to maintain the system.
\item[Churn] 
Churn is a measure of the total number of lines of code deleted and added. It represent the amount of changes made to the system, which reflects the work to the system. Interpretation of this metric is conditional. Low churn means low work load to the system, but also means consistency of the system, thus is good for systems under maintenance. However, high churn implies higher inconsistency of the source code of the system, but also means high work load, which is the normal case of a intensively developing project.
\item[Size] 
The size of software program is measured by the source lines of code (SLOC), which counts the number of lines in the text of the program's source code. SLOC is typically used to predict the amount of effort that will be required to develop a program, as well as to estimate programming productivity or effort once the software is produced.
\item[Commit] 
Commit measure the number of commitments made to the source code repository. It is recommended that programmer should make commit often and make it small and keep the integration consistent.
\item[DevTime] 
DevTime, abbreviation of Develop Time, is a measure of total human work put into the system.
\item[Build] 
Build is a count of ant build task invoked in a period of time.
\item[Test] 
Test is a count of unit tests invoked in a period of time.
\end{description}

\subsection{Vital Sign from latest value}
Latest values of software metrics are important enough to be shown separately because they represent the latest state of a project. We assign colors to latest value of the measures to indicate their performing.

\subsection{Vital Sign from historical trend via spark-line}
Historical data of software metric is as good as the latest one. It provides more information of the performance of the project over time. But in the same time it bring a great amount of data overwhelming as well. In order to reduce the historical data to an acceptable small but meaningful degree, we use spark-line to represent them. Then we assign colors to the spark-lines with several evaluation strategies: Stream Trend and Participation.

\subsubsection{Stream Trend Evaluation}
Stream trend evaluation determines the health of a stream by its trend. It takes one parameter called HigherBetter. If user define the higher value the better, then increasing trends will be considered as a healthy trend and decreasing trends will be considered as unhealthy. Stable trends are always considered as healthy because in that case it is as good as health that one don't need to pay too much attention to it, while the actual value will be shown in the latest value where the value will be judged to be good or not. And unstable trend is marked as average because it is not easy to tell if it indicates a good state or not. 

\subsubsection{Participation Evaluation}
Participation evaluation determines the health of a stream by the participations of all members in the project. When working as a group, it is not good that some of the member do most of the work and others do little, thus in this situation the stream will be classified as unhealthy stream, which mean to attract attention.

\chapter{Implementation}
The system is implemented as a part of the Hackystat Framework. It utilize higher level data from Hackystat analysis to analyze projects performance.

\section{Hackystat Framework}
Hackystat is an open source framework for collection, analysis, visualization, interpretation, annotation, and dissemination of software development process and product data.

Hackystat users typically attach software 'sensors' to their development tools, which unobtrusively collect and send "raw" data about development to a web service called the Hackystat SensorBase for storage.

The SensorBase repository can be queried by other web services to form higher level abstractions of this raw data, and/or integrate it with other internet-based communication or coordination mechanisms, and/or generate visualizations of the raw data, abstractions, or annotations.

\subsection{Daily Project Data Analysis}
The DailyProjectData(DPD) service is one of Hackystat's most important fundamental analysis service. From the raw sensor data in the SensorBase repository, it creates various abstractions of sensor data associated with a single project for a single 24 hour period, which usually represents a simple software development metric in a single day.

\subsection{Telemetry Analysis}
The Telemetry service is another fundamental analysis of Hackystat. Based on data from DPD service, it supports the creation of trend lines that show how various characteristics of software development are changing over time. To support the work practices of different organizations, it provides a domain specific language that allows the creation of custom trend lines (called telemetry "streams") and their visualization together in a specific telemetry "chart". 

Telemetry streams support various numbers of parameters. User can use them to generate more specific streams. In our Project Portfolio Analysis, which based on Telemetry service, user can configure the parameters of each Telemetry analysis. More detail will be discuss in later part.

\section {Project Browser and Wicket}
Wicket is one of the dozens of Java web frameworks, but a outstanding one. It use HTML attributes to denote components, enabling easy editing with ordinary HTML editors. The internal object structure is similar to Swing, it give easy but powerful way to develop functionality on both server and client side.

Upon it we built a web application UI to view collected data and run various analysis provided by Hackystat services, that is called Project Browser. Its goal is to simplify the usage of Hackystat service as well as provide better visualization over the analysis data.


\section {Software Intensive Care Unit}
The Software Intensive Care Unit(SICU) consist a table of metrics over each of the projects. Each metrics are shown with a spark-line and a number value, each of which will be colored conditionally. The spark-line is generated using the Google Chart API.\cite{googlechart}

\chapter{Experimental Evaluation}
We are going to evaluate this system in a classroom setting. The class contain 20 students. They are learning software development methods in the class and will use the system for about a month till the end of the semester. During this period, they work as small group of 3~4 people, and each group will work on 2 software projects.

We gather research data in two ways:
	1. We will monitor and log their usage of the system. From this data we can find out their frequency and habit of using the system. And compared with their performance of the project, we can analysis the helpfulness of the system in developer's aspect. 
	2. We will give out survey at the end of the semester to gather their opinions of the system. From this we can find out users' attitude towards the system. In additional, they may provide insightful suggestion to improve the system.

We will analyze this data along with the data from previous Hackystat studies.

\chapter{Contribution and Future Directions}
