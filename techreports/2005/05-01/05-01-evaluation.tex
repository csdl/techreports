%%%%%%%%%%%%%%%%%%%%%%%%%%%%%% -*- Mode: Latex -*- %%%%%%%%%%%%%%%%%%%%%%%%%%%%
%% 05-01-evaluation.tex -- Thesis proposal - PRI 
%% Author          : Aaron A. Kagawa
%% Created On      : Mon Sep 23 11:52:28 2004
%% Last Modified By: Aaron Kagawa
%% Last Modified On: Sat Jul  2 22:37:06 2005
%% RCS: $Id$
%%%%%%%%%%%%%%%%%%%%%%%%%%%%%%%%%%%%%%%%%%%%%%%%%%%%%%%%%%%%%%%%%%%%%%%%%%%%%%
%%   Copyright (C) 2004 Aaron A. Kagawa
%%%%%%%%%%%%%%%%%%%%%%%%%%%%%%%%%%%%%%%%%%%%%%%%%%%%%%%%%%%%%%%%%%%%%%%%%%%%%%%


\chapter{Exploratory Study Procedure}
\label{chapter:evaluation}
This chapter discusses the exploratory study procedures conducted in this
research. The main thesis of this research is that Priority Ranked
Inspection (PRI) can distinguish documents that are more in need of
inspection (MINI) from those less in need of inspection (LINI). This
chapter will describe how the main thesis was studied.


\section{Subjects Used in the Study}
\label{section:evaluation-subjects}
I studied the implementation and inspection process of the Hackystat System
\footnote{See http://www.hackystat.org for more information about the
  Hackystat System.} developed in the Collaborative Software Development
Laboratory (CSDL) \footnote{See http://csdl.ics.hawaii.edu for more
  information about CSDL projects and members}. CSDL is a research
laboratory within the Department of Information and Computer Sciences at
the University of Hawaii. Currently, CSDL is comprised of Professor Dr.
Philip Johnson and seven graduate Computer Science students, including
myself. Our mission is to provide a physical, organization, technological,
and intellectual environment conducive to collaborative development of
world-class software engineering skills. CSDL's current focus is on the
development of Java software systems that support software development
research. Hackystat \cite{Johnson05}, hackyPRI, Jupiter \cite{Jupiter},
LOCC \cite{Locc} are examples of software we have developed. All eight
members are highly experienced Java programmers and have been practicing
high quality software development. We use tools such as Eclipse, Jakarta
Tomcat, Apache Ant, and CVS. In addition, we practice several development
techniques such as Extreme Programming and inspection.

In the exploratory study of the Priority Ranked Inspection process, I
focused on studying PRI's effectiveness in aiding CSDL's inspection of
Hackystat related software code. Like most organizations, CSDL's inspection
resources are limited and therefore inspections are conducted, if at all,
on a weekly basis regardless of the number of ``ready'' documents. CSDL
primarily inspects source code grouped by Java packages; therefore, I will
use the term 'packages' when referring to CSDL's use of PRI. I will use the
term 'documents' when referring to the general idea of inspections.

Although I am a member of CSDL and have been contributing to Hackystat, I
minimized any possible data contamination by doing two things. First, I
ensured that the inspection participants are ``blind'' to the document
selection method. There are two methods of selection that were used in this
study, selection with and without aid of PRI. I worked with individual
authors to select documents based on their subjective selection or with the
aid of PRI and kept that decision a secret from the rest of the
participants. Second, although I participated in the inspections, the
defects that I discovered will not be used in the study.

%%It has been advised by my Thesis Committee, that the inspection requests
%%should be ``double blind''. ``Double blind'' means that the experimenter
%%should not know the PRI determination of MINI or LINI.  The use of this
%%technique prevents any conscious or unconscious hints via body language,
%%speech, or any other action that will give the impression of whether the
%%experimenter wants the reviewer to find defects in the document. However,
%%it is my belief that randomly selecting documents for inspection would be
%%both detrimental for CSDL and my research. Therefore, I did not use the
%%double blind technique in the evaluation methodology of this research.

CSDL has been conducting and studying inspections since the early 1990's.
CSDL's inspection process has gone through the use of many different tools
and processes. Our current inspection guidelines are published in the
Hackystat Developer Documentation: Software Review Guidelines
\cite{SoftwareReviewGuidelines}. The term ``review'' used in CSDL's process
equates to the term ``inspection''. The primary goals of the current
process includes creating an educational process that allows participants
to learn new techniques and practices about developing high quality
software design and implementation and to remove defects. The process is
lightweight and includes 5 simple steps:

\begin{enumerate}
\item \textbf{Announcement (or Review Request)} - In this step, an author
  sends an email requesting that the group inspect the specified software.
  In addition, the author lists several questions to help direct the
  participants' attention to what the author thinks is most important. This
  announcement should be sent 24 hours before the meeting.
\item \textbf{Preparation} - In the hours between the announcement and the
  meeting, the review participants must individually examine the software
  listed in the announcement and log any issues that are found. Preparation
  time is limited to no more than one hour.
\item \textbf{Meeting} - At the scheduled time the group gathers to discuss
  the validity of the issues that have been discovered in the preparation
  step.
\item \textbf{Revision} - After the review meeting, the valid issues that
  were discovered must be fixed. In this step, the author or assigned
  developer must resolve these issues.
\item \textbf{Verification} - After the revision, a quick determination is
  required to ensure that all the issues have been resolved.
\end{enumerate}

Note that the CSDL inspection process does not specify how to determine
what software should be inspected. The process simply starts with an
announcement. This missing step is evident in all traditional software
inspection processes. 

Currently, CSDL utilizes the Jupiter Eclipse Review Plugin \footnote{
  Takuya Yamashita, who is also a CSDL member, developed the Jupiter
  software.} \cite{Jupiter} to support our inspection process. Jupiter is a
lightweight tool that supports, to a varying degree, all steps of the CSDL
inspection process.  For example, individual inspectors use Jupiter during
the Preparation phase to log issues that he/she has found. In addition,
Jupiter collects various properties of the review issues generated in an
inspection process. The review issue properties that have been collected
and used in this study are severity, type, and resolution. These properties
allow the inspectors to specify additional information about the discovered
issue to accurately discuss them in future phases of the inspection
process. Table \ref{tab:jupiter-properties} provides a full listing of the
properties and their values that were used in this study.

\begin{table}[htbp]
  \begin{center}
    \caption{Jupiter Properties and Values}
    \label{tab:jupiter-properties}
    \begin{tabular}{|p{2.0cm}|p{6.0cm}|p{5.0cm}|} \hline
      {\bf Property} & {\bf Meaning} & {\bf Values} \\ \hline

Severity & Allows the inspector to note the importance of the issue. &
Critical, Major, Normal, Minor, Trivial \\ \hline 
Type & Allows the inspector to note what type of issue he/she has found. &
Coding Standards, Program Logic, Optimization, Usability, Clarity, 
Missing, Irrelevant, Suggestion, Other \\ \hline 
Resolution & Allows the team to determine the validity and necessary
actions required to resolve the issue. & Valid Needs Fixing, Valid Fix
Later, Valid Duplicate, Valid Won't Fix, Invalid Won't Fix, Unsure Validity
\\ \hline 
    \end{tabular}
  \end{center}
\end{table}

I used the issues' property value information to analyze the validity of
several claims. For example, I was able to count the number of
high-severity issues that were discovered by the inspection. For this
study, high-severity is defined as defects with a severity equal
``Critical'' or ``Major'' and a resolution not equal to ``Invalid Won't
Fix'' and ``Valid Duplicate''. A major problem that I did not address in
this research and study is the subjective opinion used when inspectors
assign values to specific properties. For example, one inspector's
subjective opinion of a Severity value could differ from another
inspector's opinion.



\section{Study Limitations}
The use of CSDL resources in my study indicates a major limitation on this
research. The most accurate and thorough evaluation of PRI should inspect
\textit{all documents} to evaluate PRI's classification of MINI and LINI
documents. However, because I am using CSDL's inspection resources, which
are limited, this was not possible.

Currently, Hackystat and its extensions are comprised of 218 packages. At
best this will take 2 hours per inspection per member, therefore totaling
3,488 hours of inspection. Requiring the use of that many hours is an
unrealistic demand on CSDL resources. Therefore, my exploratory study
investigated a small percentage of the system in hopes that a cross-section
provided adequate and acceptable results. Furthermore, CSDL conducts
inspections to increase quality and spread knowledge. It would be
detrimental to this development group, if I required the inspection of many
packages that did not provide that return on investment.

%%[TODO: add paragraph about not using Double Blind]

It is important to note two other limitations of this research. First, I am
not defining a set of PRI measures and PRI indicators that represent the
PRI ranking function for all software projects. Instead, by using hackyPRI
I will be able to go through a methodology to best calibrate the ranking
functions to accurately reflect the determination for the project I am
studying.  Second, PRI is more beneficial for organizations that have
limited inspection resources. PRI is of less use for organizations that
have the necessary resources to thoroughly inspect every document, although
this is yet to be studied.




\section{Study of Thesis Claims}
To study this thesis, I separated it into three claims based upon the
three intended benefits of PRI.

\begin{enumerate}
\item MINI documents will generate more high-severity defects than LINI
  documents.
\item PRI can enhance the volunteer-based document selection process. 
\item PRI can identify documents that need to be inspected that are not
  typically identified by volunteering.
\end{enumerate}

To evaluate my thesis claims, I created a six-part study procedure. The
study includes; questionnaires, working with authors to select documents
for inspection, and the analysis of an inspection log and results. The
different portions of the study procedure do not necessarily correlate
one-to-one with the three thesis claims. Instead, each procedure provides
supporting evidence for all of my thesis claims. The following is a short
summary of the steps of the study procedure. The following sections explain
each of the steps in more detail.

\begin{enumerate}
\item \textbf{Pre-Selection Questionnaire}: I administered a questionnaire
  to obtain the developers feelings assessing the usefulness of inspection
  and the methods they use to select documents for inspection.  In
  addition, I asked each developer to provide rankings, based on their
  current subjective opinions, for three different sets of workspaces.
  First, they ranked each top-level module within the Hackystat system
  (i.e., hackyKernel, hackyStdExt, hackyReview, etc). Second, they
  identified the top five packages throughout the whole Hackystat system
  that they thought were MINI and LINI. Last, I asked them to rank packages
  they have authored based on their opinions of what packages are MINI and
  LINI.
\item \textbf{Package Selection}: I worked with individual developers to
  select a package for inspection. The selection of packages can be made
  with or without the aid of PRI.
\item \textbf{Request for Inspection}: After a package was selected for
  inspection, I instructed the author to send an email-based request for
  inspection to our fellow Hackystat developers. I ensured that this email
  ``blinded'' the selection method.
\item \textbf{Inspection of the Selected Package}: Using the CSDL code
  review (inspection) process, the inspection participants inspected the
  package individually and met to discuss the issues that were discovered.
  The author of the package, who is also the developer I worked with in
  Step 2, did not inspect his/her own code.
\item \textbf{Post-Inspection Questionnaire}: Following the inspection I
  administered a questionnaire that asked the participants whether they
  believed the package was MINI or LINI.
\item \textbf{Record Results of Inspection}: I recorded the results of the
  inspection, the PRI ranking of the package before and after the
  inspection, and other PRI and inspection results.
\end{enumerate}




\subsection{Part 1 - Pre-Selection Questionnaire}
\label{subsection:pre-selection-questionnaire}
The first study procedure that was used is a questionnaire. The goal of the
questionnaire to obtain the authors' opinions about inspections in general,
their document selection process, and their subjective rankings of the
modules, packages, and packages that they have authored.

Appendix \ref{appendix:pre-selection-questionnaire} contains the
Pre-Selection Questionnaire. This questionnaire contains three different
sections; two general inspection questions about CSDL's inspection process,
four questions assessing the developers' document selection method, and
three tasks which gathered the developers' subjective rankings of various
packages.

The first section contains general questions about the CSDL inspection
process. These questions do not directly correlate to my thesis claims.
However, I will use this information to help validate the data that I
collect. For example, if the developers find an insignificant number of
issues in both MINI and LINI documents, then I can correlate that finding
with their enthusiasm towards inspection. In addition, one of the questions
asks whether finding defects are the most important outcome of the CSDL
inspection process and the data collected will aid future directions of
this research. It is my future-hypothesis that PRI can also aid the
selection of documents for purposes other than finding the most defects.
For example, an inspection process can be used as training or education and
PRI could aid the selection of documents that best suit that goal.

The second section contains questions about the developers' current
document selection method. This section provides important information on
the process in which developers select documents for inspection. In
addition, it provides supporting qualitative data for the quantitative data
provided in the last section.

The last section of the questionnaire asked the developers of Hackystat to
provide a numerical ranking, based on their subjective opinions, of three
different sets of packages. First, they were asked to rank the top-level
workspaces, or modules, within Hackystat according to their subjective
opinion of the quality of the modules. Hackystat contains many different
modules that can interchange depending on the situation of use
\footnote{See the Hackystat Developer Website (http://www.hackystat.org)
  for a listing of the modules in the system.}. Second, they ranked the top
five packages in the entire Hackystat project that they thought were MINI
and LINI. Last, they ranked the packages that they would volunteer for
inspection. The packages used in this last set were packages that the
developer has authored.

When analyzing the results of the developers' subjective rankings, I will
be able to compare the developers' subjective rankings against the PRI
ranking.  This comparison will indicate whether PRI is really needed. The
findings could indicate that developers can correctly distinguish, using
their own subjective reasoning, what packages need to be inspected. There
are three possible results from this study. First, I may find that
developers automatically have a sense of what code is MINI and what code is
LINI. This would indicate that PRI provides little added value.
Alternatively, I might find that, developers have no idea what code needs
to be inspected. The third possible result represents a middle ground
between the two previous results, sometimes the developers are correct and
sometimes they are wrong. The last two results will indicate that PRI
provides some benefit. Of course, these results will need to be validated
with the actual inspection of the document to validate both the developers'
subjective rankings and the PRI rankings.



\subsection{Part 2 - Package Selection}
\label{subsection:package-selection}
The second study procedure that was used is the selection of packages for
inspection. The goal of this procedure is to study the effectiveness of the
MINI and LINI determinations and PRI's ability to help the selection
process.

In this portion of the study, I worked closely with the various authors who
contribute to the Hackystat project to select a package for inspection. To
accomplish this, I created a weekly inspection schedule (See
\ref{tab:eval-timeline}). Each week I worked closely with a different
Hackystat developer to select one package for inspection. This selection
process was designed with the following steps:

%%[TODO: need to say that each developer immediately had something they
%%wanted to be inspected. And that I had to select a few LINI packages to
%%ensure that both MINI and LINI packages were inspected. In addition,
%%at some points during the evaluation, impromptu inspections were held.]

\begin{enumerate} 
\item Explain to the developer that the goal of this collaboration is
  to find the document that is most in need of inspection. Therefore, using
  PRI is not required.
\item Ask the developer if they have a package they would like to be
  inspected. If so, record the package name and the MINI or LINI
  ranking and skip to step 6. If not, continue to the next step.
\item Present the developer with a list of MINI packages that he/she has
  authored. Work with the author to select a package from this listing. If
  a document is selected, then move to step 6. If not, continue to the next
  step.
\item Present the developer with a list of LINI packages that he/she has
  authored. Work with author to select a package from this listing. If a
  document is selected, then move to step 6. If not, continue to the next
  step.
\item If we have reached this step, then it can be determined that the
  author does not believe he/she has authored any packages that are more in
  need of inspection. Therefore, I will select a document from the MINI
  listing. According to the author this package should not generate many
  high-severity issues. However, according to PRI this package should generate
  high-severity issues. The results will be recorded.
\item Once a document has been selected, the author is required to send a
  request for inspection to the inspection participants (generally all
  current CSDL members).
\end{enumerate}

This process was designed to ``blind'' the method used in selecting the
package. The inspection participants, who include all CSDL members
excluding the author and myself, did not know how the package was selected.
This ``blinding'' of the selection method was created to ensure that the
participants did not consciously or unconsciously persuade the results of
the inspection.


\subsection{Part 3 - Request for Inspection}
After a package has been selected in Part 2, the author is required to send
an email request for inspection. Again, this email ``blinded'' the
selection method from the participants. It simply stated that the author
requests the inspection of a particular package and provides the necessary
information that is required to successfully inspect the package. This
request announcement was congruent with CSDL's inspection process
\cite{SoftwareReviewGuidelines}.


\subsection{Part 4 - Inspection of Selected Package}
This part of the study procedure required little change from CSDL's
original inspection process defined in the Hackystat Software Review
Guidelines \cite{SoftwareReviewGuidelines}. The participants conducted
the Preparation and Meeting phases of the inspection process.

The Jupiter review tool \cite{Jupiter} was used to gather the issues
generated in the Preparation phase. In addition, the Jupiter tool is used
in the Meeting phase to record the validity and severity of the issues.


\subsection{Part 5 - Post-Inspection Questionnaire}
The fifth part of the study procedure is a quantitative questionnaire.
Appendix \ref{appendix:post-inspection-questionnaire} contains the
Post-Inspection Questionnaire. The goal of this questionnaire was to obtain
the developers' opinions of the MINI or LINI determination. The results of
the inspection and the developers' opinions helped to determine if the PRI
ranking for the package was correct.


\subsection{Part 6 - Record Results of Inspection}
There are two possible results of this portion of this study. First, the
packages that were selected were correctly categorized by PRI. Second, the
packages that were selected did not reflect the PRI ranking function. These
findings will provide evidence for claim 3 of my thesis statement.

During this study, CSDL has conducted nine inspections. However, in
addition to the inspections conducted under this study, I have recorded
data about eleven other inspections. In total, I have data on twenty
inspections and information on the PRI ranking functions.

Throughout my exploratory study of PRI, I monitored the validity of the PRI
ranking function throughout each inspection. To accomplish this, I
collected specific pieces of information when conducting inspections. The
following is a specific list of the information collected:

\begin{itemize}
\item Inspection date
\item Hackystat module, package, and inspection ID
\item PRI determination (MINI or LINI)
\item PRI measure values and PRI indicator ranking and weighting
\item Subjective discussion of the validity of the PRI ranking function
  before the inspection
\item Number of issues generated and the categorization of these issues
  according to severity
\item Retrospective discussion after the inspection was conducted to
  indicate possible areas of improvement. 
\end{itemize}

%%See Appendix \ref{appendix:log} for a copy of the full log. 

This information helped me keep track of the progress of the inspections
and the validity of the PRI ranking function. As I previously stated, the
calibration of the PRI ranking function is an ongoing and evolving process.
Collecting these types of information will help an organization keep track
of that evolution. The end goal of the continued study of PRI is to create
a best practices recommendation of the types of process and product
measures and their calibration that will provide the best PRI results for
different projects.

\section{Study Timeline}
The following table provides a timeline for the exploratory study of this
thesis. The developer names used in this timeline are hidden to protect
the developers' identity.


\begin{table}[htbp]
  \begin{center}
    \caption{Study Timeline}
    \label{tab:eval-timeline}
    \begin{tabular}{|p{5.0cm}|p{8.0cm}|} \hline
      {\bf Timeline} & {\bf Study Activity} \\ \hline
April 6, 2005 & Package Selection: Developer 5 \newline
Pre-Selection Questionnaire: Developer 5 \newline
Review Request: Developer 5 \newline
Review of Selected Code: CSDL \newline
Post-Inspection Questionnaire: CSDL \\ \hline

April 13, 2005 & Package Selection: Developer 6 \newline
Pre-Selection Questionnaire: Developer 6 \newline
Review Request: Developer 6 \newline
Review of Selected Code: CSDL \newline
Post-Inspection Questionnaire: CSDL \\ \hline

April 20, 2005 & Package Selection: Developer 9 \newline
Pre-Selection Questionnaire: Developer 9 \newline
Review Request: Developer 9 \newline
Review of Selected Code: CSDL \newline
Post-Inspection Questionnaire: CSDL \\ \hline

April 27, 2005 & Package Selection: Developer 7 \newline
Pre-Selection Questionnaire: Developer 7 \newline
Review Request: Developer 7 \newline
Review of Selected Code: CSDL \newline
Post-Inspection Questionnaire: CSDL \\ \hline

May 4, 2005 & Package Selection: Developer 4 \newline
Pre-Selection Questionnaire: Developer 4 \newline
Review Request: Developer 4 \newline
Review of Selected Code: CSDL \newline
Post-Inspection Questionnaire: CSDL \\ \hline

May 11, 2005 & Package Selection: Developer 3 \newline
Pre-Selection Questionnaire: Developer 3 \newline
Review Request: Developer 3 \newline
Review of Selected Code: CSDL \newline
Post-Inspection Questionnaire: CSDL \\ \hline

June 1, 2005 & Package Selection: Aaron Kagawa \newline
Review Request: Aaron Kagawa \newline
Review of Selected Code: CSDL \newline
Post-Inspection Questionnaire: CSDL \\ \hline

June 8, 2005 & Package Selection: Aaron Kagawa \newline
Review Request: Aaron Kagawa \newline
Review of Selected Code: CSDL \newline
Post-Inspection Questionnaire: CSDL \\ \hline

June 15, 2005 & Finished analyzing the results.  \\ \hline
    \end{tabular}
  \end{center}
\end{table}
















