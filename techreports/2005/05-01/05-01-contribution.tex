%%%%%%%%%%%%%%%%%%%%%%%%%%%%%% -*- Mode: Latex -*- %%%%%%%%%%%%%%%%%%%%%%%%%%%%
%% 05-01-contribution.tex -- Thesis white paper - software inspections
%% Author          : Aaron A. Kagawa
%% Created On      : Mon Sep 23 11:52:28 2004
%% Last Modified By: Aaron Kagawa
%% Last Modified On: Sat Jul  2 22:19:35 2005
%% RCS: $Id$
%%%%%%%%%%%%%%%%%%%%%%%%%%%%%%%%%%%%%%%%%%%%%%%%%%%%%%%%%%%%%%%%%%%%%%%%%%%%%%
%%   Copyright (C) 2004 Aaron A. Kagawa
%%%%%%%%%%%%%%%%%%%%%%%%%%%%%%%%%%%%%%%%%%%%%%%%%%%%%%%%%%%%%%%%%%%%%%%%%%%%%%%

\chapter{Conclusions and Future Directions}
\label{chapter:contribution}
The research conducted on the Priority Ranked Inspection process has shown
supporting evidence that it can be beneficial for organizations with
limited inspection resources. However, the conducted evaluation is very
preliminary. In addition, there are a number of other future directions
that are required to further the research of the PRI process.

\section{Future Directions}
The research presented on PRI in this thesis is the first of many possible
steps that are needed to validate the potential of the Priority Ranked
Inspection approach. Although PRI is centered on inspections, many other
research fields, like software quality, defect prevention and prediction,
software metrics, and the alike, play an equally important role.  For
example, the when calibrating PRI indicators, one can consult various
software metrics and defect prediction literature to determine the
threshold values that produce the best rankings. This section presents the
many different future directions of this research.

\subsection{More Evaluations}
This research contains many limitations. Most notably the evaluation of PRI
and hackyPRI is constrained to only one specific software project. This
fact raises many issues of adoption. For example, how hard would it be to
implement PRI at another organization? How hard would it be to calibrate
PRI for another set of product and process measures? This adoption issue
can be addressed by future evaluations of PRI in other organization
settings and other software projects. This issue will be left as a future
direction. However, I believe the evaluation that was conducted during this
research was necessary to provide evidence that PRI is a worthy concept to
try at other organizations.

\subsection{Implementation of the Hackystat PRI Extension}
In addition, future work is needed to generalize the hackyPRI extension so
that it is possible for other organizations and projects. Currently,
hackyPRI probably best supports the CSDL organization and the Hackystat
project. Also, there were many future tasks associated with the
implementation of the Hackystat PRI Extension that I've mentioned at the
end of Chapter \ref{chapter:system} that can still be addressed in future
research. For example: 

\begin{enumerate}
\item Solving the threats to data validity in hackyPRI.
\item Providing a configurable PRI indicator ranking with the JESS tool.
\item Providing other levels (i.e., modules, Java classes, methods) of
  rankings. 
\item Linking PRI with Software Project Telemetry to track changes over
  time.
\item Implementing an automatic feedback loop of inspection results to help
  automatically calibrate indicators.
\end{enumerate}

\subsection{How to Determine MINI-Threshold}
A major part of the PRI process that was not solved in this research is
Step 1c. Step 1c is the third step in creating a PRI ranking function and
states that the PRI ranking function should create a MINI-threshold, which
declares all documents below the threshold as MINI and all above as LINI.
The solution to this problem will be a major benefit to the PRI process,
because it will provide an organization with the exact number of documents
that should be inspected. They can use this information to schedule and
plan inspections. For example, an organization can find that they must
inspect 100 of their 500 documents and request the necessary inspection
resources to do so.


\subsection{Comparison of PRI with Code Smells and Crocodile}
In Chapter \ref{chapter:relatedwork}, I explained that the PRI process is
very similar to two software tools, Code Smells and Crocodile, which help
identify the right areas of a system to inspect. A future direction of
this research is the evaluation of the results obtained by all three
approaches. For example, one could generate the ``rankings'' from all three 
approaches on the same software system and pick different areas to conduct 
inspections on to determine the validity of each approach. Another possible 
evaluation could be studying the cost-effectiveness of each approach. 

On the other hand, the PRI process is more robust than the two approaches,
because it can include any type of product measures into the ranking
function. Therefore, another possible future direction is the incorporation
of the Code Smell and Crocodile measures into PRI.


\subsection{Use of PRI in Other Quality Assurance Situations}
Priority Ranked Inspection was originally created for purposes that span a
number of quality assurance techniques other than software inspections.
Originally, I proposed a technique that could identify the lowest cost
approach to increase quality of a particular piece of code. I envisioned a
Hackystat extension that could identify the right ``quality tool'' that is
needed to increase quality. For example, if the ranking showed that Unit
Tests are a problem area, then the right ``quality tool'' could be
increasing the number of Unit Tests for that particular piece of code.
However, for this research I have obviously decided to focus on one
``quality tool'', namely software inspection. I chose inspections because
the quality assurance literature suggests that this process is the most
effective way to increase quality. Another future direction for this
research is to evaluate if Priority Ranked Inspection can also identify the
right ``quality tool'' to use in specific situations.


\section{Final Thoughts}
The results of this research shows that it is very challenging to
thoroughly evaluate PRI. It is also a little ironic that the sole purpose
of PRI is to aid the inspection processes of organizations with limited
resources, but at the same time, to evaluate PRI a thorough inspection of
all ranked documents would provide the best results.

I firmly believe that the concept of PRI, which tries to identify the right
documents to inspect, has promise, although future research is needed to
provide more supporting evidence for that belief. Hopefully, one day
Priority Ranked Inspection will be as well known and evaluated as much as
other inspection processes like Software Inspections.






