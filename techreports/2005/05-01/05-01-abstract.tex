%%%%%%%%%%%%%%%%%%%%%%%%%%%%%% -*- Mode: Latex -*- %%%%%%%%%%%%%%%%%%%%%%%%%%%%
%% 05-01-abstract.tex -- Thesis white paper - software inspections
%% Author          : Aaron A. Kagawa
%% Created On      : Mon Sep 23 11:52:28 2004
%% Last Modified By: Aaron Kagawa
%% Last Modified On: Tue Jul  5 20:02:58 2005
%% RCS: $Id$
%%%%%%%%%%%%%%%%%%%%%%%%%%%%%%%%%%%%%%%%%%%%%%%%%%%%%%%%%%%%%%%%%%%%%%%%%%%%%%
%%   Copyright (C) 2004 Aaron A. Kagawa
%%%%%%%%%%%%%%%%%%%%%%%%%%%%%%%%%%%%%%%%%%%%%%%%%%%%%%%%%%%%%%%%%%%%%%%%%%%%%%%


\begin{abstract}  % 200 words
Imagine that your project manager has budgeted 200 person-hours for the
next month to inspect newly created source code. Unfortunately, in order to
inspect all of the documents adequately, you estimate that it will take 400
person-hours. However, your manager refuses to increase the budgeted
resources for the inspections. How do you decide which documents to inspect
and which documents to skip? Unfortunately, the classic definition of
inspection does not provide any advice on how to handle this situation. For
example, the notion of entry criteria used in Software Inspection
\cite{Gilb93} determines when documents are ready for inspection rather
than if it is needed at all \cite{Ebenau94}.

My research has investigated how to prioritize inspection resources and
apply them to areas of the system that need them more. It is commonly
assumed that defects are not uniformly distributed across all documents in
a system, a relatively small subset of a system accounts for a relatively
large proportion of defects \cite{Boehm01}. If inspection resources are
limited, then it will be more effective to identify and inspect the
defect-prone areas.

To accomplish this research, I have created an inspection process called
Priority Ranked Inspection (PRI). PRI uses software product and development
process measures to distinguish documents that are ``more in need of
inspection'' (MINI) from those ``less in need of inspection'' (LINI). Some
of the product and process measures include: user-reported defects, unit
test coverage, active time, and number of changes. I hypothesize that the
inspection of MINI documents will generate more defects with a higher
severity than inspecting LINI documents.

My research employed a very simple exploratory study, which includes
inspecting MINI and LINI software code and checking to see if MINI code
inspections generate more defects than LINI code inspections. The results
of the study provide supporting evidence that MINI documents do contain
more high-severity defects than LINI documents. In addition, there is some
evidence that PRI can provide developers with more information to help
determine what documents they should select for inspection.

\end{abstract}








