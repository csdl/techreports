%%%%%%%%%%%%%%%%%%%%%%%%%%%%%% -*- Mode: Latex -*- %%%%%%%%%%%%%%%%%%%%%%%%%%%%
%% project-overview.tex -- 
%% Author          : Philip Johnson
%% Created On      : Thu Oct  4 08:05:31 2001
%% Last Modified By: Philip M. Johnson
%% Last Modified On: Thu May 26 09:31:27 2005
%% RCS: $Id$
%%%%%%%%%%%%%%%%%%%%%%%%%%%%%%%%%%%%%%%%%%%%%%%%%%%%%%%%%%%%%%%%%%%%%%%%%%%%%%%
%%   Copyright (C) 2001 Philip Johnson
%%%%%%%%%%%%%%%%%%%%%%%%%%%%%%%%%%%%%%%%%%%%%%%%%%%%%%%%%%%%%%%%%%%%%%%%%%%%%%%
%% 
%%\nocite{*}

\section{Overview}

\subsection{Motivation}
\label{sec:motivation}

The NSF Next Generation Cybertools program has the ambitious goal of
producing technologies that ``not only change ways in which social and
behavioral scientists research the behavior of organizations and
individuals, but also serve sciences more broadly.'' This goal is
particularly salient because the increased automation and
``digitization'' of work creates a sea of information about
organizations and their processes.  The availability of data creates the
potential to revolutionize the way we understand, design, and manage
organizations.  To gain insight from this sea of data (rather than being
drowned by it), we need ways to find patterns, interpret them and
generalize appropriately.

In commercial organizations, opportunities to exploit improved mechanisms
for qualitative and quantitative data exist in every core business process,
such as new product development, customer support, supply chain management,
and basic accounting.  In addition to competitive pressures for process
control and improvement, which date back to the early days of scientific
management \cite{Pentland03b}, commercial organizations are facing
increased demands for compliance monitoring and internal controls
\cite{Hunton04}.  Technologies, such as Enterprise Resource Planning
systems, and continuous assurance auditing systems \cite{Varsarhelyi04}
create a virtual tidal wave of quantitative accounting data, but
organizations lack effective ways to integrate the qualitative data needed
to interpret it \cite{Hunton04}.

Many analogous opportunities exist in government and defense, as well.  For
example, military training and operations generates enormous amounts of
detailed operational data that must be analyzed and interpreted
\cite{Carolan04}. Like commercial organizations, military operations
include multiple, distributed participants, multiple hierarchical layers,
and qualitative and quantitative data from many sources. Current technology
for doing interpreting this data (e.g., Distributed Battlefield Exercise
Simulation and Debriefing) focuses on one exercise at a time
\cite{Johnston04}. As with commercial organizations, the military faces
significant challenges in gaining insights from qualitative and
quantitative data generated by diverse sources \cite{Carolan04,Johnston04}.

To frame our approach to cyberinfrastructure for organizational research
(TestBed I), we begin by describing an organization with a host of interesting
research opportunities and challenges directly related to this solicitation: the
Defense Advanced Research Projects Agency (DARPA) High Productivity
Computing Systems (HPCS) program \cite{hpcs}.

The mission of the HPCS program involves the development of next
generation, peta-scale high performance computing platforms for commercial
availability by 2010.  In a radical break with past high performance
computing initiatives, the focus of this program is not just on the
development of new and faster hardware. In addition, an explicit objective
of this program is to decrease radically the cost and time required by
organizations to perform their science and engineering activities that
require these high performance computing environments.  For example, the
development of a new climate model might currently require a team of dozens
of scientists and engineers several years to implement.  Next generation
HPC environments should simultaneously halve the size of the team and the
time required to implement such a system. DARPA is currently funding
research and development by IBM, Sun Microsystems, and Cray to better
understand the hardware, software, and organizational requirements to
achieve up to 10x productivity improvements.  

%% The HPCS ``organization'' thus
%% consists the program management group at DARPA, research and development
%% groups at IBM, Sun, and Cray, end-user organizations such as Lawrence
%% Livermore Laboratory and Los Alamos National Laboratory, and affiliated
%% academic research organizations at universities such as the University of
%% Maryland and the University of Hawaii.

Two of the principal investigators on this proposal have been associated
with the HPCS program as academic researchers. This has given us insight
into the enormous challenges associated with measuring, understanding,
assessing, and improving organizational behavior in the largely unstudied
domain of high performance computing system application development.  While
still in a very early stage, research by the vendors and affiliated
researchers has begun to generate a body of quantitative and qualitative
data concerning the behavior of developers and others in HPC organizations.

For example, pilot studies have been performed in a classroom setting with
students developing simple high performance systems, resulting in
quantitative data on the tools they used, the times at which they invoked
the tools and the results, and properties (such as the size) of the
software they produced \cite{Funk05}. Examples of qualitative data range
from interviews with administrative staff of high performance computing
centers to journals kept by professional developers as they work on HPC
software \cite{Votta05}.

%% Initial analyses of the raw data have included formal models, such as Timed
%% Markov Models fit to classroom data \cite{Smith05}.  Other case study data
%% has been used to generate semi-formal models, such as ``telemetry'' based
%% analyses \cite{csdl2-04-11}.  Still other kinds of data, such as the
%% qualitative journal data, have been subjected to qualitative encoding
%% techniques \cite{Votta05}.  Research has also led to proposals for new ways
%% to assess high performance productivity, such as Purpose-Based Benchmarks
%% \cite{Gustafson04}.

%% So far, dissemination of research data and results have been via
%% traditional, non-digital mechanisms: HPCS program meetings
%% \cite{hpcs-meeting} academic workshops \cite{pphec05,sehpcs05}, and themed
%% journal issues \cite{ijhpca04}.  No data repositories exist providing
%% access to the qualitative and quantitative data gathered by these
%% activities.

The HPCS program and its organizations are confronting a variety of
organizational research challenges directly related to the goals of the
Next Generation CyberTools program.

First, the HPCS program is revealing the need for primary research on
organizations using high performance computing environments. Basic
questions need to be answered: How are high performance computing system
applications developed and maintained?  Where are the productivity
bottlenecks? What are the organizational constraints on innovation in
technology or methods? What is the most appropriate research methodology,
or combination of methodologies, for gaining insight into these questions?
This primary research will require the collection of substantial amounts of 
qualitative and quantitative data from a variety of contexts that must be 
disseminated to a broad range of users for a diversity of analyses.

Second, the answers to these basic questions must support the design of new
technologies and organizational procedures that will yield an order of
magnitude productivity improvement in high performance computing
applications.  This requires the operational definition and empirical validation 
of a productivity measure, generation of tools to collect the data necessary
to calculate the productivity measure, and deployment of these tools in different
computational environments and application domains. 

Third, the HPCS program serves as an umbrella over many different types of
organizations, generating substantial challenges regarding the publication
and/or protection of information.  The three HPCS vendor awardees, Sun,
IBM, and Cray, are motivated to publish certain types of research
results regarding productivity in order to (for example) influence the
ultimate definition of the productivity measure used to evaluate their
systems. On the other hand, each organization also generates research
results that constitute proprietary information. The ultimate end-users of
these systems (government and military laboratories, automobile companies,
financial service institutions, etc.)  form another set of organizations.
The academic and corporate researchers form a third set of
organizations. Collection and dissemination of qualitative and quantitative
data amongst these organizations requires mechanisms for protection of privacy 
as well as proprietary trade secrets. 

Fourth, the HPCS program is distributed geographically and involves a large
number of constituent organizations and concurrent research activities.  A
major challenge to the program involves the requirement for alignment among the many
approaches to qualitative and quantitative data gathering and research
methods.  An effective alignment will enable replication, in which data
gathered to test a hypothesis at one site can be gathered in a similar
manner at another site in order to see if the hypothesis is similarly
supported.  Alignment will also enable meta-analysis, in which data from
multiple sites can be validly composed together into a larger dataset for
the purpose of certain analyses. 

We will return to the HPCS program in the Research Plan, where we will
propose to deploy our cyberinfrastructure into it as part of a case study
to evaluate our methods and technologies.  

\subsection{Cedar: Cyberinfrastructure for Empirical Data Analysis and Reuse}
\label{sec:cedar}

In this research, we propose to design, implement, and evaluate Cedar: a
CyberInfrastructure for Empirical Data Analysis and Reuse, to satisfy the
requirements for Testbed I.  Cedar is intended to be an open source
information infrastructure architecture coupled with a data management
policy mechanism that supports scalable and collaborative, qualitative and
quantitative organizational research data collection, analysis,
dissemination, and archiving.

By {\em open source}, we mean not only that Cedar's source code will be
released under a license that allows access and modification by others, but
also that we intend to create a community of developers willing and able to
maintain and enhance the Cedar system beyond the period of this grant.

By {\em information infrastructure architecture}, we mean that Cedar will
not be a monolithic system, but instead will specify a set of interfaces that
allow integration and interoperability of tools for qualitative and
quantitative data collection, analysis, and dissemination that will be
developed both by us and by others.

By {\em data management policy mechanism}, we mean that Cedar will
implement procedures that support context-sensitive publication,
suppression, or perturbation of raw or processed qualitative or
quantitative data, and support evolution in the policies applied to any
specific data item over time.  Appropriate data management policies should
also generate incentives for data contribution and dissemination.

By {\em scalable and collaborative, qualitative and quantitative
organizational research data}, we mean that Cedar will provide a
federated network of peer-to-peer servers, creating scalability to 
thousands of concurrent data collection and analysis activities, and
allowing analysis and annotation of data by many researchers across many
institutions.

Finally, by {\em collection, analysis, dissemination, and archiving}, we
mean that Cedar will support data management policies across the entire
lifecycle of qualitative and quantitative data.

Cedar is an ambitious project that will require efficient and effective
research and technology development in order to achieve its objectives
during the grant period.  At a high level, the project will focus on 
the following activities:

(1) {\em Infrastructure technology research and development.}  Through the
Hackystat Project, Principle Investigator (PI) Johnson has
developed expertise in the development of open source
collaborative systems for collection and analysis of quantitative data for
software engineeering research and experimentation.  The Hackystat system
and experiences provide a base for extension into qualitative data
collection and analysis, as well as to a peer-to-peer network of federated
servers.

(2) {\em Research on and development of policies and procedures for data
  privacy and dissemination.} PI Basili is leading a task force of
software researchers with experience in developing and maintaining software
engineering empirical data repositories with the goal of articulating prior 
problems and proposing improvements for management of future repositories. 
We will leverage this initial research and incorporate related research in 
privacy policies and technologies for integration into the Cedar infrastructure. 

(3) {\em Research on and development of models and mechanisms for
representation and integration of qualitative and quantitative
information.}  PI Pentland and PI Feldman have carried out a variety of
research on the theoretical underpinnings of qualitative and quantitative
empirical data and its appropriate interpretation.  Cedar will leverage
these insights with technological infrastructure for collection, analysis,
and dissemination of empirical data according to narrative and network theories 
for representation and analysis of qualitative and quantitative data. 

(4) {\em Case study evaluation of Cedar.}  The four PIs (Johnson, Basili,
Pentland, Feldman) have substantial prior experience in the design and
implementation of case studies across a variety of application domains and
organizational types. To test the validity of Cedar, and to understand its
strengths and limitations, we will perform a case study with selected
organizations involved in the DARPA HPCS program.






