%%%%%%%%%%%%%%%%%%%%%%%%%%%%%% -*- Mode: Latex -*- %%%%%%%%%%%%%%%%%%%%%%%%%%%%
%% project.intro.tex -- 
%% Author          : Philip Johnson
%% Created On      : Fri Jan 13 19:47:12 2012
%% Last Modified By: Philip Johnson
%% Last Modified On: Wed Jan 18 13:13:30 2012
%%%%%%%%%%%%%%%%%%%%%%%%%%%%%%%%%%%%%%%%%%%%%%%%%%%%%%%%%%%%%%%%%%%%%%%%%%%%%%%

\subsection{Introduction}

Development of the ``Smart Grid'', a modernized power infrastructure, is
one of the key technological challenges facing the United States at the
dawn of the 21st century. According to the Department of Energy (DoE) , the smart
grid should: (1) Enable active participation by consumers by providing
choices and incentives to modify electricity purchasing patterns and
behavior; (2) Accommodate all generation and storage options, including
wind and solar power.  (3) Enable new products, services, and markets
through a flexible market providing cost-benefit trade-offs to consumers
and market participants; (4) Provide reliable power that is relatively
interruption-free; (5) Optimize asset utilization and maximize operational
efficiency; (6) Provide the ability to self-heal by anticipating and
responding to system disturbances; (7) Resist attacks on physical
infrastructure by natural disasters and attacks on cyber-structure by
malware and hackers \cite{NETL:GridCharacteristics}.

Supporting ``all generation options'' implies the need to support
distributed, small scale, intermittent generation such as residential
solar.  Supporting ``all storage options'' implies the need to support
distributed, small scale, intermittent storage such as electric vehicle
batteries.  Once energy generation and storage can be just as decentralized
as energy consumption, the utility of microgrids as a building block for
the DoE Smart Grid becomes clear.  Microgrids are semi-autonomous,
self-regulating electrical systems below the subsystem level that can
involve small-scale storage and generation as well as consumption.

\subsubsection{Vision}
\label{sec:vision}

\footnote{Section \ref{sec:vision} must provide a concise description of the ``vision'' for our
  proposed sustainable energy pathway that focuses on energy transmission,
  distribution, efficiency, and use.
  We must show in our vision: a combination of scientific knowledge and
  technical innovation; a recognition of environmental, societal, and
  economic imperatives; and a promotion of education and workforce
  development.
  We need a transformative approach to SEP, not incremental advances or
  deployment of existing technologies. }

Our vision for a sustainable energy pathway begins with the creation of a
functional, semi-autonomous, self-regulating microgrid that creates a
positive energy future for the University of Hawaii at Manoa campus. This
campus currently faces severe energy challenges: while energy
consumption per square foot is among the lowest across campuses in the
nation (at approximately 65K BTU per square foot), the cost of energy per
square foot is among the highest in the nation (approximately \$4.50 per
square foot), as is the cost of energy per student FTE (approximately
\$1,300 per student FTE).  

Making matters more complicated, an aggressive retrofitting of mechanical
systems at the University over the past 10 years has largely exhausted the
traditional avenue to energy cost reduction.  In addition, because the
State of Hawaii depends on fossil fuels for almost 90\% of its energy, it
is possible, if not probable, that the cost of energy in Hawaii, already
the highest in the nation, will rise substantially in the next 20
years. The current high cost of energy and the probability of it rising
even higher, combined with the exhaustion of traditional approaches to
energy cost reduction, has led David Hafner, an Assistant Vice Chancellor
at the University of Hawaii to state, {\em ``the cost of energy represents
  an {\em existential} threat to the University of Hawaii as a Research I
  university''} \cite{Hafner2011}.

We present this background to emphasize that our choice to focus on a
sustainable energy pathway for the University of Hawaii at Manoa campus is
not because it would be ``nice to have'', it is because the status quo is
quite literally unsustainable.  To retain its current quality of campus
life, the University of Hawaii must find a way to generate a substantial
fraction of campus power, reduce the cost per kilowatt of electricity
purchased from the utility, and create a ``climate of energy conservation''
among campus members that minimizes overall demand.

Our vision involves the transformation of the campus from one which
passively delegates to the utility all responsibility for its energy needs,
to one which owns and operates a smart, sustainable microgrid involving up
to 5 MW of solar generation, short term, small scale storage, automated
demand response for the major on-campus HVAC systems, and consumer facing
information technology to engage campus members in support of the
microgrid and its goals.  This vision requires innovation in transmission,
distribution, efficiency, and use.  Our research and development plan
involves five interrelated research components, summarized as follows:

\begin{enumerate}

\item {\em Sensors and monitoring.} A responsive microgrid requires the
  ability to assess its current state and estimate its future state.  To
  enable these capabilities, this component will design and install a
  network of strategically located power and environmental sensors into the
  UH campus as well as into the neighboring vicinity. This raw data will be
  collected and stored in a server for use in modeling and analysis.

\item {\em Modeling and analysis.}  This component takes the raw power and
  environmental data and applies stochastic modeling techniques to gain
  insight into both the current and near-future state of the grid.  This
  component provides the information necessary for control and
  optimization.

\item {\em Control and optimization.}  This component uses models and
  analyses to support voltage and frequency regulation, peak shaving and
  peak shifting.  By maintaining quality of service while reducing load and
  ramp, the University microgrid will reduce the cost per kWh of
  supplemental energy bought from the utility.

\item {\em Social, economic, privacy, security, and policy implications.}
  It is explicitly not our goal to create a grid that operates
  transparently and invisibly from its users.  Indeed, we believe that part
  of the problems with our current electrical infrastructure results from
  lack of public awareness concerning the problems of reliable, sufficient,
  and sustainable energy production.  This component investigates the
  information technology necessary to inform campus members about the
  microgrid in an actionable form that lets them actively participate in
  achieving its goals.  In so doing, we must confront and address the
  privacy and security issues that result, both with respect to the
  security of the grid itself and the ways grid data could be used to
  inappropriately monitor campus member behaviors.

\item {\em Education and workforce development.} All of the PIs on this
  project are also active participants in the Center for Renewable Energy
  and Island Sustainability (REIS), a project with a central focus on
  workforce development in renewable energy.  As a natural result, our
  vision includes the development of curriculum materials about microgrid
  design, implementation, and evaluation, and students with real-world
  experience in development of the microgrid.  Our participation in REIS
  helps ensure that workforce development will span multiple disciplines
  including computer science, engineering, economics, urban planning, law,
  biology, and other disciplines.

\end{enumerate}

Our vision begins, but does not end, with addressing the energy challenges
facing the University of Hawaii at Manoa campus.  First, the creation of a
functional microgrid for the UH Manoa campus will create technology, data,
and workforce training essential for the development of similar microgrids
for other educational, government, and military ``campuses'' across the
Hawaiian islands.  Replication of the approach will yield important
insights into the transformation of a single centralized, top-down
grid into multiple, decentralized, federated microgrids. 

Second, while the mainland US does not yet feel the level of energy
pressure faced by Hawaii, we believe these pressures will rise across the
country in coming years as the price of oil rises or if a commitment to
sustainable energy pathways is made. We believe the science and engineering
produced by this research will provide significant aid to microgrid
development outside Hawaii, and thus provide a important building block in
service of the Department of Energy's goal of a nationwide Smart Grid.

\subsubsection{Integration}
\label{sec:integration}

\footnote{Section \ref{sec:integration} must summarize how we will approach the research from an
  interdisciplinary perspective that integrates science and engineering
  with a synergistic, systems approach.}

This project is designed to explore the challenges and opportunities of
microgrid development from multiple perspectives and multiple disciplines.   

From a scientific perspective, our sustainable energy pathway will produce
opportunities for the development of new analytic methods.  As further
discussed in Section \ref{sec:modeling}, this research is expected to
result in the development of new analytic methods based upon belief
propogation networks and Markov models.  It will also support the
development of controlled and semi-controlled experiments to understand both
the physical issues underlying a semi-autonomous, self-regulating microgrid
(as discussed in Section \ref{sec:optimization}), as well as the social 
issues (as discussed in Section \ref{sec:social}).

From an engineering perspective, the challenges and opportunities are
obvious and manifest.  The microgrid requires the specification,
acquisition, and/or fabrication of hardware and software components for
sensing and regulation of electrical and environmental data. These
components must be integrated into the current electrical and physical
infrastructure of the University of Hawaii campus in a manner compliant
with all state and federal regulations.  Finally, the microgrid must
interact with the centralized grid provided by the utility, both accepting
control signals from that grid as well as providing status information back
to the utility.

The scientific and engineering challenges will not be faced in isolation
but form a natural, complementary, and synergistic pair of viewpoints. For
example, the development of analytic methods will be guided by the control
and optimization goals.  Conversely, resolution of engineering challenges
such as the optimal placement of sensors will be guided by controlled
experimental procedures that are intended to produce methods for sensor
placement useful outside of the University of Hawaii context.

\subsubsection{Collaboration (Management Plan)}
\label{sec:collaboration}

\footnote{Section \ref{sec:collaboration} must discuss the roles, qualifications, and
  synergy of the multi-disciplinary team, the leadership structure, and the
  integration of the proposed activities among team members.  
  International or industrial collaborations can strengthen the proposal.
}

This team consists of four principal investigators (Professors Aleksandar
Kavcic, Philip Johnson, Anthony Kuh, and Matthias Fripp from the University
of Hawaii) along with four collaborators (Dora Nakafuji, Hawaiian Electric
Company; David Hafner and Stephen Meder, Associate Vice Chancellors for the
University of Hawaii; and Jeff Mikulina, Blue Planet Foundation).  Each of
the four collaborators have supplied a letter indicating their support for
the project and interest in participation.

Team members were carefully chosen to provide a broad, interdisciplinary,
and complementary set of skills. Professors Kuh, Kavcic, and Fripp are from
the Department of Electrical Engineering and have expertise in sensing,
modeling, and power systems.  Professor Johnson is from the Department of
Information and Computer Sciences and has expertise in software engineering
and consumer-facing interfaces to the Smart Grid.  Dr. Dora Nakafuji is Director of
Renewable Energy at Hawaiian Electric Company and has expertise in
utility-side systems. David Hafner is an Assistant Vice Chancellor and head
of Facilities Management at the University of Hawaii and has expertise in
UH power systems and requirements.  Stephen Meder is also an Assistant Vice
Chancellor and head of sustainability initiatives at the University of
Hawaii.  Finally, Jeff Mikulina is the Executive Director of Blue Planet
Foundation, a Hawaii-based environmental advocacy group which has played a
major role in renewable energy policy development in Hawaii.

For a summary of the basic roles and responsibilities of these team
members, it is useful to think of the project as consisting of two
dimensions.  The first dimension, domain of inquiry, comprises two basic
areas: science/engineering and social/environmental/policy.  The second
dimension, domain of application, also has two basic areas: internal to the
proposed microgrid (i.e. the University of Hawaii campus) and external to
the proposed microgrid (i.e. the surrounding environment).

Figure \ref{fig:team}
illustrates these two dimensions and the focal points for the team
members in a simple table.  As the table shows, at least one PI and at
least one collaborator is associated with each of the four areas in the
table. 


\begin{figure}
\begin{tabular}{|p{1.8in}|p{2.0in}|p{1.8in}|}
\hline
& {\bf Inside microgrid / UH} & {\bf Outside microgrid / UH}  \\ \hline
{\bf Science / Engineering} & Kavcic, Kuh, Johnson, Hafner & Fuji, Fripp \\ \hline
{\bf Environment / Policy} & Hafner, Meder, Kuh & Mikulina, Meder, Johnson \\ \hline
\end{tabular} 
\caption{Team members and focal areas}
\label{fig:team}
\end{figure}


The project will be led by Dr. Kuh who will manage the work, ensure
coordination among the investigators, and track milestones. The entire team
will meet monthly by conference call to discuss research and education
progress, with smaller meetings in focus areas occuring more frequently.  A
yearly summit meeting will provide an opportunity to assess overall
progress and adjust milestones if necessary.

At the University of Hawaii, the PIs will hold weekly meetings
with graduate students.  Typically, a student or faculty member will
present their research results which will be critiqued by the whole group.
The meetings will ensure that work is collaborative and will give all
investigators and students an opportunity to see the different phases of
the research project.  Dr. Nakafuji recently became an adjunct Professor at
the University of Hawaii and will join our meeting at regular intervals.

The PIs will serve as leaders for the five research components. Professor
Kuh will lead sensors and monitoring, Professor Kavcic will lead
modeling and analysis, Professor Fripp will lead control and optimization,
Professor Johnson will lead social, economic, privacy, security, and policy
implications, and Professor Kuh will (also) lead education and workforce
development. 

All PIs will work together on recruiting, retention, and outreach
efforts.  Special attention will be given to recruiting of underrepresented
students with assistance from NHSEMP and SWE.
 
