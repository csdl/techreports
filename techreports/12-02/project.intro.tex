%%%%%%%%%%%%%%%%%%%%%%%%%%%%%% -*- Mode: Latex -*- %%%%%%%%%%%%%%%%%%%%%%%%%%%%
%% project.intro.tex -- 
%% Author          : Philip Johnson
%% Created On      : Fri Jan 13 19:47:12 2012
%% Last Modified By: Philip Johnson
%% Last Modified On: Tue Jan 17 16:35:47 2012
%%%%%%%%%%%%%%%%%%%%%%%%%%%%%%%%%%%%%%%%%%%%%%%%%%%%%%%%%%%%%%%%%%%%%%%%%%%%%%%

\subsection{Introduction}

Development of the ``Smart Grid'', a modernized power infrastructure, is
one of the key technological challenges facing the United States at the
dawn of the 21st century. According to the Department of Energy (DoE) , the smart
grid should: (1) Enable active participation by consumers by providing
choices and incentives to modify electricity purchasing patterns and
behavior; (2) Accommodate all generation and storage options, including
wind and solar power.  (3) Enable new products, services, and markets
through a flexible market providing cost-benefit trade-offs to consumers
and market participants; (4) Provide reliable power that is relatively
interruption-free; (5) Optimize asset utilization and maximize operational
efficiency; (6) Provide the ability to self-heal by anticipating and
responding to system disturbances; (7) Resist attacks on physical
infrastructure by natural disasters and attacks on cyber-structure by
malware and hackers \cite{NETL:GridCharacteristics}.

Supporting all generation options implies the need to support distributed,
small scale, intermittent generation such as residential solar.  Supporting
all storage options implies the need to support distributed, small scale,
intermittent storage such as electric vehicle batteries.  Once energy
generation and storage can be just as decentralized as energy consumption,
the need for ``micro-grids'' as a building block for the DoE Smart Grid
becomes clear.  A micro-grid, based upon these requirements, is a
semi-autonomous, self-regulating electrical system below the subsystem
level in the current power grid that can involve small-scale storage and
generation as well as consumption.

\subsubsection{Vision}
\label{sec:vision}

\footnote{Section \ref{sec:vision} must provide a concise description of the ``vision'' for our
  proposed sustainable energy pathway that focuses on energy transmission,
  distribution, efficiency, and use.
  We must show in our vision: a combination of scientific knowledge and
  technical innovation; a recognition of environmental, societal, and
  economic imperatives; and a promotion of education and workforce
  development.
  We need a transformative approach to SEP, not incremental advances or
  deployment of existing technologies. }

Our vision for a sustainable energy pathway is the creation of a
functional, semi-autonomous, self-regulating micro-grid that creates a
positive energy future for the University of Hawaii at Manoa campus. Our
campus faces significant energy challenges: while energy
consumption per square foot is among the lowest across campuses in the
nation (at approximately 65K BTU per square foot), the cost of energy per
square foot is among the highest in the nation (approximately \$4.50 per
square foot), as is the cost of energy per student FTE (approximately
\$1,300 per student FTE).  

Making matters more complicated, an aggressive retrofitting of mechanical
systems at the University over the past 10 years has largely exhausted this
traditional avenue to energy cost reduction.  Because the State of Hawaii
depends on fossil fuels for almost 90\% of its energy generation, it is
possible, if not likely, that the cost of energy to the University of
Hawaii will rise substantially in the next 20 years. The current high cost
of energy and the probability of it rising much higher in the next two
decades, along with the exhaustion of traditional approaches to energy
efficiency, has led David Hafner, an Assistant Vice Chancellor at the
University of Hawaii to state, {\em ``the cost of energy represents an
  existential threat to the University of Hawaii as a Research I
  university''} \cite{Hafner2011}.

We present this background to emphasize that our motivation for a
sustainable energy pathway for the University of Hawaii at Manoa campus is
not because it would be ``nice to have'', it is because the status quo is
quite literally unsustainable.  To retain our current quality of campus
life, we must find a way to generate a substantial fraction of campus
power, reduce the cost per kilowatt of electricity purchased from the
utility through peak shaving and ramp reduction, and create a ``climate of
energy conservation'' among campus members that minimizes overall demand. 

Our vision is to transform the campus from one which passively delegates to
the utility all responsibility for its energy needs, to one which contains
a smart, sustainable micro-grid campus involving up to 5 MW of solar
generation, short term, small scale storage, automated demand response for
the major on-campus HVAC systems, and consumer facing information
technology to engage campus members in support of the micro-grid and its
goals.  This vision requires innovation in transmission, distribution,
efficiency, and use.  Our research and development plan involves five
interrelated research components, summarized as follows:

\begin{enumerate}

\item {\em Sensors and monitoring.} A responsive micro-grid requires the
  ability to assess its current state and estimate its future state.  To
  enable both of these capabilities, the ``sensors and monitoring''
  component will involve the design and installation of a network of
  strategically located power and environment sensors into the UH campus as
  well as into the neighboring vicinity. This data will be collected and
  stored in a server.

\item {\em Modeling and analysis.}  This component takes the raw power and
  environmental raw data gathered from sensors and stored in the server and
  applies stochastic modeling techniques to gain insight into both the
  current and near-term future state of the grid.  The modeling and
  analysis components will provide the information necessary for control
  and optimization of the grid, as discussed next.

\item {\em Control and optimization.}  The models and analytic techniques
  created by the last component will be applied to support voltage and
  frequency regulation, peak shaving and peak shifting (with the goal of
  achieving reduced rates from the utility) and lowered overall
  consumption. 

\item {\em Social, economic, privacy, security, and policy implications.}
  It is explicitly not our goal to create a grid that operates
  transparently and invisibly from its users.  Indeed, we believe that part
  of the problems with our current electrical infrastructure results from
  lack of public awareness concerning the problems of reliable, sufficient,
  and sustainable energy production.  This research component investigates
  the information technology necessary to inform campus members about the
  micro-grid in an actionable form that lets them actively participate in
  achieving its goals.  In so doing, we will address the privacy and
  security issues that result, both with respect to the security of the
  grid itself and the ways grid data could be used to inappropriately
  monitor campus member behaviors.

\item {\em Education and workforce development.}  All of the PIs on this
  project are also active participants in the Center for Renewable Energy
  and Island Sustainability (REIS), a project with a major focus on
  workforce development in renewable energy.  As a natural result, this
  project will result in the development of curriculum materials and
  students who have participated in the development of the micro-grid.  We
  will leverage REIS to ensure that workforce development will span
  multiple disciplines including computer science, engineering, economics,
  and other disciplines.

\end{enumerate}

Our vision begins, but does not end, with addressing the energy challenges
facing the University of Hawaii campus.  We believe the science and
engineering produced by this research will provide significant aid to
others developing micro-grids, and thus provide a building block to support
the Department of Energy's goal of a nationwide Smart Grid.

\subsubsection{Integration}
\label{sec:integration}

\footnote{Section \ref{sec:integration} must summarize how we will approach the research from an
  interdisciplinary perspective that integrates science and engineering
  with a synergistic, systems approach.}



\subsubsection{Collaboration (Management Plan)}
\label{sec:collaboration}

\footnote{Section \ref{sec:collaboration} must discuss the roles, qualifications, and
  synergy of the multi-disciplinary team, the leadership structure, and the
  integration of the proposed activities among team members.  
  International or industrial collaborations can strengthen the proposal.
}


