%%%%%%%%%%%%%%%%%%%%%%%%%%%%%% -*- Mode: Latex -*- %%%%%%%%%%%%%%%%%%%%%%%%%%%%
%% project.sensing.tex -- 
%% Author          : Philip Johnson
%% Created On      : Fri Jan 13 07:58:21 2012
%% Last Modified By: Philip Johnson
%% Last Modified On: Thu Jan 26 13:48:07 2012
%%%%%%%%%%%%%%%%%%%%%%%%%%%%%%%%%%%%%%%%%%%%%%%%%%%%%%%%%%%%%%%%%%%%%%%%%%%%%%%

\subsubsection{Research component: Sensing, monitoring, and data gathering}
\label{sec:sensing}

In this research component, we will address the problems of sensing,
monitoring, and data gathering for a microgrid with a high penetration of
distributed renewable energy generation.  Theoretical studies will be
considered along with extensive simulations in conjunction with the
development of the actual University of Hawaii at Manoa microgrid.  The UHM
microgrid will support validation of the simulation and
theoretical studies.

Today's modern electrical grid system relies on a vast web of distribution
networks (D) and transmission corridors (T) to deliver power to customers.
Tremendous control, visibility and communication resources and standards
exist to support the ``T'' portion.  The ``D'' portion has received much
less attention and poses one of the weakest links to enabling wide adoption
of more distributed resources.  The ``D'' portion tends to be more
disperse, less automated and non-standard across the nation.  Making
matters worse, the utility operators have limited or no visibility to the
distributed resources emerging on their systems.  The influx of
customer-sited PV systems and emerging plug-in-hybrid vehicle technologies
deployed at the distribution level have accelerated the need to accurately
model impacts of these distributed generation (DG) resources on the
electric system.  On the UHM microgrid this requires design and
installation of appropriate sensing equipment to monitor environmental
resource data, electrical grid data, and building energy usage.  This is
closely tied to other aspects of the project including modeling and
analysis, optimization and control, and the economic and social
implications.
 
This research considers both theoretical and practical aspects of gathering
data from sensor and monitors.  Multiple data types will be gathered
requiring construction of sensor networks and data fusion from the multiple
sensor networks.  A key is to deploy enough monitors and sensors so that
important information can be gathered and processed.  Information that is
needed are events that can will impact the operation of the grid.  This
includes detecting ramps that occur with solar energy due to changing
weather conditions (from cloudy to sunny or vice versa) and faults that can
occur on the electrical grid.  With sufficient sensor networks deployed
intelligent control and optimization strategies can be made to reduce
energy usage in buildings subject to economic considerations and energy
requirements made on labs, classrooms, and offices in the building.  In
this subsection we discuss the data types, data collection, building sensor
networks, and gathering information from the data (sensor fusion,
compression, and data mining).
 
\paragraph{Data types.}
 
To accurately model and analyze the microgrid along with designing
optimization and control strategies for the grid several sets of data need
to be monitored.  We also need to consider measuring data both spatially
and temporally.  The data that we consider in this proposal are
environmental resource data, electrical grid data, building energy usage,
and data gathered from users of energy on the microgrid.  This data will
be gathered from the UHM microgrid.
 
Environmental resource data is monitored so that we can better forecast the
energy generation of distributed renewable energy sources such as solar PV
panels.  Here we build and deploy sensor boxes that measure solar
irradiation, wind speed and direction, temperature, and humidity.  These
sensors and monitors will be placed on rooftops of UHM buildings with key
considerations being how fast we sample data, how many sensors do we
deploy, and where do we place sensors.
 
Electrical grid data will also be gathered from existing utility meters.
Additional meters will be placed at key locations on buses and electrical
lines.  Currents, voltages, complex power, and energy usage will be
measured.  Again key considerations are how fast we sample data, where to
deploy meters, and what sensors and monitors do we use that are already
deployed on the grid.
 
We would like to get better information about load data and this can be
done by deploying sensors and monitors in buildings.  Here we can monitor
energy usage by deploying sensors to measure energy usage in rooms from air
conditioning, electrical equipment, and lighting. The data will be stored
in WattDepot, an open source platform for energy data collection, storage,
analysis, and visualization developed by PI Johnson's software development 
laboratory \cite{csdl2-10-05}.
 
\paragraph{Data collection.}

There are several issues to be considered when getting data from sensors,
monitors, and meters.  How much information is to be gathered, how do we
collect this data, and how do we organize this data?  The first issue is
discussed here with the other two issues discussed in subsequent
subsections.

Where do we place the sensors?  There has been substantial research on the
sensor placement problem from researchers studying sensor networks with
applications for environmental monitoring \cite{dhillon2,dhillon3},
monitoring the electrical grid with PMUs \cite{chen-abur,xu-abur}, and
biomedical monitoring.  Tools from signal processing, statistics, and
machine learning are used.  The problem is often framed as an optimization
problem of minimizing some error criterion or maximizing some information
criterion.  Once space has been discretized it can be formulated as placing
$m$ sensors among $n>m$ possible locations.  Optimal algorithms are
computationally infeasible for large $m$ and $n$, but greedy algorithms
that run in polynomial time of $m$ and $n$ often give good approximations
to optimal algorithms \cite{li-negi-ilic,krause-singh-guestrin}.  Some key
aspects of the optimization include properties of submodularity and
monotonicity \cite{nwf}.  We study these algorithms as it applies to our
algorithm for both static and dynamic cases.  A state space model can be
used to model the dynamics of different parameters with a linear state
space model given by

\begin{displaymath}
x(k+1) = A x(k) + B u(k)
\end{displaymath}
\begin{displaymath}
y(k) = C (x(k) + v(k))
\end{displaymath}

\noindent where $x(k)$ is the state of the system (e.g. on the electrical grid,
current phasor at $n$ different nodes).  A goal is to determine the best
$C$ matrix which is a binary 0/1 matrix with at most one 1 in any column
and exactly one 1 in every row.

We must also determine how often to sample data.  The sensor placement
problem and sampling problem depend on the information we need to control
and optimize the microgrid.  This depends on the modeling and analysis of
the microgrid and also what events we are trying to detect.  Other factors
that come into play are the economics associated with placing sensors,
monitors, and meters.  A goal is to place and sample sensors so that they
give the most useful information.  With limited resources we would consider
placing sensors on the grid and to monitor the environment along an
electrical line where there is significant renewable generation and hence
intermittent generation.  This will give us a better understanding of
effects of renewable energy generation, how we can model this, and
designing control and optimization algorithms that can mitigate
intermittency subject to available resources; batteries and demand
response.

\paragraph{Sensor network implementation.}

We will create two sensor networks at the microgrid level.  One sensor
network will consist of environmental resource data collected from student
built self powered energy boxes that measure solar irradiance, wind speed
and direction, temperature, and humidity.  There will also be other
environmental sensors such as solar resource locational monitors (e.g. LM2
monitors).  A goal is to monitor environmental resource data both spatially
and temporally on the UHM campus.  Most of these sensors will be placed on
top of UHM buildings.  This will aid in understanding energy produced by
distributed PV sources placed around the campus.  Wireless networks will be
used to gather this data to a central server and storage system.

A second sensor network will monitor the electrical grid.  This will
consist of sampling existing HECO utility meters and deploying demand side
load monitors (e.g.  AMI and PMI type devices) and PMI field device data
concentrators.  These sensors will be connected together via a wireless
network and again sent to a WattDepot server for storage, analysis, and visualization.

\paragraph{Information gathering.}

We have a Smart Energy Laboratory in 493 Holmes to collect the data from
different sensor networks.  This will be stored in a data server and
storage system running WattDepot \cite{csdl2-10-05}.  Timestamped data will
be gathered from sensor networks monitoring environmental resource data,
electrical grid data, and sensor netqworks in buildings monitoring energy
usage.  In addition, as discussed in Section \ref{sec:social}, there will
be data collected from students, staff, and faculty on campus about their
energy usage.  A goal of data collection is to enable data fusion so
that information can be obtained to assist in detecting key events that
happen on the microgrid (e.g ramps in energy produced by distributed solar
PV and faults that occur on the electrical grid).

Tools are needed to organize and fuse the raw data from sensor networks.
This includes using filtering, sensor fusion, source coding, compressed
sensing, and machine learning to process the data to extract useful
information from the data.  Once this is done models can be created of the
microgrid to analyze and simulate the behavior of the grid.  Then control
and optimization algorithms can be tested on simulation networks to deal
with events such as ramping and faults.  Software packages such as
Powerworld, matlab, and PScad will be used along with visualization tools
such as TREX designed by Referentia.


 



