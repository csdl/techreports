%%%%%%%%%%%%%%%%%%%%%%%%%%%%%% -*- Mode: Latex -*- %%%%%%%%%%%%%%%%%%%%%%%%%%%%
%% project.social.tex -- 
%% Author          : Philip Johnson
%% Created On      : Fri Jan 13 07:58:21 2012
%% Last Modified By: Philip Johnson
%% Last Modified On: Fri Jan 13 08:01:15 2012
%%%%%%%%%%%%%%%%%%%%%%%%%%%%%%%%%%%%%%%%%%%%%%%%%%%%%%%%%%%%%%%%%%%%%%%%%%%%%%%

\subsubsection{Social, economic, privacy, security, and policy implications}

{\em It would be naive to assume that a smart, sustainable microgrid could
  be implemented transparently and invisibly to its users.  Indeed, we
  believe that part of the problems with our current electrical
  infrastructure results from lack of public awareness concerning the
  problems of reliable, sufficient, and sustainable energy production.
  Many people are used to virtually unlimited, low-cost, and reliable
  electrical energy produced in an unsustainable, environmentally harmful
  manner and do not yet understand why this cannot continue.

  Thus, an important part of this project is to develop information that
  can not only be used to control the grid, but also be used to inform the
  inhabitants of the microgrid about its current state in a useful,
  actionable form.  We believe this is important not only to obtain support
  from consumers for the costs and complexity associated with development
  of a smart, sustainable microgrid, but also to enable them to become
  active participants in management of the grid.  By actively
  participating, efficiencies not possible with chiller management alone
  can be realized.  In addition, it will create the data necessary for
  broader policy decisions by local government that can make future smart,
  sustainable microgrids easier to develop and deploy.
  
  We will also investigate the privacy and security aspects of this technology.

  Contributions of this part of the research will involve the design and evaluation of
  user-facing energy awareness and control systems and their impact on
  overall energy use, and an initial understanding of policy
  implications. }
