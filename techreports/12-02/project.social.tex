%%%%%%%%%%%%%%%%%%%%%%%%%%%%%% -*- Mode: Latex -*- %%%%%%%%%%%%%%%%%%%%%%%%%%%%
%% project.social.tex -- 
%% Author          : Philip Johnson
%% Created On      : Fri Jan 13 07:58:21 2012
%% Last Modified By: Philip Johnson
%% Last Modified On: Wed Jan 25 15:58:47 2012
%%%%%%%%%%%%%%%%%%%%%%%%%%%%%%%%%%%%%%%%%%%%%%%%%%%%%%%%%%%%%%%%%%%%%%%%%%%%%%%

\subsubsection{Social, economic, privacy, security, and policy implications}

As discussed in the introduction to this proposal, one of the key
properties of a smart grid according to the Department of Energy is to
``enable active participation by consumers''.  We believe that this is
especially important for a microgrid, and that it would be naive to assume
that a smart, sustainable microgrid could be implemented transparently and
invisibly to its users.  Many people are used to virtually unlimited,
low-cost, and reliable electrical energy produced in an unsustainable,
environmentally harmful manner and do not yet understand why this cannot
continue.

Thus, an important part of this project is to develop information that can
not only be used to control the grid, but also be used to inform the
inhabitants of the microgrid about its current state in a useful,
actionable form.  We believe this is important not only to obtain support
from consumers for the costs and complexity associated with development of
a smart, sustainable microgrid, but also to enable them to become active
participants in management of the grid.  By actively participating,
efficiencies not possible with chiller management alone can be realized.
In addition, it will create the data necessary for broader policy decisions
by local government that can make future smart, sustainable microgrids
easier to develop and deploy.

\paragraph{Privacy and security issues.}

Enabling active participation by consumers is made difficult by the fact
that the collection, analysis, and usage of power and environmental data
within a micro-grid creates significant new security and privacy
considerations.  The canonical security concern is the possibility of
unauthorized agents infiltrating the microgrid network and becoming capable
of injecting control signals into the microgrid leading to power and/or
equipment failure.  The canonical privacy concern is the possibility of
unauthorized agents infiltrating the microgrid network and using its data
to gain insight into the behaviors of campus members and organizations,
enabling them to better plan and execute robberies or other illegal
activities.

Smart grid security and privacy issues can be organized into categories
corresponding to primary grid componets: the PCS system, smart meters,
power system state estimation, smart grid communication protocol, and smart
grid simulation for security analysis. 

The most common PCS system is SCADA.  Traditional PCS systems were designed
with no outside network connection, so they did not typically have any
security built in.  Adding security to PCS systems is complicated due to
their real-time, low latency (sub-second), and high availability
requirements \cite{Valdes2009}. 

Smart meters provide fine-grained, near-real time information about power
use within a building or other component of the campus.  Security issues
include tampering with the smart meter readings to effectively ``steal''
power. Privacy issues include using the fine-grained data to infer the
behaviors of occupants. Berthier, Sanders, and Khurana have proposed a
comprehensive set of security tools for smart meters \cite{Berthier2010}.

Maintaining the integrity of the grid requires power system state
estimation, and yields a security risk of attackers injecting false data
into the model to create system instability or for financial gain
\cite{Xie2010}.  Allocating the processing overhead necessary to
distinguish false from real data is problematic due to the
high-availability and low-latency (sub-second) requirements for this
component. Some research has been done on how many compromised sources are
required to carry out an unobservable attack \cite{Kosut2010}.

Network communication and their associated protocols is the backbone of the
smart grid, and the security and privacy issues are diverse and dependent
upon the nature of the protocols and the types of information that are
being communicated. A particularly difficult issue is the need to interface
with legacy systems which were not designed with support for security.
Khurana et. al. have proposed a set of design guidelines for smart grid
protocols to reduce the number of vulnerabilities \cite{Khurana2010}.

Finally, smart grids cannot be taken down for testing, and so simulation
systems are used for testing instead.  While traditional grid simulation
systems are focused on availability and stability concerns, smart grids
will require these systems to support security and privacy assessment as
well. Kundar et al have begun work on a framework that provides initial
progress toward smart grid security analysis through simulation
\cite{Kundur2010}.

A contribution of this research project will be the evaluation of these
techniques in the context of microgrid design and implementation and
insights into enhanced privacy and security based upon our experiences.

\paragraph{Social and policy issues.}

To investigate social and policy issues, we plan to build upon the Kukui
Cup project experiences at the University of Hawaii. The Kukui Cup is an
advanced energy challenge that involves real-time feedback, incentives,
education, and game mechanics to investigate how to obtain sustained,
positive changes in energy behaviors. \cite{csdl2-11-03,csdl2-11-02}.

{\em More to come.}

\paragraph{Economic issues.}

The final area of investigation for this research component is the economic
implications of the microgrid.  Our principal focus in this area will be to
determine how effectively the microgrid can serve to decrease the overall
cost of energy supplied by the local utility.  Currently, the University
electrical rates for a given month are primarily a function of two
variables: the peak demand by the University during that month, and the
peak rate of increase in demand (ramp) during the month \cite{Hafner2011}.

If the microgrid design is effective, then we should be able to lower peak
demand in the following ways: (1) by integrating solar generation (which reduces demand
from the utility); (2) though automated demand response (which, when
combined with adequate prediction, should enable the system to shut down
chillers in advance of periods of high load, reducing peak requirements),
and (3) through customer-facing user interfaces, which could inform campus
members of periods of high loads and enable those with discretionary power
loads to shift them in time to periods of lighter overall demand. 

We will also be able to evaluate the ability of our prediction algorithms
to reduce the rate of increase in demand. 

By the conclusion of the project, we will be able to produce comprehensive
cost-benefit analyses of the investment required to create the sustainable,
smart microgrid and the economic benefits that accrued from its
implementation and use.  We will provide an accounting for the savings from
generation, automated demand-response, and user behavioral change.




  


