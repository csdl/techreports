%%%%%%%%%%%%%%%%%%%%%%%%%%%%%% -*- Mode: Latex -*- %%%%%%%%%%%%%%%%%%%%%%%%%%%%
%% project.modeling.tex --
%% Author          : Philip Johnson
%% Created On      : Fri Jan 13 07:58:21 2012
%% Last Modified By: Philip Johnson
%% Last Modified On: Wed Jan 18 08:10:08 2012
%%%%%%%%%%%%%%%%%%%%%%%%%%%%%%%%%%%%%%%%%%%%%%%%%%%%%%%%%%%%%%%%%%%%%%%%%%%%%%%

\subsubsection{Modeling and analysis}
\label{sec:modeling}

\footnote{Given appropriate data, the next step is to apply analytic techniques
  to create real-time and historical information useful for control and
  optimization of the microgrid.

  Two important contributions of this part of the
  research will be: (1) analytic techniques that enable us to adequately
  characterize the current state of the microgrid without a
  cost-prohibitive deployment of sensing equipment, and (2) analytic
  techniques that enable short-term prediction of various useful attributes
  of the micro-grid (such as future (potentially peak) load and ramp) and
  the surrounding environment (insolation, wind speed and direction, etc.)

  It should be noted that there is an interdependence between the ``sensing
  and monitoring'' subproject and the ``modeling and analysis'' subproject:
  we will ``tune'' the installation of sensing equipment in order to obtain
  acceptable quality of analytic outcomes for the next step, control and
  optimization. Furthermore, the chosen models and analytical tools cannot
  only be a good descriptor of the underlying physical processes, but also
  need to be matched to the signal processing (detection and estimation)
  methods, or else the signal processing methods will not be of much use.}

\paragraph{Belief Propagation:} Through or previously funded NSF project (ECCS-???),
we have already established that the factor-graph ({\sc perhaps
insert a figure here if there is space}) approach is well-suited to
modeling the electrical connections of a microgrid. We demonstrated
that, with only a few well placed sensors in the grid, we are able
to track slowly evolving behaviors of the grid as well as with more
elaborate exact maximum-likelihood estimators. This is mainly
because the electrical grid is typically a {\em tree} graph, where
correlations among renewable sources create rare loops with large
diameters. \\
\indent In this research, we plan to use belief propagation as a
tool that would also enable 1) tracking the grid in transitional
modes (e.g., if a large number of users and/or renewable
microgenerators enter and/or drop from the microgrid abruptly) and
2) short-term future prediction of the grid behavior at node-levels.
To this end, we will develop the necessary modeling and prediction
tools.

\paragraph{Modeling demand:} The standard model for demand in grid
networks is using historical {\em histograms}~\cite{several}.
However, historical histogram patterns do not reveal the true
spatiotemporal character of the demand. It is obvious that spatial
and temporal correlations will need to be exploited in order to
arrive at a model that would be useful for spatiotemporal demand
prediction, i.e., prediction that can predict the demand, say, 5
minutes in advance at any given set of spatial location locations in
the microgrid. Furthermore, in order to utilize the
belief-propagation tool, we will seek to model the demand among
various nodes in the microgrid as a spatiotemporal Markov process
with as few loops as possible (because it is the loops that
adversely affect belief propagation~\cite{a,b,c}). We note that
simple Gauss-Markov process modeling will likely not suffice because
Gaussian processes are well-suited only for tracking slow changes.
Yet, demand in campus environment can be spatially and temporally
bursty (because of correlation to class-times tarts, lab-clusters,
weather/heat conditions and distributed renewable microgenerators).
For this reason, we plan leverage our experience in modeling
Gauss-Markov processes~\cite{my-own} and finite-state
machines~\cite{my-own} to develop a comprehensive double-layer
Gauss-Markov/finite-state spatiotemporal demand model. We will
extrapolate, calibrate and test the model using measurement data
from sensors distributed around the campus.

\paragraph{Modeling renewable sources:} Much like the loads (demand)
are spatially and temporally distributed, so are the renewable
micro-generators. The UH campus already has several clusters of
photovoltaic (PV) generators and two small wind turbines (typically
used for experimental purposes) dispersed around campus. Therefore,
we will need to model their energy outputs using spatiotemporal
Markov processes (with as few loops as possible) that are capable of
modeling both slowly varying effects as well as (spatially and
temporally) localized bursts. Again, we will utilize a layered
Gauss-Markov/finite-state modeling approach to capture both effects.
However, unlike modeling demand, here we will need to heavily
correlate the model to weather conditions (night/day, cloud-cover,
wind speed and direction). We will extrapolate, calibrate and test
the model using weather measurement data from irradiance sensors,
wind velocity sensors and cloud cameras distributed around the
campus.


\paragraph{Prediction:} It is a well-known fact that the Kalman
predictor is the optimal predictor of spatial and temporal Gaussian
processes~\cite{???}. When the Gaussian process further possesses a
Markov structure (i.e., a Gauss-Markov process), the Kalman
predictor is computationally less intensive. In our case, we will
likely have a two-layer spatial and temporal
Gauss-Markov/finite-state process to track. The optimal tracker of
this process would be a spatiotemporal combination of a Kalman
predictor~\cite{} and a forward recursion of the Baum-Welch
algorithm~\cite{}. However, this approach will likely be too
computationally intensive to be practical. Therefore, we will ignore
the loops in the model, and implement the predictor using belief
propagation derived for a tree-like double-layer
(Gauss-Markov/finite-state) graph, but applied to the exact loopy
microgrid graph. \\
\indent We envision that the developed prediction tool will be used
to indicate where and how the control mechanisms (e.g.,
peak-shaving, gray-outs, etc.) should be applied. However, we must
consider the sensitivity of certain locations when applying control.
For example, it would be detrimental to shut off the air conditioner
in a sensitive climate controlled laboratory, or in computer server
clusters that must remain cool. For this reason, it is desirable to
have a prediction method that will account for these sensitivities.
It is well-known that any prediction method suffers form
uncertainly, misdetection and false alarms. Grossly false estimation
or prediction of the state of certain sensitive nodes may adversely
affect the stability of the network or sensitive campus locations.
This can be minimized in two possible ways. Firstly, we may choose
to place sensors in the vicinities of highly sensitive nodes, and
thus minimize the risk of misprediction ({\em Note: this approach should
have been discussed in the section on Sensing and Monitoring}).
Secondly, we may choose to dampen the belief-propagation algorithm
to prevent wild belief swings in the vicinities of highly sensitive
nodes. This amounts to placing weights (or other types of
constraints) on beliefs corresponding to highly sensitive nodes. We
shall term this approach {\em sensitivity weighted}
belief-propagation. To our knowledge, no such approach has been
attempted in smart-grid networks, although similar strategies are
available in the communications and error-control coding
literature~\cite{}.

\paragraph{Analysis:} The final piece of the modeling and analysis
effort is the actual analysis of the proposed models and tracking
methods. In tracking mode, we will utilize statistical signal
processing techniques to analyze the performance of the trackers.
For example, under under stationarity conditions, we can easily
bound the performance of maximum-likelihood estimators of Gausian
processes. We will extend these techniques, and combine them with
bounding techniques typical for finite-state processes~\cite{}, to
develop bounds for tracking the layered Gauss-Markov/finite-state
process in under certain stationarity assumptions.\\
\indent Under extremely non-stationary conditions, however, it will
be extremely difficult to come up with analytical tools that will be
able to bound the performance of the estimator/tracker. Indeed, in
the signal processing literature, response to extreme
non-stationarities are typically demonstrated by
simulations~\cite{signal_processing}. For this reason, it is
actually very important to design a good model that will not only be
useful for designing the detector, but also for designing the
simulator. The usefulness of this approach will be extremely
important when determining the response to {\em rare events}. Rare
events are events that are rarely (or never observed) in practice
until they happen in reality (e.g., black-outs, shut-downs,
catastrophic failures). The only way to understand how such events
would affect the system is through simulations. We will conduct
analysis studies of these rare, extremely non-stationary events
through the means of simulations using the developed process models.
