%%%%%%%%%%%%%%%%%%%%%%%%%%%%%% -*- Mode: Latex -*- %%%%%%%%%%%%%%%%%%%%%%%%%%%%
%% project.modeling.tex -- 
%% Author          : Philip Johnson
%% Created On      : Fri Jan 13 07:58:21 2012
%% Last Modified By: Philip Johnson
%% Last Modified On: Fri Jan 13 07:59:56 2012
%%%%%%%%%%%%%%%%%%%%%%%%%%%%%%%%%%%%%%%%%%%%%%%%%%%%%%%%%%%%%%%%%%%%%%%%%%%%%%%

\subsubsection{Modeling and analysis}

{\em Given appropriate data, the next step is to apply analytic techniques
  to create real-time and historical information useful for control and
  optimization of the microgrid. 

  Two important contributions of this part of the 
  research will be: (1) analytic techniques that enable us to adequately
  characterize the current state of the microgrid without a
  cost-prohibitive deployment of sensing equipment, and (2) analytic
  techniques that enable short-term prediction of various useful attributes
  of the micro-grid (such as future (potentially peak) load and ramp) and
  the surrounding environment (insolation, wind speed and direction, etc.)

  It should be noted that there is an interdependence between the ``sensing
  and monitoring'' subproject and the ``modeling and analysis'' subproject:
  we will ``tune'' the installation of sensing equipment in order to obtain
  acceptable quality of analytic outcomes for the next step, control and
  optimization.  }


