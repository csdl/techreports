%%%%%%%%%%%%%%%%%%%%%%%%%%%%%% -*- Mode: Latex -*- %%%%%%%%%%%%%%%%%%%%%%%%%%%%
%% project.conclusions.tex -- 
%% Author          : Philip Johnson
%% Created On      : Fri Jan 13 19:47:12 2012
%% Last Modified By: Philip Johnson
%% Last Modified On: Thu Jan 26 15:45:12 2012
%%%%%%%%%%%%%%%%%%%%%%%%%%%%%%%%%%%%%%%%%%%%%%%%%%%%%%%%%%%%%%%%%%%%%%%%%%%%%%%

\subsection{Conclusions}

In this proposal, we present a vision of a sustainable energy pathway that
begins with the design, implementation, and evaluation of a smart,
sustainable microgrid for the University of Hawaii at Manoa campus.  Our
vision is that the scientific findings, technological innovation, and
social and policy insights gained from this work will contribute to the
ongoing body of work that facilitates the development of microgrids across
the Hawaiian Islands and across the mainland in future years.

The contributions of this research will include the following: (1)
Development of an integrated data collection and management system for both
environmental and energy data; (2) Procedures to determine appropriate
placement of monitoring equipment within a microgrid in order to best
support modeling and control; (3) Analytic techniques for characterizing
the current state of the microgrid without a cost-prohibitive deployment of
sensing equipment; (4) Analytic techniques that enable short-term
prediction of various useful attributes of the microgrid; (5) Automated
techniques for peak shaving/shifting, ramp-rate management, and lowered
overall consumption through control of time-shiftable loads; (6) User
interfaces to enable active participation in energy conservation and
management; (7) Development and evaluation of enhanced security and privacy
mechanisms; (8) Analysis of the cost-benefits of microgrids involving
distributed, intermittent generation; (9) Curriculum and workforce
development for microgrids; and (10) Integration of members of
under-represented minorities.




