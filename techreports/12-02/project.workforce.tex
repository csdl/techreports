%%%%%%%%%%%%%%%%%%%%%%%%%%%%%% -*- Mode: Latex -*- %%%%%%%%%%%%%%%%%%%%%%%%%%%%
%% project.workforce.tex -- 
%% Author          : Philip Johnson
%% Created On      : Fri Jan 13 07:58:21 2012
%% Last Modified By: Philip Johnson
%% Last Modified On: Wed Jan 25 16:09:17 2012
%%%%%%%%%%%%%%%%%%%%%%%%%%%%%%%%%%%%%%%%%%%%%%%%%%%%%%%%%%%%%%%%%%%%%%%%%%%%%%%

\subsubsection{Education and workforce development}
  
The Hawaii Clean Energy Initiative (HECI) is a Memorandum of 
Understanding (MOU) between the state of  Hawaii and
the Department of Energy signed in 2008 that set goals for Hawaii so that by 2030
70\% of our energy will come from clean energy sources
(30\% from energy efficiency and 40\% from renewable energy sources).   
Through  the Renewable Energy and Island Sustainability (REIS) goup  we have developed 
curriculum and courses in energy and sustainability to help train this workforce that will be
needed to achieve these goals.  Much as HCEI will rely more on local energy sources the
REIS group is working to have Hawaii more reliant on locally trained experts in energy and
sustainability.
  
This proposal will work on further development of courses in the smart grid and renewable energy
areas with a focus on systems, software, and policy issues.   We will create graduate level courses in
the smart grid area and also help develop courses on energy in social sciences.  In addition we
are working through local funding agencies to develop a short twenty hour course on smart grids
and integration of renewable sources that will be available to UHM students and faculty and also
the external community.  This course will be broken up into five four hour segments: grid
overview,  policies and standards, tools and capabilities, communications and networking and
security, and integration of sources.

A major component of education and training is conducting research while using the Smart
Campus Energy Lab (SCEL) and working on the UHM micro-grid test-bed.  Both graduate and
undergraduate students will be working on research projects in conjunction with HECO engineers
and UHM facility people.  There will be a close integration between the different research
areas and also between education in the classroom where concepts are learned, analysis where
models, algorithms, optimization, and control methods are formulated,  software and hardware
simulation studies, and test-beds where  analysis and simulations are confirmed.

\paragraph{Integration of members of under-represented minorities}

This project will collaborate with the Native Hawaiian Science and
Engineering Mentorship Program (NHSEMP) to broaden the participation of
under-represented groups. NHSEMP is a successful program housed at the
University of Hawaii funded (in part) by the National Science Foundation
Louis Stokes Alliance for Minority Participation Program and the
U.S.~Department of Education Native Hawaiian Education Program. The program
is already very successful in attracting members of the Pacific Islander
minority into the undergraduate engineering program. Our next goal is to
achieve a smooth transition of the most talented minority students into the
best graduate programs nationwide. For example, we are already implementing
an REU exchange program with our collaborators (at MIT) in conjunction with
a joint project between MIT and University of Hawaii [NSF Grants
ECCS-0725555 and ECCS-0725649]. Throughout 2008 and 2009, MIT hosted
Hawaiian minority undergraduates from Professor Kavcic's group. Their
video-documented experiences can be viewed at {\bf
  http://www2.hawaii.edu/\verb+~+thanhvu/Videos.html}. We are also
presently in the process of selecting qualified minority undergraduates to
send to Pittsburgh, PA to participate in our joint project with Carnegie
Mellon University [NSF Grants ECCS- ECCS-1029081]. We will continue to draw
representatives of underrepresented into our research program in
conjunction with this project. UH minority undergraduates will spend their
semesters working as researchers with at University of Hawaii and their
summer/winter breaks as interns/researchers at HECO or visiting our
research partners on the mainland.
