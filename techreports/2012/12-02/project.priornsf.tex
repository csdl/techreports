%%%%%%%%%%%%%%%%%%%%%%%%%%%%%% -*- Mode: Latex -*- %%%%%%%%%%%%%%%%%%%%%%%%%%%%
%% project.priornsf.tex --
%% Author          : Philip Johnson
%% Created On      : Fri Jan 13 19:47:12 2012
%% Last Modified By: Philip Johnson
%% Last Modified On: Thu Jan 26 13:06:34 2012
%%%%%%%%%%%%%%%%%%%%%%%%%%%%%%%%%%%%%%%%%%%%%%%%%%%%%%%%%%%%%%%%%%%%%%%%%%%%%%%

\subsection{Results from prior NSF research}

The PIs for this project have had the following NSF-supported projects during the past five years:

\begin{enumerate}
\item P. Johnson, {\em Human centered information integration for the Smart Grid}, NSF
  Grant IIS-1017126, 8/15/10 - 8/14/13, \$381,467. The objective of this
  research is to design information technology and associated experimental
  methods to help understand what information, provided in what ways and at
  what times, enables consumers to make positive, sustained changes to
  their energy consumption behaviors. Selected publications include
  \cite{csdl2-10-05,csdl2-10-07,csdl2-11-02,csdl2-11-03,csdl2-11-07}.

\item P. Johnson, {\em Supporting development of highly dependable software through
    continuous, automated, in-process, and individualized software
    measurement validation}, NSF Grant CCF02-34568, 9/01/02 - 8/31/07,
  \$638,000.  The objective of this research was to design, implement, and
  validate software measures within a development infrastructure that
  supports the development of highly dependable software systems.  Selected
  publications for this project include
  \cite{csdl2-04-22,csdl2-04-13,csdl2-04-11,csdl2-03-12,csdl2-02-07,csdl2-03-07,csdl2-04-02,csdl2-04-04,csdl2-04-11,csdl2-06-07,csdl2-06-08,csdl2-06-13,csdl2-06-06,csdl2-09-01}.

\item A.~Kavcic, {\em Energy-Efficient Communication with Optimized ECC Decoders:
    Connecting Algorithms and Implementations}, NSF Grant ECCS-0725649, 09/01/07 -
  08/31/10, \$120,000. This was a joint project with MIT. The PI was
  responsible for reshaping Reed-Solomon soft decoding algorithms for VLSI
  implementation. Selected
  publications for this project include~\cite{Bellorado10a,Bellorado10b,Lim08,Lim08a,Lim10,Lim10a}.

\item A.~Kavcic {\em Channels with Memory -- Universal-Compression-Based
    Modeling Principles for Computing and Optimizing Information Rates}
  NSF Grant CCF-1018984, 08/01/10 - 07/31/13, \$462,000. The aim of
  the project is to utilize compression-based modeling principles
  to compute information rates of long-memory channels whose memory
  is so long that conventional channel modeling principles fail. The
  following publications are available to
  date~\cite{Lim11,Yuan11,JS1,JS2,JS3,JS4}.

\item A.~Kavcic {\em Collaborative Research: Factor-Graph Approach to
    Monitoring and Failure Assessment in Smart-Grid Networks}
  NSF Grant ECCS-1029081, 10/01/10 - 09/30/13, \$225,000.
  This is a joint project between the University of Hawaii and
  Carnegie Mellon University. University of Hawaii is responsible
  for developing the factor-graph framework for belief propagation
  monitoring, while Carnegie Mellon is responsible for devising
  microgrid control algorithms. Publications to date
  include~\cite{Hu10,Hu11,Hu11a}.

\item A.~Kavcic {\em Collaborative Research: Cross-Layer and Unified
    Signal Processing System Design for
    Ultra-High-Capacity Next-Generation Magnetic Storage}
  NSF Grant ECCS-1128705, 09/01/11 - 08/31/14, \$165,000.
  The aim of the project is to formulate a unified signal processing
  and coding framework that will integrate the magnetic and solid
  state memories into a single device. The project
  started only very recently and there are no results to report
  to date.

\item A. Kuh, {\em Incremental and Distributed Learning in Nonstationary
    Environments with Applications to Wind Forecasting}, NSF Grant ECCS-098344,
  9/01/09 - 8/31/11, \$150,251.  The objective of this research is to
  design novel nonlinear kernel online and distributed learning algorithms
  for applications including wind forecasting.   Research was also conducted
  to model the microgrid using a factor-graph framework.   This
  was done in collaboration with Prof. Kavcic and NSF Grant ECCS-1029081. 
  Selected publications for
  this project include \cite{kuhetal-10icgc,kowahl-kuh-10ijcnn,Hu10,Hu11,Hu11a}.

\item A. Kuh, {\em US-Japan Joint Seminar Information Theory}, NSF Grant 0508025
  \$35,750.  Funds used to support graduate students for conferences and
  for visit to Japan.
\end{enumerate}

Profs. Johnson, Kavcic,  Kuh, and Dr. Fripp also had NSF support dating back more than 5 years.
