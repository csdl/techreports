%%%%%%%%%%%%%%%%%%%%%%%%%%%%%% -*- Mode: Latex -*- %%%%%%%%%%%%%%%%%%%%%%%%%%%%
%% 12-17.tex --  Tech report on RAs
%% Author          : Philip Johnson
%% Created On      : Mon Sep 23 11:52:28 2002
%% Last Modified By: Philip Johnson
%% Last Modified On: Mon Jun 14 12:41:23 2010
%%%%%%%%%%%%%%%%%%%%%%%%%%%%%%%%%%%%%%%%%%%%%%%%%%%%%%%%%%%%%%%%%%%%%%%%%%%%%%%
%%   Copyright (C) 2009 Philip Johnson
%%%%%%%%%%%%%%%%%%%%%%%%%%%%%%%%%%%%%%%%%%%%%%%%%%%%%%%%%%%%%%%%%%%%%%%%%%%%%%%
%% 

\documentclass[]{article}
\usepackage[final]{graphicx}
\usepackage{cite}
\usepackage{url}
\usepackage{enumitem}
\usepackage{times}
\usepackage[margin=1in]{geometry}

% uncomment the % away on next line to produce the final camera-ready version
% and uncomment the \thispagestyle{empty} following \maketitle
%\pagestyle{empty}
\begin{document}

%\onecolumn
%\setlength{\parindent}{0cm}


\title{{\bf Looking under the lamppost for useful software analytics}}

\author{Philip M. Johnson\\
        Collaborative Software Development Laboratory\\
        Department of Information and Computer Sciences\\
        University of Hawai`i at M\=anoa\\
        Honolulu, HI 96822\\
        johnson@hawaii.edu\\
}


\maketitle

\begin{abstract}  % 150 words
Abstract goes here.
\end{abstract}

\thispagestyle{empty}


\setlength{\parskip}{3pt plus 1pt minus 1pt} 

\section{Introduction}
Noam Chomsky once said, ``Science is a bit like the joke about the drunk who is looking
under a lamppost for a key that he has lost on the other side of the street, because
that's where the light is. It has no other choice.''  For over 15 years, members of the
Collaborative Software Development Laboratory at the University of Hawaii have searched
for software analytics to understand and improve the process of software development, and
we believe Chomsky's lamppost provides a useful organizational principle for understanding our
efforts.

When it comes to software analytics regarding software development, in our experience
``looking under the lamppost'' is equivalent to ``collecting and analyzing metrics that
are easy to obtain with little social, political, or developmental impact.''  In other
words, the easier and less controversial a software analytic for software development, the
more constrained its application and utility.  For example, the data contained in a
configuration management repository is easy to collect and the intrinsically public nature
of the repository means that few developers will object to such collection and analysis,
but such analytics inevitably provide a limited perspective on development.  On the other
hand, the original version of the Personal Software Process yields extremely rich and
impactful analytics, but entails significant overhead on developers and the analytics
themselves have significant social and political implications.

The remainder of this article discusses our work from 1995 to the present on the Personal
Software Process, Project LEAP, and three projects developed using the Hackystat
Framework: Software Project Telemetry, Zorro, and the Software Intensive Care Unit.  Put
together, these five projects provide interesting case studies of the various trade-offs
that can be made between light and insight when looking for useful software analytics
about development.






\end{document}
