\begin{abstract}

Sustainability education and conservation has become an international imperative due to the rising cost of energy, increasing scarcity of natural resource and irresponsible environmental practices. Over the past decade, running energy and water challenges is a focal point for sustainability efforts at university and industry campuses. Designers of those challenges
have had three choices for information technology: (a) build their own custom in-house solution;
(b) out-source to a commercial provider; or (c) use a minimal tech solution such as a web page
and manual posting of data and results.

My research seeks to investigate the development and evaluation of information technology infrastructure that can  effectively and efficiently support the development of serious games for sustainability. We developed a serious game engine for sustainability called Makahiki, which  provides an open source, customizable, extensible IT infrastructure for creating serious games for the purpose of education and behavioral change regarding energy, water, food, and waste generation and use. Three organizations, namely, University of Hawaii at Manoa, Hawaii Pacific University and East West Center of Hawaii had successfully used the Makahiki to create their own sustainability challenges in 2012.

 In order to evaluate the effectiveness and efficiency of an IT infrastructure for serious games for sustainability, I proposed an evaluation framework that address the perspectives from different stakeholders, namely players, game designers, game managers, system admins, developers and researchers. I will apply the proposed evaluation framework to Makahiki, as well as another IT infrastructure, Lucid Dashboard. This proposal describes the experimental design of such evaluations.

The anticipated constributions of my research includes: Makahiki as an example IT infrastructure for serious games for sustainability, an evaluation framework, evidences of the effectiveness and efficiency of Makahiki and Lucid Dashboard, and the insight into the strenghths and weakness of the evaulation framework.
\end{abstract}