\chapter{Evaluation Framework}
\label{cha:framework-description}
This chapter describes the evaluation framework that can be used to access the effectiveness and efficiency of an IT infrastructure for the development of serious games for sustainability. It starts with the discussion of evaluation methodology, and is followed by the proposed evaluation framework.

\section{Evaluation Methodology}
This framework emlpoys a mixed method of case studies, with qualitative and quantitative data analysis. The qualitative analysis includes a set of interviews that will be administrated to the users of the system to gain insights about their experiences of their interaction of the system. The quantitative analysis mainly involves using the analytics data recorded by the system, such as website logs, player interaction logs,  feedbacks, resource usage, etc.

\section{Definition of effectiveness and efficiency}

Before describing the approach of the evaluation framework for IT infrastructure for serious games for sustainability, here we define the terms of effectiveness and efficiency in the context of serious games for sustainability.

\begin{itemize}
    \item \emph{Effectiveness:} We define the effectiveness of an IT infrastructure for serious games for sustainability is that the IT infrastructure can produce the desired changes in player behavior, which reduces resource consumption

    \item \emph{Efficiency:} We define the efficiency of an IT infrastructure for serious games for sustainability is that the IT infrastructure can support design,  management, administration, development, and improvement for serious games for sustainability
\end{itemize}

\section{Evaluation Mechanism}

A serious game for sustainability normally includes real-world activities and components, such as going out for a educational excursion about sustainability, installing smart meters to measure energy consumption, giving out prizes to the winners of the game, etc. There are more stakeholders in a serious game for sustainability than other kinds of serious games or entertainment games. We indentifies the stakeholders for a serious game for sustainability are the followings:

\begin{itemize}
\item \emph{Players}: the users who participate in the game play.
\item \emph{Game Designers}: the admin user(s) who design the content and game mechanics
 \item \emph{Game Managers}: the admin user(s) who manage the game during the period of the game, such as approving submission, inputing manual energy data, notifying prize winners, etc.
\item \emph{System Admins}: the IT person who installs and maintains the game system
\item \emph{Developers}: the person(s) who extend, enhance and debug the system.
\item \emph{Researchers}: the person(s) who doing research with the system.
\item \emph{Spectators}: persons who do not participate in the game play but know about and interested in the game.
\item \emph{Community partners}: persons or organizations who partner with the game organziers to help the real-world events of the game.
\item \emph{Facilities}: persons or organizations who are responsible for facilitating the energy and water meter installation and data collection.
\item \emph{Funding organizations}: the organizations who provide funding to the project.
\end{itemize}

The success of a serious game for sustainability depends on all the stakeholders. Due to our interest only in the evaluation of an IT software infrastucture, we will exclude the evaluation of spectator, community partner, facilities, and funding organizations. They are important stakeholders that contribute to the success of the serious game, but does not constribute the experience of evalauting the effectiveness and efficiency of the IT infrastrucutre.

The evaluation framework I propose is to evaluate the following stakeholders' experience with the system, and to determine the extent to which the system is effective and efficient with respect to the role of these stakeholders:
\begin{itemize}
\item Players
\item Game Designers
 \item Game Managers
\item System Admins
\item Developers
\item Researchers
\end{itemize}

The following sections describe in details the evaluation mechanism for each role in question:

\subsection{Player effectiveness}
In order to access the effectiveness of a serious game for sustainability, we will evaluate how the system affect the players regarding their sustainability education and resource consumption behavior.

There are four research questions to be investigated for the evaluation of player effectiveness:
\begin{enumerate}
\item To what extent does the system increase player's literacy in sustainability?
\item To what extent does the system produce positive player behavior change in sustainability?
\item To what extent does the system engage players?
\end{enumerate}

\subsubsection{Literacy}
One important goal of the serious game for sustainability is the education effect on the players. Literacy assessment is an indicator of such effect if there is any.

Robert Brewer's dissertation describes the pre and post survey approach to access the participation's energy literacy during a energy challenge. This framework will use the similar approach to access the player's sustainability literacy. A set of literacy survey questionnairs (pre-game) will be presented to the players at the beginning of or before the game. After the game ends, the same survey (post-game) will then be presented to the players who responded the pre-game survey. These two set of survey response data will be compared to understand if there is any changes.

The extent of player's sustainability literacy change will indicate the degree of educational effectiveness of the serious game for sustainability.

\subsubsection{Behavior}

Positive behavior change is another main goal of a serious game for sustainabiliy. A serious game for sustainability normally include some degree of resource consumption measurement. This framework will use resource consumption data before and after the game as part of the assessment for the result of the player's sustainability behavior change.  The resource consumption baseline prior to the game will be established based on the history data. During and after the game, we can compare the resource consumption with the baseline for a particular day to understand to what extent the resource consumption has changed.

We realized (citation) that discusses in details the problems with using the baseline to assess the energy reduction in the cases of dormitory energy challenge. As a framework for evaluating the effectivenss of serious game for sustainability in a broader context beyonds the dormintory challenge, we still recommend the baseline method as one way to assess the resource consumption reduction.

Besides using resource consumption change as one of the indicators of the player's behavior change, we also recommends to administrate bahavior survey to the players, to understand the change (if there is any). A pre-game survey could be presented to the players to ask about their current sustainability behavior, then after the game, a post-game survey to ask about the player's behavior again. These two set of survey response data will be compared to understand if there is any changes.

The combination of resource consumption changes and self-reported behavior changes, will be used to understand the degree of behavior effectiveness of the serious game for sustainability.

\subsubsection{Engagement}

Player's engagement is an important assessment to understand the effectiveness of a serious game. By investigating the degree of engagement, we understand that the players are actively participating in the game thus any changes in the player's literacy and and behavior, are related to the participation in the game, although we can not theoretically prove that the participation cause the changes. On the other hand, if there is no or little participation, we could safely deduce that if there is any changes in sustainability literacy and behavor, they are mostly caused by something else, not the serious game in question.

A serious game should include detailed log data for the players' interaction with the game.
These are the engagement metrics I propose to measure the player engagement for a serious games for sustainability:

\begin{itemize}
\item active participation rate
\item number of players per day
\item average session time
\item submissions per day
\item level of social engagement
\item website errors
\end{itemize}

\subsection{Game designer efficiency}
The research question to be investigated for the evaluation of game designer efficiency is:
\begin{itemize}
\item How efficient is it to design a game using the system?
\end{itemize}

In order to investigate how efficient it is to design a game, we will look at how much time it takes to design the game, and how many errors the designers encoutered during the design process.
The IT infrastracture for a serious game normally provide certain tools or interfaces for the designers to design the game. This may involve configuring gloabl settings for the game, such as how long will the game run, who are the players, how to design individual game elements.

This framework proposes to first identify the list of design tasks, then look at two set of data to assess the game designer's efficiency. One set of data is the admin log data for the interaction between the game designer and the IT infrastructure interface. From these log data, we could derise the time it took a designer to complete a certain design task using the interface, and any system error he encountered.

Another set of data could be obtained by interviewing the designers to answer:
\begin{itemize}
    \item How much time did you spend to complete each design task?
    \item What problem did you encountered?
    \item Did you find it difficult to configure? what is difficult?
    \item Did you find it difficult to design a specific game? which one, what is difficult?
    \item What did you like the least when using the system?
\end{itemize}

\subsection{Game manager efficiency}
The research question to be investigated for the evaluation of game manager efficiency is:
\begin{itemize}
\item How efficient is it to manage the game using the system?
\end{itemize}

In order to investigate how efficient it is to manage a game, we will look at how much time it takes to manage the game, and how many errors the game managers encoutered during the process.
The IT infrastracture for a serious game normally provide certain interfaces for the designers to manage the game. This may involve managing player submissions, monitoring the game state, entering manual resource data, notifying winners of the game, etc.

This framework proposes to first identify the list of managing tasks, then look at two set of data to assess the game manager's efficiency. One set of data is the admin log data for the interaction between the game manager and the IT infrastructure interface. From these log data, we could derise the time it took a manager to complete a certain managing task using the interface, and any system error he encountered.

Another set of data could be obtained by interviewing the managers to answer:
\begin{itemize}
\item How much time did you spend to complete each managing task?
\item What problem did you encountered?
\item Did you find it difficult to manage? what is difficult?
\item What did you like the least when using the system?
\end{itemize}

\subsection{System admin efficiency}
The research question to be investigated for the evaluation of system admin efficiency is:
\begin{itemize}
\item How efficient is it to install and maintain the system?
\end{itemize}

In order to investigate how efficient it is to install and maintain the system, we will look at how much time it takes to install and maintain the system, and how many errors encoutered during the process.

This framework proposes to interview the system admin to answer:
\begin{itemize}
\item How much time did you spend to install the system?
\item How much time did you spend to maintain the system?
\item What problem did you encountered?
\item Did you find it difficult to admin the system? what is difficult?
\item What did you like the least about administrating the system?
\end{itemize}

\subsection{Developer efficiency}
The research question to be investigated for the evaluation of developer efficiency is:
\begin{itemize}
\item How efficient is it to understand, extend and debug the system?
\end{itemize}

In order to investigate how efficient it is to understand, extend and debug the system, we will look at how much time it takes to install and maintain the system, and how many errors encoutered during the process.

This framework proposes to interview the developer to answer:
\begin{itemize}
\item How much time did you spend to set up the dev env?
\item How much time did you spend to develop and debug an enhancement?
\item What problem did you encountered?
\item Did you find it difficult to understand, extend and debug the system? what is difficult?
\item What did you like the least when developing with the system?
\end{itemize}

\subsection{Researcher efficiency}
The research question to be investigated for the evaluation of researcher efficiency is:
\begin{itemize}
\item How efficient is it to do research with the system?
\end{itemize}

In order to investigate how efficient it is to do research with the system, we will look at how much time it takes to use the system for specific research query, and how many errors encoutered during the process.

This framework proposes to interview the researcher to answer:
\begin{itemize}
\item How much time did you spend to collect the research data for a specific topic?
\item What problem did you encountered when collecting the data?
\item Did you find the data you collect helpful to your research? if not, what can be improved?
\item Did you find it difficult to collect the data from the system? what is difficult?
\item What did you like the least about using the system?
\end{itemize}
