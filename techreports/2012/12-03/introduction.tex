\chapter{Introduction}

\section{Sustainability Education and Behavior Change}
The rising cost, increasing scarcity, and environmental impact of fossil fuels
as an energy source makes a transition to cleaner, renewable energy sources an
international imperative.  One barrier to this transition is the relatively
inexpensive cost of current energy, which makes financial incentives less
effective. Another barrier is the success that electrical utilities have had in
making energy ubiquitous, reliable, and easy to access, thus enabling
widespread ignorance in the general population about basic energy principles
and trade-offs.  In Hawaii, the need for transition is especially acute, as the
state leads the US both in the price of energy (over \$0.30/kWh) and
reliance on fossil fuels as an energy source (over 90\% from oil and coal).

Moving away from petroleum is a technological, political, and social paradigm
shift, requiring citizens to think differently about energy policies, methods
of generation, and their own consumption than they have in the past.
Unfortunately, unlike other civic and community issues, energy has been almost
completely absent from the educational system. To give a sense for this
invisibility, public schools in the United States generally teach about the
structure and importance of our political system (via classes like ``social
studies''), nutrition and health (through ``health''), and even sports (through
``physical education'').  But there is no tradition of teaching ``energy'' as a
core subject area for an educated citizenry, even though energy appears to be
one of the most important emergent issues of the 21st century.

On the other hand, changing people's behavior with respect to energy holds significant promise in reducing energy use. Darby's survey of energy consumption research found that identical homes could differ in energy use by a factor of two or more \cite{darby-review-2006}. Data from a military housing community on Oahu show energy usage for similar homes can differ by a factor of 4 \cite{Norton2010ZeroEnergyHomes}.

\section{Collegiate dormitory sustainability competition}

Over the past decade, running energy and water challenges have become a focal point for sustain-
ability efforts at university and industry campuses, to facilitate and incentivize energy and water reduction. Designers of those competitions
have had three choices for information technology: (a) build their own custom in-house solution;
(b) out-source to a commercial provider; or (c) use a ”minimal tech” solution such as a web page
and manual posting of data and results.

Petersen et al. describe their experiences deploying a real-time feedback
system in an Oberlin College dorm energy competition in 2005 that includes 22
dormitories over a 2-week period \cite{petersen-dorm-energy-reduction}. Web
pages were used to provide feedback to students. They found a 32\% reduction in
electricity use across all dormitories. However, in a post-competition survey,
respondents indicated that some behaviors, such as turning off hallway lights
at night and unplugging vending machines were not sustainable outside the
competition period.  Overall, there has been little analysis on energy usage
after competitions finish, or how positive behavior changes could be sustained.

The Building Dashboard \cite{building-dashboard}, developed by Lucid Design
Group, is used to support Oberlin's dorm energy competition,
as well as the Campus Conservation Nationals, a nationwide electricity and
water use reduction competition on college campuses \cite{competetoreduce}.
The Building Dashboard enables viewing, comparing and sharing building energy
and water use information on the web in compelling visual interface, but the
cost of the system creates the barrier for wider adoptions. In addition, the
building dashboard solutions focus on providing energy information as
a passive media. There is little interaction between participants and the system.

\section{Serious games and Gamification}
Another emergent issue is the explosive spread of game techniques, not only in
its traditional form of entertainment, but across the entire cultural spectrum.
Games have been shown with great potential as successful
interactive media that provide engaging interfaces in various serious contexts
\cite{mcgonigal2011reality,reeves2009total}. Priebatsch attempts to build a
game layer on top of the world with his location-based service startup
\cite{Priebatsch2010ted}. The adoption of game techniques to non-traditional areas such as finance,
sales, and education has become such a phenomenon that the Gartner Group
included ``gamification'' \cite{Deterding2011mt} on its 2011 Hype List.

Reeves et al. described the design of Power House, an energy game that connects
home smart meters to an online multiple player game with the goal to improve
home energy behavior \cite{Reeves2011powerhouse}. In the game, the real world
energy data are transformed into a ``more palatable and relevant form of
feedback'', and players may be incentivized by the in-game rewards to complete
more energy-friendly real-world behaviors.

ROI Research and Recyclebank launched the Green Your Home Challenge as a case
study of employing gamification techniques online to
encourage residential green behavioral changes offline \cite{gamingforgood}.
Working with Google Analytics, the results show a 71\% increase in unique
visitors and 97\% of participants surveyed said that the challenge increased
their knowledge about how to help the environment.

\section{Serious game assessment}

One fundamental question in evaluating a serious game or a gamified application is the 
extent to which the game or application achieves its ``serious'' purpose. This is quite different from traditional entertainment games. There is an increasing
focus on the evaluation methodology in the field of serious games ~\cite{Mayer2012233}~\cite{harteveld2010triadic}. These approaches focus on evaluation of a single game, as opposed to a game {\em
  framework}. One of the benefits of 
using a game framework is that, if correctly designed, it will provide useful and
reusable ``building blocks'' with which to develop a variety of serious games. Yet how are we to know if a serious
game framework has been ``correctly designed''?

There exists some assessment tools such as GEQ (Game Engagement Questionnaire)\cite{brockmyer2009development}, QUIS (Questionnaire for User Interaction Satisfaction)\cite{harper1993improving}. We found no prior work concerning comprehensive assessment for 
the particular needs of a serious game framework.

\section{Research Description}

The overall research question that will be investigated is:
What forms of information technology infrastructure can support effective and efficient development of serious games for sustainability?

In order to address this research question, I started with two development tasks:
\begin{itemize}
    \item Develop example IT infrastructure for development of serious games for sustainability.

    \item Develop an assessment method that provides evidence of the strengths and weaknesses of the IT infrastructure for the development of serious games for sustainability.
\end{itemize}

\subsection{Makahiki}

We developed a innovated serious game framework for sustainability called Makahiki, as an example IT infrastructure for the development of sustainability challenges. Makahiki explores one section of the design space
where virtual world game mechanics are employed to affect real world sustainability
behaviors.  The ultimate goal of the Makahiki project is to learn to
not just affect behaviors during the course of the game, but to
produce long lasting, sustained change in behaviors and
outlooks by participants.

 Makahiki has a unique feature set intended
to foster more rapid innovation and development. These features include: (1) an open source li-
cense and development model which makes the technology available without charge and facilitates
collaborative development and improvement; (2) support for an ”ecosystem” of extensible, interre-
lated, customizable games and activities; (3) real-time game analytics for research
and evaluation; (4) pedagogically organized and extensible learning activities; (5) a responsive user
interface supporting mobile, tablet, and laptop displays; and (6) support for deployment to the
cloud as an inexpensive option for hosting the competition.

The Makahiki framework had been successfully used in 2012 by three organizations, namely, University of
Hawaii at Manoa, Hawaii Pacific University, EastWest Center of University of Hawaii, to implement
individually tailored sustainability challenges focusing on energy and water conservation.

\subsection{SGSEAM}
In order to assess the effectiveness and efficiency of IT infrastructure for serious games for sustainability, I proposed an assessment method called Serious Game Stakeholder Experience Assessment Method (SGSEAM). In a nutshell, SGSEAM (pronounced "sig-seam")
identifies the most important stakeholders of a
serious game framework and provides a method for gaining insight into the strengths
and shortcomings of the framework with respect to each stakeholders' needs. We consider
SGSEAM as an assessment method instead of an evaluation method. The main purpose of an
evaluation is to "determine the quality of a program by formulating a judgement"
\cite{hurteau2009legitimate}. An assessment, on the other hand, is nonjudgmental. SGSEAM does
not try to judge a framework according to a standard, instead, it is used to identify the major
strengths and shortcomings of a framework so that the community could benefit from the
assessment by learning from the strengths and improving the shortcomings.

\subsection{Evaluation}
I will apply the proposed SGSEAM to Makahiki, as well as another serious game framework, Lucid BuildingOS\cite{building-dashboard}, to gather evidences of the effectiveness and efficiency of Makahiki and BuildingOS, and to gain insight into the strenghths and weakness of SGSEAM.

\section{Outline}

The proposal is organized into the following chapters:

\begin{itemize}
	\item \autoref{cha:related-work} looks at related research, including serious game, gamification, serious game framework, and framework assessment.
	\item \autoref{cha:system-description} describes the design and implemention of the Makahiki system.
    \item \autoref{cha:framework-description} describes the evaluation framework for the serious games IT infrastructure for sustainability.
	\item \autoref{cha:ExperimentalDesign} lists our research questions and explains our plan to evaluate them.
	\item \autoref{cha:conclusion} concludes the proposal with a list of anticipated contributions and future directions.
	\item \autoref{app:qualitative-feedback} contains the questionnaire to be administered to various roles in the evaluation.
    \item \autoref{app:googleform} contains the google forms to be used in the in-lab evaluation experiments.
\end{itemize}
