% 11-07.tex
% FDG 2012 conference paper
% WORKS WITH V3.2SP OF ACM_PROC_ARTICLE-SP.CLS
% November 2011
% Author: Yongwen Xu, George Lee, Philip Johnson
%
% ----------------------------------------------------------------------------------------------------------------
% This .tex file (and associated .cls V3.2SP) *DOES NOT* produce:
%       1) The Permission Statement
%       2) The Conference (location) Info information
%       3) The Copyright Line with ACM data
%       4) Page numbering
% ---------------------------------------------------------------------------------------------------------------
% It is an example which *does* use the .bib file (from which the .bbl file
% is produced).
% REMEMBER HOWEVER: After having produced the .bbl file,
% and prior to final submission,
% you need to 'insert'  your .bbl file into your source .tex file so as to provide
% ONE 'self-contained' source file.
%

\documentclass{acm_proc_article-sp}

\begin{document}

\title{Makahiki: An Open Source Game Engine for Energy Education and Conservation}

%\numberofauthors{1} 

\author{
\smallskip
George E. Lee\\ 
\smallskip
Yongwen Xu\\ 
\smallskip
Robert S. Brewer\\ 
\smallskip
Philip M. Johnson\\
       \affaddr{Information and Computer Sciences}\\
       \affaddr{University of Hawai`i at M\=anoa}\\
       \affaddr{Honolulu, HI 96822}\\
       \email{[gelee, rbrewer, yxu, johnson]@hawaii.edu}
}

\maketitle
\begin{abstract}
  The rising cost, increasing scarcity, and climate impact of fossil
  fuels as an energy source makes a transition to cleaner, renewable energy
  sources an international imperative.  This paper presents Makahiki, an
  open source game engine for energy education and conservation.  Makahiki
  facilitates the implementation of ``serious games'' that motivate players
  to learn about energy issues, improve their intuition about the energy
  impact of appliances and behaviors, and enable them to discover how to
  use energy more efficiently in their normal life.  Makahiki has been
  used to implement ``The Quest for the Kukui Cup'', a three week energy
  challenge for over 1,000 first year students living in residence halls at
  the University of Hawaii in Fall, 2011.   Evaluation of this initial
  deployment of Makahiki has revealed useful insights into its game
  mechanics, ways to improve the next Kukui Cup challenge, and the
  challenges when adapting it to other energy contexts.
\end{abstract}

% A category with the (minimum) three required fields
\category{L.5.1}{Game-based Learning}{Gaming}

\terms{Human Factors, Games, Education, Motivation}

\keywords{Serious Games, Education, Gamification}% NOT required for Proceedings

\section{Introduction}

The rising cost, increasing scarcity, and climate impact of fossil fuels as
an energy source makes a transition to cleaner, renewable energy sources an
international imperative.  One barrier to this transition is the relatively
inexpensive cost of current energy, making financial incentives less
effective.  Another barrier is the success that electrical utilities have
had in making energy ubiquitous, reliable, and easy to access, thus
enabling widespread ignorance in the general population about basic energy
principles and trade-offs.  In Hawaii, the need for transition is
especially acute, as the state leads the nation both in the price of energy
(over \$0.30/kWh) and reliance on fossil fuels as an energy source (over
90\% from oil and coal).

Moving away from petroleum is a technological, political, and social
paradigm shift, requiring citizens to think differently about energy
policies, methods of generation, and their own consumption than they have
in the past.  Unfortunately, unlike other civic and community issues,
energy has been almost completely absent from the educational system. To
give a sense for this invisibility, public schools in the United States
generally teach about the structure and importance of our political system
(via classes like ``social studies''), monetary issues (though ``(home)
economics''), nutrition and health (through ``health''), and even sports
(through ``physical education'').  But there is no tradition of teaching
``energy'' as a core subject area for an educated citizenry, even though energy
appears to be one of the emergent issues of the 21st century.

Another emergent issue, at least for the first part of the 21st
century, is the explosive spread of game techniques, not only in its
traditional form of entertainment, but across the entire cultural spectrum.
The adoption of game techniques to non-traditional areas such as finance,
sales, and education has become such a phenomenon that the Gartner Group
included ``gamification'' on its 2011 Hype List, positioning it near the
summit of ``inflated expectations'' (after which it is projected to fall
into the ``trough of disillusionment''). 

This paper describes Makahiki, an open source game engine for energy
conservation and education, in which we attempt to create synergy between
these two emergent issues.  The result of over two years of research and
iterative development, Makahiki explores one section of the design space
where virtual world game mechanics are employed to affect real world energy
behaviors.  The ultimate goal of the Makahiki project is to learn how to
not just affect energy behaviors during the course of the game, but to
produce more long lasting, sustained change in energy behaviors and
outlooks by participants. 

We used Makahiki to create an energy challenge called The Quest for the
Kukui Cup for approximately 1,000 first year students living in the
residence halls at the University of Hawaii in Fall, 2011.  During the
three weeks of the competition, over 400 of the eligible students played
the game, for a total of 850 game play hours.  In addition to online play,
the Quest for the Kukui Cup integrated 24 real world events, including
workshops on energy-related matters, excursions to wind farms and other
energy related locations, and energy-related activities on campus. The game
mechanics were designed to create a self-reinforcing ``virtuous circle''
between the real world and virtual world activities.  The challenge was
very successfully received and plans are already underway to both repeat
the challenge in 2012 for University of Hawaii first year students, and 
adapt it to other residence halls and other universities in Hawaii.

The next section briefly reviews the research on which we based Makahiki
and the Quest for the Kukui Cup.   We then discuss the architecture and
game mechanics of the system, followed by the lessons we learned from its
first deployment. We conclude with our current goals for improvements to
the system.

\section{Related Work (Yongwen)}
Our research draws on work we've done previously
\cite{csdl2-10-05,csdl2-10-07,csdl2-11-02,csdl2-11-03}, as well as from
work done by others in the areas of energy behavior research, energy competitions, gamification and serious games. 

To reduce energy consumption, providing energy feedback is a  critical foundation. Darby's survey of 20 energy consumption studies from the past 3 decades found  that, consumption in identical homes could differ in energy use by a factor of two or more depending on the behavior of the inhabitants\cite{5}. Another survey of energy feedback conducted by Faruqui et al. found that residents that actively used the in-home displays with near-realtime feedback, averaged a 7\% reduction in energy usage\cite{6}. Darby also points out that feedback alone is not always enough, other factors such as training and social infrastructure could lead to higher rates of energy conservation\cite{7}.

Energy competitions or challenges have been introduced to college dormitories and residential homes as a way to incentivize energy reduction. Pertersen et al. describe their experiences deploying a real-time feedback system in an Oberlin College dorm energy competition in 2005 \cite{3} that includes 22 dormitories over a 2 week period. Web pages were used to provide feedback to students. They found a 32\% reduction in electricity use across all dormitories. However, in a post-competition survey, respondents indicated that some behaviors, such as turning off hallway lights at night and unplugging vending machines were not sustainable outside the competition period.  Overall, there has been little analysis on energy usage after competitions finish, or how positive behavior changes could be sustained.

Lucid Design Group's commercial system, The Building Dashboard\cite{}, is used to support Oberlin's dorm energy competition, as well as the Campus Conservation Nationals, a nationwide electricity and water use reduction competition on college campuses \cite{}. The Building Dashboard enables viewing, comparing and sharing building energy and water use information on the web in compelling visual interface, but the cost of the system creates the barrier for wider adoptions.

Besides energy competition, game design elements, or gamification techniques, has been employed in various areas of environmental behavior changes. 

ROI Research and Recyclebank launched the Green Your Home Challenge as a case study of employing gamification techniques online to encourage residential green behavioral changes offline\cite{}. Working with Google Analytics, the results show a 71\% increase in unique visitors and 97\% of participants surveyed say that the challenge increased their knowledge about how to help the environment. 

Blending of real and virtual worlds has been explored in a broader context. Mcgonigal designed the award winning Serious Alternative Reality Game (ARG) called "World Without Oil"\cite{} and later the "Evoke"\cite{} game with the goal to help empower people to come up with creative solutions to our most urgent social problems. 

Gamification has become 
ARG

\section{System Design (George)}

\subsection{Architecture}

The website is implemented using the Django web framework and the Python programming language. In addition to the modules we created, we used other third party libraries including ``django\_cas'' (an authentication plugin that allows users to log in through a CAS server), ``brabeion'' (a Django module for badges), and ``restclient'' (a library for interacting with RESTful web services). The website is hosted on a Xserve running Mac OS X Snow Leopard and Apache. To handle server loads more efficiently, we used memcached (an in-memory caching system) to cache parts of the website for quicker access.

We access another system called WattDepot for energy information. However, calculating energy (power over a period of time) can be computationally expensive if we have to calculate it every time a user accesses a page. We implemented a ``cloud cache'' that uses a Google Spreadsheet to store precomputed energy values for the different floors in the competition. The user's browser accesses the public spreadsheet using client-side javascript and the Google GData library. The server may access the cloud cache or WattDepot directly. For example, to award points in the energy goal game, a script is executed on the server once a day to check the spreadsheet and see if the individual lounges made their goal.

\subsection{Game Design}
The game mechanics is ...

\subsubsection{Smart Grid Game}
\subsubsection{Daily Energy Goal Game}
\subsubsection{Raffle Game}
\subsubsection{Social and Referral Bonuses}
\subsubsection{Confirmation Code}
\subsubsection{Quest Engine}
\subsubsection{Canopy}



\section{Lessons Learned}

Our initial deployment of Makahiki for the 2011 Kukui Cup yielded many
useful lessons for next year's Kukui Cup as well as for others wishing to
implement serious games for energy. 

These lessons derive from both qualitative and quantitative sources of data
regarding the system.  First, as noted above, Makahiki provides custom quantitative
instrumentation that enables us to track when, where, and for how long each
user accessed each page of the site.  Unlike generic logging infrastructure
like Apache Web Stats, we could track application-specific behaviors. For
example, our instrumentation enables us to determine how users allocated
and deallocated tickets to the Raffle Game, or whether they watched,
paused, or skipped over the video portion of a Smart Grid Game activity.

Second, we gathered quantitative data through a survey that players could
complete as part of a Smart Grid Game activity during the final week of the
competition. The survey asked participants to provide short, free text
answers to questions regarding the way the competition and website was
designed.   41 players completed this survey.

These two forms of data enabled us to gain insight from active players of
the game.  However, they do not provide insight into the reasons why
certain students decided not to play the game.  We contacted a random
sample of students who chose not to play the game for interviews in order
to obtain some partial understanding of this portion of our population.

\subsection{Lesson \#1: Focus groups and usability evaluations improve player
  experience.}

Somewhat unintentionally, our experience with Makahiki has provided an
example of the benefits of focus groups and usability evalutions in game
design, as well as the costs of not doing so.  We spent almost an entire
year on an iterative design process for the main portions of the site which
included extensive end-user involvement.  We performed an interface
usability study with paper prototypes, two rounds of individual user
evaluation of the onboarding experience, and a beta test with a small
number of ``friends and family'' users prior to deployment.

Both qualitative and quantitative data indicate that the portions of the
game resulting from this process were successful.  In response to a survey
question asking how the player might describe the Kukui Cup, 83\% said
``Fun'', 95\% said ``Educational'', while 7\% said ``Difficult'' and 2.3\%
said ``Boring''.  In response to the question, ``What was confusing in the
website'', 46\% of the players said ``Nothing'', and 32\% of the users also
responded ``Nothing'' in response to the question, ``What would you change
about the website? When asked what they liked most about the website, 60\%
of the survey respondents said ``ease of use''.

Instrumentation also indicates that the game was generally easy to
use. 73\% of the 418 players never accessed the ``Help'' page, and only 5\%
of the users sent a question to the administrators. 

On the other hand, we designed and implemented the ``Canopy'' level of the
game in a very short period prior to the deployment, with no time available
for user evaluation and iterative design.  The results for the Canopy
level, both qualitative and quantitative, provide a stark contrast to the
main portions of the site.  We provided access to the canopy to the top
10\% of the players during the third round, and out of these 41 players,
only 11 spent more than 10 minutes in the Canopy, and only 5 spent more
than 30 minutes. These same users spent an average of 15 hours playing the
overall game.  Qualitative feedback indicated that we made several game
design errors in the Canopy that led to reduced player motivation and
interest in this part of the site.  We believe that these problems could
have been found and avoided if we had conducted user evaluations for this
part of the site prior to deployment.  As it is, we will make these
corrections for the 2012 Kukui Cup.

Although focus groups and usability evaluation tend to be viewed as
``motherhood and apple pie'' in software engineering, the Mahahiki game
development experience provides significant evidence for the improved user
experience that results from their use, as well as the costs that result
when schedule pressure forces deployment without them.

\subsection{Lesson \#2: Serious games require serious marketing.}

The Kukui Cup achieved a relatively high adoption rate: out of the 1035
residents who were eligible, almost 40\% played the game.  This is an
impressive adoption rate, both from the perspective of residential life
administrators (who rarely succeed in designing a residence hall activity
that gets nearly half of the students to engage with it) and from the
perspective of university energy challenges (where adoption rates of
10-15\% are more common.)  

On the other hand, our qualitative and quantitative data indicates that
there is substantial room for improvement, and that the single biggest
improvement opportunity identified by the players is better marketing.  In
response to the survey question asking how to improve attendence at real
world events, 60\% of the students suggested improved marketing and/or email
announcements as the best way to improve attendance.  Only 5\% indicated
that the actual events, as opposed to the way they were advertised,
constituted the major problem.

Our attempts at marketing the inaugural Kukui Cup at times resembled a
comedy of errors. For example, to kick off the challenge, our graphic
design intern designed a series of large (2 foot by 5 foot) banners to be
hung in the lobby of each of the four towers as the challenge progressed.
These banners were attractive---so attractive that the first set of banners
for all four towers were stolen in the middle of the night within the first
three days after we hung them up.  After this happened, we requested that
the second set of banners for the four towers be displayed only when staff
was on duty to monitor them.  This request resulted in only three of the
second set of four banners being stolen: the fourth being spared because
the on-duty staff never remembered to display it.  For next year, we hope
to create display materials that are ``cool, but not too cool''.

On a more positive note, game mechanics like the ``viral marketing''
referral bonus appeared to work well, with over XX students trying out the
site as a result of encouragement from other students.  


\section{Future Directions}

We are already planning the 2012 Kukui Cup, and our efforts are encouraged
by the fact that 98\% of the students in our survey said they would play
the Kukui Cup again next year if they could. 

We will, of course, focus more energy on marketing, and our goal is to
increase participation from 418 players to at least 600, which would
achieve the highest adoption rate of any dorm energy challenge of this
type.  We intend to harness the energy of this year's top players to
support and encourage participation from next year's first year students.

We remain convinced that the Canopy page, with its emphasis on advanced
energy visualizations and analyses, has the potential to provide a rich educational
opportunity to motivated students 



\section{Acknowledgments}

Makahiki has been supported in part by grant IIS-1017126 from the National
Science Foundation, and by funding from the University of Hawaii Office of Facilities
Management.   We gratefully acknowledge the 418 players of the 2012 Kukui
Cup and the members of the Kukui Cup team in addition to the authors who made the vision a
reality:  Kaveh Abhari, Hana Bowers, Greg Burgess, Caterina Desiato,
Michelle Katchuck, Risa Khamsi, Alex Young, and Chris Zorn. 

\bibliographystyle{abbrv}
\bibliography{csdl-trs,gamification}  

\balancecolumns

\end{document}
