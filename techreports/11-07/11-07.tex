% 11-07.tex
% FDG 2012 conference paper
% WORKS WITH V3.2SP OF ACM_PROC_ARTICLE-SP.CLS
% November 2011
% Author: Yongwen Xu, George Lee, Philip Johnson
%
% ----------------------------------------------------------------------------------------------------------------
% This .tex file (and associated .cls V3.2SP) *DOES NOT* produce:
%       1) The Permission Statement
%       2) The Conference (location) Info information
%       3) The Copyright Line with ACM data
%       4) Page numbering
% ---------------------------------------------------------------------------------------------------------------
% It is an example which *does* use the .bib file (from which the .bbl file
% is produced).
% REMEMBER HOWEVER: After having produced the .bbl file,
% and prior to final submission,
% you need to 'insert'  your .bbl file into your source .tex file so as to provide
% ONE 'self-contained' source file.
%

\documentclass{acm_proc_article-sp}

\begin{document}

\title{Makahiki: An Open Source Game Engine for Energy Education and Conservation}

%\numberofauthors{1} 

\author{
George E. Lee\\
Robert S. Brewer\\
Yongwen Xu\\
Philip M. Johnson\\
       \affaddr{Information and Computer Sciences}\\
       \affaddr{University of Hawai`i at M\=anoa}\\
       \affaddr{Honolulu, HI 96822}\\
       \email{[gelee, rbrewer, yxu, johnson]@hawaii.edu}
}

\maketitle
\begin{abstract}
  The rising cost, increasing scarcity, and climate impact of using fossil
  fuels as an energy source makes a transition to cleaner, renewable energy
  sources an international imperative.  This paper presents Makahiki, an
  open source game engine for energy education and conservation.  Makahiki
  facilitates the implementation of ``serious games'' that motivate players
  to learn about energy issues, improve their intuition about the energy
  impact of appliances and behaviors, and enable them to discover how to
  use energy more efficiently in their normal life.  Makahiki has been
  used to implement ``The Quest for the Kukui Cup'', a three week energy
  challenge for over 1,000 first year students living in residence halls at
  the University of Hawaii in Fall, 2011.   Evaluation of this initial
  deployment of Makahiki has revealed useful insights into its game
  mechanics, ways to improve the next Kukui Cup challenge, and the
  challenges when adapting it to other energy contexts.
\end{abstract}

% A category with the (minimum) three required fields
\category{L.5.1}{Game-based Learning}{Gaming}

\terms{Human Factors, Games, Education, Motivation}

\keywords{Serious Games, Education, Gamification}% NOT required for Proceedings

\section{Introduction (Philip)}

The rising cost, increasing scarcity, and climate impact of using fossil
fuels as an energy source makes a transition to cleaner, renewable energy
sources an international imperative.  One barrier to this transition is the
relatively inexpensive cost of current energy, making financial incentives
less effective.   Another barrier is the success that electrical utilities
have had in making energy ubiquitous, reliable, and easy to access, thus
enabling widespread ignorance in the general population about basic energy
principles and trade-offs.  


\section{Related Work (Yongwen)}
Our research draws on work from multiple areas \cite{csdl2-10-05,csdl2-10-07,csdl2-11-02,csdl2-11-03}.

\section{System Design (George)}

\subsection{Requirements}
The requirement is ....

\subsection{Architecture}
The architecture is ...

\subsection{Game Mechanics}
The game mechanics is ...

\section{Evaluation (?)}
In Kukuicup.....

\section{Conclusions and Future Directions (Philip)}
From the above .....

\section{Acknowledgments}
Makahiki has been supported in part by grant IIs-1017126 from the National
Science Foundation, and by funding from the University of Hawaii Office of Facilities
Management. 
%
% The following two commands are all you need in the
% initial runs of your .tex file to produce the bibliography for the citations in your paper.
\bibliographystyle{abbrv}
\bibliography{csdl-trs,11-07}  
% You must have a proper ".bib" file and remember to run:
% latex bibtex latex latex
% to resolve all references
%
% ACM needs 'a single self-contained file'!
%
\balancecolumns

\end{document}
