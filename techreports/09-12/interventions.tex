\documentclass[11pt]{article}

%%% Load some useful packages:
%% "New" LaTeX2e graphics support.
\usepackage{graphicx}
%%	using final option to force graphics to be included even in draft mode
%\usepackage[final]{graphicx}
%% Tell graphicx the default directory for all figures
\graphicspath{{figures/final/}}

%% Enable subfigure support
\usepackage{subfigure}

%% Make subsubsections numbered and included in ToC
\setcounter{secnumdepth}{3}
\setcounter{tocdepth}{3}

%% Package to linebreak URLs in a sane manner.
\usepackage{url}

%% Define a new 'smallurl' style for the package that will use a smaller font.
\makeatletter
\def\url@smallurlstyle{%
  \@ifundefined{selectfont}{\def\UrlFont{\sf}}{\def\UrlFont{\small\ttfamily}}}
\makeatother
%% Now actually use the newly defined style.
\urlstyle{smallurl}

%% Define 'tinyurl' style for even smaller URLs (such as in tables)
\makeatletter
\def\url@tinyurlstyle{%
  \@ifundefined{selectfont}{\def\UrlFont{\sf}}{\def\UrlFont{\scriptsize\ttfamily}}}
\makeatother

%% Provides additional functionality for tabular environments
\usepackage{array}

%% Puts space after macros, unless followed by punctuation
\usepackage{xspace}

%%% Personal macros
%% Tired of typing CO2 so many times, requires xspace package
\newcommand{\COtwo}{CO\ensuremath{_2}\xspace}

%% Make margins less ridiculous
\usepackage{fullpage}

%% Allows insertion of fixme notes for future work
\usepackage[footnote, nomargin]{fixme}

%%%% Turned off for tech report, should be turned on for research portfolio
%% Turn on double spacing
%\usepackage{setspace}
%\doublespacing

%% Make URLs clickable
%\usepackage[colorlinks, bookmarks=false]{hyperref}
\usepackage[colorlinks, bookmarks=true]{hyperref}

%% Since I'm using the LaTeX Makefile that uses dvips, I need this
%% package to make URLs break nicely
\usepackage{breakurl}

%%% End of preamble
\begin{document}

\title{Proposal for Electricity Conservation Experiments in Saunders Hall}
\author{Robert S. Brewer \\
Collaborative Software Development Lab \\
Department of Information and Computer Sciences \\
University of Hawai`i \\
Honolulu, HI \\
rbrewer@hawaii.edu \\
\\
CSDL Technical Report 09-12 \\
\url{http://csdl.ics.hawaii.edu/techreports/09-12/09-12.pdf}
}
\date{April 2009\\[3pt]
Copyright \copyright\ Robert S. Brewer 2009}

%%% Create the title page from all the information above. Note that the
%%% titlepage is outside the front matter.
\maketitle

%% Philip suggests it needs a ToC
%\tableofcontents

\begin{abstract}
Insert abstract here.
\end{abstract}

\section{Introduction}

The University of Hawai`i at M\=anoa has set the goals of reducing its electricity usage by 30\% by 2012 and 50\% by 2015 (based on a 2003 benchmark) \cite{Moreno2006UHM-energy-goals, 2007UHM-HECO-pr}. A variety of tactics will be required to meet these aggressive goals. One promising technique is to encourage the occupants of buildings to reduce their electricity usage.

Electricity usage for Saunders Hall is now instrumented using the Obvius AcquiSuite \cite{ObviusAcquiSuite}. The total building electricity usage is tracked, as well as the individual usage of five floors of Saunders ($2^{nd}$, $3^{rd}$, $4^{th}$, $5^{th}$, and $6^{th}$ floors). The metering has turned Saunders into a living laboratory where experiments can be run on how to encourage occupants to reduce their electricity usage. The per-floor metering is particularly useful because it will allow intra-building comparisons. Different techniques for encouraging electricity conservation can be introduced on different floors, and their relative effectiveness compared. This research seeks to determine the effectiveness of different techniques (an active area of research world-wide), and can guide us towards the best ways to reduce electricity usage on campus.

There are a variety of possible interventions that may encourage occupants to reduce their electricity usage. To assess the relative effectiveness of the interventions, we plan a series of experiments in Saunders. However, the participants of each experiment will be the occupants of Saunders, rather than a set of participants recruited anew for each experiment. We expect two negative consequences of the continuity of the subjects: reduced subject interest/enthusiasm, and diminishing conservation returns.

The envisioned series of experiments will take place over at least one semester, likely multiple semesters. While we anticipate interest in the project from Saunders occupants, after the first couple of rounds, the participants may just ``tune out'' the interventions, or they may become actively annoyed by the intrusion of surveys and the like.

The other expected problem is that there is only so much discretionary electricity usage taking place in Saunders, since we do not expect occupants to make changes that significantly impact their ability to teach, learn, or otherwise do their job (work in complete darkness, permanently unplug all computers, etc). Thus, the early experimental rounds may be sufficiently effective to have virtually eliminated discretionary electricity usage. If there is no remaining ``fat'' to be trimmed, then the effectiveness of interventions in later rounds cannot be accurately assessed.

For these two reasons, the order of the interventions must be chosen carefully to prioritize the ones that are likely to make the biggest research contributions.

We examine the envisioned interventions below, as well as the literature relevant to each option. We conclude with a proposed ordering for the interventions.

\section{Energy usage website}

Many studies have demonstrated that providing real-time feedback on resource consumption to the occupants of a building will lead them to reduce their consumption by 5--15\% \cite{darby-review-2006}.



\section{Public display}

\section{Collaborative website}

\section{Competition}

\subsection{Incentives}

\subsection{Extended competition}

\section{Aggressive communications}


\section{Recommended Order}

\section{Conclusion}


%% Just for demo purposes, include all entries from bib file
%\nocite{*}

%%% Input file for bibliography
\bibliography{sustainability}
%% Use this for an alphabetically organized bibliography
\bibliographystyle{plain}
%% Use this for a reference order organized bibliography
%\bibliographystyle{unsrt}
%% Try using this BibTeX style that hopefully will print annotations in
%% the bibliography. This will allow me to make notes on papers in the
%% BibTeX file and have them readable in the references section until
%% I turn them into a conceptual literature review 
%\bibliographystyle{annotation}

\end{document}
