\begin{figure*}[tbp]
\hrule
%\vskip 0.5em
\centerline{\psfig{figure=sim.wlms.ps,height=3.5in,angle=-90}}
\vskip 0.5em
\caption{{\bf Inter-treatment Differences for Sensitive and Insensitive
Faults. (WLMS)} This figure shows a 3 $\times$ 3 matrix of subplots, with each row
corresponding to a fixed value of $p_a$ (.43, .47, or .51) and each column
to a fixed value of $p_b$ ( .34, .38, and .42).  The cells in the matrix
contain some of the simulation results for each combination of $p_a$ and $p_b$.
Within each cell the y-axis encodes the inter-treatment differences for
several fault populations. Each of these populations has both a meeting sensitive
and meeting insensitive subpopulations and is
defined by two parameters: the proportion of meeting sensitive faults in
the population and the
detection probability for meeting sensitive faults.
The proportion of meeting sensitive faults (0, .2, .4, or .6) is
encoded on the subplot's x-axis. Different
detection probabilities are plotted using differently shaped symbols (
a square for $pa \times 1$,
an octagon for $pa \times 1.25$,
a triangle for $pa \times 1.5$, and
a diamond for $pa \times 1.75$).
The results for the meeting sensitive subpopulation
are plotted with filled symbols and open symbols are used
for the meeting insensitive subpopulation. The line segments
running through each symbol mark one standard deviation in the
detection rate's estimate. Since the detection probabilities of the meeting
sensitive and meeting insensitive faults are constrained by the total
detection probability some combinations are mathematically
impossible. In these cases no symbols are plotted.}
\label{sim.wlms}
\vskip 0.5em
\hrule
\end{figure*}



%\begin{figure*}[tbp]
%\hrule
%%\centerline{\psfig{figure=sim.cruise.ps,height=4in,angle=-90}}
%\vskip 0.5em
%\caption{{\bf Inter-treatment Differences for Sensitive and 
%Insensitive Faults. (CRUISE) } For detailed explanation of the
%plot see Figure \protect{\ref{sim.wlms}}.}
%\label{sim.cruise}
%\hrule
%\end{figure*}
