\documentclass[12pt]{article}
\usepackage{times}
\usepackage{relsize}
\usepackage{microtype}

\usepackage{alltt}
\usepackage{url}
\usepackage{comment}
\usepackage{relsize}

\usepackage{geometry}

\geometry{verbose,papersize={8.5in,11in},
  left={1in},width={6.5in},top={1in},bottom={1in}}

\begin{document}
\title{ Assessment of Provisional Program\\
        {\bf The ICS Ph.D. Program}}


\author{Philip Johnson, Associate Chair \\
        Henri Casanova, Graduate Chair \\
        Department of Information and Computer Sciences \\
        University of Hawai'i,  Honolulu, HI 96822 \\
        johnson@hawaii.edu
}

\maketitle

\tableofcontents

\newpage

In this document we provide the information requested in {\em Appendix D:
Guidelines for Assessment of Provisional and Established Programs} for the
Information and Computer Sciences Ph.D. program.

\section{Assessment of need factors}

Because it provides a useful introduction to the value and importance of the ICS
Ph.D. program, we begin this report by responding to the final requested
item of information in Appendix D.

{\em
\medskip \noindent In the case of graduate programs, attention should also be given to the
following need factors:
\begin{itemize}
\item The direct relevance of the contribution of the field of study to the
  professional, economic, social, occupational and general educational
  needs of Hawaii; 
\item A ``national needs factor'' that emphasizes the direct relevance of
  the contributions of the field of study to national needs and where
  Hawaii and the University have unique or outstanding resources to
  respond with quality;
\item An ``international needs factor'' the emphasizes the direct relevance of
  the contributions of the field of study to international needs and where
  Hawaii and the University have unique or outstanding resources to
  respond with quality;
\item An educational needs factor the indicates the direct relevance of the
  field of study to basic education needs for which there is a demand by
  Hawaii's population;
\item The relevance of a field of study as a necessary supporting
  discipline for quality programs identified by the above criteria;
\end{itemize}
}

Computer Science is a fundamental discipline that generates results and
technology that impact the life of millions of people every day. In 2009, a
panel of 8 experts from the Wharton School of Business (University of
Pennsylvania) was asked to name the 20 biggest innovations of the last 30
years, with the results published in the New York Times in March of that
year. Out of the 20 innovations, 9 are directly from the field of Computer
Science and 5 are directly enabled by it. Given this impressive coverage,
it is not surprising that Computer Science graduate programs are mainstays
of virtually all research universities worldwide.  What makes Computer
Science unique is its cross-cutting impact and relevance for other
disciplines. Indeed, computers are used today not only in virtually all
disciplines of science and engineering (where computer modeling and
simulation are pervasive), but also in all the humanities (e.g., due to the
use of large-scale and distributed digital databases), with direct
involvement in fields as diverse as education (e.g., for internet
collaboration technologies for learning) and even music (e.g., for
computer-aided composition).  Far from being straightforward applications
of computers, many important developments in those fields require that
Computer Science challenges be addressed through innovative research and
development activities, such as that pursued by ICS graduate students.
Consequently, Computer Science research is fundamental for furthering human
knowledge and progress, and thus for society at large.

Since Computer Science's relevance is pervasive across so many
disciplines, our graduate program is an invaluable resource for the
university:
\begin{itemize}
  \item Many ICS graduate students are engaged in 
collaborative projects between professors in ICS and in other departments.
They are thus key contributors to the fostering of interdisciplinary
research at UHM, which is highly strategic given
the amount of federal funding available for such research. 
  \item A significant fraction of our graduate students are supported
by Research Assistantships hosted in other departments. This is because
many research projects require the type of expertise that only our
graduate students have received through the training provided in our
program at the university. We repeatedly receive several requests
from Principal Investigators on campus asking us to advertise 
Research Assistantship opportunities to our graduate students. These students
provide a sizable research workforce for the university.
  \item Our graduate program offers courses that provide advanced training 
for graduate students outside of our programs. Every semester, such students
take our graduate-level courses. For instance, Oceanography and Astronomy students have
take our high-performance computing course, Biology students have taken
our bioinformatics course, Educational Technology students have taken
our Human-Computer Interaction courses.
  \item After they graduate, several of our students obtain positions
at UHM and contribute either to research and development activities or
to information technology management.  This is the case for at least 7
of our M.S.  graduates since 2006, and 2 of out Ph.D. students (out of
the 12 that have graduated so far).
\end{itemize}

Innovations in computing drive economic growth for the state of Hawaii, not just
through the growth of the IT industry, but through productivity growth
across the entire economy.  A strong and sizable Computer Science
graduate program provides a nexus for this growth and the ICS graduate
program provides means to both build Hawai`i's capacity for
technical innovation and to staff Hawaii’s information technology
industry.  In the case of Hawai`i, the benefit goes beyond economic
growth but also encompasses (much needed) economic diversification. \emph{Consequently,
a growing ICS graduate program is a major contributor to a growing diversified
economy.}

The career paths of our graduate program is a clear testimony of the dramatic
impact that our students have on the state's economy. More than 50\%
of our M.S. students stay in Hawai`i and hold software engineering and
development positions in industry and research and development
organizations, such as in the companies hosted at the Manoa Innovation
Center, an incubator for the high-tech sector.  The impact of the ICS
graduate program is also at the level of the community.  For instance, our
students and faculty members are active contributors to TechHui,
Hawai`i's Science, Technology and New Media Community on-line.

Finally, our program also fulfills a clear local educational need. For
instance, for Fall 2009 admissions, out of the 17 applicants to our
Ph.D. program, 4 were local to Hawai`i.  Out of these 4 applicants 3
were admitted to our program. These are outstanding local students who
were exposed to research during their undergraduate experience at UHM
and, although many possibilities were offered to them, they opted for
our Ph.D. program based on their interactions with our faculty.

In summary, the national and international need for computer science
Ph.D. graduates is currently strong and will only grow stronger in future.
Regionally, the diversification of the Hawaii economy requires skilled,
innovative thinking in high technology areas which computer science
Ph.D. graduates are ideally suited to provide.  Finally, the ICS
Ph.D. program provides students who are in high demand and a valued
resource to other departments.   

We believe strongly that the ICS Ph.D. program satisfies all the need
factors required for transition from provisional to permanent status.

\section{Assessment of program organization and objectives}

{\em Following the guidelines in Appendix D, this section discusses the ICS
  Ph.D. program curriculum, requirements, advising, and counseling, with
  the goal of establishing that the ICS Ph.D. program is organized in such
  a way as to meet its objectives. }

The Ph.D. is the highest degree awarded by universities in the United
States and thus represents the pinnacle of academic achievement.  The Ph.D
Program in Information and Computer Sciences is designed for students who
want to contribute to the study of the description and representation of
information and the theory, design, analysis, implementation, and
application of algorithmic processes that transform information.

ICS Ph.D. students receive advanced training in the scientific principles and
technology required to develop and evaluate new computer systems and
applications. We equip our students with the expertise necessary to
independently perform state-of-the-art research, to formulate and develop
creative solutions to novel and existing problems, and to intelligently
manage the research of others. Our curriculum covers all major areas of
computer science, with active research in areas including artificial
intelligence, bioinformatics, human-computer interaction, software
engineering, machine learning, high performance computing, digital
democracy, computer vision, and computer systems.

An applicant may be admitted with a Bachelor's degree or with an
M.S. degree in computer science or a related field. If the applicant enters
without the M.S., the applicant will earn the M.S. before proceeding to the
"Ph.D. portion" of the program.

The ICS Ph.D. curriculum is designed to: (1) Certify the student's core
competency in computer science and address any deficiencies in this
competency as efficently as possible, so that the bulk of the student's
Ph.D. program is focused on research. (2) Prepare the student to do
research through an apprenticeship with a faculty member, assessing
readiness to do research with a research portfolio that is analogous to a
professional tenure and promotion portfolio.  We achieve these goals by
guiding the students through a curriculum with the following components:
(1) Demonstration of core competancy; (2) Participation in ICS 690; (3)
Preparation of a research portfolio; (4) Proposal defense; and (5) Final
defense.

\subsection*{Demonstration of core competency}

The ICS Ph.D. student will demonstrate core competency in computer science
by meeting the following two requirements:

\begin{enumerate}
\item Completion of a Master's degree in Computer Science or a related
  field, where what counts as ``related'' is at the discretion of the
  graduate program chair, assisted by the admissions committee;
\item Successful completion of the comprehensive exam.  The comprehensive exam
  covers core knowledge of computer science at a level that might be
  reasonably expected of a job interviewee with a Master's degree.
  Students shall take the comprehensive exam at the end of the first semester
  of the the Ph.D. portion of their students.  Students may attempt the
  comprehensive exam only twice, and must pass this exam no later than the end
  of the first year of their Ph.D. studies.  
\end{enumerate}

\subsection*{Participation in ICS 690}

According to Graduate Division guidelines, coursework is optional for
University of Hawaii Ph.D. programs.  However, the ICS Ph.D. program
requires all ICS Ph.D. students to attend and pass the seminar course ICS
690 each semester they are in the program.  ICS 690 is a one credit seminar
course that meets once a week and is directed by the Graduate Chair. It
provides an opportunity for all ICS graduate students (both M.S. and Ph.D)
to regularly discuss their research issues and problems and gain insight
from presentations by other faculty members, other graduate students, and
guest lectures by visiting academic and industry professionals.  

\subsection*{Preparation of a research portfolio}

By the end of the year following the passing of the comprehensive exam, the student
must prepare and submit a research portfolio that includes the following:

\begin{enumerate}
\item A statement of purpose, which is a one to two page description of the
  student's professional interests in research, teaching, service, and/or
  product development;
\item Evidence of core competancy, as described above;
\item Evidence of scholarly ability, i.e. the ability to identify,
  critically analyze, and research a problem, and of written communication
  skills, in the form of two items authored by the student and reviewed by
  doctoral level scholars. The first item is a written literature review in
  the proposed area of study of 20-30 pages, following the graduate
  division dissertation format and reviewing at least 20 published
  works. The second item must be one of the following:  a masters thesis by
  the student; a publication by the student in a reviewed conference or
  journal; or a technical report approved by at least two other faculty
  members. 
\item (Optional) Other evidence of professional capacity, which might
  include a professional vita of employment, professional presentations, reviewing of papers for conferences and journals, competitive fellowships, patents, teaching, and service on committees or as graduate student representatives contribute to the candidacy decision. Letters of reference may also be included. Students should report all forms of research, teaching, and service to the community and to the discipline when preparing their portfolios.
\end{enumerate}

The portfolio is approved by a two-thirds majority vote of a quorum of the ICS faculty (typically at a faculty meeting). The portfolio shall be distributed to the faculty in advance of the meeting at which it will be voted upon.

The graduate program chair shall designate one faculty to argue for the student's case and one to argue against the student, who may both vote as they see fit. Faculty that have a conflict of interest with the student (e.g., advisor or co-advisor, co-author on research articles, direct supervisor) cannot serve in these capacities.

The portfolio must be approved before undertaking the Proposal Defense.

\subsection*{Proposal Defense}

Before commencing the final dissertation research, the student shall give a
public defense of his or her Ph.D. proposal. Students prepare a research
proposal that includes a literature review in the chosen topic area (this
usually is but is not required to be derived from the literature review
from the portfolio) and a description of research topics to be
investigated. This work should be done under the direction of an
appropriate faculty adviser.  Students must also form their dissertation
committee prior to the proposal defense.

The defense includes both a presentation of the student's research
proposal and an oral examination covering their general preparation for the
research involved, as specified in the General and Graduate Information
Catalog.

It is generally advised that the proposal defense be scheduled for a time period of 3 hours. 

Once the student passes the proposal defense, they then conduct their
research and write a dissertation under the direction of their advisor and
their dissertation committee.

\subsection*{Final Defense}

The final defense is a public presentation of the student's completed
research and dissertation.  The dissertation must be presented to and
approved by a doctoral committee, as specified in the General and Graduate
Information Catalog.

We believe that our five step process of demonstrating core competancy,
participation in ICS 690, preparation of a research portfolio, proposal
defense, and final defense, when combined with our graduate curriculum and
research areas, creates an effective and efficient program for students who
wish to contribute to the study of the description and representation of
information and the theory, design, analysis, implementation, and
application of algorithmic processes that transform information. Our
program is thus organized in such a way as to meet its objectives.


\section{Assessment of student learning objectives}

{\em Following the guidelines in Appendix D, this section assesses the
  whether or not the program is meeting its learning objectives for
  students. }

We have defined nine student learning objectives for the ICS Ph.D. program,
six of which are shared with our M.S. program plus an additional three
learning objectives specific to the Ph.D. program.

The ICS M.S. graduate program provides courses for advanced education in
Computer Science and affords opportunities to conduct research. Our
objective is to help students achieve a high level of professional
competence and lifelong learning, with the following Student Learning
Objectives:

\newcounter{listcounter}
\begin{list}{\arabic{listcounter}.}{\usecounter{listcounter}}
\item Master core computer science theoretical concepts, practices and technologies;
\item Identify, formulate and solve problems employing knowledge within the discipline;
\item Contribute effectively to collaborative team oriented activities;
\item Communicate effectively about computer science topics using appropriate media;
\item Demonstrate advanced knowledge in an area of specialization within the discipline;
\item Engage in significant research in their area of specialization within the discipline and/
or in projects that respond to community and industry needs.
\end{list}

The ICS Ph.D. graduate program provides advanced, individualized training in research 
in Computer Science, preparing students for research careers in academia and industry.  
Beyond  those for the M.S. program, the Ph.D. program involves the three following 
Student Learning Objectives:

\begin{list}{\arabic{listcounter}.}{\usecounter{listcounter} \setcounter{listcounter}{6}}
\item Develop a research portfolio that demonstrates the capacity to carry out original 
research in the field;
\item Become an expert in the area of specialization including mastery of the relevant 
research skills and methods, develop a research vision, and formulate a research plan 
that will lead to novel scientific contributions;
\item Execute a research plan and demonstrate original contributions to the field, as 
shown through findings and/or publications, culminating in a Ph.D. dissertation and oral 
defense.
\end{list}

Our development of empirically based assessment procedures for these student learning
objectives is ongoing.  For example, we have planned, but not yet
implemented, an ``exit interview'' procedure in which we can gather data
directly from each graduating student regarding their subjective view as to
whether each of these student learning objectives were achieved. We also
plan to classify each course in the curriculum according to the program
SLOs that it covers, which would provide an additional level of evidence
regarding assessment and coverage by noting which courses the student took
during their program. 

Although development of assessment procedures is ongoing, we believe
strongly that the basic structure of our program as described above ensures
that successful graduates have satisfactorily achieved all of these
learning objectives, as illustrated in Table \ref{phd.slos}.

\begin{table}[htbp]
\begin{center}
\caption{Ph.D. program components and satisfaction of student learning
  objectives}
\label{phd.slos}
\begin{tabular}{|l|l|} \hline
{\bf Ph.D. program component} & {\bf Student Learning Objective(s) Addressed}  \\ \hline
Demonstration of core competancy & 1 \\
Participation in ICS 690 & 3, 4,  \\
Preparation of a research portfolio & 2, 4, 5, 7, 8 \\
Proposal Defense & 1, 2, 4, 5, 6, 8 \\
Final Defense & 1, 2, 4, 5, 6, 8, 9 \\ \hline
\end{tabular}
\end{center}
\end{table}

\section{Assessment of program resources}

{\em Following Appendix D, this section addresses whether or not program
  resources are adequate through an analysis of the number and distribution
  of faculty, faculty areas of expertise, budget and sources of funds, and
  facilities and equipment.}

Our program currently counts 42 graduate students (23 M.S. students,
19 Ph.D. students), for 21 Computer Science graduate faculty.  One
constraining factor for program size is the limited (and in some cases
dwindling) numbers of Teaching and Research Assistantships. Such
support is essential for attracting and retaining students.
Currently, all our Ph.D. students are supported and so are several of
our M.S. students. Growth of the program would be facilitated by an
increase in such support, which can probably be said of most graduate
programs at UHM.  Another constraining factor is the
student-to-faculty ratio. Currently this ratio is about to 2 (i.e., on
average a graduate faculty member advises 1 Ph.D. student and 1 M.S.
student), but it was equal to 3 only a couple of years ago. Such
fluctuation is typical and follows fluctuations in applications and
graduations. For instance, many students graduated in 2009, leading to
a somewhat lower ratio in 2010.  At any rate, due to our relatively
low student-to-faculty ratio, our students benefit from intensive
advising, which is instrumental to their successful graduations.  To
ensure that the quality of the advising remains high, a significant
increase in the student population would require an increase of our
faculty as well. While the growth of any program requires resources,
namely student support and faculty members,




\section{Assessment of program efficiency}

{\em Following Appendix D, this section assesses productivity and
  cost/benefit considerations within the overall context of campus and
  University ``mission'' and planning priorities.  It can include
  quantitative measures comparing SSH/faculty, average class size,
  etc. with other programs in the college, campus, or as appropriate other
  universities.}

One useful measure of program efficiency for a Ph.D. program is
time-to-degree (TTD).  While the TTD can be predicted fairly accurately for
students in M.S. or undergraduate programs (depending on whether they are full-time
students or whether they have full-time jobs), the same cannot be said
of the TTD for a Ph.D. program. This is due to the original research
component, whose duration depends both on the student and on the
chosen area of research within Computer Science. Variations among
students in terms of one year or more is thus common. Furthermore,
some Ph.D.  students are admitted in our program right after obtaining
their B.S., while others come into the program with a M.S. in hand,
which shortens their TTD by at least 1 year and typically 1.5 years if
that degree is in Computer Science or a related field.

According to data collected by Graduate Division, the mean TTD in our
Ph.D. program is 5.8 years, with a median of 6.0 years. We can attempt
a comparison with national averages. The report \emph{Time
To Degree of U.S. Research Doctorate Recipients} available from the
National Science Foundation (NSF) Web
site~\footnote{http://www.nsf.gov/statistics/infbrief/nsf06312}
presents data specific to Computer Science programs for academic year
2003. It reports mean ttd between 8.3 and 15.1 years depending on
student categories (Research Assistants, Teaching Assistants,
supported by fellowships, unsupported). The registered-to-degree (RTD)
metric is also reported, which accounts for time during which the
student is actually registered in graduate school, and which ranges
between 7.0 and 9.0 depending on the student category. These times are
``since obtaining a Bachelor.'' We can thus see that our program
compares favorably to nationwide averages, even accounting for the
fact that the Graduation Division data does not account for M.S.
degrees obtained in other institutions.  A recent report on nationwide
doctorate recipients is also available from the NSF Web
site~\footnote{http://www.nsf.gov/statistics/nsf10309}. It presents
data for the 2007-2008 academic year, but unfortunately does not
present data specific to Computer Science programs. Instead is shows
aggregate data for ``Physical Sciences.'' A median TTD of 6.7 years is
reported, which seems to confirm the above observations regarding our program.

The conclusion is that our program allows students to graduate at the
same or at a faster pace than the national average.  While this is good
news, we still see some students who graduate in more than 8 years and up
to 9.5 years. To try to reduce the maximum time to graduation, in 2005
we have redesigned our Ph.D. program.  Like many high-profile programs
nationwide (UC Berkeley, Univ.  of Washington, UC San Diego, etc.) we
have done away with the traditional comprehensive exams.  Instead, in
a view to engaging doctoral students in research as soon as possible,
we have put in place qualifying exams early on and a ``research
portfolio'' exam instead of the comprehensive exams. We thus expect to
maintain our relatively low average TTD but also to reduce our maximum
ttd in the future. Our first graduate for the redesigned program, Mark
Stillwell, successfully defended his dissertation in 2010. He
graduated in 4 years (he already held a M.S. degree in
Mathematics prior to applying to our program), has a very strong
publication record, and has already found a post-doctoral position with
a view to starting a promising Computer Science academic career.


\section{Assessment of program quality}

{\em Following Appendix D, this section assesses the program with respect
  to student performance, satisfaction, placement and employer
  satisfaction, awards to faculty and students, etc.}

\subsection*{Student application trends}

The average GPA of students joining our Ph.D. program over the last 5 years
is a high 3.82. The percentage of applicants that we accept in our program
has ranged between 38\% and 88\%. The last two years have had inordinate
high acceptance rates above 80\%. In fact, our acceptance rates have
increased steadily throughout the years. While this increase could be
attributed to a lowering our our admission standards, this is absolutely
\emph{not} the case. In fact, our faculty have been absolutely amazed at
the rising quality of applicants to our Ph.D. program in the last could of
years, leading to accepting 15 out of 17 applicants in 2009-2010! This
increase in quality is in part imputable to the fact that our Ph.D. program
is recent and is just gaining momentum with our graduates beginning to make
an impact.  The percentage of accepted applicants who eventually join our
program has ranged between 12.5\% and 60\% over the years. Remarkably, in
the last two years, which have seen unprecedented top quality applicants,
50\% and 60\% of these applicants have joined our program. As discussed
below, we do not believe this represents a drop in standards, but rather an increase
in the reputation and stature of our program as it matures.

The data collected by Graduate Division regarding the drop rate for
our Ph.D. program is misleading because it
accounts only for students admitted between Fall 1989 and Spring 1999.
This was when our program was in its infancy and the data is 
for 3 students only.  With the help of a Graduate Division IT
Specialist, on 8/30/2010 we obtained a full history of students in our
Ph.D. program. A total of 30 students have entered our program and not
graduated, for 32 students who have either graduated or are still in
the program. This would seem to indicate a high drop rate close to
50\%. However, out of those 30 students who never graduated, 12 never
enrolled (likely due to personal reasons or late admission to other
programs) and 6 dropped out after only one semester (likely for
similar reasons). Discounting those students, the overall drop rate of
our program is 12/48=27\%, which is basically the UHM average.  Note
that, out of these 12 students who dropped, 4 moved to a different PhD
program at UHM (e.g., CIS), and 3 left after receiving their M.S.
degree ``on the way" to the Ph.D., seizing timely
professional opportunities.  We are left with only 5 students who
entered our program, stayed in it more than one semester, and left
without a degree. One of these students was recently dismissed due to
poor academic performance.  We conclude that most students admitted to
our program are well-suited to it.

\subsection*{Student graduation and career paths}

\begin{table}[Htb]
\caption{Career Paths of Ph.D. Graduates}
\label{tab.phd}
\begin{tabular}{|l|l|l|l|}
\hline
Student & Year & Current Position & Location \\
\hline
Pei-Chia Chang & 2010 & Post-Doctoral Researcher &  \\
Mark Stillwell & 2010 & Post-Doctoral Researcher, ENS de Lyon & France \\
David Nickles & 2010 & Faculty Member at KCC & HI \\
Joshua Wingstrom & 2009 & Created his own startup company & TX\\
Xin Chen & 2009 & Software Designer, CTAHR, UHM & HI\\
Robert Fanelli & 2008 & Assistant Professor, West Point Academy & NY\\
Hongbing Kou & 2008 & Senior Software Engineer, CityGrid Media &  CA\\
Xin Zhao & 2008 & Senior Scientist, Sanjole & HI\\
Nathan Dwyer & 2007 & Senior developer, Sega Studios & CA\\
David Pautler & 2007 & Principal Investigator, Institute of HPC & Singapore\\
Qin Zhang & 2007 & Senior Software Engineer, Kofax Systems & CA\\
Matthew Chapman & 2007 & Assistant Professor, West Point Academy & NY\\
Christoph Aschwanden & 2006 & Project Manager, TRI, UHM & HI\\
Holger Mauch & 2005 & Assistant Professor, Eckerd College & FL\\
\hline
\end{tabular}
\end{table}

The recently established Ph.D. program has 14 graduates to date, as
listed in Table~\ref{tab.phd}.  Out of the 14 graduates, 6 have
obtained faculty or post-doc positions, 3 work in a research or higher education
institution, and the remaining 5 have positions in industry.

The primary mission of the ICS Ph.D. program is to produce graduates
that become leaders in their field once they have achieved at least
one major contributions to at least one of the many areas of Computer
Science research. Our graduates who went to industry all hold senior
software design and development positions, which allow them to be key
leaders in the hi-tech and information technology sector. A
perfect example of such leadership is provided by one of our 2009 graduates
who has recently created his own startup company in Texas. Most of
these graduates partake in so-called research \& development
activities, for which the research training they have acquired in our
program proves invaluable.  This training is also key for the 3
graduates that hold positions in research institutes.  Finally, 25\%
of our graduates to date hold faculty positions in research and higher
education institutions. Even though this number is typical for Ph.D.
programs nationwide, we note that students increasingly
enter our program with the goal of obtaining a faculty position in the
future. For instance, our two Fall 2010 graduates (Pei-Chia Chang and
Mark Stillwell) are moving on to post-doctoral positions as a
transition to a faculty position hopefully within two years of their graduation.

~\\
\noindent{\bf Local Impact of Graduates --} While more than 50\% of
our M.S. graduates stay in Hawai`i, about 33\% of Ph.D. graduates have
done so to date.  These 4 graduates currently contribute the Hawai`i's
economy and higher education: 1 of them is a Senior Scientist for a
local hi-tech company, 2 hold research and development positions
at UHM, and 1 holds a faculty position at a Community College (KCC). 

~\\
\noindent{\bf National Impact of Graduates --} All our graduates have
national impact in that their work and accomplishments further U.S.
economy, research, and/or education.  In general, our Ph.D. students
are all engaged in original research in many fields of Computer
Science. As a result, they publish their results in international
competitive venues, thereby contributing to the nation's (and
Hawai`i's) predominance in the international Computer Science research
arena.  The drive of these graduates to find high-profile positions
that match their research interest often entails moving to a few
specific locations nationwide.  Consequently, a large fraction of our
graduates (7 of our 12) currently hold positions on the U.S. mainland
(CA, FL, NY, TX),

~\\
\noindent{\bf International Impact of Graduates --} Those graduates
that hold positions with a strong research component have an
international impact in the sense that they further the field and the
global technology landscape (through original research publications,
patents, and products). To date, only one of our
graduates has opted for a position abroad, in a research institute in
Singapore. While we expect this number to increase (for instance, one
of our Fall 2010 graduates is about to move to France for a
post-doctoral position), this relatively low number can be simply
attributed to the fact that U.S. organizations offer highly attractive
positions for our graduates.

\section{Assessment of program outcomes}

{\em Following Appendix D, this section analyzes the number of majors,
  graduates, SSHs offered, employment, etc. in relationship to the
  objectives.}




\section{Assessment of program objectives}

{\em Following Appendix D, this section assesses whether the program
  objectives are still appropriate functions of the University mission and
  development plans, and can include evidence for the continuing need of
  the program, projections of employment opportunities for graduates, etc.}


\end{document}


