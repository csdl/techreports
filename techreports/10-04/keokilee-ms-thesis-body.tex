
\chapter{Introduction}
There have been many efforts by organizations to make the public aware of their energy usage.

\section{The Problem}

The idea of university dorm energy competitions is nothing new.  Many universities across the United States have already run competitions.  However, these early competitions have a few flaws.  First, participants have the potential to "game" the system.  Since they are encouraged to reduce their energy usage, participants will go through great lengths to perform their activities outside of their dorm.  Second, the participants have shown that they will go back to their previous habits once the competition is over.  The purpose of a dorm energy competition is to educate the participants about their own energy usage.

\section{The Kukui Cup system}

We propose a dorm energy competition system that not only shows users their energy usage, but also educates users about why they should participate.

\section{Evaluating the Kukui Cup system}

We will create an initial version of the system and present it to focus groups.

\section{Thesis Claims}

The Kukui Cup system is designed to investigate the following questions:
\begin{enumerate}
\item Does providing a competition that includes energy literacy as a component lead to retained knowledge?
\item Will the participants share their energy activities with friends in their social network (e.g. Facebook, Twitter, MySpace)?
\item Will the participants access the system while away from their computers (e.g. through a mobile device)?
\end{enumerate}

\section{Proposal Structure}

The next chapter will discuss related work, which includes other dorm energy competitions and devices that present energy usage to users.  Chapter 3 will describe the components of the Kukui Cup system.  Chapter 4 describes our proposed evaluation procedure.  Finally, chapter 5 includes the conclusion and provides a timeline for the work that is to be done to complete the thesis.

\chapter{Related Work}

\autoref{othercomps} investigates the efforts by other universities to create a dorm energy competition.

\autoref{tracking} investigates "smart meters", which are used to present energy information in a way that is easy to understand for the average person.

\section{Other Dorm Energy Competitions}
\label{othercomps}
Harvard, Yale, etc.

\section{Tracking Your Energy Usage}
\label{tracking}
TED5000, meter investigation.

\chapter{The Kukui Cup System}
\label{kukuicup}

The proposed Kukui Cup System consists of two main components.  \autoref{webapp} describes the implementation of the web application.  \autoref{socialint} describes integration with popular social networks such as Facebook and Twitter.

\section{Web Application}
\label{webapp}

Investigations of CMSs.  Django/Pinax.

\subsection{WattDepot}

Development of the WattDepot system by Robert Brewer.

\subsection{Near Real-Time Updates}

Providing sub-minute updates for each floor.

\subsection{Mobile Web Application}

Formatting the website for iPhone/Android users.

\section{Social Network Integration}
\label{socialint}

Integrating with Facebook and/or Twitter.

% \section{Other Notifications}
% \label{notifications}

\chapter{Evaluation}

Focus groups, surveys, actual competition.

\chapter{Conclusions}

It is our hope that this system will improve the participants energy literacy in a way that makes a lasting impact on their energy use outside of the competition.

\section{Anticipated Contributions}

We want to provide a system that integrates real-time energy data along with incentives for behavioral change.

\section{Thesis Timeline}

\begin{enumerate}
\item June 2010 - An early version of the Kukui Cup system will be available for evaluation.
\item October 2010 - The competition will take place.
\end{enumerate}