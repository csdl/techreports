
\chapter{Introduction}
It is becoming increasingly important to educate the public about their energy usage.  With the current state of alternative energy sources, it is impossible to support the entire world on renewable energy.  In order to ensure that current energy sources are still available for generations to come, organizations from non-profits to utility providers have advocated energy conservation.  Some organizations hope to educate the up and coming generation in order to instill lasting habits that can be carried on to future generations.

One approach to educating the younger generation is to hold college dormitory energy competitions.  The goal of these competitions is to have the residents of the dorm use as little energy as possible.  Typically, the dorm that reduces their energy use the most at the end of the competition is declared the winner.  Other smaller prizes can be awarded for accomplishing certain goals, like reducing energy usage by 10 percent in a week.  The overall energy reduction is determined either by having someone read the meters or using "smart meters" that are connected to the internet and can send out data.  Universities such as Duke (Eco-lympics) and the University of Wisconsin (Energy Apocalypse) have run competitions relating to energy conservation and awareness. 

These competitions usually involve more than just reducing energy usage.  The competition organizers also include activities that relate to energy conservation and are geared toward improving energy literacy and awareness.  Examples of activities range from viewing documentaries relating to conservation to attending recycling drives.  Participation in these activities and encouraging others to do the same can also be recognized and rewarded in the context of these dorm energy competitions.

\section{The Problem}

To aid in running the competition, many of these universities used web sites to display the resident's current usage.  While it is easy to create a content management system to display mostly static data (i.e. one that is only updated when someone reads the meter), dormitory residents are more motivated by real-time feedback\cite{oberlin-feedback}.  However, the development of such a system can be a complicated and/or expensive process.  Providing real-time feedback not only requires special meters that can communicate with other devices, but also requires software that can process the data and display the relevant information to the user.  Because of this, many organizations have turned to companies like Lucid Design Group that can provide this software and hardware at a cost.

But Lucid Design Group's software only involves the visualization of energy data and does not involve energy awareness activities.  It is designed to be embedded within a website rather than a complete competition package.  Because of this, the software is unable to immediately provide user-related information.  For example, if a dorm resident wants to view their floor's energy usage, they must interact with the visualization to get the information that they need.  In the ideal case, the user would log in using their university credentials and then be able to immediately view their current standings.

As for energy activities, this type of information can be posted on a website.  However, organizers would also like to be able to track interest and participation in these activities so that users can be rewarded.  Users also could be more motivated to participate if they see others in their floor/dorm participating.  Adding in these functions go beyond what a standard content management system does.  Developing such a module for a competition would also take more time and/or resources.

\section{Makahiki}

The goal of Makahiki is to provide a complete software package for organizations that want to hold their own dorm energy competitions.  It will have the following features:

\begin{enumerate}
	\item Integration with WattDepot as a source of energy data.
	\item Support for Central Authentication Service (CAS) for logging in users.
	\item The ability to create activities and track user participation in them.
	\item Produce visualizations of the energy data retrieved from WattDepot.
	\item Integration with social networks such as Facebook and Twitter for displaying progress and standings.
\end{enumerate}

Furthermore, Makahiki will be configurable so that organizations do not need to have all of the above components.  Also, the project will be open source so that other organizations can use and develop modules for it.

\section{Evaluating the Makahiki system}

The first step in evaluating the Makahiki system is to use it in our own dorm energy competition.  We will be holding a dorm energy competition here at the University of Hawaii at Manoa in October using both Makahiki and WattDepot.

We also hope to provide an instance of Makahiki that looks similar to another dorm energy competition (for example, the Duke Eco-lympics).

\section{Thesis Claims}

During the actual competition, we hope to answer the following questions:

\section{Proposal Structure}

The next chapter will discuss related work, which includes other dorm energy competitions and devices that present energy usage to users.  Chapter 3 will describe the components of the Makahiki.  Chapter 4 describes our proposed evaluation procedure.  Finally, chapter 5 includes the conclusion and provides a timeline for the work that is to be done to complete the thesis.

\chapter{Related Work}

\autoref{othercomps} investigates the efforts by other universities to create a dorm energy competition.

\autoref{tracking} investigates "smart meters", which are used to present energy information in a way that is easy to understand for the average person.

\section{Other Dorm Energy Competitions}
\label{othercomps}
Harvard, Yale, etc.

\section{Tracking Your Energy Usage}
\label{tracking}
TED5000, meter investigation.

\chapter{Makahiki}
\label{makahiki}

The proposed Kukui Cup System consists of two main components.  \autoref{webapp} describes the implementation of the web application.  \autoref{socialint} describes integration with popular social networks such as Facebook and Twitter.

\section{Web Application}
\label{webapp}

Investigations of CMSs.  Django/Pinax.

\subsection{WattDepot}

Development of the WattDepot system by Robert Brewer.

\subsection{Near Real-Time Updates}

Providing sub-minute updates for each floor.

\subsection{Mobile Web Application}

Formatting the website for iPhone/Android users.

\section{Social Network Integration}
\label{socialint}

Integrating with Facebook and/or Twitter.

% \section{Other Notifications}
% \label{notifications}

\chapter{Evaluation}

Focus groups, surveys, actual competition.

\chapter{Conclusions}

It is our hope that this system will improve the participants energy literacy in a way that makes a lasting impact on their energy use outside of the competition.

\section{Anticipated Contributions}

We want to provide a system that integrates real-time energy data along with incentives for behavioral change.

\section{Thesis Timeline}

\begin{enumerate}
\item August 2010 - An early version of the Makahiki will be available for evaluation.
\item October 2010 - The competition will take place.
\end{enumerate}