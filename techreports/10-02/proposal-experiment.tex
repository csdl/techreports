\chapter{Experiment}


\section{Data Sources}

\subsection{Power Usage Data}

Both instantaneous power and cummulative energy consumed on a floor by floor basis, beginning long before the competition starts and continuing indefinitely after the competition ends. The sampling rate will be at least 1 minute outside the competition period, and less than 1 minute during the competition period (target of 10 seconds). We can compute the following useful values based on this data:

\begin{itemize}

\item \emph{Minimum floor power} is the power consumed by each floor before residents move in and with all switchable devices (such as lights) turned off. This reveals the power used by the hidden infrastructure of a floor, and may be differ between floors. The value is measured by recording the kWh consumed by each floor over a period of time (preferably days to average out any periodic consumption spikes) and divided by the length of the time interval.

\item \emph{Pre-competition average floor power} is the power consumed by each floor after residents move in, but before the competition has begun. This reveals the power use profile of the floor's residents, and will almost certainly differ between floors. The value is measured by recording the kWh consumed by each floor over a long period of time (preferably weeks to average out any periodic consumption spikes) and divided by the length of the time interval.

\end{itemize}

\subsection{Pre and Post-Competition Energy Literacy Questionnaires}
\label{sec:exp-literacy-questionnaire}

Assess energy literacy of participants at start and end of competition. Assessment is through a questionnaire that is presented to participants via the contest website. Participation in the pre-test will be motivated by the award of Kukui Nut points, while the link to the post-test will be emailed to those participants that used the web site.

\subsection{Website Log Data}

The contest website will extensively log data about participants' actions on the site. All participant actions and events will be logged with timestamp.

\subsection{Post-Competition Qualitative Feedback Questionnaire}

After the competition has ended, participants that used the website will be emailed a link to a qualitative questionnaire, as part of the energy literacy post-test described in \ref{sec:exp-literacy-questionnaire}. This questionnaire will ask for participants' assessment of the competition, the website, and energy literacy in general.

\subsection{Post-Competition Qualitative Feedback Questionnaire}

After the competition has ended, participants that used the website will be emailed a link to a qualitative questionnaire, as part of the energy literacy post-test described in \ref{sec:exp-literacy-questionnaire}. This questionnaire will ask for participants' assessment of the competition, the website, and energy literacy in general.

\subsection{Post-Post-Competition Sustainable Conservation Questionnaire}

In early in the following semester (February 2011), the power data for floors will be re-examined to see whether conservation begun as part of the competition has been sustained months later. Floors with particularly high sustained conservation (compared to pre-competition average floor power), and those with low or non-conservation will be selected for an additional questionnaire, and possible face-to-face interviews to determine residents' self-assessment about why they were or were not sustaining the conservation gains made during the competition.

\section{Research Questions}

\begin{itemize}

\item How can student housing residents be motivated to reduce their electricity usage?

\item How sustainable are any electricity usage reductions after the study is complete?

\item How can energy literacy be assessed?

\item What tasks are most helpful in improving participants' energy literacy?

\item How much does a real-time electricity display help a dorm to achieve electricity conservation?

\item How well do Kukui Nut scores correlate with post-test energy literacy?

\end{itemize}


\section{Hypotheses}

\begin{itemize}

\item Participants with Kukui Nut scores will have higher and more improved post-test energy literacy scores.

\item Improving residents' energy literacy will lead to sustained electricity conservation.

\item Floors with large electricity conservation during the competition period but with low energy literacy scores post-test will have greater rebound effect than floors with higher energy literacy scores post-test.

\item Residents that complete more activities (as specified by the competition website) will improve their energy literacy more than residents that do not participate.

\end{itemize}