%%%%%%%%%%%%%%%%%%%%%%%%%%%%%% -*- Mode: Latex -*- %%%%%%%%%%%%%%%%%%%%%%%%%%%%
%% uhtest-appendix.tex -- 
%% Author          : Robert Brewer
%% Created On      : Fri Oct  2 16:31:12 1998
%% Last Modified By: Robert Brewer
%% Last Modified On: Mon Oct  5 14:41:05 1998
%% RCS: $Id: uhtest-appendix.tex,v 1.1 1998/10/06 02:07:03 rbrewer Exp $
%%%%%%%%%%%%%%%%%%%%%%%%%%%%%%%%%%%%%%%%%%%%%%%%%%%%%%%%%%%%%%%%%%%%%%%%%%%%%%%
%%   Copyright (C) 1998 Robert Brewer
%%%%%%%%%%%%%%%%%%%%%%%%%%%%%%%%%%%%%%%%%%%%%%%%%%%%%%%%%%%%%%%%%%%%%%%%%%%%%%%
%% 

\appendix
\chapter{Energy Literacy Questions}

This appendix lists the questions that assess participants' energy literacy. The questions are separated into sections based on the topic they are addressing. We provide an even number of questions for each concept being tested, so that they can be used to assess energy literacy both before and after the competition. To determine if the phrasing of the question impacts the results, half of the participants will be given the first question in pre-test, while the other half will get the second question in pre-test, and vice versa in the post test. Keywords have been attached to each question to indicate which subjects they attempt to assess. This is useful to ensure that there exists energy literacy content that addresses the concept represented by each keyword.

The questions are intended to be displayed one at a time without the ability for the participant to go back to previous questions, as later questions may imply the answer to previous questions. When administered via a web site, this is straightforward to accomplish. When administered on paper, each question could be printed on a separate sheet of paper, and participants could be directed to not turn back to previous pages.

\section{Power and Energy Concepts}

\subsection{Watt definition}

\begin{question}
	\item The watt is a unit of:
\end{question}

\begin{answer}
	\item energy
	\item power
	\item distance
	\item force
\end{answer}

Correct answer: power

Keywords: power, units

\begin{question}
	\item Power is commonly measured in units of:
\end{question}

\begin{answer}
	\item BTU
	\item joule
	\item kilowatt-hour
	\item watt
\end{answer}

Correct answer: watt

Keywords: power, units

\subsection{Watt abbreviation}

\begin{question}
	\item The watt is abbreviated as:
\end{question}

\begin{answer}
	\item wt
	\item Wh
	\item W
	\item tt
\end{answer}

Correct answer: W

Keywords: power, units

\begin{question}
	\item The abbreviation "W" refers to what unit:
\end{question}

\begin{answer}
	\item watt-hour
	\item wind power
	\item wave power
	\item watt
\end{answer}

Correct answer: watt

Keywords: power, units

\subsection{Watt-hour definition}

\begin{question}
	\item The watt-hour is a unit of:
\end{question}

\begin{answer}
	\item energy
	\item power
	\item distance
	\item force
\end{answer}

Correct answer: energy

Keywords: energy, units

\begin{question}
	\item Electrical energy is commonly measured in units of:
\end{question}

\begin{answer}
	\item BTU
	\item joule
	\item watt-hour
	\item watt
\end{answer}

Correct answer: watt-hour

Keywords: energy, units

\subsection{Watt-hour abbreviation}

\begin{question}
	\item The watt-hour is abbreviated as:
\end{question}

\begin{answer}
	\item Wh
	\item wth
	\item W
	\item erg
\end{answer}

Correct answer: Wh

Keywords: energy, units

\begin{question}
	\item The abbreviation "Wh" refers to what unit:
\end{question}

\begin{answer}
	\item watt
	\item wind-hour
	\item watt-hour
	\item power
\end{answer}

Correct answer: Wh

Keywords: energy, units

\subsection{Power/energy calculations}

\begin{question}
	\item A compact fluorescent lightbulb uses 13 W. If it is run for 2 hours, how much energy does it use?
\end{question}

\begin{answer}
	\item 7.5 Wh
	\item 13 Wh
	\item 26 Wh
	\item 52 Wh
\end{answer}

Correct answer: 26 Wh

Keywords: power, energy, unit-intuition, calculation

\begin{question}
	\item A compact fluorescent lightbulb (CFL) used 26 Wh after running for 2 hours. How much power does the bulb consume?
\end{question}

\begin{answer}
	\item 7.5 W
	\item 13 W
	\item 26 W
	\item 52 W
\end{answer}

Correct answer: 13 W

Keywords: power, energy, unit-intuition, calculation

\begin{question}
	\item If your game console uses 200 W when turned on, how much energy would it waste if you left it on all weekend while you were away?
\end{question}

\begin{answer}
	\item 15000 Wh
	\item 100 Wh
	\item 960 kWh
	\item 9.6 kWh
\end{answer}

Correct answer: 9.6 kWh

Keywords: power, energy, unit-intuition, calculation

\begin{question}
	\item While reading your electric bill you notice that you used 72 kWh more than the previous month. You search your apartment for anything out of the ordinary and find you left a fan running in a closet all month long! Approximately how much power does the fan use?
\end{question}

\begin{answer}
	\item 100 W
	\item 10 W
	\item 300 kWh
	\item 1 kWh
\end{answer}

Correct answer: 100 W

Keywords: power, energy, unit-intuition, calculation


\section{Energy Intuition}

\subsection{Consumption intuition}

\begin{question}
	\item Roughly how much power does a normal compact fluorescent lightbulb (CFL) use when running?
\end{question}

\begin{answer}
	\item 20 mW
	\item 3 W
	\item 60 W
	\item 13 W
\end{answer}

Correct answer: 13 W

Keywords: power, unit-intuition

\begin{question}
	\item Roughly how much power does an electric oven use when turned to its highest setting?
\end{question}

\begin{answer}
	\item 100 W
	\item 500 W
	\item 1 kW
	\item 2.5 kW
\end{answer}

Correct answer: 2.5 kW

Keywords: power, unit-intuition

\begin{question}
	\item On average, how much electrical energy does a home in Hawaii use per day?
\end{question}

\begin{answer}
	\item 13 kWh
	\item 4 kWh
	\item 57 kWh
	\item 328 kWh
\end{answer}

Correct answer: 13 kWh

Keywords: energy, unit-intuition, Hawaii 

\begin{question}
	\item On average, how much electrical energy does a home in Hawaii use per month?
\end{question}

\begin{answer}
	\item 37 kWh
	\item 104 kWh
	\item 390 kWh
	\item 2000 kWh
\end{answer}

Correct answer: 390 kWh

Keywords: energy, unit-intuition, Hawaii 

\subsection{Solar intuition}

\begin{question}
	\item What is the approximate maximum power generated from a single standard rooftop solar panel?
\end{question}

\begin{answer}
	\item 25 W
	\item 50 W
	\item 200 W
	\item 800 W
\end{answer}

Correct answer: 200 W

Keywords: power, unit-intuition, generation, PV

\begin{question}
	\item Approximately how much energy does single standard rooftop solar panel in Hawaii generate each day?
\end{question}

\begin{answer}
	\item 100 Wh
	\item 1000 Wh
	\item 10 kWh
	\item 100 W
\end{answer}

Correct answer: 1000 Wh

Keywords: energy, unit-intuition, generation, PV


\section{Grid knowledge}

\subsection{Generation}

\begin{question}
	\item What is the source of approximately 80\% of Hawaii's electricity?
\end{question}

\begin{answer}
	\item coal
	\item wind
	\item solar
	\item oil
\end{answer}

Correct answer: oil

Keywords: generation, utility, Hawaii

\begin{question}
	\item Burning oil is used to generate approximately what percentage of Hawaii's electricity?
\end{question}

\begin{answer}
	\item 100\%
	\item 50\%
	\item 78\%
	\item 17.5\%
\end{answer}

Correct answer: 78\%

Keywords: generation, utility, Hawaii

\subsection{Demand}

\begin{question}
	\item What is the approximate maximum electrical power demand for the entire island of Oahu?
\end{question}

\begin{answer}
	\item 560 kW
	\item 3800 kW
	\item 11.8 GW
	\item 1.2 GW
\end{answer}

Correct answer: 1.2 GW

Keywords: power, unit-intuition, generation, utility, Hawaii

[Need another question on demand, maybe demand for entire state??]

\begin{question}
	\item What is the electrical grid demand curve?
\end{question}

\begin{answer}
	\item A graph of the amount of power used on the grid over time
	\item The number of efficient appliances demanded by consumers
	\item A graph of the amount of energy used on the grid over time
	\item The amount overhead power lines can bend before breaking
\end{answer}

Correct answer: A graph of the amount of power used on the grid over time

Keywords: power, generation, utility

\begin{question}
	\item Why is the shape of the electrical grid demand curve important?
\end{question}

\begin{answer}
	\item Less efficient power plants must be used if there are peaks in the curve 
	\item A flat curve means nobody is using any electricity
	\item The shape shows how many power plants are running
	\item The curve is lower at night if there is a lot of solar power in the grid
\end{answer}

Correct answer: Less efficient power plants must be used if there are peaks in the curve

Keywords: power, generation, utility, Hawaii

\subsection{Hawaii Clean Energy Initiative}

\begin{question}
	\item What is the goal of the Hawaii Clean Energy Initiative?
\end{question}

\begin{answer}
	\item Maintain Hawaii's energy use at current levels forever
	\item Decrease Hawaii's oil use by 20\% by 2020
	\item Get 70\% of Hawaii's energy from clean sources by 2030
	\item Get 50\% of Hawaii's energy from wind by 2050
\end{answer}

Correct answer: Get 70\% of Hawaii's energy from clean sources by 2030

Keywords: energy, generation, utility, Hawaii

\begin{question}
	\item What is the breakdown of the clean energy mandated by the Hawaii Clean Energy Initiative?
\end{question}

\begin{answer}
	\item 50\% from renewable sources, 10\% from conservation
	\item 30\% from solar, 30\% from wind, 10\% from waves
	\item 30\% from renewable sources, 20\% from conservation, 10\% from natural gas
	\item 30\% from energy conservation, 40\% from renewable sources
\end{answer}

Correct answer: 30\% from energy conservation, 40\% from renewable sources

Keywords: energy, generation, conservation, utility, Hawaii

\section{Climate change}

\begin{question}
	\item What are the effects of climate change?
\end{question}

\begin{answer}
	\item Global temperatures increasing by a few degrees on average
	\item Changes in seasonal rainfall patterns (droughts, floods)
	\item A significant rise in the sea level
	\item All of the above
\end{answer}

Correct answer: All of the above

Keywords: climate change

\begin{question}
	\item What is the primary cause of climate change?
\end{question}

\begin{answer}
	\item Melting glaciers in Greenland
	\item Carbon dioxide released from burning fossil fuels
	\item Natural solar cycles
	\item Radioactive waste from nuclear power plants
\end{answer}

Correct answer: Carbon dioxide released from burning fossil fuels

Keywords: climate change

\begin{question}
	\item Approximately how much rise in sea level is predicted by the end of the century?
\end{question}

\begin{answer}
	\item 2 inches
	\item 6 inches
	\item 1 foot
	\item 3 feet
\end{answer}

Correct answer: 3 feet

Keywords: climate change

\begin{question}
	\item Approximately how much carbon dioxide is in the atmosphere now, and what level is considered safe/acceptable?
\end{question}

\begin{answer}
	\item 450 ppm, 500 ppm
	\item 387 ppm, 350 ppm
	\item 331 ppm, 350 ppm
	\item 600 ppm, 450 ppm
\end{answer}

Correct answer: 387 ppm, 350 ppm

Keywords: climate change

