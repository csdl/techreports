\section{Temporal models, concepts and operators} \label{tconcepts}
The SAX transformation procedure described in the previous section yields symbolic time-series based on the real-valued data. Having such a symbolic representation, also called in the literature as \textit{symbolic temporal data}, is very adventageous comparing to the real-valued data in terms of easy pattern search, indexing, clustering and low computational and space complexity. 

This section built around the technical report by Fabian M\"orchen \cite{citeulike:1748833} which aggregated many of work done in the field of data mining from symbolic temporal data. Two figures taken from this work constitute the Figure \ref{fig:concepts1} and depict a hierarchy of time-points (left panel) and time-intervals (right panel) data models, concepts and operators.

\begin{figure}[tbp]
   \centering
   \includegraphics[height=45mm]{concepts1.eps}
   \caption{The temporal concepts and operators from \cite{citeulike:1748833} for both: time point and time interval data models.}
   \label{fig:concepts1}
\end{figure}

\subsection{Temporal data models} \label{tconcepts_models}
The time intervals temporal data model built upon the concept of \textit{duration} which is a repetition of the property over several time-points. More precisely, \textit{time-intervals} are continous groups of discrete time instants and some algorithms and applications operate with them rather than individual points. Time-intervals essentially are sets of two or more continous points and two successive time points define a minimal interval which starts at the earlier point and continues to the latter point inclusevily. Allen \cite{citeulike:191348} says: ``In English, we can refer times as points or as intervals...'' giving next two examples: ``We found the letter at twelve noon.'' and ``We found the letter while John was away.'' pointing that there are temporal relations involved.

\subsection{Temporal concepts} \label{tconcepts_concepts}
The concept of \textit{concurrency} as described by the author explains the closeness of two time-points in time without considering their ordering, - a coincidence of events in time is important. The \textit{synchronicity} is a special case of concurrency where events occur synchonously in time.

The \textit{order}, and \textit{synchronicity} concepts in the Time intervals model are anlogous ones in the Time points model, whether the Time intervals \textit{coincidence} describes an intersection of intervals in time.

\subsection{Temporal operators} \label{tconcepts_opertaors}
The time point operators from the Figure \ref{fig:concepts1}: \textit{before}, \textit{after} and \textit{equals} precizely define the relation of points in time. The \textit{close} operator is a ``fuzzy extension for temporal reasoning'' since it encapsulates other three. Note that some threshold can be used to relax or constrain these operators, for example we can consider points equal to each other even if they are less than $k$ time units apart.

The Time intervals operators a more complex and were examined in many mork. In 1983 Allen \cite{citeulike:191348} proposed therteen basic relations between time intervals which are distinct, exhausting and qualitative. In his work Allen showed that thirteen relationships are sufficient to model the relationship between any two intervals. Figure \ref{fig:allen} depicts Allen's relations. This relations and operations on them are forming \textit{Allen's Interval Algebra}.

Despite the exhausting property of Allen's relations, while working on the multimedia data analysis, Snoek \& Worring \cite{citeulike:272197} discovered two practical problems: first is that ``in video analysis, exact alignment of start- or end- points seldom occurs due to noise... '' and the second is that ``two time intervals will always have a relation even if they are far apart in time...''. In order to resolve these issues authors had to relax a set of Allen's relations and introduce a new \textit{NoRelation} relation. This new relaxed set of 14 relations named TIME (Time Interval Multimedia Event) was built with two time-interval parameters: $T_{1}$ for the neighborhood in which impresize boundary segments considered synchronous and $T_{2}$ which assigns two intervals as \textit{NoRelation} if they are more than $T_{2}$ time apart.

citeulike:4991332

\begin{figure}[tbp]
   \centering
   \includegraphics[height=20mm]{allen.eps}
   \caption{The Allen's thirteen basic relations (from \cite{citeulike:4072008}) sorted by the degree to which $a$ begins and later ends before $b$. All but \textit{equals} can be inverted.}
   \label{fig:allen}
\end{figure}
