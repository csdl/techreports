\chapter{Related work}
Although, process mining in the business domain is a well-established field with many work done and software developed up to date (ERP, WFM and other systems), the Business Process Intelligence tools usually do not allow to perform process discovery and typically offer relatively simple analyses that depend upon a correct a-priori process model \cite{citeulike:3718014} \cite{citeulike:5044991}. This fact restricts a direct application of the business domain process mining techniques to the general process mining and especially to the software engineering, where processes are usually performed concurrently by many agents, more complex and typically have a higher level of noise. Taking this fact in account, I will review only some of the existing approaches to the general process mining which expressed possible applicability to the software process mining. 

Three papers are reviewed in this chapter: first one, by Cook \& Wolf \cite{citeulike:328044}, discusses an event-based framework for the process discovery based on the grammar inference and finite state machines. Authors directly applied their framework to the SCM logs demonstrating satisfiable results. Second paper, by van der Aalst et al \cite{citeulike:3718014}, demonstrates the applicability of the Transition Systems and labeled Petri nets to the process discovery in general. While authors are not inferring the direct application to the software process in this paper, previous work by van der Aalst and Rubin \cite{citeulike:1885717} discusses this. The third reviewed work by Jensen \& Scacchi while not presenting existing software or experimental validation describes an interesting and related framework of mining OSS repositories and archived communications aiming the process discovery. 

Other work has been done in the process mining and discovery which extends the reviewed papers in terms of dealing with concurrency. Among others, Weijters \& van der Aalst in \cite{citeulike:5128101} propose heuristics application to handle concurrency and noise issues, while van der Aalst et al in \cite{citeulike:5128101} discuss a genetic programming application.