\chapter{Related work} \label{related.work}
Although process mining in the business domain is a well-established field with much software developed up to date (ERP, WFM and other systems), ``Business Process Intelligence'' tools usually do not perform process discovery and typically offer relatively simple analyses that depend upon a correct a-priori process model \cite{citeulike:3718014} \cite{citeulike:5044991}. This fact restricts direct application of business domain process mining techniques to software engineering, where processes are usually performed concurrently by many agents, are more complex and typically have a higher level of noise. Taking this fact in account, I will review only the approaches to the mining for which applicability to software process mining was expressed. 

Three papers are reviewed in this chapter: 
\begin{itemize}
	\item Cook \& Wolf in \cite{citeulike:328044} discuss an event-based framework for process discovery based on grammar inference and finite state machines. The authors directly applied their framework to Software Configuration Management (SCM) logs demonstrating satisfactory results. 
	\item Van der Aalst et al \cite{citeulike:3718014} demonstrate the applicability of Transition Systems and labeled Petri nets to process discovery in general. While this paper does not apply its results directly to software process, the subsequent work by van der Aalst and Rubin \cite{citeulike:1885717} discusses software process application.
	\item The third paper, by Jensen \& Scacchi \cite{citeulike:5043664}, while not presenting a pattern mining strategy, describes an interesting framework built upon an universal generic meta-model and specific to the observed processes models which are iteratively built and revised during case studies. The value of this paper is in the demonstration of the importance of the correct mapping between process artifacts and process entities as well as a demonstration of iterative, human-involved technique of process revision which is emphasizing importance of pre-existing domain knowledge in the effective pruning of the search space.
\end{itemize}
As pointed by authors in the reviewed papers, the proposed methods have difficulties dealing with concurrency, which, in turn, is inevitable in the software process usually performed by many agents. Much successive work has been done extending reviewed approaches to the concurrent processes. Among others, Weijters \& van der Aalst in \cite{citeulike:5128101} propose heuristics to handle concurrency and noise issues, while van der Aalst et al in \cite{citeulike:5128110} discuss a genetic programming application. 
