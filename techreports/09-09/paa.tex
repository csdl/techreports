\subsection{Piecewise Aggregate Approximation (PAA)} \label{paa}
According to Yi \& Faloutsos \cite{citeulike:2946589}, most of the prior research in the time series indexing was centered around the Euclidean distance ($L_{2}$) applied to time sequences, where the method proposed by the authors enable efficient multi-modal similarity search. Supporting the claim, the authors explain some of pitfalls of previously published spectral-decomposition methods such as DFT, DCT, SVD etc. whose core algorithm employs Euclidean distance based metrics over a set of transform coefficients is shown to be inefficient over other distance functions.

The proposed method performs a time-series feature extraction based on segmented means. Given a time-series $X$ of length $n$ transformed into vector $\bar{X} = ( \bar{x}_{1}, ..., \bar{x}_{M} )$ of any arbitrary length $M \leq n$ where each of $\bar{x_{i}}$ is calculated by the following formula:
\begin{equation}
\bar{x}_{i} = \frac{M}{n} \sum_{j=n/M(i-1)+1}^{(n/M)i} x_{j}
\label{eq:paa}
\end{equation}

This simply means that in order to reduce the dimensionality from $n$ to $M$, we first divide the original time-series into $M$ equally sized frames and secondly compute the mean values for each frame. The sequence assembled from the mean values is the PAA transform of the original time-series. It was shown by Keogh et al. that the complexity of the PAA transform can be reduced from $O(NM)$ (\ref{eq:paa}) to $O(Mm)$ where $m$ is the number of sliding windows (frames). The satisfaction of the transform to a bounding condition in order to guarantee no false dismissals was also shown by Yi \& Faloutsos for any $L_{p}$ norms and by Keogh et al. \cite{citeulike:3000416} by introducing the distance function:
\begin{equation}
D_{PAA}(\bar{X}, \bar{Y}) \equiv \sqrt{\frac{n}{M}} \sqrt{ \sum_{i=1}^{M} 
\left(  \bar{x}_{i} - \bar{y}_{i} \right)}
\label{eq:paa_distance}
\end{equation}
and showing that $D_{PAA}(\bar{X}, \bar{Y}) \leq D(X,Y)$.

