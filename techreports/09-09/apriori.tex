\section{Apriori algorithm}
We have seen a broad application of the Apriori algorithm while reviewing temporal patterns for symbolic points and time-series. AprioriAll algorithm was published in 1995 by Agrawal \& Srikant in \cite{citeulike:775528} and is based on the naive approach of Apriori association rule stating that any sub-pattern of a frequent pattern must be frequent. Authors has shown its application to the mining of sequential patterns from a customer transaction database. 

The pattern support function used by authors in their algorithm implementation is defined as the fraction of the customers supporting such a pattern. This definition allows us to state the problem precisely: given a database of customer transactions, the problem is to find the maximal sequences among all others that have at least user-specified support

Apriori algorithm starts by building maximal sequences through finding all ``candidate'' patterns of size 1 with support greater or equal to the minimal support value. On the next step algorithm generates successive set of candidate patterns by extending each of the candidate patterns by 1 and testing it against the database for sufficient support. Algorithms iterates over this second step until it terminates when no further extension is possible yielding a set of maximal sequences

\begin{table}
\begin{center}
    \begin{tabular}{ | c | c | c |}
    \hline
    Large 3-sequences & Candidate 4-sequences                     & Candidate 4-sequences \\ 
                      & after join                                & after pruning \\ 
    \hline
    $\left\{ 1, 2, 3 \right\} $ & $ \left\{ 1, 2, 3, 4 \right\} $ & $ \left\{ 1, 2, 3, 4 \right\} $ \\ 
    \hline
    $\left\{ 1, 2, 4 \right\} $ & $ \left\{ 1, 2, 4, 3 \right\} $ & \\ 
    \hline
    $\left\{ 1, 3, 4 \right\} $ & $ \left\{ 1, 3, 4, 5 \right\} $ & \\ 
    \hline
    $\left\{ 1, 3, 5 \right\} $ & $ \left\{ 1, 3, 5, 4 \right\} $ & \\ 
    \hline
    $\left\{ 2, 3, 4 \right\} $  &                                & \\ 
    \hline
    \end{tabular}
    \caption{Illustration of 4-sequences candidate generation from large 3sequences and pruning of the generated 4-sequences set.}
    \label{fig:apriori}
    \end{center}
\end{table}

Due to the high time cost of the scanning process (determined by the time of a single pass over the database and the number of candidates)  Agrawal \& Srikant in their work improved the naive Aprori approach using a hash-tree and breadth-first search to speed-up the search itself with a clever generative function which prunes candidate pattern sequences before a scanning phase (Figure \ref{fig:apriori}). 

AprioriAll was the very first algorithm for sequential pattern mining. While not being efficient due to the need for for many passes over the database it was used in many work and served as a basis for a family of algorithms based on the same Apriori principle. In 1996 Srikant \& Agrawal extended their original work with GSP (Generalized Sequential Pattern) algorithm. GSP allows time constraints and relaxes the definition of transaction, additional improvement was added by considering the knowledge of taxonomies which improves proning by excluding non-interesting sequences. Wang et al in 2001 proposed a GSP based MFS(Mining Frequent Sequences) \cite{citeulike:5164952}