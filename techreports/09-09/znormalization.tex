\section{Normalization to Zero Mean and Unit Standard Deviation} \label{sect:normalization}
While the time-series retrieved from the Hackystat services are used throughout the Trajectory application before the each of the comparison or matching steps of implemented algorithms the whole series or sub-section used always get normalized to zero mean and unit of standard deviation. This type of normalization is also called ``Normalization to Zero Mean and Unit of Energy'' in research literature and first found in \cite{citeulike:3815880}. It ensures that all elements of the input vector are transformed into the output vector whose mean is approximately $0$ while the standard deviation (and variance) are in a range close to $1$.
This procedure uses mean $\mu$ and standard deviation which calculated as 
\begin{equation}
\sigma = \sqrt{ \frac{ \sum_{i=1}^{N} (x_{i} - \mu)^{2} }{ N - 1 } }
\end{equation}
or equivalently
\begin{equation}
\sigma = \sqrt{
	 \frac{
       N \left( \sum_{i=1}^{N} x_{i}^{2}  \right) - 
       \left( \sum_{i=1}^{N} x_{i} \right) ^{2}}
       {N(N-1)}}
\end{equation}
The normalization itself is 
\begin{equation}
x_{i}^{'} = \frac{x_{i} - \mu}{\sigma}, \: i \in \mathbb{N}
\end{equation}
and yields the vector $X_{i}^{'}$ such as $\mu_{X^{'}} \approx 0$ and $\sigma_{X^{'}} \approx 1$.
