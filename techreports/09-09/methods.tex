\chapter{Methods} \label{methods}
My research in the software process discovery mainly rests on a behavioral representation of software processes as a sequences of events performed by individual developers or automaton with or without concurrency. As we seen in the related work section, it is possible not only to infer the high-level known processes from observing such a low-level artifacts but also to discover a novel processes through the use of the discussed process mining techniques. 

In practice, the collection of development artifacts is performed by Hackystat, a framework for automated software process and product metrics collection and analysis. The event streams which are provided by Hackystat are very rich in information and contain not only atomic events such as invoking a build tool or performing a test but enriched with a temporal data and metrics of an effort applied to the software development. Many other low-level products can be retrieved such as buffer transfers in the editor or compilation activities. All of these low-level artifacts characterize the dynamic behavior of a software process and can be further extended for analyses by aggregating some of the repetitive event patterns into low-level sub processes or aggregating them into single events.

In order to perform this my system extracts the data with desired granularity from the Hackystat and converts it into the symbolic representation by performing Piecewise Aggregate Approximation (\ref{paa}) and Symbolic Aggregate approXimation.  Both approximation methods are discussed in the Section \ref{sax}.

The Temporal Concepts section (\ref{tconcepts}) introduces data models of \textit{time-points} and \textit{time-intervals} on the symbolic sequences along with \textit{temporal concepts} and applicable \textit{temporal operators}. 

The last section of this chapter, Temporal patterns and indexing (\ref{tpatterns}), defines \textit{temporal patterns} (\textit{motifs} and \textit{surprise}) along with discussing relevant pattern search algorithms and data structures used for patterns indexing.