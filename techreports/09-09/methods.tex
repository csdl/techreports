\chapter{Methods} \label{methods}
My research in the software process discovery mainly rests on a behavioral representation of software processes as a sequences of events performed by individual developers or automation with or without concurrency. As we seen in the related work section, it is possible not only to infer the high-level known processes from observing such a low-level artifacts but also to discover a novel processes through the use of a process mining techniques.

In the context of my research, the collection of development artifacts is performed by Hackystat, a framework for automated software process and product metrics collection and analysis. The event streams which are provided by Hackystat are very rich in terms of enclosed information which provides temporal data about atomic events, such as invoking a build tool or performing a test along with a great variety of process and product metrics such as cyclomatic complexity of the code, amount of effort applied to the software development and other. It is possible to retrieve other very low-level artifacts such as buffer transfers wthin the IDE editor or background compilation activities. All of these artifacts characterize the dynamic behavior of a software process in the very details.

In order to perform analyses in my system, I am extracting the necessery process data with a desired granularity from the Hackystat and converting it into the symbolic representation by performing Piecewise Aggregate Approximation (section \ref{paa}) and Symbolic Aggregate approXimation (section \ref{sax}. This symbolic representation of the observed software process is used in the Trajectory analyses which built upon discussed further in this chapter Symbolic Temporal Data Models (section \ref{tconcepts_models}), Temporal Concepts (section \ref{tconcepts}) and Temporal Operators (section \ref{tconcepts_operators}).

The last section of this chapter - Temporal patterns and indexing (\ref{tpatterns}) defines \textit{temporal patterns} of \textit{motif} and \textit{surprise} along with discussing relevant pattern search algorithms and data structures used for the indexing.