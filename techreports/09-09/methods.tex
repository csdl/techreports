\section{Temporal data mining} \label{methods}
My research in software process discovery mainly rests on the mining of recurrent behavioral patterns from a representation of software processes as a temporal sequences of events performed by individual developers or automated tools with or without concurrency. As we saw in the previous sections of this Chapter, it is possible not only to infer the known high-level processes from observing low-level artifacts, but also to discover novel processes through the use of a unsupervised process mining techniques.

In my research, the collection of development artifacts is performed by Hackystat, a framework for automated software process and product metrics collection and analysis. The event streams provided by Hackystat are very rich in information. They provide temporal data about atomic events, such as invoking of a build tool or performing a test, along with a great variety of process and product metrics such as cyclomatic complexity of the code, or amount of effort applied to software development and many other. It is possible to retrieve other very low-level artifacts such as buffer transfers within the IDE editor or background compilation activities. All of these artifacts characterize the dynamic behavior of a software process in great detail.

In order to perform analyses in my system, I am extracting the necessary process data with a desired granularity from the Hackystat and converting it into a symbolic representation by performing a direct mapping based on the taxonomy of events or by approximating telemetry streams (time-series) with Piecewise Aggregate Approximation (section \ref{paa}) and Symbolic Aggregate approXimation (section \ref{sax}). This symbolic representation of the observed software processes are used in the Software Trajectory analyses which build upon Symbolic Temporal Data Models (section \ref{tconcepts_models}), Temporal Concepts (section \ref{tconcepts}) and Temporal Operators (section \ref{tconcepts_operators}). 

Section \ref{tpatterns} defines \textit{temporal patterns} of \textit{motif} and \textit{surprise} along with discussing relevant pattern search algorithms and data structures used for the indexing. Section \ref{apriori} presents AprioriAll algorithm for unsupervised pattern mining.