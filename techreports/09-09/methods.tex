\chapter{Methods}

\section{Piecewise Aggregate Approximation (PAA)}
According to Yi \& Faloutsos \cite{citeulike:2946589}, most of the prior research in the time series indexing was centered around the Euclidean distance ($L_{2}$) applied to time sequences, whether the method proposed by authors bears an ability to perform an efficient multi-modal similarity search. Supporting the claim, authors explain some of pitfalls of previously published spectral-decomposition methods such as DFT, DCT, SVD etc. which core algorithm employs Euclidean-distance based comparison over the set set of transform coefficients and shown to be inefficient over other distance functions.

The proposed method performs a time-series feature extraction (in other words ``approximates time-series'' or ``performs dimensionality reduction'') based on the segmented means. Given time-series $X$ of length $n$ transformed into vector $\bar{X} = ( \bar{x}_{1}, ..., \bar{x}_{M} )$ of any arbitrary length length $M \leq n$ where each of $\bar{x_{i}}$ is calculated by following the next formula:
\begin{equation}
\bar{x}_{i} = \frac{M}{n} \sum_{j=n/M(i-1)+1}^{(n/M)i} x_{j}
\label{eq:paa}
\end{equation}

This simply means that in order to reduce the dimensionality from $n$ to $M$, we first divide the original time-series into $M$ equally sized frames and secondly compute the mean values for each frame. The sequence assembled from the mean values is the PAA transform of the original time-series. It was shown by Keogh et al that the complexity of the PAA transform can be reduced from $O(NM)$ (\ref{eq:paa}) to $O(Mm)$ where $m$ is the number of sliding windows (frames). The satisfaction of the transform to bounding condition in order to guarantee no false dismissals was also shown by Yi \& Faloutsos for any $L_{p}$ norms and by Keogh et al \cite{citeulike:3000416} by introducing the distance:
\begin{equation}
D_{PAA}(\bar{X}, \bar{Y}) \equiv \sqrt{\frac{n}{M}} \sqrt{ \sum_{i=1}^{M} 
\left(  \bar{x}_{i} - \bar{y}_{i} \right)}
\label{eq:paa_distnace}
\end{equation}
and showing that $D_{PAA}(\bar{X}, \bar{Y}) \leq D(X,Y)$.


\section{Symbolic Aggregate approXimation (SAX)}
This section pretty much repeats the litreview section too, with somewhat modified references, will put a bit more info about mindist function and the proper approach for sliding window moves.

\section{Temporal concepts}
Discuss the concepts of Events and Intervals here: data point -> duration -> interval, temporal operators

\section{Temporal patterns and time-series indexing}
