\section{Temporal patterns and indexing} \label{tpatterns}
In previous chapters we have shown the SAX algorithm for conversion of real value time series into symbolic representation along with introducing temporal data models, concepts and operators. All these was a necessery background to categorize existing approaches for unsupervised pattern mining from symbolic temporal data. In this section we will review sequential pattern mining algorithms from time points for univariate and multivariate data along with mining algorithms for time interval data.

\subsection{Time points patterns}
According to M\"orchen, the most commonly searched pattern within univariate symbolic time series is order. This search for particular order of symbols within subsequence is not necesserely requires symbols to be consecutive, usually gaps and substitution allowed. The very similar problem of string matching is one of the central in the computation biology \cite{citeulike:465665} and many algorithms are very similar. The suffix tree algorithm is a standard approach for pattern discovery from the string time-series according to Palopoli et al \cite{citeulike:5003338}. Authors discussing algorithms of automatic discovery of frequent structured patterns (\textit{motifs}) in ``exact'' or ``approximate'' forms. 

As an opposite to motif finding problem, a \textit{surprise pattern} problem explored too. In many fields: biology, physics, astronomy and statistics various algorithms proposed. Keogh et al in \cite{citeulike:3025877} discuss methods of finding a surprise patterns from the temporal data and propose their ``TARZAN'' algorithm which is reporting surprising patterns occuring with substantially different frequency from that expected by chance.