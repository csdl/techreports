\section{Temporal patterns and indexing} \label{tpatterns}
In previous sections of this chapter we have shown the PAA and SAX algorithms for conversion of real valued time series into the symbolic representation along with temporal data models, concepts and operators. All these was a necessary background in order to continue with review of approaches for unsupervised knowledge mining from a symbolic temporal data. In this section we will review sequential pattern mining algorithms from time points and time-intervals data. We will base the review on the univariate data extending it to the multivariate.


Before discussing algorithms we need to provide a formal difinition of pattern. The essential property which defines a pattern is called \textit{support}. The support, roughly speaking, is the frequency of occurence of the certain pattern in the observed data. It is generally assumed that each of the possible patterns have a certain probability to be seen in the dataset just by chance, this probability is called \textit{expected} probability and defines the expected support for the pattern. When the actual observed support (or frequency) of a pattern significantly differs from the expected one, it called \textit{significant support} and indicates that pattern might have some meaningful knowledge artefact attached to it. Although support different from the expected one does not guarantee useful or interesting rules, it is used for a powerful pruning of a search space since most possible patterns will not have sufficient support.

There are two well-established categories of patterns with significant support. The first category of patterns, frequently occuring ones (with support higher than expected), is very important in many data mining areas such as medicine, motion-capture, robotics, video surveillance, meteorology and others. Patterns from this category usually named as \textit{repeated},\textit{approximately repeated} or \textit{motifs}. The second category of patterns, with the support lower than expected, contains patterns named \textit{surprise} or (\textit{novelty}). Novelty patterns also have a great value for many applications: for example it is important to detect unusual semi-repeated pattern in the ECG data diagnosing heartbeat abnormalities, or detecting unusual activity patterns in video surveillance recognizing a suspicious activity.

Note, that the temporal motif finding problem is very similar to one of the central problems in the field of Computation Biology \cite{citeulike:465665} and many algorithms are very similar, but, from other hands, in Biology motifs are usually informative and bear some information about evolutionary artifacts, it is not true in the field of time-series analysis \cite{citeulike:3978085}.

\subsection{Time points patterns}
According to M\"orchen, the most commonly searched pattern within univariate symbolic time series is order. This search for particular order of symbols within subsequence is called \textit{sequential pattern mining} \cite{citeulike:775528} and not necesserely requires symbols to be consecutive, usually gaps and substitution allowed.

The classical suffix tree algorithm \cite{citeulike:707616} with some modifications is a standard approach for pattern discovery from the string time-series according to Palopoli et al \cite{citeulike:5003338}, in this paper authors discussing algorithms of automatic discovery of frequent structured patterns (\textit{motifs}) in ``exact'' or ``approximate'' forms. There are two approaches usually used for the suffix tree building used: \textit{generative}, when algorithms generates all possible patterns and tests their appearence frequency \cite{citeulike:5012661} and \textit{scanning}, when the sliding window used to scan over the sequences available and construct the tree on the fly \cite{citeulike:5012661}. 

The certain limitation of the algorithm is that the maximum length of a pattern needs to be specified upon tree construction since all sub-sequences of this length are extracted from the time series with a sliding window. In \cite{citeulike:5003404} Jiang \& Hamilton compare traversal strategies for suffix trees: breadth-first, depth-first and the heuristic depth-first algorithms implementations for temporal data mining.

As per \textit{surprise pattern} finding problem, Keogh et al in \cite{citeulike:3025877} discuss methods of finding a surprise patterns from the temporal data and propose their ``TARZAN'' algorithm which is based upon suffix tree and Markov model and reporting surprising patterns occurring with a frequency substantially different from that one expected by a chance.

