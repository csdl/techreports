\chapter{Appendices} \label{appendix}

\section{Classroom survey design} \label{survey}
I plan to conduct at least two classroom surveys evaluating both: the usability of the Software Trajectory framework and the usefulness of performed analyses. This surveys will be conducted under approved by Committee on Human Subjects application: \textit{CHS \#16520 - Evaluation of Hackystat}.

The approach I am taking in the conducting this surveys resembles a Software Project Telemetry system evaluation performed by Qin Zhang in \cite{csdl2-06-05}. As pointed by Zhang, the choice of anonymous survey format when collecting feedback in a classroom, is more advantageous then interview. It eliminates students' concerns about final grades and thus provides less biased feedback. The fixed and limited questionnaire is the only disadvantage of the survey format. 

This questionnaire will contain no more than 10 questions and will cover Software Trajectory analyses utility, its usefulness and usability of interface. Another important question will be the students' perception of whether or not Software Trajectory analyses is a reasonable approach to software process discovery, analysis and improvement in ``real world'' settings. I plan to design questions as complete statements like ``\textit{Software Trajectory analyses revealed recurrent behaviors in my software process}'' providing ranking options in the form of statements:
\begin{itemize*}
	\item Strongly Disagree
	\item Disagree
	\item Neutral
	\item Agree
	\item Strongly Agree
\end{itemize*}
In addition to this, an option to skip the question and option to write an extended comment for elaboration of an opinion will be provided. By the end of the questionnaire I will ask students to provide additional impressions, suggestions and comments about the Software Trajectory framework. 

It is most likely that questionnaire will be offered at the last day of instructions preceding final examination. At this time I am not familiar with a curriculum of both courses and this creates certain difficulties to design a complete questionnaire reflecting curriculum settings. If software process behavioral models (patterns) will be taught in the class, and students will follow these in their assignments, I will include questions targeting their experience with confirming development behavior through Software Trajectory analyses. Once responses will be collected I will perform detailed statistical analyses and will correlate students' responses with observations of the Software Trajectory system usage recalled from the system logs. 

At the time of writing this proposal I expect to find that the use of Software Trajectory framework by students will be monotonously increasing during the experiment timeframe and this growth will correlate with the growth of the amount of discovered patterns. If so, this correlation will confirm the intended utility of the Trajectory as an aid in the software process development. If the behavior of usage will be different, for example declining within certain interval of time, I will investigate the reasons behind this phenomena.
