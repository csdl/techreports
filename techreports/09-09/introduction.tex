\chapter{Introduction}
There is a long history of software process improvement through proposing specific patterns of software development process. For example, the Waterfall Model process proposes a sequential pattern in which developers first create a Requirements document, then create a Design, then create an Implementation, and finally develop Tests. The Test Driven Development process proposes an iterative pattern in which the developer must first write a test case, then write the code to implement that test case, then refactor the system for maximum clarity and minimal code duplication. One problem with the traditional top-down approach to process development is that it requires the developer or manager to notice a recurrent pattern of behavior in the first place \cite{citeulike:5043104}. 

In my research, I will apply knowledge discovery and data mining techniques to the domain of software engineering in order to evaluate their ability to automatically notice interesting recurrent patterns of behavior. As a simple example, consider a development team in which committing code to a repository triggers a build of the system. Sometimes the build passes, and sometimes the build fails. To improve the productivity of the team, it would be useful to be aware of any recurrent behaviors of the developers. My system might generate one recurrent pattern consisting of a) implementing code b) running unit tests, c) committing code and d) a passed build: $i \rightarrow u \rightarrow c \rightarrow s $, and another recurrent pattern consisting of a) implementing code, b) committing code, and c) a failed build: $i \rightarrow c \rightarrow f $. The automated generation of these recurrent patterns can provide actionable knowledge to developers; in this case, the insight that running test cases prior to committing code reduces the frequency of build failures.

Although, latest trends in the software process study emphasize mining of the software process artifacts and behaviours \cite{citeulike:5043664} \cite{citeulike:1885717} \cite{citeulike:5112229} \cite{citeulike:1885717}, to the best of my knowledge, the approach I am taking through the mining of 
automatically collected, low-level product and process data has never been attempted.