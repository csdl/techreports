\chapter{Experimental evaluation} \label{experiments}
Three case studies � a pilot study, a classroom case study, and a public data case study will be conducted in order to empirically evaluate capabilities and performance of the Hackystat Trajectory framework. The primary goal of these studies will be to assess the ability of the framework to reproduce well known recurrent behavioral patterns (for example TDD), as well as the ability to discover novel ones.

\section{Pilot project evaluation}
In order to demonstrate the ability of the current system implementation to perform telemetry indexing and temporal recurrent patterns extraction I have conducted two experiments. 

For the first experiment, aiming an unsupervised classification of Telemetry streams, I was using the real data collected during the Spring'09 software engineering class. This dataset represents Hackystat metrics collected during sixty days of the classroom project development conducted by eight students. 


 The second experiment, aiming a discovery of sequential patterns was also conducted by using a real data from my own concurrent development of two software projects.

In order to compare and classify behavior of telemetry streams and individual developers I have built SAX indexes for 
