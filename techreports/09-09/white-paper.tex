\documentclass[11pt,oneside]{article}
\begin{document}
\title{Dissertation proposal abstract}
\author{Pavel Senin \\
 \texttt{senin@hawaii.edu}
}
\date{July 2009}
\maketitle

\section{Motivation}
Since the first computers were build and first programs written many research was done on understanding of both: \textit{programming} and \textit{programs}. In contemporary definitions the human activity called ``\textit{programming}'' consists of many iterative phases and interleaving activities such as planning, writing code, testing, debugging, and maintaining the source code of computer programs. All these high-level phases are also aggregeting many low-level processes and episodes. In addition to that, the social interactions among developers and between developers and users are adding even more complexity into each phase of the programming. The computer program (or \textit{system}, or \textit{software}) itself has it's own lifecycle and in many cases ``orchestrates'' needed programming activities and social interactions.

Since the process of ``understanding'' is connected with measurements, over the years many work has been done in order to discover and standardize metrics for both processes: for programming as the human activity and for the computer systems evolution (life cycle). Currently we have many software utilities aiding the collection and analysis of software metrics. The Hackystat system, originated from University of Hawaii, is one of them and provides users with an ``one-stop shopping place'' for metrics collection utilities, storage and analysis engine and visualization modules. The latest Hackystat implementation is a sophisticated distributed, service oriented system which provides users not only with all the metrics, but aids the understanding of metrics trends dynamic through the system of rules and indicators.

All of these creates a rich background for a further investigation 




\end{document}
