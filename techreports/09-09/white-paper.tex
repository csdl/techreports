\documentclass[11pt,oneside]{article}
\usepackage{fullpage}
%%% Load some useful packages:
%% "New" LaTeX2e graphics support.
\usepackage{graphicx}
%%	using final option to force graphics to be included even in draft mode
%\usepackage[final]{graphicx}
%% Tell graphicx the default directory for all figures
\graphicspath{{figures/}}

%% Enable subfigure support
\usepackage{subfigure}

\begin{document}
\title{Dissertation proposal abstract}
\author{Pavel Senin \\
 \texttt{senin@hawaii.edu}
}
\date{July 2009}
\maketitle

\section{Introduction}
For my dissertation research, I propose to implement and evaluate a novel approach for discovering recurrent patterns of software development behaviors based upon automatically collected, low-level product and process data. There is a long tradition in software engineering of proposing specific patterns of software behaviors in order to produce high quality software. For example, the Waterfall Model process describes a sequential pattern in which developers first create a Requirements document, then create a Design, then create an Implementation, and finally develop Tests. The Test Driven Development process describes an iterative pattern in which the developer must first write a test case, then write the code to implement that test case, then refactor the system for maximum clarity and minimal code duplication.

One problem with the traditional top-down approach to process development is that it requires the developer or manager to notice a recurrent pattern of behavior in the first place. In my research, I will apply knowledge discovery and data mining techniques to the domain of software engineering in order to evaluate its ability to automatically notice interesting recurrent patterns of behavior. As a simple example, consider a development team in which committing code to a repository triggers a build of the system. Sometimes the build passes, and sometimes the build fails. To improve the productivity of the team, it would be useful to understand the recurrent behaviors of the developers. 

My system might generate one recurrent pattern consisting of a) implementing code b) running unit tests, c) committing code and d) a passed build: $i \rightarrow u \rightarrow c \rightarrow s $, while another recurrent pattern is a) implementing code, b) committing code, and c) a failed build: $i \rightarrow c \rightarrow f $. Such automated generation of recurrent patterns can provide actionable knowledge to developers; in this case, the insight that running test cases prior to committing code reduces the frequency of build failures.

As an ultimate goal of my PhD research I am seeing not implementation of such a system (shown at Figure \ref{system_overview.eps}) for aiding a discovery of novel software process knowledge, but my own experimental work and possible contribution to the set of established software process patterns.

\begin{figure}[tbp]
   \centering
   \includegraphics[height=65mm]{system_overview.eps}
   \caption{The high-level system overview. Data collected from users and integration system collected and aggregated by Hackystat, later SAX indexes built. Data mining tools perform unsupervised pattern discovery on demand constrained by the domain knowledge. The GUI provides interface for discovered patterns and knowledge base aiding iterative refinement of knowledge.}
   \label{fig:system_overview}
\end{figure}


\section{Approach and Methods}

\begin{figure}[tbp]
   \centering
   \includegraphics[height=90mm]{fig2.eps}
   \caption{The transformation of Hackystat streams into uni- and multi-variate symbolic time-series and interval series along with the pattern identification.}
   \label{fig:fig2}
\end{figure}

In my opinion many temporal knowledge discovery and data mining methods developed in the last decade can be applied to the software process domain. In my propotyping work I have implemented a Symbolic Aggregate approXimation algorithm \cite{citeulike:2821475} which transforms Hackystat telemetry streams (Figure \ref{fig:fig2}, panels $a$, $b$, $c$) into symbolic representation (Figure \ref{fig:fig2}, panels $d$, $f$, $h$). It is fairly easy to extend this algorithm with an interval series option (Figure \ref{fig:fig2}, panels $e$ and $g$). Currently I am using a relational database storing this symbolic data and KDD and clustering algorithms data requirements addressed with SQL queries: for example it's very easy to get the motif frequency vectors for each of the streams, find most frequent temporal motifs across the subset of streams etc. By using this rich data field I am planning to experiment with Interagon Query Language (IQL) for symbolic temporal data \cite{citeulike:5043086}, AprioriAll \cite{citeulike:775528} and Pattern-Growth algorithms \cite{citeulike:5043097} as well as with Episodes \cite{citeulike:5043099} and Partial Order patterns \cite{citeulike:5043101}. For time-interval data I will investigate applicability of Allen's interval algebra \cite{citeulike:191348} and it's derivatives (UTG \cite{citeulike:5043086} and TSKR\cite{citeulike:3978076}) for the software process domain. 

\section{Current state of the research}
Description of the current software implementation, planned process of KDD and demonstration of the preliminary results - 1 page.

%%% Input file for bibliography
\bibliography{seninp}
%% Use this for an alphabetically organized bibliography
\bibliographystyle{plain}

\end{document}
