\documentclass[man]{apa} %can be jou (for journal), man (manuscript) or doc (document)
%
%
%these next packages extend the apa class to allow for including statistical and graphic commands
\usepackage{url}   %this allows us to cite URLs in the text
\usepackage{graphicx}  %allows for graphic to float when doing jou or doc style

% 11-07 is the FDG paper.

\title{Lights Off.  Game On?  Myths and misperceptions in energy challenge game design
  discovered by the Kukui Cup}
\author{\smallskip Philip M. Johnson \\ 
        \smallskip Yongwen Xu \\ 
        \smallskip Robert S. Brewer}
\affiliation{Collaborative Software Development Laboratory \\ Information and Computer
  Sciences \\ University of Hawaii \\ Honolulu, HI USA \\ johnson@hawaii.edu}

\abstract{

  The Kukui Cup project investigates the use of ``meaningful play'' to facilitate energy
  awareness, conservation and behavioral change.  The Kukui Cup combines real
  world and online environments in an attempt to synergistically combine information
  technology, game mechanics, educational pedagogy, and incentives.  Our goal is to entice
  players into both acquiring more sophistication about energy concepts and
  re-evaluating their behaviors both micro (turning off the light) and macro (desirable
  political positions by candidates).

  To inform the design of the inaugural 2011 Kukui Cup, we relied heavily for guidance on
  the design of prior energy challenges, of which there have been over 100 in the past
  five years.  While our inaugural challenge was quite successful, it also uncovered many
  unwarranted assumptions made in prior challenges. 

  In this paper, we describe the Kukui Cup, the design ``myths and misperceptions'' it
  uncovered, and how we believe they should be addressed in future energy challenges. 

}


\acknowledgements{The Kukui Cup has been supported in part by grant IIS-1017126 from the
  National Science Foundation, and by funding from the University of Hawaii Office of
  Facilities Management.  We gratefully acknowledge the 418 players of the 2012 Kukui Cup
  and the members of the Kukui Cup team in addition to the authors who made the vision a
  reality: Kaveh Abhari, Hana Bowers, Greg Burgess, Caterina Desiato, Michelle Kat\-chuck,
  Risa Khamsi, Alex Young, and Chris Zorn.}

\shorttitle{Myths and misperceptions in energy challenge game design}
%\rightheader{Right header}
%\leftheader{Left Header}

\begin{document}
\maketitle 

\section{Introduction}  

The rising cost, increasing scarcity, and environmental impact of fossil fuels as an
energy source makes a transition to cleaner, renewable energy sources an international
imperative.  One barrier to this transition is the relatively inexpensive cost of current
energy, which makes financial incentives less effective. Another barrier is the historical
success of electrical utilities in making energy ubiquitous, reliable, and easy to access,
thus enabling widespread ignorance in the general population about basic energy principles
and trade-offs.  In Hawaii, the need for transition is especially acute, as our state
leads the nation both in the price of energy and in reliance on fossil fuels as an energy
source.

Moving away from petroleum is a technological, political, and social paradigm shift,
requiring citizens to think differently about energy policies, methods of generation, and
their own consumption. There is no tradition of teaching ``energy'' as a core subject area
for an educated citizenry, even though energy appears to be one of the most important
emergent issues of the 21st century.

The Kukui Cup project attempts to address this need by combining information technology,
community-based social marketing, serious games, and educational pedagogy to support
sustained change in sustainability-related behaviors. 

A defining feature of Kukui Cup challenges is a blend of real world and virtual world
activities, all tied together through game mechanics.  In the real world, players
participate in workshops, geocaching, scavenger hunts, artistic/musical events, win
prizes, and most importantly, learn about their current lifestyle and its impact on
sustainability behaviors.  In the virtual world of the Kukui Cup web application, players
earn points, achieve badges, increase their sustainability "literacy" through readings and
videos, and use social networking mechanisms to engage with friends and family about the
issues raised. The challenge is designed to make real-world and virtual-world activities
complementary and synergistic.

Each Kukui Cup Challenge is typically designed with the following goals for its
participants:
\begin{itemize}
\item Increase knowledge about sustainability issues;
\item Gain insight about the impact of one's current behaviors and how to change them for sustainability;
\item Build community, through awareness of local and national sustainability organizations and initiatives;
\item Create commitment, from minor (turn off the lights when not in use) to major (pursue a profession related to sustainability).
\end{itemize}

To create sophisticated games based upon energy consumption, it must be possible to
collect real-time energy data from meters, store the data, perform analyses on the data, and
visualize the results. We developed WattDepot to provide an open source, vendor-neutral
framework for energy data collection, storage, analysis, and visualization.  WattDepot is
useful not only as technology infrastructure for the Kukui Cup, but as infrastructure for
other energy-related initiative such as the Smart Grid.

Implementation of game mechanics is provided by another system we developed called
Makahiki.  It provides an open source, component-based, extensible environment for
developing sustainability challenges such as the Kukui Cup and tailoring them to the needs
of different organizations.  One configures the Makahiki framework to produce a "challenge
instance" with a specific set of game mechanics, user interface features, and experimental
goals.  Makahiki provides sophisticated instrumentation to support evaluation of how well
the game mechanics supported the organization's goals for the challenge.


\section{Design of the 2011 Kukui Cup}

Design stuff goes here.


\section{Myths and misperceptions in energy challenge game design}

\subsection{\#1: Valid baselines are easy to calculate}

Explain how using data immediately prior to competition can skew results depending upon
whether consumption is increasing, decreasing, or stable.

\subsection{\#2: Outcomes account for confounding variables}

Explain how external factors like weather, holidays, and occupancy volatility can
artificially affect reduction results.

\subsection{\#3: The good guys win}

Explain how energy hogs can more easily obtain larger percentage reductions than their
more sustainability oriented peers.

\subsection{\#4: Results reflect sustainable change}

It is trivial to get a "perfect" 100\% reduction in energy; just flip the breaker.
Behaviors like unplugging vending machines are exactly similar.

\subsection{\#5: Prizes are good}

Extrinsic vs. intrinsic motivation; possible that prizes could decrease intrinsic
motivation to conserve energy.

\subsection{\#6: Results are comparable}

It is tempting to look at literature for guidance on how much reduction can be achieved.  In reality, results are from widely different geographic regions and infrastructures and do not provide any predictive guidance.  Would you use the weather forecast for Hawaii as a way of predicting the temperature in Duluth?

\subsection{\#7: Energy challenges measure the right thing}

Focusing on in-competition energy consumption is not meaningful.  Instead, to understand
the true value of energy competitions, one should instead measure: (1) post-competition
energy consumption; changes in participant knowledge as a result of competition; changes
in participant behavior (interpreted broadly: not just micro-behaviors like turning off
the lights but macro-behaviors such as choice of major)

Like looking under the lightpost for keys.

\section{Other stuff}

``In the classroom, a lack of energy education'':
\url{http://www.indyweek.com/indyweek/in-the-classroom-a-lack-of-energy-education/Content?oid=1547796}

\bibliography{example}

\end{document}



