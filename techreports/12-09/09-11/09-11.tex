%%%%%%%%%%%%%%%%%%%%%%%%%%%%%% -*- Mode: Latex -*- %%%%%%%%%%%%%%%%%%%%%%%%%%%%
%% nsf.tex         : 2009 CPATH Proposal
%% Author          : Philip Johnson
%% Created On      : Tue Mar 31 11:16:58 2009
%% Last Modified By: Philip Johnson
%% Last Modified On: Fri May 08 08:59:57 2009
%%%%%%%%%%%%%%%%%%%%%%%%%%%%%%%%%%%%%%%%%%%%%%%%%%%%%%%%%%%%%%%%%%%%%%%%%%%%%%%
%%   Copyright (C) 2009 
%%%%%%%%%%%%%%%%%%%%%%%%%%%%%%%%%%%%%%%%%%%%%%%%%%%%%%%%%%%%%%%%%%%%%%%%%%%%%%%
%% 
 
\documentclass{proposalnsf}
\usepackage[final]{graphicx}

% NSF proposal generation template style file.
% based on latex stylefiles  written by Stefan Llewellyn Smith and
% Sarah Gille, with contributions from other collaborators.

\renewcommand{\refname}{\centerline{References cited}}
\newcommand{\eCT}{{\it e}CT}

% Fix things so that figures tend to stay away from the last page. 
\renewcommand{\topfraction}{0.85}
\renewcommand{\textfraction}{0.1}
\renewcommand{\floatpagefraction}{0.75}

% this handles hanging indents for publications
\def\rrr#1\\{\par
\medskip\hbox{\vbox{\parindent=2em\hsize=6.12in
\hangindent=4em\hangafter=1#1}}}

\def\baselinestretch{1}

\begin{document}
\title{Empirical Computational Thinking}

\author{Philip M. Johnson \\
      Collaborative Software Development Laboratory \\
      Department of Information and Computer Sciences \\
      University of Hawai'i \\
      Honolulu, HI 96822 \\
      johnson@hawaii.edu\\
}

\maketitle

\begin{abstract}
\noindent This technical report presents an edited version of a proposal to the NSF 
CPATH program. The vision of this proposal is to develop and
institutionalize a new approach to computational thinking where abstraction
and automation combine to transform the use of {\em empirical thinking} in
software development.  We call this approach ``empirical computational
thinking'', or \eCT.  The goal of this research is to explore, evaluate,
and institutionalize techniques and technologies for \eCT, building upon
research and education by ourselves and others in empirically-based
software development.
\end{abstract}

\input{project-09-11}
\bibliography{csdl-trs,hackystat,psp,cpath}
\bibliographystyle{jponew}


\end{document}
