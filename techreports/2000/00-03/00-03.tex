%%%%%%%%%%%%%%%%%%%%%%%%%%%%%% -*- Mode: Latex -*- %%%%%%%%%%%%%%%%%%%%%%%%%%%%
%% 00-03.tex -- 
%% Author          : Philip Johnson
%% Created On      : Fri Jun 23 10:19:09 2000
%% Last Modified By: Philip Johnson
%% Last Modified On: Thu Oct 26 14:46:17 2000
%% RCS: $Id$
%%%%%%%%%%%%%%%%%%%%%%%%%%%%%%%%%%%%%%%%%%%%%%%%%%%%%%%%%%%%%%%%%%%%%%%%%%%%%%%
%%   Copyright (C) 2000 Philip Johnson
%%%%%%%%%%%%%%%%%%%%%%%%%%%%%%%%%%%%%%%%%%%%%%%%%%%%%%%%%%%%%%%%%%%%%%%%%%%%%%%
%% 

\documentstyle[11pt,twocolumn,icse2000,times]{article}
% Psfig/TeX 
\def\PsfigVersion{1.9}
% dvips version
%
% All psfig/tex software, documentation, and related files
% in this distribution of psfig/tex are 
% Copyright 1987, 1988, 1991 Trevor J. Darrell
%
% Permission is granted for use and non-profit distribution of psfig/tex 
% providing that this notice is clearly maintained. The right to
% distribute any portion of psfig/tex for profit or as part of any commercial
% product is specifically reserved for the author(s) of that portion.
%
% *** Feel free to make local modifications of psfig as you wish,
% *** but DO NOT post any changed or modified versions of ``psfig''
% *** directly to the net. Send them to me and I'll try to incorporate
% *** them into future versions. If you want to take the psfig code 
% *** and make a new program (subject to the copyright above), distribute it, 
% *** (and maintain it) that's fine, just don't call it psfig.
%
% Bugs and improvements to trevor@media.mit.edu.
%
% Thanks to Greg Hager (GDH) and Ned Batchelder for their contributions
% to the original version of this project.
%
% Modified by J. Daniel Smith on 9 October 1990 to accept the
% %%BoundingBox: comment with or without a space after the colon.  Stole
% file reading code from Tom Rokicki's EPSF.TEX file (see below).
%
% More modifications by J. Daniel Smith on 29 March 1991 to allow the
% the included PostScript figure to be rotated.  The amount of
% rotation is specified by the "angle=" parameter of the \psfig command.
%
% Modified by Robert Russell on June 25, 1991 to allow users to specify
% .ps filenames which don't yet exist, provided they explicitly provide
% boundingbox information via the \psfig command. Note: This will only work
% if the "file=" parameter follows all four "bb???=" parameters in the
% command. This is due to the order in which psfig interprets these params.
%
%  3 Jul 1991	JDS	check if file already read in once
%  4 Sep 1991	JDS	fixed incorrect computation of rotated
%			bounding box
% 25 Sep 1991	GVR	expanded synopsis of \psfig
% 14 Oct 1991	JDS	\fbox code from LaTeX so \psdraft works with TeX
%			changed \typeout to \ps@typeout
% 17 Oct 1991	JDS	added \psscalefirst and \psrotatefirst
%

% From: gvr@cs.brown.edu (George V. Reilly)
%
% \psdraft	draws an outline box, but doesn't include the figure
%		in the DVI file.  Useful for previewing.
%
% \psfull	includes the figure in the DVI file (default).
%
% \psscalefirst width= or height= specifies the size of the figure
% 		before rotation.
% \psrotatefirst (default) width= or height= specifies the size of the
% 		 figure after rotation.  Asymetric figures will
% 		 appear to shrink.
%
% \psfigurepath#1	sets the path to search for the figure
%
% \psfig
% usage: \psfig{file=, figure=, height=, width=,
%			bbllx=, bblly=, bburx=, bbury=,
%			rheight=, rwidth=, clip=, angle=, silent=}
%
%	"file" is the filename.  If no path name is specified and the
%		file is not found in the current directory,
%		it will be looked for in directory \psfigurepath.
%	"figure" is a synonym for "file".
%	By default, the width and height of the figure are taken from
%		the BoundingBox of the figure.
%	If "width" is specified, the figure is scaled so that it has
%		the specified width.  Its height changes proportionately.
%	If "height" is specified, the figure is scaled so that it has
%		the specified height.  Its width changes proportionately.
%	If both "width" and "height" are specified, the figure is scaled
%		anamorphically.
%	"bbllx", "bblly", "bburx", and "bbury" control the PostScript
%		BoundingBox.  If these four values are specified
%               *before* the "file" option, the PSFIG will not try to
%               open the PostScript file.
%	"rheight" and "rwidth" are the reserved height and width
%		of the figure, i.e., how big TeX actually thinks
%		the figure is.  They default to "width" and "height".
%	The "clip" option ensures that no portion of the figure will
%		appear outside its BoundingBox.  "clip=" is a switch and
%		takes no value, but the `=' must be present.
%	The "angle" option specifies the angle of rotation (degrees, ccw).
%	The "silent" option makes \psfig work silently.
%

% check to see if macros already loaded in (maybe some other file says
% "\input psfig") ...
\ifx\undefined\psfig\else\endinput\fi

%
% from a suggestion by eijkhout@csrd.uiuc.edu to allow
% loading as a style file. Changed to avoid problems
% with amstex per suggestion by jbence@math.ucla.edu

\let\LaTeXAtSign=\@
\let\@=\relax
\edef\psfigRestoreAt{\catcode`\@=\number\catcode`@\relax}
%\edef\psfigRestoreAt{\catcode`@=\number\catcode`@\relax}
\catcode`\@=11\relax
\newwrite\@unused
\def\ps@typeout#1{{\let\protect\string\immediate\write\@unused{#1}}}
\ps@typeout{psfig/tex \PsfigVersion}

%% Here's how you define your figure path.  Should be set up with null
%% default and a user useable definition.

\def\figurepath{./}
\def\psfigurepath#1{\edef\figurepath{#1}}

%
% @psdo control structure -- similar to Latex @for.
% I redefined these with different names so that psfig can
% be used with TeX as well as LaTeX, and so that it will not 
% be vunerable to future changes in LaTeX's internal
% control structure,
%
\def\@nnil{\@nil}
\def\@empty{}
\def\@psdonoop#1\@@#2#3{}
\def\@psdo#1:=#2\do#3{\edef\@psdotmp{#2}\ifx\@psdotmp\@empty \else
    \expandafter\@psdoloop#2,\@nil,\@nil\@@#1{#3}\fi}
\def\@psdoloop#1,#2,#3\@@#4#5{\def#4{#1}\ifx #4\@nnil \else
       #5\def#4{#2}\ifx #4\@nnil \else#5\@ipsdoloop #3\@@#4{#5}\fi\fi}
\def\@ipsdoloop#1,#2\@@#3#4{\def#3{#1}\ifx #3\@nnil 
       \let\@nextwhile=\@psdonoop \else
      #4\relax\let\@nextwhile=\@ipsdoloop\fi\@nextwhile#2\@@#3{#4}}
\def\@tpsdo#1:=#2\do#3{\xdef\@psdotmp{#2}\ifx\@psdotmp\@empty \else
    \@tpsdoloop#2\@nil\@nil\@@#1{#3}\fi}
\def\@tpsdoloop#1#2\@@#3#4{\def#3{#1}\ifx #3\@nnil 
       \let\@nextwhile=\@psdonoop \else
      #4\relax\let\@nextwhile=\@tpsdoloop\fi\@nextwhile#2\@@#3{#4}}
% 
% \fbox is defined in latex.tex; so if \fbox is undefined, assume that
% we are not in LaTeX.
% Perhaps this could be done better???
\ifx\undefined\fbox
% \fbox code from modified slightly from LaTeX
\newdimen\fboxrule
\newdimen\fboxsep
\newdimen\ps@tempdima
\newbox\ps@tempboxa
\fboxsep = 3pt
\fboxrule = .4pt
\long\def\fbox#1{\leavevmode\setbox\ps@tempboxa\hbox{#1}\ps@tempdima\fboxrule
    \advance\ps@tempdima \fboxsep \advance\ps@tempdima \dp\ps@tempboxa
   \hbox{\lower \ps@tempdima\hbox
  {\vbox{\hrule height \fboxrule
          \hbox{\vrule width \fboxrule \hskip\fboxsep
          \vbox{\vskip\fboxsep \box\ps@tempboxa\vskip\fboxsep}\hskip 
                 \fboxsep\vrule width \fboxrule}
                 \hrule height \fboxrule}}}}
\fi
%
%%%%%%%%%%%%%%%%%%%%%%%%%%%%%%%%%%%%%%%%%%%%%%%%%%%%%%%%%%%%%%%%%%%
% file reading stuff from epsf.tex
%   EPSF.TEX macro file:
%   Written by Tomas Rokicki of Radical Eye Software, 29 Mar 1989.
%   Revised by Don Knuth, 3 Jan 1990.
%   Revised by Tomas Rokicki to accept bounding boxes with no
%      space after the colon, 18 Jul 1990.
%   Portions modified/removed for use in PSFIG package by
%      J. Daniel Smith, 9 October 1990.
%
\newread\ps@stream
\newif\ifnot@eof       % continue looking for the bounding box?
\newif\if@noisy        % report what you're making?
\newif\if@atend        % %%BoundingBox: has (at end) specification
\newif\if@psfile       % does this look like a PostScript file?
%
% PostScript files should start with `%!'
%
{\catcode`\%=12\global\gdef\epsf@start{%!}}
\def\epsf@PS{PS}
%
\def\epsf@getbb#1{%
%
%   The first thing we need to do is to open the
%   PostScript file, if possible.
%
\openin\ps@stream=#1
\ifeof\ps@stream\ps@typeout{Error, File #1 not found}\else
%
%   Okay, we got it. Now we'll scan lines until we find one that doesn't
%   start with %. We're looking for the bounding box comment.
%
   {\not@eoftrue \chardef\other=12
    \def\do##1{\catcode`##1=\other}\dospecials \catcode`\ =10
    \loop
       \if@psfile
	  \read\ps@stream to \epsf@fileline
       \else{
	  \obeyspaces
          \read\ps@stream to \epsf@tmp\global\let\epsf@fileline\epsf@tmp}
       \fi
       \ifeof\ps@stream\not@eoffalse\else
%
%   Check the first line for `%!'.  Issue a warning message if its not
%   there, since the file might not be a PostScript file.
%
       \if@psfile\else
       \expandafter\epsf@test\epsf@fileline:. \\%
       \fi
%
%   We check to see if the first character is a % sign;
%   if so, we look further and stop only if the line begins with
%   `%%BoundingBox:' and the `(atend)' specification was not found.
%   That is, the only way to stop is when the end of file is reached,
%   or a `%%BoundingBox: llx lly urx ury' line is found.
%
          \expandafter\epsf@aux\epsf@fileline:. \\%
       \fi
   \ifnot@eof\repeat
   }\closein\ps@stream\fi}%
%
% This tests if the file we are reading looks like a PostScript file.
%
\long\def\epsf@test#1#2#3:#4\\{\def\epsf@testit{#1#2}
			\ifx\epsf@testit\epsf@start\else
\ps@typeout{Warning! File does not start with `\epsf@start'.  It may not be a PostScript file.}
			\fi
			\@psfiletrue} % don't test after 1st line
%
%   We still need to define the tricky \epsf@aux macro. This requires
%   a couple of magic constants for comparison purposes.
%
{\catcode`\%=12\global\let\epsf@percent=%\global\def\epsf@bblit{%BoundingBox}}
%
%
%   So we're ready to check for `%BoundingBox:' and to grab the
%   values if they are found.  We continue searching if `(at end)'
%   was found after the `%BoundingBox:'.
%
\long\def\epsf@aux#1#2:#3\\{\ifx#1\epsf@percent
   \def\epsf@testit{#2}\ifx\epsf@testit\epsf@bblit
	\@atendfalse
        \epsf@atend #3 . \\%
	\if@atend	
	   \if@verbose{
		\ps@typeout{psfig: found `(atend)'; continuing search}
	   }\fi
        \else
        \epsf@grab #3 . . . \\%
        \not@eoffalse
        \global\no@bbfalse
        \fi
   \fi\fi}%
%
%   Here we grab the values and stuff them in the appropriate definitions.
%
\def\epsf@grab #1 #2 #3 #4 #5\\{%
   \global\def\epsf@llx{#1}\ifx\epsf@llx\empty
      \epsf@grab #2 #3 #4 #5 .\\\else
   \global\def\epsf@lly{#2}%
   \global\def\epsf@urx{#3}\global\def\epsf@ury{#4}\fi}%
%
% Determine if the stuff following the %%BoundingBox is `(atend)'
% J. Daniel Smith.  Copied from \epsf@grab above.
%
\def\epsf@atendlit{(atend)} 
\def\epsf@atend #1 #2 #3\\{%
   \def\epsf@tmp{#1}\ifx\epsf@tmp\empty
      \epsf@atend #2 #3 .\\\else
   \ifx\epsf@tmp\epsf@atendlit\@atendtrue\fi\fi}


% End of file reading stuff from epsf.tex
%%%%%%%%%%%%%%%%%%%%%%%%%%%%%%%%%%%%%%%%%%%%%%%%%%%%%%%%%%%%%%%%%%%

%%%%%%%%%%%%%%%%%%%%%%%%%%%%%%%%%%%%%%%%%%%%%%%%%%%%%%%%%%%%%%%%%%%
% trigonometry stuff from "trig.tex"
\chardef\psletter = 11 % won't conflict with \begin{letter} now...
\chardef\other = 12

\newif \ifdebug %%% turn me on to see TeX hard at work ...
\newif\ifc@mpute %%% don't need to compute some values
\c@mputetrue % but assume that we do

\let\then = \relax
\def\r@dian{pt }
\let\r@dians = \r@dian
\let\dimensionless@nit = \r@dian
\let\dimensionless@nits = \dimensionless@nit
\def\internal@nit{sp }
\let\internal@nits = \internal@nit
\newif\ifstillc@nverging
\def \Mess@ge #1{\ifdebug \then \message {#1} \fi}

{ %%% Things that need abnormal catcodes %%%
	\catcode `\@ = \psletter
	\gdef \nodimen {\expandafter \n@dimen \the \dimen}
	\gdef \term #1 #2 #3%
	       {\edef \t@ {\the #1}%%% freeze parameter 1 (count, by value)
		\edef \t@@ {\expandafter \n@dimen \the #2\r@dian}%
				   %%% freeze parameter 2 (dimen, by value)
		\t@rm {\t@} {\t@@} {#3}%
	       }
	\gdef \t@rm #1 #2 #3%
	       {{%
		\count 0 = 0
		\dimen 0 = 1 \dimensionless@nit
		\dimen 2 = #2\relax
		\Mess@ge {Calculating term #1 of \nodimen 2}%
		\loop
		\ifnum	\count 0 < #1
		\then	\advance \count 0 by 1
			\Mess@ge {Iteration \the \count 0 \space}%
			\Multiply \dimen 0 by {\dimen 2}%
			\Mess@ge {After multiplication, term = \nodimen 0}%
			\Divide \dimen 0 by {\count 0}%
			\Mess@ge {After division, term = \nodimen 0}%
		\repeat
		\Mess@ge {Final value for term #1 of 
				\nodimen 2 \space is \nodimen 0}%
		\xdef \Term {#3 = \nodimen 0 \r@dians}%
		\aftergroup \Term
	       }}
	\catcode `\p = \other
	\catcode `\t = \other
	\gdef \n@dimen #1pt{#1} %%% throw away the ``pt''
}

\def \Divide #1by #2{\divide #1 by #2} %%% just a synonym

\def \Multiply #1by #2%%% allows division of a dimen by a dimen
       {{%%% should really freeze parameter 2 (dimen, passed by value)
	\count 0 = #1\relax
	\count 2 = #2\relax
	\count 4 = 65536
	\Mess@ge {Before scaling, count 0 = \the \count 0 \space and
			count 2 = \the \count 2}%
	\ifnum	\count 0 > 32767 %%% do our best to avoid overflow
	\then	\divide \count 0 by 4
		\divide \count 4 by 4
	\else	\ifnum	\count 0 < -32767
		\then	\divide \count 0 by 4
			\divide \count 4 by 4
		\else
		\fi
	\fi
	\ifnum	\count 2 > 32767 %%% while retaining reasonable accuracy
	\then	\divide \count 2 by 4
		\divide \count 4 by 4
	\else	\ifnum	\count 2 < -32767
		\then	\divide \count 2 by 4
			\divide \count 4 by 4
		\else
		\fi
	\fi
	\multiply \count 0 by \count 2
	\divide \count 0 by \count 4
	\xdef \product {#1 = \the \count 0 \internal@nits}%
	\aftergroup \product
       }}

\def\r@duce{\ifdim\dimen0 > 90\r@dian \then   % sin(x+90) = sin(180-x)
		\multiply\dimen0 by -1
		\advance\dimen0 by 180\r@dian
		\r@duce
	    \else \ifdim\dimen0 < -90\r@dian \then  % sin(-x) = sin(360+x)
		\advance\dimen0 by 360\r@dian
		\r@duce
		\fi
	    \fi}

\def\Sine#1%
       {{%
	\dimen 0 = #1 \r@dian
	\r@duce
	\ifdim\dimen0 = -90\r@dian \then
	   \dimen4 = -1\r@dian
	   \c@mputefalse
	\fi
	\ifdim\dimen0 = 90\r@dian \then
	   \dimen4 = 1\r@dian
	   \c@mputefalse
	\fi
	\ifdim\dimen0 = 0\r@dian \then
	   \dimen4 = 0\r@dian
	   \c@mputefalse
	\fi
%
	\ifc@mpute \then
        	% convert degrees to radians
		\divide\dimen0 by 180
		\dimen0=3.141592654\dimen0
%
		\dimen 2 = 3.1415926535897963\r@dian %%% a well-known constant
		\divide\dimen 2 by 2 %%% we only deal with -pi/2 : pi/2
		\Mess@ge {Sin: calculating Sin of \nodimen 0}%
		\count 0 = 1 %%% see power-series expansion for sine
		\dimen 2 = 1 \r@dian %%% ditto
		\dimen 4 = 0 \r@dian %%% ditto
		\loop
			\ifnum	\dimen 2 = 0 %%% then we've done
			\then	\stillc@nvergingfalse 
			\else	\stillc@nvergingtrue
			\fi
			\ifstillc@nverging %%% then calculate next term
			\then	\term {\count 0} {\dimen 0} {\dimen 2}%
				\advance \count 0 by 2
				\count 2 = \count 0
				\divide \count 2 by 2
				\ifodd	\count 2 %%% signs alternate
				\then	\advance \dimen 4 by \dimen 2
				\else	\advance \dimen 4 by -\dimen 2
				\fi
		\repeat
	\fi		
			\xdef \sine {\nodimen 4}%
       }}

% Now the Cosine can be calculated easily by calling \Sine
\def\Cosine#1{\ifx\sine\UnDefined\edef\Savesine{\relax}\else
		             \edef\Savesine{\sine}\fi
	{\dimen0=#1\r@dian\advance\dimen0 by 90\r@dian
	 \Sine{\nodimen 0}
	 \xdef\cosine{\sine}
	 \xdef\sine{\Savesine}}}	      
% end of trig stuff
%%%%%%%%%%%%%%%%%%%%%%%%%%%%%%%%%%%%%%%%%%%%%%%%%%%%%%%%%%%%%%%%%%%%

\def\psdraft{
	\def\@psdraft{0}
	%\ps@typeout{draft level now is \@psdraft \space . }
}
\def\psfull{
	\def\@psdraft{100}
	%\ps@typeout{draft level now is \@psdraft \space . }
}

\psfull

\newif\if@scalefirst
\def\psscalefirst{\@scalefirsttrue}
\def\psrotatefirst{\@scalefirstfalse}
\psrotatefirst

\newif\if@draftbox
\def\psnodraftbox{
	\@draftboxfalse
}
\def\psdraftbox{
	\@draftboxtrue
}
\@draftboxtrue

\newif\if@prologfile
\newif\if@postlogfile
\def\pssilent{
	\@noisyfalse
}
\def\psnoisy{
	\@noisytrue
}
\psnoisy
%%% These are for the option list.
%%% A specification of the form a = b maps to calling \@p@@sa{b}
\newif\if@bbllx
\newif\if@bblly
\newif\if@bburx
\newif\if@bbury
\newif\if@height
\newif\if@width
\newif\if@rheight
\newif\if@rwidth
\newif\if@angle
\newif\if@clip
\newif\if@verbose
\def\@p@@sclip#1{\@cliptrue}


\newif\if@decmpr

%%% GDH 7/26/87 -- changed so that it first looks in the local directory,
%%% then in a specified global directory for the ps file.
%%% RPR 6/25/91 -- changed so that it defaults to user-supplied name if
%%% boundingbox info is specified, assuming graphic will be created by
%%% print time.
%%% TJD 10/19/91 -- added bbfile vs. file distinction, and @decmpr flag

\def\@p@@sfigure#1{\def\@p@sfile{null}\def\@p@sbbfile{null}
	        \openin1=#1.bb
		\ifeof1\closein1
	        	\openin1=\figurepath#1.bb
			\ifeof1\closein1
			        \openin1=#1
				\ifeof1\closein1%
				       \openin1=\figurepath#1
					\ifeof1
					   \ps@typeout{Error, File #1 not found}
						\if@bbllx\if@bblly
				   		\if@bburx\if@bbury
			      				\def\@p@sfile{#1}%
			      				\def\@p@sbbfile{#1}%
							\@decmprfalse
				  	   	\fi\fi\fi\fi
					\else\closein1
				    		\def\@p@sfile{\figurepath#1}%
				    		\def\@p@sbbfile{\figurepath#1}%
						\@decmprfalse
	                       		\fi%
			 	\else\closein1%
					\def\@p@sfile{#1}
					\def\@p@sbbfile{#1}
					\@decmprfalse
			 	\fi
			\else
				\def\@p@sfile{\figurepath#1}
				\def\@p@sbbfile{\figurepath#1.bb}
				\@decmprtrue
			\fi
		\else
			\def\@p@sfile{#1}
			\def\@p@sbbfile{#1.bb}
			\@decmprtrue
		\fi}

\def\@p@@sfile#1{\@p@@sfigure{#1}}

\def\@p@@sbbllx#1{
		%\ps@typeout{bbllx is #1}
		\@bbllxtrue
		\dimen100=#1
		\edef\@p@sbbllx{\number\dimen100}
}
\def\@p@@sbblly#1{
		%\ps@typeout{bblly is #1}
		\@bbllytrue
		\dimen100=#1
		\edef\@p@sbblly{\number\dimen100}
}
\def\@p@@sbburx#1{
		%\ps@typeout{bburx is #1}
		\@bburxtrue
		\dimen100=#1
		\edef\@p@sbburx{\number\dimen100}
}
\def\@p@@sbbury#1{
		%\ps@typeout{bbury is #1}
		\@bburytrue
		\dimen100=#1
		\edef\@p@sbbury{\number\dimen100}
}
\def\@p@@sheight#1{
		\@heighttrue
		\dimen100=#1
   		\edef\@p@sheight{\number\dimen100}
		%\ps@typeout{Height is \@p@sheight}
}
\def\@p@@swidth#1{
		%\ps@typeout{Width is #1}
		\@widthtrue
		\dimen100=#1
		\edef\@p@swidth{\number\dimen100}
}
\def\@p@@srheight#1{
		%\ps@typeout{Reserved height is #1}
		\@rheighttrue
		\dimen100=#1
		\edef\@p@srheight{\number\dimen100}
}
\def\@p@@srwidth#1{
		%\ps@typeout{Reserved width is #1}
		\@rwidthtrue
		\dimen100=#1
		\edef\@p@srwidth{\number\dimen100}
}
\def\@p@@sangle#1{
		%\ps@typeout{Rotation is #1}
		\@angletrue
%		\dimen100=#1
		\edef\@p@sangle{#1} %\number\dimen100}
}
\def\@p@@ssilent#1{ 
		\@verbosefalse
}
\def\@p@@sprolog#1{\@prologfiletrue\def\@prologfileval{#1}}
\def\@p@@spostlog#1{\@postlogfiletrue\def\@postlogfileval{#1}}
\def\@cs@name#1{\csname #1\endcsname}
\def\@setparms#1=#2,{\@cs@name{@p@@s#1}{#2}}
%
% initialize the defaults (size the size of the figure)
%
\def\ps@init@parms{
		\@bbllxfalse \@bbllyfalse
		\@bburxfalse \@bburyfalse
		\@heightfalse \@widthfalse
		\@rheightfalse \@rwidthfalse
		\def\@p@sbbllx{}\def\@p@sbblly{}
		\def\@p@sbburx{}\def\@p@sbbury{}
		\def\@p@sheight{}\def\@p@swidth{}
		\def\@p@srheight{}\def\@p@srwidth{}
		\def\@p@sangle{0}
		\def\@p@sfile{} \def\@p@sbbfile{}
		\def\@p@scost{10}
		\def\@sc{}
		\@prologfilefalse
		\@postlogfilefalse
		\@clipfalse
		\if@noisy
			\@verbosetrue
		\else
			\@verbosefalse
		\fi
}
%
% Go through the options setting things up.
%
\def\parse@ps@parms#1{
	 	\@psdo\@psfiga:=#1\do
		   {\expandafter\@setparms\@psfiga,}}
%
% Compute bb height and width
%
\newif\ifno@bb
\def\bb@missing{
	\if@verbose{
		\ps@typeout{psfig: searching \@p@sbbfile \space  for bounding box}
	}\fi
	\no@bbtrue
	\epsf@getbb{\@p@sbbfile}
        \ifno@bb \else \bb@cull\epsf@llx\epsf@lly\epsf@urx\epsf@ury\fi
}	
\def\bb@cull#1#2#3#4{
	\dimen100=#1 bp\edef\@p@sbbllx{\number\dimen100}
	\dimen100=#2 bp\edef\@p@sbblly{\number\dimen100}
	\dimen100=#3 bp\edef\@p@sbburx{\number\dimen100}
	\dimen100=#4 bp\edef\@p@sbbury{\number\dimen100}
	\no@bbfalse
}
% rotate point (#1,#2) about (0,0).
% The sine and cosine of the angle are already stored in \sine and
% \cosine.  The result is placed in (\p@intvaluex, \p@intvaluey).
\newdimen\p@intvaluex
\newdimen\p@intvaluey
\def\rotate@#1#2{{\dimen0=#1 sp\dimen1=#2 sp
%            	calculate x' = x \cos\theta - y \sin\theta
		  \global\p@intvaluex=\cosine\dimen0
		  \dimen3=\sine\dimen1
		  \global\advance\p@intvaluex by -\dimen3
% 		calculate y' = x \sin\theta + y \cos\theta
		  \global\p@intvaluey=\sine\dimen0
		  \dimen3=\cosine\dimen1
		  \global\advance\p@intvaluey by \dimen3
		  }}
\def\compute@bb{
		\no@bbfalse
		\if@bbllx \else \no@bbtrue \fi
		\if@bblly \else \no@bbtrue \fi
		\if@bburx \else \no@bbtrue \fi
		\if@bbury \else \no@bbtrue \fi
		\ifno@bb \bb@missing \fi
		\ifno@bb \ps@typeout{FATAL ERROR: no bb supplied or found}
			\no-bb-error
		\fi
		%
%\ps@typeout{BB: \@p@sbbllx, \@p@sbblly, \@p@sbburx, \@p@sbbury} 
%
% store height/width of original (unrotated) bounding box
		\count203=\@p@sbburx
		\count204=\@p@sbbury
		\advance\count203 by -\@p@sbbllx
		\advance\count204 by -\@p@sbblly
		\edef\ps@bbw{\number\count203}
		\edef\ps@bbh{\number\count204}
		%\ps@typeout{ psbbh = \ps@bbh, psbbw = \ps@bbw }
		\if@angle 
			\Sine{\@p@sangle}\Cosine{\@p@sangle}
	        	{\dimen100=\maxdimen\xdef\r@p@sbbllx{\number\dimen100}
					    \xdef\r@p@sbblly{\number\dimen100}
			                    \xdef\r@p@sbburx{-\number\dimen100}
					    \xdef\r@p@sbbury{-\number\dimen100}}
%
% Need to rotate all four points and take the X-Y extremes of the new
% points as the new bounding box.
                        \def\minmaxtest{
			   \ifnum\number\p@intvaluex<\r@p@sbbllx
			      \xdef\r@p@sbbllx{\number\p@intvaluex}\fi
			   \ifnum\number\p@intvaluex>\r@p@sbburx
			      \xdef\r@p@sbburx{\number\p@intvaluex}\fi
			   \ifnum\number\p@intvaluey<\r@p@sbblly
			      \xdef\r@p@sbblly{\number\p@intvaluey}\fi
			   \ifnum\number\p@intvaluey>\r@p@sbbury
			      \xdef\r@p@sbbury{\number\p@intvaluey}\fi
			   }
%			lower left
			\rotate@{\@p@sbbllx}{\@p@sbblly}
			\minmaxtest
%			upper left
			\rotate@{\@p@sbbllx}{\@p@sbbury}
			\minmaxtest
%			lower right
			\rotate@{\@p@sbburx}{\@p@sbblly}
			\minmaxtest
%			upper right
			\rotate@{\@p@sbburx}{\@p@sbbury}
			\minmaxtest
			\edef\@p@sbbllx{\r@p@sbbllx}\edef\@p@sbblly{\r@p@sbblly}
			\edef\@p@sbburx{\r@p@sbburx}\edef\@p@sbbury{\r@p@sbbury}
%\ps@typeout{rotated BB: \r@p@sbbllx, \r@p@sbblly, \r@p@sbburx, \r@p@sbbury}
		\fi
		\count203=\@p@sbburx
		\count204=\@p@sbbury
		\advance\count203 by -\@p@sbbllx
		\advance\count204 by -\@p@sbblly
		\edef\@bbw{\number\count203}
		\edef\@bbh{\number\count204}
		%\ps@typeout{ bbh = \@bbh, bbw = \@bbw }
}
%
% \in@hundreds performs #1 * (#2 / #3) correct to the hundreds,
%	then leaves the result in @result
%
\def\in@hundreds#1#2#3{\count240=#2 \count241=#3
		     \count100=\count240	% 100 is first digit #2/#3
		     \divide\count100 by \count241
		     \count101=\count100
		     \multiply\count101 by \count241
		     \advance\count240 by -\count101
		     \multiply\count240 by 10
		     \count101=\count240	%101 is second digit of #2/#3
		     \divide\count101 by \count241
		     \count102=\count101
		     \multiply\count102 by \count241
		     \advance\count240 by -\count102
		     \multiply\count240 by 10
		     \count102=\count240	% 102 is the third digit
		     \divide\count102 by \count241
		     \count200=#1\count205=0
		     \count201=\count200
			\multiply\count201 by \count100
		 	\advance\count205 by \count201
		     \count201=\count200
			\divide\count201 by 10
			\multiply\count201 by \count101
			\advance\count205 by \count201
			%
		     \count201=\count200
			\divide\count201 by 100
			\multiply\count201 by \count102
			\advance\count205 by \count201
			%
		     \edef\@result{\number\count205}
}
\def\compute@wfromh{
		% computing : width = height * (bbw / bbh)
		\in@hundreds{\@p@sheight}{\@bbw}{\@bbh}
		%\ps@typeout{ \@p@sheight * \@bbw / \@bbh, = \@result }
		\edef\@p@swidth{\@result}
		%\ps@typeout{w from h: width is \@p@swidth}
}
\def\compute@hfromw{
		% computing : height = width * (bbh / bbw)
	        \in@hundreds{\@p@swidth}{\@bbh}{\@bbw}
		%\ps@typeout{ \@p@swidth * \@bbh / \@bbw = \@result }
		\edef\@p@sheight{\@result}
		%\ps@typeout{h from w : height is \@p@sheight}
}
\def\compute@handw{
		\if@height 
			\if@width
			\else
				\compute@wfromh
			\fi
		\else 
			\if@width
				\compute@hfromw
			\else
				\edef\@p@sheight{\@bbh}
				\edef\@p@swidth{\@bbw}
			\fi
		\fi
}
\def\compute@resv{
		\if@rheight \else \edef\@p@srheight{\@p@sheight} \fi
		\if@rwidth \else \edef\@p@srwidth{\@p@swidth} \fi
		%\ps@typeout{rheight = \@p@srheight, rwidth = \@p@srwidth}
}
%		
% Compute any missing values
\def\compute@sizes{
	\compute@bb
	\if@scalefirst\if@angle
% at this point the bounding box has been adjsuted correctly for
% rotation.  PSFIG does all of its scaling using \@bbh and \@bbw.  If
% a width= or height= was specified along with \psscalefirst, then the
% width=/height= value needs to be adjusted to match the new (rotated)
% bounding box size (specifed in \@bbw and \@bbh).
%    \ps@bbw       width=
%    -------  =  ---------- 
%    \@bbw       new width=
% so `new width=' = (width= * \@bbw) / \ps@bbw; where \ps@bbw is the
% width of the original (unrotated) bounding box.
	\if@width
	   \in@hundreds{\@p@swidth}{\@bbw}{\ps@bbw}
	   \edef\@p@swidth{\@result}
	\fi
	\if@height
	   \in@hundreds{\@p@sheight}{\@bbh}{\ps@bbh}
	   \edef\@p@sheight{\@result}
	\fi
	\fi\fi
	\compute@handw
	\compute@resv}

%
% \psfig
% usage : \psfig{file=, height=, width=, bbllx=, bblly=, bburx=, bbury=,
%			rheight=, rwidth=, clip=}
%
% "clip=" is a switch and takes no value, but the `=' must be present.
\def\psfig#1{\vbox {
	% do a zero width hard space so that a single
	% \psfig in a centering enviornment will behave nicely
	%{\setbox0=\hbox{\ }\ \hskip-\wd0}
	%
	\ps@init@parms
	\parse@ps@parms{#1}
	\compute@sizes
	%
	\ifnum\@p@scost<\@psdraft{
		%
		\special{ps::[begin] 	\@p@swidth \space \@p@sheight \space
				\@p@sbbllx \space \@p@sbblly \space
				\@p@sbburx \space \@p@sbbury \space
				startTexFig \space }
		\if@angle
			\special {ps:: \@p@sangle \space rotate \space} 
		\fi
		\if@clip{
			\if@verbose{
				\ps@typeout{(clip)}
			}\fi
			\special{ps:: doclip \space }
		}\fi
		\if@prologfile
		    \special{ps: plotfile \@prologfileval \space } \fi
		\if@decmpr{
			\if@verbose{
				\ps@typeout{psfig: including \@p@sfile.Z \space }
			}\fi
			\special{ps: plotfile "`zcat \@p@sfile.Z" \space }
		}\else{
			\if@verbose{
				\ps@typeout{psfig: including \@p@sfile \space }
			}\fi
			\special{ps: plotfile \@p@sfile \space }
		}\fi
		\if@postlogfile
		    \special{ps: plotfile \@postlogfileval \space } \fi
		\special{ps::[end] endTexFig \space }
		% Create the vbox to reserve the space for the figure.
		\vbox to \@p@srheight sp{
		% 1/92 TJD Changed from "true sp" to "sp" for magnification.
			\hbox to \@p@srwidth sp{
				\hss
			}
		\vss
		}
	}\else{
		% draft figure, just reserve the space and print the
		% path name.
		\if@draftbox{		
			% Verbose draft: print file name in box
			\hbox{\frame{\vbox to \@p@srheight sp{
			\vss
			\hbox to \@p@srwidth sp{ \hss \@p@sfile \hss }
			\vss
			}}}
		}\else{
			% Non-verbose draft
			\vbox to \@p@srheight sp{
			\vss
			\hbox to \@p@srwidth sp{\hss}
			\vss
			}
		}\fi	



	}\fi
}}
\psfigRestoreAt
\let\@=\LaTeXAtSign





\begin{document}

\title{Empirically-Guided Software Effort Guesstimation}

\author{
        %\hspace*{-2ex}
        %\parbox{6.0in} 
        %\begin{center}
        {\bf Philip M. Johnson \hfill Carleton A. Moore}\\ 
        {\bf Joseph A. Dane \hfill  Robert S. Brewer }\\ 
        Collaborative Software Development Laboratory\\
        Information \& Computer Sciences Department\\
        University of Hawaii\\
        Honolulu, Hawaii 96822 USA \\
        johnson@hawaii.edu
        %\end{center} 
        }
\maketitle
%
% If you want to print drafts of the paper with a draft 
% notice in the copyright space, comment out the \copyrightspace
% line above and include the \submitspace line below instead.
%
%\copyrightspace
\submitspace{Appearing in IEEE Software, November 2000}

\begin{abstract}
  Software project effort estimation is frequently seen as complex and
  expensive for individual software engineers.  In Project LEAP, we are
  investigating tools and methods to support low-cost, empirically-based software
  developer improvement. In a recent case study, we investigated effort
  estimation and the relative accuracy of a dozen different analytic estimation
  procedures. Student programmers could estimate the effort required using
  any of the analytic methods, or else provide their own ``guesstimate''.
  Our study provides evidence that ``guesstimates'', when informed by
  low-cost analytical methods, may be the most accurate of all. 
\end{abstract}

\noindent {\bf Keywords: } software estimation, software quality, software planning

\pagestyle{plain}

\section{Introduction}

Most software engineers (and their managers) fervently wish they could get
home from the office earlier at night and come in less on weekends. Typically,
escaping the tyranny of crunch mode requires greater
control over the development process, and reasonably accurate project
estimates provide essential support for acquiring this control.

Given the undisputable lifestyle benefits that can result, it is ironic
that many developers continue to view project estimation as a kind of
software engineering version of the ``np-complete problem''. In other
words, some fear that creating an accurate project estimate might be, in the worst
case, as costly as simply building the system itself.

The state of the art in software project estimation does not deserve this
reputation, but the fact remains that modern methods are time-consuming and
complex. Estimation methods such as COCOMO II \cite{Boehm95} are oriented
toward the needs of large and very large software engineering projects for
creating estimates of cost, effort, and schedule.  In COCOMO II, an
organization provides values characterizing organizational factors such as
``precedentedness'', ``development flexibility'', ``team cohesion'', and
``process maturity'' as part of the estimation process.  On the other hand,
methods such as the Personal Software Process (PSP) \cite{Humphrey95} support
small/individual project estimation accuracy.  In the PSP,
individuals collect personal data concerning software size, effort, and
defects for their own code-level work products, then use a regression-based
analytic technique called PROBE to generate effort predictions for future
projects based upon historical data.   Although both of these approaches
have demonstrated success in organizations with the resources necessary to
adopt them, the process overhead involved in their implementation can often
be inconsistent with the resource-constrained nature of smaller or even
startup level development organizations.

For several years, we have pursued an initiative called Project LEAP, whose
goal is the improvement of individual developers though lightweight,
empirical, anti-measurement dysfunction, and portable software engineering
tools and methods \cite{csdl2-00-01}.  One result of this project is the
LEAP toolkit, a publically available, Java-based suite of applications for
collection and analysis of an individual's software engineering data.
Among other things, LEAP contains tools to simplify the collection of size
data (at the level of lines of code, methods, and classes) and effort data
(in developer minutes). The collected size and effort data serves as input
to a set of estimation tools that can produce over a dozen different
analytical estimates of the effort required for a new project given an
estimate of its size, using various estimation methods such as linear,
logarithmic, or exponential regressions.  During project planning, the
developer can review and select one of the estimates produced by the
analytical tools, or else substitute their own ``guesstimate'' based upon
their own experience and review of the analytical estimates.

During the Fall of 1999, we performed a case study using the LEAP toolkit
in a graduate software engineering class.  One of the goals of the study
was to evaluate the various analytical estimation methods made available by
the toolkit.  We were curious as to whether a single method would prove
most accurate, or whether the most accurate method would depend upon the
type of project or the specific developer.  To our surprise, we found that,
in most cases, the developer-generated ``guesstimates'' were more accurate
than the analytical estimates.  We also found that the PROBE method of the
Personal Software Process \cite{Humphrey95}, perhaps the most widely
publicized analytical approach to personal effort estimation, was the sixth
most accurate method.  Finally, we found that access to a range of
analytical estimation methods appeared to be useful to developers in
generating their guesstimates and improving them over time. 

Due to the small number of participants in the case study, not all of our
results are statistically significant.  Replication is necessary 
to better understand its generality and applicability.  However, our
initial findings do support some provocative conjectures concerning the
research and practice of project estimation. First, much of the research on
project estimation focuses on the evaluation of a single estimation
method.  Success is often demonstrated by increased estimation accuracy
over time. Our study suggests that more research should be done in which a
variety of different estimation methods are investigated
simultaneously, since even suboptimal estimation methods may exhibit
improvement over time.  Second, our research suggests that a practical
approach to small-project software estimation might be ``empirically guided
guesstimation'', whereby a variety of simple analytical methods inform
developer intuition to create a low-cost yet useful software project
estimates.

The remainder of this article elaborates on these ideas. The next section
briefly describes how estimation is performed in the LEAP toolkit.  The
following section presents selected results from the case study. The final
section provides some recommendations for future research and practice.

\section{Estimation using the LEAP toolkit}

The LEAP toolkit provides a suite of tools for collecting and analyzing 
personal software engineering data (see Sidebar).  For the purposes of
this article, only the tools relating to size collection, effort collection,
and project estimation are relevant. 

\begin{figure*}[t]
  {\centerline{\psfig{width=5.5in,figure=leap6-time-est.ps}}}
  \caption{The LEAP tool interface for effort estimation}
  \label{fig:time-est}
\end{figure*}

Effort collection in LEAP is straightforward: tools allow developers to
enter effort data either after the fact or interactively while they
work. Size collection uses a tool called LOCC \cite{csdl-99-10} that 
counts non-comment source lines, methods, and classes for any language for
which a JavaCC-compliant grammar is available.  Unlike most other size
counting tools, LOCC also includes a ``diff'' facility which allows the
developer to count the number of source lines, methods, and classes that {\em changed}
between two versions of the system, which is critical for useful
estimation in an evolutionary development setting.  Together, these tools
provide cross-platform, cross-language support for low-cost size and effort
collection. 

Producing an estimate of the effort required for a new software project
using LEAP involves the following basic steps:
\begin{enumerate}
\item Select the completed projects whose size and effort data you wish to use as input 
to the analytical estimation methods.
\item Generate an estimate of the expected size (in lines, methods, or
classes).
\item Browse the effort estimates produced by the various analytical methods.
\item Record an effort estimate to use, either from those proposed by the
analytical methods or by generating your own ``guesstimate''. 
\end{enumerate}

Steps three and four are illustrated in the accompanying figures.  Figure
\ref{fig:time-est} shows the LEAP interface to the analytical effort estimation
method browser.  In the top panel, the user can indicate the type of
historical data to be used (planned or actual), the trend line (i.e. the
type of analytical model to be applied to the data), and the size metric
(in this case, Java-based lines, methods, and classes).  The displayed
graph and the three estimates in the bottom panel indicate the results of
estimating the effort required to produce an 80 method software system, based
upon eight prior software projects, using linear regression.

Choosing different trend lines and size metrics enables the developer
to generate and then save a variety of estimates.
Figure \ref{fig:hee} shows one pane in the LEAP project
planning tool.  The upper half of the window provides fields in which the
developer enters their planned effort. The lower half of the window shows a
table in which the developer can collect together for review a set of
estimates generated using the effort estimation tool. (The ``Add to Hee''
button in Figure \ref{fig:time-est} adds the data currently displayed to
the table in Figure \ref{fig:hee}.)  As you can see in Figure
\ref{fig:hee}, the analytical effort estimates range in value from 947
minutes to 1304 minutes, yet the developer has entered a planned effort of
1200 minutes, which differs from all of the analytically-derived
estimates. 

\begin{figure*}[t]
  {\centerline{\psfig{width=5in,figure=leap8-time-est-bound.ps}}}
  \caption{One pane in the LEAP project plan tool.  The table in the lower
  half of the screen lists the estimates used in coming up with the plan. }
  \label{fig:hee}
\end{figure*}

The LEAP toolkit allows substantial flexibility in the kinds of work
products and size metrics involved in estimation.  While LOCC provides
``shrink-wrapped'' support for counting object-oriented programming
languages like Java and C++, LEAP provides a size metric definition
mechanism that allows users to integrate new measures for new document
types. For example, the size of a high level requirements document might be
counted in pages, while a design specification might be counted in function
points. 

The ability of LEAP to support multiple estimation approaches and represent 
planned and actual efforts for projects enabled us to explore an interesting
research question: would users pick the most accurate estimate, and if so,
which estimation technique would it be?  The following case study provides
some initial insight into this question. 

\section{Results from a case study}

To gain some initial data concerning estimation using the LEAP toolkit, we
conducted a case study in a graduate software engineering class of 16
students at the University of Hawaii.  Seven of the 16 students were
``experienced'' software developers, with five or more years of prior
programming experience. The 16 students developed 8 software projects each
for a total of 128 projects. However, the students used the first three
projects to generate an initial set of historical data, and then practiced
the estimation technique described above on the remaining five projects.
Thus, only the last five projects by the 16 students, or 80 projects total were used to
generate the following results concerning estimation.

Over the course of this semester, the students improved in their estimation 
capabilities in a manner similar to the results obtained from other
small-project empirical methods such as the Personal Software Process
\cite{Hayes97,Emam96,Ferguson97,csdl-98-13}.  For
example, the average student size estimate was off by approximately 50\% on
the third project but decreased to less than 15\% by the eighth project.
Similarly, the average effort estimate was off by approximately 25\% on the
third project but decreased to less than 10\% on average on the eighth
project. 

What is novel about this case study is not that the students became
better at estimation, which is by now a well-established property of
curricula of this type.  What is novel is the ability to investigate
the relative accuracy of the estimates chosen relative to the other
estimates that were available.  In other words, did the students pick the
right estimate to use amongst those that were available?   
To assess this, we calculated the error in the estimates by subtracting the 
planned value from the actual value, and performed an analysis of variance
(ANOVA) test to see if the average error for one approach was statistically 
different from another approach. We interpret the results as statistically
significant if the results could be due to chance less than 2\% of the
time (p $<$ .02).  More details are available in \cite{csdl2-00-01}.

The results from the case study are provocative.  First, over the course of
the 80 projects in which estimation was performed, an analytical estimate
was used as the student's planned effort estimate less than 10\% of the
time. Students vastly preferred to enter their own ``guesstimate'' rather
than use one of the analytical values, although they typically recorded
around three analytical estimates before entering their guesstimate.  For 8
out of 16 students, these ``guesstimates'' were on average more accurate
(i.e.  had the smallest average error) than any of the analytical
estimates. For two of the eight students, the guesstimates were
significantly more accurate than any of the analytical methods (p $<$ .02). For the
other six students, their guesstimates were more accurate on average but
not significantly more accurate than the next most accurate method from a
statistical point of view.  Interestingly, there was only one student for
which an analytical technique (exponential regression using actual LOC) was
the best estimate overall and significantly better than any other
estimation technique (p $<$ .02).

When the average relative error is computed for each estimation method over
the class as a whole, it reveals that the student guesstimates were
significantly more accurate than any of the analytical methods (p 
 $<$ 0.001).  The next most accurate estimation technique (which was also,
incidentally, significantly more accurate than any of the other analytical
estimation techniques) was exponential regression using actual
methods. 

The PROBE analytical method from the Personal Software Process is perhaps
the most widely researched current method for individual effort estimation,
and so we were particularly interested to see how it fared against the
other methods.  For one student, the PROBE method was the most accurate
estimation method on average (although not significantly more accurate than
the next most accurate method from a statistical point of view). However,
for the other 15 students, their own guesstimates were more accurate on
average than the PROBE estimate.  When the class data is viewed as a whole,
the PROBE method was the sixth most accurate method.

\section{Lessons Learned}

One must be careful when interpreting these results.  Even though some of
our results are statistically significant, the teaching method, development 
environment, choice of programming projects, and other factors 
greatly influence the data and outcomes.  There are, of course, a myriad of 
differences between an academic and professional development environment
that might influence such data. That said, we present the following as
the major lessons we believe can be learned from our experiences with
estimation in the LEAP toolkit:

\begin{enumerate}
  
\item {\bf Software process data collection is costly. Even more automated
support than that currently provided by the LEAP toolkit will be useful.}  Collecting effort and size data, even with the advanced
  tool support provided by LEAP, is still viewed by developers as a
  distraction from the task at hand, even when the future benefits to them
  are clear.  As a result, we are currently investigating
  ``ultra-lightweight'' project estimation support, in which deep
  integration with a specific development environment could provide almost
  total automation of effort and size data collection for personal
  small-project estimation. 

\item {\bf Providing multiple estimation techniques and size measures adds value.}
  Our users found it interesting and useful to ``browse'' the various
  estimation methods to look for similarities and differences. In some
  cases, users would ``triangulate'' their estimate by choosing a value
  midway between several of the analytical estimates.  In addition, while
  estimation based upon lines of code may be best for small projects,
  estimation based upon the number of methods may be preferable as the size
  of the system increases.
  
\item {\bf Analytical estimation methods need not be complex.}  The
  estimation methods included in LEAP range from the very simple (such as a
  method based upon average, minimum, and maximum productivity) to the
  relatively complicated (the PROBE method, in which the developer must
  choose between three analytical methods based upon the strength of
  correlation between planned and actual data).  Our preliminary results
  suggest that complicated methods may not necessarily yield a more
  accurate estimate, particularly when developers can incorporate their own
  intuition into the estimate.  During the postmortem interview on each
  project, students would often explain that they decided to deviate from the
  analytical estimates to account for various idiosyncracies in the
  project, their background, or other factors.  Automating the breadth of 
  knowledge brought to bear on the estimation problem by these users within 
  an analytical estimation technique will be problematic at best. 
  
\item {\bf Empirical data is useful.} Although our users
  rarely adopted an analytical estimate without change, it was also clear
  that they found the data of tremendous utility as a way of ``getting in
  the right ballpark.''  When a user's estimate departed radically from the
  analytical range of values, they invariably had an excellent rationale
  for their decision.
  
  The importance of empirical data is highlighted by an analysis of
  estimation accuracy between two groups in the case study: the
  ``experienced'' developers (those with five or more years of experience)
  and the ``inexperienced'' developers (those with less than five years of
  experience).  To our surprise, we discovered that
  experienced developers substantially underestimated the effort required
  for their projects at the beginning of the study. Their guesstimates were 
  so far off that they were not only worse than the analytic estimates,
  they were even less accurate than those of the inexperienced developers!
  It seems
  as though inexperienced developers appear to ``trust'' the
  empirical data and let it guide their guesstimates from the beginning, while
  experienced developers initially ignored the empirical data.  However,
  this effect was temporary.  By the end of the study, experienced
  developers seemed to rely more on the empirical data in forming their
  guesstimates, and their estimation accuracy improved to a level similar to that of
  the inexperienced developers. 


\end{enumerate}

\section{Acknowledgements}

This research was supported in part by a grant from the National Science
Foundation (CCR-9804010). 

\section{Sidebar: The LEAP toolkit}

``LEAP'' is an acronym for four of the fundamental design principles
underlying the toolkit.  Lightweight, Empirical, Anti-measurement
dysfunction, and Portable.  {\em Lightweight} means that the tool does not force
you (or force the colleagues that work with you) to conform to its own highly
structured and/or constrained development process.  {\em Empirical} means that
the toolkit helps you to collect and analyze a variety of numerical
measures which can help give you insight into your strengths and
weaknesses. (Non-numerical, qualitative insight is also supported in the
toolkit.)  {\em Anti-measurement dysfunction} \cite{Austin96} means that the toolkit is
designed with a recognition that attaching numbers to people can be a
social or professional liability in certain organizations, and so the tool
is designed to support privacy concerns.  Finally, {\em portable} acknowledges
the highly mobile nature of current software professionals both within and
across company boundaries, and thus the need for personal data collection
and analysis software that can move with the professional into different
platforms, organizations, and application development domains. 

LEAP consists of over a dozen integrated tools, supporting collection of
defect data, effort data, size data, checklists, and patterns.  LEAP also
allows definition of work product types, defect types, severity levels,
development phases, and size measures. The project planning tool supports
size and effort estimation, GQM (goal-question-metric) \cite{Basili84}
documentation about the nature and use of the data collected, and various
analyses concerning productivity, defect rates, and so forth.

The Leap toolkit is written in Java. Since 1997, we have made over 30 public
releases of the Leap toolkit. As of January, 2000, the Leap toolkit consisted
of 44,000 lines of code, 2,209 methods, 287 classes, and 14 packages.  You can
download the latest version of the Leap toolkit from
$<$http://csdl.ics.hawaii.edu/\linebreak[0]Tools/LEAP/LEAP.html$>$.  The
LOCC toolkit is also publically available, and can be downloaded from 
$<$http://csdl.ics.hawaii.edu/\linebreak[0]Tools/LOCC/LOCC.html$>$.

\bibliography{/export/home/csdl/bib/psp,/export/home/csdl/bib/csdl-trs}
%\bibliographystyle{plain}
\bibliographystyle{abbrv}

 \end{document}

