%%%%%%%%%%%%%%%%%%%%%%%%%%%%%% -*- Mode: Latex -*- %%%%%%%%%%%%%%%%%%%%%%%%%%%%
%% essay.tex -- 
%% Author          : Robert Brewer
%% Created On      : Wed Jan 27 17:17:21 1999
%% Last Modified By: Robert Brewer
%% Last Modified On: Mon Jan 31 12:29:00 2000
%% RCS: $Id: essay.tex,v 1.1 2000/01/31 22:12:58 rbrewer Exp rbrewer $
%%%%%%%%%%%%%%%%%%%%%%%%%%%%%%%%%%%%%%%%%%%%%%%%%%%%%%%%%%%%%%%%%%%%%%%%%%%%%%%
%%   Copyright (C) 1999 Robert Brewer
%%%%%%%%%%%%%%%%%%%%%%%%%%%%%%%%%%%%%%%%%%%%%%%%%%%%%%%%%%%%%%%%%%%%%%%%%%%%%%%
%% 

\section{Essay Question}
%No more than 700 words, currently exactly 700 words (not counting heading and quote)

\begin{quote}
  {\em As an Aspect Technology Fund grant recipient, how would you contribute
    to the field of technology and promote the spirit of entrepreneurship?}
\end{quote}

As a grant recipient I will be helping the growth of high-tech industry here in
Hawaii. In the last few months it seems like all politicians and the media can
do is talk about high-tech and what it can do for Hawaii. In his 2000 ``State
of the State'' address the governor highlighted technology as an area where the
state needs to continue to expand. But it takes more than words to build a
business. It takes people actually willing to take risks and make a commitment
to hard work.

I have already been involved in one successful high-tech startup here in
Hawaii: LavaNet. I left the MS program in ICS to help start LavaNet in 1994.
Through a lot of hard work, in 1997 LavaNet had grown to employ 23 people and
serve 6,500 customers. At that time I decided to come back to UH to finish my
degree.

One of the very real problems with the high-tech industry here in Hawaii is the
migration of talented people to the mainland. The booming Silicon Valley
economy makes it difficult to convince technical people to stay in Hawaii. My
experiences at LavaNet I have enabled me to evangelize the concept of
entrepreneurship with first hand experience. Many students here at UH have not
seriously considered being part of a startup company here in Hawaii as a career
path which is unfortunate. This grant would form the foundation of a software
business based here in Hawaii.  Beyond the initial implementation, developers
will be needed to enhance existing spam tests and add new ones. The server
product will require a skilled technical support staff in order to help ISPs
install the software. There is a lot of potential for this kind of enterprise
because for information technology, Hawaii's geographical isolation is
irrelevant. Also, information technology has virtually no environmental impact
compared to other industries, which makes it sustainable.

I'm also pleased to hear that there is more venture capital available for
high-technology now than ever before. Most of the credit for this must go to
the incredible success of the Digital Island IPO. In my opinion the best way to
promote the spirit of entrepreneurship is to lead by example. Each successful
technology company further demonstrates that it is possible (and profitable) to
start up in Hawaii which encourages others to take the entrepreneurial leap.
And those that payoff big for their investors will provide the seed funds for
more companies (as Digital Island has done).

In conclusion, I have the experience and the ability to take an idea and turn
it into a company. Every high tech company that starts up and succeeds in
Hawaii provides another beacon for future entrepreneurs.

% LocalWords:  http opensource LocalWords tex www org Exp
