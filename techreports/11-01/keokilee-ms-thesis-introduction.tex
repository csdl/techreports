Because of the questionable future of our current energy sources, it has become increasingly important to educate people about their energy usage.  To ensure that current energy sources are still available for future generations, organizations from non-profits~\cite{blueplanet} to utility providers~\cite{heco} have advocated energy conservation.  Some organizations have targeted children in order to instill these habits early~\cite{energyhog}.

One approach to educating the younger generation is to hold college residence hall energy competitions.  The goal of these competitions is to have the residents of the dorm reduce their energy usage.  Typically, the dorm that reduces their energy use the most at the end of the competition is declared the winner.  Other smaller prizes can be awarded for accomplishing certain goals, like reducing energy usage by 10 percent in a week.  The overall energy reduction is determined either by having someone read the meters or using ``smart meters" that are connected to the internet and can send out data.  Universities such as Duke (Eco-Olympics) and the University of Wisconsin (Energy Apocalypse) have run competitions relating to energy conservation and awareness. 

\section{The Problem}

To aid in running the competition, many of these universities used web sites to display the dormitory's current usage.  While it is easy to create a content management system to display mostly static data (i.e. one that is only updated when someone reads the meter), dormitory residents are more motivated by real-time feedback~\cite{oberlin-feedback}.  However, the development of such a system can be a complicated and/or expensive process.  Providing real-time feedback not only requires special meters that can communicate with other devices, but also requires software that can process the data and display the relevant information to the user.  Because of this, many organizations have turned to companies like Lucid Design Group that can provide this software and hardware at a cost.  Peterson et al estimated that the cost for their experiment was about \textdollar5,000 for each of the two dorms, though they suggest a conservative estimate of \textdollar10,000 for each dorm.

However, Lucid Design Group's software only involves the visualization of energy data for the building or dorm as a whole.  Because of this, the software is unable to immediately provide user-related information.  For example, if a dorm resident wants to view their floor's energy usage, they must interact with the visualization to get the information that they need.  In the ideal case, the user would log in using their university credentials and then be able to immediately view their current standings.

If another organization is handling the development and maintenance of the software and hardware, who owns the data related to the competition? While some organizations may feel the benefits of the competition will outweigh the loss of control of the data, it can be used to make improvements to the overall design of the competition in the real world. Data of interest to an organization include a more granular look at the energy readings and the web server logs. For example, organizations can see how effective marketing and real-world events are by seeing how they drive traffic to the website.

Another issue is that many of the web sites for these dorm energy competitions have not kept up with the latest trends in technology.  Two trends in particular are social networks and mobile phone interfaces.  Social networks, like Facebook, are extremely popular especially among college students.  A few colleges/universities have energy competition pages or groups on Facebook.  These two things provide a place for students to discuss what is going on in the competition.  However, providing the ability to share user activity within Makahiki can help spread the word and encourage other potential users to sign in.

The rise of mobile computing devices such as Android devices, iPhones, and iPod Touches have web developers designing interfaces for smaller screens.  While a normal website is viewable on an iPhone, users will have to zoom in to read sections of the site, which can be a little cumbersome.  Also, these devices may not support browser-based plugins like Adobe's Flash.  If the website only uses Flash for a few elements, then everything else will load except for those elements.  If the website uses Flash for their entire interface, the page may not load at all.

Finally, energy competitions that only involve energy reduction may result in the energy use going back up after the competition is over. Promoting energy \emph{literacy} in addition to the energy reduction competition can provide the necessary context for individuals to drive sustained energy reduction past the end of the competition. Many energy competitions involve events that help people be more aware of their energy usage. However, participation in these events is typically tracked manually and is updated periodically. Furthermore, many systems do not have such a component built in. Our goal is to provide a system that also promotes energy literacy and makes it as easy as possible to manage.

\section{Goals for Makahiki}

The goal of Makahiki is to provide a complete software package for organizations that want to hold their own dorm energy competitions that addresses the problems described above.  It will have the following features:

\begin{enumerate}
	\item Near-real time energy data by integrating with WattDepot.
	\item Personalized user information.
	\item The ability to create and track participation in activities, commitments, and goals.
	\item Integration with social networks such as Facebook and Twitter for displaying progress and standings.
	\item Free and open source project.
	\item Support for an energy literacy points competition with content created by administrators.
\end{enumerate}

WattDepot~\cite{wattdepot} is an open source web service in development here at the University of Hawaii at Manoa.  Its purpose is to collect power data from sources and to store it. By combining Makahiki with WattDepot, competition organizers have an automated way of tracking the energy usage for buildings.  While compatible meters still need to be purchased, the software comes at no additional cost.  WattDepot also provides data at near-real time intervals, meaning that dorm residents can immediately see the results of their actions.

While WattDepot is able to collect all of the power data, the data still needs to be processed and presented to users in a visually appealing way.  Through the use Google Visualizations, electricity data can be presented in a way that is easy to understand and dynamic.  Users will be able to see their past and current electricity usage and be able to compare it to other floors.  Competition standings and goal status can also be displayed to users.

While a competition is active, competition participants will want to see personalized information about how well their floor is doing in the dorm.  To accomplish this, Makahiki will have the ability to create user accounts for participants.  There, they can view personalized energy data as well as view and participate in activities, events, and/or commitments.  Users also have a public profile that they can customize, including the ability to upload a profile picture.

In order to promote energy literacy, Makahiki will also have support for creating actions, commitments, and daily energy goals.  Actions can range from replacing light bulbs in a desk lamp to attending meetings by sustainability organizations.  Commitments are typically small ``pledges" that dorm residents can accept, like committing to turning off the lights in the lounge.  Goals are actions that entire dorm floors participate in.  Daily energy goals involve floor members voting on how much they plan on reducing their floor's energy and then attempting to accomplish the goal.  Competition participants can participate in these items in order to gain points.

Finally, Makahiki will be open source.  This means that competition organizers can design the visual look of the application to fit their organization.  Also, advanced users can add or tweak modules in the application to fit their needs.  \autoref{feature-comparison} outlines our desired features and compares Makahiki with other dorm energy competition implementations.

\begin{table}[h]
	\begin{tabular}{| l || l | l | l | l | l | l | }
		\hline
		Feature & Makahiki & Lucid & Duke & Harvard & Indiana & Stanford \\
		\hline
		Near-real time energy data & Yes & Yes & No & No & No & No \\
		Information personalization & Yes & No & No & No & No & No \\
		Activity competition management & Yes & No & No & No & No & No\\
		Social network support & Yes & No & No & No & No & No\\
		Mobile device support & Yes & No & No & No & No & No \\
		Development cost & Free & High & Low & Low & Low & Low \\
		\hline
	\end{tabular}
	\caption{Features of Makahiki compared to other dorm energy competitions}
	\label{feature-comparison}
\end{table}

\section{Evaluation}

In order to evaluate the design of the system, we needed to bring in subjects to interact with the system. We also had multiple evaluation phases. First, we designed mockups and had subjects ``interact'' with them as if they were the real system. Based on their feedback, we implemented the system and then recruited subjects to evaluate the web site in our lab. We did two rounds of these evaluations. After this, we held a beta evaluation where we ran a small-scale version of the actual energy competition. Subjects in the beta evaluation interacted with the system during their own time. We used the contents of the database, logs, and surveys to gain insight into how subjects interacted with the system.

The final step in evaluating the Makahiki system was to use it in our own dorm energy competition. We held a dorm energy competition here at the University of Hawaii at Manoa in October 2010 using both Makahiki and WattDepot.  Much like the beta evaluation, we used the contents of the system and surveys to gain insight into how these users interacted with Makahiki.

\section{Research Questions}

The evaluation of Makahiki is intended to address two primary research goals.

\begin{enumerate}
  \item How can we provide a system that supports researchers interested in users participating in an energy competition?
	\item How can we effectively test the website and the overall design of the competition?
\end{enumerate}

\section{Thesis Structure}

Chapter 2 will discuss related works, which includes other dorm energy competitions.  Chapter 3 will describe the overall design of Makahiki.  Chapter 4 describes our evaluation procedure.  Chapter 5 will present results and discussion and Chapter 6 presents contributions and future directions of the research.