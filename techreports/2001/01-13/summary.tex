%%%%%%%%%%%%%%%%%%%%%%%%%%%%%% -*- Mode: Latex -*- %%%%%%%%%%%%%%%%%%%%%%%%%%%%
%% summary.tex -- 
%% RCS:            : $Id: nsf93-summary.tex,v 1.11 93/10/06 16:52:35 johnson Exp $
%% Author          : Philip Johnson
%% Created On      : Wed Aug 11 12:55:46 1993
%% Last Modified By: Philip Johnson
%% Last Modified On: Mon Nov  5 16:43:47 2001
%% Status          : Unknown
%%%%%%%%%%%%%%%%%%%%%%%%%%%%%%%%%%%%%%%%%%%%%%%%%%%%%%%%%%%%%%%%%%%%%%%%%%%%%%%
%%   Copyright (C) 1993 University of Hawaii
%%%%%%%%%%%%%%%%%%%%%%%%%%%%%%%%%%%%%%%%%%%%%%%%%%%%%%%%%%%%%%%%%%%%%%%%%%%%%%%
%% 
%% History
%% 11-Aug-1993          Philip Johnson  
%%    

\section{Executive Summary}


Collection and analysis of empirical software project data is central to
modern techniques for improving software quality, programmer productivity,
and the economics of software project development.  Unfortunately,
effective collection and analysis of software project data is rare in
mainstream software development. Prior research suggests that three primary
barriers are: (1) {\em cost}: gathering empirical software engineering
project data is frequently expensive in resources and time; (2) {\em
  quality}: it is often difficult to validate the accuracy of the data; and
(3) {\em utility}: many metrics programs succeed in collecting data but
fail to make that data useful to developers.

This report describes Hackystat, a technology initiative and research
project that explores the strengths and weaknesses of a {\em
  developer-centric}, {\em in-process}, and {\em non-disruptive} approach
to empirical software project data collection and analysis. 
Hackystat makes available to developers a set of custom sensors that they
voluntarily attach to their development tools.  Once installed, these
sensors automatically monitor characteristics of the developer's process
and products and send data to a centralized web service.  The web service
maintains a repository of process and product data for each developer,
performs analyses on the repository, and automatically sends the developer
an email when new, unexpected, and/or potentially interesting analysis
results become available.

Hackystat is developer-centric because all data is collected directly from
developer activities, and all analyses provided back to that same
developer. It is in-process because data is collected regularly throughout
project development, and analysis results are returned and intended to be
useful during development. It is non-disruptive because developers do not
need to interact directly with the sensors during development to enable
collection or analysis.

Hackystat is designed to accelerate adoption of empirically guided software 
project measurement by providing a new approach to to addressing the
barriers of cost, quality, and utility identified above.  We will evaluate
the Hackystat project through the following
research components:

\begin{enumerate}
  
\item {\em Bootstrap and ongoing technology development.}  The
  ``bootstrap'' phase will create a critical mass of sensors
  and analysis mechanisms required for experimentation.
 Additional development will occur throughout the project.
  
\item 
{\em Verification and validation.}  Verification focuses on assessing
the fidelity of the sensors; in other words, does a sensor that is intended 
to detect ``idle time'' actually detect it with sufficient accuracy to
support related analyses? Validation focuses on assessing the utility of
the analyses: do developers find the analyses to be useful, and do they
actually make changes based upon the feedback they receive?
  
\item {\em A comparative study of data collection and analysis in Hackystat
    and the PSP}.  We will contrast our approach with the Personal Software
  Process (PSP):  a developer-centric, in-process, {\em disruptive}
  approach to software project data collection and analysis.

\item {\em A case study of automated data collection and analysis for
Extreme Programming.} This case study will explore whether the Hackystat
approach can add value and provide new insight into ``agile''
development methods such as XP.

\item {\em A longitudinal study of software development skill maturation.}
  By the end of the three years of the study, we will have in-process
  software development data from students over two years of course work.
  This study will provide insights into the development of advanced
  programmers, with the goal of improving educational practice.

\end{enumerate}  
 






