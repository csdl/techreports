\section{Experiences}

To better understand the strengths and weaknesses of the Makahiki+WattDepot software stack, we have been designing and implementing an ``Energy Challenge'' called the Kukui Cup.  Development of the Kukui Cup challenge began in 2009, and the first Kukui Cup challenge was held in 2011 for over 1,000 first year students living in the residence halls at the University of Hawaii (UH) in Fall, 2011.  In Fall 2012, the second Kukui Cup challenge was held at the University of Hawaii using Makahiki+WattDepot.  In addition, Hawaii Pacific University (HPU) held a Kukui Cup challenge using Makahiki+WattDepot. Finally, an international organization called the East-West Center (EWC) held a Kukui Cup challenge using just Makahiki (their energy data was manually gathered by reading meters and entering the data by hand, so WattDepot was not needed for their challenge).    

The successful creation of four challenges by three different organizations over two years provides evidence that the software stack can be tailored to the differing needs of separate organizations.  First, UH uses meters by Electro-Industries Inc., while HPU uses meters by EGauge Inc., and EWC collected their energy data by hand. Second, while UH and HPU challenges involved only energy consumption data, the EWC challenge involved both energy and water consumption data.  Third, the IT infrastructure at UH and HPU provided authentication services using CAS and LDAP, while EWC used the built-in Django authentication. Fourth, the user interface was customized to ``brand'' each challenge with the logo and other thematic elements of the sponsoring organization. 

On the other hand, it should be recognized that these organizations are in other ways quite similar: they are all institutions of post-secondary education, and they are all based in Hawaii.  These organizational similarities are mostly due to the desire by the 2012 challenges to reuse a significant amount of the content developed in 2011, which was oriented toward the Hawaii-based, college aged demographic. For 2013 and beyond, we hope to expand our experiences with the software stack  ``downward'' into primary and secondary schools, and well as ``outward'' into residences and businesses. 

User response to the 2011 UH Kukui Cup challenge was positive, and provided evidence regarding the software stack's usability, functionality, and performance characteristics.   Over 400 students participated, for an adoption rate of approximately 40\%.  In a user survey conducted near the end of the challenge, over 90\% of users said they would participate in the challenge again if offered an opportunity.  60\%  said ``ease of use'' was the thing they liked best about the website.  40\% responded ``Nothing'' when asked what was confusing about the website, and 32\% responded ``Nothing'' when asked what they would change about the website.  The survey did yield insights into what could be improved, including the ability to introduce new games at points during the challenge, to provide better access to other player data, and simplify navigation.  There was virtually no downtime during the 2011 challenge, and only one significant bug in the system (affecting scoring) was discovered during the challenge, which was fixed within a day of its discovery.

The 2012 challenges are ongoing as of the time of writing, so the following results must be viewed as preliminary, but our current experience is similarly positive to 2011.  The UH challenge participation rate so far appears to be slightly lower than last year, at about 33\%, though the HPU challenge participation rate so far is higher (approximately 50\%), and the EWC participation rate is much lower (around 6\%). None of the challenge instnaces have experienced significant downtime, and so far only 1 significant bug (affecting scoring) has been reported (and has again been fixed within a day).  Load testing of the software stack just prior to the 2012 challenge indicates a hypothetical throughput of around 200 concurrent users with acceptable page loading times, though we have not experienced that level of load in the current challenges. 

Our experiences over the past two years yields the following lessons learned regarding the Makahiki+WattDepot software stack:

{\em Cloud-based hosting simplifies installation.}  During 2012, we have gained experience with both cloud-based hosting as well as local installation for the Makahiki+WattDepot software stack.  We have found that cloud-based hosting significantly simplifies the installation process and avoids certain types of installation-related bugs from occurring.  On the other hand, cloud-based hosting incurs costs (in our experience, between \$50-\$100/month for these challenges) and may incur constraints (for example, the Heroku hosting platform currently has minimal support for LDAP authentication).  The Makahiki Manual \cite{MakahikiManual} provides instructions for both cloud-based and local installation, providing some idea of the differences between the two approaches.

{\em Challenge design is time consuming.} Despite the freely providing the Makahiki+WattDepot software stack to HPU and EWC, along with content developed specifically for college-age residents of Hawaii, the administrators  still expressed surprise at the how time consuming it was to design their respective challenges.  This appears to be due to the fact that the software stack enables a variety of game mechanics (such as the Smart Grid Game, Raffle Game, Badges, and point-based Prizes) not present in more simplistic energy challenges.  For example, the Smart Grid Game requires configuration of the widget including what activities to include and when/where they appear in the game.  The Raffle Game and point-based awards requires the collection of appropriate prizes. While we provided a library of almost 100 activities from 2011, all of the 2012 challenges required the definition of at least a few new events.  Thus, although the result is a more sophisticated experience for the participants, the up front design was more than anticipated. The Kukui Cup Challenge Planning Guide \cite{KukuiCupChallengePlanningGuide} provides more details on this process.

{\em Challenge administration overhead is significant.}  




