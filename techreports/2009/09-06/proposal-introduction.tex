\chapter{Introduction}

It is widely recognized that the global climate is warming due to anthropogenic sources (see \autoref{sec:climate-change}). There are an increasing number of people interested in making personal changes to reduce their contribution to climate change. We focus our efforts on these people who are actively seeking to reduce their carbon footprint. These users have questions about how best to direct their efforts, such as “how much additional electricity does increasing the thermostat on the air conditioner by one degree consume?” or “how much less carbon is released by carpooling with someone who lives nearby rather than driving alone?” We need to provide a system that allows users to perform informal experiments related to their daily lives and provide rapid feedback on the results of those experiments.

Another important question these users face is ``what are the relative contributions of different activities to my carbon footprint (driving, air travel, heating/cooling home, entertainment, food, consumer purchases)?'' While tracking usage in individual areas (home electricity usage, automobile gasoline consumption) is important, the comparative contributions to the user’s carbon footprint must be determined for rational decision–making. This approach allows users to prioritize among the many possible ways they can reduce their environmental impact.

\section{Thesis Statement}
