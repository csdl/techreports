%%%%%%%%%%%%%%%%%%%%%%%%%%%%%% -*- Mode: Latex -*- %%%%%%%%%%%%%%%%%%%%%%%%%%%%
%% uhtest-abstract.tex -- 
%% Author          : Robert Brewer
%% Created On      : Fri Oct  2 16:30:18 1998
%% Last Modified By: Robert Brewer
%% Last Modified On: Fri Oct  2 16:30:25 1998
%% RCS: $Id: uhtest-abstract.tex,v 1.1 1998/10/06 02:06:30 rbrewer Exp $
%%%%%%%%%%%%%%%%%%%%%%%%%%%%%%%%%%%%%%%%%%%%%%%%%%%%%%%%%%%%%%%%%%%%%%%%%%%%%%%
%%   Copyright (C) 1998 Robert Brewer
%%%%%%%%%%%%%%%%%%%%%%%%%%%%%%%%%%%%%%%%%%%%%%%%%%%%%%%%%%%%%%%%%%%%%%%%%%%%%%%
%% 

\begin{abstract}
The Personal Environmental Tracker (PET) is a proposed system for helping
people to track their impact on the environment, and to make changes to reduce
that impact, creating a personal feedback loop. PET consists of sensors that
collect data such as home electricity or gasoline usage and send it to a
database for analysis and presentation to the user. By collecting data from
diverse sources, PET can help users decide what aspect of their lives they
should make changes in first to maximize their reduction in environmental
impact. PET's open architecture will allow other ubiquitous sustainability
researchers to leverage the infrastructure for research in sensors, data
analysis, or presentation of data.
\end{abstract}
