
\begin{abstract}
In software engineering, the importance of measurement is well understood, and many efficient software development metrics have been developed to help measurement. However, as the number of metrics increases, the effort required to collect data, analyze them and interpret the results quickly becomes overwhelming. This problem is even more critical in educational approaches regarding empirical software engineering.

The Software Intensive Care Unit is a new approach to facilitating software measurement and control with multiple software development metrics. It uses the Hackystat system to achieve automated data collection and analysis, then uses the collected analysis data to create a monitoring interface for multiple ``vital signs''. A vital sign is a wrapper of a software metric with an easy to use presentation. It consists of a historical trend and a newest state value, both of which are colored according to the ``health'' state. 

My research deployed and evaluated the Software ICU in a senior-level software engineering course. Students' usage was logged in the system, and a survey was conducted. The results provide supporting evidence that Software ICU does help students in course project development and project team organization. In addition, the results of the study also discover some limitations of the system, including inappropriate vital sign presentation and measurement dysfunction.

\end{abstract}
