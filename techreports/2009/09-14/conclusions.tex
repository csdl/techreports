\chapter{Conclusion}\label{chapter_conclusion}
In this dissertation I have proposed the Software Trajectory Analysis -- a generic framework for recurrent behaviors discovery from software process and product artifacts, whose ultimate premise is to provide means for empirical guidance of developers and project management in software development and decision-making processes. To aid the discovery of recurrent behaviors, I have also proposed a novel approach for time series classification, that not only enables the discovery and ranking of class-characteristic patterns, but, as I have shown, aids in interpretability of both: the classification results and the data specificity. This chapter summarizes my research, discusses its significance, and suggests future directions. 

\section{Dissertation summary}
This dissertation covers a novel approach to the problem of recurrent behaviors discovery from software process artifacts. The research field-specific data type, that is \textit{software trajectory}, its analysis paradigm, that is \textit{Software Trajectory Analysis}, and a novel technique for time series classification and characteristic patterns discovery called \textit{SAX-VSM} are proposed and evaluated.

In Chapter \ref{chapter_introduction}, I have described background for the explored research problem concerned with software process analysis. Specifically, I have emphasized the importance of an ability to discover recurrent behaviors offline by mining  public software repositories. The concept of software trajectory, that is a temporally ordered sequence of software artifact measurements, and the Software Trajectory Analysis paradigm were introduced in the same Chapter.

Next, in Chapter \ref{chapter_background_work}, I have discussed software metrology and the relevant work from research area of mining software repositories, while focusing on the recurrent behaviors discovery. 

In Chapter \ref{chapter_sax_vsm}, addressing the problem of \textit{unsupervised} knowledge discovery from software trajectories, and in particular the problem of time series class-characteristic patterns discovery, I have proposed and evaluated a novel technique for interpretable time series classification called SAX-VSM, which enables the discovery of class-characteristic patterns.

Finally, in Chapter \ref{chapter_sta}, I have shown and evaluated a reference implementations of based on SAX-VSM Software Trajectory Analysis framework which provides end-to-end generic and customizable solution for the problem of recurrent behaviors discovery from software trajectories. The implemented system capabilities and limitations were also discussed.

\section{Research summary}
In contrast to the previous body of work in the area of software process analysis, that has been mostly concerned with identification of \textit{previously known} behaviors for the purpose of software project management, the major distinction of this work is that it offers an ability to discover novel, \textit{previously unknown} recurrent behaviors offline and in the automated manner.

\section{Contributions}
While the detailed list of contributions has been provided in Section \ref{section_contributions}, to summarize, I would like to emphasize two significant outcomes of my research.

First is the novel generic algorithm for interpretable time series classification which is yet to be used by the data mining community. Mining time series data will be an important area of research in coming years because of the growing ubiquity of time series. I expect SAX-VSM to play important role in the future development of time series data mining and serve the practitioners with valuable insights.

The second important result of my research is that despite discovering best software trajectory class-characteristic patterns, their corresponding recurrent behaviors were found difficult to interpret without the domain knowledge and understanding of the studied phenomena's context. This result emphasizes, that the software process design is inseparable from accounting for a project internal and external constraints as well as for human-specific aspects. This finding reflects the discussed in Section \ref{oss_processes} specificities of OSS processes and shall aid in the future studies design.

\section{Future work}\label{section_future_work}
A number of future directions suggests themselves. These can be divided into two categories - those that address current 
limitations of SAX-VSM and those that are concerned with the future STA-based research. Some immediate extension to the 
discussed in this dissertation work are:
\begin{itemize}
 \item \textit{\textbf{SAX-VSM ranking schema improvement.}} This addresses the possibility of a single software trajectory study, 
 the two classes patterns ranking problem, and the patterns numerosity. 
 Based on my current experience with the application of grammatical inference to discretized time series \cite{grammarviz2}, 
 I plan to develop a threshold-based extension of the SAX-VSM weighting schema, 
 explore the possibility of a relevance-feedback algorithm application \cite{intro_ir_Manning}, 
 and to implement a similar to the MDL principle \cite{mdl} solution based on the minimal grammar size.
 \item \textit{\textbf{Variable-length characteristic pattern discovery.}} This addresses the fixed sliding window length. 
 It is possible that the best class-characteristic patterns have different lengths among classes, moreover, the capacity to 
 work with variable length patterns should mitigate for the discussed in Section \ref{sta_limitations} effect of the class-characteristic pattern elimination by \textbf{idf}. Based on the previous application of grammatical inference to time series \cite{grammarviz}, and my own work \cite{grammarviz2}, an extension of SAX-VSM was developed and currently being evaluated \cite{saxvsm2}.
 \item \textit{\textbf{Multivariate software trajectories mining.}} As I have pointed out in Section \ref{section_sta_overview}, 
 it is highly desirable to extend STA capabilities to multivariate trajectories analysis. This direction was previously explored 
 by  Ord\'{o}\={n}ez et al in \cite{oates1, oates2} and the proposed solution can be used.
 \item \textit{\textbf{In-depth study of a software project.}} This shall address the discovered recurrent behavior
 interpretation shortcoming and to allow a thorough evaluation of the proposed methodology through online interactions with the development team and project managers. 
\end{itemize}

I expect this thesis will continue to play important role in the future development of time series data mining and serve the practitioners in the field of software repository mining with valuable insights into this fascinating area of research.
