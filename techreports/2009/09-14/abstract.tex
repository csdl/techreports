\begin{abstract}
%Abstract goes here if needed.
Software development activity is carried out by individuals and groups of people, and 
it is widely recognized, that the success of every software project is heavily affected by 
complexity and unpredictability of individual and collaborative processes. 
In an effort to control these processes and in order
to bring projects to success, many formal models of software development activities 
were proposed in past decades. Some of these models, after adoption by software development community,
proven to be efficient and being adopted as industrial standards, such as
CMMI, ISO, and SPICE; some of the models are still in development or in the evaluation stages,
while others were considered inefficient and were discarded. 
Today, formal process models, standards, guidelines and recommendations provide teams 
and individuals with easy-to-understand methodologies and allow a great flexibility for 
processes for project needs.

Nevertheless, all of these models are not proven to deliver consistently. Moreover, they fail with 
equal probability. This phenomena was widely observed and firstly identified as a ``Software crisis'' 
in 1968. Despite all of the mentioned above development, the study by Standish 
Group (Rubinstein, ``Chaos Reports'', 2006) indicates that while `\textit{`Software development shops are 
doing a better job creating software than they were 12 years ago}'', still, only ``\textit{35\% of software projects 
in 2006 can be categorized as successful meaning they were completed on time, on budget and met 
user requirements}”. Ultimately, project managers, researchers and educators recognize, that despite 
all of the effort aiming on understanding and control of software process, many challenges exist 
when developing a software product.

As alternative approaches to this problem, the open-source software model (OSS) gained significant credibility,
as well as the ``software development process as a craft'' idea, which recently emerged and being successfully 
adopted. While OSS software process paradigm emphasizes extensive collaboration, frequent releases and basically
removes the boundaries between developers and customers, the craftsmanship approach is focused on 
the roles of highly motivated, creative and skilled individuals in software creation. 

However, software processes taught by these three schools, while supported by excellent research work 
and industrial success stories, 
mostly follow a conventional ``top-down'' technique - at first someone has to invent a
process, design and implement its building blocks and empirically evaluate it after. 
The problem with this approach clearly shown by van der Alast in \cite{citeulike:9758924} - 
the process inventors, limited in their scope, always assume an idealized versions of real processes 
and tend to produce ``paper lions'' - process models which are likely to be disruptive and 
unacceptable for end users, at least in their proposed form.

In my work I attempt to explore an opposite, ``down-top'' approach for the software process analysis - 
through the discovery of recurrent behaviors (habits) from software process artifacts trails.
Potentially, if there will be a mechanism in place allowing one to recognize recurrent behaviors and associate
them with successful larger processes, it is possible not only to refine existing formal models,
but to design novel software development methodologies.

My research rests on the previous contribution by many people to three research areas: software repository 
mining, time-series analysis and knowledge discovery through the text mining. By adopting current 
state-of-the art methods and practices from these domains, I have developed a methodology and implemented 
a toolkit enabling researchers and practitioners to discover significantly supported recurrent behaviors. 
By applying data mining techniques to publicly-available software change repositories I introduce the concept 
of software process and software change entropy for quantifying development effort.
Presented in this thesis results from empirical studies of open-source projects identify development effort 
and software change patterns, present analysis and visualization methodology, 
and propose avenues for further research.

\end{abstract}