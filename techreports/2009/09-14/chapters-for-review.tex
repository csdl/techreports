\documentclass[11pt,oneside]{article}
\usepackage{fullpage}

%%% Load some useful packages:
\usepackage{graphicx}
\graphicspath{{figures/}}
\usepackage{subfigure}
\usepackage{bbm}
\usepackage{tabularx}
\usepackage{setspace}
\onehalfspacing
%% Package to linebreak URLs in a sane manner.
\usepackage{url}

%% Define a new 'smallurl' style for the package that will use a smaller font.
\makeatletter
\def\url@smallurlstyle{%
  \@ifundefined{selectfont}{\def\UrlFont{\sf}}{\def\UrlFont{\small\ttfamily}}}
\makeatother
%% Now actually use the newly defined style.
\urlstyle{smallurl}

%% Define 'tinyurl' style for even smaller URLs (such as in tables)
\makeatletter
\def\url@tinyurlstyle{%
  \@ifundefined{selectfont}{\def\UrlFont{\sf}}{\def\UrlFont{\scriptsize\ttfamily}}}
\makeatother

%% Provides additional functionality for tabular environments
\usepackage{array}

%% Puts space after macros, unless followed by punctuation
\usepackage{xspace}

%% Make margins less ridiculous
\usepackage{fullpage}

%% Allows insertion of fixme notes for future work
\usepackage[footnote, nomargin]{fixme}

%% Make URLs clickable
\usepackage[colorlinks, bookmarks=true]{hyperref}

\begin{document}
\title{Contribution chapter abstract for Ph.D. dissertation: \\
       \textsc{Software Trajectory Analysis:} \\
       \textsc{An empirically based method for automated software process discovery} \\
       \author{Pavel Senin \\
               Collaborative Software Development Laboratory \\
               Department of Information and Computer Sciences \\
               University of Hawaii \\[0.3cm]
               \texttt{senin@hawaii.edu} \\[0.3cm]
               CSDL Technical Report 09-13 \\
               \url{http://csdl.ics.hawaii.edu/techreports/09-13/09-13.pdf}
       }
       \date{February 2013}
}
\maketitle

\clearpage


\section{Introduction}
Software development is a unique area of engineering blessed with having a very little or no 
associated cost with materials and fabrication which dominate other areas of engineering. 
However, ironically, it is suffering from the costs and challenges associated with continuous 
re-design of the software design processes, which is rarely seen at all in other engineering 
areas. 

In order to efficiently deal with this issue, many software development methodologies 
were proposed up to date. For example, the Waterfall Model process proposes a sequential 
pattern in which developers first create a Requirements document, then create a Design, 
then create an Implementation, and finally develop Tests. 
The Test Driven Development process proposes an iterative pattern in which the developer 
must first write a test case, then write the code to implement that test case, then refactor 
the system for maximum clarity and minimal code duplication. One problem with the 
traditional top-down approach to software process design is that it requires
the developer or manager to notice a recurrent pattern of behavior in the first place 
\cite{citeulike:5043104}. 

In my research, I have explored a possibility of an alternative approach to software 
process design through automated discovery of recurrent behaviors, which, in turn,
offer insights on the performed processes. For this, I have designed a process analysis 
methodology built upon a novel technique for knowledge discovery from temporal data. 
These two pieces - the novel generic technique of knowledge discovery from temporal data,
and explorative study of its application to software development artifacts trails are
the contributions of my thesis.

\subsection{Temporal data mining}
Many data mining tasks are based on datasets containing sequential characteristics, such
as web search queries, medical monitoring data, motion capture records, and astronomical
observations. In these and many other applications, a time series is a concise yet expressive
representation. A wealth of current data mining research on time series is focused on providing exact solutions in such small datasets. However, advances in storage techniques and
the increasing ubiquity of distributed systems make realistic time series datasets orders of
magnitude larger than the size that most of those solutions can handle due to computational
resource limitations. On the other hand, proposed approximate solutions such as dimensionality reduction and sampling suffer from two drawbacks: they do not adapt to available
computational resources and they often require complicated parameter tuning to produce high
quality results.

aTo alleviate This phenomena still poorly understood and thus, there is nhas driven software process research 
This e reason for this continuous d

Its been \cite{citeulike:11061107}
feature of software engineering, recognition includes two fundamental tasks: description and classification. Given
a process, a recognition system first generates a description of it (i.e., reconstructs the process in
full or partially) and then classifies the process based on that description (i.e., the recognition).

Two general approaches for implementing pattern recognition systems, statistical and structural, employ differ-
ent techniques for description and classification. Statistical approaches to pattern recognition use
decision-theoretic concepts to discriminate among objects belonging to different groups based upon
their quantitative features. Structural approaches to pattern recognition use syntactic grammars
to discriminate among objects belonging to different groups based upon the arrangement of their
morphological (i.e., shape-based or structural) features. Hybrid approaches to pattern recognition
combine aspects of both statistical and structural pattern recognition.

Many engineering, scientific, and production fields (such as movie-making or advertisement) have explicit 
and formalized design processes which are well studied to at least some degree. In contrast, in software 
development we are treating the process of design itself as a thing to be designed and, potentially, 
re-designed along the way. While there are ``best practices'', they prone to fail and it is commonplace
to alter these through combination or a systematic change.



Software artifacts are abundant and thought to carry a significant amount of information about performed 
processes.

However, the vast majority of artifacts are concerned about the software itself and largely associated 
with a specific development methodology. Examples of such artifacts are design documents, use cases, class 
diagrams and requirements, user manuals, etc. The payload of this artifacts aids in understanding of 
a function, architecture, and the design of software, while carrying a very little information about the 
applied effort and underlying behaviors. Due to this fact, I put such artifacts outside of this thesis
immediate attention.

What is studied in this work, is the informational content of software development process byproducts 
which accompany software change. It is long known that change Change not only provides an evidence about performed activities, and, potentially, 
carry an informational load about
recurrent behaviors. Such artifacts span in time, as behaviors do, and usually reflect both: the applied 
effort (process), and the evolution of the software itself. 
Examples of such artifacts are source code changes, bug reports, and developers discussions.

Note, however, that developers do not intentionally create these artifacts to enable research, or to keep 
things in some order - mainly, these artifacts are the pure byproduct helping to the development of a 
software project. Thus, we must assume here, that this data is inconsistent, that any kind of annotations 
used by developers might be erroneous, and the amount of disclosed information could simply be not enough
to determine the actual generative behavior - which ultimately leads to uncertainty of any claims about
process correctness, ``productivity'', or any other performance-related metrics. 

The focus of my work is to explore the informational content of software process artifacts designing 
a toolkit capable to handle the discovery of recurrent behaviors automatically. Ideally, such a toolkit 
must have following properties:
\begin{itemize}
 \item it must be Effective: the reported findings, with respect to behaviors reconstruction, must agree 
 with human intuition.
 \item it should be Scalable: currently software process artifact trails for a single project could easily 
       grow beyond dozens of gigabytes, thus, the computation technique should ideally be able to utilize 
       parallelization and be capable to pre-compute intermediate results alleviating the overall space-time 
       complexity to enable an online (fast turn-around) interactive mining.
 \item Efficient: the set of reported findings should not exceed a certain threshold simly becoming an 
        overwhelming stream of spurious facts.
\end{itemize}


\section{Methodology}
Given multiple trails of software process artifacts, how to find recurrent behaviors? In this chapter, 
I will describe the design and an implementation of research methodology employed in Software Trajectory 
Analysis. In short, the implemented approach enables aggregation, indexing and mining of software 
artifact trails allowing the discovery of recurrent behaviors. 

\section{Temporal attributes of software process artifacts}
The close examination and analysis of temporal dynamics of artifact-generating events laying the foundation 
of STA methodology. The extraction of software process-related metrics, their temporal partitioning and the
ability of finding the relevant information is the 



\end{document}
