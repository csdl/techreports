\chapter{Software Process Discovery}
\section{Software process recovery from SCM system}
As Ball et al. \cite{citeulike:9004378} and Zimmermann \& WeiBgerber
\cite{citeulike:5058462} point out - all of the contemporary version control
systems provide considerably large amount of auxiliary information about
software change. In particular, version control system, when coupled with a
mailing lists and (or) bug and issue tracking system is capable of providing
information \textit{who} changed \textit{what} and \textit{why}. Which seems to
be a fair amount of information needed for one's opinion about the change. It is
possible to get an overall understanding of the change necessity through the
analysis of bug and issue reports. Version control itself provides quantitative
data about files and LOC and changed, added or deleted. The analysis of code
snapshots (versions) allows to quantify the change in terms of various software
metrics like complexity, cohesion etc. When considered in time all this data
provides a solid background for a software evolution research.

However, what is very difficult to know from any contemporary SCM system is that
a software process behind the changes. Nevertheless many research in the field
of MSR was done in order to shed a light on the software process itself.
\cite{citeulike:9007622} There are only traces of such information present in
version control transactions. In order to recover some insights about the
performed software process information behind a software change statistics
behind the change can show us some behavioral patterns blanks in the single
transactions can be restored by statistics
outliers effect can be diminished by statistics

