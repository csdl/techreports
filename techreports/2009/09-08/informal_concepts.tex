 Appendix: Informal Review of Statistical Concepts

    * Recall that the goal of probability theory is to determine the likelihood of a given event given a probability distribution (e.g. how likely is it to get 5,300 heads in 10,000 flips of a fair coin?).
      The goal of statistics is to determine a probability distribution given a series of observations or at least to disprove a null hypothesis (e.g. is a fair coin a reasonable model if I get 8,000 heads in 10,000 flips?).
    * In parametric statistics, one knows the form of the target probability distribution but not the value of certain parameters, e.g. coin flips are binomial but the probability of a head may be unknown.
      In non-parametric statistics, one does not know the form of the target probability distribution.
      In finance, most models are parametric (autoregression, option pricing). When models aren't, people use queries and eyeballs to figure out what to do.
    * Stationary process : one whose statistics (mean and variance) do not vary with time. Stationarity is a fundamental assumption of pairs trading and options pricing.
    * Correlation: a measure of the association between two series, e.g. the option open interest and the price of a security 5 days later. If cov(x,y) represents the covariance between x and y and sigma(x) is the standard deviation of x, then
      correlation(x,y) = cov(x,y)/(sigma(x)*sigma(y))
      so is entirely symmetric and lies always between -1 and 1.
    * Partial correlation : suppose you are looking at the one day returns of Merck and Pfizer (two drug companies). You can look at them as raw data or you can subtract out the market influence via a least squares estimate and use the correlation of the residuals.
    * Volatility : a measure of the standard deviation of the value of a variable over a specific time, e.g. the annualized standard deviation of the returns. The return at time t is ln(p(t)/p(t-1)). This is a critical parameter in options pricing, because it determines the probability that a price will exceed a certain price range.
    * Alpha, Beta, and Regression: suppose we estimate the relationship between the percentage change in price of some stock S vs. the percentage change in some market index M using a best fit (least squares) linear relationship:
      s = a + bm
      Then the parameter alpha (a) is the change in S independent of M and beta (b) is the slope of the best fit line. A riskless investment has a positive alpha and a zero beta, but most investments have a zero alpha and a positive beta. If beta is greater than 1, then for a given change in the market, you can expect a greater change in S. If beta is negative, then S moves in the opposite direction from the market. Note that beta is different from correlation (and can be arbitrarily large or small) because it is not symmetric:
      beta = cov(S,M)/(sigma(M)*sigma(M))
    * ANOVA: analysis of variance in cases when there is no missing data. This is used to model situations in which several factors can play a role and one wants to tease out a probabilistic model that describes their interaction. For example, product, location and customer income may be factors that influence buying behavior. ANOVA helps to figure out how to weight each one. More significant variants of this include principal components analysis and factor analysis . In finance, one might use one of these to figure out what determines the price movement of a stock (perhaps half general market movement, one third interest rates, etc.). In psychology, one can ask a person 100 questions and then categorize the person according to a weighted sum of a few questions.
    * Autoregression: a statistical model which predicts future values from one or more previous ones. This generalizes trend forecasting as used to predict sales. Financial traders use this sparingly since models that look at the recent past often just follow a short term trend. As one trader put it: ``they follow a trend and are always a day late and many dollars short.'' In general, regression of y on x is a determination of how y depends on x.
    * Maximum likelihood method: suppose you are given a training set consisting of observations and the categories to which the observations belong. The maximum likelihood method selects the probability distribution that best explains the training set. For example, if you toss a coin 10,000 times and observe that heads comes up 8,000, you assign a probability to the heads that maximizes the probability of this event. This will make the probability of heads be greater than 1/2. In finance, the maximum likelihood method is often used for forecasting based on previously seen patterns.
    * Regularization A technique for smoothing a function to make it have nice mathematical properties such as differentiability. Moving averages are an example of regularization.
    * Bootstrapping (i) Divide the training set (set of (observation, category) pairs) into pieces. (ii) Infer the model from some pieces. (iii) Test it on the other pieces. 