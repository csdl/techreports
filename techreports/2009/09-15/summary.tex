%%%%%%%%%%%%%%%%%%%%%%%%%%%%%% -*- Mode: Latex -*- %%%%%%%%%%%%%%%%%%%%%%%%%%%%
%% summary.tex -- 
%% Author          : Philip Johnson
%% Created On      : Tue Mar 31 11:42:10 2009
%% Last Modified By: Philip Johnson
%% Last Modified On: Wed Dec 16 16:20:25 2009
%% RCS: $Id$
%%%%%%%%%%%%%%%%%%%%%%%%%%%%%%%%%%%%%%%%%%%%%%%%%%%%%%%%%%%%%%%%%%%%%%%%%%%%%%%
%%   Copyright (C) 2009 
%%%%%%%%%%%%%%%%%%%%%%%%%%%%%%%%%%%%%%%%%%%%%%%%%%%%%%%%%%%%%%%%%%%%%%%%%%%%%%%
%% 

\section*{Project Summary}
\renewcommand{\thepage} {A--\arabic{page}}

%% {\em The proposal must contain a summary of the proposed activity suitable for
%% publication, not more than one page in length. It should not be an abstract
%% of the proposal, but rather a self-contained description of the activity
%% that would result if the proposal were funded. The summary should be
%% written in the third person and include a statement of objectives and
%% methods to be employed. It must clearly address in separate statements
%% (within the one-page summary):

%% (1) the intellectual merit of the proposed activity; and

%% (2)the broader impacts resulting from the proposed activity. 

%% It should be informative to other persons working in the same or related
%% fields and, insofar as possible, understandable to a scientifically or
%% technically literate lay reader. Proposals that do not separately address
%% both merit review criteria within the one-page Project Summary will be
%% returned without review.

% What is the intellectual merit of the proposed activity?
% How important is the proposed activity to advancing knowledge and
% understanding within its own field or across different fields? How well
% qualified is the proposer (individual or team) to conduct the project? (If
% appropriate, the reviewer will comment on the quality of the prior work.)
% To what extent does the proposed activity suggest and explore creative,
% original, or potentially transformative concepts? How well conceived and
% organized is the proposed activity? Is there sufficient access to
% resources?

% What are the broader impacts of the proposed activity?
% How well does the activity advance discovery and understanding while
% promoting teaching, training, and learning? How well does the proposed
% activity broaden the participation of underrepresented groups (e.g.,
% gender, ethnicity, disability, geographic, etc.)? To what extent will it
% enhance the infrastructure for research and education, such as facilities,
% instrumentation, networks, and partnerships? Will the results be
% disseminated broadly to enhance scientific and technological understanding?
% What may be the benefits of the proposed activity to society? 

% NSF staff also will give careful consideration to the following in making
% funding decisions:

% Integration of Research and Education
% One of the principal strategies in support of NSF's goals is to foster
% integration of research and education through the programs, projects, and
% activities it supports at academic and research institutions. These
% institutions provide abundant opportunities where individuals may
% concurrently assume responsibilities as researchers, educators, and
% students and where all can engage in joint efforts that infuse education
% with the excitement of discovery and enrich research through the diversity
% of learning perspectives.


% Integrating Diversity into NSF Programs, Projects, and Activities
% Broadening opportunities and enabling the participation of all citizens --
% women and men, underrepresented minorities, and persons with disabilities
% -- is essential to the health and vitality of science and engineering. NSF
% is committed to this principle of diversity and deems it central to the
% programs, projects, and activities it considers and supports.

\noindent {\bf Overview.}  The ``Smart Grid'' represents a new vision for
the electrical infrastructure of the United States, whose goals include
more active participation by consumers, new generation and storage options
including renewable energy, and new products, services, and markets.  To
reach its full potential, the Smart Grid must provide information to
consumers in a way that enables positive, sustained changes to
energy-related behaviors.  

The central question to be pursued in this research is: {\em What kinds of
  information, provided in what ways and at what times, enables consumers
  to make positive, sustained changes to their energy consumption
  behaviors?}  Prior research indicates that such changes can potentially
be motivated by an appropriate combination of personalized information,
general and specific commitments, achievable goals, social reinforcement,
feedback, and financial incentives.

\medskip

\noindent {\bf Intellectual Merit.} This project will develop a collection
of open source components called WattBlocks, which  will provide novel and
useful scientific infrastructure for investigating the ways in which
energy-related information can affect human behavior. The project will also
develop eSpheres, a novel social networking application that provides users
with access to energy-related communities at configurable levels of
scale. The combination of WattBlocks and eSpheres will lower the technological 
efforts required for empirical, replicable studies of human energy-related
behaviors.    

The project will use this infrastructure in a series of two case studies, one
involving campus dormitory energy competitions and one involving community
home energy challenges.  The project will investigate a number of
important research questions, including: (1) What are the requirements
for consumer-facing, open source, scientific energy information
infrastructure? (2) What are the strengths and weaknesses of a dedicated
social network technology like eSpheres for energy behavior change? (3)
What combination of behavioral change motivators, under what conditions,
induce positive change? (4) What factors influence the sustainability of
these changes? (5) What is the influence of energy data feedback latency
(i.e. 1 minute, 15 minutes, 1 hour, 1 day) on behavioral change? 

The research group is well-qualified for the development challenges, as
they have over ten years experience in the development of open source,
component-based software for software engineering process and product data
through the successful Hackystat Project.  They also have broad prior experience
in empirical experimentation, data collection, and analysis.  The project will
gain additional strength from partnerships with other University of Hawaii
groups (REIS, Sustainable UH) and community groups (Kanu Hawaii, Blue
Planet Foundation).

\medskip 

\noindent{\bf Broader Impacts.}  
This project will serve underrepresented populations, as the University of
Hawaii is in an EPSCOR state. Approximately 84\% of undergraduates at the
University of Hawaii are minorities. WattBlocks and eSphere will be
released as open source software, providing new and useful infrastructure
for others interested in investigating energy-related human behavior.
Finally, the resulting insights can inform the design of consumer-facing
Smart Grid information systems, with potentially significant cost-saving or
service benefits for the U.S. population.

\medskip

\noindent {\bf Key Words:} smart grid; human behavior; social networks;
energy.




