%%%%%%%%%%%%%%%%%%%%%%%%%%%%%% -*- Mode: Latex -*- %%%%%%%%%%%%%%%%%%%%%%%%%%%%
%% 06-06.tex -- ICSE 2007 submission
%% Author          : Philip Johnson
%% Created On      : Wed Dec 08 2004
%% Last Modified By: 
%% Last Modified On: Thu Aug 10 09:36:13 2006
%% RCS: $Id$
%%%%%%%%%%%%%%%%%%%%%%%%%%%%%%%%%%%%%%%%%%%%%%%%%%%%%%%%%%%%%%%%%%%%%%%%%%%%%%%
%%   Copyright (C) 2002 Philip Johnson
%%%%%%%%%%%%%%%%%%%%%%%%%%%%%%%%%%%%%%%%%%%%%%%%%%%%%%%%%%%%%%%%%%%%%%%%%%%%%%%
%% 

\documentclass[10pt,twocolumn]{article} 
% Psfig/TeX 
\def\PsfigVersion{1.9}
% dvips version
%
% All psfig/tex software, documentation, and related files
% in this distribution of psfig/tex are 
% Copyright 1987, 1988, 1991 Trevor J. Darrell
%
% Permission is granted for use and non-profit distribution of psfig/tex 
% providing that this notice is clearly maintained. The right to
% distribute any portion of psfig/tex for profit or as part of any commercial
% product is specifically reserved for the author(s) of that portion.
%
% *** Feel free to make local modifications of psfig as you wish,
% *** but DO NOT post any changed or modified versions of ``psfig''
% *** directly to the net. Send them to me and I'll try to incorporate
% *** them into future versions. If you want to take the psfig code 
% *** and make a new program (subject to the copyright above), distribute it, 
% *** (and maintain it) that's fine, just don't call it psfig.
%
% Bugs and improvements to trevor@media.mit.edu.
%
% Thanks to Greg Hager (GDH) and Ned Batchelder for their contributions
% to the original version of this project.
%
% Modified by J. Daniel Smith on 9 October 1990 to accept the
% %%BoundingBox: comment with or without a space after the colon.  Stole
% file reading code from Tom Rokicki's EPSF.TEX file (see below).
%
% More modifications by J. Daniel Smith on 29 March 1991 to allow the
% the included PostScript figure to be rotated.  The amount of
% rotation is specified by the "angle=" parameter of the \psfig command.
%
% Modified by Robert Russell on June 25, 1991 to allow users to specify
% .ps filenames which don't yet exist, provided they explicitly provide
% boundingbox information via the \psfig command. Note: This will only work
% if the "file=" parameter follows all four "bb???=" parameters in the
% command. This is due to the order in which psfig interprets these params.
%
%  3 Jul 1991	JDS	check if file already read in once
%  4 Sep 1991	JDS	fixed incorrect computation of rotated
%			bounding box
% 25 Sep 1991	GVR	expanded synopsis of \psfig
% 14 Oct 1991	JDS	\fbox code from LaTeX so \psdraft works with TeX
%			changed \typeout to \ps@typeout
% 17 Oct 1991	JDS	added \psscalefirst and \psrotatefirst
%

% From: gvr@cs.brown.edu (George V. Reilly)
%
% \psdraft	draws an outline box, but doesn't include the figure
%		in the DVI file.  Useful for previewing.
%
% \psfull	includes the figure in the DVI file (default).
%
% \psscalefirst width= or height= specifies the size of the figure
% 		before rotation.
% \psrotatefirst (default) width= or height= specifies the size of the
% 		 figure after rotation.  Asymetric figures will
% 		 appear to shrink.
%
% \psfigurepath#1	sets the path to search for the figure
%
% \psfig
% usage: \psfig{file=, figure=, height=, width=,
%			bbllx=, bblly=, bburx=, bbury=,
%			rheight=, rwidth=, clip=, angle=, silent=}
%
%	"file" is the filename.  If no path name is specified and the
%		file is not found in the current directory,
%		it will be looked for in directory \psfigurepath.
%	"figure" is a synonym for "file".
%	By default, the width and height of the figure are taken from
%		the BoundingBox of the figure.
%	If "width" is specified, the figure is scaled so that it has
%		the specified width.  Its height changes proportionately.
%	If "height" is specified, the figure is scaled so that it has
%		the specified height.  Its width changes proportionately.
%	If both "width" and "height" are specified, the figure is scaled
%		anamorphically.
%	"bbllx", "bblly", "bburx", and "bbury" control the PostScript
%		BoundingBox.  If these four values are specified
%               *before* the "file" option, the PSFIG will not try to
%               open the PostScript file.
%	"rheight" and "rwidth" are the reserved height and width
%		of the figure, i.e., how big TeX actually thinks
%		the figure is.  They default to "width" and "height".
%	The "clip" option ensures that no portion of the figure will
%		appear outside its BoundingBox.  "clip=" is a switch and
%		takes no value, but the `=' must be present.
%	The "angle" option specifies the angle of rotation (degrees, ccw).
%	The "silent" option makes \psfig work silently.
%

% check to see if macros already loaded in (maybe some other file says
% "\input psfig") ...
\ifx\undefined\psfig\else\endinput\fi

%
% from a suggestion by eijkhout@csrd.uiuc.edu to allow
% loading as a style file. Changed to avoid problems
% with amstex per suggestion by jbence@math.ucla.edu

\let\LaTeXAtSign=\@
\let\@=\relax
\edef\psfigRestoreAt{\catcode`\@=\number\catcode`@\relax}
%\edef\psfigRestoreAt{\catcode`@=\number\catcode`@\relax}
\catcode`\@=11\relax
\newwrite\@unused
\def\ps@typeout#1{{\let\protect\string\immediate\write\@unused{#1}}}
\ps@typeout{psfig/tex \PsfigVersion}

%% Here's how you define your figure path.  Should be set up with null
%% default and a user useable definition.

\def\figurepath{./}
\def\psfigurepath#1{\edef\figurepath{#1}}

%
% @psdo control structure -- similar to Latex @for.
% I redefined these with different names so that psfig can
% be used with TeX as well as LaTeX, and so that it will not 
% be vunerable to future changes in LaTeX's internal
% control structure,
%
\def\@nnil{\@nil}
\def\@empty{}
\def\@psdonoop#1\@@#2#3{}
\def\@psdo#1:=#2\do#3{\edef\@psdotmp{#2}\ifx\@psdotmp\@empty \else
    \expandafter\@psdoloop#2,\@nil,\@nil\@@#1{#3}\fi}
\def\@psdoloop#1,#2,#3\@@#4#5{\def#4{#1}\ifx #4\@nnil \else
       #5\def#4{#2}\ifx #4\@nnil \else#5\@ipsdoloop #3\@@#4{#5}\fi\fi}
\def\@ipsdoloop#1,#2\@@#3#4{\def#3{#1}\ifx #3\@nnil 
       \let\@nextwhile=\@psdonoop \else
      #4\relax\let\@nextwhile=\@ipsdoloop\fi\@nextwhile#2\@@#3{#4}}
\def\@tpsdo#1:=#2\do#3{\xdef\@psdotmp{#2}\ifx\@psdotmp\@empty \else
    \@tpsdoloop#2\@nil\@nil\@@#1{#3}\fi}
\def\@tpsdoloop#1#2\@@#3#4{\def#3{#1}\ifx #3\@nnil 
       \let\@nextwhile=\@psdonoop \else
      #4\relax\let\@nextwhile=\@tpsdoloop\fi\@nextwhile#2\@@#3{#4}}
% 
% \fbox is defined in latex.tex; so if \fbox is undefined, assume that
% we are not in LaTeX.
% Perhaps this could be done better???
\ifx\undefined\fbox
% \fbox code from modified slightly from LaTeX
\newdimen\fboxrule
\newdimen\fboxsep
\newdimen\ps@tempdima
\newbox\ps@tempboxa
\fboxsep = 3pt
\fboxrule = .4pt
\long\def\fbox#1{\leavevmode\setbox\ps@tempboxa\hbox{#1}\ps@tempdima\fboxrule
    \advance\ps@tempdima \fboxsep \advance\ps@tempdima \dp\ps@tempboxa
   \hbox{\lower \ps@tempdima\hbox
  {\vbox{\hrule height \fboxrule
          \hbox{\vrule width \fboxrule \hskip\fboxsep
          \vbox{\vskip\fboxsep \box\ps@tempboxa\vskip\fboxsep}\hskip 
                 \fboxsep\vrule width \fboxrule}
                 \hrule height \fboxrule}}}}
\fi
%
%%%%%%%%%%%%%%%%%%%%%%%%%%%%%%%%%%%%%%%%%%%%%%%%%%%%%%%%%%%%%%%%%%%
% file reading stuff from epsf.tex
%   EPSF.TEX macro file:
%   Written by Tomas Rokicki of Radical Eye Software, 29 Mar 1989.
%   Revised by Don Knuth, 3 Jan 1990.
%   Revised by Tomas Rokicki to accept bounding boxes with no
%      space after the colon, 18 Jul 1990.
%   Portions modified/removed for use in PSFIG package by
%      J. Daniel Smith, 9 October 1990.
%
\newread\ps@stream
\newif\ifnot@eof       % continue looking for the bounding box?
\newif\if@noisy        % report what you're making?
\newif\if@atend        % %%BoundingBox: has (at end) specification
\newif\if@psfile       % does this look like a PostScript file?
%
% PostScript files should start with `%!'
%
{\catcode`\%=12\global\gdef\epsf@start{%!}}
\def\epsf@PS{PS}
%
\def\epsf@getbb#1{%
%
%   The first thing we need to do is to open the
%   PostScript file, if possible.
%
\openin\ps@stream=#1
\ifeof\ps@stream\ps@typeout{Error, File #1 not found}\else
%
%   Okay, we got it. Now we'll scan lines until we find one that doesn't
%   start with %. We're looking for the bounding box comment.
%
   {\not@eoftrue \chardef\other=12
    \def\do##1{\catcode`##1=\other}\dospecials \catcode`\ =10
    \loop
       \if@psfile
	  \read\ps@stream to \epsf@fileline
       \else{
	  \obeyspaces
          \read\ps@stream to \epsf@tmp\global\let\epsf@fileline\epsf@tmp}
       \fi
       \ifeof\ps@stream\not@eoffalse\else
%
%   Check the first line for `%!'.  Issue a warning message if its not
%   there, since the file might not be a PostScript file.
%
       \if@psfile\else
       \expandafter\epsf@test\epsf@fileline:. \\%
       \fi
%
%   We check to see if the first character is a % sign;
%   if so, we look further and stop only if the line begins with
%   `%%BoundingBox:' and the `(atend)' specification was not found.
%   That is, the only way to stop is when the end of file is reached,
%   or a `%%BoundingBox: llx lly urx ury' line is found.
%
          \expandafter\epsf@aux\epsf@fileline:. \\%
       \fi
   \ifnot@eof\repeat
   }\closein\ps@stream\fi}%
%
% This tests if the file we are reading looks like a PostScript file.
%
\long\def\epsf@test#1#2#3:#4\\{\def\epsf@testit{#1#2}
			\ifx\epsf@testit\epsf@start\else
\ps@typeout{Warning! File does not start with `\epsf@start'.  It may not be a PostScript file.}
			\fi
			\@psfiletrue} % don't test after 1st line
%
%   We still need to define the tricky \epsf@aux macro. This requires
%   a couple of magic constants for comparison purposes.
%
{\catcode`\%=12\global\let\epsf@percent=%\global\def\epsf@bblit{%BoundingBox}}
%
%
%   So we're ready to check for `%BoundingBox:' and to grab the
%   values if they are found.  We continue searching if `(at end)'
%   was found after the `%BoundingBox:'.
%
\long\def\epsf@aux#1#2:#3\\{\ifx#1\epsf@percent
   \def\epsf@testit{#2}\ifx\epsf@testit\epsf@bblit
	\@atendfalse
        \epsf@atend #3 . \\%
	\if@atend	
	   \if@verbose{
		\ps@typeout{psfig: found `(atend)'; continuing search}
	   }\fi
        \else
        \epsf@grab #3 . . . \\%
        \not@eoffalse
        \global\no@bbfalse
        \fi
   \fi\fi}%
%
%   Here we grab the values and stuff them in the appropriate definitions.
%
\def\epsf@grab #1 #2 #3 #4 #5\\{%
   \global\def\epsf@llx{#1}\ifx\epsf@llx\empty
      \epsf@grab #2 #3 #4 #5 .\\\else
   \global\def\epsf@lly{#2}%
   \global\def\epsf@urx{#3}\global\def\epsf@ury{#4}\fi}%
%
% Determine if the stuff following the %%BoundingBox is `(atend)'
% J. Daniel Smith.  Copied from \epsf@grab above.
%
\def\epsf@atendlit{(atend)} 
\def\epsf@atend #1 #2 #3\\{%
   \def\epsf@tmp{#1}\ifx\epsf@tmp\empty
      \epsf@atend #2 #3 .\\\else
   \ifx\epsf@tmp\epsf@atendlit\@atendtrue\fi\fi}


% End of file reading stuff from epsf.tex
%%%%%%%%%%%%%%%%%%%%%%%%%%%%%%%%%%%%%%%%%%%%%%%%%%%%%%%%%%%%%%%%%%%

%%%%%%%%%%%%%%%%%%%%%%%%%%%%%%%%%%%%%%%%%%%%%%%%%%%%%%%%%%%%%%%%%%%
% trigonometry stuff from "trig.tex"
\chardef\psletter = 11 % won't conflict with \begin{letter} now...
\chardef\other = 12

\newif \ifdebug %%% turn me on to see TeX hard at work ...
\newif\ifc@mpute %%% don't need to compute some values
\c@mputetrue % but assume that we do

\let\then = \relax
\def\r@dian{pt }
\let\r@dians = \r@dian
\let\dimensionless@nit = \r@dian
\let\dimensionless@nits = \dimensionless@nit
\def\internal@nit{sp }
\let\internal@nits = \internal@nit
\newif\ifstillc@nverging
\def \Mess@ge #1{\ifdebug \then \message {#1} \fi}

{ %%% Things that need abnormal catcodes %%%
	\catcode `\@ = \psletter
	\gdef \nodimen {\expandafter \n@dimen \the \dimen}
	\gdef \term #1 #2 #3%
	       {\edef \t@ {\the #1}%%% freeze parameter 1 (count, by value)
		\edef \t@@ {\expandafter \n@dimen \the #2\r@dian}%
				   %%% freeze parameter 2 (dimen, by value)
		\t@rm {\t@} {\t@@} {#3}%
	       }
	\gdef \t@rm #1 #2 #3%
	       {{%
		\count 0 = 0
		\dimen 0 = 1 \dimensionless@nit
		\dimen 2 = #2\relax
		\Mess@ge {Calculating term #1 of \nodimen 2}%
		\loop
		\ifnum	\count 0 < #1
		\then	\advance \count 0 by 1
			\Mess@ge {Iteration \the \count 0 \space}%
			\Multiply \dimen 0 by {\dimen 2}%
			\Mess@ge {After multiplication, term = \nodimen 0}%
			\Divide \dimen 0 by {\count 0}%
			\Mess@ge {After division, term = \nodimen 0}%
		\repeat
		\Mess@ge {Final value for term #1 of 
				\nodimen 2 \space is \nodimen 0}%
		\xdef \Term {#3 = \nodimen 0 \r@dians}%
		\aftergroup \Term
	       }}
	\catcode `\p = \other
	\catcode `\t = \other
	\gdef \n@dimen #1pt{#1} %%% throw away the ``pt''
}

\def \Divide #1by #2{\divide #1 by #2} %%% just a synonym

\def \Multiply #1by #2%%% allows division of a dimen by a dimen
       {{%%% should really freeze parameter 2 (dimen, passed by value)
	\count 0 = #1\relax
	\count 2 = #2\relax
	\count 4 = 65536
	\Mess@ge {Before scaling, count 0 = \the \count 0 \space and
			count 2 = \the \count 2}%
	\ifnum	\count 0 > 32767 %%% do our best to avoid overflow
	\then	\divide \count 0 by 4
		\divide \count 4 by 4
	\else	\ifnum	\count 0 < -32767
		\then	\divide \count 0 by 4
			\divide \count 4 by 4
		\else
		\fi
	\fi
	\ifnum	\count 2 > 32767 %%% while retaining reasonable accuracy
	\then	\divide \count 2 by 4
		\divide \count 4 by 4
	\else	\ifnum	\count 2 < -32767
		\then	\divide \count 2 by 4
			\divide \count 4 by 4
		\else
		\fi
	\fi
	\multiply \count 0 by \count 2
	\divide \count 0 by \count 4
	\xdef \product {#1 = \the \count 0 \internal@nits}%
	\aftergroup \product
       }}

\def\r@duce{\ifdim\dimen0 > 90\r@dian \then   % sin(x+90) = sin(180-x)
		\multiply\dimen0 by -1
		\advance\dimen0 by 180\r@dian
		\r@duce
	    \else \ifdim\dimen0 < -90\r@dian \then  % sin(-x) = sin(360+x)
		\advance\dimen0 by 360\r@dian
		\r@duce
		\fi
	    \fi}

\def\Sine#1%
       {{%
	\dimen 0 = #1 \r@dian
	\r@duce
	\ifdim\dimen0 = -90\r@dian \then
	   \dimen4 = -1\r@dian
	   \c@mputefalse
	\fi
	\ifdim\dimen0 = 90\r@dian \then
	   \dimen4 = 1\r@dian
	   \c@mputefalse
	\fi
	\ifdim\dimen0 = 0\r@dian \then
	   \dimen4 = 0\r@dian
	   \c@mputefalse
	\fi
%
	\ifc@mpute \then
        	% convert degrees to radians
		\divide\dimen0 by 180
		\dimen0=3.141592654\dimen0
%
		\dimen 2 = 3.1415926535897963\r@dian %%% a well-known constant
		\divide\dimen 2 by 2 %%% we only deal with -pi/2 : pi/2
		\Mess@ge {Sin: calculating Sin of \nodimen 0}%
		\count 0 = 1 %%% see power-series expansion for sine
		\dimen 2 = 1 \r@dian %%% ditto
		\dimen 4 = 0 \r@dian %%% ditto
		\loop
			\ifnum	\dimen 2 = 0 %%% then we've done
			\then	\stillc@nvergingfalse 
			\else	\stillc@nvergingtrue
			\fi
			\ifstillc@nverging %%% then calculate next term
			\then	\term {\count 0} {\dimen 0} {\dimen 2}%
				\advance \count 0 by 2
				\count 2 = \count 0
				\divide \count 2 by 2
				\ifodd	\count 2 %%% signs alternate
				\then	\advance \dimen 4 by \dimen 2
				\else	\advance \dimen 4 by -\dimen 2
				\fi
		\repeat
	\fi		
			\xdef \sine {\nodimen 4}%
       }}

% Now the Cosine can be calculated easily by calling \Sine
\def\Cosine#1{\ifx\sine\UnDefined\edef\Savesine{\relax}\else
		             \edef\Savesine{\sine}\fi
	{\dimen0=#1\r@dian\advance\dimen0 by 90\r@dian
	 \Sine{\nodimen 0}
	 \xdef\cosine{\sine}
	 \xdef\sine{\Savesine}}}	      
% end of trig stuff
%%%%%%%%%%%%%%%%%%%%%%%%%%%%%%%%%%%%%%%%%%%%%%%%%%%%%%%%%%%%%%%%%%%%

\def\psdraft{
	\def\@psdraft{0}
	%\ps@typeout{draft level now is \@psdraft \space . }
}
\def\psfull{
	\def\@psdraft{100}
	%\ps@typeout{draft level now is \@psdraft \space . }
}

\psfull

\newif\if@scalefirst
\def\psscalefirst{\@scalefirsttrue}
\def\psrotatefirst{\@scalefirstfalse}
\psrotatefirst

\newif\if@draftbox
\def\psnodraftbox{
	\@draftboxfalse
}
\def\psdraftbox{
	\@draftboxtrue
}
\@draftboxtrue

\newif\if@prologfile
\newif\if@postlogfile
\def\pssilent{
	\@noisyfalse
}
\def\psnoisy{
	\@noisytrue
}
\psnoisy
%%% These are for the option list.
%%% A specification of the form a = b maps to calling \@p@@sa{b}
\newif\if@bbllx
\newif\if@bblly
\newif\if@bburx
\newif\if@bbury
\newif\if@height
\newif\if@width
\newif\if@rheight
\newif\if@rwidth
\newif\if@angle
\newif\if@clip
\newif\if@verbose
\def\@p@@sclip#1{\@cliptrue}


\newif\if@decmpr

%%% GDH 7/26/87 -- changed so that it first looks in the local directory,
%%% then in a specified global directory for the ps file.
%%% RPR 6/25/91 -- changed so that it defaults to user-supplied name if
%%% boundingbox info is specified, assuming graphic will be created by
%%% print time.
%%% TJD 10/19/91 -- added bbfile vs. file distinction, and @decmpr flag

\def\@p@@sfigure#1{\def\@p@sfile{null}\def\@p@sbbfile{null}
	        \openin1=#1.bb
		\ifeof1\closein1
	        	\openin1=\figurepath#1.bb
			\ifeof1\closein1
			        \openin1=#1
				\ifeof1\closein1%
				       \openin1=\figurepath#1
					\ifeof1
					   \ps@typeout{Error, File #1 not found}
						\if@bbllx\if@bblly
				   		\if@bburx\if@bbury
			      				\def\@p@sfile{#1}%
			      				\def\@p@sbbfile{#1}%
							\@decmprfalse
				  	   	\fi\fi\fi\fi
					\else\closein1
				    		\def\@p@sfile{\figurepath#1}%
				    		\def\@p@sbbfile{\figurepath#1}%
						\@decmprfalse
	                       		\fi%
			 	\else\closein1%
					\def\@p@sfile{#1}
					\def\@p@sbbfile{#1}
					\@decmprfalse
			 	\fi
			\else
				\def\@p@sfile{\figurepath#1}
				\def\@p@sbbfile{\figurepath#1.bb}
				\@decmprtrue
			\fi
		\else
			\def\@p@sfile{#1}
			\def\@p@sbbfile{#1.bb}
			\@decmprtrue
		\fi}

\def\@p@@sfile#1{\@p@@sfigure{#1}}

\def\@p@@sbbllx#1{
		%\ps@typeout{bbllx is #1}
		\@bbllxtrue
		\dimen100=#1
		\edef\@p@sbbllx{\number\dimen100}
}
\def\@p@@sbblly#1{
		%\ps@typeout{bblly is #1}
		\@bbllytrue
		\dimen100=#1
		\edef\@p@sbblly{\number\dimen100}
}
\def\@p@@sbburx#1{
		%\ps@typeout{bburx is #1}
		\@bburxtrue
		\dimen100=#1
		\edef\@p@sbburx{\number\dimen100}
}
\def\@p@@sbbury#1{
		%\ps@typeout{bbury is #1}
		\@bburytrue
		\dimen100=#1
		\edef\@p@sbbury{\number\dimen100}
}
\def\@p@@sheight#1{
		\@heighttrue
		\dimen100=#1
   		\edef\@p@sheight{\number\dimen100}
		%\ps@typeout{Height is \@p@sheight}
}
\def\@p@@swidth#1{
		%\ps@typeout{Width is #1}
		\@widthtrue
		\dimen100=#1
		\edef\@p@swidth{\number\dimen100}
}
\def\@p@@srheight#1{
		%\ps@typeout{Reserved height is #1}
		\@rheighttrue
		\dimen100=#1
		\edef\@p@srheight{\number\dimen100}
}
\def\@p@@srwidth#1{
		%\ps@typeout{Reserved width is #1}
		\@rwidthtrue
		\dimen100=#1
		\edef\@p@srwidth{\number\dimen100}
}
\def\@p@@sangle#1{
		%\ps@typeout{Rotation is #1}
		\@angletrue
%		\dimen100=#1
		\edef\@p@sangle{#1} %\number\dimen100}
}
\def\@p@@ssilent#1{ 
		\@verbosefalse
}
\def\@p@@sprolog#1{\@prologfiletrue\def\@prologfileval{#1}}
\def\@p@@spostlog#1{\@postlogfiletrue\def\@postlogfileval{#1}}
\def\@cs@name#1{\csname #1\endcsname}
\def\@setparms#1=#2,{\@cs@name{@p@@s#1}{#2}}
%
% initialize the defaults (size the size of the figure)
%
\def\ps@init@parms{
		\@bbllxfalse \@bbllyfalse
		\@bburxfalse \@bburyfalse
		\@heightfalse \@widthfalse
		\@rheightfalse \@rwidthfalse
		\def\@p@sbbllx{}\def\@p@sbblly{}
		\def\@p@sbburx{}\def\@p@sbbury{}
		\def\@p@sheight{}\def\@p@swidth{}
		\def\@p@srheight{}\def\@p@srwidth{}
		\def\@p@sangle{0}
		\def\@p@sfile{} \def\@p@sbbfile{}
		\def\@p@scost{10}
		\def\@sc{}
		\@prologfilefalse
		\@postlogfilefalse
		\@clipfalse
		\if@noisy
			\@verbosetrue
		\else
			\@verbosefalse
		\fi
}
%
% Go through the options setting things up.
%
\def\parse@ps@parms#1{
	 	\@psdo\@psfiga:=#1\do
		   {\expandafter\@setparms\@psfiga,}}
%
% Compute bb height and width
%
\newif\ifno@bb
\def\bb@missing{
	\if@verbose{
		\ps@typeout{psfig: searching \@p@sbbfile \space  for bounding box}
	}\fi
	\no@bbtrue
	\epsf@getbb{\@p@sbbfile}
        \ifno@bb \else \bb@cull\epsf@llx\epsf@lly\epsf@urx\epsf@ury\fi
}	
\def\bb@cull#1#2#3#4{
	\dimen100=#1 bp\edef\@p@sbbllx{\number\dimen100}
	\dimen100=#2 bp\edef\@p@sbblly{\number\dimen100}
	\dimen100=#3 bp\edef\@p@sbburx{\number\dimen100}
	\dimen100=#4 bp\edef\@p@sbbury{\number\dimen100}
	\no@bbfalse
}
% rotate point (#1,#2) about (0,0).
% The sine and cosine of the angle are already stored in \sine and
% \cosine.  The result is placed in (\p@intvaluex, \p@intvaluey).
\newdimen\p@intvaluex
\newdimen\p@intvaluey
\def\rotate@#1#2{{\dimen0=#1 sp\dimen1=#2 sp
%            	calculate x' = x \cos\theta - y \sin\theta
		  \global\p@intvaluex=\cosine\dimen0
		  \dimen3=\sine\dimen1
		  \global\advance\p@intvaluex by -\dimen3
% 		calculate y' = x \sin\theta + y \cos\theta
		  \global\p@intvaluey=\sine\dimen0
		  \dimen3=\cosine\dimen1
		  \global\advance\p@intvaluey by \dimen3
		  }}
\def\compute@bb{
		\no@bbfalse
		\if@bbllx \else \no@bbtrue \fi
		\if@bblly \else \no@bbtrue \fi
		\if@bburx \else \no@bbtrue \fi
		\if@bbury \else \no@bbtrue \fi
		\ifno@bb \bb@missing \fi
		\ifno@bb \ps@typeout{FATAL ERROR: no bb supplied or found}
			\no-bb-error
		\fi
		%
%\ps@typeout{BB: \@p@sbbllx, \@p@sbblly, \@p@sbburx, \@p@sbbury} 
%
% store height/width of original (unrotated) bounding box
		\count203=\@p@sbburx
		\count204=\@p@sbbury
		\advance\count203 by -\@p@sbbllx
		\advance\count204 by -\@p@sbblly
		\edef\ps@bbw{\number\count203}
		\edef\ps@bbh{\number\count204}
		%\ps@typeout{ psbbh = \ps@bbh, psbbw = \ps@bbw }
		\if@angle 
			\Sine{\@p@sangle}\Cosine{\@p@sangle}
	        	{\dimen100=\maxdimen\xdef\r@p@sbbllx{\number\dimen100}
					    \xdef\r@p@sbblly{\number\dimen100}
			                    \xdef\r@p@sbburx{-\number\dimen100}
					    \xdef\r@p@sbbury{-\number\dimen100}}
%
% Need to rotate all four points and take the X-Y extremes of the new
% points as the new bounding box.
                        \def\minmaxtest{
			   \ifnum\number\p@intvaluex<\r@p@sbbllx
			      \xdef\r@p@sbbllx{\number\p@intvaluex}\fi
			   \ifnum\number\p@intvaluex>\r@p@sbburx
			      \xdef\r@p@sbburx{\number\p@intvaluex}\fi
			   \ifnum\number\p@intvaluey<\r@p@sbblly
			      \xdef\r@p@sbblly{\number\p@intvaluey}\fi
			   \ifnum\number\p@intvaluey>\r@p@sbbury
			      \xdef\r@p@sbbury{\number\p@intvaluey}\fi
			   }
%			lower left
			\rotate@{\@p@sbbllx}{\@p@sbblly}
			\minmaxtest
%			upper left
			\rotate@{\@p@sbbllx}{\@p@sbbury}
			\minmaxtest
%			lower right
			\rotate@{\@p@sbburx}{\@p@sbblly}
			\minmaxtest
%			upper right
			\rotate@{\@p@sbburx}{\@p@sbbury}
			\minmaxtest
			\edef\@p@sbbllx{\r@p@sbbllx}\edef\@p@sbblly{\r@p@sbblly}
			\edef\@p@sbburx{\r@p@sbburx}\edef\@p@sbbury{\r@p@sbbury}
%\ps@typeout{rotated BB: \r@p@sbbllx, \r@p@sbblly, \r@p@sbburx, \r@p@sbbury}
		\fi
		\count203=\@p@sbburx
		\count204=\@p@sbbury
		\advance\count203 by -\@p@sbbllx
		\advance\count204 by -\@p@sbblly
		\edef\@bbw{\number\count203}
		\edef\@bbh{\number\count204}
		%\ps@typeout{ bbh = \@bbh, bbw = \@bbw }
}
%
% \in@hundreds performs #1 * (#2 / #3) correct to the hundreds,
%	then leaves the result in @result
%
\def\in@hundreds#1#2#3{\count240=#2 \count241=#3
		     \count100=\count240	% 100 is first digit #2/#3
		     \divide\count100 by \count241
		     \count101=\count100
		     \multiply\count101 by \count241
		     \advance\count240 by -\count101
		     \multiply\count240 by 10
		     \count101=\count240	%101 is second digit of #2/#3
		     \divide\count101 by \count241
		     \count102=\count101
		     \multiply\count102 by \count241
		     \advance\count240 by -\count102
		     \multiply\count240 by 10
		     \count102=\count240	% 102 is the third digit
		     \divide\count102 by \count241
		     \count200=#1\count205=0
		     \count201=\count200
			\multiply\count201 by \count100
		 	\advance\count205 by \count201
		     \count201=\count200
			\divide\count201 by 10
			\multiply\count201 by \count101
			\advance\count205 by \count201
			%
		     \count201=\count200
			\divide\count201 by 100
			\multiply\count201 by \count102
			\advance\count205 by \count201
			%
		     \edef\@result{\number\count205}
}
\def\compute@wfromh{
		% computing : width = height * (bbw / bbh)
		\in@hundreds{\@p@sheight}{\@bbw}{\@bbh}
		%\ps@typeout{ \@p@sheight * \@bbw / \@bbh, = \@result }
		\edef\@p@swidth{\@result}
		%\ps@typeout{w from h: width is \@p@swidth}
}
\def\compute@hfromw{
		% computing : height = width * (bbh / bbw)
	        \in@hundreds{\@p@swidth}{\@bbh}{\@bbw}
		%\ps@typeout{ \@p@swidth * \@bbh / \@bbw = \@result }
		\edef\@p@sheight{\@result}
		%\ps@typeout{h from w : height is \@p@sheight}
}
\def\compute@handw{
		\if@height 
			\if@width
			\else
				\compute@wfromh
			\fi
		\else 
			\if@width
				\compute@hfromw
			\else
				\edef\@p@sheight{\@bbh}
				\edef\@p@swidth{\@bbw}
			\fi
		\fi
}
\def\compute@resv{
		\if@rheight \else \edef\@p@srheight{\@p@sheight} \fi
		\if@rwidth \else \edef\@p@srwidth{\@p@swidth} \fi
		%\ps@typeout{rheight = \@p@srheight, rwidth = \@p@srwidth}
}
%		
% Compute any missing values
\def\compute@sizes{
	\compute@bb
	\if@scalefirst\if@angle
% at this point the bounding box has been adjsuted correctly for
% rotation.  PSFIG does all of its scaling using \@bbh and \@bbw.  If
% a width= or height= was specified along with \psscalefirst, then the
% width=/height= value needs to be adjusted to match the new (rotated)
% bounding box size (specifed in \@bbw and \@bbh).
%    \ps@bbw       width=
%    -------  =  ---------- 
%    \@bbw       new width=
% so `new width=' = (width= * \@bbw) / \ps@bbw; where \ps@bbw is the
% width of the original (unrotated) bounding box.
	\if@width
	   \in@hundreds{\@p@swidth}{\@bbw}{\ps@bbw}
	   \edef\@p@swidth{\@result}
	\fi
	\if@height
	   \in@hundreds{\@p@sheight}{\@bbh}{\ps@bbh}
	   \edef\@p@sheight{\@result}
	\fi
	\fi\fi
	\compute@handw
	\compute@resv}

%
% \psfig
% usage : \psfig{file=, height=, width=, bbllx=, bblly=, bburx=, bbury=,
%			rheight=, rwidth=, clip=}
%
% "clip=" is a switch and takes no value, but the `=' must be present.
\def\psfig#1{\vbox {
	% do a zero width hard space so that a single
	% \psfig in a centering enviornment will behave nicely
	%{\setbox0=\hbox{\ }\ \hskip-\wd0}
	%
	\ps@init@parms
	\parse@ps@parms{#1}
	\compute@sizes
	%
	\ifnum\@p@scost<\@psdraft{
		%
		\special{ps::[begin] 	\@p@swidth \space \@p@sheight \space
				\@p@sbbllx \space \@p@sbblly \space
				\@p@sbburx \space \@p@sbbury \space
				startTexFig \space }
		\if@angle
			\special {ps:: \@p@sangle \space rotate \space} 
		\fi
		\if@clip{
			\if@verbose{
				\ps@typeout{(clip)}
			}\fi
			\special{ps:: doclip \space }
		}\fi
		\if@prologfile
		    \special{ps: plotfile \@prologfileval \space } \fi
		\if@decmpr{
			\if@verbose{
				\ps@typeout{psfig: including \@p@sfile.Z \space }
			}\fi
			\special{ps: plotfile "`zcat \@p@sfile.Z" \space }
		}\else{
			\if@verbose{
				\ps@typeout{psfig: including \@p@sfile \space }
			}\fi
			\special{ps: plotfile \@p@sfile \space }
		}\fi
		\if@postlogfile
		    \special{ps: plotfile \@postlogfileval \space } \fi
		\special{ps::[end] endTexFig \space }
		% Create the vbox to reserve the space for the figure.
		\vbox to \@p@srheight sp{
		% 1/92 TJD Changed from "true sp" to "sp" for magnification.
			\hbox to \@p@srwidth sp{
				\hss
			}
		\vss
		}
	}\else{
		% draft figure, just reserve the space and print the
		% path name.
		\if@draftbox{		
			% Verbose draft: print file name in box
			\hbox{\frame{\vbox to \@p@srheight sp{
			\vss
			\hbox to \@p@srwidth sp{ \hss \@p@sfile \hss }
			\vss
			}}}
		}\else{
			% Non-verbose draft
			\vbox to \@p@srheight sp{
			\vss
			\hbox to \@p@srwidth sp{\hss}
			\vss
			}
		}\fi	



	}\fi
}}
\psfigRestoreAt
\let\@=\LaTeXAtSign




\usepackage{/export/home/csdl/tex/icse2003/latex8}
\usepackage{times}
%% A verbatim-like environment which allows font changes
%%\usepackage{alltt}
%% New LaTeX2e graphics support
\usepackage[final]{graphicx}
% uncomment the % away on next line to produce the final camera-ready version
% and uncomment the \thispagestyle{empty} following \maketitle
\pagestyle{empty}

\begin{document}

\title{Requirement and Design Trade-offs in Hackystat: An In-Process
Software Engineering Measurement and Analysis System}

\author{Philip M. Johnson \\
\em  Collaborative Software Development Laboratory \\
\em  Department of Information and Computer Sciences \\
\em  University of Hawai'i \\
\em  Honolulu, HI 96822 \\
\em  johnson@hawaii.edu \\
}
\maketitle
\thispagestyle{empty}

\begin{abstract}
Abstract here.
\end{abstract}

\Section{Introduction}
\label{sec:intro}
Most software engineers will agree that measurement can be useful in
software development.  The disagreements begin when deciding what, when,
where, how, and why to measure.  What to measure can range from process
measures such as build failure rate to product measures such as the size
of the system.  When to measure can range from in-process measures that
require daily or hourly data collection, to out-of-process measures that
are collected, for example, after a development project is done as part of
a post-mortem.  How to measure can range from manual techniques that
require a software process group to collect and analyze the data, to
automated techniques that require no human involvement at all for
collection and analysis (but still currently require human involvement for
interpretation of the analyses and subsequent decision-making.)  Finally,
why to measure can range from the building of predictive models to estimate
future cost or quality, to assessment of current project
characteristics.

Since 2001, we have been developing and evaluating an open source,
extensible application framework called Hackystat for in-process software
engineering measurement and analysis (ISEMA).  The client-server systems
resulting from instantiation of the framework enable developers to attach
small software plugins called ``sensors'' to their development tools which
unobtrusively monitor the tools and sends low-level data about their
behavior and/or results to a Hackystat web application using SOAP. The
extensible set of sensors currently includes support for IDEs (Eclipse,
Emacs, JBuilder, Vim, Visual Studio), testing (JUnit, CppUnit, Emma), build
(Ant, Make), configuration management (CVS, Subversion), static analysis
(Checkstyle, FindBugs, PMD), bug tracking (Jira), size metrics for over
twenty five programming languages (SCLC, LOCC, CCCC), and management
(Microsoft Office, OpenOffice.org). The low-level data sent by sensors is
represented by an extensible set of abstractions called ``sensor data
types'', such as Activity, CodeIssue, Coverage, or FileMetric, which
facilitate data consistency and simplify higher level processing.  On the
server side, an extensible set of analysis modules process the raw sensor
data to create higher-level abstractions to support software development
research and management. For example, the Software Project Telemetry module
provides support for trend analysis of multiple sensor data streams to aid
in-process decision-making \cite{csdl2-04-11}, the Zorro module provides
support for automated recognition of Test Driven Development
\cite{csdl2-06-02}, the MDS module provides support for build process
analysis for NASA's Mission Data System project \cite{csdl2-03-07}, the HPC
module supports analysis of high performance computing software development
\cite{csdl2-04-22}, the CGQM module provides a ``continuous'' approach to
the Goal-Question-Mtric paradigm \cite{csdl2-05-09}, and the Course module
supports software engineering education \cite{csdl2-03-12}.  An
organization can use Hackystat to instantiate a tailored ISEMA system by
selecting components from our public repository, and can also combine these
public components with proprietary Hackystat components they develop for
themselves.

When we started the Hackystat Project, we had the idealistic (and naive)
goal of designing a truly ``generic'' ISEMA framework, one that would
provide appropriate infrastructure to any organization desiring in-process
software engineering measurement and analysis.  After five years of
research and development, we have learned that while Hackystat can be
effectively applied to a range of problems, the domain of in-process
software engineering measurement and analysis is much too broad for a ``one
size fits all'' solution.  Indeed, over the past five years, at least five
other ISEMA system development projects have been initiated, including EPM
\cite{EPM}, 6th Sense Analytics \cite{SixthSenseAnalytics}, PROM
\cite{PROM}, ECG \cite{ECG}, and SUMS \cite{SUMS}. On the one hand, this
surge of activity by the software engineering community appears to validate
the utility and potential of ISEMA systems.  On the other hand, if
Hackystat was truly generic, why where these other projects initiated?

Over the course of its development, Hackystat has had over forty public
releases, undergone seven major architectural revisions, 
been used by hundreds of developers, and grown to over 300,000 lines of
code. The system and its architecture appears to be relatively mature and
stable. Our experiences as developers and the feedback we have received
from our users over the years have made it apparent that the requirement
and design decisions made during development of an ISEMA system entail
trade-offs along the dimensions of usability, simplicity, tailorability,
marketability, and performance.  We believe that the essential nature of
these trade-offs is an important reason for the rise of alternative ISEMA
systems.  For example, the decision to make Hackystat extensible with respect to
sensors, sensor data types, and analyses also makes Hackystat more
complicated to install and use than an ISEMA tool like ECG, which performs
a single type of analysis (micro-process analysis) for a single developer
tool (Eclipse). 

This paper presents results from our first five years of research on ISEMA,
including an analysis of 12 requirement and design trade-offs made in
Hackystat, a discussion of how these trade-offs influence the buy-vs-build
decision, and implications for an ISEMA research agenda.  We believe this
information will help (a) potential users of ISEMA systems to better
evaluate the relative strengths and weaknesses of current and future
systems, (b) potential developers of new ISEMA systems to better understand
some of the important requirement and design trade-offs that they must
make, and finally (c) accelerate progress by helping the community identify
promising directions for future research and development.

While we will refer to other ISEMA systems during the presentation of the
trade-offs, and in fact provide a brief overview of them in the next
section, this paper is not a comparative analysis of current ISEMA systems.
As developers of the Hackystat system, it would be very hard for us to
provide a truly unbiased comparison of the various approaches. Furthermore,
while we have almost complete knowledge about the current design and
history of the Hackystat system, our knowledge of other systems is limited
to externally published documentation. Thus, we believe that we can make
the greatest contribution to the community by providing new insights about
our own system and our experiences with it, and leaving comparative
analysis as a possible future research direction for an unbiased third
party.  That said, the next section provides a brief introduction to the
current landscape of ISEMA systems to clarify the various approaches
underway.

\Section{ISEMA Systems}

ISEMA is a relatively recent approach to software engineering measurement.
The more traditional approach is out of process measurement, in which data
is collected about a set of previously completed projects and used to make
predictions about a future as-yet-unstarted project.  One of the most
successful applications of out of process measurement is COCOMO
\cite{Boehm00}, in which data about the cost, size, and characteristics of
previously developed systems is used to produce a predictive model that
provides estimates of cost and time for new system development projects
based upon various parameters.  Such an approach is out of process since
the system is basically used after the completion of old projects but
before the initiation of new projects.  Other non-ISEMA approaches to
software engineering measurement and analysis include PSP/TSP, the IFPUG
function point data repository, and the NASA/SEL metrics repository. The
ISEMA approach, in contrast, accumulates data about a current project in
order to provide feedback and decision-making value back into the very same
project. This section briefly presents some recent ISEMA projects to help
clarify the range of approaches in consideration.

{\bf PROM.} The Professional Metrics (PROM) system \cite{PROM} is sponsored
by the Center for Applied Software Engineering at the Free University of
Bolzano-Bozen in Italy. PROM supports an approach similar to Hackystat, in which
plugins unobtrusively monitor development activities and send process and
product data to a centralized server for analysis.  PROM provides plugins
for Microsoft Office, OpenOffice, and Eclipse. It can
extract code metrics for Java and Smalltalk. Finally, the Trace tool can
support tracking of operating system calls.  PROM has been used in case
studies of agile methodologies, open source tool evaluation, and knowledge
database integration.

{\bf EPM.} The Empirical Project Monitor (EPM) system \cite{EPM} is
sponsored by the EASE Project, which is an academic-industrial alliance in
Japan that includes the Nara Institute of Science and Technology, Osaka
University, NTT, and Hitachi.  EPM does not use a sensor-based approach,
but instead ``pulls'' data from three types of development tools: the CVS
configuration management system, the GNATS issue tracking system, and the
MailMan mail archiving system. Applications of EPM include analyzing the 
similarity and diversity of software preojcts, code clone detection, and 
comparative analysis of open source project repositories.

{\bf Sixth Sense Analytics.} This company \cite{SixthSenseAnalytics}
provides software measurement and analysis services based upon the use of
sensors that send data to a centralized server.  Unlike Hackystat, users
cannot download the server software, install it locally, and store their
data locally. Sensors are available for a variety of IDEs including
Eclipse, Emacs, Vim, JBuilder, and Visual Studio. Analyses support two
proxies for developer effort: Active Time and Flow Time. 

{\bf ElectroCodeoGram.}  The ElectroCodeoGram (ECG) project \cite{ECG} is
sponsored by the Software Engineering research group at the Free University
of Berlin. ECG is a plug-in to Eclipse that monitors developer activities
in order to represent ``micro-processes'' during software development.
Examples include the ``copy-paste-change'' micro-process, which is a common
way of producing similar functionality in multiple locations in a software
system, but which is has been hypothesized to produce defects more often
than a ``refactoring'' micro-process, in which the common functionality is
extracted out into a new method and called from both locations.  ECG is
intended to support experimental investigation into these and other
micro-processes.

 {\bf SUMS.} The Standardized User Monitoring Suite (SUMS) project
\cite{SUMS} is sponsored by the Pittsburgh Supercomputing Center and IBM.
SUMS provides unobtrusive monitoring of developers, but accomplishes this
not through individual sensors for specific tools, but rather through
low-level operating system monitoring.  SUMS has been used within a
specially instrumented lab to collect data on student programmers in order
to better understand the use of next generation high performance computing
languages and tools.

All of these ISEMA systems share one basic requirement: after installation
and configuration of the system, data collection must be automatic. This is
because in-process metrics are inherently voluminous and impractical to
collect manually.  On the other hand, none of these other systems appears
to have gone as far down the road to ``genericity'' as Hackystat.  The next
section begins our discussion of Hackystat requirement and 
design trade-offs by focusing on those we made in order to puruse our 
goal of genericity.

\Section{Primary trade-offs for ISEMA genericity}

{\bf (1) Sensor-server architecture.} An ISEMA system must perform two basic
activities: data collection and data analysis, and almost always performs
a third: data storage.  There are a variety of top-level architectures one
can choose to accomplish these goals, ranging from a single user approach where
everything occurs on a single computer, to a client-server approach where data
is collected on a client and sent to a central server for storage (either before
or after analysis).  Another architectural possibility is peer-to-peer, in which 
data is stored on individual computers but shared with others as required. 

In Hackystat, we decided upon a client-server architecture in which the
``clients'' consisted of custom sensors developed for each tool to be
monitored in the environment.  The cost of this decision is the requirement
that a custom software component must be created for a tool before its data
can be included for analysis.  The benefit is that the sensor can include
domain knowledge about the tool whose behavior is being monitored. For
example, the sensor for the Eclipse IDE can potentially monitor the
invocation of subsystems like the debugger, which can provide valuable
insight into the process of development.  

By convention, Hackystat sensors collect relatively ``raw'' data and send
it to the server where all significant analysis occurs.  This minimizes the
processing overhead on the client computer. It also allows new analyses to
be developed, deployed on the server, and then run retrospectively over
previously collected sensor data.

ISEMA systems with different architectures illustrate the trade-offs.  For
example, SUMS does not require specialized sensors for each development
tool, but instead instruments the operating system. This enables SUMS to
transparently monitor any tool used by the developers, though the type of
data that can be collected by monitoring OS-level events is more limited
than what can be obtained by custom software for each tool.  Additionally,
the SUMS instrumentation is specific to a single operating system.

EPM is another ISEMA system that does not use a sensor-server architecture.
Instead, it ``pulls'' data from its tools using their public reporting
interface. The trade-off in this case is the requirement that a tool have
some reporting interface in order for it to be accessable to EPM using this
technique.

{\bf (2) Workspaces.}  In an ISEMA system, measurements associated with
files must be collected automatically.  A generic ISEMA system
must confront the following problem: the same file could be named many
different ways. For example, the file ``Foo.java'' might be associated with
the file path c:$\backslash$svn$\backslash$projectA$\backslash$Foo.java on
developer A's computer and /usr/home/smith/svn-sandbox/projectA/Foo.java on
developer B's computer.

In Hackystat, this issue is addressed by server-side post-processing of
file names to create a canonical location known as a ``Workspace'', which
has a common file separator character. The user must also provide a
``Workspace Root'' during configuration of their account, which enables the
system to determine that directories rooted at c:$\backslash$svn on
Developer A's computer and /usr/home/smith/svn-sandbox/ on Developer B's
computer might contain the same files. 

An ISEMA system can avoid the need for the complexity of Workspaces in
several ways.  One way is to simply disallow client-side collection of data
about file artifacts, and instead collect this information from a single
location, such as a configuration management system.  Another way
is to limit the ISEMA system to analysis of one user's data (i.e. not
supporting aggregate analyses over groups of developers working on a common
project), or requiring all programmers to use a common file system. EPM, 
ECG, and SUMS all avoid the need for workspaces through one or more of these 
simplifying assumptions.

{\bf (3) Projects.} Most software engineers work on multiple tasks
concurrently, and each task might involve a different set of collaborators. 
Many kinds of ISEMA analyses require the ability to organize the process
and product data into collections according to the task underway.  

In Hackystat, this issue is addressed through a server-side abstraction
called ``Projects''.  When a Hackystat user defines a Project, they specify
a time interval, a set of Workspaces, and a set of email addresses
corresponding to other Hackystat users.  The server generates an email
``inviting'' the invited users to join this Project. Accepting
the invitation enables the system to perform project-level analyses that
aggregate together the process and product data associated with each of the
users. The specified set of Workspaces allows the system to filter out 
unrelated sensor data. 

One way to avoid the need for Projects is for the ISEMA system to guarantee
that all data sent from a user is associated with a single task. This is
the approach taken by the SUMS system, which is deployed in a laboratory
setting under controlled conditions. Another way is to focus on analyses
that are independent of particular tasks. For example, ECG identification
of copy-paste-change micro-processes can be useful without tying their
occurrence to a specific Project.

{\bf (4) Data Quality Assurance.} The requirement of regular, unobtrusive
process and product data collection creates a number of challenges related
to data completeness and correctness for a generic ISEMA system.  For
example, one cannot assume connectivity to the Hackystat server at
all times: developers often work offline (such as when traveling), and the
server can crash due to power outages or other problems.  Second, sensors
for different tools that perform the same type of function (for example,
two configuration management tools such as CVS and SVN) should collect data
in a standardized way and format so that analyses are not completely
tool-specific. Finally, sensors can and will ``drop out'' occasionally due
to power outages, platform changes, implementation bugs, and so forth.

In Hackystat, we provide a variety of mechanisms to address these
data quality issues.  First, a middleware application
called the SensorShell provides infrastructure for Hackystat sensor
development.  The SensorShell provides a high-level API to sensor designers
that insulates them from the low-level details of data transmission. It
also transparently implements client-side offline data caching and
re-transmission.  Thus, if a developer is working on a plane or the server
is unavailable, their data will be collected and cached on their laptop
until she lands and re-establishes a server connection.  Upon the next
invocation of a sensor-enabled tool, the accumulated data will be sent to
the server.  The SensorShell also buffers sensor data locally and sends the
collected data in a single SOAP request every few minutes, dramatically lowering
the overhead of sensor data transmission.

Second, to ensure consistent data collection across different tools, we
developed the ``Sensor Data Type'' abstraction.  Sensor Data Types allow
you to specify both required fields indicating data that must be sent by
all sensors for this type, as well as a ``property list'' field that
supports an arbitrary amount of optional key-value data.  For example, the
``Commit'' sensor data type includes required fields specifying data that
all configuration management sensors must provide, but also allows a sensor
for a specific tool to send additional optional data that may only be
available for that particular tool type. This enables the development of
``generic'' analyses for Commit data that are independent of the specific
configuration management tool, as well as specialized analyses for data
that may be available on only one type of tool.

Third, the requirement for unobtrusive data collection means that it is
possible for one or more sensors to crash or otherwise stop sending data
without any notification. On the other hand, sometimes sensor data is not
sent simply because developers are not currently working with those tools.
In Hackystat, our approach is the development of telemetry streams that
allow a project manager to passively monitor sensor data streams and look
for anomolies.  For example, on one occasion this telemetry revealed a
developer that was sending IDE, Build, and Commit data for a number of days
without any corresponding Unit Test data, which revealed that his test
sensor had become misconfigured during a recent upgrade.

One way to reduce the complexity of data quality assurance is to not use a
sensor-based mechanism for data collection, and/or minimize the type of
data that is collected. For example, the EPM project ``pulls'' data from
three kinds of software engineering data repositories: CVS, GNATS, and
MailMan.  Such an approach avoids many of the data quality issues we have
needed to address in Hackystat.

{\bf (5) Configurability.} A generic ISEMA system can be applied to many
different measurement tasks, and we quickly discovered that providing every
single implemented analysis, sensor data type, and sensor in a Hackystat
release significantly impacted upon its usability.  Java web application
developers, for example, disliked wading through analyses focused on MPI
programming using C++. Second, supporting configurations enables Hackystat
to more easily be applied to new domains, and to allow these new domains to
leverage code implemented previously. Finally, allowing end-user
configuration and enhancement has enabled Hackystat to grow more easily and
avoid the need for all extensions to be implemented by a central ``gate
keeper'' organization.

Clearly, a generic ISEMA system must be able to be tailored to the specific
measurement and analysis concerns of an organization. In Hackystat, this is
accomplished through two mechanisms. First, the build procedure allows an
administrator both exclude public Hackystat modules implementing
unnecessary functionality and include privately developed Hackystat modules
implementing organization-specific functionality. Second, a set of
extension points allow modules to implement generic functionality that can
be extended for organization-specific purposes.  

These mechanisms have resulted in a current Hackystat architecture
containing over seventy modules organized into four ``subsystems''.  A
``core'' subsystem provides essential functions that are independent of the
specific sensors, sensor data types, and analyses contained in a
configuration.  Core functions include the sensor and sensor data type
definition facilities, the SOAP data transmission capabilities, features
for the web application interface, and the configuration mechanism itself.
Most instantiations of the Hackystat framework include all of the core
modules. The ``Sensor'' subsystem contains modules each implementing sensor
support for a development tool; each module in the ``SDT'' subsystem
implement a single sensor data type, and the ``App'' subsystem contains
modules that operate over sensors and sensor data types to provide higher
level analyses that enable an organization to use the data for project
monitoring, quality assurance, and decision-making.

Configurability using modules and (an extensible) extension point mechanism
add significant complexity to Hackystat development, installation, and use.
First, the Hackystat build process must represent and manage module
dependencies. For example, a configuration that includes a sensor that
generates data using the UnitTest SDT must be sure to include the module
defining that sensor data type.  Second, as Hackystat has grown to over
seventy modules, 300,000 lines of code, and a half dozen different
configurations, it has become impractical for developers to test each of
their changes over the entire system, leading to the need for automated
daily build and error reports.  Third, configurations make the build
process more complicated, and have required a substantial amount of
documentation to be developed, which must also be configurable!

The complexity of Hackystat configurations comes from our desire to
minimize constraints on the kind of tailoring that can be done.  For
example, a system like ECG which is constrained to the Eclipse platform and
micro-process analyses can provide a much simpler extension mechanism for
this specific platform and analysis domain.

\Section{Emergent trade-offs}

We consider the sensor-server architecture, workspaces, projects, data
quality assurance, and configurability to be ``primary'' trade-offs: design
decisions that follow more-or-less directly from our goal to make Hackystat
as generic an ISEMA system as possible.  Interestingly, a number of
additional trade-offs follow as a consequence of these primary trade-offs.

{\bf (6) Non-real time analysis.}  We have been contacted several times by
researchers who have been interested in evaluating Hackystat for use in
domains involving ``real-time'' measurement and analysis, in other words, a
domains requiring feedback to the user within a second or two of a
measurement event.  For example, ``cyclomatic complexity'' is a well-known
measure of a method or function's complexity, and it is easy in Hackystat
to provide a sensor for a tool such as NCSS that computes this metric.  An
example of a real-time application of this measure is a plug-in to an
IDE that continuously monitors the cyclomatic complexity of a function and
pops up a window as soon as the complexity exceeds a certain threshold
value.

Many such ``real-time'' systems for software engineering measurement and
analysis can be envisioned, but Hackystat is not the appropriate
infrastructure for their development.  This trade-off results from our
dependency on the use of the SensorShell middleware component which buffers
data to reduce transmission overhead and enables offline sensor data
collection, and the more general assumption that sensor data may appear on
the server minutes, hours, or even days after it has been collected by the
client. This assumption allows flexibility in the way sensors are
implemented.  For example, we implement our Subversion configuration
management sensor as a timer-based system that runs once a day and collects
all CM events from the previous day.  Unlike the more real-time,
``hook-based'' design, our approach does not require root-level privileges
for its installation and use. 

Our experiences suggest to us that the decision to support ``real-time''
vs.  ``non-real time'' ISEMA is a significant trade-off.  We believe that an
ISEMA architecture will be significantly simpler and support its domain
more effectively if it constrains itself to either real-time or non-real
time applications but not both.  Interestingly, we know of no systems that
focus explicitly on generic support for real-time ISEMA applications.

{\bf (7) Sensor data type evolution.} As a consequence of its goal of 
genericity, Hackystat does not presuppose what types of measurement data 
will be collected and how this data will be structured.  However, to 
facilitate understanding and correct analysis of measurement data, 
Hackystat provides the sensor data type definition facility, which 
among other things differentiates between ``required'' and ``optional''
data.  Hackystat also does not presuppose the specific tools from which
sensor data will be collected, but does mandate that all tools send
their sensor data as instances of one or more sensor data types. 

The ability to define new sensors and sensor data types over time enables
an incremental and exploratory approach to ISEMA system development. For
example, one can implement a sensor data type to support a single kind of 
size counting tool, such as LOCC, and then add sensors for additional 
size counting tools such as CCCC, NCSS, and SCLC that use the same 
sensor data type.  

The problem, of course, is that experience with a broader set of tools
often reveals inadequacies in the original sensor data type definition.
For example, it is quite common when first defining a sensor data type to
build in assumptions about the nature of the data that turn out to be
peculiar to the first tool you are instrumenting. These hidden assumptions
only become apparent once sensors for additional tools of that type are
under development.   

Once a sensor data type definition is found to be inadequate and the
decision is made to improve it, one must deal with the question of what to
do with the sensor data already collected under the old definition.  For
the first few years of Hackystat's development, the system required you to
throw away the data collected under the old definition if you wished to
upgrade it.  This is an expensive solution, and in several cases led us to
simply live with a ``bad'' sensor data type definition simply because we
did not want to lose access to the data we had already collected.

Hackystat now provides the ability to ``evolve'' sensor data type
definitions to incorporate new insights about the most appropriate set of
required and optional data.  The evolution is implemented in terms of a
distinguished method in the sensor data type definition class which ``lazily'' 
evolves older versions of the sensor data upon access. This approach 
enables both old data stored on the server to be upgraded to the new
definition when retrieved for analysis, as well as data received from
clients that are still using an old version of a sensor that has not been
upgraded to use the new SDT definition.  

Sensor data type evolution adds complexity to the representation and
implementation of sensor data types, but this trade-off does enable a more
exploratory style of development while preserving the benefits of typed
data.  To our knowledge, no other ISEMA system implements sensor data type
evolution. We believe they deal with this issue using one or more of the
following trade-offs: (1) represent sensor data in an unstructured, non-typed
format, (2) perform a thorough domain analysis prior to sensor data type
definition in order to assure that the definition is correct, or (3) force
users to throw away old sensor data if evolution in the sensor data type is
required.

{\bf (8) Intermediate abstractions.} The design decisions to send sensor data
in ``raw'' form, combined with the Project abstraction for representing
group activities on subsets of sensor data have led to the need for a
variety of intermediate abstractions to support server-side analyses.

To see why this is so, consider two simple measures: an integer indicating
the total time in minutes spent editing Project-related files by all
members for a given day, and an integer indicating the change in size (LOC)
of the system under development for this Project during that same day.
There are two interesting things to note about the computation of these two
integers. First, they are not particularly domain-specific measures: many
of the Hackystat application domains, from software project telemetry to
high performance computing to NASA's Mission Data System find these
measures to be useful. Second, computing these two integers requires
retrieving all of the sensor data sent by all project members for the week
of interest, filtering out the sensor data sent by members not related to
this Project, and processing the remaining data appropriately. In
Hackystat, several thousand sensor data points might require processing to
compute each of these measures, and recurrent analyses like software
project telemetry might want these measures for several weeks or months at
a time on a regular basis.

Thus, an emergent trade-off in Hackystat is the need for a set of cached, 
intermediate abstractions that basically represent ``partial analyses'' of
the raw sensor data.   Examples of these abstractions include ``DailyAnalysis'', which 
represents individual developer activities for a given day in five minute chunks,
``DailyProjectData'', which represents one or more measures for a given Project
and day, ``Reduction Functions'', which compute sequences of measures over a given
time period, and ``SDSA Episodes'', which partition a stream of developer behavioral
events in discrete episodes suitable for later classification. 

\Section{Scalability trade-offs}

A third category of requirement and design trade-offs in Hackystat relates
to ``scalability.''  Scalability trade-offs address the total number of
users that can be supported by a running server, of course, but also
includes scalability with respect to analyses and public accessability.

{\bf (9) Usage scalability.}  The most obvious scalability trade-off involves
the total number of users who can access a running system with acceptable
responsiveness.  Hackystat's server runs within Tomcat, and our public
server currently accomodates several hundred users on a low-end Dell server
with 2 GB RAM and 80 GB disk space. 

Assessing usage scalability in an ISEMA system occurs along two primary
dimensions that can be assessed independently.  The first dimension is
scalability with respect to data collection: in other words, how many
concurrent users can be sending data to a server with acceptable
responsiveness?  Along this dimension, the trade-off in Hackystat to
support only non-real time analysis, which enables client-side buffering,
enables Hackystat to scale quite well with respect to data
collection. Given that a single user will transmit data to a Hackystat
server approximately every 10 minutes, and that a typical sensor data
transmission involves only a few thousand bytes of data, it is easy to see
that a single Hackystat server can provide adequate responsiveness to
hundreds of concurrent users with a low-end server hardware configuration.

The second dimension is responsiveness with respect to user-initiated
analyses.  This assessment is of course entirely dependent on the specifics
of the analysis, but we can offer one interesting insight from our
experience with Hackystat. When we began development of Hackystat in 2001,
we decided to store sensor data using an XML-based flat file organization,
where a separate file would be used to store the sensor data of a given
type for a given user on a given day. We viewed this as a ``spike''
solution that would simplify installation and debugging in the short-term,
but would be replaced with a more robust and efficient RDBMS such as MySQL
or Derby when system performance became constrained by this design
decision.

To our surprise, after a small amount of performance optimization early on,
our profiling activities have never since revealed our XML storage approach
to be the bottleneck in responsiveness to user-initiated analyses.  We
believe this is due to Hackystat's use of cached intermediate abstractions,
which significantly reduce the amount of raw sensor data access.  Instead,
our performance optimization efforts have focussed on this level, and have
involved tuning the cache mechanism to avoid excessive heap usage as well
as improvements to the intermediate abstraction implementations, such as
pre-computing frequently used analysis results to minimize redundent sensor
data access.


{\bf (10) Analysis scalability.} Another scalability trade-off involves the
ability of users to customize specific analyses.  For example, the software
project telemetry application allows users to monitor sets of
measurement trends over time and looking for co-variances that suggest
causal dependencies.  For example, if one notices that code coverage is
decreasing over time and that the build failure rate is increasing, it
suggests that these two measures might be interrelated and that one could
potentially decrease the build failure rate by improving test quality.  A
usability problem with software project telemetry is that the number of
possible measures and co-variances increases exponentially with the number
of available sensor data streams, so that even a half dozen sensor data
streams leads to many thousands of potential telemetry reports.

Another example of this analysis scalability problem in Hackystat is the
Software Development Stream Analysis (SDSA) application, which supports
workflow analysis over sequences of developer behavioral events, such as
opening a file, invoking a test case, refactoring code, and so forth. 
Once again, the range of approaches to partitioning the sequence of events into
episodes and then classifying them as workflow states is very large.  

In both cases, analysis scalability was improved through the implementation
of a domain-specific language.  The DSL for software project telemetry was
implemented using a grammar, the JavaCC parser generator, and a custom
interpretor.  The DSL for SDSA was implemented using JESS and a set of
CLISP rules. 

{\bf (11) Data access scalability.}  One of the most important and
complicated trade-offs in scalability for ISEMA systems involves the degree
and form in which data collected from a system and its developers is made
available to others.  For example, in Hackystat, a measure called ``Active
Time'' represents the number of minutes that a developer spent actively
editing files associated with a project during a given time interval.  A
few developers have told us that they would never allow Hackystat in their
organization because of the potential for this kind of data to be collected
and the potential misuse of this data by management. Their (legitimate)
worry is that measuring just the time spent editing files could be
misinterpreted by management as a meaningful measurement of individual
developer productivity, resulting in counterproductive measurement
dysfunction within the organization.  In these kinds of situations, we have
recommended that an organization adopting Hackystat begin by collecting
only product measures and do not require sensors to be installed into their
developer's IDEs.

At one end of the spectrum, an ISEMA system could declare that all data is
completely private and restricted to the user that collected it.  While
this ``solves'' the privacy issue arising from data access, it also
severely impacts on the potential utility of the system. For example, it
would not be possible to generate analyses that represent aggregate time,
defects, churn, and so forth for a group of developers working on a single
project.

Hackystat currently takes a more moderate approach to data access
scalability, in which users can participate in Projects which enable their
raw sensor data to be accessed for project-level analysis, although
participation in a Project does not allow members to access each others raw
sensor data directly.

Data access scalability beyond the project-level could provide very useful 
software engineering insights. For example, when paired with demographic information
about the project and its developers, ISEMA systems could become a rich source 
of information about development issues and approaches at the language or
application level.  For example, the current Hackystat repository contains rich 
data about Java-related software development over the past five years, and we 
have already performed preliminary design work for ``federated'' Hackystat servers 
that can share and exchange qualitative and quantitative data \cite{csdl2-05-02}, 
yet implementation awaits the definition of suitable policy for privacy protection. 


{\bf (12) Developer community scalability.} Our final trade-off concerns
the potential for community involvement in the enhancement and
customization of an ISEMA system.  At one end, an ISEMA system can be
developed as a completely closed source system, with no possibility for
end-user extension.  Such an approach minimizes the cost of developing the
resources and mechanisms to support a broader developer community, and also
enables all information about the design of the system to be kept
proprietary.

Hackystat has chosen the opposite end of the spectrum, in which the source
is freely available, the system is modularized for third party extension,
mailing lists are provided for developers and users, extensive developer
documentation has been developed, and a significant number of
customizations to our build (Ant), test (Junit), and documentation
framework (DocBook) have been developed to support community involvement.

Other ISEMA systems make different trade-offs. For example, Sixth Sense
Software keeps the server proprietary, but allows end-user involvement in
sensor development.

\Section{Buy vs. Build}

One practical application of these trade-offs is to assist in the buy
vs. build decision for an organization that wishes to introduce an ISEMA
system.

First, for organizations that wish to avoid the investment in developing an
ISEMA system in-house, these trade-offs can help you to evaluate the
features and capabilities of various systems.  For example, if you work in
a multi-platform, collaborative setting, then whatever ISEMA system you
choose should address the issues that were solved in Hackystat by the
Workspace and Project abstractions.

Second, if you are considering developing your own ISEMA system, these
trade-offs reveal the requirement and decision complexities that can result
from the desire to provide genericity with respect to the types of data
collected, the way in which data is collected, and the way in which data is
processed.  In some cases, a significantly simpler ISEMA solution is
possible by eliminating one or more dimensions of genericity.  For example,
if the only development tool you use is Eclipse, then much of the
complexity of Hackystat is not needed.  If your organization works on a
single platform, then you might not need Workspaces.

Third, regardless of whether you are buying or building, you must actively
consider the issue of data privacy.  An ISEMA system that can monitor
developer behavior, such as the IDE sensors for Hackystat, has the
potential to be viewed as ``Big Brother'' by developers and arouse fears
that the ISEMA data will be used to evaluate developer productivity.  Such
counter-productive applications have been termed ``measurement
dysfunction'', and for a more comprehensive treatment of this subject, we
recommend that you start with Robert Austin's excellent introduction to
this problem and ways to manage it \cite{Austen??}.

\Section{A Research Agenda for ISEMA}

When we began the Hackystat Project five years ago, we were not aware of any other 
active ISEMA development projects.  We are gratified that a community of users and
developers interested in in-process software engineering measurement and analysis 
has arisen over this time, and that Hackystat is in the vanguard of these efforts. 

To conclude this paper, we wish to reflect on the implications of these
requirement and design trade-offs for the software engineering research
community in general and the ISEMA research community in particular.  What
are important issues to be addressed by ISEMA systems during the next five
years? Here are six directions we believe could have significant pay-off. 

{\bf Real-time ISEMA.} One of the notable gaps in current ISEMA
research and development is the lack of support for real-time
responsiveness, particularly in a distributed, collaborative setting.
Infrastructure for real-time ISEMA could take at least two forms: real-time
ISEMA feedback, and real-time ISEMA notification.  Similar to the way in
which spelling checkers went from a stand-alone, batch mode to a real-time
``squiggly underline'' mode, real-time ISEMA feedback could enable
developers to monitor the measurement impact of their system after each
keystroke.  Real-time ISEMA notification is a form of remote
feedback--instead of getting immediate ISEMA feedback on your own
keystrokes, you get real-time immediate feedback on your fellow developer's
keystrokes.  Many interesting research issues exist in this domain, from the 
kinds of measures that would be generally useful in real-time, to the contextual
information required to avoid ``false positive'' notifications. 

{\bf ISEMA interoperability.}  As noted above, one result of our research
is the insight that there can be no ``one-size-fits-all'' ISEMA system for
the same reasons that there can be no ``one-size-fits-all'' programming
language: the same design decisions that make Ruby an excellent language
for small-scale web application development also make it inferior to
Fortran for climate modeling.  There will be a community of ISEMA systems
that satisfy different organizational contexts and measurement goals.

A community of ISEMA systems provides the opportunity to establish
standards for data representation and interoperation that would create new
measurement and analysis alternatives for organizations.  First, data
collected in one environment could be analyzed using another environment.
Second, standards for interoperability could enable ``meta analysis'', in
which common and comparable features from a set of analyses from different
environments are extracted and used to form more general
conclusions. Finally, the process of standard setting could help facilitate
propogation of experience across the ISEMA development landscape,
increasing the rate of progress and bring stability to measurement and
analysis definitions.

{\bf Support for qualitative data.} ISEMA systems current focus on
collecting and analyzing numeric data.  However, rich insight into software
engineering practices can be obtained through non-numeric, qualitative
approaches such as grounded theory \cite{Seaman??}.  In our research on the
software engineering of high performance computing systems, we have already
discovered the need to integrate qualitative data (from developer
interviews, journals, and researcher analyses) with quantitative process
and product data.  Yet no ISEMA system provides facilities for qualitative
data collection and analysis comparable to what they provide for
quantitative data collection and analysis.

{\bf Improved data privacy, anonymity, and access.}  One of the most
important research issues for ISEMA systems is to better understand and
manage the inherent tension between data privacy, anonymity, and access. As
discussed above, guaranteeing complete privacy hinders even simple forms of
analysis, such as the aggregate impact of group activities on a project.
However, even the {\em potential} availability of certain types of measures
to management can become a barrier to data collection itself, and/or lead
to measurement dysfunction.  On the third hand, providing broader access to
the data collected by ISEMA systems has significant potential to provide
new insights about software engineering.

Providing broader access to ISEMA data must somehow resolve the following
conundrum: in order for ISEMA data to be interpreted correctly by the
broader research community, contextual data about its collection and
analysis must be available.  On the other hand, such contextual data often
reveals, either implicitly or explicitly, the identity of the organization,
projects, and/or individuals from which the data was collected.  This
violates their privacy.  Basili and others have began to work on this issue
\cite{Basili}.

{\bf Automated decision making.}  Like other ISEMA systems, Hackystat
collects data, and provides analyses based upon this data, but does not
``act'' upon these analyses. Instead, it is left to a developer or manager
to review the data and decide what, if any, action to take based upon the
analysis.  As our experience and confidence in ISEMA systems grows, a
natural next step is for the system to begin ``closing the loop'', by
actually initiating development actions based upon its analyses.

As a simple example, large systems often develop test case suites that are
so large and expensive to execute that they can be run in their entirety
only intermittently.  For example, the Mission Data System test suite was
run only on weekends since it took over a day to execute and required
machine resources not available during the working week.  In such
situations, developers must rely on partial testing before committing their
code.  An test case selection mechanism that includes ISEMA data could be
integrated into a daily build mechanism and automatically decide which set
of test cases to execute that would be most likely to reveal introduced
defects while staying within the resource constraints of the environment.

{\bf Measurement and analysis validation.} At the end of the day,
collecting software engineering process and product measurements, and even
inventing plausibly interesting analyses about this raw data, is not very
hard.  What is hard is validation: ensuring that (a) the collection
mechanisms are actually collecting the data that they are supposed to be
collecting and (b) the analyses performed on the collected data actually
provide accurate and useful insight into the corresponding software
development.  As the ISEMA community matures, much more effort must be
devoted to measurement validation.

Measurement validation almost always requires an independent source of
information about the measure being validated.  For example, we recently
performed a pilot validation of the Zorro system for recognizing Test
Driven Design \cite{csdl2-06-02}.  Zorro is a Hackystat-based system that
collects sensor data and analyzes it using a rule-based system that
partitions the developer's activities into ``episodes'', each of which is
then classified as an instance of TDD or non-TDD.

While the system seemed reasonable in theory, we didn't really know if the
sensors were collecting the right kind of data to support the right kind of
episode partitioning, and if the resulting episodes were being classified
correctly, and we couldn't use Zorro's data and analyses to validate
itself.  To perform the validation, we implemented a separate system,
called the Eclipse Screen Recorder, that created a QuickTime movie of the
Eclipse screen while the developer was working.  By comparing this
independent source of data about the developer's activities to the Zorro
analyses, we were able to determine that Zorro was collecting the right
data and analyzing it appropriately around 90\% of the time.  


%\Section{Acknowledgements}
%\label{sec:acknowledgements}
%We gratefully acknowledge support for the Hackystat Project from NSF grants [...].
%We would also like to thank Koji Torii from the EPM Project, Alberto Sillitti from the PROM project, 
%Frank from the ECG project, and Todd Olson from Sixth Sense Analytics for their comments on this
%paper.
 
\bibliographystyle{/export/home/csdl/tex/icse2003/latex8}
\bibliography{/export/home/csdl/bib/csdl-trs,/export/home/csdl/bib/hackystat,/export/home/csdl/bib/psp,/export/home/csdl/bib/hpcs}
\end{document}
 










