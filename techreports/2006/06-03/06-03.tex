%%%%%%%%%%%%%%%%%%%%%%%%%%%%%% -*- Mode: Latex -*- %%%%%%%%%%%%%%%%%%%%%%%%%%%%
%% 06-03.tex -- Tutorial proposal submission to XP 2006
%% Author          : Hongbing Kou
%% Created On      : Mon Sep 23 11:52:28 2002
%% Last Modified By: 
%% Last Modified On: Sat Feb 25 10:47:56 2006
%%%%%%%%%%%%%%%%%%%%%%%%%%%%%%%%%%%%%%%%%%%%%%%%%%%%%%%%%%%%%%%%%%%%%%%%%%%%%%%
%%   Copyright (C) 2005 Hongbing Kou
%%%%%%%%%%%%%%%%%%%%%%%%%%%%%%%%%%%%%%%%%%%%%%%%%%%%%%%%%%%%%%%%%%%%%%%%%%%%%%%
%% 

\documentclass[runningheads]{llncs}
\input{psfig.sty}

\begin{document}

\pagestyle{headings}
\mainmatter

\title{Lightweight, automated process and product\\
metrics collection and analysis with Hackystat}

\titlerunning{Introduction to the Hackystat Framework} 

\author{Philip M. Johnson}

\institute{
Collaborative Software Development Laboratory \\
Department of Information and Computer Sciences \\
University of Hawai'i \\
Honolulu, HI 96822, USA \\
\email{johnson@hawaii.edu}}

\maketitle

Hackystat is an open source framework for automated collection and analysis
of software engineering process and product data that has been under
development since 2001. Increasing usage by a variety of academic and professional 
organizations indicates its utility to organizations interested in 
lightweight, yet sophisticated metrics collection and analysis.

Hackystat differs from other metrics collection and analysis frameworks in
one or more of the following ways.

First, Hackystat uses sensors to unobtrusively collect data from development
environment tools; there is no chronic overhead on developers to collect
product and process data. Over two dozen sensors are publically available,
including sensors for IDEs (Emacs, Eclipse, JBuilder, Vim, VisualStudio),
configuration management (CVS, Subversion), bug tracking (Jira), testing
and coverage (JUnit, CppUnit, Emma, JBlanket), system builds and packaging
(Ant), static analysis (Checkstyle, PMD, FindBugs, LOCC, SCLC), and so
forth.

Second, Hackystat is tool, environment, process, and application
agnostic. The architecture does not suppose a specific operating system
platform, a specific integrated development environment, a specific
software process, or specific application area. A Hackystat installation is
configured from a set of modules that determine what tools are supported,
what data is collected, and what analyses are run on this data.

Third, Hackystat is intended to provide in-process project management
support. Many traditional software metrics approaches are based upon the
"project repository" method, in which data from prior completed projects
are used to make predictions about or support control of a current
project. In contrast, Hackystat is designed to collect data from a current,
ongoing project, and use that data as feedback into the current project.

Fourth, Hackystat provides infrastructure for empirical experimentation. For
those wishing to compare alternative approaches to development, or for
those wishing to do longitudinal studies over time, Hackystat can provide a
low-cost approach to gathering certain forms of project data.
 
Fifth, Hackystat is open source and is made available for no charge.

Hackystat has been applied in a variety of agile and non-agile contexts.
The Zorro Project involves the development of a sensor for Eclipse that
attempts to automatically detect when users are employing test-driven
design practices. An pilot validation study found that Zorro correctly
identified TDD episodes 89\% of the time \cite{csdl2-06-02}.  Software
Project Telemetry is an approach to metrics analysis and visualization that
provides a new style of in-process project management \cite{csdl2-04-11}.
Hackystat is being applied as a measurement technology for the DARPA High
Productivity Computing Systems program \cite{csdl2-04-22}.  Finally,
Hackystat has been used extensively for software engineering education
\cite{csdl2-03-12}.

Developing proficiency with the Hackystat Framework involves a progression
through three levels of expertise. The first level is the ``user'', who is
able to install sensors and run the analyses provided by a server.  The
next level is the ``administrator'', who is able to install and maintain a
Hackystat server, as well as assemble new configurations of Hackystat from
the 70 publically available modules to customize the system's capabilities
for their organizational needs.  The final level is the ``developer'', who
can implement new modules to support additional process or product data and
analyses for entirely new software development domains.

The goal of this tutorial is to provide attendees with a clear
understanding of the opportunities and challenges associated with
Hackystat-based software engineering measurement collection and analysis.
This understanding can help attendees decide whether or not Hackystat is
appropriate for their organization, as well as improve their ability to evaluate
software engineering metrics frameworks in general. 

The tutorial will work toward this goal in a highly experiential way.
Attendees are encouraged to bring a laptop (Unix, Mac, or Windows), upon
which they will install a Hackystat server and sensors, collect their own
sample metrics, and perform analyses on them. These direct experiences will
be augmented by case studies on large scale usage of Hackystat.

\bibliographystyle{/export/home/csdl/tex/splncs}
\bibliography{/export/home/csdl/bib/zorro,/export/home/csdl/bib/tdd,/export/home/csdl/bib/csdl-trs,/export/home/csdl/bib/hackystat,/export/home/csdl/bib/psp}


\newpage

\section{Agenda}

\noindent 08:00-08:45: Overview of  Hackystat

\bigskip 
\noindent 08:45-09:30: Lab 1: Installing Hackystat sensors, collecting initial data

\bigskip 
\noindent 09:30-10:00: Break

\bigskip 
\noindent 10:00-10:45: Basic Hackystat data analysis Project summaries and telemetry:

\bigskip 
\noindent 10:45-11:30: Lab 2: Basic analysis with Hackystat

\bigskip 
\noindent 11:30-12:00: Advanced analysis with Hackystat (TDD inference)

\bigskip 
\noindent 12:00-01:00: Lunch

\bigskip 
\noindent 01:00-01:30: Lab 3: Installing and configuring a Hackystat server

\bigskip 
\noindent 01:30-02:00: Hackystat and metrics data administration

\bigskip 
\noindent 02:00-02:30: Lab 4: Requirements for software engineering metrics in your organization

\bigskip 
\noindent 02:30-03:00: Break

\bigskip 
\noindent 03:00-03:30: Customizing Hackystat for specific organizations (JPL) 

\bigskip 
\noindent 03:30-04:00: Roadmap for Hackystat 2007-2008:

\bigskip 
\noindent 04:00-whenever: Open questions and answers

\end{document}











