%%%%%%%%%%%%%%%%%%%%%%%%%%%%%% -*- Mode: Latex -*- %%%%%%%%%%%%%%%%%%%%%%%%%%%%
%% summary.tex -- 
%% RCS:            : $Id: nsf93-summary.tex,v 1.11 93/10/06 16:52:35 johnson Exp $
%% Author          : Philip Johnson
%% Created On      : Wed Aug 11 12:55:46 1993
%% Last Modified By: Philip M. Johnson
%% Last Modified On: Fri Jun 17 10:53:02 2005
%% Status          : Unknown
%%%%%%%%%%%%%%%%%%%%%%%%%%%%%%%%%%%%%%%%%%%%%%%%%%%%%%%%%%%%%%%%%%%%%%%%%%%%%%%
%%   Copyright (C) 1993 University of Hawaii
%%%%%%%%%%%%%%%%%%%%%%%%%%%%%%%%%%%%%%%%%%%%%%%%%%%%%%%%%%%%%%%%%%%%%%%%%%%%%%%
%% 
%% History
%% 11-Aug-1993          Philip Johnson  
%%    

\documentclass[11pt]{article} 
\usepackage{/export/home/csdl/tex/icse2003/latex8}
\usepackage{times}

\pagestyle{empty}

\begin{document}
\section*{Project Summary}

This research proposal presents a new, evidence-based approach to the
generation and adoption of best practices.  Instead of looking outward into
the community for best practices, and attempting to adapt them to one's own
environment, our research will investigate how best practices can emerge
organically from within one's current organizational and project context.
Instead of relying on politics or persuasiveness for adoption, our research
approach involves instrumentation that generates empirical data that can be
used to either argue for the benefits of adoption, or else provide evidence
that the practice is not actually effective in the current context.
Finally, our research will involve analytic approaches designed to generate
candidate best practices from analysis of process and product data.  To
accomplish this, we will synthesize and extend four streams of research:
(a) software project telemetry, which provides a mechanism for in-process
monitoring of software engineering data streams; (b) software development
stream analysis (SDSA), a mechanism for recognition of certain best practices
(such as test-driven design) from low-level developer behaviors, (c) pattern
discovery, a collection of data mining techniques for discovery of patterns
in event data, and (d) evidence-based software engineering, an approach to 
better empirically-based research.  Our project has seven objectives:

  
(1) Enhancement of the Software Development Stream Analysis mechanism to
  support a variety of current best practices, and determination of the
  kinds of abstractions, automation, and best practices that are amenable
  to recognition using SDSA.

(2) Development of integration mechanisms between SDSA and Software
  Project Telemetry in order to allow users to determine how practices
  recognized by SDSA relate to telemetry data at any particular point in time.
  
(3) Development of a pattern discovery subsystem in Hackystat to support
automated recognition of behavioral patterns by developers as they use tools, 
abstractions, and automation, and the use of Software Project Telemetry to 
determine whether these behavioral patterns are potential candidates for best
practices. 
  
(4) Classroom-based, case study evaluation of the proposed techniques. We will 
apply these techniques to gain evidence regarding programmer productivity and variability 
with respect to the Test Driven Design best practice. This activity will also 
refine the technology, develop curriculum materials, and ready the approach for 
industrial evaluation. 

(5) Industry-based evaluation of the proposed techniques. Following
classroom evaluation, we will carry out two industry-based case studies
to gather evidence regarding best practices related to high performance computing 
and agile software development. This activity will also assess our approach
in industrial settings. 
  
(6) Packaging of the system and methods for widespread dissemination.  We
will continue the process used by the open source Hackystat Project of making
our technology available to the software engineering community.  In addition, 
we will package and disseminate our experimental methods to support external
evidence-based software engineering efforts. 
  
(7) Development of curriculum materials regarding continuous,
  evidence-based discovery and assessment of software engineering best
  practices. As with the Hackystat Project, we will develop software engineering
  curriculum materials and assignments that enable the study and analysis
  of this approach in academic settings.

The intellectual merit of this research includes the application of novel
data gathering and analysis techniques for the discovery and evaluation of 
software engineering best practices, and the evaluation of this technique
through classroom and industrial case studies. 

The broader impact of this research includes the development of a
sophisticated, freely available, open source software system for use by
researchers and practitioners to study software engineering best practices,
the generation of new evidence-based research results, and curriculum
materials to support education and technology transfer.  As the University
of Hawaii is a university with 75\% minority students in an EPSCOR state,
this project will provide novel research opportunities to underrepresented
groups.

\end{document}

 







