\Section{Thesis Statements}
\label{sec:statement}

Developers like unit testing because it brings confidence \cite{Hunt:03} on
their code being created on a solid base and can also serve as regression
testing to check system is working or not in the new circumstance.
Everybody knows testing is important and everybody knows they don't do
enough testing \cite{Beck:00}.  In Test-Driven Development tests are
written before coding and they drive the software design as well.Initial
studies in \cite{Edwards:04}, \cite{George:2003}, \cite{Maximilien:2003}
all gave postive results on TDD; however, our experiences indicate that we
do unit testing moderately but we usually don't do TDD in Collborative
Software Development Lab and unit testing is more likely being done in
postmortem fashion. Tests are created when we don't have enough confidence
on the code being created and the continuous integration system will run
all tests automatically everyday. We believe our reluctance on TDD adoption
could be generic and with Hackystat support we will be able to find out the
reasons. Initially there are three kinds of testing styles:

\begin{enumerate}
\item Write unit tests before code, which is Test-Driven Development or
Test-First Design.
\item Write unit tests after code is written, which is used as validation. 
\item Hybrid, sometime in TDD but not exactly.
\end{enumerate}

With Hackystat in-process metric data it is feasible to categorize them and
find patterns of each. A series analyses can be created to support the
categorization and quantatively interpret the actual development process.

Previous case studies show that TDD developers yielded higher quality code
and are more productive than non-TDD developers.  An informatic survey also
found that 87.5\% of developers reported better requirement understanding
and 95.8\% of developers reported reduced debugging effort
\cite{HawleyBlog}. In my study I will revisit these claims and provide
in-process metric data to cross-validate and refine them.












