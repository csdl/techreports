\begin{abstract} 

Adopting and deploying best practices is a common practice in software
organizations to improve development capabilities and process maturities.
Best practice contains ``a set of guidelines and recommendations'' for
doing something in either software development or business management
\cite{BestPracticeWiki}. For instance, design patterns are depicted in
Unified Modeling Language (UML) diagram to illustrate the generically
accepted designs to similiar software problems. In the development of
software engineering discipline, researchers came up with waterfall model,
software review, extreme programming and aspect programming etc out of best
practices. As one of the most well-known practice, waterfall model plays a
vital role in the history of software engineering and it still exists in
many modern software projects' development. Best practice helps to yield
good software processes and improve software development. Extreme
programming, one of the most famous agile process, consists of 14 best
practices \cite{Jeffries:00}.

Best practice varies from complicated and heavy practice such as waterfall
model to light-weight practice such as uniform code style in a software
project. In software engineering, best practices are summarized and
abstracted from successful development experiences and they are constantly
being improved by practitioners. Meanwhile, a process model may be
developed to enforce the best practice discipline as extreme programming
shows. Despite its importance, best practice is often ignored or discarded
by developers and software organizations intentionally or unintentionally.
One critical reason for this is up to the nature of best practice -- best
practice has no solid theoretical support, and usually best practice is
very descriptive and most time it has no rigid execution plan. The
uncertainness and vagueness of best practice make it hard to understand and
study best practice execution in software development. Janzen's research
found that ``Measuring the use of particular software development
methodology is hard.  Many organizations might be using the methodology
without talking about it.  Others might claim to be using a methodology
when in fact they are misapplying it. '' \cite{Janzen:05} Without good
understanding, mentoring and consultancy, practitioners will lack the
ability to judge whether they misconduct best practice or not, and it is
hard to tell whether the consistency is maintained or not.

In our research work, we introduced a substantial step forward to help
developers and organizations inspect software development process with
Hackystat. A framework, software development stream analysis (SDSA) is
proposed and implemented to direct and evaluate best practice's execution
in software development. Procedure and key steps of best practices are
represented as rules in SDSA framework to study software development
process stream. Unlike software processes such as waterfall model and
rational unified process (RUP), which emphesize on work flow and
collaboration in software project development, our approach is bottom-up
and we emphesize on the procedure how developers implement software with
in-process software metric data. Our methodology is microprocess, which is
contrary to macroprocess such as waterfall model, CMM or Unified Process. I
will use SDSA microprocess to study how effectively it can support best
practice evaluation and coach. All the studies in this thesis will be on
best practice Test-Driven Development.
\end{abstract}










