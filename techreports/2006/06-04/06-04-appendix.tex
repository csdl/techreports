\appendix
\chapter{Consent form of classroom case study}
\label{app:consent}
\noindent Thank you for agreeing to participate in our research on
understanding test-driven development practices. This research is being
conducted by Hongbing Kou as part of his Ph.D. research in Computer Science
under the supervision of Professor Philip Johnson.\\*[3mm]
As part of this research, you will be asked to develop or modify a program
and part of your course project using test-driven design practices and the
Eclipse development environment. While you are working on the program, we
will be collecting data about how you program, including the statements
that you write, the test cases that you develop, the times that you invoke
the tests and their outcomes, and so forth.  The development or
modification activity will typically take no more than a few hours of work,
depending upon how consistently you work on it.\\*[3mm]
There is no ``right'' or ``wrong'' way to do software development in this
research. Just develop the software as best you can. We are interested in
observing what real programmers do when working in a test-driven
development setting.\\*[3mm]
The data that we collect will be kept anonymously, and there will be no
identifying information about you in any analyses of this data.\\*[3mm]
Your participation is voluntary, and you may decide to stop participation
at any time, including after your data has been collected.  If you are
doing this task as part of a course, your participation or lack of
participation will not affect your grade.\\*[3mm]
If you have questions regarding this research, you may contact Professor
Philip Johnson, Department of Information and Computer Sciences, University
of Hawaii, 1680 East-West Road, Honolulu, HI 96822, 808-956-3489.  If you
have questions or concerns related to your treatment as a research subject,
you can contact the University of Hawaii Committee on Human Studies, 2540
Maile Way, Spalding Hall 253, University of Hawaii, Honolulu, HI 96822,
808-539-3955. \\*[3mm]
Please sign below to indicate that you have read and agreed to these
conditions.\\*[3mm]
Thanks very much! \\*[3mm]
----------------------------- \\
Your name/signature\\
Cc: A copy of this consent form will be provided to you to keep.


\chapter{Pre-survey on test-driven development}
\label{app:pre-survey}
\noindent Your email: \ldots\ldots\ldots\ldots\ldots\ldots\ldots\ldots\ldots\ldots\ldots\ldots\ldots\ldots\dots\ldots\ldots\ldots\\*[3mm]
Please circulate your answers to the following claims made on test-driven
development.\\*[3mm]
1. Test-driven development helps me develop code in less time: \\
(1) Strongly Disagree (2) Disagree (3) Neutral (4) Agree (5) Strongly Agree\\*[3mm]
2. Test-driven development drives me understand requirements and specification
better.\\
(1) Strongly Disagree (2) Disagree (3) Neutral (4) Agree (5) Strongly Agree\\*[3mm]
3. Test-driven development improves code quality. \\
(1) Strongly Disagree (2) Disagree (3) Neutral (4) Agree (5) Strongly Agree\\*[3mm]
4. Test-driven development helps to yield simple design.\\
(1) Strongly Disagree (2) Disagree (3) Neutral (4) Agree (5) Strongly Agree\\*[3mm]
5. Test-driven development saves time on debugging. \\
(1) Strongly Disagree (2) Disagree (3) Neutral (4) Agree (5) Strongly Agree\\*[3mm]
6. Test-driven development is more effective than other methods.\\
(1) Strongly Disagree (2) Disagree (3) Neutral (4) Agree (5) Strongly Agree\\*[3mm]
7. I tried to do test-driven development all the time.\\
(1) Strongly Disagree (2) Disagree (3) Neutral (4) Agree (5) Strongly Agree\\*[3mm]
8. It is hard for me to get used to test-drive style programming.\\
(1) Strongly Disagree (2) Disagree (3) Neutral (4) Agree (5) Strongly Agree\\*[3mm]

\chapter{Post-survey on test-driven development}
\label{app:post-survey}
\noindent Your email: \ldots\ldots\ldots\ldots\ldots\ldots\ldots\ldots\ldots\ldots\ldots\ldots\ldots\ldots\dots\ldots\ldots\ldots\\*[3mm]
Please circulate your answers to the following claims made on test-driven
development according to your experience.\\*[3mm]
1. Can you estimate how much percent of your development is in test-driven
development?\\
(1) 90-100\% (2) 75-90\% (3) 50-75\% (4) 20-50\% (5) less than 20\% \\*[3mm]
2. It is hard for me to do test-driven development? \\
(1) Strongly Disagree (2) Disagree (3) Neutral (4) Agree (5) Strongly Agree\\*[3mm]
3. Test-driven development helps me develop code in less time. \\
(1) Strongly Disagree (2) Disagree (3) Neutral (4) Agree (5) Strongly Agree\\*[3mm]
4. Test-driven development drives me understand requirements and
specification better.\\
(1) Strongly Disagree (2) Disagree (3) Neutral (4) Agree (5) Strongly Agree\\*[3mm]
5. Test-driven development improves code quality. \\
(1) Strongly Disagree (2) Disagree (3) Neutral (4) Agree (5) Strongly Agree\\*[3mm]
6. Test-driven development helps to yield simple design.\\
(1) Strongly Disagree (2) Disagree (3) Neutral (4) Agree (5) Strongly Agree\\*[3mm]
7. Test-driven development is more effective than other methods.\\
(1) Strongly Disagree (2) Disagree (3) Neutral (4) Agree (5) Strongly Agree\\*[3mm]
8. I will continue using test-driven development in the rest of my course proejct development. \\
(1) Strongly Disagree (2) Disagree (3) Neutral (4) Agree (5) Strongly Agree\\*[3mm]
9 What are the drawbacks of test-driven development with your experience in
this class?\\
\ldots\dotfill\ldots \\
\ldots\dotfill\ldots

\chapter{Participation solicitation letter}
\label{app:letter}

\noindent Hi, Dear TDD developers,\\*[3mm]
I am writing an invitational letter of participation in a TDD study of
validation of Zorro, a software system that infers existence of TDD with development event data.\\*[3mm]
As members of test-driven community, we are proud of being test-infected
and we appreciate the confidence that test-driven development brings to us.
But, outside of our community, there are still a lot of doubts and
questions on test-driven development.  A significant one of them is that it
is impossible to do test-driven development all the time, even most of the
time. In another word, are we disciplined in doing test-driven development?\\*[3mm]
In my Ph.D research work, I developed a software system called Zorro to
infer existence of test-driven development with developer behavior data
automatically in an obtrusive manner(silent event data collection provided
by Hackystat http://hackydev.ics.hawaii.edu and http://www.hackystat.org).
In this paper (http://csdl.ics.hawaii.edu/techreports/06-02/06-02.pdf), we
presented Zorro software system and a pilot study on its validation.  Zorro
correctly recognizes 90\% of development episodes, which are
equivalent to iterations of test-driven development.\\*[3mm]
I am planning to conduct a replication case study on Zorro with experienced
test-driven developers. Your experience and expertice on test-driven
development will help us improve our understanding on test-driven
development and refine Zorro, an open source software that can recognize
test-driven development. Please refer to
http://hackydev.ics.hawaii.edu/hackyDevSite/external/docbook/ch07s06.html
for present analyses provided by Zorro.\\*[3mm]
We will be very appreciated for your participation in this study. Please
email me at hongbing@hawaii.edu if you are interested\\\\*[3mm]
Yours Sincerely,\\*[3mm]
Hongbing

\chapter{Consent form for experienced develoeprs}
\label{app:consent2}
\noindent Thank you for agreeing to participate in our research on
understanding test-driven development practices. This research is being
conducted by Hongbing Kou as part of his Ph.D. research in Computer Science
under the supervision of Professor Philip Johnson.\\*[3mm]
As part of this research, you will be asked to develop or modify a program
using test-driven design practice and Eclipse IDE. While you are working on
the program, we will be collecting data about how you program, including
the statements that you write, the test cases that you develop, the times
that you invoke the tests and their outcomes, and so forth. The development
or modification activity will typically take an anour to no more than a few
hours of work, depending upon how which program you choose to work on.\\*[3mm]
There is no ``right'' or ``wrong'' way to do software development in this
research. Just develop the software as best you can. We are interested in
observing what real programmers do when working in a test-driven
development setting.\\*[3mm]
The data that we collect will be kept anonymously, and there will be no
identifying information about you in any analyses of this data.\\*[3mm]
Your participation is voluntary, and you may decide to stop participation
at any time, including after your data has been collected.  If you are
doing this task as part of a course, your participation or lack of
participation will not affect your grade.\\*[3mm]
If you have questions regarding this research, you may contact Professor
Philip Johnson, Department of Information and Computer Sciences, University
of Hawaii, 1680 East-West Road, Honolulu, HI 96822, 808-956-3489.  If you
have questions or concerns related to your treatment as a research subject,
you can contact the University of Hawaii Committee on Human Studies, 2540
Maile Way, Spalding Hall 253, University of Hawaii, Honolulu, HI 96822,
808-539-3955. \\*[3mm]
Please sign below to indicate that you have read and agreed to these
conditions.\\*[3mm]
Thanks very much! \\*[3mm]
----------------------------- \\
Your name/signature\\
Cc: A copy of this consent form will be provided to you to keep.

\chapter{Problemsets used in validation study}
\label{app:problems}
\section{Stack}
\noindent This tutorial gives step-wise guideline on how to implement a 
stack data structure with Test-Driven Development. Stack works in
last-in-first-out (LIFO) principle. In a stack, new value will always be
pushed onto the top of the stack and the topmost value is always the first
one to be popped off from the stack. Stack includes four basic operations:
Push, Pop, Top, and isEmpty.
\begin{itemize}
\item The \textit{Push} function inserts an element onto the top of the
  Stack.
\item The \textit{Pop} function removes the topmost element and returns it.
\item The \textit{Top} operation returns the topmost element but does not
  remove it from the Stack.
\item The \textit{isEmpty} function returns truth when there are no
  elements on the Stack and false otherwise.
\end{itemize}

\section{Roman numeral}
\noindent Roman numerals are written as combinations of the seven letters 
in the table \ref{tab:RomanNumeral} (excerpted from
\cite{DictRomanNumeral}).
\begin{table}[!htbp]
  \centering
  \begin{tabular}{|l|l|} \hline
  I=1 & C=100 \\ \hline
  V=5 & D=500 \\ \hline
  X=10 & M=1000 \\ \hline
  L=50 & \\ \hline
  \end{tabular}
  \caption{Roman Numerals} \ref{tab:RomanNumeral}
\end{table}
If smaller numbers follow larger numbers, the numbers are added. 
If a smaller number precedes a larger number, the smaller number is
subtracted from the larger. For example:
\begin{itemize}
\item VIII =   5 +   3 =  8
\item IX   = 10  -   1 =  9
\item XL  =  50  - 10 = 40
\end{itemize}
You are about to write a conversion program to translate integer numbers 0
- 50 to roman numeral. See more explanation and bi-directional conversion
at \cite{RomanNumeralSize}.

\section{Bolwing game}
\noindent A single bowling game consists of ten \textit{frames}. The object 
in each frame is to roll a ball at ten bowling pins arranged in an equilateral
triangle and to knock down as many pins as possible.\\*[3mm]
For each frame, a bowler is allowed a maximum of \textit{two rolls} to
knock down all ten pins. If the bowler knocks them all down on the first
attempt, the frame is scored as a \textit{strike}. If the bowler does not
knock them down on the first attempt in the frame the bowler is allowed a
second attempt to knock down the remaining pins. If the bowler succeeds in
knocking the rest of the pins down in the second attempt, the frame is
scored as a \textit{spare}.\\*[3mm]
The score for a bowling game consists of sum of the scores for each frame.
The score for each frame is the total number of pins knocked down in the
frame, \textit{plus} bonuses for strikes and spares. In particular, if a
bowler scores a \textit{strike} in a particular frame, the score for that
frame is ten plus the sum of the next two rolls. If a bowler scores a spare
in a particular frame, the score for that frame is ten plus the score of
the next roll. If a bowler scores a strike in the tenth (final) frame, the
bowler is allowed \textit{two more rolls}. Similarly, a bowler scoring a
\textit{spare} in the tenth frame is allowed \textit{one more roll}. The
bonus is only used to calculate the last frame and it won't be treated as a
normal frame.\\*[3mm]
The maximum possible score in a game of bowling (strikes in all ten frames
plus two extra strikes for the tenth frame strike) is 300.\\*[3mm]
Figure \ref{fig:scoreboard1} and \ref{fig:scoreboard2} are two bowling
scoreboard. Each frame except for the tenth has one little squares in the
upper right, where scoring for the frame's own throws is kept. The first
throw is written outside the little square; the second is recorded inside
the little square; and the cumulative score goes in the big part of the
square.\\*[3mm]
In the score board, a solid square stands for a strike and solid triangle
stand for a spare. You should only work on the bowling game model so that
it can compute the game score correctly. Please pay attention this program
does not require GUI; you write test cases to drive the implementation of
the data model.

