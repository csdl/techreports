\begin{abstract}

Software development is slow, expensive and error prone, often resulting in products with a large number of defects which cause serious problems in usability, reliability, and performance. To combat this problem, software measurement provides a systematic and empirically-guided approach to control and improve software development processes and final products. However, due to the high cost associated with ``metrics collection'' and difficulties in ``metrics decision-making,'' measurement is not widely adopted by software organizations.

This dissertation proposes a novel metrics-based program called ``software project telemetry'' to address the problems. It uses software sensors to collect metrics automatically and unobtrusively. It employs a domain-specific language to represent telemetry trends in software product and process metrics. Project management and process improvement decisions are made by detecting changes in telemetry trends and comparing trends between different periods of the same project. Software project telemetry avoids many problems inherent in traditional metrics models, such as the need to accumulate a historical project database and ensure that the historical data remain comparable to current and future projects.

The claim of this dissertation is that software project telemetry provides an effective approach to (1) automated metrics collection and analysis, and (2) in-process, empirically-guided software development process problem detection and diagnosis. Two empirical studies were carried out to evaluate the claim: one in software engineering classes, and the other in the Collaborative Software Development Lab. The results suggested that software project telemetry had acceptably-low metrics collection and analysis overhead, and that it provided decision-making value at least in the exploratory context of the two studies. 



%Software development is slow, expensive and error prone, often resulting in products with a large number of defects which cause serious problems in usability, reliability, and performance. To combat this problem, software measurement provides a systematic and empirically-guided approach to control and improve development processes and final products. Experience has shown excellent results so long as measurement programs are conscientiously implemented and followed. However, due to the high cost associated with ``metrics collection'' and difficulties in ``metrics decision-making'', many organizations fail to benefit from measurement programs.
%
%In this thesis, I propose a light-weight approach to software measurement called \textit{``software project telemetry''}. 
%It addresses the ``metrics collection cost problem'' through highly automated measurement machinery: software sensors are written to collect metrics automatically and unobtrusively. 
%It addresses the ``metrics decision-making problem'' through a domain-specific language designed for the representation of high-level perspectives on software development process.
%In contrast to traditional model-based approaches which involve the comparison of metrics from different projects, software project telemetry focuses on the comparison of metrics from the same project at different times. This within-project data comparison involves a much smaller time scale: typically with intervals of days or weeks. The idea is that comparison can be made more effectively between two different periods of the same project than between two different projects. It thus avoids many problems a model-based approach suffers from, such as spending the cost of accumulating a historical project database first and then constantly worrying about whether the current project is �comparable� to those in the database. In software project telemetry, the metrics from the initial period of the project are used to establish a baseline and bootstrap the process. Project management and process improvement decisions are made by detecting changes in telemetry trends and comparing trends between two periods of the same project. In-process control for a project that is still under development is made possible precisely because comparisons are made within the same project instead of across projects.
%
%The claim of this thesis is that software project telemetry provides an effective approach to (1) automated metrics collection and analysis, and (2) in-process, empirically-guided software development process problem detection and diagnosis.
%Two empirical studies were carried out to evaluate the claim: one in software engineering classes, and the other in the Collaborative Software Development Lab (CSDL). 
%The classroom study was relatively simple. I distributed a questionnaire at the end of the semester to collect the student's opinions about software project telemetry. To increase my confidence in the validity of their self-reported opinions, I analyzed their telemetry analysis invocation pattern to determine the extent to which their opinions were based on the actual system usage. 
%In the CSDL study, I pursued a much more in-depth data collection and analysis strategy over a much longer period of time. I introduced software project telemetry as a metrics-based process improvement program and explored its use in the lab extensively. I collected data from observations and interviews, and used the grounded theory approach to generate hypotheses. I also tested the hypotheses in a limited way by making changes to the telemetry system or implementing new facilities to see if the hypothesized outcome would come true. 
%The results from the two studies suggested that software project telemetry had acceptably-low metrics collection and analysis overhead, and that it provided decision-making value at least in the exploratory context of the two studies. 
%
%The studies also generated a number of insights with respect to real-world use of software project telemetry. Software project telemetry delivers best value when it can be customized to the specific needs of an organization. The customization includes both setting up sensors to collect metrics and designing charts to perform analyses. Due to automated nature of metrics collection, a broken sensor might be undetected for some time. Special-purpose telemetry charts can be designed to help make a quick assessment of whether the underlying sensors are likely broken or not by flagging suspicious metrics. Data privacy concerns, especially with effort-related personal process metrics, might be an adoption barrier for the technology. Though the implementation provides a mechanism to limit the kinds of data that could be accessed by people other than the owner, overcoming this issue seems largely dependent on what the data are used for in an organization.
%
%Two pieces of software are made publicly available as a result from this research. One of them is a server-side component that enables a user to ``actively'' explore relationships among different metrics using a web browser. The other is a client-side application that can be configured to retrieve and display a sequence of telemetry charts automatically and continuously, providing the user with ``passive'' awareness of project status.

\end{abstract}