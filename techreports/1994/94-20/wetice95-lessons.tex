%%%%%%%%%%%%%%%%%%%%%%%%%%%%%% -*- Mode: Latex -*- %%%%%%%%%%%%%%%%%%%%%%%%%%%%
%% wetice95-lessons.tex -- 
%% Author          : Carleton Moore
%% Created On      : Thu Dec  8 10:55:03 1994
%% Last Modified By: Carleton Moore
%% Last Modified On: Wed May 10 14:43:25 1995
%% Status          : Unknown
%% RCS: $Id: wetice95-lessons.tex,v 1.8 1995/05/11 00:43:31 cmoore Exp $
%%%%%%%%%%%%%%%%%%%%%%%%%%%%%%%%%%%%%%%%%%%%%%%%%%%%%%%%%%%%%%%%%%%%%%%%%%%%%%%
%%   Copyright (C) 1994 University of Hawaii
%%%%%%%%%%%%%%%%%%%%%%%%%%%%%%%%%%%%%%%%%%%%%%%%%%%%%%%%%%%%%%%%%%%%%%%%%%%%%%%
%% 


\section{Lessons learned about strong collaboration}

In this section, we summarize some the principal findings from our first
evaluation of AEN.  

\begin{enumerate}

\item {\em Users as well as artifacts should be visible.}

  All collaborative hypertext document systems provide access to the artifacts
  created by the authors, but few provide access to the authors
  themselves.  In AEN, providing knowledge of who was using AEN, where
  they were, and a means to communicate with them created many new
  opportunities for strong collaboration without requiring face-to-face
  interaction.
  

\item {\em Provide direct and indirect authoring mechanisms.}

  AEN supports two methods for collaborative authoring, which we term
  {\em proof-reading} and {\em trading the lock}.  In the proof-reading
  method, the author creates a node and publishes it by allowing read and
  annotate access.  This allow others to make comments suggesting
  changes, new ideas, or just general comments on the subject.  The
  author can read the comments and make changes to the document or
  comment on the comments.
  In the trading the lock method, the author creates a node and allows
  other authors to edit it by providing read and write access.  Each
  author can lock the node, make a change, save the node, and unlock it,
  which updates the contents of the node displayed on each author's
  screen.   Strong collaboration is enhanced by providing authors with
  these degrees of control over the document and styles of
  interaction with others. 

\item {\em Provide context-sensitive ``what's new.''}

  When a group divides into subgroups and is actively and incrementally
  building hypertext documents, context-sensitive mechanisms to
  automatically inform users of what has changed are
  essential. Otherwise, the users will suffer from either lack of
  knowledge about what is changing (if no change-related mechanisms
  exist) or a low signal-to-noise ratio (if context-free change-related
  mechanisms are used, in which case users are informed of many changes
  that are irrelevant to them.)  In AEN, the combination of access
  control, unread nodes, and Hyperstar Bulletin provides a very nice
  means of selectively propagating change-related information across
  groups.  The daily Hyperstar Bulletin encourages users to log in to AEN
  only when necessary to see changes, and access control allows users to
  control the visibility of their changes.

\item {\em Provide access to intermediate work products.}

  One of the strongest enablers of strong collaboration is easy
  accessibility to intermediate work products.  Synergy is nurtured
  by permission to review another's admittedly rough, first pass at an idea,
  where comments and suggestions are aimed at refining and enhancing,
  rather than confirming or denying, as is frequently the case with final,
  polished presentations.


\item {\em Maintaining database integrity is essential to effective
  collaboration.}

  This might seem like a no-brainer, but we learned the hard way that
  collaboration can be completely undermined by lost data.  Even
  ``standard'' measures may not be enough: soon after instituting a daily
  backup mechanism, we lost user data due to the convergence of (a) a
  holiday weekend (Thanksgiving), (b) a filled disk drive, and (c) a
  power failure.  We've moved to four backups of the system daily, in
  version 2.3.13.


\item {\em An agent-based architecture may be necessary for strong
  collaboration.}

  Several key facilities for strong collaboration, such as snoopy and
  unread nodes, were easily implemented in terms of agents that monitored
  the HBS/user event stream and used this to build and maintain auxiliary
  state information.  Such an approach was crucial to preventing
  excessive overhead on the central HBS server or individual user client
  processes.  As shown in Figure \ref{fig:aen-architecture}, the current
  version of AEN contains five independent agent processes for such
  purposes. Given the ``state-full'' requirements for strong
  collaboration, we suspect that an agent-based architecture will be
  essential to providing advanced support for such collaboration for
  significant numbers of users with acceptable system responsiveness.


\item {\em The WWW is not an effective vehicle for strong collaboration.}

  While we are great enthusiasts for the WWW, we believe it is not
  well-suited to strong collaboration.  Our experiences with AEN have
  convinced us that strong collaboration occurs through an entwining of
  {\em authoring} and {\em learning}. Support for this entwining requires
  the system to contain substantial knowledge about the current state of
  each user, and mechanisms to propagate the appropriate kind of state
  information about each user to other users when needed.  The WWW is a
  fundamentally stateless system, and so cannot provide full support for
  strong collaboration.

\end{enumerate}






