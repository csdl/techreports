%%%%%%%%%%%%%%%%%%%%%%%%%%%%%% -*- Mode: Latex -*- %%%%%%%%%%%%%%%%%%%%%%%%%%%%
%% report.tex -- 
%% Author          : Sang-Woo Han
%% Created On      : Thu Feb  3 10:16:21 1994
%% Last Modified By: Taeho Yum-499SM94
%% Last Modified On: Mon Jun  6 22:27:07 1994
%% RCS: $Id: jmis.tex,v 1.5 94/02/18 11:13:31 johnson Exp $
%% Status          : Unknown
%%%%%%%%%%%%%%%%%%%%%%%%%%%%%%%%%%%%%%%%%%%%%%%%%%%%%%%%%%%%%%%%%%%%%%%%%%%%%%%
%%   Copyright (C) 1994 University of Hawaii
%%%%%%%%%%%%%%%%%%%%%%%%%%%%%%%%%%%%%%%%%%%%%%%%%%%%%%%%%%%%%%%%%%%%%%%%%%%%%%%
%%%  
\documentstyle{article}

\begin{document}
{\Huge{\bf Deep thoughts }}\\
\\
\\
\begin{flushright}
by jsj, shan, tyum\\
\end{flushright}
\begin{enumerate}
\item {\bf Priority of matches}\\
The simplest method of matching priority is probably a first come
first serve system.  The first match that can be found will be given
priority.

\item  {\bf Expiration of requests}\\
What if someone enters a request to sell and doesn't find a match for
a couple of weeks and that person goes and sells it to someone outside
the system?  Can we trust users to send in cancelation requests?  What
would be a fair procedure for canceling requests?  I personally don't
think we can rely on users to send in cancelations, so the system
should do this automatically so that the database does not grow out of
control.  One method would be to cancel a request if a match notice is
not answered in a reasonable amount of time.

\item  {\bf Pricing}\\
How will the books be priced?  I think people would rather set their
own prices.  If sellers are going to set their own prices, then we
probably have to have a more elaborate matching scheme.  We might have
to expand it to handle people's asking price and prices offered for
books.  Are we going to allow people to do some dealing?  One of the
beauties of capitalism.  I don't think there will be a lot of wheeling
and dealing, but I think it's reasonable for people to be able to take
a few bucks less than they asked or for someone to pay a few bucks
more for than they offered.  This will probably complicate the
matching process.  I'll have to think about this one a lot more.


\item  {\bf The transaction}\\
Now lets pretend a match is made and all is agreed upon.  How will the
transaction be handled.  Should we have a pickup and drop off place?
Should we have students meet on their own?  This does not seem as
important as the rest of the system at the moment.


\item  {\bf Contition of books}\\
How do we handle this?  There is no way for anyone to tell the
condition of a book unless the book is actually inspected.  How do we
prevent misunderstandings about the condition of books.  Maybe we can
define a set of criteria for levels of book conditions.  Now this
issue further complicates the matching process.


\item  {\bf Some notes about a possible model for the bookstore}\\
An email system.  People will email requests to sell or requests to
buy.  A match will be made by the system and the respective people
will be notifed by mail.  If the match is ackowledged by both people,
the transaction will be made.

We decided to keep the the buyers and sellers anonymous so that they
will not bypass the system.  This would give us the chance to monitor
the usage of the system.  One possible way to prevent exchange of
unwanted information in the system is by accepting a formatted
message.

We can have an intro mail message with instructions on how to
construct a request mail.  Just had a though.  How about if we had an
interactive session to send in requests?  Maybe a telnet session.  How
about if we actually let the people see the 'inventory' of books
available and books wanted?  Gonna have to give those a lot more
though.  I'll sleep on it and report more later.


\item {\bf Free Market ?}\\

I think we do have to keep this a free market system.  We have to let
people set their own prices.  It makes it really hard for us, but we
are diligent people who want to serve the university students.


\item  {\bf More on matching}\\

I think given the highest priority, a person would want to best price
and the best condition.  What would be more important?  In what way do
we let someone specify what they want?  Maybe we can let them tell us
the lowest condition book they are looking for?
Maybe we can set thresholds for condition and price.


\item {\bf Fairness in selecting books}\\

In a regular bookstore, usually the first person to pay for a book is
the first one to get it--not the first one to see the book.  We were
thinking of letting someone reserve a book provided they have
priority.  Is it fair for someone who was given a match to wait a day
or two while other eager buyers have cash now?  Even if that person
requested a book first?  This can cause a lot of unnecesary delays.


\item  {\bf Waiting periods}\\

If someone can reserve a book, how long do we wait until we free up
the book for someone else?  With different scenarios, a person can
wait long periods for a book that people reserved and never got.
How often do we expect students to check mail?


\item {\bf  Penalize for over reservations}\\

We might alleviate the situation of reserving books, if we kept people
in check with their reservations.  They should in some way be held to
their offers.  I think in the real-estate market, if someone makes an
offer and the offer is accepted, the person is legally responsible to
some extent to honner that offer.


\item  {\bf Without reservations}\\

How would we work this thing, if we didn't let people reserve books
based on priority?  Do we send multiple offers to one person and let
him take the best one?  How long do we give that person to respond?
This seems to be agreed upon for the moment.


\item {\bf Speed over personal satisfaction}\\

This brings up the issue of speed over person satisfaction that
someone got the best deal possible for a given amount of time.  The
fastest method would be to match offers to buy and sell directly as
soon as possible.  That would sell all books as fast as possible
provided that people will actually go with it.  If we throw in a lot
of things like the condition of books and if we let people choose what
prices they want, things will seemingly go a lot slower, but it also
seems like people will get better deal.  It would just take longer.


\item  {\bf What does a student selling expect to do?}\\

They would expect to send in the request at the very least.  Would
they want to field a bunch of different offers, or would they rather
just be notified if there is a buyer for a book?
We are lazy, but mostly poor so we would rather field a bunch of
offers to get the best deal.


\item  {\bf What does a student buying expect to do ?}\\

Does a student buying a book want to just be notified if there is a
book available, or does that person want to get their hands dirty and
find the best deal they can?  Same as above.


\item  {\bf Interactive system for browsing}\\

We could have an interactive system for the buyer to look at the
inventory and with respective prices.  That would simplifiy the
requests to buy.  A buyer can instead look through the inventory and
give an offer to a book he is interested in.  Of'course we let them
only offer once for the same book.  We have to give a time limit for
offers to be accepted.


\item  {\bf Mail vs Interactive Telnet}\\

We can easily give sellers to use either mail or a telnet session.
They just have to follow a specific format to enter their sales.

Buyers can specify what books they want and we can send them a list of
books and prices.  Of'course, if they choose that, they will not get
up to date prices.


\item {\bf Programmers benefits}\\

Programmers get 10 karma points for each succesful transaction made by
the system.  With enough karma points, programmers can then shave
their head and experience nirvana.  :)\\
That's it for now I guess.  Will think more later when more hair grows
back.
\end{enumerate}
\end{document}


