%%%%%%%%%%%%%%%%%%%%%%%%%%%%%% -*- Mode: Latex -*- %%%%%%%%%%%%%%%%%%%%%%%%%%%%
%% bookstore.tex -- 
%% Author          : Robert Brewer
%% Created On      : Thu Feb  3 10:16:21 1994
%% Last Modified By: Philip Johnson
%% Last Modified On: Fri Jun  3 13:08:06 1994
%% RCS: $Id: jmis.tex,v 1.5 94/02/18 11:13:31 johnson Exp $
%% Status          : Unknown
%%%%%%%%%%%%%%%%%%%%%%%%%%%%%%%%%%%%%%%%%%%%%%%%%%%%%%%%%%%%%%%%%%%%%%%%%%%%%%%
%%   Copyright (C) 1994 University of Hawaii
%%%%%%%%%%%%%%%%%%%%%%%%%%%%%%%%%%%%%%%%%%%%%%%%%%%%%%%%%%%%%%%%%%%%%%%%%%%%%%%

%%%  


\documentstyle [11pt,/group/csdl/tex/definemargins,
                /group/csdl/tex/lmacros,
                /group/csdl/tex/ps-times]{article}

\begin{document}

\title{Requirements for a Virtual Bookstore}
\author{Philip M. Johnson\\
  Collaborative Software Development Laboratory\\
  Department of Information and Computer Sciences\\
  University of Hawaii\\
  Honolulu, HI 96822\\
  Tel: (808) 956-3489\\
  Email: johnson@hawaii.edu}

\date{ICS/CSDL Technical Report 94-05}


\maketitle 

\section{Motivation}

The University of Hawaii, as with all other major universities, supports a
flourishing used book business.  This business can be roughly characterized
in the following way:

\begin{enumerate}
\item A student decides that a book that they own is no longer useful to them.
\item The student offers the book to the bookstore.
\item The bookstore determines whether or not a market exists for the book
     (i.e. that a professor will require the book the following term, and
     that course enrollment justifies "investing" in this book.)
\item The bookstore evaluates the condition of the book and makes an offer
     to the student.
\item The student accepts or rejects the offer.
\item The bookstore offers the book for sale with a sizable markup (typically 50-100%).
\item The bookstore either succeeds or fails in selling the book during the
     following semester.  If the bookstore fails to sell the used book, it
     must decide whether or not to: (a) keep it for the following semester
     and attempt to sell it again, in which case it incurs overhead for storage;
     (b) sell the book to a national used book distributor, or (c) throw
     away the book.  
\end{enumerate}

The purpose of this research project is to explore the requirements for a
``virtual bookstore'' by exploring the process alternatives and outcomes
for how such a bookstore would be organized and run.  After this initial
analysis, an instrumented, collaborative environment will be designed and
built that will allow students to buy and sell books where the current
``physical'' bookstore has been replaced by an on-line, ``virtual''
bookstore running on uhunix.  A simple electronic mail interface will be
developed to allow any current UH student with basic computer skills and a
uhunix account to successfully use the system.  More elaborate interfaces
will also be developed for skilled computer users, as well as for bookstore
administrators.  Several different process models for this virtual
bookstore will be developed and experimentally evaluated through actual use
of the system by the UH student community.  Systematic changes to the
bookstore process model will explore the impact upon fairness, efficiency,
and effectiveness of variations in information access and information
requirements for the roles of buyer and seller.

\section{Process Modelling a Virtual Bookstore}

In general, the goal of a virtual bookstore is very simple: create a more
efficient and thus lower-cost method for students to exchange used books by
eliminating the physical bookstore from the transaction.  This can benefit
both buyer and seller: for example, a selling price mid-way between the
physical bookstore's buying cost and its subsequent markup can
simultaneously give the student selling the book more money and save the
student buying the book an equivalent amount of money.  

The increased efficiency of a virtual bookstore is produced through
eliminating the overhead and risk incurred by a physical bookstore: no
salaries are paid to buy, sell, and stock books, nor is there a risk of
buying a book that cannot be sold.  In some respects, the virtual bookstore
is similar to a ``just-in-time'' inventory system.

While this goal is simple and obvious, the design of a process requires
more reflection than might first be apparent.  First, the design of the
process should satisfy, as much as possible and as often as possible (if
not guarantee), the following constraints:

\begin{itemize}
\item {\bf Fairness.}  The process should lead to the buyer paying a reasonable and
    equitable price for the goods received. The process might also try to
    guarantee that the longer a buyer waits for a book, the higher the
    probability that the buyer will be able to buy the next instance of the
    book that becomes available for sale.

\item {\bf Efficiency.} The process should occur fast: a seller wanting to sell a
    book should be able to do so as quickly as possible, and a buyer
    wanting to buy a book should be able to do so as quickly as possible.

\item {\bf Effectiveness.}  The process should lead to all sellers being able to
    sell their books if buyers are available, and all buyers being able to
    buy their books if sellers are available.
\end{itemize}

To illustrate just a few of the issues and interactions, consider the
following example process model.

\subsection{The ``Public BBS" process model}

This first process model is perhaps the simplest: sellers post a book for
sale to a public database, and buyers peruse the database and contact
sellers if a book is offered that they have an interest in.  While simple,
this process model can fail to satisfy all three essential requirements:

{\em Fairness.} Sellers can be unfair by falsely advertising a book
as being in ``excellent" condition when it is actually in poor
condition.  Buyers can be unfair by getting their friends to ``flood"
the database with false offers for the books of interest at unfairly
low prices.  This would artificially depress the perceived ``going
rate" for a book, leading to an unfairly low price for the honest
sellers.

The process also does not support fairness in the sense of
preventing ``starvation''---a buyer may never be able to buy a book,
even though instances of the desired book become available, simply
because other buyers continually ``get there first".

{\em Efficiency.} The bookstore must handle hundreds, if not thousands of
transactions, but the process model does not specify how to expire listings
or how to organize the database.  Sellers must presumably decide for
themselves how and where to post the notice, which may lead to
inconsistencies and ``lost" postings.

The process model states only that buyers can browse for books that they
want.  To maximize effectiveness, then, buyers must be continuously
checking the bookstore to see if a desired book has just been added, which
is very inefficient.

{\em Effectiveness.} As noted above, if not well organized, the
bookstore will lead to ``lost'' postings---situations in which a buyer
and seller do not connect, either because the buyer cannot find the
seller's listing (and thus the seller loses out), or because the
buyer does not check frequently enough and another buyer gets it (in
which case the buyer loses out).

\section{Evaluation of Requirements for a Virtual Bookstore} 

My intuition says that there will be no ``perfect" process model for a
virtual bookstore---one that simultaneously satisfies the fairness,
efficiency, and effectiveness requirements.  This is great, because it
makes the design process an interesting one: which requirements must be
sacrificed, and to what degree, in order to fulfill others?  

My intuition also says that although it may be possible to theoretically
sketch out the pros and cons of a particular process model, that in fact
there is no way to really understand its strengths and weaknesses except by
experimentation. 

To investigate the requirements for a virtual bookstore, I recommend work
on the following tasks:

\begin{itemize}
  
\item {\em Literature Review.} It will be helpful to learn more about such
  systems.  A virtual bookstore system was recently designed for the Anderson
  Graduate School of Management \cite{bookstore-AGSM94}.  How does the
  process/data model suggested in this report fair with respect to fairness,
  effectiveness, and efficiency?

\item {\em Process Model Design.} Concurrently with literature research,
  try constructing a few process models for a virtual bookstore using Egret.
  Specify the process and data model as precisely as you can, using whatever
  notation you feel is useful.

\item {\em Process Model Evaluation.} For each model, evaluate its
  strengths and weaknesses with respect to fairness, efficiency, and
  effectiveness.  Can you discover other criteria to be used in addition to
  these three?
  
\item {\em Process Model Measurement.} For each model, design on-line
  measurements of the process that can provide an empirical basis for
  evaluating the fairness, efficiency, and effectiveness of the process.
  
\item {\em Process Model Experimentation.} Eventually, we will decide upon
  one or more models to implement using Egret and evaluate using the
  specified measures.  This experimentation will be designed and 
  evaluated according to the CSDL research process model \cite{csdl-ro-93-01}. 

\end{itemize}

\bibliography{/group/csdl/bib/bookstore,/group/csdl/objectives/csdl-objectives}
\bibliographystyle{plain}

\end{document}
