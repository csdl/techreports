\documentstyle[nftimes,12pt,/group/csdl/tex/definemargins]{article}
\input{/group/csdl/tex/psfig/psfig}                

\begin{document}
\noindent

\title{Virtual Bookstore Model}

\author{Tae Ho Yum\\
	Sang Woo Han\\
	John S. Johnson\\
        Department of Information and Computer Sciences\\
        University of Hawaii\\
        Honolulu, HI 96822}

\date{CSDL Techreport 94-05}


\maketitle

\section{Intro}

This is a proposal for an electronic bookstore.  This document will
address the basic model of the bookstore.  The actual specifics will
be introduced later.\\
\\
Our direction came down to two types of systems.  We dubbed them the
"Lazy Transaction System" and the "Eager Transaction System"\\
\\
The Eager system was basically a system that operated like an
electronic marketplace.  It basically served as a central place to do
trading.  In this system, a student is assured to get a better deal or
to sell their book quickly if they were eager to do a little work.
Thus the name "Eager Transaction System"\\
\\
The Lazy system is a system that does all the work.  A student will
basically send in a request to buy or sell and the system will do all
of the work.  It will try to match up as many sales as possible and
keep everyone fairly satisfied.  All a student has to do is put their
faith in the system.\\
\\
From our experiences with university students, we opted for the Lazy
system.  The more we thought about it, the more we came to conclude
that buying and selling books is not interesting.  It is, however,
still a necesary chore that would be best if carried out in the most
efficient manner.  Efficiency in this case would be the greatest
number of sales for the least amount of human activity.  Who better to
give the uninteresting work than a computer?\\

\section{Some principles of the model}

Users will send in requests to buy or sell describing exactly what
they want.\\
They will also specify ranges of prices and conditions of
books.\\
If a match is found within the user specified thresholds, then it will
be assumed that they will take the deal.\\
If the computer cannot make a match within user specified thresholds,
then it will try to make the best fit.\\
In any case where the computer makes a decision outside of what the
user allowed it to, the user will be allowed to bargain.\\
Time is also a factor.  Users can specify the amount of time they need
to sell the book by.\\

\section{The basic workings of the model}

\subsection{Condition of books}

For our model we will be considering the condition of the books.  We will
define a linear scale for the condition of books.  A linear scale should make
it easier to implement a matching algorithm.  We will clearly define the
criteria for the categories of books and hopefully most people will agree.  For
this writing we will use the letters A,B,C,D,F for the book categories.\\

\subsection{Data input from buyers and sellers}

Seller:\\
1.	Book\\
2.	Condition\\
3.	Asking price\\
4.	Lowest accepted offer\\
5.	Time to sell book by\\
6.	Personal information\\
\\
Buyer:\\
1.	Book\\
2.	Conditions wanted\\
3.	Asking price\\
4.	Highest accepted offer\\
5.	Time to buy book by\\
6.	Personal information\\
\\
We will specify for them to be strict in entering price and condition ranges so
that they will pretty much accept if there is something in that range.  The
system will still find matches for them even if it's not in their range.
This data can be entered either by a telnet session or by a formatted email
message.\\
\\
\\

\subsection{The matching process}

We decided to have the system do the actual matchings at the end of the day or
after some other fixed time period.  For now we will assume that there will be
enough entrys by the end of the day.\\
\\
\\
Best fit matches:\\
\\
Here we had an argument over the "best fit".  Should we try to obtain as many
matches as possible within user specified thresholds at the expense of breaking
up some perfect matches (where both asking prices and conditions match)?  The
argument there is what percentage of these "close" matches will actually result
in sales.  Or could we try to get as many perfect matches as possible at the expense of the
number of close matches?  The argument there is that that these perfect matches
should result in a guaranteed sale.
It's basically low risk-low yield or high risk-high yield.\\
\\
For now we will go with the most possible matches.\\
\\
\\
We will make the most matches possible and send out notices to the people
matched.  Those requests are now on hold until we get a confirmation from the
users.  All requests that have not been matched will cary over to be matched
the next day.  If both parties agree, then great, we release personal
information and let the transaction take place.  If someone does not agree,
then we put that request back as active again.  We are assuming that most
people will take offers that are withing their specified range.\\
If a match is not made by a certain amount of time, then the computer should go
and make a match that is close, but not within the user specified threshold.
That time period could be a week or so.  So someone using the system can expect
a match every week.  Notices will be send like the other matches.  If either
party refuses, then we can at that point let them bargain.  All possible (buyer
and seller exists) matches will be made by the end of that time period.  This
is so we don't have any requests that float in the system forever.\\
If a user specified a time period they need to sell the book by, then a match
will be made for the person by that time no matter what.\\
\\
\\

\section{Algorithm specifics}

Matches within user specified thresholds:\\
\\
Scenario:  If a person wanted \$30-25 for a book and a person is willing to pay
\$30-25 for a book, then at what price do we notify the parties to buy/sell.
Should we let them meet half way or go to either end.  It would probably be
fair to give both of them the half way price in any overlap.\\
\\
This would really be a simple model if we didn't have to deal with the
condition of the book.  If a buyer specified A-C condition and \$30-50, he
most likely does not want a C book for \$50.  The graph would look something
like this:\\

\begin{verbatim}

	A              xxxxxxxxx
	B      	       xxxxxxxxx
	C              xxxxxxxxx
	D
	F

	   0  10  20  30  40  50  60



They probably want a graph that looks more like this:

	A                   xxxx
	B      	         xxxx
	C              xxxx
	D
	F

	   0  10  20  30  40  50  60

\end{verbatim}


Should we let them specify the price range for each book?\\

\end{document}
