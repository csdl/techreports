%%%%%%%%%%%%%%%%%%%%%%%%%%%%%% -*- Mode: Latex -*- %%%%%%%%%%%%%%%%%%%%%%%%%%%%
%% ispw94.tex -- 
%% RCS:            : $Id: icse94.tex,v 1.2 94/02/15 13:49:55 johnson Exp $
%% Author          : Philip Johnson
%% Created On      : Thu Feb 10 11:15:01 1994
%% Last Modified By: Philip Johnson
%% Last Modified On: Tue Mar  1 10:15:12 1994
%% Status          : Unknown
%%%%%%%%%%%%%%%%%%%%%%%%%%%%%%%%%%%%%%%%%%%%%%%%%%%%%%%%%%%%%%%%%%%%%%%%%%%%%%%
%%   Copyright (C) 1994 University of Hawaii
%%%%%%%%%%%%%%%%%%%%%%%%%%%%%%%%%%%%%%%%%%%%%%%%%%%%%%%%%%%%%%%%%%%%%%%%%%%%%%%
%% 
%% History
%% 10-Feb-1994		Philip Johnson	
%%    

%% Submission to the 9th International Software Process Workshop

\documentstyle[twocolumn,ieee-hacked,/group/csdl/tex/lmacros]{article}
\input{/usr/uh/lib/tex/TeXPS/macros/psfig}

\begin{document}

\makeieeetitle
  {Toward an Enactable Process Modelling Language\\
           for Formal Technical Review}
  {Philip M. Johnson\\
   Department of Information and Computer Sciences\\
   University of Hawaii\\
   Honolulu, HI 96822\\
   (808) 956-3489\\
   {\tt johnson@hawaii.edu}}

   
   \makeieeeabstract 
   {
   Formal technical review (FTR) is an essential
   component of all software quality assessment, assurance, and
   improvement techniques.  However, current FTR practice leads to
   significant expense, clerical overhead, group process obstacles, and
   research methodology problems.
  
   CSRS is an instrumented, computer-supported cooperative work
   environment for formal technical review.  This position paper
   discusses our current redesign of CSRS. Previously, the system
   implemented a single FTR method.  The upcoming version will be a
   toolkit for FTR method definition providing a data and process model
   language for formal technical review. This language is expressive
   enough to define virtually all currently published formal technical
   review methods.}



\section{Background}

Formal technical review (FTR) is a cornerstone of software quality
assurance.  However, the labor-intensive and manual nature of review, along
with basic unresolved questions about its process and products, means that
review is typically under-utilized or inefficiently applied within the
software development process.  

For the past two years, we have been designing, implementing, and
experimenting with a computer-supported cooperative work environment for
formal technical review called CSRS \cite{csdl-93-17}.  CSRS runs in a
Unix/X-windows/Ethernet environment, and is built using a client-server
collaborative framework where a single C++ database back-end
services multiple Lucid Emacs client front-end processes \cite{csdl-92-01}.
Users perform review asynchronously within a hypertext environment that
links source code artifacts to raised issues to resolutions, and so on.
Figure \ref{fig:issue-screen} gives a sense for the look and feel of CSRS.

\section{From CSRS 2.x to 3.x}

CSRS was initially designed with two major goals.  First, it should improve
the efficiency of review activities through a variety of computational
services to reduce clerical and adminstrative overhead.  Second, it should
support empirical investigation into improved review methods through
fine-grained, high quality instrumentation of the review process and
products.

The currently implemented design of CSRS (Versions 2.x) meets these goals.
First, it provides a specific data and process model for formal technical
review called FTArm (for Formal, Technical, Asynchronous Review).  FTArm
leverages off results from research into both formal technical review and
collaborative work environments to provide a method that overcomes many
of the group process problems inherent in manual review practice. Figure
\ref{fig:process-model} illustrates the major phases in this process.

Second, CSRS provides extensive tool support. For example, tools facilitate
and enforce the reviewer's duties, help the moderator control the review
process and consolidate the results of review, and generate a LaTeX
document that serves as a comprehensive report of the review.  Finally,
tools provide fine-grained measurementof the review process at a far
greater level of detail than is possible with manual approaches.  In CSRS,
for example, it is possible to obtain insight into the number of minutes
spent by each participant on individual components of the review artifact
(such as each individual function and variable definition, in the case of
code review). CSRS also provides unique insight into the behavior of
participants during review (such as the sequence in which the components of
the artifact are studied and critiqued, the number of sessions required to
finish review and their individual length, etc.).  Such instrumentation
provides valuable information for process assessment and improvement.

Unfortunately, the data and process model is ``hardwired'' in Version 2.x
of CSRS, which leads to a variety of problems. To resolve them, we are
implementing a new version of CSRS.  In Version 3, CSRS implements a
framework (or toolkit) for defining computer-supported formal technical
review processes.  The FTArm review method that was hard-wired into CSRS
2.x will be simply one among many possible review methods that can be
implemented using the CSRS Version 3 framework.

Version 3 will improve the quality of a current Ph.D. research project in
our lab, in which differences between three different review methods are
under experimental assessment.  With Version 3, creating a family of
related review methods that are guaranteed to be the same in certain
respects while differing in other respects is both quick and reliable,
since a declarative language is provided to specify review methods.  With
Version 2 of CSRS, creating a family of related review methods is both slow
and unreliable, since one must physically rewrite the implementation to
create these different behaviors.

Version 3 will also facilitate technology transfer.  Through experience
with Version 2, conversations with many prospective corporate sponsors, and
study of technology transfer literature, we have realized that the step
from a manual FTR method (or no method at all) to CSRS/FTArm may be too
much and/or simply not appropriate for many organizations.  Instead, a more
incremental, organization-centric approach will have greater chance
of success.  This involves assessing the current state of FTR within the
organization, then designing an initial computer-supported FTR method that
represents only a small change from their current process, but which adds
useful value and reduces the current costs of FTR.  Once that initial
method is adopted, the organization may be ready to move toward a more
sophisticated method that provides new value and reduces costs further by
additional changes to the process. Such an approach does not assume that
all organizations will eventually converge upon FTArm as their ultimate
review method.  While FTArm is a very promising review method with many
fine properties, we do not believe it to be the holy grail of FTR.

The final motivation for CSRS 3.0 is more theoretical. In the design of
CSRS 3.0, we are attempting to create an executable data and process
modelling language that can represent the most central features of all
currently published review methods.  We believe that success in this
attempt will be of significance to the software engineering community,
since few current process modelling systems can claim to be both general
and enactable across a given domain.

\section{Language Summary}

We conclude this position paper with a very brief description of the major
process and data language constructs in Version 3.  Given the declaration
of the process and data model using these constructs, CSRS Version 3.0 will
generate a system and interface providing automated support and conformance
to this model.

\begin{itemizenoindent}
  \item {\em Define-method.}  Defines the sequence of phases in the FTR
  method. All current FTR methods are essentially sequential in nature;
  there are few conditionals and virtually no iteration (between phases).
  \item {\em Define-phase.} Defines a phase in terms of its synchronicity,
  the review technique, and its entry and exit conditions.
  \item {\em Define-participant.}  Defines a participant, including their
  e-mail address and the roles they will play during review.
  \item {\em Define-role.}  Defines a role, such as moderator, producer,
  reviewer, reader, etc.
  \item {\em Define-operation.} Defines a review operation, and provides
  for context-sensitivity by constraining its validity to specific roles,
  phases, node-schemas, or field-schemas.  
  \item {\em Define-node-schema.}  Defines a node type, and specifies its
  position within the inheritance hierarchy of nodes.
  \item {\em Define-field-schema.}  Defines a component within a node, and
  its representation, layout, and storage properties. 
  \item {\em Define-link-schema.} Defines a link type, and constrains its
  source and destination node types, along with other properties. 
\end{itemizenoindent}

The above constitute approximately half of the language constructs, and
these descriptions provide only the most superficial perspective on their
meaning and use.  The language does not trivialize the definition of FTR
data and process models.  For example, we have redefined FTArm using this
model and found it requires 106 calls to our language, in a file containing
over 1000 lines.  Nonetheless, this provides an fundamental shift in our
architecture from a procedural specification of process to a declarative,
domain-specific one.

This research provides insights into several topics of interest to this
workshop, including: the modelling of humans and tools; automation of
manual processes; the design and implementation of experiments and
experimental data; and the relationship between process modelling and
computer-supported cooperative work technology.  I look forward to
contributing these insights and profiting from the insights of others.


\begin{figure*} 
 {\centerline{\psfig{figure=/group/csdl/techreports/93-22/issue-private.ps}}}
\caption{{\em A CSRS screen illustrating the generation of an issue.}}
\label{fig:issue-screen}
\end{figure*}

\begin{figure*} 
 {\centerline{\psfig{figure=/group/csdl/techreports/93-22/process.ps}}}
\caption{{\em The seven phases in the FTArm method, along with the primary
entry condition for each phase.}}
\label{fig:process-model}
\end{figure*}


\bibliography{/group/csdl/objectives/csdl-objectives,/group/csdl/bib/c++,/group/csdl/bib/research-process,/group/csdl/bib/code-inspection,/group/csdl/bib/csdl-trs}

\bibliographystyle{plain}

\end{document}




