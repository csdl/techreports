%%%%%%%%%%%%%%%%%%%%%%%%%%%%%% -*- Mode: Latex -*- %%%%%%%%%%%%%%%%%%%%%%%%%%%%
%% scharding-tech-transfer.tex -- 
%% Author          : Philip Johnson
%% Created On      : Sun Jun 19 08:01:45 1994
%% Last Modified By: Philip Johnson
%% Last Modified On: Sun Jun 19 09:58:28 1994
%% Status          : Unknown
%% RCS: $Id$
%%%%%%%%%%%%%%%%%%%%%%%%%%%%%%%%%%%%%%%%%%%%%%%%%%%%%%%%%%%%%%%%%%%%%%%%%%%%%%%
%%   Copyright (C) 1994 University of Hawaii
%%%%%%%%%%%%%%%%%%%%%%%%%%%%%%%%%%%%%%%%%%%%%%%%%%%%%%%%%%%%%%%%%%%%%%%%%%%%%%%
%% 

\section{CSRS: The organization-centered design}

\subsection{Motivation}

Our first publications on CSRS attracted a great deal of interest from
industry.  During 1993, almost 100 organizations contact our research group
to express an interest in learning more about this approach to
computer-supported FTR.  While successful collaborative arrangements have
been initiated with several companies, we were surprised at the number and
degree of negative responses to our design.  These responses revealed the
following major issues with our principle-centered design:

\begin{itemize}
\item {\em Process Complexity.} We failed to take into account the
  incremental effect to the method's complexity of designing the ``mother
  of all FTR methods.''  Whenever necessary to resolve a problem, we added
  yet another phase or computational mechanism to the method.  While most
  manual FTR methods had 3 or 4 phases, ours ended up with seven.  Many
  organizations expressed a fear that this process model was simply too
  complex for them to implement.  Such organizations were typically ones in
  which manual FTR methods had proven too complicated or difficult to
  administer, and who had begun investigating computer-based FTR with the
  hope that it would be more simple than the manual alternative.  In fact,
  we were offering them a far more complex method.

\item {\em Instrumentation.} Many organizations expressed fears about the
  level of instrumentation present in our system. The ability to record
  what an employee is doing on a minute by minute basis is frequently
  perceived as an invasion of privacy; an Orwellian ``big brother''
  mentality.  While the literature on FTR is adamant that FTR measurements
  not be used for management purposes (i.e. for assessing employee
  performance for hiring/firing/promotion), many people believed such data
  would prove irresistable to certain managers.
  
\item {\em On-line nature.} Some organizations found the all-or-nothing
  nature of our on-line support to be problematic.  From our
  principle-centered design perspective, it was necessary that all aspects
  of review be carried out on-line in order to maximize our computational
  support, to provide control over the process and products of FTR, and to
  provide the highest quality measurements possible.  Many organizations,
  however, believed that it would be pragmatically impossible to carry out
  all review activities on-line. 
  
\item {\em Combined research and practice.} The principle-centered design of
  CSRS combined together two orthogonal issues in FTR: research and
  practice.  Addressing both of these issues added substantial design
  complexity and made the process of installing and using CSRS difficult.
  Many organizations objected to the notion implicit in the design that
  they would need to simultaneously researchers and practitioners in FTR in
  order to use CSRS.
  
\item {\em Unix, X-window, and Emacs dependence.} The choice of platform
  for CSRS was guided by our own expertise and resource availability.  For
  many organizations, our choices are arcane and create difficulties for
  adoption in a PC-based environment.  One organization questioned why we
  simply didn't use Lotus Notes.

\end{itemize}

As we began soliciting opinion and support from organizations and learned
of these issues, we realized that the problems with our design were deeper
than simply the choice of platform or the granularity of measurement, but
were actually a result of our fundamental driving force---a
principle-centered design.  To respond to these objections required not
simply minor changes to the implementation, but rather a re-evaluation and
re-focussing of the premise behind the design of computer supported FTR.
This process, which is still underway in our research group, involves a
shift in focus from prior research findings to current and future
organizational needs---an organization-centered design process.

Our current organization-centered design perspective has produced a
dramatic change in our perspective on the requirements for
computer-supported FTR, and has prompted a complete redesign of CSRS.
We no longer view the goal of a computer-supported FTR environment as one
which should address as many issues in FTR research (and/or practice) as
possible.  From our organization-centered design perspective, the essential
requirements for a computer-supported FTR environment are:

\begin{itemize}
\item To facilitate the transition of a company from a high cost, low
  effectiveness FTR method to a lower-cost, higher effectiveness FTR method,
  
\item To support an evolutionary, incremental process involving a
  succession of ``transitional'' FTR methods,
  
\item To support the fact that FTR methods are ultimately context and
  environment specific; that no single FTR method is optimal for all
  organizations.
\end{itemize}

From an organization-centered design perspective, FTArm is still an
interesting and significant contribution to the literature on FTR, because
it provides a concise illustration of how many common issues in FTR can be
resolved through computer-support.  However, FTArm must also be viewed as
somewhat ``pedagogical'' in nature: it is not the method to which all
organizations should aspire, and it should certainly not be the first
computer-supported FTR method that an organization should adopt.

\subsection{Overview of CSRS 3.0}

The new requirements for computer-supported FTR forced a complete
reconceptualization of CSRS.  The essential outcome of that process was the
recognition that CSRS should provide a ``fourth generation language'' for
FTR: a means for organizations to leverage off the insights of FTArm in an
incremental and organization-specific manner.  To provide this, the
essential components of FTArm (as well as other common FTR methods) were
abstracted into a process and data modelling language for FTR.  An
organization wishing to implement an FTR method can declaratively specify
the phases, roles, review artifact types, and so forth, and from this CSRS
generates a collaborative system to enact the method.  Organizations are
free to disable such FTArm essentials as instrumentation, for example, if
they believe that such a feature would impede adoption of the method.  On
the other hand, such features are easily added on in an incremental fashion
if the organization later reaches a point where empirical process
measurement and improvement can be successfully instituted.

The following briefly summarizes the important language constructs in CSRS
3.0:

\begin{itemize}
\item {\em Define-method.} Defines the sequence of phases in the FTR
  method. All current FTR methods are essentially sequential in nature;
  there are few conditionals and virtually no iteration (between phases).
  
\item {\em Define-phase.} Defines a phase in terms of its synchronicity,
  the review technique, and its entry and exit conditions.
  
\item {\em Define-participant.} Defines a participant, including their
  e-mail address and the roles they will play during review.
  
\item {\em Define-role.} Defines a role, such as moderator, producer,
  reviewer, reader, etc.
  
\item {\em Define-operation.} Defines a review operation, and provides
  for context-sensitivity by constraining its validity to specific roles,
  phases, node-schemas, or field-schemas.
  
\item {\em Define-node-schema.} Defines a node type, and specifies its
  position within the inheritance hierarchy of nodes.
  
\item {\em Define-field-schema.} Defines a component within a node, and its
  representation, layout, and storage properties.
  
\item {\em Define-link-schema.} Defines a link type, and constrains its
  source and destination node types, along with other properties.
\end{itemize}

The above constitute approximately half of the language constructs, and
these descriptions provide only the most superficial perspective on their
meaning and use.  The language does not trivialize the definition of FTR
data and process models.  For example, we have redefined FTArm using this
model and found it requires 106 calls to our language, in a file containing
over 1000 lines.

With this new version of CSRS, we will be able to embark upon a
qualitatively different form of collaboration with interested industrial
sponsors.  Rather than attempting to ``sell'' the benefits of adopting
FTArm to them based upon the principles of FTR as developed in the research
literature, we can instead attempt to work with them to discover whether,
when, and in what form computer-supported FTR can improve their software
quality.  Once we have collaboratively developed a perspective on what form
of computer-based FTR could help them, we can then begin work on designing
an incremental process of technology transfer and adoption.  Even if an
organization believes that fine-grained measurement would provide
significant benefits, for example, it may also be possible that immediate
implementation of such a feature could inhibit the adoption of the
technology.  

We envision that with CSRS 3.0, methods used by organizations for initial
adoption of computer-supported FTR will always be very simple, consisting
of just one or two phases and with very general goals, objectives, and
measurements.  Once a simple method has been successfully adopted within
the organization, requirements for enhancements will emerge naturally to
drive the evolution of the method.

















