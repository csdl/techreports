%%%%%%%%%%%%%%%%%%%%%%%%%%%%%% -*- Mode: Latex -*- %%%%%%%%%%%%%%%%%%%%%%%%%%%%
%% citl-old.tex -- 
%% Author          : Philip Johnson
%% Created On      : Sat Sep 17 11:11:57 1994
%% Last Modified By: Philip Johnson
%% Last Modified On: Thu Oct 13 15:24:18 1994
%% Status          : Unknown
%% RCS: $Id$
%%%%%%%%%%%%%%%%%%%%%%%%%%%%%%%%%%%%%%%%%%%%%%%%%%%%%%%%%%%%%%%%%%%%%%%%%%%%%%%
%%   Copyright (C) 1994 University of Hawaii
%%%%%%%%%%%%%%%%%%%%%%%%%%%%%%%%%%%%%%%%%%%%%%%%%%%%%%%%%%%%%%%%%%%%%%%%%%%%%%%
%% 

\documentstyle[nftimes,twocolumn,/group/csdl/tex/lmacros]{article}
\input{/group/csdl/tex/psfig/psfig}

\begin{document}

\title {Collaboration in the Small \\ vs. \\ Collaboration in the Large}
\author  {Philip M. Johnson\\
          Collaborative Software Development Laboratory\\
          Department of Information and Computer Sciences\\
          University of Hawaii\\
          Honolulu, HI 96822\\
          (808) 956-3489\\
          {\tt johnson@hawaii.edu}}

\maketitle

\begin{abstract}

  Our work building and evaluating collaborative systems has led us to
  the conclusion that there are some fundamental architectural decisions
  that must be made to support different styles of collaboration.  In
  this paper, we characterize two such styles, which we term
  ``collaboration in the small'' and ``collaboration in the large''.  We
  then discuss the architectural requirements for each of these styles
  and their mutual incompatibility, and illustrate with common systems.
  Our thesis is that designers of collaborative systems must make an
  explicit decision on which style they intend to support and design the
  architecture appropriately.

\end{abstract}

\section{Introduction}


Since 1991, the Collaborative Software Development Laboratory at the
University of Hawaii has been designing, implementing, and evaluating
various forms of collaborative systems.  We have implemented CLARE, a
system for collaborative learning (cite), CSRS, a system for collaborative
software review and quality improvement (cite), and URN, a collaborative
USENET reader (cite).  Recently, we have designed and implemented a
collaborative system called AEN, which creates a "virtual classroom" and is
being used to explore issues in collaborative learning and authoring.

All of these systems have been built using Egret, a framework for
implementing domain-specific, collaborative hypertext applications (cite).
Egret has gone through three distinct and total redesigns during its three
years of existance.  The first version of Egret was essentially a proof-of-
concept prototype which immediately taught us some hard facts of
collaborative life: client- server, multi-user, interactive, hypertext
database applications are extremely difficult to build from scratch.  Our
attempts to build systems with the first version of Egret did not scale
beyond a few hundred nodes, were slow and inflexible, and offered little
that could not be implemented more quickly and reliably with a decent,
fourth generation database package.

However, the first version of Egret provided us with a vision of what a
collaborative system could be.  This vision had the following facets.

First, and fundamentally, the database and its fundamental services should
be accessable both interactively (so that users could exploit them) and
programmatically (so that domain-specific applications could be built on
top of them).

Second, a client-server architecture leads inexorably to a processing
bottleneck.  To delay the impact of this as long as possible, Egret should
provide customizable local caching of the structure and contents of the
database.  If such caches could be implemented efficiently, and if
applications were intelligent about their use, then queries over the
database could be answered in many cases entirely by the client without a
database access, or at least the amount of database access could be cut
down substantially.

In the second version of Egret, we set out to satisfy these goals by
redesigning our client-side software, which was based upon Epoch (foot:
Historically speaking, Epoch is the cousin of Emacs 18, the spouse of Lucid
Emacs Version 19, and the grandparent of Xemacs).  We did not redesign the
server at that time, which helped us to realize that programmatic
manipulation of the database can support the creation of agents, or
autonomous client processes.  The second version of Egret provided one
agent, which served as a kind of auxiliary server process by maintaining
certain kinds of structural information about the database.

The second version of Egret, while more sophisticated, efficient, and
scalable, still suffered from performance and reliability problems.  The
root of these problems lay in the server: it had not been designed to
support the kind of architecture Egret had evolved into, where basic
structural information (such as the types of nodes and links) were being
maintained by what the server perceived as "client" processes.

The third, current version of Egret is the product of the redesign and
reimplementation of the server to support the Egret architecture, along
with yet another complete redesign of the client side software to
incorporate the lessons learned from this first three years of research.
The resulting system is very impressive, at least in comparison to its
ancestors: the raw speed of the system has increased 5-7 fold (primitive
node creation has decreased from over 1 second per node to .2 seconds/node
under the normal network load of our campus ethernet system).  Just as
importantly, we have redesigned the protocol between client and server to
more efficiently communicate information: many queries and updates can now
be accomplished in half the number of server operations required under
Version 2 of Egret.

In its current version, Egret is powerful, efficient, and reliable enough
for us to realize our vision.  We can finally build the kind of domain-
specific collaborative applications we've wanted without taxing the
computational platform.  For example, AEN provides a "virtual classroom"
experience, a kind of cross between a hypertext document, a review system,
and a MUD.  So far this semester, we've had over a dozen simultaneous users
of AEN, and the time it takes to pull down a Lucid Emacs menu often exceeds
the time it takes to perform the operation selected.

Now that we have finally succeeded in building an architecture that
fulfills our vision for collaboration, it has clarified what this vision
actually is.  The catalyst for this clearing was the choice in Egret
Version 3 to use a variant of HTML as part of the internal node
representation language.  Once we decided to do this, we had to begin
asking ourselves some frightening, but exhilarating questions: what is the
difference between an Egret database and WWW?  Is Egret simply an in-house
version of Mosaic?  Why not just use Mosaic instead of Egret?  Our
tentative answers to these questions have led to the concepts of
collaboration in the small, which Egret is good at and Mosaic is bad at,
and collaboration in the large, which Mosaic is good at and Egret is bad
at.  Let's leave Egret at this crossroads for a moment and discuss WWW and
Mosaic.

\end{document}



 
