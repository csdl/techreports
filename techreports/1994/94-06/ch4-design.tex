%%%%%%%%%%%%%%%%%%%%%%%%%%%%%% -*- Mode: Latex -*- %%%%%%%%%%%%%%%%%%%%%%%%%%%%
%% ch4-design.tex -- 
%% Author          : Philip Johnson
%% Created On      : Fri Jun 17 11:15:03 1994
%% Last Modified By: Philip Johnson
%% Last Modified On: Fri Jun 17 11:25:51 1994
%% Status          : Unknown
%% RCS: $Id$
%%%%%%%%%%%%%%%%%%%%%%%%%%%%%%%%%%%%%%%%%%%%%%%%%%%%%%%%%%%%%%%%%%%%%%%%%%%%%%%
%%   Copyright (C) 1994 University of Hawaii
%%%%%%%%%%%%%%%%%%%%%%%%%%%%%%%%%%%%%%%%%%%%%%%%%%%%%%%%%%%%%%%%%%%%%%%%%%%%%%%
%% 

\chapter{The CSDL Way: Design Conventions}

\silentfootnote{This chapter is a revised version of CSDL-TR-93-02.}

\section{Introduction}

This chapter presents an general overview of the software engineering
principles underlying the Egret system and its applications. It 
defines the notion of object oriented design as it relates to Egret
applications written in Emacs lisp, and what conventions have evolved 
to support this process.


\section{Notational Conventions}
\label{conventions}

This section introduces notational conventions.  With rare exceptions, {\em all}
Emacs Lisp functions written as part of an Egret application should conform
to these conventions.   Adherence to these conventions are particularly
important since automated tools (such as DSB) are designed to leverage off
this knowledge about the structure of Egret systems, as revealed by the
use of identifier names. 

\subsection{Identifier Syntax}

Function and variable names in Egret have a standard format.
Each name must adhere to the following template:

\small\begin{verbatim}
<sys-name>:<sys-vis><class-name><class-vis><name>
\end{verbatim}\normalsize

\noindent where:
\begin{itemize}
  
\item {\tt <sys-name>} is a single character that identifies the
  subsystem membership of the object.  Currently, {\bf u} identifies
  the utilities subsystem, {\bf gi} identifies the generic-interface
  subsystem, {\bf t} identifies the type subsystem, and {\bf s}
  identifies the server subsystem.  Sys-names have historically been chosen 
  to be unique across *all* Egret applications, though this will probably
  not continue to be true in future. 

\item {\tt <class-name>} is an identifier for the class name.
  
\item {\tt <sys-vis>} and {\tt <class-vis>} are single characters
  indicating the external visibility of the object. The character
  {\bf *} indicates that the object is public, the character {\bf !}
  indicates that the object is private, and the character {\bf @}
  indicates that the object is a system administration function to be
  manipulated only by distinguished users at special times (such as
  database initialization, recovery, and so forth).  

  \item {\tt <name>} is the actual name of the operation or
  attribute.

\end{itemize}

For example, the function {\bf t*node*set-name} is the type subsystem
public operation {\sf set-name} from the class {\sf node}.  These prefixes
both serve to document the class and subsystem membership of operations,
and to create separate namespaces.  For example, the server subsystem also
defines a node class with a set-name operation, but the two operations have
distinct names: {\bf s*node*set-name}.

In this document, boldface font is used when referring to a specific Egret
operation or attribute, such as {\bf s*hbs*make}.  Sans serif font
is used when referring to class names or operation names in general, such
as {\sf node} or {\sf delete}.  The typewriter font is used when referring
to Emacs or Epoch functions, such as {\tt title}, or when displaying
sections of code that may include Egret and Emacs functions, such as:

\small\begin{verbatim} 
(foobar (i*screen*name scrn) "New Screen")
\end{verbatim}\normalsize

\subsection{Naming Conventions}

Beyond naming syntax, Egret also has conventions for naming certain
kinds of semantically related operations.  These conventions may be
repeated elsewhere in this document, but are collected here for
handy reference:

\begin{itemize}

\item Constructor and instantiation operations are prefixed by {\bf make-}.

\item Deletion operations are prefixed by {\bf delete-}.

\item Operations that set the value of an attribute are prefixed by {\bf set-}.

\item Recovery operation names are by convention prefixed with {\bf reset-},
and whenever possible suffixed with the name of the attribute or
operation whose functioning they repair.  For example, the server
subsystem node class attribute {\bf s*node*incoming-links}
would have a corresponding recovery operation called {\bf s*node!reset-incoming-links}.

\end{itemize}

\section{Object Oriented Design Conventions}

The Egret design is object-oriented.  However, ``object oriented'' can
mean very different things to different people.  This section
clarifies our use of object orientation by describing the major
entities in our design: classes, objects, attributes, operations, and
collections.  It also describes how public and private interfaces for
class attributes and operations are defined\foot{The mechanisms for
inheritance and visibility control do not attempt to extend the state
of the art: inheritance is implemented ``manually''.  The goal here is
to introduce a set of concepts for object orientation and visibility
that provide the greatest benefit for the lowest overhead.}.


\subsection{Classes}  

A {\em class}\/ is a collection of objects with related structure and
behavior.  {\sf Node} and {\sf link} are obvious classes in Egret . Each class
has associated with it a set of superclasses, a set of subclasses, a set of
attributes, and a set of operations.  The design of Egret  is  neutral on the
issue of single vs.  multiple inheritance, although no occurrences of multiple
inheritance in the design currently exist.

Classes frequently have constructor and destructor operations for the
objects associated with them.  These operations are always called {\sf
make} and {\sf delete}. If these operations are not specified, then
the class is an {\em abstract class}. Abstract classes exist to
collect together related structure that is inherited by other classes
that do have constructor and destructor operations.

Classes are, in some sense, purely a design-level representation. No
explicit Emacs Lisp support for defining classes exists.  The
reification of classes in Emacs Lisp occurs purely through the use of
naming conventions for functions and variables, and voluntary
observance by programmers of the visibility constraints.  However,
designing and programming in an object oriented fashion will greatly
ease understanding of the system architecture, as well as supporting
migration to other languages providing direct support for these
concepts.

\subsection{Objects} 

Classes are an organizational entity--what actually exists during
execution are particular instances of classes, or {\em objects}.

Each object has a unique identifier associated with it, and this
unique identifier (or the object itself) is virtually always the first
argument to the operations associated with a class.  This convention
implements a simple kind of message passing behavior, where the first
argument to a function call indicates the particular object that the
message (the function) is sent to.  For example, the {\bf t*node*lock}
operation takes a node-ID as its first argument, which can be
interpreted as having the effect of sending a message to that
particular node to lock itself.

A rough design heuristic is that if an entity does not have a
distinguishing unique ID, or if one is never needed to implement its
behavior, then the entity should probably exist as a part of another
class.  Similarly, deciding the primary home for an object operation
can sometimes be resolved by considering the most appropriate first
argument to it.

\subsection{Attributes}  

Each class instance has a set of characteristics, or {\em
attributes}\/ associated with it.  For example, each screen has a
geometry, color, name, and so forth.  These are attributes of the
class.  If a class has an attribute defined for it, then there exists
a function that returns the value of the attribute.  For example, {\bf
t*node*name} returns the {\sf name} attribute for an instance of the
class {\sf node}.

Attributes need not be ``static''.  For example, a reasonable
attribute of the class {\sf node-buffer} could potentially be {\sf
link-under-mouse}---an attribute that returns the link ID of the link
label currently under the mouse cursor (or {\tt nil} if the mouse
cursor is not currently over any link label).

Attribute functions take a single argument, an instance of the class
for which this state value should be retrieved.

Some attributes are {\em setable}.  This means that an operation
corresponding to the attribute is defined to change its value. This
operation is named by prefixing the attribute name with {\bf set-}.
For example, if the {\bf t*node*name} attribute is setable,
then the operation {\bf t*node*set-name} is defined to change
the value of that operation.

Not all attributes should be setable.  For example, it probably
doesn't make sense to have the {\bf node-buffer*link-under-mouse}
attribute mentioned above be setable\foot{Although, you could
conceivably argue for making that attribute setable as a way of
warping the mouse cursor to a particular link label.  But we won't.}.

Attributes can be private or public.  Public attributes are indicated
notationally by the use of {\bf *} as the visibility token; {\bf !} as
the visibility token indicates a private attribute.  If an attribute is
private, then functions outside the implementation of the class should
not access or set the attribute's value.  This, of course, is not
enforced in Emacs Lisp, but is rather a way of helping programmers to 
understand what parts of a subsystem are internal details and can thus
be safely ignored.  If you find yourself as a class implementor needing 
access to a private attribute of another class, you should publicize this
observation immediately.  It may be the case that there is a public mechanism
that you are aware of and should use instead.  Alternatively, it may be
the case that the private attribute should be changed to public.

It is common to have an attribute be publically accessable but only
privately setable.  To do this, define both the {\bf *} and {\bf !}
forms of the attribute, but only define the set operation for the {\bf
!} form. For example, one could define two accessors for screen names,
{\bf gi*screen*name} and {\bf gi*screen!name}, the first public for use
by any class, and the second private to the implementation of the
screen class.  By defining the set operation {\bf gi!screen!set-name},
you provide a consistent mechanism for updating the screen title
internally without making this capability public.

\subsection{Operations}

While attributes refer to characteristics of classes, {\em operations}
refer to the behaviors of the instances of the class.  For example,
one behavior of a node is to be written out or retrieved from the
hyperbase.  These behaviors are represented by the operations
{\sf write} and {\sf retrieve}, with corresponding Emacs functions {\bf
t*node*write} and {\bf t*node*read}.

Operations may take any number of arguments, but the first argument
is, by convention, an object representing the class instance.  In the
case of objects like nodes or links, the first argument is their
unique ID.  In the case of objects like buffers or screens, the first
argument may be the actual buffer or screen object itself.  Operations
must always document what form their first (as well as the other)
argument must take. The operation may take additional required or
optional arguments depending upon the nature of the behavior.


Operations can be public, indicated by the visibility token ``*'',
or private, indicated by ``{\bf !}''.  Public operations are accessable to
other classes, while private operations are only accessable to the
internal implementation of the class. As noted above, this is purely
convention, and unfortunately Emacs Lisp does not enforce this
encapsulation.  To repeat, if you find yourself as a class implementor
needing access to a private operation of another class, you should
publicize this observation immediately.  It may be the case that there
is a public mechanism that you are aware of and should use instead.
Alternatively, it may be the case that the private operation should be
changed to public.

All instantiable (i.e. non-abstract) classes have a constructor and a
destructor operation associated with them.  By convention, these
operations are named {\sf make} and {\sf delete}, and are normally
public.  (It is often useful to implement two constructor operations,
for example {\bf s*node*make} and {\bf s*node!make}.  The first is the
public operation with publically-supplied arguments; the second is the
internal, down-and-dirty constructor that may take many more arguments
or return an internal object that should be hidden from the external
interface.

\subsection{Collections}

Operations, as defined above, operate on individual instances of the
class. There is another kind of operation, however, that operates on
the set of all objects that exist in the class. These are called {\em
collection operations}.  For example, an operation that returns a list
of all nodes of a specified type is a collection operation.  These
operations, taken together, represent many of the classical database
query and retrieval operations supported in Egret .

Collection operations are always defined relative to a class, and are
named by wrapping {\bf \{} and {\bf \}} around the class name.  Unlike
conventional class arguments, collection operations do not have a
distinguished first argument, since their name indicates which
collection the function operates on\foot{Future revisions to this
design may lead to the presence of a first argument that is passed a
collection object, consisting of some subset of objects of the named
type.}.  For example, the collection operation that returns all the
IDs for server-level nodes is {\bf s*\{node\}*IDs}.  This operation
does not take any arguments, since the class that it operates on is
implicit in its name.

Collection operations can be public or private, and their visibility is
indicated in the standard manner by {\bf *} or {\bf !}.
