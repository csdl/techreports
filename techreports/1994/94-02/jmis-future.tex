%%%%%%%%%%%%%%%%%%%%%%%%%%%%%% -*- Mode: Latex -*- %%%%%%%%%%%%%%%%%%%%%%%%%%%%
%% jmis-future.tex -- 
%% RCS:            : $Id: jmis-future.tex,v 1.5 94/02/18 13:56:38 rbrewer Exp $
%% Author          : Robert Brewer
%% Created On      : Thu Feb  3 10:24:28 1994
%% Last Modified By: Robert Brewer
%% Last Modified On: Fri Feb 18 13:56:33 1994
%% Status          : Unknown
%%%%%%%%%%%%%%%%%%%%%%%%%%%%%%%%%%%%%%%%%%%%%%%%%%%%%%%%%%%%%%%%%%%%%%%%%%%%%%%
%%   Copyright (C) 1994 University of Hawaii
%%%%%%%%%%%%%%%%%%%%%%%%%%%%%%%%%%%%%%%%%%%%%%%%%%%%%%%%%%%%%%%%%%%%%%%%%%%%%%%
%% 
%% History
%% 3-Feb-1994		Robert Brewer	
%%    Created


\section{Future directions}
\label{sec:conclusion}


This paper has presented our initial results in designing collaborative
classification and evaluation mechanisms for large, dynamically structured
information systems such as Usenet. The system, URN, combines a
collaboratively built representation for the keywords associated with an
article with an adaptive interface that prioritizes articles based on votes
on previous articles.  A two week experimental evaluation of URN provides
quantitative evidence supporting the design of weighting functions to
automatically and incrementally build a representation of user's interests.
These experiences with URN constitute the first step in a project involving
both short and long term research directions.

\subsection{Extend the URN paradigm}

One short term direction is to gather more experimental data on URN over a
longer time frame. Through straightforward refinements to the current
experimental design, we can generate data that will provide new insights
into the strengths and limitations of this approach.  For example, we would
like to more clearly assess the contributions of collaboratively built
weighting functions by comparing the weights generated using those to the
weights generated using a default mechanism (such as the contents of the
Subject and Author lines without modification).

A further short-term direction is motivated by user-suggested improvements
to the weighting function mechanism. One suggestion is the ability for URN
to provide users with direct access to their weighting functions. On many
occasions users wished that they could add a weighting function to their
profile directly because they were sure that they were interested or
uninterested in a particular keyword. Users also desired better control
over keywords, such as the ability to create synonyms or select from a menu
of keywords.  Finally, users also desired enhanced collaborative
capabilities, such as the ability to recommend a specific article to
another individual.

\subsection{Knowledge condensation through URN}

Along with better understanding the use of collaboration for classification
and evaluation, we will also be exploring a process we call {\em knowledge
condensation} in future research with URN.  Current newsreaders (and
systems such as Usenet in general) suffer from a problem of archiving: how
does one preserve the information obtained through this source in a usable
format.  Current approaches, such as FAQs or simple storage of original
postings quickly become unwieldy and prone to the same problems as Usenet
itself.  The essential problem of archiving is that the structure of
information appropriate to the ``news'' paradigm is not appropriate to an
archival information source.  Returning to our original metaphor, the daily
newspaper is not well-structured for archival purposes: to learn about
World War II, one would not generally desire to read through the daily
newspaper for this four year period.  A far more efficient approach is to
read a book, which is effectively a restructured and condensed version of
the daily events covered by the newspapers.

To support knowledge condensation, users must go beyond simple annotation
of postings with keywords. In addition, high quality postings must be
actively restructured with the addition of hypertext links to related
information archived from prior postings.  While such annotations might
create prohibitive overhead for a single user, we hypothesize that a
collaborative approach can lead to incremental creation and structuring of
a richly interlinked knowledge base of information about a common topic of
interest.  Egret provides excellent infrastructure for this research
project, since it supports distributed, client-server communication, and
strong hypertext facilities. Most importantly, Egret provides a dynamic
type system that supports incremental creation and modification of the
schema-level structure of its underlying database \cite{csdl-93-09}.
Our approach to knowledge condensation may provide useful technology not
only to Usenet but also to future on-line information sources such as 
the Interpedia Project.

\subsection{Active, Agent-based Information Acquisition}

Refinements to URN, and addition of collaboration-centered knowledge
condensation mechanisms will provide the substrate for a longer range
research direction.  We intend to develop our system into an entirely new
paradigm for collaborative knowledge management.  In this paradigm, users
would unite together for a common purpose, such as to learn about a new
programming language, the figure skaters at the Olympics, or networked
organizations.  They would begin reading related Usenet newsgroups, but
only in order to ``teach'' the system about their interests.

Once the system acquires confidence in its representation of the user's
interests, it would begin spawning autonomous agents to search the Internet
through mechanisms such as Gopher, Mosaic, or World Wide Web for related
information in repositories other than the Usenet. The information
retrieved by these agents would be classified, evaluated, and restructured
by participants, which would further improve the capabilities of the agents
to retrieve information relevant to the group.

\section{Acknowledgments}

We would like to thank the other members of Collaborative Software Development
Lab: Rosemary Andrada, Carleton Moore, Danu Tjahjono, and Dadong Wan for their
assistance in preparing this manuscript as well as in the development of
Egret. Robert Brewer would also like to thank Yuka Nagashima
for her help in reviewing this document.
Support for this research was partially provided by the National Science
Foundation Research Initiation Award CCR-9110861.
