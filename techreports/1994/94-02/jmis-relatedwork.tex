%%%%%%%%%%%%%%%%%%%%%%%%%%%%%% -*- Mode: Latex -*- %%%%%%%%%%%%%%%%%%%%%%%%%%%%
%% jmis-relatedwork.tex -- 
%% RCS:            : $Id: jmis-relatedwork.tex,v 1.3 94/02/18 13:52:24 rbrewer Exp $
%% Author          : Robert Brewer
%% Created On      : Thu Feb 17 13:40:17 1994
%% Last Modified By: Philip Johnson
%% Last Modified On: Tue Nov  1 08:52:09 1994
%% Status          : Unknown
%%%%%%%%%%%%%%%%%%%%%%%%%%%%%%%%%%%%%%%%%%%%%%%%%%%%%%%%%%%%%%%%%%%%%%%%%%%%%%%
%%   Copyright (C) 1994 University of Hawaii
%%%%%%%%%%%%%%%%%%%%%%%%%%%%%%%%%%%%%%%%%%%%%%%%%%%%%%%%%%%%%%%%%%%%%%%%%%%%%%%
%% 
%% History
%% 17-Feb-1994		Robert Brewer	
%%    Created

\section{Related Work}

Many different programs have been written to read Usenet. What follows is a
brief survey of some that are related to URN's goals.

\subsection{trn, GNUS, etc}

These are the standard programs used by most Usenet users. They allow
subscription to newsgroups and threads are explicitly represented. Typically a
user will be shown a list of all the threads from a group, and the user can
select any number to be read. The only filtering technique provided is kill
files. There is primitive support for the automated creation of kill files
(i.e. a user may ask to have the Subject line from the article they are reading
deposited in their kill file). There is no support for signalling articles as
especially interesting, nor is there any kind of collaboration support. These
systems are well suited to browsing Usenet, but they are slow and cumbersome
when the number of articles in each newsgroup becomes large. Some newsreaders
(xrn, Tknews, NewsWatcher for the Macintosh) have moved towards graphical user
interfaces to Usenet. While these make it easier for new users to start reading
and participating in Usenet, they don't address the issue of information
overload.

\subsection{strn}

``Scan Threaded Read News'' is an enhanced version of trn. In strn, users can
write patterns that assign a ``score'' to articles that match the pattern,
which are similar to URN's weighting functions.  These patterns must be created
by the user, and the patterns can only search the header portion of the
article. Once articles have been scored, they can be displayed in order of
descending weight. When displaying articles ranked by weight it, like URN, does
not take into account the thread structure of the articles.

strn also adds the concept of virtual newsgroups which can be named by the
user. Each virtual newsgroup consists of articles selected from multiple
newsgroups that match patterns specified by the user. Again, these patterns and
virtual newsgroups must be explicitly specified by the user. The author has
suggested that users might share their scoring files through some mechanism,
which would allow any number of users to moderate a newsgroup.

\subsection{INFOSCOPE}

This newsreading tool allows the creation of virtual newsgroups via filtering
\cite{chi-infoscope-91,cacm-infoscope-92}. Virtual newsgroups are groups of
articles selected from multiple newsgroups that match some series of patterns.
The filters are generated by background agents that monitor users' activity and
display their findings to the user as suggestions. Users can then choose to
either accept or reject the filter suggestions. If an filter is rejected, that
information is stored so that the agent will not attempt to suggest that filter
again in the near future.

The filtering is based solely on individual users' actions and INFOSCOPE does
not provide any collaboration support.  The filters permitted are a subset of
boolean logic, and they can only search header lines of an article, never the
body. INFOSCOPE also provides a graphical user interface for browsing through
the Usenet hierarchy as a tree structure. Because it derives all its
information about articles through their header lines, it cannot determine that
a thread's contents have changed from the actual Subject line which is a
problem ubiquitous to Usenet.

\subsection{Tapestry}

This collaborative document filtering tool \cite{cacm-tapestry-92} contains a
complex query language (TQL) based on SQL which users can use to write their
own queries.  Filters written in TQL can access ``annotations'' or
``endorsements'' created by other users in order to filter a message. For
example, a user can write a TQL script that shows all articles from
comp.unix.wizards that were responded to by another user named Natasha. These
endorsements are similar to votes in URN, and allow for ``virtual moderation''
of newsgroups.  However, filters must all be created manually by the user using
TQL.

Because queries can depend on other users' actions which might occur at any
point in time, Tapestry has adopted {\em continuous semantics} where TQL
queries' results should return the value they would if they were executed at
every instant in time.

Articles can be examined using a Tapestry browser or forwarded via email. As
discussed in \cite{cacm-tapestry-92}, articles were sent via email and are only
prioritized for display in the last step of the Tapestry process using a email
client program that does not have access to the full TQL language.

\subsection{GroupLens}

GroupLens \cite{Resnick94} is a distributed system for gathering and
desseminating ratings of USENET articles via specially modified USENET
clients acting in conjunction with autonomous processes called ``Better Bit
Bureaus.''  Users provide a single digit rating from 1 to 5 on each posting
they read.  These postings are then propogated throughout the USENET
community via newsgroups designed for that purpose, whose contents are then
interpreted by Better Bit Bureaus which use these ratings to predict how
much each user will like an article.  

GroupLens differs from URN in several interesting ways.  First, the
essential goal of GroupLens is to enable users to correlate their tastes in
USENET: if I can find another user with very similar tastes to my own, then
their evaluation of a posting may provide a useful prediction of my own
interest in it. An open question is whether, given the extraordinary
diversity and rate of change in the topics and content in USENET, such 
``aesthetic correlations'' can be made, maintained, and exploited.  

URN, by utilizing the power of collaboration to both classify and rate
articles, eliminates the need for aesthetic correlations: users leverage
off each other by providing a better representation for the content of the
article.  URN users do not need to share the same tastes in USENET in order
to profit from each other's work. 

Second, the information generated by GroupLens clients, due to its
simplicity and the clever use of newsgroups as a transport mechanism, has
the potential to provide an information resource on the scale of USENET
itself.  URN, due to its more sophisticated representation of postings,
cannot scale as easily or effectively as GroupLens: it appears limited to 
relatively small groups (on the order of five to fifty participants) where 
the keyword representation mechanism can be managed.  
