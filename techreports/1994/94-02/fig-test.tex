%%%%%%%%%%%%%%%%%%%%%%%%%%%%%% -*- Mode: Latex -*- %%%%%%%%%%%%%%%%%%%%%%%%%%%%
%% fig-test.tex -- 
%% RCS:            : $Id: jmis.tex,v 1.2 94/02/17 14:02:50 johnson Exp Locker: rbrewer $
%% Author          : Robert Brewer
%% Created On      : Thu Feb  3 10:16:21 1994
%% Last Modified By: Philip Johnson
%% Last Modified On: Fri Feb 18 10:40:11 1994
%% Status          : Unknown
%%%%%%%%%%%%%%%%%%%%%%%%%%%%%%%%%%%%%%%%%%%%%%%%%%%%%%%%%%%%%%%%%%%%%%%%%%%%%%%
%%   Copyright (C) 1994 University of Hawaii
%%%%%%%%%%%%%%%%%%%%%%%%%%%%%%%%%%%%%%%%%%%%%%%%%%%%%%%%%%%%%%%%%%%%%%%%%%%%%%%
%% 
%% History
%% ??-Feb-1994		Robert Brewer	
%%    Created

%%% CSCW '94 Submission on URN and its magical weighing functions.


\documentstyle [12pt,/group/csdl/tex/definemargins,
                /group/csdl/tex/lmacros]{article}

\input{/usr/uh/lib/tex/TeXPS/macros/psfig}

\begin{document}

\ls{2.0}

Users were instructed to read articles using URN, and to vote on each article
based on how interesting it was to them personally. They were also asked to add
or delete keywords from articles when they felt it was appropriate. In order to
gain better insight into URN's weighting function generation capabilities,
users were asked to read all articles, even if they had very large negative
weights. This is quite different from `normal' usage where users
would probably mark all articles below a certain threshold weight as read
automatically.

Users were instructed to read articles using URN, and to vote on each article
based on how interesting it was to them personally. They were also asked to add
or delete keywords from articles when they felt it was appropriate. In order to
gain better insight into URN's weighting function generation capabilities,
users were asked to read all articles, even if they had very large negative
weights. This is quite different from `normal' usage where users
would probably mark all articles below a certain threshold weight as read
automatically.


\begin{figure}[htb]
  \begin{center}
  \small
  \begin{tabular} {|l|c|} \hline
    Number of users & 6 \\ \hline
    Duration of experiment & 10 days \\ \hline
    Number of articles input & 175 \\ \hline
    Total time spent by all users & ~ 13.5 hrs \\ \hline
    Size of body text & 343 Kbytes \\ \hline
  \end{tabular}
  \caption{\ls{1} {\em Statistics from URN experiment. Statistics from
  URN experiment. Statistics from URN experiment. Statistics from URN
  experiment.}}
  \label{tab:general-stat}
  \normalsize
  \end{center}
\end{figure}

Users were instructed to read articles using URN, and to vote on each article
based on how interesting it was to them personally. They were also asked to add
or delete keywords from articles when they felt it was appropriate. In order to
gain better insight into URN's weighting function generation capabilities,
users were asked to read all articles, even if they had very large negative
weights. This is quite different from `normal' usage where users
would probably mark all articles below a certain threshold weight as read
automatically.

Users were instructed to read articles using URN, and to vote on each article
based on how interesting it was to them personally. They were also asked to add
or delete keywords from articles when they felt it was appropriate. In order to
gain better insight into URN's weighting function generation capabilities,
users were asked to read all articles, even if they had very large negative
weights. This is quite different from `normal' usage where users
would probably mark all articles below a certain threshold weight as read
automatically.

Users were instructed to read articles using URN, and to vote on each article
based on how interesting it was to them personally. They were also asked to add
or delete keywords from articles when they felt it was appropriate. In order to
gain better insight into URN's weighting function generation capabilities,
users were asked to read all articles, even if they had very large negative
weights. This is quite different from `normal' usage where users
would probably mark all articles below a certain threshold weight as read
automatically.

\bibliography{/group/csdl/bib/csdl-trs,/group/csdl/bib/urn}
%\bibliography{csrs}
\bibliographystyle{plain}

\end{document}
