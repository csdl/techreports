%%%%%%%%%%%%%%%%%%%%%%%%%%%%%% -*- Mode: Latex -*- %%%%%%%%%%%%%%%%%%%%%%%%%%%%
%% jmis-title.tex -- 
%% RCS:            : $Id: jmis-title.tex,v 1.4 94/02/18 12:10:41 johnson Exp Locker: rbrewer $
%% Author          : Robert Brewer
%% Created On      : Thu Feb  3 13:36:43 1994
%% Last Modified By: Robert Brewer
%% Last Modified On: Fri Feb 18 14:04:05 1994
%% Status          : Unknown
%%%%%%%%%%%%%%%%%%%%%%%%%%%%%%%%%%%%%%%%%%%%%%%%%%%%%%%%%%%%%%%%%%%%%%%%%%%%%%%
%%   Copyright (C) 1994 University of Hawaii
%%%%%%%%%%%%%%%%%%%%%%%%%%%%%%%%%%%%%%%%%%%%%%%%%%%%%%%%%%%%%%%%%%%%%%%%%%%%%%%
%% 
%% History
%% 3-Feb-1994		Robert Brewer	
%%    Created

\title {Toward Collaborative Knowledge Management within\\
Large, Dynamically Structured Information Systems\silentfootnote{This paper
is a revised and expanded version of one currently in submission to the 
1994 ACM Conference on Computer Supported Cooperative Work.}}

%%% \author{Robert S. Brewer and Philip M. Johnson\\
%%%   Collaborative Software Development Laboratory\\
%%%   Department of Information and Computer Sciences\\
%%%   University of Hawaii\\
%%%   Honolulu, HI 96822\\
%%%   Tel: (808) 956-3489\\
%%%   Email: rbrewer@uhics.ics.hawaii.edu, johnson@hawaii.edu}

\author{}

\maketitle


\begin{abstract}
  
  Usenet is an example of the potential and problems of the nascent
  National Information Infrastructure. While Usenet makes an enormous
  amount of useful information available to its users, the daily data
  overwhelms any user who tries to read more than a fraction of it. This
  paper presents a collaboration-oriented approach to knowledge
  management and evaluation for very large, dynamic database structures
  such as Usenet. Our approach is implemented in a system called URN, a
  multi-user, collaborative, hypertextual Usenet reader.  Empirical
  evaluation of this system demonstrates that this collaborative method,
  coupled with an adaptive interface, improves the overall relevance
  level of information presented to a user.  Finally, the design of this
  system provides important insights into general collaborative knowledge
  management mechanisms for very large, dynamically structured database
  systems such as Usenet and the upcoming Information Superhighway.

\end{abstract}

\paragraph{KEYWORDS:}
collaborative filtering, classification, evaluation, usenet, adaptive
interfaces, large databases





