%%%%%%%%%%%%%%%%%%%%%%%%%%%%%% -*- Mode: Latex -*- %%%%%%%%%%%%%%%%%%%%%%%%%%%%
%% 03-04-ch3.tex --
%% Author          : Hongbing Kou
%% Created On      : Fri Sep  5 13:50:18 1997
%% Last Modified By: 
%% Last Modified On: Wed Nov 12 18:09:44 2003
%% RCS: $Id$
%%%%%%%%%%%%%%%%%%%%%%%%%%%%%%%%%%%%%%%%%%%%%%%%%%%%%%%%%%%%%%%%%%%%%%%%%%%%%%%
%%   Copyright (C) 1998 Robert Brewer
%%%%%%%%%%%%%%%%%%%%%%%%%%%%%%%%%%%%%%%%%%%%%%%%%%%%%%%%%%%%%%%%%%%%%%%%%%%%%%%
%%

\chapter{Data Collection with Hackystat}

The IDE for TDD study is Eclipse because it has good refactoring and
JUnit supports. Hackystat sensor plug-in for Eclipse is required to
collect development activity data. In order to catch the possible fast
editing work we will set the state change interval for Eclipse will be much
shorter than thirty seconds.

Chidamber-Kermer metrics of the active java file are collected to study
dynamic metrics behavior of TDD. Beyond the current metrics we can get
super class of the active object is needed for TDD analyses to identify
test classes. 

\begin{table}[!ht]
\caption{Hackystat sensors and Data to be collected}
\begin{tabular}{|l|l|}
\hline Hackystat Sensor & Data \\
\hline Eclipse sensor & Editing, compilation, State Change, buffer
transition and refactoring data \\
\hline BCML sensor (in Eclispe sensor) & Chidamber-Kermer metrics  \\
\hline JUnit Sensor (in Eclipse sensor)& Test Invocation, duration and results \\
\hline
\end{tabular}
\end{table}










