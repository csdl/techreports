%%%%%%%%%%%%%%%%%%%%%%%%%%%%%% -*- Mode: Latex -*- %%%%%%%%%%%%%%%%%%%%%%%%%%%%
%% 03-04-ch1.tex --
%% Author          : Hongbing Kou
%% Created On      : Fri Sep  5 13:50:18 1997
%% Last Modified By: 
%% Last Modified On: Thu Nov 13 17:34:26 2003
%% RCS: $Id$
%%%%%%%%%%%%%%%%%%%%%%%%%%%%%%%%%%%%%%%%%%%%%%%%%%%%%%%%%%%%%%%%%%%%%%%%%%%%%%%
%%   Copyright (C) 1998 Robert Brewer
%%%%%%%%%%%%%%%%%%%%%%%%%%%%%%%%%%%%%%%%%%%%%%%%%%%%%%%%%%%%%%%%%%%%%%%%%%%%%%%
%%

\chapter{Related Work}

``Test-Driven Development (TDD) is a software development practice in which
test cases are incrementally written prior to code
implementation.''\cite{George_2003} Here test case is the unit test, which
is a piece of code written by a developer that exercises a very small,
specific area functionality of the code being tested. The rational of TDD
is to ``Analyze a little, test a little, code a little, and test a
little,repeat.'' The goal of TDD is to write ``clean code that
works''\cite{Beck_TDD_2003} and it has two basic rules. \cite{Beck_TDD_2003}:

\begin{enumerate}
\item Write new code only if an automated test has failed.
\item Eliminate duplication (Refactoring).
\end{enumerate}

\begin{description}
\item They imply an order to do programming task \cite{Beck_TDD_2003}:
\begin{enumerate}
\item \emph{Red}\newline Write a little test that does not work, and perhaps
does not even compile at first.
\item \emph{Green}\newline Make the test work quickly, committing whatever
sins(for example, constant, fake implementation) necessary in the process.
\item \emph{Refactor}\newline Eliminate all the duplications created in
merely getting test work.
\end{enumerate}
\end{description}

In book ``Test-Driven Development by Example'', Ken Beck claimed that
red/green/refactor rhythm is the mantra of TDD, once you have an automated
suite of tests you will never go back. It gives incredible confidence in
your code. He also said TDD will naturally generate 100\% code coverage and a
coverage tool is not necessary at all if you do TDD perfectly.

Unit test is important in software development because it is easy to check
whether the program is functioning properly or not with unit test
suites. It allows the programmer to refactor code at a later date, and be
sure the model is still functioning properly\cite{UnitTest}.The automatic
unit test suite created in the development process could be used as
regression tests too. ``A unit test behaves as executable documentation,
showing how you expect code to behave under the variation condition you've
considered.''\cite{Andy&Dave_2003}

In common software practices unit test process is not disciplined and is
usually done as an afterthought. Occasionally no tests are created at all,
especially with tight schedules. With TDD, before writing implementation,
the automated unit tests are written first and they are created on the
ground of requirements, which leads to creating better and effective test
suites.
  
However, software developers are conditioned to do afterward tests not test
first design thus TDD requires discipline, training and good tool
support. Kent Beck suggested the ``xUnit'' framework when he proposed
Test-Driven Development. ``xUnit'' has many variances and it has been
ported to more than 30 languages' support\cite{XPSoftware}. But to some
applications like Graphic User Interface, database querying etc it is hard to
write small unit test to deploy them or not practical. TDD practitioners
suggest mocking technique to forge the time consuming operations. Two most
famous mock tools are MockObject and EasyMock. With mocked object developer
can write test and program faster thus writing and executing unit test
suites will not take significant amount of time.

Boby George's analysis and quantification of test-driven development
concluded that both students and professional TDD developers appear to have
higher code quality\cite{George_2002}. E. Michael Maximilien and Laurie
A. Williams found that defect rate of project IBM Retail Store
Solution was reduced by 50\% compared to another system built with ad-hoc
unit test approach \cite{Maximilien_2003} and TDD developers passed 18\% more
functional black-box tests than non-TDD developers. 

