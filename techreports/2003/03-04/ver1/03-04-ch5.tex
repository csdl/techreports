%%%%%%%%%%%%%%%%%%%%%%%%%%%%%% -*- Mode: Latex -*- %%%%%%%%%%%%%%%%%%%%%%%%%%%%
%% 03-04-ch5.tex --
%% Author          : Hongbing Kou
%% Created On      : Fri Sep  5 13:50:18 1997
%% Last Modified By: 
%% Last Modified On: Wed Nov 12 18:34:51 2003
%% RCS: $Id$
%%%%%%%%%%%%%%%%%%%%%%%%%%%%%%%%%%%%%%%%%%%%%%%%%%%%%%%%%%%%%%%%%%%%%%%%%%%%%%%
%%   Copyright (C) 1998 Robert Brewer
%%%%%%%%%%%%%%%%%%%%%%%%%%%%%%%%%%%%%%%%%%%%%%%%%%%%%%%%%%%%%%%%%%%%%%%%%%%%%%%
%%

\chapter{Thesis Executive Preliminary}
As we discussed Eclispe will be used as the targeted IDE to evaluate TDD
because it is easy and ready for use, popular in Java development/research
community extensible and we have rich experience on using Eclipse. The good
thing with a universal IDE is that we can eliminate the bias introduced by
environment adaptive issue in the evaluation.

\section{Hackystat Eclipse Activity Sensor}
Hackystat activity sensor for Eclipse was originally written by Takuya, who
is a research assistant and Eclipse evangelist in CSDL. It supports open
file, close file, state change with the given interval, Chidamber-Kermer
metrics of object editing, and unit test invocation with result in Eclipse
IDE. In Test-Driven Development we also care the build, file renaming,
deletion, creation and removing to observe what happens. The new Eclipse
sensor will include the following activities:
\begin{enumerate}
\item Editing
\item Compile/build (Including Eclipse build/rebuild/build all/rebuild all)
\item Delete file   (.java files only)
\item New file      (.java file only)
\item Rename dile   (.java file only)
\item Move file     (From ... to ...)
\end{enumerate}

We add these activities because they are part of or related with the
refactoring. In TDD refactoring is to remove the duplications created in
development process. I am not quite sure whether we can solve the
refactoring identification problem with these new activities but we will
find out with the project progressing.

\section{What's the test case?}
With JUnit framework we believe objects which extend {TestCase} will be
test classes so we could separate test classes and implementation classes
by gathering each object's super class. We can go ahead to add one more
super class to Chidamber-Kermer metric set. It might be a solution but
there is a better one. In TDD practice all tests cases are exercised and
the Eclipse JUnit sensor knows which test casese are executed and test
results. It means we can tell which object is the test class without
matching the activity data and Chidamber-Kermer metrics.  (\emph{This
solution is kind of lame. Basically we could add file to unit test sensor
so that we can get which classes are test classes directly without using
super classes.})

Hackystat JUnit sensor was written for ANT batch execution use such that
there is no file associated with test runs. Fully qualified test class names
and test case names (test methods) are included in unit test sensor data
type. Test class file name as a useful field for TDD analysis will be added
into unit test sensor data type.

\section{Refactoring}
As to Kent Beck's explanation and demonistration in Test-Driven Development
by Example \cite{Beck_TDD_2003} the refactoring in TDD is to remove the
duplication (method, implementation) created. My understanding is that TDD
refactoring includes the following cases:
\begin{enumerate}
\item Move method around (upward or downward)
\item Delete useless class
\item Eliminate outdated testcases (method)
\end{enumerate}

Class file deletion is easy to be recognized with the proposed new activity
sensor. It is not an easy thing to say whether adding/deleting method is
refactoring. We will find more after we dig into the domain.















