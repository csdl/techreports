%%%%%%%%%%%%%%%%%%%%%%%%%%%%%% -*- Mode: Latex -*- %%%%%%%%%%%%%%%%%%%%%%%%%%%%
%% 03-04-ch4.tex --
%% Author          : Hongbing Kou
%% Created On      : Fri Sep  5 13:50:18 1997
%% Last Modified By: 
%% Last Modified On: Thu Nov 13 17:38:41 2003
%% RCS: $Id$
%%%%%%%%%%%%%%%%%%%%%%%%%%%%%%%%%%%%%%%%%%%%%%%%%%%%%%%%%%%%%%%%%%%%%%%%%%%%%%%
%%   Copyright (C) 1998 Robert Brewer
%%%%%%%%%%%%%%%%%%%%%%%%%%%%%%%%%%%%%%%%%%%%%%%%%%%%%%%%%%%%%%%%%%%%%%%%%%%%%%%
%%

\chapter{TDD Recognition and Evaluation}

As the leading factor that drives development, unit test should be light
weight in TDD so that it will not be the reason that defers development
process. Time required to run tests does not take significant amount of
time.  In TDD each step is subtle and the interval between testing and coding
could be just a few seconds in certain conditions. In order to catch this
sort of R/G/R transition we need to set the state change interval, defined
by Hackystat, to a small value such as 10 seconds or less to record
developers' programming activities accurately. Also the existing Hackystat
analyses are created based on 5 minutes most active file abstraction
mechanism which is coarse and not applicable to TDD recognition and
evaluation. The substituted solution is to use raw sensor data to conduct
analyses.

\section{Graphic Representation of TDD}
TDD practice contains lot of iterations of RGR iterations, each of them
starts with a failed test and ends with green bar. The red bar opens the
start of iteration and green bar closes the iteration. Between green bar
and red bar there must have some activities associate with the unit
test. To turn the bar from red and green developers will work on either
implementation or test code. After test passes it is the time to remove the
duplication (refactoring). The token of refactoing is that it starts with
green and ends with green and it is also possible that there may not have
any refactoring at all in a RGR iteration.

As long as we can recognize the RGR iteration of TDD we can represent TDD
process graphically as as Hakan proposed. 

\begin{description}
\item \emph{Requirement}: 
\begin{enumerate}
\item Development activity data include file editing and refactoring activities
\item Unit test invocation and test results
\item Object Metrics such as number of children, line of code
\end{enumerate}
\end{description}
\emph{Limitation}:
We are not sure whether 1 minutes or 10 seconds state change interval is
applicable to TDD or not at this moment. The start and end of RGR iteration
could be hard to define.

\begin{description}
\item\emph{Procedure}:
\begin{enumerate}
\item Get all sensor working well to include refactoring activities
\item Write analysis to recognize RGR iteration
\item Continue working on the analysis to understand the scenarios of RGR
\end{enumerate}
\end{description}

\section{Dynamic Coverage}
Clover/JBlanket data will be used to study coverage evolution. As it is
claimed the statement coverage should be 100\% or close to 100\% excluding
untestable classes. 

\begin{description}
\item\emph{Requirements}:
   ANT sensor for Clover or JBlanket. And we need to have continuous
   integration configuration to observe coverage of the developing projects.
\end{description}

\begin{description}
\item\emph{Procedure}:
\begin{enumerate}
\item Create Clover ANT task
\item Configure cruise control or manually check out projects to calculate
project test coverage
\end{enumerate}
\end{description}

%%\section{Coupling and Cohesion Study}
%%This analysis computes the changing curve of the system's coupling and
%%cohesion. For each data, we will compute average coupling and cohesion of
%%the system.  Drawing changing curve of coupling and cohesion can help us
%%understand whether can maximize the cohesion and minize cohesion between
%%objects.


%%\emph{Requirements}: BCML sensor

\section{Effort Distribution in TDD process}
In most situations, one the code is implemented the work is almost
done. Even though unit tests are required, sometimes developers may
overlook or don't emphasize on unit test especially when the schedule is
very tight\cite{Maximilien_2003}. From developer's and stakeholder's views
unit test are not the end product so that they easily overlook the
importance of unit test. TDD emphesize unit tests. ``If you cannot write
test for what you are about to code, the you should not ever be thinking
about coding.'' L. William's research suggests TDD only takes 16\%  more
effort than the ad-hoc process. With Hackystat we can get activity effort
on test, test invocation and implementation to study the effort
distribution in TDD practice.

\begin{description}
\item\emph{Requirements}: Hackystat Activity Sensor
\end{description}




