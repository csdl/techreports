%%%%%%%%%%%%%%%%%%%%%%%%%%%%%% -*- Mode: Latex -*- %%%%%%%%%%%%%%%%%%%%%%%%%%%%
%% 03-05-ch1.tex --
%% Author          : Robert Brewer
%% Created On      : Fri Sep  5 13:50:18 1997
%% Last Modified By: Aaron Kagawa
%% Last Modified On: Tue Sep 21 17:58:23 2004
%% RCS: $Id: thesis-body.tex,v 1.4 2000/03/17 21:28:10 rbrewer Exp $
%%%%%%%%%%%%%%%%%%%%%%%%%%%%%%%%%%%%%%%%%%%%%%%%%%%%%%%%%%%%%%%%%%%%%%%%%%%%%%%
%%   Copyright (C) 1998 Robert Brewer
%%%%%%%%%%%%%%%%%%%%%%%%%%%%%%%%%%%%%%%%%%%%%%%%%%%%%%%%%%%%%%%%%%%%%%%%%%%%%%%
%%

\chapter{Introduction}

Managing large software projects is intrinsically difficult.  Although,
high software projects is a definite must, other issues like time and cost
plays major roles in large software development.  ``The fundamental issue
is an economical imperative: Produce function at a cost, quality, and
schedule that meet users' needs \cite{humphrey_1985}.''  For example,
if a software company can produce the highest quality products but cannot
predict how long and how much it is going to cost, that company will not
have any business. 

There are many ways to measure software development in order to assess
quality and productivity.  Traditionally, managers asess quality through
measures like defect density, and productivity is assessed through measures 
like lines of code per hour.  The thesis of this research is that, for a
large scale programming projects is it possible to predict quality and
productivity by measuring and monitoring the system's build process.  If it 
is possible to prodict problems with a large software proejct, then is it
possible to prevent these problems from ever occuring.  I will address
these research questions through the development of specific measurements
and analyses of a specific large scale programming project: the Mission
Data System (MDS) at the Jet Propulsion Laboratory.  If sucessful, my
project will not only provide useful support to the software development of 
MDS, but it will also lead to more general hypotheses.  First, I will be
able to make hypotheses concerning the kinds of build measurements and
analyses that are useful for predicting quality and producitivity.  Second, 
the kinds of large scale programming projects that would be amemble to this
research.  And last of all, the kinds of improvements in quality and
productivity that can result.


\section{The Problem of Large Software Systems}
``The scale of large systems has grown by three orders of magnitude in the
last thirty years, and this rate of growth is likely to continue or even
increase in the future \cite{humphrey_1985}.''  Software is growing in size and
maybe even complexity, yet the understanding of large software systems are
relatively low.  NEED MORE GENERAL INFORMATION ABOUT THE PROBLEMS OF LARGE
SOFTWARE SYSTEMS.


%%\section{Build Systems}
%%HAVEN'T FOUND ANY INFORMATION ABOUT LARGE SOFTWARE BUILD SYSTEMS (I haven't 
%%looked that hard though).


\section{The Mission Data System}
The Mission Data System at the Jet Propulsion Laboratory is a project to
provide all space mission software with a common architecture.  It also
provides autonomous tracking of the state of the spacecraft and the system
uses that information to make its own autonomous decisions. \cite{x2000mds}


%%\section{The Harvest Configuration Management and Build System}


\section{The Hackystat Jet Propulsion Laboratory Build System: A system to
predict and prevent problems in MDS.}
The Hackystat Jet Propulsion Laboratory Build System (hackyJPLBuild) is a
specialized extension of the Hackystat System.  Hackystat is a framework
for unobtrusive metric collection and analysis.  To use Hackystat,
developers download and install ``sensors'' in development tools.  These
sensors collect data of the development that occurs, send this data to a
server, which analyzes that data to provide insights back to the developers 
to support improvements in their work products and process.

The hackyJPLBuild system takes alters the traditional use of the ``sensor'' 
and attaches a sensor to the MDS CCC Harvest build tool.  The MDS CCC
Harvest build tool sensor will collect data pertaining to the build process 
of MDS.  The hackyJPLBuild system uses this data to analyze certain aspects 
of the build process.

MORE DETAILS ABOUT HACKYSTAT AND hackyJPLBuild to come.

\section{Thesis Statement}
This research investigates the effectiveness of hackyJPLBuild system and
gathers qualitative and quanititative data in order to assess the following 
general hypothesis:

\begin{description}
\item Thesis Statement:
\begin{enumerate}
\item hackyJPLBuild can accurately represent the software process of the
Mission Data System.
\item hackyJPLBuild will identify threshold values that indicate above
average attributes in the Mission Data System.
\item hackyJPLBuild will predict software problems in the Mission Data
System.
\item hackyJPLBuild will help improve developers' and managers'
  understanding of the MDS system and the systems build process.
\end{enumerate}
\end{description}

The first hypothesis claims that the hackyJPLBuild system along with the
Hackystat System has the ability to accurately represent the software
process of the Mission Data System.  To evaluate this claim, I will conduct
Case Study 1 - Secion 4.1, which validates several implementation level
details of the hackyJPLBuild System.

The second hypothesis claims that the hackyJPLBuild system will be able to
identify threshold values that indicate ``above average'' attributes with in
the software process of the Mission Data System.  An example of a threshold 
is the age measurement, which measures the time it takes for a MDS Package
to reach the Release Harvest State.  To evaluate if hackyJPLBuild can
recognize these thresholds, I will conduct Case Study 2 - Section 4.2, in
which I will attempt to create specialized Hackystat analyses to identify
threshold values for the Age and Throughput measurements.

The third hypothesis claims that the hackyJPLBuild system will be able to
predict software process problems using the threshold values generated in
claim 2.  To evaluate this claim I will conduct Case Study 3 - Section
4.3, in which I will use threshold values to predict future software
process problems.

The last hypothesis claims that the hackyJPLBuild system will increase
the improve the understanding that the developers and managers have of the
Mission Data System and the system's build process.  To evaluate this
claim, I will conduct several qualitative surveys and telephone conferences 
to with the developers and managers of the MDS system.


\section{Structure of the Proposal}
The remainder of this thesis proposal is as follows.  Chapter 2 discusses
previous studies that influenced this research.  Chapter 3 describes the
functionality and architecture of the hackyJPLBuild system.  Chapter 4
discusses the experimental design.  Chapter 5 discusses the expected
results and sample analyses.  Finally, Chapter 6 contains my thesis plan.











