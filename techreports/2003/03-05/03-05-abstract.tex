%%%%%%%%%%%%%%%%%%%%%%%%%%%%%% -*- Mode: Latex -*- %%%%%%%%%%%%%%%%%%%%%%%%%%%%
%% 03-05-abstract.tex -- 
%% Author          : Joy M. Agustin
%% Created On      : Fri Jun  9 09:43:42 1995
%% Last Modified By: Aaron Kagawa
%% Last Modified On: Thu Nov 20 14:47:58 2003
%% Status          : Unknown
%% RCS: $Id: thesis-abstract.tex,v 1.1 1998/09/19 01:24:42 jagustin Exp $
%%%%%%%%%%%%%%%%%%%%%%%%%%%%%%%%%%%%%%%%%%%%%%%%%%%%%%%%%%%%%%%%%%%%%%%%%%%%%%%
%%   Copyright (C) 1995 University of Hawaii
%%%%%%%%%%%%%%%%%%%%%%%%%%%%%%%%%%%%%%%%%%%%%%%%%%%%%%%%%%%%%%%%%%%%%%%%%%%%%%%
%% 


\begin{abstract}

asdfdasf


Managing large software projects is intrinsically difficult.  Although,
high software quailty is a definite must, other issues like time and cost
play major roles in large software development.  For example, if a software 
company can produce the highest quality products but cannot predict how
long and how much it is going to cost, then that company will not have any
business.

Software metrics are one answer to those problems.  Software metrics are
the measurement of periodic progress towards a goal \cite{moller_1993}.
Metrics are used to indicate various problems in a development process.

Currently, there are a large number of documented metrics.  However, there
does not exist one perfect formula to to satisfy every development's quality
and productivity.  The key to a good software metrics program is to be able
to identify the specific goals of the development and be able to assist in
reaching these goals.

I will addres this concept through the development of specific measurements 
and analyses that will improve the quality of a specific system, the
Mission Data System (MDS) at the Jet Propulsion Laboratory.  I will attempt 
to identify certain software metrics that can help JPL reach their
development goals.

To accomplish this I have created the Hackystat Jet Propulstion Laboratory
Build System (hackyJPLBuild).  This system measures and analyzes the build
system of MDS.  The research question of this thesis is, is it possible to
collect data from a build process of a large scale software project, in
order to understand, predict, and prevent problems in the quality and
productivity of the actual system.  To evaluate this research question I
will conduct three case studies: (1) can the hackyJPLBuild system
accurately represent the build process of MDS, (2) can threshold values
indicate problematic issues in MDS, and (3) can hackyJPLBuild predict
future problematic issues in MDS.

Initial results of case study 1 indicate that hackyJPLBuild can accurately
represent the build process of MDS.  In fact, hackyJPLBuild has already
identified some pontetial flaws in the MDS build process.  Case studies 2
and 3 have not been conducted yet.

If this thesis project is successful, I will be able to identify a generic set
of build process software metrics that are applicable to identify, predict
and prevent software qualilty and productivity problems.


\end{abstract}
