\documentclass[11pt,oneside]{article}
\usepackage{fullpage}
%%% Load some useful packages:
%% "New" LaTeX2e graphics support.
\usepackage{graphicx}
%%	using final option to force graphics to be included even in draft mode
%\usepackage[final]{graphicx}
%% Tell graphicx the default directory for all figures
\graphicspath{{figures/}}

%% Enable subfigure support
\usepackage{subfigure}

\begin{document}
\title{Extended abstract for: \\
       \textsc{Software Trajectory Analysis:} \\
       \textsc{An empirically based method for automated software process discovery} \\
       \author{Pavel Senin \\
							 Collaborative Software Development Laboratory \\
							 Department of Information and Computer Sciences \\
							 University of Hawaii \\
         \texttt{senin@hawaii.edu}
       }
       \date{July 2009}
}
\maketitle



\section{Introduction}
For my dissertation research, I propose to implement and evaluate a novel approach for discovering recurrent patterns of software development behaviors based upon automatically collected, low-level product and process data. There is a long tradition in software engineering of proposing specific patterns of software behaviors in order to produce high quality software. For example, the Waterfall Model process describes a sequential pattern in which developers first create a Requirements document, then create a Design, then create an Implementation, and finally develop Tests. The Test Driven Development process describes an iterative pattern in which the developer must first write a test case, then write the code to implement that test case, then refactor the system for maximum clarity and minimal code duplication.

One problem with the traditional top-down approach to process development is that it requires the developer or manager to notice a recurrent pattern of behavior in the first place \cite{citeulike:5043104}. In my research, I will apply knowledge discovery and data mining techniques to the domain of software engineering in order to evaluate its ability to automatically notice interesting recurrent patterns of behavior. As a simple example, consider a development team in which committing code to a repository triggers a build of the system. Sometimes the build passes, and sometimes the build fails. To improve the productivity of the team, it would be useful to understand the recurrent behaviors of the developers. 

My system might generate one recurrent pattern consisting of a) implementing code b) running unit tests, c) committing code and d) a passed build: $i \rightarrow u \rightarrow c \rightarrow s $, while another recurrent pattern is a) implementing code, b) committing code, and c) a failed build: $i \rightarrow c \rightarrow f $. Such automated generation of recurrent patterns can provide actionable knowledge to developers; in this case, the insight that running test cases prior to committing code reduces the frequency of build failures.

The contribution of my research will include: a) the implementation of a system aiding in discovery of novel software process knowledge (shown at the Figure \ref{fig:system_overview}); b) my experimental evaluation of the system which will provide insight into its strengths and weaknesses, and c) the possible discovery of useful new software process patterns.
\begin{figure}[tbp]
   \centering
   \includegraphics[height=65mm]{system_overview.eps}
   \caption{The high-level system overview. Data from users and integration system collected and aggregated by Hackystat, later SAX and event indexes built. Data mining tools perform unsupervised pattern discovery on demand constrained by the domain knowledge. The GUI provides an expert interface for discovered patterns and knowledge base aiding iterative refinement of a discovered phenomena.}
   \label{fig:system_overview}
\end{figure}

\section{Approach and Methods}
\begin{figure}[tbp]
   \centering
   \includegraphics[height=90mm]{fig2.eps}
   \caption{The transformation of Hackystat telemetry streams into uni- and multi-variate symbolic time-series and interval series along with the pattern identification.}
   \label{fig:fig2}
\end{figure}

Many temporal knowledge discovery and data mining methods developed in the last decade can be applied to the software process domain. In my pilot project I have implemented a Symbolic Aggregate approXimation algorithm \cite{citeulike:2821475} which transforms Hackystat telemetry streams (Figure \ref{fig:fig2}, panels $a$, $b$, $c$) into symbolic representation (Figure \ref{fig:fig2}, panels $d$, $f$, $h$). It is fairly easy to extend this algorithm with an interval series option (Figure \ref{fig:fig2}, panels $e$ and $g$). Currently I am using a relational database storing this symbolic data. This approach allows to address data requirements of implemented KDD and clustering algorithms with SQL queries: for example it's very easy to get the temporal motifs frequency vector for each of the telemetry streams or find a set of most frequent motifs across the subset of streams etc. By using this rich data field I am planning to experiment with Interagon Query Language (IQL) for symbolic temporal data \cite{citeulike:5043086}, AprioriAll \cite{citeulike:775528} and Pattern-Growth algorithms \cite{citeulike:5043097} as well as with Episodes \cite{citeulike:5043099} and Partial Order patterns \cite{citeulike:5043101}. For time-interval data I will investigate the applicability of Allen's interval algebra \cite{citeulike:191348} and it's derivatives (UTG \cite{citeulike:5043086} and TSKR\cite{citeulike:3978076}) for the software process domain. 

\section{Related work}
To my best knowledge, the approach I am taking in my research was not explored yet. Most of the current trends among researchers and practitioners in the software process improvements are based on the descriptive process modeling \cite{citeulike:5043670}, though recently, benefits of the mining of software repositories and archived communications were broadly recognized \cite{citeulike:5043676}. The most relevant to my research software process mining framework defined as \textit{incremental workflow mining} was developed by Rubin et al. \cite{citeulike:1885717}. This system is built upon a business process mining framework ProM by van Dongen et al.\cite{citeulike:5043673} which synthesises a Petri Net (PN) from a discovered through the SCM mining transition system. Jensen \& Scacchi in \cite{citeulike:5043664} proposed a similar reference model for OSS software processes discovery by extending CVS mining with domain-specific knowledge discovered from the developers communications and some additional artifacts such as a tools and resource usage. 

\section{Current state of the research}
The methods I have tackled so far in my research and in the pilot project development include a wide variety of a time-series approximation, similarity search and pattern matching. While started indexing data with spectral decomposition of time-series, I experimented with Dynamic Time Warping based similarity search and currently working with SAX approximation. During this time I have implemented many of the algorithms in Java and contributed this code to the community by creating JMotif library. The latest version of Trajectory software is capable of indexing time-series with successive visualization of the strong sequential patterns (Figure \ref{fig:fig3} panel $b$); further analyses of the temporal motif frequencies allow classification of developers and telemetry streams by activity patterns (Figure \ref{fig:fig3} panels $c$ and $d$). 

Currently I am working on the capturing of low-level activity patterns from individual telemetry streams aiming creation of a multivariate event streams for the sequential and episodical patterns discovery.

\begin{figure}[tbp]
   \centering
   \includegraphics[height=115mm]{fig3.eps}
   \caption{Trajectory analyses: panel a) shows DevTime activity strems in the Hackystat ProjectBrowser; panel b) shows sequential growth pattern identified by the Trajectory; panel c) depicts developers classification by activity patterns and panel d) shows classification of telemetry streams by frequency of activity patterns.}
   \label{fig:fig3}
\end{figure}

\section{Planning experimental validation}


\section{Timeline}

%%% Input file for bibliography
\bibliography{seninp}
%% Use this for an alphabetically organized bibliography
\bibliographystyle{plain}

\end{document}
