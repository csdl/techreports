\chapter{Conclusion}
\label{cha:conclusion}

This proposal laid out a research plan to investigate the sustainability of energy conservation in a dorm energy competition using a competition website that integrates techniques from environmental psychology in an attempt to improve participants' energy literacy. The competition will employ floor-level near-realtime power meters to allow competition between floors, and make participants more aware of their energy usage. The competition website creates a parallel competition for Kukui Nut points through completion and verification of tasks intended to increase energy literacy. To examine the relationship between energy literacy and the website, and the relationship between energy literacy and energy conservation, an energy literacy survey has been developed and will be administered to the participants.

\section{Anticipated Contributions}

The anticipated contributions of this research are:

\begin{itemize}
	\item An increased understanding of the energy use of residence halls after an energy competition ends.
	\item Insight into the effect of energy literacy on energy conservation in a University residence hall context.
	\item Experience in designing a website intended to foster behavior changes related to energy use, and detailed data about participants' use of the website.
	\item An increase in energy literacy among the participants of the competition.
	\item A permanent metering infrastructure in two residence halls that will permit future competitions and research on those competitions.
	\item Institutional knowledge and logistical infrastructure for performing future competitions.
	\item A reduction in energy use (and therefore cost savings to Student Housing) for the two residence halls in the competition.
\end{itemize}

\section{Future Directions}

There are a variety of directions that can be pursued once this research is complete, such as:

\begin{itemize}
	\item Repeating the energy competition in future years (possibly in more buildings if funding is available), using insights gained from this research. Freshmen are a renewable resource, so the competition can be run once a year with new participants. Professor Johnson already plans to run future competitions, and has submitted an NSF grant proposal to that end. If the data indicates some subset of the website tasks are particularly useful, future competitions could switch to a treatment-based design to investigate those effects more robustly.
	\item Moving beyond residence halls to other buildings on the UHM campus. Does a competition make sense for buildings where faculty and staff are the primary occupants? Outside the dorm, long-term financial incentives generated by returning a portion of financial savings to the departments that conserve energy might make more sense than prizes.
	\item Fostering energy conservation in homes through behavior change. With the growth of the smart grid, near-realtime power usage data will be available to more and more homes. While the direct feedback coupled with the incentive of lower utility bills is likely to lead to some energy conservation, web-based tools have the potential to help motivate behavior change on a large scale.
\end{itemize}

\section{Timeline}

The planned timeline for the research is given below. Note that if the competition should be delayed due to one of the factors in \autoref{sec:risks}, the competition might take place in February 2011 rather than October 2010. Even in this scenario, we believe it is possible to complete the dissertation by the end of the Spring semester in May 2011.

\begin{itemize}
	\item Spring 2010: competition design, website design, buy-in from stakeholders
	\item Summer 2010: install meters in dorms
	\item September 2010: competition website complete
	\item October 2010: competition takes place
	\item November 2010: data analysis and dissertation writing begin in earnest
	\item February 2011: followup study takes place
	\item May 2011: dissertation defense
\end{itemize}
