%%%%%%%%%%%%%%%%%%%%%%%%%%%%%% -*- Mode: Latex -*- %%%%%%%%%%%%%%%%%%%%%%%%%%%%
%% project-background.tex -- 
%% Author          : Philip Johnson
%% Created On      : Tue Mar 31 11:44:58 2009
%% Last Modified By: Philip Johnson
%% Last Modified On: Wed Dec 16 15:29:18 2009
%% RCS: $Id$
%%%%%%%%%%%%%%%%%%%%%%%%%%%%%%%%%%%%%%%%%%%%%%%%%%%%%%%%%%%%%%%%%%%%%%%%%%%%%%%
%%   Copyright (C) 2009 
%%%%%%%%%%%%%%%%%%%%%%%%%%%%%%%%%%%%%%%%%%%%%%%%%%%%%%%%%%%%%%%%%%%%%%%%%%%%%%%
%% 

\section{Background and significance}

% {\em The proposal must provide adequate background for reviewers to judge the novelty, uniqueness, and significance of the proposed research.}

The OPQ research project involves the development of three basic components: the OPQBox power quality monitoring hardware, the OPQHub service for crowdsourced power quality data, and analytics to make the collected data useful. 

\subsection{Power quality monitoring hardware}
\label{sec:background-hardware}

Starting at the high end, a phasor measurement unit (PMU) captures measurements of voltage or current at a rate of 30-60 Hz, and uses GPS to ensure that the timestamps recorded between different PMUs are accurate to approximately 1 microsecond \cite{Zhang2007}. As of 2014, there were approximately 1100 installed PMUs in the North American power grid. The cost of a single PMU hardware device and its installation on a transmission line can reach \$150,000. The PMU user community consists of utilities who install and maintain PMUs at substations or generation plants in order to assess grid stability.

The Wide-Area Frequency Monitoring Network (FNET) is a project by researchers at Virginia Tech based upon a GPS-synchronized single-phase ``frequency disturbance recorder'' (FDR) that can be installed at ordinary 120V outlets \cite{Zhang2010}. Currently, FNET gathers frequency and voltage data from approximately 80 FDRs installed across North America.  By monitoring changes, FNET can detect generator trips (which cause a decline in frequency) and load shedding (which cause an increase in frequency). Because the geographical location of each FDR is known, and because the timestamps are synchronized to less than a microsecond, FNET can be used to triangulate both the original size and location of such events. The FNET user community is a consortium consisting of utilities, power companies, and government groups who pay \$10,000 per year to gain real-time access to the data. 

% In summary, PMUs and FDRs are both wide-area, distributed devices for monitoring power quality at the grid-level through highly synchronized, time-stamped data. PMUs monitor voltage and current, while FDRs monitor frequency and voltage.  Both 
% user communities are utilities and large power technology companies. 

Industrial manufacturing companies form a different user community for power quality monitoring. These companies are not concerned with overall grid quality but only with the quality of the power received at their buildings.  For example, devices such as PQube \cite{pqube} connect to AC power and can collect a variety of power data including voltage, frequency, THD (total harmonic distortion), and reactive power (VAR). PQube data is highly accurate and each device comes with an NIST calibration certificate. A single PQube device can cost over \$5,000.  Companies such as Fluke, PowerSight, and Tektronix also sell devices for measuring power quality problems in industrial or laboratory settings, for prices generally starting around \$1,000.

A partnership consisting of the California Institute for Energy and the Environment, Power Standards Lab, Lawrence Berkeley National Lab, and UC Berkeley are extending PQubes with custom hardware to create ``micro'' PMUs \cite{Meier2013,Meier2014}.  The goal is to manufacture PMUs at a low enough price point to justify their installation below the transmission level, and thus provide utilities with additional data that improves their situational awareness and faster service restoration.  

Residential consumers form a relatively unexplored user community for power quality monitoring.   One of the few commercial products for this user community is the AC Scout \cite{acscout}.  This device plugs into 120V power outlets and can monitor voltage and frequency. The AC Scout is designed only to monitor a single household, and is designed to write entries to a log file when pre-defined thresholds for voltage and frequency are exceeded. The log file can be off-loaded from the device to a computer via a USB cable or sent via email if the ethernet cable connection to the Internet is provided. Since data from one AC Scout is not intended for comparison with others, there is no attempt at synchronization.

\begin{figure}[h]
  \begin{tabular}{l|r|p{2in}|p{1in}|p{1in}} \hline
  {\bf Device} & {\bf Cost} & {\bf Measurements} & {\bf Synchronization} & {\bf Communication} \\ 
  PMU & \$100,000 & frequency, voltage, current & GPS & Secure LAN  \\
  FDR & \$2,500 & frequency, voltage & GPS & Internet \\
  PQube & \$5,000+ & frequency, voltage, THD, VARs & (none) & (none)  \\
  mPMU & \$5,000+ & frequency, voltage, THD, VARs & GPS & Custom network \\
  AC Scout & \$200+  & frequency, voltage & (none) & (none) \\
  OPQ & \$60 & frequency, voltage, THD & NTP & HTTP/SSE \\
  \hline
  \end{tabular}
  \caption{\em \small Comparison of hardware devices for power quality monitoring}
  \label{fig:hardware-table}
\end{figure} 

Figure \ref{fig:hardware-table} summarizes how our OPQ hardware compares to other devices. First, we will monitor total harmonic distortion in addition to frequency and voltage, and we will also store waveform data for the past 24 hours in order to explore the possibility of diagnosis when thresholds are exceeded.  Second, we expect to be able to produce the devices for approximately \$60, a price point similar to conventional power strips with surge protectors. Third, our devices will use WiFi for communication with our cloud-based service via HTTP and Server-Sent Events protocols.  Fourth, our devices will use Network Time Protocol (NTP) for synchronization.  While GPS-based synchronization provides an accuracy below a single microsecond, NTP-based synchronization provides a much lower accuracy of a few milliseconds. Use of NTP limits the kinds of analytics that can be performed with OPQ, but reduces both the cost and constraints upon installation location. 

\subsection{Crowdsourced power quality data}

Up to now, geographically distributed power quality data has been collected by and oriented toward the needs of utilities. The OPQ project proposes an alternative, crowdsourced approach, in which collection is by and analysis is oriented toward the needs of consumers.   

According to Estelles-Aroles, crowdsourcing is ``a type of participative online activity in which an individual, an institution, a non-profit organization, or company proposes to a group of individuals of varying knowledge, heterogeneity, and number, via a flexible open call, the voluntary undertaking of a task. The undertaking of the task [...] always entails mutual benefit. The user will receive the satisfaction of a given type of need, be it economic, social recognition, self-esteem, or the development of individual skills, while the crowdsourcer will obtain and utilize to their advantage that what the user has brought to the venture, whose form will depend on the type of activity undertaken.'' \cite{Estelles-Aroles2012}.

To our knowledge, crowdsourcing has never been successfully used for the purpose of collecting and analyzing power quality data.  That said, Hammack proposed this very idea as ``citizen engineering'' in 2010, suggesting that the deployment of several thousand FNET frequency disturbance devices could be combined with publicly accessable online tools for visualization and analytics to enable consumers to see how their frequency data relates to that of the rest of the nation \cite{Hammack2010}.   We believe his idea did not gain traction due to the cost of the devices and the relatively low benefits to individuals of a nation-wide perspective on frequency changes.  In contrast, the OPQ project is designed to address both of these issues by the design of less expensive devices, and the deployment into a geographically small user community that is actively ``suffering'' from issues related to power quality.  

Privacy is an important issue when crowdsourcing data about individuals or their environment. One successful example of a crowdsourced project that has addressed this issue is the Personal Genome Project (PGP), in which individuals are asked to share their genetic data in order to create a public repository that can advance the science of health care \cite{Church2005}. PGP data is placed in the public domain and contributors are required to sign an ``open consent'' form which states that the researchers cannot guarantee anonymity. In the OPQ project, we address privacy by allowing contributors to ``coarsen'' the published geographical location of their power quality data in order to address privacy concerns.  We will also investigate the relationship of our data to broader smart meter privacy issues \cite{Balough2011}.

\subsection{Analytics}

Power quality data is traditionally used for four purposes: (1) {\em detection} of anomalies, (2) {\em diagnosis} of the cause and/or originating location of the anomaly, (3) real-time {\em control} of grid stability, and (4) {\em prediction} of future anomalies. Whether or not a given data set can be used for any of these purposes depends upon the characteristics of its generating device. A relatively simple device like the AC Scout can be used only for detection, FNET can be used for both detection and diagnosis \cite{Markham2012}, and PMU data can be applied to detection, diagnosis \cite{Zhao2009}, control \cite{Liu2010}, and prediction \cite{Liu2009,Gao2012}). 
The OPQ project is designed to collect power quality data that can be used for detection of anomalies and a limited form of diagnosis: if a user lives in a neighborhood where several OPQ devices are installed, then an anomaly report will indicate if it was limited to the users household or co-occurred elsewhere in the neighborhood. Our devices are not sufficiently synchronized, nor can the data be communicated to grid operators quickly enough to support real-time utility-scale control. That said, when combined with IoT devices, OPQ can potentially enable useful control in limited circumstances, such as voltage collapse scenarios where effective reponse time is measured in seconds, not milliseconds. We will also investigate whether or not OPQ data can support some forms of prediction when combined with other environmental, consumption, and generation data. From this traditional perspective on the use of power quality data, the OPQ project results in a data set that has somewhat more capability than the data collected from devices like the AC Scout, yet less capability than the data collected by devices like FDRs and PMUs. 

However, in addition to the four traditional purposes, the OPQ project is designed to investigate whether power quality data can be used for an entirely new purpose: (5) {\em enabling active participation by consumers}. To do this, we will develop some simple guidelines for interpretation of the data, helping consumers to decide if: their problems are severe enough to contact the utilities; their problems correlate with environmental events such as thunderstorms and can be ameliorated by unplugging; or their problems are frequent, severe, and unpredictable enough to recommend installation of UPS systems to protect sensitive electronics.  This interpretation makes power quality data {\em actionable} even without integration with IoT.

We observe that one cause of the societal and political issues involving the Smart Grid in Hawaii involves a fundamental disengagement by consumers from the grid: they want the utility to support unlimited installation of distributed generation by consumers (and resulting lower utility bills) without any perceivable impact on quality, price, or availability.  We hypothesize that a low-cost approach to providing consumers with increased visibility into grid stability can increase engagement (by enabling consumers to see how power quality varies from time to time and from neighborhood to neighborhood) followed by active participation (through policy and civic engagement) to create a Smart Grid satisfying everyone's needs.  Our project is designed to provide preliminary evidence regarding whether or not this fifth purpose can occur in reality.

In summary, our OPQ project will design and produce an innovative hardware device called OPQBox for power quality monitoring. We will implement an innovative, crowdsourced approach to power quality data collection through a software service called OPQHub. Finally, we will investigate the application of this data to both traditional and entirely new purposes.   