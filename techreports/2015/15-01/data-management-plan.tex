%%%%%%%%%%%%%%%%%%%%%%%%%%%%%% -*- Mode: Latex -*- %%%%%%%%%%%%%%%%%%%%%%%%%%%%
%% supplemental.tex -- 
%% Author          : Philip Johnson
%% Created On      : Tue Mar 31 11:42:10 2009
%% Last Modified By: Philip Johnson
%% Last Modified On: Thu Dec 10 09:09:23 2009
%% RCS: $Id$
%%%%%%%%%%%%%%%%%%%%%%%%%%%%%%%%%%%%%%%%%%%%%%%%%%%%%%%%%%%%%%%%%%%%%%%%%%%%%%%
%%   Copyright (C) 2009 
%%%%%%%%%%%%%%%%%%%%%%%%%%%%%%%%%%%%%%%%%%%%%%%%%%%%%%%%%%%%%%%%%%%%%%%%%%%%%%%
%% 

\section*{Data Management Plan}

\subsubsection*{Types of data}

As part of this project, many types of data will be collected. This includes power quality data including frequency, voltage, and THD, environmental data gathered from public weather sources including temperature, humidity, wind direction and speed, and solar irradiance. It will also include end-use power consumption and generation data.

These data types will be transmitted through wireless and wired networks to a data server and storage system that will be hosted through the University of Hawaii Information and Technology Services, or through a commercial provider such as Amazon Web Services, CloudBees, or Heroku.

We will also be collecting questionnaire response data.  This data will be stored in computers located in the Collaborative Software Development Laboratory.

\subsubsection*{Data and metadata standards}

While there are many systems that have been developed to store environmental sensor data and electrical grid data, such as Geo-CENS, Pachube, and the Berkeley Sensor Database, there are no commonly recognized standards for the formatting, storage, or transmission of the data we will be collecting. Because of this, we plan to store our data using a custom, but publicly documented set of database schemas in an internet-accessible server running a standard open source such as LAMP (Linux, Apache, MySQL, Python) stack. 

\subsubsection*{Policies for access and sharing and provisions for appropriate protection and privacy}

The power quality and environmental data that we will collect, store, analyze, and publicize will not reveal personal characteristics of users. We will restrict real-time access to power and/or energy consumption data, or aggregate this data as required, in order to prevent users from being able to gain behavioral insight from patterns of energy usage.  Questionnaire response data will also be anonymous and not provide identifying characteristics.

\subsubsection*{Policies and provisions for re-use and re-distribution}

We plan to make the data collected in this research freely available under the Open Data Commons Open Database License. This license will let others use, distribute, and analyze the data we collect without restriction as long as they credit us as the original creators of the data.

We believe that the data we collect will be of interest to others developing ``smart microgrid'' systems as it will provide, at a minimum, baseline data for environmental conditions we experienced during the course of our research.

\subsubsection*{Plans for presentation, archiving, and preservation of data}

During the course of this research, we will be storing the data in a web accessible database as discussed above. Archiving the data past the conclusion of the study will be done via data archiving services provided by the University of Hawaii. All research reports and collateral documents created in response to the data will also be permanently archived through the University of Hawaii technical report services.   Research results will also be presented through journal publications, presentations at conferences, and seminars to students, faculty, and industry collaborators at the University of Hawaii.

The University of Hawaii implements standard best practices for data storage backup and retrieval, including off-site storage, redundant power supplies, RAID disk storage, and so forth.  Sites for storage will be at the Collaborative Software Development Laboratory. 

\subsubsection*{Dissemination of research and educational work}

Research and educational results will be disseminated through journals, dissertations, conferences (oral and poster presentations), and website demonstrations.  The Collaborative Software Development Laboratory website will also contain information about courses, and technical reports (which will be turned into journal and conference publications).  We also plan to lso give short energy and smart grid courses to the community.  These courses will be funded through the University of Hawaii, the city of Honolulu, and the State of Hawaii and will provide another method where knowledge and material can be disseminated to the community.  For example, a course on Software Engineering for the Smart Grid was offered in Fall, 2012.

