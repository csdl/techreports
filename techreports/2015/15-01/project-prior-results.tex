%%%%%%%%%%%%%%%%%%%%%%%%%%%%%% -*- Mode: Latex -*- %%%%%%%%%%%%%%%%%%%%%%%%%%%%
%% project-background.tex -- 
%% Author          : Philip Johnson
%% Created On      : Tue Mar 31 11:44:58 2009
%% Last Modified By: Philip Johnson
%% Last Modified On: Wed Dec 16 15:29:18 2009
%% RCS: $Id$
%%%%%%%%%%%%%%%%%%%%%%%%%%%%%%%%%%%%%%%%%%%%%%%%%%%%%%%%%%%%%%%%%%%%%%%%%%%%%%%
%%   Copyright (C) 2009 
%%%%%%%%%%%%%%%%%%%%%%%%%%%%%%%%%%%%%%%%%%%%%%%%%%%%%%%%%%%%%%%%%%%%%%%%%%%%%%%
%% 

\section{Results from prior NSF support}

P. Johnson, {\em Human centered information integration for the Smart Grid}, NSF Grant IIS-1017126, 8/15/10 to 7/31/14, \$413,467. 
{\em Intellectual Merit:} The research provided novel insight into: the inadequacy of baseline data for energy competition research, the design of experimental studies for assessing energy behaviors, the design of energy competitions incorporating educational activities. 
{\em Broader Impacts} include: the creation and distribution of two open source systems, WattDepot and Makahiki, that can be used for collection and analysis of energy data and the design and implementation of sustainability games; the publication of data regarding the impact of energy
education and gamification techniques on energy literacy and behavior; the training of approximately 9 undergraduate students, 3 M.S. students, and 3 Ph.D. students in research techniques, sustainability concepts, and software design and development.
Selected publications:
  \cite{csdl2-10-05,csdl2-10-07,csdl2-10-08,csdl2-11-02,csdl2-11-03,csdl2-12-06,csdl2-11-07, csdl2-12-12,csdl2-13-10,csdl2-13-05,csdl2-13-03}. Research products are available at the Kukui Cup site \cite{kukuicup}, the WattDepot site \cite{wattdepot}, and the Makahiki site \cite{makahiki}.

A. Kuh and M. Fripp, {\em Sensing, Modeling, and Control of Smart Sustainable Microgrids}, NSF Grant ECCS-1310634, 7/1/2013-6/30/2016, \$360,000. 
{\em Intellectual Merit:} This project explores the potential of future power distribution by designing, building and evaluating a microgrid that offsets local energy usage using distributed generation (i.e. rooftop PV).
{\em Broader Impacts} include: By developing an operational structure analogous to the larger utility grid, the campus microgrid is being instrumented to serve as a platform for gathering field data on load, generation, local grid quality and usage patterning. Using this platform, students, faculty and partners can immediately advance their knowledge on the challenges of operating, monitoring and assessing the conditions of a modern grid. Selected publications: \cite{Fatemi2014}.

% \vspace{-.1in}

% \item A. Kuh, {\em Incremental and Distributed Learning in Nonstationary
%     Environments with Applications to Wind Forecasting}, NSF Grant ECCS-098344,
%   9/01/09 - 8/31/13, \$150,251.  The objective of this research is to
%   design novel nonlinear kernel online and distributed learning algorithms
%   for applications including wind forecasting.   Research was also conducted
%   to model the microgrid using a factor-graph framework. 
%   Selected publications for
%   this project include \cite{kuh-etal10,kowahl-kuh,hu-etal10,hu-kuh-yang-kavcic,hu-kuh-kavcic-nakafuji,ji-wei-kuh,uddin12,kuh-isess,carland,navid}.
  
% %\vspace{-.1in}

% \item A. Kuh, {\em US-Japan Joint Seminar Information Theory}, NSF Grant 0508025
%   \$35,750.  Funds used to support graduate students for conferences and
%   for visit to Japan.


%\vspace{-.1in}



