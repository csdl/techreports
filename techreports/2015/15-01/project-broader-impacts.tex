%%%%%%%%%%%%%%%%%%%%%%%%%%%%%% -*- Mode: Latex -*- %%%%%%%%%%%%%%%%%%%%%%%%%%%%
%% project-plan.tex -- 
%% Author          : Philip Johnson
%% Created On      : Tue Mar 31 11:44:58 2009
%% Last Modified By: Philip Johnson
%% Last Modified On: Wed Dec 16 15:29:18 2009
%% RCS: $Id$
%%%%%%%%%%%%%%%%%%%%%%%%%%%%%%%%%%%%%%%%%%%%%%%%%%%%%%%%%%%%%%%%%%%%%%%%%%%%%%%
%%   Copyright (C) 2009 
%%%%%%%%%%%%%%%%%%%%%%%%%%%%%%%%%%%%%%%%%%%%%%%%%%%%%%%%%%%%%%%%%%%%%%%%%%%%%%%
%% 

\section{Broader Impacts}
\label{sec:merit}

% {\em The Project Description must contain, as a separate section within the narrative, a discussion of the broader impacts of the proposed activities. Broader impacts may be accomplished through the research itself, through the activities that are directly related to specific research projects, or through activities that are supported by, but are complementary to the project. NSF values the advancement of scientific knowledge and activities that contribute to the achievement of societally relevant outcomes. Such outcomes include, but are not limited to: full participation of women, persons with disabilities, and underrepresented minorities in science, technology, engineering, and mathematics (STEM); improved STEM education and educator development at any level; increased public scientific literacy and public engagement with science and technology; improved well-being of individuals in society; development of a diverse, globally competitive STEM workforce; increased partnerships between academia, industry, and others; improved national security; increased economic competitiveness of the United States; and enhanced infrastructure for research and education.
% }


% A report to the U.S. Department of Energy in September 2009 on the
% Principle Characteristics of the Smart Grid echoes our research orientation
% when it states: {\em ``Achieving consumer participation means
%   making participation easy and understandable.  And essential to this will
%   be providing a user interface that successfully motivates and supports
%   consumer action. [...] Today's communications and electronic technologies
%   create options that were just not viable in the past.''}
% \cite{NETL:EnablesActiveParticipation}

We believe this project defines an ambitious, aggressive, yet feasible approach to obtaining significant insight into the following important sustainability questions: Can crowdsourced power quality data enable active participation in the Smart Grid?  What are the technical, social, behavioral, and economic requirements for crowdsourced data that make it effective for detection, monitoring, prediction and diagnosis of selected Smart Grid power quality issues? And finally, how can our project outcomes improve ``citizen science'' in general and the kinds of intrinsic and extrinsic motivators needed for successful outcomes?

We will gain new insights into these questions through a number of innovations.  We will develop low cost, open source hardware for residential power quality monitoring of voltage, frequency, and total harmonic distortion. This data will be uploaded to an open source cloud-based internet service we have designed for storage, retrieval, and analysis. The collected data will be open source, and we will address privacy concerns by allowing consumers to ``coarsen'' their locational information when providing the data to others.  We will combine power quality data with other environmental data from publicly available sources in order to investigate relationships that may aid in prediction and diagnosis.  We will use a pre and post-test experimental design in order to gain insight into the effect that power quality data has upon consumers with respect to their attitudes and behaviors toward the Smart Grid and the utility implementing it.  According to LaCommare \cite{LaCommare2004}, there is no publicly available dataset regarding power quality at the household level, and so our project data has the potential to provide unprecedented insight into power quality issues at the household level.

To achieve these innovations, we will build upon our prior experiences with open source software development, empirical software engineering, energy challenge game design, power systems for renewable energy sources, environmental planning, and community ecology. We will build upon established relationships with University of Hawaii organizations (Renewable Energy and Island Sustainabity Group, Sustainable UH) and community organizations (Kanu Hawaii, Blue Planet Foundation, Hawaiian Electric Company).  Hawaii is an EPSCOR state and approximately 84\% of University of Hawaii undergraduates are minorities, so this research will benefit under-represented populations.

Crowdsourcing is a growing technique for citizen science. We believe this project will provide useful new insights into the strengths and weaknesses of this approach. Can intrinsic motivation suffice for this form of data collection? How does this effort compare to other citizen science projects? 

The project will create a interdisciplinary community of researchers including professors, graduate students and undergraduates from computer science, electrical engineering, and urban and regional planning.  Through the development of graduate seminars and open source repositories, we will pursue workforce development related to power quality, crowdsourcing, user interface design, community development, and the Smart Grid.

This research creates a mechanism for rapid implementation and deployment of community-based power quality monitoring. While Hawaii is the ideal location to develop this capability due to its nation-leading penetration of distributed renewables, we expect that other communities will find it useful in future.




