%%%%%%%%%%%%%%%%%%%%%%%%%%%%%% -*- Mode: Latex -*- %%%%%%%%%%%%%%%%%%%%%%%%%%%%
%% project-plan.tex -- 
%% Author          : Philip Johnson
%% Created On      : Tue Mar 31 11:44:58 2009
%% Last Modified By: Philip Johnson
%% Last Modified On: Wed Dec 16 15:29:18 2009
%% RCS: $Id$
%%%%%%%%%%%%%%%%%%%%%%%%%%%%%%%%%%%%%%%%%%%%%%%%%%%%%%%%%%%%%%%%%%%%%%%%%%%%%%%
%%   Copyright (C) 2009 
%%%%%%%%%%%%%%%%%%%%%%%%%%%%%%%%%%%%%%%%%%%%%%%%%%%%%%%%%%%%%%%%%%%%%%%%%%%%%%%
%% 

\section{Broader Impacts}
\label{sec:merit}

% {\em The Project Description must contain, as a separate section within the narrative, a discussion of the broader impacts of the proposed activities. Broader impacts may be accomplished through the research itself, through the activities that are directly related to specific research projects, or through activities that are supported by, but are complementary to the project. NSF values the advancement of scientific knowledge and activities that contribute to the achievement of societally relevant outcomes. Such outcomes include, but are not limited to: full participation of women, persons with disabilities, and underrepresented minorities in science, technology, engineering, and mathematics (STEM); improved STEM education and educator development at any level; increased public scientific literacy and public engagement with science and technology; improved well-being of individuals in society; development of a diverse, globally competitive STEM workforce; increased partnerships between academia, industry, and others; improved national security; increased economic competitiveness of the United States; and enhanced infrastructure for research and education.
% }


% A report to the U.S. Department of Energy in September 2009 on the
% Principle Characteristics of the Smart Grid echoes our research orientation
% when it states: {\em ``Achieving consumer participation means
%   making participation easy and understandable.  And essential to this will
%   be providing a user interface that successfully motivates and supports
%   consumer action. [...] Today's communications and electronic technologies
%   create options that were just not viable in the past.''}
% \cite{NETL:EnablesActiveParticipation}

We believe this project defines an ambitious, aggressive, yet feasible approach to obtaining significant insight into the following important sustainability questions: Can crowdsourced power quality data enable active participation in the Smart Grid?  What are the technical, social, behavioral, and economic requirements for crowdsourced data that make it effective for detection, monitoring, prediction, control, and diagnosis of selected Smart Grid power quality issues? 

We will gain new insights into these questions through a number of innovations.  We will develop low cost, open source hardware and software for residential power quality monitoring of voltage, frequency, and total harmonic distortion. The collected data will be open source, and we will address privacy concerns by allowing consumers to ``coarsen'' their locational information when providing the data to others.  We will combine power quality data with other environmental data to support prediction and diagnosis, and investigate IoT for control.  We will use a pre and post-test experimental design in order to gain insight into the effect that power quality data has upon consumers with respect to their attitudes and behaviors toward the Smart Grid and the utility implementing it.  According to LaCommare \cite{LaCommare2004}, there is no publicly available dataset regarding power quality at the household level.

The project will create a interdisciplinary community of researchers including professors, graduate students and undergraduates from computer science, electrical engineering, and urban and regional planning.  Through the development of graduate seminars and open source repositories, we will pursue workforce development related to power quality, crowdsourcing, user interface design, community development, and the Smart Grid. Hawaii is an EPSCOR state and approximately 84\% of University of Hawaii undergraduates are minorities, so this research will benefit under-represented populations.

While Hawaii is the ideal location to develop this capability due to its nation-leading penetration of distributed renewables and current consumer dissatisfaction, our results will have broader applicability: according to a report by the North Carolina Clean Energy Technology Center, rooftop PV is now cheaper than utility-supplied power in 42 of America's 50 largest cities \cite{Kennerly2015}. The power quality problems Hawaii faces now may soon be faced by many other communities across the nation.




