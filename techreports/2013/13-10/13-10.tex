\documentclass{sigchi}

% Use this command to override the default ACM copyright statement (e.g. for preprints). 
% Consult the conference website for the camera-ready copyright statement.

\toappear{Copyright is held by the author.}

%% EXAMPLE BEGIN -- HOW TO OVERRIDE THE DEFAULT COPYRIGHT STRIP -- (July 22, 2013 - Paul Baumann)
% \toappear{Permission to make digital or hard copies of all or part of this work for personal or classroom use is 	granted without fee provided that copies are not made or distributed for profit or commercial advantage and that copies bear this notice and the full citation on the first page. Copyrights for components of this work owned by others than ACM must be honored. Abstracting with credit is permitted. To copy otherwise, or republish, to post on servers or to redistribute to lists, requires prior specific permission and/or a fee. Request permissions from permissions@acm.org. \\
% {\emph{CHI'14}}, April 26--May 1, 2014, Toronto, Canada. \\
% Copyright \copyright~2014 ACM ISBN/14/04...\$15.00. \\
% DOI string from ACM form confirmation}
%% EXAMPLE END -- HOW TO OVERRIDE THE DEFAULT COPYRIGHT STRIP -- (July 22, 2013 - Paul Baumann)


% Arabic page numbers for submission. 
% Remove this line to eliminate page numbers for the camera ready copy
\pagenumbering{arabic}


% Load basic packages
\usepackage{balance}  % to better equalize the last page
\usepackage{graphics} % for EPS, load graphicx instead
\usepackage{times}    % comment if you want LaTeX's default font
\usepackage{url}      % llt: nicely formatted URLs

% llt: Define a global style for URLs, rather that the default one
\makeatletter
\def\url@leostyle{%
  \@ifundefined{selectfont}{\def\UrlFont{\sf}}{\def\UrlFont{\small\bf\ttfamily}}}
\makeatother
\urlstyle{leo}


% To make various LaTeX processors do the right thing with page size.
\def\pprw{8.5in}
\def\pprh{11in}
\special{papersize=\pprw,\pprh}
\setlength{\paperwidth}{\pprw}
\setlength{\paperheight}{\pprh}
\setlength{\pdfpagewidth}{\pprw}
\setlength{\pdfpageheight}{\pprh}

% Make sure hyperref comes last of your loaded packages, 
% to give it a fighting chance of not being over-written, 
% since its job is to redefine many LaTeX commands.
\usepackage[pdftex]{hyperref}
\hypersetup{
pdftitle={Three Shifts for Sustainable HCI: Scalable, Sticky, and Multidisciplinary},
pdfauthor={LaTeX},
pdfkeywords={Sustainability, sustainable HCI},
bookmarksnumbered,
pdfstartview={FitH},
colorlinks,
citecolor=black,
filecolor=black,
linkcolor=black,
urlcolor=black,
breaklinks=true,
}

% create a shortcut to typeset table headings
\newcommand\tabhead[1]{\small\textbf{#1}}

%% Puts space after macros, unless followed by punctuation
\usepackage{xspace}

%%% Personal macros
%% Tired of typing CO2 so many times, requires xspace package
\newcommand{\COtwo}{CO\ensuremath{_2}\xspace}
%% Hawai`i with okina
\newcommand{\Hawaii}{Hawai`i\xspace}
%% Hawai`ian with okina
\newcommand{\Hawaiian}{Hawai`ian\xspace}
%% Manoa with kahako
\newcommand{\Manoa}{M\=anoa\xspace}
%% Formatting W, Wh, kW, kWh properly as units
\newcommand{\W}{\,W\xspace}
\newcommand{\Wh}{\,Wh\xspace}
\newcommand{\kW}{\,kW\xspace}
\newcommand{\kWh}{\,kWh\xspace}

% Customize enumeration environments
\usepackage{enumitem}
% reduce space before enumerations: http://stackoverflow.com/a/1073140/140430
\setlist{nolistsep}

% End of preamble. Here it comes the document.
\begin{document}

\title{Three Shifts for Sustainable HCI: \\ Scalable, Sticky, and Multidisciplinary}

\numberofauthors{1}
\author{
  \alignauthor Robert S. Brewer\\
    \affaddr{Dept. of Computer Science}\\
    \affaddr{Aarhus University}\\
    \affaddr{8200 Aarhus N, Denmark}\\
    \email{rbrewer@cs.au.dk}\\
%  \alignauthor Johanne Mose Entwistle\\
%    \affaddr{Alexandra Institute}\\
%    \affaddr{8200 Aarhus N, Denmark}\\
%    \email{johanne.mose@alexandra.dk}\\
}

\maketitle

% ~150 word abstract
\begin{abstract}
While there has been a steady increase in sustainable HCI research, there remains a lack of consensus on how to ensure this research moves us towards achieving sustainability. This paper suggests three ways the sustainable HCI community might shift to better address the challenge of achieving global sustainability. First, we should shift from creating only small-scale solutions to systems and solutions that are scalable to many users and environments because the problem of sustainability is vast in scale. Second, we should shift from short-term solutions to `sticky' solutions that will continue to have an impact over decades, because sustainability is a problem that will span generations. Third, the sustainable HCI community must shift from an insular focus on our community to a broad engagement and collaboration with other research communities involved in sustainability research.
\end{abstract}

\keywords{
	Sustainability; sustainable HCI
}

\category{K.4.0.}{Computers and society}{General}

\section{Introduction}

The area of sustainable HCI has exploded, leading to many researchers working in this area. However, as described in the workshop extended abstract~\cite{2014Silberman-CHIworkshop}, it is unclear how much this research has been brought into practice. Drawing on experiences from the Kukui Cup project \cite{csdl2-13-05,csdl2-12-08,csdl2-10-08} and the EcoSense project (\url{http://ecosense.au.dk}), I suggest three shifts in sustainable HCI research:
\begin{itemize}
	\item from small-scale prototypes and conclusions to solutions with built-in scalability,
	\item from solutions that are unlikely to be of long-term utility to solutions that are sticky for long-term use, and
	\item from an insular focus on our own HCI community to collaborations with the broad range of researchers and practitioners in sustainability.
\end{itemize}

This paper addresses the following questions from the workshop CFP:
\begin{enumerate}
  % question 2 means counter set to 1
  \setcounter{enumi}{1}
  \item What do we know, from within and beyond HCI, about how sustainability might be achieved?
  % question 4 means counter set to 3
  \setcounter{enumi}{3}
  \item How can HCI research help achieve sustainability?
  % question 7 means counter set to 6
  \setcounter{enumi}{6}
  \item How can we make better use of sustainability knowledge from outside HCI?
  % question 8 means counter set to 7
  \setcounter{enumi}{7}
  \item How can we encourage work that contributes substantively to practical efforts to achieve sustainability?
\end{enumerate}


\section{Scalable}

%With respect to scalability, I agree with you but think you could be slightly clearer that the problem is not small scale studies per se.  Indeed, now that I think about it, you seem to be making the claim that even creating "replicable" small-scale studies is not sufficient: "replicability is different from scalability". I would find a way to make that point in the paper because it's interesting and provocative. 

The problem of sustainability is multifaceted and immense in scale. For sustainable HCI research to make any substantial practical impacts, the results of the research will not only have to be replicable but also scalable to meet the size of the challenge. One aspect of scalability is the scale of the evaluation. Research that is evaluated through a two week study with 20 people that is never followed up with additional work is unlikely to have substantial practical impact. To demonstrate that our research is having an impact beyond novelty, we need larger studies conducted for longer periods of time. One way to achieve this result could be partnering with colleagues in other fields with experience conducting larger and longer studies (also suggested by Froehlich et al.~\cite{Froehlich2010}). Furthermore, adapting a systematic framework from the field of evaluation that applies both qualitative and quantitative methods will ensure that we don't make superficial and flawed impact evaluations of technology based on inadequate data~\cite{Blunck2013CEE}.

Another aspect of scalability is the ability of the research outputs to reach a large scale. When one of the outputs of HCI research is software, it should be developed to the point where it can be used by a broad audience. For example, publishing software on GitHub (\url{https://github.com/}) with an open source license can help to broaden adoption, and allow others to extend or adapt the software to their needs.

If the software provides a service component, which is increasingly common, the service should make use of modern service deployment methods to make it easy for other developers to deploy the service, and to allow the service to scale easily. With the growth of cloud computing, it is now quite feasible for a single developer to economically deploy a system which can scale up computing resources dynamically. For example, Heroku (\url{https://www.heroku.com/}) provides a service that developers can easily use to deploy their systems in a scalable fashion. Designing open source software for cloud deployment also allows other groups to deploy their own versions of services. These shifts would allow research systems to scale to hundreds of thousands or millions of users. In some cases, it might make sense to commercialize software and services, if that is the best way to make them available to a broad audience in a sustainable manner.
 
For outputs that are theoretical in nature or implications for design, the results must be broadly disseminated outside of the HCI community in order to have an impact. The broader audience may not only include other research communities, but also the general public in order to popularize those ideas.


\section{Sticky}

As described in the workshop extended abstract, sustainability is a complex, ``wicked'' problem. There will be no single solution, because sustainability is not a single problem in the same way cancer is not a single disease. However, one thing that is clear: achieving the goal of global sustainability will be a long process that takes place over decades of effort. Therefore, any technologies that attempt to help achieve sustainability need to be \emph{sticky}: they must achieve sustained results over long periods of time.

One way to achieve stickiness in technology is through gamification~\cite{Deterding-2011b} or serious games~\cite{csdl2-13-05}. The incorporation of game play holds the potential for long-term engagement, as shown by the passion and time-commitment of online game players~\cite{mcgonigal2011reality}. However, game play in sustainability applications can also lead to an excessive focus on individual action and on easily-measured metrics which might not accurately reflect progress towards the desired goal. For example, as a result of our Kukui Cup research on energy competitions conducted in college residence halls, we found that the metric of kilowatt-hours saved was often misleading due to difficulty in computing the energy use baseline~\cite{csdl2-12-08}. 

The growing focus on practice-orientated approaches instead of resource consumption~\cite{Pierce2013Practice,Strengers2011} also provides potential avenues for stickiness. In the practice orientation, the consumption of resources is an unintended side-effect of the activities that people engage in every day. People do not think of themselves as consumers of kilowatt-hours any more than people eating a meal view themselves primarily as consumers of calories. By a thorough understanding of these everyday practices, there is the potential to introduce new practices that will be readily adopted, but are also more sustainable as a side-effect.

For example, for many people in the United States, the practice of getting to and from work or school is accomplished using a personal automobile, which provides ample convenience. An alternative practice would be to use public transportation, which can be less expensive, but is often perceived as less convenient due to the need to consult route maps and schedules to determine how to arrive at the desired destination on time. Providing easier-to-use interfaces to public transit schedule information could increase the convenience of public transit, thereby making it a more desirable practice. The switch from driving a car versus taking public transit would be adopted, with no appeal to being `green'. The fact that using public transit is more sustainable than using a personal automobile is a consequence, rather than the primary focus of the new practice.

From this perspective, research outside the typical sustainable HCI areas can actually contribute to the goal of sustainability. In the field of software development, security and privacy are no longer be the domain of a small group of security experts: writing secure software is now every developer's responsibility. Perhaps sustainability now needs to be an attribute that all HCI researchers think about when designing systems.

Actions by individuals alone will be insufficient in achieving sustainability. Sustainability will require changes in policies, infrastructure, and social norms. Achieving these changes will require further community and political action; therefore, HCI research areas like digital democracy and online movement organization can also be seen as potential paths to sustainability.


\section{Multidisciplinary}

Many sustainable HCI researchers are not aware of work in sustainability being done outside the CHI community. Froehlich et al. found that many eco-feedback systems developed by the CHI community did not cite or seem aware of the decades of prior research in the area from other fields like psychology~\cite{Froehlich2010}. HCI researchers may also fail to address new areas such as emerging energy systems that are well-known by many working in the energy field, as pointed out by Pierce and Paulos~\cite{Pierce2012-BEM}. This lack of connection to other streams of research outside our discipline can lead to false assumptions and repeated work, or even irrelevance.

The sustainable HCI community must actively engage with the wider sustainability community. We must encourage external experts to participate in the HCI community through joint authorship of publications and inviting them to participate in HCI conferences. We must also participate in conferences outside the HCI sphere not only to reach, but also to learn from the broader sustainability community. For example, the Behavior, Energy, and Climate Change (BECC) conference (\url{http://beccconference.org/}) is filled with sustainability practitioners and researchers from government, industry, non-profits, and academia. Though BECC in particular is somewhat known in the HCI community, we must continue to seek out ways to connect outside HCI, since they won't be looking for us.


\section{Acknowledgments}

This work has been supported by The Danish Council for Strategic Research as part of the EcoSense project (11-115331) and has been partly funded by the Danish Energy Agency project: Virtual Power Plant for Smartgrid Ready Buildings and Customers (no. 12019).

I would like to thank the members of the Ubiquitous Computing and Interaction group, the EcoSense team, and the members of the Collaborative Software Development Laboratory for the many conversations that have helped shape the views expressed in this paper. Comments from Yuka Nagashima and Johanne Mose Entwistle helped to clarify the themes of this paper.


%\begin{figure}[!h]
%\centering
%\includegraphics[width=0.9\columnwidth]{Figure1}
%\caption{With Caption Below, be sure to have a good resolution image
%  (see item D within the preparation instructions).}
%\label{fig:figure1}
%\end{figure}

%\begin{table}
%  \centering
%  \begin{tabular}{|c|c|c|}
%    \hline
%    \tabhead{Objects} &
%    \multicolumn{1}{|p{0.3\columnwidth}|}{\centering\tabhead{Caption --- pre-2002}} &
%    \multicolumn{1}{|p{0.4\columnwidth}|}{\centering\tabhead{Caption --- 2003 and afterwards}} \\
%    \hline
%    Tables & Above & Below \\
%    \hline
%    Figures & Below & Below \\
%    \hline
%  \end{tabular}
%  \caption{Table captions should be placed below the table.}
%  \label{tab:table1}
%\end{table}



% Balancing columns in a ref list is a bit of a pain because you
% either use a hack like flushend or balance, or manually insert
% a column break.  http://www.tex.ac.uk/cgi-bin/texfaq2html?label=balance
% multicols doesn't work because we're already in two-column mode,
% and flushend isn't awesome, so I choose balance.  See this
% for more info: http://cs.brown.edu/system/software/latex/doc/balance.pdf
%
% Note that in a perfect world balance wants to be in the first
% column of the last page.
%
% If balance doesn't work for you, you can remove that and
% hard-code a column break into the bbl file right before you
% submit:
%
% http://stackoverflow.com/questions/2149854/how-to-manually-equalize-columns-
% in-an-ieee-paper-if-using-bibtex
%
% Or, just remove \balance and give up on balancing the last page.
%
\balance

\bibliographystyle{acm-sigchi}
\bibliography{sustainability,csdl-trs,gamification}
\end{document}
