\chapter{Conclusion}
\label{cha:conclusion}

This dissertation investigated the design, implementation, and evaluation of the serious game framework for sustainability called Makahiki and a stakeholder experience based assessment method for serious game framework. This chapter summarizes the results of the research, the contributions of the research, and possible future directions.

\section{Research Summary}
This research investigates the information technology infrastructure that can support effective and efficient development of serious games for sustainability. The research includes the development of an innovative serious game framework for sustainability that combining education and behavior change, and an assessment method accessing the effectiveness and efficiency of the IT infrastructure for serious games for sustainability with regarding the most important stakeholder's perspective.

%% TODO. more on SGSEAM

%% TODO. more on result of MAKAHIKI applications on several kukui cup

%% TODO. more on SGSEAM on makahiki result

\section{Contributions}

The contributions of this research are:

%% TODO: expand on contribution

\begin{itemize}
	\item Makahiki: open source information technology for development of serious games for sustainability.
	\item SGSEAM: an assessment method for assessing serious game framework.
	\item Evidence regarding the effectiveness and efficiency of Makahiki as a framework for development of serious games for sustainability.
	\item Evidence regarding the effectiveness and efficiency of a second system (BuildingOS) as a framework for development of serious games for sustainability.
	\item Insights into the strengths and weaknesses of the assessment method.
\end{itemize}

\section{Future Directions}

There are a variety of directions that can be pursued once this research is complete. One of them is the evaluation of the SGSEAM itself. The design of SGSEAM creates a research question of what are the strengths and weaknesses of this assessment method. 
To better answer this question, SGSEAM should be applied to another serious game development environment. BuildingOS\cite{building-dashboard} by Lucid Design Group is such a serious game framework that is suitable for SGSEAM evaluation. Our research lab had made the effort to contact Lucid Design group for the collaboration. I created the assessment plan for them which hope to minimize the effort spent from their side. But due to the workload of the Lucid design group, which is still a newly found startup company, the collaboration did not continue. A further evaluation of SGSEAM by applying to another serious game framework is still an ongoing research direction.

Other future direction of this research includes:
\begin{itemize}
	\item Evaluate the other stakeholders’ experiences

    \item Build a community to expand content and game library for Makahiki

    \item Scale and expand Makahiki to support other geographical and cultural different locations.

\end{itemize}

This chapter describes some enhancement projects for Makahiki that we believe would be interesting and useful for the framework.

3.11.1. Real-time player awareness
It is not possible in Makahiki to know who is currently “on line” and playing the game. Creating this awareness opens up new social gaming opportunities (performing tasks together), new opportunities for communication (chat windows), and potentially entirely new games (play “against” another online player).

The goal of this enhancement is to extend the framework with a general purpose API that provides the identities of those who are online, and then the development of one or more user interface enhancements to exploit this capability.

3.11.2. Deep Facebook integration
Makahiki currently supports a “shallow” form of Facebook integration: you can request that your Facebook photo be used as your Makahiki profile picture, and you are given an oppportunity to post to Facebook when the system notifies you of an accomplishment.

For this task, expand the current Facebook integration. One way is to deepen the connection between user Facebook pages and their game play. This might involve more automated forms of notification (i.e. the same way Spotify playlists are posted to your Facebook wall), or ways in which your activities on Facebook could impact on your Makahiki challenge status (for example, posting a sustainability video to Facebook, or liking a Sustainability organization could earn you points.)

A different type of enhancement is to allow challenge designers to specify a Facebook page as the official Challenge Facebook information portal, and have the system automatically post information to that Facebook page as the challenge progresses.

3.11.3. Action Library Management System
Makahiki currently ships with over 100 possible “actions” already developed for the Smart Grid Game. However, the current implementation suffers from a number of problems:

There is no convenient way to display and peruse the current set of actions. This has led to a duplicate representation of the smart grid game, implemented using a Google Docs spreadsheet linked to Google Sites pages. This approach has a lot of problems: it duplicates content, it does not provide a way to edit or manage content, it is already out of date.
The content is intimately tied to the Smart Grid Game implementation. The SGG is just one of many ways that the sustainability content could be presented to players. By separating “content” from the “presentation”, more games can be developed using this content.
For this task, you will enhance Makahiki to provide a “content management system” for “actions”. This involves the following changes to the current system:

A new set of database tables must be defined to hold Library actions.
Library actions can be “instantiated” (i.e. copied) into the Smart Grid Game. At that point they are assigned a category and a row within the Smart Grid Game.
An editor is provided to create action content and preview it in a formatted manner.
A new set of pages can be (optionally) made available to allow others to peruse Library content.
Library content can be exported and imported into systems in order to support sharing. A public repository can be provided on GitHub. The format is likely JSON.