\chapter{SGSEAM Improvement Action Report for Makahiki}
\label{app:makahiki-improvement-report}

This appendix describes the improvement action report after applying the SGSEAM assessment to Makahiki. It is the deliverable for the final step of the SGSEAM process. It first describes findings from the data analysis of the SGSEAM assessment for each stakeholder, followed by a list of the strengths and weaknesses of the Makahiki framework from the perspectives of different stakeholders, and suggestions of any improvement actions to the framework.

\section{Makahiki Player Assessment Report}

\subsection{Player Assessment Results}

Following the SGSEAM assessment plan for Makahiki (\autoref{app:makahiki-assessment-plan}), four assessment approaches were carried out to assess the player's experiences with the Makahiki framework. The real world Makahiki instances of the UHM Kukui Cup challenge were used for these assessments. 

The Pre-post effectiveness study shows that the players of the Makahiki framework effectively achieved the energy literacy improvement. Since the average energy reduction found in the study was low, the study did not provide strong evidence that the challenge results in effective reduction of energy consumption by players. 

The self-reported effectiveness survey study shows that the players perceived their experiences with Makahiki were positive and there were impacts to their sustainability awareness and behaviors because of playing the game. 

The self-reported usability survey study reveals that most of the usability in Makahiki is good. It discovered a few usability issues in the 2014 KC instance such as the slow performance of certain pages and the confusion of scoreboard display. There were also issues with the SmartGrid game content which had several broken links causing players not able to complete the actions. 

The engagement metrics study indicates that Makahiki is successful in achieving player engagement for a 4 week period in this kind of serious game. After this period, engagement tampered off. 

\subsection{Strengths and Weaknesses from Player's Perspective}

From the above player assessment results, we concluded that the strengths of the Makahiki framework from player's experience perspective as listed below:   

\begin{compactenum}
    \item The framework can be successfully used to create a game that effectively improve player's energy literacy.
    \item The framework can be effective to increase player's awareness and behavior change in sustainability.
    \item The framework can create a series of games with good usability.
    \item The framework can have a high level of player engagement for a duration of 4 weeks.
\end{compactenum}               

We found the weaknesses of the Makahiki framework from the player's experience perspective are:
\begin{compactenum}
     \item The games did not show effective reduction of energy consumption.
     \item Player engagement is low for a duration that is longer than 4 weeks. 
     \item The rotating carousel display of the scoreboard widget is confusing to some players.
     \item There seems to be a performance downgrade in the latest software release.
     \item There is no tool to validate the game content such as the broken links
\end{compactenum}  

\subsection{Improvement Actions from Player's Perspective}

Based on the findings of the weaknesses of the framework, we recommend several improvement actions to the Makahiki framework from the player stakeholder's perspective, as listed in \autoref{table:player-report}.

\begin{table}[ht!]
\begin{shadebox}

{\bf Improvement Actions:}
 \begin{compactenum}
	\item Investigate other methods to measure the effectiveness of resource consumption. 
	\item Research the benefits and approaches to maintain high level of engagements for long competition duration.
	\item Investigate the performance downgrade in the latest release.
	\item Investigate alternative to improve the rotating scoreboard widget display.
	\item Fix the broken links to the educational video.
	\item Provide a content validation tool to minimize content related issues.
\end{compactenum}
\end{shadebox}
\caption{SGSEAM Improvement Action Report from Player's Perspective}

\label{table:player-report}
\end{table}

\section{Makahiki System Admin Assessment Report}

\subsection{System Admin Assessment Results}

Two SGSEAM assessment approaches were carried out to assess the system admin's experiences with the Makahiki framework. 

The in-lab installation experiment study identified that the ``Install and configure database'' step of the local installation took the longest average time. It also has the most problems reported from the participants. In the case of cloud installation, the study identified that the ``Setup cloud environment'' step took the longest average time. The most participant-reported problems were confusions in setting up Amazon S3. It also identified a few issues with the installation documents and the diagnosability with the install script when errors occurred.

The real-world system admin interview study revealed difficulty in integrating Makahiki with an organization's LDAP and email server as well as using SSL in a local installation scenario. The real-world system admin reported no issues in maintaining the Makahiki system with backup and monitoring once the installation was completed.

The study show that the installation experiences varied greatly between the participants in the study. For the participants that may have better system admin skill, it only took 0.9 hour to install the system locally and 0.7 hour to install in the cloud. This indicated Makahiki is relatively easy to install in both scenarios for experienced system admin.

\subsection{Strengths and Weaknesses from System Admin's Perspective}

From the above results of the system admin assessments, we can conclude that the strengths of the Makahiki framework from system admin's experience perspective are:
    \begin{compactenum}
    \item The framework is easy to install for experienced system admin.
    \item Cloud installation is easier than local installation for experienced system admin.
    \item Easy to maintain the system once the installation is completed.
    \end{compactenum}               
    
We found the weaknesses of the Makahiki framework from the system admin's experience perspective are:
    \begin{compactenum}
    \item Database installation documentation is not easy to follow for inexperience database admins.
    \item It is not easy to diagnose problems from the output of the installation script when problems occur.
    \item It is difficult to integrate with an organization's LDAP and email server
    \item It is difficult to use SSL with the Makahiki server
    \end{compactenum} 

\subsection{Improvement Actions from System Admin's Perspective}

Based on the findings of the weaknesses of the framework, we recommend several improvement actions to the Makahiki framework from the system admin stakeholder's perspective, as listed in \autoref{table:sysadmin-report}.

\begin{table}[ht!]
\begin{shadebox}
{\bf Improvement Actions:}
\begin{compactenum}
\item Improve the database configuration document.
\item Improve automation of DB installation, configuration, cloud environment setup process as much as possible.
\item Improve the usability of the installation script, with clear error message, redirecting the verbose output into installation log.
\item Develop a tool to verify and diagnose the dependency and environment setup.
\item Improve the documentation on LDAP and email server integration.
\item Improve the documentation on configuring SSL.
\end{compactenum}
\end{shadebox}
\caption{SGSEAM Improvement Action Report from System Admin's Perspective}
\label{table:sysadmin-report}
\end{table}

\section{Makahiki Game Designer Assessment Report}

\subsection{Game Designer Assessment Results}

Two SGSEAM assessment approaches were carried out to assess the game designer's experiences with the Makahiki framework. 

The in-lab game design study revealed the two most time consuming tasks in Makahiki design were ``Smart Grid Game Design'' and ``Configure Challenge Settings''. The complexity of the predicate system and lack of documentation caused difficulty in defining dependencies between game activities when designing the SmartGrid game. There was also difficulty encountered when generating attendance codes for the event activities. The study discovered a few bugs in the game design interface when configuring the challenge settings.

The real world game designer interview study also revealed the confusions in the game design interface for generating the event attendance code. It was reported that the user interface is not intuitive enough for easy game content creation, and it is time consuming to design the SmartGrid game.

\subsection{Strengths and Weaknesses from Game Designer's Perspective}
    
From the above results of the game designer assessments, we can conclude that the strengths of the Makahiki framework from game designer's experience perspective are:
    \begin{compactenum}
    \item Most of the game design interface are easy to use.
    \item It is easy to create a game with small modification to the default content.
    \end{compactenum} 

We found that the weaknesses of the Makahiki framework from the game designer's experience perspective are:
    \begin{compactenum}
    \item The predicate system is difficulty and lack of documentation. 
    \item It is difficulty to generate event attendance code.
    \item Content creation is not WYSIWYG.
    \item Designing the smart grid game is time consuming.
    \end{compactenum} 

\subsection{Improvement Actions from Game Designer's Perspective}

Based on the above findings, we recommend several improvement actions to the Makahiki framework from the game designer stakeholder's perspective, as listed in \autoref{table:designer-report}.

\begin{table}[ht!]
\begin{shadebox}
{\bf Improvement Actions:}
\begin{compactenum}
\item Improve usability in event confirmation code generation admin interface
\item Provide an WYSIWYG content authoring and design tool
\item Improve the documentation on how to use the predicates
\item Investigate a new design of predicate system to improve usability
\item Fix the reported bugs in the challenge setting configuration interface  
\end{compactenum}
\end{shadebox}
\caption{SGSEAM Improvement Action Report from Game Designer's Perspective}
\label{table:designer-report}
\end{table}

\section{Makahiki Game Manager Assessment Report}

\subsection{Game Manager Assessment Results}

The post-hoc game manager interview approach was carried out for the game manager assessment. The real-world game manager interview study uncovered a few problems with Makahiki game management. The reported problems included the confusions in finding the event confirmation codes from the game managing interface, and not able to automatically send out game status emails.  The interview study show that the submission approval interface is easy to use, and the game analytics page is very helpful when managing the game.

\subsection{Strengths and Weaknesses from Game Manager's Perspective}
   
From the above results of the system admin assessments, we can conclude that the strengths of the Makahiki framework from game manager's experience perspective are: 

    \begin{compactenum}
    \item The submission approval interface is easy to use.
    \item The batch approval feature is useful.
    \item The game analytics ``Status'' page is very useful.
    \end{compactenum} 
    
The weaknesses of the Makahiki framework from the game manager's experience perspective are:

    \begin{compactenum}
    \item It is not easy to find the event confirmation code from the admin interface
    \item The game site is not available after the competition is over
    \item The framework do not support automatically sending out game status emails 
    \end{compactenum} 

\subsection{Improvement Actions from Game Manager's Perspective}

Based on the above findings, we recommend several improvement actions to the Makahiki framework from the game manager stakeholder's perspective, as listed in \autoref{table:manager-report}.
     
\begin{table}[ht!]
\begin{shadebox}
{\bf Improvement Actions:}
	\begin{compactenum}
	\item Improve usability of the interface for looking up event confirmation code
	\item Support access to the game site after the game over
	\item Support automatic emailing of game status periodically
	\end{compactenum}
\end{shadebox}
\caption{SGSEAM Improvement Action Report from Game Manager's Perspective}
\label{table:manager-report}
\end{table}

\section{Makahiki Developer Assessment Report}

\subsection{Developer Assessment Results}

The in-lab game development approach was carried to assess the developer's experience with Makahiki. The study revealed many problems with the development support in Makahiki. The most reported problem was the lack of documentation regarding the development with Makahiki. The simple ``HelloWorld''-type development was well documented, but beyond that, the documentation was lacking. There was a steep learning curve to develop with the framework.

\subsection{Strengths and Weaknesses from Developer's Perspective}
    
The strengths of the Makahiki framework from developer's experience perspective are listed here:
    \begin{enumerate}
    \item The framework can be used to develop other serious games with less effort
    \item It is easy to develop a simple enhancement.
    \end{enumerate} 
The weaknesses of the Makahiki framework from the game designer's experience perspective are:
    \begin{enumerate}
    \item It is difficult to develop complex enhancements with the current documentation.
    \item Some documents are confusing and inconsistent.
    \item The learning curve for development is steep. 
    \end{enumerate} 

\subsection{Improvement Actions from Developer's Perspective}

Based on the above findings, we recommend several improvement actions to the Makahiki framework from the developer stakeholder's perspective, as listed in \autoref{table:developer-report}.

\begin{table}[t!]
\begin{shadebox}
{\bf Improvement Actions:}
\begin{compactenum}
\item Create documentation about widget development overview, data models, class diagrams.
\item Improve documentation on existing structure, modules, APIs.
\item Include more examples or tutorials in developing with the framework
\item Investigate the plugin development mechanism
\end{compactenum}
\end{shadebox}
\caption{SGSEAM Improvement Action Report from Developer's Perspective}
\label{table:developer-report}
\end{table}
