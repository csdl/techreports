\documentclass[11pt]{article}

%%% Load some useful packages:
%% "New" LaTeX2e graphics support.
\usepackage{graphicx}
%%	using final option to force graphics to be included even in draft mode
%\usepackage[final]{graphicx}
\usepackage{paralist} % compact lists

%% Support sub-figures.
\usepackage{subfigure}

%% Make subsubsections numbered and included in ToC
\setcounter{secnumdepth}{3}
\setcounter{tocdepth}{3}

%% Package to linebreak URLs in a sane manner.
\usepackage{url}

%% Define a new 'smallurl' style for the package that will use a smaller font.
\makeatletter
\def\url@smallurlstyle{%
  \@ifundefined{selectfont}{\def\UrlFont{\sf}}{\def\UrlFont{\small\ttfamily}}}
\makeatother
%% Now actually use the newly defined style.
\urlstyle{smallurl}

%% Make margins less ridiculous
\usepackage{fullpage}

%% Allows insertion of fixme notes for future work
\usepackage[footnote, nomargin]{fixme}

%%%% Turned off for tech report, should be turned on for research portfolio
%% Turn on double spacing
%\usepackage{setspace}
%\doublespacing

%% Make URLs clickable
\usepackage[colorlinks, bookmarks=true]{hyperref}
\usepackage[all]{hypcap}


%% Since I'm using the LaTeX Makefile that uses dvips, I need this
%% package to make URLs break nicely
\usepackage{breakurl}

\usepackage{array}

%% Make table cross pages.
\usepackage{longtable}

\begin{document}

\title{SGSEAM Assessment Plan for Assessing Lucid BuildingOS and BuildingDashboard}

\author{
	 Yongwen Xu \\
\em  Collaborative Software Development Laboratory \\
\em  Department of Information and Computer Sciences \\
\em  University of Hawai'i at Manoa\\
     yxu@hawaii.edu \\
}

\date{\today}
\maketitle

\tableofcontents

\graphicspath{{figures/}} 
\DeclareGraphicsExtensions{.eps}

\section{Introduction}
This document describes the proposed approach to assess the Lucid BuildingOS and BuildingDashboard using the Serious Game Stakeholder Experience Assessment Method (SGSEAM). Lucid BuildingOS and BuildingDashboard provides the software framework to create energy competitions that engage the building occupants to become active participants in energy management \cite{building-dashboard}. The goal of SGSEAM assessment is to identify the major strengths and shortcomings of the framework from the perspectives of user experiences of major stakeholders. The benefits of this assessment are for the developers of the framework to learn from the findings of the assessment and identify any actionable improvements.

This approach is based on the SGSEAM user guide \cite{csdl2-13-06} and tailored to the assessment of the Lucid system.

\section{SGSEAM assessment}

\subsection{Step 1: Identify Stakeholders}
Identify the person(s) that use Lucid BuildingOS and BuildingDashboard system and categorize them into SGSEAM stakeholders.

\begin{itemize}
\item Player:
    residents living in the buildings which are participated in the competition

\item System admins:
    IT staffs who are responsible for setting up and maintain the software infrastructure for the competition.
    
\item Game Designers:
    Competition organizers who design/configure the competition to achieve the sustainability goal.
    They may include content experts, instructional designers, etc.

\item Game Managers:
    Competition organizers who is responsible for running the competition
    They may include Residential Life staff, Sustainability Coordinator

\item Game Developers:
    Software developers who use the game framework to customize, extend and enhance their games.    
\end{itemize}

It is desirable to identify the stakeholder persons from different competitions using the same version of the software. For example, many schools will participate in the Campus Conservation National (CCN) 2014. They will use the BuildingOS and BuildingDashboard to create their own competitions. We could select stakeholders from several competitions and use SGSEAM to assess the experiences from them. The more data we collect, the more insights we get.

\subsection{Step 2: Identify Tasks} 
For each stakeholder, identify the tasks that interact with the building OS and building dashboard.

\paragraph{Player:} Interact with the building dashboard interface including:
    \begin{compactitem}
    \item go to the homepage of the competition website
    \item actively look up information about the consumptions of one or several buildings
    \item actively look up standings, prizes
    \item actively participate in the commitments
    \item actively participate in the social sharing
    \end{compactitem}

\paragraph{System admin:} Set up and maintain the software infrastructure:
    \begin{compactitem}
    \item install the software
    \item configuring connectivity with building smart meters if available
    \item backup the data
    \item monitor the performance
    \item scaling the system
    \item patching
    \end{compactitem}
        
\paragraph{Game designer:} Design the competition:
    \begin{compactitem}
    \item Before the competition: use BuildingOS interface to configure and design the competition:\\
        - decide competition period\\
        - set up participant building information (occupancy, energy related LEED certification, manual or automated meters)\\
        - decide baseline period
    
    \item During the competition: monitor competition state, looking up scoreboard info and analytics
    \end{compactitem}

\paragraph{Game manager:} Manage the competition: 
    \begin{compactitem}
    \item enter the data manually in the case of manual meters (at least twice weekly)
    \item manage real world activities, such as events, marketing, handing out prizes
    \item monitor competition state, looking up scoreboard info and analytics
    \end{compactitem}
    
\paragraph{Game Developer:} Use the game framework to customize, extend and enhance their games:
    \begin{compactitem}
    \item use API to get data in and/or out of the system
    \item customize the interface
    \item extend the system to support new meters
    \item enhancement
    \end{compactitem}
    
\subsection{Step 3: Determine assessment approaches}
 Determine the appropriate assessment approaches for each stakeholder and carry out the assessment.

\subsubsection{Player assessment}
 
\paragraph{1. pre-post study:}

One of the goals of the competition is (but not limited to) energy consumption reduction. To assess the effectiveness of this goal, we will need to determine the metrics that may be measured before and 
after the competition to determine the effect of the competition.

Lucid Dashboard calculates percentage of reduction of energy consumption for each participated building, based on the baseline usage of the previous two weeks. 
We could use this metrics at the end of the competition to assess this aspect of the effectiveness of the competition with the respective to the players.
    
\paragraph{2. self-reported metrics: } 

We could conduct a player survey during or after the competition. A number of players (minimum of 20) could be randomly selected to participate in this survey. The survey could be administrated online via tools such as survey monkey. We could design the survey questionnaire as the following:
    
Open-ended questions: 
\begin{compactitem}
\item What did you like most about the website?
\item What did you found confusing?
\item What issues did you have while using the site?
\item What was the thing you liked the least about the site?
\item What can we do to improve the site?\\
\end{compactitem}

Close-ended questions with Likert scale from "Strong disagree" to "Strongly agree": 
\begin{compactitem}
\item It was easy to find what I was looking for on the website
\item The website was responsive 
\item I understood how to play
\item this is something my friends should participate in
\end{compactitem}

Once the survey is created online, the survey administrator could send it out via emails to the selected players with the link to the online survey and the instruction to fill out the survey online.
The survey result will be analyzed to understand the player's experience with the competition interface.
 
\paragraph{3. engagement metrics: }

This approach will gather the website usage data, which requires detailed logging of user interaction within the website. These logging includes http web server logs and/or user action logs which identify every user click on the web page. By using this website usage data, we could calculate the following metrics:

\begin{compactitem}
\item  number of players per day
\item  play time of a player per day
\item  commitment submissions of all player per day
\item  social interaction of all player per day
\item  website errors per day
\end{compactitem}

Distribution of the above metrics across of the period of the competition could provide insights on 
the extent of engagement in different time of the competition. For example, it may be typical that
the first few days of the competition may have higher engagement metrics because of the launch. Another
example of engagement metrics spur could be an announcement of an interesting real-world event. 
    
\subsubsection{System admin assessment}

Due to the cost of recruiting testing subjects and set up the experiments, in-lab experiment assessment may not be appropriate in our case. Instead, we recommend to system admin interview approach. Once we identify the contact info of the system admin of the system, the interview could be administrated by using an online questionnaire form followed by an optional phone interview if needed. We could design the interview with the following questionnaire:

\begin{compactitem}
\item How much time did you spend to install the system and the dependencies?
\item How much time did you spend to configure the meters?
\item How much time did you spend to maintain the system such as backup, patching, monitoring?
\item Did you need to scale the system? if Yes, how much time did you spend?
\item What problems did you encounter?
\item Did you find it difficult to admin the system? What was difficult?
\item Do you agree for us to call you for a short phone interview if we have more questions regarding your experience with the system?
\end{compactitem}

\subsubsection{Game designer assessment}
Similar to system admin assessment, we choose interview approach for game designer assessment. The interview could be administrated by using an online questionnaire form followed by an optional phone interview if needed. Several game designers of different competitions could be contacted for this interview. The more data we collect, the more insights we get. The interview could be designed with the following questionnaire:

\begin{compactitem}
\item How much time did you spend to set up the buildings including meters?
\item How much time did you spend to setup the competition (competition periods, baseline period, participants)?
\item How much time did you spend to setup the homepage by deciding which widgets to include?
\item How much time did you spend to monitor analytical data to understand the state of the game
\item What problems did you encounter?
\item Did you find it difficult to use the interface? What was difficult?
\item Do you agree for us to call you for a short phone interview if we have more questions regarding your experience with the system?
\end{compactitem}
    
\subsubsection{Game manager assessment}

Similar to game designer assessment, we choose interview approach for game manager assessment. The interview could be administrated by using an online questionnaire form followed by an optional phone interview if needed. Several game managers of different competitions could be contacted for this interview. The more data we collect, the more insights we get. The interview could be designed with the following questionnaire: 

\begin{compactitem}
\item How much time did you spend to enter the meter data manually for the baseline period?
\item How much time did you spend to enter the meter data manually for the competition period?    
\item What problems did you encounter?
\item How much time did you spend to monitor analytical data to understand the state of the game
\item Did you find it difficult to manage? What was difficult?\\
\end{compactitem}
    
\subsubsection{Game developer assessment}

BuildingOS and Dashboard may have APIs for developing apps to tie into the framework. We could use the API to develop an extension or customization of the system, for example, create a new widget to be available in the home page, or support the automated energy data collection from a new type of meter.

We could ask developer(s) to implement such enhancement or customization, using the APIs provided by the framework. The developers could be Lucid internal developers or some one outside of Lucid. After the developers completes the task, we will interview the developers to assess his experience for this development task. The interview could be designed with the following questionnaire:

\begin{compactitem}
\item How much time did you spend developing the customization using the game framework?
\item What problem(s) did you encounter?
\item Did you find it difficult to understand, extend and debug the system? What was difficult?\\
\end{compactitem}    
    
%% Use this for an alphabetically organized bibliography
\bibliography{sustainability,csdl-trs,gamification,yxu}
\bibliographystyle{plain}

\end{document}
