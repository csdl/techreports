%%%%%%%%%%%%%%%%%%%%%%%%%%%%%% -*- Mode: Latex -*- %%%%%%%%%%%%%%%%%%%%%%%%%%%%
%% 10-05.tex --      IEEE Smart Grid Comm paper
%% Author          : Philip Johnson
%% Created On      : Mon Sep 23 11:52:28 2002
%% Last Modified By: Philip Johnson
%% Last Modified On: Thu Apr  8 09:00:26 2010
%%%%%%%%%%%%%%%%%%%%%%%%%%%%%%%%%%%%%%%%%%%%%%%%%%%%%%%%%%%%%%%%%%%%%%%%%%%%%%%
%%   Copyright (C) 2009 Philip Johnson
%%%%%%%%%%%%%%%%%%%%%%%%%%%%%%%%%%%%%%%%%%%%%%%%%%%%%%%%%%%%%%%%%%%%%%%%%%%%%%%
%% 

%% Home page: http://www.ieee-smartgridcomm.org/submission.html

%% Must submit to one of the 12 symposia:
%% http://www.ieee-smartgridcomm.org/symposia.html

%% It appears that this is the most relevant symposia:
%% http://www.ieee-smartgridcomm.org/hibfn.html

%% For ``peer review mode'', do:
%%   \documentclass[conference,compsoc,peerreview]{IEEEtran}
%% and
%%   \IEEEpeerreviewmaketitle  (after the abstract).

%\documentclass[conference,compsoc,peerreview]{IEEEtran}
\documentclass[conference,compsoc]{IEEEtran}
\usepackage[final]{graphicx}
\usepackage{cite}
\usepackage{url}
% uncomment the % away on next line to produce the final camera-ready version
% and uncomment the \thispagestyle{empty} following \maketitle
%\pagestyle{empty}

\begin{document}

\title{WattDepot: An open source software ecosystem for enterprise-scale
  energy data collection, storage, analysis, and visualization}

\author{Robert Brewer\\
             Philip M. Johnson \\
\em  Collaborative Software Development Laboratory \\
      Department of Information and Computer Sciences \\
      University of Hawai'i \\
      Honolulu, HI 96822 \\
      rbrewer@hawaii.edu \\
}


\maketitle
%\IEEEpeerreviewmaketitle
%\thispagestyle{empty}

\begin{abstract}  % 200 words
  WattDepot is an open source, ethernet-based, service-oriented framework,
  for collection, storage, analysis, and visualization of energy data.
  WattDepot differs from other energy management solutions in one or more
  of the following ways: it is not tied to any specific metering
  technology; it provides high-level support for meter aggregation and data
  interpolation; it supports carbon intensity analysis; it is
  architecturally decoupled from the underlying storage technology; it
  supports both hosted and local energy services; it can provide near-real
  time data collection and feedback; and the software is open source and
  freely available.  In this paper, we introduce the framework, provide
  examples of its use, and discuss its application to research and
  understanding of the Smart Grid.
\end{abstract}


\section{Introduction}
\label{sec:intro}

Recent interest in the Smart Grid has produced a variety of proposed
standards and technologies for energy data collection, storage, and
analysis.  Some of the most prominent include: (...).  For our research on
producing sustained, positive behavioral change among energy consumers, we
needed a way to collect and store energy data, perform basic analyses on
it, and visualize it in a number of ways.  We found that none of the
existing technologies satisfied our needs.  Rather than implement a special
purpose solution for our research, we decided instead to develop a generic
solution under an open source development paradigm, so that others could
benefit from our efforts (and so that we could benefit from the help of
others).




\bibliographystyle{IEEEtran}
\bibliography{tdd,zorro,csdl-trs,hackystat,psp}
\end{document}











