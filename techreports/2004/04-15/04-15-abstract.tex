\begin{abstract} 

Adopting and deploying best practices is a common practice in software
organizations to improve development capabilities and process maturities.
Best practice contains ``a set of guidelines and recommendations'' for
doing something in either software development or business management
\cite{BestPracticeWiki}. For instance, design patterns are depicted in
Unified Modeling Language (UML) diagram to illustrate the generically
accepted designs to similar software problems. In the development of
software engineering discipline, researchers and practitioners came up with
best practices waterfall model, software review, extreme programming and
aspect programming etc. out of software practices. As one of the most
well-known practice, waterfall model plays a vital role in the history of
software engineering and it still exists in many modern software projects'
development. Best practice helps to yield good software processes and
improve software development. Extreme programming, one of the most famous
agile process, consists of 12 best practices \cite{Jeffries:00}.

Best practice varies from complicated and heavy practice such as waterfall
model to light-weight practice such as uniform coding style in a software
project. In software engineering, best practices are summarized and
abstracted from successful development experiences and they are constantly
being improved by practitioners. Meanwhile, a process model may be
developed to enforce the best practice discipline as extreme programming
shows. Despite its importance, developers and software organizations often
ignore and discard best practice intentionally or unintentionally.  One
important reason for this is up to the nature of best practices -- they
usually do not have solid theoretical foundation; on the contrary, best
practices are very descriptive and narrative. Lacking of strict execution
plan gives practitioners the flexibility to interpret and customize best
practices in their environments but it also brings uncertainness and
vagueness. On one hand, researchers and practitioners do not know how well
best practices are being deployed; on another hand, developers are probably
not aware of what they are doing. Janzen articulated that ``Measuring the
use of particular software development methodology is hard.  Many
organizations might be using the methodology without talking about it.
Others might claim to be using a methodology when in fact they are
misapplying it. '' \cite{Janzen:05} Without good understanding, mentor
and consultation, practitioners will lack the ability to conclude whether
they misconduct best practice or not, and it is hard to tell whether
discipline is maintained or not as well.

In our research work, we introduced a substantial step forward to help
developers and organizations retrospectively inspect software development
process with Hackystat\cite{csdl2-02-07}. On the top of Hackystat platform,
we designed and implemented software development stream analysis (SDSA)
framework to evaluate execution of best practices in software development.
In SDSA framework, procedure and key steps of best practices are represented
as rules to study software process. It measures microprocess to inspect
individual developer's development work on micro level from bottom up
starting with development activities. Microprocess is light weight on the
contrary to the traditional documentation and management oriented heavy
processes such as waterfall model and Unified Process (UP). In my thesis
work I will use SDSA framework to automatically measure and evaluate best
practice Test-Driven Development (TDD), one of the most well-known extreme
programming (XP) practices.
\end{abstract}










