%%%%%%%%%%%%%%%%%%%%%%%%%%%%%% -*- Mode: Latex -*- %%%%%%%%%%%%%%%%%%%%%%%%%%%%
%% 04-14-hackystat.tex -- Thesis proposal - software inspections
%% Author          : Aaron A. Kagawa
%% Created On      : Mon Sep 23 11:52:28 2004
%% Last Modified By: Aaron Kagawa
%% Last Modified On: Wed Jan 26 11:25:36 2005
%% RCS: $Id$
%%%%%%%%%%%%%%%%%%%%%%%%%%%%%%%%%%%%%%%%%%%%%%%%%%%%%%%%%%%%%%%%%%%%%%%%%%%%%%
%%   Copyright (C) 2004 Aaron A. Kagawa
%%%%%%%%%%%%%%%%%%%%%%%%%%%%%%%%%%%%%%%%%%%%%%%%%%%%%%%%%%%%%%%%%%%%%%%%%%%%%%%
%% 

\chapter{The Hackystat System}
\label{chapter:hackystat}

The Priority Ranked Inspection process is a theoretical process that can be
implemented in many different ways. For this proposed research I will
utilize the Hackystat system to aid the implementation of PRI. This chapter
is an introduction to the Hackystat system which was invented by Dr. Philip 
M. Johnson, in the Collaborative Software Development Laboratory,
Department of Information and Computer Sciences, University of Hawaii at
Manoa. 

\section{Overview of the Hackystat System}
The Hackystat system is an open-source software framework for the automated
collection and analysis of software product and process measures. Product
measures can be defined as, measures that are obtainable from direct
analysis of source code. For example, some product measures can include:
lines of code, complexity, and the number of unit tests. Process measures
can be defined as, measures that are obtainable from the actual development
process which creates the source code. For example, the number of
developers, the developers' ``effort'', the number of major releases and
the number of defects, are examples of process measures. 

The following list summarizes the features that Hackystat provides: 
\cite{Johnson05}:

\begin{enumerate}
\item Hackystat utilizes custom ``sensors'' that are ``attached'' to
  various software development tools. Theses sensors unobtrusively collect
  data on various software product and development process measures. 
\item Hackystat supports any and all software projects, development
  processes, software development environments, operating systems, and
  development tools. 
\item Hackystat supports in-process project management by providing a set
  of extendible analyses of the product and process measures that are
  collect by the sensors.
\item Hackystat is well suited for empirical software development
  experimentation.
\end{enumerate}

The Hackystat system is a mature and extendible software system. Currently, 
Hackystat is being utilized by NASA's Jet Propulsion Laboratory, Sun
Microsystems, IBM, University of Torino, University of Maryland, and of
course the University of Hawaii. 

%%\subsection{Hackystat Sensors}

%%\subsection{Basic Usage Scenario}

%%\subsection{Example Hackystat Analyses}










