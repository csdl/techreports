%%%%%%%%%%%%%%%%%%%%%%%%%%%%%% -*- Mode: Latex -*- %%%%%%%%%%%%%%%%%%%%%%%%%%%%
%% 04-14-initalresults.tex -- Thesis white paper - software inspections
%% Author          : Aaron A. Kagawa
%% Created On      : Mon Sep 23 11:52:28 2004
%% Last Modified By: Aaron Kagawa
%% Last Modified On: Wed Nov 17 01:39:38 2004
%% RCS: $Id$
%%%%%%%%%%%%%%%%%%%%%%%%%%%%%%%%%%%%%%%%%%%%%%%%%%%%%%%%%%%%%%%%%%%%%%%%%%%%%%
%%   Copyright (C) 2004 Aaron A. Kagawa
%%%%%%%%%%%%%%%%%%%%%%%%%%%%%%%%%%%%%%%%%%%%%%%%%%%%%%%%%%%%%%%%%%%%%%%%%%%%%%%
%% 

\Section{Initial Results}
The use of the Hackystat Quality Extension system to provide the
determination of ``most'' and ``least'' need of inspection has been promising.
The initial implementation of the system has proven that it is technically
possible to do what I have envisioned. In addition, I have already
recommended the inspection of a package that was in ``most need of a inspection''
and the defects and issues identified have confirmed that the package had
low quality.

Of course, I will continue to discover new attributes to define quality,
fine tune the numerical weights associated with the attributes, and
continue to recommend inspections until I believe my mechanism is ready for a
thorough evaluation.






