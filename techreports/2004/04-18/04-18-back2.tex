%%%%%%%%%%%%%%%%%%%%%%%%%%%%%% -*- Mode: Latex -*- %%%%%%%%%%%%%%%%%%%%%%%%%%%%
%% 04-18.tex -- Improving review process quality with Jupiter-Hackystat paradigm
%% Author          : Takuya Yamashita
%% Created On      : Mon Sep 23 11:52:28 2002
%% Last Modified By:
%% Last Modified On: Fri Sep 24 16:22:32 2004
%% RCS: $Id$
%%%%%%%%%%%%%%%%%%%%%%%%%%%%%%%%%%%%%%%%%%%%%%%%%%%%%%%%%%%%%%%%%%%%%%%%%%%%%%%
%%   Copyright (C) 2002 Philip Johnson
%%%%%%%%%%%%%%%%%%%%%%%%%%%%%%%%%%%%%%%%%%%%%%%%%%%%%%%%%%%%%%%%%%%%%%%%%%%%%%%
%%

%% This is a sample file showing how to produce CSDL TechReports in ICSE
%% conference style using LaTeX.  It can be adapted to thesis structure
%% with very minor changes.

\documentclass[11pt,twocolumn]{article}
\input{/export/home/csdl/tex/psfig/psfig}
\usepackage{/export/home/csdl/tex/icse2003/latex8}
\usepackage{times}
%% A verbatim-like environment which allows font changes
%%\usepackage{alltt}
%% New LaTeX2e graphics support
\usepackage[final]{graphicx}
% uncomment the % away on next line to produce the final camera-ready version
% and uncomment the \thispagestyle{empty} following \maketitle
\pagestyle{empty}

\begin{document}

\title{Evaluating Automated Review Framework \\ with Jupiter and Hackystat tools}

\author{\protect\begin{tabular}{ccc}
Takuya Yamashita  \\
\end{tabular}\\
\em  Collaborative Software Development Laboratory \\
\em  Department of Information and Computer Sciences \\
\em  University of Hawai'i \\
\em  Honolulu, HI 96822 \\
\em  takuyay@hawaii.edu} \maketitle \thispagestyle{empty}

\begin{abstract}  % 200 words

Over many years, there is general agreement that software inspection
reduces development costs and improves product quality by finding
defects in early software development, and these software tools help
the inspection process efficiently and effectively. Even though the
inspection tools are evaluated to be useful enough to provide
efficiency and effectiveness to reviewers, it is still hard to
understand the efficiency of the sequence of the inspection process
only by the evaluation of inspection tools.

By using Jupiter (review tool) and Hackystat (automated metric
collection system), the issue must be addressed if the automated
review framework provides the better understanding of the efficient
sequence of the inspection process in such a way that reviewers and
team can be aware of useful review metrics to reflect the next
review session. The framework could gather the review active time to
find who does not have enough preparation for team review, gather
defect type and severity to find the review trend, and educate
reviewers to be aware of the kind of issues.

To investigate my research questions, I propose a controlled review
experiment in software engineering class to give Java based
programming assignments to around 20 students. In first several
round, students are forced to learn both text editor based and
Jupiter based review. Qualitative evaluation will be conducted
before and after the round to see the usefulness of review tool. In
the next several round, Hackystat metric collection system will be
introduced to gather review metrics. It could provide the review
analysis such as prepared reviewers, categorized defect types and
severity, number of confirmed issues, and so forth. The second
qualitative evaluation will be conducted before and after the
introduction to see the usefulness of automated review framework.

The expected results would be that automated review framework,
including the review tool, provides some useful aspects for
sequential review process. This result would be one milestone for
empirical software review community.

Finally, this thesis is going to be done by not later than May, 2005
with first milestone for the implementation of analysis tool by
January 2005, second milestone for the evaluation of the review tool
by March 2005, and final milestone for the evaluation of automated
review system by May 2005.

\end{abstract}


%\bibliographystyle{/export/home/csdl/tex/icse2003/latex8}
%\bibliography{/export/home/csdl/bib/review}
\end{document}
