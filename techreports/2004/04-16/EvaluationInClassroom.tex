\chapter{Case Study in a Classroom}  \label{Chapter:EvaluationInClassroom}
 

This chapter reports on a pilot case study with senior undergraduate and introductory graduate software engineering students in 2005 Spring semester at University of Hawaii. The study had dual purposes:
\begin{itemize}
	\item To test drive the software project telemetry system in preparation for the next two full-scale case studies as described in Chapter \ref{Chapter:EvaluationInCSDL} and \ref{Chapter:EvaluationInIkayzo}.
	\item To collect the students' opinion about software project telemetry when the adoption of the technology is mandated by their instructor.
\end{itemize}

The design was the simple one shot case study. The students used software project telemetry to collect and analyze their metrics while performing software development tasks during the semester. At the end of the semester, a survey questionnaire was distributed to collect their opinions with respect to software project telemetry. The actual telemetry analysis invocation data \textcolor{red}{will be} used to corroborate the qualitative findings.

This chapter begins with a description of the classroom setting in Section \ref{EvaluationInClassroom:Setting}.
Section \ref{EvaluationInClassroom:Design} outlines the design and strategies of this case study and offers rationale for the decisions.
Section \ref{EvaluationInClassroom:Role} describes the researcher's role.
Section \ref{EvaluationInClassroom:DataAnalysis} introduces data collection and analysis procedures.
Section \ref{EvaluationInClassroom:Threats} discuses the limitations and threats. 
Section \ref{EvaluationInClassroom:Results} reports the results. 
Section \ref{EvaluationInClassroom:Conclusion} concludes the chapter with a summary of the insights learned from this case study.




%%%%%%%%%%%%%%%%%%%%%%%%%%%%%%%%%%%%%%%%%%%%%%%%%%%%%%%%%
%                                                       %
%                   S E C T I O N                       %
%                                                       %
%%%%%%%%%%%%%%%%%%%%%%%%%%%%%%%%%%%%%%%%%%%%%%%%%%%%%%%%%

\section{Classroom Setting} \label{EvaluationInClassroom:Setting}

The case study was conducted in Dr. Philip Johnson's senior undergraduate (ICS 414) and introductory graduate (ICS 613) software engineering classes in Spring 2005 at University of Hawaii. The two classes followed the same basic curriculum except that the graduate level one had more supplementary readings. The curriculum had two equally important components:

\begin{enumerate}
	\item \textbf{Techniques and tools for Java-based software development} --- The students in the classes were divided into teams of 2 to 4 members and worked on different projects. The techniques taught in class covered code formatting and documentation best practices, design patterns, configuration management, code review, and agile development practice; while the tools used included Eclipse (an IDE primarily for Java development), CVS (a configuration management system), Ant (a Java-based build tool), and JUnit (a Java unit test framework).

	\item \textbf{Software product and process metrics collection and analysis} --- The two classes required the students to collect and analyze their software process and product metrics while performing software development tasks. The purpose was to help them acquire hands-on experience in collecting, analyzing, and interpreting software metrics. 	
% The curriculum takes an experiential-centered approach.
\end{enumerate}

The students' product and process metrics were collected and analyzed using the software project telemetry system. The implementation of the system was based on the Hackystat framework. It was first incorporated in Dr. Philip Johnson's software engineering class in 2004 Fall semester. This was the second semester that it was used. The system was deployed on a university public server\footnote{http://hackystat.ics.hawaii.edu}. It had a web interface to allow the students to manage their projects and invoke telemetry analysis over their data to gain insights into their software development processes. In order to gather product and process metrics, the students were required to instrument their development environment and build tools with sensors. These sensors collected a wide variety of information which included the time each project member spent editing source files, code structural metrics, the occurrence of unit tests and their results, and system test coverage.





%%%%%%%%%%%%%%%%%%%%%%%%%%%%%%%%%%%%%%%%%%%%%%%%%%%%%%%%%
%                                                       %
%                   S E C T I O N                       %
%                                                       %
%%%%%%%%%%%%%%%%%%%%%%%%%%%%%%%%%%%%%%%%%%%%%%%%%%%%%%%%%

\section{Case Study Design and Strategies} \label{EvaluationInClassroom:Design}

The software engineering classroom was an environment where the use of software project telemetry could be mandated. Part of the course goal was to let the students gain hands-on experience with metrics collection and analysis. The software project telemetry system was used as a tool by the students to collect and analyze their product and process metrics.

Apart from test driving the software project telemetry system, the purpose of this case study was to collect the students' opinion about software project telemetry such as which part of technology worked, which part didn't, which part the students liked, which part they didn't like, and how the system could be improved.  

The mixed methods paradigm was adopted in this case study. Since this was mainly a naturalistic study exploring how software project telemetry worked for software engineering students, priority was skewed toward qualitative information obtained during this case study. Quantitative data \textcolor{red}{will be} used to corroborate the qualitative findings.

The two most popular approaches in qualitative data collection are survey (either interview or questionnaire) and ethnographic observation. The advantage of ethnography was that it could enable me to observe how the students interacted with software project telemetry and how the technology facilitated them make project management and process improvement decisions. However, there were 25 students enrolled in the two classes. Obviously following those 25 students around and watching them developing software was an infeasible solution. As a result, I decided to use a survey. The interview method was not chosen because out of concern that the students might feel pressured to give more favorable opinions to bias the study results. Instead, an anonymous questionnaire survey method was my final choice. It had advantages such as ease of administration, rapid turn around in data collection.
% and guarantee of 100\% response rate in the classroom setting. (students can decide not to participate)

The decision to use questionnaire survey, at the same time, implied that I had to rely on the students' self-reported opinions to evaluate software project telemetry. This threat was mitigated through data triangulation. All telemetry analyses invoked by the students were logged. If they ran the analyses on a regular basis, then I would put more confidence in their survey responses. On the other hand, if it turned out that they seldom invoked the telemetry analyses, then I had to discount their opinions. 





%%%%%%%%%%%%%%%%%%%%%%%%%%%%%%%%%%%%%%%%%%%%%%%%%%%%%%%%%
%                                                       %
%                   S E C T I O N                       %
%                                                       %
%%%%%%%%%%%%%%%%%%%%%%%%%%%%%%%%%%%%%%%%%%%%%%%%%%%%%%%%%

\section{Researcher's Role} \label{EvaluationInClassroom:Role}

The instructor of the two software engineering classes in which this case study was conducted was Dr. Philip Johnson. He is my dissertation adviser. The software project telemetry system is implemented by me. It was used as a tool by the students to collect and analyze their software product and process metrics. I helped the instructor predefine the telemetry charts and reports that we thought were most useful to the students. However, I did not participate in the teaching of the course.




%%%%%%%%%%%%%%%%%%%%%%%%%%%%%%%%%%%%%%%%%%%%%%%%%%%%%%%%%
%                                                       %
%                   S E C T I O N                       %
%                                                       %
%%%%%%%%%%%%%%%%%%%%%%%%%%%%%%%%%%%%%%%%%%%%%%%%%%%%%%%%%

\section{Data Collection and Analysis Procedures} \label{EvaluationInClassroom:DataAnalysis}

This case study collected data from 2 sources:
\begin{itemize}
	\item A survey questionnaire was distributed at the end of the semester asking the students' opinion of software project telemetry.
	\item The software project telemetry system was instrumented. It logged all telemetry analysis invocation information.
\end{itemize}
The 2 sources of data were integrated at data interpretation phase, with priority skewed toward qualitative information and quantitative data corroborating the qualitative findings.


\subsection{Survey Questionnaire} \label{EvaluationInClassroom:Survey}

The survey was conducted through a written questionnaire administered on the last day of instruction. The questions covered metrics collection overhead, telemetry analysis usability and utility, as well as the students' perception whether software project telemetry was a reasonable approach to process improvement and project management in ``real world'' settings.

Each question was represented by a statement. For example, when collecting information about telemetry analysis utility, I made the statement \textit{``telemetry analyses have shown me valuable insight into my and my team's software development process''}. I then asked the students to rank their feelings toward the statement on a scale from 1 to 5:

\begin{itemize}
	\item 1 = strongly disagree
	\item 2 = disagree
	\item 3 = neutral
	\item 4 = agree
	\item 5 = strongly agree
\end{itemize}

The 6th option \textit{``not applicable''} was provided as well to allow the students to skip the questions which they were unable to answer. At the end of each question, I provided large empty space to allow the students to record any related comments such as justification or elaboration of the answer.

The last page of the questionnaire was a free response section where the students were encouraged to supply any additional comments such as their general opinions toward software measurement, their concerns about the way software project telemetry handled their personal process data, their suggestions on how the system could be improved, etc.

The actual questionnaire is available in Appendix \ref{Appendix:EvaluationInClassroom} for further reference.


\subsection{System Invocation Log} \label{EvaluationInClassroom:InvocationLog}

The software project telemetry system exposes a web interface though which users can invoke telemetry analyses over his own or his team's software product and process metrics. The system deployed for the classroom use was instrumented with automatic logging facility. Analysis invocation time, user name, and full request string are recorded.  



%%%%%%%%%%%%%%%%%%%%%%%%%%%%%%%%%%%%%%%%%%%%%%%%%%%%%%%%%
%                                                       %
%                   S E C T I O N                       %
%                                                       %
%%%%%%%%%%%%%%%%%%%%%%%%%%%%%%%%%%%%%%%%%%%%%%%%%%%%%%%%%

\section{Threats, Verification, and Validation} \label{EvaluationInClassroom:Threats}


\subsection{Measurement Validity}
	
Measurement validity concerns about whether we are measuring what we are supposed to measure and whether the instrument can measure the data correctly. 

The survey was conducted less than one week before the final examination. Students might be concerned that their response would influence their final grades and thus would comment more favorably toward software project telemetry. The threat was addressed in several different ways:

\begin{itemize}
	\item First, the survey was anonymous and the students were specifically instructed not to reveal their names in their responses.
	
	\item Second, I personally assured the students that their responses would be sealed until after their instructor had turned in their final grades.
	
	\item Third, this entire research was conducted under the approval of the ``committee for the protection of human subjects'' at University of Hawaii. All research participants are protected by the committee.
	
	\item Lastly, all telemetry analysis invocations were logged. The students's analysis invocation information were used to assess ``truthfulness'' of their opinion (see Section \ref{EvaluationInClassroom:Results:InvocationLog}).
	
\end{itemize}



 
\subsection{Internal Validity}
 
Internal validity is related to cause and effect, in other words, whether the treatment (using software project telemetry) actually causes the observed results (an improvement in software development process). 

This case study did not employ ethnography to observe how software project telemetry impacted the students' software development process, because it was infeasible to track the activities of all the 25 students at the same time. Instead, a survey questionnaire was used. To mitigate this threat, I included a question asking the students whether they felt that their development actions had a causal relationship to telemetry streams (survey question 5). In other words, does telemetry provided a model for development that actually reflected changes in their behavior? This might not be enough to get a handle on causality, but this question will be further addressed in Chapter \ref{Chapter:EvaluationInCSDL} where ethnography was employed to observer the impact of software project telemetry on CSDL developers' build process.

 
 
 

\subsection{External Validity}

External validity refers to the extent to which the results obtained in this case study can be generalized to a larger population. 

This case study drew results from a limited sample size (25 students). All survey participants were computer science students taking either senior undergraduate or introductory graduate software engineering classes at the University of Hawaii in Spring 2005. The students might have a relatively homogeneous background in software development which might not be representative of software developers at a whole. The context of this study was a course setting. Course projects tend to be smaller and narrower in scope. 

These external threats are limitations of this case study. The best way to address them is to conduct more case studies in different environments. Chapter \ref{Chapter:EvaluationInCSDL} and \ref{Chapter:EvaluationInIkayzo} report on two more case studies in CSDL and Ikayzo respectively.


%Though software software project telemetry offers potentially powerful information, understanding and interpreting telemetry data requires commitment to metrics-based project management and process improvement. The students have different time constraints, and they are likely to have different motivation to produce high quality software products than professional developers. %It is one thing to get a less than ideal grade for doing poor homework, but it is completely another thing to get fired for performing a lousy job. 








%%%%%%%%%%%%%%%%%%%%%%%%%%%%%%%%%%%%%%%%%%%%%%%%%%%%%%%%%
%                                                       %
%                   S E C T I O N                       %
%                                                       %
%%%%%%%%%%%%%%%%%%%%%%%%%%%%%%%%%%%%%%%%%%%%%%%%%%%%%%%%%

\section{Results} \label{EvaluationInClassroom:Results}

All of the 25 students enrolled in the two software engineering classes participated in this study, of which 9 were from the senior undergraduate section and 16 were from the introductory graduate section. The students had fairly diversified background. Their total programming experience, as defined from the first ``Hello World'' toy application, ranged from 3 to 25 years, with a mean of 6.92 and a standard deviation of 4.43. Their paid professional experience\footnote{In this question, I specifically asked the students to exclude the experience of half-time or less than half-time on-campus employment, such as student helper or research assistant, even if they were paid to program.} ranged from 0 to 8 years, with a mean of 1.27 and a standard deviation of 2.10 \footnote{One student did not answer this question and thus was not included in the statistics.}. The survey was conducted in a normal class session (on the last day of instruction), and the response rate is 100\%.


\subsection{Results from Individual Question}

The individual survey questions are listed below along with the results. Each question was in the form of a statement, and the students were asked to to circle the number that most closely matched their feelings about the statement. The options were 1 (strongly disagree), 2 (disagree), 3 (neutral), 4 (agree), 5 (strongly agree), and 6 (not applicable). The resulting statistics were computed by excluding those ``not applicable'' answers.


\newpage
\textbf{Statement 1: I have no trouble installing and configuring the sensors.}

Software project telemetry utilizes sensors to collect metrics, which intends to make the data collection process transparent and unobtrusive to developers. The sensors must be installed and configured properly before they can do their jobs. This question was designed to gather information about the one-time setup cost of the sensors. 

\begin{quote}\end{quote} % make some lines

\begin{figure}[h]
  \center
  \includegraphics[width=0.60\textwidth]{figures/ClassroomSurvey-Q1}
  \label{fig:InClassSurvey-Q1}
\end{figure}

\begin{center}Response Rate: 25/25\end{center}
\begin{table}[h]
	\centering
		\begin{tabular}{|c|c|c|c|c|} 
			\hline
			\textbf{Strongly Disagree} & \textbf{Disagree} & \textbf{Neutral} & \textbf{Agree} & \textbf{Strongly Agree} \\
			\hline
			\textit{1} & \textit{10} & \textit{3} & \textit{6} &\textit{5} \\
			\hline
		\end{tabular}
	\label{table:InClassSurvey-Q1}
\end{table}

%\begin{table}[h]
%	\centering
%		\begin{tabular}{|c|c|c|c|} 
%			\hline
%			\textbf{Mean Response} & \textbf{Standard Deviation} & \textbf{Response Range} & \textbf{Response Rate} \\
%			\hline
%			\textit{3.16} & \textit{1.28} & \textit{[1, 5]} & \textit{25/25} \\
%			\hline
%		\end{tabular}
%	\label{table:InClassSurvey-Q1}
%\end{table}
%
%\begin{center}
%\textit{1 = strongly disagree, 2 = disagree, 3 = neutral, 4 = agree, 5 = strongly agree}
%\end{center}
%\begin{quote}\end{quote} % make some lines

It turned out that installation and configuration of the sensors involved quite complex procedures. 40\% of the respondents did not agree with the statement. One of the students wrote that he/she was still having troubles with the installation of some of the sensors at the end of the semester. Most students expressed the wish to have an all-in-one intelligent graphical user interface to install and configure the sensors.

%3.16 (1.28) [1, 5]
%*I didn't read much documentation. I followed directions in class.
%*I thought the instructions were much too detailed. I got lost with details.
%*No all-in-one installer. To much manual work.
%*It took some time to install it.
%*Should really consider creating installer scripts.
%*I could not figure out that I need to put sensor.properties to .hackystat folder.
%*I still have problems with some sensors.
%*Problem with regional time format settings.
%*Not hard, just sometimes it seemed data didn't show up.
%*Need GUI driver process. 


\newpage
\textbf{Statement 2: After sensors are installed and configured, there is no overhead collecting metrics.}

This question was designed to gather information about software metrics collection overhead as well. However, it had different goal than the previous question. This question focused on long-term chronic metrics collection overhead, while the previous one emphasized on the up-front sensor setup cost.

\begin{quote}\end{quote} % make some lines

\begin{figure}[h]
  \center
  \includegraphics[width=0.60\textwidth]{figures/ClassroomSurvey-Q2}
  \label{fig:InClassSurvey-Q2}
\end{figure}

\begin{center}Response Rate: 24/25\end{center}
\begin{table}[h]
	\centering
		\begin{tabular}{|c|c|c|c|c|} 
			\hline
			\textbf{Strongly Disagree} & \textbf{Disagree} & \textbf{Neutral} & \textbf{Agree} & \textbf{Strongly Agree} \\
			\hline
			\textit{0} & \textit{0} & \textit{6} & \textit{7} &\textit{11} \\
			\hline
		\end{tabular}
	\label{table:InClassSurvey-Q2}
\end{table}

%\begin{table}[h]
%	\centering
%		\begin{tabular}{|c|c|c|c|} 
%			\hline
%			\textbf{Mean Response} & \textbf{Standard Deviation} & \textbf{Response Range} & \textbf{Response Rate} \\
%			\hline
%			\textit{4.21} & \textit{0.83} & \textit{[3, 5]} & \textit{24/25} \\
%			\hline
%		\end{tabular}
%	\label{table:InClassSurvey-Q2}
%\end{table}
%
%\begin{center}
%\textit{1 = strongly disagree, 2 = disagree, 3 = neutral, 4 = agree, 5 = strongly agree}
%\end{center}
%\begin{quote}\end{quote} % make some lines

The sensor-based metrics collection approach adopted in software project telemetry appeared to have achieved its design goal of eliminating long-term chronic data collection overhead. No one disagreed with the statement. The reason that some students chose ``neutral'' was mainly because they encountered bugs in one of the sensors released prematurely in order to meet the course schedule.


%4.21 (0.83) [3, 5] (one answered N/A)
%*Problems occurred with the new sensors themselves. They failed to collect/send data.
%*Sometimes the sensor do not send data and you don't know until late.
%*Broadband won't feel a thing.


\newpage
\textbf{Statement 3: It's simple to invoke predefined telemetry chart and report analyses.}

The software project telemetry implementation offered both an expert interface where users could use telemetry language to interact with the system and an express interface which allowed users to retrieve predefined telemetry charts and reports. The telemetry language was not introduced in the classes, nor was the expert interface. The instructor and I predefined dozens of telemetry charts and reports that we thought were most useful to the students. This question was intended to gather information about the usability of the express interface.

\begin{quote}\end{quote} % make some lines

\begin{figure}[h]
  \center
  \includegraphics[width=0.60\textwidth]{figures/ClassroomSurvey-Q3}
  \label{fig:InClassSurvey-Q3}
\end{figure}

\begin{center}Response Rate: 24/25\end{center}
\begin{table}[h]
	\centering
		\begin{tabular}{|c|c|c|c|c|} 
			\hline
			\textbf{Strongly Disagree} & \textbf{Disagree} & \textbf{Neutral} & \textbf{Agree} & \textbf{Strongly Agree} \\
			\hline
			\textit{0} & \textit{2} & \textit{5} & \textit{11} &\textit{6} \\
			\hline
		\end{tabular}
	\label{table:InClassSurvey-Q3}
\end{table}

%\begin{table}[h]
%	\centering
%		\begin{tabular}{|c|c|c|c|} 
%			\hline
%			\textbf{Mean Response} & \textbf{Standard Deviation} & \textbf{Response Range} & \textbf{Response Rate} \\
%			\hline
%			\textit{3.88} & \textit{0.90} & \textit{[2, 5]} & \textit{24/25} \\
%			\hline
%		\end{tabular}
%	\label{table:InClassSurvey-Q3}
%\end{table}
%
%\begin{center}
%\textit{1 = strongly disagree, 2 = disagree, 3 = neutral, 4 = agree, 5 = strongly agree}
%\end{center}
%\begin{quote}\end{quote} % make some lines

Though most students agreed that the express interface was easy to use, they thought it could be improved. It turned out that a major problem involved the input of telemetry analysis parameter values. Different telemetry charts or reports had different parameter requirements, but it was hard to tell from the user interface what parameters were expected.




%3.88 (0.90) [ 2-5] (one answered N/A)
%*Some tasks are somewhat not visible.
%*Some require unknown parameters.
%*Some. Others require parameters, but no instructions on what those parameters might be.
%*They don't work so well due to the last parameter option. What goes there? How about a help link for each option?
%*The report names aren't that descriptive, and the the parameters needed were confusing.

\newpage
\textbf{Statement 4: Telemetry analyses have shown me valuable insight into my and / or my team's software development process.}

One of the design goals of software project telemetry is to make the development process as transparent as possible so that software development problems can be detected early. This question was intended to measure whether software project telemetry had achieved that goal from the perspective of software developers.

\begin{quote}\end{quote} % make some lines

\begin{figure}[h]
  \center
  \includegraphics[width=0.60\textwidth]{figures/ClassroomSurvey-Q4}
  \label{fig:InClassSurvey-Q4}
\end{figure}

\begin{center}Response Rate: 25/25\end{center}
\begin{table}[h]
	\centering
		\begin{tabular}{|c|c|c|c|c|} 
			\hline
			\textbf{Strongly Disagree} & \textbf{Disagree} & \textbf{Neutral} & \textbf{Agree} & \textbf{Strongly Agree} \\
			\hline
			\textit{0} & \textit{0} & \textit{5} & \textit{10} &\textit{10} \\
			\hline
		\end{tabular}
	\label{table:InClassSurvey-Q4}
\end{table}

%\begin{table}[h]
%	\centering
%		\begin{tabular}{|c|c|c|c|} 
%			\hline
%			\textbf{Mean Response} & \textbf{Standard Deviation} & \textbf{Response Range} & \textbf{Response Rate} \\
%			\hline
%			\textit{4.20} & \textit{0.76} & \textit{[3, 5]} & \textit{25/25} \\
%			\hline
%		\end{tabular}
%	\label{table:InClassSurvey-Q4}
%\end{table}
%
%\begin{center}
%\textit{1 = strongly disagree, 2 = disagree, 3 = neutral, 4 = agree, 5 = strongly agree}
%\end{center}
%\begin{quote}\end{quote} % make some lines

No one disagreed with the statement. This indicated that software project telemetry had largely achieved the goal of making development process transparent. However, some students expressed concern about data privacy. Though steps have been taken during system design to only allow data sharing among project members, it seems that we have to do more work with data privacy.


%4.20 (0.76) [3-5]
%*Mostly it shows who doesn't work.
%*Dr. Johnson can track our progress. Yikes!.
%*Have not done enough projects to get a pattern.
%*Sometimes the data is not precise.
%*Makes me more aware of what others are doing (good or bad)
%*Good tool.


\newpage
\textbf{Statement 5: Telemetry analyses have helped me improve my software development process.}

Software project telemetry helps a developer improve his/her development process by making the development process transparent and the information available to the user, but it's not its design goal to make process improvement decisions on behalf of the user. Whether a developer can improve his/her development process depends crucially on (1) his/her ability to interpret telemetry results and take actions accordingly, and (2) whether telemetry analysis is able to deliver the relevant information in an easy-to-understand way. This question was intended to ask the students whether there was any self-perceived process improvement after using the system.

\begin{quote}\end{quote} % make some lines

\begin{figure}[h]
  \center
  \includegraphics[width=0.60\textwidth]{figures/ClassroomSurvey-Q5}
  \label{fig:InClassSurvey-Q5}
\end{figure}

\begin{center}Response Rate: 25/25\end{center}
\begin{table}[h]
	\centering
		\begin{tabular}{|c|c|c|c|c|} 
			\hline
			\textbf{Strongly Disagree} & \textbf{Disagree} & \textbf{Neutral} & \textbf{Agree} & \textbf{Strongly Agree} \\
			\hline
			\textit{1} & \textit{2} & \textit{6} & \textit{13} &\textit{3} \\
			\hline
		\end{tabular}
	\label{table:InClassSurvey-Q5}
\end{table}

%\begin{table}[h]
%	\centering
%		\begin{tabular}{|c|c|c|c|} 
%			\hline
%			\textbf{Mean Response} & \textbf{Standard Deviation} & \textbf{Response Range} & \textbf{Response Rate} \\
%			\hline
%			\textit{3.60} & \textit{0.96} & \textit{[1, 5]} & \textit{25/25} \\
%			\hline
%		\end{tabular}
%	\label{table:InClassSurvey-Q5}
%\end{table}
%
%\begin{center}
%\textit{1 = strongly disagree, 2 = disagree, 3 = neutral, 4 = agree, 5 = strongly agree}
%\end{center}
%\begin{quote}\end{quote} % make some lines

Most students concurred with the statement that telemetry analyses had helped them improve their software development process. However, due to the limitation of this study, questions remained. I was not able to tell whether the students' self-perceived improvement in their development process was due to the use of the software project telemetry system, or the fact that they learned new development best practice in class, or both.

%3.6 (0.96) [1-5]
%The first key is that analyses have made me aware.
%During the development, not really. Afterwards, it's fun to look at it.
%Maybe consider more information gathering to give programmers insights v.s. giving manager insight.
%(Telemetry analyses have helped me improve my software development process) compared to what I learned in class, from students, and on the web.


\newpage
\textbf{Statement 6: If I was a professional software developer, I will want to use telemetry analyses in my development projects.}

This question was intended to ask the students whether they perceived software project telemetry as a reasonable approach to process improvement in ``real'' software development settings from the perspective of a developer. 

\begin{quote}\end{quote} % make some lines

\begin{figure}[h]
  \center
  \includegraphics[width=0.60\textwidth]{figures/ClassroomSurvey-Q6}
  \label{fig:InClassSurvey-Q6}
\end{figure}

\begin{center}Response Rate: 25/25\end{center}
\begin{table}[h]
	\centering
		\begin{tabular}{|c|c|c|c|c|} 
			\hline
			\textbf{Strongly Disagree} & \textbf{Disagree} & \textbf{Neutral} & \textbf{Agree} & \textbf{Strongly Agree} \\
			\hline
			\textit{1} & \textit{1} & \textit{7} & \textit{11} &\textit{5} \\
			\hline
		\end{tabular}
	\label{table:InClassSurvey-Q6}
\end{table}

%\begin{table}[h]
%	\centering
%		\begin{tabular}{|c|c|c|c|} 
%			\hline
%			\textbf{Mean Response} & \textbf{Standard Deviation} & \textbf{Response Range} & \textbf{Response Rate} \\
%			\hline
%			\textit{3.72} & \textit{0.98} & \textit{[1, 5]} & \textit{25/25} \\
%			\hline
%		\end{tabular}
%	\label{table:InClassSurvey-Q6}
%\end{table}
%
%\begin{center}
%\textit{1 = strongly disagree, 2 = disagree, 3 = neutral, 4 = agree, 5 = strongly agree}
%\end{center}
%\begin{quote}\end{quote} % make some lines

The majority of the students confirmed the value of software project telemetry as a reasonable approach to software process improvement. However, some of them expressed the concern about sharing private personal process data with others. For example, one student said \textit{``I don't want to show the data to my boss.''}. Data privacy is a serious issue. If it can not be handled properly, it might be a significant adoption barrier to the technology.


%3.72 (0.98) [1-5]
%*Only after getting more experience with the tool would I feel confortable to use it in a prefessional development environment.
%*I'll be concetrating on my work. Stats don't really matter.
%*I don't want to show the data to my boss.
%*More useful for management.
%*Only if I was in a development team. If I was alone I am not sure I'd use it.


\newpage
\textbf{Statement 7: If I was a project manager, I will want to use telemetry analyses in my development projects.}

Software project telemetry can be used by a project manager to monitor development progress. This question was intended to ask the students whether they perceived software project telemetry as a reasonable approach to project management in ``real'' software development settings from the perspective of a project manager. 

\begin{quote}\end{quote} % make some lines

\begin{figure}[h]
  \center
  \includegraphics[width=0.60\textwidth]{figures/ClassroomSurvey-Q7}
  \label{fig:InClassSurvey-Q7}
\end{figure}

\begin{center}Response Rate: 25/25\end{center}
\begin{table}[h]
	\centering
		\begin{tabular}{|c|c|c|c|c|} 
			\hline
			\textbf{Strongly Disagree} & \textbf{Disagree} & \textbf{Neutral} & \textbf{Agree} & \textbf{Strongly Agree} \\
			\hline
			\textit{0} & \textit{0} & \textit{3} & \textit{12} &\textit{10} \\
			\hline
		\end{tabular}
	\label{table:InClassSurvey-Q7}
\end{table}

%\begin{table}[h]
%	\centering
%		\begin{tabular}{|c|c|c|c|} 
%			\hline
%			\textbf{Mean Response} & \textbf{Standard Deviation} & \textbf{Response Range} & \textbf{Response Rate} \\
%			\hline
%			\textit{4.28} & \textit{0.68} & \textit{[3, 5]} & \textit{25/25} \\
%			\hline
%		\end{tabular}
%	\label{table:InClassSurvey-Q7}
%\end{table}
%
%\begin{center}
%\textit{1 = strongly disagree, 2 = disagree, 3 = neutral, 4 = agree, 5 = strongly agree}
%\end{center}
%\begin{quote}\end{quote} % make some lines

Nobody disagreed with this statement. When used as a management tool, there is no data privacy issue from the perspective of a project manager. The more information that is available to the project manager, the better informed decision he/she can make. 



%4.28 (0.68) [3-5]
%*See above comment - would add training of development team. (above comment refers to the student's comment for Q6: Only after getting more experience with the tool would I feel confortable to use it in a prefessional development environment).
%*I want to know what people are doing.
%*Can gather data for future projects.






\newpage
\subsection{Results from Free Response Section}

The students provided a lot of textual feedback. From the feedback, several themes were identified. The identified themes are listed in bold face followed by the students' actual comments.

\begin{itemize}
	\item \textbf{Having an automated and unobtrusive mechanism for
	              metrics collection is helpful.}
    \begin{quote} \textit{``Overall it is an incredible tool that generally makes 
           software development metric collection effortless.''}\end{quote}


  \item \textbf{Software project telemetry offers powerful information 
                and insights into the development process.}
    \begin{quote} \textit{``There is powerful information and insights 
           to gain.''} \end{quote} 
    \begin{quote} \textit{``The first key is that analyses have made 
           me aware.''} \end{quote}
    \begin{quote} \textit{``(Telemetry analyses have helped me improve 
           my software development process) compared to what I learned 
           in class, from students, and on the web.''} \end{quote}
    \begin{quote} \textit{``(Telemetry analysis) makes me more aware of 
           what others are doing -- good or bad?''} \end{quote}
    

  \item \textbf{Understanding and interpreting telemetry data is the key to 
                get the value from the tool. This requires commitment 
                to metrics-based process improvement.}
    \begin{quote} \textit{``I would say that understanding and interpreting 
           the results to benefit the development process is the key ingredient 
           to getting value from the data. Does a team of software developers 
           understand the domain to make use of the information?''} \end{quote}
    \begin{quote} \textit{``Have not done enough projects to get a
           pattern.''} \end{quote}
              
 
  \item \textbf{Sensor installation and configuration are too complex.}
    \begin{quote} \textit{``It took some time to install it.''} \end{quote}
    \begin{quote} \textit{``I thought the instructions were much too 
           detailed. I got lost with details.''} \end{quote}
    \begin{quote} \textit{``No all-in-one installer. Too much manual 
           work.''} \end{quote}  
    \begin{quote} \textit{``I could not figure out that I need to put 
           sensor.properties to .hackystat folder.''} \end{quote}
    \begin{quote} \textit{``I still have problems with some sensors
           (at the end of the semester).''} \end{quote} 
    \begin{quote} \textit{``Please make the installation process 
           easier.''} \end{quote}
    \begin{quote} \textit{``(It) should have some mechanism to switch on 
           the sensors easily instead of going into the sensor property file 
           to change the true to false.''} \end{quote}
    \begin{quote} \textit{``(You) should really consider creating 
           installer scripts.''} \end{quote}  
    \begin{quote} \textit{``(They) need GUI driver process.''} \end{quote}           
           

  \item \textbf{Some sensors do not seem working correctly.}  
    
    This is a complex issue. There are several reasons that could cause sensors seemingly not working as expected:
    (1)programming bugs in sensor code, 
    (2)incorrect sensor configurations or project settings, 
    or (3) inappropriate interpretation of metrics data.
  
    \begin{quote} \textit{``The sensors of the Jupiter does not work 
           correctly sometimes, and did not sense any data and send 
           it to the server.''} \end{quote}
    \begin{quote} \textit{``I had a lot of problems getting Jblanket 
           to work on a web page with http unit. For some version, 
           it kept closing the web application from running on the server, 
           even with no dependencies.''} \end{quote}  
    \begin{quote} \textit{``I spend 2 hours one night to debug a problem 
           with the Ant build file and end up with only 15 minutes
           on Hackystat.''} \end{quote}


  \item \textbf{The web interface of telemetry analysis could be more user-friendly.}
    \begin{quote} \textit{``The website interface is really improvable.''} \end{quote}
    \begin{quote} \textit{``I think the Hackystat server website can be more user
           friendly. It took me a while to get used to the page and find the 
           relevant (telemetry) charts I wanted.''} \end{quote}  
    \begin{quote} \textit{``The user interface is a little bit confusing. 
           It's hard to click on Extras (link) when there is no information about 
           what Extra (link) does.''} \end{quote}
    \begin{quote} \textit{``Interface to Hackystat website 
          (for telemetry analysis) yields too many options on pages. 
           Could use a simplified design.''} \end{quote}
    \begin{quote} \textit{``I think the way a developer views a (telemetry) report 
           needs to be simplified. Of course, I can see some people would want to
           customize their own (telemetry) reports.''} \end{quote}  
    \begin{quote} \textit{``Some (telemetry analysis invocation) require 
           unknown parameters.''} \end{quote}
    \begin{quote} \textit{``Others (telemetry analysis invocations) require 
           parameters, but no instructions on what those parameters 
           might be.''} \end{quote}
    \begin{quote} \textit{``They (telemetry analysis invocations) don't work 
           so well due to the last parameter option. What goes there? 
           How about a help link for each option?''} \end{quote}  
    \begin{quote} \textit{``The (telemetry) report names aren't that descriptive, 
           and the the parameters needed were confusing.''} \end{quote}



  \item \textbf{Developers have privacy concerns with their personal process data.}
    \begin{quote} \textit{``(Software project telemetry data are) good
           if used correctly.''} \end{quote}
    \begin{quote} \textit{``(Telemetry analyses) makes me more aware of 
           what others are doing -- good or bad?''} \end{quote}
    \begin{quote} \textit{``I don't want to show the data to my 
           boss.''} \end{quote}
    \begin{quote} \textit{``Maybe (you should) consider more information 
           gathering to give programmers insights v.s. giving manager 
           insight.''} \end{quote}
  
  
  \item \textbf{Software project telemetry is better suited as a project management tool.} 
    \begin{quote} \textit{``(It is) more useful for management.''} \end{quote}  
    \begin{quote} \textit{``I want to know what people are doing.''} \end{quote}    
   
  
  \item \textbf{Miscellaneous Comments}   
    \begin{quote} \textit{``Eclipse sensor updates too often. 
           Ever time when I start Eclipse I had to download new version. 
           It was too much overhead for me. But I do not want to disable 
           automatic update, because I will forget to update if it is 
           disabled. Can you limit the sensor updates to once a week or 
           twice a month?''} \end{quote}  
    \begin{quote} \textit{``I'll be concentrating on my work. Stats don't 
           really matter.''} \end{quote}     
    \begin{quote} \textit{``(Telemetry analysis is useful) only if I was 
           in a development team. If I was alone I am not sure 
           I'd use it.''} \end{quote} 


\end{itemize}





\subsection{\textcolor{red}{Results from Telemetry Analysis Invocation Log}}
\label{EvaluationInClassroom:Results:InvocationLog}

The telemetry analysis invocation log was kept in space-delimited text files. I have just finished a program that imports the log records into a database, but it takes some time to analyze the data.

I will plot each student's invocation count on a daily or weekly bar chart. The information gives indication of the overall popularity of the telemetry mechanism. It could be used to triangulate the previous survey findings. There are three possible scenarios:

\begin{itemize}
	\item The survey results revealed that most of the students though software project telemetry was a useful approach to software project management and process improvement in general, though certain aspects of it could be improved. I will have more confidence in the survey results if the telemetry analysis invocation log indicates that most of them invoked the telemetry mechanism a lot, especially on a regular basis. 
	
	\item The survey did not reveal any overly negative comments about software project telemetry. If the log indicates that most of the students did not invoke the telemetry mechanism very much or they stopped using it after an initial period, then I will have to discount the survey results. Perhaps the inconsistency is caused by the fact that the students were concerned that negative comments would bring negative grades.

  \item The survey was anonymous, which means that there was not enough information to connect individual telemetry analysis invocation to individual survey response. If the log indicates that some of the students invoked telemetry mechanism a lot while others did not use it very much or stopped using it after a while, then I will not be able to use the data to triangulate the survey findings.
\end{itemize}

Once the analysis of invocation log is complete, I will know which scenario I am in.

%	If the students ran the analyses on a regular basis, then I would put more confidence in their survey responses. On the other hand, if it turned out that they seldom invoked the telemetry analyses, then I had to discount their opinions. 

	 %if you found that students who say they found telemetry useful also actually invoked the mechanism a lot, and students who say they didn't find telemetry useful did not invoke the mechanism very much (or stopped using it after an initial period), then you have more confidence in the opinion being "true" because it is supported by more than one source of data.



%%%%%%%%%%%%%%%%%%%%%%%%%%%%%%%%%%%%%%%%%%%%%%%%%%%%%%%%%
%                                                       %
%                   S E C T I O N                       %
%                                                       %
%%%%%%%%%%%%%%%%%%%%%%%%%%%%%%%%%%%%%%%%%%%%%%%%%%%%%%%%%

\section{Conclusion}  \label{EvaluationInClassroom:Conclusion}


The case study yielded valuable insights into software project telemetry as well as the current implementation of the system.

An automated and unobtrusive metrics collection mechanism is crucial to the success of a metrics program. From the student's feedback, sensor-based metrics collection approach appears to have eliminated long-term data collection overhead successfully. However, the one-time setup cost of the current sensors is still too high. While this is not a major issue in a setting where the use of the technology is mandatory, it could cause adoption barrier in other environment. Many survey participants have expressed the wish to have an all-in-one intelligent graphical user interface to install and configure the sensors. Fortunately, such an installer is now available.
	
Software project telemetry provides powerful insights by making the software development process transparent. Participants in this study generally agreed that they were made more aware of both their own and their team's development process as a result of using the system. An important factor to benefit from software project telemetry is the understanding and appropriate interpretation of telemetry metrics. For example, there were several reports during the semester that the sensors did not seem to collect metrics correctly, or the analyses did not seem to compute the data as expected. Some were caused by inappropriate interpretation of the results. It seemed that effort-related metrics were most susceptible to mis-interpretation. As far as the implementation is concerned, many participants voiced that the current telemetry analysis interface worked but could be more user-friendly. 		
	
There was a data privacy issue, especially with effort-related metrics data. Some students concerned that their personal process data might be misused. We were very well aware of the issue while designing the system, and had taken steps to limit the scope that the data could be accessed. However, it seemed hard to reconcile the difference between project management metrics requirements and perfect personal privacy protection. Some participants expressed that they would not want to share personal metrics with others, while other participants said they would like to know what other people were doing. As with all successful metrics programs, correct interpretation and proper use of metric data are crucial, as well as developer understanding, trust, and support.
	


%This study is conducted in the classroom setting. Though we have taken necessary precautions not to bias the results, the background and experience of the students may not be representative of professional software developers, and course projects are typically smaller and narrower in scope than industrial projects. There is always possibility that the results obtained in this study may not be generalized in ``real world settimgs''. The next 2 chapters are aimed to address this problem by evaluating software project telemetry in different contexts.









%\begin{itemize}
%	\item The sociological impact of telemetry system on developers. Are they more aware of their processes? Does the increased awareness help to improve development processes?
%	\item What are results? How are the results used to improve development processes?
%\end{itemize}
