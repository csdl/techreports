%%%%%%%%%%%%%%%%%%%%%%%%%%%%%% -*- Mode: Latex -*- %%%%%%%%%%%%%%%%%%%%%%%%%%%%
%% 04-17.tex -- Characterizing the Degree of Parallelism and Active Time 
%%              in High Performance Source Code
%% Author          : Michael Paulding
%% Created On      : Wed Sep 22
%% Last Modified By: 
%% Last Modified On: Mon Sep 27
%% RCS: $Id$
%%%%%%%%%%%%%%%%%%%%%%%%%%%%%%%%%%%%%%%%%%%%%%%%%%%%%%%%%%%%%%%%%%%%%%%%%%%%%%%
%%   Copyright (C) 2004 Michael Paulding
%%%%%%%%%%%%%%%%%%%%%%%%%%%%%%%%%%%%%%%%%%%%%%%%%%%%%%%%%%%%%%%%%%%%%%%%%%%%%%%
%% 

\documentclass[11pt,twocolumn]{article} 
\input{/export/home/csdl/tex/psfig/psfig}
\usepackage{/export/home/csdl/tex/icse2003/latex8}
\usepackage{times}
%% A verbatim-like environment which allows font changes
%%\usepackage{alltt}
%% New LaTeX2e graphics support
\usepackage[final]{graphicx}
% uncomment the % away on next line to produce the final camera-ready version
% and uncomment the \thispagestyle{empty} following \maketitle
\pagestyle{empty}

\begin{document}

\title{Characterizing the Degree of Parallelism and Active Time \\
in High Performance Source Code}


\author{\protect\begin{tabular}{ccc}
Michael Paulding \\
\end{tabular}\\
\em  Collaborative Software Development Laboratory \\
\em  Department of Information and Computer Sciences \\
\em  University of Hawai'i \\
\em  Honolulu, HI 96822 \\
\em  mpauldin@hawaii.edu}
\maketitle
\thispagestyle{empty}

\begin{abstract}  % 200 words

High performance computing systems are being applied to an
increasingly wide variety of domains and are also becoming radically
less expensive.  However, the DARPA High Productivity Computing
Systems Program notes that dramatic increases in low-level HPCS
benchmarks of execution speed do not necessarily translate into
increase development productivity.  In addition, traditional
benchmarks do not address the workflow characteristics of HPCS
development, such as active time spent developing sequential code and
that of parallel code.

This paper describes an implementation of a Purpose-Based Benchmark
(PBB) to solve the Optimal Truss Problem with the intention of gaining
further understanding of HPCS development from project inception
through completion.  With the data gathered from and analysis of the
Optimal Truss PBB implementation, a metric is proposed to measure the
ratio between degree of parallelism and active time for source code
files.

\end{abstract}

\Section{Introduction}
\label{sec:intro}

In 2002, the DARPA High Productivity Computing Systems Program began
funding an initiative to build the next generation of supercomputers,
increasing the performance and productivity of current HPCS
architectures by a factor of 10.  The project is divided into 3
phases, which include: Concept Formation (Phase 1), Risk Reduction
(Phase 2), and Full Scale Development.

DARPA intially awarded 5 vendors (Sun, IBM, Cray, SGI and HP) with
funding for Phase 1 research to conceptualize and design a prototype
of the next generation architecture.  Phase 1 was completed in 2003,
and 3 out of the 5 vendors (Sun, IBM, Cray) were selected to progress to
the next phase.  Phase 2, Risk Reduction, is slated to be completed by
2006, with at most 2 vendors selected to participate in Phase 3, in
which the architecture designed in earlier phases will be built in
full scale production.

One of the primary goals of the DARPA HPCS Program is to provide a
definition of productivity to evaluate whether vendors have achieved a
10x improvement\cite{Faulk04}.  Current benchmarking mechansims are
performance, rather than productivity based.  For example,
Activity-Based Benchmarks are used to measure system performance
during runtime, typically in floating-point operations per second
(FLOPS).  Unfortunately, this type of benchmark ignores the majority
of effort invested in the development of the software before reaching
the point at which it is ready for execution.

To address the shortcomings of traditional Activity-Based Benchmarks,
Gustafson \cite{Gustafson04} have proposed the idea of a Purpose-Based
Benchmark (PBB).  A Purpose-Based Benchmark varies from an
Activity-Based Benchmark in that it accounts for all project effort
from its inception (design phase) through to its terminal phase,
which is usually verification/validation and maintenance.  Another
characteristic of PBBs is that each one must model a problem of ``direct
interest to humans'' \cite{Gustafson04}.  For example, a PBB might be
created to accurately predict the weather over the next 3 days or to
simulate the events of an automobile collision.

In this paper, results are presented from an implementation of the
Optimal Truss PBB.  Simply stated, the Optimal Truss problem is
defined as follows:  Given two attachment points on a vertical surface
and a load-bearing point some distance from that surface, find a
pin-connected, steel truss structure that uses as little mass as
possible to bear the load with a safety factor of 50 percent.  An initial
implementation of the Optimal Truss PBB was composed over the Spring of
2004 and is used as a basis of data for results in this paper.  More
details about the Optimal Truss PBB appear in Section 4.

Concurrently with the development of the Optimal Truss PBB, I gathered
software engineering data for each day of development on the PBB in
the form of Hackystat diaries and journal entries.  Hackystat is a
tool that provides software developers with automated support for
collecting and analyzing useful measures of the process and products
of software development.  The Hackystat diaries are automatically
constructed from activities that take place within an editor
(e.g. Emacs or Vim).  Hackystat sensors for these tools collect
activity data and send it to a Hackystat server for storage and
analysis.  The journal entries I compiled during PBB development were
personal, handwritten entries reflecting on processes or products that
I felt Hackystat did not capture.

One of the primary reasons for maintaining the software engineering
data related to the Optimal Truss PBB was to gain insight into time
allocation of HPCS developers.  In order to realize productivity
gains, one must first understand where the bottlenecks exist.
Specifically, it is important to characterize active time between the
``sequential'' aspects of development and those that are considered
``parallel''.  For example, the workflow of an HPCS developer might be
classified into the general categories:

\begin{enumerate}
  \item Developing the sequential workhorse - including the translation of
algorithms and mathematics from model to source code.
  \item Parallelizing the sequential workhorse - including the task
distribution and communication handling for multiple nodes.
\end{enumerate}

One of the funtamental goals of this research is to produce a metric
that classifies the ratio between ``sequential'' and ``parallel''
active time, on a per file basis to isolate and identify bottlenecks,
which can then be addressed and corrected to increase overall
productivity in future projects.

Therefore, from the experience of developing a PBB on a full scale HPCS
architecture and the software engineering data gathered automatically
by Hackystat diaries and manually through personal journals, I propose
a two-part thesis claim that:
\\
\\
1. The software engineering of HPCS systems can be improved by
understanding the way programmers allocate their time between the
``parallel'' and ``sequential'' aspects of a software implementation.
\\
\\
2. A metric can be defined to measure the ratio between the degree of
parallelism and active time of HPCS source code.  This metric will be
defined through an analysis of developer workflow and the
incorporation of automatic tool support.

\Section{Methodology}
\label{sec:methodology}

To evaluate the thesis claims made in the previous section, I plan to
perform an analysis of the software engineering data gathered during
the implementation of the Optimal Truss PBB.  Specifically, I intend
to characterize my own workflow between ``parallel'' and
``sequential'' portions of development.  Using a combination of the
active time data from Hackystat diaries and self reported time in the
developer journals, I intend to create a time chart indicating the
time spent on each portion, on a per file basis.  For example, if the
file ``seq\_workhorse.cc'' contains no parallel constructs (in this
case, MPI calls), I will credit the sequential portion with the active
time reported in the Hackystat diaries and developer journals.

Once I have performed this analysis on each file, I will be able to
create an initial ratio of active time allocated between ``parallel''
and ``sequential'' development.  I will also be able to identify
potential productivity bottlenecks, by uncovering the files that
required a disproportionate amount of active time.

In parallel with the data analysis, I plan on building a tool to
classify whether a file of source code is ``parallel'' or not.  As an
initial pass, I will designate a ``parallel'' file as one that
contains one or more MPI constructs.  This tool can be achieved by
extending a C++ parser or tokenizer to gather the name and count of
MPI constructs appearing in the source code.

After the tool is built, I hope to integrate it with Hackystat to
couple the degree of parallelism (e.g. whether file is parallel) with
the Hackystat active time analysis.  This should produce a ratio
similar in definition to the one produced manually, but this will be
automatic with little overhead.

With the tool support in place, I hope to apply it to future HPCS
development of my own and in other student environments to
characterize workflow and suggest productivity bottlenecks.

\Section{Optimal Truss PBB Problem}
\label{sec:trusspbb}

As mentioned in Section 1, Introduction, the Optimal Truss PBB is
defined as finding a pin-connected steel truss structure that uses as
little mass as possible to support a load connected from three
attachment points on a wall to the load-bearing point away from the
wall.

Before delving into the design details of the Optimal Truss PBB
implementation, there are a few terms the reader should be familiar
with.  These include:
\\
\\
{\bf Topology} - the set of all trusses connecting the attachment points to
the load bearing point.
\\ \\
{\bf Truss} - the set of joints and members connecting a single
attachment point to the load bearing point.
\\ \\
{\bf Joint} - a cylindrical piece of metal
connecting at least two members.  In the Optimal Truss PBB, all joints
have the same dimensions, material and mass.  
\\ \\
{\bf Member} - a length of metal connecting two joints in a truss.  A member
can either be a strut or a cable.
\\ \\
{\bf Strut} - a rigid piece of metal with measurable length and width.
\\ \\
{\bf Cable} - a flexible piece of metal with measurable length and
cross-sectional area.
\\ \\
In the implementation I developed in the spring of 2003, I divided
the problem into several components.  The first was designated as the
sequential workhorse, which included the task of solving a truss once
all its elements were defined.  Solving a truss includes the
calculation of its mass, which is determined by summing the mass of each of its
components (e.g. all steel joints and members).  In addition to mass
calculation, solving a truss also includes verifying equilibrium and
deformational constraints.  Equilibrium constraints require that all
forces and moments within a truss net zero magnitude, thus ensuring
that the truss is not accelerating.  Deformational constraints require
that the length of members (strut or cable) used in the truss do not
exceed construction safety limits.  These limits are well defined and
are known prior to runtime.

\begin{figure}[h]
  \centering
  \includegraphics[width=0.2\textwidth]{simple_truss.eps}
  \caption{An unoptimized solution to the Truss PBB}
  \label{fig:simple_truss}
\end{figure}

\begin{figure}[h]
  \centering
  \includegraphics[width=0.2\textwidth]{multijoint_truss.eps}
  \caption{An optimized solution to the Truss PBB}
  \label{fig:multijoint_truss}
\end{figure}

\pagebreak

The next component of the Optimal Truss PBB was to permute all
possible topologies of trusses within the domain space, ensuring that
the configuration with minimal mass is a global minimum.  The domain
space I dealt with was a 2-dimensional mesh of points, defining the
rectangle formed between the attachment points and the load bearing
point.  Exploring all possible configurations results in a
combinatorial explosion as the mesh size is increased and this served
as the first point of parallelism in the implementation.  The task of
topology generation was equally divided among the available nodes.
Each row of the mesh was assigned to a different processor to permute.

Finally, the last component was to perform geometry assignment for all
trusses.  After generation, a topology defines the path of each truss
from the attachment points to the load bearing point, but it does not
specify what type of member connects each joint.  In this stage, a
strut and a cable is substituted for each member, flushing out all
permutations.  Once the geometry has been assigned, it can be passed
to a processor to compute the sequential workhorse and determine
whether the topology is valid under the equilibrium and deformational
constraints.

\Section{Collecting Degree of Parallelism/Active Time Metric}
\label{sec:dopatmetric}
This section will describe how the metric for degree of
parallelism/active time will be collected.  (Needs to be implemented)

\Section{Identifying Productivity Bottlenecks}
\label{prodbottle}
This section will contain examples of how using the tool can identify
files with potential productivity bottlenecks.

\Section{Related Work}
\label{sec:relwork}
This section will contain related work.

\Section{Contributions}
\label{sec:contrib}
This section will contain contributions made to the HPCS community as
a result of this research.

\bibliographystyle{/export/home/csdl/tex/icse2003/latex8}
\bibliography{/export/home/csdl/bib/hpcs}
\end{document}
 










