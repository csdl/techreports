%%%%%%%%%%%%%%%%%%%%%%%%%%%%%% -*- Mode: LaTeX -*- %%%%%%%%%%%%%%%%%%%%%%%%%%%%
%% 96-04-process.tex -- 
%% Author          : David C. Brauer
%% Created On      : Fri Mar 15 15:25:29 1996
%% Last Modified By: David C. Brauer
%% Last Modified On: Mon Mar 18 13:48:39 1996
%% RCS: $Id: 96-04-process.tex,v 1.7 1996/03/18 23:49:04 dave Exp $
%%%%%%%%%%%%%%%%%%%%%%%%%%%%%%%%%%%%%%%%%%%%%%%%%%%%%%%%%%%%%%%%%%%%%%%%%%%%%%%
%%   Copyright (C) 1996 David C. Brauer
%%%%%%%%%%%%%%%%%%%%%%%%%%%%%%%%%%%%%%%%%%%%%%%%%%%%%%%%%%%%%%%%%%%%%%%%%%%%%%%
%% 

\section{THE HI-TIME PROCESS}


With the HI-TIME Process, we were attempting to draw a picture of 
Hawaii's future with an advanced information and telecommunications 
infrastructure or "Information Superhighway" and to answer questions 
like:

\begin{itemize}
\item How will we use this new technology to enrich our lives?

\item What are the fundamental principles we should use to guide its 
development and use?

\item What role should the government play in its development and use?

\item How should the government proceed with respect to protecting 
fundamental rights and resolving critical issues?

\end{itemize}

A schematic of the HI-TIME Strategic Planning Process is shown in Figure
\ref{fig:hi-time-flow}.  The process began with the collaborative
development of the HI-TIME Vision.  The Vision was expressed as high level
goals and principles which were indended to focus subsequent activities in the
planning process.  Initial concepts for the HI-TIME Vision were developed by
drawing upon a variety of efforts already underway within the State of
Hawaii.  The Vision was then reviewed and refined by the "Process
Participants" (general public, key user groups, telecommunications
stakeholders, government decision makers, and providers of
telecommunications and information infrastructure both public and private.)
Along with the goals and principles, general metrics were developed to
ascertain progress towards meeting the goals as well as their compliance
with the guiding principles.

\begin{figure*}[htbp]
  \centerline{\psfig{figure=hi-time-flow.ps}} 
\caption{The Project HI-TIME Strategic Planning Process.}
\label{fig:hi-time-flow}
\end{figure*}


A Status assessment of the current telecommunications and information 
infrastructure was planned to be conducted to determine how well it meets the 
Vision.  A variety of mechanisms, including surveys, document reviews, 
and field research, were intended to be used along with input, review and commentary 
from Process Participants.  For the first iteration, a preliminary 
Status Assessment was derived from existing sources to be used as a starting 
point.

Process Participants were then to Identify Disparities between the HI-TIME 
Vision and the current Status.  Reviews and discussions with 
stakeholders/key users, decision makers, and providers would serve to identify 
gaps and shortfalls.

The next step was for stakeholders/key users, decision makers, and 
providers to articulate a Strategy to address the gaps and shortfalls. 
This Strategy was to be expressed as a set of measurable objectives to be 
accomplished in order to meet the goals and principles of the Vision.  
All Process Participants were to have the opportunity to review the Strategy.  
Metrics for determining the progress towards and achievement of each 
strategic objective were also to be defined.

Decision makers and providers in the public and private sector were then 
to develop a detailed Plan which recommends specific, measurable, 
agreed-upon, realistic and time-framed tasks to be accomplished in the 
following year.  The Plan would also recommend responsibility and resource 
assignments.  All Process Participants were to have the opportunity to review 
and comment on the recommended tasks.  The tasks, by their nature, would serve 
as metrics to help determine progress on the Plan.

Implementation of the plan was to be accomplished by the responsible parties 
using designated resources.  As implementation proceeded, data on 
progress and results would be fed back into the process for evaluation 
against the Vision, Strategy and Plan metrics.  On an annual basis the 
current status of the implementation of the plan and data on all metrics 
would be fed back into the initial stages of the Strategic Planning Process.  
Status assessment and Identification of Disparities the would reflect the new 
state of the telecommunications and information infrastructure and the 
Vision, Strategy and Plan would be updated accordingly.  Through the 
application of this process, all parties (the general public, 
stakeholders/key users, decision makers and providers) would become better 
informed with respect to the issues, technologies, and most promising 
applications of the telecommunications infrastructure.  Throughout this 
process, metrics arewould be used to evaluate how the process itself was working 
and to determine how it could be improved.



