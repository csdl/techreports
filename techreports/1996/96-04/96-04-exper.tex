%%%%%%%%%%%%%%%%%%%%%%%%%%%%%% -*- Mode: Latex -*- %%%%%%%%%%%%%%%%%%%%%%%%%%%%
%% 96-04-exper.tex -- 
%% Author          : Philip Johnson
%% Created On      : Sun Mar 10 07:13:19 1996
%% Last Modified By: Philip Johnson
%% Last Modified On: Mon Mar 18 11:17:57 1996
%% RCS: $Id: 96-04-exper.tex,v 1.3 1996/03/18 21:18:01 johnson Exp $
%%%%%%%%%%%%%%%%%%%%%%%%%%%%%%%%%%%%%%%%%%%%%%%%%%%%%%%%%%%%%%%%%%%%%%%%%%%%%%%
%%   Copyright (C) 1996 Philip Johnson
%%%%%%%%%%%%%%%%%%%%%%%%%%%%%%%%%%%%%%%%%%%%%%%%%%%%%%%%%%%%%%%%%%%%%%%%%%%%%%%
%% 

\section{EXPERIENCES WITH HI-TIME}

The initial HI-TIME process plan was to create a Vision document.  This
document was intended to be generic, to solicit the public's input, and to
reflect the desires of Hawaii's people for the future of Telecommunications
infrastructure.

\subsection{September 1995: Initial Release --- The HI-TIME Vision}

To seed the HI-TIME process, we created a rough outline for the
telecommunications strategic plan.  This outline consisted of four main
sections

\begin{itemize}
\item Future Scenarios --- This section was an area for the public to
  input their ideas and visions for the future of Hawaii's
  telecommunications.
\item Guiding Principles --- This section was for proposing and
  discussing the different principles that would guide the
  telecommunications plan.
\item Issues --- This section was intended for soliciting different issues
  about the future scenarios, guiding principles or any other issues that the
  public might have.
\item Background --- This section was intended to be a store house of
  information about Telecommunications.  It would hold pointers to different
  documents and information sources about telecommunications infrastructure
  at the State, Nationals and International levels.
\end{itemize}

When the initial HI-TIME vision was released we received thirteen user
registrations.  These thirteen registrations were primarily the members of
the HI-TIME planning committee.  

From the beginning, we had trouble obtaining comments.  Most of the
comments we received were about the format of the document and not about
telecommunications infrastructure.  After the initial rush of comments
about the structure of the document was complete, we redesigned the HI-TIME
document.

\subsection{October - November 1995: Building the HI-TIME Strategy}

The HI-TIME Strategy's structure was built out of the Vision's building
blocks.  The Guiding Principles were broken down into Principles,
Objectives, and Metrics.  HI-TIME's hypertext structure allowed us to
rearrange the different components into new organizations easily.  The new
document structure was as follows:
\begin{itemize}
\item Introduction --- A brief introduction to the HI-TIME strategic
  planning process.
\item HI-TIME Strategy --- A hierarchical organization of Principles,
  Objectives, Metrics and Results.  This section is the main body of the
  HI-TIME document.  When filled out it would act as the strategic plan.
\item Appendix A. Issues --- This section is identical to the previous
  version.  It was moved to an appendix.
\item Appendix B. Future Scenarios --- The future scenarios was also
  moved to an appendix since it did not directly relate to the strategic
  plan.
\end{itemize}
We received initial feed back from the HI-TIME committee saying that the
format of just asking for comments did not get the public involved enough.
The committee worried that the general public would not know what to
comment on.  We added a few tickler questions to get the reader of the
HI-TIME document to think about the issues.  These questions were supposed
to encourage the reader to provide input to the HI-TIME document. 

At the end of November we submitted the current state of the document to
the HI-TIME committee.  They did not like the lack of input and the lack of
a detailed concrete plan to implement the telecommunications strategy.

\subsection{November 1995 - March 1996: Lack of Input --- Increasing Hits}

Since the end of November the HI-TIME document has been available on the
web.  The HI-TIME site was announced to several World Wide Web indexes and
search engines in late November.  The number of hits to the HI-TIME web
pages has been increasing since November 1995 while the number of comments
and inputs have been falling.  Prior to December 1, 1995 the HI-TIME web
pages were averaging 152 accesses per month.  After December 1, 1995 the
pages are averaging 553 accesses per month.


Even with a reasonable amount of ``hits'', we are not receiving much public
input.  Since the HI-TIME system went on-line we have accepted only 22
comments and 23 extensions to the base HI-TIME document. 

In February 1996, the State Government decided to change the HI-TIME
process.  Instead of using the HI-TIME system to create the
telecommunications strategic plan from the public's input, the State would
form a committee to write the initial draft of the plan, which may
eventually be presented for public comment through the HI-TIME Web pages.


