%%%%%%%%%%%%%%%%%%%%%%%%%%%%%% -*- Mode: Latex -*- %%%%%%%%%%%%%%%%%%%%%%%%%%%%
%% 96-04-telecom.tex -- 
%% Author          : Philip Johnson
%% Created On      : Sun Mar 10 07:15:06 1996
%% Last Modified By: Philip Johnson
%% Last Modified On: Mon Mar 18 11:06:40 1996
%% RCS: $Id: 96-04-telecom.tex,v 1.5 1996/03/18 21:06:45 johnson Exp $
%%%%%%%%%%%%%%%%%%%%%%%%%%%%%%%%%%%%%%%%%%%%%%%%%%%%%%%%%%%%%%%%%%%%%%%%%%%%%%%
%%   Copyright (C) 1996 Philip Johnson
%%%%%%%%%%%%%%%%%%%%%%%%%%%%%%%%%%%%%%%%%%%%%%%%%%%%%%%%%%%%%%%%%%%%%%%%%%%%%%%
%% 

\section{TELECOMMUNICATIONS POLICY PLANNING}

The Hawaii Telecommunications Infrastructure Modernization and Expansion
(HI-TIME) project sought to establish a Strategic Planning Process for the
State of Hawaii that can serve as a model for Telecommunications and
Information Infrastructure planning throughout the nation \cite{TIIAP95}.
The primary goal was to provide the government of the State of Hawaii with
public policy principles, strategic objectives, and a plan of recommended
actions to guide decision making.  The two key features of the proposed
strategic planning process were:

\begin{itemize} 

\item Methods and mechanisms for engaging the general public in a
  meaningful dialog to develop appropriate public and social policies with
  respect to the telecommunications and information infrastructure.

\item A methodology and supporting information infrastructure for an
  on-going strategic planning process which can take into account the rapid
  changes in technology and regulation in the telecommunications field and
  which can evolve over time to effectively address new issues and
  opportunities which arise.

\end{itemize} 
 
Telecommunications policy planning has historically been the exclusive
domain of the State Public Utilities Commission (PUC), the Federal
Communications Commission (FCC), and various legislative and regulatory
bodies in conjunction with the private telecommunications carriers they
regulate.  The process of planning and implementing a telecommunications
infrastructure has typically been carried out in adversarial legal
proceedings before the PUC and FCC.  While the general public is usually
afforded some opportunity for comment on the results of a proceeding, they
seldom have any real opportunity to influence the proceeding itself.
 
Lack of widespread and informed public input to telecommunications
infrastructure planning has made it more difficult to resolve such critical
issues as \cite{NII95}:
 
\begin{itemize} 
\item the definition of universal service, 
\item how to deal with information have-nots, 
\item which government services should be freely provided via the infrastructure, 
\item equal access for rural and disadvantaged communities, and 
\item low-cost access for educational, health and community service organizations. 
\end{itemize} 
 
An expression of public will with respect to these issues is necessary to
formulate appropriate levels of public funding (including tax base support)
as well as policy to ensure telecommunications providers work towards the
desired social goals.  Informed, organized public input will help guide
decision makers towards actions which meet the true needs of their
constituents. The Public Utilities Commission, the Consumer Advocate's
Office, the State Administration and the Legislature are all intended
beneficiaries of the outputs from this process.
 
Another frequently cited problem with telecommunications planning through regulatory proceedings is the length of the process.  Proceedings seldom yield a decision from the commission (a order) within a few months and can frequently drag on for one (or several) years.  Without an on-going, evolutionary strategic planning process, decision makers cannot hope to effectively respond to rapid changes in telecommunications regulation and technology.  A one-time "Strategic Plan" report will be obsolete almost as soon as it is produced, requiring a subsequent effort to update the report.  A more cost effective approach is to recognize the need for an on-going planning process and to develop mechanisms, both automated and organizational, for supporting that process. 
 
In May 1993, responding to legislative mandate, the Public Utilities
Commission of the State of Hawaii, instituted PUC docket 7702 \cite{PUC93},
a proceeding on telecommunications infrastructure for the
State of Hawaii. In conducting these proceedings,
the PUC elected to convene a series of facilitated collaborative sessions
involving all registered intervenors and participants in the docket.  These
collaborative sessions proved to be highly fruitful in identifying and
proposing solutions for key problems in the deployment of Hawaii's
communications infrastructure.  By June of 1994, the "Communications
Infrastructure Collaborative" issued its final report to the PUC detailing
a framework for introducing competition in the Hawaii telecommunications
market and for accelerating the deployment of a "world-class"
infrastructure \cite{PUC95}.  The report was the foundation for landmark legislation in
the 1995 Hawaii legislature which moved Hawaii to the forefront in
innovative telecommunications regulatory policy \cite{HI95}. 
 
The success of the Communications Infrastructure Collaborative led the
State Administration to believe that telecommunications planning was an
area of public discourse that would greatly benefit from improved
mechanisms for collaboration.  A hierarchy of telecommunications
commmittees was formed and new initiatives were launched to further
accelerate improvements in Hawaii's infrastructure.  Project HI-TIME
(Hawaii Information and Telecommunications Infrastructure Modernization and
Expansion) was formed to increase the involvement of the general public in
the planning process and to produce an initial set of planning documents
which would stimulate on-going strategic planning in the telecommunications
domain.
 

 



