%%%%%%%%%%%%%%%%%%%%%%%%%%%%%% -*- Mode: Latex -*- %%%%%%%%%%%%%%%%%%%%%%%%%%%%
%% 96-04-lessons.tex -- 
%% Author          : Philip Johnson
%% Created On      : Sun Mar 10 07:14:24 1996
%% Last Modified By: David C. Brauer
%% Last Modified On: Mon Mar 18 13:41:06 1996
%% RCS: $Id: 96-04-lessons.tex,v 1.6 1996/03/18 23:41:24 dave Exp $
%%%%%%%%%%%%%%%%%%%%%%%%%%%%%%%%%%%%%%%%%%%%%%%%%%%%%%%%%%%%%%%%%%%%%%%%%%%%%%%
%%   Copyright (C) 1996 Philip Johnson
%%%%%%%%%%%%%%%%%%%%%%%%%%%%%%%%%%%%%%%%%%%%%%%%%%%%%%%%%%%%%%%%%%%%%%%%%%%%%%%
%% 
   
\section{LESSONS LEARNED}

The high level of collaboration experienced in docket 7702, which was the
impetus behind Project HI-TIME, never materialized as expected. Reliance on
the HI-TIME Web site as the primary mechanism for input to the strategic
planning process proved to be a serious mistake. The very low participation
of the general public on the HI-TIME Web Site was not surprising since
access to the Internet and the World Wide Web is still not widespread.
What did surprise the project team was that members of the community and
representatives of the State government directly involved in
telecommunications and information infrastructure issues, who did have
internet access, still chose to not participate.  The majority of the input
to the Web site was generated by only a few individuals (despite the fact
that Web site statistics show that HI-TIME was accessed from several
hundred unique addresses).
 
The cause of this low level of participation has been the subject of much
speculation.  We have developed two general categories of hypotheses
regarding factors which may
have contributed to the demise of the Project HI-TIME process.  The
first category covers problems regarding the execution of the process.  The second
category examines potential problems with the underlying collaborative
tools.  The next two  sections present these hypotheses in the hope that they may serve
to guide future collaborative efforts.


\subsection{Problems with the HI-TIME Process}

\begin{itemize}

\item {\bf The topic was not intrinsically motivating}

Telecommunications policy planning is a rather arcane topic, whose impact
upon the general public is not necessarily obvious. Furthermore, the
general public does not necessarily know what they can contribute to
this topic that will be of genuine use.  One goal of this project was to
enable the public to participate in the process by increasing the
visibility of the process, and lowering the barrier to entry into it.

However, the public did not make much use of the system during its brief
lifetime. We suspect that we did not put enough emphasis into educating
the public on why this topic was of importance to them. We assumed that
if they had access, they would use it.  This was not the case.  The level
of contributory participation may increase as community planning centers
are opened and initial training/review sessions are held.

We are also testing this assumption by fielding another system based upon
the CA/M architecture which has a much broader appeal.  Ke Ala Hoku:
Community Benchmarking for Hawaii, is an initiative which began with a
vision of a preferred future for Hawaii articulated by over 3000 K-12
students from across the State.  Ke Ala Hoku now seeks to translate that
vision into measurable benchmarks, which will be used to guide the
allocation of public and private resources to specific actions in pursuit
of the vision.  The benchmarks touch upon all aspects of modern society;
the environment, drug abuse and crime, education, intolerance, appropriate
application of technology, sustainability, etc.  This breadth of topics
gives us confidence that we can isolate the variable of motivation to comment.

\item {\bf Incentives were neither defined nor maintained}

When we began this project, we also assumed that the use of this tool would
be mandated, or at least supported vigorously by the State government.  
This official coercion (or at least blessing) from the political
community would help overcome any natural aversion that the 
telecommmunications industry might have toward working together. 
Unfortunately, due to the low level of "official" participation by State
Government policy personnel, telecommunications stakeholders may not
consider HI-TIME as the right vehicle to raise their concerns. There was little motivation for them
to invest the time (and reveal their strategies) through 
involvement in this system.

\item {\bf The project was not promoted}

The root cause may be simply that the HI-TIME Web site has not yet been heavily
promoted both to the on-line community and in the local media.  Informal
surveys of community members and media reveals that very few people even
know project HI-TIME exists.  Several recommendations were made to the
State of Hawaii officials in charge of this project that a high level media
event be used to kick-off the system.  In fact, two press releases were
drafted but never released for reasons unknown to the authors. Promotion of
a world wide web collaborative tool only on the world wide web is clearly
insufficient to bring an audience from the 'general public' to the site.

\item {\bf The target was not articulated clearly enough}

Because the process was oriented toward an incremental, emergent approach
to telecommunications policy plan generation, there was little need felt
to come up with a concrete, exemplary telecommunications planning
document as an example of what the system should strive to create.  The
lack of a clear "target" led to miscommunication between the State
Agency funding the project and the project team as to the ultimate goal
of the system and the process.  It may also have contributed to some degree
of confusion regarding how to interact with the web site.  Some of the
early feedback from HI-TIME participants indicated that they
didn't know how to respond.  This led to the development of tickler
questions and a more structured document, however, these modifications did
not yield any greater degree of participation.  In fact, participation on
the site was even lower after these enhancments.

We are now in the process of testing this assumption.  The State of
Hawaii has elected to have a blue ribbon panel draft the Strategic Plan
which will the be disseminated for public comment.  The HI-TIME system will
be used as the vehicle to collect public commentary.  If we experience a
much higher degree of participation, then it may be due in part to a
clearly articulated target, although political incentives and promotion of
the site will also be contributory factors.

\end{itemize}

\subsection{Problems with the HI-TIME Collaborative Tools}

\begin{itemize}

\item {\bf Collaboration was developed too independently of content}

One mistake made in this project was the creation of a division between the
"collaboration architecture" people and the "telecommunications planning"
people.  The architecture people worked relatively independently of the
telecommunications planning people on the system. The telecommunications
planning people waited until the system was running to try to introduce any
content.  The architecture people proceeded from the assumption that a
generic CSCW architecture could be instantiated for this domain, which was
not necessarily true.  We discovered soon after the launch of the Web site
that the
telecommunications planners in the State preferred a much more structured
document and more clearly defined areas for comment.  Had the
telecommunications planners been more intimately involved in the initial
design of the system, an entirely different form of collaboration may have
emerged.  To our embarassment as software engineers, we negleted the time
tested truth -- know and communicate with your users.

\item {\bf Tool support was an insufficient condition for collaboration}

When we began this project, we assumed that the existance of 
the system would precipitate extensive involvement from at least the 
major players in telecommunications.  Because the process of
planning and the artifacts were publically visible over the web, 
we then expected to draw in the general public. 

However, the telecommunications industry is intrinsically competitive, not
cooperative. Just having a tool does not instantly create a collaborative
community where one did not previously exist.  Although we had evidence
that this community could collaborate by virtue of the results of PUC 7702,
we overlooked one glaring fact.  The ``Communication Infrastructure
Collaborative'' came about because of a PUC order.  The participants and
intervenors in PUC 7702 had to participate or risk detrimental outcomes in
the formal proceedings.  They had a strong legal and economic incentive to
collaborate.  HI-TIME, on the other hand, asked them to collaborate ``for the
good of the general public''.

\item {\bf The latency between submission and publication was too high}

By placing a moderator in the loop to protect the information integrity of
the evolving project HI-TIME, we introduced a rather significant latency
time between the submission of a comment and its appearance in the on-line
document.  Most modern computer users have come to expect
immediate (or at least timely) feedback from interactive information
systems.  In HI-TIME, a user does receive immediate feedback that their
comment has been forwarded to the moderator for review.  In most cases,
this type of feedback would seem to be sufficient. However, we discovered
through our own use of the system that this latency time actually made it
difficult to execute the HI-TIME process as defined.  The process dictated
that principles were further defined by objectives which were then
amplified by metrics.  Because of the latency time, it was difficult to
propose a new principle, then follow up immediately with related objectives
and metrics.  There was no way of referring to the principle you just
submitted!  We were forced by the collaborative tool to work the process
breadth first rather than depth first.  

\item {\bf Reducing the signal to noise ratio also reduced the signal}

Early on in the design of the CSCW system, we decided for a moderator-based
system to prevent the low signal-to-noise ratio often seen on public
systems such as USENET. In addition, we wanted to make sure that we could
block any attempts to impersonate or misrepresent the positions of State of
Hawaii officials, Telecommuncations companies, and others.  The moderator
mechanism was our solution to these potential problems.  But these sorts of
problems usually only emerge in heavily used systems like UseNet.  Our
system had a very limited number of users.

In retrospect, our concerns may have been better addressed by having two
areas for feedback.  The ``unofficial'' feedback area would immediately
display user's comments and ideas, perhaps stimulating dialog with other
users.  At regular intervals, moderators would then review the ``unofficial''
area, filtering out inappropriate feedback and condensing the information,
and place the filtered information in the ``official'' feedback area.


\end{itemize}

\section{SUMMARY}

Unfortunately, with respect to the Project HI-TIME collaborative planning
process, we will never have the opportunity to determine which of our
hypotheses actually led to its demise. The State of Hawaii
Telecommunications Strategic Plan must be completed in a timely fashion.
It is our hope that these experiences will support the design of better
public collaborative processes and tools in the future.

