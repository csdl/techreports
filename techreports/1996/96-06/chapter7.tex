\chapter {Experimental Results}
\label{ch:results}
This chapter presents the results of the experiment comparing the
review performance of real group reviews using EGSM and nominal
group reviews using EIAM.  

Section \ref{sec:summary} describes the summary of the experimental
procedures. Section \ref{sec:overview}
provides an overview of the experimental results obtained from
analyzing the quantitative data.
The following sections detail the data analysis. It starts
with the analysis of main hypothesis (Section 
\ref{sec:main-hypothesis}) comparing the detection effectiveness and
the detection cost of the two review methods EGSM and EIAM. This is
followed by the analysis of other research questions (Section
\ref{sec:other-research-question}), and the qualitative analysis of
questionnaires (section \ref{sec:questionnaires}). 
It concludes with a discussion of the
findings, related studies, and the lessons learned (section
\ref{sec:discussion}). 


\section {Summary of the experiments}
\label{sec:summary}

The experiment was conducted in the Spring of 1995 with subjects
recruited from ICS-313 (27 students) and ICS-411 (45 students), as
described in the previous chapter.
The ICS-411 experiment occurred after the ICS-313 experiment.

Both experiments were started approximately 3-4 weeks after the students
completed their assignments.
Each group completed one training and two actual review
sessions of about two hours each. 

As discussed earlier, each group reviewed two sets of source code,
once using EGSM and once using EIAM. The source code was broken down
into a set of source nodes, each corresponding to a function, class,
or type declaration. Only source nodes of type ``Function'' were
seeded with errors. 

The groups conducted the review sessions
in Keller 204 or Keller 304 at a pre-scheduled time. They
completed all three sessions on three separate days.

To optimize the use of workstations, two or
more EIAM groups were allowed to hold their reviews at the same time
and  place. For EGSM sessions, only one group worked at a time.

The source code and review methods were assigned
randomly to the groups. However, in the ICS-313 experiment, all groups reviewed
source-1 first (using EGSM or EIAM) followed by source-2 (using EIAM
or EGSM). In the ICS-411 experiment, the order of the source code was
also randomly assigned, so that the number of groups who reviewed source-1
first was about the same as the number of groups who reviewed
source-2 first.
Due to a scheduling problem, the number of groups who
used EGSM followed by EIAM, or vice versa, was slightly unequal.
Tables \ref{group-assignments-313} and \ref{group-assignments-411}
show the source code and group  
assignments for ICS-313 and ICS-411 respectively. The superscripts $^1$
or $^2$ indicate the session number.


\begin{table}[htb]
  \caption{ICS-313 Source Code and Group Assignments}
  \begin{center}
    \begin{tabular} {|l|l|l|}
      \hline
      & {\bf Employee1} & {\bf Employee2}\\
      \hline
      & & \\
      {\bf EGSM} & G1$^1$,G6$^1$,G8$^1$,G9$^1$ & G2$^2$,G3$^2$,G4$^2$,G5$^2$,G7$^2$ \\
      & & \\
      \hline
      & & \\
      {\bf EIAM} & G2$^1$,G3$^1$,G4$^1$,G5$^1$,G7$^1$ & G1$^2$,G6$^2$,G8$^2$,G9$^2$ \\
      & &  \\
      \hline
     \end{tabular}
  \end{center}
  \label{group-assignments-313}
\end{table}


\begin{table}[htb]
  \caption{ICS-411 Source Code and Group Assignments}
  \begin{center}
    \begin{tabular} {|l|l|l|}
      \hline
      & {\bf Pass1} & {\bf Pass2}\\
      \hline
      & & \\
      {\bf EGSM} & G3$^2$,G4$^2$,G9$^2$,G10$^2$, & G1$^2$,G2$^2$,G5$^2$,G6$^1$, \\
                 & G11$^1$,G12$^2$,G13$^1$ & G7$^2$,G8$^1$,G14$^1$,G15$^2$\\
      & & \\
      \hline
      & & \\
      {\bf EIAM} & G1$^1$,G2$^1$,G5$^1$,G6$^2$, & G3$^1$,G4$^1$,G9$^1$,G10$^1$, \\
                 & G7$^1$,G8$^2$,G14$^2$,G15$^1$ & G11$^2$,G12$^1$,G13$^2$\\
      & &  \\
      \hline
     \end{tabular}
  \end{center}
  \label{group-assignments-411}
\end{table}


\section{Summary of the experimental results}
\label{sec:overview}

This section summarizes the experimental results
concerning the performance of EGSM versus EIAM. In most cases,
the results were analyzed for three subject groups: ICS-313, ICS-411
and All groups (ICS-313 and ICS-411 combined). For this reason, most
of the analysis was based on percentage numbers instead of 
absolute numbers.

Table \ref{summary} shows the summary of the findings comparing the
performance of EGSM and EIAM for the specified review metrics. A
`-' indicates there were no significant differences between
EGSM and EIAM, `$>$' indicates EGSM performance was
significantly higher (p$<$.05) than EIAM for the corresponding review
metric, and `$<$' indicates EGSM performance was significantly lower
(p$<$.05) than EIAM.


As shown in the table, with respect to overall
detection effectiveness, the analysis found no significant differences
between EGSM and EIAM in all three cases.  
However, for the specific error type C4 
(mistyped statements), EIAM significantly outperformed EGSM in
detection effectiveness.

With respect to detection cost, the analysis found no significant
differences between EGSM and EIAM for ICS-411, but for ICS-313 and All 
groups, EGSM had a higher detection cost than EIAM.
For detection effort, EGSM required significantly more effort than EIAM
for ICS-313 and All groups, but was insignificant for ICS-411. For
detection rate (i.e., number of errors found per hour), EGSM was
significantly lower than EIAM for ICS-313 and All groups, but was
insignificant for ICS-411.
In terms of paraphrasing rate (i.e., number of lines inspected per
hour), there were no significant differences found.

When the group performance in EGSM was compared to the average single
individual performance in EIAM, EGSM outperformed EIAM.
On the other hand, when the group performance in EGSM was compared to the
total individual performance in EIAM regardless of error duplication
(i.e., the total number of issue nodes created), EIAM outperformed EGSM. One
conclusion from these two results is that 
error duplication in EIAM could potentially make a significant
difference in overall review performance.

In terms of false positives, EGSM generated significantly
fewer false positives than EIAM.

The experiment also found the presence of group synergy in EGSM.
However, this synergy was not large enough to
create a significant difference between the two methods.
Furthermore, the group interaction in EGSM
did not boost the performance of individual members; in other words,
the smartest participants still catch the most errors, while the
slowest ones catch the fewest regardless of whether they work in
a group (EGSM) or individually (EIAM).


Finally, the analysis of the questionnaires indicates that
the participants overwhelmingly preferred EGSM over EIAM. This
includes preferences in the review process, the review system, and
the perceived quality of the review outcomes.

Subsequent sections will look at the detail of this data
analysis. The raw data of the experiment can be found in
\cite{csdl-95-13}.  


\begin{table}[htb]
 \caption{Summary of the experimental results (EGSM vs. EIAM)}
  \begin{center}
  \begin{tabular}{|l|l|c|c|c|}
   \hline
No &  Review metric         & ICS-313 & ICS-411 & All \\
   \hline
1 &   Detection effectiveness & -       & -       & - \\
2 &   Detection eff. of error type C4   & n/a     & $<$     & n/a \\   
3 &   Detection cost          & $>$     & -       & $>$ \\  
4 &   Detection effort        & $>$     & -       & $>$ \\
5 &   Detection rate          & $<$     & -       & $<$ \\
6 &   Paraphrasing rate       &  -      & -       & - \\
7 &  Detection effectiveness &       &        &  \\
   &  (group vs single individual) & $>$   & $>$     & $>$ \\
8 & Total issue nodes created &  $<$    & $<$     & $<$ \\
9 &   False positives         & $<$     & $<$     & $<$ \\
  \hline
   \end{tabular}
  \end{center}
 \label{summary}
\end{table}


\section {Analysis of main hypotheses}
\label{sec:main-hypothesis}

This section describes the analysis of the main hypotheses, that is,
whether there are significant differences in detection effectiveness
(Hypothesis H1) and detection cost (Hypothesis H2) between  real groups
(EGSM) and nominal groups (EIAM).  Figure \ref{main-hypothesis} shows
a summary of the experimental data related to the main hypothesis for
ICS-313 and ICS-411. The
figure shows the total number of correct/valid errors found by a
group, the percentage of these errors from the total errors seeded
(``Percent Err''),
the review time (``Time''), the detection cost (``Cost''),
and the detection rate (``Rate'').  The
review time for the EIAM method is the sum of the group
members' review time, while the review time for the EGSM method is the
presenter's time multiplied by the number of participants (since the
participants were synchronized with the presenter). The detection cost
is the amount of review time (in hours) to detect an error. The
detection rate is 
the number of errors found per hour.

\begin{figure}[htb]
 {\centerline{\psfig{figure=/group/csdl/techreports/95-08/figures/main-hypothesis.epsi}}}
 \caption{The main hypothesis data}
 \label{main-hypothesis}
\end{figure}


\subsection{Detection effectiveness} 
\label{sec:detection-effectiveness}

Detection effectiveness measures the percentage of valid errors
detected by the participants. Figure \ref{main-hypothesis}, Column
``Percent Err'', shows the resulting data, and Table
\ref{detection-effectiveness-analysis} shows the  
Wilcoxon paired-test analysis of this data.
As shown in the table, the P-Value for ICS-313, ICS-411 and All groups
are 0.3743, 0.3066 and 0.1374 respectively.
In all of these cases, the P-Values are greater than
$\alpha$ (= 0.05). Therefore,  the null hypothesis cannot be rejected, that
is, there was no evidence that the two methods (EGSM and EIAM) were
significantly different in their detection effectiveness. 
In other words, these results do not support the predicted outcome that
EGSM would outperform EIAM in detection effectiveness. 


\begin{table}[htb]
 \caption{Detection effectiveness analysis}
 \begin{center}
 \begin{tabular}{|l|r|r|r|}
  \hline
  \multicolumn{4}{|c|}{\bf Wilcoxon Signed Rank Test for Percent(EGSM), Percent(EIAM)}\\
  \hline
  & {\bf ICS-313} & {\bf ICS-411} & {\bf All} \\ 
  \hline
  \# 0 Differences & 0     & 0      &  0 \\
  \# Ties          & 0     & 1      &  1 \\
  Z-Value          & -.889 & -1.022 & -1.486\\
  P-Value          & .3743 & .3066  & .1374\\
  Tied Z-Value     & -.889 & -1.023 & -1.486\\
  Tied P-Value     & .3743 & .3065  & .1373\\
  \# Ranks $<$ 0   & 6     & 8      & 14\\
  \# Ranks $>$ 0   & 3     & 7      & 10\\
  \hline
   \end{tabular}
  \end{center}
  \label{detection-effectiveness-analysis}
\end{table} 

\subsection{Detection cost}
Detection cost measures the amount of effort or review time to detect
a valid error. 
Figure \ref{main-hypothesis}, column ``Cost'', shows the resulting
data, and Table \ref{detection-cost-analysis} shows 
the Wilcoxon analysis of this data. 
The analysis shows that the P-value for ICS-313 is around 0.05,
for All groups is $<$0.05, and for ICS-411 is $>$0.05.
Thus, EGSM was significantly more costly than EIAM for ICS-313 and All
groups at the significant level of $\alpha<=$0.05. 

With respect to detection effort (review time), Table
\ref{detection-effort-analysis} shows the Wilcoxon analysis for
the variable ``Time'' of Figure \ref{main-hypothesis}. The analysis shows
that the detection effort of EGSM was significantly higher than EIAM
(P-value $<$0.05) for both ICS-313 and All groups. For ICS-411, it was
insignificant (P-value $>$0.05).

\begin{table}[htb]
 \caption{Detection cost analysis}
 \begin{center}
 \begin{tabular}{|l|r|r|r|}
  \hline
  \multicolumn{4}{|c|}{\bf Wilcoxon Signed Rank Test for Cost(EGSM),
Cost(EIAM)}\\
  \hline
  & {\bf ICS-313} & {\bf ICS-411} & {\bf All} \\
  \hline
  \# 0 Differences & 0 & 0 & 0 \\
  \# Ties          & 0 & 0 & 1 \\
  Z-Value          & -1.955 & -1.533 & -2.429\\
  P-Value          & .0506 & .1252 & .0152\\
  Tied Z-Value     & -1.955 & -1.533 & -2.429\\
  Tied P-Value     & .0506 & .1252 & .0152\\
  \# Ranks $<$ 0   & 1     & 5 & 6\\
  \# Ranks $>$ 0   & 8     & 10 & 18\\
  \hline
     \end{tabular}
  \end{center}
  \label{detection-cost-analysis} \end{table}


\begin{table}[htb]
 \caption{Detection effort analysis}
 \begin{center}
 \begin{tabular}{|l|r|r|r|}
  \hline
  \multicolumn{4}{|c|}{\bf Wilcoxon Signed Rank Test for Time(EGSM),
Time(EIAM)}\\
  \hline
  & {\bf ICS-313} & {\bf ICS-411} & {\bf All} \\
  \hline
  \# 0 Differences & 0 & 0 & 0 \\
  \# Ties          & 0 & 0 & 0 \\
  Z-Value         & -2.666 & -1.079 & -2.714\\
  P-Value         & .0077 & .2805 & .0066\\
  Tied Z-Value    & -2.666 & -1.079 & -2.714\\
  Tied P-Value    & .0077 & .2805 & .0066\\
  \# Ranks $<$ 0  & 0 & 5 & 5\\
  \# Ranks $>$ 0  & 9 & 10 & 19\\
  \hline
     \end{tabular}
  \end{center}
  \label{detection-effort-analysis} \end{table}

In general, these results support the main hypothesis
that EGSM had a higher detection cost than EIAM, and that this cost
might be attributed primarily to the higher detection effort required
in EGSM as indicated by Table \ref{detection-effort-analysis}.  With
respect to 
detection effort, it appears that higher detection effort in EGSM was
due to the extra effort to discuss the findings among group members. 
The nature of group discussion in
 EGSM was typically to reject or accept issues raised by any group member. In
EIAM, there were no such discussions, however, individual participants also
spent some time to judge the validity of issues, namely, to consider
themselves whether to accept or reject the issues. One may
argue that the EGSM group should be 
able to make this decision faster than the individuals in EIAM because
the group members would help each other out in deciding the validity
of errors. 
However, my observation during the experiment indicated that in many
cases, EGSM
groups needed to spend extra time explaining the logic of the code
repeatedly to some group members, or 
arguing the validity of an issue excessively with other group members.
The participants also indicated in the questionnaires that it took
time  to sort out disagreements.

\section {Analysis of other research questions}
\label{sec:other-research-question}
This section presents the analysis of other research questions
concerning the performance of EGSM and EIAM as described in Chapter
\ref{ch:exp-design} 

\subsection{Detection effectiveness for specific error types}
\label{sec:error-type}
This subsection analyzes whether  there were any differences in
detection effectiveness between EGSM and EIAM with respect
to specific error types. The source1 and source2 programs in the ICS-411
experiment were designed to explore this question. 
Both source programs were seeded with the
same number and type of errors, as well as the same relative
location of the errors.

Figure \ref{errorno} shows the specific error numbers caught
by ICS-411 
groups using EGSM (denoted by label ``G'')  and EIAM (denoted by label
``I''). For EIAM, the mark indicates that there was at least one
member of the group who successfully identified the error.

Figure \ref{errorclass} shows the classification of these errors according
to the following error type or class (see also Section
\ref{sec:source-seeded-errors}):  
\begin{table}[htb]
 \caption{Error Class}
  \begin{center}
  \begin{tabular}{|c|l|}
   \hline
    Class  & Description \\
   \hline
   C1 & Missing initialization \\
   C2 & Incorrect operators \\
   C3 & Incorrect condition \\
   C4 & Mistyped statements \\
   C5 & Incorrect algorithm \\
   C6 & Incorrect computation \\
  \hline
   \end{tabular}
  \end{center}
 \label{tb:error-class}
\end{table}


C1 consists of error E3, E11 and E19; C2
consists of error E5, E15, and E18, and so forth (see Figure
\ref{errorclass}). 

Table \ref{errorclass-analysis} shows the resulting Wilcoxon 
analysis for these error types.
As shown in the analysis, only C4 exhibited a P-value $<$0.05 (=
0.0431) with EGSM $<$ EIAM, the rest the error classes exhibited P-values
($>$0.05). In conclusion, EIAM was significantly more effective in
detecting errors of type C4 than EGSM. 


\begin{figure}[htb]
 {\centerline{\psfig{figure=/group/csdl/techreports/95-08/figures/ErrorNo.epsi}}}
 \caption{Specific error no caught by EGSM and EIAM groups}
 \label{errorno}
\end{figure}


\begin{figure}[htb]
 {\centerline{\psfig{figure=/group/csdl/techreports/95-08/figures/ErrorClass.epsi}}}
 \caption{Number of errors by type/class} 
 \label{errorclass}
\end{figure}

\begin{table}[htb]
 \caption{Error class analysis }
 \begin{center}
 \begin{tabular}{|l|r|r|r|r|r|r|}
  \hline
  \multicolumn{7}{|c|}{\bf Wilcoxon Signed Rank Test for EGSM, EIAM}\\
  \hline
     & {\bf C1} & {\bf C2} & {\bf C3} & {\bf C4} & {\bf C5} & {\bf C6}\\
  \hline
  \# 0 Differences & 6     & 9    & 6     & 10    & 8     & 6\\
  \# Ties          & 2     & 1     & 2     & 1     & 1     & 1\\ 
  Z-Value          & -.889 & -.629 & -.711 & -2.023 & -1.014 & -1.066\\
  P-Value          & .3743 & .5294 & .4772 & .0431 & .3105 & .2863\\
  Tied Z-Value     & -.921 & -.647 & -.724 & -2.121 & -1.134 & -1.155\\
  Tied P-Value     & .3573 & .5176 & .4693 & .0339 & .2568 & .2482\\
  \# Ranks $<$ 0   & 6     & 3     & 4     & 5     & 2     & 3\\
  \# Ranks $>$ 0   & 3     & 3     & 5     & 0     & 5     & 6\\
  \hline
  \end{tabular}
  \end{center}
  \label{errorclass-analysis}
\end{table} 


Error type C4 is a mistyped statement, which consists of  error 7 (E7) and 
error 17 (E17). As shown in Figure \ref{errorno}, there are
13 EIAM groups compared to only 9 EGSM groups who successfully identified
E17. For E7, the differences are insignificant (14 EGSM groups versus
15 EIAM groups). 

One example of E17 is incorrect typing of keyword ``STORE'' in the
following code fragment:
\small
\begin{verbatim}
     Access_Symtab(SEARCH, &symtabret, source.labl, &address);
     if (symtabret == NOTFOUND) {
       address = locctr;
       Access_Symtab(SEARCH, &symtabret, source.labl, &address);
       /*SEARCH should have been STORE */
\end{verbatim}
\normalsize
This code is supposed to add a label (``source.labl'') into a
symbol table. The first 
call to ``Access\_Symtab'' searches the label in the symbol table to ensure
that the label is not already there, then it calls
``Access\_Symtab'' for the second time to store the label.
Instead, the second call issues ``SEARCH'' for the symbol for the
second time.  

My observation during the EGSM session indicated that
this error was often missed because of a bad paraphrasing by the
presenter. The presenter simply read the statement aloud without
questioning the logic of the code, and the reviewers often
went along with the presenter. %%perhaps the reviewers didn't have
                               %%enough time or influenced by the presenter

One example of error E7 is missing ``else'' keyword in the following
if-then-else code fragment:
\small
\begin{verbatim}
     if (ch == 'T')
        source->comline = true
     source->comline = false;  /*missing keyword ``else'' */
\end{verbatim}
\normalsize

My observation indicated that the group did not catch this
error because the presenter unconsciously added the word ``else'' when
paraphrasing the statement, and again, the reviewers went along
with the presenter.  

Although the rest of the error types were found to be statistically
insignificant, informal analysis  on the specific error
number in Figure \ref{errorno} led to the following conclusions:

\begin{itemize}

\item EIAM seemed  more effective in catching E9 than EGSM (with
3 groups differences). Error E9 is incorrect use of logical operator
$||$ or \&\&. 
One example of this error is as follows:
\small
\begin{verbatim}
   /*discard remaining characters till newline
     (implementation of readln in PASCAL) */ 
    while (ch!=EOF || ch != '\n') /*|| should have been && */
        ch=getc(srcfile);
\end{verbatim}
\normalsize
This code  discards any character read from the file ``srcfile'' while
the character is neither EOF nor a newline.
My observation indicated that the presenter did pronounce the word
``OR'' ($||$) when paraphrasing the code, instead of 
``AND'' (the correct logic). It appears that when one paraphrases the
above statement orally 
using ``OR'', the statement sounds correct. In other
words, the use of ``OR'' or ``AND'' does not seem to make
any difference orally.
EIAM groups made fewer mistakes of this type perhaps
because the logic ``OR'' was not evaluated orally.


\item EGSM seemed more effective in catching error E4 than EIAM (with 3
groups differences). Error E4 is incorrect use of algorithm, as in 
the following code:
\small
\begin{verbatim}
   do{
     mid = (low + high) / 2;
     if (strncmp(mnemonic,OPTAB[mid].mnemonic,sizeof(CHAR6))< 0)
        high = mid + 1; /*should have been mid - 1 */
     else
        low = mid - 1; /*should have been mid + 1 */
   }while (strncmp(mnemonic, OPTAB[mid].mnemonic, sizeof(CHAR6))&&
        high >= low);
\end{verbatim}
\normalsize

This code uses incorrect binary search algorithm. 
It appears that working in a group (EGSM) to inspect a complex
algorithm is more effective than working alone (EIAM).
My observation indicated
that the group seriously worked together to inspect the algorithm. One
member explained how the algorithm worked, others confirmed or
rejected the explanation. This was repeated until all members agreed
with the explanation. 

\item Both EGSM and EIAM had difficulty catching error E10 and E19. 

Error E10 is an error of type C6, or deviation from the
specification. This error is strongly related
to the domain knowledge. One example is missing conditional statements
that test for ``comment'' lines in  Pass1:
\begin{verbatim}
   /*initializes operand and comment with blanks*/
   memcpy(source->operand, BLANK18, sizeof(CHAR18)); 
   memcpy(source->comment, BLANK31, sizeof(CHAR31));

   /*** should have checked for comment line here ***/ 

   /*process each line*/
   if (source->line[0] >= 'A' && source->line[0] <= 'Z' ){...
\end{verbatim}
This code processes the input text line by line, except 
comment lines. However, there are no  conditional
statements that test for the comment lines.
Although this was clearly stated in the
specification, most participants failed to catch the error.

%%May help with active checklist

Error E19 is incorrect re-initialization of a global variable (error
type C1). One example is as follows:
\small
\begin{verbatim}
  void P2_Assemble_Inst (...)
  ...
  if(FIRSTSTMT && !strncmp(source.operation,"START ",sizeof(CHAR6))){
   /*the first source statement (except for comments) must be start*/
    *errorsfound = true;
    errorflags[15] = true; /*missing or misplaced start statement*/
  }

  /* missing reinitialization of FIRSTSTMT to false here*/
  ...
\end{verbatim}
\normalsize
The global variable FIRSTSTMT is initialized in the main
function to a boolean value ``true''. This variable is used as a flag 
to indicate whether the first statement has been processed by the
assembler or not. It needs to be reset to ``false'' in this function.
There were only
one EGSM group and two EIAM groups who successfully caught this error.

On the other hand, missing initialization of a variable within a
function (E3 and E11) was caught easily by all participants (both EGSM
and EIAM). 

\item Overall, all seeded errors were caught in this experiment. EGSM
groups caught all errors, while EIAM groups missed one error only
(E10). 

Similarly, in ICS-313 experiments, there were only 2 out of 23 errors in
source1 and 2 out of 25 errors in source2 that were not caught by any
group. These errors, however, were not well understood by the
students, since no one submitted the final assignment without making these
errors. 

Appendix \ref{app:error-313} and \ref{app:error-411} contains the
complete list of the seeded errors 
used in the experiment for both ICS-313 and ICS-411.

\end{itemize}


\subsection {Detection rate}

Detection rate measures the number of errors detected per unit time
(hour). Figure \ref{main-hypothesis} (column ``Rate'') shows the 
detection rate  for EGSM and EIAM groups. Actually, this rate was
obtained from the number of errors caught divided by the
total review time, which included both the time to detect
and record the errors, although in this experiment the recording time was
relatively insignificant compared to the detection time.

In practice, detection rate is often used as a measure for the
effectiveness of defect removal techniques, including testing. It
refers to the number of defects found per man-hour invested. 
However, the term ``detection rate'' is not uniformly used.
Russell referred to this term as ``detection efficiency''
\cite{Russell91}.  Weller used ``detection rate'' to refer to the
number of defects per KLOC \cite{Weller93}. Porter and Votta  used  
``detection rate'' to refer to the percentage of defects found
per total defects in the source \cite{Porter94}. %%Also Schneider
Only Selby used ``detection rate'' as number of faults detected per
hour similar to the one used in this study \cite{Selby85}.

Table \ref{detection-rate-analysis} shows the resulting Wilcoxon
analysis for ICS-313, ICS-411, and All groups.
The analysis found that the detection rate of EIAM was
significantly higher than EGSM for both ICS-313
and All groups, although for ICS-411, the data was insignificant.

\begin{table}[htb]
 \caption{Detection rate analysis}
 \begin{center}
 \begin{tabular}{|l|r|r|r|}
  \hline
  \multicolumn{4}{|c|}{\bf Wilcoxon Signed Rank Test for Rate(EGSM), Rate(EIAM)}\\
  \hline
  & {\bf ICS-313} & {\bf ICS-411} & {\bf All} \\ 
  \hline
  \# 0 Differences & 0      & 0 & 0\\
  \# Ties          & 0      & 0 & 0\\
  Z-Value          & -2.192 & -1.306 & -2.429\\
  P-Value          & .0284  & .1914 & .0152\\
  Tied Z-Value     & -2.192 & -1.306 & -2.429\\
  Tied P-Value     & .0284  & .1914 & .0152\\
  \# Ranks $<$ 0   & 8      & 10 & 18\\
  \# Ranks $>$ 0   & 1      & 5 & 6\\
  \hline
     \end{tabular}
  \end{center}
  \label{detection-rate-analysis}
\end{table} 



\subsection {Paraphrasing rate}
\label{sec:paraphrasing-rate}
Paraphrasing rate measures the amount of materials reviewed per
unit time. For code review, it refers to the number of lines of code
reviewed per hour.

In EGSM, paraphrasing involved reading the code aloud statement by
statement by the 
presenter.  In EIAM, paraphrasing was performed by individual participants
silently. For the latter, however, I had no way of knowing
whether the participants read every single statement as instructed
in the review guideline. 

Several studies have found that paraphrasing rate is 
a good predictor of review effectiveness
\cite{Russell91,Humphrey90,Christenson92}. In fact, 
Russell suggested that the effective paraphrasing rate should be
around 150 lines/hour.
This section compares the paraphrasing rate of EGSM and EIAM.

In the experiment, the source materials were broken down
into a set source nodes, each corresponded to a function/procedure
definition, a class definition, or a type/constant declaration.
Only source nodes of type ``Function'' were seeded with
errors. Others were included in the review materials for references
only.  
In the paraphrasing rate analysis, the latter were not included.
Similarly, very small functions
(less than 9 lines of code) were also excluded from the analysis.

Tables \ref{source-nodes-313} and \ref{source-nodes-411}
show the source nodes, type, and the node size (\# lines without the
specification) for ICS-313 and ICS-411. 
The number ($s_{ij}$) refers to the source node that was selected for
this analysis. Figure \ref{paraphrasing-313} and Figure
\ref{paraphrasing-411} show the resulting paraphrasing data
for these nodes.
This data was obtained by dividing the source node size with the
amount time spent in reading the node. The latter
was collected by the system automatically.
In EIAM, the data refers to the average paraphrasing rate
of the group members for the specified source node.
The last column ``Visit'' in Figure \ref{paraphrasing-313} and Figure
\ref{paraphrasing-411} indicates the average number of source node
visitations. For example, on average, the source nodes were
visited 1.9 times by group 1 using EGSM (see Figure
\ref{paraphrasing-313}).

The results show that in general, the average paraphrasing rates for
both EGSM and EIAM were 
slightly higher than the recommended rate of 150 lines/hour by Russell.
Table \ref{paraphrasing-rate-analysis} shows the resulting Wilcoxon
analysis for this data for EGSM and EIAM. As shown in the table, in all
three cases the P-values were  
greater than 0.05, or there were no significant differences in
paraphrasing rate between EGSM and EIAM. 

With respect to the number of visitations, it appears that the source
nodes in EIAM were visited more often than the ones in EGSM
(for example, 1.8 for ICS-313 EGSM and 2.9 for ICS-313 EIAM). 
Since there were no differences in paraphrasing rate between EGSM and
EIAM, this might indicate that the source nodes in EIAM were
re-inspected more often but less amount of time compared to the nodes
in EGSM. 
In any case, this did not seem to affect the number of errors caught.


\footnotesize
\begin{table}[htb]
  \caption {ICS-313 Source nodes}
  \begin{center}
   \begin{tabular} {|l|l|l|r||l|l|l|r|}
   \hline
   \multicolumn{4}{|c||} {Source1 (Employee1)} &
   \multicolumn{4}{c|} {Source2 (Employee2)} \\
 \hline    
  No & Node & Type & Size & No & Node & Type & Size \\
   &      &      & (loc)&    &      &      & (loc) \\
 \hline
s11 &  Company1::Company1 & Func & 10   & s21 &  Emp::$\sim$Emp & Func & 9\\        
s12 &  Emp::$\sim$Emp & Func & 9             & s22 &  Emp::setName & Func & 11\\     
s13 &  Emp::setName & Func & 19         & s23 &  Emp::setSocSecurity & Func & 27\\
s14 &  Company1::addEmp & Func & 39     & s24 &  Company2::$\sim$Company2 & Func & 13\\
s15 &  Emp::setSocSecurity & Func & 33  & s25 &  Emp::setAge & Func & 9\\       
s16 &  Company1::findEmp & Func & 10    & s26 &  Company2::findEmp & Func & 21\\
s17 &  Emp::setAge & Func & 11          & s27 &  Emp::setNumDep & Func & 10\\
s18 &  Company1::deleteEmp & Func & 20  & s28 &  Company2::addEmp & Func & 32\\ 
s19 &  Emp::setNumDep & Func & 11 & s29 &  Emp::print & Func & 9\\         
s110 &  Company1::print & Func & 12     & s210 &  Company2::deleteEmp & Func & 23\\ 
 &   Emp & Class & 19                   & s211 &  Company2::print & Func & 18\\  
 &   Company1 & Class & 14             &  &   Emp & Class & 19\\               
 &   Emp::Emp & Func & 8               &   &   EmpNode & Class & 12\\           
 &   Emp::print & Func & 8             &  &   Company2 & Class & 14\\           
 &   Emp::getSocSecurity & Func & 5    &  &   Emp::Emp & Func & 8\\              
 &   Company1::$\sim$Company1 & Func & 6    &  &   Emp::getSocSecurity & Func & 5\\   
 &   Constant & Cons & 2           &  &   EmpNode::EmpNode & Func & 6\\     

 &         &           &             &  &   EmpNode::$\sim$EmpNode & Func & 5\\  
 &        &            &            &  &   Company2::$\sim$Company2 & Func & 7\\
\hline
 &       & Total       & 236        &  &    & Total & 258 \\
\hline
     \end{tabular}
  \end{center}
  \label{source-nodes-313}
\end{table}

\footnotesize
\begin{table}[htb]
  \caption {ICS-411 Source nodes}
  \begin{center}
   \begin{tabular} {|l|l|l|r||l|l|l|r|}
   \hline
   \multicolumn{4}{|c||} {Source1 (Pass1)} &
   \multicolumn{4}{c|} {Source2 (Pass2)} \\
 \hline    
  No & Node & Type & Size & No & Node & Type & Size \\
   &      &      & (loc)&    &      &      & (loc) \\
 \hline
s11 &   hextonum & Func & 27 &      s21 &   dectonum & Func & 21\\                
s12 &   Access\_Symtab & Func & 58 & s22 &   Read\_Int\_File & Func & 59\\            
s13 &   Write\_Int\_File & Func & 36 &s23 &   P2\_Search\_Optab & Func & 28\\          
s14 &   P1\_Read\_Source & Func & 76 &s24 &   P2\_Proc\_START & Func & 19\\            
s15 &   P1\_Proc\_START & Func & 30 & s25 &   P2\_Proc\_BYTE & Func & 50\\            
s16 &   P1\_Proc\_RESW & Func & 31 &  s26 &   P2\_Assemble\_Inst & Func & 42\\         
s17 &   P1\_Assign\_Loc & Func & 29 & s27 &   P2\_Write\_Obj & Func & 72\\              
s18 &   P1\_Assign\_Sym & Func & 27 & s28 &   Pass\_2 & Func & 38\\                   
s19 &   Pass\_1 & Func & 36 &         & Type and var decl & Decl & 135\\
     & Type and var decl & Decl & 101 &                 &   &  & \\
 \hline
                & & Total & 451 &      &               & Total & 464 \\
\hline
     \end{tabular}
  \end{center}
  \label{source-nodes-411}
\end{table}
\normalsize


\begin{figure}[htb]
 {\centerline{\psfig{figure=/group/csdl/techreports/95-08/figures/paraphrasing-313.epsi}}}
 \caption{Paraphrasing rate for ICS-313}
 \label{paraphrasing-313}
\end{figure}


\begin{figure}[htb]
 {\centerline{\psfig{figure=/group/csdl/techreports/95-08/figures/paraphrasing-411.epsi}}}
 \caption{Paraphrasing rate for ICS-411}
 \label{paraphrasing-411}
\end{figure}

     
\begin{table}[htb]
 \caption{Paraphrasing rate analysis }
 \begin{center}
 \begin{tabular}{|l|r|r|r|}
  \hline
  \multicolumn{4}{|c|}{\bf Wilcoxon Signed Rank Test for Paraphrasing
  Rate(EGSM), (EIAM)}\\
  \hline
  & {\bf ICS-313} & {\bf ICS-411} & {\bf All} \\ 
  \hline
  \# 0 Differences & 0 & 0 & 0\\
  \# Ties & 0 & 0 & 0\\
  Z-Value & -.296 & -.398 & -.343\\
  P-Value & .7671 & .6909 & .7317\\
  Tied Z-Value & -.296 & -.398 & -.343\\
  Tied P-Value & .7671 & .6909 & .7317\\
  \# Ranks $<$ 0 & 6 & 8 & 14\\
  \# Ranks $>$ 0 & 3 & 7 & 10\\
  \hline
  \end{tabular}
  \end{center}
  \label{paraphrasing-rate-analysis}
\end{table} 


\subsection {Duplicate versus distinct errors}

In EGSM, all errors found by the  groups were distinct errors. In
EIAM,  some errors were duplicate/overlapped, that is, the same errors
might be found by more than one reviewer working privately.
This section analyses the nature of duplicate errors versus
distinct errors.

Figure \ref{duplicate} compares the performance of EGSM and EIAM for
distinct and duplicate errors. All numbers shown in this figure
pertain to valid errors. 
The column ``\%Dup'' shows the percentage of duplicate errors in EIAM.
``\%Group'' shows the percentage of distinct errors
found by the group. ``\%Indiv'' shows the percentage of errors found by 
any single individual in EIAM, which is equal to the total issue nodes
found by the group (Tot Found) divided by the number of seeded error
(Seed) and the number of reviewers (=3).
``\%Tot'' shows the total issue nodes found by the
EIAM group (Tot Found) divided by the number of seeded errors. Actually,
\%Tot simply measures the total number issue nodes regardless of issue
duplication discovered by EIAM 
groups. However, since different source code may contain different number of
seeded errors, this number is normalized against the
number of seeded errors; thus, some entries may contain values $>$
100\%. 

The data shows that about 28.7\% error duplication occurred in
ICS-313, 31.1\% in ICS-411, and 30.2\% in All groups.
It also shows that the performance of any single individual
in detecting errors (``\%Indiv'') was generally lower than 
the corresponding EGSM group performance (``\%Group'' in EGSM). 
Any single individual in EIAM found  only 21.6\% 
(ICS313), or  23.5\% 
(ICS411) of the total seeded errors compared to 40.6\% (ICS313), or 44.2\%
(ICS411) found by EGSM groups. 
The Wilcoxon tests between the variable ``\%Group'' (EGSM) and ``\%Indiv''
(EIAM) shown in Table \ref{max-duplication-analysis},
indicate P-value $<$ 0.05 for all three cases, or,
EGSM group detected
significantly more errors than any single individual in EIAM.
In other words,
when an individual worked in a group (EGSM), his/her group performance was
significantly better than his/her performance alone in EIAM.
However, the collective performance as an EIAM group (\%Group EIAM)
was not found to be significantly different from the 
EGSM group (\%Group EGSM). 
One possible explanation for this result is that error duplication in
EIAM was somewhat low (30.2\% for All groups), such that
different individuals of the same group had a slight
tendency to catch  different errors.  

When simply comparing the total number of issue nodes raised
by EIAM groups (\%Tot EIAM) to EGSM  
(\%Group EGSM), EIAM outperformed EGSM.
The Wilcoxon tests on these variables shown in Table
\ref{min-duplication-analysis} indicate a P-value $<$
0.05 for all three cases. Thus, EGSM groups raised significantly fewer
issue nodes than EIAM groups.
In other words, if all individuals in EIAM were to
raise distinct errors (no duplications), their group performance
(\%Tot EIAM) would be greater than the EGSM group performance
(\%Group EGSM). 

In conclusion, error duplication significantly affected review
performance. 


\begin{figure}[htb]
 {\centerline{\psfig{figure=/group/csdl/techreports/95-08/figures/duplicate.epsi}}}
 \caption{Detection effectiveness with distinct and duplicate errors}
 \label{duplicate}
\end{figure}

\begin{table}[htb]
 \caption{Detection effectiveness analysis for group vs. single individual} 
 \begin{center}
 \begin{tabular}{|l|r|r|r|}
  \hline
  \multicolumn{4}{|c|}{\bf Wilcoxon Signed Rank Test for \%Group(EGSM), \%Indiv(EIAM)}\\
  \hline
  & {\bf ICS-313} & {\bf ICS-411} & {\bf All} \\ 
  \hline
  \# 0 Differences & 0      & 0 & 0\\
  \# Ties          & 0      & 0 & 0\\
  Z-Value          & -2.547 & -3.408 & -4.257\\
  P-Value          & .0109  & .0007 & $<$.0001\\
  Tied Z-Value     & -2.547 & -3.408 & -4.257\\
  Tied P-Value     & .0109  & .0007 & $<$.0001\\
  \# Ranks $<$ 0   & 1      & 0 & 1\\
  \# Ranks $>$ 0   & 8      & 15 & 23\\
  \hline
     \end{tabular}
  \end{center}
  \label{max-duplication-analysis}
\end{table} 

\begin{table}[htb]
 \caption{Analysis of total issues found}
 \begin{center}
 \begin{tabular}{|l|r|r|r|}
  \hline
  \multicolumn{4}{|c|}{\bf Wilcoxon Signed Rank Test for \%Group(EGSM), \%Tot(EIAM)}\\
  \hline
  & {\bf ICS-313} & {\bf ICS-411} & {\bf All} \\ 
  \hline
  \# 0 Differences & 0      & 0 & 0\\
  \# Ties          & 0      & 1 & 1\\
  Z-Value          & -2.429 & -3.408 & -4.171\\
  P-Value          & .0152  & .0007 & $<$.0001\\
  Tied Z-Value     & -2.429 & -3.408 & -4.172\\
  Tied P-Value     & .0152  & .0007 & $<$.0001\\
  \# Ranks $<$ 0   & 8      & 15 & 23\\
  \# Ranks $>$ 0   & 1      & 0 & 1\\
  \hline
     \end{tabular}
  \end{center}
  \label{min-duplication-analysis}
\end{table} 


\subsection{False Positives}

False positives measures the number of wrong issues (or invalid
errors) raised by the participants.

Figure \ref{false-positives} shows the number false positives generated
by EGSM and EIAM groups. ``Issues'' is the
total number of issues raised by the group. ``Correct'' is the number
of correct issues/errors identified by the group, ``Wrong'' is the
number of wrong issues. W1, W2, W3, W4, W5 and W6 are further
classification of the ``Wrong'' issues as discussed later.

The data shows that EIAM groups
generated significantly more issues than EGSM groups. 
On average, EIAM groups generated 36.5 issues compared to only 14.4
issues in EGSM (All groups). 
However, many of these issues were 
false positives. The amount of false positives  were
significantly higher in EIAM than EGSM as shown by the column
``Wrong'' in Figure \ref{false-positives}. 
On average, EIAM groups raised 22.0 false positives compared to
only 5.3 in EGSM for All groups.

Since the participants were not penalized for raising false positives,
it was possible that they raised issues without evaluating
their validity.
To further understand the nature of these false positives, the false
positives were classified according to the following types:
\begin{itemize}
\item {\it Misinterpretation of logic (W1)}. This type of false
positive  was due to misinterpretation of the logic in the code.
This includes a failure to understand the conditional
statement, the algorithm, the input/output data, the use of primitive
functions, the specification implemented by the code, etc.
One typical example of this false positive is as follows:
\small
\begin{verbatim}
      if (!strncmp (string1,string2,string_size)){ 
         ... 
      }
\end{verbatim}
\normalsize
The function ``strncmp'' tests whether ``string1'' is equal to
``string2''; if so, the function returns 0 (boolean false in C), otherwise
it returns non-zero (boolean true in C). The operator 
``!'' negates the returned value of strncmp: if string1 is
equal to string2, the ``if predicate'' evaluates to true, and thus,
the statements within \{..\} will be executed. In other words, the
above statement should be read as ``if string1 equals string2''.
Unfortunately, some participants interpreted the above statement
incorrectly as ``if not string1 equals string2''.

\item {\it Re-occurrence of the same mistakes (W2)}. This type of false
positive was due to errors in the previous statement.
One example is the following ``missing else'' error:
\small
\begin{verbatim}
      if (condition)
         statement1;
      statement2;
      statement3;
      statement4; 
\end{verbatim}
\normalsize
The participants had already caught the ``missing else'' error, and
yet they stated, for example,  
that statement3 was wrong because statement2 was executed. 
Another example is raising this same issue again in statement4. 

In the review guideline, the participants were specifically instructed
not to raise this type of issue.

\item {\it Syntactical mistakes (W3)}. This type of false positive
was due to syntax errors in the program, such as
incorrect type assignment, missing semicolon, etc.

In the review guidelines, the participants were told that the code
had no syntax errors, that is, it had been compiled
successfully.  Yet, some participants still raised
this type of issues.

\item {\it Stylistic mistakes (W4)}. This type of false positive was
due to improper programming styles, such as
the use of  ``for'' instead of ``while''.

In the review guidelines, the participants were instructed not to raise
stylistic issues, or to propose alternative code; instead,
they should look for programming errors.

\item {\it Misinterpretation of the specification (W5)}. This type of
false positive  was due to misinterpretation of the given specification,
usually, over generalizing the specification.
For example, one subject stated that the file
needed to be opened before being used, although the specification
never mentioned it.

In the review guidelines, the participants were instructed to review
the code within the given specification. Anything not provided in the
specification should be considered correct. The specification itself
should always be considered correct.


\item {\it Guessing (W6)}. This type of false positive was due to
guessing. The participants simply pointed out
the location of the error without describing what the actual
error was. 

In the review guideline, the participants were instructed to provide
the detailed explanation of the error, as well as the location of the
error.

\end{itemize}

In general, only W1 may be considered as ``true'' false positives, the
other types as  ``procedural errors'', or misinterpretation of the
review guideline. 

Figure \ref{false-positives} shows the number of false positives
according to the above types generated by EGSM and EIAM groups.
Table \ref{false-positives-analysis} shows the resulting Wilcoxon
analysis of EGSM and EIAM false positives of type W1. 
With P-value $<$0.05 for all three cases, it is obvious that
 EIAM significantly raised more ``true'' false positives than EGSM.

Actually, this result was not surprising considering that in EGSM, one
had a chance to 
cross check his or her findings with other group members, whereas in EIAM,
there was no such chance. 
In general, when an issue was brought for  group discussion (in
EGSM), the majority had to agree with it before the presenter recorded
it as a valid error. Although, this mode of consensus was not required
in the experiment, that is, the presenter might record an issue
even though the majority did not agree, in practice, the majority
 would decide whether to record an issue or not. As a result,
bad issues (false positives) tend to be rejected  by EGSM groups.
However, I also observed that good issues
(valid errors) were sometimes rejected because the majority did not
agree.  In the extreme case, the majority even  
convinced an individual that a valid issue was really involved.


In sum, the result shows that group discussion (in EGSM) was more
effective than individual judgment (in EIAM) in filtering out false
positives. 

\begin{figure}[htb]
 {\centerline{\psfig{figure=/group/csdl/techreports/95-08/figures/wrong.epsi}}}
 \caption{False positives}
 \label{false-positives}
\end{figure}

\begin{table}[htb]
 \caption{False positives analysis (type W1:logic misinterpretation)}
 \begin{center}
 \begin{tabular}{|l|r|r|r|}
  \hline
  \multicolumn{4}{|c|}{\bf Wilcoxon Signed Rank Test for W1(EGSM), W1(EIAM)}\\
  \hline
  & {\bf ICS-313} & {\bf ICS-411} & {\bf All} \\ 
  \hline
  \# 0 Differences & 0      & 1 & 1\\
  \# Ties          & 2      & 2 & 6\\
  Z-Value          & -2.666 & -3.233 & -4.152\\
  P-Value          & .0077  & .0012 & $<$.0001\\
  Tied Z-Value     & -2.670 & -3.242 & -4.157\\
  Tied P-Value     & .0076  & .0012 & $<$.0001\\
  \# Ranks $<$ 0   & 9      & 13 & 22\\
  \# Ranks $>$ 0   & 0      & 1 & 1\\
  \hline
     \end{tabular}
  \end{center}
  \label{false-positives-analysis}
\end{table} 


\subsection{Group synergy}

Many researchers believe that group synergy plays an important role in
group reviews. In this experiment, group synergy (in EGSM) was measured
by the number of errors  discovered by one member of the group as the
result of inspiration by other group members.

Figure \ref{synergy} shows the number of correct errors,
and the percentage of these errors that were caught with and without group
synergy. The column ``Without Synergy'' was obtained by counting the
number of valid issue nodes discovered solely by one member
of the group, or issue nodes with
the ``Suggested-by'' field = ``1''(Me).  As described in the previous
chapter, ``Suggested-by'' field may contain value of 1 (Me), 2 (Me,
but inspired by others), 3 (Other but also occurred to me), or 4
(Other and had not occurred to me). 

Theoretically, the column ``With Synergy'' should refer to those
issues with value ``2'' only. In reality, some issues had 
no 1s or 2s, but only 3s or 4s. In other words, the participants
could not identify who suggested the issues in the first place, and
therefore should have been the result of group synergy.
The column ``With Synergy'' was obtained by subtracting 
``Without Synergy'' issues from the total ``Correct'' issues.


The data shows that the group synergy was indeed present in EGSM,
however, it did not contribute greatly to the number of errors found
by the group. Only  28.9\% of the total errors were discovered as a
result of group synergy, compared to
71.1\% discovered without group synergy (for All groups).

A typical example of group synergy observed during the experiment
involved the following scenarios:
\begin{enumerate}
\item The presenter was in the midst of paraphrasing this code:
\small
\begin{verbatim}
   Employee* Company1::findEmployee(char* SSN){
     /*loop through all workers*/
     for (int i=0; Workers[i] != 0; i++){
         /*if SSN matches the current worker, return the worker*/
         if(strcmp(Workers[i]->getSocSecurity()),SSN ==0)
             return Workers[i];
     return 0; //NULL pointer
    }
\end{verbatim}
\normalsize
\item Suddenly, one reviewer interrupted the presenter with comment:
``Check the case for the last worker in the array''.
\item The presenter then paraphrased the loop focusing on 
the case for the last worker, and did not find anything wrong.
\item However, at the end of paraphrasing, the presenter suddenly
raised a question ``What happens when the array is full ?''
\item Other group member soon realized the potential problem and
responded that {\tt Workers[i]} would never
equal to NULL, and thus the loop would never terminate.
\end{enumerate}


\begin{figure}[htb]
 {\centerline{\psfig{figure=/group/csdl/techreports/95-08/figures/synergy.epsi}}}
 \caption{Number of errors discovered with or without group synergy}
 \label{synergy}
\end{figure}


\subsection{Individual performance correlation}
One related question concerning group synergy is
 to understand whether there was a
positive correlation between the performance of individual group
members in EGSM and their respective performance in EIAM. In other words,  
whether individuals working in a group (using EGSM) contributed
more or less the same  percentage of errors as when working alone
(using EIAM). Theoretically, working in a group and feeling the presence
of others may boost one's performance, a phenomenon often
called social facilitation \cite{Myers90}. 

In EGSM, individual contributions were measured by counting the
percentage of valid issue nodes that were first suggested by the
individuals regardless who inspired them 
(i.e., Suggested-by field value is equal to 1 or 2).
In EIAM, individual contributions were measured by counting the
percentage of valid issue nodes created by each individual. 
Specifically, one's contribution was calculated as the percentage of
one's valid issues over the total of valid issues found by the group
regardless of issue duplications. Thus, if two participants found the same
issue, each participant's contribution for the issue would be 50\%.

Figure \ref{correlation-313} and Figure \ref{correlation-411} show
the correlation charts of individual contributions in EGSM and EIAM for
ICS-313 and ICS-411. 
Table \ref{correlation-analysis} shows the {\it
Correlation Z Test} \cite{StatView92}. The
test shows that for all three cases, there was significant
correlation (with P-value $<$ 0.05) between individual contributions
in EGSM and EIAM.

This result suggests that the performance of
individual participants was constant no matter what method (EGSM or
EIAM) was used. In other words, social facilitation did not occur in
real groups (EGSM); smart participants still caught the most
errors, while the slow ones caught the fewest.


\begin{figure}[htb]
 {\centerline{\psfig{figure=/group/csdl/techreports/95-08/figures/correlation-313.epsi}}}
 \caption{Individual performance correlations (ICS-313)}
 \label{correlation-313}
\end{figure}


\begin{figure}[htb]
 {\centerline{\psfig{figure=/group/csdl/techreports/95-08/figures/correlation-411.epsi}}}
 \caption{Individual performance correlations (ICS-411)}
 \label{correlation-411}
\end{figure}

\begin{table}[htb]
 \caption{Analysis of individual performance contribution}
 \begin{center}
 \begin{tabular}{|l|r|r|r|}
  \hline
  \multicolumn{4}{|c|}{\bf Correlation Z Test for \%Contrib(EGSM),\%Contrib(EIAM)}\\
  \hline
             & {\bf ICS-313} & {\bf ICS-411} & {\bf All} \\ 
  \hline
   Correlation   & .632   & .483 & .530\\
   Count         & 27     & 45   &  72\\
   Z-Value       & 3.648  & 3.411 & 4.898\\
   P-Value       & .0003  & .0006 & $<$.0001\\
   95\% lower     & .332 & .220  & .340\\ 
   95\% upper     & .816 & .680 & .678\\
  \hline
   \end{tabular}
  \end{center}
  \label{correlation-analysis}
\end{table} 


\section {Analysis of questionnaires}
\label{sec:questionnaires}
This section presents the analysis of qualitative data collected
through questionnaires. The 
questionnaires consisted of three sets of questions concerning the use
of EGSM, EIAM, and the final preference of the review methods.
Most questions were assessed using the scores of 1-5; their
interpretations depended upon the specific questions asked.
Some questions were open ended, and asked the participants to write out
the answers if they desired.
Appendix \ref{app:questionnaire-egsm}, \ref{app:questionnaire-eiam},
\ref{app:questionnaire-post} show these questionnaires in detail.

In general, the questionnaires assessed the qualitative aspects
of the review processes, the methods, and the systems used in the
experiment. Specifically, they assessed the participants'
perceived quality of EGSM versus EIAM, the problems associated with group
dynamics,  the appropriateness of the review processes and the
review systems, and their potential problems. 

Table \ref{questionnaires} shows the summary of the questionnaires
findings for All groups (72 participants). The data shows three
category responses corresponding to the percentage of participants
who responded with a value of 3 (mediocre), less than 3 (1 or 2),  or 
greater than 3 (4 or 5). 


\scriptsize
\begin{table}[htb]
 \caption{Results of questionnaires}
 \begin{center}
 \begin{tabular}{|l|r|r|r|c|}
  \hline
                   & {\bf $<$3 (\%)} & {\bf $=$3 (\%)} & {\bf $>$3 (\%)}  & {\bf Note}\\ 
  \hline
  \multicolumn{5}{|l|} {\bf Perceived quality of EGSM versus EIAM}\\
  \hline
   Understanding the source code before the review (EGSM)& 18 & 44 &38&b) \\%G1  
   Understanding the source code before the review (EIAM)& 24 & 40 &36 &b) \\%I1
   Understanding the source code after the review (EGSM)& 4 & 26 & 70 &b)\\%G2 
   Understanding the source code after the review (EIAM)& 5 & 38 & 57 & b) \\%I2
   Improvement in understanding the programming language (EGSM)& 21 & 35 & 44 & a)\\%G3
   Improvement in understanding the programming language (EIAM)& 25 &33  & 42 & a) \\%I3
   More confidence in the issues raised when using EGSM & 7 & 21 & 72 & a)\\ %G7
   More confidence in the issues raised when using EIAM & 40  & 38  & 22 & a)\\%I6
   Overall confidence about the review quality using EGSM & 4 & 18  & 78 & b)\\%G13 
   Overall confidence about the review quality using EIAM & 21 &37 & 42 &b) \\%I10
   Perceived productivity in using EGSM vs EIAM & 72 & 18 & 10 & c) \\ %P5
   Overall preference in using EGSM vs EIAM & 63 & 30 & 7 & d)\\      %P4
  \hline
  \multicolumn{5}{|l|} {\bf Group dynamics}\\
  \hline
   There were much disagreement among group members &65 &27 & 8 & a)\\ %G17
   I could express my opinion freely & 5 & 14 & 81 & a) \\%G21
   Some people dominated the discussion &49 &30 & 21 & a)\\ %G18
   There was some hostility in the group  &79  &13 & 8 & a)\\     %% G28
   Overall satisfaction with the group interaction &3 & 12& 85 & b)\\ %G26
  \hline
  \multicolumn{5}{|l|} {\bf Review process} \\
  \hline
   EGSM training was sufficient & 7 & 22 & 71 & a)\\ %G9
   EIAM training was sufficient & 11 & 22 & 67 & a)\\ %I7
   The source code was easy to understand (EGSM)& 14 &36 & 50 & a)\\ %G4
   The source code was easy to understand (EIAM) & 25 &36 & 39 & a)\\ %I4
   Review time was sufficient in EGSM & 25 & 21 & 54 & a)\\%G10
   Review time was sufficient in EIAM & 11 & 11 & 78 & a)\\%I8
   Paraphrasing in general was useful & 15 & 40 & 45 & a)\\%G34
   Paraphrasing in EGSM inspired me in finding errors &25 & 40 &35 & a)\\ %G33
   Presenter was too fast & 71 & 22 & 7 & a)\\%G32
   Presenter had done a good job & 5 & 28 & 67 & a)\\  %G35
   Overall satisfaction with EGSM process & 0 & 19 & 81 &b) \\ %G37
   Overall satisfaction with EIAM process & 14 & 25 & 61 &b) \\ %I14
  \hline
  \multicolumn{5}{|l|} {\bf Review system} \\
  \hline
   EGSM system is easy to use & 3 & 12 & 85 & a) \\ %G40
   EIAM system is easy to use & 3 & 21 & 76 & a)\\ %I17
   EGSM system is useful& 4 & 17 & 79 & a)\\ %G41
   EIAM system is useful & 15 & 22 & 63 & a)\\ %I18
   Overall satisfaction with EGSM system & 3 & 11 & 86 &b) \\ %G42
   Overall satisfaction with EIAM system & 11 & 22 & 67& b) \\%I19
   Overall satisfaction in using CSRS & 7 & 16 & 77 & b)\\ %P3
   I rather used pencil and paper than CSRS & 72 & 17  & 11 & a)\\ %P1
  \hline
  \hline
  \multicolumn{5}{|l|} {\bf Note}\\
   \hline
    \multicolumn{5}{|l|} {a) 1=Not at all true ... 5=Very true}\\
    \multicolumn{5}{|l|} {b) 1=Very low ... 5=Very high}\\
    \multicolumn{5}{|l|} {c) 1=Much EGSM, 2=Somewhat EGSM,3=Equal,4=Somewhat EIAM,5=Much EIAM}\\
    \multicolumn{5}{|l|} {d) 1=Strongly EGSM,2=Somewhat EGSM,3=Equal,4=Somewhat EIAM,5=Strongly EIAM}\\
   \hline
   \end{tabular}
  \end{center}
  \label{questionnaires}
\end{table} 
\normalsize



\subsection {Perceived quality of EGSM versus EIAM}
\label{sec:perceived-quality}
In the questionnaires, the participants were asked to rate their
perceptions concerning EGSM and EIAM,
specifically, their degree of understanding the source
code/programming language, their confidence in the issues they
raised, their perceived review qualities and productivities, and their
overall satisfactions.

From Table \ref{questionnaires}, it shows that 
most participants had  sufficient or more than sufficient degree of
understanding the source code 
before the reviews (82\% of EGSM participants and 76\% of EIAM
participants). 
The participants also felt that their degree of understanding the
source code was improved  after the reviews (96\% of EGSM participants
and 95\% of EIAM participants), with much more EGSM participants improved
their skills beyond sufficient (70\% of EGSM compared to 57\% of
EIAM). There were also a slight improvement in understanding
programming languages in general (44\% for EGSM, and 42 \% for EIAM). 

The participants also had more confidence about the issues they raised
when using EGSM (72\% of the participants) as opposed to EIAM
(only 22\%).  They also had more confidence about their overall
review qualities when using EGSM (78\%) versus EIAM (42\%). They also
felt more productive using EGSM  than EIAM  (72\%). 
Overall, they strongly preferred EGSM over EIAM (63\%).

The following comments were made by participants to explain why they
preferred working in a group (EGSM):
\begin{itemize}
\item Working in a group helped me find errors faster and better.
\item Working in a group helped me understand the code better.
\item You got analysis from  different angles that
others might not see it. If one missed the error, others might find it.  
\item Working in a group was better because I was not too familiar with
the code and/or programming language myself; others could contribute
more errors than me.
\item When working alone, I had to rely on my own knowledge, which was
rather limited.
\item Sometimes you got ideas from others that you had never thought
of, and you felt less stress.
\item Working in a group made you felt more confident; when you raised
an issue, others might support you.
\item Working in a group could stimulate more ideas and was far more
effective than working alone. If one raised an issue, others could
correct it immediately to avoid unnecessary mistakes.
\end{itemize}


The few subjects who preferred working individually made these
comments: 
\begin{itemize}
\item It was hard to work efficiently in a group. All members should
be well familiar with the code and each other.
\item Working alone provided me with a better concentration.
\item In a group, more time was spent for discussion and less time was
used to find the errors.
\item You had endless arguments in a group.
\item You spent less time arguing the issue validity when working alone. 
\item I could get a lot more done in a given amount of time when working
by myself.
\end{itemize}

The overwhelming support for EGSM might be explained by the
following:
\begin{enumerate}
\item The participants felt that the number of errors they caught
individually (EIAM) were far fewer than those caught when working in a
group (EGSM). This was indeed true. The analysis on Figure 
\ref{duplicate} does show that the number of errors any single
individuals 
caught (``\%Indiv'') was significantly lower than their collective
performance in EGSM (``\%Group'' in EGSM).

\item The participants felt the presence of group synergy
when working in a group, although the finding shows that
its actual proportion was less than non-synergy effort.

\item The participants felt they could {\it free ride} on the group
effort when using EGSM.  In other words, they did not have to work as
hard as they would have worked in EIAM. Some of the comments
preferring EGSM listed above clearly indicated this.

I also experienced the presence of ``free riding'' among
some group members during the experiment; when I asked one group to do
EIAM on their first review session, some participants strongly
objected because they felt they have not prepared well; they insisted
on doing EGSM first.

\end{enumerate}


\subsection{Group dynamics}

In the questionnaires, the participants were asked to indicate their
perceptions concerning common problems in a group session (EGSM),
which included level of disagreements, domination, and group pressure.

The data in Table \ref{questionnaires} indicates that there was not
much disagreement among group members (65\%),
and more importantly  the participants felt that they could express
their opinions freely (81\%). Also, there was not much
domination by some group members (49\%), nor did they feel much
pressure/hostility by some group members (79\%). Overall, the
participants were highly satisfied with the group interaction (85\%).

Hence, the results suggest that there were no significant
problems with group dynamics in EGSM. This also implies that 
group dynamics did not bias the experimental results.


\subsection {Review process}
In the questionnaires, the participants were asked to indicate the
appropriateness of the review process, specifically, 
the training session, the source code difficulty,
the review time, the presenter's performance, and the paraphrasing
technique.

The results shown in Table \ref{questionnaires} indicate that most
participants felt that the training was sufficient or more than
sufficient for both EGSM and EIAM (93\% for EGSM and 89\% for EIAM).
The source code was very easy to understand for both
EGSM and EIAM, with a slightly more
EGSM participants  than EIAM participants (50\% for EGSM and
39\% for EIAM). 
The review time was also sufficient or more than sufficient for both
EGSM and EIAM (76\% for EGSM and 89\% for EIAM),
although more EGSM participants (25\%) considered the review
time was slightly insufficient compared to EIAM (11\%).

With respect to paraphrasing, the participants generally 
felt that the technique was very useful (45\%) and was able to inspire
them to find errors (35\%), although about equal number of
participants also felt the technique was mediocre (40\%).

As for the
presenter's performance, most of them did not feel the presenter went
too fast (71\%). Overall, they felt the presenter had done an
excellent job (67\%).

Specific comments regarding paraphrasing technique expressed by the
participants:  
\begin{itemize}
\item It was hard to visualize what happened to the program simply by
reading it (without running it). 
\item Sometimes it seemed ridiculous to paraphrase statements which
were obvious. 
\item The paraphrasing was tedious, but I understand it was necessary.
\item If the code was not clear, the paraphrasing could not proceed
smoothly (i.e., hard to find the right sentences).
\item It took time to get used to paraphrasing.
\item The paraphrasing should be done in laymen's terms.
\end{itemize}

A typical problem with paraphrasing was that some presenters were
incapable of explaining the code effectively.
During the pilot study, I noticed that the presenters simply read the
code literally without adding ``meaning'' into it; this usually
led to unnecessary and tedious paraphrasing.
During the training session, the presenters were provided with an
example of good paraphrasing, namely, by summarizing the code in
the presenter's own words. 
In any case, since the reviewers 
already had a general understanding what the code did, 
suboptimal paraphrasing should suffice for effective reviews,
and this should have been the case in this experiment.

Overall, most participants were quite satisfied or more than satisfied
with the review processes of both EGSM and EIAM,
with a slightly more participants favoring EGSM over EIAM  (81\% for
EGSM and 61\% for EIAM). 

Some problems regarding EGSM process expressed by the participants
involved the amount of time it took to sort out
disagreements. One participant also commented that there was
a potential slow down of the EGSM review
process when one participant did not know much the review materials. 

As for problems with the EIAM process, some participants expressed
a lack of confidence, an inability to verify the findings with other
members, an inability to run the code and see the output, and the lack of 
manuals to look up built-in functions.
The underlying reason for all these problems seemed to stem from the
inability to discuss the findings with others.
Some participants even suggested that they would like
to add one extra review phase in EIAM where they could discuss their
findings with other group members. 

\subsection {Review system}
In the questionnaires, the participants were asked to indicate the
usability of the review systems (EGSM and EIAM), specifically their
ease of use and usefulness. The participants were also asked to rate
their overall satisfaction with the systems, and their relative
preferences when using manual review based on pencil and paper
instead.  

The results shown in Table \ref{questionnaires} indicate
that the participants generally had no problems with the systems. 
Both EGSM and EIAM were perceived as extremely easy to use (85\%
for EGSM and 76\% for EIAM), and very useful (79\% for EGSM and
63\% for EIAM), with a slightly more participants perceived EGSM as 
more useful than EIAM.

Overall, the participants were highly satisfied or more than satisfied
with the systems  (97\% for EGSM, 89\% for EIAM and 93\% for CSRS in
general). 
Again, more participants were satisfied with EGSM than EIAM.

Finally, although the participants had never experienced manual
review, they indicated that they were less likely to prefer manual
review than online review/CSRS (72\%), perhaps because CSRS 
was very easy to use. 

One problem with the EGSM system expressed by some participants concerned
the use of separate workstations for individual
participants. Some participants indicated that on many occasions they
found themselves looking at the presenter's workstation,  
instead of looking at their own workstations. 

It is not known whether the use of one large screen for
EGSM would produce different results.
I believe that this mode of presentation would not make any
difference, primarily because the majority of the participants were
already comfortable with the current mode of looking at one's
own workstation while listening to the presenter.  
In any case, this issue may need further investigation.

As for the problems with EIAM system, some participants commented
that the window/font size was too small, and that they desired an
online help to look up the documentation of built-in functions.  
For the latter issue, the review instruction actually allowed
the participants to direct questions about built-in functions
as well as questions about user commands to me during the
experiment. 
However, some participants may not have been aware of this.

As for system stability, only two participants expressed their
concerns in the questionnaires. 
I did observe that on a few occasions the systems 
froze and needed to be restarted. However, most of the time the
system performed very reliably.

Finally, the following comments about the system and the experiment
expressed by the participants were made:
\begin{itemize}
\item I really enjoyed doing this stuff.
\item I enjoyed the review thoroughly.
\item I think this kind of program is very useful for learning the
code written by other people. This really helped me better
understanding the logic of the code and the issues that 
I never thought to be of importance and usually ignored it. I would
like to see all ICS classes have this practice as well.
\item I think the system is very useful for source code review. One of
the attribute that I like most is that you can jump between functions
rather easily.
\item The system was very interesting to work with. The commands
weren't that hard.
\item Overall, I felt that this was an excellent way to conduct
reviews on source code.
\item It was well worth the time, it's a good system.
\end{itemize}

\section{Discussion}
\label{sec:discussion}

\subsection{Reasoning about the experimental results}

The goal of the experiment was to investigate whether there were
performance differences between EGSM and EIAM review methods,
specifically their detection effectiveness and detection costs. Both
review methods included only one phase
where the participants searched for errors in the source materials
using the paraphrasing technique. In EGSM, the participants worked
together as a group and were led by the presenter, while in EIAM, the
participants worked privately.  
Figure \ref{fig:review-model} illustrates the issue identification process
of these two review methods:
\begin{enumerate}
\item It starts with the presenter (in EGSM) or the individual
reviewers (in EIAM) paraphrasing the source materials 
(i.e., walking through the source materials). In
EIAM, the reviewers paraphrase the source silently/mentally.
\item This paraphrasing can lead to identification of a potential 
issue in the source. In EGSM, one reviewer would interrupt the
presenter when he or she spotted the potential issue.
\item This issue is then evaluated for either acceptance or
rejection. In EGSM, group members evaluate the issue through face
to face discussion. In most cases, this involves explaining the issue to
every group member and convincing each of them that the issue is indeed
correct. In EIAM, individual participants evaluate the
issue privately. 
\item Finally, when an issue is accepted, it is recorded as
error. However, this error may not necessarily be
a valid error; it can be a false positive.
Similarly, when an issue is rejected, it can be simply
a non-error, or a false negative (a valid error that is rejected).
\end{enumerate}
Again, in EGSM this entire process was performed explicitly through
group  discussion, while in EIAM it was performed by individual
participants implicitly.

\begin{figure}[htb]
 {\centerline{\psfig{figure=/group/csdl/techreports/95-08/figures/review-model.epsi}}}
 \caption{The issue identification process in EGSM and EIAM}
 \label{fig:review-model}
\end{figure}


The experiment found that there were no significant differences
between the number of valid errors found (detection effectiveness) for
EGSM and EIAM. However, EIAM groups found significantly more potential issues
(due to false positives) than EGSM groups. 

%The contributions of individual participants in the real group were
%also found to be correlated with the corresponding nominal group. In
%other words, the group members who spotted the potential issues in the
%first place made more or less the same contributions as when working
%privately. Although the analysis did reveal the
%presence of group synergy, this synergy was not large enough to
%create a significant difference between the two methods.

The experiment also found that EGSM significantly cost more than EIAM
in one case (All groups). 
Further analysis indicated that this cost (effort/error)
was due to significantly higher detection effort in EGSM. From the
model in Figure 
\ref{fig:review-model}, these efforts can be broken down into paraphrasing
effort and evaluation effort, and leads to the question of whether
paraphrasing or evaluation had different costs in the two methods?
The paraphrasing
analysis indicated that there were no significant differences in paraphrasing
rates between EGSM and EIAM (see section
\ref{sec:paraphrasing-rate}). Unfortunately, the paraphrasing data 
collected in the experiment did not distinguish between the time spent
on paraphrasing versus the time spent on evaluation.
Based on
observation during the experiment, as well as on the participants'
comments in the questionnaires, the evaluation effort for EGSM seemed
much greater than that for EIAM. 
In other words, EGSM groups seemed to
spend a much greater amount of time evaluating/discussing potential
issues than individual group members in EIAM.
This conjecture is also supported by the fact that EGSM groups
caught significantly fewer false positives than EIAM groups. Thus, the
extra time 
spent by EGSM groups for evaluation was  perhaps used to filter out false
positives.  

Other observations also revealed that a certain type of group discussion
was neither paraphrasing nor evaluation. It was strictly for
education/learning purposes. For example, a participant asked
the presenter to explain in great detail the function currently being
paraphrased  or related function. This type of discussion was also
identified as one shortcoming of EGSM (see also Section
\ref{sec:perceived-quality}). 

In conclusion, although the findings suggest that
EGSM was more costly than EIAM, the extra cost incurred by EGSM was
not totally worthless; some was used for filtering out false
positives. 

With respect to false negatives, I did observe that EGSM
participants sometimes rejected valid errors. However, no
quantitative data concerning false negatives was collected during the
experiment.  


%The author believes that this evaluation
%activity  was the bottle-neck of a synchronous group review such as
%EGSM. 
%Although the author (as the moderator) was somewhat successfully
%enforced the group process so that the participants always worked
%together as a group  when searching for errors (i.e., were
%synchronized with the presenter). 


\subsection{Which review method is better ?} 

The results of the experiment did not yield strong
evidence supporting one method over the other (EGSM or EIAM). 
Statistically, there were no significant differences between the two
methods with respect to detection effectiveness.
Although the detection effort of EGSM was more costly than EIAM, 
this effort resulted in fewer false positives.
However, when all the data is viewed together, the following
conjectures appears to be supported:

\subsubsection{A modified EIAM might outperform EGSM}
\label{sec:eiam-may-be-better}
Although the analysis did not indicate strongly significant differences in
detection performance between EGSM and EIAM,
other results seem to favor EIAM over EGSM.  
First, in all three cases of the data (ICS-313, ICS-411 and all
groups), the  average detection effectiveness for EIAM was greater
than EGSM. For 
example, for All groups, EIAM found 46.4\% of the errors compared to
42.8\% found by EGSM (see Figure \ref{main-hypothesis}). The Wilcoxon
test on this data shows a rather low P-value of 0.1374
(see Table \ref{detection-effectiveness-analysis}). Thus,
EIAM would outperform EGSM at the significant level of $\alpha<$0.15.
Second, the effect of group synergy that was supposed to stimulate
errors  discovery among group members in EGSM was relatively small
compared to non synergy effects (28.9\% of errors found was due to synergy
compared to 71.1\% found without synergy). 
This was further supported by the fact that social
facilitation did not appear to occur in EGSM. In other words,
individual members appeared to contribute 
more or less the same amount regardless whether they worked alone or
in a group. Third, EIAM has more room for improvement: 1) the duplication
analysis provides evidence that with fewer duplications,
EIAM could further outperform EGSM. This could be achieved, for example,
by mixing participants' skills or introducing 
specialists to the group; 2) EIAM might scale up better with
increasing group size than EGSM.
Figure \ref{group-size-variation} shows the increasing trends of
detection  
effectiveness of EIAM with increasing group size (for ICS-411). The
data were obtained from randomly assigning the EIAM  participants into  
groups of size 1-10. 

\begin{figure}[htb]
 {\centerline{\psfig{figure=/group/csdl/techreports/95-08/figures/group-size.epsi}}}
 \caption{EIAM with varying group sizes}
 \label{group-size-variation}
\end{figure}


It is unlikely that the same trends will hold for EGSM considering
that more efforts will be needed to maintain the group process,
especially the paraphrasing process that
holds the key to overall group performance. 
The following lists potential problems with 
paraphrasing as observed during the experiment:

\begin{itemize}

\item {\bf Reading  instead of analysis.}
Some presenters read the code literally without putting meaning into
it. In other words, the presenter himself/herself might have little
understanding the code being paraphrased. This usually led to 
overlooking obvious errors, such as ``mistyped statement'' discussed
in section \ref{sec:error-type}. The paraphrasing process became
very routine and tedious, and the reviewers often lost their
concentration as the result.

\item {\bf Presenter is too fast.}
This is a common problem in paraphrasing. Several studies also cited
this problem \cite{Russell91,Deimel91}.
My observation indicated that in these circumstances the presenters
were generally very knowledgeable about the code being
paraphrased. The presenter paraphrased the 
code as if he was talking to himself, and without pausing to ask the
reviewers whether they were following; when he spotted a potential issue, he
immediately recorded the issue then consulted with the group
afterwards. In the experiment, however, the moderator normally
intervened and asked the presenter to slow down. Nevertheless,
the presenter often repeated this problem.

\item {\bf Reviewer is too passive.}
I observed that reviewers often kept silent during the
entire paraphrasing process until the presenter brought up an issue
and solicited the reviewers' opinions.
In other words, the reviewers never questioned the code being
paraphrased or the explanations provided by the presenter. As the result,
sometimes the presenter was
not sure about the logic of the code, but somehow managed to come
up with an assumption (i.e., faulty assumption) to back up his
explanation, and yet no reviewers ever questioned this assumption.
Once the presenter misses a potential issue, the entire group
suffers. In contrast, when one group member misses a potential 
issue in EIAM, other members working independently might catch it.


\item {\bf Participant is too opinionated.}
In some cases, the reviewer was too opinionated. He constantly
interrupted the presenter and argued that the statement being
paraphrased was wrong.  
This usually led to a very slow paraphrasing process; the presenter had to
stop frequently and discuss the issue. Most of the time, 
it led to heated arguments with the presenter, and after a
while the presenter became very defensive and less objective in his
evaluation. 

Similarly, the presenter could be very opinionated. He might
ignore the reviewers' opinions in favor of his. 
As the result, potential issues from the reviewers
might be dropped from further evaluation/discussion.

\item {\bf False interpretation due to verbal paraphrasing.}
As discussed in Section \ref{sec:error-type}, some programming statements
are easy to paraphrase incorrectly, such as logical {\it and/or}, and 
the conditional statement {\it if-then-else}.

\end{itemize}


\subsubsection{Two-phase EIAM may be required}
\label{sec:two-phase}
In practice, a single phase EIAM may not be sufficient. There 
were too many false positives generated. One benefit of EGSM 
was its ability to filter out false positives. Thus, one may want to
introduce a group meeting into the second phase of EIAM to allow
the participants' issues to be  evaluated by the group, i.e., the
objective of the phase is consolidation.
This second phase can also increase participants'
confidence and facilitate learning.
Actually, this two-phase review method is commonly
practiced \cite{Humphrey90,Votta93,Porter94} (see also Chapter
\ref{ch:framework}). 

However, as pointed out by Votta \cite{Votta93}, this second phase could
be inefficient if not carefully administered. For example, instead of
having the entire group members participate in the second phase,
one might select only a subset of the participants.

\subsubsection{N-fold replications can improve overall performance}
\label{sec:n-fold-inspection}
The experiment found that on average,  each group only discovered
a relatively small percentage of the total errors (less than 50\%;
42.8\% for EGSM and 46.4\% for EIAM). These findings are consistent
with the  literature. For example, Myers's
study of code inspection found 38\% of total errors discovered by the
group of size 3 \cite{Myers78}. Martin and Tsai's study of 
user requirements document inspection found 
27.2\% of errors discovered by the group of size 4 \cite{Martin90}
(see also Table \ref{tb:ftr-empirical-studies}).
%Schneider, Martin and Tsai later repeated this study and found 35.1\%
%of errors discovered by the group of size 3 \cite{Schneider92}.

However,  replicating the number of groups N fold
shows dramatic improvement in the overall detection effectiveness.
Figure \ref{nfold} shows the detection effectiveness plots for ICS-411
for N=1,2..7. The data was obtained by  combining the existing
groups randomly, and counting their detection effectiveness. As shown
in the figure, the groups find almost all the errors
(90\%) after N=7. It also shows that both EGSM and EIAM yield more or
less the same results using this technique.
These results are consistent with Martin and Tsai's study \cite{Martin90}.

\begin{figure}[htb]
 {\centerline{\psfig{figure=/group/csdl/techreports/95-08/figures/nfold.epsi}}}
 \caption{N-fold replications of EGSM and EIAM}
 \label{nfold}
\end{figure}

In general, I believe that any review method that combines
results from different individual participants/groups has the
potential to improve
overall review performance, since the individual
participants/groups tend to discover different errors (see also section
\ref{sec:eiam-may-be-better} ).

\subsection{Related studies}
As described in Chapter \ref{ch:exp-design},  
one objective of this study was to follow up the empirical study by
Votta, Porter and Eick et al. \cite{Votta93,Porter94a,Eick92}.
These studies found that group meetings (collection
meeting) are not as beneficial as claimed by Fagan \cite{Fagan76}.
However, as described in Chapter \ref{ch:exp-design}, the review 
methods used in these studies are slightly different from Fagan's
method.  For example, in Votta's
method, the objective of the meeting phase is
collection/consolidation, whereas in Fagan's method, the objective is
examination/defects finding. Furthermore, Fagan's method also involves
a paraphrasing technique. 

The study presented in this chapter investigated group meetings
(EGSM) in which the objective was to find defects and the technique was
paraphrasing. The results obviously did not support Votta's
study that group meetings were not beneficial, since the group
performance (EGSM) was no worse than reviews without meetings
(EIAM). This study 
also found that group synergy was indeed happening, although its
contribution was relatively small compared to non-synergy effort.

Again, the review method in this study is different from Votta's
method. Votta's method is based on a two-phase review process; in the first
phase, individual participants search for errors like EIAM; in
the second phase, the participants meet together to discuss their findings
(i.e., collection meeting). His study found that the collection
meeting is ineffective for finding defects and no group synergy is
observed. The study presented in this chapter did not involve a
collection meeting. In fact, I believe that such a two-phase
review method can be effective (see Section \ref{sec:two-phase}). 

The review method in this study is also different from Fagan's
method. Fagan's 
method defines at least two review phases; in the first phase, the
participants prepare themselves individually by reading the source
materials; in the second phase, they meet as a group
searching for errors using paraphrasing technique like EGSM. 
The review method presented in this chapter did not have the
preparation phase. 

Finally, with respect to the FTR framework, 
this study differs from Votta's study in that
this study investigated the ``interaction'' component of review methods,
(i.e., whether the group interaction was more effective than no
interaction/individual effort), whereas Votta's study
investigated the effectiveness 
of the ``phase'' component of review methods (i.e., whether
the meeting phase was effective in traditional inspection methods).

There are many other empirical studies concerning software
reviews and inspections as shown in Table \ref{tb:review-vs-test}
and Table \ref{tb:ftr-empirical-studies} of Chapter
\ref{ch:related-work}.
However, the review methods in these studies investigate different
components of the methods. No studies have investigated the 
``interaction'' components of review methods.
For example,
Hetzel's, Basili's and Porter's studies \cite{Hetzel76,Basili87}
investigated the ``examination technique'' components of review
methods. 
In both Hetzel's and Basili's method, the examination technique (code
reading) was compared to testing; no group interaction was
involved. In Porter's study, three different examination (detection)
techniques were compared: Ad Hoc, Checklist and Scenario
\cite{Porter94}. 

Myers', Eick's, and Knight's studies \cite{Myers78,Eick92,Knight93}
investigated review methods as a whole  
(i.e., without focusing on specific review components). 
Myer's study compared code walkthrough/inspection method with testing.
Eick's study involved a two-phase review method to
estimate the number of software faults before coding; Knight's study
investigated the effectiveness of Phased Inspection method.

Other FTR studies investigated the effect of generic review attributes
on review performance, such as group size and replication. 
Martin's and Schneider's studies \cite{Martin90,Schneider92}
found that replicating the review group N fold can significantly
improve review performance. Their findings
are also substantiated by this experiment (see section
\ref{sec:n-fold-inspection}).  
The effect of group size observed in this study seems to contradict the
Buck's study cited in \cite{Bisant89}. 
This study found that review performance would increase with a larger
group size, whereas the latter study found no
difference in performance between 3, 4, or 5-person inspection teams.
This study also agrees with Weller's study that found four-person
teams are twice 
as effective, and more than twice as efficient as three-person teams
\cite{Weller93}. However, unlike Weller, who attributed the
increase in performance to increase in team's level of expertise, this
study found that it was caused by the increase in the number of unique
errors caught. This study also found that the performance would start
to level off after N=6, as seen in Figure \ref{group-size-variation}.

Finally, Votta's study also found that  group meetings (which are
similar to EGSM in this research) minimize
false positives \cite{Votta93}. 

%%Finally, from behavioral science


\subsection {Lessons learned about review experiments}
The following presents the lessons learned from
administering the experiment, observing the review process and
implementing the review system used in the experiment.
\begin{enumerate}
\item {\bf Stick with the experimental schedules.}
The actual experiment (excluding the training) lasted about two weeks
for ICS-313 and another two weeks for ICS-411. During the time,
the groups kept changing their schedules. 
At one time, I had to accommodate 6 groups in one day from 9 am to
11 pm to fit everyone's schedules. This was unfortunately at the
expense of the experimental design.
For example, instead of having more or less the
same number of groups performing EGSM followed by EIAM and vice versa,
I unintentionally scheduled more EIAM groups to do their
reviews first.

The lesson is {\it do not always grant the students' requests for schedule
changes as they may affect the experimental design.
Ask the students to keep their original schedules}.

\item {\bf Interrupting group discussion is not trivial.}
In this experiment, the moderator was supposed to ensure that the
group process was 
carried out properly. For example, the moderator should interrupt and
stop any group discussion that lasted more than 10 minutes.
In many cases, it was very difficult to decide when to
cut the discussion, especially when it involved arguments about the
logic of the code. Different participants presented different views
on how the code worked. Interrupting this type of discussion
might impede the paraphrasing process that followed. Allowing the
discussion to continue, on the other hand, might lead to digression.

The lesson is {\it prepare to interrupt unproductive group
discussions as necessary}. 

\item {\bf Dialog box for prompting user's action is desirable.}
In general, the users found it easy to use the review system
(EGSM/EIAM). The system implements hypertext navigational style, as
well as pulldown and context sensitive popup menus. 
However, some prompts for user's action appeared on the
minibuffer (i.e., the text area appearing on the bottom of each window).
The users often missed this prompt. They thought that the system was
frozen and constantly tried to interrupt it.

The lesson is {\it use a dialog box that pops in the middle of
the window  for user's prompt whenever possible}.

\item {\bf Disable unused system features.}
The review systems EGSM and EIAM were implemented from a generic
review system CSRS. As a generic system, CSRS provides a rich set of
commands and features.
Unfortunately, many of these features were not used for the
experiment. Some of them were even undesirable. For example, in
EGSM system, the presenter was supposed to retrieve a node that was
displayed in other participants' screens. However, individual
participants can still retrieve other nodes if they want.
Some participants did exactly this although it was not allowed. The
author explicitly instructed the participants not to review the nodes
by themselves; they had to discuss any potential issue with all group
members and asked the presenter to retrieve any source node.
During the experiment, I (the external moderator) monitored the review
process, and therefore could prevent any such behavior from re-occurring.

The lesson is {\it disable system features that are
not used in the experiment}.

\item {\bf Final evaluation is not trivial.}
All issues generated in the experiment were evaluated manually for their
correctness. 
Some issues were vague and difficult to judge whether the participants
really understood what the real problems were or they were just
guessing. In the latter case, there was no credit given.
This process was even painful for EIAM since there were
a significant number of false positives.

The lesson is {\it ask the participants not to record vague issues}.

\item {\bf Effective paraphrasing requires practice.}
Although the presenters generally had done a good job during the
experiment, I also believes that effective paraphrasing needs
a lot of practice. This applies for both the presenter and the
reviewers. The issue for the presenter is how to present the code
in such a way that stimulates member participation.
I noticed that the presenters sometimes gave up on
paraphrasing and let individual members search for the errors
themselves. Of course, the moderator would soon reprimand the
presenters. 
As for the reviewers, the issue is how to actively participate in the
process and help the presenter uncovers errors.
I noticed that the participants sometimes felt sleepy or did not
pay attention to the presenter.

Other than the problems identified in Section
\ref{sec:eiam-may-be-better},  I noticed that the key to
effective paraphrasing is practice. 

The lesson is {\it effective paraphrasing requires practice}.
\end{enumerate}


%Things to ponder:
%In general, I believe EGSM is more thorough because reviewers are forced
%to read  the code statement by statement or all statements are
%guaranted to be covered. Unfortunately, this process may not be effective;
%many participants didn't question what being paraphrased (tend to agree).
%This is one reason why EGSM took longer because it run more slowly,
%but EIAM visited the functions more often than EGSM. ??
%Is it true that presenter caught errors more than any other
%participants. ?


