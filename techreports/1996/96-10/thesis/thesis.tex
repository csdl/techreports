%%%%%%%%%%%%%%%%%%%%%%%%%%%%%%% -*- Mode: Latex -*- %%%%%%%%%%%%%%%%%%%%%%%%%%%%
%% thesis.tex -- 
%% Author          : Jennifer Geis
%% Created On      : Fri Sep  5 13:50:18 1997
%% Last Modified By: Jennifer Geis
%% Last Modified On: Mon Jan 12 12:17:37 1998
%% RCS: $Id$
%%%%%%%%%%%%%%%%%%%%%%%%%%%%%%%%%%%%%%%%%%%%%%%%%%%%%%%%%%%%%%%%%%%%%%%%%%%%%%%
%%   Copyright (C) 1996 Jennifer Geis
%%%%%%%%%%%%%%%%%%%%%%%%%%%%%%%%%%%%%%%%%%%%%%%%%%%%%%%%%%%%%%%%%%%%%%%%%%%%%%%
%% 

\documentstyle[nftimes,11pt,/group/csdl/tex/definemargins,
/group/csdl/tex/lmacros]{report} 
\input{/group/csdl/tex/psfig/psfig}
\setcounter{secnumdepth}{6}
\setcounter{tocdepth}{6}
\begin{document}


\title{JavaWizard: An Automated Code Checker For Java} \author{Jennifer
  Geis\\ Collaborative Software Development Laboratory,\\ Department of
  Information and Computer Sciences\\ University of Hawaii, Manoa\\ 
  Honolulu, Hawaii 96822\\ {\tt jgeis@uhics.ics.hawaii.edu}} \maketitle

\tableofcontents

% people to thank: Mom, Dad, Mike O, Johnson, all members of CSDL, Digital, 
% Bruce Foster, Will Kling, Steve Rodgers, Bill McKeeman, Dale Skrien, 
% Jeremy Lueck, Mitchell Goodman, 

\chapter{Introduction}
Every programmer hates debugging their work.  If you were to guarantee a
software developer that she would never have to spend another minute
tracking down bugs in her code, she would probably worship you for life.
All programmers can remember some horrible late night searching for
whatever is causing the strange behavior of their program.  Many had the
thought ``but it's supposed to work'' until they found the obscure error in
their code.

JavaWizard (JWiz) can't prevent those late nights, but maybe it can make
them happen a little less often.  JWiz is a Java source code analyzer.  It
scans through the code looking for common programming constructs which are
likely not what the programmer intended.  JWiz assumes the code compiles,
it does not concern itself with syntactical errors.  Instead, JWiz notifies
the user of possible semantic problems.  The following code prompts a
notice from Jwiz.

String str = ``hi'';

if (str == ``hi'') { // do something }

JWiz would give the warning ``Comparing strings using '==' instead of
'.equals'.''  Usually, programmers want to compare the contents of two
strings while the above code compares the addresses of the strings.  Since
the strings are not the same object, the if statement will always return
false.  This is the case even though the string's contents are identical..

In the Fall of 1995 I participated in a course on Watts Humphrey's Personal
Software Process (PSP).  During this course, students were required to keep
track of all defects they found in their programs.  For each defect the
data collected included a description of the defect, what phase of
development the defect was injected into the program, the phase in which it
was removed, and how long it took the developer to find and remove it.

In looking over the data, I noticed that the defects that took the most
time to find and remove were the ones that made it past the compilation
phase and didn't rear their ugly heads until the testing phase.  These were
the defects that resulted in my late nights.

In the summer of 1997, I was invited to do a summer internship at Digital
Equipment Corporation in New Hampshire.  Purely by coincidence, I found the
description of JavaWizard among the list of possible projects that Digital
offered me.  With my previously collected defect data, I figured this was
the perfect project and accepted the offer.

JWiz will be under constant development until the summer of 1998 at which
time I will return to Digital to hand over the finished program.  The final
version of the program is expected to have 70 defects checks and a gui
interface.

As JWiz was under development during the time the experiment was conducted
and this thesis written, the number of defects checks available ranges from
30 to 50. However, these defect checks include the most common and time
consuming errors that I have found.  I believe these will be sufficient to
predict the future usefulness of JWiz to the Java developer.

\section{Thesis Statement} 
JavaWizard is an accurate and effective tool for the discovery of Java
programming errors.
\section{Objective}


\chapter{Related Work}
\section{Automated Debugging}
Automated debugging involves using a computer program that scans through
code and looks for possible problem areas. Usually the program just
notifies the developer of the potential problem and the developer decides
whether or not it really is a defect.
\subsection{Lint-type tools}
LINT's original purpose was to locate bugs and inefficiencies in C source
code.  Later, it was expanded to handle portability issues.  The
portability problems arise from programs including system dependent
operations.  LINT shows where you have written system-dependent
(non-portable) code.
\subsection{AI tools}
\subsection{HCI tools}
\subsection{Program Comprehension}
\section{Non-Automated Debugging}
\subsection{Individual}
Non-Automated individual debugging typically involves the developer acting
as reviewer of their own product.
\subsubsection{Personal Software Process}
The Personal Software Process is a process of developing software designed
by Watts Humphrey of the Software Engineering Institute.  The method
includes doing personal reviews of both design and code.  The purpose of
these reviews is to identify defects before the next stage of development.
The idea is that the longer a defect stays in the product, the more costly
it is to remove it.  By performing reviews, the developer can identify the
problem areas and fix the defects prior to the next phase.  Through the use
of the Personal Software Process developers have managed to decrease the
number of defects found in compile and test by 50 percent or more.
\subsection{Group}
\subsubsection{Formal Technical Review}




\chapter{JavaWizard}
\section{Motivation}

\subsection{Preliminary Case Study}
\subsubsection{ICS 414 and CSDL}
\subsubsection{Initial Observations}
\paragraph{Common Mistakes}
\paragraph{Defect Frequency}
\paragraph{Testing Time}

\subsection{Possible benefits}
\subsubsection{Shorter overall development cycle}
\subsubsection{LOC/HR increase}
\subsubsection{Test Time as a Smaller Percentage of Overall Cycle}

\section{Design}
\subsection{100 Percent Pure Java}
\subsubsection{Platform Independent}
\subsection{JJTree}
\subsection{Symbol Table}
\subsection{How JWiz Works} 
\subsection{What JWiz Looks For}
\subsection{Current State}
\subsection{Future Additions/Modifications}

\section{Development Issues}
\subsubsection{Extensibility}
\subsection{Performance}
\subsection{What Makes a Defect?} 
\subsection{Complexity: Number of Tests}





\chapter{Experiment}

\section{Questions}
Who will benefit most by using JWiz?

At what phase does JWiz provide the most benefit?

How accurate is JWiz? (real defects vs. false positives)

How effective is JWiz? (real defects/KLOC)

% mention law of decreasing returns.

\section{Means of Data Collection}
JWiz was provided in zipped format for download/installation.  Along with
the JWiz program itself, the user was provided a graphical user interface.
This interface allowed the user to select the directory and files upon
which to execute JWiz.  Once selected, JWiz scanned the files and provided
a listing of possible defects found in the code along with the defect's
locations.  Next to each defect, there was a toggle button by which the
user identifed which warnings are valid, and conversely, which were just
'false positives.'  When the user indicated they were finished, the program
created an HTML page listing all the defects found in the program, and the
size of the program parsed.  The program size was given in terms of LOC,
number of classes, and number of methods. At this point, the data was
anonymously emailed to the author of this paper.

\section{Sources of Java Code}
The following are the sources from which I collected Java code and/or the
JWiz results.

Collaborative Software Development Laboratory: CSDL is a research group
within the ICS department at the University of Hawaii.  The members of this
group served as the original guinea pigs and testers for the JWiz program
itself and the accompanying data collection tool.

ICS 111: I was allowed to provide JWiz to Dr. Feng Gao's Introduction to
Computer Science students.  This proved to be useful to me in answering the
question of to whom JWiz provides the most benefit.

Hawaii Java User's Group (HJUG): Composed of Java developers from both the
academic and industrial communities, HJUG provided a forum for addressing a
range of experience and goals.

Digital Equipment Corporation (DEC): The Java development group at Digital
provided me with (state number of lines of code) in all stages of
development.

% Include Jan Stelovsky's Biotope program?
% Include WorldPoint

\section{Data Collected}
In addition to collecting the defects that JWiz found, I also collected
data on a number of other things in order to answer my questions.  As part
of my data, I asked the developer to indicate their experience in Java.
%other langs?
I also asked the developer to indicate if their organization was academic
or industrial  WORKING HERE!!!!!!!!!!!!!! 

Using a Java LOC counter included with the JWiz package, I recorded the
size of the program. 
% The LOC counter included the number of LOC as well as the number of 
% classes and methods.   ??





\section{Data Stratisfication}
I believed that there would be different defects found depending on a
variety of factors.  As a result, I strastified the data collected based on
the characteristics of the programs JWiz was run on, the program's
developer's skill, and the phase of development in which JWiz was run.

% explain what the significance of each would be.
\subsection{Academic/Industrial}
\subsection{Size}
The size of the program is useful in determining JWiz' effectiveness.  By
effectiveness, I mean the number of defects found per KLOC.  If Jwiz
reports only one error, the usefulness of JWiz to that developer varies
drastically if the program was one thousand lines of code or if it was only
ten.

\subsection{Developer Experience}
The developer's experience is a factor in what kinds of errors JWiz is
likely to find.  If the developer is a first year introductory student,
they aren't likely to be doing anything like inheritance and inner-classes,
so these defect checks are not likely to be invoked.  On the other hand,
these students will probably make the error of not creating a listener for
events, or adding multiple components to the same area of a BorderLayout
(only one will be displayed).

If the developer is experienced with Java, they could make the same
mistakes, but they are more likely to make mistakes like calling
Thread.suspend() (it causes your program to hang).  A beginning student
probably won't be doing anything with threads, so they won't encounter this
problem.

I believed that the number of effective JWiz tests would be directly
proportional to the developer's experience.

 
\subsection{Release Phase}
\subsection{Defect Rules}

\section{How Data Answers Preceeding Questions}
Percentage of valid JWiz warnings (aggregate measure of accuracy).  Density
of valid JWiz warnings (aggregate measure of effectiveness).




\chapter{Results}





\chapter{Conclusion}
\section{Discussion}
\subsection{JWiz Effectiveness}
\subsection{State of Java Software Quality}
\subsection{Suggested Future Software Analysis Tools}
what future tools might be useful as suggested by this analysis?


\chapter{Timeline}
\section{Weekly Schedule}

\subsection{January 12 '98}
\subsubsection{Chapter 4 draft}
\subsubsection{Data collection Application}

\subsection{January 19 '98}
\subsubsection{January 23, File For Degree App.}
\subsubsection{Chapter 3 draft}
\subsubsection{Chapter 4 revision}
\subsubsection{3 Defect Checks}

\subsection{January 26 '98}
\subsubsection{Chapter 3 revision}
\subsubsection{Chapter 1 draft}
\subsubsection{3 Defect Checks}

\subsection{February 2 '98}
\subsubsection{Chapter 1 revision}
\subsubsection{Chapter 2 draft}
\subsubsection{CSDL Data Collection}
\subsubsection{3 Defect Checks}

\subsection{February 9 '98}
\subsubsection{Chapter 2 revision}
\subsubsection{CSDL Data Collection}
\subsubsection{Non-CSDL Defect Data Collection}
\subsubsection{3 Defect Checks}

\subsection{February 16 '98}
\subsubsection{3 Defect Checks}
\subsubsection{Defect Data Analysis}
\subsubsection{Chapter 5 draft}
\subsubsection{Chapter 6 draft}

\subsection{February 23 '98}
\subsubsection{Chapter 5 revision}
\subsubsection{Chapter 6 revision}

\subsection{March 2 '98}
\subsubsection{March 2 - Thesis Due To Committee Members}
\subsubsection{3 Defect Checks}

\subsection{March 9 '98}
\subsubsection{3 Defect Checks}

\subsection{March 16 '98}
\subsubsection{March 16 - Thesis Defense}
\subsubsection{3 Defect Checks}

\subsection{March 23 '98}
\subsubsection{3 Defect Checks}

\subsection{March 30 '98}
\subsubsection{3 Defect Checks}

\subsection{April 6 '98}
\subsubsection{April 8 - Thesis Due in Graduate Division}
\subsubsection{3 Defect Checks}

\subsection{April 13 '98}
\subsubsection{4 Defect Checks}

\subsection{April 20 '98}
\subsubsection{4 Defect Checks}

\subsection{April 27 '98}
\subsubsection{4 Defect Checks}

\subsection{May '98}
\subsubsection{May 17, Commencement}

\end{document}








