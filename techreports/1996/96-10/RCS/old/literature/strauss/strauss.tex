\documentclass{article}

\begin{document}
Title: Software Inspection Process
Authors: Robert G. Ebenau, Susan H. Strauss
Bibcard reference: Strauss.bib

Summary:

Benefits and limitations of in-process inspections, and show where
inspections fit into an overall quality management program.  Describes the
inspection process, through its stages from preparation to follow-up,
including how to implement in-process inspections and how to keep them
running smoothly.  Includes a discussion of management of in-process
inspections and its impact on the success of an inspection program.  Also
described are techniques for the analysis of inspection data to evaluate
the product, the development processes, and the inspection procedures.
Includes an in-process inspection training program.


Notes:
\begin{enumerate}
\item Software Inspections were introduced by Michael Fagan while he was a
software development manager at IBM in 1972. (ix in preface)
\item Inspections are not an individual skill, but a coordinated group
effort concentrating on improving the quality of a project's
product. (x in preface)
\item Bell labs model (x-xi): 
  \begin{itemize}
  \item Focus on the project as the client
  \item Emphasis on universal inspection training
  \item Procedures for inspection management
  \item Existence of an inspection coordinator
  \item Tailoring inspections to a project's needs
  \item Inclusion of entry criteria
  \item Application of inspections to other areas
  \end{itemize}
\item In-process inspection is a result of the concept that time and delays
mean money.  In-process inspection operates on the idea that by putting
more effort into achieving quality during the early stages of a project,
you will save more time and have fewer delays than if you attempted to
obtain that quality at a later stage. (1)
\item 
  \begin{quote}
    Cease dependence on inspection to achieve quality.  Eliminate the need
    for inspection on a mass basis by building quality into the product
    in the first place 
  \end{quote} -Edwards Deming
\item In-Process inspection refers to the use of the inspection method
during the process of development, by examining intermediate
representations of the product for defects at each stage of their
preparation. The results of in-process inspections are analyzed and fed
back to control the production process to further reduce the occurrence of
defects - while the process is ongoing. (2)

\end{enumerate}

\end{document}
