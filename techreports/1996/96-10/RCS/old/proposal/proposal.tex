%%%%%%%%%%%%%%%%%%%%%%%%%%%%%%% -*- Mode: Latex -*- %%%%%%%%%%%%%%%%%%%%%%%%%%%%
%% proposal.tex -- 
%% Author          : Jennifer Geis
%% Created On      : Wed Sep 18 11:36:45 1996
%% Last Modified By: Jennifer Geis
%% Last Modified On: Wed Nov 13 08:58:04 1996
%% RCS: $Id$
%%%%%%%%%%%%%%%%%%%%%%%%%%%%%%%%%%%%%%%%%%%%%%%%%%%%%%%%%%%%%%%%%%%%%%%%%%%%%%%
%%   Copyright (C) 1996 Jennifer Geis
%%%%%%%%%%%%%%%%%%%%%%%%%%%%%%%%%%%%%%%%%%%%%%%%%%%%%%%%%%%%%%%%%%%%%%%%%%%%%%%
%% 



\documentstyle[nftimes,11pt,/group/csdl/tex/definemargins,
/group/csdl/tex/lmacros]{report} 
\input{/group/csdl/tex/psfig/psfig}

\begin{document}

\title{Kona: An experimental platform for automating FTR} \author{Jennifer
  Geis\\ Collaborative Software Development Laboratory,\\ Department of
  Information and Computer Sciences\\ 2565 The Mall\\ University of Hawaii,
  Manoa\\ Honolulu, Hawaii 96822\\ {\tt jgeis@uhics.ics.hawaii.edu}}
\maketitle

\tableofcontents

\chapter{Introduction}
It is my belief that an appropriately designed environment that integrates
Formal Technical Review(FTR) with the Personal Software Process(PSP) will
prove to be highly beneficial to its' users.

The Personal Software Process (PSP) consists of collecting and analyzing
data about the way you personally develop a program.  In other words, the
PSP aids in the analysis of your software development habits with the aim
of improving them.

According to studies done by Watts Humphrey, the creator of the PSP,
individuals who utilize the PSP properly can lower their defect rates,
shorten their compile and test times, and produce higher quality programs.

Formal Technical Review involves having a group of people inspect some
product (traditionally source code), with the intent of finding errors.
When done properly, FTR has been proven to result in products with fewer
defects.

It is my hypothesis that through the combination of these two systems,
users can reap benefits not obtainable by the enactment of one or both of
these systems separately.

Initial observations indicate that individuals doing the personal software
process tend to make many errors, some of which are serious enough to
nullify the user's efforts if not corrected.

By having a group of people perform FTR on another's PSP data, these
process errors are more likely to be found, thereby resulting in better PSP
data collection and analysis.  The emphasis here is that Formal Technical
Review is to be performed on the agent's process, not their product.

Other advantages to an integrated environment is that errors found in one
person's data can generate a check for the same error in other agent's data
through a type of cross-fertilization mechanism.  In this way, one
individual's experience benefits all other system users.

One final thing that I want to discuss is the possibility of changes to the
FTR process.  FTR is normally used in the context of reviewing a product
such as source code or a requirements document.  Applying FTR to the
process instead of the product could result in some changes to the way FTR
is conducted within this new environment.
\section{Thesis Statement} 
The thesis of this research is that an appropriately designed environment
that integrates PSP and FTR will provide benefits not attainable by either
or both systems utilized separately.
\section{Motivation}
\section{Objective/What I Hope to Learn}

\chapter{Related Work}
\section{The Personal Software Process}
\subsection{History}
\subsection{Concept}
\subsection{Method}
\subsection{Benefits}
\subsection{Previous Experiments/Case Studies}
\subsubsection{Humphreys' Students}
\section{Formal Technical Review}
\subsection{History}
\subsection{Concept}
\subsection{Method}
\subsection{Benefits}
\subsection{Previous Experiments/Case Studies}
\subsubsection{IBM}

\chapter{An Integrated Automated FTR/PSP Environment}
\section{Intent of the System}
\section{Benefits of an Integrated Environment}
\section{Architecture}
\section{Design}
\section{Issues}
\subsection{Personal Information}

\chapter{Case Study: ICS 414}
\section{Subject Matter}
\section{Participants/Class makeup}
\section{Time Frame/Duration}
\section{Initial Observations}
\subsection{Common Errors}
\subsection{What Helps/Hurts}
\subsection{More Forms!!!}
\section{Issues}
\section{Evaluation}   Data Analysis?
\section{Questionnaire}

\chapter{Research Plan}
\section{Personal Thesis Process (PTP)}
\end{document}


