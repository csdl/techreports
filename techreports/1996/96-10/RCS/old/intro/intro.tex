\documentclass{article}

\begin{document}
It is my belief that an appropriately designed environment that integrates
Formal Technical Review with the Personal Software Process will prove to be
highly beneficial to its' users.

The Personal Software Process (PSP) consists of collecting and analyzing
data about the way you personally develop a program.  In other words, the
PSP aids in the analysis of your software development habits.

According to studies done by the creator of the PSP, individuals who
utilize the PSP properly can lower their defect rates, shorten their
compile and test times, and produce higher quality programs.

Formal Technical Review involves having a group of people inspect some
product (traditionally source code), with the intent of finding errors.

Formal Technical Review, when done properly, has been proven to result in
products with fewer defects.

It is my hypothesis that through the combination of these two systems,
users can reap benefits not obtainable by the enactment of one or both of
these systems separately.

Initial observations indicate that individuals doing the personal software process
tend to make many errors, some of which are serious enough to nullify the
agent's errors if not corrected.

By having a group of people perform FTR on another's PSP data, these
process errors are more likely to be found resulting in better PSP data
collection and analysis.  The emphasis here is that Formal Technical Review
is to be performed on the agent's process, not their product.

Other advantages to an integrated environment is that errors found in one
person's data can generate a check for the same error in other agent's data
through a type of cross-fertilization mechanism.  In this way, one
individual's experience benefits all other system users.

The thesis of this research is that an appropriately designed environment
that integrates PSP and FTR will provide benefits not attainable by either
or both systems utilized separately.

\end{document}
