%%%%%%%%%%%%%%%%%%%%%%%%%%%%%% -*- Mode: Latex -*- %%%%%%%%%%%%%%%%%%%%%%%%%%%%
%% TekInspectv1SRS.tex -- 
%% Author          : Carleton Moore
%% Created On      : Tue Nov 26 15:42:53 1996
%% Last Modified By: Carleton Moore
%% Last Modified On: Tue Nov 26 16:21:27 1996
%% RCS: $Id: TekInspectv1SRS.tex,v 1.1 1996/11/27 02:25:27 cmoore Exp $
%%%%%%%%%%%%%%%%%%%%%%%%%%%%%%%%%%%%%%%%%%%%%%%%%%%%%%%%%%%%%%%%%%%%%%%%%%%%%%%
%%   Copyright (C) 1996 Carleton Moore
%%%%%%%%%%%%%%%%%%%%%%%%%%%%%%%%%%%%%%%%%%%%%%%%%%%%%%%%%%%%%%%%%%%%%%%%%%%%%%%
%% 

\documentstyle[nftimes,11pt,/group/csdl/tex/definemargins,
/group/csdl/tex/lmacros]{article} 
\input{/group/csdl/tex/psfig/psfig}

\begin{document}

%%%%%%%%%%%%%%%%%%%%%%%%%%%%%% -*- Mode: Latex -*- %%%%%%%%%%%%%%%%%%%%%%%%%%%%
%% titlepage.tex -- 
%% Author          : Carleton Moore
%% Created On      : Fri Oct 18 16:18:14 1996
%% Last Modified By: Carleton Moore
%% Last Modified On: Thu Nov  7 16:14:03 1996
%% RCS: $Id: titlepage.tex,v 1.3 1996/11/08 02:14:23 cmoore Exp $
%%%%%%%%%%%%%%%%%%%%%%%%%%%%%%%%%%%%%%%%%%%%%%%%%%%%%%%%%%%%%%%%%%%%%%%%%%%%%%%
%%   Copyright (C) 1996 Carleton Moore
%%%%%%%%%%%%%%%%%%%%%%%%%%%%%%%%%%%%%%%%%%%%%%%%%%%%%%%%%%%%%%%%%%%%%%%%%%%%%%%
%% 

\begin{titlepage}
\begin{center}
  
  \vspace{1.5in}
  
  {\Large\bf AFTR:  An Automated Formal Technical Review System}
  
  \vspace{0.5in}
  
  {\large\bf A Ph.D. Dissertation Proposal}
  
  \vspace{0.5in}
  
  
  by   
  
  \vspace{.2in}
  
  {\large  Cam Moore}
  
  \vspace{4.5in}
 
  \today
%  October, 1996
  
  \vfill
  
  {\sc Department of Communication and Information Sciences
  
  University of Hawaii at Manoa}
\end{center}
\end{titlepage}


\section{Introduction}

\subsection{Purpose of this document}
Describes the purpose of the document, and the intended audience.

\subsection{Scope of this document}
Describes the scope of this requirements definition effort. Describes who
was involved in the requirements elicitation effort.

This section also details any constraints that were placed upon the
requirements elicitation process, such as schedules, costs, or people.

\subsection{Overview}
Provides a brief overview of the system defined as a result of the
requirements elicitation process.

\subsection{Organizational Context}
Provides an overview of the business organization sponsoring the
development of this system. This overview should provide insight into the
organizational context relevant to the development of this project. Why is
the organization interested? What does it intend to get from this project?

\section{General Description}

\subsection{System Functionality}
Introduces the general functionality of the system.

\subsection{Similar Systems}
Describes the relationship of this system with any other systems. Specifies
if this system is intended to be stand-alone, or else used as a component
of a larger system. If the latter, this section discusses the relationship
of this system to the larger system.

\subsection{User Characteristics}
Describes the features of the user community, including their expected
expertise with software in general and the application domain in
particular.

\subsection{User Objectives}
This section describes the set of objectives and requirements for the
system from the user's perspective. It may include a "wish list" of
desirable characteristics, along with more feasible solutions that are in
line with the business objectives.

\section{Operational Scenarios}

This section should describe a set of scenarios that illustrate, from the
user's perspective, what will be experienced when utilizing the system
under various situations.

In the article Inquiry-Based Requirements Analysis (IEEE Software, March
1994), scenarios are defined as follows:

In the broad sense, a scenario is simply a proposed specific use of the
system. More specifically, a scenario is a description of one or more
end-to-end transactions involving the required system and its environment.
Scenarios can be documented in different ways, depending up on the level of
detail needed. The simplest form is a use case, which consists merely of a
short description with a number attached.  More detailed forms are called
scripts. These are usually represented as tables or diagrams and involved
identifying an action and the agent (doer) of the action. For this reason,
a script can also be called an action table.

Although scenarios are useful in acquiring and validating requirements,
they are not themselves requirements, because the describe the system's
behavior only in specific situations; a specification, on the other hand,
describes what the system should do in general.

A sufficient number of scenarios should be provided such that the
end-to-end use of the system under normal, abnormal, and "interesting"
conditions is illustrated.

\section{Functional Requirements}

This section lists the functional requirements in ranked order. Functional
requirements describes the possible effects of a software system, in other
words, what the system must accomplish. Other kinds of requirements (such
as interface requirements, performance requirements, or reliability
requirements) describe how the system accomplishes its functional
requirements. Each functional requirement should be specified in a format
similar to the following:

      Short, imperative sentence stating highest ranked functional requirement.

\begin{enumerate}
      
    \item{\em Description} A full description of the requirement.
      
    \item{\em Technical issues} Describes any design or implementation issues
      involved in satisfying this requirement.
      
    \item{\em Risks} Describes the circumstances under which this requirement
      might not able to be satisfied, and what actions can be taken to
      reduce the probability of this occurrence.
      
    \item{\em Dependencies} Describes interactions with other requirements.

\end{enumerate}

      Next requirement... 

\section{Interface Requirements}

This section describes how the system interfaces with other software
systems and its users for input and output.

\subsection{User Interface}
Describes how this system interfaces with the user. Should include screen
shots, etc.

\subsection{API Interface}
Describes how this system interfaces with other tools. Can cross reference
the relevant conceptual design classes listed below.

\section{Persistant Data Requirements}

This section specifies the internal structure and interrelationships
between any persistant data that is written or read by the system. This can
be described as a set of persistant classes, and reference the relevant
conceptual design classes listed below.

\section{Conceptual Design}

This section presents a description of the classes and methods that will be
required to implement the system. It corresponds to the JavaDoc
documentation for a conceptual design.

Class Defect:
Class ScribeSheet:
Class ReviewerSheet:

\section{Preliminary Schedule}

This section provides an initial version of the project plan, including the
major tasks to be accomplished, their interdependencies, and their
tentative start/stop dates.

\subsection{Milestones}
\begin{table}[htbp]
  \caption{System Milestones}
  
  \begin{center}
    \begin{tabular}{|l|l|}\hline
      {\bf Task}&{\bf Dates / Milestones}\\ \hline \hline
      SRS Draft&13 Dec 1996\\ \hline
      Release System& Feb 1997\\ \hline
    \end{tabular}
  \end{center}
\end{table}


\end{document}