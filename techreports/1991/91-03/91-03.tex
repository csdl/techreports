\documentstyle [11pt,/usr3/ics/johnson/Tex/definemargins,
                     /usr3/ics/johnson/Tex/lmacros,
                     /usr3/ics/johnson/Tex/functiondoc]{article}  

\begin{document}
\definemargins{1.0in}{1.0in}{1.0in}{1.0in}{0.3in}{0.3in}
\pagestyle{empty}

\begin{center}
{\Large\bf The Egret Project:}

{\Large\bf Exploring Open, Evolutionary, and Emergent}

{\Large\bf Collaborative Systems}

\bigskip

Philip Johnson\\
Department of Information and Computer Sciences\\
University of Hawaii\\
Honolulu, HI 96822\\
(808) 956-3489\\
{\tt johnson@uhics.ics.hawaii.edu}

\bigskip

{\bf CSDL-TR-91-03}

\bigskip

{\em In Proceedings of the 1991 European Conference on Computer
Supported Cooperative Work, Developer's Workshop.}

\end{center}

\ls{1.1}
\section*{Introduction}

The EGRET Project at the University of Hawaii is pursuing a research
program designed to investigate evolution in collaborative systems.
Evolution is of central concern due to the exploratory nature of many
CSCW application areas, including software development, document
preparation, issue generation and discussion, and so forth.  In
exploratory domains, the ill-defined or changing nature of both the
problem and an acceptable solution to it impose special demands.  For
example, the criteria for an acceptable solution, or the solution
generation method may be an an emergent property of the exploratory
process.

To accomodate these exploratory, emergent properties of CSCW
applications, computation support environments must not only support
problem solving in a specific domain, but also evolutionary processes,
as the problem solution, the process, and the domain itself undergoes
change.  To make progress on this problem, our research paradigm has two
general phases.  In one phase, we construct instrumented collaborative
support environments that allow us to observe and analyze evolution in
CSCW application domains.  From this data, we change the support
environment to better support the evolutionary phenomena, as well as
to improve our instrumentation for observing it.  In this manner we
are gradually bootstrapping our understanding of evolutionary,
emergent collaborative support environment.  This paradigm and our
approach in general is influenced by our previous work in software
structure evolution and exploratory development
\cite{Johnson89b,Johnson90,Johnson91}, as well as by other research 
on evolution crossing several fields of study
(for example, \cite{Salthe85,Skarra87,Lehman85,Beer85}). 

We believe that understanding emergent and evolutionary properties are
key to the development of open CSCW systems.  In this position paper,
we briefly review some of the important architectural features of our
approach to collaborative support environments, and describe our
current research status and plans.

\section*{Supporting Evolution in Collaboration}

In the Egret Project, we have identified the need to provide support
for two kinds of evolution, which we term {\em syntactic evolution}
and {\em semantic evolution}\foot{These terms were chosen for their
suggestive, figurative nature; we do not wish to rekindle a fruitless
syntax-vs-semantics debate.}.

Syntactic evolution refers to change in the structure of the
representation language used by collaborators.  Our environments
support a traditional collaborative architecture where developers post
typed {\sl nodes} containing information, and relate nodes to each
other through typed {\sl links}.  Research on systems such as gIBIS
\cite{Conklin88} and SIBYL \cite{Lee90} can be characterized in some
sense as ``the search for the `right' representation''---the optimal
set of node and link types for a particular application domain.  While
we feel this research approach is valid and provides useful
information, our emphasis on exploratory environments compels us to
view the `right' representation as an emergent and volatile phenomena
that will evolve along with the group composition, problem domain, and
process history.

In the Egret Project, therefore, we are investigating how to support
the collaborative process not only at the level of the application
domain, but also at the level of the representation used for
collaboration.  Structural support for syntactic evolution requires
facilities for specialization, generalization, fission, or fusion of
node or link types.  Impact analysis mechanisms are also required to
help understand the ramifications of change to the representation.
Finally, support for syntactic evolution allows collaborators to
dynamically extend and optimize query-based navigation over the
collaborative network, since node and instance types can be
incrementally specialized to provide more fine-grained retrieval.

Semantic evolution refers to change in the content and organization of
the node and link instance structure, rather than to change in
the types of nodes and links possible in the conversation.  Semantic
evolution is required in collaborative design work in order to fight
a general phenomena we think of as the ``historical decrease in signal-to-noise
ratio.''   Simply put, the percentage of noise in a collaborative network
tends to increase with the age of the network, and must be combatted through
systematic instance-level maintenance activities such as restructuring,
garbage collection, and so forth. 

We aggregate sets of nodes and links together into composite entities 
called {\sl layers}.  Currently, we are designing two
fundamental flavors of layer--- {\sl revision} layers and {\sl meta}
layers.  

Revision layers allow re-organization of the collaborative network by
replacing pieces of the collaborative network with new pieces.  For
example, in a collaborative authoring scenario, one person might post
a node containing a draft of some portion of the paper, to which other
annotation and criticism nodes might be linked.  At an appropriate
point, the author may create a new revision layer with a single node
containing a new draft that incorporates some subset of the comments
posted at the original layer.  By producing a revision layer, the
author can clearly indicate which parts of the collaborative network
he has attempted to incorporate into the revision and which parts he
has not.  

Structural support is also needed for the process by which a decision
to produce a revision is reached, and the scope and content of the
revision.  This support is provided through meta layers.  Meta layers
provide an orthogonal collaborative ``space'' in which to discuss the
organization, content, and process of the layer below the meta layer.
Continuing our example, after posting the draft section and receiving
comments about it, but before posting the revision, the original
author might introduce a meta-level conversation to indicate his
intention to produce a revision and projected scope and content.  This
intent could then be debated by the other collaborators---perhaps
another person was in the midst of producing a comment and wanted the
original author to hold off on posting the revision until the
comment-in-progress was completed.  Besides supporting this semantic
evolution of the network, meta layers also provide a mechanism for
syntactic evolution---i.e. the ability to discuss the current set of
node, link, and layer types and propose extensions or modifications to
them.  Thus, the architectural facilities for syntactic and semantic
evolution are not orthogonal, but rather entertwined and synergistic.

Layers can themselves be organized into composites called {\sl
surfaces}.  Surfaces allow collaborators to group together a coherent
set of revisions, similar in intent to configurations in source code
control systems.  We choose the more spatial terms of layer and
surface to reflect our work on a three dimensional user
interface for the Egret architecture, noted below.


\section*{Current Research Activities}

The Egret Project is currently attacking the issues identified above
on a number of fronts.  Earlier this year, we finished our
first prototype collaborative support environment called Plover.
This relatively simple environment was intended primarily
to get us acquainted with our current platform---X windows and an
Emacs/Epoch interface on top of Unix.  

We are now implementing the second version of a more sophisticated
collaborative support environment, called CoReView. This environment is
designed to support a collaborative approach to literature review and
analysis.  It is scheduled to be used in a graduate seminar on CSCW
in the Spring of 1992, where it will provide
computational support for group analysis and discussion both inside
and outside the seminar boundaries.  CoReView is being built using a
preliminary version of the Egret structural representation language,
which is designed to support syntactic evolution. We are using 
HyperBase \cite{Wiil90} as a back-end for concurrency control.  CoReView is
instrumented to collect data about evolutionary processes.

We are currently exploring presentation issues of layers and
surfaces in a research project on three dimensional interfaces for
collaborative support environments.  In a prototype system now under
development using AutoCAD, we are investigating new forms of
navigation strategies and cognitive orientation models in a simulated
three-dimensional space.  We intend to reimplement this system in 
an X windows environment after gaining some initial experience with the 
prototype.


\subsection*{Acknowledgements}

The extensive contributions of Dadong Wan, Danu Tjahjono, and Ram
Hariharan to the Egret project are gratefully
acknowledged.

\bibliography{91-03}
\bibliographystyle{alpha}

 
\end{document}







