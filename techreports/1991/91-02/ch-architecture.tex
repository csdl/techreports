\chapter{Architectural Overview}
\section{Introduction}

The top-level architecture of Egret  currently consists of the following
major subsystems:  Utilities, Server, Type, and Interface.  Each of these
are briefly introduced below, and detailed in subsequent sections.

\begin{itemizenoindent}
  
\item {\bf Utilities.} The utilities subsystem provides functions
  that are useful to classes across all other subsystems. Object classes
defined in this
  module typically represent extensions of Emacs Lisp built-in constructs.
  For example, {\sf u*hook} and {\sf u*error} are built directly on top
  of the Emacs error and hook facilities. {\sf u*hash} extend Emacs Lisp
  to support hashtables in a similar manner to that provided by Common
  Lisp. 
  
\item {\bf Server.} The server subsystem serves as the essential platform for
Egret.
   Its classes provide an Emacs interface specialized to the particular
database server.  For example, the server subsystem currently implements
  {\sf s*process-server}, {\sf s*node}, and {\sf s*link} classes whose
  structure and behavior are specialized for use with the hbserver
  database server application.  In general, the server subsystem encapsulates
details of network access and low-level
  retrieval operations from the other subsystems, and provides
  operations implementing a persistant, fixed, monotyped, lockable
  record structure.
  
\item {\bf Type.} This subsystem uses the server subsystem's node
  and link definition facilities to implement ECTS, the exploratory
collaborative type system.  As noted previously and in other 
documents \cite{csdl-92-01},  ECTS
  supports exploration and consolidation of structure by groups of 
developers.   The type level provides its own node and link classes,
thus ``shadowing'' the node and link classes at the server level.
  
\item {\bf Interface.} This subsystem provides various forms of input
  and output to and from the system.  Textual and graphical interfaces
  are current targets for interface-level functionality.
  
  The interface level also contains domain-specific specializations for
  collaborative object oriented design, collaborative research,
  collaborative code inspection, and so forth.  These specializations
  are implemented in terms of calls to operations of the type subsystem
  to define domain-specific types, calls to operations of the type
  subsystem to define domain-specific aggregations, as well as
  functionality defined in this level for traversal, creation,
  modification, and communication in a domain-specific fashion.

\end{itemizenoindent}

One motivation for this top-level decomposition is to isolate those
parts of the system that appear most volatile, while still providing
high performance.  For example, the use of the hbserver application as
the underlying database server may be reconsidered in the near future
if a high quality object oriented database mechanism becomes
available. Such a database change will obviate the need for some of
the type subsystem mechanisms.  Similarly, current interface choices
and domain specializations will evolve over time.

The following sections provide more detail about each of these subsystems. 

\section{Utilities Subsystem}

\subsection{Services Provided}

The utility subsystem provides a collection of general utilities to be used
by other Egret modules. Similar to the server subsystem which extends the
functionality of the remote database server, the utility subsystem adds
richness to the Emacs Lisp environment. For example, Elisp does not have
built-in support for hashtable, which is available in other environments
such as  Common Lisp.

This concept is further abstracted in the class {\it table\/}, which 
implements both hashtable and alist structures, and which allows the user to
choose between the two based on the type of data to be stored and the type of
operations to be performed.  


\subsection{Component Classes}

One unique feature of the utility subsystem is that its object classes are
relatively independent. For example, {\it error \/} is totally orthogonal to {\it
hook\/}.   Membership in the utility class is determined simply by the
generality of the class and broad  applicability, rather than by
inter-relationships among the member classes themselves.

The utility subsystem currently consists of four classes: {\sf
u*hashtable}, {\sf u*table}, {\sf u*error}, and {\sf u*hook}. 

{\sf U*hashtable} provides the classic random-access data structure. It is
implemented using obarray and Elisp symbol primitive operations {\it
intern\/} and {\it intern-soft\/}, since intern and obarrays are internally
implemented through a hashtable algorithm.  The resulting hashtable
mechanism is not as general as that in Common Lisp, for example, but is useful
nonetheless.





The {\sf u*table} class implements an abstract table structure
for which the user can choose between either a hashtable or 
an association list as the internal representation.

An association list is not appropriate for  large tables requiring quick lookup,
since it uses linear search. On the other hand, it does possess certain nice
properties which the hashtable mechanism does not have.  For example,
association lists can straightforwardly implement both forward and reverse
matching  through {\tt assoc} and  {\tt rassoc}. Hashtables, as implemented
through obarrays, can only permit one-way (i.e., key to data) lookup.

Typically, a user wishing to build a large table with quick lookup will choose
the hashtable representation, while those requiring reverse access will
choose the alist representation.  It is also possible to  obtain quick lookup for
large tables and reverse access when specifying the hashtable
representation.  In this case, an auxiliary list of  value to key mappings is
maintained in addition to the hashtable  obarray.  Since this introduces
significant overhead in terms of additional space, it should only be chosen
when the need for quick lookup on a large table and reverse access outweighs
the additional space required.

Both {\sf u*error} and {\sf u*hook} are extensions of built-in Elisp
constructs. Like their Elisp counterparts, error objects in Egret are
represented as symbols, and their attributes as property lists.  The Egret
extensions  involve mainly additions of new operations, including tracking
error objects defined,  and executing a list of forms with error protection.
Similarly, extensions to the built-in hook facility include support for ordering
functions on a given hook.


\section{Server Subsystem}

\subsection{Services Provided}

The server subsystem interfaces to as well as extends the functionality of
the remote database server. As an interface, it acts like the server
itself, allowing transparent access to the functions of the remote server.
As an extension, the server implements services to improve the performance
via local caching of the remote data.   The major functions of the 
server subsystem are:
\begin{itemize}
\item  Persistent data storage, retrieval and update. 
  
\item ``Atomic'' node and link objects. The server subsystem implements
  primitive entities for data storage (nodes) and relations between them
  (links).

\item ``System'' node objects with specialized semantics for storage
and retrieval of shared global state information.
  
\item An event-based mechanism for two-way communication 
  between the local clients and the global database server mechanism.
  The server subsystem relies on the event mechanism to communicate with
the remote server, to synchronize the local and global states, and to control
concurrent access to the centralized database by multiple users.

\item Session management, e.g., connect and disconnect, and local
  structure initialization; etc.
\end{itemize}

A key requirement of the server subsystem is it make no assumptions about
the kind of information being stored in nodes, the semantics of the
relationship represented by a link between two nodes, or the meaning of an
event. For example, while it is the server subsystem's responsibility to
install all event handlers properly and make sure they are invoked in the right
order, the server does not know what the effect of these event handlers will
be to the local or global state.  For the same reason, at the server subsystem
level, the contents of fields are uniformly represented as either strings or
integers, and no assumptions are made by the server operations about the
format or structure of stored data. 

Currently, the server subsystem implements an interface to one database
server application---the HyperBase system from the University of Aalborg.
The internal structure of the server subsystem is therefore somewhat
specialized to this application. 

\subsection{Component Classes}

The server subsystem consists of six classes: {\sf s*node}, {\sf s*link}, {\sf
s*server-process}, {\sf s*snode}, {\sf s*serror}, and {\sf s*hook}. The {\sf
s*node} and {\sf s*link} classes provide a set of primitive operations on
nodes and links. {\sf S*server-process } maintains information on individual
servers, and manages sessions and synchronizes the states of the remote and
local data. {\sf S*snode} provides operations on a set of special ``system''
nodes used primarily to implement  shared global data structures. The {\sf
s*serror }  and  {\sf s*hook} classes aggregate together all error and hook
instances in the Server subsystem.

The structure of the node and link classes follows directly from the fixed (at
compilation time) structure of nodes and links in the Hyperbase. For example,
each field in a Hyperbase node corresponds to an attribute of the
{\sf s*node} class.  Operations on  instances of {\sf s*node} objects
support the retrieval, setting, and locking of Hyperbase nodes. 

A drawback of this design is the fact that the class-level structure
of the server subsystem must change in the event that the fixed
structure of nodes, links, or events changes.  This can also be viewed
as a benefit, in that it clearly reifies within the design the obvious
dependencies that the structure of the system as a whole has upon the
atomic, internal structure of the server application.  

The {\sf s*server-process} class provides attributes helpful in identifying the
database being connected to, the location of the data files, and so forth.

Retrieval in the server subsystem is accomplished purely on a
field-by-field basis; no aggregate retrieval operations for multiple
node fields are provided.  This eliminates the need to provide local
caching of node state within the server layer, and makes performance
considerations purely the responsibility of higher subsystems.  Upper
subsystems must balance the benefits of aggregating information into
the DATA field (in order to retrieve multiple items of information
with one network access) against the costs of unpacking each item from
the DATA field (which may become significant when a single item of
information in the DATA field is required on a frequent basis, such as
during a lengthy traversal of the database).  Since this cost-benefit
analysis is domain-specific, it should be resolved in a higher level,
more domain-specific subsystem.

\section{Type Subsystem}

\subsection{Services Provided}

The type subsystem implements a typed, flexibly structured data
model called ECTS on top of the monotyped, fixed structure data model
provided by the server subsystem. 

ECTS departs from traditional type systems by allowing first class
definition of fields at the instance as well as at the class level.
Traditionally, the set of fields associated with any instance is
determined by its type. In other words, the type specification serves
as a template for creating instances, and all instances are
constrained to exactly the structure indicated by the type
specification.  While ECTS class definition facilities do provide a
template for instance creation, instances are not subsequently
constrained to that structure. Instead, individual instances of typed
nodes can dynamically define new fields, or delete some of its
original fields. More about ECTS can be found in Appendix \ref{ects-dr}.

To make ECTS work in a collaborative context, impact analysis and other
mechanisms must also be provided in order to control the proliferation of
instance-level variants. The current impact analysis mechanisms consist of
metrics of convergence and divergence.

To support the more mundane aspects of exploratory development, the
type subsystem is constructed to facilitate ease in renaming or
modifying other attributes of of the constituant objects (i.e. nodes,
links, fields, and schemas).  Thus, node contents always store
references to attributes of links, fields, and schemas
symbolically---by their corresponding ID.  After retrieving a node,
the local client has the responsibility of mapping the ID to the
corresponding name.  This allows changes to field or link names to be
instantly reflected in all nodes referencing them without requiring a
traversal of the entire database looking for instances and textually
changing them.  The cost is that local tables for node, schema, field,
and link instances that associates their ID with all relevent
attributes must be maintained.

\subsection{Component Classes}

The type subsystem has classes defining node and link schemas, node and
link instances, fields, layers, errors, and events.  Each of these are introduced
briefly below.

The {\sf t*node-schema} class defines the consensually 
agreed upon structural features (i.e. the set of fields)
for a set of node instances.  However, the node instances in
this set may not necessarily conform to these structural 
features.   To assess how far the instances of a node-schema
are drifting from their consensual structure, the class defines
the operation  {\bf t*node-schema*divergence}. The {\sf t*link-schema}
class performs an analogous function for links.

The class {\sf t*field-schema} identifies and defines the internal structural
components of node instances.  The internal structure and value of a field can be
assessed and enforced via a user-specified validity function.

While type-level nodes are logically composed of distinct fields,
there is not a one-to-one correspondence between the hyperbase (or
server subsystem) node fields and the type-level node fields.  This is
because at the hyperbase (and server) level, nodes are composed of a
fixed number of fields, while at the type level, nodes are composed of
a variable number of fields that may vary at run-time.  To resolve
this, the type-level ``packs'' its fields into the single
distinguished variable length {\tt data} field of the hyperbase.  This
leads to an interesting tension between conceptual clarity and
efficiency that is discussed in Appendix \ref{type-dr}.  We now
overview the manner in which several type-level fields are packed into
a single hyperbase node.

Fields are packed by delimiting their contents with a marker
containing the field schema ID. For example, if a ``Name" field has the
field-schema-ID 37, and its value is ``Foo", then its packed
representation is:

\small\begin{verbatim}
<<## 37 ##>>Foo<<## 37 ##>>
\end{verbatim}\normalsize

Note that presentational issues, such as the labelling
of the field, or its font or color are determined by
reference to the associated field-schema.

Links are packed similarly, except that the link-ID
and destination node-ID are embedded in the marker as
well as the link-schema-ID. Thus,

\small\begin{verbatim}
<<$$ 62 128 34 $$>> 
\end{verbatim}\normalsize

represents an occurrence of the link-ID 62, associated
with the link-schema 128, and pointing to the node-ID
34.  

The following example illustrates a typical use of links
within fields:

\small\begin{verbatim}
Description: See [-> more-info]. 
\end{verbatim}\normalsize

This ``Description" field is packed as follows:

\small\begin{verbatim}
<<## 23 ##>>See <<$$ 12 35 98 $$>>.<<## 23 ##>>
\end{verbatim}\normalsize

Here, 23 is the field-schema-ID for the Description
field, 12 is the link-ID for the link called more-info,
35 is its link-schema-ID, and 98 is the node-ID pointed
to by more-info.

The {\sf t*layer} class implements an aggregation mechanism for representing
and supporting the exploration and consolidation phases of exploratory
collaboration. The contents of each layer instance contains links to each
member node and link instance contained within it.\foot{Currently, we believe
that maintaining schema and field information outside of layers leads to greater
clarity in understanding exploratory phenomena.} Layer instances also maintain
relationships to prior and subsequently defined layers.  The operations upon a
layer include computations of convergence and divergence for the layer.

Layers do not support functional partitions of the database, nor do
they support versioning of particular nodes, fields, or schemas.
Their sole purpose is to reflect the cyclic periods of exploration
followed by consolidation.  Thus, they tend to partition the database
into something like a temporally ordered set of ``onion skins", where
each onion skin reflects a single period of exploration followed by
consolidation for a particular portion of the database.

Individual node, field, and schema instances can be members of more than
one layer.  This occurs when an instance is both actively used and 
structurally stable, so that it is relevent across multiple phases
of exploration and consolidation.

This concludes the overview of the architectural features of Egret. 
Part \ref{detailed-design} presents a snapshot of the Designbase, the
on-line design specification for Egret.   The final part of this document
contains a snapshot of the currently defined test cases.
