
\chapter{Utilities}
\label{Utilities}

\begin{description}
\item [Name:]  Utilities

\item [Description:]
This subsystem contains generic classes and utilities for the
entire Egret environment.  Object classes and their operations
defined in this module are available for use and
specialization by all other subsystems. Due to the general
nature of this subsystem, it should always be loaded before
all other subsystems.

\item [Public-classes:]
\item {\sl u*hash}\hfill(page~\pageref{u*hash})
\item {\sl u*table}\hfill(page~\pageref{u*table})
\item {\sl u*error}\hfill(page~\pageref{u*error})
\item {\sl u*hook}\hfill(page~\pageref{u*hook})




\end{description}
\horizontalline

\section{u*hash}
\label{u*hash}

\begin{description}
\item [Name:]  u*hash

\item [Layer:]
{\sl Utilities}\hfill(page~\pageref{Utilities})

\item [Description:]

U*HASH is an implementation of hashtable in Elisp.
Hashtable is a random access data structure that allows
flexible and fast retrieval.

U*HASH table is implemented as obarrays. Each element of
U*HASH can be arbitrary Lisp object, such as symbols,
structs. Though hashtable typically allows only one-way
mapping, i.e., from key to data, U*HASH provides
restricted reverse searching, i.e., from data to key. It
does so via maintaining an auxiliary alist of all
elements in the table.

\item [Attributes:]

\item [Operations:]
\item {\sl u*hash*make}\hfill(page~\pageref{u*hash*make})
\item {\sl u*hash*get}\hfill(page~\pageref{u*hash*get})
\item {\sl u*hash*set}\hfill(page~\pageref{u*hash*set})
\item {\sl u*hash*rem}\hfill(page~\pageref{u*hash*rem})
\item {\sl u*hash*map}\hfill(page~\pageref{u*hash*map})

\item [Collections:]

\item [Subclasses:]

\item [Superclasses:]

\item [Instances:]



\end{description}
\horizontalline

\subsection{u*hash*make}
\label{u*hash*make}

\begin{description}
\item [Name:]  u*hash*make

\item [Class:]
{\sl u*hash}\hfill(page~\pageref{u*hash})

\item [Parameters:]
\item {\sl size}:  integer (prime number)


\item [Return-value:] obarray of required size with all cells set
to 0.   

\item [Description:]

Creates an obarray as hashtable. SIZE, if supplied, should
be a prime number. Otherwise, defaults to default-size,
i.e., 511. 

\item [Public:]



\end{description}
\horizontalline

\subsection{u*hash*get}
\label{u*hash*get}

\begin{description}
\item [Name:]  u*hash*get

\item [Class:]
{\sl u*hash}\hfill(page~\pageref{u*hash})

\item [Parameters:]
\item {\sl hash-key}:  symbol

\item {\sl table}:  symbol


\item [Return-value:] 
data element of the table entry, or NIL if not found.

\item [Description:]
Retrieves data element from hashtable TABLE using
HASH-KEY as key.

\item [Public:]



\end{description}
\horizontalline

\subsection{u*hash*set}
\label{u*hash*set}

\begin{description}
\item [Name:]  u*hash*set

\item [Class:]
{\sl u*hash}\hfill(page~\pageref{u*hash})

\item [Parameters:]
\item {\sl hash-key}:  symbol

\item {\sl data}:  any Lisp object

\item {\sl table}:  symbol


\item [Return-value:] new data value

\item [Description:]
Stores DATA in table TABLE using HASH-KEY as key. 

\item [Public:]



\end{description}
\horizontalline

\subsection{u*hash*rem}
\label{u*hash*rem}

\begin{description}
\item [Name:]  u*hash*rem

\item [Class:]
{\sl u*hash}\hfill(page~\pageref{u*hash})

\item [Parameters:]
\item {\sl hash-key}:  symbol

\item {\sl table}:  symbol


\item [Return-value:]  t if deletion is successful or
nil otherwise.

\item [Description:]
Removes data element corresponding to key HASH-KEY from
hashtable TABLE. 

\item [Public:]



\end{description}
\horizontalline

\subsection{u*hash*map}
\label{u*hash*map}

\begin{description}
\item [Name:]  u*hash*map

\item [Class:]
{\sl u*hash}\hfill(page~\pageref{u*hash})

\item [Parameters:]
\item {\sl table}:  symbol

\item {\sl fn}:  function symbol


\item [Return-value:] returns nil.

\item [Description:]
Calls FN on each entry in hashtable TABLE. FN must
accept two arguments: a key and its associated value. 

\item [Public:]



\end{description}
\horizontalline

\section{u*table}
\label{u*table}

\begin{description}
\item [Name:]  u*table

\item [Layer:] {\sl Utilities}\hfill(page~\pageref{Utilities})

\item [Description:]

U*TABLE is a generic class that consists of paired objects: KEY
and DATA: KEY is either string or symbol, and DATA, any lisp
object.  U*TABLE is typically used to store data for quick
lookup. For example, the Server subsystem uses this class
extensively as cache structures for storing selected node and
link attributes so that lookup operations can be done
efficiently (i.e., no remote access. 

U*TABLE supports both hashtable and alist. For a given table,
one can specify whether to use hashtable or alist, depending on
the requirement, e.g., size, operation types. The default is
alist.

\item [Attributes:]

\item [Operations:]
\item {\sl u*table*define}\hfill(page~\pageref{u*table*define})

\item [Collections:]

\item [Subclasses:]

\item [Superclasses:]



\end{description}
\horizontalline

\subsection{u*table*define}
\label{u*table*define}

\begin{description}
\item [Name:]  u*table*define

\item [Class:]
{\sl u*table}\hfill(page~\pageref{u*table})

\item [Parameters:]
\item {\sl table-name}:  symbol

\item {\sl fn-prefix}:  string

\item {\sl rlookup-fn}:  slot accessor function

\item {\sl table-size}:  integer (prime number)

\item {\sl hashp}:  Boolean 

\item {\sl init-value}:  list of Lisp objects.


\item [Return-value:] macro with no interesting return value.

\item [Description:]

Defines a table called TABLE-NAME with supplied parameter
constraints. Table operations include PUT, GET, DELETE,
GET-KEY, GET-COMPLETION-LIST, and INITIALIZE. They are prefixed with
FN-PREFIX.

Keyword TABLE-NAME is an optional symbol indicating the name of
table. When nil, defaults to the global table.  
Keyword FN-PREFIX is a required string used as prefix of table
operations, e.g., s!node*.
Keyword HASHP is a Boolean indicating if the table should be a
hashtable or not.  Using hashtable if t or alist otherwise.
Keyword RLOOKUP-FN is an optional arg indicating DATA slot on
which reverse lookup might be done. Currently, reverse lookup is
restricted to only one slot.
Keyword TABLE-SIZE is an optional prime number indicating the
size of hashtable. Used only when HASHP is t.
Keyworkd COMPLETION-FN is an optional arg indicating DATA slot on
which completion list will be maintained. Currently, completion can
only done on one data slot.
Keyword INITIAL-VALUES is a set of initial values for the table.
It must be provided as an alist whose elements are in the form
of (KEY . DATA)

U*table*define implicitly defines the following operations (each of
which will be prefixed with FN-PREFIX):

  PUT (key data) ={\tt>} data 
    Adds key/data pair to the table.

  GET (key) ={\tt>} data
    Returns data corresponding to key in table, or NIL if not found.

  DELETE (key) ={\tt>} T or NIL.
    Returns T if key was deleted from table, NIL if not found.

  GET-COMPLETION-LIST () ={\tt>} list
   Returns the completion list maintained with this table; requiring
aux-table.

  GET-KEY (data) ={\tt>} KEY
   Returns key that matches data; requiring aux-table. 

  MAP (fn) ={\tt>} NIL
    Maps through table, calling FN with each key and its associated
    value.

\item [Public:]



\end{description}
\horizontalline

\section{u*error}
\label{u*error}

\begin{description}
\item [Name:]  u*error

\item [Layer:] {\sl Utilities}\hfill(page~\pageref{Utilities})

\item [Description:]

U*ERROR is a global error class which subjects to specialization in
each subsystem. In other words, each subsystem has its own error
class, which is a subclass of U*ERROR. Inevitably, they inherits all
attributes and operations of U*ERROR. These error subclasses normally
add no new attributes and operations. Instead, they contain explicit
links to to all error instances that belong to that subclass.

U*ERROR follows the same structure as that of the Emacs Lisp error
object, i.e., each error object contains three attributes or
properties: error symbol or name, error conditions, and error message.
The attribute operations defined below are also applicable to Emacs
built-in error objects listed on pp. 535-536 of the Gnus Emacs Lisp
Reference Manual.

\item [Attributes:]
\item {\sl u*error*name}\hfill(page~\pageref{u*error*name})
\item {\sl u*error*conditions}\hfill(page~\pageref{u*error*conditions})
\item {\sl u*error*message}\hfill(page~\pageref{u*error*message})

\item [Operations:]
\item {\sl u*error*define}\hfill(page~\pageref{u*error*define})
\item {\sl u*error*protected}\hfill(page~\pageref{u*error*protected})

\item [Collections:]

\item [Subclasses:]
\item {\sl s*serror}\hfill(page~\pageref{s*serror})
\item {\sl t*error}\hfill(page~\pageref{t*error})

\item [Superclasses:]




\end{description}
\horizontalline

\subsection{u*error*name}
\label{u*error*name}

\begin{description}
\item [Name:]  u*error*name

\item [Class:]
{\sl u*error}\hfill(page~\pageref{u*error})

\item [Contents:] symbol 

\item [Description:]  error name

\item [Setf-able:] no

\item [Public:]



\end{description}
\horizontalline

\subsection{u*error*conditions}
\label{u*error*conditions}

\begin{description}
\item [Name:]  u*error*conditions

\item [Class:]
{\sl u*error}\hfill(page~\pageref{u*error})

\item [Contents:] list of symbols

\item [Description:] 
Specifying a set of conditions under which error
message is to be sent to the user. Normally, this
list includes the error symbol itself adn the symbol
ERROR. Occassionally it includes additional symbols,
which are intermediate classifications, narrower than
error but broader than a single error symbol.

\item [Setf-able:]

\item [Public:]



\end{description}
\horizontalline

\subsection{u*error*message}
\label{u*error*message}

\begin{description}
\item [Name:]  u*error*message

\item [Class:]
{\sl u*error}\hfill(page~\pageref{u*error})

\item [Contents:] message to be sent to the user upon the 
occurance of the error. 

\item [Description:]

\item [Setf-able:] no

\item [Public:]



\end{description}
\horizontalline

\subsection{u*error*define}
\label{u*error*define}

\begin{description}
\item [Name:]  u*error*define

\item [Class:]
{\sl u*error}\hfill(page~\pageref{u*error})

\item [Parameters:]
\item {\sl error-msg}:  string

\item {\sl error-symbol}:  symbol


\item [Return-value:] 
macro with no interesting return value

\item [Description:]
Defines a new error symbol, ERROR-SYMBOL, with two
default condition names: value of ERROR-SYMBOL and
ERROR. Default error message is ERROR-MSG.

\item [Public:]



\end{description}
\horizontalline

\subsection{u*error*protected}
\label{u*error*protected}

\begin{description}
\item [Name:]  u*error*protected

\item [Class:]
{\sl u*error}\hfill(page~\pageref{u*error})

\item [Parameters:]
\item {\sl error-symbols}:  symbol or list of symbols

\item {\sl body}:  list of Lisp forms


\item [Return-value:] 
error object from ERROR-SYMBOLS if error occurs in the
execution of BODY or return value of last form of BODY.

\item [Description:]
executes forms in BODY and catches errors specified in
ERROR-SYMBOLS.  ERROR-SYMBOLS is a required symbol or
list of symbols that represent error conditions to be
caught.  

\item [Public:]



\end{description}
\horizontalline

\section{u*hook}
\label{u*hook}

\begin{description}
\item [Name:]  u*hook

\item [Layer:] {\sl Utilities}\hfill(page~\pageref{Utilities})

\item [Description:] 

U*HOOK is an extension of the Emacs hook mechanism. Its value
is a function or a list of functions to be called under certain
pre-defined conditions. The main purpose of hooks is for
customization and modularity. The main extension is that U*HOOK
allows ordering constraints to be imposed on the functions
installed on a given hook.

\item [Attributes:]

\item [Operations:]
\item {\sl u*hook*insert}\hfill(page~\pageref{u*hook*insert})

\item [Collections:]

\item [Subclasses:]

\item [Superclasses:]



\end{description}
\horizontalline

\subsection{u*hook*insert}
\label{u*hook*insert}

\begin{description}
\item [Name:]  u*hook*insert

\item [Class:]
{\sl u*hook}\hfill(page~\pageref{u*hook})

\item [Parameters:]
\item {\sl hook-var}:  symbol

\item {\sl hook-fn}:  symbol

\item {\sl before-hook-fns}:  list of symbols


\item {\sl after-hook-fns}:  list of symbols



\item [Return-value:]
Returns t if the new event function can be
inserted into the sequence of hook functions in such
a manner as to satisfy the before and after 
constraints, or {\sl conflicting-hook-constraints} (page~\pageref{conflicting-hook-constraints})
otherwise.

\item [Description:]
Installs hook function HOOK-FN on to designated hook
variable.  Returns hook variable value if the new hook fn can
be inserted into the sequence of hook fns that satisfy
required before- and after- constraints; otherwise returns
error object CONFLICTING-HOOK-CONSTRAINTS


\item [Public:]



\end{description}
