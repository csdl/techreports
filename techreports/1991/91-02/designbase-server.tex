\chapter{Server}
\label{Server}

\begin{description}
\item [Name:]  Server Subsystem

\item [Description:]
The server subsystem is the interface between the remote
database server and other modules of the Egret system. As
such, it provides transparent and efficient access to the
functions of the server via such facilities as caching,
events, etc. It also offers encapsulation via
comprehensive error checking and additional services
(e.g., global nodes) essential for the implementation of
advanced Egret functions.

The server module consists of six classes: s*node, s*link,
s*server-process, s*sys-node, s*serror, and s*event.

S*node and s*link provide a set of primitive operations
on nodes and links.  S*server-process manages connection
sessions and synchronizes the states of the remote and
local data.  S*sys-node supports a set of special nodes
which contain data used by various Egret modules for
internal purposes.  S*serror tracks all error objects,
and S*event, all public events in the server system.

\item [Public-classes:]
\item {\sl s*node}\hfill(page~\pageref{s*node})
\item {\sl s*link}\hfill(page~\pageref{s*link})
\item {\sl s*server-process}\hfill(page~\pageref{s*server-process})
\item {\sl s*snode}\hfill(page~\pageref{s*snode})
\item {\sl s*serror}\hfill(page~\pageref{s*serror})
\item {\sl s*event}\hfill(page~\pageref{s*event})  

\item [Private-classes:]




\end{description}
\horizontalline

\section{s*node}
\label{s*node}

\begin{description}
\item [Name:]  s*node

\item [Layer:] {\sl Server}\hfill(page~\pageref{Server})

\item [Description:]
This class implements primitive operations on nodes, including
creation, deletion, field content retrieval and updating,
locking, and so forth. Most of these functions are available from
the remote server. The primary purpose of this class is to
provide a uniform and location-independent access, i.e., the
caller does not need to know whether the returned data is from
local cache or directly from the remote server.

Note that except for the two node repairing functions, which are
accessable only to the system administrator, all other operations
and attribute functions are public. 

\item [Attributes:]
\item {\sl s*node*name}\hfill(page~\pageref{s*node*name})
\item {\sl s*node*data}\hfill(page~\pageref{s*node*data})
\item {\sl s*node*created-by}\hfill(page~\pageref{s*node*created-by})
\item {\sl s*node*created-date}\hfill(page~\pageref{s*node*created-date})
\item {\sl s*node*last-modified-date}\hfill(page~\pageref{s*node*last-modified-date})
\item {\sl s*node*last-modified-by}\hfill(page~\pageref{s*node*last-modified-by})
\item {\sl s*node*font}\hfill(page~\pageref{s*node*font})
\item {\sl s*node*geometry}\hfill(page~\pageref{s*node*geometry})
\item {\sl s*node*locked-by}\hfill(page~\pageref{s*node*locked-by})
\item {\sl s*node*incoming-links}\hfill(page~\pageref{s*node*incoming-links})
\item {\sl s*node*outgoing-links}\hfill(page~\pageref{s*node*outgoing-links})

\item [Operations:]
\item {\sl s*node*make}\hfill(page~\pageref{s*node*make})
\item {\sl s*node*delete}\hfill(page~\pageref{s*node*delete})
\item {\sl s*node*lock}\hfill(page~\pageref{s*node*lock})
\item {\sl s*node*unlock}\hfill(page~\pageref{s*node*unlock})
\item {\sl s*node*set-name}\hfill(page~\pageref{s*node*set-name})
\item {\sl s*node*set-data}\hfill(page~\pageref{s*node*set-data})
\item {\sl s*node*set-geometry}\hfill(page~\pageref{s*node*set-geometry})
\item {\sl s*node*set-font}\hfill(page~\pageref{s*node*set-font})

\item {\sl s*\{node\}*IDs}\hfill(page~\pageref{s*node*IDs})
\item {\sl s*\{node\}*mapc-IDs}\hfill(page~\pageref{s*node*mapc-IDs})

\item {\sl s*node@reset-incoming-links}\hfill(page~\pageref{s*node@reset-incoming-links})
\item {\sl s*node@reset-outgoing-links}\hfill(page~\pageref{s*node@reset-outgoing-links})



\item [Collections:]

\item [Subclasses:]

\item [Superclasses:]



\end{description}
\horizontalline

\subsection{s*node*name}
\label{s*node*name}

\begin{description}

\item [Name:]  s*node*name

\item [Class:] {\sl s*node}\hfill(page~\pageref{s*node}) 

\item [Contents:] String (40)

\item [Description:]
The name of the node. Node names are not
required to be unique at the server subsystem
level, although higher subsystems may wish to
enforce this constraint.

\item [Setf-able:] See s*node*set-name

\item [Public:]



\end{description}
\horizontalline

\subsection{s*node*data}
\label{s*node*data}

\begin{description}

\item [Name:]  s*node*data

\item [Class:] {\sl s*node}\hfill(page~\pageref{s*node})

\item [Contents:]  Unspecified 

\item [Description:]
This is the famous, variable length data
field, of which the server subsystem is
permitted to know practically nothing about.

\item [Setf-able:]


\item [Public:]



\end{description}
\horizontalline

\subsection{s*node*created-by}
\label{s*node*created-by}

\begin{description}

\item [Name:]  s*node*created-by

\item [Class:] {\sl s*node}\hfill(page~\pageref{s*node})

\item [Contents:] String (20)

\item [Description:]

The original author of the node.  This is 
automatically set by the server during 
node creation, but its value is accessable.

\item [Setf-able:]


\item [Public:]



\end{description}
\horizontalline

\subsection{s*node*created-date}
\label{s*node*created-date}

\begin{description}

\item [Name:]  s*node*created-date

\item [Class:] {\sl s*node}\hfill(page~\pageref{s*node})

\item [Contents:] String

\item [Description:]
Node creation date. Automatically set by 
the server subsystem during node creation.

\item [Setf-able:]

\item [Public:]



\end{description}
\horizontalline

\subsection{s*node*last-modified-date}
\label{s*node*last-modified-date}

\begin{description}

\item [Name:]  s*node*last-modified-date

\item [Class:] {\sl s*node}\hfill(page~\pageref{s*node})

\item [Contents:] string

\item [Description:]
Last modified date.  Automatically 
set by the server subsystem during 
node field updating.

\item [Setf-able:]


\item [Public:]



\end{description}
\horizontalline

\subsection{s*node*last-modified-by}
\label{s*node*last-modified-by}

\begin{description}

\item [Name:]  s*node*last-modified-by

\item [Class:] {\sl s*node}\hfill(page~\pageref{s*node})

\item [Contents:] String (20)

\item [Description:]
Automatically maintained by server subsystem
during node field updates.

\item [Setf-able:]

\item [Public:]



\end{description}
\horizontalline

\subsection{s*node*font}
\label{s*node*font}

\begin{description}

\item [Name:]  s*node*font

\item [Class:] {\sl s*node}\hfill(page~\pageref{s*node})

\item [Contents:] String

\item [Description:]
A font specification.  This might be checked
for validity.  This attribute is not
maintained by the server.

\item [Setf-able:]


\item [Public:]



\end{description}
\horizontalline

\subsection{s*node*geometry}
\label{s*node*geometry}

\begin{description}

\item [Name:]  s*node*geometry

\item [Class:] {\sl s*node}\hfill(page~\pageref{s*node})

\item [Contents:] string

\item [Description:]
This should be a valid geometry
specification.  Its value is not
maintained by the server.

\item [Setf-able:]


\item [Public:]



\end{description}
\horizontalline

\subsection{s*node*locked-by}
\label{s*node*locked-by}

\begin{description}

\item [Name:]  s*node*locked-by

\item [Class:] {\sl s*node}\hfill(page~\pageref{s*node})

\item [Contents:] string

\item [Description:]

This attribute is either nil (if the node is
unlocked) or a user-name (if the node is locked by
the specified user). It can also return an error
object {\sl show-lock-fails} (page~\pageref{show-lock-fails}).

This value is maintained by the server
subsystem.

\item [Setf-able:]


\item [Public:]



\end{description}
\horizontalline

\subsection{s*node*incoming-links}
\label{s*node*incoming-links}

\begin{description}

\item [Name:]  s*node*incoming-links

\item [Class:] {\sl s*node}\hfill(page~\pageref{s*node})

\item [Contents:] List of link-IDs

\item [Description:]
This attribute holds a list of link-IDs
whose destination node-IDs are equal 
to the ID of this node instance.

Since this is corruptable local information,
the function s*node@reset-incoming-links
exists to rebuild this node's attribute.

\item [Setf-able:]


\item [Public:]



\end{description}
\horizontalline

\subsection{s*node*outgoing-links}
\label{s*node*outgoing-links}

\begin{description}

\item [Name:]  s*node*outgoing-links

\item [Class:] {\sl s*node}\hfill(page~\pageref{s*node})
 
\item [Contents:] A list of link-IDs

\item [Description:] 
A list of link-IDs corresponding to the
links pointing away from this node.

Since this is local, corruptable information,
the operation s*node!reset-links rebuilds
this by reference to the hbserver.

\item [Setf-able:]


\item [Public:]



\end{description}
\horizontalline

\subsection{s*node*make}
\label{s*node*make}

\begin{description}
\item [Name:]  s*node*make

\item [Class:] {\sl s*node}\hfill(page~\pageref{s*node})

\item [Parameters:]
\item {\sl node-name}:  
A valid node name. This currently means that it is a
string of less than 40 characters, and that it does
not contain leading space(s) or tabs.
  

\item [Return-value:] 
A node-ID if successful.

Error object {\sl invalid-node-name} (page~\pageref{invalid-node-name}) if node-name
violates node naming conventions.

Error object {\sl create-node-fails} (page~\pageref{create-node-fails}) if node
creation hb-call fails.

\item [Description:]
This is the primitive function for obtaining
new node-IDs from the remote server.

On successful node creation, the following node
attributes are set:

(1) s*node*created-by and s*node*last-modified-by are
set to the original author of the node; (2)
s*node*created-date and s*node*last-modified-date are
set to the creation date; (3) s*node*geometry and
s*node*font are set to repective user defaults, or
system defaults if user defaults are not set.

\item [Public:]





\end{description}
\horizontalline

\subsection{s*node*delete}
\label{s*node*delete}

\begin{description}
\item [Name:]  s*node*delete

\item [Class:] {\sl s*node}\hfill(page~\pageref{s*node})

\item [Parameters:] 
\item {\sl node-ID}:   An integer representing
a valid hbserver node ID.
 

\item [Return-value:]
NODE-ID if the node-ID was successfully deleted.

Error object {\sl node-still-referenced} (page~\pageref{node-still-referenced}) if the
target node still contains incoming links.

Error object  {\sl node-still-locked} (page~\pageref{node-still-locked}) if the target
node is still locked.

Error object {\sl unknown-hb-error} (page~\pageref{unknown-hb-error}) if node
deletion hb-operation fails for reasons other
than the above.

\item [Description:]
Permanently removes node NODE-ID from remote database.

\item [Public:]


\end{description}
\horizontalline

\subsection{s*node*lock}
\label{s*node*lock}

\begin{description}
\item [Name:]  s*node*lock

\item [Class:] {\sl s*node}\hfill(page~\pageref{s*node})

\item [Parameters:] 
\item {\sl node-ID}:   An integer representing
a valid hbserver node ID.


\item [Return-value:]
T if the lock was successfully obtained, nil otherwise.

\item [Description:]
Attempts to get a lock on node-ID.  Will fail if 
node-ID is already locked by another user.

\item [Public:]



\end{description}
\horizontalline

\subsection{s*node*unlock}
\label{s*node*unlock}

\begin{description}
\item [Name:]  s*node*unlock

\item [Class:] {\sl s*node}\hfill(page~\pageref{s*node})

\item [Parameters:] 
\item {\sl node-ID}:   An integer representing
a valid hbserver node ID.
 

\item [Return-value:]
T if node-ID was successfully unlocked, nil otherwise.

\item [Description:]
Unlocks node-ID. The user who is trying to unlock the node
must be the same as the one who locks the node.

\item [Public:]



\end{description}
\horizontalline

\subsection{s*node*set-name}
\label{s*node*set-name}

\begin{description}
\item [Name:]  s*node*set-name

\item [Class:] {\sl s*node}\hfill(page~\pageref{s*node})

\item [Parameters:]
\item {\sl node-ID}:   An integer representing
a valid hbserver node ID.

\item {\sl node-name}:  
A valid node name. This currently means that it is a
string of less than 40 characters, and that it does
not contain leading space(s) or tabs.


\item [Return-value:]
A node-ID if successful.

Error object {\sl invalid-node-name} (page~\pageref{invalid-node-name}) if node-name
violates node name conventions. 

Error object {\sl write-attribute-fails} (page~\pageref{write-attribute-fails}) if rename
hb-operation fails.

\item [Description:]
Resets the name of the node. Involves a write
out to the database.

\item [Public:]


\end{description}
\horizontalline

\subsection{s*node*set-data}
\label{s*node*set-data}

\begin{description}
\item [Name:]  s*node*set-data

\item [Class:] {\sl s*node}\hfill(page~\pageref{s*node})

\item [Parameters:]
\item {\sl node-ID}:   An integer representing
a valid hbserver node ID.

\item {\sl node-data}:  string


\item [Return-value:]
A node-ID if successful.

Error object {\sl write-attribute-fails} (page~\pageref{write-attribute-fails}) if set-data
hb-operation fails.

\item [Description:]
Saves data to the persistent datastore on the remote
database server.


\item [Public:]



\end{description}
\horizontalline

\subsection{s*node*set-geometry}
\label{s*node*set-geometry}

\begin{description}
\item [Name:]  s*node*set-geometry

\item [Class:] {\sl s*node}\hfill(page~\pageref{s*node})

\item [Parameters:]
\item {\sl node-ID}:   An integer representing
a valid hbserver node ID.

\item {\sl node-geometry}:  A string containing a valid 
geometry specification. 
 

\item [Return-value:]
NODE-GEOMETRY if successful.

Error object {\sl write-attribute-fails} (page~\pageref{write-attribute-fails}) if set geometry
hb-operation fails.

\item [Description:]
Resets the geometry attribute of node-ID.  
Involves a write out to the database.
Since geometry is inherently a kind of 
domain-specific idea, the server subsystem
does not checks for the validity of its
value. Instead, it treats it simply as
string. 

\item [Public:]



\end{description}
\horizontalline

\subsection{s*node*set-font}
\label{s*node*set-font}

\begin{description}
\item [Name:]  s*node*set-font

\item [Class:] {\sl s*node}\hfill(page~\pageref{s*node})

\item [Parameters:]
\item {\sl node-ID}:   An integer representing
a valid hbserver node ID.

\item {\sl node-font}:  a string corresponding to a valid font name.


\item [Return-value:]
NODE-FONT if successful.

Error object {\sl write-attribute-fails} (page~\pageref{write-attribute-fails}) if set font
hb-operation fails.

\item [Description:]
Resets the font attribute.  Since this
is relatively domain-specific, the server
subsystem is not responsible for its
validity checking.

\item [Public:]



\end{description}
\horizontalline

\subsection{s*\{node\}*IDs}
\label{s*node*IDs}

\begin{description}
\item [Name:]  s*\{node\}*IDs

\item [Class:] {\sl s*node}\hfill(page~\pageref{s*node})

\item [Parameters:] none

\item [Return-value:]

A list of s*node IDs.

\item [Description:]

Returns a freshly consed list of all currently
defined s*node IDs.


\item [Public:]



\end{description}
\horizontalline

\subsection{s*\{node\}*mapc-IDs}
\label{s*node*mapc-IDs}

\begin{description}
\item [Name:]  s*\{node\}*mapc-IDs

\item [Class:] {\sl s*node}\hfill(page~\pageref{s*node})

\item [Parameters:]
\item {\sl map-ID-fn}:  A function that takes one argument, an ID,
and which performs some side-effect based upon that
value.

 

\item [Return-value:] nil

\item [Description:]

Calls map-ID-fn once one each currently defined
s*node ID.

\item [Public:]



\end{description}
\horizontalline

\subsection{s*node@reset-incoming-links}
\label{s*node@reset-incoming-links}

\begin{description}
\item [Name:]  s*node@reset-incoming-links
\item [Class:]
{\sl s*node}\hfill(page~\pageref{s*node})

\item [Parameters:]
\item {\sl node-ID}:   An integer representing
a valid hbserver node ID.


\item [Return-value:] 
list of link-IDs

\item [Description:]

Rebuilds the list of link-IDs that point to this node
by directly calling the remote server to traverse 
all node and link objects to determine which ones point
to this node. 
This is curruption repairing function.

\item [Public:]



\end{description}
\horizontalline

\subsection{s*node@reset-outgoing-links}
\label{s*node@reset-outgoing-links}

\begin{description}
\item [Name:]  s*node@reset-outgoing-links

\item [Class:]
{\sl s*node}\hfill(page~\pageref{s*node})

\item [Parameters:]
\item {\sl node-ID}:   An integer representing
a valid hbserver node ID.
 

\item [Return-value:] 
List of link-IDs.

\item [Description:]
Rebuilds the list of link-IDs that point away from this
node by calling the remote server to traverse all nodes
and links determine which ones point away.  
This is a corruption recovery function.

\item [Public:]



\end{description}
\horizontalline

\section{s*link}
\label{s*link}

\begin{description}
\item [Name:]  s*link

\item [Layer:] {\sl Server}\hfill(page~\pageref{Server})

\item [Description:]
The S*LINK class implements a set of primitive
operations on links, including link creation, deletion,
link attribute retrieval and update, and so forth.
Most of these operations are straightforward
translation of the corresponding remote database server
functions. Because of the asymmetric handling of links
by the current server, S*LINK has to maintain a
separate structure to keep track of the source and
destination nodes of a link. Nevertheless, certain
operations, such as S*LINK*SET-SOURCE-NODE, can still
not be provided due to the built-in constraints on
links.

\item [Attributes:]
\item {\sl s*link*name}\hfill(page~\pageref{s*link*name})
\item {\sl s*link*created-by}\hfill(page~\pageref{s*link*created-by})
\item {\sl s*link*created-date}\hfill(page~\pageref{s*link*created-date})
\item {\sl s*link*last-modified-by}\hfill(page~\pageref{s*link*last-modified-by})
\item {\sl s*link*last-modified-date}\hfill(page~\pageref{s*link*last-modified-date})
\item {\sl s*link*source-node}\hfill(page~\pageref{s*link*source-node})
\item {\sl s*link*destination-node}\hfill(page~\pageref{s*link*destination-node})

\item [Operations:]
\item {\sl s*link*make}\hfill(page~\pageref{s*link*make})
\item {\sl s*link*delete}\hfill(page~\pageref{s*link*delete})
\item {\sl s*link*set-name}\hfill(page~\pageref{s*link*set-name})
\item {\sl s*link*set-destination-node}\hfill(page~\pageref{s*link*set-destination-node})

\item {\sl s*\{link\}*IDs}\hfill(page~\pageref{s*link*IDs})
\item {\sl s*\{link\}*mapc-IDs}\hfill(page~\pageref{s*link*mapc-IDs})

\item [Collections:]

\item [Subclasses:]

\item [Superclasses:]



\end{description}
\horizontalline

\subsection{s*link*name}
\label{s*link*name}

\begin{description}
\item [Name:]  s*link*name

\item [Class:] {\sl s*link}\hfill(page~\pageref{s*link})

\item [Contents:] string (30)

\item [Description:]
The name of the link; limited to thirty chars.

\item [Setf-able:]

\item [Public:]



\end{description}
\horizontalline

\subsection{s*link*created-by}
\label{s*link*created-by}

\begin{description}
\item [Name:]  s*link*created-by

\item [Class:] {\sl s*link}\hfill(page~\pageref{s*link})

\item [Contents:] string

\item [Description:] 
Name of the user who created this link.

\item [Setf-able:] no

\item [Public:]



\end{description}
\horizontalline

\subsection{s*link*created-date}
\label{s*link*created-date}

\begin{description}

\item [Name:]  s*link*created-date

\item [Class:] {\sl s*link}\hfill(page~\pageref{s*link})

\item [Contents:] string

\item [Description:] 
the creation date for this link

\item [Setf-able:]

\item [Public:]



\end{description}
\horizontalline

\subsection{s*link*last-modified-by}
\label{s*link*last-modified-by}

\begin{description}

\item [Name:]  s*link*last-modified-by

\item [Class:] {\sl s*link}\hfill(page~\pageref{s*link})

\item [Contents:] string

\item [Description:] 
The name of user who was last in modifying this
link.

\item [Setf-able:]

\item [Public:]



\end{description}
\horizontalline

\subsection{s*link*last-modified-date}
\label{s*link*last-modified-date}

\begin{description}

\item [Name:]  s*link*last-modified-date

\item [Class:] {\sl s*link}\hfill(page~\pageref{s*link})

\item [Contents:] string

\item [Description:] 
The date of the last modification of this link.

\item [Setf-able:]


\item [Public:]



\end{description}
\horizontalline

\subsection{s*link*source-node}
\label{s*link*source-node}

\begin{description}
\item [Name:]  s*link*source-node

\item [Class:] {\sl s*link}\hfill(page~\pageref{s*link})

\item [Contents:] node-ID

\item [Description:]
The node-ID of the source node of this link.

\item [Setf-able:]


\item [Public:]



\end{description}
\horizontalline

\subsection{s*link*destination-node}
\label{s*link*destination-node}

\begin{description}

\item [Name:]  s*link*destination-node

\item [Class:] {\sl s*link}\hfill(page~\pageref{s*link})

\item [Contents:] node-ID

\item [Description:] 
The destination node for this link.

\item [Setf-able:]

\item [Public:]



\end{description}
\horizontalline

\subsection{s*link*make}
\label{s*link*make}

\begin{description}
\item [Name:]  s*link*make

\item [Class:] {\sl s*link}\hfill(page~\pageref{s*link})

\item [Parameters:]
\item {\sl link-name}:  string (30); a valid link name

\item {\sl from-node-ID}:  node-ID

\item {\sl to-node-ID}:  node-ID




\item [Return-value:]
link-ID if successful.

error object {\sl invalid-link-name} (page~\pageref{invalid-link-name}) if LINK-NAME
contains leading space(s), or its length exceeds 30.

error object {\sl create-link-fails} (page~\pageref{create-link-fails}) if create-link
hb-calls fails. 

error object {\sl write-attribute-fails} (page~\pageref{write-attribute-fails}) if
initialization of link attributes fails.

error object {\sl update-link-info-fails} (page~\pageref{update-link-info-fails}) if update 
operations on LINK-INFO node fails.

\item [Description:]
This function validates link-name, calls remote server
to allocate a new link num; initializes
link-attributes: created by, created ate, last update
by, and last updated date; and runs s*link!make-hooks. 

\item [Public:]





\end{description}
\horizontalline

\subsection{s*link*delete}
\label{s*link*delete}

\begin{description}
\item [Name:]  s*link*delete
\item [Class:] {\sl s*link}\hfill(page~\pageref{s*link})

\item [Parameters:]
\item {\sl link-ID}:  
valid HB link ID number (integer)

\item {\sl from-node-ID}:  node-ID


\item [Return-value:]
LINK-ID if the link deletion was successful, or 
error object otherwise.

\item [Description:]
Permanently remove the link from remote database.


\item [Public:]




\end{description}
\horizontalline

\subsection{s*link*set-name}
\label{s*link*set-name}

\begin{description}
\item [Name:]  s*link*set-name

\item [Class:] {\sl s*link}\hfill(page~\pageref{s*link})

\item [Parameters:]
\item {\sl link-ID}:  
valid HB link ID number (integer)

\item {\sl link-name}:  string (30); a valid link name
 

\item [Return-value:]
LINK-NAME if successful.

error object {\sl invalid-link-name} (page~\pageref{invalid-link-name}) if LINK-NAME contains
leading space(s)/tab(s), or its length exceeds 30.

error object {\sl write-attribute-fails} (page~\pageref{write-attribute-fails}) if hb-write
operation fails.

\item [Description:]
Resets the name of the link. Note that
the link name does not necessarily
correspond to the link label.

\item [Public:]



\end{description}
\horizontalline

\subsection{s*link*set-destination-node}
\label{s*link*set-destination-node}

\begin{description}
\item [Name:]  s*link*set-destination-node

\item [Class:]
{\sl s*link}\hfill(page~\pageref{s*link})

\item [Parameters:]
\item {\sl node-ID}:   An integer representing
a valid hbserver node ID.

\item {\sl link-ID}:  
valid HB link ID number (integer)


\item [Return-value:]
NODE-ID if the destination node of link LINK-ID can be
set to NODE-ID, or error object {\sl move-link-fails} (page~\pageref{move-link-fails})
otherwise. 

\item [Description:]
Reset the target node of the current link to another
node. Note that there is no reciprocal operation,
i.e., one cannot reset the source node of a link
without necessitating the creation of a new link.

\item [Public:]



\end{description}
\horizontalline

\subsection{s*\{link\}*IDs}
\label{s*link*IDs}

\begin{description}
\item [Name:]  s*\{link\}*IDs

\item [Class:] {\sl s*link}\hfill(page~\pageref{s*link})

\item [Parameters:] none

\item [Return-value:]

A list of s*link IDs.

\item [Description:]

Returns a freshly consed list of all currently
defined s*link IDs.

\item [Public:]



\end{description}
\horizontalline

\subsection{s*\{link\}*mapc-IDs}
\label{s*link*mapc-IDs}

\begin{description}
\item [Name:]  s*\{link\}*mapc-IDs

\item [Class:] {\sl s*link}\hfill(page~\pageref{s*link})

\item [Parameters:]
\item {\sl map-ID-fn}:  A function that takes one argument, an ID,
and which performs some side-effect based upon that
value.



\item [Return-value:] nil

\item [Description:]

Calls map-ID-fn once on each currently defined
s*link ID.

\item [Public:]



\end{description}
\horizontalline

\section{s*server-process}
\label{s*server-process}

\begin{description}
\item [Name:]  s*server-process

\item [Layer:] {\sl Server}\hfill(page~\pageref{Server})

\item [Description:]
The server-process class implements connection and
communication operations and attribute functions
related  to the server process, e.g., the directory the
remote databases resides, the name of machine on which
the server is running, the description of the database,
etc. Currently, S*SERVER-PROCESS consists of only one
instance, i.e., hbserver. But the class is designed to
accommodate arbitrary number of server processes. 

\item [Attributes:]
\item {\sl s*sp*name}\hfill(page~\pageref{s*sp*name})
\item {\sl s*sp*description}\hfill(page~\pageref{s*sp*description})
\item {\sl s*sp*directory-path}\hfill(page~\pageref{s*sp*directory-path})
\item {\sl s*sp*ip}\hfill(page~\pageref{s*sp*ip})

\item [Operations:]
\item {\sl s*sp*connect}\hfill(page~\pageref{s*sp*connect})
\item {\sl s*sp*disconnect}\hfill(page~\pageref{s*sp*disconnect})

Administration Operations
\item {\sl s*sp@defserver}\hfill(page~\pageref{s*sp@defserver})

\item [Collections:]

\item [Subclasses:]

\item [Superclasses:]



\end{description}
\horizontalline

\subsection{s*sp*name}
\label{s*sp*name}

\begin{description}

\item [Name:]  s*sp*name

\item [Class:] {\sl s*server-process}\hfill(page~\pageref{s*server-process})

\item [Contents:] String 

\item [Description:]  
Name of server database. It is the same as machine name,
e.g. "uhics.ics.hawaii.edu".

\item [Setf-able:] 

\item [Public:]



\end{description}
\horizontalline

\subsection{s*sp*description}
\label{s*sp*description}

\begin{description}
\item [Name:]  s*sp*description

\item [Class:] {\sl s*server-process}\hfill(page~\pageref{s*server-process})

\item [Contents:] String

\item [Description:]

A short description of the contents or purpose of 
this database.

\item [Setf-able:]


\item [Public:]



\end{description}
\horizontalline

\subsection{s*sp*directory-path}
\label{s*sp*directory-path}

\begin{description}

\item [Name:]  s*sp*directory-path

\item [Class:] {\sl s*server-process}\hfill(page~\pageref{s*server-process})

\item [Contents:] String

\item [Description:]
The pathname to the directory where the datafiles for 
this database are kept.

\item [Setf-able:]


\item [Public:]



\end{description}
\horizontalline

\subsection{s*sp*ip}
\label{s*sp*ip}

\begin{description}
\item [Name:]  s*sp*ip

\item [Class:]
{\sl s*server-process}\hfill(page~\pageref{s*server-process})

\item [Contents:] string.

\item [Description:]
Contains full IP address, eg. "128.171.2.5"

\item [Setf-able:]

\item [Public:]



\end{description}
\horizontalline

\subsection{s*sp*connect}
\label{s*sp*connect}

\begin{description}
\item [Name:]  s*sp*connect

\item [Class:] {\sl s*server-process}\hfill(page~\pageref{s*server-process})

\item [Parameters:]
\item {\sl machine-name}:  
A string which is a valid internet machine address. 
	 

\item [Return-value:] 
t if the connection was made successfully.

error object {\sl missing-required-arg} (page~\pageref{missing-required-arg}) if MACHINE-NAME is
not supplied and public variable S*SP*CURRENT-SERVER is not
set. 

error object {\sl connection-is-on} (page~\pageref{connection-is-on}) if the user tries to 
connect while the connection is already on.

\item [Description:]
This function sets up three network stream processes:
read, write, and event; sets global variable
S*SP*CURRENT-SERVER; subscribes the initial set of
events; initializes local caches of nodes and links, and
runs S*SP!CONNECT-HOOKS.


\item [Public:]






\end{description}
\horizontalline

\subsection{s*sp*disconnect}
\label{s*sp*disconnect}

\begin{description}

\item [Name:]  s*sp*disconnect

\item [Class:] {\sl s*server-process}\hfill(page~\pageref{s*server-process})

\item [Parameters:] none

\item [Return-value:]
t if the disconnection from the server was successful.

Error object {\sl connection-is-off} (page~\pageref{connection-is-off}) if the user
attempts to disconnect while the connection is not on. 


\item [Description:]
This function deletes all three network processes (ie.,
read, write, and event) and sets global status variable
S*SP!CONNECTED to nil.

\item [Public:]




\end{description}
\horizontalline

\subsection{s*sp@defserver}
\label{s*sp@defserver}

\begin{description}
\item [Name:]  s*sp@defserver

\item [Class:]
{\sl s*server-process}\hfill(page~\pageref{s*server-process})

\item [Parameters:]
\item {\sl name}:  string, eg, "zero.ics".


\item {\sl description}:  string, e.g., "DesignBase", "Testbase". 


\item {\sl path}:  string. It must be absolute path name. 


\item {\sl ip}:  string. e.g., "128.71.4.4"



\item [Return-value:]
newly-defined server struct.

\item [Description:]
Defines a new instance of server structure.  Note that
the system administrator should predefine all servers
accessable to the user. Attempts to access an undefined
server is an error condition.

\item [Public:]



\end{description}
\horizontalline

\section{s*snode}
\label{s*snode}

\begin{description}
\item [Name:]  s*snode

\item [Layer:]
{\sl Server}\hfill(page~\pageref{Server})

\item [Description:]

The SYS-NODE is provided by the server subsystem to other
EGRET modules (including the server itself) as a uniform
mechanism for handling a set of special nodes which store
data used by these modules in implementing advanced
functions. The major features of SYS-NODES in comparison
with the regular Egret nodes are:

* They are "hidden" from the users of the system.

* They are internal to the modules in which they are
  defined and used.

* They are susceptible to corruption and thus require
  recovery operations.
*  They are normally not deleted once created.

* Operations on them are normally performed via special
  SYS-NODE specific event functions. A public function
  also exists to update the contents of the system node.

* Multiple subsystems can share the same system node. For
  example, unread node uses the done-node data defined by
  the to-do module.

* SYS-NODEs store and retrieve a single, variable length
  string, whose contents are interpreted by its utilizing
  systems.

* Unique IDs for system nodes (snode-ID) are not integers
  but a unique string.

* Concurrent access is maintained by the server
  subsystem. Users cannot explicitly lock or unlock a
  system node. All operations upon SNODEs are assumed to
  require a brief amount of time; they are reliably
  unlocked regardless of whether operations terminate
  successfully or signal an error.

* Definition and instantiation of a SYS-NODE are separate
  operations. Definition describes the event operations
  that will run if the instance of the system node exists
  on the server. A separate operation (s*snode!make)
  exists to actually create the entity. 

\item [Attributes:]

\item [Operations:]
\item {\sl s*snode*define}\hfill(page~\pageref{s*snode*define})
\item {\sl s*snode*with-data}\hfill(page~\pageref{s*snode*with-data})
\item {\sl s*snode*with-data-locked}\hfill(page~\pageref{s*snode*with-data-locked})

\item {\sl s*snode@make}\hfill(page~\pageref{s*snode@make})

\item [Collections:]


\item [Subclasses:]


\item [Superclasses:]


\item [Instances:]



\end{description}
\horizontalline

\subsection{s*snode*define}
\label{s*snode*define}

\begin{description}
\item [Name:]  s*snode*define

\item [Class:]
{\sl s*snode}\hfill(page~\pageref{s*snode})

\item [Parameters:]
\item {\sl snode-name}:  string

\item {\sl data-event}:  symbol

\item {\sl connect}:  symbol

\item {\sl disconnect}:  symbol


\item [Return-value:]
Macro with side effect of defining an SNODE object.

\item [Description:]
This macro defines a specialized system node class with a
single instance SNODE-NAME. It registers SNODE-NAME on
the global system node data structure, and data update
event and connection event functions, if any, on
respective hooks.

\item [Public:]



\end{description}
\horizontalline

\subsection{s*snode*with-data}
\label{s*snode*with-data}

\begin{description}
\item [Name:]  s*snode*with-data

\item [Class:]
{\sl s*snode}\hfill(page~\pageref{s*snode})

\item [Parameters:]
\item {\sl snode-name}:  string

\item {\sl body}:  list of Lisp forms


\item [Return-value:] 
Macro with no interesting return value.

\item [Description:]
Retrieves sys-node NODE-NAME and then evals BODY.
NODE-NAME must be a name of an existent node. 
BODY is a list of forms to be executed. 

This macro temporarily resets the current buffer to a
special system node buffer that contains the contents of
the system node, position cursor at tbe begining of
buffer, evals forms in BODY, and restores the
previous buffer context. It is evaled for side effect. 

\item [Public:]




\end{description}
\horizontalline

\subsection{s*snode*with-data-locked}
\label{s*snode*with-data-locked}

\begin{description}
\item [Name:]  s*snode*with-data-locked

\item [Class:]
{\sl s*snode}\hfill(page~\pageref{s*snode})

\item [Parameters:]
\item {\sl snode-name}:  string

\item {\sl body}:  list of Lisp forms


\item [Return-value:]
Macro with no interesting return value if successful or
error object {\sl lock-snode-fails} (page~\pageref{lock-snode-fails}) if s*snode-lock 
operation fails..

\item [Description:]
Retrieves snode NODE-NAME with lock and then evals BODY.
NODE-NAME must be a name of an existent node.  
BODY is a list of forms to be executed.  

This macro temporarily resets the current buffer to a
special system node buffer that contains the contents of
the system node, position cursor at tbe begining of
buffer, evals the forms in BODY, which ought to alter
the content of the buffer, saves the updated buffer to
the database, and restores the previous buffer context.
The macro is evaled for side effect.


\item [Public:]



\end{description}
\horizontalline

\subsection{s*snode@make}
\label{s*snode@make}

\begin{description}
\item [Name:]  s*snode@make

\item [Class:]
{\sl s*snode}\hfill(page~\pageref{s*snode})

\item [Parameters:]
\item {\sl snode-name}:  string

\item {\sl initial-value}:  string


\item [Return-value:] SNODE-NAME

\item [Description:]
Instantiates an snode, i.e., creating node in remote
database with name SNODE-NAME, and stores INITIAL-VALUE
in the node. Note this is an administrative function. If
SNODE-NAME already exists, re-initializes it to
INIT-VALUE.

\item [Public:]



\end{description}
\horizontalline

\section{s*serror}
\label{s*serror}

\begin{description}
\item [Name:]  s*serror

\item [Layer:]
{\sl Server}\hfill(page~\pageref{Server}) 

\item [Description:]
This class contains all error objects in the Server subsystem.
As a general rule, server level functions trap and report but
do not handle errors. It is the responsility of the calling
function to interpret various error objects and decide what
actions to take. Lower-level server functions use SIGNAL to
return to top-level server functions in case of error.
Top-level server functions must catch all server-level errors,
and guaranttee returning to the calling function with
approciate error objects.

\item [Attributes:] See U*ERROR

\item [Operations:] See U*ERROR

\item [Collections:]

\item [Subclasses:]

\item [Superclasses:]
\item {\sl u*error}\hfill(page~\pageref{u*error})

\item [Instances:]
\item {\sl lock-snode-fails}\hfill(page~\pageref{lock-snode-fails})
\item {\sl missing-required-arg}\hfill(page~\pageref{missing-required-arg})
\item {\sl write-attribute-fails}\hfill(page~\pageref{write-attribute-fails})
\item {\sl read-attribute-fails}\hfill(page~\pageref{read-attribute-fails})
\item {\sl get-entity-IDs-fails}\hfill(page~\pageref{get-entity-IDs-fails})
\item {\sl conflicting-hook-constraints}\hfill(page~\pageref{conflicting-hook-constraints})
\item {\sl unknown-hb-error}\hfill(page~\pageref{unknown-hb-error})
\item {\sl uninstantiated-system-node}\hfill(page~\pageref{uninstantiated-system-node})

\item {\sl parse-event-fails}\hfill(page~\pageref{parse-event-fails})
\item {\sl subscribe-event-fails}\hfill(page~\pageref{subscribe-event-fails})
\item {\sl unsubscribe-event-fails}\hfill(page~\pageref{unsubscribe-event-fails})

\item {\sl invalid-node-name}\hfill(page~\pageref{invalid-node-name})
\item {\sl show-lock-fails}\hfill(page~\pageref{show-lock-fails})
\item {\sl lock-node-fails}\hfill(page~\pageref{lock-node-fails})
\item {\sl create-node-fails}\hfill(page~\pageref{create-node-fails})
\item {\sl delete-node-fails}\hfill(page~\pageref{delete-node-fails})
\item {\sl node-still-referenced}\hfill(page~\pageref{node-still-referenced})
\item {\sl node-still-locked}\hfill(page~\pageref{node-still-locked})
\item {\sl node-not-found}\hfill(page~\pageref{node-not-found})

\item {\sl invalid-link-name}\hfill(page~\pageref{invalid-link-name})
\item {\sl move-link-fails}\hfill(page~\pageref{move-link-fails})
\item {\sl create-link-fails}\hfill(page~\pageref{create-link-fails})
\item {\sl link-not-found}\hfill(page~\pageref{link-not-found})
\item {\sl update-link-info-fails}\hfill(page~\pageref{update-link-info-fails})

\item {\sl connection-is-on}\hfill(page~\pageref{connection-is-on})
\item {\sl connection-is-off}\hfill(page~\pageref{connection-is-off})
\item {\sl server-not-found}\hfill(page~\pageref{server-not-found})









\end{description}
\horizontalline

\subsection{lock-snode-fails}
\label{lock-snode-fails}

\begin{description}
\item [Name:]  lock-snode-fails

\item [Class:]
{\sl s*serror}\hfill(page~\pageref{s*serror})

\item [Description:]
Attempt to lock a system node fails.


\end{description}
\horizontalline

\subsection{missing-required-arg}
\label{missing-required-arg}

\begin{description}
\item [Name:]  missing-required-arg


\item [Class:]
{\sl s*serror}\hfill(page~\pageref{s*serror})


\item [Description:] 
Required arg is either missing or unccaptable. 



\end{description}
\horizontalline

\subsection{write-attribute-fails}
\label{write-attribute-fails}

\begin{description}
\item [Name:]  write-attribute-fails


\item [Class:]
{\sl s*serror}\hfill(page~\pageref{s*serror})


\item [Description:] 
Data fails to be written to remote persistent store.


\end{description}
\horizontalline

\subsection{read-attribute-fails}
\label{read-attribute-fails}

\begin{description}

\item [Name:]  read-attribute-fails


\item [Class:]
{\sl s*serror}\hfill(page~\pageref{s*serror})


\item [Description:] 
Field data fails to be retrieved from remote database
server.



\end{description}
\horizontalline

\subsection{get-entity-IDs-fails}
\label{get-entity-IDs-fails}

\begin{description}
\item [Name:]  get-entity-IDs-fails

\item [Class:]
{\sl s*serror}\hfill(page~\pageref{s*serror})

\item [Description:]
Operation on retrieving the list of either all node or link
IDs from remote server fails.



\end{description}
\horizontalline

\subsection{conflicting-hook-constraints}
\label{conflicting-hook-constraints}

\begin{description}
\item [Name:]  conflicting-hook-constraints

\item [Class:]
{\sl s*serror}\hfill(page~\pageref{s*serror})

\item [Description:]
The supplied ordering constraints are conflicting with each
other and thus cannot be satisfied.

\end{description}
\horizontalline

\subsection{unknown-hb-error}
\label{unknown-hb-error}

\begin{description}

\item [Name:]  unknown-hb-error


\item [Class:]
{\sl s*serror}\hfill(page~\pageref{s*serror})


\item [Description:] 
Hyperbase operation fails due to some unknown reason(s).



\end{description}
\horizontalline

\subsection{uninstantiated-system-node}
\label{uninstantiated-system-node}

\begin{description}
\item [Name:]  uninstantiated-system-node

\item [Class:]
{\sl s*serror}\hfill(page~\pageref{s*serror})


\item [Description:] 
A SYS-NODE must be initialized before use. Fails to do so
would cause this error to be signaled.



\end{description}
\horizontalline

\subsection{parse-event-fails}
\label{parse-event-fails}

\begin{description}
\item [Name:]  parse-event-fails


\item [Class:]
{\sl s*serror}\hfill(page~\pageref{s*serror})


\item [Description:]
Incoming event string from the remote server cannot be
parsed.. 


\end{description}
\horizontalline

\subsection{subscribe-event-fails}
\label{subscribe-event-fails}

\begin{description}
\item [Name:]  subscribe-event-fails


\item [Class:]
{\sl s*serror}\hfill(page~\pageref{s*serror})


\item [Description:]
Operation for subscribing an event on remote server fails. 

\end{description}
\horizontalline

\subsection{unsubscribe-event-fails}
\label{unsubscribe-event-fails}

\begin{description}

\item [Name:]  unsubscribe-event-fails


\item [Class:]
{\sl s*serror}\hfill(page~\pageref{s*serror})


\item [Description:]
Attempt to unsubscribe an event previsouly subscribed
fails.


\end{description}
\horizontalline

\subsection{invalid-node-name}
\label{invalid-node-name}

\begin{description}

\item [Name:]  invalid-node-name


\item [Class:]
{\sl s*serror}\hfill(page~\pageref{s*serror})


\item [Description:]
Node naming violation, i.e., exceeding 40 characters or
containing leading space or tabs.


\end{description}
\horizontalline

\subsection{show-lock-fails}
\label{show-lock-fails}

\begin{description}
\item [Name:]  show-lock-fails


\item [Class:]
{\sl s*serror}\hfill(page~\pageref{s*serror})


\item [Description:]
cannot determine who has the lock to the requested node.


\end{description}
\horizontalline

\subsection{lock-node-fails}
\label{lock-node-fails}

\begin{description}

\item [Name:]  lock-node-fails


\item [Class:]
{\sl s*serror}\hfill(page~\pageref{s*serror})


\item [Description:]
cannot lock a node.


\end{description}
\horizontalline

\subsection{create-node-fails}
\label{create-node-fails}

\begin{description}
\item [Name:]  create-node-fails


\item [Class:]
{\sl s*serror}\hfill(page~\pageref{s*serror})


\item [Description:]
Attempt to create new node fails.

\end{description}
\horizontalline

\subsection{delete-node-fails}
\label{delete-node-fails}

\begin{description}
\item [Name:]  delete-node-fails


\item [Class:]
{\sl s*serror}\hfill(page~\pageref{s*serror})


\item [Description:]
Attempt to delete a node form remote persistent store fails.


\end{description}
\horizontalline

\subsection{node-still-referenced}
\label{node-still-referenced}

\begin{description}

\item [Name:]  node-still-referenced


\item [Class:]
{\sl s*serror}\hfill(page~\pageref{s*serror})


\item [Description:]
Attempts to delete a node which still has incoming links.


\end{description}
\horizontalline

\subsection{node-still-locked}
\label{node-still-locked}

\begin{description}
\item [Name:]  node-still-locked


\item [Class:]
{\sl s*serror}\hfill(page~\pageref{s*serror})


\item [Description:]
Trying to delete a node which is locked. 


\end{description}
\horizontalline

\subsection{node-not-found}
\label{node-not-found}

\begin{description}
\item [Name:]  node-not-found

\item [Class:]
{\sl s*serror}\hfill(page~\pageref{s*serror})

\item [Description:] requested node does not exist.



\end{description}
\horizontalline

\subsection{invalid-link-name}
\label{invalid-link-name}

\begin{description}
\item [Name:]  invalid-link-name


\item [Class:]
{\sl s*serror}\hfill(page~\pageref{s*serror})


\item [Description:]
Link naming violation, i.e., either the name exceeds 30
characters or contains leading space or tabs.


\end{description}
\horizontalline

\subsection{move-link-fails}
\label{move-link-fails}

\begin{description}
\item [Name:]  move-link-fails


\item [Class:]
{\sl s*serror}\hfill(page~\pageref{s*serror})

\item [Description:]
Attempt to reset the target node of a link fails.


\end{description}
\horizontalline

\subsection{create-link-fails}
\label{create-link-fails}

\begin{description}

\item [Name:]  create-link-fails


\item [Class:]
{\sl s*serror}\hfill(page~\pageref{s*serror})


\item [Description:]
Attempt to create new link fails.


\end{description}
\horizontalline

\subsection{link-not-found}
\label{link-not-found}

\begin{description}
\item [Name:]  link-not-found

\item [Class:]
{\sl s*serror}\hfill(page~\pageref{s*serror})

\item [Description:] requested link does not exist. 



\end{description}
\horizontalline

\subsection{update-link-info-fails}
\label{update-link-info-fails}

\begin{description}
\item [Name:]  update-link-info-fails

\item [Class:]
{\sl s*serror}\hfill(page~\pageref{s*serror})

\item [Description:] 
attempts to update the link-info system node fails.



\end{description}
\horizontalline

\subsection{connection-is-on}
\label{connection-is-on}

\begin{description}
\item [Name:]  connection-is-on


\item [Class:]
{\sl s*serror}\hfill(page~\pageref{s*serror})


\item [Description:]
Trying to connect to the remote server while the
current client is already connected.


\end{description}
\horizontalline

\subsection{connection-is-off}
\label{connection-is-off}

\begin{description}

\item [Name:]  connection-is-off


\item [Class:]
{\sl s*serror}\hfill(page~\pageref{s*serror})


\item [Description:]
Trying to disconnect while the client is already
disconnected from the remote server.


\end{description}
\horizontalline

\subsection{server-not-found}
\label{server-not-found}

\begin{description}
\item [Name:]  server-not-found

\item [Class:]
{\sl s*serror}\hfill(page~\pageref{s*serror})

\item [Description:] requested server is not defined.



\end{description}
\horizontalline

\section{s*event}
\label{s*event}

\begin{description}
\item [Name:]  s*event

\item [Layer:]
{\sl Server}\hfill(page~\pageref{Server})

\item [Description:]
S*EVENT represents a class of a predefined event types. It is
implemented in terms of U*HOOK, which in turn is an extension of
Emacs Lisp hook facility. S*EVENT allows arbitrary ordering
constraints to be imposed on functions to be intalled on any
given event dispatching queue.

S*EVENT is special in that all its instances are predefined; the
user of these events are not allowed to instantiated them, though
they may add functions to or delete functions from them, or
reinitialize them.  Currently, there are currently elevent (11)
event types, all of which are documentated below.  The execution
of functions on these event queues is triggered by events from
the remote database server, rather than directed by the Server
itself.

\item [Attributes:]
\item {\sl s*event*event-handlers}\hfill(page~\pageref{s*event*event-handlers})

\item [Operations:]
\item {\sl s*event*initialize}\hfill(page~\pageref{s*event*initialize})
\item {\sl s*event*add-event-handler-fn}\hfill(page~\pageref{s*event*add-event-handler-fn})
\item {\sl s*event*remove-event-handler-fn}\hfill(page~\pageref{s*event*remove-event-handler-fn})

\item [Collections:]

\item [Subclasses:]

\item [Superclasses:]

\item [Instances:]
\item {\sl s*node!newname-event-hooks}\hfill(page~\pageref{s*node!newname-event-hooks})
\item {\sl s*node!rename-event-hooks}\hfill(page~\pageref{s*node!rename-event-hooks})
\item {\sl s*node!delete-event-hooks}\hfill(page~\pageref{s*node!delete-event-hooks})
\item {\sl s*node!data-event-hooks}\hfill(page~\pageref{s*node!data-event-hooks})
\item {\sl s*node!lock-event-hooks}\hfill(page~\pageref{s*node!lock-event-hooks})
\item {\sl s*node!unlock-event-hooks}\hfill(page~\pageref{s*node!unlock-event-hooks})
\item {\sl s*node!font-event-hooks}\hfill(page~\pageref{s*node!font-event-hooks})
\item {\sl s*node!geometry-event-hooks}\hfill(page~\pageref{s*node!geometry-event-hooks})
\item {\sl s*link!newname-event-hooks}\hfill(page~\pageref{s*link!newname-event-hooks})
\item {\sl s*link!rename-event-hooks}\hfill(page~\pageref{s*link!rename-event-hooks})
\item {\sl s*link!delete-event-hooks}\hfill(page~\pageref{s*link!delete-event-hooks})












\end{description}
\horizontalline

\subsection{s*event*event-handlers}
\label{s*event*event-handlers}

\begin{description}
\item [Name:]  s*event*event-handlers

\item [Class:]
{\sl s*event}\hfill(page~\pageref{s*event})

\item [Contents:] List of functions

\item [Description:] 
An ordered list of functions to be invoked upon
receiving a given type of event.


\item [Setf-able:] no

\item [Public:]



\end{description}
\horizontalline

\subsection{s*event*initialize}
\label{s*event*initialize}

\begin{description}
\item [Name:]  s*event*initialize

\item [Class:]
{\sl s*event}\hfill(page~\pageref{s*event})

\item [Parameters:]
\item {\sl s*event-instance}:  symbol


\item [Return-value:] t

\item [Description:] Removes all current event handlers from 
S*EVENT-INSTANCE.

\item [Public:]



\end{description}
\horizontalline

\subsection{s*event*add-event-handler-fn}
\label{s*event*add-event-handler-fn}

\begin{description}
\item [Name:]  s*event*add-event-handler-fn

\item [Class:]
{\sl s*event}\hfill(page~\pageref{s*event})

\item [Parameters:]
\item {\sl s*event-instance}:  symbol

\item {\sl handler-fn-name}:  function symbol

\item {\sl before-handlers}:  functional symbol

\item {\sl after-handlers}:  function symbol


\item [Return-value:] 
List of function symbols or error object
CONFLICTING-HOOK-CONSTRAINTS. 

\item [Description:]
Inserts HANDLER-FN-NAME into S*EVENT S*EVENT-INSTANCE
so that BEFORE-HANDLERS and AFTER-HANDLERS constraintsw
are satisfied.

\item [Public:]



\end{description}
\horizontalline

\subsection{s*event*remove-event-handler-fn}
\label{s*event*remove-event-handler-fn}

\begin{description}
\item [Name:]  s*event*remove-event-handler-fn

\item [Class:]
{\sl s*event}\hfill(page~\pageref{s*event})

\item [Parameters:]
\item {\sl s*event-instance}:  symbol

\item {\sl handler-fn-name}:  function symbol


\item [Return-value:] Symbol of function being removed

\item [Description:] 
Removes function HANDLER-FN-NAME from S*EVENT-INSTANCE. 

\item [Public:]



\end{description}
\horizontalline

\subsection{s*node!newname-event-hooks}
\label{s*node!newname-event-hooks}

\begin{description}
\item [Name:]  s*node!newname-event-hooks

\item [Class:]
{\sl s*event}\hfill(page~\pageref{s*event})

\item [Description:]
Holds a list of functions executed upon receiving a
'new-node-name' event, i.e., with incoming event type 'n
name', entity type is 'node' and the target node can be
found in the local cache structure. 


\end{description}
\horizontalline

\subsection{s*node!rename-event-hooks}
\label{s*node!rename-event-hooks}

\begin{description}
\item [Name:]  s*node!rename-event-hooks

\item [Class:]
{\sl s*event}\hfill(page~\pageref{s*event})

\item [Description:]
Holds a list of functions executed upon receiving a
'rename-node-name' event, i.e., with incoming event
type 'n name', entity type is 'node' and the target
node cannot be found in the local cache structure.



\end{description}
\horizontalline

\subsection{s*node!delete-event-hooks}
\label{s*node!delete-event-hooks}

\begin{description}
\item [Name:]  s*node!delete-event-hooks

\item [Class:]
{\sl s*event}\hfill(page~\pageref{s*event})

\item [Description:]
Holds a list of functions executed upon receiving
delete-node event.  Each function is passed a node-ID.



\end{description}
\horizontalline

\subsection{s*node!data-event-hooks}
\label{s*node!data-event-hooks}

\begin{description}
\item [Name:]  s*node!data-event-hooks

\item [Class:]
{\sl s*event}\hfill(page~\pageref{s*event})

\item [Description:]
Holds a list of functions executed upon receiving a
'write data' event.



\end{description}
\horizontalline

\subsection{s*node!lock-event-hooks}
\label{s*node!lock-event-hooks}

\begin{description}

\item [Name:]  s*node!lock-event-hooks


\item [Class:]
{\sl s*event}\hfill(page~\pageref{s*event})


\item [Description:]
Holds a list of functions executed upon
receiving a 'lock' event.


\end{description}
\horizontalline

\subsection{s*node!unlock-event-hooks}
\label{s*node!unlock-event-hooks}

\begin{description}
\item [Name:]  s*node!unlock-event-hooks

\item [Class:]
{\sl s*event}\hfill(page~\pageref{s*event})

\item [Description:]
Holds a list of functions executed upon
receiving a 'unlock' event.


\end{description}
\horizontalline

\subsection{s*node!font-event-hooks}
\label{s*node!font-event-hooks}

\begin{description}
\item [Name:]  s*node!font-event-hooks

\item [Class:]
{\sl s*event}\hfill(page~\pageref{s*event})

\item [Description:]
Holds a list of functions executed upon receiving
a 'write-font' event.


\end{description}
\horizontalline

\subsection{s*node!geometry-event-hooks}
\label{s*node!geometry-event-hooks}

\begin{description}

\item [Name:]  s*node!geometry-event-hooks


\item [Class:]
{\sl s*event}\hfill(page~\pageref{s*event})


\item [Description:]
Holds a list of functions executed upon receiving a
'write-geometry' event.


\end{description}
\horizontalline

\subsection{s*link!newname-event-hooks}
\label{s*link!newname-event-hooks}

\begin{description}
\item [Name:]  s*link!newname-event-hooks

\item [Class:]
{\sl s*event}\hfill(page~\pageref{s*event})

\item [Description:]
Holds a list of functions executed upon receiving
a 'write-geometry' event.


\end{description}
\horizontalline

\subsection{s*link!rename-event-hooks}
\label{s*link!rename-event-hooks}

\begin{description}
\item [Name:]  s*link!rename-event-hooks

\item [Class:]
{\sl s*event}\hfill(page~\pageref{s*event})

\item [Description:]
Holds a list of functions executed upon receiving a
'new-link-name' event, i.e., with incoming event
type 'n name', entity type is 'link' and the target
link can be found in the local cache structure.


\end{description}
\horizontalline

\subsection{s*link!delete-event-hooks}
\label{s*link!delete-event-hooks}

\begin{description}
\item [Name:]  s*link!delete-event-hooks

\item [Class:]
{\sl s*event}\hfill(page~\pageref{s*event})

\item [Description:]
Holds a list of functions executed upon receiving
delete-link event.  Each function is passed link-ID.


\end{description}
