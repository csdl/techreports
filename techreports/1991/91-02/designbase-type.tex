\chapter{Type}
\label{Type}

\begin{description}
\item [Name:]  Type

\item [Description:]

The Type subsystem defines a collaborative, extensible
data model on top of the fixed Server subsystem node
and link types.  

\item [Public-classes:]
\item {\sl t*node-schema}\hfill(page~\pageref{t*node-schema})
\item {\sl t*node-instance}\hfill(page~\pageref{t*node-instance})

\item {\sl t*link-schema}\hfill(page~\pageref{t*link-schema})
\item {\sl t*link-instance}\hfill(page~\pageref{t*link-instance})

\item {\sl t*field-schema}\hfill(page~\pageref{t*field-schema})

\item {\sl t*layer}\hfill(page~\pageref{t*layer})

\item {\sl t*error}\hfill(page~\pageref{t*error})
\item {\sl t*event}\hfill(page~\pageref{t*event})

\end{description}
\horizontalline

\section{t*node-schema}
\label{t*node-schema}

\begin{description}
\item [Name:]  t*node-schema

\item [Layer:] {\sl Type}\hfill(page~\pageref{Type})

\item [Description:]

Each instance of a node-schema defines the consensually 
agreed upon structural features (i.e. the set of fields)
for a set of node instances.  However, these instances
may not necessarily conform to these structural 
features. 

\item [Attributes:]
\item {\sl t*node-schema*name}\hfill(page~\pageref{t*node-schema*name})
\item {\sl t*node-schema*node-IDs}\hfill(page~\pageref{t*node-schema*node-IDs})
\item {\sl t*node-schema*field-IDs}\hfill(page~\pageref{t*node-schema*field-IDs})

\item [Operations:]
\item {\sl t*node-schema*delete}\hfill(page~\pageref{t*node-schema*delete})
\item {\sl t*node-schema*divergence}\hfill(page~\pageref{t*node-schema*divergence})
\item {\sl t*node-schema*set-name}\hfill(page~\pageref{t*node-schema*set-name})
\item {\sl t*node-schema*delete-fields}\hfill(page~\pageref{t*node-schema*delete-fields})
\item {\sl t*node-schema*add-fields}\hfill(page~\pageref{t*node-schema*add-fields})
\item {\sl t*node-schema*instantiate}\hfill(page~\pageref{t*node-schema*instantiate})
\item {\sl t*node-schema*make}\hfill(page~\pageref{t*node-schema*make})

\item {\sl t*\{node-schema\}*IDs}\hfill(page~\pageref{t*node-schema*IDs})
\item {\sl t*\{node-schema\}*mapc-IDs}\hfill(page~\pageref{t*node-schema*mapc-IDs})

\item [Subclasses:]


\item [Superclasses:]


\item [Instances:]




\end{description}
\horizontalline

\subsection{t*node-schema*name}
\label{t*node-schema*name}

\begin{description}
\item [Name:]  t*node-schema*name

\item [Class:] 

\item [Contents:] Symbol 

\item [Description:] The name of the node-schema

\item [Setf-able:] see t*node-schema*set-name

\item [Public:] yes.



\end{description}
\horizontalline

\subsection{t*node-schema*node-IDs}
\label{t*node-schema*node-IDs}

\begin{description}
\item [Name:]  t*node-schema*node-IDs

\item [Class:] {\sl t*node-schema}\hfill(page~\pageref{t*node-schema})

\item [Contents:] a list of node-IDs

\item [Description:]

This list of node-IDs are those that currently have
this node-schema as their associated consensual structure.

\item [Setf-able:] Updated automatically by type system. 

\item [Public:] yes



\end{description}
\horizontalline

\subsection{t*node-schema*field-IDs}
\label{t*node-schema*field-IDs}

\begin{description}
\item [Name:]  t*node-schema*field-IDs

\item [Class:] {\sl t*node-schema}\hfill(page~\pageref{t*node-schema})

\item [Contents:] a list of field-schema-IDs

\item [Description:]

Returns a list of the field-schema-IDs currently 
associated with this node-schema.

\item [Setf-able:] See t*node-schema*add-fields and 
t*node-schema*delete-fields
 
\item [Public:] yes



\end{description}
\horizontalline

\subsection{t*node-schema*delete}
\label{t*node-schema*delete}

\begin{description}
\item [Name:]  t*node-schema*delete

\item [Class:] {\sl t*node-schema}\hfill(page~\pageref{t*node-schema})

\item [Parameters:]
\item {\sl node-schema-ID}:  a server-level node-ID that corresponds to an 
instance of a type-level node-schema. 



\item [Return-value:]
NODE-SCHEMA-ID if successful. 

Error object {\sl invalid-node-schema-ID} (page~\pageref{invalid-node-schema-ID}) if bad 
node-schema-ID. 

Error object {\sl unknown-hb-error} (page~\pageref{unknown-hb-error}) if this call
fails for some other reason.

\item [Description:]

This operation "marks" node-schema-ID for deletion, 
thus ensuring that no future use of node-schema-ID 
will occur (once this deletion event has been propogated
to all other connected users.)  Node-schema-ID will
continue to exist in the database and its prior 
references and instances will continue to exist. 

More specifically, a deleted node-schema-ID will be
considered an invalid argument to the node-schema
operations: instantiate, add-fields, delete-fields,
set-name, and delete. The node-instance operation clone
will also be disabled for instances of this node-schema.
However, the attributes and the divergence operation will
continue to operate normally when passed this
node-schema-ID.

\item [Public:]



\end{description}
\horizontalline

\subsection{t*node-schema*divergence}
\label{t*node-schema*divergence}

\begin{description}
\item [Name:]  t*node-schema*divergence

\item [Class:] {\sl t*node-schema}\hfill(page~\pageref{t*node-schema})

\item [Parameters:]
\item {\sl node-schema-ID}:  a server-level node-ID that corresponds to an 
instance of a type-level node-schema. 



\item [Return-value:]
An integer corresponding to the divergence metric for 
this node-schema and its instances if successful.

Error object {\sl invalid-node-schema-ID} (page~\pageref{invalid-node-schema-ID}) if bad 
node-schema-ID.

Error object {\sl unknown-hb-error} (page~\pageref{unknown-hb-error}) if this computation
fails for some other reason.

\item [Description:]

Computes the structural divergence of the instances
of this node schema.

\item [Public:]



\end{description}
\horizontalline

\subsection{t*node-schema*set-name}
\label{t*node-schema*set-name}

\begin{description}
\item [Name:]  t*node-schema*set-name

\item [Class:] {\sl t*node-schema}\hfill(page~\pageref{t*node-schema})

\item [Parameters:]
\item {\sl node-schema-ID}:  a server-level node-ID that corresponds to an 
instance of a type-level node-schema. 


\item {\sl node-name}:  
A valid node name. This currently means that it is a
string of less than 40 characters, and that it does
not contain leading space(s) or tabs.


\item [Return-value:]
Returns the new name if successful.

Error object {\sl invalid-node-schema-ID} (page~\pageref{invalid-node-schema-ID}) if bad 
node-schema-ID. 

Error object {\sl invalid-node-name} (page~\pageref{invalid-node-name}) if node-name 
violates node name conventions.

Error object {\sl unknown-hb-error} (page~\pageref{unknown-hb-error}) if this call
fails for some other reason.

\item [Description:]
Renames the node-schema-ID to node-name.  
(Note that node names need not be unique at the 
type level.)

\item [Public:]



\end{description}
\horizontalline

\subsection{t*node-schema*delete-fields}
\label{t*node-schema*delete-fields}

\begin{description}
\item [Name:]  t*node-schema*delete-fields

\item [Class:] {\sl t*node-schema}\hfill(page~\pageref{t*node-schema})

\item [Parameters:]
\item {\sl node-schema-ID}:  a server-level node-ID that corresponds to an 
instance of a type-level node-schema. 


\item {\sl field-schema-IDs}:  list of field-schema-ID
 

\item [Return-value:]
The updated list of field-schema-IDs if successful.

Error object {\sl invalid-node-schema-ID} (page~\pageref{invalid-node-schema-ID}) if
node-schema-ID was not a t*node-schema.

Error object {\sl invalid-field-ID} (page~\pageref{invalid-field-ID}) if field-schema-IDs 
are invalid or are not present in node-schema-ID.

Error object {\sl unknown-hb-error} (page~\pageref{unknown-hb-error}) if the call fails
for some other reason.

\item [Description:]

Removes one or more fields from node-schema-ID. 
The parameter field-schema-IDs can also be a simple 
field-schema-ID, as well as a list of field-schema-IDs.


\item [Public:]



\end{description}
\horizontalline

\subsection{t*node-schema*add-fields}
\label{t*node-schema*add-fields}

\begin{description}
\item [Name:]  t*node-schema*add-fields

\item [Class:] {\sl t*node-schema}\hfill(page~\pageref{t*node-schema})

\item [Parameters:]
\item {\sl node-schema-ID}:  a server-level node-ID that corresponds to an 
instance of a type-level node-schema. 


\item {\sl field-schema-IDs}:  list of field-schema-ID


\item [Return-value:] 
The updated list of field-schema-IDs if successful.

Error object {\sl invalid-node-schema-ID} (page~\pageref{invalid-node-schema-ID}) if 
node-schema-ID was not a t*node-schema.

Error object {\sl invalid-field-ID} (page~\pageref{invalid-field-ID}) if any one of the 
field-schema-IDs is not a t*field-schema, or if any
one of the field-schemas already exists in node-schema-ID.

Error object {\sl unknown-hb-error} (page~\pageref{unknown-hb-error}) if the call 
fails for some other reason. 

\item [Description:]  Adds one or more field-schema-IDs to 
this node-schema. The parameter field-schema-IDs
can also be a simple field-schema-ID, as well as
a list of field-schema-IDs.

\item [Public:]



\end{description}
\horizontalline

\subsection{t*node-schema*instantiate}
\label{t*node-schema*instantiate}

\begin{description}
\item [Name:]  t*node-schema*instantiate

\item [Class:] {\sl t*node-schema}\hfill(page~\pageref{t*node-schema})

\item [Parameters:]
\item {\sl node-name}:  
A valid node name. This currently means that it is a
string of less than 40 characters, and that it does
not contain leading space(s) or tabs.

\item {\sl node-schema-ID}:  a server-level node-ID that corresponds to an 
instance of a type-level node-schema. 



\item [Return-value:] 
A newly created node-id with name node-name if successful.

Error object {\sl invalid-node-schema-ID} (page~\pageref{invalid-node-schema-ID}) if 
node-schema-ID was not a t*node-schema.

Error object {\sl invalid-node-name} (page~\pageref{invalid-node-name}) if node-name was
invalid.

Error object {\sl unknown-hb-error} (page~\pageref{unknown-hb-error}) if call fails
for some other reason.

\item [Description:]

Creates a new type-level node-instance based upon this
node-schema.

\item [Public:]



\end{description}
\horizontalline

\subsection{t*node-schema*make}
\label{t*node-schema*make}

\begin{description}
\item [Name:]  t*node-schema*make

\item [Class:] {\sl t*node-schema}\hfill(page~\pageref{t*node-schema})

\item [Parameters:]
\item {\sl node-name}:  
A valid node name. This currently means that it is a
string of less than 40 characters, and that it does
not contain leading space(s) or tabs.

\item {\sl field-schema-IDs}:  list of field-schema-ID


\item [Return-value:] 
A newly created node-schema-ID if successful.

Error object {\sl invalid-node-name} (page~\pageref{invalid-node-name}) if the node-name
was syntactically illegal.

Error object {\sl invalid-field-ID} (page~\pageref{invalid-field-ID}) if one or more of the
field-IDs is not a t*field-schema.

Error object {\sl unknown-hb-error} (page~\pageref{unknown-hb-error}) if the call fails
for some other reason.
 
\item [Description:]

Creates a new node-schema with field-IDs as its
structure, and returns the corresponding node-schema-ID.

\item [Public:]



\end{description}
\horizontalline

\subsection{t*\{node-schema\}*IDs}
\label{t*node-schema*IDs}

\begin{description}
\item [Name:]  t*\{node-schema\}*IDs

\item [Class:]
{\sl t*node-schema}\hfill(page~\pageref{t*node-schema})

\item [Parameters:] none

\item [Return-value:]

A list of t*node-schema IDs.

\item [Description:]

Returns a freshly consed list of all currently
defined t*node-schema IDs. 

\item [Public:]



\end{description}
\horizontalline

\subsection{t*\{node-schema\}*mapc-IDs}
\label{t*node-schema*mapc-IDs}

\begin{description}
\item [Name:]  t*\{node-schema\}*mapc-IDs

\item [Class:] {\sl t*node-schema}\hfill(page~\pageref{t*node-schema})

\item [Parameters:]
\item {\sl map-ID-fn}:  A function that takes one argument, an ID,
and which performs some side-effect based upon that
value.



\item [Return-value:] nil

\item [Description:]

Calls map-ID-fn on each currently defined node-schema-ID.

\item [Public:]



\end{description}
\horizontalline

\section{t*node-instance}
\label{t*node-instance}

\begin{description}
\item [Name:]  t*node-instance

\item [Layer:] {\sl Type}\hfill(page~\pageref{Type})

\item [Description:]

Each instance of this class corresponds to an actual
content-bearing node in Egret.  Each node instance has
an associated schema, which represents the current
consensus in the group about the appropriate field-level
structure for this node.  This agreement may or may not
correspond to the actual field-level structure of any
particular node-instance.

Note that node-instance is abbreviated to "node" in
the operation and attribute names. 

The contents of node instances are stored in a form
termed the "packed field" representation. Packed fields
are a presentation-independent way of storing field
values. 

Fields are packed by delimiting their contents with a
marker containing the field schema ID. For example, if a
"Name" field has the field-schema-ID 37, and its value
is "Foo", then its packed representation is:

{\tt<}{\tt<}\#\# 37 \#\#{\tt>}{\tt>}Foo{\tt<}{\tt<}\#\# 37 \#\#{\tt>}{\tt>}

Note that presentational issues, such as the labelling
of the field, or its font or color are determined by
reference to the associated field-schema.

Links are packed similarly, except that the link-ID
and destination node-ID are embedded in the marker as
well as the link-schema-ID. Thus,

{\tt<}{\tt<}\$\$ 62 128 34 \$\${\tt>}{\tt>} 

represents an occurrence of the link-ID 62, associated
with the link-schema 128, and pointing to the node-ID
34.  

The following example illustrates a typical use of links
within fields:

Description: See [-{\tt>} more-info]. 

This "Description" field is packed as follows:

{\tt<}{\tt<}\#\# 23 \#\#{\tt>}{\tt>}See {\tt<}{\tt<}\$\$ 12 35 98 \$\${\tt>}{\tt>}.{\tt<}{\tt<}\#\# 23 \#\#{\tt>}{\tt>}

Here, 23 is the field-schema-ID for the Description
field, 12 is the link-ID for the link called more-info,
35 is its link-schema-ID, and 98 is the node-ID that
more-info points to.


\item [Attributes:]
\item {\sl t*node*name}\hfill(page~\pageref{t*node*name})
\item {\sl t*node*schema-ID}\hfill(page~\pageref{t*node*schema-ID})
\item {\sl t*node*field-schema-IDs}\hfill(page~\pageref{t*node*field-schema-IDs})
\item {\sl t*node*incoming-link-IDs}\hfill(page~\pageref{t*node*incoming-link-IDs})
\item {\sl t*node*outgoing-link-IDs}\hfill(page~\pageref{t*node*outgoing-link-IDs})
\item {\sl t*node*layer-IDs}\hfill(page~\pageref{t*node*layer-IDs})

\item [Operations:]
\item {\sl t*node*set-name}\hfill(page~\pageref{t*node*set-name})
\item {\sl t*node*clone}\hfill(page~\pageref{t*node*clone})
\item {\sl t*node*delete}\hfill(page~\pageref{t*node*delete})
\item {\sl t*node*set-schema-ID}\hfill(page~\pageref{t*node*set-schema-ID})
\item {\sl t*node*add-field-schema-IDs}\hfill(page~\pageref{t*node*add-field-schema-IDs})
\item {\sl t*node*delete-field-schema-IDs}\hfill(page~\pageref{t*node*delete-field-schema-IDs})
\item {\sl t*node*add-layer-ID}\hfill(page~\pageref{t*node*add-layer-ID})
\item {\sl t*node*delete-layer-ID}\hfill(page~\pageref{t*node*delete-layer-ID})
\item {\sl t*node*field-values}\hfill(page~\pageref{t*node*field-values})
\item {\sl t*node*set-field-values}\hfill(page~\pageref{t*node*set-field-values})
\item {\sl t*node*lock}\hfill(page~\pageref{t*node*lock})
\item {\sl t*node*unlock}\hfill(page~\pageref{t*node*unlock})
\item {\sl t*node*convergence}\hfill(page~\pageref{t*node*convergence})
 
\item {\sl t*\{node\}*IDs}\hfill(page~\pageref{t*node*IDs})
\item {\sl t*\{node\}*mapc-IDs}\hfill(page~\pageref{t*node*mapc-IDs})

\item [Subclasses:]


\item [Superclasses:]


\item [Instances:]



\end{description}
\horizontalline

\subsection{t*node*name}
\label{t*node*name}

\begin{description}
\item [Name:]  t*node*name

\item [Class:] {\sl t*node-instance}\hfill(page~\pageref{t*node-instance})

\item [Contents:] string 

\item [Description:]

The name of this node. 

\item [Setf-able:] See t*node*set-name


\item [Public:]



\end{description}
\horizontalline

\subsection{t*node*schema-ID}
\label{t*node*schema-ID}

\begin{description}
\item [Name:]  t*node*schema-ID

\item [Class:] {\sl t*node-instance}\hfill(page~\pageref{t*node-instance})

\item [Contents:] a node-schema-ID

\item [Description:]

The consensual schema for this node.

\item [Setf-able:]


\item [Public:]



\end{description}
\horizontalline

\subsection{t*node*field-schema-IDs}
\label{t*node*field-schema-IDs}

\begin{description}
\item [Name:]  t*node*field-schema-IDs

\item [Class:] {\sl t*node-instance}\hfill(page~\pageref{t*node-instance})

\item [Contents:] a list of field-schema-IDs

\item [Description:]

The internal structure of the node instance. 

\item [Setf-able:]


\item [Public:]



\end{description}
\horizontalline

\subsection{t*node*incoming-link-IDs}
\label{t*node*incoming-link-IDs}

\begin{description}
\item [Name:]  t*node*incoming-link-IDs

\item [Class:] {\sl t*node-instance}\hfill(page~\pageref{t*node-instance})

\item [Contents:] a list of link-IDs

\item [Description:]

A list of the link-IDs pointing to this node.

\item [Setf-able:]


\item [Public:]



\end{description}
\horizontalline

\subsection{t*node*outgoing-link-IDs}
\label{t*node*outgoing-link-IDs}

\begin{description}
\item [Name:]  t*node*outgoing-link-IDs

\item [Class:] {\sl t*node-instance}\hfill(page~\pageref{t*node-instance})

\item [Contents:] a list of link-IDs

\item [Description:]

A list of the link-IDs outgoing from this node.

\item [Setf-able:]


\item [Public:]



\end{description}
\horizontalline

\subsection{t*node*layer-IDs}
\label{t*node*layer-IDs}

\begin{description}
\item [Name:]  t*node*layer-IDs

\item [Class:] {\sl t*node-instance}\hfill(page~\pageref{t*node-instance})

\item [Contents:] a list of layer-IDs

\item [Description:]

The list of layers to which this node instance 
currently belongs. 

\item [Setf-able:]


\item [Public:]



\end{description}
\horizontalline

\subsection{t*node*set-name}
\label{t*node*set-name}

\begin{description}
\item [Name:]  t*node*set-name

\item [Class:] {\sl t*node-instance}\hfill(page~\pageref{t*node-instance})

\item [Parameters:]
\item {\sl node-ID}:   An integer representing
a valid hbserver node ID.

\item {\sl node-name}:  
A valid node name. This currently means that it is a
string of less than 40 characters, and that it does
not contain leading space(s) or tabs.


\item [Return-value:]
The new node-name if successful.

Error object {\sl invalid-node-name} (page~\pageref{invalid-node-name}) if node-name not valid.

Error object {\sl unknown-hb-error} (page~\pageref{unknown-hb-error}) if call fails
for any other reason.

\item [Description:]

Updates the name of node-ID to node-name.

\item [Public:]



\end{description}
\horizontalline

\subsection{t*node*clone}
\label{t*node*clone}

\begin{description}
\item [Name:]  t*node*clone

\item [Class:] {\sl t*node-instance}\hfill(page~\pageref{t*node-instance})

\item [Parameters:]
\item {\sl node-name}:  
A valid node name. This currently means that it is a
string of less than 40 characters, and that it does
not contain leading space(s) or tabs.

\item {\sl node-ID}:   An integer representing
a valid hbserver node ID.


\item [Return-value:]
A new node-ID with name node-name corresponding to 
the cloned instance if successful.

Error object {\sl invalid-node-ID} (page~\pageref{invalid-node-ID}) if node-ID is not
an instance of t*node-instance.

Error object {\sl unknown-hb-error} (page~\pageref{unknown-hb-error}) if the call fails
for some other reason. 

\item [Description:]

Creates and initializes a new t*node-instance with 
the node-schema-ID and field-schema-ID structure of 
node-ID. However, it does not copy the contents of 
node-ID.

\item [Public:]



\end{description}
\horizontalline

\subsection{t*node*delete}
\label{t*node*delete}

\begin{description}
\item [Name:]  t*node*delete

\item [Class:] {\sl t*node-instance}\hfill(page~\pageref{t*node-instance})

\item [Parameters:]
\item {\sl node-ID}:   An integer representing
a valid hbserver node ID.


\item [Return-value:]
node-ID if successfully deleted.

Error object {\sl invalid-node-ID} (page~\pageref{invalid-node-ID}) if node-ID is 
not an instance of t*node-instance.

Error object {\sl node-still-locked} (page~\pageref{node-still-locked}) if node is locked
by some other user. 

Error object {\sl node-still-referenced} (page~\pageref{node-still-referenced}) if node
has incoming links. 

Error object {\sl unknown-hb-error} (page~\pageref{unknown-hb-error}) if call fails
for some other reason.

\item [Description:]

Deletes node-ID from the hyperbase.

\item [Public:]



\end{description}
\horizontalline

\subsection{t*node*set-schema-ID}
\label{t*node*set-schema-ID}

\begin{description}
\item [Name:]  t*node*set-schema-ID

\item [Class:] {\sl t*node-instance}\hfill(page~\pageref{t*node-instance})

\item [Parameters:]
\item {\sl node-ID}:   An integer representing
a valid hbserver node ID.

\item {\sl node-schema-ID}:  a server-level node-ID that corresponds to an 
instance of a type-level node-schema. 



\item [Return-value:]
The new node-schema-ID if successful.

Error object {\sl invalid-node-ID} (page~\pageref{invalid-node-ID}) if bad node-ID.

Error object {\sl invalid-node-schema-ID} (page~\pageref{invalid-node-schema-ID}) if bad
node-schema-ID.

Error object {\sl lock-node-fails} (page~\pageref{lock-node-fails}) if a lock can't be
obtained in preparation for the update. 

Error object {\sl unknown-hb-error} (page~\pageref{unknown-hb-error}) if call fails for
some other reason.

\item [Description:]

Sets the schema of node-ID to node-schema-ID. Note
that this does not change the existing structure of
node-ID at all. (It does potentially change the
convergence and divergence metric values for this
node schema and instance.)


\item [Public:]






\end{description}
\horizontalline

\subsection{t*node*add-field-schema-IDs}
\label{t*node*add-field-schema-IDs}

\begin{description}
\item [Name:]  t*node*add-field-schema-IDs

\item [Class:] {\sl t*node-instance}\hfill(page~\pageref{t*node-instance})

\item [Parameters:]
\item {\sl node-ID}:   An integer representing
a valid hbserver node ID.

\item {\sl field-schema-IDs}:  list of field-schema-ID


\item [Return-value:]
The updated list of field-schema-IDs if successful.

Error object {\sl invalid-node-ID} (page~\pageref{invalid-node-ID}) if node-ID is not an
instance of t*node-instance.

Error object {\sl invalid-field-ID} (page~\pageref{invalid-field-ID}) if any of 
field-schema-IDs are invalid or are already members of 
node-ID.

Error object {\sl lock-node-fails} (page~\pageref{lock-node-fails}) if a lock on node-ID
cannot be obtained in preparation for the update.

Error object {\sl unknown-hb-error} (page~\pageref{unknown-hb-error}) if the call fails
for any other reason. 

\item [Description:]

Adds the list of field-schema-IDs to the structure of
node-ID. The parameter field-schema-IDs can also be a 
simple field-schema-ID, as well as a list of 
field-schema-IDs.


\item [Public:]



\end{description}
\horizontalline

\subsection{t*node*delete-field-schema-IDs}
\label{t*node*delete-field-schema-IDs}

\begin{description}
\item [Name:]  t*node*delete-field-schema-IDs

\item [Class:] {\sl t*node-instance}\hfill(page~\pageref{t*node-instance})

\item [Parameters:]
\item {\sl node-ID}:   An integer representing
a valid hbserver node ID.

\item {\sl field-schema-IDs}:  list of field-schema-ID


\item [Return-value:]
The updated list of field-schema-IDs if successful.

Error object {\sl invalid-node-ID} (page~\pageref{invalid-node-ID}) if node-ID is not
a t*node-instance.

Error object {\sl invalid-field-ID} (page~\pageref{invalid-field-ID}) if any of the 
field-schema-IDs are not legal members of this
node-ID.

Error object {\sl lock-node-fails} (page~\pageref{lock-node-fails}) if node-ID cannot
be locked in preparation for the deletion. 

Error object {\sl unknown-hb-error} (page~\pageref{unknown-hb-error}) if call fails
for any other reason.

\item [Description:]

Deletes fields from an individual node instance. 
The parameter field-schema-IDs can also be a 
simple field-schema-ID, as well as a list of 
field-schema-IDs.

\item [Public:]



\end{description}
\horizontalline

\subsection{t*node*add-layer-ID}
\label{t*node*add-layer-ID}

\begin{description}
\item [Name:]  t*node*add-layer-ID

\item [Class:] {\sl t*node-instance}\hfill(page~\pageref{t*node-instance})

\item [Parameters:]
\item {\sl node-ID}:   An integer representing
a valid hbserver node ID.

\item {\sl layer-ID}:  a unique ID for layers (possibly a node-ID?)



\item [Return-value:]
Layer-ID if successful.

Error object {\sl invalid-node-ID} (page~\pageref{invalid-node-ID}) if node-ID is not
a t*node-instance.

Error object {\sl invalid-layer-ID} (page~\pageref{invalid-layer-ID}) if layer-ID is
not the ID of a layer instance, or if it already
contains this node-ID as a member.

Error object {\sl lock-node-fails} (page~\pageref{lock-node-fails}) if a lock on
node-ID cannot be obtained in preparation for this
update.

Error object {\sl unknown-hb-error} (page~\pageref{unknown-hb-error}) if call fails
for any other reason.

\item [Description:]

Makes node-ID a member of layer-ID.

\item [Public:]



\end{description}
\horizontalline

\subsection{t*node*delete-layer-ID}
\label{t*node*delete-layer-ID}

\begin{description}
\item [Name:]  t*node*delete-layer-ID

\item [Class:] {\sl t*node-instance}\hfill(page~\pageref{t*node-instance})

\item [Parameters:]
\item {\sl node-ID}:   An integer representing
a valid hbserver node ID.

\item {\sl layer-ID}:  a unique ID for layers (possibly a node-ID?)



\item [Return-value:]
The deleted layer-ID if successful.

Error object {\sl invalid-node-ID} (page~\pageref{invalid-node-ID}) if node-ID is
not a t*node-instance.

Error object {\sl invalid-layer-ID} (page~\pageref{invalid-layer-ID}) if bad layer-ID,
or if node-ID is not currently a member of layer-ID.

Error object {\sl lock-node-fails} (page~\pageref{lock-node-fails}) if a lock cannot
be obtained on node-ID (and perhaps layer-ID?).

Error object {\sl unknown-hb-error} (page~\pageref{unknown-hb-error}) if call fails
for any other reason.


\item [Description:]
Deletes this instance from the associated layer.

\item [Public:]



\end{description}
\horizontalline

\subsection{t*node*field-values}
\label{t*node*field-values}

\begin{description}
\item [Name:]  t*node*field-values

\item [Class:] {\sl t*node-instance}\hfill(page~\pageref{t*node-instance})

\item [Parameters:]
\item {\sl node-ID}:   An integer representing
a valid hbserver node ID.

\item {\sl field-schema-IDs}:  list of field-schema-ID

\item {\sl buffer-instance}:  an Emacs buffer instance object


\item [Return-value:]
Returns a buffer-instance containing a set of 
packed fields corresponding to field-schema-IDs
if successful.

Error object {\sl invalid-buffer-instance} (page~\pageref{invalid-buffer-instance}) if 
buffer-instance argument is supplied and not a 
legal buffer-instance.

Error object {\sl invalid-node-ID} (page~\pageref{invalid-node-ID}) if node-ID is
not a t*node-instance.

Error object {\sl invalid-field-ID} (page~\pageref{invalid-field-ID}) if
field-schema-IDs is supplied and any of the 
them are not t*field-schemas or not present in
node-ID. 

Error object {\sl unknown-hb-error} (page~\pageref{unknown-hb-error}) if call fails
for any other reason.

\item [Description:]

Calls the hyperbase and returns a buffer-instance
containing the packed field representation for the
fields corresponding to field-schema-IDs in node-ID.

Field-schema-IDs is an optional argument, and 
defaults to the field-schema-IDs in the data field
of the hyperbase node corresponding to node-ID.

Buffer-instance, if supplied, is a buffer-instance
that can be erased and returned with the requested
packed fields.  If not supplied, a new
buffer-instance is created and returned.

\item [Public:]



\end{description}
\horizontalline

\subsection{t*node*set-field-values}
\label{t*node*set-field-values}

\begin{description}
\item [Name:]  t*node*set-field-values

\item [Class:] {\sl t*node-instance}\hfill(page~\pageref{t*node-instance})

\item [Parameters:]
\item {\sl node-ID}:   An integer representing
a valid hbserver node ID.

\item {\sl field-schema-IDs}:  list of field-schema-ID

\item {\sl buffer-instance}:  an Emacs buffer instance object


\item [Return-value:]
T if node-ID is successfully updated with new
field values.

Error object {\sl invalid-node-ID} (page~\pageref{invalid-node-ID}) if node-ID
is not a t*node-instance.

Error object {\sl invalid-field-ID} (page~\pageref{invalid-field-ID}) if argument
field-schema-IDs is specified and if any of them
are not t*field-schemas or not present in node-ID.

Error object {\sl invalid-buffer-instance} (page~\pageref{invalid-buffer-instance}) if the 
buffer-instance argument is not a buffer instance,
or does not contain the fields specified in 
field-schema-IDs.

Error object {\sl lock-node-fails} (page~\pageref{lock-node-fails}) if lock cannot 
be obtained on node-ID in preparation for the update.

Error object {\sl unknown-hb-error} (page~\pageref{unknown-hb-error}) if call fails
for some other reason.

\item [Description:]
Updates the contents of node-ID with new field
values.

The optional parameter field-schema-IDs, if supplied,
indicates which fields in node-ID should be updated
from the set of packed fields contained in
buffer-instance.  Buffer-instance must contain the
packed representation of these fields, or an error is
signalled. 

If field-schema-IDs is not supplied, then
buffer-instance is assumed to contain the packed
field representation for all fields in node-ID (that
are stored in the hbserver data field).

Buffer-instance must contain exactly the packed field
representations for the field-schema-IDs specified by
the field-schema-ID argument (or implicitly by its
absence.)


\item [Public:]



\end{description}
\horizontalline

\subsection{t*node*lock}
\label{t*node*lock}

\begin{description}
\item [Name:]  t*node*lock

\item [Class:] {\sl t*node-instance}\hfill(page~\pageref{t*node-instance})

\item [Parameters:]
\item {\sl node-ID}:   An integer representing
a valid hbserver node ID.


\item [Return-value:]
T if the lock can be successfully obtained, or if
user already has a lock on node-ID.

NIL if another user has locked node-ID.

Error object {\sl invalid-node-ID} (page~\pageref{invalid-node-ID}) if node-ID is
not a t*node-instance.

Error object {\sl unknown-hb-error} (page~\pageref{unknown-hb-error}) if call fails
for some other reason.

\item [Description:]

Attempts to obtain a lock on node-ID. 

\item [Public:]



\end{description}
\horizontalline

\subsection{t*node*unlock}
\label{t*node*unlock}

\begin{description}
\item [Name:]  t*node*unlock

\item [Class:] {\sl t*node-instance}\hfill(page~\pageref{t*node-instance})

\item [Parameters:]
\item {\sl node-ID}:   An integer representing
a valid hbserver node ID.


\item [Return-value:] 
T if node-ID is successfully unlocked.

Error object {\sl invalid-node-ID} (page~\pageref{invalid-node-ID}) if node-ID is
not a t*node-instance.

Error object {\sl unknown-hb-error} (page~\pageref{unknown-hb-error}) if the call
fails for some other reason.

\item [Description:]

Releases the lock (if present) on node-ID by user.

Will succeed when node-ID is already unlocked.

Will fail if node-ID is locked by another user.


\item [Public:]



\end{description}
\horizontalline

\subsection{t*node*convergence}
\label{t*node*convergence}

\begin{description}
\item [Name:]  t*node*convergence

\item [Class:] {\sl t*node-instance}\hfill(page~\pageref{t*node-instance})

\item [Parameters:]
\item {\sl node-ID}:   An integer representing
a valid hbserver node ID.


\item [Return-value:] 
Integer convergence value if successful.

Error object {\sl invalid-node-ID} (page~\pageref{invalid-node-ID}) if node-ID is not a 
valid t*node-instance.

Error object {\sl unknown-hb-error} (page~\pageref{unknown-hb-error}) if the call fails
for some other reason.

\item [Description:]

Computes and returns an integer value representing 
the degree of convergence between the node
instance and its schema.

\item [Public:]



\end{description}
\horizontalline

\subsection{t*\{node\}*IDs}
\label{t*node*IDs}

\begin{description}
\item [Name:]  t*\{node\}*IDs

\item [Class:] {\sl t*node-instance}\hfill(page~\pageref{t*node-instance})

\item [Parameters:] none

\item [Return-value:] A list of node-IDs

\item [Description:]

Returns a freshly consed list of all the currently
defined type-level node-instances.  

Note that this operation may be expensive. 

\item [Public:]



\end{description}
\horizontalline

\subsection{t*\{node\}*mapc-IDs}
\label{t*node*mapc-IDs}

\begin{description}
\item [Name:]  t*\{node\}*mapc-IDs

\item [Class:] {\sl t*node-instance}\hfill(page~\pageref{t*node-instance})

\item [Parameters:]
\item {\sl map-ID-fn}:  A function that takes one argument, an ID,
and which performs some side-effect based upon that
value.



\item [Return-value:] nil

\item [Description:]

Calls map-ID-fn once with a node-ID corresponding
to each t*node-instance. 

\item [Public:]



\end{description}
\horizontalline

\section{t*link-schema}
\label{t*link-schema}

\begin{description}
\item [Name:]  t*link-schema

\item [Layer:] {\sl Type}\hfill(page~\pageref{Type})

\item [Description:]

This class defines the structural properties of link
types.

\item [Attributes:]
\item {\sl t*link-schema*name}\hfill(page~\pageref{t*link-schema*name})
\item {\sl t*link-schema*to-nodes}\hfill(page~\pageref{t*link-schema*to-nodes})
\item {\sl t*link-schema*from-nodes}\hfill(page~\pageref{t*link-schema*from-nodes})
\item {\sl t*link-schema*link-IDs}\hfill(page~\pageref{t*link-schema*link-IDs})

\item [Operations:]
\item {\sl t*link-schema*make}\hfill(page~\pageref{t*link-schema*make})
\item {\sl t*link-schema*instantiate}\hfill(page~\pageref{t*link-schema*instantiate})
\item {\sl t*link-schema*add-to-node}\hfill(page~\pageref{t*link-schema*add-to-node})
\item {\sl t*link-schema*add-from-node}\hfill(page~\pageref{t*link-schema*add-from-node})
\item {\sl t*link-schema*delete-to-node}\hfill(page~\pageref{t*link-schema*delete-to-node})
\item {\sl t*link-schema*delete-from-node}\hfill(page~\pageref{t*link-schema*delete-from-node})
\item {\sl t*link-schema*set-name}\hfill(page~\pageref{t*link-schema*set-name})
\item {\sl t*link-schema*divergence}\hfill(page~\pageref{t*link-schema*divergence})
\item {\sl t*link-schema*delete}\hfill(page~\pageref{t*link-schema*delete})

\item {\sl t*\{link-schema\}*IDs}\hfill(page~\pageref{t*link-schema*IDs})
\item {\sl t*\{link-schema\}*mapc-IDs}\hfill(page~\pageref{t*link-schema*mapc-IDs})

\item [Subclasses:]


\item [Superclasses:]


\item [Instances:]



\end{description}
\horizontalline

\subsection{t*link-schema*name}
\label{t*link-schema*name}

\begin{description}
\item [Name:]  t*link-schema*name

\item [Class:] {\sl t*link-schema}\hfill(page~\pageref{t*link-schema})

\item [Contents:] a symbol

\item [Description:]

The name of the link-schema.

\item [Setf-able:]


\item [Public:]



\end{description}
\horizontalline

\subsection{t*link-schema*to-nodes}
\label{t*link-schema*to-nodes}

\begin{description}
\item [Name:]  t*link-schema*to-nodes

\item [Class:] {\sl t*link-schema}\hfill(page~\pageref{t*link-schema})

\item [Contents:] A link constraint expression.

\item [Description:]

An expression of the form \{T {\tt|}({\tt<}node-schema-ID{\tt>}*)\}.
T indicates that any node-schema-ID is valid; otherwise,
the list indicates the set of legal node-schema-IDs.

\item [Setf-able:]


\item [Public:]



\end{description}
\horizontalline

\subsection{t*link-schema*from-nodes}
\label{t*link-schema*from-nodes}

\begin{description}
\item [Name:]  t*link-schema*from-nodes

\item [Class:] {\sl t*link-schema}\hfill(page~\pageref{t*link-schema})

\item [Contents:] A link constraint expression

\item [Description:]

An expression of the form \{T {\tt|}({\tt<}node-schema-ID{\tt>}*)\}.
T indicates that any node-schema-ID is valid; otherwise,
the list indicates the set of legal node-schema-IDs.

\item [Setf-able:]


\item [Public:]



\end{description}
\horizontalline

\subsection{t*link-schema*link-IDs}
\label{t*link-schema*link-IDs}

\begin{description}
\item [Name:]  t*link-schema*link-IDs

\item [Class:] {\sl t*link-schema}\hfill(page~\pageref{t*link-schema})

\item [Contents:] A list of link-IDs

\item [Description:]

The current links of this type.

\item [Setf-able:]


\item [Public:]



\end{description}
\horizontalline

\subsection{t*link-schema*make}
\label{t*link-schema*make}

\begin{description}
\item [Name:]  t*link-schema*make

\item [Class:] {\sl t*link-schema}\hfill(page~\pageref{t*link-schema})

\item [Parameters:]
\item {\sl link-name}:  string (30); a valid link name

\item {\sl from-constraint-exp}:  A link constraint expression.


\item {\sl to-constraint-exp}:  A link constraint expression



\item [Return-value:]

Returns a new link-schema-ID with 
associated link-name and to and from node constraints. 

Error object {\sl invalid-constraint-exp} (page~\pageref{invalid-constraint-exp}) if the to
or from constraint expressions are illegal.

Error object {\sl unknown-hb-error} (page~\pageref{unknown-hb-error}) if call fails for
any other reason. 


\item [Description:]


\item [Public:]



\end{description}
\horizontalline

\subsection{t*link-schema*instantiate}
\label{t*link-schema*instantiate}

\begin{description}
\item [Name:]  t*link-schema*instantiate

\item [Class:] {\sl t*link-schema}\hfill(page~\pageref{t*link-schema})

\item [Parameters:]
\item {\sl link-schema-ID}:  an ID for a link schema.

\item {\sl link-name}:  string (30); a valid link name

\item {\sl from-node-ID}:  node-ID

\item {\sl to-node-ID}:  node-ID



\item [Return-value:]
The newly created link-ID if successful.

Error object {\sl invalid-link-schema-ID} (page~\pageref{invalid-link-schema-ID}) if not
a legal link-schema ID.

Error object {\sl invalid-link-name} (page~\pageref{invalid-link-name}) if not a 
legal link name.

Error object {\sl invalid-node-ID} (page~\pageref{invalid-node-ID}) if either of 
the node-IDs are invalid or violate this link schema's
constraints. 

Error object {\sl unknown-hb-error} (page~\pageref{unknown-hb-error}) if call fails for
any other reason.

\item [Description:]

Creates a new link instance conforming to the constraints
specified for link-schema-ID. 

\item [Public:]



\end{description}
\horizontalline

\subsection{t*link-schema*add-to-node}
\label{t*link-schema*add-to-node}

\begin{description}
\item [Name:]  t*link-schema*add-to-node

\item [Class:] {\sl t*link-schema}\hfill(page~\pageref{t*link-schema})

\item [Parameters:]
\item {\sl link-schema-ID}:  an ID for a link schema.

\item {\sl node-schema-ID}:  a server-level node-ID that corresponds to an 
instance of a type-level node-schema. 



\item [Return-value:]
The updated to-node constraint expression if successful.

Error object {\sl invalid-node-schema-ID} (page~\pageref{invalid-node-schema-ID}) if second
arg is not a t*node-schema.

Error object {\sl invalid-link-schema-ID} (page~\pageref{invalid-link-schema-ID}) if first
arg is not a t*link-schema.

Error object {\sl unknown-hb-error} (page~\pageref{unknown-hb-error}) if call fails for 
any other reason.

\item [Description:]

Adds node-schema-ID to the list of to-node constraints,
and returns the new constraint expression.

If node-schema-ID is the symbol T, then the constraint
expression is set to T.

\item [Public:]



\end{description}
\horizontalline

\subsection{t*link-schema*add-from-node}
\label{t*link-schema*add-from-node}

\begin{description}
\item [Name:]  t*link-schema*add-from-node

\item [Class:] {\sl t*link-schema}\hfill(page~\pageref{t*link-schema})

\item [Parameters:]
\item {\sl link-schema-ID}:  an ID for a link schema.

\item {\sl node-schema-ID}:  a server-level node-ID that corresponds to an 
instance of a type-level node-schema. 



\item [Return-value:]
Returns the updated constraint expression if successful.

Error object {\sl invalid-link-schema-ID} (page~\pageref{invalid-link-schema-ID}) if first
arg is not a t*link-schema.

Error object {\sl invalid-node-schema-ID} (page~\pageref{invalid-node-schema-ID}) if the 
second arg is not a t*node-schema, or the distinguished
symbol T.

Error object {\sl unknown-hb-error} (page~\pageref{unknown-hb-error}) if the call fails
for any other reason. 

\item [Description:]

Adds node-schema-ID to the list of acceptable node
types for this from-node constraint expression, or 
sets the constraint expression to T if that symbol
is supplied.

\item [Public:]



\end{description}
\horizontalline

\subsection{t*link-schema*delete-to-node}
\label{t*link-schema*delete-to-node}

\begin{description}
\item [Name:]  t*link-schema*delete-to-node

\item [Class:] {\sl t*link-schema}\hfill(page~\pageref{t*link-schema})

\item [Parameters:]
\item {\sl link-schema-ID}:  an ID for a link schema.

\item {\sl node-schema-ID}:  a server-level node-ID that corresponds to an 
instance of a type-level node-schema. 



\item [Return-value:]
Returns the updated constraint expression if successful.
If the current to-node constraint expression is T, 
then this value is replaced by the current value
of t*\{node-schema\}*IDs before performing the deletion.

Error object {\sl invalid-link-schema-ID} (page~\pageref{invalid-link-schema-ID}) if first
arg is not a t*link-schema.

Error object {\sl invalid-node-schema-ID} (page~\pageref{invalid-node-schema-ID}) if second
arg is not a t*node-schema in the constraint 
expression for t*link-schema's to-nodes, or the 
distinguished symbol NIL. 

Error object {\sl unknown-hb-error} (page~\pageref{unknown-hb-error}) if call fails
for any other reason.

\item [Description:]

Deletes node-schema-ID from the to-node constraint
expression. If node-schema-ID is NIL, then the 
constraint expression is set to NIL.

\item [Public:]



\end{description}
\horizontalline

\subsection{t*link-schema*delete-from-node}
\label{t*link-schema*delete-from-node}

\begin{description}
\item [Name:]  t*link-schema*delete-from-node

\item [Class:] {\sl t*link-schema}\hfill(page~\pageref{t*link-schema})

\item [Parameters:]
\item {\sl link-schema-ID}:  an ID for a link schema.

\item {\sl node-schema-ID}:  a server-level node-ID that corresponds to an 
instance of a type-level node-schema. 



\item [Return-value:]
The updated constraint expression if successful.
If the current from-node constraint expression is T, 
then this value is replaced by the current value
of t*\{node-schema\}*IDs before performing the deletion.

Error object {\sl invalid-link-schema-ID} (page~\pageref{invalid-link-schema-ID}) if first
arg is not a t*link-schema.

Error object {\sl invalid-node-schema-ID} (page~\pageref{invalid-node-schema-ID}) if second
argument is not a node-schema-ID currently present
in the constraint expression, or the distinguished
value NIL.

Error object {\sl unknown-hb-error} (page~\pageref{unknown-hb-error}) if call fails
for any other reason.

\item [Description:]

Deletes node-schema-ID from the constraint expression
for link from-nodes.  If NIL is supplied rather than
a node-schema-ID, then the constraint expression
is set to NIL.


\item [Public:]



\end{description}
\horizontalline

\subsection{t*link-schema*set-name}
\label{t*link-schema*set-name}

\begin{description}
\item [Name:]  t*link-schema*set-name

\item [Class:] {\sl t*link-schema}\hfill(page~\pageref{t*link-schema})

\item [Parameters:]
\item {\sl link-schema-ID}:  an ID for a link schema.

\item {\sl link-name}:  string (30); a valid link name



\item [Return-value:]
T if successful.

Error object {\sl invalid-link-name} (page~\pageref{invalid-link-name}) if illegal link-name.

Error object {\sl invalid-link-schema-ID} (page~\pageref{invalid-link-schema-ID}) if first
argument is not a t*link-schema.

Error object {\sl unknown-hb-error} (page~\pageref{unknown-hb-error}) if call fails for
any other reason.

\item [Description:]

Changes the name of link-schema-ID to link-name.

\item [Public:]



\end{description}
\horizontalline

\subsection{t*link-schema*divergence}
\label{t*link-schema*divergence}

\begin{description}
\item [Name:]  t*link-schema*divergence

\item [Class:] {\sl t*link-schema}\hfill(page~\pageref{t*link-schema})

\item [Parameters:]
\item {\sl link-schema-ID}:  an ID for a link schema.


\item [Return-value:]
An integer divergence value if successful.

Error object {\sl unknown-hb-error} (page~\pageref{unknown-hb-error}) if call fails
for any reason.

\item [Description:]

Computes the divergence metric for link-schema 
instances.

\item [Public:]



\end{description}
\horizontalline

\subsection{t*link-schema*delete}
\label{t*link-schema*delete}

\begin{description}
\item [Name:]  t*link-schema*delete

\item [Class:] {\sl t*link-schema}\hfill(page~\pageref{t*link-schema})

\item [Parameters:]
\item {\sl link-schema-ID}:  an ID for a link schema.


\item [Return-value:]
The deleted link-schema-ID if successful.

Error object {\sl invalid-link-schema-ID} (page~\pageref{invalid-link-schema-ID}) if argument
is not a link-schema-ID.

Error object {\sl unknown-hb-error} (page~\pageref{unknown-hb-error}) if call fails
for some other reason.

\item [Description:]

Deletes t*link-schema-ID.  Note that this does not
physically remove link-schema-ID from the database.
Rather, it simply prevents any new instances of 
link-schema-ID from being created.

\item [Public:]



\end{description}
\horizontalline

\subsection{t*\{link-schema\}*IDs}
\label{t*link-schema*IDs}

\begin{description}
\item [Name:]  t*\{link-schema\}*IDs

\item [Class:] {\sl t*link-schema}\hfill(page~\pageref{t*link-schema})

\item [Parameters:] none

\item [Return-value:]

A list of link-schema-IDs.

\item [Description:]

Returns a freshly consed list of all currently 
defined link-schema IDs.

\item [Public:]



\end{description}
\horizontalline

\subsection{t*\{link-schema\}*mapc-IDs}
\label{t*link-schema*mapc-IDs}

\begin{description}
\item [Name:]  t*\{link-schema\}*mapc-IDs

\item [Class:] {\sl t*link-schema}\hfill(page~\pageref{t*link-schema})

\item [Parameters:]
\item {\sl map-ID-fn}:  A function that takes one argument, an ID,
and which performs some side-effect based upon that
value.



\item [Return-value:] nil

\item [Description:]

Calls map-ID-fn on each currently defined link-schema ID.

\item [Public:]



\end{description}
\horizontalline

\section{t*link-instance}
\label{t*link-instance}

\begin{description}
\item [Name:]  t*link-instance

\item [Layer:] {\sl Type}\hfill(page~\pageref{Type})

\item [Description:]

The instances of this class define links between nodes. 

\item [Attributes:]
\item {\sl t*link*name}\hfill(page~\pageref{t*link*name})
\item {\sl t*link*schema-ID}\hfill(page~\pageref{t*link*schema-ID})
\item {\sl t*link*to-node-ID}\hfill(page~\pageref{t*link*to-node-ID})
\item {\sl t*link*from-node-ID}\hfill(page~\pageref{t*link*from-node-ID})
\item {\sl t*link*layer-IDs}\hfill(page~\pageref{t*link*layer-IDs})

\item [Operations:]
\item {\sl t*link*set-name}\hfill(page~\pageref{t*link*set-name})
\item {\sl t*link*clone}\hfill(page~\pageref{t*link*clone})
\item {\sl t*link*delete}\hfill(page~\pageref{t*link*delete})
\item {\sl t*link*set-schema-ID}\hfill(page~\pageref{t*link*set-schema-ID})
\item {\sl t*link*set-to-node}\hfill(page~\pageref{t*link*set-to-node})
\item {\sl t*link*add-to-constraint}\hfill(page~\pageref{t*link*add-to-constraint})
\item {\sl t*link*add-from-constraint}\hfill(page~\pageref{t*link*add-from-constraint})
\item {\sl t*link*delete-to-constraint}\hfill(page~\pageref{t*link*delete-to-constraint})
\item {\sl t*link*delete-from-constraint}\hfill(page~\pageref{t*link*delete-from-constraint})
\item {\sl t*link*add-layer-ID}\hfill(page~\pageref{t*link*add-layer-ID})
\item {\sl t*link*delete-layer-ID}\hfill(page~\pageref{t*link*delete-layer-ID})
\item {\sl t*link*convergence}\hfill(page~\pageref{t*link*convergence})

\item {\sl t*\{link\}*IDs}\hfill(page~\pageref{t*link*IDs})
\item {\sl t*\{link\}*mapc-IDs}\hfill(page~\pageref{t*link*mapc-IDs})

\item [Subclasses:]


\item [Superclasses:]


\item [Instances:]



\end{description}
\horizontalline

\subsection{t*link*name}
\label{t*link*name}

\begin{description}
\item [Name:]  t*link*name

\item [Class:] {\sl t*link-instance}\hfill(page~\pageref{t*link-instance})

\item [Contents:]
String.

\item [Description:]

The name of this link.

\item [Setf-able:] See t*link*set-name.


\item [Public:]



\end{description}
\horizontalline

\subsection{t*link*schema-ID}
\label{t*link*schema-ID}

\begin{description}
\item [Name:]  t*link*schema-ID

\item [Class:] {\sl t*link-instance}\hfill(page~\pageref{t*link-instance})

\item [Contents:]
A link-schema-ID

\item [Description:]

The link schema from this link instance.

\item [Setf-able:]


\item [Public:]



\end{description}
\horizontalline

\subsection{t*link*to-node-ID}
\label{t*link*to-node-ID}

\begin{description}
\item [Name:]  t*link*to-node-ID

\item [Class:] {\sl t*link-instance}\hfill(page~\pageref{t*link-instance})

\item [Contents:]
A node-ID.

\item [Description:]

The node-ID to which this link instance points. 

\item [Setf-able:]


\item [Public:]



\end{description}
\horizontalline

\subsection{t*link*from-node-ID}
\label{t*link*from-node-ID}

\begin{description}
\item [Name:]  t*link*from-node-ID

\item [Class:] {\sl t*link-instance}\hfill(page~\pageref{t*link-instance})

\item [Contents:] 
A node-ID

\item [Description:]

The node-ID from which this link originates.

\item [Setf-able:]


\item [Public:]



\end{description}
\horizontalline

\subsection{t*link*layer-IDs}
\label{t*link*layer-IDs}

\begin{description}
\item [Name:]  t*link*layer-IDs

\item [Class:] {\sl t*link-instance}\hfill(page~\pageref{t*link-instance})

\item [Contents:] A list of layer-IDs

\item [Description:]

The list of layer-IDs to which this link instance belongs.

\item [Setf-able:]


\item [Public:]



\end{description}
\horizontalline

\subsection{t*link*set-name}
\label{t*link*set-name}

\begin{description}
\item [Name:]  t*link*set-name

\item [Class:] {\sl t*link-instance}\hfill(page~\pageref{t*link-instance})

\item [Parameters:]
\item {\sl link-ID}:  
valid HB link ID number (integer)

\item {\sl link-name}:  string (30); a valid link name


\item [Return-value:]

Returns the updated link name if successful.

Error object {\sl invalid-link-ID} (page~\pageref{invalid-link-ID}) if not a 
t*link. 

Error object {\sl invalid-link-name} (page~\pageref{invalid-link-name}) if not a 
legal link name.

Error object {\sl unknown-hb-error} (page~\pageref{unknown-hb-error}) if call fails
for any other reason.

\item [Description:]

Updates the name of link-ID.

\item [Public:]



\end{description}
\horizontalline

\subsection{t*link*clone}
\label{t*link*clone}

\begin{description}
\item [Name:]  t*link*clone   

\item [Class:] {\sl t*link-instance}\hfill(page~\pageref{t*link-instance})

\item [Parameters:]
\item {\sl link-name}:  string (30); a valid link name

\item {\sl link-ID}:  
valid HB link ID number (integer)

\item {\sl from-node-ID}:  node-ID

\item {\sl to-node-ID}:  node-ID



\item [Return-value:]
The newly created link-ID with name link-name
and the same schema, to, and from nodes as link-ID.

\item [Description:]

Analogously to t*node*clone, this operation is
intended to clone link instances. However, the 
semantics of this operation are not yet clear.

Its implementation is temporarily deferred.

\item [Public:]



\end{description}
\horizontalline

\subsection{t*link*delete}
\label{t*link*delete}

\begin{description}
\item [Name:]  t*link*delete

\item [Class:] {\sl t*link-instance}\hfill(page~\pageref{t*link-instance})

\item [Parameters:]
\item {\sl link-ID}:  
valid HB link ID number (integer)


\item [Return-value:]
The deleted link-ID if successful.

Error object {\sl invalid-link-ID} (page~\pageref{invalid-link-ID}) if not a t*link-instance.

Error object {\sl unknown-hb-error} (page~\pageref{unknown-hb-error}) if call fails
for any other reason.

\item [Description:]

Deletes link-ID.

\item [Public:]



\end{description}
\horizontalline

\subsection{t*link*set-schema-ID}
\label{t*link*set-schema-ID}

\begin{description}
\item [Name:]  t*link*set-schema-ID

\item [Class:] {\sl t*link-instance}\hfill(page~\pageref{t*link-instance})

\item [Parameters:]
\item {\sl link-ID}:  
valid HB link ID number (integer)

\item {\sl link-schema-ID}:  an ID for a link schema.


\item [Return-value:]
The updated link-schema-ID if successful.

Error object {\sl invalid-link-ID} (page~\pageref{invalid-link-ID}) if not a t*link-ID.

Error object {\sl invalid-link-schema-ID} (page~\pageref{invalid-link-schema-ID}) if not
a t*link-schema.

Error object {\sl unknown-hb-error} (page~\pageref{unknown-hb-error}) if call fails
for any other reason.

\item [Description:]

Updates the schema associated with this 

\item [Public:]

























\end{description}
\horizontalline

\subsection{t*link*set-to-node}
\label{t*link*set-to-node}

\begin{description}
\item [Name:]  t*link*set-to-node

\item [Class:] {\sl t*link-instance}\hfill(page~\pageref{t*link-instance})

\item [Parameters:]
\item {\sl link-ID}:  
valid HB link ID number (integer)

\item {\sl node-ID}:   An integer representing
a valid hbserver node ID.


\item [Return-value:]
Returns the updated to-node ID if successful.

Error object {\sl invalid-link-ID} (page~\pageref{invalid-link-ID}) if not a t*link.

Error object {\sl invalid-node-ID} (page~\pageref{invalid-node-ID}) if not a t*node.

Error object {\sl unknown-hb-error} (page~\pageref{unknown-hb-error}) if call fails
for any other reason.

\item [Description:]

Updates link-ID to point to node-ID.

\item [Public:]



\end{description}
\horizontalline

\subsection{t*link*add-to-constraint}
\label{t*link*add-to-constraint}

\begin{description}
\item [Name:]  t*link*add-to-constraint

\item [Class:] {\sl t*link-instance}\hfill(page~\pageref{t*link-instance})

\item [Parameters:]
\item {\sl link-ID}:  
valid HB link ID number (integer)

\item {\sl node-schema-ID}:  a server-level node-ID that corresponds to an 
instance of a type-level node-schema. 



\item [Return-value:]
The updated constraint expression if successful.

Error object {\sl invalid-link-ID} (page~\pageref{invalid-link-ID}) if not a t*link-instance.

Error object {\sl invalid-node-schema-ID} (page~\pageref{invalid-node-schema-ID}) if not a
t*node-schema.

Error object {\sl unknown-hb-error} (page~\pageref{unknown-hb-error}) if call fails for
any other reason.

\item [Description:]

Adds a to-node constraint. If node-schema-ID is the
distinguished symbol T, then the constraint expression
is set to T.

\item [Public:]



\end{description}
\horizontalline

\subsection{t*link*add-from-constraint}
\label{t*link*add-from-constraint}

\begin{description}
\item [Name:]  t*link*add-from-constraint

\item [Class:] {\sl t*link-instance}\hfill(page~\pageref{t*link-instance})

\item [Parameters:]
\item {\sl link-ID}:  
valid HB link ID number (integer)

\item {\sl node-schema-ID}:  a server-level node-ID that corresponds to an 
instance of a type-level node-schema. 



\item [Return-value:]
The updated from-node constraint expression if successful.

Error object {\sl invalid-link-ID} (page~\pageref{invalid-link-ID}) if not a t*link-instance.

Error object {\sl invalid-node-schema-ID} (page~\pageref{invalid-node-schema-ID}) if not
a t*node-schema.

Error object {\sl unknown-hb-error} (page~\pageref{unknown-hb-error}) if call fails for
any other reason.

\item [Description:]

Adds node-schema-ID to the from-constraints for this
link instance. Node-schema-ID may also be the distinguished
symbol T, which sets the constraint expression itself
to T.

\item [Public:]



\end{description}
\horizontalline

\subsection{t*link*delete-to-constraint}
\label{t*link*delete-to-constraint}

\begin{description}
\item [Name:]\item [Name:]  t*link*delete-to-constraint

\item [Class:]\item [Class:] {\sl t*link-instance}\hfill(page~\pageref{t*link-instance}){\sl t*link-instance}\hfill(page~\pageref{t*link-instance})

\item [Parameters:]\item [Parameters:]
{\sl node-schema-ID}:  a server-level node-ID that corresponds to an 
instance of a type-level node-schema. 

updated to node constraint expression if successful.

Error object {\sl invalid-link-ID} (page~\pageref{invalid-link-ID}) if not a t*link-ID.

Error object {\sl invalid-node-schema-ID} (page~\pageref{invalid-node-schema-ID}) if not a 
t*node-schema, and if not currently a member of the 
to-node constraints.

\item [Description:]

Deletes a to-node constraint.

\item [Public:]



\end{description}
\horizontalline

\subsection{t*link*delete-from-constraint}
\label{t*link*delete-from-constraint}

\begin{description}
\item [Name:]  t*link*delete-from-constraint

\item [Class:] {\sl t*link-instance}\hfill(page~\pageref{t*link-instance})

\item [Parameters:]
\item {\sl link-ID}:  
valid HB link ID number (integer)

\item {\sl node-schema-ID}:  a server-level node-ID that corresponds to an 
instance of a type-level node-schema. 



\item [Return-value:]
Returns the updated from-node constraint expression
if successful.

Error object {\sl invalid-link-ID} (page~\pageref{invalid-link-ID}) if not a t*link-instance.

Error object {\sl invalid-node-schema-ID} (page~\pageref{invalid-node-schema-ID}) if not
a t*node-schema, and not a member of the constraint
expression.

Error object {\sl unknown-hb-error} (page~\pageref{unknown-hb-error}) if call fails for
any other reason.

\item [Description:]

Deletes a from constraint from this node instance's
constraint expression.

\item [Public:]



\end{description}
\horizontalline

\subsection{t*link*add-layer-ID}
\label{t*link*add-layer-ID}

\begin{description}
\item [Name:]  t*link*add-layer-ID

\item [Class:] {\sl t*link-instance}\hfill(page~\pageref{t*link-instance})

\item [Parameters:]
\item {\sl link-ID}:  
valid HB link ID number (integer)

\item {\sl layer-ID}:  a unique ID for layers (possibly a node-ID?)



\item [Return-value:]
The updated list of layer-IDs if successful.

Error object {\sl invalid-link-ID} (page~\pageref{invalid-link-ID}) if not a legal
t*link-ID.

Error object {\sl invalid-layer-ID} (page~\pageref{invalid-layer-ID}) if not a legal
layer-ID, or if it already occurs in this link-instance.

Error object {\sl unknown-hb-error} (page~\pageref{unknown-hb-error}) if call fails
for any other reason.

\item [Description:]

Makes link-ID a member of layer-ID.

\item [Public:]



\end{description}
\horizontalline

\subsection{t*link*delete-layer-ID}
\label{t*link*delete-layer-ID}

\begin{description}
\item [Name:]  t*link*delete-layer-ID

\item [Class:] {\sl t*link-instance}\hfill(page~\pageref{t*link-instance})

\item [Parameters:]
\item {\sl link-ID}:  
valid HB link ID number (integer)

\item {\sl layer-ID}:  a unique ID for layers (possibly a node-ID?)

 

\item [Return-value:]
The updated list of layer-IDs if successful.

Error object {\sl invalid-link-ID} (page~\pageref{invalid-link-ID}) if link-ID was
not a t*link-instance.

Error object {\sl invalid-layer-ID} (page~\pageref{invalid-layer-ID}) if layer-ID was
not a t*layer, or not a current member of this link
instance's layer-IDs.

Error object {\sl unknown-hb-error} (page~\pageref{unknown-hb-error}) if call fails
for any other reason. 

\item [Description:]

Deletes a layer-ID membership from this link.

\item [Public:]



\end{description}
\horizontalline

\subsection{t*link*convergence}
\label{t*link*convergence}

\begin{description}
\item [Name:]  t*link*convergence

\item [Class:] {\sl t*link-instance}\hfill(page~\pageref{t*link-instance})

\item [Parameters:]
\item {\sl link-ID}:  
valid HB link ID number (integer)



\item [Return-value:]

An integer convergence value.

\item [Description:]

Returns the convergence between this link instance and
the set of link-schemas if successful.

Error object {\sl unknown-hb-error} (page~\pageref{unknown-hb-error}) if call fails for
any reason.

\item [Public:]



\end{description}
\horizontalline

\subsection{t*\{link\}*IDs}
\label{t*link*IDs}

\begin{description}
\item [Name:]  t*\{link\}*IDs

\item [Class:] {\sl t*link-instance}\hfill(page~\pageref{t*link-instance})

\item [Parameters:] none

\item [Return-value:] A list of link-IDs

\item [Description:]

Returns a freshly consed list of link-IDs.

\item [Public:]



\end{description}
\horizontalline

\subsection{t*\{link\}*mapc-IDs}
\label{t*link*mapc-IDs}

\begin{description}
\item [Name:]  t*\{link\}*mapc-IDs

\item [Class:] {\sl t*link-instance}\hfill(page~\pageref{t*link-instance})

\item [Parameters:]
\item {\sl map-ID-fn}:  A function that takes one argument, an ID,
and which performs some side-effect based upon that
value.



\item [Return-value:] nil

\item [Description:]

Calls map-ID-fn on each currently defined link-ID.

\item [Public:]



\end{description}
\horizontalline

\section{t*field-schema}
\label{t*field-schema}

\begin{description}
\item [Name:]  t*field-schema

\item [Layer:] {\sl Type}\hfill(page~\pageref{Type})

\item [Description:]

Field schemas identify and define the internal structure
of node instances.  This structure is enforced via
a user-specified validity function.

\item [Attributes:]
\item {\sl t*field-schema*name}\hfill(page~\pageref{t*field-schema*name})
\item {\sl t*field-schema*validity-fn}\hfill(page~\pageref{t*field-schema*validity-fn})


\item [Operations:]
\item {\sl t*field-schema*make}\hfill(page~\pageref{t*field-schema*make})
\item {\sl t*field-schema*delete}\hfill(page~\pageref{t*field-schema*delete})
\item {\sl t*field-schema*set-name}\hfill(page~\pageref{t*field-schema*set-name})
\item {\sl t*field-schema*set-validity-fn}\hfill(page~\pageref{t*field-schema*set-validity-fn})

\item {\sl t*\{field-schema\}*IDs}\hfill(page~\pageref{t*field-schema*IDs})
\item {\sl t*\{field-schema\}*mapc-IDs}\hfill(page~\pageref{t*field-schema*mapc-IDs}) 

\item [Subclasses:]


\item [Superclasses:]


\item [Instances:]



\end{description}
\horizontalline

\subsection{t*field-schema*name}
\label{t*field-schema*name}

\begin{description}
\item [Name:]  t*field-schema*name

\item [Class:] {\sl t*field-schema}\hfill(page~\pageref{t*field-schema})

\item [Contents:] a symbol

\item [Description:]

The name of the field. 

\item [Setf-able:] See t*field-schema*set-name.


\item [Public:]



\end{description}
\horizontalline

\subsection{t*field-schema*validity-fn}
\label{t*field-schema*validity-fn}

\begin{description}
\item [Name:]  t*field-schema*validity-fn

\item [Class:] {\sl t*field-schema}\hfill(page~\pageref{t*field-schema})

\item [Contents:] A list

\item [Description:]

This is an Emacs-Lisp function (in lambda list format) for 
assessing the validity of an object to be stored in a 
field. 

The validity function takes one argument, the field 
value, and should return T if the field value is legal
and NIL if the field value is not. 

\item [Setf-able:] See t*field-schema*set-validity-fn

\item [Public:]



\end{description}
\horizontalline

\subsection{t*field-schema*make}
\label{t*field-schema*make}

\begin{description}

\item [Name:]  t*field-schema*make


\item [Class:]
{\sl t*field-schema}\hfill(page~\pageref{t*field-schema})


\item [Parameters:]
\item {\sl node-name}:  
A valid node name. This currently means that it is a
string of less than 40 characters, and that it does
not contain leading space(s) or tabs.

\item {\sl field-validity-fn}:  

A lambda list representing a funcallable Emacs-Lisp function for
assessing the contents of a field.

The function takes one argument, the contents of the field.
See t*field-schema for more information on the field content
structure.



\item [Return-value:]
The newly created field-schema-ID if successful. 

Error object {\sl invalid-node-name} (page~\pageref{invalid-node-name}) if node-name is not
legal.

Error object {\sl unknown-hb-error} (page~\pageref{unknown-hb-error}) if call fails for 
any other reason.

\item [Description:]

Creates a new field schema.  

\item [Public:]



\end{description}
\horizontalline

\subsection{t*field-schema*delete}
\label{t*field-schema*delete}

\begin{description}
\item [Name:]  t*field-schema*delete

\item [Class:] {\sl t*field-schema}\hfill(page~\pageref{t*field-schema})

\item [Parameters:]
\item {\sl field-schema-ID}:  an integer signifying a field-schema.


\item [Return-value:]
The deleted field-schema-ID if successful.

Error object {\sl invalid-field-ID} (page~\pageref{invalid-field-ID}) if field-schema-ID
is not a t*field-schema.

Error object {\sl unknown-hb-error} (page~\pageref{unknown-hb-error}) if call fails
for any other reason.

\item [Description:]

Deletes the field-schema.  This does not physically
remove the schema from the database, but instead
prevents any new instances of it from being made.

For the time being, node schemas and instances are
not automatically flagged as containing a deleted
field schema. 


\item [Public:]



\end{description}
\horizontalline

\subsection{t*field-schema*set-name}
\label{t*field-schema*set-name}

\begin{description}
\item [Name:]  t*field-schema*set-name

\item [Class:] {\sl t*field-schema}\hfill(page~\pageref{t*field-schema})

\item [Parameters:]
\item {\sl field-schema-ID}:  an integer signifying a field-schema.

\item {\sl node-name}:  
A valid node name. This currently means that it is a
string of less than 40 characters, and that it does
not contain leading space(s) or tabs.


\item [Return-value:]
T if successful.

Error object {\sl invalid-node-name} (page~\pageref{invalid-node-name}) if node-name
is syntactically incorrect.

Error object {\sl invalid-field-ID} (page~\pageref{invalid-field-ID}) if field-schema-ID
is not a t*field-schema.

Error object {\sl unknown-hb-error} (page~\pageref{unknown-hb-error}) if the call fails
for any other reason.

\item [Description:]

Sets the name attribute of the field-schema.

\item [Public:]



\end{description}
\horizontalline

\subsection{t*field-schema*set-validity-fn}
\label{t*field-schema*set-validity-fn}

\begin{description}
\item [Name:]  t*field-schema*set-validity-fn

\item [Class:] {\sl t*field-schema}\hfill(page~\pageref{t*field-schema})

\item [Parameters:]
\item {\sl field-schema-ID}:  an integer signifying a field-schema.

\item {\sl field-validity-fn}:  

A lambda list representing a funcallable Emacs-Lisp function for
assessing the contents of a field.

The function takes one argument, the contents of the field.
See t*field-schema for more information on the field content
structure.




\item [Return-value:] 
T if successful.

Error object {\sl invalid-field-ID} (page~\pageref{invalid-field-ID}) if field-schema-ID is not a valid t*field-schema.

Error object {\sl unknown-hb-error} (page~\pageref{unknown-hb-error}) if the call fails
for any other reason. 

\item [Description:]

Sets the validity function associated with field-schema-ID
to a new value. 


\item [Public:]



\end{description}
\horizontalline

\subsection{t*\{field-schema\}*IDs}
\label{t*field-schema*IDs}

\begin{description}
\item [Name:]  t*\{field-schema\}*IDs

\item [Class:] {\sl t*field-schema}\hfill(page~\pageref{t*field-schema})

\item [Parameters:] none

\item [Return-value:] A list of field-schema IDs.

\item [Description:]

Returns a freshly consed list of all currently defined
field-schema IDs.

\item [Public:]



\end{description}
\horizontalline

\subsection{t*\{field-schema\}*mapc-IDs}
\label{t*field-schema*mapc-IDs}

\begin{description}
\item [Name:]  t*\{field-schema\}*mapc-IDs

\item [Class:] {\sl t*field-schema}\hfill(page~\pageref{t*field-schema})

\item [Parameters:]
\item {\sl map-ID-fn}:  A function that takes one argument, an ID,
and which performs some side-effect based upon that
value.



\item [Return-value:] nil

\item [Description:]

Calls map-ID-fn on each currently defined field-schema-ID.

\item [Public:]



\end{description}
\horizontalline

\section{t*layer}
\label{t*layer}

\begin{description}
\item [Name:]  t*layer

\item [Layer:] {\sl Type}\hfill(page~\pageref{Type})

\item [Description:]

Defines the exploratory namespace mechanism in Egret.

\item [Attributes:]
\item {\sl t*layer*name}\hfill(page~\pageref{t*layer*name})
\item {\sl t*layer*node-IDs}\hfill(page~\pageref{t*layer*node-IDs})
\item {\sl t*layer*link-IDs}\hfill(page~\pageref{t*layer*link-IDs})

\item [Operations:]
\item {\sl t*\{layer\}*mapc-IDs}\hfill(page~\pageref{t*layer*mapc-IDs})
\item {\sl t*\{layer\}*IDs}\hfill(page~\pageref{t*layer*IDs})
\item {\sl t*layer*make}\hfill(page~\pageref{t*layer*make})
\item {\sl t*layer*add-node-ID}\hfill(page~\pageref{t*layer*add-node-ID})
\item {\sl t*layer*delete-node-ID}\hfill(page~\pageref{t*layer*delete-node-ID})
\item {\sl t*layer*add-link-ID}\hfill(page~\pageref{t*layer*add-link-ID})
\item {\sl t*layer*delete-link-ID}\hfill(page~\pageref{t*layer*delete-link-ID})
\item {\sl t*layer*divergence}\hfill(page~\pageref{t*layer*divergence})


\item [Subclasses:]


\item [Superclasses:]


\item [Instances:]
























\end{description}
\horizontalline

\subsection{t*layer*name}
\label{t*layer*name}

\begin{description}
\item [Name:]  t*layer*name

\item [Class:] {\sl t*layer}\hfill(page~\pageref{t*layer})

\item [Contents:]
symbol

\item [Description:]
The name of the layer.

\item [Setf-able:]


\item [Public:]



\end{description}
\horizontalline

\subsection{t*layer*node-IDs}
\label{t*layer*node-IDs}

\begin{description}
\item [Name:]  t*layer*node-IDs

\item [Class:] {\sl t*layer}\hfill(page~\pageref{t*layer})

\item [Contents:]
A list of node-IDs.

\item [Description:]

The node-IDs currently belonging to this layer.

\item [Setf-able:]


\item [Public:]



\end{description}
\horizontalline

\subsection{t*layer*link-IDs}
\label{t*layer*link-IDs}

\begin{description}
\item [Name:]  t*layer*link-IDs

\item [Class:] {\sl t*layer}\hfill(page~\pageref{t*layer})

\item [Contents:] 
A list of link-IDs

\item [Description:]

The link-IDs belonging to this layer.

\item [Setf-able:]


\item [Public:]



\end{description}
\horizontalline

\subsection{t*\{layer\}*mapc-IDs}
\label{t*layer*mapc-IDs}

\begin{description}
\item [Name:]  t*\{layer\}*mapc-IDs

\item [Class:] {\sl t*layer}\hfill(page~\pageref{t*layer})

\item [Parameters:]
\item {\sl map-ID-fn}:  A function that takes one argument, an ID,
and which performs some side-effect based upon that
value.



\item [Return-value:] nil

\item [Description:]

Calls map-ID-fn on each currently defined layer-ID.

\item [Public:]



\end{description}
\horizontalline

\subsection{t*\{layer\}*IDs}
\label{t*layer*IDs}

\begin{description}
\item [Name:]  t*\{layer\}*IDs

\item [Class:] {\sl t*layer}\hfill(page~\pageref{t*layer})

\item [Parameters:] none

\item [Return-value:]

A list of layer-IDs.

\item [Description:]

Returns a freshly consed list of all currently
defined layer-IDs.

\item [Public:]



\end{description}
\horizontalline

\subsection{t*layer*make}
\label{t*layer*make}

\begin{description}
\item [Name:]  t*layer*make

\item [Class:] {\sl t*layer}\hfill(page~\pageref{t*layer})

\item [Parameters:]
\item {\sl layer-name}:  a string less than 40 characters, obeying
all node-name restrictions. 


\item {\sl node-IDs}:  a list of node-IDs


\item {\sl link-IDs}:  a list of link-IDs


\item [Return-value:]
A layer-ID for the newly created layer if successful.

Error object {\sl invalid-node-ID} (page~\pageref{invalid-node-ID}) if any of the node-IDs
are invalid.

Error object {\sl invalid-link-ID} (page~\pageref{invalid-link-ID}) if any of the link-IDs
are invalid.

Error object {\sl unknown-hb-error} (page~\pageref{unknown-hb-error}) if the call fails
for any other reason.

\item [Description:]

Creates and initializes a new layer. 

\item [Public:]



\end{description}
\horizontalline

\subsection{t*layer*add-node-ID}
\label{t*layer*add-node-ID}

\begin{description}
\item [Name:]  t*layer*add-node-ID

\item [Class:] {\sl t*layer}\hfill(page~\pageref{t*layer})

\item [Parameters:]
\item {\sl layer-ID}:  a unique ID for layers (possibly a node-ID?)


\item {\sl node-ID}:   An integer representing
a valid hbserver node ID.


\item [Return-value:]
The updated list of node-IDs in layer-ID if successful.

Error object {\sl invalid-layer-ID} (page~\pageref{invalid-layer-ID}) if not a t*layer.

Error object {\sl invalid-node-ID} (page~\pageref{invalid-node-ID}) if not a 
t*node-instance, or if currently a member of t*layer.

\item [Description:]

Adds node-ID to layer-ID.

\item [Public:]



\end{description}
\horizontalline

\subsection{t*layer*delete-node-ID}
\label{t*layer*delete-node-ID}

\begin{description}
\item [Name:]  t*layer*delete-node-ID

\item [Class:] {\sl t*layer}\hfill(page~\pageref{t*layer})

\item [Parameters:]
\item {\sl layer-ID}:  a unique ID for layers (possibly a node-ID?)


\item {\sl node-ID}:   An integer representing
a valid hbserver node ID.


\item [Return-value:]
The updated list of node-IDs in layer-ID if successful.

Error object {\sl invalid-layer-ID} (page~\pageref{invalid-layer-ID}) if not a t*layer.

Error object {\sl invalid-node-ID} (page~\pageref{invalid-node-ID}) if not a
t*node-instance, or if not in layer-ID.

Error object {\sl unknown-hb-error} (page~\pageref{unknown-hb-error}) if call fails for
any other reason.

\item [Description:]

Deletes node-ID from layer-ID.

\item [Public:]



\end{description}
\horizontalline

\subsection{t*layer*add-link-ID}
\label{t*layer*add-link-ID}

\begin{description}
\item [Name:]  t*layer*add-link-ID

\item [Class:] {\sl t*layer}\hfill(page~\pageref{t*layer})

\item [Parameters:]
\item {\sl layer-ID}:  a unique ID for layers (possibly a node-ID?)


\item {\sl link-ID}:  
valid HB link ID number (integer)


\item [Return-value:]
The updated set of link-IDs in layer-ID if successful.

Error object {\sl invalid-layer-ID} (page~\pageref{invalid-layer-ID}) if not a t*layer.

Error object {\sl invalid-link-ID} (page~\pageref{invalid-link-ID}) if not a
t*link-instance, or if link-ID is not a member of 
layer-ID.

Error object {\sl unknown-hb-error} (page~\pageref{unknown-hb-error}) if call fails
for any other reason.

\item [Description:]

Deletes link-ID from layer-ID.

\item [Public:]



\end{description}
\horizontalline

\subsection{t*layer*delete-link-ID}
\label{t*layer*delete-link-ID}

\begin{description}
\item [Name:]  t*layer*delete-link-ID

\item [Class:] {\sl t*layer}\hfill(page~\pageref{t*layer})

\item [Parameters:]
\item {\sl layer-ID}:  a unique ID for layers (possibly a node-ID?)


\item {\sl link-ID}:  
valid HB link ID number (integer)


\item [Return-value:]
The updated set of link-IDs in this layer if successful.

Error object {\sl invalid-layer-ID} (page~\pageref{invalid-layer-ID}) if not a t*layer.

Error object {\sl invalid-link-ID} (page~\pageref{invalid-link-ID}) if not a t*link,
or if not currently a member of this layer.

Error object {\sl unknown-hb-error} (page~\pageref{unknown-hb-error}) if call fails
for any other reason.

\item [Description:]

Deletes a link instance from this layer.

\item [Public:]



\end{description}
\horizontalline

\subsection{t*layer*divergence}
\label{t*layer*divergence}

\begin{description}
\item [Name:]  t*layer*divergence

\item [Class:] {\sl t*layer}\hfill(page~\pageref{t*layer})

\item [Parameters:]
\item {\sl layer-ID}:  a unique ID for layers (possibly a node-ID?)



\item [Return-value:]
The divergence value for layer-ID if successful.

Error object {\sl invalid-layer-ID} (page~\pageref{invalid-layer-ID}) if not a t*layer.

Error object {\sl unknown-hb-error} (page~\pageref{unknown-hb-error}) if call fails
for any other reason.

\item [Description:]

Computes the aggregate divergence for all of the
schemas with instances in this layer. 

\item [Public:]



\end{description}
\horizontalline

\section{t*error}
\label{t*error}

\begin{description}
\item [Name:]  t*error

\item [Layer:] {\sl Type}\hfill(page~\pageref{Type})

\item [Description:]
The instances of this class consist of the type-level
error objects. In some cases, however, the type system
may return error objects that are instances of the 
server system s*serror class, such as the error object
for an invalid node name. Thus, this class defines 
all of the errors semantically associated with the
type system facilities, but not all of the possible 
errors that the type system could return.

\item [Attributes:]

\item [Operations:]

\item [Subclasses:]

\item [Superclasses:]
\item {\sl u*error}\hfill(page~\pageref{u*error})

\item [Instances:]
\item {\sl invalid-link-ID}\hfill(page~\pageref{invalid-link-ID})
\item {\sl invalid-constraint-exp}\hfill(page~\pageref{invalid-constraint-exp})
\item {\sl invalid-link-schema-ID}\hfill(page~\pageref{invalid-link-schema-ID})
\item {\sl invalid-event-instance}\hfill(page~\pageref{invalid-event-instance})
\item {\sl invalid-layer-ID}\hfill(page~\pageref{invalid-layer-ID})
\item {\sl invalid-buffer-instance}\hfill(page~\pageref{invalid-buffer-instance})
\item {\sl invalid-node-ID}\hfill(page~\pageref{invalid-node-ID})
\item {\sl invalid-node-schema-ID}\hfill(page~\pageref{invalid-node-schema-ID})
\item {\sl invalid-field-ID}\hfill(page~\pageref{invalid-field-ID})




\end{description}
\horizontalline

\subsection{invalid-link-ID}
\label{invalid-link-ID}

\begin{description}
\item [Name:]  invalid-link-ID

\item [Class:] {\sl t*error}\hfill(page~\pageref{t*error})

\item [Description:]

Indicates the passed link instance ID was invalid.


\end{description}
\horizontalline

\subsection{invalid-constraint-exp}
\label{invalid-constraint-exp}

\begin{description}
\item [Name:]  invalid-constraint-exp

\item [Class:]
{\sl t*error}\hfill(page~\pageref{t*error})


\item [Description:]

Error object indicating the presence of an 
invalid link constraint expression.

\end{description}
\horizontalline

\subsection{invalid-link-schema-ID}
\label{invalid-link-schema-ID}

\begin{description}
\item [Name:]  invalid-link-schema-ID

\item [Class:] {\sl t*error}\hfill(page~\pageref{t*error})

\item [Description:]

Error object indicating an invalid link-schema-ID 
object.


\end{description}
\horizontalline

\subsection{invalid-event-instance}
\label{invalid-event-instance}

\begin{description}
\item [Name:]  invalid-event-instance

\item [Class:] {\sl t*error}\hfill(page~\pageref{t*error})

\item [Description:]

An error indicating that an object that wasn't a
t*event instance was passed. 


\end{description}
\horizontalline

\subsection{invalid-layer-ID}
\label{invalid-layer-ID}

\begin{description}
\item [Name:]  invalid-layer-ID

\item [Class:] {\sl t*error}\hfill(page~\pageref{t*error})

\item [Description:]

An error object indicating that the argument was
not a valid layer-ID.


\end{description}
\horizontalline

\subsection{invalid-buffer-instance}
\label{invalid-buffer-instance}

\begin{description}
\item [Name:]  invalid-buffer-instance

\item [Class:] {\sl t*error}\hfill(page~\pageref{t*error})

\item [Description:]

Error object indicating that the passed object
was not an Emacs buffer instance.


\end{description}
\horizontalline

\subsection{invalid-node-ID}
\label{invalid-node-ID}

\begin{description}
\item [Name:]  invalid-node-ID

\item [Class:] {\sl t*error}\hfill(page~\pageref{t*error})

\item [Description:]

An error object representing the invalidity of the 
passed value as a node-ID.


\end{description}
\horizontalline

\subsection{invalid-node-schema-ID}
\label{invalid-node-schema-ID}

\begin{description}
\item [Name:]  invalid-node-schema-ID

\item [Class:]

\item [Description:]

Error object indicating an invalid type-level node-schema
ID.


\end{description}
\horizontalline

\subsection{invalid-field-ID}
\label{invalid-field-ID}

\begin{description}
\item [Name:]  invalid-field-ID

\item [Class:] {\sl t*error}\hfill(page~\pageref{t*error})

\item [Description:]

The error object indicating that a bad field-schema-ID
value was detected.


\end{description}
\horizontalline

\section{t*event}
\label{t*event}

\begin{description}
\item [Name:]  t*event

\item [Layer:] {\sl Type}\hfill(page~\pageref{Type})

\item [Description:]

This class defines the set of events to be handled
by the type system.  Events are the communication
mechanism between the hyperbase and its currently
connected clients that allow changes made by one
client to be propogated to other clients. 

\item [Attributes:]
\item {\sl t*event*handlers}\hfill(page~\pageref{t*event*handlers})


\item [Operations:]
\item {\sl t*event*remove-handler}\hfill(page~\pageref{t*event*remove-handler})
\item {\sl t*event*add-handler}\hfill(page~\pageref{t*event*add-handler})
\item {\sl t*event*initialize}\hfill(page~\pageref{t*event*initialize})

\item [Subclasses:]

\item [Superclasses:]

\item [Instances:]
Due to a bug in cv-1.3, the following three links
all point to the same node,
(t*event*update-lschema-from-constraints),
even though two other event nodes exist and should
be pointed to (t*event*update-lschema-name and 
t*event*update-lschema-to-constraints). 

\item {\sl t*event*update-lschema-from-constraints}\hfill(page~\pageref{t*event*update-lschema-from-constraints})
\item {\sl t*event*update-lschema-to-constraints}\hfill(page~\pageref{t*event*update-lschema-to-constraints})
\item {\sl t*event*update-lschema-name}\hfill(page~\pageref{t*event*update-lschema-name})
\item {\sl t*event*delete-link-schema}\hfill(page~\pageref{t*event*delete-link-schema})
\item {\sl t*event*new-link-schema}\hfill(page~\pageref{t*event*new-link-schema})
\item {\sl t*event*update-fschema-val-fn}\hfill(page~\pageref{t*event*update-fschema-val-fn})
\item {\sl t*event*update-fschema-name}\hfill(page~\pageref{t*event*update-fschema-name})
\item {\sl t*event*delete-field-schema}\hfill(page~\pageref{t*event*delete-field-schema})
\item {\sl t*event*new-field-schema}\hfill(page~\pageref{t*event*new-field-schema})
\item {\sl t*event*delete-ninstance-field}\hfill(page~\pageref{t*event*delete-ninstance-field})
\item {\sl t*event*update-ninstance-field}\hfill(page~\pageref{t*event*update-ninstance-field})
\item {\sl t*event*update-ninstance-name}\hfill(page~\pageref{t*event*update-ninstance-name})
\item {\sl t*event*delete-node-instance}\hfill(page~\pageref{t*event*delete-node-instance})
\item {\sl t*event*new-node-instance}\hfill(page~\pageref{t*event*new-node-instance})
\item {\sl t*event*update-nschema-nodes}\hfill(page~\pageref{t*event*update-nschema-nodes})
\item {\sl t*event*update-nschema-fields}\hfill(page~\pageref{t*event*update-nschema-fields})
\item {\sl t*event*update-nschema-name}\hfill(page~\pageref{t*event*update-nschema-name})
\item {\sl t*event*delete-node-schema}\hfill(page~\pageref{t*event*delete-node-schema})
\item {\sl t*event*new-node-schema}\hfill(page~\pageref{t*event*new-node-schema})

























\end{description}
\horizontalline

\subsection{t*event*handlers}
\label{t*event*handlers}

\begin{description}
\item [Name:]  t*event*handlers

\item [Class:] {\sl t*event}\hfill(page~\pageref{t*event})

\item [Contents:] List of symbols

\item [Description:]

This attribute returns an ordered list of event
handler function names. 

\item [Setf-able:] See t*event*add-event-handlers and 
t*event*remove-event-handlers.


\item [Public:]



\end{description}
\horizontalline

\subsection{t*event*remove-handler}
\label{t*event*remove-handler}

\begin{description}
\item [Name:]  t*event*remove-handler

\item [Class:] {\sl t*event}\hfill(page~\pageref{t*event})

\item [Parameters:]
\item {\sl t*event-instance}:  an instance of t*event


\item {\sl handler-fn-name}:  function symbol


\item [Return-value:]
T if successful.

Error object {\sl invalid-event-instance} (page~\pageref{invalid-event-instance}) if first
arg is not a t*event-instance.

\item [Description:]

Removes handler-fn-name from t*event-instance.

\item [Public:]



\end{description}
\horizontalline

\subsection{t*event*add-handler}
\label{t*event*add-handler}

\begin{description}
\item [Name:]  t*event*add-handler

\item [Class:] {\sl t*event}\hfill(page~\pageref{t*event})

\item [Parameters:]
\item {\sl t*event-instance}:  an instance of t*event


\item {\sl handler-fn-name}:  function symbol

\item {\sl before-handlers}:  functional symbol

\item {\sl after-handlers}:  function symbol


\item [Return-value:]
T if successful.

Error object {\sl invalid-event-instance} (page~\pageref{invalid-event-instance}) if first
argument was not a legal t*event-instance.

Error object {\sl conflicting-hook-constraints} (page~\pageref{conflicting-hook-constraints}) if the 
constraints cannot be satisfied.

\item [Description:]

Adds handler-fn-name to the set of function handlers for
t*event-instance.

\item [Public:]



\end{description}
\horizontalline

\subsection{t*event*initialize}
\label{t*event*initialize}

\begin{description}
\item [Name:]  t*event*initialize

\item [Class:] {\sl t*event}\hfill(page~\pageref{t*event})

\item [Parameters:]
\item {\sl t*event-instance}:  an instance of t*event



\item [Return-value:]
T if successfully re-initialized the set of event-handlers
for event-instance to NIL.

Error object {\sl invalid-event-instance} (page~\pageref{invalid-event-instance}) if the argument
was not a t*event instance.

\item [Description:]

Resets the list of event handler functions for event-instance
to NIL.

\item [Public:]



\end{description}
\horizontalline

\subsection{t*event*update-lschema-from-constraints}
\label{t*event*update-lschema-from-constraints}

\begin{description}
\item [Name:]  t*event*update-lschema-from-constraints

\item [Class:] {\sl t*event}\hfill(page~\pageref{t*event})

\item [Description:]

Handler functions are passed a link-schema-ID and
its from node constraint expression.


\end{description}
\horizontalline

\subsection{t*event*update-lschema-to-constraints}
\label{t*event*update-lschema-to-constraints}

\begin{description}
\item [Name:]  t*event*update-lschema-to-constraints

\item [Class:] {\sl t*event}\hfill(page~\pageref{t*event})

\item [Description:]

Handler functions are passed a link-schema-ID and
its new to-node constraint expression.


\end{description}
\horizontalline

\subsection{t*event*update-lschema-name}
\label{t*event*update-lschema-name}

\begin{description}
\item [Name:]  t*event*update-lschema-name

\item [Class:] {\sl t*event}\hfill(page~\pageref{t*event})

\item [Description:]

Handler functions are passed a link-schema-ID and
its new name.


\end{description}
\horizontalline

\subsection{t*event*delete-link-schema}
\label{t*event*delete-link-schema}

\begin{description}
\item [Name:]  t*event*delete-link-schema

\item [Class:] {\sl t*event}\hfill(page~\pageref{t*event})

\item [Description:]

Handler functions are passed the link-schema-ID for
the successfully deleted link schema.

Note that schema "deletion" has a special interpretation
in Egret: the schema is not physically removed, just
marked as not available for future instantiation.


\end{description}
\horizontalline

\subsection{t*event*new-link-schema}
\label{t*event*new-link-schema}

\begin{description}
\item [Name:]  t*event*new-link-schema

\item [Class:] {\sl t*event}\hfill(page~\pageref{t*event})

\item [Description:]

Handler functions are passed a new link-schema-ID,
its name, and its to and from node constraint expressions.



\end{description}
\horizontalline

\subsection{t*event*update-fschema-val-fn}
\label{t*event*update-fschema-val-fn}

\begin{description}
\item [Name:]  t*event*update-fschema-val-fn

\item [Class:] {\sl t*event}\hfill(page~\pageref{t*event})

\item [Description:]

Handler functions are passed a field-schema-ID and
the new validity function.




\end{description}
\horizontalline

\subsection{t*event*update-fschema-name}
\label{t*event*update-fschema-name}

\begin{description}
\item [Name:]  t*event*update-fschema-name

\item [Class:] {\sl t*event}\hfill(page~\pageref{t*event})

\item [Description:]

Handler functions are passed a field-schema-ID and
its new name.


\end{description}
\horizontalline

\subsection{t*event*delete-field-schema}
\label{t*event*delete-field-schema}

\begin{description}
\item [Name:]  t*event*delete-field-schema

\item [Class:] {\sl t*event}\hfill(page~\pageref{t*event})

\item [Description:]

Handler functions are passed a field-schema-ID after
it has been successfully deleted.

Note that deletion of schemas does not result in 
removal, since their presence is required to interpret
previously created nodes. Rather, deletion indicates
that new new uses of the schema are allowed.


\end{description}
\horizontalline

\subsection{t*event*new-field-schema}
\label{t*event*new-field-schema}

\begin{description}
\item [Name:]  t*event*new-field-schema

\item [Class:] {\sl t*event}\hfill(page~\pageref{t*event})

\item [Description:]

Handler functions are passed a new field-schema-ID,
its name, and its validity function.


\end{description}
\horizontalline

\subsection{t*event*delete-ninstance-field}
\label{t*event*delete-ninstance-field}

\begin{description}
\item [Name:]  t*event*delete-ninstance-field

\item [Class:] {\sl t*event}\hfill(page~\pageref{t*event})

\item [Description:]

Handler functions are passed a node-ID and its
field-schema-ID that has just been successfully
deleted.


\end{description}
\horizontalline

\subsection{t*event*update-ninstance-field}
\label{t*event*update-ninstance-field}

\begin{description}
\item [Name:]  t*event*update-ninstance-field

\item [Class:] {\sl t*event}\hfill(page~\pageref{t*event})

\item [Description:]

Handler functions are passed a node-ID, a 
field-schema-ID, and the new field contents.

This event is also used when a new field is added
to a node.

\end{description}
\horizontalline

\subsection{t*event*update-ninstance-name}
\label{t*event*update-ninstance-name}

\begin{description}
\item [Name:]  t*event*update-ninstance-name

\item [Class:] {\sl t*event}\hfill(page~\pageref{t*event})

\item [Description:]

Handler functions are passed a node-ID and its
new name.


\end{description}
\horizontalline

\subsection{t*event*delete-node-instance}
\label{t*event*delete-node-instance}

\begin{description}
\item [Name:]  t*event*delete-node-instance

\item [Class:] {\sl t*event}\hfill(page~\pageref{t*event})

\item [Description:]

Handler functions are passed the node-ID that has just
been successfully deleted.


\end{description}
\horizontalline

\subsection{t*event*new-node-instance}
\label{t*event*new-node-instance}

\begin{description}
\item [Name:]  t*event*new-node-instance

\item [Class:] {\sl t*event}\hfill(page~\pageref{t*event})

\item [Description:]

Handler function are passed the new node-ID and
its name.


\end{description}
\horizontalline

\subsection{t*event*update-nschema-nodes}
\label{t*event*update-nschema-nodes}

\begin{description}
\item [Name:]  t*event*update-nschema-nodes

\item [Class:] {\sl t*event}\hfill(page~\pageref{t*event})

\item [Description:]

Handler functions are passed a node-schema-ID and
a list of its new node-IDs.


\end{description}
\horizontalline

\subsection{t*event*update-nschema-fields}
\label{t*event*update-nschema-fields}

\begin{description}
\item [Name:]  t*event*update-nschema-fields

\item [Class:] {\sl t*event}\hfill(page~\pageref{t*event})

\item [Description:]

Handler functions are passed a node-schema-ID and
a list of its new field-IDs.


\end{description}
\horizontalline

\subsection{t*event*update-nschema-name}
\label{t*event*update-nschema-name}

\begin{description}
\item [Name:]  t*event*update-nschema-name

\item [Class:] {\sl t*event}\hfill(page~\pageref{t*event})

\item [Description:]

Handler functions are passed a node-schema-ID and
its new name.


\end{description}
\horizontalline

\subsection{t*event*delete-node-schema}
\label{t*event*delete-node-schema}

\begin{description}
\item [Name:]  t*event*delete-node-schema

\item [Class:] {\sl t*event}\hfill(page~\pageref{t*event})

\item [Description:]

Handler functions for this instance are passed
the node-schema-ID of the just deleted node schema.


\end{description}
\horizontalline

\subsection{t*event*new-node-schema}
\label{t*event*new-node-schema}

\begin{description}
\item [Name:]  t*event*new-node-schema

\item [Class:] {\sl t*event}\hfill(page~\pageref{t*event})

\item [Description:]

Handler functions are passed the new node-schema-ID,
its name, and a list of its field-schema-IDs.




\end{description}
