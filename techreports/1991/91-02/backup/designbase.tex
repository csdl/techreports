\documentstyle [11pt,/home/13/csdl/tex/definemargins,
/home/13/csdl/tex/lmacros]{report}
\begin{document}
\pagenumbering{roman}
\tableofcontents
\newpage
\pagenumbering{arabic}

\chapter{Utilities}
\label{Utilities}

\begin{description}
\item [Name:]  Utilities

\item [Description:]
This subsystem contains generic classes and utilities for the
entire Egret environment.  Object classes and their operations
defined in this module are available for use and
specialization by all other subsystems. Due to the general
nature of this subsystem, it should always be loaded before
all other subsystems.

\item [Public-classes:]
\item {\sl u*hash}\hfill(page~\pageref{u*hash})
\item {\sl u*table}\hfill(page~\pageref{u*table})
\item {\sl u*error}\hfill(page~\pageref{u*error})
\item {\sl u*hook}\hfill(page~\pageref{u*hook})




\end{description}
\horizontalline

\section{u*hash}
\label{u*hash}

\begin{description}
\item [Name:]  u*hash

\item [Layer:]
{\sl Utilities}\hfill(page~\pageref{Utilities})

\item [Description:]

U*HASH is an implementation of the hash table data
structure in Emacs Lisp. Hash tables trade off space for
time, potentially providing faster retrieval than that
allowed by association lists, the typical alternative
Lisp table structure, which uses sequential search.

U*HASH is implemented as vectors. Each element of U*HASH
can be arbitrary Lisp object, ranging from symbols to
complex structures. Though hash tables are oriented
toward one-way mapping, i.e., from key to data,
U*HASH provides restricted reverse searching, i.e., from
data to key. It does so via maintaining an auxiliary
alist of all elements in the table.

\item [Attributes:]

\item [Operations:]
\item {\sl u*hash*make}\hfill(page~\pageref{u*hash*make})
\item {\sl u*hash*get}\hfill(page~\pageref{u*hash*get})
\item {\sl u*hash*set}\hfill(page~\pageref{u*hash*set})
\item {\sl u*hash*rem}\hfill(page~\pageref{u*hash*rem})
\item {\sl u*hash*existp}\hfill(page~\pageref{u*hash*existp})

\item [Collections:]

\item [Subclasses:]

\item [Superclasses:]

\item [Instances:]



\end{description}
\horizontalline

\subsection{u*hash*make}
\label{u*hash*make}

\begin{description}
\item [Name:]  u*hash*make

\item [Class:]
{\sl u*hash}\hfill(page~\pageref{u*hash})

\item [Parameters:]
\item {\sl size}:  integer (prime number)


\item [Return-value:] obarray of required size with all cells set
to nil.   

\item [Description:]

Creates an obarray as hashtable. SIZE, if supplied, should
be a prime number. Otherwise, defaults to default-size,
i.e., 511. 

\item [Public:]



\end{description}
\horizontalline

\subsection{u*hash*get}
\label{u*hash*get}

\begin{description}
\item [Name:]  u*hash*get

\item [Class:]
{\sl u*hash}\hfill(page~\pageref{u*hash})

\item [Parameters:]
\item {\sl hash-key}:  symbol

\item {\sl table}:  symbol


\item [Return-value:] 
data element of the table entry.

\item [Description:]
Retrieves data element from the hashtable. If optional
arg TABLE is supplied, use that table. Otherwise, use
the system default table.

\item [Public:]



\end{description}
\horizontalline

\subsection{u*hash*set}
\label{u*hash*set}

\begin{description}
\item [Name:]  u*hash*set

\item [Class:]
{\sl u*hash}\hfill(page~\pageref{u*hash})

\item [Parameters:]
\item {\sl hash-key}:  symbol

\item {\sl data}:  any Lisp object

\item {\sl table}:  symbol


\item [Return-value:] new data value

\item [Description:]
Stores DATA in table TABLE using HASH-KEY as key.  If
optional arg TABLE is not supplied, defaults to system
table. 

\item [Public:]



\end{description}
\horizontalline

\subsection{u*hash*rem}
\label{u*hash*rem}

\begin{description}
\item [Name:]  u*hash*rem

\item [Class:]
{\sl u*hash}\hfill(page~\pageref{u*hash})

\item [Parameters:]
\item {\sl hash-key}:  symbol

\item {\sl table}:  symbol


\item [Return-value:]  t if deletion is successful or
nil otherwise.

\item [Description:]

Removes data element corresponding to key HASH-KEY from
table TABLE. If TABLE is not supplied, defaults to
system table. 

\item [Public:]



\end{description}
\horizontalline

\subsection{u*hash*existp}
\label{u*hash*existp}

\begin{description}
\item [Name:]  u*hash*existp

\item [Class:]
{\sl u*hash}\hfill(page~\pageref{u*hash})

\item [Parameters:]
\item {\sl hash-key}:  symbol

\item {\sl table}:  symbol


\item [Return-value:] 
data value if found or nil otherwise.

\item [Description:]
Check the existence of the item with key HASH-KEY in
table TABLE. If TABLE is not supplied, defaults
to system table.

\item [Public:]




\end{description}
\horizontalline

\section{u*table}
\label{u*table}

\begin{description}
\item [Name:]  u*table

\item [Layer:] {\sl Utilities}\hfill(page~\pageref{Utilities})

\item [Description:]
U*TABLE is a generic class that consists of paired objects: KEY and
DATA: KEY is either string or symbol, and DATA, any lisp object.
U*TABLE is typically used to store data for quick lookup purposes. For
example, the Server subsystem uses this class extensively as cache
structures for storing selected node and link attributes so that
lookup operations can be done efficiently (i.e., no remote access.
Important U*TABLE operations include GET, PUT, DELETE, EXISTP,
INITIALIZE, and so forth.

U*TABLE is implemented using both hashtable and alist. For a given
table, one can specify whether to use hashtable or alist, depending on
the requirement, e.g., size, operation types. The default is alist. 

\item [Attributes:]

\item [Operations:]
\item {\sl u*table*define}\hfill(page~\pageref{u*table*define})

\item [Collections:]

\item [Subclasses:]

\item [Superclasses:]



\end{description}
\horizontalline

\subsection{u*table*define}
\label{u*table*define}

\begin{description}
\item [Name:]  u*table*define

\item [Class:]
{\sl u*table}\hfill(page~\pageref{u*table})

\item [Parameters:]
\item {\sl table-name}:  symbol

\item {\sl fn-prefix}:  string

\item {\sl rlookup-fn}:  slot accessor function

\item {\sl table-size}:  integer (prime number)

\item {\sl hashp}:  Boolean 

\item {\sl init-value}:  list of Lisp objects.


\item [Return-value:]

\item [Description:]
Defines a table called TABLE-NAME with requested parameter
constraints. Table operations include PUT, GET, DELETE, EXISTP,
GET-KEY, GET-COMPLETION-LIST, and INITIALIZE. They are prefixed with
FN-PREFIX.
Keyword TABLE-NAME is an optional symbol indicating the name of
table. When nil, defaults to the global table.  
Keyword FN-PREFIX is a required string used as prefix of table
operations, e.g., s!node*.
Keyword HASHP is a Boolean indicating if the table should be a
hashtable or not.  Using hashtable if t or alist otherwise.
Keyword RLOOKUP-FN is an optional arg indicating DATA slot on
which reverse lookup might be done. Currently, only one such
slot is allowed.
Keyword TABLE-SIZE is an optional prime number indicating the
size of hashtable.  Used only when HASHP is t.
Keyworkd COMPLETION-FN is an optional arg indicating DATA slot
on which completion list will be maintained. Currently, only one
such list is allowed.
Keyword INITIAL-VALUES is a set of initial values for the table.
It must be provided as an alist whose elements are in the form
of (KEY . DATA)

\item [Public:]



\end{description}
\horizontalline

\section{u*error}
\label{u*error}

\begin{description}
\item [Name:]  u*error

\item [Layer:] {\sl Utilities}\hfill(page~\pageref{Utilities})

\item [Description:]
U*ERROR is a global error class which subjects to specialization in
each subsystem. In other words, each subsystem has its own error
class, which is a subclass of U*ERROR. Inevitably, they inherits all
attributes and operations of U*ERROR. These error subclasses normally
add no new attributes and operations. Instead, they contain explicit
links to to all error instances that belong to that subclass.

U*ERROR follows the same structure as that of the Emacs Lisp error
object, i.e., each error object contains three attributes or
properties: error symbol or name, error conditions, and error message.
The attribute operations defined below are also applicable to Emacs
built-in error objects listed on pp. 535-536 of the Gnus Emacs Lisp
Reference Manual

\item [Attributes:]
\item {\sl u*error*name}\hfill(page~\pageref{u*error*name})
\item {\sl u*error*conditions}\hfill(page~\pageref{u*error*conditions})
\item {\sl u*error*message}\hfill(page~\pageref{u*error*message})

\item [Operations:]
\item {\sl u*error*define}\hfill(page~\pageref{u*error*define})
\item {\sl u*error*protected}\hfill(page~\pageref{u*error*protected})

\item [Collections:]

\item [Subclasses:]
\item {\sl s*serror}\hfill(page~\pageref{s*serror})
\item {\sl t*error}\hfill(page~\pageref{t*error})

\item [Superclasses:]




\end{description}
\horizontalline

\subsection{u*error*name}
\label{u*error*name}

\begin{description}
\item [Name:]  u*error*name

\item [Class:]
{\sl u*error}\hfill(page~\pageref{u*error})

\item [Contents:] symbol 

\item [Description:]  error name

\item [Setf-able:] no

\item [Public:]



\end{description}
\horizontalline

\subsection{u*error*conditions}
\label{u*error*conditions}

\begin{description}
\item [Name:]  u*error*conditions

\item [Class:]
{\sl u*error}\hfill(page~\pageref{u*error})

\item [Contents:] list of symbols

\item [Description:] 
Specifying a set of conditions under which error
message is to be sent to the user. Normally, this
list includes the error symbol itself adn the symbol
ERROR. Occassionally it includes additional symbols,
which are intermediate classifications, narrower than
error but broader than a single error symbol.

\item [Setf-able:]

\item [Public:]



\end{description}
\horizontalline

\subsection{u*error*message}
\label{u*error*message}

\begin{description}
\item [Name:]  u*error*message

\item [Class:]
{\sl u*error}\hfill(page~\pageref{u*error})

\item [Contents:] message to be sent to the user upon the 
occurance of the error. 

\item [Description:]

\item [Setf-able:] no

\item [Public:]



\end{description}
\horizontalline

\subsection{u*error*define}
\label{u*error*define}

\begin{description}
\item [Name:]  u*error*define

\item [Class:]
{\sl u*error}\hfill(page~\pageref{u*error})

\item [Parameters:]
\item {\sl error-msg}:  string

\item {\sl error-symbol}:  symbol


\item [Return-value:] 
macro with no interesting return value

\item [Description:]
Defines a new error symbol, ERROR-SYMBOL, with two
default condition names: value of ERROR-SYMBOL and
ERROR. Default error message is ERROR-MSG.

\item [Public:]



\end{description}
\horizontalline

\subsection{u*error*protected}
\label{u*error*protected}

\begin{description}
\item [Name:]  u*error*protected

\item [Class:]
{\sl u*error}\hfill(page~\pageref{u*error})

\item [Parameters:]
\item {\sl error-symbols}:  symbol or list of symbols

\item {\sl body}:  list of Lisp forms


\item [Return-value:] 
error object from ERROR-SYMBOLS if error occurs in the
execution of BODY or return value of last form of BODY.

\item [Description:]
executes forms in BODY and catches errors specified in
ERROR-SYMBOLS.  ERROR-SYMBOLS is a required symbol or
list of symbols that represent error conditions to be
caught.  

\item [Public:]



\end{description}
\horizontalline

\section{u*hook}
\label{u*hook}

\begin{description}
\item [Name:]  u*hook

\item [Layer:] {\sl Utilities}\hfill(page~\pageref{Utilities})

\item [Description:] 
U*HOOK is an extension of the Emacs hook mechanism whose value
is a function or a list of functions to be called under certain
pre-defined conditions. The main purpose of hooks is for
customization and modularity. The main extension is that U*HOOK
allows ordering constraints to be imposed on the functions
installed on a given hook.

Like U*ERROR, U*HOOK is a generic object class that subjects to 
specialization in various subsystems. These hook subclasses contain
links to individual instances within respective subsystems. 

\item [Attributes:]

\item [Operations:]
\item {\sl u*hook*insert}\hfill(page~\pageref{u*hook*insert})

\item [Collections:]

\item [Subclasses:]

\item [Superclasses:]



\end{description}
\horizontalline

\subsection{u*hook*insert}
\label{u*hook*insert}

\begin{description}
\item [Name:]  u*hook*insert

\item [Class:]
{\sl u*hook}\hfill(page~\pageref{u*hook})

\item [Parameters:]
\item {\sl hook-var}:  symbol

\item {\sl hook-fn}:  symbol

\item {\sl before-hook-fns}:  list of symbols


\item {\sl after-hook-fns}:  list of symbols



\item [Return-value:]
Returns t if the new event function can be
inserted into the sequence of hook functions in such
a manner as to satisfy the before and after 
constraints, or {\sl conflicting-hook-constraints} (page~\pageref{conflicting-hook-constraints})
otherwise.

\item [Description:]
Installs hook function HOOK-FN on to designated hook
variable.  Returns hook variable value if the new hook fn can
be inserted into the sequence of hook fns that satisfy
required before- and after- constraints; otherwise returns
error object CONFLICTING-HOOK-CONSTRAINTS


\item [Public:]



\end{description}
\horizontalline

\chapter{Server}
\label{Server}

\begin{description}
\item [Name:]  Server Subsystem

\item [Description:]
The server subsystem is the interface between the remote
database server and other modules of the Egret system. As
such, it provides transparent and efficient access to the
functions of the server via such facilities as caching,
events, etc. It also offers encapsulation via
comprehensive error checking and additional services
(e.g., global nodes) essential for the implementation of
advanced Egret functions.

The server module consists of six classes: s*node, s*link,
s*server-process, s*sys-node, s*serror, and s*event.

S*node and s*link provide a set of primitive operations
on nodes and links.  S*server-process manages connection
sessions and synchronizes the states of the remote and
local data.  S*sys-node supports a set of special nodes
which contain data used by various Egret modules for
internal purposes.  S*serror tracks all error objects,
and S*event, all public events in the server system.

\item [Public-classes:]
\item {\sl s*node}\hfill(page~\pageref{s*node})
\item {\sl s*link}\hfill(page~\pageref{s*link})
\item {\sl s*server-process}\hfill(page~\pageref{s*server-process})
\item {\sl s*snode}\hfill(page~\pageref{s*snode})
\item {\sl s*serror}\hfill(page~\pageref{s*serror})
\item {\sl s*event}\hfill(page~\pageref{s*event})  

\item [Private-classes:]




\end{description}
\horizontalline

\section{s*node}
\label{s*node}

\begin{description}
\item [Name:]  s*node

\item [Layer:] {\sl Server}\hfill(page~\pageref{Server})

\item [Description:]
This class implements primitive operations on nodes, including
creation, deletion, field content retrieval and updating,
locking, and so forth. Most of these functions are available from
the remote server. The primary purpose of this class is to
provide a uniform and location-independent access, i.e., the
caller does not need to know whether the returned data is from
local cache or directly from the remote server.

Note that except for the two node repairing functions, which are
accessable only to the system administrator, all other operations
and attribute functions are public. 

\item [Attributes:]
\item {\sl s*node*name}\hfill(page~\pageref{s*node*name})
\item {\sl s*node*data}\hfill(page~\pageref{s*node*data})
\item {\sl s*node*created-by}\hfill(page~\pageref{s*node*created-by})
\item {\sl s*node*created-date}\hfill(page~\pageref{s*node*created-date})
\item {\sl s*node*last-modified-date}\hfill(page~\pageref{s*node*last-modified-date})
\item {\sl s*node*last-modified-by}\hfill(page~\pageref{s*node*last-modified-by})
\item {\sl s*node*font}\hfill(page~\pageref{s*node*font})
\item {\sl s*node*geometry}\hfill(page~\pageref{s*node*geometry})
\item {\sl s*node*locked-by}\hfill(page~\pageref{s*node*locked-by})
\item {\sl s*node*incoming-links}\hfill(page~\pageref{s*node*incoming-links})
\item {\sl s*node*outgoing-links}\hfill(page~\pageref{s*node*outgoing-links})

\item [Operations:]
\item {\sl s*node*make}\hfill(page~\pageref{s*node*make})
\item {\sl s*node*delete}\hfill(page~\pageref{s*node*delete})
\item {\sl s*node*lock}\hfill(page~\pageref{s*node*lock})
\item {\sl s*node*unlock}\hfill(page~\pageref{s*node*unlock})
\item {\sl s*node*set-name}\hfill(page~\pageref{s*node*set-name})
\item {\sl s*node*set-data}\hfill(page~\pageref{s*node*set-data})
\item {\sl s*node*set-geometry}\hfill(page~\pageref{s*node*set-geometry})
\item {\sl s*node*set-font}\hfill(page~\pageref{s*node*set-font})

\item {\sl s*\{node\}*IDs}\hfill(page~\pageref{s*node*IDs})
\item {\sl s*\{node\}*mapc-IDs}\hfill(page~\pageref{s*node*mapc-IDs})

\item {\sl s*node@reset-incoming-links}\hfill(page~\pageref{s*node@reset-incoming-links})
\item {\sl s*node@reset-outgoing-links}\hfill(page~\pageref{s*node@reset-outgoing-links})



\item [Collections:]

\item [Subclasses:]

\item [Superclasses:]



\end{description}
\horizontalline

\subsection{s*node*name}
\label{s*node*name}

\begin{description}

\item [Name:]  s*node*name

\item [Class:] {\sl s*node}\hfill(page~\pageref{s*node}) 

\item [Contents:] String (40)

\item [Description:]
The name of the node. Node names are not
required to be unique at the server subsystem
level, although higher subsystems may wish to
enforce this constraint.

\item [Setf-able:] See s*node*set-name

\item [Public:]



\end{description}
\horizontalline

\subsection{s*node*data}
\label{s*node*data}

\begin{description}

\item [Name:]  s*node*data

\item [Class:] {\sl s*node}\hfill(page~\pageref{s*node})

\item [Contents:]  Unspecified 

\item [Description:]
This is the famous, variable length data
field, of which the server subsystem is
permitted to know practically nothing about.

\item [Setf-able:]


\item [Public:]



\end{description}
\horizontalline

\subsection{s*node*created-by}
\label{s*node*created-by}

\begin{description}

\item [Name:]  s*node*created-by

\item [Class:] {\sl s*node}\hfill(page~\pageref{s*node})

\item [Contents:] String (20)

\item [Description:]

The original author of the node.  This is 
automatically set by the server during 
node creation, but its value is accessable.

\item [Setf-able:]


\item [Public:]



\end{description}
\horizontalline

\subsection{s*node*created-date}
\label{s*node*created-date}

\begin{description}

\item [Name:]  s*node*created-date

\item [Class:] {\sl s*node}\hfill(page~\pageref{s*node})

\item [Contents:] String

\item [Description:]
Node creation date. Automatically set by 
the server subsystem during node creation.

\item [Setf-able:]

\item [Public:]



\end{description}
\horizontalline

\subsection{s*node*last-modified-date}
\label{s*node*last-modified-date}

\begin{description}

\item [Name:]  s*node*last-modified-date

\item [Class:] {\sl s*node}\hfill(page~\pageref{s*node})

\item [Contents:] string

\item [Description:]
Last modified date.  Automatically 
set by the server subsystem during 
node field updating.

\item [Setf-able:]


\item [Public:]



\end{description}
\horizontalline

\subsection{s*node*last-modified-by}
\label{s*node*last-modified-by}

\begin{description}

\item [Name:]  s*node*last-modified-by

\item [Class:] {\sl s*node}\hfill(page~\pageref{s*node})

\item [Contents:] String (20)

\item [Description:]
Automatically maintained by server subsystem
during node field updates.

\item [Setf-able:]

\item [Public:]



\end{description}
\horizontalline

\subsection{s*node*font}
\label{s*node*font}

\begin{description}

\item [Name:]  s*node*font

\item [Class:] {\sl s*node}\hfill(page~\pageref{s*node})

\item [Contents:] String

\item [Description:]
A font specification.  This might be checked
for validity.  This attribute is not
maintained by the server.

\item [Setf-able:]


\item [Public:]



\end{description}
\horizontalline

\subsection{s*node*geometry}
\label{s*node*geometry}

\begin{description}

\item [Name:]  s*node*geometry

\item [Class:] {\sl s*node}\hfill(page~\pageref{s*node})

\item [Contents:] string

\item [Description:]
This should be a valid geometry
specification.  Its value is not
maintained by the server.

\item [Setf-able:]


\item [Public:]



\end{description}
\horizontalline

\subsection{s*node*locked-by}
\label{s*node*locked-by}

\begin{description}

\item [Name:]  s*node*locked-by

\item [Class:] {\sl s*node}\hfill(page~\pageref{s*node})

\item [Contents:] string

\item [Description:]

This attribute is either nil (if the node is
unlocked) or a user-name (if the node is locked by
the specified user). It can also return an error
object {\sl show-lock-fails} (page~\pageref{show-lock-fails}).

This value is maintained by the server
subsystem.

\item [Setf-able:]


\item [Public:]



\end{description}
\horizontalline

\subsection{s*node*incoming-links}
\label{s*node*incoming-links}

\begin{description}

\item [Name:]  s*node*incoming-links

\item [Class:] {\sl s*node}\hfill(page~\pageref{s*node})

\item [Contents:] List of link-IDs

\item [Description:]
This attribute holds a list of link-IDs
whose destination node-IDs are equal 
to the ID of this node instance.

Since this is corruptable local information,
the function s*node@reset-incoming-links
exists to rebuild this node's attribute.

\item [Setf-able:]


\item [Public:]



\end{description}
\horizontalline

\subsection{s*node*outgoing-links}
\label{s*node*outgoing-links}

\begin{description}

\item [Name:]  s*node*outgoing-links

\item [Class:] {\sl s*node}\hfill(page~\pageref{s*node})
 
\item [Contents:] A list of link-IDs

\item [Description:] 
A list of link-IDs corresponding to the
links pointing away from this node.

Since this is local, corruptable information,
the operation s*node!reset-links rebuilds
this by reference to the hbserver.

\item [Setf-able:]


\item [Public:]



\end{description}
\horizontalline

\subsection{s*node*make}
\label{s*node*make}

\begin{description}
\item [Name:]  s*node*make

\item [Class:] {\sl s*node}\hfill(page~\pageref{s*node})

\item [Parameters:]
\item {\sl node-name}:  
A valid node name. This currently means that it is a
string of less than 40 characters, and that it does
not contain leading space(s) or tabs.
  

\item [Return-value:] 
A node-ID if successful.

Error object {\sl invalid-node-name} (page~\pageref{invalid-node-name}) if node-name
violates node naming conventions.

Error object {\sl create-node-fails} (page~\pageref{create-node-fails}) if node
creation hb-call fails.

\item [Description:]
This is the primitive function for obtaining
new node-IDs from the remote server.

On successful node creation, the following node
attributes are set:

(1) s*node*created-by and s*node*last-modified-by are
set to the original author of the node; (2)
s*node*created-date and s*node*last-modified-date are
set to the creation date; (3) s*node*geometry and
s*node*font are set to repective user defaults, or
system defaults if user defaults are not set.

\item [Public:]





\end{description}
\horizontalline

\subsection{s*node*delete}
\label{s*node*delete}

\begin{description}
\item [Name:]  s*node*delete

\item [Class:] {\sl s*node}\hfill(page~\pageref{s*node})

\item [Parameters:] 
\item {\sl node-ID}:   An integer representing
a valid hbserver node ID.
 

\item [Return-value:]
NODE-ID if the node-ID was successfully deleted.

Error object {\sl node-still-referenced} (page~\pageref{node-still-referenced}) if the
target node still contains incoming links.

Error object  {\sl node-still-locked} (page~\pageref{node-still-locked}) if the target
node is still locked.

Error object {\sl unknown-hb-error} (page~\pageref{unknown-hb-error}) if node
deletion hb-operation fails for reasons other
than the above.

\item [Description:]
Permanently removes node NODE-ID from remote database.

\item [Public:]


\end{description}
\horizontalline

\subsection{s*node*lock}
\label{s*node*lock}

\begin{description}
\item [Name:]  s*node*lock

\item [Class:] {\sl s*node}\hfill(page~\pageref{s*node})

\item [Parameters:] 
\item {\sl node-ID}:   An integer representing
a valid hbserver node ID.


\item [Return-value:]
T if the lock was successfully obtained, nil otherwise.

\item [Description:]
Attempts to get a lock on node-ID.  Will fail if 
node-ID is already locked by another user.

\item [Public:]



\end{description}
\horizontalline

\subsection{s*node*unlock}
\label{s*node*unlock}

\begin{description}
\item [Name:]  s*node*unlock

\item [Class:] {\sl s*node}\hfill(page~\pageref{s*node})

\item [Parameters:] 
\item {\sl node-ID}:   An integer representing
a valid hbserver node ID.
 

\item [Return-value:]
T if node-ID was successfully unlocked, nil otherwise.

\item [Description:]
Unlocks node-ID. The user who is trying to unlock the node
must be the same as the one who locks the node.

\item [Public:]



\end{description}
\horizontalline

\subsection{s*node*set-name}
\label{s*node*set-name}

\begin{description}
\item [Name:]  s*node*set-name

\item [Class:] {\sl s*node}\hfill(page~\pageref{s*node})

\item [Parameters:]
\item {\sl node-ID}:   An integer representing
a valid hbserver node ID.

\item {\sl node-name}:  
A valid node name. This currently means that it is a
string of less than 40 characters, and that it does
not contain leading space(s) or tabs.


\item [Return-value:]
A node-ID if successful.

Error object {\sl invalid-node-name} (page~\pageref{invalid-node-name}) if node-name
violates node name conventions. 

Error object {\sl write-attribute-fails} (page~\pageref{write-attribute-fails}) if rename
hb-operation fails.

\item [Description:]
Resets the name of the node. Involves a write
out to the database.

\item [Public:]


\end{description}
\horizontalline

\subsection{s*node*set-data}
\label{s*node*set-data}

\begin{description}
\item [Name:]  s*node*set-data

\item [Class:] {\sl s*node}\hfill(page~\pageref{s*node})

\item [Parameters:]
\item {\sl node-ID}:   An integer representing
a valid hbserver node ID.

\item {\sl node-data}:  string


\item [Return-value:]
A node-ID if successful.

Error object {\sl write-attribute-fails} (page~\pageref{write-attribute-fails}) if set-data
hb-operation fails.

\item [Description:]
Saves data to the persistent datastore on the remote
database server.


\item [Public:]



\end{description}
\horizontalline

\subsection{s*node*set-geometry}
\label{s*node*set-geometry}

\begin{description}
\item [Name:]  s*node*set-geometry

\item [Class:] {\sl s*node}\hfill(page~\pageref{s*node})

\item [Parameters:]
\item {\sl node-ID}:   An integer representing
a valid hbserver node ID.

\item {\sl node-geometry}:  A string containing a valid 
geometry specification. 
 

\item [Return-value:]
NODE-GEOMETRY if successful.

Error object {\sl write-attribute-fails} (page~\pageref{write-attribute-fails}) if set geometry
hb-operation fails.

\item [Description:]
Resets the geometry attribute of node-ID.  
Involves a write out to the database.
Since geometry is inherently a kind of 
domain-specific idea, the server subsystem
does not checks for the validity of its
value. Instead, it treats it simply as
string. 

\item [Public:]



\end{description}
\horizontalline

\subsection{s*node*set-font}
\label{s*node*set-font}

\begin{description}
\item [Name:]  s*node*set-font

\item [Class:] {\sl s*node}\hfill(page~\pageref{s*node})

\item [Parameters:]
\item {\sl node-ID}:   An integer representing
a valid hbserver node ID.

\item {\sl node-font}:  a string corresponding to a valid font name.


\item [Return-value:]
NODE-FONT if successful.

Error object {\sl write-attribute-fails} (page~\pageref{write-attribute-fails}) if set font
hb-operation fails.

\item [Description:]
Resets the font attribute.  Since this
is relatively domain-specific, the server
subsystem is not responsible for its
validity checking.

\item [Public:]



\end{description}
\horizontalline

\subsection{s*\{node\}*IDs}
\label{s*node*IDs}

\begin{description}
\item [Name:]  s*\{node\}*IDs

\item [Class:] {\sl s*node}\hfill(page~\pageref{s*node})

\item [Parameters:] none

\item [Return-value:]

A list of s*node IDs.

\item [Description:]

Returns a freshly consed list of all currently
defined s*node IDs.


\item [Public:]



\end{description}
\horizontalline

\subsection{s*\{node\}*mapc-IDs}
\label{s*node*mapc-IDs}

\begin{description}
\item [Name:]  s*\{node\}*mapc-IDs

\item [Class:] {\sl s*node}\hfill(page~\pageref{s*node})

\item [Parameters:]
\item {\sl map-ID-fn}:  A function that takes one argument, an ID,
and which performs some side-effect based upon that
value.

 

\item [Return-value:] nil

\item [Description:]

Calls map-ID-fn once one each currently defined
s*node ID.

\item [Public:]



\end{description}
\horizontalline

\subsection{s*node@reset-incoming-links}
\label{s*node@reset-incoming-links}

\begin{description}
\item [Name:]  s*node@reset-incoming-links
\item [Class:]
{\sl s*node}\hfill(page~\pageref{s*node})

\item [Parameters:]
\item {\sl node-ID}:   An integer representing
a valid hbserver node ID.


\item [Return-value:] 
list of link-IDs

\item [Description:]

Rebuilds the list of link-IDs that point to this node
by directly calling the remote server to traverse 
all node and link objects to determine which ones point
to this node. 
This is curruption repairing function.

\item [Public:]



\end{description}
\horizontalline

\subsection{s*node@reset-outgoing-links}
\label{s*node@reset-outgoing-links}

\begin{description}
\item [Name:]  s*node@reset-outgoing-links

\item [Class:]
{\sl s*node}\hfill(page~\pageref{s*node})

\item [Parameters:]
\item {\sl node-ID}:   An integer representing
a valid hbserver node ID.
 

\item [Return-value:] 
List of link-IDs.

\item [Description:]
Rebuilds the list of link-IDs that point away from this
node by calling the remote server to traverse all nodes
and links determine which ones point away.  
This is a corruption recovery function.

\item [Public:]



\end{description}
\horizontalline

\section{s*link}
\label{s*link}

\begin{description}
\item [Name:]  s*link

\item [Layer:] {\sl Server}\hfill(page~\pageref{Server})

\item [Description:]
The S*LINK class implements a set of primitive
operations on links, including link creation, deletion,
link attribute retrieval and update, and so forth.
Most of these operations are straightforward
translation of the corresponding remote database server
functions. Because of the asymmetric handling of links
by the current server, S*LINK has to maintain a
separate structure to keep track of the source and
destination nodes of a link. Nevertheless, certain
operations, such as S*LINK*SET-SOURCE-NODE, can still
not be provided due to the built-in constraints on
links.

\item [Attributes:]
\item {\sl s*link*name}\hfill(page~\pageref{s*link*name})
\item {\sl s*link*created-by}\hfill(page~\pageref{s*link*created-by})
\item {\sl s*link*created-date}\hfill(page~\pageref{s*link*created-date})
\item {\sl s*link*last-modified-by}\hfill(page~\pageref{s*link*last-modified-by})
\item {\sl s*link*last-modified-date}\hfill(page~\pageref{s*link*last-modified-date})
\item {\sl s*link*source-node}\hfill(page~\pageref{s*link*source-node})
\item {\sl s*link*destination-node}\hfill(page~\pageref{s*link*destination-node})

\item [Operations:]
\item {\sl s*link*make}\hfill(page~\pageref{s*link*make})
\item {\sl s*link*delete}\hfill(page~\pageref{s*link*delete})
\item {\sl s*link*set-name}\hfill(page~\pageref{s*link*set-name})
\item {\sl s*link*set-destination-node}\hfill(page~\pageref{s*link*set-destination-node})

\item {\sl s*\{link\}*IDs}\hfill(page~\pageref{s*link*IDs})
\item {\sl s*\{link\}*mapc-IDs}\hfill(page~\pageref{s*link*mapc-IDs})

\item [Collections:]

\item [Subclasses:]

\item [Superclasses:]



\end{description}
\horizontalline

\subsection{s*link*name}
\label{s*link*name}

\begin{description}
\item [Name:]  s*link*name

\item [Class:] {\sl s*link}\hfill(page~\pageref{s*link})

\item [Contents:] string (30)

\item [Description:]
The name of the link; limited to thirty chars.

\item [Setf-able:]

\item [Public:]



\end{description}
\horizontalline

\subsection{s*link*created-by}
\label{s*link*created-by}

\begin{description}
\item [Name:]  s*link*created-by

\item [Class:] {\sl s*link}\hfill(page~\pageref{s*link})

\item [Contents:] string

\item [Description:] 
Name of the user who created this link.

\item [Setf-able:] no

\item [Public:]



\end{description}
\horizontalline

\subsection{s*link*created-date}
\label{s*link*created-date}

\begin{description}

\item [Name:]  s*link*created-date

\item [Class:] {\sl s*link}\hfill(page~\pageref{s*link})

\item [Contents:] string

\item [Description:] 
the creation date for this link

\item [Setf-able:]

\item [Public:]



\end{description}
\horizontalline

\subsection{s*link*last-modified-by}
\label{s*link*last-modified-by}

\begin{description}

\item [Name:]  s*link*last-modified-by

\item [Class:] {\sl s*link}\hfill(page~\pageref{s*link})

\item [Contents:] string

\item [Description:] 
The name of user who was last in modifying this
link.

\item [Setf-able:]

\item [Public:]



\end{description}
\horizontalline

\subsection{s*link*last-modified-date}
\label{s*link*last-modified-date}

\begin{description}

\item [Name:]  s*link*last-modified-date

\item [Class:] {\sl s*link}\hfill(page~\pageref{s*link})

\item [Contents:] string

\item [Description:] 
The date of the last modification of this link.

\item [Setf-able:]


\item [Public:]



\end{description}
\horizontalline

\subsection{s*link*source-node}
\label{s*link*source-node}

\begin{description}
\item [Name:]  s*link*source-node

\item [Class:] {\sl s*link}\hfill(page~\pageref{s*link})

\item [Contents:] node-ID

\item [Description:]
The node-ID of the source node of this link.

\item [Setf-able:]


\item [Public:]



\end{description}
\horizontalline

\subsection{s*link*destination-node}
\label{s*link*destination-node}

\begin{description}

\item [Name:]  s*link*destination-node

\item [Class:] {\sl s*link}\hfill(page~\pageref{s*link})

\item [Contents:] node-ID

\item [Description:] 
The destination node for this link.

\item [Setf-able:]

\item [Public:]



\end{description}
\horizontalline

\subsection{s*link*make}
\label{s*link*make}

\begin{description}
\item [Name:]  s*link*make

\item [Class:] {\sl s*link}\hfill(page~\pageref{s*link})

\item [Parameters:]
\item {\sl link-name}:  string (30); a valid link name

\item {\sl from-node-ID}:  node-ID

\item {\sl to-node-ID}:  node-ID




\item [Return-value:]
link-ID if successful.

error object {\sl invalid-link-name} (page~\pageref{invalid-link-name}) if LINK-NAME
contains leading space(s), or its length exceeds 30.

error object {\sl create-link-fails} (page~\pageref{create-link-fails}) if create-link
hb-calls fails. 

error object {\sl write-attribute-fails} (page~\pageref{write-attribute-fails}) if
initialization of link attributes fails.

error object {\sl update-link-info-fails} (page~\pageref{update-link-info-fails}) if update 
operations on LINK-INFO node fails.

\item [Description:]
This function validates link-name, calls remote server
to allocate a new link num; initializes
link-attributes: created by, created ate, last update
by, and last updated date; and runs s*link!make-hooks. 

\item [Public:]





\end{description}
\horizontalline

\subsection{s*link*delete}
\label{s*link*delete}

\begin{description}
\item [Name:]  s*link*delete
\item [Class:] {\sl s*link}\hfill(page~\pageref{s*link})

\item [Parameters:]
\item {\sl link-ID}:  
valid HB link ID number (integer)

\item {\sl from-node-ID}:  node-ID


\item [Return-value:]
LINK-ID if the link deletion was successful, or 
error object otherwise.

\item [Description:]
Permanently remove the link from remote database.


\item [Public:]




\end{description}
\horizontalline

\subsection{s*link*set-name}
\label{s*link*set-name}

\begin{description}
\item [Name:]  s*link*set-name

\item [Class:] {\sl s*link}\hfill(page~\pageref{s*link})

\item [Parameters:]
\item {\sl link-ID}:  
valid HB link ID number (integer)

\item {\sl link-name}:  string (30); a valid link name
 

\item [Return-value:]
LINK-NAME if successful.

error object {\sl invalid-link-name} (page~\pageref{invalid-link-name}) if LINK-NAME contains
leading space(s)/tab(s), or its length exceeds 30.

error object {\sl write-attribute-fails} (page~\pageref{write-attribute-fails}) if hb-write
operation fails.

\item [Description:]
Resets the name of the link. Note that
the link name does not necessarily
correspond to the link label.

\item [Public:]



\end{description}
\horizontalline

\subsection{s*link*set-destination-node}
\label{s*link*set-destination-node}

\begin{description}
\item [Name:]  s*link*set-destination-node

\item [Class:]
{\sl s*link}\hfill(page~\pageref{s*link})

\item [Parameters:]
\item {\sl node-ID}:   An integer representing
a valid hbserver node ID.

\item {\sl link-ID}:  
valid HB link ID number (integer)


\item [Return-value:]
NODE-ID if the destination node of link LINK-ID can be
set to NODE-ID, or error object {\sl move-link-fails} (page~\pageref{move-link-fails})
otherwise. 

\item [Description:]
Reset the target node of the current link to another
node. Note that there is no reciprocal operation,
i.e., one cannot reset the source node of a link
without necessitating the creation of a new link.

\item [Public:]



\end{description}
\horizontalline

\subsection{s*\{link\}*IDs}
\label{s*link*IDs}

\begin{description}
\item [Name:]  s*\{link\}*IDs

\item [Class:] {\sl s*link}\hfill(page~\pageref{s*link})

\item [Parameters:] none

\item [Return-value:]

A list of s*link IDs.

\item [Description:]

Returns a freshly consed list of all currently
defined s*link IDs.

\item [Public:]



\end{description}
\horizontalline

\subsection{s*\{link\}*mapc-IDs}
\label{s*link*mapc-IDs}

\begin{description}
\item [Name:]  s*\{link\}*mapc-IDs

\item [Class:] {\sl s*link}\hfill(page~\pageref{s*link})

\item [Parameters:]
\item {\sl map-ID-fn}:  A function that takes one argument, an ID,
and which performs some side-effect based upon that
value.



\item [Return-value:] nil

\item [Description:]

Calls map-ID-fn once on each currently defined
s*link ID.

\item [Public:]



\end{description}
\horizontalline

\section{s*server-process}
\label{s*server-process}

\begin{description}
\item [Name:]  s*server-process

\item [Layer:] {\sl Server}\hfill(page~\pageref{Server})

\item [Description:]
The server-process class implements connection and
communication operations and attribute functions
related  to the server process, e.g., the directory the
remote databases resides, the name of machine on which
the server is running, the description of the database,
etc. Currently, S*SERVER-PROCESS consists of only one
instance, i.e., hbserver. But the class is designed to
accommodate arbitrary number of server processes. 

\item [Attributes:]
\item {\sl s*sp*name}\hfill(page~\pageref{s*sp*name})
\item {\sl s*sp*description}\hfill(page~\pageref{s*sp*description})
\item {\sl s*sp*directory-path}\hfill(page~\pageref{s*sp*directory-path})
\item {\sl s*sp*ip}\hfill(page~\pageref{s*sp*ip})

\item [Operations:]
\item {\sl s*sp*connect}\hfill(page~\pageref{s*sp*connect})
\item {\sl s*sp*disconnect}\hfill(page~\pageref{s*sp*disconnect})

Administration Operations
\item {\sl s*sp@defserver}\hfill(page~\pageref{s*sp@defserver})

\item [Collections:]

\item [Subclasses:]

\item [Superclasses:]



\end{description}
\horizontalline

\subsection{s*sp*name}
\label{s*sp*name}

\begin{description}

\item [Name:]  s*sp*name

\item [Class:] {\sl s*server-process}\hfill(page~\pageref{s*server-process})

\item [Contents:] String 

\item [Description:]  
Name of server database. It is the same as machine name,
e.g. "uhics.ics.hawaii.edu".

\item [Setf-able:] 

\item [Public:]



\end{description}
\horizontalline

\subsection{s*sp*description}
\label{s*sp*description}

\begin{description}
\item [Name:]  s*sp*description

\item [Class:] {\sl s*server-process}\hfill(page~\pageref{s*server-process})

\item [Contents:] String

\item [Description:]

A short description of the contents or purpose of 
this database.

\item [Setf-able:]


\item [Public:]



\end{description}
\horizontalline

\subsection{s*sp*directory-path}
\label{s*sp*directory-path}

\begin{description}

\item [Name:]  s*sp*directory-path

\item [Class:] {\sl s*server-process}\hfill(page~\pageref{s*server-process})

\item [Contents:] String

\item [Description:]
The pathname to the directory where the datafiles for 
this database are kept.

\item [Setf-able:]


\item [Public:]



\end{description}
\horizontalline

\subsection{s*sp*ip}
\label{s*sp*ip}

\begin{description}
\item [Name:]  s*sp*ip

\item [Class:]
{\sl s*server-process}\hfill(page~\pageref{s*server-process})

\item [Contents:] string.

\item [Description:]
Contains full IP address, eg. "128.171.2.5"

\item [Setf-able:]

\item [Public:]



\end{description}
\horizontalline

\subsection{s*sp*connect}
\label{s*sp*connect}

\begin{description}
\item [Name:]  s*sp*connect

\item [Class:] {\sl s*server-process}\hfill(page~\pageref{s*server-process})

\item [Parameters:]
\item {\sl machine-name}:  
A string which is a valid internet machine address. 
	 

\item [Return-value:] 
t if the connection was made successfully.

error object {\sl missing-required-arg} (page~\pageref{missing-required-arg}) if MACHINE-NAME is
not supplied and public variable S*SP*CURRENT-SERVER is not
set. 

error object {\sl connection-is-on} (page~\pageref{connection-is-on}) if the user tries to 
connect while the connection is already on.

\item [Description:]
This function sets up three network stream processes:
read, write, and event; sets global variable
S*SP*CURRENT-SERVER; subscribes the initial set of
events; initializes local caches of nodes and links, and
runs S*SP!CONNECT-HOOKS.


\item [Public:]






\end{description}
\horizontalline

\subsection{s*sp*disconnect}
\label{s*sp*disconnect}

\begin{description}

\item [Name:]  s*sp*disconnect

\item [Class:] {\sl s*server-process}\hfill(page~\pageref{s*server-process})

\item [Parameters:] none

\item [Return-value:]
t if the disconnection from the server was successful.

Error object {\sl connection-is-off} (page~\pageref{connection-is-off}) if the user
attempts to disconnect while the connection is not on. 


\item [Description:]
This function deletes all three network processes (ie.,
read, write, and event) and sets global status variable
S*SP!CONNECTED to nil.

\item [Public:]




\end{description}
\horizontalline

\subsection{s*sp@defserver}
\label{s*sp@defserver}

\begin{description}
\item [Name:]  s*sp@defserver

\item [Class:]
{\sl s*server-process}\hfill(page~\pageref{s*server-process})

\item [Parameters:]
\item {\sl name}:  string, eg, "zero.ics".


\item {\sl description}:  string, e.g., "DesignBase", "Testbase". 


\item {\sl path}:  string. It must be absolute path name. 


\item {\sl ip}:  string. e.g., "128.71.4.4"



\item [Return-value:]
newly-defined server struct.

\item [Description:]
Defines a new instance of server structure.  Note that
the system administrator should predefine all servers
accessable to the user. Attempts to access an undefined
server is an error condition.

\item [Public:]



\end{description}
\horizontalline

\section{s*snode}
\label{s*snode}

\begin{description}
\item [Name:]  s*snode

\item [Layer:]
{\sl Server}\hfill(page~\pageref{Server})

\item [Description:]

The SYS-NODE is provided by the server subsystem to other EGRET modules
(including the server itself) as a uniform mechanism for handling a set
of special nodes which store data used by these modules in implementing
advanced functions. The major features of SYS-NODES in comparison with
the regular Egret nodes are:

(1)  They are "hidden" from the users of the system.

(2)  They are internal to the modules in which they are defined
     and used.

(3)  They are susceptible to corruption and thus require recovery 
     operations.

(4)  They are normally not deleted once created.

(5)  Operations on them are normally performed via special SYS-NODE
     specific event functions. A public function also exists to
     update the contents of the system node.
 
(6)  Multiple subsystems can share the same system node. For example,
     unread node uses the done-node data defined by the to-do module.

(7)  SYS-NODEs store and retrieve a single, variable length string,
     whose contents are interpreted by its utilizing systems.

(8)  Unique IDs for system nodes (snode-ID) are not integers but a unique
     string.
 
(9)  Concurrent access is maintained by the server subsystem. Users cannot
     explicitly lock or unlock a system node. All operations upon SYS-
     NODEs are assumed to require a brief amount of time; they are 
     reliably unlocked regardless of whether operations terminate 
     successfully or signal an error.

(10) Definition and instantiation of a SYS-NODE are separate
     operations. Definition describes the event operations that will 
     run if the instance of the system node exists on the server. A 
     separate operation (s*snode!make) exists to actually create 
     the entity. 

\item [Attributes:]

\item [Operations:]
\item {\sl s*snode*define}\hfill(page~\pageref{s*snode*define})
\item {\sl s*snode*with-data}\hfill(page~\pageref{s*snode*with-data})
\item {\sl s*snode*with-data-locked}\hfill(page~\pageref{s*snode*with-data-locked})

\item {\sl s*snode@make}\hfill(page~\pageref{s*snode@make})

\item [Collections:]


\item [Subclasses:]


\item [Superclasses:]


\item [Instances:]



\end{description}
\horizontalline

\subsection{s*snode*define}
\label{s*snode*define}

\begin{description}
\item [Name:]  s*snode*define

\item [Class:]
{\sl s*snode}\hfill(page~\pageref{s*snode})

\item [Parameters:]
\item {\sl snode-name}:  string

\item {\sl data-event}:  symbol

\item {\sl connect}:  symbol

\item {\sl disconnect}:  symbol


\item [Return-value:]
Macro with side effect of defining an SNODE object.

\item [Description:]
This macro defines a specialized system node class with a
single instance SNODE-NAME. It registers SNODE-NAME on
the global system node data structure, and data update
event and connection event functions, if any, on
respective hooks.

\item [Public:]



\end{description}
\horizontalline

\subsection{s*snode*with-data}
\label{s*snode*with-data}

\begin{description}
\item [Name:]  s*snode*with-data

\item [Class:]
{\sl s*snode}\hfill(page~\pageref{s*snode})

\item [Parameters:]
\item {\sl snode-name}:  string

\item {\sl body}:  list of Lisp forms


\item [Return-value:] 
Return value of the last form in BODY.

\item [Description:]
Retrieves sys-node NODE-NAME and then executes BODY.
NODE-NAME must be a name of an existent node. 
BODY is a list of forms to be executed.  This macro is
evaluated for its side effects.

\item [Public:]




\end{description}
\horizontalline

\subsection{s*snode*with-data-locked}
\label{s*snode*with-data-locked}

\begin{description}
\item [Name:]  s*snode*with-data-locked

\item [Class:]
{\sl s*snode}\hfill(page~\pageref{s*snode})

\item [Parameters:]
\item {\sl snode-name}:  string

\item {\sl body}:  list of Lisp forms


\item [Return-value:]
Return value of last form in BODY or error object
{\sl lock-snode-fails} (page~\pageref{lock-snode-fails}) if s*snode-lock operation fails..

\item [Description:]
Retrieves sys-node NODE-NAME with lock and then
executes BODY.  NODE-NAME must be a name of an
existent node.  BODY is a list of forms to be
executed.  This macro is evaluated for its side
effects

\item [Public:]



\end{description}
\horizontalline

\subsection{s*snode@make}
\label{s*snode@make}

\begin{description}
\item [Name:]  s*snode@make

\item [Class:]
{\sl s*snode}\hfill(page~\pageref{s*snode})

\item [Parameters:]
\item {\sl snode-name}:  string

\item {\sl initial-value}:  string


\item [Return-value:] SNODE-NAME

\item [Description:]
Instantiates an snode, i.e., creating node in remote
database with name SNODE-NAME, and stores INITIAL-VALUE
in the node. Note this is an administrative function. If
SNODE-NAME already exists, re-initializes it to
INIT-VALUE.

\item [Public:]



\end{description}
\horizontalline

\section{s*serror}
\label{s*serror}

\begin{description}
\item [Name:]  s*serror

\item [Layer:]
{\sl Server}\hfill(page~\pageref{Server}) 

\item [Description:]
This class contains all error objects in the Server subsystem.
As a general rule, server level functions trap and report but
do not handle errors. It is the responsility of the calling
function to interpret various error objects and decide what
actions to take. Lower-level server functions use SIGNAL to
return to top-level server functions in case of error.
Top-level server functions must catch all server-level errors,
and guaranttee returning to the calling function with
approciate error objects.

\item [Attributes:] See U*ERROR

\item [Operations:] See U*ERROR

\item [Collections:]

\item [Subclasses:]

\item [Superclasses:]
\item {\sl u*error}\hfill(page~\pageref{u*error})

\item [Instances:]
\item {\sl lock-snode-fails}\hfill(page~\pageref{lock-snode-fails})
\item {\sl missing-required-arg}\hfill(page~\pageref{missing-required-arg})
\item {\sl write-attribute-fails}\hfill(page~\pageref{write-attribute-fails})
\item {\sl read-attribute-fails}\hfill(page~\pageref{read-attribute-fails})
\item {\sl get-entity-IDs-fails}\hfill(page~\pageref{get-entity-IDs-fails})
\item {\sl conflicting-hook-constraints}\hfill(page~\pageref{conflicting-hook-constraints})
\item {\sl unknown-hb-error}\hfill(page~\pageref{unknown-hb-error})
\item {\sl uninstantiated-system-node}\hfill(page~\pageref{uninstantiated-system-node})

\item {\sl parse-event-fails}\hfill(page~\pageref{parse-event-fails})
\item {\sl subscribe-event-fails}\hfill(page~\pageref{subscribe-event-fails})
\item {\sl unsubscribe-event-fails}\hfill(page~\pageref{unsubscribe-event-fails})

\item {\sl invalid-node-name}\hfill(page~\pageref{invalid-node-name})
\item {\sl show-lock-fails}\hfill(page~\pageref{show-lock-fails})
\item {\sl lock-node-fails}\hfill(page~\pageref{lock-node-fails})
\item {\sl create-node-fails}\hfill(page~\pageref{create-node-fails})
\item {\sl delete-node-fails}\hfill(page~\pageref{delete-node-fails})
\item {\sl node-still-referenced}\hfill(page~\pageref{node-still-referenced})
\item {\sl node-still-locked}\hfill(page~\pageref{node-still-locked})
\item {\sl node-not-found}\hfill(page~\pageref{node-not-found})

\item {\sl invalid-link-name}\hfill(page~\pageref{invalid-link-name})
\item {\sl move-link-fails}\hfill(page~\pageref{move-link-fails})
\item {\sl create-link-fails}\hfill(page~\pageref{create-link-fails})
\item {\sl link-not-found}\hfill(page~\pageref{link-not-found})
\item {\sl update-link-info-fails}\hfill(page~\pageref{update-link-info-fails})

\item {\sl connection-is-on}\hfill(page~\pageref{connection-is-on})
\item {\sl connection-is-off}\hfill(page~\pageref{connection-is-off})
\item {\sl server-not-found}\hfill(page~\pageref{server-not-found})









\end{description}
\horizontalline

\subsection{lock-snode-fails}
\label{lock-snode-fails}

\begin{description}
\item [Name:]  lock-snode-fails

\item [Class:]
{\sl s*serror}\hfill(page~\pageref{s*serror})

\item [Description:]
Attempt to lock a system node fails.


\end{description}
\horizontalline

\subsection{missing-required-arg}
\label{missing-required-arg}

\begin{description}
\item [Name:]  missing-required-arg


\item [Class:]
{\sl s*serror}\hfill(page~\pageref{s*serror})


\item [Description:] 
Required arg is either missing or unccaptable. 



\end{description}
\horizontalline

\subsection{write-attribute-fails}
\label{write-attribute-fails}

\begin{description}
\item [Name:]  write-attribute-fails


\item [Class:]
{\sl s*serror}\hfill(page~\pageref{s*serror})


\item [Description:] 
Data fails to be written to remote persistent store.


\end{description}
\horizontalline

\subsection{read-attribute-fails}
\label{read-attribute-fails}

\begin{description}

\item [Name:]  read-attribute-fails


\item [Class:]
{\sl s*serror}\hfill(page~\pageref{s*serror})


\item [Description:] 
Field data fails to be retrieved from remote database
server.



\end{description}
\horizontalline

\subsection{get-entity-IDs-fails}
\label{get-entity-IDs-fails}

\begin{description}
\item [Name:]  get-entity-IDs-fails

\item [Class:]
{\sl s*serror}\hfill(page~\pageref{s*serror})

\item [Description:]
Operation on retrieving the list of either all node or link
IDs from remote server fails.



\end{description}
\horizontalline

\subsection{conflicting-hook-constraints}
\label{conflicting-hook-constraints}

\begin{description}
\item [Name:]  conflicting-hook-constraints

\item [Class:]
{\sl s*serror}\hfill(page~\pageref{s*serror})

\item [Description:]
The supplied ordering constraints are conflicting with each
other and thus cannot be satisfied.

\end{description}
\horizontalline

\subsection{unknown-hb-error}
\label{unknown-hb-error}

\begin{description}

\item [Name:]  unknown-hb-error


\item [Class:]
{\sl s*serror}\hfill(page~\pageref{s*serror})


\item [Description:] 
Hyperbase operation fails due to some unknown reason(s).



\end{description}
\horizontalline

\subsection{uninstantiated-system-node}
\label{uninstantiated-system-node}

\begin{description}
\item [Name:]  uninstantiated-system-node

\item [Class:]
{\sl s*serror}\hfill(page~\pageref{s*serror})


\item [Description:] 
A SYS-NODE must be initialized before use. Fails to do so
would cause this error to be signaled.



\end{description}
\horizontalline

\subsection{parse-event-fails}
\label{parse-event-fails}

\begin{description}
\item [Name:]  parse-event-fails


\item [Class:]
{\sl s*serror}\hfill(page~\pageref{s*serror})


\item [Description:]
Incoming event string from the remote server cannot be
parsed.. 


\end{description}
\horizontalline

\subsection{subscribe-event-fails}
\label{subscribe-event-fails}

\begin{description}
\item [Name:]  subscribe-event-fails


\item [Class:]
{\sl s*serror}\hfill(page~\pageref{s*serror})


\item [Description:]
Operation for subscribing an event on remote server fails. 

\end{description}
\horizontalline

\subsection{unsubscribe-event-fails}
\label{unsubscribe-event-fails}

\begin{description}

\item [Name:]  unsubscribe-event-fails


\item [Class:]
{\sl s*serror}\hfill(page~\pageref{s*serror})


\item [Description:]
Attempt to unsubscribe an event previsouly subscribed
fails.


\end{description}
\horizontalline

\subsection{invalid-node-name}
\label{invalid-node-name}

\begin{description}

\item [Name:]  invalid-node-name


\item [Class:]
{\sl s*serror}\hfill(page~\pageref{s*serror})


\item [Description:]
Node naming violation, i.e., exceeding 40 characters or
containing leading space or tabs.


\end{description}
\horizontalline

\subsection{show-lock-fails}
\label{show-lock-fails}

\begin{description}
\item [Name:]  show-lock-fails


\item [Class:]
{\sl s*serror}\hfill(page~\pageref{s*serror})


\item [Description:]
cannot determine who has the lock to the requested node.


\end{description}
\horizontalline

\subsection{lock-node-fails}
\label{lock-node-fails}

\begin{description}

\item [Name:]  lock-node-fails


\item [Class:]
{\sl s*serror}\hfill(page~\pageref{s*serror})


\item [Description:]
cannot lock a node.


\end{description}
\horizontalline

\subsection{create-node-fails}
\label{create-node-fails}

\begin{description}
\item [Name:]  create-node-fails


\item [Class:]
{\sl s*serror}\hfill(page~\pageref{s*serror})


\item [Description:]
Attempt to create new node fails.

\end{description}
\horizontalline

\subsection{delete-node-fails}
\label{delete-node-fails}

\begin{description}
\item [Name:]  delete-node-fails


\item [Class:]
{\sl s*serror}\hfill(page~\pageref{s*serror})


\item [Description:]
Attempt to delete a node form remote persistent store fails.


\end{description}
\horizontalline

\subsection{node-still-referenced}
\label{node-still-referenced}

\begin{description}

\item [Name:]  node-still-referenced


\item [Class:]
{\sl s*serror}\hfill(page~\pageref{s*serror})


\item [Description:]
Attempts to delete a node which still has incoming links.


\end{description}
\horizontalline

\subsection{node-still-locked}
\label{node-still-locked}

\begin{description}
\item [Name:]  node-still-locked


\item [Class:]
{\sl s*serror}\hfill(page~\pageref{s*serror})


\item [Description:]
Trying to delete a node which is locked. 


\end{description}
\horizontalline

\subsection{node-not-found}
\label{node-not-found}

\begin{description}
\item [Name:]  node-not-found

\item [Class:]
{\sl s*serror}\hfill(page~\pageref{s*serror})

\item [Description:] requested node does not exist.



\end{description}
\horizontalline

\subsection{invalid-link-name}
\label{invalid-link-name}

\begin{description}
\item [Name:]  invalid-link-name


\item [Class:]
{\sl s*serror}\hfill(page~\pageref{s*serror})


\item [Description:]
Link naming violation, i.e., either the name exceeds 30
characters or contains leading space or tabs.


\end{description}
\horizontalline

\subsection{move-link-fails}
\label{move-link-fails}

\begin{description}
\item [Name:]  move-link-fails


\item [Class:]
{\sl s*serror}\hfill(page~\pageref{s*serror})

\item [Description:]
Attempt to reset the target node of a link fails.


\end{description}
\horizontalline

\subsection{create-link-fails}
\label{create-link-fails}

\begin{description}

\item [Name:]  create-link-fails


\item [Class:]
{\sl s*serror}\hfill(page~\pageref{s*serror})


\item [Description:]
Attempt to create new link fails.


\end{description}
\horizontalline

\subsection{link-not-found}
\label{link-not-found}

\begin{description}
\item [Name:]  link-not-found

\item [Class:]
{\sl s*serror}\hfill(page~\pageref{s*serror})

\item [Description:] requested link does not exist. 



\end{description}
\horizontalline

\subsection{update-link-info-fails}
\label{update-link-info-fails}

\begin{description}
\item [Name:]  update-link-info-fails

\item [Class:]
{\sl s*serror}\hfill(page~\pageref{s*serror})

\item [Description:] 
attempts to update the link-info system node fails.



\end{description}
\horizontalline

\subsection{connection-is-on}
\label{connection-is-on}

\begin{description}
\item [Name:]  connection-is-on


\item [Class:]
{\sl s*serror}\hfill(page~\pageref{s*serror})


\item [Description:]
Trying to connect to the remote server while the
current client is already connected.


\end{description}
\horizontalline

\subsection{connection-is-off}
\label{connection-is-off}

\begin{description}

\item [Name:]  connection-is-off


\item [Class:]
{\sl s*serror}\hfill(page~\pageref{s*serror})


\item [Description:]
Trying to disconnect while the client is already
disconnected from the remote server.


\end{description}
\horizontalline

\subsection{server-not-found}
\label{server-not-found}

\begin{description}
\item [Name:]  server-not-found

\item [Class:]
{\sl s*serror}\hfill(page~\pageref{s*serror})

\item [Description:] requested server is not defined.



\end{description}
\horizontalline

\section{s*event}
\label{s*event}

\begin{description}
\item [Name:]  s*event

\item [Layer:]
{\sl Server}\hfill(page~\pageref{Server})

\item [Description:]
S*EVENT represents a class of a predefined event types. It is
implemented in terms of U*HOOK, which in turn is an extension of
Emacs Lisp hook facility. S*EVENT allows arbitrary ordering
constraints to be imposed on functions to be intalled on any
given event dispatching queue.

S*EVENT is special in that all its instances are predefined; the
user of these events are not allowed to instantiated them, though
they may add functions to or delete functions from them, or
reinitialize them.  Currently, there are currently elevent (11)
event types, all of which are documentated below.  The execution
of functions on these event queues is triggered by events from
the remote database server, rather than directed by the Server
itself.

\item [Attributes:]
\item {\sl s*event*event-handlers}\hfill(page~\pageref{s*event*event-handlers})

\item [Operations:]
\item {\sl s*event*initialize}\hfill(page~\pageref{s*event*initialize})
\item {\sl s*event*add-event-handler-fn}\hfill(page~\pageref{s*event*add-event-handler-fn})
\item {\sl s*event*remove-event-handler-fn}\hfill(page~\pageref{s*event*remove-event-handler-fn})

\item [Collections:]

\item [Subclasses:]

\item [Superclasses:]

\item [Instances:]
\item {\sl s*node!newname-event-hooks}\hfill(page~\pageref{s*node!newname-event-hooks})
\item {\sl s*node!rename-event-hooks}\hfill(page~\pageref{s*node!rename-event-hooks})
\item {\sl s*node!delete-event-hooks}\hfill(page~\pageref{s*node!delete-event-hooks})
\item {\sl s*node!data-event-hooks}\hfill(page~\pageref{s*node!data-event-hooks})
\item {\sl s*node!lock-event-hooks}\hfill(page~\pageref{s*node!lock-event-hooks})
\item {\sl s*node!unlock-event-hooks}\hfill(page~\pageref{s*node!unlock-event-hooks})
\item {\sl s*node!font-event-hooks}\hfill(page~\pageref{s*node!font-event-hooks})
\item {\sl s*node!geometry-event-hooks}\hfill(page~\pageref{s*node!geometry-event-hooks})
\item {\sl s*link!newname-event-hooks}\hfill(page~\pageref{s*link!newname-event-hooks})
\item {\sl s*link!rename-event-hooks}\hfill(page~\pageref{s*link!rename-event-hooks})
\item {\sl s*link!delete-event-hooks}\hfill(page~\pageref{s*link!delete-event-hooks})












\end{description}
\horizontalline

\subsection{s*event*event-handlers}
\label{s*event*event-handlers}

\begin{description}
\item [Name:]  s*event*event-handlers

\item [Class:]
{\sl s*event}\hfill(page~\pageref{s*event})

\item [Contents:] List of functions

\item [Description:] 
An ordered list of functions to be invoked upon
receiving a given type of event.


\item [Setf-able:] no

\item [Public:]



\end{description}
\horizontalline

\subsection{s*event*initialize}
\label{s*event*initialize}

\begin{description}
\item [Name:]  s*event*initialize

\item [Class:]
{\sl s*event}\hfill(page~\pageref{s*event})

\item [Parameters:]
\item {\sl s*event-instance}:  symbol


\item [Return-value:] t

\item [Description:] Removes all current event handlers from 
S*EVENT-INSTANCE.

\item [Public:]



\end{description}
\horizontalline

\subsection{s*event*add-event-handler-fn}
\label{s*event*add-event-handler-fn}

\begin{description}
\item [Name:]  s*event*add-event-handler-fn

\item [Class:]
{\sl s*event}\hfill(page~\pageref{s*event})

\item [Parameters:]
\item {\sl s*event-instance}:  symbol

\item {\sl handler-fn-name}:  function symbol

\item {\sl before-handlers}:  functional symbol

\item {\sl after-handlers}:  function symbol


\item [Return-value:] 
List of function symbols or error object
CONFLICTING-HOOK-CONSTRAINTS. 

\item [Description:]
Inserts HANDLER-FN-NAME into S*EVENT S*EVENT-INSTANCE
so that BEFORE-HANDLERS and AFTER-HANDLERS constraintsw
are satisfied.

\item [Public:]



\end{description}
\horizontalline

\subsection{s*event*remove-event-handler-fn}
\label{s*event*remove-event-handler-fn}

\begin{description}
\item [Name:]  s*event*remove-event-handler-fn

\item [Class:]
{\sl s*event}\hfill(page~\pageref{s*event})

\item [Parameters:]
\item {\sl s*event-instance}:  symbol

\item {\sl handler-fn-name}:  function symbol


\item [Return-value:] Symbol of function being removed

\item [Description:] 
Removes function HANDLER-FN-NAME from S*EVENT-INSTANCE. 

\item [Public:]



\end{description}
\horizontalline

\subsection{s*node!newname-event-hooks}
\label{s*node!newname-event-hooks}

\begin{description}
\item [Name:]  s*node!newname-event-hooks

\item [Class:]
{\sl s*event}\hfill(page~\pageref{s*event})

\item [Description:]
Holds a list of functions executed upon receiving a
'new-node-name' event, i.e., with incoming event type 'n
name', entity type is 'node' and the target node can be
found in the local cache structure. 


\end{description}
\horizontalline

\subsection{s*node!rename-event-hooks}
\label{s*node!rename-event-hooks}

\begin{description}
\item [Name:]  s*node!rename-event-hooks

\item [Class:]
{\sl s*event}\hfill(page~\pageref{s*event})

\item [Description:]
Holds a list of functions executed upon receiving a
'rename-node-name' event, i.e., with incoming event
type 'n name', entity type is 'node' and the target
node cannot be found in the local cache structure.



\end{description}
\horizontalline

\subsection{s*node!delete-event-hooks}
\label{s*node!delete-event-hooks}

\begin{description}
\item [Name:]  s*node!delete-event-hooks

\item [Class:]
{\sl s*event}\hfill(page~\pageref{s*event})

\item [Description:]
Holds a list of functions executed upon receiving
delete-node event.  Each function is passed a node-ID.



\end{description}
\horizontalline

\subsection{s*node!data-event-hooks}
\label{s*node!data-event-hooks}

\begin{description}
\item [Name:]  s*node!data-event-hooks

\item [Class:]
{\sl s*event}\hfill(page~\pageref{s*event})

\item [Description:]
Holds a list of functions executed upon receiving a
'write data' event.



\end{description}
\horizontalline

\subsection{s*node!lock-event-hooks}
\label{s*node!lock-event-hooks}

\begin{description}

\item [Name:]  s*node!lock-event-hooks


\item [Class:]
{\sl s*event}\hfill(page~\pageref{s*event})


\item [Description:]
Holds a list of functions executed upon
receiving a 'lock' event.


\end{description}
\horizontalline

\subsection{s*node!unlock-event-hooks}
\label{s*node!unlock-event-hooks}

\begin{description}
\item [Name:]  s*node!unlock-event-hooks

\item [Class:]
{\sl s*event}\hfill(page~\pageref{s*event})

\item [Description:]
Holds a list of functions executed upon
receiving a 'unlock' event.


\end{description}
\horizontalline

\subsection{s*node!font-event-hooks}
\label{s*node!font-event-hooks}

\begin{description}
\item [Name:]  s*node!font-event-hooks

\item [Class:]
{\sl s*event}\hfill(page~\pageref{s*event})

\item [Description:]
Holds a list of functions executed upon receiving
a 'write-font' event.


\end{description}
\horizontalline

\subsection{s*node!geometry-event-hooks}
\label{s*node!geometry-event-hooks}

\begin{description}

\item [Name:]  s*node!geometry-event-hooks


\item [Class:]
{\sl s*event}\hfill(page~\pageref{s*event})


\item [Description:]
Holds a list of functions executed upon receiving a
'write-geometry' event.


\end{description}
\horizontalline

\subsection{s*link!newname-event-hooks}
\label{s*link!newname-event-hooks}

\begin{description}
\item [Name:]  s*link!newname-event-hooks

\item [Class:]
{\sl s*event}\hfill(page~\pageref{s*event})

\item [Description:]
Holds a list of functions executed upon receiving
a 'write-geometry' event.


\end{description}
\horizontalline

\subsection{s*link!rename-event-hooks}
\label{s*link!rename-event-hooks}

\begin{description}
\item [Name:]  s*link!rename-event-hooks

\item [Class:]
{\sl s*event}\hfill(page~\pageref{s*event})

\item [Description:]
Holds a list of functions executed upon receiving a
'new-link-name' event, i.e., with incoming event
type 'n name', entity type is 'link' and the target
link can be found in the local cache structure.


\end{description}
\horizontalline

\subsection{s*link!delete-event-hooks}
\label{s*link!delete-event-hooks}

\begin{description}
\item [Name:]  s*link!delete-event-hooks

\item [Class:]
{\sl s*event}\hfill(page~\pageref{s*event})

\item [Description:]
Holds a list of functions executed upon receiving
delete-link event.  Each function is passed link-ID.


\end{description}
\horizontalline

\chapter{Type}
\label{Type}

\begin{description}
\item [Name:]  Type

\item [Description:]

The Type subsystem defines a collaborative, extensible
data model on top of the fixed Server subsystem node
and link types.  

\item [Public-classes:]
\item {\sl t*node-schema}\hfill(page~\pageref{t*node-schema})
\item {\sl t*node-instance}\hfill(page~\pageref{t*node-instance})

\item {\sl t*link-schema}\hfill(page~\pageref{t*link-schema})
\item {\sl t*link-instance}\hfill(page~\pageref{t*link-instance})

\item {\sl t*field-schema}\hfill(page~\pageref{t*field-schema})

\item {\sl t*layer}\hfill(page~\pageref{t*layer})

\item {\sl t*error}\hfill(page~\pageref{t*error})
\item {\sl t*event}\hfill(page~\pageref{t*event})

\end{description}
\horizontalline

\section{t*node-schema}
\label{t*node-schema}

\begin{description}
\item [Name:]  t*node-schema

\item [Layer:] {\sl Type}\hfill(page~\pageref{Type})

\item [Description:]

Each instance of a node-schema defines the consensually 
agreed upon structural features (i.e. the set of fields)
for a set of node instances.  However, these instances
may not necessarily conform to these structural 
features. 

\item [Attributes:]
\item {\sl t*node-schema*name}\hfill(page~\pageref{t*node-schema*name})
\item {\sl t*node-schema*node-IDs}\hfill(page~\pageref{t*node-schema*node-IDs})
\item {\sl t*node-schema*field-IDs}\hfill(page~\pageref{t*node-schema*field-IDs})

\item [Operations:]
\item {\sl t*node-schema*delete}\hfill(page~\pageref{t*node-schema*delete})
\item {\sl t*node-schema*divergence}\hfill(page~\pageref{t*node-schema*divergence})
\item {\sl t*node-schema*set-name}\hfill(page~\pageref{t*node-schema*set-name})
\item {\sl t*node-schema*delete-fields}\hfill(page~\pageref{t*node-schema*delete-fields})
\item {\sl t*node-schema*add-fields}\hfill(page~\pageref{t*node-schema*add-fields})
\item {\sl t*node-schema*instantiate}\hfill(page~\pageref{t*node-schema*instantiate})
\item {\sl t*node-schema*make}\hfill(page~\pageref{t*node-schema*make})

\item {\sl t*\{node-schema\}*IDs}\hfill(page~\pageref{t*node-schema*IDs})
\item {\sl t*\{node-schema\}*mapc-IDs}\hfill(page~\pageref{t*node-schema*mapc-IDs})

\item [Subclasses:]


\item [Superclasses:]


\item [Instances:]




\end{description}
\horizontalline

\subsection{t*node-schema*name}
\label{t*node-schema*name}

\begin{description}
\item [Name:]  t*node-schema*name

\item [Class:] 

\item [Contents:] Symbol 

\item [Description:] The name of the node-schema

\item [Setf-able:] see t*node-schema*set-name

\item [Public:] yes.



\end{description}
\horizontalline

\subsection{t*node-schema*node-IDs}
\label{t*node-schema*node-IDs}

\begin{description}
\item [Name:]  t*node-schema*node-IDs

\item [Class:] {\sl t*node-schema}\hfill(page~\pageref{t*node-schema})

\item [Contents:] a list of node-IDs

\item [Description:]

This list of node-IDs are those that currently have
this node-schema as their associated consensual structure.

\item [Setf-able:] Updated automatically by type system. 

\item [Public:] yes



\end{description}
\horizontalline

\subsection{t*node-schema*field-IDs}
\label{t*node-schema*field-IDs}

\begin{description}
\item [Name:]  t*node-schema*field-IDs

\item [Class:] {\sl t*node-schema}\hfill(page~\pageref{t*node-schema})

\item [Contents:] a list of field-schema-IDs

\item [Description:]

Returns a list of the field-schema-IDs currently 
associated with this node-schema.

\item [Setf-able:] See t*node-schema*add-fields and 
t*node-schema*delete-fields
 
\item [Public:] yes



\end{description}
\horizontalline

\subsection{t*node-schema*delete}
\label{t*node-schema*delete}

\begin{description}
\item [Name:]  t*node-schema*delete

\item [Class:] {\sl t*node-schema}\hfill(page~\pageref{t*node-schema})

\item [Parameters:]
\item {\sl node-schema-ID}:  a server-level node-ID that corresponds to an 
instance of a type-level node-schema. 



\item [Return-value:]
NODE-SCHEMA-ID if successful. 

Error object {\sl invalid-node-schema-ID} (page~\pageref{invalid-node-schema-ID}) if bad 
node-schema-ID. 

Error object {\sl unknown-hb-error} (page~\pageref{unknown-hb-error}) if this call
fails for some other reason.

\item [Description:]

This operation "marks" node-schema-ID for deletion, 
thus ensuring that no future use of node-schema-ID 
will occur (once this deletion event has been propogated
to all other connected users.)  Node-schema-ID will
continue to exist in the database and its prior 
references and instances will continue to exist. 

More specifically, a deleted node-schema-ID will be
considered an invalid argument to the node-schema
operations: instantiate, add-fields, delete-fields,
set-name, and delete. The node-instance operation clone
will also be disabled for instances of this node-schema.
However, the attributes and the divergence operation will
continue to operate normally when passed this
node-schema-ID.

\item [Public:]



\end{description}
\horizontalline

\subsection{t*node-schema*divergence}
\label{t*node-schema*divergence}

\begin{description}
\item [Name:]  t*node-schema*divergence

\item [Class:] {\sl t*node-schema}\hfill(page~\pageref{t*node-schema})

\item [Parameters:]
\item {\sl node-schema-ID}:  a server-level node-ID that corresponds to an 
instance of a type-level node-schema. 



\item [Return-value:]
An integer corresponding to the divergence metric for 
this node-schema and its instances if successful.

Error object {\sl invalid-node-schema-ID} (page~\pageref{invalid-node-schema-ID}) if bad 
node-schema-ID.

Error object {\sl unknown-hb-error} (page~\pageref{unknown-hb-error}) if this computation
fails for some other reason.

\item [Description:]

Computes the structural divergence of the instances
of this node schema.

\item [Public:]



\end{description}
\horizontalline

\subsection{t*node-schema*set-name}
\label{t*node-schema*set-name}

\begin{description}
\item [Name:]  t*node-schema*set-name

\item [Class:] {\sl t*node-schema}\hfill(page~\pageref{t*node-schema})

\item [Parameters:]
\item {\sl node-schema-ID}:  a server-level node-ID that corresponds to an 
instance of a type-level node-schema. 


\item {\sl node-name}:  
A valid node name. This currently means that it is a
string of less than 40 characters, and that it does
not contain leading space(s) or tabs.


\item [Return-value:]
Returns the new name if successful.

Error object {\sl invalid-node-schema-ID} (page~\pageref{invalid-node-schema-ID}) if bad 
node-schema-ID. 

Error object {\sl invalid-node-name} (page~\pageref{invalid-node-name}) if node-name 
violates node name conventions.

Error object {\sl unknown-hb-error} (page~\pageref{unknown-hb-error}) if this call
fails for some other reason.

\item [Description:]
Renames the node-schema-ID to node-name.  
(Note that node names need not be unique at the 
type level.)

\item [Public:]



\end{description}
\horizontalline

\subsection{t*node-schema*delete-fields}
\label{t*node-schema*delete-fields}

\begin{description}
\item [Name:]  t*node-schema*delete-fields

\item [Class:] {\sl t*node-schema}\hfill(page~\pageref{t*node-schema})

\item [Parameters:]
\item {\sl node-schema-ID}:  a server-level node-ID that corresponds to an 
instance of a type-level node-schema. 


\item {\sl field-schema-IDs}:  list of field-schema-ID
 

\item [Return-value:]
The updated list of field-schema-IDs if successful.

Error object {\sl invalid-node-schema-ID} (page~\pageref{invalid-node-schema-ID}) if
node-schema-ID was not a t*node-schema.

Error object {\sl invalid-field-ID} (page~\pageref{invalid-field-ID}) if field-schema-IDs 
are invalid or are not present in node-schema-ID.

Error object {\sl unknown-hb-error} (page~\pageref{unknown-hb-error}) if the call fails
for some other reason.

\item [Description:]

Removes one or more fields from node-schema-ID. 

\item [Public:]



\end{description}
\horizontalline

\subsection{t*node-schema*add-fields}
\label{t*node-schema*add-fields}

\begin{description}
\item [Name:]  t*node-schema*add-fields

\item [Class:] {\sl t*node-schema}\hfill(page~\pageref{t*node-schema})

\item [Parameters:]
\item {\sl node-schema-ID}:  a server-level node-ID that corresponds to an 
instance of a type-level node-schema. 


\item {\sl field-schema-IDs}:  list of field-schema-ID


\item [Return-value:] 
The updated list of field-schema-IDs if successful.

Error object {\sl invalid-node-schema-ID} (page~\pageref{invalid-node-schema-ID}) if 
node-schema-ID was not a t*node-schema.

Error object {\sl invalid-field-ID} (page~\pageref{invalid-field-ID}) if any one of the 
field-schema-IDs is not a t*field-schema, or if any
one of the field-schemas already exists in node-schema-ID.

Error object {\sl unknown-hb-error} (page~\pageref{unknown-hb-error}) if the call 
fails for some other reason. 

\item [Description:]  Adds one or more field-schema-IDs to 
this node-schema.

\item [Public:]



\end{description}
\horizontalline

\subsection{t*node-schema*instantiate}
\label{t*node-schema*instantiate}

\begin{description}
\item [Name:]  t*node-schema*instantiate

\item [Class:] {\sl t*node-schema}\hfill(page~\pageref{t*node-schema})

\item [Parameters:]
\item {\sl node-name}:  
A valid node name. This currently means that it is a
string of less than 40 characters, and that it does
not contain leading space(s) or tabs.

\item {\sl node-schema-ID}:  a server-level node-ID that corresponds to an 
instance of a type-level node-schema. 



\item [Return-value:] 
A newly created node-id with name node-name if successful.

Error object {\sl invalid-node-schema-ID} (page~\pageref{invalid-node-schema-ID}) if 
node-schema-ID was not a t*node-schema.

Error object {\sl invalid-node-name} (page~\pageref{invalid-node-name}) if node-name was
invalid.

Error object {\sl unknown-hb-error} (page~\pageref{unknown-hb-error}) if call fails
for some other reason.

\item [Description:]

Creates a new type-level node-instance based upon this
node-schema.

\item [Public:]



\end{description}
\horizontalline

\subsection{t*node-schema*make}
\label{t*node-schema*make}

\begin{description}
\item [Name:]  t*node-schema*make

\item [Class:] {\sl t*node-schema}\hfill(page~\pageref{t*node-schema})

\item [Parameters:]
\item {\sl node-name}:  
A valid node name. This currently means that it is a
string of less than 40 characters, and that it does
not contain leading space(s) or tabs.

\item {\sl field-schema-IDs}:  list of field-schema-ID


\item [Return-value:] 
A newly created node-schema-ID if successful.

Error object {\sl invalid-node-name} (page~\pageref{invalid-node-name}) if the node-name
was syntactically illegal.

Error object {\sl invalid-field-ID} (page~\pageref{invalid-field-ID}) if one or more of the
field-IDs is not a t*field-schema.

Error object {\sl unknown-hb-error} (page~\pageref{unknown-hb-error}) if the call fails
for some other reason.
 
\item [Description:]

Creates a new node-schema with field-IDs as its
structure, and returns the corresponding node-schema-ID.

\item [Public:]



\end{description}
\horizontalline

\subsection{t*\{node-schema\}*IDs}
\label{t*node-schema*IDs}

\begin{description}
\item [Name:]  t*\{node-schema\}*IDs

\item [Class:]
{\sl t*node-schema}\hfill(page~\pageref{t*node-schema})

\item [Parameters:] none

\item [Return-value:]

A list of t*node-schema IDs.

\item [Description:]

Returns a freshly consed list of all currently
defined t*node-schema IDs. 

\item [Public:]



\end{description}
\horizontalline

\subsection{t*\{node-schema\}*mapc-IDs}
\label{t*node-schema*mapc-IDs}

\begin{description}
\item [Name:]  t*\{node-schema\}*mapc-IDs

\item [Class:] {\sl t*node-schema}\hfill(page~\pageref{t*node-schema})

\item [Parameters:]
\item {\sl map-ID-fn}:  A function that takes one argument, an ID,
and which performs some side-effect based upon that
value.



\item [Return-value:] nil

\item [Description:]

Calls map-ID-fn on each currently defined node-schema-ID.

\item [Public:]



\end{description}
\horizontalline

\section{t*node-instance}
\label{t*node-instance}

\begin{description}
\item [Name:]  t*node-instance

\item [Layer:] {\sl Type}\hfill(page~\pageref{Type})

\item [Description:]

Each instance of this class corresponds to an actual
content-bearing node in Egret.  Each node instance has
an associated schema, which represents the current
consensus in the group about the appropriate field-level
structure for this node.  This agreement may or may not
correspond to the actual field-level structure of any
particular node-instance.

Note that node-instance is abbreviated to "node" in
the operation and attribute names. 

The contents of node instances are stored in a form
termed the "packed field" representation. Packed fields
are a presentation-independent way of storing field
values. 

Fields are packed by delimiting their contents with a
marker containing the field schema ID. For example, if a
"Name" field has the field-schema-ID 37, and its value
is "Foo", then its packed representation is:

<<\#\# 37 \#\#>>Foo<<\#\# 37 \#\#>>

Note that presentational issues, such as the labelling
of the field, or its font or color are determined by
reference to the associated field-schema.

Links are packed similarly, except that the link-ID
and destination node-ID are embedded in the marker as
well as the link-schema-ID. Thus,

<<\$\$ 62 128 34 \$\$>> 

represents an occurrence of the link-ID 62, associated
with the link-schema 128, and pointing to the node-ID
34.  

The following example illustrates a typical use of links
within fields:

Description: See [-> more-info]. 

This "Description" field is packed as follows:

<<\#\# 23 \#\#>>See <<\$\$ 12 35 98 \$\$>>.<<\#\# 23 \#\#>>

Here, 23 is the field-schema-ID for the Description
field, 12 is the link-ID for the link called more-info,
35 is its link-schema-ID, and 98 is the node-ID that
more-info points to.


\item [Attributes:]
\item {\sl t*node*name}\hfill(page~\pageref{t*node*name})
\item {\sl t*node*schema-ID}\hfill(page~\pageref{t*node*schema-ID})
\item {\sl t*node*field-schema-IDs}\hfill(page~\pageref{t*node*field-schema-IDs})
\item {\sl t*node*incoming-link-IDs}\hfill(page~\pageref{t*node*incoming-link-IDs})
\item {\sl t*node*outgoing-link-IDs}\hfill(page~\pageref{t*node*outgoing-link-IDs})
\item {\sl t*node*layer-IDs}\hfill(page~\pageref{t*node*layer-IDs})

\item [Operations:]
\item {\sl t*node*set-name}\hfill(page~\pageref{t*node*set-name})
\item {\sl t*node*clone}\hfill(page~\pageref{t*node*clone})
\item {\sl t*node*delete}\hfill(page~\pageref{t*node*delete})
\item {\sl t*node*set-schema-ID}\hfill(page~\pageref{t*node*set-schema-ID})
\item {\sl t*node*add-field-schema-IDs}\hfill(page~\pageref{t*node*add-field-schema-IDs})
\item {\sl t*node*delete-field-schema-IDs}\hfill(page~\pageref{t*node*delete-field-schema-IDs})
\item {\sl t*node*add-layer-ID}\hfill(page~\pageref{t*node*add-layer-ID})
\item {\sl t*node*delete-layer-ID}\hfill(page~\pageref{t*node*delete-layer-ID})
\item {\sl t*node*field-values}\hfill(page~\pageref{t*node*field-values})
\item {\sl t*node*set-field-values}\hfill(page~\pageref{t*node*set-field-values})
\item {\sl t*node*lock}\hfill(page~\pageref{t*node*lock})
\item {\sl t*node*unlock}\hfill(page~\pageref{t*node*unlock})
\item {\sl t*node*convergence}\hfill(page~\pageref{t*node*convergence})
 
\item {\sl t*\{node\}*IDs}\hfill(page~\pageref{t*node*IDs})
\item {\sl t*\{node\}*mapc-IDs}\hfill(page~\pageref{t*node*mapc-IDs})

\item [Subclasses:]


\item [Superclasses:]


\item [Instances:]



\end{description}
\horizontalline

\subsection{t*node*name}
\label{t*node*name}

\begin{description}
\item [Name:]  t*node*name

\item [Class:] {\sl t*node-instance}\hfill(page~\pageref{t*node-instance})

\item [Contents:] string 

\item [Description:]

The name of this node. 

\item [Setf-able:] See t*node*set-name


\item [Public:]



\end{description}
\horizontalline

\subsection{t*node*schema-ID}
\label{t*node*schema-ID}

\begin{description}
\item [Name:]  t*node*schema-ID

\item [Class:] {\sl t*node-instance}\hfill(page~\pageref{t*node-instance})

\item [Contents:] a node-schema-ID

\item [Description:]

The consensual schema for this node.

\item [Setf-able:]


\item [Public:]



\end{description}
\horizontalline

\subsection{t*node*field-schema-IDs}
\label{t*node*field-schema-IDs}

\begin{description}
\item [Name:]  t*node*field-schema-IDs

\item [Class:] {\sl t*node-instance}\hfill(page~\pageref{t*node-instance})

\item [Contents:] a list of field-schema-IDs

\item [Description:]

The internal structure of the node instance. 

\item [Setf-able:]


\item [Public:]



\end{description}
\horizontalline

\subsection{t*node*incoming-link-IDs}
\label{t*node*incoming-link-IDs}

\begin{description}
\item [Name:]  t*node*incoming-link-IDs

\item [Class:] {\sl t*node-instance}\hfill(page~\pageref{t*node-instance})

\item [Contents:] a list of link-IDs

\item [Description:]

A list of the link-IDs pointing to this node.

\item [Setf-able:]


\item [Public:]



\end{description}
\horizontalline

\subsection{t*node*outgoing-link-IDs}
\label{t*node*outgoing-link-IDs}

\begin{description}
\item [Name:]  t*node*outgoing-link-IDs

\item [Class:] {\sl t*node-instance}\hfill(page~\pageref{t*node-instance})

\item [Contents:] a list of link-IDs

\item [Description:]

A list of the link-IDs outgoing from this node.

\item [Setf-able:]


\item [Public:]



\end{description}
\horizontalline

\subsection{t*node*layer-IDs}
\label{t*node*layer-IDs}

\begin{description}
\item [Name:]  t*node*layer-IDs

\item [Class:] {\sl t*node-instance}\hfill(page~\pageref{t*node-instance})

\item [Contents:] a list of layer-IDs

\item [Description:]

The list of layers to which this node instance 
currently belongs. 

\item [Setf-able:]


\item [Public:]



\end{description}
\horizontalline

\subsection{t*node*set-name}
\label{t*node*set-name}

\begin{description}
\item [Name:]  t*node*set-name

\item [Class:] {\sl t*node-instance}\hfill(page~\pageref{t*node-instance})

\item [Parameters:]
\item {\sl node-ID}:   An integer representing
a valid hbserver node ID.

\item {\sl node-name}:  
A valid node name. This currently means that it is a
string of less than 40 characters, and that it does
not contain leading space(s) or tabs.


\item [Return-value:]
The new node-name if successful.

Error object {\sl invalid-node-name} (page~\pageref{invalid-node-name}) if node-name not valid.

Error object {\sl unknown-hb-error} (page~\pageref{unknown-hb-error}) if call fails
for any other reason.

\item [Description:]

Updates the name of node-ID to node-name.

\item [Public:]



\end{description}
\horizontalline

\subsection{t*node*clone}
\label{t*node*clone}

\begin{description}
\item [Name:]  t*node*clone

\item [Class:] {\sl t*node-instance}\hfill(page~\pageref{t*node-instance})

\item [Parameters:]
\item {\sl node-ID}:   An integer representing
a valid hbserver node ID.


\item [Return-value:]
A new node-ID corresponding to the cloned instance 
if successful.

Error object {\sl invalid-node-ID} (page~\pageref{invalid-node-ID}) if node-ID is not
an instance of t*node-instance.

Error object {\sl unknown-hb-error} (page~\pageref{unknown-hb-error}) if the call fails
for some other reason. 

\item [Description:]

Creates and initializes a new t*node-instance with 
the node-schema-ID and field-schema-ID structure of 
node-ID. However, it does not copy the contents of 
node-ID.

\item [Public:]



\end{description}
\horizontalline

\subsection{t*node*delete}
\label{t*node*delete}

\begin{description}
\item [Name:]  t*node*delete

\item [Class:] {\sl t*node-instance}\hfill(page~\pageref{t*node-instance})

\item [Parameters:]
\item {\sl node-ID}:   An integer representing
a valid hbserver node ID.


\item [Return-value:]
node-ID if successfully deleted.

Error object {\sl invalid-node-ID} (page~\pageref{invalid-node-ID}) if node-ID is 
not an instance of t*node-instance.

Error object {\sl node-still-locked} (page~\pageref{node-still-locked}) if node is locked
by some other user. 

Error object {\sl node-still-referenced} (page~\pageref{node-still-referenced}) if node
has incoming links. 

Error object {\sl unknown-hb-error} (page~\pageref{unknown-hb-error}) if call fails
for some other reason.

\item [Description:]

Deletes node-ID from the hyperbase.

\item [Public:]



\end{description}
\horizontalline

\subsection{t*node*set-schema-ID}
\label{t*node*set-schema-ID}

\begin{description}
\item [Name:]  t*node*set-schema-ID

\item [Class:] {\sl t*node-instance}\hfill(page~\pageref{t*node-instance})

\item [Parameters:]
\item {\sl node-ID}:   An integer representing
a valid hbserver node ID.

\item {\sl node-schema-ID}:  a server-level node-ID that corresponds to an 
instance of a type-level node-schema. 



\item [Return-value:]
The new node-schema-ID if successful.

Error object {\sl invalid-node-ID} (page~\pageref{invalid-node-ID}) if bad node-ID.

Error object {\sl invalid-node-schema-ID} (page~\pageref{invalid-node-schema-ID}) if bad
node-schema-ID.

Error object {\sl lock-node-fails} (page~\pageref{lock-node-fails}) if a lock can't be
obtained in preparation for the update. 

Error object {\sl unknown-hb-error} (page~\pageref{unknown-hb-error}) if call fails for
some other reason.

\item [Description:]

Sets the schema of node-ID to node-schema-ID. Note
that this does not change the existing structure of
node-ID at all. (It does potentially change the
convergence and divergence metric values for this
node schema and instance.)


\item [Public:]






\end{description}
\horizontalline

\subsection{t*node*add-field-schema-IDs}
\label{t*node*add-field-schema-IDs}

\begin{description}
\item [Name:]  t*node*add-field-schema-IDs

\item [Class:] {\sl t*node-instance}\hfill(page~\pageref{t*node-instance})

\item [Parameters:]
\item {\sl node-ID}:   An integer representing
a valid hbserver node ID.

\item {\sl field-schema-IDs}:  list of field-schema-ID


\item [Return-value:]
The updated list of field-schema-IDs if successful.

Error object {\sl invalid-node-ID} (page~\pageref{invalid-node-ID}) if node-ID is not an
instance of t*node-instance.

Error object {\sl invalid-field-ID} (page~\pageref{invalid-field-ID}) if any of 
field-schema-IDs are invalid or are already members of 
node-ID.

Error object {\sl lock-node-fails} (page~\pageref{lock-node-fails}) if a lock on node-ID
cannot be obtained in preparation for the update.

Error object {\sl unknown-hb-error} (page~\pageref{unknown-hb-error}) if the call fails
for any other reason. 

\item [Description:]

Adds the list of field-schema-IDs to the structure of
node-ID.

\item [Public:]



\end{description}
\horizontalline

\subsection{t*node*delete-field-schema-IDs}
\label{t*node*delete-field-schema-IDs}

\begin{description}
\item [Name:]  t*node*delete-field-schema-IDs

\item [Class:] {\sl t*node-instance}\hfill(page~\pageref{t*node-instance})

\item [Parameters:]
\item {\sl node-ID}:   An integer representing
a valid hbserver node ID.

\item {\sl field-schema-IDs}:  list of field-schema-ID


\item [Return-value:]
The updated list of field-schema-IDs if successful.

Error object {\sl invalid-node-ID} (page~\pageref{invalid-node-ID}) if node-ID is not
a t*node-instance.

Error object {\sl invalid-field-ID} (page~\pageref{invalid-field-ID}) if any of the 
field-schema-IDs are not legal members of this
node-ID.

Error object {\sl lock-node-fails} (page~\pageref{lock-node-fails}) if node-ID cannot
be locked in preparation for the deletion. 

Error object {\sl unknown-hb-error} (page~\pageref{unknown-hb-error}) if call fails
for any other reason.

\item [Description:]

Deletes fields from an individual node instance. 

\item [Public:]



\end{description}
\horizontalline

\subsection{t*node*add-layer-ID}
\label{t*node*add-layer-ID}

\begin{description}
\item [Name:]  t*node*add-layer-ID

\item [Class:] {\sl t*node-instance}\hfill(page~\pageref{t*node-instance})

\item [Parameters:]
\item {\sl node-ID}:   An integer representing
a valid hbserver node ID.

\item {\sl layer-ID}:  a unique ID for layers (possibly a node-ID?)



\item [Return-value:]
Layer-ID if successful.

Error object {\sl invalid-node-ID} (page~\pageref{invalid-node-ID}) if node-ID is not
a t*node-instance.

Error object {\sl invalid-layer-ID} (page~\pageref{invalid-layer-ID}) if layer-ID is
not the ID of a layer instance, or if it already
contains this node-ID as a member.

Error object {\sl lock-node-fails} (page~\pageref{lock-node-fails}) if a lock on
node-ID cannot be obtained in preparation for this
update.

Error object {\sl unknown-hb-error} (page~\pageref{unknown-hb-error}) if call fails
for any other reason.

\item [Description:]

Makes node-ID a member of layer-ID.

\item [Public:]



\end{description}
\horizontalline

\subsection{t*node*delete-layer-ID}
\label{t*node*delete-layer-ID}

\begin{description}
\item [Name:]  t*node*delete-layer-ID

\item [Class:] {\sl t*node-instance}\hfill(page~\pageref{t*node-instance})

\item [Parameters:]
\item {\sl node-ID}:   An integer representing
a valid hbserver node ID.

\item {\sl layer-ID}:  a unique ID for layers (possibly a node-ID?)



\item [Return-value:]
The deleted layer-ID if successful.

Error object {\sl invalid-node-ID} (page~\pageref{invalid-node-ID}) if node-ID is
not a t*node-instance.

Error object {\sl invalid-layer-ID} (page~\pageref{invalid-layer-ID}) if bad layer-ID,
or if node-ID is not currently a member of layer-ID.

Error object {\sl lock-node-fails} (page~\pageref{lock-node-fails}) if a lock cannot
be obtained on node-ID (and perhaps layer-ID?).

Error object {\sl unknown-hb-error} (page~\pageref{unknown-hb-error}) if call fails
for any other reason.


\item [Description:]
Deletes this instance from the associated layer.

\item [Public:]



\end{description}
\horizontalline

\subsection{t*node*field-values}
\label{t*node*field-values}

\begin{description}
\item [Name:]  t*node*field-values

\item [Class:] {\sl t*node-instance}\hfill(page~\pageref{t*node-instance})

\item [Parameters:]
\item {\sl node-ID}:   An integer representing
a valid hbserver node ID.

\item {\sl field-schema-IDs}:  list of field-schema-ID

\item {\sl buffer-instance}:  an Emacs buffer instance object


\item [Return-value:]
Returns a buffer-instance containing a set of 
packed fields corresponding to field-schema-IDs
if successful.

Error object {\sl invalid-buffer-instance} (page~\pageref{invalid-buffer-instance}) if 
buffer-instance argument is supplied and not a 
legal buffer-instance.

Error object {\sl invalid-node-ID} (page~\pageref{invalid-node-ID}) if node-ID is
not a t*node-instance.

Error object {\sl invalid-field-ID} (page~\pageref{invalid-field-ID}) if
field-schema-IDs is supplied and any of the 
them are not t*field-schemas or not present in
node-ID. 

Error object {\sl unknown-hb-error} (page~\pageref{unknown-hb-error}) if call fails
for any other reason.

\item [Description:]

Calls the hyperbase and returns a buffer-instance
containing the packed field representation for the
fields corresponding to field-schema-IDs in node-ID.

Field-schema-IDs is an optional argument, and 
defaults to the field-schema-IDs in the data field
of the hyperbase node corresponding to node-ID.

Buffer-instance, if supplied, is a buffer-instance
that can be erased and returned with the requested
packed fields.  If not supplied, a new
buffer-instance is created and returned.

\item [Public:]



\end{description}
\horizontalline

\subsection{t*node*set-field-values}
\label{t*node*set-field-values}

\begin{description}
\item [Name:]  t*node*set-field-values

\item [Class:] {\sl t*node-instance}\hfill(page~\pageref{t*node-instance})

\item [Parameters:]
\item {\sl node-ID}:   An integer representing
a valid hbserver node ID.

\item {\sl field-schema-IDs}:  list of field-schema-ID

\item {\sl buffer-instance}:  an Emacs buffer instance object


\item [Return-value:]
T if node-ID is successfully updated with new
field values.

Error object {\sl invalid-node-ID} (page~\pageref{invalid-node-ID}) if node-ID
is not a t*node-instance.

Error object {\sl invalid-field-ID} (page~\pageref{invalid-field-ID}) if argument
field-schema-IDs is specified and if any of them
are not t*field-schemas or not present in node-ID.

Error object {\sl invalid-buffer-instance} (page~\pageref{invalid-buffer-instance}) if the 
buffer-instance argument is not a buffer instance,
or does not contain the fields specified in 
field-schema-IDs.

Error object {\sl lock-node-fails} (page~\pageref{lock-node-fails}) if lock cannot 
be obtained on node-ID in preparation for the update.

Error object {\sl unknown-hb-error} (page~\pageref{unknown-hb-error}) if call fails
for some other reason.

\item [Description:]
Updates the contents of node-ID with new field
values.

The optional parameter field-schema-IDs, if supplied,
indicates which fields in node-ID should be updated
from the set of packed fields contained in
buffer-instance.  Buffer-instance must contain the
packed representation of these fields, or an error is
signalled. 

If field-schema-IDs is not supplied, then
buffer-instance is assumed to contain the packed
field representation for all fields in node-ID (that
are stored in the hbserver data field).

Buffer-instance must contain exactly the packed field
representations for the field-schema-IDs specified by
the field-schema-ID argument (or implicitly by its
absence.)


\item [Public:]



\end{description}
\horizontalline

\subsection{t*node*lock}
\label{t*node*lock}

\begin{description}
\item [Name:]  t*node*lock

\item [Class:] {\sl t*node-instance}\hfill(page~\pageref{t*node-instance})

\item [Parameters:]
\item {\sl node-ID}:   An integer representing
a valid hbserver node ID.


\item [Return-value:]
T if the lock can be successfully obtained, or if
user already has a lock on node-ID.

NIL if another user has locked node-ID.

Error object {\sl invalid-node-ID} (page~\pageref{invalid-node-ID}) if node-ID is
not a t*node-instance.

Error object {\sl unknown-hb-error} (page~\pageref{unknown-hb-error}) if call fails
for some other reason.

\item [Description:]

Attempts to obtain a lock on node-ID. 

\item [Public:]



\end{description}
\horizontalline

\subsection{t*node*unlock}
\label{t*node*unlock}

\begin{description}
\item [Name:]  t*node*unlock

\item [Class:] {\sl t*node-instance}\hfill(page~\pageref{t*node-instance})

\item [Parameters:]
\item {\sl node-ID}:   An integer representing
a valid hbserver node ID.


\item [Return-value:] 
T if node-ID is successfully unlocked.

Error object {\sl invalid-node-ID} (page~\pageref{invalid-node-ID}) if node-ID is
not a t*node-instance.

Error object {\sl unknown-hb-error} (page~\pageref{unknown-hb-error}) if the call
fails for some other reason.

\item [Description:]

Releases the lock (if present) on node-ID by user.

Will succeed when node-ID is already unlocked.

Will fail if node-ID is locked by another user.


\item [Public:]



\end{description}
\horizontalline

\subsection{t*node*convergence}
\label{t*node*convergence}

\begin{description}
\item [Name:]  t*node*convergence

\item [Class:] {\sl t*node-instance}\hfill(page~\pageref{t*node-instance})

\item [Parameters:]
\item {\sl node-ID}:   An integer representing
a valid hbserver node ID.


\item [Return-value:] 
Integer convergence value if successful.

Error object {\sl invalid-node-ID} (page~\pageref{invalid-node-ID}) if node-ID is not a 
valid t*node-instance.

Error object {\sl unknown-hb-error} (page~\pageref{unknown-hb-error}) if the call fails
for some other reason.

\item [Description:]

Computes and returns an integer value representing 
the degree of convergence between the node
instance and its schema.

\item [Public:]



\end{description}
\horizontalline

\subsection{t*\{node\}*IDs}
\label{t*node*IDs}

\begin{description}
\item [Name:]  t*\{node\}*IDs

\item [Class:] {\sl t*node-instance}\hfill(page~\pageref{t*node-instance})

\item [Parameters:] none

\item [Return-value:] A list of node-IDs

\item [Description:]

Returns a freshly consed list of all the currently
defined type-level node-instances.  

Note that this operation may be expensive. 

\item [Public:]



\end{description}
\horizontalline

\subsection{t*\{node\}*mapc-IDs}
\label{t*node*mapc-IDs}

\begin{description}
\item [Name:]  t*\{node\}*mapc-IDs

\item [Class:] {\sl t*node-instance}\hfill(page~\pageref{t*node-instance})

\item [Parameters:]
\item {\sl map-ID-fn}:  A function that takes one argument, an ID,
and which performs some side-effect based upon that
value.



\item [Return-value:] nil

\item [Description:]

Calls map-ID-fn once with a node-ID corresponding
to each t*node-instance. 

\item [Public:]



\end{description}
\horizontalline

\section{t*link-schema}
\label{t*link-schema}

\begin{description}
\item [Name:]  t*link-schema

\item [Layer:] {\sl Type}\hfill(page~\pageref{Type})

\item [Description:]

This class defines the structural properties of link
types.

\item [Attributes:]
\item {\sl t*link-schema*name}\hfill(page~\pageref{t*link-schema*name})
\item {\sl t*link-schema*to-nodes}\hfill(page~\pageref{t*link-schema*to-nodes})
\item {\sl t*link-schema*from-nodes}\hfill(page~\pageref{t*link-schema*from-nodes})
\item {\sl t*link-schema*link-IDs}\hfill(page~\pageref{t*link-schema*link-IDs})

\item [Operations:]
\item {\sl t*link-schema*make}\hfill(page~\pageref{t*link-schema*make})
\item {\sl t*link-schema*instantiate}\hfill(page~\pageref{t*link-schema*instantiate})
\item {\sl t*link-schema*add-to-node}\hfill(page~\pageref{t*link-schema*add-to-node})
\item {\sl t*link-schema*add-from-node}\hfill(page~\pageref{t*link-schema*add-from-node})
\item {\sl t*link-schema*delete-to-node}\hfill(page~\pageref{t*link-schema*delete-to-node})
\item {\sl t*link-schema*delete-from-node}\hfill(page~\pageref{t*link-schema*delete-from-node})
\item {\sl t*link-schema*set-name}\hfill(page~\pageref{t*link-schema*set-name})
\item {\sl t*link-schema*divergence}\hfill(page~\pageref{t*link-schema*divergence})
\item {\sl t*link-schema*delete}\hfill(page~\pageref{t*link-schema*delete})

\item {\sl t*\{link-schema\}*IDs}\hfill(page~\pageref{t*link-schema*IDs})
\item {\sl t*\{link-schema\}*mapc-IDs}\hfill(page~\pageref{t*link-schema*mapc-IDs})

\item [Subclasses:]


\item [Superclasses:]


\item [Instances:]



\end{description}
\horizontalline

\subsection{t*link-schema*name}
\label{t*link-schema*name}

\begin{description}
\item [Name:]  t*link-schema*name

\item [Class:] {\sl t*link-schema}\hfill(page~\pageref{t*link-schema})

\item [Contents:] a symbol

\item [Description:]

The name of the link-schema.

\item [Setf-able:]


\item [Public:]



\end{description}
\horizontalline

\subsection{t*link-schema*to-nodes}
\label{t*link-schema*to-nodes}

\begin{description}
\item [Name:]  t*link-schema*to-nodes

\item [Class:] {\sl t*link-schema}\hfill(page~\pageref{t*link-schema})

\item [Contents:] A link constraint expression.

\item [Description:]

An expression of the form \{T |(<node-schema-ID>*)\}.
T indicates that any node-schema-ID is valid; otherwise,
the list indicates the set of legal node-schema-IDs.

\item [Setf-able:]


\item [Public:]



\end{description}
\horizontalline

\subsection{t*link-schema*from-nodes}
\label{t*link-schema*from-nodes}

\begin{description}
\item [Name:]  t*link-schema*from-nodes

\item [Class:] {\sl t*link-schema}\hfill(page~\pageref{t*link-schema})

\item [Contents:] A link constraint expression

\item [Description:]

An expression of the form \{T |(<node-schema-ID>*)\}.
T indicates that any node-schema-ID is valid; otherwise,
the list indicates the set of legal node-schema-IDs.

\item [Setf-able:]


\item [Public:]



\end{description}
\horizontalline

\subsection{t*link-schema*link-IDs}
\label{t*link-schema*link-IDs}

\begin{description}
\item [Name:]  t*link-schema*link-IDs

\item [Class:] {\sl t*link-schema}\hfill(page~\pageref{t*link-schema})

\item [Contents:] A list of link-IDs

\item [Description:]

The current links of this type.

\item [Setf-able:]


\item [Public:]



\end{description}
\horizontalline

\subsection{t*link-schema*make}
\label{t*link-schema*make}

\begin{description}
\item [Name:]  t*link-schema*make

\item [Class:] {\sl t*link-schema}\hfill(page~\pageref{t*link-schema})

\item [Parameters:]
\item {\sl link-name}:  string (30); a valid link name

\item {\sl to-constraint-exp}:  A link constraint expression


\item {\sl from-constraint-exp}:  A link constraint expression.



\item [Return-value:]

Returns a new link-schema-ID with 
associated link-name and to and from node constraints. 

Error object {\sl invalid-constraint-exp} (page~\pageref{invalid-constraint-exp}) if the to
or from constraint expressions are illegal.

Error object {\sl unknown-hb-error} (page~\pageref{unknown-hb-error}) if call fails for
any other reason. 


\item [Description:]


\item [Public:]



\end{description}
\horizontalline

\subsection{t*link-schema*instantiate}
\label{t*link-schema*instantiate}

\begin{description}
\item [Name:]  t*link-schema*instantiate

\item [Class:] {\sl t*link-schema}\hfill(page~\pageref{t*link-schema})

\item [Parameters:]
\item {\sl link-schema-ID}:  an ID for a link schema.

\item {\sl link-name}:  string (30); a valid link name

\item {\sl to-node-ID}:  node-ID


\item {\sl from-node-ID}:  node-ID


\item [Return-value:]
The newly created link-ID if successful.

Error object {\sl invalid-link-schema-ID} (page~\pageref{invalid-link-schema-ID}) if not
a legal link-schema ID.

Error object {\sl invalid-link-name} (page~\pageref{invalid-link-name}) if not a 
legal link name.

Error object {\sl invalid-node-ID} (page~\pageref{invalid-node-ID}) if either of 
the node-IDs are invalid or violate this link schema's
constraints. 

Error object {\sl unknown-hb-error} (page~\pageref{unknown-hb-error}) if call fails for
any other reason.

\item [Description:]

Creates a new link instance conforming to the constraints
specified for link-schema-ID. 

\item [Public:]



\end{description}
\horizontalline

\subsection{t*link-schema*add-to-node}
\label{t*link-schema*add-to-node}

\begin{description}
\item [Name:]  t*link-schema*add-to-node

\item [Class:] {\sl t*link-schema}\hfill(page~\pageref{t*link-schema})

\item [Parameters:]
\item {\sl link-schema-ID}:  an ID for a link schema.

\item {\sl node-schema-ID}:  a server-level node-ID that corresponds to an 
instance of a type-level node-schema. 



\item [Return-value:]
The updated to-node constraint expression if successful.

Error object {\sl invalid-node-schema-ID} (page~\pageref{invalid-node-schema-ID}) if second
arg is not a t*node-schema.

Error object {\sl invalid-link-schema-ID} (page~\pageref{invalid-link-schema-ID}) if first
arg is not a t*link-schema.

Error object {\sl unknown-hb-error} (page~\pageref{unknown-hb-error}) if call fails for 
any other reason.

\item [Description:]

Adds node-schema-ID to the list of to-node constraints,
and returns the new constraint expression.

If node-schema-ID is the symbol T, then the constraint
expression is set to T.

\item [Public:]



\end{description}
\horizontalline

\subsection{t*link-schema*add-from-node}
\label{t*link-schema*add-from-node}

\begin{description}
\item [Name:]  t*link-schema*add-from-node

\item [Class:] {\sl t*link-schema}\hfill(page~\pageref{t*link-schema})

\item [Parameters:]
\item {\sl link-schema-ID}:  an ID for a link schema.

\item {\sl node-schema-ID}:  a server-level node-ID that corresponds to an 
instance of a type-level node-schema. 



\item [Return-value:]
Returns the updated constraint expression if successful.

Error object {\sl invalid-link-schema-ID} (page~\pageref{invalid-link-schema-ID}) if first
arg is not a t*link-schema.

Error object {\sl invalid-node-schema-ID} (page~\pageref{invalid-node-schema-ID}) if the 
second arg is not a t*node-schema, or the distinguished
symbol T.

Error object {\sl unknown-hb-error} (page~\pageref{unknown-hb-error}) if the call fails
for any other reason. 

\item [Description:]

Adds node-schema-ID to the list of acceptable node
types for this from-node constraint expression, or 
sets the constraint expression to T if that symbol
is supplied.

\item [Public:]



\end{description}
\horizontalline

\subsection{t*link-schema*delete-to-node}
\label{t*link-schema*delete-to-node}

\begin{description}
\item [Name:]  t*link-schema*delete-to-node

\item [Class:] {\sl t*link-schema}\hfill(page~\pageref{t*link-schema})

\item [Parameters:]
\item {\sl link-schema-ID}:  an ID for a link schema.

\item {\sl node-schema-ID}:  a server-level node-ID that corresponds to an 
instance of a type-level node-schema. 



\item [Return-value:]
Returns the updated constraint expression if successful.
If the current to-node constraint expression is T, 
then this value is replaced by the current value
of t*\{node-schema\}*IDs before performing the deletion.

Error object {\sl invalid-link-schema-ID} (page~\pageref{invalid-link-schema-ID}) if first
arg is not a t*link-schema.

Error object {\sl invalid-node-schema-ID} (page~\pageref{invalid-node-schema-ID}) if second
arg is not a t*node-schema in the constraint 
expression for t*link-schema's to-nodes, or the 
distinguished symbol NIL. 

Error object {\sl unknown-hb-error} (page~\pageref{unknown-hb-error}) if call fails
for any other reason.

\item [Description:]

Deletes node-schema-ID from the to-node constraint
expression. If node-schema-ID is NIL, then the 
constraint expression is set to NIL.

\item [Public:]



\end{description}
\horizontalline

\subsection{t*link-schema*delete-from-node}
\label{t*link-schema*delete-from-node}

\begin{description}
\item [Name:]  t*link-schema*delete-from-node

\item [Class:] {\sl t*link-schema}\hfill(page~\pageref{t*link-schema})

\item [Parameters:]
\item {\sl link-schema-ID}:  an ID for a link schema.

\item {\sl node-schema-ID}:  a server-level node-ID that corresponds to an 
instance of a type-level node-schema. 



\item [Return-value:]
The updated constraint expression if successful.
If the current from-node constraint expression is T, 
then this value is replaced by the current value
of t*\{node-schema\}*IDs before performing the deletion.

Error object {\sl invalid-link-schema-ID} (page~\pageref{invalid-link-schema-ID}) if first
arg is not a t*link-schema.

Error object {\sl invalid-node-schema-ID} (page~\pageref{invalid-node-schema-ID}) if second
argument is not a node-schema-ID currently present
in the constraint expression, or the distinguished
value NIL.

Error object {\sl unknown-hb-error} (page~\pageref{unknown-hb-error}) if call fails
for any other reason.

\item [Description:]

Deletes node-schema-ID from the constraint expression
for link from-nodes.  If NIL is supplied rather than
a node-schema-ID, then the constraint expression
is set to NIL.


\item [Public:]



\end{description}
\horizontalline

\subsection{t*link-schema*set-name}
\label{t*link-schema*set-name}

\begin{description}
\item [Name:]  t*link-schema*set-name

\item [Class:] {\sl t*link-schema}\hfill(page~\pageref{t*link-schema})

\item [Parameters:]
\item {\sl link-schema-ID}:  an ID for a link schema.

\item {\sl link-name}:  string (30); a valid link name



\item [Return-value:]
T if successful.

Error object {\sl invalid-link-name} (page~\pageref{invalid-link-name}) if illegal link-name.

Error object {\sl invalid-link-schema-ID} (page~\pageref{invalid-link-schema-ID}) if first
argument is not a t*link-schema.

Error object {\sl unknown-hb-error} (page~\pageref{unknown-hb-error}) if call fails for
any other reason.

\item [Description:]

Changes the name of link-schema-ID to link-name.

\item [Public:]



\end{description}
\horizontalline

\subsection{t*link-schema*divergence}
\label{t*link-schema*divergence}

\begin{description}
\item [Name:]  t*link-schema*divergence

\item [Class:] {\sl t*link-schema}\hfill(page~\pageref{t*link-schema})

\item [Parameters:]
\item {\sl link-schema-ID}:  an ID for a link schema.


\item [Return-value:]
An integer divergence value if successful.

Error object {\sl unknown-hb-error} (page~\pageref{unknown-hb-error}) if call fails
for any reason.

\item [Description:]

Computes the divergence metric for link-schema 
instances.

\item [Public:]



\end{description}
\horizontalline

\subsection{t*link-schema*delete}
\label{t*link-schema*delete}

\begin{description}
\item [Name:]  t*link-schema*delete

\item [Class:] {\sl t*link-schema}\hfill(page~\pageref{t*link-schema})

\item [Parameters:]
\item {\sl link-schema-ID}:  an ID for a link schema.


\item [Return-value:]
The deleted link-schema-ID if successful.

Error object {\sl invalid-link-schema-ID} (page~\pageref{invalid-link-schema-ID}) if argument
is not a link-schema-ID.

Error object {\sl unknown-hb-error} (page~\pageref{unknown-hb-error}) if call fails
for some other reason.

\item [Description:]

Deletes t*link-schema-ID.  Note that this does not
physically remove link-schema-ID from the database.
Rather, it simply prevents any new instances of 
link-schema-ID from being created.

\item [Public:]



\end{description}
\horizontalline

\subsection{t*\{link-schema\}*IDs}
\label{t*link-schema*IDs}

\begin{description}
\item [Name:]  t*\{link-schema\}*IDs

\item [Class:] {\sl t*link-schema}\hfill(page~\pageref{t*link-schema})

\item [Parameters:] none

\item [Return-value:]

A list of link-schema-IDs.

\item [Description:]

Returns a freshly consed list of all currently 
defined link-schema IDs.

\item [Public:]



\end{description}
\horizontalline

\subsection{t*\{link-schema\}*mapc-IDs}
\label{t*link-schema*mapc-IDs}

\begin{description}
\item [Name:]  t*\{link-schema\}*mapc-IDs

\item [Class:] {\sl t*link-schema}\hfill(page~\pageref{t*link-schema})

\item [Parameters:]
\item {\sl map-ID-fn}:  A function that takes one argument, an ID,
and which performs some side-effect based upon that
value.



\item [Return-value:] nil

\item [Description:]

Calls map-ID-fn on each currently defined link-schema ID.

\item [Public:]



\end{description}
\horizontalline

\section{t*link-instance}
\label{t*link-instance}

\begin{description}
\item [Name:]  t*link-instance

\item [Layer:] {\sl Type}\hfill(page~\pageref{Type})

\item [Description:]

The instances of this class define links between nodes. 

\item [Attributes:]
\item {\sl t*link*name}\hfill(page~\pageref{t*link*name})
\item {\sl t*link*schema-ID}\hfill(page~\pageref{t*link*schema-ID})
\item {\sl t*link*to-node-ID}\hfill(page~\pageref{t*link*to-node-ID})
\item {\sl t*link*from-node-ID}\hfill(page~\pageref{t*link*from-node-ID})
\item {\sl t*link*layer-IDs}\hfill(page~\pageref{t*link*layer-IDs})

\item [Operations:]
\item {\sl t*link*set-name}\hfill(page~\pageref{t*link*set-name})
\item {\sl t*link*clone}\hfill(page~\pageref{t*link*clone})
\item {\sl t*link*delete}\hfill(page~\pageref{t*link*delete})
\item {\sl t*link*set-schema-ID}\hfill(page~\pageref{t*link*set-schema-ID})
\item {\sl t*link*set-to-node}\hfill(page~\pageref{t*link*set-to-node})
\item {\sl t*link*add-to-constraint}\hfill(page~\pageref{t*link*add-to-constraint})
\item {\sl t*link*add-from-constraint}\hfill(page~\pageref{t*link*add-from-constraint})
\item {\sl t*link*delete-to-constraint}\hfill(page~\pageref{t*link*delete-to-constraint})
\item {\sl t*link*delete-from-constraint}\hfill(page~\pageref{t*link*delete-from-constraint})
\item {\sl t*link*add-layer-ID}\hfill(page~\pageref{t*link*add-layer-ID})
\item {\sl t*link*delete-layer-ID}\hfill(page~\pageref{t*link*delete-layer-ID})
\item {\sl t*link*convergence}\hfill(page~\pageref{t*link*convergence})

\item {\sl t*\{link\}*IDs}\hfill(page~\pageref{t*link*IDs})
\item {\sl t*\{link\}*mapc-IDs}\hfill(page~\pageref{t*link*mapc-IDs})

\item [Subclasses:]


\item [Superclasses:]


\item [Instances:]



\end{description}
\horizontalline

\subsection{t*link*name}
\label{t*link*name}

\begin{description}
\item [Name:]  t*link*name

\item [Class:] {\sl t*link-instance}\hfill(page~\pageref{t*link-instance})

\item [Contents:]
String.

\item [Description:]

The name of this link.

\item [Setf-able:] See t*link*set-name.


\item [Public:]



\end{description}
\horizontalline

\subsection{t*link*schema-ID}
\label{t*link*schema-ID}

\begin{description}
\item [Name:]  t*link*schema-ID

\item [Class:] {\sl t*link-instance}\hfill(page~\pageref{t*link-instance})

\item [Contents:]
A link-schema-ID

\item [Description:]

The link schema from this link instance.

\item [Setf-able:]


\item [Public:]



\end{description}
\horizontalline

\subsection{t*link*to-node-ID}
\label{t*link*to-node-ID}

\begin{description}
\item [Name:]  t*link*to-node-ID

\item [Class:] {\sl t*link-instance}\hfill(page~\pageref{t*link-instance})

\item [Contents:]
A node-ID.

\item [Description:]

The node-ID to which this link instance points. 

\item [Setf-able:]


\item [Public:]



\end{description}
\horizontalline

\subsection{t*link*from-node-ID}
\label{t*link*from-node-ID}

\begin{description}
\item [Name:]  t*link*from-node-ID

\item [Class:] {\sl t*link-instance}\hfill(page~\pageref{t*link-instance})

\item [Contents:] 
A node-ID

\item [Description:]

The node-ID from which this link originates.

\item [Setf-able:]


\item [Public:]



\end{description}
\horizontalline

\subsection{t*link*layer-IDs}
\label{t*link*layer-IDs}

\begin{description}
\item [Name:]  t*link*layer-IDs

\item [Class:] {\sl t*link-instance}\hfill(page~\pageref{t*link-instance})

\item [Contents:] A list of layer-IDs

\item [Description:]

The list of layer-IDs to which this link instance belongs.

\item [Setf-able:]


\item [Public:]



\end{description}
\horizontalline

\subsection{t*link*set-name}
\label{t*link*set-name}

\begin{description}
\item [Name:]  t*link*set-name

\item [Class:] {\sl t*link-instance}\hfill(page~\pageref{t*link-instance})

\item [Parameters:]
\item {\sl link-ID}:  
valid HB link ID number (integer)

\item {\sl link-name}:  string (30); a valid link name


\item [Return-value:]

Returns the updated link name if successful.

Error object {\sl invalid-link-ID} (page~\pageref{invalid-link-ID}) if not a 
t*link. 

Error object {\sl invalid-link-name} (page~\pageref{invalid-link-name}) if not a 
legal link name.

Error object {\sl unknown-hb-error} (page~\pageref{unknown-hb-error}) if call fails
for any other reason.

\item [Description:]

Updates the name of link-ID.

\item [Public:]



\end{description}
\horizontalline

\subsection{t*link*clone}
\label{t*link*clone}

\begin{description}
\item [Name:]  t*link*clone   

\item [Class:] {\sl t*link-instance}\hfill(page~\pageref{t*link-instance})

\item [Parameters:]
\item {\sl link-ID}:  
valid HB link ID number (integer)

\item {\sl to-node-ID}:  node-ID


\item {\sl from-node-ID}:  node-ID


\item [Return-value:]
The newly created link-ID with the same schema, 
to, and from nodes as link-ID.

\item [Description:]

Analogously to t*node*clone, this operation is
intended to clone link instances. However, the 
semantics of this operation are not yet clear.

Its implementation is temporarily deferred.

\item [Public:]



\end{description}
\horizontalline

\subsection{t*link*delete}
\label{t*link*delete}

\begin{description}
\item [Name:]  t*link*delete

\item [Class:] {\sl t*link-instance}\hfill(page~\pageref{t*link-instance})

\item [Parameters:]
\item {\sl link-ID}:  
valid HB link ID number (integer)


\item [Return-value:]
The deleted link-ID if successful.

Error object {\sl invalid-link-ID} (page~\pageref{invalid-link-ID}) if not a t*link-instance.

Error object {\sl unknown-hb-error} (page~\pageref{unknown-hb-error}) if call fails
for any other reason.

\item [Description:]

Deletes link-ID.

\item [Public:]



\end{description}
\horizontalline

\subsection{t*link*set-schema-ID}
\label{t*link*set-schema-ID}

\begin{description}
\item [Name:]  t*link*set-schema-ID

\item [Class:] {\sl t*link-instance}\hfill(page~\pageref{t*link-instance})

\item [Parameters:]
\item {\sl link-ID}:  
valid HB link ID number (integer)

\item {\sl link-schema-ID}:  an ID for a link schema.


\item [Return-value:]
The updated link-schema-ID if successful.

Error object {\sl invalid-link-ID} (page~\pageref{invalid-link-ID}) if not a t*link-ID.

Error object {\sl invalid-link-schema-ID} (page~\pageref{invalid-link-schema-ID}) if not
a t*link-schema.

Error object {\sl unknown-hb-error} (page~\pageref{unknown-hb-error}) if call fails
for any other reason.

\item [Description:]

Updates the schema associated with this 

\item [Public:]

























\end{description}
\horizontalline

\subsection{t*link*set-to-node}
\label{t*link*set-to-node}

\begin{description}
\item [Name:]  t*link*set-to-node

\item [Class:] {\sl t*link-instance}\hfill(page~\pageref{t*link-instance})

\item [Parameters:]
\item {\sl link-ID}:  
valid HB link ID number (integer)

\item {\sl node-ID}:   An integer representing
a valid hbserver node ID.


\item [Return-value:]
Returns the updated to-node ID if successful.

Error object {\sl invalid-link-ID} (page~\pageref{invalid-link-ID}) if not a t*link.

Error object {\sl invalid-node-ID} (page~\pageref{invalid-node-ID}) if not a t*node.

Error object {\sl unknown-hb-error} (page~\pageref{unknown-hb-error}) if call fails
for any other reason.

\item [Description:]

Updates link-ID to point to node-ID.

\item [Public:]



\end{description}
\horizontalline

\subsection{t*link*add-to-constraint}
\label{t*link*add-to-constraint}

\begin{description}
\item [Name:]  t*link*add-to-constraint

\item [Class:] {\sl t*link-instance}\hfill(page~\pageref{t*link-instance})

\item [Parameters:]
\item {\sl link-ID}:  
valid HB link ID number (integer)

\item {\sl node-schema-ID}:  a server-level node-ID that corresponds to an 
instance of a type-level node-schema. 



\item [Return-value:]
The updated constraint expression if successful.

Error object {\sl invalid-link-ID} (page~\pageref{invalid-link-ID}) if not a t*link-instance.

Error object {\sl invalid-node-schema-ID} (page~\pageref{invalid-node-schema-ID}) if not a
t*node-schema.

Error object {\sl unknown-hb-error} (page~\pageref{unknown-hb-error}) if call fails for
any other reason.

\item [Description:]

Adds a to-node constraint. If node-schema-ID is the
distinguished symbol T, then the constraint expression
is set to T.

\item [Public:]



\end{description}
\horizontalline

\subsection{t*link*add-from-constraint}
\label{t*link*add-from-constraint}

\begin{description}
\item [Name:]  t*link*add-from-constraint

\item [Class:] {\sl t*link-instance}\hfill(page~\pageref{t*link-instance})

\item [Parameters:]
\item {\sl link-ID}:  
valid HB link ID number (integer)

\item {\sl node-schema-ID}:  a server-level node-ID that corresponds to an 
instance of a type-level node-schema. 



\item [Return-value:]
The updated from-node constraint expression if successful.

Error object {\sl invalid-link-ID} (page~\pageref{invalid-link-ID}) if not a t*link-instance.

Error object {\sl invalid-node-schema-ID} (page~\pageref{invalid-node-schema-ID}) if not
a t*node-schema.

Error object {\sl unknown-hb-error} (page~\pageref{unknown-hb-error}) if call fails for
any other reason.

\item [Description:]

Adds node-schema-ID to the from-constraints for this
link instance. Node-schema-ID may also be the distinguished
symbol T, which sets the constraint expression itself
to T.

\item [Public:]



\end{description}
\horizontalline

\subsection{t*link*delete-to-constraint}
\label{t*link*delete-to-constraint}

\begin{description}
\item [Name:]\item [Name:]  t*link*delete-to-constraint

\item [Class:]\item [Class:] {\sl t*link-instance}\hfill(page~\pageref{t*link-instance}){\sl t*link-instance}\hfill(page~\pageref{t*link-instance})

\item [Parameters:]\item [Parameters:]
{\sl node-schema-ID}:  a server-level node-ID that corresponds to an 
instance of a type-level node-schema. 

updated to node constraint expression if successful.

Error object {\sl invalid-link-ID} (page~\pageref{invalid-link-ID}) if not a t*link-ID.

Error object {\sl invalid-node-schema-ID} (page~\pageref{invalid-node-schema-ID}) if not a 
t*node-schema, and if not currently a member of the 
to-node constraints.

\item [Description:]

Deletes a to-node constraint.

\item [Public:]



\end{description}
\horizontalline

\subsection{t*link*delete-from-constraint}
\label{t*link*delete-from-constraint}

\begin{description}
\item [Name:]  t*link*delete-from-constraint

\item [Class:] {\sl t*link-instance}\hfill(page~\pageref{t*link-instance})

\item [Parameters:]
\item {\sl link-ID}:  
valid HB link ID number (integer)

\item {\sl node-schema-ID}:  a server-level node-ID that corresponds to an 
instance of a type-level node-schema. 



\item [Return-value:]
Returns the updated from-node constraint expression
if successful.

Error object {\sl invalid-link-ID} (page~\pageref{invalid-link-ID}) if not a t*link-instance.

Error object {\sl invalid-node-schema-ID} (page~\pageref{invalid-node-schema-ID}) if not
a t*node-schema, and not a member of the constraint
expression.

Error object {\sl unknown-hb-error} (page~\pageref{unknown-hb-error}) if call fails for
any other reason.

\item [Description:]

Deletes a from constraint from this node instance's
constraint expression.

\item [Public:]



\end{description}
\horizontalline

\subsection{t*link*add-layer-ID}
\label{t*link*add-layer-ID}

\begin{description}
\item [Name:]  t*link*add-layer-ID

\item [Class:] {\sl t*link-instance}\hfill(page~\pageref{t*link-instance})

\item [Parameters:]
\item {\sl link-ID}:  
valid HB link ID number (integer)

\item {\sl layer-ID}:  a unique ID for layers (possibly a node-ID?)



\item [Return-value:]
The updated list of layer-IDs if successful.

Error object {\sl invalid-link-ID} (page~\pageref{invalid-link-ID}) if not a legal
t*link-ID.

Error object {\sl invalid-layer-ID} (page~\pageref{invalid-layer-ID}) if not a legal
layer-ID, or if it already occurs in this link-instance.

Error object {\sl unknown-hb-error} (page~\pageref{unknown-hb-error}) if call fails
for any other reason.

\item [Description:]

Makes link-ID a member of layer-ID.

\item [Public:]



\end{description}
\horizontalline

\subsection{t*link*delete-layer-ID}
\label{t*link*delete-layer-ID}

\begin{description}
\item [Name:]  t*link*delete-layer-ID

\item [Class:] {\sl t*link-instance}\hfill(page~\pageref{t*link-instance})

\item [Parameters:]
\item {\sl link-ID}:  
valid HB link ID number (integer)

\item {\sl layer-ID}:  a unique ID for layers (possibly a node-ID?)

 

\item [Return-value:]
The updated list of layer-IDs if successful.

Error object {\sl invalid-link-ID} (page~\pageref{invalid-link-ID}) if link-ID was
not a t*link-instance.

Error object {\sl invalid-layer-ID} (page~\pageref{invalid-layer-ID}) if layer-ID was
not a t*layer, or not a current member of this link
instance's layer-IDs.

Error object {\sl unknown-hb-error} (page~\pageref{unknown-hb-error}) if call fails
for any other reason. 

\item [Description:]

Deletes a layer-ID membership from this link.

\item [Public:]



\end{description}
\horizontalline

\subsection{t*link*convergence}
\label{t*link*convergence}

\begin{description}
\item [Name:]  t*link*convergence

\item [Class:] {\sl t*link-instance}\hfill(page~\pageref{t*link-instance})

\item [Parameters:]
\item {\sl link-ID}:  
valid HB link ID number (integer)



\item [Return-value:]

An integer convergence value.

\item [Description:]

Returns the convergence between this link instance and
the set of link-schemas if successful.

Error object {\sl unknown-hb-error} (page~\pageref{unknown-hb-error}) if call fails for
any reason.

\item [Public:]



\end{description}
\horizontalline

\subsection{t*\{link\}*IDs}
\label{t*link*IDs}

\begin{description}
\item [Name:]  t*\{link\}*IDs

\item [Class:] {\sl t*link-instance}\hfill(page~\pageref{t*link-instance})

\item [Parameters:] none

\item [Return-value:] A list of link-IDs

\item [Description:]

Returns a freshly consed list of link-IDs.

\item [Public:]



\end{description}
\horizontalline

\subsection{t*\{link\}*mapc-IDs}
\label{t*link*mapc-IDs}

\begin{description}
\item [Name:]  t*\{link\}*mapc-IDs

\item [Class:] {\sl t*link-instance}\hfill(page~\pageref{t*link-instance})

\item [Parameters:]
\item {\sl map-ID-fn}:  A function that takes one argument, an ID,
and which performs some side-effect based upon that
value.



\item [Return-value:] nil

\item [Description:]

Calls map-ID-fn on each currently defined link-ID.

\item [Public:]



\end{description}
\horizontalline

\section{t*field-schema}
\label{t*field-schema}

\begin{description}
\item [Name:]  t*field-schema

\item [Layer:] {\sl Type}\hfill(page~\pageref{Type})

\item [Description:]

Field schemas identify and define the internal structure
of node instances.  This structure is enforced via
a user-specified validity function.

\item [Attributes:]
\item {\sl t*field-schema*name}\hfill(page~\pageref{t*field-schema*name})
\item {\sl t*field-schema*validity-fn}\hfill(page~\pageref{t*field-schema*validity-fn})


\item [Operations:]
\item {\sl t*field-schema*make}\hfill(page~\pageref{t*field-schema*make})
\item {\sl t*field-schema*delete}\hfill(page~\pageref{t*field-schema*delete})
\item {\sl t*field-schema*set-name}\hfill(page~\pageref{t*field-schema*set-name})
\item {\sl t*field-schema*set-validity-fn}\hfill(page~\pageref{t*field-schema*set-validity-fn})

\item {\sl t*\{field-schema\}*IDs}\hfill(page~\pageref{t*field-schema*IDs})
\item {\sl t*\{field-schema\}*mapc-IDs}\hfill(page~\pageref{t*field-schema*mapc-IDs}) 

\item [Subclasses:]


\item [Superclasses:]


\item [Instances:]



\end{description}
\horizontalline

\subsection{t*field-schema*name}
\label{t*field-schema*name}

\begin{description}
\item [Name:]  t*field-schema*name

\item [Class:] {\sl t*field-schema}\hfill(page~\pageref{t*field-schema})

\item [Contents:] a symbol

\item [Description:]

The name of the field. 

\item [Setf-able:] See t*field-schema*set-name.


\item [Public:]



\end{description}
\horizontalline

\subsection{t*field-schema*validity-fn}
\label{t*field-schema*validity-fn}

\begin{description}
\item [Name:]  t*field-schema*validity-fn

\item [Class:] {\sl t*field-schema}\hfill(page~\pageref{t*field-schema})

\item [Contents:] A list

\item [Description:]

This is an Emacs-Lisp function (in lambda list format) for 
assessing the validity of an object to be stored in a 
field. 

The validity function takes one argument, the field 
value, and should return T if the field value is legal
and NIL if the field value is not. 

\item [Setf-able:] See t*field-schema*set-validity-fn

\item [Public:]



\end{description}
\horizontalline

\subsection{t*field-schema*make}
\label{t*field-schema*make}

\begin{description}

\item [Name:]  t*field-schema*make


\item [Class:]
{\sl t*field-schema}\hfill(page~\pageref{t*field-schema})


\item [Parameters:]
\item {\sl node-name}:  
A valid node name. This currently means that it is a
string of less than 40 characters, and that it does
not contain leading space(s) or tabs.

\item {\sl field-validity-fn}:  

A lambda list representing a funcallable Emacs-Lisp function for
assessing the contents of a field.

The function takes one argument, the contents of the field.
See t*field-schema for more information on the field content
structure.



\item [Return-value:]
The newly created field-schema-ID if successful. 

Error object {\sl invalid-node-name} (page~\pageref{invalid-node-name}) if node-name is not
legal.

Error object {\sl unknown-hb-error} (page~\pageref{unknown-hb-error}) if call fails for 
any other reason.

\item [Description:]

Creates a new field schema.  

\item [Public:]



\end{description}
\horizontalline

\subsection{t*field-schema*delete}
\label{t*field-schema*delete}

\begin{description}
\item [Name:]  t*field-schema*delete

\item [Class:] {\sl t*field-schema}\hfill(page~\pageref{t*field-schema})

\item [Parameters:]
\item {\sl field-schema-ID}:  an integer signifying a field-schema.


\item [Return-value:]
The deleted field-schema-ID if successful.

Error object {\sl invalid-field-ID} (page~\pageref{invalid-field-ID}) if field-schema-ID
is not a t*field-schema.

Error object {\sl unknown-hb-error} (page~\pageref{unknown-hb-error}) if call fails
for any other reason.

\item [Description:]

Deletes the field-schema.  This does not physically
remove the schema from the database, but instead
prevents any new instances of it from being made.

For the time being, node schemas and instances are
not automatically flagged as containing a deleted
field schema. 


\item [Public:]



\end{description}
\horizontalline

\subsection{t*field-schema*set-name}
\label{t*field-schema*set-name}

\begin{description}
\item [Name:]  t*field-schema*set-name

\item [Class:] {\sl t*field-schema}\hfill(page~\pageref{t*field-schema})

\item [Parameters:]
\item {\sl field-schema-ID}:  an integer signifying a field-schema.

\item {\sl node-name}:  
A valid node name. This currently means that it is a
string of less than 40 characters, and that it does
not contain leading space(s) or tabs.


\item [Return-value:]
T if successful.

Error object {\sl invalid-node-name} (page~\pageref{invalid-node-name}) if node-name
is syntactically incorrect.

Error object {\sl invalid-field-ID} (page~\pageref{invalid-field-ID}) if field-schema-ID
is not a t*field-schema.

Error object {\sl unknown-hb-error} (page~\pageref{unknown-hb-error}) if the call fails
for any other reason.

\item [Description:]

Sets the name attribute of the field-schema.

\item [Public:]



\end{description}
\horizontalline

\subsection{t*field-schema*set-validity-fn}
\label{t*field-schema*set-validity-fn}

\begin{description}
\item [Name:]  t*field-schema*set-validity-fn

\item [Class:] {\sl t*field-schema}\hfill(page~\pageref{t*field-schema})

\item [Parameters:]
\item {\sl field-schema-ID}:  an integer signifying a field-schema.

\item {\sl field-validity-fn}:  

A lambda list representing a funcallable Emacs-Lisp function for
assessing the contents of a field.

The function takes one argument, the contents of the field.
See t*field-schema for more information on the field content
structure.




\item [Return-value:] 
T if successful.

Error object {\sl invalid-field-ID} (page~\pageref{invalid-field-ID}) if field-schema-ID is not a valid t*field-schema.

Error object {\sl unknown-hb-error} (page~\pageref{unknown-hb-error}) if the call fails
for any other reason. 

\item [Description:]

Sets the validity function associated with field-schema-ID
to a new value. 


\item [Public:]



\end{description}
\horizontalline

\subsection{t*\{field-schema\}*IDs}
\label{t*field-schema*IDs}

\begin{description}
\item [Name:]  t*\{field-schema\}*IDs

\item [Class:] {\sl t*field-schema}\hfill(page~\pageref{t*field-schema})

\item [Parameters:] none

\item [Return-value:] A list of field-schema IDs.

\item [Description:]

Returns a freshly consed list of all currently defined
field-schema IDs.

\item [Public:]



\end{description}
\horizontalline

\subsection{t*\{field-schema\}*mapc-IDs}
\label{t*field-schema*mapc-IDs}

\begin{description}
\item [Name:]  t*\{field-schema\}*mapc-IDs

\item [Class:] {\sl t*field-schema}\hfill(page~\pageref{t*field-schema})

\item [Parameters:]
\item {\sl map-ID-fn}:  A function that takes one argument, an ID,
and which performs some side-effect based upon that
value.



\item [Return-value:] nil

\item [Description:]

Calls map-ID-fn on each currently defined field-schema-ID.

\item [Public:]



\end{description}
\horizontalline

\section{t*layer}
\label{t*layer}

\begin{description}
\item [Name:]  t*layer

\item [Layer:] {\sl Type}\hfill(page~\pageref{Type})

\item [Description:]

Defines the exploratory namespace mechanism in Egret.

\item [Attributes:]
\item {\sl t*layer*name}\hfill(page~\pageref{t*layer*name})
\item {\sl t*layer*node-IDs}\hfill(page~\pageref{t*layer*node-IDs})
\item {\sl t*layer*link-IDs}\hfill(page~\pageref{t*layer*link-IDs})

\item [Operations:]
\item {\sl t*\{layer\}*mapc-IDs}\hfill(page~\pageref{t*layer*mapc-IDs})
\item {\sl t*\{layer\}*IDs}\hfill(page~\pageref{t*layer*IDs})
\item {\sl t*layer*make}\hfill(page~\pageref{t*layer*make})
\item {\sl t*layer*add-node-ID}\hfill(page~\pageref{t*layer*add-node-ID})
\item {\sl t*layer*delete-node-ID}\hfill(page~\pageref{t*layer*delete-node-ID})
\item {\sl t*layer*add-link-ID}\hfill(page~\pageref{t*layer*add-link-ID})
\item {\sl t*layer*delete-link-ID}\hfill(page~\pageref{t*layer*delete-link-ID})
\item {\sl t*layer*divergence}\hfill(page~\pageref{t*layer*divergence})


\item [Subclasses:]


\item [Superclasses:]


\item [Instances:]
























\end{description}
\horizontalline

\subsection{t*layer*name}
\label{t*layer*name}

\begin{description}
\item [Name:]  t*layer*name

\item [Class:] {\sl t*layer}\hfill(page~\pageref{t*layer})

\item [Contents:]
symbol

\item [Description:]
The name of the layer.

\item [Setf-able:]


\item [Public:]



\end{description}
\horizontalline

\subsection{t*layer*node-IDs}
\label{t*layer*node-IDs}

\begin{description}
\item [Name:]  t*layer*node-IDs

\item [Class:] {\sl t*layer}\hfill(page~\pageref{t*layer})

\item [Contents:]
A list of node-IDs.

\item [Description:]

The node-IDs currently belonging to this layer.

\item [Setf-able:]


\item [Public:]



\end{description}
\horizontalline

\subsection{t*layer*link-IDs}
\label{t*layer*link-IDs}

\begin{description}
\item [Name:]  t*layer*link-IDs

\item [Class:] {\sl t*layer}\hfill(page~\pageref{t*layer})

\item [Contents:] 
A list of link-IDs

\item [Description:]

The link-IDs belonging to this layer.

\item [Setf-able:]


\item [Public:]



\end{description}
\horizontalline

\subsection{t*\{layer\}*mapc-IDs}
\label{t*layer*mapc-IDs}

\begin{description}
\item [Name:]  t*\{layer\}*mapc-IDs

\item [Class:] {\sl t*layer}\hfill(page~\pageref{t*layer})

\item [Parameters:]
\item {\sl map-ID-fn}:  A function that takes one argument, an ID,
and which performs some side-effect based upon that
value.



\item [Return-value:] nil

\item [Description:]

Calls map-ID-fn on each currently defined layer-ID.

\item [Public:]



\end{description}
\horizontalline

\subsection{t*\{layer\}*IDs}
\label{t*layer*IDs}

\begin{description}
\item [Name:]  t*\{layer\}*IDs

\item [Class:] {\sl t*layer}\hfill(page~\pageref{t*layer})

\item [Parameters:] none

\item [Return-value:]

A list of layer-IDs.

\item [Description:]

Returns a freshly consed list of all currently
defined layer-IDs.

\item [Public:]



\end{description}
\horizontalline

\subsection{t*layer*make}
\label{t*layer*make}

\begin{description}
\item [Name:]  t*layer*make

\item [Class:] {\sl t*layer}\hfill(page~\pageref{t*layer})

\item [Parameters:]
\item {\sl layer-name}:  a string less than 40 characters, obeying
all node-name restrictions. 


\item {\sl node-IDs}:  a list of node-IDs


\item {\sl link-IDs}:  a list of link-IDs


\item [Return-value:]
A layer-ID for the newly created layer if successful.

Error object {\sl invalid-node-ID} (page~\pageref{invalid-node-ID}) if any of the node-IDs
are invalid.

Error object {\sl invalid-link-ID} (page~\pageref{invalid-link-ID}) if any of the link-IDs
are invalid.

Error object {\sl unknown-hb-error} (page~\pageref{unknown-hb-error}) if the call fails
for any other reason.

\item [Description:]

Creates and initializes a new layer. 

\item [Public:]



\end{description}
\horizontalline

\subsection{t*layer*add-node-ID}
\label{t*layer*add-node-ID}

\begin{description}
\item [Name:]  t*layer*add-node-ID

\item [Class:] {\sl t*layer}\hfill(page~\pageref{t*layer})

\item [Parameters:]
\item {\sl layer-ID}:  a unique ID for layers (possibly a node-ID?)


\item {\sl node-ID}:   An integer representing
a valid hbserver node ID.


\item [Return-value:]
The updated list of node-IDs in layer-ID if successful.

Error object {\sl invalid-layer-ID} (page~\pageref{invalid-layer-ID}) if not a t*layer.

Error object {\sl invalid-node-ID} (page~\pageref{invalid-node-ID}) if not a 
t*node-instance, or if currently a member of t*layer.

\item [Description:]

Adds node-ID to layer-ID.

\item [Public:]



\end{description}
\horizontalline

\subsection{t*layer*delete-node-ID}
\label{t*layer*delete-node-ID}

\begin{description}
\item [Name:]  t*layer*delete-node-ID

\item [Class:] {\sl t*layer}\hfill(page~\pageref{t*layer})

\item [Parameters:]
\item {\sl layer-ID}:  a unique ID for layers (possibly a node-ID?)


\item {\sl node-ID}:   An integer representing
a valid hbserver node ID.


\item [Return-value:]
The updated list of node-IDs in layer-ID if successful.

Error object {\sl invalid-layer-ID} (page~\pageref{invalid-layer-ID}) if not a t*layer.

Error object {\sl invalid-node-ID} (page~\pageref{invalid-node-ID}) if not a
t*node-instance, or if not in layer-ID.

Error object {\sl unknown-hb-error} (page~\pageref{unknown-hb-error}) if call fails for
any other reason.

\item [Description:]

Deletes node-ID from layer-ID.

\item [Public:]



\end{description}
\horizontalline

\subsection{t*layer*add-link-ID}
\label{t*layer*add-link-ID}

\begin{description}
\item [Name:]  t*layer*add-link-ID

\item [Class:] {\sl t*layer}\hfill(page~\pageref{t*layer})

\item [Parameters:]
\item {\sl layer-ID}:  a unique ID for layers (possibly a node-ID?)


\item {\sl link-ID}:  
valid HB link ID number (integer)


\item [Return-value:]
The updated set of link-IDs in layer-ID if successful.

Error object {\sl invalid-layer-ID} (page~\pageref{invalid-layer-ID}) if not a t*layer.

Error object {\sl invalid-link-ID} (page~\pageref{invalid-link-ID}) if not a
t*link-instance, or if link-ID is not a member of 
layer-ID.

Error object {\sl unknown-hb-error} (page~\pageref{unknown-hb-error}) if call fails
for any other reason.

\item [Description:]

Deletes link-ID from layer-ID.

\item [Public:]



\end{description}
\horizontalline

\subsection{t*layer*delete-link-ID}
\label{t*layer*delete-link-ID}

\begin{description}
\item [Name:]  t*layer*delete-link-ID

\item [Class:] {\sl t*layer}\hfill(page~\pageref{t*layer})

\item [Parameters:]
\item {\sl layer-ID}:  a unique ID for layers (possibly a node-ID?)


\item {\sl link-ID}:  
valid HB link ID number (integer)


\item [Return-value:]
The updated set of link-IDs in this layer if successful.

Error object {\sl invalid-layer-ID} (page~\pageref{invalid-layer-ID}) if not a t*layer.

Error object {\sl invalid-link-ID} (page~\pageref{invalid-link-ID}) if not a t*link,
or if not currently a member of this layer.

Error object {\sl unknown-hb-error} (page~\pageref{unknown-hb-error}) if call fails
for any other reason.

\item [Description:]

Deletes a link instance from this layer.

\item [Public:]



\end{description}
\horizontalline

\subsection{t*layer*divergence}
\label{t*layer*divergence}

\begin{description}
\item [Name:]  t*layer*divergence

\item [Class:] {\sl t*layer}\hfill(page~\pageref{t*layer})

\item [Parameters:]
\item {\sl layer-ID}:  a unique ID for layers (possibly a node-ID?)



\item [Return-value:]
The divergence value for layer-ID if successful.

Error object {\sl invalid-layer-ID} (page~\pageref{invalid-layer-ID}) if not a t*layer.

Error object {\sl unknown-hb-error} (page~\pageref{unknown-hb-error}) if call fails
for any other reason.

\item [Description:]

Computes the aggregate divergence for all of the
schemas with instances in this layer. 

\item [Public:]



\end{description}
\horizontalline

\section{t*error}
\label{t*error}

\begin{description}
\item [Name:]  t*error

\item [Layer:] {\sl Type}\hfill(page~\pageref{Type})

\item [Description:]
The instances of this class consist of the type-level
error objects. In some cases, however, the type system
may return error objects that are instances of the 
server system s*serror class, such as the error object
for an invalid node name. Thus, this class defines 
all of the errors semantically associated with the
type system facilities, but not all of the possible 
errors that the type system could return.

\item [Attributes:]

\item [Operations:]

\item [Subclasses:]

\item [Superclasses:]
\item {\sl u*error}\hfill(page~\pageref{u*error})

\item [Instances:]
\item {\sl invalid-link-ID}\hfill(page~\pageref{invalid-link-ID})
\item {\sl invalid-constraint-exp}\hfill(page~\pageref{invalid-constraint-exp})
\item {\sl invalid-link-schema-ID}\hfill(page~\pageref{invalid-link-schema-ID})
\item {\sl invalid-event-instance}\hfill(page~\pageref{invalid-event-instance})
\item {\sl invalid-layer-ID}\hfill(page~\pageref{invalid-layer-ID})
\item {\sl invalid-buffer-instance}\hfill(page~\pageref{invalid-buffer-instance})
\item {\sl invalid-node-ID}\hfill(page~\pageref{invalid-node-ID})
\item {\sl invalid-node-schema-ID}\hfill(page~\pageref{invalid-node-schema-ID})
\item {\sl invalid-field-ID}\hfill(page~\pageref{invalid-field-ID})




\end{description}
\horizontalline

\subsection{invalid-link-ID}
\label{invalid-link-ID}

\begin{description}
\item [Name:]  invalid-link-ID

\item [Class:] {\sl t*error}\hfill(page~\pageref{t*error})

\item [Description:]

Indicates the passed link instance ID was invalid.


\end{description}
\horizontalline

\subsection{invalid-constraint-exp}
\label{invalid-constraint-exp}

\begin{description}
\item [Name:]  invalid-constraint-exp

\item [Class:]
{\sl t*error}\hfill(page~\pageref{t*error})


\item [Description:]

Error object indicating the presence of an 
invalid link constraint expression.

\end{description}
\horizontalline

\subsection{invalid-link-schema-ID}
\label{invalid-link-schema-ID}

\begin{description}
\item [Name:]  invalid-link-schema-ID

\item [Class:] {\sl t*error}\hfill(page~\pageref{t*error})

\item [Description:]

Error object indicating an invalid link-schema-ID 
object.


\end{description}
\horizontalline

\subsection{invalid-event-instance}
\label{invalid-event-instance}

\begin{description}
\item [Name:]  invalid-event-instance

\item [Class:] {\sl t*error}\hfill(page~\pageref{t*error})

\item [Description:]

An error indicating that an object that wasn't a
t*event instance was passed. 


\end{description}
\horizontalline

\subsection{invalid-layer-ID}
\label{invalid-layer-ID}

\begin{description}
\item [Name:]  invalid-layer-ID

\item [Class:] {\sl t*error}\hfill(page~\pageref{t*error})

\item [Description:]

An error object indicating that the argument was
not a valid layer-ID.


\end{description}
\horizontalline

\subsection{invalid-buffer-instance}
\label{invalid-buffer-instance}

\begin{description}
\item [Name:]  invalid-buffer-instance

\item [Class:] {\sl t*error}\hfill(page~\pageref{t*error})

\item [Description:]

Error object indicating that the passed object
was not an Emacs buffer instance.


\end{description}
\horizontalline

\subsection{invalid-node-ID}
\label{invalid-node-ID}

\begin{description}
\item [Name:]  invalid-node-ID

\item [Class:] {\sl t*error}\hfill(page~\pageref{t*error})

\item [Description:]

An error object representing the invalidity of the 
passed value as a node-ID.


\end{description}
\horizontalline

\subsection{invalid-node-schema-ID}
\label{invalid-node-schema-ID}

\begin{description}
\item [Name:]  invalid-node-schema-ID

\item [Class:]

\item [Description:]

Error object indicating an invalid type-level node-schema
ID.


\end{description}
\horizontalline

\subsection{invalid-field-ID}
\label{invalid-field-ID}

\begin{description}
\item [Name:]  invalid-field-ID

\item [Class:] {\sl t*error}\hfill(page~\pageref{t*error})

\item [Description:]

The error object indicating that a bad field-schema-ID
value was detected.


\end{description}
\horizontalline

\section{t*event}
\label{t*event}

\begin{description}
\item [Name:]  t*event

\item [Layer:] {\sl Type}\hfill(page~\pageref{Type})

\item [Description:]

This class defines the set of events to be handled
by the type system.  Events are the communication
mechanism between the hyperbase and its currently
connected clients that allow changes made by one
client to be propogated to other clients. 

\item [Attributes:]
\item {\sl t*event*handlers}\hfill(page~\pageref{t*event*handlers})


\item [Operations:]
\item {\sl t*event*remove-handler}\hfill(page~\pageref{t*event*remove-handler})
\item {\sl t*event*add-handler}\hfill(page~\pageref{t*event*add-handler})
\item {\sl t*event*initialize}\hfill(page~\pageref{t*event*initialize})

\item [Subclasses:]

\item [Superclasses:]

\item [Instances:]
Due to a bug in cv-1.3, the following three links
all point to the same node,
(t*event*update-lschema-from-constraints),
even though two other event nodes exist and should
be pointed to (t*event*update-lschema-name and 
t*event*update-lschema-to-constraints). 

\item {\sl t*event*update-lschema-from-constraints}\hfill(page~\pageref{t*event*update-lschema-from-constraints})
\item {\sl t*event*update-lschema-to-constraints}\hfill(page~\pageref{t*event*update-lschema-to-constraints})
\item {\sl t*event*update-lschema-name}\hfill(page~\pageref{t*event*update-lschema-name})
\item {\sl t*event*delete-link-schema}\hfill(page~\pageref{t*event*delete-link-schema})
\item {\sl t*event*new-link-schema}\hfill(page~\pageref{t*event*new-link-schema})
\item {\sl t*event*update-fschema-val-fn}\hfill(page~\pageref{t*event*update-fschema-val-fn})
\item {\sl t*event*update-fschema-name}\hfill(page~\pageref{t*event*update-fschema-name})
\item {\sl t*event*delete-field-schema}\hfill(page~\pageref{t*event*delete-field-schema})
\item {\sl t*event*new-field-schema}\hfill(page~\pageref{t*event*new-field-schema})
\item {\sl t*event*delete-ninstance-field}\hfill(page~\pageref{t*event*delete-ninstance-field})
\item {\sl t*event*update-ninstance-field}\hfill(page~\pageref{t*event*update-ninstance-field})
\item {\sl t*event*update-ninstance-name}\hfill(page~\pageref{t*event*update-ninstance-name})
\item {\sl t*event*delete-node-instance}\hfill(page~\pageref{t*event*delete-node-instance})
\item {\sl t*event*new-node-instance}\hfill(page~\pageref{t*event*new-node-instance})
\item {\sl t*event*update-nschema-nodes}\hfill(page~\pageref{t*event*update-nschema-nodes})
\item {\sl t*event*update-nschema-fields}\hfill(page~\pageref{t*event*update-nschema-fields})
\item {\sl t*event*update-nschema-name}\hfill(page~\pageref{t*event*update-nschema-name})
\item {\sl t*event*delete-node-schema}\hfill(page~\pageref{t*event*delete-node-schema})
\item {\sl t*event*new-node-schema}\hfill(page~\pageref{t*event*new-node-schema})

























\end{description}
\horizontalline

\subsection{t*event*handlers}
\label{t*event*handlers}

\begin{description}
\item [Name:]  t*event*handlers

\item [Class:] {\sl t*event}\hfill(page~\pageref{t*event})

\item [Contents:] List of symbols

\item [Description:]

This attribute returns an ordered list of event
handler function names. 

\item [Setf-able:] See t*event*add-event-handlers and 
t*event*remove-event-handlers.


\item [Public:]



\end{description}
\horizontalline

\subsection{t*event*remove-handler}
\label{t*event*remove-handler}

\begin{description}
\item [Name:]  t*event*remove-handler

\item [Class:] {\sl t*event}\hfill(page~\pageref{t*event})

\item [Parameters:]
\item {\sl t*event-instance}:  an instance of t*event


\item {\sl handler-fn-name}:  function symbol


\item [Return-value:]
T if successful.

Error object {\sl invalid-event-instance} (page~\pageref{invalid-event-instance}) if first
arg is not a t*event-instance.

\item [Description:]

Removes handler-fn-name from t*event-instance.

\item [Public:]



\end{description}
\horizontalline

\subsection{t*event*add-handler}
\label{t*event*add-handler}

\begin{description}
\item [Name:]  t*event*add-handler

\item [Class:] {\sl t*event}\hfill(page~\pageref{t*event})

\item [Parameters:]
\item {\sl t*event-instance}:  an instance of t*event


\item {\sl handler-fn-name}:  function symbol

\item {\sl before-handlers}:  functional symbol

\item {\sl after-handlers}:  function symbol


\item [Return-value:]
T if successful.

Error object {\sl invalid-event-instance} (page~\pageref{invalid-event-instance}) if first
argument was not a legal t*event-instance.

Error object {\sl conflicting-hook-constraints} (page~\pageref{conflicting-hook-constraints}) if the 
constraints cannot be satisfied.

\item [Description:]

Adds handler-fn-name to the set of function handlers for
t*event-instance.

\item [Public:]



\end{description}
\horizontalline

\subsection{t*event*initialize}
\label{t*event*initialize}

\begin{description}
\item [Name:]  t*event*initialize

\item [Class:] {\sl t*event}\hfill(page~\pageref{t*event})

\item [Parameters:]
\item {\sl t*event-instance}:  an instance of t*event



\item [Return-value:]
T if successfully re-initialized the set of event-handlers
for event-instance to NIL.

Error object {\sl invalid-event-instance} (page~\pageref{invalid-event-instance}) if the argument
was not a t*event instance.

\item [Description:]

Resets the list of event handler functions for event-instance
to NIL.

\item [Public:]



\end{description}
\horizontalline

\subsection{t*event*update-lschema-from-constraints}
\label{t*event*update-lschema-from-constraints}

\begin{description}
\item [Name:]  t*event*update-lschema-from-constraints

\item [Class:] {\sl t*event}\hfill(page~\pageref{t*event})

\item [Description:]

Handler functions are passed a link-schema-ID and
its from node constraint expression.


\end{description}
\horizontalline

\subsection{t*event*update-lschema-to-constraints}
\label{t*event*update-lschema-to-constraints}

\begin{description}
\item [Name:]  t*event*update-lschema-to-constraints

\item [Class:] {\sl t*event}\hfill(page~\pageref{t*event})

\item [Description:]

Handler functions are passed a link-schema-ID and
its new to-node constraint expression.


\end{description}
\horizontalline

\subsection{t*event*update-lschema-name}
\label{t*event*update-lschema-name}

\begin{description}
\item [Name:]  t*event*update-lschema-name

\item [Class:] {\sl t*event}\hfill(page~\pageref{t*event})

\item [Description:]

Handler functions are passed a link-schema-ID and
its new name.


\end{description}
\horizontalline

\subsection{t*event*delete-link-schema}
\label{t*event*delete-link-schema}

\begin{description}
\item [Name:]  t*event*delete-link-schema

\item [Class:] {\sl t*event}\hfill(page~\pageref{t*event})

\item [Description:]

Handler functions are passed the link-schema-ID for
the successfully deleted link schema.

Note that schema "deletion" has a special interpretation
in Egret: the schema is not physically removed, just
marked as not available for future instantiation.


\end{description}
\horizontalline

\subsection{t*event*new-link-schema}
\label{t*event*new-link-schema}

\begin{description}
\item [Name:]  t*event*new-link-schema

\item [Class:] {\sl t*event}\hfill(page~\pageref{t*event})

\item [Description:]

Handler functions are passed a new link-schema-ID,
its name, and its to and from node constraint expressions.



\end{description}
\horizontalline

\subsection{t*event*update-fschema-val-fn}
\label{t*event*update-fschema-val-fn}

\begin{description}
\item [Name:]  t*event*update-fschema-val-fn

\item [Class:] {\sl t*event}\hfill(page~\pageref{t*event})

\item [Description:]

Handler functions are passed a field-schema-ID and
the new validity function.




\end{description}
\horizontalline

\subsection{t*event*update-fschema-name}
\label{t*event*update-fschema-name}

\begin{description}
\item [Name:]  t*event*update-fschema-name

\item [Class:] {\sl t*event}\hfill(page~\pageref{t*event})

\item [Description:]

Handler functions are passed a field-schema-ID and
its new name.


\end{description}
\horizontalline

\subsection{t*event*delete-field-schema}
\label{t*event*delete-field-schema}

\begin{description}
\item [Name:]  t*event*delete-field-schema

\item [Class:] {\sl t*event}\hfill(page~\pageref{t*event})

\item [Description:]

Handler functions are passed a field-schema-ID after
it has been successfully deleted.

Note that deletion of schemas does not result in 
removal, since their presence is required to interpret
previously created nodes. Rather, deletion indicates
that new new uses of the schema are allowed.


\end{description}
\horizontalline

\subsection{t*event*new-field-schema}
\label{t*event*new-field-schema}

\begin{description}
\item [Name:]  t*event*new-field-schema

\item [Class:] {\sl t*event}\hfill(page~\pageref{t*event})

\item [Description:]

Handler functions are passed a new field-schema-ID,
its name, and its validity function.


\end{description}
\horizontalline

\subsection{t*event*delete-ninstance-field}
\label{t*event*delete-ninstance-field}

\begin{description}
\item [Name:]  t*event*delete-ninstance-field

\item [Class:] {\sl t*event}\hfill(page~\pageref{t*event})

\item [Description:]

Handler functions are passed a node-ID and its
field-schema-ID that has just been successfully
deleted.


\end{description}
\horizontalline

\subsection{t*event*update-ninstance-field}
\label{t*event*update-ninstance-field}

\begin{description}
\item [Name:]  t*event*update-ninstance-field

\item [Class:] {\sl t*event}\hfill(page~\pageref{t*event})

\item [Description:]

Handler functions are passed a node-ID, a 
field-schema-ID, and the new field contents.

This event is also used when a new field is added
to a node.

\end{description}
\horizontalline

\subsection{t*event*update-ninstance-name}
\label{t*event*update-ninstance-name}

\begin{description}
\item [Name:]  t*event*update-ninstance-name

\item [Class:] {\sl t*event}\hfill(page~\pageref{t*event})

\item [Description:]

Handler functions are passed a node-ID and its
new name.


\end{description}
\horizontalline

\subsection{t*event*delete-node-instance}
\label{t*event*delete-node-instance}

\begin{description}
\item [Name:]  t*event*delete-node-instance

\item [Class:] {\sl t*event}\hfill(page~\pageref{t*event})

\item [Description:]

Handler functions are passed the node-ID that has just
been successfully deleted.


\end{description}
\horizontalline

\subsection{t*event*new-node-instance}
\label{t*event*new-node-instance}

\begin{description}
\item [Name:]  t*event*new-node-instance

\item [Class:] {\sl t*event}\hfill(page~\pageref{t*event})

\item [Description:]

Handler function are passed the new node-ID and
its name.


\end{description}
\horizontalline

\subsection{t*event*update-nschema-nodes}
\label{t*event*update-nschema-nodes}

\begin{description}
\item [Name:]  t*event*update-nschema-nodes

\item [Class:] {\sl t*event}\hfill(page~\pageref{t*event})

\item [Description:]

Handler functions are passed a node-schema-ID and
a list of its new node-IDs.


\end{description}
\horizontalline

\subsection{t*event*update-nschema-fields}
\label{t*event*update-nschema-fields}

\begin{description}
\item [Name:]  t*event*update-nschema-fields

\item [Class:] {\sl t*event}\hfill(page~\pageref{t*event})

\item [Description:]

Handler functions are passed a node-schema-ID and
a list of its new field-IDs.


\end{description}
\horizontalline

\subsection{t*event*update-nschema-name}
\label{t*event*update-nschema-name}

\begin{description}
\item [Name:]  t*event*update-nschema-name

\item [Class:] {\sl t*event}\hfill(page~\pageref{t*event})

\item [Description:]

Handler functions are passed a node-schema-ID and
its new name.


\end{description}
\horizontalline

\subsection{t*event*delete-node-schema}
\label{t*event*delete-node-schema}

\begin{description}
\item [Name:]  t*event*delete-node-schema

\item [Class:] {\sl t*event}\hfill(page~\pageref{t*event})

\item [Description:]

Handler functions for this instance are passed
the node-schema-ID of the just deleted node schema.


\end{description}
\horizontalline

\subsection{t*event*new-node-schema}
\label{t*event*new-node-schema}

\begin{description}
\item [Name:]  t*event*new-node-schema

\item [Class:] {\sl t*event}\hfill(page~\pageref{t*event})

\item [Description:]

Handler functions are passed the new node-schema-ID,
its name, and a list of its field-schema-IDs.




\end{description}
\horizontalline
\end{document}