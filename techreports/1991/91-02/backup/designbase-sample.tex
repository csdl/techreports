\chapter{Utilities}
\label{Utilities}

\begin{description}
\item [Name:]  Utilities

\item [Description:]
This subsystem contains generic classes and utilities for the
entire Egret environment.  Object classes and their operations
defined in this module are available for use and
specialization by all other subsystems. Due to the general
nature of this subsystem, it should always be loaded before
all other subsystems.

\item [Public-classes:]
\item {\sl u*hash}\hfill(page~\pageref{u*hash})
\item {\sl u*table}\hfill(page~\pageref{u*table})
\item {\sl u*error}\hfill(page~\pageref{u*error})
\item {\sl u*hook}\hfill(page~\pageref{u*hook})




\end{description}
\horizontalline

\section{u*hash}
\label{u*hash}

\begin{description}
\item [Name:]  u*hash

\item [Layer:]
{\sl Utilities}\hfill(page~\pageref{Utilities})

\item [Description:]

U*HASH is an implementation of the hash table data
structure in Emacs Lisp. Hash tables trade off space for
time, potentially providing faster retrieval than that
allowed by association lists, the typical alternative
Lisp table structure, which uses sequential search.

U*HASH is implemented as vectors. Each element of U*HASH
can be arbitrary Lisp object, ranging from symbols to
complex structures. Though hash tables are oriented
toward one-way mapping, i.e., from key to data,
U*HASH provides restricted reverse searching, i.e., from
data to key. It does so via maintaining an auxiliary
alist of all elements in the table.

\item [Attributes:]

\item [Operations:]
\item {\sl u*hash*make}\hfill(page~\pageref{u*hash*make})
\item {\sl u*hash*get}\hfill(page~\pageref{u*hash*get})
\item {\sl u*hash*set}\hfill(page~\pageref{u*hash*set})
\item {\sl u*hash*rem}\hfill(page~\pageref{u*hash*rem})
\item {\sl u*hash*existp}\hfill(page~\pageref{u*hash*existp})

\item [Collections:]

\item [Subclasses:]

\item [Superclasses:]

\item [Instances:]



\end{description}
\horizontalline

\subsection{u*hash*make}
\label{u*hash*make}

\begin{description}
\item [Name:]  u*hash*make

\item [Class:]
{\sl u*hash}\hfill(page~\pageref{u*hash})

\item [Parameters:]
\item {\sl size}:  integer (prime number)


\item [Return-value:] obarray of required size with all cells set
to nil.   

\item [Description:]

Creates an obarray as hashtable. SIZE, if supplied, should
be a prime number. Otherwise, defaults to default-size,
i.e., 511. 

\item [Public:]



\end{description}
\horizontalline

\subsection{u*hash*get}
\label{u*hash*get}

\begin{description}
\item [Name:]  u*hash*get

\item [Class:]
{\sl u*hash}\hfill(page~\pageref{u*hash})

\item [Parameters:]
\item {\sl hash-key}:  symbol

\item {\sl table}:  symbol


\item [Return-value:] 
data element of the table entry.

\item [Description:]
Retrieves data element from the hashtable. If optional
arg TABLE is supplied, use that table. Otherwise, use
the system default table.

\item [Public:]



\end{description}
\horizontalline

\subsection{u*hash*set}
\label{u*hash*set}

\begin{description}
\item [Name:]  u*hash*set

\item [Class:]
{\sl u*hash}\hfill(page~\pageref{u*hash})

\item [Parameters:]
\item {\sl hash-key}:  symbol

\item {\sl data}:  any Lisp object

\item {\sl table}:  symbol


\item [Return-value:] new data value

\item [Description:]
Stores DATA in table TABLE using HASH-KEY as key.  If
optional arg TABLE is not supplied, defaults to system
table. 

\item [Public:]



\end{description}
\horizontalline

\subsection{u*hash*rem}
\label{u*hash*rem}

\begin{description}
\item [Name:]  u*hash*rem

\item [Class:]
{\sl u*hash}\hfill(page~\pageref{u*hash})

\item [Parameters:]
\item {\sl hash-key}:  symbol

\item {\sl table}:  symbol


\item [Return-value:]  t if deletion is successful or
nil otherwise.

\item [Description:]

Removes data element corresponding to key HASH-KEY from
table TABLE. If TABLE is not supplied, defaults to
system table. 

\item [Public:]



\end{description}
\horizontalline

\subsection{u*hash*existp}
\label{u*hash*existp}

\begin{description}
\item [Name:]  u*hash*existp

\item [Class:]
{\sl u*hash}\hfill(page~\pageref{u*hash})

\item [Parameters:]
\item {\sl hash-key}:  symbol

\item {\sl table}:  symbol


\item [Return-value:] 
data value if found or nil otherwise.

\item [Description:]
Check the existence of the item with key HASH-KEY in
table TABLE. If TABLE is not supplied, defaults
to system table.

\item [Public:]




\end{description}
\horizontalline
\chapter{Utilities}
\label{Utilities}

\begin{description}
\item [Name:]  Utilities

\item [Description:]
This subsystem contains generic classes and utilities for the
entire Egret environment.  Object classes and their operations
defined in this module are available for use and
specialization by all other subsystems. Due to the general
nature of this subsystem, it should always be loaded before
all other subsystems.

\item [Public-classes:]
\item {\sl u*hash}\hfill(page~\pageref{u*hash})
\item {\sl u*table}\hfill(page~\pageref{u*table})
\item {\sl u*error}\hfill(page~\pageref{u*error})
\item {\sl u*hook}\hfill(page~\pageref{u*hook})




\end{description}
\horizontalline

\section{u*hash}
\label{u*hash}

\begin{description}
\item [Name:]  u*hash

\item [Layer:]
{\sl Utilities}\hfill(page~\pageref{Utilities})

\item [Description:]

U*HASH is an implementation of the hash table data
structure in Emacs Lisp. Hash tables trade off space for
time, potentially providing faster retrieval than that
allowed by association lists, the typical alternative
Lisp table structure, which uses sequential search.

U*HASH is implemented as vectors. Each element of U*HASH
can be arbitrary Lisp object, ranging from symbols to
complex structures. Though hash tables are oriented
toward one-way mapping, i.e., from key to data,
U*HASH provides restricted reverse searching, i.e., from
data to key. It does so via maintaining an auxiliary
alist of all elements in the table.

\item [Attributes:]

\item [Operations:]
\item {\sl u*hash*make}\hfill(page~\pageref{u*hash*make})
\item {\sl u*hash*get}\hfill(page~\pageref{u*hash*get})
\item {\sl u*hash*set}\hfill(page~\pageref{u*hash*set})
\item {\sl u*hash*rem}\hfill(page~\pageref{u*hash*rem})
\item {\sl u*hash*existp}\hfill(page~\pageref{u*hash*existp})

\item [Collections:]

\item [Subclasses:]

\item [Superclasses:]

\item [Instances:]



\end{description}
\horizontalline

\subsection{u*hash*make}
\label{u*hash*make}

\begin{description}
\item [Name:]  u*hash*make

\item [Class:]
{\sl u*hash}\hfill(page~\pageref{u*hash})

\item [Parameters:]
\item {\sl size}:  integer (prime number)


\item [Return-value:] obarray of required size with all cells set
to nil.   

\item [Description:]

Creates an obarray as hashtable. SIZE, if supplied, should
be a prime number. Otherwise, defaults to default-size,
i.e., 511. 

\item [Public:]



\end{description}
\horizontalline

\subsection{u*hash*get}
\label{u*hash*get}

\begin{description}
\item [Name:]  u*hash*get

\item [Class:]
{\sl u*hash}\hfill(page~\pageref{u*hash})

\item [Parameters:]
\item {\sl hash-key}:  symbol

\item {\sl table}:  symbol


\item [Return-value:] 
data element of the table entry.

\item [Description:]
Retrieves data element from the hashtable. If optional
arg TABLE is supplied, use that table. Otherwise, use
the system default table.

\item [Public:]



\end{description}
\horizontalline

\subsection{u*hash*set}
\label{u*hash*set}

\begin{description}
\item [Name:]  u*hash*set

\item [Class:]
{\sl u*hash}\hfill(page~\pageref{u*hash})

\item [Parameters:]
\item {\sl hash-key}:  symbol

\item {\sl data}:  any Lisp object

\item {\sl table}:  symbol


\item [Return-value:] new data value

\item [Description:]
Stores DATA in table TABLE using HASH-KEY as key.  If
optional arg TABLE is not supplied, defaults to system
table. 

\item [Public:]



\end{description}
\horizontalline

\subsection{u*hash*rem}
\label{u*hash*rem}

\begin{description}
\item [Name:]  u*hash*rem

\item [Class:]
{\sl u*hash}\hfill(page~\pageref{u*hash})

\item [Parameters:]
\item {\sl hash-key}:  symbol

\item {\sl table}:  symbol


\item [Return-value:]  t if deletion is successful or
nil otherwise.

\item [Description:]

Removes data element corresponding to key HASH-KEY from
table TABLE. If TABLE is not supplied, defaults to
system table. 

\item [Public:]



\end{description}
\horizontalline

\subsection{u*hash*existp}
\label{u*hash*existp}

\begin{description}
\item [Name:]  u*hash*existp

\item [Class:]
{\sl u*hash}\hfill(page~\pageref{u*hash})

\item [Parameters:]
\item {\sl hash-key}:  symbol

\item {\sl table}:  symbol


\item [Return-value:] 
data value if found or nil otherwise.

\item [Description:]
Check the existence of the item with key HASH-KEY in
table TABLE. If TABLE is not supplied, defaults
to system table.

\item [Public:]




\end{description}
\horizontalline
