\chapter{Type Subsystem Tests}
\section {Class: t*field-schema}
\subsection {Operation: t*field-schema*make}
\subsubsection {Test-case scenario: normal, Field-schema-name-0}


This test case tests that a field-schema can be created in a normal fashion.
Its precondition ensures that a connection exists to the server.
Then it calls the function t*field-schema*make  with a valid name "Field Schema 0", and expects field-schema-ID (integer).
Afterwards, its postcondition tests that the attribute t*field-schema*name exists and equal to Field Schema 0, then it tests that the attribute t*field-schema*validity-fn is either nil (for the 1st test data ) or equal to symbol Validity-fn-0 (for the 2nd test data).


\noindent {\bf Result: FAILED}\
\begin {itemize}
\item 	Error-name             : system-abort
\item Error-data             : (void-variable s*sp*current-server)
\item Operation and Parameter: tf*connection ("claudia.ics.hawaii.edu")
\item Expected-return-type   : object
\item Expected-return-value  : t
\item Location               : Precondition



\end {itemize}
\subsubsection {Test-case scenario: normal, Field-schema-name-0-Validity-fn-0}


This test case tests that a field-schema can be created in a normal fashion.
Its precondition ensures that a connection exists to the server.
Then it calls the function t*field-schema*make  with a valid name "Field Schema 0" and a symbol 'Validity-fn-0, and expects field-schema-ID (integer).
Afterwards, its postcondition tests that the attribute t*field-schema*name exists and equal to Field Schema 0, then it tests that the attribute t*field-schema*validity-fn is either nil (for the 1st test data ) or equal to symbol Validity-fn-0 (for the 2nd test data).


\noindent {\bf Result: FAILED}\
\begin {itemize}
\item 	Error-name             : system-abort
\item Error-data             : (void-variable s*sp*current-server)
\item Operation and Parameter: tf*connection ("claudia.ics.hawaii.edu")
\item Expected-return-type   : object
\item Expected-return-value  : t
\item Location               : Precondition



\end {itemize}
\subsubsection {Test-case scenario: invalid-field-schema-name, Nil-data}


This test case tests that field-schema cannot be created with an invalid name.
Its precondition ensures that a connection exists to the server.
Then it calls the function t*field-schema*make  with a nil data, and expects return value error object invalid-node-name.



\noindent {\bf Result: FAILED}\
\begin {itemize}
\item 	Error-name             : system-abort
\item Error-data             : (void-variable s*sp*current-server)
\item Operation and Parameter: tf*connection ("claudia.ics.hawaii.edu")
\item Expected-return-type   : object
\item Expected-return-value  : t
\item Location               : Precondition



\end {itemize}
\subsubsection {Test-case scenario: invalid-field-schema-name, Empty-string}


This test case tests that field-schema cannot be created with an invalid name.
Its precondition ensures that a connection exists to the server.
Then it calls the function t*field-schema*make  with an empty-string, and expects return value error object invalid-node-name.



\noindent {\bf Result: FAILED}\
\begin {itemize}
\item 	Error-name             : system-abort
\item Error-data             : (void-variable s*sp*current-server)
\item Operation and Parameter: tf*connection ("claudia.ics.hawaii.edu")
\item Expected-return-type   : object
\item Expected-return-value  : t
\item Location               : Precondition



\end {itemize}
\subsubsection {Test-case scenario: invalid-field-schema-name, String-preceded-by-spaces}


This test case tests that field-schema cannot be created with an invalid name.
Its precondition ensures that a connection exists to the server.
Then it calls the function t*field-schema*make  with a string preceded by two spaces, and expects return value error object invalid-node-name.



\noindent {\bf Result: FAILED}\
\begin {itemize}
\item 	Error-name             : system-abort
\item Error-data             : (void-variable s*sp*current-server)
\item Operation and Parameter: tf*connection ("claudia.ics.hawaii.edu")
\item Expected-return-type   : object
\item Expected-return-value  : t
\item Location               : Precondition



\end {itemize}
\subsection {Operation: t*field-schema*delete}
\subsubsection {Test-case scenario: normal, Field-schema-id-1}


This test case tests that a field-schema can be deleted in a normal manner.
Its precondition ensures that a connection exists to the server, then it creates Field Schema 1.
Then it calls the function t*field-schema*delete  with a valid field-schema-id of the field-schema-1, and expects the deleted field-schema-id.
Afterwards, its postcondition tests that field schema 1 no longer exists by retrieving its name attribute.


\noindent {\bf Result: FAILED}\
\begin {itemize}
\item 	Error-name             : system-abort
\item Error-data             : (void-variable s*sp*current-server)
\item Operation and Parameter: tf*connection ("claudia.ics.hawaii.edu")
\item Expected-return-type   : object
\item Expected-return-value  : t
\item Location               : Precondition



\end {itemize}
\subsubsection {Test-case scenario: invalid-field, Non-extant-field-schema-id-1}


This test case tests that the t*field-schema*delete operation with invalid field-schema-id will fail.
Its precondition ensures that a connection exists to the server, then it creates Field Schema 1, then it deletes field-schema-1, then it creates a non type level node using s*node*make.
Then it calls the function t*field-schema*delete  with the deleted field schema id 1, and expects error invalid-field-ID.



\noindent {\bf Result: FAILED}\
\begin {itemize}
\item 	Error-name             : system-abort
\item Error-data             : (void-variable s*sp*current-server)
\item Operation and Parameter: tf*connection ("claudia.ics.hawaii.edu")
\item Expected-return-type   : object
\item Expected-return-value  : t
\item Location               : Precondition



\end {itemize}
\subsubsection {Test-case scenario: invalid-field, Non-type-level-node-id}


This test case tests that the t*field-schema*delete operation with invalid field-schema-id will fail.
Its precondition ensures that a connection exists to the server, then it creates Field Schema 1, then it deletes field-schema-1, then it creates a non type level node using s*node*make.
Then it calls the function t*field-schema*delete  with a server node-id, and expects error invalid-field-ID.



\noindent {\bf Result: FAILED}\
\begin {itemize}
\item 	Error-name             : system-abort
\item Error-data             : (void-variable s*sp*current-server)
\item Operation and Parameter: tf*connection ("claudia.ics.hawaii.edu")
\item Expected-return-type   : object
\item Expected-return-value  : t
\item Location               : Precondition



\end {itemize}
\subsection {Operation: t*field-schema*set-name}
\subsubsection {Test-case scenario: normal, Field-schema-id-1-New-field-schema-name}


This test case tests that a valid field-schema can be successfully renamed.
Its precondition ensures that a connection exists to the server, then it creates Field Schema 1.
Then it calls the function t*field-schema*set-name  with a valid field schema 1 id and a new name: Field Schema 2, and expects t.
Afterwards, its postcondition tests that the name "Field Schema 2" exists.


\noindent {\bf Result: FAILED}\
\begin {itemize}
\item 	Error-name             : system-abort
\item Error-data             : (void-variable s*sp*current-server)
\item Operation and Parameter: tf*connection ("claudia.ics.hawaii.edu")
\item Expected-return-type   : object
\item Expected-return-value  : t
\item Location               : Precondition



\end {itemize}
\subsubsection {Test-case scenario: invalid-name, Field-schema-id-1-invalid-name}


This test case tests that the operation t*field-schema*set-name will fail when given invalid name.
Its precondition ensures that a connection exists to the server, then it creates Field Schema 1.
Then it calls the function t*field-schema*set-name  with a valid field schema id 1 with invalid name containing leading tab, and expects error object invalid-node-name.



\noindent {\bf Result: FAILED}\
\begin {itemize}
\item 	Error-name             : system-abort
\item Error-data             : (void-variable s*sp*current-server)
\item Operation and Parameter: tf*connection ("claudia.ics.hawaii.edu")
\item Expected-return-type   : object
\item Expected-return-value  : t
\item Location               : Precondition



\end {itemize}
\subsubsection {Test-case scenario: invalid-field-id, Non-extant-field-schema-id-1-Valid-name}


This test case tests that the operation t*field-schema*set-name will fail when given invalid field-schema-id.
Its precondition ensures that a connection exists to the server, then it creates Field Schema 1, then it deletes field-schema-1, then it creates a non type level node using s*node*make.
Then it calls the function t*field-schema*set-name  with non extant field schema id 1 with valid name, and expects error object invalid-field-ID.



\noindent {\bf Result: FAILED}\
\begin {itemize}
\item 	Error-name             : system-abort
\item Error-data             : (void-variable s*sp*current-server)
\item Operation and Parameter: tf*connection ("claudia.ics.hawaii.edu")
\item Expected-return-type   : object
\item Expected-return-value  : t
\item Location               : Precondition



\end {itemize}
\subsubsection {Test-case scenario: invalid-field-id, Non-type-level-id-Valid-name}


This test case tests that the operation t*field-schema*set-name will fail when given invalid field-schema-id.
Its precondition ensures that a connection exists to the server, then it creates Field Schema 1, then it deletes field-schema-1, then it creates a non type level node using s*node*make.
Then it calls the function t*field-schema*set-name  with a server node id and valid name, and expects error object invalid-field-ID.



\noindent {\bf Result: FAILED}\
\begin {itemize}
\item 	Error-name             : system-abort
\item Error-data             : (void-variable s*sp*current-server)
\item Operation and Parameter: tf*connection ("claudia.ics.hawaii.edu")
\item Expected-return-type   : object
\item Expected-return-value  : t
\item Location               : Precondition



\end {itemize}
\section {Class: t*node-schema}
\subsection {Operation: t*node-schema*make}
\subsubsection {Test-case scenario: normal, Node-schema-name-0-List-of-field-schema-id-1}


This test case tests that a node schema can be created in a normal manner.
Its precondition ensures that a connection exists to the server, then it creates Field Schema 1, then it creates Field Schema 2.
Then it calls the function t*node-schema*make  with a valid name "Node Schema 0" and a field-schema-id 1, and expects a valid node schema-id (an integer).
Afterwards, its postcondition tests that the name "Node Schema 0" exists, then it tests that this node schema has either a list of field-schema-1 or field-schema-1-2.


\noindent {\bf Result: FAILED}\
\begin {itemize}
\item 	Error-name             : system-abort
\item Error-data             : (void-variable s*sp*current-server)
\item Operation and Parameter: tf*connection ("claudia.ics.hawaii.edu")
\item Expected-return-type   : object
\item Expected-return-value  : t
\item Location               : Precondition



\end {itemize}
\subsubsection {Test-case scenario: normal, Node-schema-name-0-Field-schema-id-1-2}


This test case tests that a node schema can be created in a normal manner.
Its precondition ensures that a connection exists to the server, then it creates Field Schema 1, then it creates Field Schema 2.
Then it calls the function t*node-schema*make  with a valid name "Node Schema 0" and two valid field-schema-ids, and expects a valid node schema-id (an integer).
Afterwards, its postcondition tests that the name "Node Schema 0" exists, then it tests that this node schema has either a list of field-schema-1 or field-schema-1-2.


\noindent {\bf Result: FAILED}\
\begin {itemize}
\item 	Error-name             : system-abort
\item Error-data             : (void-variable s*sp*current-server)
\item Operation and Parameter: tf*connection ("claudia.ics.hawaii.edu")
\item Expected-return-type   : object
\item Expected-return-value  : t
\item Location               : Precondition



\end {itemize}
\subsubsection {Test-case scenario: valid-name-invalid-field-ids, Node-schema-name-0-List-of-non-extant-field-schema-id-1}


This test case tests that t*node-schema*make failed when given invalid-field-ids.
Its precondition ensures that a connection exists to the server, then it creates Field Schema 1, then it deletes field-schema-1, then it creates a non type level node using s*node*make.
Then it calls the function t*node-schema*make  with a valid name "Node Schema 0" and the deleted field schema id 1, and expects error invalid-field-ID.



\noindent {\bf Result: FAILED}\
\begin {itemize}
\item 	Error-name             : system-abort
\item Error-data             : (void-variable s*sp*current-server)
\item Operation and Parameter: tf*connection ("claudia.ics.hawaii.edu")
\item Expected-return-type   : object
\item Expected-return-value  : t
\item Location               : Precondition



\end {itemize}
\subsubsection {Test-case scenario: valid-name-invalid-field-ids, Node-schema-name-0-List-of-non-type-level-id}


This test case tests that t*node-schema*make failed when given invalid-field-ids.
Its precondition ensures that a connection exists to the server, then it creates Field Schema 1, then it deletes field-schema-1, then it creates a non type level node using s*node*make.
Then it calls the function t*node-schema*make  with a valid name "Node Schema" and a server node-id, and expects error invalid-field-ID.



\noindent {\bf Result: FAILED}\
\begin {itemize}
\item 	Error-name             : system-abort
\item Error-data             : (void-variable s*sp*current-server)
\item Operation and Parameter: tf*connection ("claudia.ics.hawaii.edu")
\item Expected-return-type   : object
\item Expected-return-value  : t
\item Location               : Precondition



\end {itemize}
\subsection {Operation: t*node-schema*delete}
\subsubsection {Test-case scenario: normal, Node-schema-id-1}


This test case tests that node-schema can be deleted in a normal manner.
Its precondition ensures that a connection exists to the server, then it creates Field Schema 1, then it creates Node Schema 1 with field-schema: field-schema 1.
Then it calls the function t*node-schema*delete  with a valid schema id (Node Schema 1), and expects the deleted node-schema-id-1.
Afterwards, its postcondition tests that Node Schema 1 does not exist.


\noindent {\bf Result: FAILED}\
\begin {itemize}
\item 	Error-name             : system-abort
\item Error-data             : (void-variable s*sp*current-server)
\item Operation and Parameter: tf*connection ("claudia.ics.hawaii.edu")
\item Expected-return-type   : object
\item Expected-return-value  : t
\item Location               : Precondition



\end {itemize}
\subsubsection {Test-case scenario: invalid-node-schema-id, Non-extant-node-schema-id-1}


This test case tests that t*node-schema*delete fails when given invalid node schema id.
Its precondition ensures that a connection exists to the server, then it creates Field Schema 1, then it creates Node Schema 1 with field-schema: field-schema 1, then it deletes node schema 1, then it creates a non type level node using s*node*make.
Then it calls the function t*node-schema*delete  with a deleted node-schema-id 1, and expects error object invalid-node-schema-ID.



\noindent {\bf Result: FAILED}\
\begin {itemize}
\item 	Error-name             : system-abort
\item Error-data             : (void-variable s*sp*current-server)
\item Operation and Parameter: tf*connection ("claudia.ics.hawaii.edu")
\item Expected-return-type   : object
\item Expected-return-value  : t
\item Location               : Precondition



\end {itemize}
\subsubsection {Test-case scenario: invalid-node-schema-id, Non-type-level-node-id}


This test case tests that t*node-schema*delete fails when given invalid node schema id.
Its precondition ensures that a connection exists to the server, then it creates Field Schema 1, then it creates Node Schema 1 with field-schema: field-schema 1, then it deletes node schema 1, then it creates a non type level node using s*node*make.
Then it calls the function t*node-schema*delete  with a server node-id, and expects error object invalid-node-schema-ID.



\noindent {\bf Result: FAILED}\
\begin {itemize}
\item 	Error-name             : system-abort
\item Error-data             : (void-variable s*sp*current-server)
\item Operation and Parameter: tf*connection ("claudia.ics.hawaii.edu")
\item Expected-return-type   : object
\item Expected-return-value  : t
\item Location               : Precondition



\end {itemize}
\subsection {Operation: t*node-schema*set-name}
\subsubsection {Test-case scenario: normal, Node-schema-id-1-Node-schema-name-2}


This test case tests that node-schema can be successfully renamed.
Its precondition ensures that a connection exists to the server, then it creates Field Schema 1, then it creates Node Schema 1 with field-schema: field-schema 1.
Then it calls the function t*node-schema*set-name  with a valid node schema id 1 and new name Node Schema 2, and expects t.
Afterwards, its postcondition tests that node-schema-1 has a new name "Node Schema 2".


\noindent {\bf Result: FAILED}\
\begin {itemize}
\item 	Error-name             : system-abort
\item Error-data             : (void-variable s*sp*current-server)
\item Operation and Parameter: tf*connection ("claudia.ics.hawaii.edu")
\item Expected-return-type   : object
\item Expected-return-value  : t
\item Location               : Precondition



\end {itemize}
\subsubsection {Test-case scenario: invalid-name, Node-schema-id-1-Invalid-name}


This test case tests that the operation t*node-schema*set-name will fail when given invalid name.
Its precondition ensures that a connection exists to the server, then it creates Field Schema 1, then it creates Node Schema 1 with field-schema: field-schema 1.
Then it calls the function t*node-schema*set-name  with a valid node schema id 1 with invalid name containing leading tab, and expects error object invalid-node-name.



\noindent {\bf Result: FAILED}\
\begin {itemize}
\item 	Error-name             : system-abort
\item Error-data             : (void-variable s*sp*current-server)
\item Operation and Parameter: tf*connection ("claudia.ics.hawaii.edu")
\item Expected-return-type   : object
\item Expected-return-value  : t
\item Location               : Precondition



\end {itemize}
\subsubsection {Test-case scenario: invalid-node-id, Non-extant-node-schema-id-1-Valid-name}


This test case tests that the operation t*node-schema*set-name will fail when given invalid node-schema-id.
Its precondition ensures that a connection exists to the server, then it creates Field Schema 1, then it creates Node Schema 1 with field-schema: field-schema 1, then it deletes node schema 1, then it creates a non type level node using s*node*make.
Then it calls the function t*node-schema*set-name  with non extant node schema id 1 with valid name, and expects error object invalid-node-ID.



\noindent {\bf Result: FAILED}\
\begin {itemize}
\item 	Error-name             : system-abort
\item Error-data             : (void-variable s*sp*current-server)
\item Operation and Parameter: tf*connection ("claudia.ics.hawaii.edu")
\item Expected-return-type   : object
\item Expected-return-value  : t
\item Location               : Precondition



\end {itemize}
\subsubsection {Test-case scenario: invalid-node-id, Non-type-level-id-Valid-name}


This test case tests that the operation t*node-schema*set-name will fail when given invalid node-schema-id.
Its precondition ensures that a connection exists to the server, then it creates Field Schema 1, then it creates Node Schema 1 with field-schema: field-schema 1, then it deletes node schema 1, then it creates a non type level node using s*node*make.
Then it calls the function t*node-schema*set-name  with a server node id and valid name, and expects error object invalid-node-ID.



\noindent {\bf Result: FAILED}\
\begin {itemize}
\item 	Error-name             : system-abort
\item Error-data             : (void-variable s*sp*current-server)
\item Operation and Parameter: tf*connection ("claudia.ics.hawaii.edu")
\item Expected-return-type   : object
\item Expected-return-value  : t
\item Location               : Precondition



\end {itemize}
\subsection {Operation: t*node-schema*delete-fields}
\subsubsection {Test-case scenario: normal, Node-schema-id-3-List-of-field-schema-id-2}


This test case tests that an existing field can be successfully deleted from a valid node schema.
Its precondition ensures that a connection exists to the server, then it creates Field Schema 1, then it creates Field Schema 2, then it creates Node Schema 3 with field-schema: field-schema-1 and field-schema-2.
Then it calls the function t*node-schema*delete-fields  with a valid node schema id 3 and the list field schema id 2, and expects the deleted field-schema-ids.
Afterwards, its postcondition tests that node schema 3 contains field-schema-1 only.


\noindent {\bf Result: FAILED}\
\begin {itemize}
\item 	Error-name             : system-abort
\item Error-data             : (void-variable s*sp*current-server)
\item Operation and Parameter: tf*connection ("claudia.ics.hawaii.edu")
\item Expected-return-type   : object
\item Expected-return-value  : t
\item Location               : Precondition



\end {itemize}
\subsubsection {Test-case scenario: invalid-node-schema-id, Non-extant-node-schema-id-1-List-of-field-schema-id-1}


This test case tests that t*node-schema*delete-fields with invalid node schema id will result in error.
Its precondition ensures that a connection exists to the server, then it creates Field Schema 1, then it creates Node Schema 1 with field-schema: field-schema 1, then it deletes node schema 1, then it creates a non type level node using s*node*make.
Then it calls the function t*node-schema*delete-fields  with a deleted node-schema-id 1 with a valid list of field schema id 1, and expects error object invalid-node-schema-ID.



\noindent {\bf Result: FAILED}\
\begin {itemize}
\item 	Error-name             : system-abort
\item Error-data             : (void-variable s*sp*current-server)
\item Operation and Parameter: tf*connection ("claudia.ics.hawaii.edu")
\item Expected-return-type   : object
\item Expected-return-value  : t
\item Location               : Precondition



\end {itemize}
\subsubsection {Test-case scenario: invalid-node-schema-id, Non-type-level-id-List-of-field-schema-id-1}


This test case tests that t*node-schema*delete-fields with invalid node schema id will result in error.
Its precondition ensures that a connection exists to the server, then it creates Field Schema 1, then it creates Node Schema 1 with field-schema: field-schema 1, then it deletes node schema 1, then it creates a non type level node using s*node*make.
Then it calls the function t*node-schema*delete-fields  with non type level id but valid list of field-schema-id 1, and expects error object invalid-node-schema-ID.



\noindent {\bf Result: FAILED}\
\begin {itemize}
\item 	Error-name             : system-abort
\item Error-data             : (void-variable s*sp*current-server)
\item Operation and Parameter: tf*connection ("claudia.ics.hawaii.edu")
\item Expected-return-type   : object
\item Expected-return-value  : t
\item Location               : Precondition



\end {itemize}
\subsubsection {Test-case scenario: invalid-field-schema-id, Node-schema-id-1-List-of-field-schema-id-2}


This test case tests that t*node-schema*delete-fields will fail given invalid field schema id.
Its precondition ensures that a connection exists to the server, then it creates Field Schema 1, then it creates Node Schema 1 with field-schema: field-schema 1, then it creates Field Schema 2, then it creates Node Schema 2 with field-schema: field-schema 2, then it creates a non type level node using s*node*make.
Then it calls the function t*node-schema*delete-fields  with a valid node schema id 1 and a valid field schema id 2, and expects error object invalid-node-schema-ID.



\noindent {\bf Result: FAILED}\
\begin {itemize}
\item 	Error-name             : system-abort
\item Error-data             : (void-variable s*sp*current-server)
\item Operation and Parameter: tf*connection ("claudia.ics.hawaii.edu")
\item Expected-return-type   : object
\item Expected-return-value  : t
\item Location               : Precondition



\end {itemize}
\subsubsection {Test-case scenario: invalid-field-schema-id, Node-schema-id-1-Non-type-level-id}


This test case tests that t*node-schema*delete-fields will fail given invalid field schema id.
Its precondition ensures that a connection exists to the server, then it creates Field Schema 1, then it creates Node Schema 1 with field-schema: field-schema 1, then it creates Field Schema 2, then it creates Node Schema 2 with field-schema: field-schema 2, then it creates a non type level node using s*node*make.
Then it calls the function t*node-schema*delete-fields  with the valid node schema id 1 and non type level id, and expects error object invalid-node-schema-ID.



\noindent {\bf Result: FAILED}\
\begin {itemize}
\item 	Error-name             : system-abort
\item Error-data             : (void-variable s*sp*current-server)
\item Operation and Parameter: tf*connection ("claudia.ics.hawaii.edu")
\item Expected-return-type   : object
\item Expected-return-value  : t
\item Location               : Precondition



\end {itemize}
\subsection {Operation: t*node-schema*add-fields}
\subsubsection {Test-case scenario: normal, Node-schema-id-1-List-of-field-schema-id-2}


This test case tests that a field can be added to the existing node schema in a normal manner.
Its precondition ensures that a connection exists to the server, then it creates Field Schema 1, then it creates Node Schema 1 with field-schema: field-schema 1, then it creates Field Schema 2.
Then it calls the function t*node-schema*add-fields  with a valid node schema id 1 and a valid field schema id 2, and expects the updated field-schema-ids which is a list of field schema 1 and 2.



\noindent {\bf Result: FAILED}\
\begin {itemize}
\item 	Error-name             : system-abort
\item Error-data             : (void-variable s*sp*current-server)
\item Operation and Parameter: tf*connection ("claudia.ics.hawaii.edu")
\item Expected-return-type   : object
\item Expected-return-value  : t
\item Location               : Precondition



\end {itemize}
\subsubsection {Test-case scenario: invalid-field-schema, Node-schema-id-1-List-of-non-extant-field-schema-id-1}


This test case tests that t*node-schema*add-fields fails when given invalid field schema id.
Its precondition ensures that a connection exists to the server, then it creates Field Schema 1, then it creates Node Schema 1 with field-schema: field-schema 1, then it deletes node schema 1, then it creates a non type level node using s*node*make.
Then it calls the function t*node-schema*add-fields  with a valid node schema id 1 and the deleted field schema id 1, and expects error object invalid-node-schema-ID.



\noindent {\bf Result: FAILED}\
\begin {itemize}
\item 	Error-name             : system-abort
\item Error-data             : (void-variable s*sp*current-server)
\item Operation and Parameter: tf*connection ("claudia.ics.hawaii.edu")
\item Expected-return-type   : object
\item Expected-return-value  : t
\item Location               : Precondition



\end {itemize}
\subsubsection {Test-case scenario: invalid-field-schema, Node-schema-id-1-List-of-non-type-level-id}


This test case tests that t*node-schema*add-fields fails when given invalid field schema id.
Its precondition ensures that a connection exists to the server, then it creates Field Schema 1, then it creates Node Schema 1 with field-schema: field-schema 1, then it deletes node schema 1, then it creates a non type level node using s*node*make.
Then it calls the function t*node-schema*add-fields  with a valid node schema id 1 and non type level node id, and expects error object invalid-node-schema-ID.



\noindent {\bf Result: FAILED}\
\begin {itemize}
\item 	Error-name             : system-abort
\item Error-data             : (void-variable s*sp*current-server)
\item Operation and Parameter: tf*connection ("claudia.ics.hawaii.edu")
\item Expected-return-type   : object
\item Expected-return-value  : t
\item Location               : Precondition



\end {itemize}
\subsection {Operation: t*node-schema*instantiate}
\subsubsection {Test-case scenario: normal, Node-instance-name-0-Node-schema-id-1}


This test case tests that a valid node schema can be instantiated in a normal manner .
Its precondition ensures that a connection exists to the server, then it creates Field Schema 1, then it creates Node Schema 1 with field-schema: field-schema 1.
Then it calls the function t*node-schema*instantiate  with a valid node instance name "Node Instance 0" and a valid node schema id 1, and expects a valid node-id (an integer).
Afterwards, its postcondition tests that node-schema-1 is a valid attribute of node instance 1, then it tests that field-schema-1 is a valid attribute of node instance 1.


\noindent {\bf Result: FAILED}\
\begin {itemize}
\item 	Error-name             : system-abort
\item Error-data             : (void-variable s*sp*current-server)
\item Operation and Parameter: tf*connection ("claudia.ics.hawaii.edu")
\item Expected-return-type   : object
\item Expected-return-value  : t
\item Location               : Precondition



\end {itemize}
\subsubsection {Test-case scenario: invalid-schema-id, Non-extant-node-schema-id-1}


This test case tests that node schema cannot be instantiated given invalid schema id.
Its precondition ensures that a connection exists to the server, then it creates Field Schema 1, then it creates Node Schema 1 with field-schema: field-schema 1, then it deletes node schema 1.
Then it calls the function t*node-schema*instantiate  with a deleted node-schema-id 1, and expects error object invalid-node-schema-ID.



\noindent {\bf Result: FAILED}\
\begin {itemize}
\item 	Error-name             : system-abort
\item Error-data             : (void-variable s*sp*current-server)
\item Operation and Parameter: tf*connection ("claudia.ics.hawaii.edu")
\item Expected-return-type   : object
\item Expected-return-value  : t
\item Location               : Precondition



\end {itemize}
\subsubsection {Test-case scenario: invalid-schema-id, Non-type-level-node-id}


This test case tests that node schema cannot be instantiated given invalid schema id.
Its precondition ensures that a connection exists to the server, then it creates Field Schema 1, then it creates Node Schema 1 with field-schema: field-schema 1, then it deletes node schema 1.
Then it calls the function t*node-schema*instantiate  with a server node-id, and expects error object invalid-node-schema-ID.



\noindent {\bf Result: FAILED}\
\begin {itemize}
\item 	Error-name             : system-abort
\item Error-data             : (void-variable s*sp*current-server)
\item Operation and Parameter: tf*connection ("claudia.ics.hawaii.edu")
\item Expected-return-type   : object
\item Expected-return-value  : t
\item Location               : Precondition



\end {itemize}
\section {Class: t*link-schema}
\subsection {Operation: t*link-schema*make}
\subsubsection {Test-case scenario: normal, Link-schema-name-0-Constraint-exp-1}


This test case tests that a link schema can be created in a normal manner.
Its precondition ensures that a connection exists to the server, then it creates Field Schema 1, then it creates Node Schema 1 with field-schema: field-schema 1, then it creates Field Schema 2, then it creates Node Schema 2 with field-schema: field-schema 2.
Then it calls the function t*link-schema*make  with a valid name "Link Schema 0" and both to and from constraint-exps are t, and expects a valid link schema-id (integer).
Afterwards, its postcondition tests that the attribute t*link-schema*name is equal to Link Schema 0, then it tests that the attribute t*link-schema*to-nodes is valid for all constraint-exp 1 2 3 and 4, then it tests that the attribute t*link-schema*from-nodes is valid for all constraint-exp 1, 2, 3 and 4.


\noindent {\bf Result: FAILED}\
\begin {itemize}
\item 	Error-name             : system-abort
\item Error-data             : (void-variable s*sp*current-server)
\item Operation and Parameter: tf*connection ("claudia.ics.hawaii.edu")
\item Expected-return-type   : object
\item Expected-return-value  : t
\item Location               : Precondition



\end {itemize}
\subsubsection {Test-case scenario: normal, Link-schema-name-0-Constraint-exp-2}


This test case tests that a link schema can be created in a normal manner.
Its precondition ensures that a connection exists to the server, then it creates Field Schema 1, then it creates Node Schema 1 with field-schema: field-schema 1, then it creates Field Schema 2, then it creates Node Schema 2 with field-schema: field-schema 2.
Then it calls the function t*link-schema*make  with a valid name "Link Schema 0" and from-const-exp is node-schema-id-1 and to-const-exp is node-schema-id-2, and expects a valid link schema-id (integer).
Afterwards, its postcondition tests that the attribute t*link-schema*name is equal to Link Schema 0, then it tests that the attribute t*link-schema*to-nodes is valid for all constraint-exp 1 2 3 and 4, then it tests that the attribute t*link-schema*from-nodes is valid for all constraint-exp 1, 2, 3 and 4.


\noindent {\bf Result: FAILED}\
\begin {itemize}
\item 	Error-name             : system-abort
\item Error-data             : (void-variable s*sp*current-server)
\item Operation and Parameter: tf*connection ("claudia.ics.hawaii.edu")
\item Expected-return-type   : object
\item Expected-return-value  : t
\item Location               : Precondition



\end {itemize}
\subsubsection {Test-case scenario: normal, Link-schema-name-0-Constraint-exp-3}


This test case tests that a link schema can be created in a normal manner.
Its precondition ensures that a connection exists to the server, then it creates Field Schema 1, then it creates Node Schema 1 with field-schema: field-schema 1, then it creates Field Schema 2, then it creates Node Schema 2 with field-schema: field-schema 2.
Then it calls the function t*link-schema*make  with a valid name "Link Schema 0" and from-const-exp is t and to-const-exp is node-schema-id-1, and expects a valid link schema-id (integer).
Afterwards, its postcondition tests that the attribute t*link-schema*name is equal to Link Schema 0, then it tests that the attribute t*link-schema*to-nodes is valid for all constraint-exp 1 2 3 and 4, then it tests that the attribute t*link-schema*from-nodes is valid for all constraint-exp 1, 2, 3 and 4.


\noindent {\bf Result: FAILED}\
\begin {itemize}
\item 	Error-name             : system-abort
\item Error-data             : (void-variable s*sp*current-server)
\item Operation and Parameter: tf*connection ("claudia.ics.hawaii.edu")
\item Expected-return-type   : object
\item Expected-return-value  : t
\item Location               : Precondition



\end {itemize}
\subsubsection {Test-case scenario: normal, Link-schema-name-0-Constraint-exp-4}


This test case tests that a link schema can be created in a normal manner.
Its precondition ensures that a connection exists to the server, then it creates Field Schema 1, then it creates Node Schema 1 with field-schema: field-schema 1, then it creates Field Schema 2, then it creates Node Schema 2 with field-schema: field-schema 2.
Then it calls the function t*link-schema*make  with a valid name "Link Schema 0" and from-const-exp is t and to-const-exp is a list of node-schema-id-1 and node-schema-id-2, and expects a valid link schema-id (integer).
Afterwards, its postcondition tests that the attribute t*link-schema*name is equal to Link Schema 0, then it tests that the attribute t*link-schema*to-nodes is valid for all constraint-exp 1 2 3 and 4, then it tests that the attribute t*link-schema*from-nodes is valid for all constraint-exp 1, 2, 3 and 4.


\noindent {\bf Result: FAILED}\
\begin {itemize}
\item 	Error-name             : system-abort
\item Error-data             : (void-variable s*sp*current-server)
\item Operation and Parameter: tf*connection ("claudia.ics.hawaii.edu")
\item Expected-return-type   : object
\item Expected-return-value  : t
\item Location               : Precondition



\end {itemize}
\subsubsection {Test-case scenario: invalid-constraint-exp, Link-schema-name-0-Non-extant-from-const-1-Valid-to-const}


This test case tests that t*link-schema*make fails when given invalid constraint expressions.
Its precondition ensures that a connection exists to the server, then it creates Field Schema 1, then it creates Node Schema 1 with field-schema: field-schema 1, then it deletes node schema 1.
Then it calls the function t*link-schema*make  with valid name, non extant node-schema-id-1 as the from-constraint-exp, and valid to-constraint-exp (t), and expects error invalid-contraint-exp.



\noindent {\bf Result: FAILED}\
\begin {itemize}
\item 	Error-name             : system-abort
\item Error-data             : (void-variable s*sp*current-server)
\item Operation and Parameter: tf*connection ("claudia.ics.hawaii.edu")
\item Expected-return-type   : object
\item Expected-return-value  : t
\item Location               : Precondition



\end {itemize}
\subsubsection {Test-case scenario: invalid-constraint-exp, Link-schema-name-0-Valid-from-const-Non-extant-to-const-1}


This test case tests that t*link-schema*make fails when given invalid constraint expressions.
Its precondition ensures that a connection exists to the server, then it creates Field Schema 1, then it creates Node Schema 1 with field-schema: field-schema 1, then it deletes node schema 1.
Then it calls the function t*link-schema*make  with a valid link schema name 1, a valid from-constraint exp (t), and non extant node-schema-id-1 as the to-constraint-exp, and expects error invalid-contraint-exp.



\noindent {\bf Result: FAILED}\
\begin {itemize}
\item 	Error-name             : system-abort
\item Error-data             : (void-variable s*sp*current-server)
\item Operation and Parameter: tf*connection ("claudia.ics.hawaii.edu")
\item Expected-return-type   : object
\item Expected-return-value  : t
\item Location               : Precondition



\end {itemize}
\subsubsection {Test-case scenario: invalid-constraint-exp, Link-schema-name-0-Non-type-level-from-const}


This test case tests that t*link-schema*make fails when given invalid constraint expressions.
Its precondition ensures that a connection exists to the server, then it creates Field Schema 1, then it creates Node Schema 1 with field-schema: field-schema 1, then it deletes node schema 1.
Then it calls the function t*link-schema*make  with a valid link schema name 0, invalid type from-constraint-exp (t), valid to-constraint-expression, and expects error invalid-contraint-exp.



\noindent {\bf Result: FAILED}\
\begin {itemize}
\item 	Error-name             : system-abort
\item Error-data             : (void-variable s*sp*current-server)
\item Operation and Parameter: tf*connection ("claudia.ics.hawaii.edu")
\item Expected-return-type   : object
\item Expected-return-value  : t
\item Location               : Precondition



\end {itemize}
\subsection {Operation: t*link-schema*add-to-node}
\subsubsection {Test-case scenario: normal, Link-schema-id-2-Node-schema-id-2}


This test case tests that a valid node-schema-id can be added to the to-node constraint of the link-schema.
Its precondition ensures that a connection exists to the server, then it creates Field Schema 1, then it creates Node Schema 1 with field-schema: field-schema 1, then it creates Link Schema 2 with from-constraint-exp equalto t and to-constraint-exp equal to node-schema-id-1., then it creates Field Schema 2, then it creates Node Schema 2 with field-schema: field-schema 2.
Then it calls the function t*link-schema*add-to-node  with a valid link-schema id 1 and a valid node-schema id 2, and expects updated to-node-constraint expression.



\noindent {\bf Result: FAILED}\
\begin {itemize}
\item 	Error-name             : system-abort
\item Error-data             : (void-variable s*sp*current-server)
\item Operation and Parameter: tf*connection ("claudia.ics.hawaii.edu")
\item Expected-return-type   : object
\item Expected-return-value  : t
\item Location               : Precondition



\end {itemize}
\subsubsection {Test-case scenario: normal, Link-schema-id-2-and-t}


This test case tests that a valid node-schema-id can be added to the to-node constraint of the link-schema.
Its precondition ensures that a connection exists to the server, then it creates Field Schema 1, then it creates Node Schema 1 with field-schema: field-schema 1, then it creates Link Schema 2 with from-constraint-exp equalto t and to-constraint-exp equal to node-schema-id-1., then it creates Field Schema 2, then it creates Node Schema 2 with field-schema: field-schema 2.
Then it calls the function t*link-schema*add-to-node  with a valid link-schema id 1 and t, and expects updated to-node-constraint expression.



\noindent {\bf Result: FAILED}\
\begin {itemize}
\item 	Error-name             : system-abort
\item Error-data             : (void-variable s*sp*current-server)
\item Operation and Parameter: tf*connection ("claudia.ics.hawaii.edu")
\item Expected-return-type   : object
\item Expected-return-value  : t
\item Location               : Precondition



\end {itemize}
\subsubsection {Test-case scenario: invalid-node-schema, Link-schema-id-1-Non-extant-node-schema-id-1}


This test case tests that t*link-schema*add-to-node fails when given invalid to-node constraint expression.
Its precondition ensures that a connection exists to the server, then it creates Link Schema 1 with both from-constraint-exp and to-constraint-exp equals to t, then it creates Field Schema 1, then it creates Node Schema 1 with field-schema: field-schema 1, then it deletes node schema 1, then it creates a non type level node using s*node*make.
Then it calls the function t*link-schema*add-to-node  with a valid link-schema-id 1 and non-extant node-schema-id 1, and expects error object invalid-node-schema-ID.



\noindent {\bf Result: FAILED}\
\begin {itemize}
\item 	Error-name             : system-abort
\item Error-data             : (void-variable s*sp*current-server)
\item Operation and Parameter: tf*connection ("claudia.ics.hawaii.edu")
\item Expected-return-type   : object
\item Expected-return-value  : t
\item Location               : Precondition



\end {itemize}
\subsubsection {Test-case scenario: invalid-node-schema, Link-schema-id-1-Non-type-level-id}


This test case tests that t*link-schema*add-to-node fails when given invalid to-node constraint expression.
Its precondition ensures that a connection exists to the server, then it creates Link Schema 1 with both from-constraint-exp and to-constraint-exp equals to t, then it creates Field Schema 1, then it creates Node Schema 1 with field-schema: field-schema 1, then it deletes node schema 1, then it creates a non type level node using s*node*make.
Then it calls the function t*link-schema*add-to-node  with a valid link-schema-id 1 and non type-level-node id, and expects error object invalid-node-schema-ID.



\noindent {\bf Result: FAILED}\
\begin {itemize}
\item 	Error-name             : system-abort
\item Error-data             : (void-variable s*sp*current-server)
\item Operation and Parameter: tf*connection ("claudia.ics.hawaii.edu")
\item Expected-return-type   : object
\item Expected-return-value  : t
\item Location               : Precondition



\end {itemize}
\subsection {Operation: t*link-schema*delete-from-node}
\subsubsection {Test-case scenario: normal, Link-schema-id-5-Node-schema-id-1}


This test case tests that a valid node-schema-id can be deleted from the from-node constraint of the link-schema.
Its precondition ensures that a connection exists to the server, then it creates Field Schema 1, then it creates Node Schema 1 with field-schema: field-schema 1, then it creates Field Schema 2, then it creates Node Schema 2 with field-schema: field-schema 2, then it creates Link Schema 4 with both  from-constraint-exp and to-constraint-exp equal to node-schema-id-1., then it creates Link Schema 5 with from-constraint-exp is equal to the list of node-schema-1 and node-schema-2 and the to-constraint-exp equal to t.
Then it calls the function t*link-schema*delete-from-node  with a valid link schema id 5 and node schema id 1, and expects updated from-node-constraint expression.



\noindent {\bf Result: FAILED}\
\begin {itemize}
\item 	Error-name             : system-abort
\item Error-data             : (void-variable s*sp*current-server)
\item Operation and Parameter: tf*connection ("claudia.ics.hawaii.edu")
\item Expected-return-type   : object
\item Expected-return-value  : t
\item Location               : Precondition



\end {itemize}
\subsubsection {Test-case scenario: normal, Link-schema-id-4-Node-schema-id-1}


This test case tests that a valid node-schema-id can be deleted from the from-node constraint of the link-schema.
Its precondition ensures that a connection exists to the server, then it creates Field Schema 1, then it creates Node Schema 1 with field-schema: field-schema 1, then it creates Field Schema 2, then it creates Node Schema 2 with field-schema: field-schema 2, then it creates Link Schema 4 with both  from-constraint-exp and to-constraint-exp equal to node-schema-id-1., then it creates Link Schema 5 with from-constraint-exp is equal to the list of node-schema-1 and node-schema-2 and the to-constraint-exp equal to t.
Then it calls the function t*link-schema*delete-from-node  with a valid link schema id 4 and node schema id 1, and expects updated from-node-constraint expression.



\noindent {\bf Result: FAILED}\
\begin {itemize}
\item 	Error-name             : system-abort
\item Error-data             : (void-variable s*sp*current-server)
\item Operation and Parameter: tf*connection ("claudia.ics.hawaii.edu")
\item Expected-return-type   : object
\item Expected-return-value  : t
\item Location               : Precondition



\end {itemize}
\subsubsection {Test-case scenario: invalid-node-schema, Link-schema-id-3-Node-schema-id-2}


This test case tests that t*link-schema*delete-from-node fails when given invalid to-node constraint expression.
Its precondition ensures that a connection exists to the server, then it creates Field Schema 1, then it creates Node Schema 1 with field-schema: field-schema 1, then it creates Link Schema 1 with from-constraint-exp equalto node-schema-1 and to-constraint-exp equal to t, then it creates Field Schema 1, then it creates Node Schema 2 with field-schema: field-schema 2, then it creates a non type level node using s*node*make.
Then it calls the function t*link-schema*add-to-node  with a valid link schema id 3 and a valid node schema id 2 but node-schema-id-2 is not present in link-schema-3, and expects error object invalid-node-schema-ID.



\noindent {\bf Result: FAILED}\
\begin {itemize}
\item 	Error-name             : system-abort
\item Error-data             : (void-variable s*sp*current-server)
\item Operation and Parameter: tf*connection ("claudia.ics.hawaii.edu")
\item Expected-return-type   : object
\item Expected-return-value  : t
\item Location               : Precondition



\end {itemize}
\subsubsection {Test-case scenario: invalid-node-schema, Link-schema-id-3-Non-type-level-id}


This test case tests that t*link-schema*delete-from-node fails when given invalid to-node constraint expression.
Its precondition ensures that a connection exists to the server, then it creates Field Schema 1, then it creates Node Schema 1 with field-schema: field-schema 1, then it creates Link Schema 1 with from-constraint-exp equalto node-schema-1 and to-constraint-exp equal to t, then it creates Field Schema 1, then it creates Node Schema 2 with field-schema: field-schema 2, then it creates a non type level node using s*node*make.
Then it calls the function t*link-schema*add-to-node  with a valid link schema id 3 and non-type-level node id, and expects error object invalid-node-schema-ID.



\noindent {\bf Result: FAILED}\
\begin {itemize}
\item 	Error-name             : system-abort
\item Error-data             : (void-variable s*sp*current-server)
\item Operation and Parameter: tf*connection ("claudia.ics.hawaii.edu")
\item Expected-return-type   : object
\item Expected-return-value  : t
\item Location               : Precondition



\end {itemize}
\subsection {Operation: t*link-schema*set-name}
\subsubsection {Test-case scenario: normal, Link-schema-id-1-Link-schema-name-2}


This test case tests that link-schema can be successfully renamed.
Its precondition ensures that a connection exists to the server, then it creates Field Schema 1, then it creates Node Schema 1 with field-schema: field-schema 1, then it creates Link Schema 1 with both from-constraint-exp and to-constraint-exp equals to t.
Then it calls the function t*link-schema*set-name  with a valid link schema id 1 and a valid name Link Schema 2, and expects t.
Afterwards, its postcondition tests that the attribute t*link-schema*name of link-schema 1 is equal to Link Schema 2.


\noindent {\bf Result: FAILED}\
\begin {itemize}
\item 	Error-name             : system-abort
\item Error-data             : (void-variable s*sp*current-server)
\item Operation and Parameter: tf*connection ("claudia.ics.hawaii.edu")
\item Expected-return-type   : object
\item Expected-return-value  : t
\item Location               : Precondition



\end {itemize}
\subsubsection {Test-case scenario: invalid-name, Link-schema-id-1-Invalid-name}


This test case tests that the operation t*link-schema*set-name will fail when given invalid name.
Its precondition ensures that a connection exists to the server, then it creates Field Schema 1, then it creates Node Schema 1 with field-schema: field-schema 1, then it creates Link Schema 1 with both from-constraint-exp and to-constraint-exp equals to t.
Then it calls the function t*link-schema*set-name  with a valid link schema id 1 and a too long name string (31 chars), and expects error object invalid-node-name.



\noindent {\bf Result: FAILED}\
\begin {itemize}
\item 	Error-name             : system-abort
\item Error-data             : (void-variable s*sp*current-server)
\item Operation and Parameter: tf*connection ("claudia.ics.hawaii.edu")
\item Expected-return-type   : object
\item Expected-return-value  : t
\item Location               : Precondition



\end {itemize}
\subsubsection {Test-case scenario: invalid-link-schema-id, Non-extant-link-schema-id-1-Valid-name}


This test case tests that the operation t*link-schema*set-name will fail when given invalid link-schema-id.
Its precondition ensures that a connection exists to the server, then it creates Link Schema 1 with both from-constraint-exp and to-constraint-exp equals to t, then it deletes Link Schema 1, then it creates a non type level node using s*node*make.
Then it calls the function t*link-schema*set-name  with non extant node schema id 1 with valid name, and expects error object invalid link-schema-id.



\noindent {\bf Result: FAILED}\
\begin {itemize}
\item 	Error-name             : system-abort
\item Error-data             : (void-variable s*sp*current-server)
\item Operation and Parameter: tf*connection ("claudia.ics.hawaii.edu")
\item Expected-return-type   : object
\item Expected-return-value  : t
\item Location               : Precondition



\end {itemize}
\subsubsection {Test-case scenario: invalid-link-schema-id, Non-type-level-id-Valid-name}


This test case tests that the operation t*link-schema*set-name will fail when given invalid link-schema-id.
Its precondition ensures that a connection exists to the server, then it creates Link Schema 1 with both from-constraint-exp and to-constraint-exp equals to t, then it deletes Link Schema 1, then it creates a non type level node using s*node*make.
Then it calls the function t*link-schema*set-name  with a server node id and valid name, and expects error object invalid link-schema-id.



\noindent {\bf Result: FAILED}\
\begin {itemize}
\item 	Error-name             : system-abort
\item Error-data             : (void-variable s*sp*current-server)
\item Operation and Parameter: tf*connection ("claudia.ics.hawaii.edu")
\item Expected-return-type   : object
\item Expected-return-value  : t
\item Location               : Precondition



\end {itemize}
\subsection {Operation: t*link-schema*delete}
\subsubsection {Test-case scenario: normal, Link-schema-id-1}


This test case tests that link-schema can be deleted in a normal manner.
Its precondition ensures that a connection exists to the server, then it creates Link Schema 1 with both from-constraint-exp and to-constraint-exp equals to t.
Then it calls the function t*link-schema*delete  with a valid link schema id 1, and expects the deleted link-schema-id which is an integer.
Afterwards, its postcondition checks that Link Schema 1 does not exist.


\noindent {\bf Result: FAILED}\
\begin {itemize}
\item 	Error-name             : system-abort
\item Error-data             : (void-variable s*sp*current-server)
\item Operation and Parameter: tf*connection ("claudia.ics.hawaii.edu")
\item Expected-return-type   : object
\item Expected-return-value  : t
\item Location               : Precondition



\end {itemize}
\subsubsection {Test-case scenario: invalid-link-schema-id, Non-extant-link-schema-id-1}


This test case tests that invalid deletion on link-schema results in error object.
Its precondition ensures that a connection exists to the server, then it creates Link Schema 1 with both from-constraint-exp and to-constraint-exp equals to t, then it deletes Link Schema 1, then it creates a non type level node using s*node*make.
Then it calls the function t*link-schema*delete  with a deleted link-schema-id, and expects error object invalid-link-schema-ID.



\noindent {\bf Result: FAILED}\
\begin {itemize}
\item 	Error-name             : system-abort
\item Error-data             : (void-variable s*sp*current-server)
\item Operation and Parameter: tf*connection ("claudia.ics.hawaii.edu")
\item Expected-return-type   : object
\item Expected-return-value  : t
\item Location               : Precondition



\end {itemize}
\subsubsection {Test-case scenario: invalid-link-schema-id, Non-type-level-node-id}


This test case tests that invalid deletion on link-schema results in error object.
Its precondition ensures that a connection exists to the server, then it creates Link Schema 1 with both from-constraint-exp and to-constraint-exp equals to t, then it deletes Link Schema 1, then it creates a non type level node using s*node*make.
Then it calls the function t*link-schema*delete  with a server node-id, and expects error object invalid-link-schema-ID.



\noindent {\bf Result: FAILED}\
\begin {itemize}
\item 	Error-name             : system-abort
\item Error-data             : (void-variable s*sp*current-server)
\item Operation and Parameter: tf*connection ("claudia.ics.hawaii.edu")
\item Expected-return-type   : object
\item Expected-return-value  : t
\item Location               : Precondition



\end {itemize}
\subsection {Operation: t*link-schema*instantiate}
\subsubsection {Test-case scenario: normal-no-constraint, Link-schema-id-1-Link-name-0-Node-instance-1-Node-instance-2}


This test case tests that a link can be instantiated from a valid link schema without link expression constraint in a normal manner..
Its precondition ensures that a connection exists to the server, then it creates Link Schema 1 with both from-constraint-exp and to-constraint-exp equals to t, then it creates Field Schema 1, then it creates Node Schema 1 with field-schema: field-schema 1, then it creates a valid "Node Instance 1" from instantiation of node-schema-1 containing field-schema-1, then it creates Field Schema 2, then it creates Node Schema 2 with field-schema: field-schema 2, then it creates a valid "Node Instance 2" from instantiation of node-schema-2 containing field-schema-2.
Then it calls the function t*link-schema*instantiate  with a valid link schema id 1, a valid link name 0, a valid node instance 1 and a valid node instance 2, and expects a valid link-id (an integer).
Afterwards, its postcondition tests the attribute link-schema-ID of this link-instance is equal to link-schema-id-1, then it tests that the name of the link is "Link 0", then it tests that from-node is pointing to node-instance-1, then it tests that to-node is pointing to node-instance-2.


\noindent {\bf Result: FAILED}\
\begin {itemize}
\item 	Error-name             : system-abort
\item Error-data             : (void-variable s*sp*current-server)
\item Operation and Parameter: tf*connection ("claudia.ics.hawaii.edu")
\item Expected-return-type   : object
\item Expected-return-value  : t
\item Location               : Precondition



\end {itemize}
\subsubsection {Test-case scenario: normal-with-constraint, Link-schema-id-4-Node-instance-id-1-Node-instance-id-1a}


This test case tests that a link can be instantiated from a valid link schema with link expression constraint in a normal manner..
Its precondition ensures that a connection exists to the server, then it creates Field Schema 1, then it creates Node Schema 1 with field-schema: field-schema 1, then it creates Link Schema 4 with both  from-constraint-exp and to-constraint-exp equal to node-schema-id-1., then it creates a valid "Node Instance 1" from instantiation of node-schema-1 containing field-schema-1, then it creates a valid "Node Instance 1a" from instantiation of node-schema-1 containing field-schema-1.
Then it calls the function t*link-schema*instantiate  with a valid link-schema-id-4 with from-node-id is node-instance-1 and to-node-id is node-instance-1a, and expects a valid link-id (an integer).
Afterwards, its postcondition tests the attribute link-schema-ID of this link-instance is equal to link-schema-id-4, then it tests that from-node is pointing to node-instance-1, then it tests that to-node is pointing to node-instance-1a.


\noindent {\bf Result: FAILED}\
\begin {itemize}
\item 	Error-name             : system-abort
\item Error-data             : (void-variable s*sp*current-server)
\item Operation and Parameter: tf*connection ("claudia.ics.hawaii.edu")
\item Expected-return-type   : object
\item Expected-return-value  : t
\item Location               : Precondition



\end {itemize}
\subsubsection {Test-case scenario: invalid-link-schema-id, Non-extant-link-schema-id-1-Node-instance-id-2-Node-instance-id-2a}


This test case tests that instantiating link schema with invalid schema id will result in error.
Its precondition ensures that a connection exists to the server, then it creates Link Schema 1 with both from-constraint-exp and to-constraint-exp equals to t, then it deletes Link Schema 1, then it creates Field Schema 2, then it creates Node Schema 2 with field-schema: field-schema 2, then it creates a valid "Node Instance 2" from instantiation of node-schema-2 containing field-schema-2, then it creates a valid "Node Instance 2a" from instantiation of node-schema-2 containing field-schema-2, then it creates a non type level node using s*node*make.
Then it calls the function t*link-schema*instantiate  with the deleted link-schema-id-1 with from-node-id is node-instance-2 and to-node-id is node-instance-2a, and expects error object invalid-link-schema-ID.



\noindent {\bf Result: FAILED}\
\begin {itemize}
\item 	Error-name             : system-abort
\item Error-data             : (void-variable s*sp*current-server)
\item Operation and Parameter: tf*connection ("claudia.ics.hawaii.edu")
\item Expected-return-type   : object
\item Expected-return-value  : t
\item Location               : Precondition



\end {itemize}
\subsubsection {Test-case scenario: invalid-link-schema-id, Non-type-level-id-Node-instance-id-2-Node-instance-id-2a}


This test case tests that instantiating link schema with invalid schema id will result in error.
Its precondition ensures that a connection exists to the server, then it creates Link Schema 1 with both from-constraint-exp and to-constraint-exp equals to t, then it deletes Link Schema 1, then it creates Field Schema 2, then it creates Node Schema 2 with field-schema: field-schema 2, then it creates a valid "Node Instance 2" from instantiation of node-schema-2 containing field-schema-2, then it creates a valid "Node Instance 2a" from instantiation of node-schema-2 containing field-schema-2, then it creates a non type level node using s*node*make.
Then it calls the function t*link-schema*instantiate  with a non type level id and  from-node-id is node-instance-2 and to-node-id is node-instance-2a, and expects error object invalid-link-schema-ID.



\noindent {\bf Result: FAILED}\
\begin {itemize}
\item 	Error-name             : system-abort
\item Error-data             : (void-variable s*sp*current-server)
\item Operation and Parameter: tf*connection ("claudia.ics.hawaii.edu")
\item Expected-return-type   : object
\item Expected-return-value  : t
\item Location               : Precondition



\end {itemize}
\subsubsection {Test-case scenario: invalid-node-id, Link-schema-id-1-Non-extant-node-instance-id-2-Node-instance-id-2a}


This test case tests that instantiating link schema with valid schema id 1 and invalid from-node-id will result in error.
Its precondition ensures that a connection exists to the server, then it creates Link Schema 1 with both from-constraint-exp and to-constraint-exp equals to t, then it creates Field Schema 2, then it creates Node Schema 2 with field-schema: field-schema 2, then it creates a valid "Node Instance 2" from instantiation of node-schema-2 containing field-schema-2, then it creates a valid "Node Instance 2a" from instantiation of node-schema-2 containing field-schema-2, then it deletes node-instance-2, then it creates a non type level node using s*node*make.
Then it calls the function t*link-schema*instantiate  with the valid link-schema-id-1 with non extant from-node-id and to-node-id is node-instance-2a, and expects error object invalid-link-schema-ID.



\noindent {\bf Result: FAILED}\
\begin {itemize}
\item 	Error-name             : system-abort
\item Error-data             : (void-variable s*sp*current-server)
\item Operation and Parameter: tf*connection ("claudia.ics.hawaii.edu")
\item Expected-return-type   : object
\item Expected-return-value  : t
\item Location               : Precondition



\end {itemize}
\subsubsection {Test-case scenario: invalid-node-id, Link-schema-id-1-Node-instance-id-2-Non-type-level-id}


This test case tests that instantiating link schema with valid schema id 1 and invalid from-node-id will result in error.
Its precondition ensures that a connection exists to the server, then it creates Link Schema 1 with both from-constraint-exp and to-constraint-exp equals to t, then it creates Field Schema 2, then it creates Node Schema 2 with field-schema: field-schema 2, then it creates a valid "Node Instance 2" from instantiation of node-schema-2 containing field-schema-2, then it creates a valid "Node Instance 2a" from instantiation of node-schema-2 containing field-schema-2, then it deletes node-instance-2, then it creates a non type level node using s*node*make.
Then it calls the function t*link-schema*instantiate  with the valid link-schema-id-1 with valid from-node-id-2 and non-type-level to-node-id, and expects error object invalid-link-schema-ID.



\noindent {\bf Result: FAILED}\
\begin {itemize}
\item 	Error-name             : system-abort
\item Error-data             : (void-variable s*sp*current-server)
\item Operation and Parameter: tf*connection ("claudia.ics.hawaii.edu")
\item Expected-return-type   : object
\item Expected-return-value  : t
\item Location               : Precondition



\end {itemize}
\section {Class: t*node}
\subsection {Operation: t*node*clone}
\subsubsection {Test-case scenario: normal, Node-instance-name-0-Node-instance-id-1}


This test case tests that a valid node instance can be cloned in a normal manner.
Its precondition ensures that a connection exists to the server, then it creates Field Schema 1, then it creates Node Schema 1 with field-schema: field-schema 1, then it creates a valid "Node Instance 1" from instantiation of node-schema-1 containing field-schema-1.
Then it calls the function t*node*clone  with a valid name "Node 0 (cloned)" and a valid node instance id 1, and expects a new node instance id (integer).
Afterwards, its postcondition tests that this node instance has name attribute "Node 0 (cloned)", then it tests the attribute node-schema-id-1 exists in this node instance, then it tests the attribute field-schema-id-1 exists in this node.


\noindent {\bf Result: FAILED}\
\begin {itemize}
\item 	Error-name             : system-abort
\item Error-data             : (void-variable s*sp*current-server)
\item Operation and Parameter: tf*connection ("claudia.ics.hawaii.edu")
\item Expected-return-type   : object
\item Expected-return-value  : t
\item Location               : Precondition



\end {itemize}
\subsubsection {Test-case scenario: invalid-node-id, Node-instance-name-0-Non-extant-node-instance-id-1}


This test case tests that an invalid node-id cannot be cloned.
Its precondition ensures that a connection exists to the server, then it creates Field Schema 1, then it creates Node Schema 1 with field-schema: field-schema 1, then it creates a valid "Node Instance 1" from instantiation of node-schema-1 containing field-schema-1, then it deletes node-instance-1, then it creates a non type level node using s*node*make.
Then it calls the function t*node*clone  with a valid node name and a deleted node instance id 1, and expects error invalid-node-ID.



\noindent {\bf Result: FAILED}\
\begin {itemize}
\item 	Error-name             : system-abort
\item Error-data             : (void-variable s*sp*current-server)
\item Operation and Parameter: tf*connection ("claudia.ics.hawaii.edu")
\item Expected-return-type   : object
\item Expected-return-value  : t
\item Location               : Precondition



\end {itemize}
\subsubsection {Test-case scenario: invalid-node-id, Node-instance-name-0-Non-type-level-node-id}


This test case tests that an invalid node-id cannot be cloned.
Its precondition ensures that a connection exists to the server, then it creates Field Schema 1, then it creates Node Schema 1 with field-schema: field-schema 1, then it creates a valid "Node Instance 1" from instantiation of node-schema-1 containing field-schema-1, then it deletes node-instance-1, then it creates a non type level node using s*node*make.
Then it calls the function t*node*clone  with a valid node name and a non type level node id, and expects error invalid-node-ID.



\noindent {\bf Result: FAILED}\
\begin {itemize}
\item 	Error-name             : system-abort
\item Error-data             : (void-variable s*sp*current-server)
\item Operation and Parameter: tf*connection ("claudia.ics.hawaii.edu")
\item Expected-return-type   : object
\item Expected-return-value  : t
\item Location               : Precondition



\end {itemize}
\subsection {Operation: t*node*set-schema-ID}
\subsubsection {Test-case scenario: normal, Node-instance-id-1-Node-schema-id-2}


This test case tests that a new  schema-id can be set to a valid node instance.
Its precondition ensures that a connection exists to the server, then it creates Field Schema 1, then it creates Node Schema 1 with field-schema: field-schema 1, then it creates a valid "Node Instance 1" from instantiation of node-schema-1 containing field-schema-1, then it creates Field Schema 2, then it creates Node Schema 2 with field-schema: field-schema 2.
Then it calls the function t*node*set-schema-ID  with the valid node instance id 1 and node-schema id 2, and expects node-schema-id 2.



\noindent {\bf Result: FAILED}\
\begin {itemize}
\item 	Error-name             : system-abort
\item Error-data             : (void-variable s*sp*current-server)
\item Operation and Parameter: tf*connection ("claudia.ics.hawaii.edu")
\item Expected-return-type   : object
\item Expected-return-value  : t
\item Location               : Precondition



\end {itemize}
\subsection {Operation: t*node*add-field-schema-IDs}
\subsubsection {Test-case scenario: normal, Node-instance-id-1-and-field-schema-id-2}


This test case tests that a valid field-schema ID can be added to the existing node instance in a normal manner.
Its precondition ensures that a connection exists to the server, then it creates Field Schema 1, then it creates Node Schema 1 with field-schema: field-schema 1, then it creates a valid "Node Instance 1" from instantiation of node-schema-1 containing field-schema-1, then it creates Field Schema 2.
Then it calls the function t*node*add-field-schema-IDs  with a valid node instance id and a valid field-schema id, and expects the updated field schema ids (1 and 2).
Afterwards, its postcondition tests the attribute field-schema-id-1 and 2 exists in this node.


\noindent {\bf Result: FAILED}\
\begin {itemize}
\item 	Error-name             : system-abort
\item Error-data             : (void-variable s*sp*current-server)
\item Operation and Parameter: tf*connection ("claudia.ics.hawaii.edu")
\item Expected-return-type   : object
\item Expected-return-value  : t
\item Location               : Precondition



\end {itemize}
\subsubsection {Test-case scenario: invalid-field-schema-id, Node-instance-id-1-Non-extant-field-id-2}


This test case tests that invalid field schema cannot be added to a valid node instance.
Its precondition ensures that a connection exists to the server, then it creates Field Schema 1, then it creates Node Schema 1 with field-schema: field-schema 1, then it creates a valid "Node Instance 1" from instantiation of node-schema-1 containing field-schema-1, then it creates Field Schema 2, then it deletes field-schema-2, then it creates a non type level node using s*node*make.
Then it calls the function t*node*add-field-schema-IDs  with a valid node instance id and non extant field-schema id 2, and expects error object invalid-field-ID.



\noindent {\bf Result: FAILED}\
\begin {itemize}
\item 	Error-name             : system-abort
\item Error-data             : (void-variable s*sp*current-server)
\item Operation and Parameter: tf*connection ("claudia.ics.hawaii.edu")
\item Expected-return-type   : object
\item Expected-return-value  : t
\item Location               : Precondition



\end {itemize}
\subsubsection {Test-case scenario: invalid-field-schema-id, Node-instance-id-1-non-type-level-id}


This test case tests that invalid field schema cannot be added to a valid node instance.
Its precondition ensures that a connection exists to the server, then it creates Field Schema 1, then it creates Node Schema 1 with field-schema: field-schema 1, then it creates a valid "Node Instance 1" from instantiation of node-schema-1 containing field-schema-1, then it creates Field Schema 2, then it deletes field-schema-2, then it creates a non type level node using s*node*make.
Then it calls the function t*node*add-field-schema-IDs  with a valid node instance id and non type level node id, and expects error object invalid-field-ID.



\noindent {\bf Result: FAILED}\
\begin {itemize}
\item 	Error-name             : system-abort
\item Error-data             : (void-variable s*sp*current-server)
\item Operation and Parameter: tf*connection ("claudia.ics.hawaii.edu")
\item Expected-return-type   : object
\item Expected-return-value  : t
\item Location               : Precondition



\end {itemize}
\subsection {Operation: t*node*delete-field-schema-IDs}
\subsubsection {Test-case scenario: normal, Node-instance-id-1-Field-schema-id-1}


This test case tests that a valid field-schema can be deleted from a valid node instance in a normal manner.
Its precondition ensures that a connection exists to the server, then it creates Field Schema 1, then it creates Node Schema 1 with field-schema: field-schema 1, then it creates a valid "Node Instance 1" from instantiation of node-schema-1 containing field-schema-1.
Then it calls the function t*node*delete-field-schema-IDs  with a valid node instance id 1 and a valid field schema id 1, and expects the deleted field-schema-id 1.
Afterwards, its postcondition tests that this node instance 1 has no field schema.


\noindent {\bf Result: FAILED}\
\begin {itemize}
\item 	Error-name             : system-abort
\item Error-data             : (void-variable s*sp*current-server)
\item Operation and Parameter: tf*connection ("claudia.ics.hawaii.edu")
\item Expected-return-type   : object
\item Expected-return-value  : t
\item Location               : Precondition



\end {itemize}
\subsubsection {Test-case scenario: invalid-field-schema-id, Node-instance-id-1-Non-extant-field-id-2}


This test case tests that t*node*delete-field-schema-IDs given invalid field schema id will result in error.
Its precondition ensures that a connection exists to the server, then it creates Field Schema 1, then it creates Node Schema 1 with field-schema: field-schema 1, then it creates a valid "Node Instance 1" from instantiation of node-schema-1 containing field-schema-1, then it creates Field Schema 2, then it deletes field-schema-2, then it creates a non type level node using s*node*make.
Then it calls the function t*node*delete-field-schema-IDs  with a valid node instance id and non extant field-schema id 2, and expects return value invalid-field-ID.



\noindent {\bf Result: FAILED}\
\begin {itemize}
\item 	Error-name             : system-abort
\item Error-data             : (void-variable s*sp*current-server)
\item Operation and Parameter: tf*connection ("claudia.ics.hawaii.edu")
\item Expected-return-type   : object
\item Expected-return-value  : t
\item Location               : Precondition



\end {itemize}
\subsubsection {Test-case scenario: invalid-field-schema-id, Node-instance-id-1-non-type-level-id}


This test case tests that t*node*delete-field-schema-IDs given invalid field schema id will result in error.
Its precondition ensures that a connection exists to the server, then it creates Field Schema 1, then it creates Node Schema 1 with field-schema: field-schema 1, then it creates a valid "Node Instance 1" from instantiation of node-schema-1 containing field-schema-1, then it creates Field Schema 2, then it deletes field-schema-2, then it creates a non type level node using s*node*make.
Then it calls the function t*node*delete-field-schema-IDs  with a valid node instance id and non type level node id, and expects return value invalid-field-ID.



\noindent {\bf Result: FAILED}\
\begin {itemize}
\item 	Error-name             : system-abort
\item Error-data             : (void-variable s*sp*current-server)
\item Operation and Parameter: tf*connection ("claudia.ics.hawaii.edu")
\item Expected-return-type   : object
\item Expected-return-value  : t
\item Location               : Precondition



\end {itemize}
\section {Class: t*link}
\subsection {Operation: t*link*clone}
\subsubsection {Test-case scenario: normal, Link-instance-name-0-Link-instance-id-1-Node-instance-1-Node-instance-2}


This test case tests that a valid link instance can be cloned in a normal manner.
Its precondition ensures that a connection exists to the server, then it creates Field Schema 1, then it creates Node Schema 1 with field-schema: field-schema 1, then it creates a valid "Node Instance 1" from instantiation of node-schema-1 containing field-schema-1, then it creates Field Schema 2, then it creates Node Schema 2 with field-schema: field-schema 2, then it creates a valid "Node Instance 2" from instantiation of node-schema-2 containing field-schema-2, then it creates Link Schema 1 with both from-constraint-exp and to-constraint-exp equals to t, then it creates a link instance 1 instantiated from  link-schema-1 (no constraint) with name "Link Instance 1", and from-node-id is node-instance-1 and to-node-id is node-instance-2.
Then it calls the function t*link*clone  with valid link name "Link 0", link instance id 1, from-node instance id 1 and to-node node instance 2, and expects a new link instance id (integer).
Afterwards, its postcondition tests the attribute link-schema-ID of this link-instance is equal to link-schema-id-1, then it tests that the attribute from-node-id of this link instance exists and equal to node-instance-1, then it tests that the attribute to-node-id of this link instance exists and equal to node-instance-2.


\noindent {\bf Result: FAILED}\
\begin {itemize}
\item 	Error-name             : system-abort
\item Error-data             : (void-variable s*sp*current-server)
\item Operation and Parameter: tf*connection ("claudia.ics.hawaii.edu")
\item Expected-return-type   : object
\item Expected-return-value  : t
\item Location               : Precondition



\end {itemize}
\subsubsection {Test-case scenario: invalid-link-id, Link-instance-name-0-Non-extant-link-instance-id-1}


This test case tests that cloning operation will fail given an invalid link-id.
Its precondition ensures that a connection exists to the server, then it creates Link Schema 1 with both from-constraint-exp and to-constraint-exp equals to t, then it creates a link instance 1 instantiated from  link-schema-1 (no constraint) with name "Link Instance 1", and from-node-id is node-instance-1 and to-node-id is node-instance-2, then it deletes link-instance-1, then it creates a non type level node using s*node*make.
Then it calls the function t*link*clone  with valid link name "Link 0" and a deleted link instance id 1, and expects error invalid-link-ID.



\noindent {\bf Result: FAILED}\
\begin {itemize}
\item 	Error-name             : system-abort
\item Error-data             : (void-variable s*sp*current-server)
\item Operation and Parameter: tf*connection ("claudia.ics.hawaii.edu")
\item Expected-return-type   : object
\item Expected-return-value  : t
\item Location               : Precondition



\end {itemize}
\subsubsection {Test-case scenario: invalid-link-id, Link-instance-name-0-Non-type-level-id}


This test case tests that cloning operation will fail given an invalid link-id.
Its precondition ensures that a connection exists to the server, then it creates Link Schema 1 with both from-constraint-exp and to-constraint-exp equals to t, then it creates a link instance 1 instantiated from  link-schema-1 (no constraint) with name "Link Instance 1", and from-node-id is node-instance-1 and to-node-id is node-instance-2, then it deletes link-instance-1, then it creates a non type level node using s*node*make.
Then it calls the function t*link*clone  with valid link name "Link 0" and a non type level id, and expects error invalid-link-ID.



\noindent {\bf Result: FAILED}\
\begin {itemize}
\item 	Error-name             : system-abort
\item Error-data             : (void-variable s*sp*current-server)
\item Operation and Parameter: tf*connection ("claudia.ics.hawaii.edu")
\item Expected-return-type   : object
\item Expected-return-value  : t
\item Location               : Precondition



\end {itemize}
\subsection {Operation: t*link*set-schema-ID}
\subsubsection {Test-case scenario: normal, Link-instance-id-1-Link-schema-id-2}


This test case tests that a new link-schema can be set to the existing link instance in a normal manner.
Its precondition ensures that a connection exists to the server, then it creates Link Schema 1 with both from-constraint-exp and to-constraint-exp equals to t, then it creates a link instance 1 instantiated from  link-schema-1 (no constraint) with name "Link Instance 1", and from-node-id is node-instance-1 and to-node-id is node-instance-2, then it creates Link Schema 2 with from-constraint-exp equalto t and to-constraint-exp equal to node-schema-id-1..
Then it calls the function t*link*set-schema-ID  with a valid link instance and a different schema id, and expects return value t.



\noindent {\bf Result: FAILED}\
\begin {itemize}
\item 	Error-name             : system-abort
\item Error-data             : (void-variable s*sp*current-server)
\item Operation and Parameter: tf*connection ("claudia.ics.hawaii.edu")
\item Expected-return-type   : object
\item Expected-return-value  : t
\item Location               : Precondition



\end {itemize}
\subsection {Operation: t*link*add-to-constraint}
\subsubsection {Test-case scenario: normal, Link-instance-id-1-Node-schema-id-2}


This test case tests that a valid node-schema-ID can be added to the existing to-node constraint of a link instance in a normal manner.
Its precondition ensures that a connection exists to the server, then it creates a link instance 1 instantiated from  link-schema-1 (no constraint) with name "Link Instance 1", and from-node-id is node-instance-1 and to-node-id is node-instance-2, then it creates Field Schema 2, then it creates Node Schema 2 with field-schema: field-schema 2.
Then it calls the function t*link*add-to-constraint  with a valid link instance id and a valid node-schema id, and expects the return value the updated to-node constraint expression (equal to t).



\noindent {\bf Result: FAILED}\
\begin {itemize}
\item 	Error-name             : system-abort
\item Error-data             : (void-variable s*sp*current-server)
\item Operation and Parameter: tf*connection ("claudia.ics.hawaii.edu")
\item Expected-return-type   : object
\item Expected-return-value  : t
\item Location               : Precondition



\end {itemize}
\subsubsection {Test-case scenario: invalid-node-schema, Link-instance-id-1-Non-extant-node-schema-id-2}


This test case tests that invalid node schema id cannot be added to the existing to-node constraint expression.
Its precondition ensures that a connection exists to the server, then it creates Field Schema 1, then it creates Node Schema 1 with field-schema: field-schema 1, then it creates a valid "Node Instance 1" from instantiation of node-schema-1 containing field-schema-1, then it creates Field Schema 2, then it creates Node Schema 2 with field-schema: field-schema 2, then it creates a valid "Node Instance 2" from instantiation of node-schema-2 containing field-schema-2, then it creates Link Schema 1 with both from-constraint-exp and to-constraint-exp equals to t, then it creates a link instance 1 instantiated from  link-schema-1 (no constraint) with name "Link Instance 1", and from-node-id is node-instance-1 and to-node-id is node-instance-2, then it creates Field Schema 2, then it creates Node Schema 2 with field-schema: field-schema 2, then it creates a non type level node using s*node*make.
Then it calls the function t*link*add-to-constraint  with a valid link instance id 1 and non extant node-schema id 2, and expects error object invalid-node-schema-ID.



\noindent {\bf Result: FAILED}\
\begin {itemize}
\item 	Error-name             : system-abort
\item Error-data             : (void-variable s*sp*current-server)
\item Operation and Parameter: tf*connection ("claudia.ics.hawaii.edu")
\item Expected-return-type   : object
\item Expected-return-value  : t
\item Location               : Precondition



\end {itemize}
\subsubsection {Test-case scenario: invalid-node-schema, Link-instance-id-1-Non-type-level-id}


This test case tests that invalid node schema id cannot be added to the existing to-node constraint expression.
Its precondition ensures that a connection exists to the server, then it creates Field Schema 1, then it creates Node Schema 1 with field-schema: field-schema 1, then it creates a valid "Node Instance 1" from instantiation of node-schema-1 containing field-schema-1, then it creates Field Schema 2, then it creates Node Schema 2 with field-schema: field-schema 2, then it creates a valid "Node Instance 2" from instantiation of node-schema-2 containing field-schema-2, then it creates Link Schema 1 with both from-constraint-exp and to-constraint-exp equals to t, then it creates a link instance 1 instantiated from  link-schema-1 (no constraint) with name "Link Instance 1", and from-node-id is node-instance-1 and to-node-id is node-instance-2, then it creates Field Schema 2, then it creates Node Schema 2 with field-schema: field-schema 2, then it creates a non type level node using s*node*make.
Then it calls the function t*link*add-to-constraint  with a valid link instance id and non node-schema id, and expects error object invalid-node-schema-ID.



\noindent {\bf Result: FAILED}\
\begin {itemize}
\item 	Error-name             : system-abort
\item Error-data             : (void-variable s*sp*current-server)
\item Operation and Parameter: tf*connection ("claudia.ics.hawaii.edu")
\item Expected-return-type   : object
\item Expected-return-value  : t
\item Location               : Precondition



\end {itemize}
\subsection {Operation: t*link*delete-from-constraint}
\subsubsection {Test-case scenario: normal, Link-instance-id-5-Node-schema-id-2}


This test case tests that a valid node-schema-id in the from-constraint expression of an existing link instance can be deleted in a normal manner.
Its precondition ensures that a connection exists to the server, then it creates Field Schema 1, then it creates Node Schema 1 with field-schema: field-schema 1, then it creates Field Schema 2, then it creates Node Schema 2 with field-schema: field-schema 2, then it creates Link Schema 5 with from-constraint-exp is equal to the list of node-schema-1 and node-schema-2 and the to-constraint-exp equal to t, then it creates a link instance with name "Link instance 5" and which is instantiated from link-schema-5, and which  from-node-id is node-instance-1 and to-node-id is node-instance-2.
Then it calls the function t*link*delete-from-constraint  with valid link instance 5 and node schema id 2, and expects the updated from constraint expression which is node-schema-1.
Afterwards, its postcondition checks that the from-constraint expression of the link instance is node-schema-1.


\noindent {\bf Result: FAILED}\
\begin {itemize}
\item 	Error-name             : system-abort
\item Error-data             : (void-variable s*sp*current-server)
\item Operation and Parameter: tf*connection ("claudia.ics.hawaii.edu")
\item Expected-return-type   : object
\item Expected-return-value  : t
\item Location               : Precondition



\end {itemize}
\subsubsection {Test-case scenario: invalid-node-schema, Link-instance-id-4-Node-schema-id-2}


This test case tests that the deletion of invalid node-schema from the link-constraint expression  will result in error.
Its precondition ensures that a connection exists to the server, then it creates Field Schema 1, then it creates Node Schema 1 with field-schema: field-schema 1, then it creates Link Schema 4 with both  from-constraint-exp and to-constraint-exp equal to node-schema-id-1., then it creates Field Schema 2, then it creates Node Schema 2 with field-schema: field-schema 2, then it creates a valid "Node Instance 1" from instantiation of node-schema-1 containing field-schema-1, then it creates a valid "Node Instance 2" from instantiation of node-schema-2 containing field-schema-2, then it creates a link instance with name "Link instance 4" which is instantiated from link-schema-4, and which  from-node-id is node-instance-1 and to-node-id is node-instance-2, then it creates a non type level node using s*node*make.
Then it calls the function t*link*delete-from-constraint  with valid link instance id 4 and valid node schema id 2, and expects return value invalid-node-schema-ID.



\noindent {\bf Result: FAILED}\
\begin {itemize}
\item 	Error-name             : system-abort
\item Error-data             : (void-variable s*sp*current-server)
\item Operation and Parameter: tf*connection ("claudia.ics.hawaii.edu")
\item Expected-return-type   : object
\item Expected-return-value  : t
\item Location               : Precondition



\end {itemize}
\subsubsection {Test-case scenario: invalid-node-schema, Link-instance-id-4-Non-type-level-id}


This test case tests that the deletion of invalid node-schema from the link-constraint expression  will result in error.
Its precondition ensures that a connection exists to the server, then it creates Field Schema 1, then it creates Node Schema 1 with field-schema: field-schema 1, then it creates Link Schema 4 with both  from-constraint-exp and to-constraint-exp equal to node-schema-id-1., then it creates Field Schema 2, then it creates Node Schema 2 with field-schema: field-schema 2, then it creates a valid "Node Instance 1" from instantiation of node-schema-1 containing field-schema-1, then it creates a valid "Node Instance 2" from instantiation of node-schema-2 containing field-schema-2, then it creates a link instance with name "Link instance 4" which is instantiated from link-schema-4, and which  from-node-id is node-instance-1 and to-node-id is node-instance-2, then it creates a non type level node using s*node*make.
Then it calls the function t*link*delete-from-constraint  with  valid link instance id 4 and non node-schema id, and expects return value invalid-node-schema-ID.



\noindent {\bf Result: FAILED}\
\begin {itemize}
\item 	Error-name             : system-abort
\item Error-data             : (void-variable s*sp*current-server)
\item Operation and Parameter: tf*connection ("claudia.ics.hawaii.edu")
\item Expected-return-type   : object
\item Expected-return-value  : t
\item Location               : Precondition



\end {itemize}
\subsection {Operation: t*link*set-to-node}
\subsubsection {Test-case scenario: normal, Link-instance-id-1-Node-instance-id-3}


This test case tests that the to-node of this link can be set to a different node.
Its precondition ensures that a connection exists to the server, then it creates Field Schema 1, then it creates Node Schema 1 with field-schema: field-schema 1, then it creates a valid "Node Instance 1" from instantiation of node-schema-1 containing field-schema-1, then it creates Field Schema 2, then it creates Node Schema 2 with field-schema: field-schema 2, then it creates a valid "Node Instance 2" from instantiation of node-schema-2 containing field-schema-2, then it creates Field Schema 3, then it creates Node Schema 3 with field-schema: field-schema-1 and field-schema-2, then it creates a valid "Node Instance 3" from instantiation of node-schema-3 containing field-schema-3, then it creates Link Schema 1 with both from-constraint-exp and to-constraint-exp equals to t, then it creates a link instance 1 instantiated from  link-schema-1 (no constraint) with name "Link Instance 1", and from-node-id is node-instance-1 and to-node-id is node-instance-2.
Then it calls the function t*link*set-to-node  with  valid link instance id 1 and valid node instance id 3, and expects the updated to-node ID.
Afterwards, its postcondition checks that the to-constraint expression of the link instance 1 is node-schema-3.


\noindent {\bf Result: FAILED}\
\begin {itemize}
\item 	Error-name             : system-abort
\item Error-data             : (void-variable s*sp*current-server)
\item Operation and Parameter: tf*connection ("claudia.ics.hawaii.edu")
\item Expected-return-type   : object
\item Expected-return-value  : t
\item Location               : Precondition



\end {itemize}
\subsubsection {Test-case scenario: invalid-node-instance, Link-instance-id-1-Non-extant-node-instance-id-3}


This test case tests that a valid link cannot be set to an invalid node .
Its precondition ensures that a connection exists to the server, then it creates Field Schema 1, then it creates Node Schema 1 with field-schema: field-schema 1, then it creates a valid "Node Instance 1" from instantiation of node-schema-1 containing field-schema-1, then it creates Field Schema 2, then it creates Node Schema 2 with field-schema: field-schema 2, then it creates a valid "Node Instance 2" from instantiation of node-schema-2 containing field-schema-2, then it creates Field Schema 3, then it creates Node Schema 3 with field-schema: field-schema-1 and field-schema-2, then it creates a valid "Node Instance 3" from instantiation of node-schema-3 containing field-schema-3, then it creates Link Schema 1 with both from-constraint-exp and to-constraint-exp equals to t, then it creates a link instance 1 instantiated from  link-schema-1 (no constraint) with name "Link Instance 1", and from-node-id is node-instance-1 and to-node-id is node-instance-2, then it deletes node-instance-3, then it creates a non type level node using s*node*make.
Then it calls the function t*link*set-to-node  with  valid link instance id 1 and deleted node instance id 3, and expects error object invalid-node-id.



\noindent {\bf Result: FAILED}\
\begin {itemize}
\item 	Error-name             : system-abort
\item Error-data             : (void-variable s*sp*current-server)
\item Operation and Parameter: tf*connection ("claudia.ics.hawaii.edu")
\item Expected-return-type   : object
\item Expected-return-value  : t
\item Location               : Precondition



\end {itemize}
\subsubsection {Test-case scenario: invalid-node-instance, Link-instance-id-1-Non-type-level-id}


This test case tests that a valid link cannot be set to an invalid node .
Its precondition ensures that a connection exists to the server, then it creates Field Schema 1, then it creates Node Schema 1 with field-schema: field-schema 1, then it creates a valid "Node Instance 1" from instantiation of node-schema-1 containing field-schema-1, then it creates Field Schema 2, then it creates Node Schema 2 with field-schema: field-schema 2, then it creates a valid "Node Instance 2" from instantiation of node-schema-2 containing field-schema-2, then it creates Field Schema 3, then it creates Node Schema 3 with field-schema: field-schema-1 and field-schema-2, then it creates a valid "Node Instance 3" from instantiation of node-schema-3 containing field-schema-3, then it creates Link Schema 1 with both from-constraint-exp and to-constraint-exp equals to t, then it creates a link instance 1 instantiated from  link-schema-1 (no constraint) with name "Link Instance 1", and from-node-id is node-instance-1 and to-node-id is node-instance-2, then it deletes node-instance-3, then it creates a non type level node using s*node*make.
Then it calls the function t*link*set-to-node  with a valid link instance id and non node-schema id, and expects error object invalid-node-id.



\noindent {\bf Result: FAILED}\
\begin {itemize}
\item 	Error-name             : system-abort
\item Error-data             : (void-variable s*sp*current-server)
\item Operation and Parameter: tf*connection ("claudia.ics.hawaii.edu")
\item Expected-return-type   : object
\item Expected-return-value  : t
\item Location               : Precondition
\end {itemize}

