\chapter{Testing Concepts}
\section{Introduction}

Standard software engineering practice for testing involves several
interrelated concepts.

The {\em test framework}\/ describes and implements the set of
functions and procedures that support definition of the test plan,
test cases, test data, and test log.

The {\em test environment}\/ specifies the required state of the
environment that must exist in order to start running the test.

The {\em test plan}\/ specifies the flow of control to be followed
during the running of the test. In particular, it specifies the
sequence of test cases to be run, and how this sequence changes in
response to positive or negative results of previous test cases.  The
test plan also specifies the results of the test in terms of the
format of the test log.

Each {\em test case}\/ specifies a test of some particular behavior,
function, or other aspect of the system.  For example, each public
operation and attribute of the system will have a corresponding test
case.  Test cases can be further broken down into sets of test data.

Each {\em test data}\/ item specifies (a) input to the system, (b) the
state of the system during the test, and (c) the expected output.  For
example, the server subsystem test case might have a test data item
that requests connection to a particular server while the system is
already connected to a server, and which expects the connection
function to return a specific value that indicates this particular
error.

An integral part of the Egret environment is a testing environment,
which supports the incremental definition of test cases and 
regression testing of any subset of the Egret system after changes
are made.  More specifically, the Egret testing environment 
satisfies the following requirements:

\begin{itemize}
\item {\bf Background, batch operation.} Egret tests can be invoked
  with a single command and then left to run test cases unattended.
  The results of this run are accumulated in a log file.
  
\item {\bf Testing of all public operations and attributes.} All
  public operations for a given class and subsystem have associated
  test cases.  In general, all arguments to all public operations are
  tested with valid values, invalid values, and invalid types of
  values.  A test case for one public operation can implicitly presume
  that all other public operations are implemented correctly.
  
\item {\bf Independence from private operations or encapsulated
  representations.} The Egret test methodology is ``black box''.  Test
  cases are generated from reference to the Design specification for
  classes, not by reference to the code.  
  
\item {\bf Verification, not diagnosis.} The goal of a test run is
  determine that the system either (a) passes testing, or (b) fails
  testing.  The Egret test environment does not currently support any
  diagnosis of the reasons for test failure.
  
\end{itemize}

\section{Test Case Design Heuristics}

While the types of tests for a particular class are in general
domain-specific, there are a few broadly applicable guidelines 
useful in generating test cases across many classes. These are
enumerated below:

\begin{itemize}
  
\item Invoke each operation with invalid data at the type and the
  value level.  For example, the attribute {\bf s*node*name} accepts a
  node-ID as its argument.  This attribute should be tested by passing
  {\bf s*node*name} both a string (i.e. the wrong type of argument) and
  an integer that is guaranteed not to be a valid node-ID (such as -1).
  
\item Set, then test the state of the system as reified by attributes.
  For example, invoke {\bf s*node*set-name} on a newly created node to
  change its name, then use {\bf s*node*name} to see if the name was
  actually changed.
  
\item Set, then test the state of the system as reified by global
  structures.  For example, an implicit result of creating a node
  should be a change in the collection operation {\bf s*\{node\}*IDs}.
  
\item Test the recovery operations.  This is difficult, since
  corrupting the tested state normally requires access to the
  internal, private state.  An alternative method is to assume the
  current state is correct, invoke the recovery operations, and see if
  the rebuilt state matches the previous one.
  
\item Test concurrent and cooperative access. Automating this kind
  of test requires a shell script that starts up two Emacs processes,
  changes the user-name of one of them, then tests locking, event
  propogation, and so forth while sychronizing the two subprocess
  accesses to the database.  The Egret test framework does not 
  currently provide these facilities, although test cases (which 
  always fail) could be written to describe the kinds of tests that
  should be performed manually by test teams.
  
\end{itemize}

The next chapter presents a more detailed look at the Egret testing 
environment.
