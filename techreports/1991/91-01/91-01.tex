
\documentstyle [nftimes,11pt,/group/csdl/tex/definemargins,/group/csdl/tex/lmacros]{article}

\definemargins{1in}{1in}{1in}{1in}{.3in}{.3in}

\begin{document}

\title{{\bf A CSDL Primer}}
\author {Philip Johnson\\
Collaborative Software Development Laboratory\\ 
Department of Information and Computer Sciences\\
University of Hawaii\\
Honolulu, HI 96822\\
(808) 956-3489\\
CSDL Technical Report 91-01}

\date{Last Revised: \today}

\maketitle

\begin{abstract}

This document provides an overview of research and development
practice in CSDL.  It provides a brief introduction to the major
tools, environments, and methods used by CSDL in the pursuit of high
quality software and research.

\end{abstract}

\newpage
\tableofcontents

\newpage 
\section{Introduction}

The Collaborative Software Development Laboratory (CSDL), formed in
1991, is a research group within the Department of Information and
Computer Sciences at the University of Hawaii.  CSDL pursues research
along two general fronts: the development of computer systems to
support group activities (cooperative software), as well as research
on the process of developing software in a collaborative setting
(cooperative development).

Since 1991, we have developed a broad array of tools and techniques to
help us in the pursuit of high quality software and research.  The
goal of this document is to briefly touch on the facilities we have
developed, so that new members can more quickly come up to speed in
our environment.

This primer is organized as a set of sections, but need not be
necessarily read in a sequential order.  Please let us know of
corrections, ambiguities, or additions you feel would improve its
quality and usefulness.

\section {The CSDL Unix Environment}

CSDL conducts most of its business in the directory
<file://uhics.ics.hawaii.edu/group/csdl>.  To begin work in CSDL, you
will need to obtain an account on uhics, and be added as a member to 
the CSDL group.  

You will also need to make sure that group csdl appears first on your
list of group memberships associated with your login account.  This is
to ensure that when you provide group access to directories in
/group/csdl, access will be provided to CSDL members and not some
other Unix group.  The ICS Department system administrator will
perform these tasks for you.

You will also need to be added to the CSDL mailing list:
csdl@uhics.ics.hawaii.edu.  Philip will do this for you. Thereafter,
you can send mail to csdl@uhics.ics and the message will go out to all
members of CSDL.

You should peruse the CSDL subdirectory and become familiar with its
contents.  The top-level /group/csdl directory primarily contains
subdirectories for each of the software systems developed by CSDL. 
The primary exceptions are the following special subdirectories:
\begin{itemize}

\item The /bib directory contains a set of bibtex files where literature
citations for the various research projects in CSDL are stored.  All
CSDL research reports are written using LaTeX, and all citations are
represented in Bibtex format.  CSDL uses LaTeX for all document
processing because it is powerful and because we have a great deal of
tools that process and/or generate LaTeX format.   

\item The /public_html directory contains the CSDL WWW pages. 

\item The /techreports directory contains a set of subdirectories containing
all the research reports and publications produced by CSDL members.

\item The /tex directory contains LaTeX utilities.

\item the /emacs directory contains Emacs utilities. 
\end{itemize}

Although there is an actual csdl account, you do not, under normal
circumstances, require the password to that account. Once you are 
a member of the unix csdl group, you will be able to create
subdirectories in /group/csdl as necessary.

\section{Emacs}


\section{WWW}


\section{Defsys/DSB}

\section{Basic Lisp Development Standards}



\end{document}








