%%%%%%%%%%%%%%%%%%%%%%%%%%%%% -*- Mode: Latex -*- %%%%%%%%%%%%%%%%%%%%%%%%%%%%
%% curriculum-vitae.tex -- 
%% RCS:            : $Id: nsf93-bio.tex,v 1.4 93/10/06 16:50:04 johnson Exp $
%% Author          : Philip Johnson
%% Created On      : Wed Aug 11 14:09:06 1993
%% Last Modified By: Philip Johnson
%% Last Modified On: Tue Sep 14 09:58:39 2010
%% Status          : Unknown
%%%%%%%%%%%%%%%%%%%%%%%%%%%%%%%%%%%%%%%%%%%%%%%%%%%%%%%%%%%%%%%%%%%%%%%%%%%%%%%
%%   Copyright (C) 1993 University of Hawaii
%%%%%%%%%%%%%%%%%%%%%%%%%%%%%%%%%%%%%%%%%%%%%%%%%%%%%%%%%%%%%%%%%%%%%%%%%%%%%%%
%% 
%% History
%% 11-Aug-1993          Philip Johnson  
%%    

\documentstyle[times,11pt,lmacros,definemargins]{article}

\begin{document}

%%      Command to define new categories:
\newcommand{\newcategory}[1]{\newenvironment{#1}
 {\sectionheading{#1}\par\vspace*{0.03in}\par\hrule\par\begin{description}}{\end{description}\par}}
\newcommand{\sectionheading}[1]{\medskip\pagebreak[2]\par\noindent
 {\bf #1}\nopagebreak}

\newcategory{Degrees}
\newcategory{Tutorial Presentations}
\newcategory{Journal Publications}
\newcategory{Conference Publications}
\newcategory{Workshop Publications}
\newcategory{Publications}
\newcategory{Book Chapters}
\newcategory{Invited Talks}
\newcategory{Awarded Grant Support}
\newcategory{Pending Grant Support}
\newcategory{Professional Activities}
\newcategory{Research and Teaching Experience}
\newcategory{Industry Experience}
\newcategory{Awards and Honors}

\definemargins{1in}{1in}{1in}{1in}{.3in}{.3in}


\begin{center}
{\bf Philip M. Johnson}\\
Information and Computer Sciences \hfill (808) 956-3489

University of Hawaii              \hfill fax: (808) 956-3548

1680 East-West Road               \hfill johnson@hawaii.edu

Honolulu, HI~~~96822              \hfill http://csdl.hawaii.edu/$\sim$johnson/

\end{center}

\begin{Degrees} 
\item Ph.D.~in Computer Science, University of Massachusetts, Amherst. 1990
\item M.S.~in Computer Science, University of Massachusetts, Amherst. 1985 
\item B.S.~in Computer Science,  University of Michigan, Ann Arbor. 1980 
\item B.S.~in Biology, University of Michigan, Ann Arbor. 1980  
\end{Degrees}

\begin{Research and Teaching Experience}
  
\item {\em Professor} \hfill 2001---present 
\vspace*{-10pt}
\item {\em Associate Professor} \hfill 1995---2001
\vspace*{-10pt}
\item {\em Assistant Professor} \hfill  1990---1995
\vspace*{-10pt}
\item Department of Information and Computer Sciences, University of Hawaii.  
  
 $\bullet$  Director, Collaborative Software Development Laboratory, 2001 - present \newline
 $\bullet$  Associate Chair, Information and Computer Sciences, 2010 - present \newline
 $\bullet$  Graduate Chair, Information and Computer Sciences, 1998-2001 \newline

\item {\em Visiting Professor} \hfill   2006
\vspace*{-10pt}
\item School of Engineering, Blekinge Institute of Technology, Sweden.

 $\bullet$  Research on software measurement. \newline

\item {\em Senior Research Fellow} \hfill   1997
\vspace*{-10pt}
\item Distributed Systems Technology Centre, University of Queensland, 
Brisbane, Australia.

 $\bullet$  Research on computer-supported cooperative work technologies \newline

  
\item {\em Research Assistant} \hfill   1984---1986, 1987---1990
\vspace*{-10pt}
\item Department of Computer Science,  University of Massachusetts.

 $\bullet$ Ph.D. thesis on structural evolution in software development \newline
 $\bullet$ Research on automated Ada package restructuring \newline
 $\bullet$ Research on natural language processing tools. 
  
\item {\em Teaching Assistant} \hfill  1983---1984
\vspace*{-10pt}
\item Department of Computer Science, University  of Massachusetts.

 $\bullet$ Coursework support in software engineering.
  
\item {\em Lecturer} \hfill  1981---1982
\vspace*{-10pt}
\item Department of Computer and Communication Sciences, University of Michigan.

 $\bullet$ Taught introductory programming course.  

\end{Research and Teaching Experience}

\begin{Industry Experience}

\item {\em Member, Technical Advisory Board} \hfill 2006-2009
\vspace*{-10pt}
\item Sixth Sense Analytics, Raleigh, North Carolina.

  $\bullet$ Sixth Sense Analytics (bought by Borland) provided software measurement
and analysis services. 

\item {\em Member, Board of Directors} \hfill 1999-2007
\vspace*{-10pt}
\item Hawaii Strategic Development Corporation, Honolulu, Hawaii.

  $\bullet$ HSDC is a state-sponsored organization that works
to support the growth of the venture capital industry in Hawaii.

\item {\em Member, Board of Directors} \hfill 2003-2005
\vspace*{-10pt}
\item Tiki Technologies, Inc., Honolulu, Hawaii.

  $\bullet$ Tiki Technologies (now defunct) developed internet software including
spam detection systems. 

\item {\em Member, Board of Directors} \hfill 2002-2005
\vspace*{-10pt}
\item Lavanet, Inc., Honolulu, Hawaii.

  $\bullet$ Lavanet is an Internet Service Provider, 
Network Engineering, and Web Development Services company.


\item {\em Member, Professional Advisory Board} \hfill 2000-present
\vspace*{-10pt}
\item BreastCancer.org, Philadelphia, PA.

  $\bullet$ BreastCancer.org is a non-profit organization dedicated 
  to helping those living with breast cancer. 

\item {\em Member, Board of Directors} \hfill 2000-2004
\vspace*{-10pt}
\item High Technology Development Corporation, Honolulu, Hawaii.

  $\bullet$ HTDC is a state-sponsored organization whose mission
  is to support the growth of the high technology industry in Hawaii.


\item {\em Co-Founder} \hfill 2000-2004
\vspace*{-10pt}
\item hotU, Inc., Honolulu, Hawaii.

  $\bullet$ Served as interim Chief Technology Officer during initial
  formation of company to provide Internet-based services to college student market. Also served on Technology Advisory Board.


\item {\em Consulting Software Engineer} \hfill 1994---present
\vspace*{-10pt}
\item Honolulu, Hawaii.

  $\bullet$ Providing project management and software engineering services to local and national companies.

\item {\em Member, Professional Advisory Board} \hfill 2000-2002
\vspace*{-10pt}
\item Referentia, Inc., Honolulu, HI. 

  $\bullet$ Referentia develops E-learning multimedia packages.

\item {\em Programmer} \hfill  1986---1987  
\vspace*{-10pt}
\item Department of Computer Science, University of Massachusetts. 
  
  $\bullet$ Developed control shell for GBB, an AI knowledge base
  and inference environment. 
 
\item {\em Systems Programmer} \hfill  1982---1983 
\vspace*{-10pt}
\item Software Services Corporation, Ann Arbor, MI.

  $\bullet$ Developed software quality assurance and validation tools for
  Ford Motor Company. 
  
\item {\em Systems Analyst} \hfill 1981---1983
\vspace*{-10pt}
\item Veterans Hospital, Ann Arbor, MI.

  $\bullet$ Developed real-time data acquisition and signal processing
  software to control hardware for psychophysiological experimentation.
  
\item {\em Programmer} \hfill 1978
\vspace*{-10pt}
\item  Great Lakes Software Systems, Ann Arbor, MI.

  $\bullet$ Implemented accounting software in COBOL. 

\end{Industry Experience}


\begin{Journal Publications}

\item H. Kou and P. Johnson and H. Erdogmus, 
{\em Operational Definition and Automated Inference of Test-Driven Development with Zorro},
Automated Software Engineering, Volume 16, Number 4, December, 2009.

\item V. Basili and M. Zelkowitz and D. Sjoberg and P. Johnson and T. Cowling,
{\em Protocols in the use of Empirical Software Engineering Artifacts}, 
Empirical Software Engineering, Volume 12, February, 2007.

\item L. Hochstein and T. Nakamura and V. Basili and S. Asgari and 
M. Zelkowitz and J. Hollingsworth and F. Shull and J. Carver and 
M. Voelp and N. Zazworka and P. Johnson, {\em Experiments to 
understand HPC time to development}, CTWatch Quarterly, 
November, 2006.

\item P.~M.~Johnson and H.~Kou and M.~Paulding and Q.~Zhang and
and A.~Kagawa and T.~Yamashita, {\em Improving software development
management through software project telemetry}, 
IEEE Software, Vol. 22, No. 4, July 2005.

\item S.~Faulk and J.~Gustafson and P.~Johnson and A.~Porter and W.~Tichy 
and L.~Votta, {\em Measuring {HPC} Productivity}, 
International Journal of High Performance Computing Applications, December 2004. 

\item P.~M.~Johnson and M.~L.~Moffett and B.~T.~Pentland, {\em
Lessons learned from VCommerce: A virtual
environment for interdisciplinary learning about software entrepreneurship}.
Communications of the ACM, Vol. 46, No. 12, December, 2003.

\item P.~M.~Johnson and C.~A.~Moore and J.~A.~Dane and R.~S.~Brewer, {\em
Empirically Guided Software Effort Guesstimation}.
IEEE Software, Vol. 17, No. 6, December 2000.

\item P.~M.~Johnson and A.~M.~Disney, {\em A Critical Analysis of PSP Data Quality: Results from a Case Study}.
Journal of Empirical Software Engineering, Volume 4, December, 1999.


\item P.~M.~Johnson and A.~M.~Disney, {\em The Personal Software Process: 
A Cautionary Case Study}.
  In IEEE Software, Volume 15, No. 6, 
  November, 1998.

\item P.~M.~Johnson, {\em Reengineering Inspection}.
  In Communications of the ACM, Volume 41, No. 2, 
  February, 1998.

\item P.~M.~Johnson and D.~Tjahjono, {\em Does Every Inspection Really Need A Meeting?}, 
      In Journal of Empirical Software Engineering, Volume 4,
      No. 1, January 1998.

\item A.~A.~Porter and P.~M.~Johnson, {\em Assessing Software Review
    Meetings: Results of a Comparative Analysis of Two Experimental
    Studies}.  In IEEE Transactions on Software Engineering, vol. 23, no. 3, March
    1997.

\item P.~M.~Johnson, {\em Design for Instrumentation: High Quality
  Measurement of Formal Technical Review}.  Software
  Quality Journal, Volume 5, March, 1996.
  
\item D.~Wan and P.~M.~Johnson, {\em Experiences with CLARE: a
  Computer-Supported Collaborative Learning Environment}.  
 In the International Journal of Human-Computer Studies, Volume 41, 
 December, 1994.

%\item R.~S.~Brewer and P.~M.~Johnson, {\em Toward Collaborative Knowledge
% Management within Large, Dynamically Structured Information Systems}.
% Submitted to the ACM Transactions on Information Systems, 1994.
  
\item P.~M.~Johnson, {\em Experiences with EGRET: An Exploratory Group
  Work Environment}. In Collaborative Computing 1(1), March, 1994.
  
\item B.~Walker, M.~Walker, S.~Achem, P.~Johnson, and R.~Gregg.
     {\em The Clinical Significance of Electrogastrography}.
     In Psychophysiology 20: 1983.

\end{Journal Publications}  



\begin{Book Chapters}

\item P.~M.~Johnson, {\em An Instrumented Approach to Improving Software
  Quality through Formal Technical Review}. In Software Inspection: 
  An Industry Best Practice.  David A. Wheeler, Bill Brykczynski, and
  Reginald N. Meeson, Jr., Editors.  IEEE Computer Society Press. 1996

  Also appearing in the Proceedings of the 16th International Conference on
  Software Engineering, Sorrento,  Italy. 1994.
  
\item P.~M.~Johnson and W.~G.~Lehnert, {\em Beyond Exploratory Programming:
  A Methodology and Environment for Natural Language Processing}. In
  Artificial Intelligence and Software Engineering, D.~Partridge, editor.
  Ablex, 1990. 
  
  Also appearing in Proceedings of the Fifth National Conference on
  Artificial Intelligence (AAAI-86), Philadelphia, PA.

\item D.~Corkill, K.~Gallagher, and P.~M.~Johnson, {\em Achieving
  Flexibility, Efficiency, and Generality in Blackboard Architectures}. In
  Readings in Distributed Artificial Intelligence, A.~Bond and L.~Gasser,
  editors.  Morgan-Kaufman, 1988. 
  
  Also appearing in Proceedings of the Sixth National Conference on
  Artificial Intelligence (AAAI-87), Seattle, WA.
  

\end{Book Chapters}

\begin{Conference Publications}

\item R. Brewer, Y. Xu, G. Lee, M. Katchuck, C. Moore, P. Johnson, 
{\em Energy Feedback for Smart Grid Consumers: Lessons Learned from the Kukui Cup}, 
Proceedings of Energy 2013, March, 2013.

\item P. Johnson, Y. Xu, R. Brewer,  C. Moore,  G. Lee, A. Connell, 
{\em Makahiki+WattDepot: An open source software stack for next generation energy research
  and education}, 
Proceedings of the 2012 Conference on Information and Communication Technologies for
Sustainability, February 2013.

\item P. Johnson, Y. Xu, R. Brewer, G. Lee, M. Katchuck, C. Moore,
{\em Beyond kWh: Myths and fixes for energy competition game design}, 
Proceedings of Meaningful Play 2012, October, 2012.


\item R. Brewer, G. Lee, and P. Johnson, {\em The Kukui Cup: a Dorm Energy Competition Focused on Sustainable Behavior Change and Energy Literacy}, Proceedings of the 43rd Hawaii International Conference on System Sciences, January, 2011.

\item R. Brewer and P. Johnson, {\em WattDepot: An open source software ecosystem for enterprise-scale energy data collection, storage, analysis, and visualization}, Proceedings of the First IEEE International Conference on Smart Grid Communications, October 2010.

\item P. Johnson and S. Zhang, {\em We need more coverage, stat! Experiences with the Software ICU},
Proceedings of the 2009 Conference on Empirical Software Engineering and Measurement, Orlando, Florida, October, 2009.

\item P. Johnson, {\em Requirement and Design Trade-offs in Hackystat: An
in-process software engineering measurement and analysis system},
Proceedings of the 2007 International Symposium on Empirical Software
Engineering and Measurement, Madrid, Spain, September, 2007.

\item P. Johnson and H. Kou, {\em Automated Recognition of Test-Driven Development with Zorro}, 
Proceedings of Agile 2007, Washington, D.C., August, 2007. 

\item H. Scott and P. Johnson, {\em Generalizing fault contents from a few
classes}, Proceedings of the 2007 International Symposium on Empirical
Software Engineering and Measurement, Madrid, Spain, September, 2007.

\item P.~M.~Johnson, H.~Kou, J.~Agustin, Q.~Zhang, A.~Kagawa, T.~Yamashita,
{\em Practical automated process and product metric collection and analysis
in a classroom setting: {L}essons learned from {Hackystat-UH}}, In
Proceedings of the 2004 Symposium on Empirical Software Engineering, Los
Angeles, CA., August 2004.

\item P.~M.~Johnson, H.~Kou, J.~Agustin, C.~Chan, C.~Moore, 
J.~Miglani, S.~Zhen, and W.~Doane, {\em Beyond
the Personal Software Process: Metrics collection and analysis for the
differently disciplined}, In Proceedings of the 2003 International
Conference on Software Engineering, Portland, OR., May, 2003.

\item P.~M.~Johnson, {\em Leap: A ``Personal Information Environment'' for
Software Engineers}. In
  Proceedings of the 1999 International Conference on Software Engineering,
  Los Angeles, CA., May 1999.


\item A.~M.~Disney, P.~M.~Johnson, {\em Investigating Data Quality Problems in the PSP}.
In Proceedings of the Sixth International Symposium on the 
Foundations of Software Engineering, Orlando, FL., November, 1998.

  
\item P.~M.~Johnson, D.~Tjahjono, {\em Assessing software review
    meetings: A controlled experimental study using CSRS}. In
  Proceedings of the 1997 International Conference on Software Engineering,
  Boston, MA., May 1997.

\item D.~Wan and P.~M.~Johnson, {\em Computer Supported Collaborative
  Learning using CLARE: the Approach and Experimental Findings}.  
  In Proceedings of the 1994 ACM Conference on Computer Supported
  Cooperative Work, Chapel Hill, NC.  1994.
  
\item P.~M.~Johnson, {\em Supporting Technology Transfer of Formal
  Technical Review through a Computer Supported Collaborative Review
  System}.  In Proceedings of the Fourth International
  Conference on Software Quality, Reston, VA. 1994
  
\item P.~M.~Johnson, {\em An Instrumented Approach to Improving Software
  Quality through Formal Technical Review}. In Proceedings of
  the 16th International Conference on Software Engineering, Sorrento,
  Italy. 1994.
  
\item P.~M.~Johnson, D.~Tjahjono, D.~Wan, R.~Brewer, {\em Experiences
  with CSRS: An Instrumented Software Review Environment}.  In Proceedings
  of the 11th Annual Pacific Northwest Software Quality Conference,
  Portland, OR. 1993.
  
\item P.~M.~Johnson, D.~Tjahjono, {\em Improving Software Quality through
  Computer Supported Collaborative Review}.  In Proceedings of the Third
  European Conference on Computer Supported Cooperative Work, Milan, Italy.
  1993.
  
\item P.~M.~Johnson, {\em Supporting Exploratory CSCW with the EGRET
  Framework}. In Proceedings of the ACM 1992 Conference on Computer
  Supported Cooperative Work, Toronto, Canada. 1992.

\item P.~M.~Johnson, D.~Hildum, A.~Kaplan, C.~Kay, and J.~Wileden,
  {\em An Ada Restructuring Assistant}.  In Proceedings of the
  Fourth Annual Conference on Artificial Intelligence and Ada,
  Fairfax, VA. 1988.

\item D.~Corkill, K.~Gallagher, and P.~M.~Johnson, {\em Achieving
  Flexibility, Efficiency, and Generality in Blackboard Architectures}. 
  In Proceedings of the Sixth National Conference on
  Artificial Intelligence (AAAI-87), Seattle, WA.

\item P.~M.~Johnson and W.~G.~Lehnert, {\em Beyond Exploratory Programming:
  A Methodology and Environment for Natural Language Processing}. In
  Proceedings of the Fifth National Conference on
  Artificial Intelligence (AAAI-86), Philadelphia, PA.

\end{Conference Publications}  


\begin{Workshop Publications}

\item R. Brewer, G. Lee, Y. Xu, C. Desiato, M. Katchuck, P. Johnson, 
{\em Lights Off. Game On. The Kukui Cup: A Dorm Energy Competition},
Proceedings of the 2011 CHI Workshop on Gamification, May, 2011.

\item P. Johnson, {\em Ultra-automation and ultra-autonomy for software
engineering management of ultra-large-scale systems}, Proceedings of the
2007 Workshop on Ultra Large Scale Systems, Minneapolis, Minnesota, May,
2007.

\item H. Kou, P.~M.~Johnson, 
{\em Automated recognition of low-level process: A pilot validation study
of Zorro for test-driven development}, Proceedings of the 2006 
International Workshop on Software Process, Shanghai, China, May 2006.

\item P.~M.~Johnson, M.~G.~Paulding, 
  {\em Understanding HPC Development through Automated Process and Product Measurement with Hackystat},
  Proceedings of the Second Workshop on Productivity and Performance in High-End Computing (P-PHEC), 
 February, 2005.

\item P.~M.~Johnson, 
  {\em You can't even ask them to push a button: Toward ubiquitous,
  developer-centric,
empirical software engineering}, Proceedings of the Workshop
on New Visions for Software Design and Productivity: Research
and Applications, December, 2001.

\item P.~M.~Johnson, 
  {\em Project {LEAP}, Lightweight, Empirical, Anti-measurement
  dysfunction, and Portable Software Developer Improvement}, 
  Software Engineering Notes, Volume 24, Number 6, December 1999.


\item P.~M.~Johnson, 
  {\em Egret: A Framework for Advanced CSCW Applications}, 
  Software Engineering Notes, Volume 21, Number 5, September 1996.


\item P.~M.~Johnson,
  {\em Assessing software review meetings: An empirical study using CSRS}, 
  Appearing in the 1996 International Software Engineering Research Network
  Meeting (ISERN'96), Sydney, Australia, August, 1996.


\item P.~M.~Johnson and Carleton Moore,
  {\em Investigating Strong Collaboration with the Annotated Egret Navigator}, 
  Appearing in the Fourth IEEE Workshop on Enabling 
  Technologies: Infrastructure for Collaborative Enterprises (WET ICE 95),
  April, 1995.

  
\item P.~M.~Johnson, {\em Computer Supported Formal Technical Review with
    CSRS}, Software Inspection and Review Organization Newsletter, Volume
  5, Number 3, December, 1994.

\item P.~M.~Johnson, 
  {\em Collaboration-in-the-large vs. Collaboration-in-the-small}.  
  Appearing in Proceedings of the 1994 CSCW Workshop on Software Architectures for
  Cooperative Systems, Chapel Hill, VA. October, 1994.

\item P.~M.~Johnson, {\em From Principle-centered to
  Organization-centered Design: A Case Study of Evolution in a  Computer-Supported Formal Technical Review Environment}. 
  Appearing in the Proceedings of the 15th Interdisciplinary Workshop
  on Informatics and Psychology, Scharding, Austria, 1994.
  
\item P.~M.~Johnson, 
      {\em Report from the 1993 ECSCW Workshop on
           Tools and Technologies.} SIGOIS Bulletin, April, 1994.

\item P.~M.~Johnson,
      {\em Methodological Issues in CSCW Research}.  Position paper for 
      the 1993 European Conference on Computer Supported Cooperative Work,
      Workshop on Tools and Technologies, Milan, Italy.  1993.

\item P.~M.~Johnson, 
     {\em An Architectural Perspective on EGRET}.
     In Proceedings of the ACM 1992 Conference on Computer Supported
     Cooperative Work, Workshop on Tools and Technologies,  Toronto, Canada. 
     1992.

\item P.~M.~Johnson, 
     {\em Collaborative Software Review for Capturing Design Rationale}.
     In Proceedings of the 1992 AAAI Workshop on AI and Design Rationale,
     San Jose, CA.
     1992.

\item D.~Wan and P.~M.~Johnson, 
     {\em Supporting Scientific Learning and Research Review using COREVIEW}.
     In Proceedings of the 1992 AAAI Workshop on Communicating Scientific
     and Technical Knowledge, San Jose, CA.
     1992.
     
\item P.~M.~Johnson, {\em EGRET: Exploring Open, Evolutionary, and
     Emergent Collaborative Systems}. In Proceedings of the 1991 European
     Conference on Computer Supported Cooperative Work, Tools and
     Technologies Workshop, Amsterdam, The Netherlands. 1991.

\item P.~M.~Johnson, 
     {\em Structural Evolution in Exploratory Software Development}.
     In Proceedings of the 1989 AAAI Spring Symposium on AI and Software
     Engineering, Stanford University, CA.
     1989.

\item S.~Founds and P.~M.~Johnson, 
     {\em A Knowledge-based Rhythm Composition Tool}.
     In Proceedings of the 1989 IJCAI Workshop on Artificial Intelligence
     and Music, Detroit, MI. 1989.


\item P.~M.~Johnson, 
     {\em Integrating BB1-style Control into the Generic Blackboard System}.
     In Proceedings of the 1987 AAAI Workshop on Blackboard Systems,
     Seattle, WA.
     1987.
     

\item P.~M.~Johnson, 
     {\em Combining Software Engineering and Artificial Intelligence}.
     In Proceedings of the First International Workshop on Computer-Aided
     Software Engineering, Cambridge, MA.
     1987.

\item D.~Corkill, K.~Gallagher, and P.~M.~Johnson, 
     {\em From Prototype to Product: Evolutionary Development from within
     the Blackboard Paradigm}.
     In Proceedings of the Workshop on High-level Tools for Knowledge-based
     Systems, Columbus, OH. 1986.

\item P.~M.~Johnson, 
     {\em Requirements Definition for a PLUMber's Apprentice}.
     In Proceedings of the Second Annual Workshop on Theoretical Issues in
     Conceptual Information Processing, New Haven, CT.  1985. 


\end{Workshop Publications}


%% \begin{Tutorial Presentations}


%% \item {\em Java: What's it all about?} Half-day tutorial presented at
%% the Pacific New Media Center, Honolulu, HI, November, 1997.

%% \item {\em The TekInspect Software Review Method: Reviewer Training} and 
%% {\em The TekInspect Software Review Method: Moderator Training}. 
%% Two day tutorial presented at Tektronix, Inc, Beaverton, OR. July, 1996.

%% \item {\em Inspection Quick Start: An accelerated, pragmatic introduction
%% for software engineers and managers.}
%% Full-day tutorial presented at Tektronix, Inc, Beaverton, OR. May, 1996.

%% \item {\em Improved Formal Technical Reviews: Beyond Fagan Code
%% Inspections}.  Half-day tutorial presented at the 17th International
%% Conference on Software Engineering, Seattle, WA. April, 1995.

%% \item {\em Formal Technical Review: Theory and Practice --- Past,
%%   Present, and Future}.  Half-day tutorial presented at the Fourth
%%   International Conference on Software Quality, Reston, VA. 1994

%% \item {\em UNIX and Trends in Next Generation Operating Systems}.  Half-day tutorial
%%   presented at the Japan-America Institute for Management Science. Honolulu,
%%   HI. 1992, 1993, 1994.

%% \end{Tutorial Presentations}


%% \begin{Invited Talks}
%% \item {\em  Tool Support for Experimentation.} Panel chair and member, presented at
%% the 2003 meeting of the International Software Engineering Research
%% Network, Rome, Italy, 2004.

%% \item {\em  e-World Experimentation.} Panel chair and member, presented at
%% the 2001 meeting of the International Software Engineering Research
%% Network, Strathclyde, Scotland, August, 2001.

%% \item {\em  Experimental Software Engineering in Internet Startups: An
%% oxymoron?} Talk presented at the 2000 meeting of the International Software Engineering Research
%% Network, Honolulu, HI. October, 2000.

%% \item {\em  Process improvement is dead!  Long live the rising tide!} Talk 
%% presented at Microsoft Corporation, Seattle, Washington.  November, 1999.

%% \item {\em Questioning assumptions in the {PSP}, Part {II}}. Talk presented 
%% at the 1999 meeting of the International Software Engineering Research
%% Network, Oulu, Finland. June, 1999.

%% \item {\em Introduction to Software Engineering in the Collaborative
%% Software Development Laboratory}.  Talk presented at
%% ``A presentation of leading-edge scientific projects and programs
%% for the Honorable Benjamin J. Cayetano, Governor, State of Hawaii''.
%% University of Hawaii, April, 1999.


%% \item {\em Investigating Data Quality Problems in the PSP}.  Talk presented at the 
%% Sixth International Symposium on the Foundations of Software
%% Engineering, Orlando, FL, November, 1998.

%% \item {\em From the PSP to Project LEAP}.  Talk presented at the 
%% 1998 Meeting of the International Software Engineering Research
%% Network, Naperville, IL, October, 1998.


%% \item {\em What's Quality Got To Do With It?}  Talk presented at the 
%% School of Information Technology, University of Queensland, St. Lucia,
%% Australia, July, 1997.
  
%% \item {\em Java: What's so Special?}  Talk presented at the Annual
%%   Meeting of the Data Processing Management Association, Western Region.
%%   Honolulu, HI, September, 1996.
  
%% \item {\em Object Oriented Programming: From Scandinavia to the South
%%     Pacific, Simula to OO Cobol}.  Talk presented to the Data Processing
%%   Management Association, Honolulu Chapter, April 1996.

%% \item {\em Reengineering Inspection: The Future of Formal 
%%   Technical Review}. Talk presented at: Tektronix, Inc, Beaverton, OR;
%%   Andersen Consulting, Chicago, IL; Motorola, Inc, Austin, TX. 1996.
  
%% \item {\em Supporting Software Quality through Computer Supported Formal
%%   Technical Review}. Invited talk for the Distinguished Software
%%   Scientists Visiting Speaker Program, Tektronix, Inc. Beaverton,
%%   OR. 1994.

%% \item {\em Supporting Technology Transfer of Formal
%%   Technical Review through a Computer Supported Collaborative Review
%%   System}.  Talk presented at the Fourth International
%%   Conference on Software Quality, Reston, VA. 1994
  
%% \item {\em Lessons Learned from Designing Computer-mediated Collaboration
%%   for Formal Technical Review.} Talk presented at the 15th
%%   Interdisciplinary Workshop on Informatics and Psychology, Sch\"{a}rding,
%%   Austria, 1994.
  
%% \item {\em An Instrumented Approach to Improving Software Quality through
%%   Formal Technical Review.} Talk presented at: the 16th International
%%   Conference on Software Engineering, Sorrento, Italy; the University of
%%   Aalborg, Denmark. 1994.

%% \item {\em Experiences with CSRS: An Instrumented Software Review
%%   Environment.} Talk presented at the Eleventh Annual Pacific Northwest
%%   Quality Conference, Portland, OR. October, 1993.
  
%% \item {\em Improving Software Quality through Computer-Supported Formal
%%   Technical Review.} Invited talk presented at the Naval Command, Control,
%%   and Ocean Surveillance Center, San Diego, CA. October, 1993.

%% \item {\em Improving Software Quality through Computer-Supported
%%   Collaborative Review.} Talk presented at the 1993 European Conference on
%%   Computer Supported Cooperative Work. Milan, Italy. 1993.
  
%% \item {\em Why CSCW Developers are Bad at CSCW Research.} Keynote address
%%   presented at the 1993 European Conference on Computer Supported Cooperative
%%   Work, Tools and Technologies Workshop. Milan, Italy. 1993.

%% \item {\em CSRS: A Collaborative Software Review Environment}.  Poster
%%   session at the 15th International Conference on Software Engineering.
%%   Baltimore, MD. 1993.

%% \item {\em Working Notes on Collaborative Design of a Collaborative
%%   Spreadsheet}. Talk presented at the 1992 International Conference on
%%   Computer Supported Cooperative Work, Workshop on Tools and Technologies.
%%   Toronto, Canada.  1992.
  
%% \item {\em Supporting Exploratory CSCW with the Egret Framework}.  Talk
%%   presented at the 1992 International Conference on Computer Supported
%%   Cooperative Work. Toronto, Canada. 1992.
  
%% \item {\em Coordination in the Egret Framework}.  Talk presented at the
%%   1992 AAAI Workshop on AI and Design Rationale. San Jose, CA. 1992. 
  
  
%% \item {\em International Aspects of Computer Supported Cooperative Work}.
%%   Talk presented at the Hawaii International Software Conference. Honolulu,
%%   HI. 1991
  
%% \item {\em Type Flow Analysis for Exploratory Software Development}.
%%   Talk presented at: MITRE Corporation, University of South Carolina at
%%   Columbia, Siemens Corporation, University of California at San Diego,
%%   MCC, Portland State University, University of Hawaii at Manoa, and the
%%   University of Massachusetts at Amherst. 1990.
  
%% \item {\em An Ada Restructuring Assistant}. Talk presented at the Fourth
%%   Annual Conference on Artificial Intelligence and Ada. Fairfax, VA. 1988.
  
%% \item {\em Integrating BB1-Style Control into the Generic Blackboard System}.
%%   Talk presented at the 1987 AAAI Workshop on Blackboard Systems.
%%   Seattle, WA. 1987.
  
%% \item {\em How can Knowledge-based Approaches Help Computer-Aided
%%   Software Engineering?}  Panel member at the First International Workshop
%%   on Computer-Aided Software Engineering. Cambridge, MA. 1987.
  
%% \item {\em Combining Software Engineering and Artificial Intelligence.}
%%   Talk presented at the First International Workshop on Computer-Aided
%%   Software Engineering. Cambridge, MA. 1987.
  
%% \item {\em Beyond Exploratory Programming: A Methodology and Environment
%%   for Natural Language Processing.} Talk presented at AAAI 1986.
%%   Philadelphia, PA.  1986.
  
%% \item {\em An Introduction to the Plumber's Apprentice}.  Talk presented
%%   at General Electric Corporation Research and Development. Schenectady, NY.
%%   1986.
  
%% \item {\em Requirements Definition for a Plumber's Apprentice.} Talk
%%   presented at the 2nd Annual Workshop on Theoretical Issues in Conceptual
%%   Information Processing. New Haven, CT. 1985.
  
%% \item {\em Computers in Psychophysiology: Hardware and Software
%%   Considerations}.  Lecture series presented at the Veterans Hospital, Ann
%%   Arbor, MI. 1982.

%% \end{Invited Talks}


\begin{Awarded Grant Support}

\item {\em Supporting next generation energy education with the Kukui Cup},
P.M. Johnson, Principal Investigator,
HEI Charitable Foundation, \$10,777. 2012.

\item {\em Research Experience for Undergraduates (REU) Supplemental Funding}, P.~M.~Johnson, Principal Investigator.
  National Science Foundation.
    \$16,000. 2012.

\item {\em Research Experience for Undergraduates (REU) Supplemental Funding}, P.~M.~Johnson, Principal Investigator.
  National Science Foundation.
    \$16,000. 2011.

\item {\em Human centered information integration for the smart grid}, P.~M.~Johnson, Principal Investigator.
  National Science Foundation.
    \$381,000. 2010-2013.

\item {\em Renewable Energy and Island Sustainability}, Principal Investigator: T. Kuh; Co-Principal Investigators:  O. Boric-Lubecke,
B. Chao, M. Coffman, D. Garmire, M. Nejhad, R. Ghorbani, P. Johnson, A. Kavcic, D. Konan, B. Liaw, E. Miller, W. Qu, M. Teng, 
X. Zhou. University of Hawaii. \$1,000,000. 2009-2011.

\item {\em Google Summer of Code Sponsored Project (Hackystat)}, P.~M.~Johnson, Principal Investigator. Google, Inc.  \$45,000. 2008-2009.

\item {\em CSDL Grant}, P.~M.~Johnson, Principal Investigator. Expedia, Inc.  \$25,000. 2008.

\item {\em CSDL Grant}, P.~M.~Johnson, Principal Investigator. Sixth Sense Analytics, Inc.  \$25,000. 2006.

\item {\em Student Engagement Grant}, P.~M.~Johnson, Principal
Investigator. University of Hawaii and Maui High Performance Computing Center.  \$42,000. 2004, 2005.

\item {\em Eclipse Innovation Grant Award}, P.~M.~Johnson, Principal
Investigator.  IBM Corporation. \$15,000. 2004.

\item {\em Supporting development of highly dependable software through
continuous, automated, in-process, and individualized software measurement
validation.}  P.~M.~Johnson, Principal Investigator.  Joint NSF/NASA 
Highly Dependable Computing Program.  \$638,000.  2002-2006.

\item {\em  Aligning the financial services, fulfillment distribution
infrastructure, and small business sectors in Hawaii through B2B technology 
innovation.}    
P.~M.~Johnson, Principal Investigator.
University of Hawaii New Economy Research Grant Program.
\$30,000.  2000-2001.


\item {\em  Internet Entrepreneurship: Theory and Practice. 
}
  P.~M.~Johnson and Glen Taylor, Principal Investigators.
  University of Hawaii Entrepreneurship Course Development Grant. 
    \$10,000. 1999-2000.


\item {\em Java-based software engineering technology
                   for high quality development in "Internet Time"
                   organizations.}
  P.~M.~Johnson, Principal Investigator.
  Sun Microsystems Academic Equipment Grant Program.
    \$39,205. 1999.


\item {\em Project LEAP: Lightweight, Empirical, Anti-measurement
dysfunction, and Portable Software Developer Improvement.}
  P.~M.~Johnson, Principal Investigator.
  National Science Foundation.
    \$265,000. 1998-2001.


\item {\em Internet-enabled Engineering Tool for Dynamically Analyzing and
Planning World-Wide Subsea Cable and Array Installations.}
  P.~M.~Johnson, Principal Investigator.
  Makai Ocean Engineering, Inc.
    \$83,286. 1998-1999.

\item {\em Kona: A distributed, collaborative technical review environment.}
  P.~M.~Johnson, Principal Investigator.
  Digital Equipment Corporation External Research Program.
    \$101,413. 1997.
    
  \item {\em Collaborative Software Development Laboratory Industrial
      Affiliates Program: Makai Ocean Engineering, Inc.}, P.~M.~Johnson,
    Principal Investigator. \$10,000. 1997.
  
\item {\em Collaborative Software Development Laboratory Industrial
    Affiliates Program: Tektronix, Inc.}, P.~M.~Johnson, Principal
    Investigator. \$45,000. 1996-1998.

\item {\em Improving Software Quality through Instrumented Formal Technical
  Review}, P.~M.~Johnson, Principal Investigator.  National
  Science Foundation.  \$161,754. 1995-1997.
  
\item {\em Collaboration Mechanisms for Project HI-TIME: Hawaii
  Telecommunications Infrastructure Modernization and Expansion: A Model for
  Statewide Strategic Planning}, P.~M.~Johnson, Principal Investigator.
  Subcontract with the Pacific International Center for High Technology
  Research.  \$30,280. 1995

\item {\em Three Dimensional Interfaces for Evolving Collaborative Systems.} P.
  Johnson, Principal Investigator.  University of Hawaii Research Council
  Seed Money Grant, \$5,000. 1992-1993.
  
\item {\em Support for Structural Evolution in Exploratory Software
  Development}.  P.~M.~Johnson, Principal Investigator.  National Science
  Foundation Research Initiation Award Program in Software Engineering.
  \$54,810.  1991-1993.
  
\item {\em An Investigation of Software Structure Evolution}.
  P.~M.~Johnson, Principal Investigator.  University of Hawaii Research
  Council Seed Money Grant, \$6,000.  1990-1991.

\end{Awarded Grant Support}


%\begin{Pending Grant Support}
%\item {\em Kona: A distributed, collaborative technical review environment.}
%   P. Johnson, Principal Investigator.
%   Submitted to the Digital Equipment Corporation External Research Program
%   June, 1996.  \$90,694.
%%% 
%%% %\item {\em Continuous Software Quality Improvement through
%%% %  Industry-centered Computer-supported Formal Technical Review}, P. Johnson,
%%% %  J. Lee, Principal Investigators.  Submitted to the NSF
%%% %  Transformations to Quality Organizations program, May, 1994. \$400,336
%%%   
%%%   
%%% \item {\em An Automated Electronic Review System: A Collaborative Task
%%%   with the University of Hawaii}, I. Kierk, Principal Investigator.
%%%   Submitted to the Jet Propulsion Laboratory Continuous Improvement
%%%   Initiative Program, April, 1994.  \$180,000.
%%% 
%%% %\item {\em National Science Foundation Young Investigator Award},
%%% %  P.~M.~Johnson, Nominee.  Submitted to the National Science
%%% %  Foundation, January 1994. \$500,000 (includes \$375,000 on a matching
%%% %  funds basis).
%%% 
%%% 
%%% %%% \item {\em Automated Support for Process Maturation of Formal Technical
%%% %%%   Review}, P.~M.~Johnson, Principal Investigator.  Submitted to the Advanced
%%% %%%   Research Projects Agency, Department of Defense.  \$722,969.
%%% 
%%% \end{Pending Grant Support}
%%% 


\begin{Professional Activities}

\item {\em Program Committee Member}, Energy 2011, 2012.

\item {\em Program Co-Chair}, First International Workshop on In-Process
Software Engineering Measurement and Analysis, Montreal, Canada, October, 2007.

\item {\em Program Committee}, Third International Workshop on Software
Engineering for High Performance Computing System Applications, 
Minneapolis, MN,  May, 2007.

\item {\em Editorial Board}, Journal of Empirical Software Engineering, 2004-2008.

\item {\em Program Committee Member}, PROFES 2005-2010.

\item {\em Program Committee Member}, Workshop on Productivity and
Performance in High-End Computing, 2005-2006.

\item {\em Program Chair}, Second International Workshop on Software
Engineering for High Performance Computing System Applications, St. Louis,
MO, May, 2005.

\item {\em Program Chair}, First International Workshop on Software
Engineering for High Performance Computing System Applications, Edinburgh,
Scotland, May, 2004.

\item {\em Program Committee Member}, XP/Agile Universe,  Calgary, CA, August 2004.

\item {\em Program Committee Member}, International Software Metrics Symposium, 2003-2004. 

\item {\em Program Committee Member}, International Symposium on Empirical Software Engineering, 2002-2004.

\item {\em Editorial Board}, IEEE Transactions on Software Engineering, 2000-2004.

\item {\em Program Chair}, International Software Engineering Research Network Annual Meeting, Honolulu, HI, 2000.

\item {\em Member}, State of Hawaii Millenium Workforce Development Initiative, 1999.

\item {\em Program Committee Member}, 
  European Conference on Computer Supported Cooperative Work, 
  Copenhagen, Denmark, 1999.

\item {\em Judge}, 
  Hawaii State Science Fair,
  Honolulu, Hawaii, 1998-present.

\item {\em Founder and Chair}, 
  Hawaii Java Users Group,
  Honolulu, Hawaii, 1996-present.

\item {\em Member}, International Software Engineering Research Network
(ISERN), 1996-present. 

\item {\em Program Committee Member}, 
  European Conference on Computer Supported Cooperative Work, Lancaster,
  England, 1997.

\item {\em Advisory Board Member}, 
  The International Journal of Computer Supported Cooperative Work, 1997-2004.

\item {\em Editor}, the WWW Formal Technical Review Archive.  \newline
  http://www.ics.hawaii.edu/$\sim$johnson/FTR/.

\item {\em Editor}, the WWW Software Inspection and Review Organization
(SIRO) Home Page. \newline
  http://www.ics.hawaii.edu/$\sim$siro/.

\item {\em Program Organizer}, Software Architectures for Cooperative
  Systems Workshop, 1994 ACM Conference on Computer Supported Cooperative Work,
  Chapel Hill, North Carolina.

\item {\em Program Chair}, CSCW Tools and Technologies Workshop, 1993
  European Conference on Computer Supported Cooperative Work, Milan, Italy.
  
\item {\em Program Organizer}, CSCW Tools and Technologies Workshop,
  1992 ACM Conference on Computer Supported Cooperative Work, Toronto, Canada.
  
\item {\em Reviewer}, IEEE Transactions on Software Engineering, IEEE
Software, ACM Transactions on Software Engineering and
  Methodology, Hawaii International Conference on System Sciences, Sixth International Conference on Computing and Information,
  IEEE Computer, the 1993 Conference on Organizational Computing Systems, the
  1993 International Conference on Computer Applications in Industry and
  Engineering, the Journal of Collaborative Computing, Artificial
  Intelligence in Engineering, Design, and Manufacturing (AI-EDAM), ACM
  Transactions on Programming Languages and Systems, The AI Handbook, Volume
  4 (Chapter on AI and Software Engineering), the 1991 Conference on Software
  Maintenance.


\end{Professional Activities}


\begin{Awards and Honors}

\item Guitarist, Mabanzi Marimba Band, Hawaii Music Award Winner, World Music Category, 2006.
\item Coach of the Year, American Youth Soccer Association, Region 100, 2001.
\item Honorary member, Golden Key International Honour Society, 2001.
\item University of Hawaii Presidential Citation for Meritorious Teaching, 1994.
\item Computer and Information Science Department Fellowship, 1989-1990.
\item University of Massachusetts Graduate School Fellowship, 1985-1986.

\end{Awards and Honors}

\end{document}






