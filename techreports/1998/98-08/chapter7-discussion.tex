
\chapter{Discussion} 

The 1539 errors found in this case study give at least partial support to
my hypotheses that
\begin{enumerate}
\item The PSP suffers from a collection data quality problem.
\item Manual PSP suffers from an analysis data quality problem.
\end{enumerate}
In this chapter I will evaluate the collection and analysis data quality
problems, discuss some implications of these problems in the evaluation of
existing PSP data, and outline possible directions for future research.


\section{Further refinement of the PSP model} 

The next few sections will cover the causes and prevention of collection
and analysis stage errors.  To summarize the concepts covered in previous
chapters, I expand the model shown in Figure \ref{fig:model} to outline the
nature of the errors associated with each stage and their correction
methods:

% \begin{figure} [bht] 
% \begin{center}
% \caption{\label{fig:errorModel} PSP Error Model}
% \begin{tabular}{|l|l|l|}\hline 
% {\bf Collection stage errors:}
% & Detectable & Correct value is clear or
%                        unimportant \\ \hline
%                        &            & Correct value is important, but
%                        unclear     \\ \hline 
%                        & Hidden     & Missing values \\ \hline
%                        &            & Incorrect values \\ \hline
%Analysis stage errors   & & \\ \hline   
%\end{tabular}
%\end{center}
%\end{figure}


\begin{itemize}
\item {\bf Collection stage errors:} Occur in such fields as Time Recording
  Log start/stop times, all Defect Recording Log fields, and program size
  measures.
   \begin{itemize}
   \item {\bf Detectable:} Errors that are clearly present, despite being
     in primary data.  Examples include PSP phases without Time Recording
     Log entries or missing dates in the Defect Recording Log.  There are
     two types of detectable errors:
      \begin{itemize}
      \item Correct value is clear or unimportant.  When attempting to fix
        these errors, the value is not significant (e.g. a Defect Recording
        Log date) or can be reliably deduced from other data (e.g. a
        missing Time Recording Log date for a phase that has a
        corresponding entry with a date in the Defect Recording Log).
      \item Correct value is important but unclear.  These errors can never be corrected
        with complete confidence.  An attempt can be made to fix them using
        a set of rules to provide consistency.
      \end{itemize}
    \item{\bf Hidden:} Errors that could only have been detected by direct
      observation of the person using PSP and subsequent comparison with
      the recorded data. These errors can be made either intentionally or
      unintentionally.
      \begin{itemize}
      \item Missing values.  Primarily missing Time and Defect Recording
        Log entries.
      \item Incorrect data. Data values that exist but do not reflect what
        actually happened.
      \end{itemize}
   \end{itemize}
 \item {\bf Analysis stage errors:} Given reliable primary data, these can
   be reliably fixed by re-doing the appropriate calculations or PSP
   operations.
\end{itemize}


Hidden errors, which are of special interest, could have several sources:
\begin{itemize}
\item Missing data values of which the programmer is not aware. This
  includes defects not recorded because the programmer became completely
  absorbed in fixing a tricky bug.
\item Inaccurate data values of which the programmer is not aware.  For
  example, a faulty {\it Start} value in a Time Recording Log entry.
\item Data values created in an attempt to recover lost information or
  information not recorded due to preoccupation with programming tasks,
  etc.  For example, if a programmer is coding and is interrupted by a
  phone call, but forgets to time the interruption, he or she will have to
  guess the number of minutes to enter in the {\it Interruption Time}
  column on the Time Recording Log.
\item Willful recording (or non-recording) of inaccurate data due to
  outside pressures such as an employer performing evaluations based on a
  PSP measure or a class that requires the use of PSP from a person not
  committed to using PSP.
\end{itemize}

\section{The Collection Problem}

In the two-stage model of the PSP illustrated in Figure \ref{fig:model},
the collection stage starts with actual work and ends with records of work.
Problems in this stage occur when the records of work (size, time, and
defect data,) do not accurately reflect the actual work done. As stated in
Section \ref{section:DirectCollection}, 90 errors were found from the
collection stage.  These 90 errors lend at least some support to the
hypothesis that PSP suffers from a collection data quality problem.
However, as outlined in Section \ref{section:IndirectCollection}, there are
indications that the actual number of collection stage errors was much
higher.  Analysis stage errors are relatively easy to find, since the
analysis steps can be duplicated and the resulting data sets compared.  The
collection stage cannot be duplicated, so unless there are internal data
conflicts, there is no reliable way to determine if a student padded time
log entries, failed to record defects, or incorrectly measured program
size.  

Humphrey does acknowledge that there can be problems in this area:
\begin{quote}
Data gathering can be time consuming and tedious.  To be consistently
effective in gathering data, you must be convinced of its value to you.  If 
you do not intend to use the data, the odds are you will either not gather
them or the data you gather will be incomplete or inaccurate.  This is
especially true for the PSP... The principal issue is whether the data you
gather are for your personal use or for someone else's.  If you are
gathering data for someone who sets your pay or evaluates your work, you
will likely be careful to show good results.  If it is for your personal
use, however, you can be more objective (p. 226, 227) \cite{Humphrey95}.
\end{quote}

\subsection{Causes}

One possible cause of the collection problem is measurement dysfunction.
Chapter 2 discusses this concept in more depth, but briefly, measurement
dysfunction in software development is a situation in which organizational
forces lead to the conscious or unconscious skewing of data to support the
trends desired by management, even when the true trend is the opposite of
that portrayed by the data.  As a simple example, using the PSP should lead
to lower defect density over time.  It is easy for a programmer to achieve
such an effect: simply record fewer and fewer defects in the Defect
Recording Log.

The problem with teaching PSP in a classroom setting is that any grades
given must be based, at least in part, on the completed PSP forms.  It is
difficult for the instructor to communicate convincingly to the students
that the actual values recorded (if the result of correctly doing the PSP)
do not impact their grades.  Students can have trouble understanding, for
example, that the instructor is evaluating their Defect Recording Logs
based on completeness and correctness, but {\it not} on the number of
defects injected while programming.  Similarly, with LOC per hour, students
may not understand that the instructor is looking for a correct computation
rather than high productivity or a tendency for improvement over time.
Therefore, a situation develops which is ripe for measurement dysfunction.
Consciously or unconsciously, students feel pressure to improve the
accuracy of their estimates and the quality of their programs.  By
manipulating data gathered in the collection stage, it is very easy to
change the outcome of the derived measures such as LOC per hour, yield, and
the cost-performance index.

Another possible cause of problems in the collection stage is pressure for
time.  Given that defect data collection is quite time consuming, and given
that students are under significant time pressures later in the semester,
it is possible that observed downward trends in PSP defect data could be
due more to increasing demands upon student time than to improvements in
their development skill.

Developers using the PSP collect time and defect data during the
development phases of design, code, compile, and test.  It can be
distracting to do the work involved in these phases while simultaneously
recording measures about the work.  This can lead to mistakes in recording
or can even cause developers to forget to record data altogether, if they
become so absorbed in development that they forget that they're doing PSP
at the same time.  (In a worst-case scenario, a person could be so
distracted by recording defects during coding that he or she could actually
inject more defects than would have occurred without the PSP.)  Even if a
developer becomes aware that he or she has forgotten to record, say, a
phone call interruption, just estimating the interruption at a later time
is a source of error in itself.  Errors caused by distraction could
especially affect the accuracy of defect fix times.  Often after fixing a
defect, a programmer has no idea whether the fix took 15 or 35 minutes, and
the PSP forms do not facilitate the collection of this piece of data.

Simple laziness, even fitful periods of laziness, is also a source of
collection error.  It is very easy to skip the recording of ``little''
defects or typos, especially when fixing a defect takes less time than
recording it.  Of course, this is closely related to a person's motivation
for doing the PSP at all.  If students are recording time and defects
simply to get forms filled out so that they can get a good grade in a
class, the completeness of the data means very little to them.  Undeniably,
there are times, such as when a defect is injected while fixing another
defect, when defect recording is irritating to anyone.  However, an
internally motivated programmer is much more likely to take the small extra
steps that produce accurate work measures.  

Finally, there are plenty of ways to make ``stupid mistakes'', even when
collecting data as simple as time and defects.  For example, a user could
write down 8:20 for planning start time when he really meant 9:20, give
defects the wrong types or phases, or mix up lines added and lines deleted
when copying results from a line-counting program to the PSP form.
% \footnote{Stupid mistakes? At least programmers are better than users. An
%  AST tech support person reported that one customer complained that her
%  mouse was hard to control with the dust cover on it.  The dust cover
%  turned out to be the plastic bag in which the mouse was packaged...  At
%  Dell, a customer called to say he couldn't get his computer to fax
%  anything.  After 40 minutes, the tech discovered the man was trying to
%  fax a piece of paper by holding it in front of the screen and hitting the
%  ``send'' key...  At Compaq, one customer was having diskette problems.
%  After trouble shooting for a while (magnets, heat, etc.), tech asked the
%  customer what else was being done with the diskette.  Response: ``I put a
%  label on the diskette, roll it into the typewriter...''.  }

\subsection{Prevention}

If a programmer's PSP data is used, or even viewed, by any other person, it
is possible for measurement dysfunction to exist.  Therefore, to prevent
collection errors, any such situation should be examined to see if
measurement dysfunction is occurring.  If the person with access to a
programmer's PSP data is an instructor or manager, the problems should be
obvious, since such a person has the ability to create negative
consequences for ``bad'' PSP measures, or to reward ``good'' ones.
However, even peers can exert pressure.  Because PSP makes personal
measures such as LOC/hour very visible to peers, unspoken competition might
develop among co-workers, or a student might have bad feelings about being
a ``worse'' programmer than others in the class.  In both these situations, a
person could feel tempted to record fewer defects or less time for a
project, therefore improving certain quality or productivity measures.

Well-designed automation would probably also reduce the number of
collection errors.  A good application could help to prevent many
``stupid'' mistakes by automatically filling in values such as start and
stop times.  The user would still have to indicate that he or she was
leaving a particular phase, but the actual time stamp and delta time could
be created by the computer.  The application could also provide timing for
defect fixes, which would preventing collection errors generated by
requiring the user to guess fix times.  It could automatically measure
programs used in a project, calculating LOC added, modified, and deleted;
and store the numbers for future computations in the analysis stage.

An automated version of the PSP could make doing PSP much simpler.  This
would free the user from some of the inherent distraction involved in using
the PSP, leaving more mental room to remember to take the necessary steps
for data collection.  It would also allow the user to remain physically
focused on the terminal and keyboard, which could particularly help in
defect recording.  All unwanted hand, arm, body, and eye movements away
from the screen to the desktop could be eliminated; defects could be
entered much faster.  This might make a person more inclined to record a
defect encountered during compile or test.

Finally, to prevent collection errors, PSP users must truly desire to
improve their work. Even with no measurement dysfunction and a
smooth-functioning automated PSP application, a person can still be sloppy
about data collection.  Motivation, therefore, is highly important.
However, producing and maintaining a positive, meticulous mind set is a
challenge that each programmer must address individually.  Humphrey
suggests that skilled coaches can help in this area:
\begin{quote}
  Without motivation, professionals do not excel; with motivation, they
  will tackle and surmount unbelievable challenges.  While there are many
  keys to motivation, the coach's first task is to find these keys and
  devise the mix of goals, rewards, and demands that will achieve
  consistently high performance.
\end{quote}

\section{The Analysis Problem}

In the two-stage model of the PSP illustrated in Figure \ref{fig:model},
the analysis stage starts with records of work and ends with analyzed work.
Problems in this stage occur when PSP users make any kind of errors in this
analysis, whether incorrectly performing computations, providing the wrong
data for computations, or choosing the wrong statistical methods. As
outlined in Section \ref{section:Analysis}, 1479 of the errors found in the
case study were committed during the analysis stage.  This lends strong
support to the hypothesis that manual PSP suffers from an analysis data
quality problem.  The support is only strengthened by the fact that these
1479 errors do not include the thousands of fields with incorrect values
that were the result of subjects performing subsequent analysis steps using
this flawed data.  As shown in Section \ref{section:Correction}, steps
taken to correct analysis stage errors created a second set of data that
included substantial changes for some significant measures, demonstrating
that the analysis errors found were not merely ``noise''.

\subsection{Causes}

Why did the subjects make so many mistakes? Part of the problem lies in the
nature of manual PSP.  To begin with there are a lot of forms.  PSP0 starts
out with four scripts and four forms.  PSP1 has five scripts and eight
forms.  By PSP2 there are five scripts, ten forms, and two checklists.
Seven of these forms and checklists are likely to extend to multiple pages
for even a medium size project. 
% \footnote{PSP3? It would be easy to make
%  errors just {\em counting} them: seven scripts, two
%  checklists, and 16 forms. \\
%  A programmer once went to a school, \\
%  Thought to learn PSP - 'twould be cool. \\
%  But, oh!, all those forms, \\
%  Regression and norms -- \\
%  She lit a fire with Humphrey as fuel. \\
%  (The book, not the man - don't worry.)} 
Not only are there are a lot of forms, there are a lot of fields on those
forms.  It isn't possible to give an exact number of fields per process,
since many of the forms, such as the Defect Recording Log, have a variable
number of entries.  However, Dr. Johnson estimates that the average number
of fields per project probably ranges from about 200 for PSP0 to at least
530 for PSP2.  There are 177 fields on the PSP2 Project Plan Summary form
alone.

Just having a lot of forms with a lot of fields shouldn't make things
overwhelmingly complicated. However, there are other factors involved.
First, on these numerous and complex forms, not all fields are applicable
to the current phase -- whatever phase that happens to be.  It is easy to
get confused about what form and field you are supposed to be filling in
now and what should be saved for a later phase.  Second, there are data
dependencies between the forms for a single project which involve a
constant transferring of data from paper to paper.  Almost every form has
data that must be summarized and sent to another form, or must itself have
data from another form in order for the user to complete the calculations.
Every one of these transfers is an opportunity for an error to be made,
either by transferring the wrong number or by transferring the right number
to the wrong place.  Third, there are many calculations and operations that
involve prior projects.  Just locating the correct project, form, and field
can be frustrating and time-consuming, and there are the same opportunities
for transfer errors.  Particularly for size and time estimation, the user
must leaf through a pile of old projects, or rely on a possibly inaccurate
list of the pertinent values such as planned and actual size and time.  At
best, this list is yet another form to maintain. 
% \footnote{And one should
%  be careful about adding new forms or modifying processes: Q. How many
%  workers at Rocky Flats, the former nuclear weapon components plant,
%  should it take to change a light bulb?\\ A. Sixteen -- and that's no
%  joke.  An internal memo written by a manager at the U.S. Department of
%  Energy plant recommended a new safety procedure for ``the replacement of
%  a light bulb in a criticality beacon.''  The beacon, similar to the
%  revolving red lamp atop a police car, warns workers of nuclear accidents.
%  The memo said the job should take at least 16 people over 60 hours to
%  replace the light.  It added that the same job used to take 12 workers
%  4.15 hours.} 
Additionally, when learning the PSP, all the scripts and the set of forms
change with each new process.  Therefore, as the user learns the PSP,
familiar forms and scripts are constantly changing.

Another factor contributing to analysis error is the textbook
\cite{Humphrey95} itself.  Admittedly, the material it covers is both
complex and extensive, but in some areas the instructions are not very
clear.  The main problem, however, is not the clarity of the instructions,
but their location(s).  For a single form, a PSP user might have to locate
three or more references in the book.  For example, the Size Estimating
Template is introduced on page 120 with a sample form showing a sample
project.  It is discussed over the following four pages.  The full
instructions are not given until page 684 (Appendix C, PSP Process
Elements).  A sample form is shown there, but no examples - all the fields
are blank.  In the instructions, references are made to Appendix A, Table
A27 (no page number given) for the procedure to calculate regression
parameters, and to Appendix A, Table A29 (no page number given) for the
procedure to calculate the prediction interval.  There are also some notes
about size estimation on page 621 (Appendix C, PSP Process Contents).
Further instructions about size estimation appear on page 679-80 in the
PROBE estimating script, phases 4-8.  However, there is no information
about how to combine the steps in this script with the seemingly
overlapping set of instructions on page 684.

Furthermore, instructions for new fields appear at inconsistent places in
the text.  The formula for {\it Cost Performance Index} appears in Appendix
C, Process Contents; while the formulas for the {\it Cost-of-Quality}
fields appear in the text itself, in Chapter 9.  An example worked out for
the Size Estimating Template appears in Chapter 5, but the example for the
PSP1 Project Plan Summary form is in Appendix C, Process Contents.
Instructions for carrying out time estimation are given in chapter 6, but
there is no index entry for ``time, estimation''.  The PSP user must look
under ``estimating, development time''.  All this makes it difficult for a
developer to find answers, or to feel confident that he or she has all the
pertinent information even after finding a useful reference.

The actual computations are another source of analysis errors.
Fortunately, the book does have a good 80-page appendix covering such
subjects as statistical distributions, numerical integration, tests for
normality, linear regression, multiple regression, prediction intervals, and
Gauss's method.  The appendix includes explanations and examples as well as
formulas.  However, even after gaining a good understanding of these
things, it is all too easy for a programmer to make a mistake when working
out a range calculation by hand:
\begin{displaymath}
\mbox{Range = }
t(\alpha / 2,n - 2)\sigma  \mbox{ }
\sqrt{1 + \frac{1}{n} + 
\frac{(x_{k} - x_{avg})^{2}}  {\sum_{i=1}^n (x_{i} - x_{avg})^{2}}
}
\end{displaymath}
or when writing a helper program to calculate correlation:
\begin{displaymath}
r(x,y) = \frac{n\sum^n_{i=1}x_iy_i - \sum^n_{i=1} x_i \sum^n_{i=1} y_i}
{\sqrt{
\left[ n\sum^n_{i=1}x^2_i - \left( \sum^n_{i=1}x_i \right)^2\right]
\left[ n\sum^n_{i=1}y^2_i - \left( \sum^n_{i=1}y_i \right)^2\right]
}}
\end{displaymath}
In the standard PSP curriculum, these particular equations are among those
that students implement as they create PSP tools while completing the
outlined assignments.  The students then use the tools to do the PSP in
future assignments, so if students don't accurately implement the
equations, the very tools they use to do the PSP can introduce analysis
errors.

It is also easy to make mistakes in choosing among the various statistical
methods to use, particularly for size and time estimation.  The user must
also make decisions about which completed projects to include as historical
data and which ones to treat as outliers.

Finally, it is important to remember that the PSP doesn't take place in a
vacuum.  The user has to deal with all the forms, scripts, instructions,
and computations {\em at the same time} that he or she is carrying out
another highly demanding intellectual task - software development.  The
part of the planning phase which contains the most difficult statistical
steps occurs right between the conceptual design and detailed design steps.
The Time Recording Log and Defect Recording Log must be filled out in the
midst of the design, code, compile and test phases.  This probably accounts
for the high incidence of ``stupid errors'' in trivial calculations.
Finding the number of minutes between start and stop times is not a
difficult task, but for 4 out of 9 assignments, this mistake affected 7 to
9\% of Time
Recording Log entries. % see /usr/thesis/timelog.errors
Programmers practice the PSP during development, and their attention is
inevitably split between the two tasks.
%  \footnote{
%  A developer started to write \\
%  A routine for sorting - so trite.\\
%  But he used PSP \\
%  Couldn't focus, you see, \\
%  And it ended out taking all night. \\}


\subsection{\label{section:AnalysisErrorPrevention}Prevention}

When considering ways to prevent analysis errors, the main answer is
automation.  Virtually every analysis error I discovered in the student
data could have been prevented if the students had used a well-designed PSP
tool. Such a tool could completely free users from the maze of phases,
forms, scripts, and instructions. The tool could ``understand'' the process
scripts well enough to guide the user through the phases, correctly
ordering the forms and presenting only the needed fields in the proper
order. The 275 blank field errors, 142 entry errors and 16 sequence errors
point to the need for these features.  The tool could, of course, do all
calculations automatically.  The 705 calculation errors show that this
should be a primary requirement. The tool could provide context sensitive
help at all times.  It could handle all intra-project and inter-project
data transfers and take over the management of project groups.  A user
would not have to remember anything about prior projects in order to do
size and time estimation or view to date values.  212 inter-project
transfer errors and 99 intra-project transfer errors show that this kind of
functionality is necessary.
% \footnote{This tool should probably not be developed by the X Windows
%  architects: ``If the designers of X Windows built cars, there would be no
%  fewer than 5 steering wheels hidden about the cockpit, none of which
%  would follow the same principles - but you'd be able to shift gears with
%  your car stereo, a useful feature at that.''}

An all-in-one approach appears essential in designing a useful PSP
application.  During the PSP course, students had many tools available to
them, ranging from calculators to spreadsheets to Java programs for size
estimation and line counting.  However, besides being a source of error, it
is irritating, distracting, and time-consuming to use six or seven separate
tools when it would be possible to seamlessly include every needed service
within one package.  An integrated approach could help to shift the user's
focus from the complexities of ``doing PSP'', to actually looking at the
data.  When steps for the postmortem phase take two minutes, (as opposed to
the average postmortem time in this study -- about 90 minutes) %92.66 minutes
it seems more likely that a user would take the time to really look at the
results and think about the insights revealed.  This kind of approach could
also allow the user to see results of data analysis in many different
formats and give the user access to raw data for further study.

Although Humphrey describes a manual approach to PSP in {\it A Discipline for
Software Engineering} \cite{Humphrey95} and {\it Introduction to the Personal
Software Process} \cite{Humphrey97}, he also indicates that automation is
desirable.  For example, 
\begin{quote}
Tool support can make the methods described in
this book more efficient and easier to use.  Such standard aids as word
processors, spreadsheets, and statistical packages provide an adequate base 
initially, but ultimately CASE environments are needed that embody the PSP
methods in engineering workstations, in addition to all the other tools
generally available (p. 26) \cite{Humphrey95}.
\end{quote}
Also, in reference to the collection stage,
\begin{quote}
Tools to assist in data gathering are feasible and could certainly
help.  They would probably not save a lot of time, but they could
significantly improve data accuracy and completeness... Once the data are
gathered and are in a database, many automatic analysis tools could assist
in estimating, planning, and progress reporting (p. 219) \cite{Humphrey95}.
\end{quote}

Moving away from automation, what are other ways to prevent analysis
errors?  Manual PSP could be improved by making some simple modifications
to the forms.  Many fields contain numbers that must be transferred to
other forms.  Often the value in a field must be added to the value in the
same field from a different project to provide a to date value.  Students
often correctly transferred a value from the wrong field when doing these
kind of transfers.  These errors could be prevented by visually linking the
field being transferred from with the field being transferred to.  For
example, if {\it Program Size (LOC), Total New \& Changed (N)} fields were
printed as boxes drawn with a double line, and {\it Program Size (LOC),
  Total LOC (T)} fields were printed as boxes drawn with a dotted line, I
believe it would cut down significantly on the transfer errors for size and
time estimation.  Figure \ref{fig:formBefore} shows a portion of the
current Project Plan Summary for PSP2.  Figure \ref{fig:formChanged} shows
one possibility for formatting changes that could reduce the number of
transfer errors.

\ls{1}
\begin{figure}[p] 
    {\centerline{\psfig{figure=formBefore.eps}}}
    \caption[Current Project Plan Summary Form]{\label{fig:formBefore} 
      {\em Portion of Current Project Plan Summary Form for PSP2}}
\end{figure}
\ls{1.5}

\ls{1}
\begin{figure}[p] 
  {\centerline{\psfig{figure=formChanged.eps}}}
    \caption[Possible Modifications to Project Plan Summary Form]{\label{fig:formChanged} 
      {\em Possible Modifications to Project Plan Summary Form for PSP2}}
\end{figure}
\ls{1.5}


Finally, to cut down on problems with finding and interpreting PSP
instructions, it would be useful to have a PSP reference booklet for each
process (such as PSP0, PSP0.1, etc.).  The booklets could contain samples
of all the forms used in the process and complete instructions for each
one. They could also include process scripts and definitions for key
concepts. Obviously a lot of the information would be duplicated from
booklet to booklet, but only pertinent information would be included, and
instructions could be much better ordered.


\section{Evaluating PSP Data}

Because of the questions this study raises about the quality of PSP data, I
believe that PSP data should not be used to evaluate the PSP method itself.
In other words, it is not appropriate to infer that changes in PSP measures
during or after a training course accurately reflect changes in the
underlying developer behavior.  A statement such as, ``The improvement in
average defect levels for engineers who complete the course is 58.0\%'', if
based on PSP data alone, might only reflect a decreasing trend in the
recording of defects, rather than a decreased trend in the defects present
in the work product.

\newpage
This is not to imply that all PSP evaluations are based upon PSP data
alone.  For example, in one of the case studies \cite{Ferguson97}, evidence
for the utility of the PSP method was based upon reductions in acceptance
test defect density for products subsequent to the introduction of PSP
practices.  Although alternative explanations for this trend can be
hypothesized (such as the PSP-based projects were more simple than those
before and thus acceptance test defect density would have decreased
anyway), at least the evaluation measure is independent of the PSP data
and is not subject to PSP data quality problems.

It could be argued 
%\footnote{... and has!} 
that this conclusion is invalid because Dr. Johnson did not focus on data
quality during the course and that the study just revealed this isolated
problem after the fact.  After all, students will make errors, and are
unlikely to improve without faculty guidance; with proper faculty guidance,
error rates should drop significantly.  However, as outlined in Section
\ref{section:ModifiedPSP}, Dr. Johnson addressed data quality even before
the first lecture, and continued to focus on it throughout the semester.
Also, before teaching the course, he had already taught the PSP for one
semester at the graduate level and had instituted it within his research
group, the Collaborative Software Development Laboratory.  By the time of
the study, he was quite experienced as both a teacher and a user of the
PSP. Not only did he consider it 
% \footnote{... at the time, anyway} 
to be one of the most powerful software engineering practices he had
acquired in his career, but most of the students were very positive about
the PSP in their post-course evaluations:

\begin{itemize}
\item ``I think PSP is one of the few things I learned in this school that
is useful.  It will be definitely useful on my job.''
\item ``[At the beginning of the course], I didn't know anything about
software engineering.  Now I know a great deal thanks to PSP.  I now know
the importance of why a process is used to finish a task.  Software
development is not easy and using a process helps in development.  I have
learned that my design skills aren't that great but my debugging skills is
(sic) pretty good.''
\item At the beginning, I just coded to finish the project or solve the
problem.  Now I take an in-depth look at the problem and think about it for 
a while before trying to develop a solution.  By executing and learning
this process I know way more about software engineering than when I started 
this course.''
\end{itemize}
 
The projects I examined should have been of {\it at least} average data
quality.  Additionally, the analysis I did, even using automated tool
support, was extremely tedious and time consuming, often requiring two or
three hours for a single project.  It is unlikely the average PSP
instructor has the time or motivation to do this on a week-to-week basis.

It could also be argued that data quality problems are mainly confined to
student projects. While it is true that students may be less motivated and
less experienced (and therefore less accurate) than professional software
developers, the most severe errors that are outlined in Section
\ref{section:SevereErrors} are of the type that can happen to anyone.  None
of these error types are likely to be eliminated just by pointing out that
the particular type of error is occurring.  Repeatedly I found myself
stumbling over these same errors (delta time calculation, inter-project and
intra-project transfers) while trying myself to verify the student data.
Moreover, it would seem that external pressures and distractions would be
greater for software engineers in the field, and they could be even more
likely to make these kinds of errors.  Analysis of error age, covered in
Section \ref{section:ErrorAge}, showed that most PSP errors did not appear
to be a function of learning how to do PSP.  Instead, errors continued to
occur in various measures many projects after the first introduction of the
PSP steps that produced those measures.  As far as collection stage errors,
I can speak from personal experience and say that I still have days when I
don't record all my defects or get messed up on time recording.  There may
be people who do perfect data collection, but for everyone else it is
important to understand the areas in which problems can occur and the
effect such problems can have on PSP measures.


\section{Summary}
In conclusion, this thesis describes a case study of PSP to assess data
quality issues.  It has been shown that PSP users can make substantial
numbers of errors in both the collection and analysis stages, and that
these errors can have a clear impact upon measures of quality and
productivity.  However, these results do not imply that the PSP method is
wholly unuseful in improving time estimation and software quality.  On the
contrary, student evaluation of the PSP method was positive, and even if
certain numerical values are incorrect, the process still provides students
with profound new insight into the software development process.  Instead,
these results could be useful in motivating two essential types of
improvement to the PSP process: attention to measurement dysfunction issues
and integrated automated tool support.  Until questions raised by this
study with respect to PSP data quality can be resolved, PSP data should not
be used to evaluate the PSP method itself.
 
\section{Future Directions} 

Although the PSP tool developed for this case study was adequate for the
purpose of uncovering analysis errors, there is room for improvement, as
outlined in Section \ref{section:ImprovementIdeas}.  The tool could be
expanded to include all forms and processes through PSP3 and could be
enhanced to work more smoothly in some areas and to provide more
flexibility when accessing outside applications such as text editors.
When these improvements are complete, it would be possible to make the tool 
publicly available as open source software.

Since many people still do PSP manually or with limited tool support, it
would be useful to make formatting changes to the PSP forms, as outlined in 
Section \ref{section:AnalysisErrorPrevention}, and then empirically
evaluate whether such changes help to reduce the number of analysis errors.

\newpage
Finally, it would be very useful to attempt to isolate and study the
effects of measurement dysfunction on PSP data.  I do not know exactly how
this could be done.  This case study gives some idea of the scope of the
analysis problem, but more work is needed to determine the severity and
causes of the collection problem.



\bibliography{/group/csdl/bib/psp} 
\bibliographystyle{plain}

\end{document}

