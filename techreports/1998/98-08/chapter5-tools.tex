
\chapter{Software Tools}

In order to carry out the case study, I developed two software tools: the
first to automate the PSP and the second to track errors found in the PSP
projects.  Both tools were written using the Progress 4GL/RDBMS
\cite{Progress98}, version 6.3F01, running on SCO Unix, Release 3.2v4.2,
using a character based interface.  This chapter will describe the two
tools; including requirements, operational scenarios, structure, and areas
for improvement.

\section{PSP Tool}

\subsection{Requirements}

Obviously, the main requirement for this package was the ability to
reliably automate all the PSP calculations so that I could check the
original PSP data for errors.  Since for this purpose I was the only user,
a clear user interface and flexible functionality weren't very important.
However, I also wanted to go beyond what was necessary for simple data
checking and explore some ideas about a fully integrated set of PSP tools.

When I first learned the PSP, I almost immediately began creating various
small tools to help me follow the PSP processes more efficiently.  Some of
these, such as a Java LOC counter were assignments in the PSP course
itself.  I created others, such as a size estimation applet, simply because
I got tired of doing various processes by hand.  Even though these tools
relieved me of a lot of tedious work, it still seemed as though there was a
tremendous amount of overhead involved in using the PSP to do actual
software development.  I would have a stack of PSP forms (with the correct
one never on top) to the left of my keyboard, a stack of prior projects
beyond that, {\it A Discipline for Software Engineering} (never open to the
right page) for process scripts and form instructions under my elbow, and
to the right all the papers needed for whatever software project I was
working on.  Of course, my pencil, eraser, and calculator always seemed to
be {\it under} one of these papers whenever I needed them! On the computer,
I had to continuously invoke various tools, provide input, and write down
results on one of the PSP forms.  All this led me to form the requirement
that the new tool should completely eliminate any need for paper, including
the need for the textbook for process scripts and form instructions.  (Of
course, the book is still needed to {\it learn} the PSP, but not as a
day-to-day reference guide.)  It should seamlessly combine all tools in
such a way that the user never has to consciously invoke a tool or transfer
data from one tool to another.

Closely related to this was the requirement that the new tool should
provide as much guidance as possible in following process scripts.  The
user should not have to remember the order of the phases, the order of the
forms within each phase, or which fields on a particular form are required
for a particular phase. The tools should provide context-sensitive help
screens and help messages.  The computer should also calculate all possible
values.  It should provide derived values, such as {\it LOC per hour,
  Actual}, to the user in display rather than update mode.  For other
values, such as {\it Time in Phase, Total, Plan}, it should at least give
the user a default number to override.

Since PSP results aren't any better than the data that goes into them,
another requirement was that the tool should do as much collection of
primary data (time, size, and defects) as possible. Data, such as defects,
should be easy to collect so that a user is less likely to have an
incomplete record of his or her work.  The system should also collect data
automatically, whenever possible, so that a user is less likely to have an
inaccurate record of his or her work.  For example, the user should not
have to fill in time log entries or tell the tool how many LOC are in a
certain program.  However, the user should be able to override or correct
any automatically collected data.

One of the most aggravating things about doing PSP by hand is the transfer
of data between projects.  Size and time estimation require data from as
many prior projects as possible.  The user must either maintain a
spreadsheet of this data, or continually leaf through a stack of former
projects, looking for the relevant fields.  Even more data is required from
the most recently completed project, in both the planning and postmortem
phases.  The user must find the PSP forms for this project, and then
reference various values at least 45 times (for PSP2), not counting size
and time estimation.  Even worse is the situation where a programmer
discovers an error in PSP data for an older project or decides to declare a
prior project an outlier.  Recalculating to date values through prior
projects up to the current project is no fun at all!  Therefore, a
requirement for the new tool was the ability to do inter-project
management.  The tool should allow the user to categorize each project by
PSP process, programming language, and process type; as well as allowing
certain projects to be marked as outliers.  Other than the PSP process, the
user should be able to change any of these characteristics even after the
project is complete.  The tool should have the ability to recalculate all
PSP data for all prior projects at any time.  Furthermore, when the user is
working on a new project, the tool should automatically reference relevant
prior projects to provide all appropriate ``carry forward'' values.

Since reports make collected PSP data even more useful, the ability to do
at least the reports outlined in {\it A Discipline for Software Engineering}
was another requirement.

A final requirement was that the tool should be as flexible as possible.
Since Humphrey intended that developers should modify the PSP for their own
needs, users should be able to do such things as define new processes,
change existing process definitions, maintain individual coding standards,
or add new defect types.  Users should also be able to customize the system
by such actions as modifying help screens, adding or modifying table
entries for programming languages, or changing menu option descriptions.

\subsection{Operational Scenarios}

Operational scenarios are descriptions of users actually using a system.
They provide a story-like rather than manual-like overview of an
application.  They are not intended to cover all system functions or every
possible action by the the user.  Instead, they provide small
representations of what it would be like to actually use an application.
These scenarios are examples of the way I actually used the tool while
checking the student data, and illustrate how the tool could be used by
future PSP users.

\subsubsection{Scenario One: Starting a Project Using PSP1}

A developer decides to start work on a new software project using PSP1.
She goes to the main PSP menu and selects ``Work on a Project''.  The menu
disappears and the header for the main project screen comes up. This header
can be seen on the top part of the screen shown in Figure
\ref{fig:toolSample1}. The system prompts her for a project number.  Since
she is starting a new project, she follows the instructions in the help
message and hits return on the blank field.  

\begin{figure} [htp]
    {\centerline{\psfig{figure=sample1.eps}}}
    \caption[PSP Tool: Main Project Screen]{\label{fig:toolSample1}
    {\em PSP Tool: Main Project Screen}}
\end{figure}

The system looks in her personal profile for the next project number to
assign, verifies that it hasn't already been used, and assigns it to the
new project, displaying ``29'' on the screen. It then prompts her for the
other data elements needed for project management: process, programming
language, programmer, project name, project type, date started, and outlier
status.  Based on system information and her personal profile, the system
provides defaults for process, programming language, programmer, date
started, and outlier status.  The developer accepts these values and fills
in the project name and assigns the project a type of ``N'' for new.  She
then presses F1.

The system sees from the header data that she will be using PSP1.  It finds
the database record previously created to define process ``PSP1'', builds a
list of the phases that are involved, and displays their names on the
screen.  There are two additional columns for {\it Planned} time in phase
and {\it Actual} time in phase.  The developer knows that although she can
arrow up and down over the phase list and select any one, the first phase
listed is the phase she should use first if she plans to follow the
standard process script.  Therefore, she presses return on ``Plan''.
Immediately an asterisk appears by ``Plan'' on the screen, showing that
this is the current phase.  Under the {\it Actual} Time in Phase column,
the amount of time spent in Planning is incremented second by second.

Immediately, a box appears titled, ``Problem Description''.  She types in
``Client LYX requests WP interface to dictation sequences.'' and presses
F1.  The box disappears and a new box comes up titled, ``Requirements
Statement''.  She types in a number which references the hardcopy
requirements document generated in a planning meeting the previous week,
and presses F1.  A text editor then fills the screen with a document
already named, created, and pulled up for her, titled ``Produce a
Conceptual Design''.  She takes about half an hour to do this work, and
exits the editor. A new box appears titled ``Program List'', which prompts
her for the code location and the names of the programs she will be adding
or modifying.  The code location defaults to the one stored in her personal
profile, so she just types in the name of the one program she will be
modifying and the names of four new ones. After she presses F1, the system
automatically counts and displays the number of lines in each program (zero
for the new ones).  It also makes backup copies so that after development
it can determine the number of lines that were added, modified, or deleted.

Then the ``Size Estimating Template'' box appears, which prompts the
developer for all the fields in the top part of the standard Size
Estimating Template.  After she fills them in and presses F1, another box
appears, showing the results of all calculations done from the lower part
of the standard Size Estimating Template.  There is a message in the box
informing her that calculations have been done, ``Using projects from
current language/process/type.''  The ``Project Plan Summary'' box appears,
showing default values for {\it Program Size (LOC), Plan}: base, deleted,
modified, added, reused, and total new reused.  She glances at them and
presses F1.  Then the time estimation box appears, with an estimate for
total time and a message ``From regression calculation on estimated object
LOC and actual hours.''  She accepts the default value of 661 minutes and
presses F1.  The system uses to date percentages from her most recent
similar project to distribute the time across the phases, and again
presents her with default values.  She presses F1 again and is returned to
the main project screen.  The screen now displays the planned values for
each phase, and ``00:33:35'' under {\it Actual} for the phase ``Plan'', as
shown in Figure \ref{fig:toolSample1}.  She moves the cursor down one line
to ``Design''.  Invisibly, the system creates a time log entry for the
planning phase.


\subsubsection{Scenario Two: Defect Density Report} 
A software engineer has been conscientiously using the PSP for about six
months, and wonders if the quality of his work on new projects is improving
or not.  He goes to the main PSP menu and selects ``Reports''.  The system
displays the Reports Menu and he selects ``Defect Densities''.  The system
hides the Reports Menu and displays the driver screen for the Defect
Density Report.  The screen prompts him for process, project type,
language, date range and display type. Using his default values stored in
database file, it fills in ``PSP2'' for process, ``N'' for project type
(new projects), ``Progress'' for language, 06/12/98 and 09/04/98 for date
range (first and last project completed of type ``N'' using PSP2), and
``D'' for display type.  He doesn't need to change anything, and presses F1
to proceed.  Instantly, a box appears with a line for each of the 25 new
projects he completed in Progress using PSP2 during the selected date
range.  For each project the system displays project number, new and
changed LOC, number of defects, and defects per KLOC.  The last line
provides total values and the average number of defects per KLOC, as shown
in Figure \ref{fig:toolSample2}.  He is happy to see that the later
projects do appear to have fewer defects.

\begin{figure} [htp]
    {\centerline{\psfig{figure=sample2.eps}}}
    \caption[PSP Tool: Defect Density Report]{\label{fig:toolSample2}
    {\em PSP Tool: Last Page of Multi-Page Defect Density Report}}
\end{figure}

\subsubsection{Scenario Three: Adding a Defect Model}
Note: In the PSP tool, {\it defect types} are used as defined in {\it A
Discipline for Software Engineering}, and refer to general defect
categories such as Syntax, System, Environment, and Data. I have added a
new classification, {\it Defect models}, which refer to specific defects
within a defect type, such as ``Missing semicolon'' or ``Confusing field
label''.  There are three reasons for this setup: 1) The user can record in
the Defect Recording Log exactly what went wrong.  This eliminates the
mental overhead of deciding which defect type to use and cuts down on
typing in explanatory comments.  2) It provides for a more consistent
classification of defects.  3) It allows for finer grained defect reporting
without extra work and without sacrificing any of the functionality
required to produce the standard PSP defect reports using defect type.

A programmer notices that when recording defects he is continually using
the defect model code ``GSYN'' for ``General Syntax Error'' and then using
the comment line to explain ``CUM function set error''.  He decides that he
wants to add a new defect model to the system to make his defect recording
process faster. He goes to the main menu and selects the ``Maintain
Tables'' menu option.  A new menu appears.  He then selects the ``Defect
Models'' menu option.  The menu disappears, and the Defect Model Code
Maintenance screen appears.  

Following the instructions in the help message, he decides on a new code
``CE'' and types it in.  He then enters a description ``CUM functions:
incorrect use'' and moves to the next field which is ``Defect Type Code''.
He doesn't know what the defect type codes are, so he presses F6 and a list
of all defect types appears.  Using the down arrow key, he moves through
the list until ``Syntax'' is highlighted and presses return.  The defect
type code ``SYN'' is filled in automatically, and the addition of a new
defect model is complete.

The next time he is testing, he finds a defect caused by his incorrect use
of a CUM function.  He opens a new defect in the Defect Recording Log.  The
system numbers the new defect, fills in the date automatically, and places
his cursor on the ``Model'' code field.  He doesn't remember the new code
for CUM errors, so he presses F6 and is presented with a list of all defect
models.  He types in ``CUM'' and presses return.  ``CUM functions: used
incorrectly'' appears under his cursor, as shown in Figure
\ref{fig:toolSample3}, and he presses return again.  ``CE'' is
automatically placed in the defect model code field.  No comment is
necessary.

\begin{figure} [htb]
    {\centerline{\psfig{figure=sample3.eps}}}
    \caption[PSP Tool: Defect Recording Log]{\label{fig:toolSample3}
    {\em PSP Tool: Browsing for a defect model while adding a new entry to the Defect Recording Log}}
\end{figure}

\subsubsection{Scenario Four: Correcting a Completed Project}
A software developer has an extra half hour late one Friday afternoon.  She
doesn't want to start anything new, so she starts going through the stack
of papers that tends to accumulate on one of the back corners of her desk.
She finds a slip of paper that says: ``McNall lab interface, 3 hours design,
03/22''.  She remembers that one Sunday back in March she did some work at
home and never recorded it in her PSP database.  

She turns to her screen and selects ``Work on a Project'' from her main PSP
tool menu.  The system displays the main project screen and prompts her for
``Project\#''.  Of course she can't remember the project number, so she
presses F6 and selects the correct project from the list that appears.
When the header fields appear, she changes the project status to
``Incomplete'' and presses F1.  Following the help message on the bottom of
the screen, she presses F5 and the project time log appears.  She verifies
that no time was recorded for March 22.  Again following the help message,
she presses F9 and creates a time log entry for the three hours of design
work.  Then she presses F4 and returns to the main project screen.  She
selects the postmortem phase.  All the recalculations take about 1 second.
She then presses the space bar as 6 boxes showing postmortem data appear.
The system automatically marks the project as complete, and returns her to
the main project screen, where she presses F4 twice.  When the system
returns her to the main PSP menu, she selects ``Recalculate To-Date
Values''.  A message appears ``Working on project number 1.''  The project
number quickly changes as the system recalculates the to date values for
all appropriate projects using the new time measures for the changed
project.  10 seconds later, the system is completely updated.

\subsection{Structure}

The PSP tool consists of about 80 programs and include files (subroutines)
and a Progress database. The programs and include files add up to
approximately 7000 lines of code. The database has 24 files (tables), as
shown in Table \ref{table:pspTables}, containing 237 fields (columns).  Each
field definition includes a default validation expression, help message,
display format, field description, and screen label.

\begin{table} 
\begin{center}
\caption{\label{table:pspTables}Database Files for the PSP Tool} 
\begin{tabular}{|l|l|} \hline 
File Name      & Description                                     \\ \hline\hline 

s-browse       & Ties field names to code look-up browses        \\ \hline
s-menu         & Numbers and descriptions of menus               \\ \hline
s-option       & Numbers and descriptions of menu options        \\ \hline\hline

c-control      & Default values for individual users             \\ \hline
c-dmodel       & Defect models (general defect categories)       \\ \hline
c-dtype        & Defect types (specific defect categories)       \\ \hline
c-form         & PSP forms                                       \\ \hline
c-help         & Help screens by process, phase, form            \\ \hline
c-language     & Programming languages                           \\ \hline
c-otype        & Object types (for size estimation)              \\ \hline
c-phase        & PSP phases                                      \\ \hline
c-process      & Holds ordered sets of phases for processes      \\ \hline
c-prophase     & Ties phases to processes                        \\ \hline
c-worktype     & Project types                                   \\ \hline\hline

defect         & Defect Recording Log entries                    \\ \hline
lesson         & Lessons learned/notes                           \\ \hline
PIP            & Process Improvement Proposals                   \\ \hline
prob-desc      & Problem descriptions                            \\ \hline
program-set    & Programs created or modified during development \\ \hline
project        & Projects                                        \\ \hline
requirements   & Requirements                                    \\ \hline
size-template  & Size Estimating Template data                   \\ \hline
test-report    & Test Report Template entries                    \\ \hline
time-log       & Time Recording Log entries                      \\ \hline
\end{tabular} 
\end{center} 
\end{table}

I designed the system so that certain database records (rows) must be
pre-loaded for the system to function.  These records all belong to files
whose names begin with the prefix ``s-'', and include menus, menu options,
and links between certain fields and programs providing ``browses'' of
user-defined acceptable values.  Unless the user modifies the system
through further programming, nothing should be added to or deleted from
this record set. The only thing that may be modified are the names of menus
and the names and numbers of menu options.

Unless a user wants to define his or her own processes completely from
scratch, all files beginning with the prefix ``c-'' should also have a
basic set of records pre-loaded before the user tries to start work.  These
records will define the standard PSP processes, forms, defect types,
phases, etc.  They will also provide a starting list of programming
languages, object types, and defect types. Full facilities are provided to
the user to modify these files and to add new records.

All the remaining files are records of the user's work and PSP data.  The
central file to this group is ``Project''.  Records in the other files have a
one-to-one or many-to-one relationship with records in the ``Project''
file. For example, there is one ``requirements'' record per project record
and many ``time-log'' records.

I tried to design the system in such a way that programs rarely call other
programs.  Instead, database records determine which program to run next.
Each menu option record contains a field for ``program to run''.  After
writing a simple report program, a user can add it to his or her system
simply by creating a new menu option record referencing the new program
(using the maintenance functions already written).  The records used to
define PSP forms also contain a ``program to run'' field.  To define a new
process, a user builds a list of the phases that the process will contain,
then indicates the sequence of forms to use in each phase. This makes it
easier for someone to write new programs and add them into the system,
since he or she doesn't have to understand complicated program
interactions.

\subsection{\label{section:ImprovementIdeas}Areas for Improvement}

As with most applications, the PSP tool has several areas that could use
more work.  The main deficiency is that the standard PSP has not been
completely implemented.  The only processes that are truly complete are
PSP0 and PSP0.1.  PSP1 is missing an entry mechanism for the Test Report
Template.  This is because I was never able to devise a smooth way of
switching between this form and the Defect Recording Log.  PSP1.1 does not
include the Task Planning Template and the Schedule Planning Template.
The Design and Code Review Checklists are not available for PSP2.
Finally, PSP3 has not been implemented at all.

There are also two problem areas.  The first is measuring program size.  At
least for programs written in the Progress 4GL, this is not a simple matter
of counting the number of lines in the program, since what I really want to
measure is the number of statements used.  This involves a tremendous
amount of string parsing.  I was able to implement this accurately in the
Progress 4GL, but it takes a long time - about .1 seconds per line.  Since
this is tightly integrated with the rest of the PSP tool, it is still an
improvement over other line counting methods, but the function should
probably be rewritten in C++ or even a Unix script, and then called by the
PSP tool.  The second problem involves time recording.  When a user moves
too rapidly between phases, a time log entry is not recorded.  I'm not sure
if this is a problem with my code or the outcome of a combination of quirks
in the Progress 4GL.

Of course, there are also areas that could be enhanced.  It would be nice
to add a table for editor types which would include pointers to programs to
run to create, update, and access documents using editors other than vi!
Then each user's personal profile could keep track of their preferred
editor and automatically use it when needed.  In another area, the {\it
  Actual} time in phase column on the main project screen should
automatically refresh itself after the user manually changes the time log.
Similarly, when a user changes any header data for a completed project, the
system should automatically recalculate all to date values for the projects
that follow it, rather than requiring the user to remember to take a second
step.  Finally, the programs that provide maintenance of the defect types,
defect models, phases, forms, etc., do not allow the user to delete
entries.  The user should be able to request a deletion, then the system
should check all relevant records to see if the code has been used at all.
If not, the system should delete the record, otherwise it should issue a
warning message to the user.

\subsection{Personal Experiences With PSP Automation}
My progress toward the goal of automating the PSP was recorded in various
journal entries over the eight months after my first introduction to the
PSP. Initially, I was not aware of any problems with PSP data quality;
instead, I was concerned with the amount of time manual PSP required.

It began with a interest in the PSP and the insights it provided into my
own software development habits:
\begin{quote}
  February 5, 1996\\
  I'm really enthusiastic about the PSP. ...  My software development group
  at work has absolutely no process of any kind.  I've tried using similar
  things before, but they take too much time and don't do much good.  This
  one seems to be different, however.  I've been using it for three or four
  days at work, and I think it's improving the quality of my work already.
  After 10 years of programming, not many bugs actually survive beyond
  development, but now I realize how much time I was spending killing bugs
  in the testing phase that could have been prevented by spending just five
  minutes or so more in design.  Besides, ``watching'' myself work keeps me
  from getting bored, and after almost five years on the current project
  that is getting to be a problem.
\end{quote}

But very soon, the time required began to trouble me:
\begin{quote}
  February 23, 1996\\
  Less and less convinced that the current part of the PSP [size and time
  estimation] will be helpful in my work.  Thinking back over the large
  software projects that my company has done for other organizations (where
  estimates were important), it would have been impossible to get the kind
  of detail that [PSP] requires from the prospective clients.  In addition,
  even if we could have gotten it, we could not have afforded to spend the
  weeks it would have required to design the system to the point where all
  the input screens were defined for a job that we had not even won yet!
  That would have cost us a good deal of our profit.  Besides, on every
  project that we've worked on, the requirements have changed so
  tremendously along the course of development that it would have been
  wasted time.  Not to mention all the detailed records you have to keep
  for every little programming process that take hours to maintain and
  process.  I'm disappointed, because this was getting interesting, but
  maybe I'll find something useful on which to build something useful to
  me.
\end{quote}

\begin{quote}
  March 11, 1996\\
  I'm wondering the best way to use PSP to cover as many programming
  scenarios as possible.  Right now the place I'm having trouble in is with
  [problem] fixes.  It's impossible to complete even the planning phase
  until the [problem] is verified and you know what caused it and how to
  fix it.  By then, I often need about two minutes programming time to fix
  it, but all the PSP stuff takes about 10 minutes.  So is this worth it?
\end{quote}

Despite my concerns, I continued to use PSP at work. It was helping to
improve the quality of my programs, although there were setbacks at times:
\begin{quote}
  March 11, 1996\\
  I let a major bug get into our new release (using PSP).  I'm upset about
  it - I remember working a long time on the logic, and testing it well -
  and yet it was so obvious looking at it a few days ago that it would not
  work.  What happened?  Maybe just a bad day, but that's no excuse. With
  PSP1.1 or whatever we're on now [in the PSP class at school] there is the
  test report sheet, so maybe that would have helped me to catch it.  I
  HATE it when this kind of thing happens.
\end{quote}

But I was beginning to move into the PSP mind set:
\begin{quote}
  March 15, 1996\\
  Last night, my office called with a request to have a prototype for some
  new reports planned, coded, compiled, tested, and on the computer in
  Pennsylvania by 8AM EST for an important demo.  My first impulse was to
  say ``This is such a huge job, and I don't want to stay up all night, so
  I'll just skip the PSP stuff.''  But then my second thought followed
  automatically, ``No, this will be some valuable data to collect for this
  type of project.''  I couldn't believe it!
\end{quote}

About that time I became discouraged trying to use manual PSP in the ``real
world''.  I stopped using it at work, but started to design an automated
system:

\begin{quote}
  March 20, 1996\\
  Yesterday I went into my office to find about 20 pages of faxes pouring
  out of my fax machine and all over the floor.  They were all [problems]
  to fix or [small] improvements to provide ``instant gratification'' for
  some of our pickier users.  After spending an hour on the phone clearing
  up questions, I had five hours to get it all done - so decided not to use
  PSP for this little stuff.  According to my data :-) it's taking at least
  15 minutes per project to do the PSP work.  Felt bad not to do it, but
  after I get all the programs done and the database built for the PSP
  support, it will be much faster.  It was such a relief to do the
  programming without filling out all those forms and timing every step.
\end{quote}

\begin{quote}
  April 3, 1996\\
  Haven't used PSP at work since March 20.  Need to develop the tools to
  make it easier to use.  I'm just sick of filling out all those forms.  As
  soon as school is out ...
\end{quote}

Two months after starting to use the PSP, I became aware for the first time 
of problems with data quality, and the ambiguities of defect recording:

\begin{quote}
  April 12, 1996\\
  According to my PSP data, I have one design error in about two months of
  programming.  I've mostly been doing spot improvements or error fixing,
  but I must not be recording design errors.  How do you know when you find
  one?  Sometimes I might change my mind about how to do something in the
  middle of coding, or see a better way to do it, or sometimes the
  requirements change in the middle of coding - guess that's what I should
  count.  But those aren't really defects.  And how can you possibly inject
  a defect in the planning phase?
\end{quote}

I began to get more serious about my automation project:
\begin{quote}
  Friday, April 26, 1996\\
  Worked some more on the PSP project.  Going well so far, have the menu
  structure, on-line help and browse functions, and started the database
  definition.  There are a lot of fields, but the structure is obvious.
  Getting the defect log and time log going are the most important things -
  most of the other data can be figured out later.  Having the usual
  difficulties deciding whether to store calculated values, and if so,
  which ones.
\end{quote}

But it took another four months to really be usable.  At that time I was
able to start observing differences between manual and automated PSP:
\begin{quote}
  August 5, 1996\\
  PSP0 is basically done!  Just need to add the recalculate and timer
  modification features.
  
  I'm much better about recording defects with the computer than with
  paper.  However, I'm not recording every one (some are tricky) and I'm
  not always thinking back to the correct injection point for the defect.
  
  The postmortem is sort of anticlimactic now.  I remember when I was using
  the paper and calculating it all out that it used to be kind of exciting
  to see where all my defects came from and the percent of time in each
  phase.  Now it's hardly interesting.  Maybe because I've done higher PSP
  levels since then, or maybe this is just a feature of the automation.
\end{quote}

I began to realize that automation alone would not solve all collection
problems: 

\begin{quote}
  August 26, 1996\\
  The most frustrating part of PSP at the moment is in keeping track of
  defects.  After just a day or two of using automated PSP, it became
  second nature to flip over to the first [virtual terminal screen] and
  enter the defect before starting to fix it.  The problems are:\\
  a. deciding where it was injected.  My usual impulse is to pick ``coding'',
  but I need to think back to design, or most often, a bug injected not
  because of a design error, but because of a sketchy design that left out
  the area that caused the bug.\\
  b.  deciding whether a problem counts as a defect or not.  For example,
  what if when testing a screen entry program, I don't like one of the
  labels (but no typo or mistake).  Is that a defect?  What if I was not
  the one that defined the table (includes label definition)?  What if I
  defined the table as part of another project than my current one?
\end{quote}

\begin{quote}
  September 3, 1996\\
  Yes, it seems that the real problem with [my] PSP right now is not the
  automation, but correctly tracking the defects.  Another example: say my
  design is sloppy, and I'm in the coding phase.  I have to redesign, tear
  out a little code, and implement the new design.  How do I measure the
  length of time that defect takes to remove?
\end{quote}

Eventually, things began to look up, and I committed myself to making
automated PSP an integral part of my work habits:
\begin{quote}
  September 10, 1996\\
  Now that PSP0.1 is done (except for PIP), it is more exciting to use PSP.
  Things have settled down so that the figures I see are correct, etc.  And
  the process is much less intrusive, either because I'm getting used to it
  or because it really is smoother to use.
\end{quote}

Currently, I use the PSP tool for most work-related projects, and have 121
of these projects on file.  Most are done using Progress, but I've also
used the PSP tool to record data for SPlus projects.  Processes range from
PSP0 to PSP1.1, as well as 14 projects done using a non-standard process
used to test the tool's flexibility.


\section{Error Tracking Tool}

\subsection{Requirements}

From the beginning, I viewed the error tracking tool as a special-purpose,
single-user application; with the special purpose of recording errors
found while analyzing PSP data, and the single user as myself.  Therefore,
the requirements were rather simple.

The primary requirement was that the tool should have an appropriate
database structure for recording data about the PSP errors and should
provide an entry screen that would allow me to record the errors quickly
and accurately.  If a field on the entry screen corresponded to a list of
acceptable values, then the tool should allow me to enter only one of those
values.

As I found errors, I wanted to keep track of 11 pieces of data for each
one; such as person who injected, person who discovered, assignment,
process, phase, form, etc.  (This does not include three data elements that
did not apply to every error: incorrect value, correct value, and comment
lines).  9 of the 11 fields would contain codes that would represent items
from lists of acceptable values.  Some of the lists, such as severity
levels, would contain a fixed set of values.  But others, such as defect
types, would probably grow as I worked through the case study.  Therefore,
another requirement was that the lists of acceptable values should have
maintenance programs available for adding and modifying entries, and that
each list should appear in selection mode after the user pressed F6 from
the corresponding field on the error entry screen.

A final requirement was that the tool should allow me to analyze the error
data after it had been entered.  It should do this by creating various
reports.  Since these were to be special-purpose reports created for a
single user, there was no need for fancy output formats or drivers
with long lists of options.

After implementing this tool and entering about half the projects, I
realized that I needed to keep track of estimated LOC, actual LOC, and
actual minutes as they were {\it originally} recorded on the student
projects.  This was necessary to allow me to efficiently duplicate time and
size estimation and LOC calculation using incorrect data.  Therefore, this
became the main requirement for a small addition to the original
application.

\subsection{Structure}
 
\begin{table}[ht]
\begin{center}    
\caption{\label{table:errorTables}Database Files for the Error Tracking Tool}
\begin{tabular}{|l|l|} \hline 
File Name      & Description                                             \\ \hline\hline 

s-browse       & Ties field names to code look-up browses                \\ \hline
s-menu         & Numbers and descriptions of menus                       \\ \hline
s-option       & Numbers and descriptions of menu options                \\ \hline\hline
c-dmodel       & Defect models (general defect categories)               \\ \hline
c-dtype        & Defect types (specific defect categories)               \\ \hline
c-form         & PSP forms                                               \\ \hline
c-language     & Programming languages                                   \\ \hline
c-person       & People (instructor, students, myself)                   \\ \hline
c-phase        & PSP phases                                              \\ \hline
c-process      & PSP processes                                           \\ \hline
c-severity     & Defect severity levels (impact of errors on other data  \\ \hline
c-worktype     & Project types                                           \\ \hline\hline

defect         & Errors found in student PSP data                        \\ \hline
hist           & Summary of project data by author/assignment            \\
               & (Used to keep key data on-line so that when doing       \\
               & calculations by hand I didn't have to look through      \\
               & stacks of prior projects)                               \\ \hline 
        
\end{tabular}
\end{center} 
\end{table}


The error tracking tool consists of about 60 programs and include files
(subroutines) and a Progress database. The programs and include files add
up to approximately 2000 lines of code. The database has 14 files (tables),
as shown in Table \ref{table:errorTables}, containing 49 fields (columns).
Each field definition includes a default validation expression, help
message, display format, field description, and screen label.  Besides
providing tracking of errors and historical data, it generates 12 reports.
A sample report is shown in Figure \ref{fig:toolSample4}.  This report
shows specific errors types, sorted by the number of times each error was
found during the case study.

\begin{figure} [htb]
    {\centerline{\psfig{figure=sample4.eps}}}
    \caption[Error Tracking Tool: Sample Report]{\label{fig:toolSample4}
    {\em Error Tracking Tool: Sample Report}}
\end{figure}

