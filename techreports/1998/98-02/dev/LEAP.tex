%%%%%%%%%%%%%%%%%%%%%%%%%%%%%% -*- Mode: Latex -*- %%%%%%%%%%%%%%%%%%%%%%%%%%%%
%% LEAP.tex -- 
%% Author          : Carleton Moore
%% Created On      : Tue Feb  3 11:34:30 1998
%% Last Modified By: Carleton Moore
%% Last Modified On: Mon Sep 21 10:56:47 1998
%% RCS: $Id$
%%%%%%%%%%%%%%%%%%%%%%%%%%%%%%%%%%%%%%%%%%%%%%%%%%%%%%%%%%%%%%%%%%%%%%%%%%%%%%%
%%   Copyright (C) 1998 Carleton Moore
%%%%%%%%%%%%%%%%%%%%%%%%%%%%%%%%%%%%%%%%%%%%%%%%%%%%%%%%%%%%%%%%%%%%%%%%%%%%%%%
%% 

\chapter{Project LEAP}
\label{sec:LEAP}

\section{Problem}
Project LEAP results from a recognition that many software
process improvement initiatives suffer from one or more of
the following problems:
\begin{itemize}


\item{\bf Heavyweight development process constraints.}
For example, many process improvement initiatives require adherence to
strict documentation, audit, and development phase constraints.

\item{\bf Measurement dysfunction.} The use of process metrics for
employee performance evaluation can lead to "dysfunctional" behavior which
skews the metric in the desired direction while compromising overall
organizational performance.

\item{\bf Organization-level analysis and improvement.} Typical process
measurements aggregate data collected from multiple projects and
organizations. Such data takes time to accumulate, analyze, and produce
meaningful process improvements.

\item{\bf Manual data gathering.} Measurement may involve
time-consuming clerical overhead that lowers the quality of the data and
produces resistance to its collection.

\end{itemize}


\section{Goal}
The goal of Project LEAP is to produce tools and techniques to support 
software process improvement for individual software engineers that
satisfy the LEAP constraints, detailed below.

In Project LEAP, tools and methods for software developer
improvement must satisfy four major criteria: 

\begin{itemize}
  
\item{\bf Light-weight.} LEAP support must be "light-weight".  It must be
  easy to learn, easy to integrate with existing methods and tools, and
  above all, not impose significant new overhead on the developer unless
  that investment of overhead will provide a direct return-on-investment to
  that same developer.
  
\item{\bf Empirical.} LEAP support should be quantitative as well as
  qualitative. Software developer improvement should be able to be shown
  through measurements of effort, defects, size, and so forth.
  
\item{\bf Automated.} Light-weight support for empirically-based developer
  improvement virtually demands some form of automation.  On the other
  hand, automation does not by itself guarantee light-weight processes or
  meaningful empirical evidence of improvement.
  
\item{\bf Portable.} As a developer-oriented approach, Project LEAP
  recognizes that any long-term improvement mechanism must accommodate the
  fact that software developers change jobs and companies on a regular
  basis. Useful support cannot be locked into a particular organization
  such that the developer must "give up" the data and tools when they leave
  the organization. Rather, LEAP support will be a kind of "personal
  information assistant" for their software engineering skill set.

\end{itemize}

\section{LEAP Tool Set}

\subsection{File structure}

\subsection{Architecture}

\subsection{Design}

\subsection{Metrics collected}

\subsubsection{Defect recording}
 
\subsubsection{Time recording}

\subsubsection{Size recording}

\subsection{Analysis}



\begin{itemize}
\item{Individual reviewers will control where their data is stored.  e.g. a
private disk or access controlled file.}
\item{System will aggregate the individual data and not store individual
data.  The aggregate will be accessible only to the group/review members.
The data will have mean and standard deviation so members will be able to
see how they are doing vs the group.}
\item{Number and severity of defects found will only be aggregated into
group numbers then deleted.  Only the author will have a record of the
defects and locations.  Once the review meeting is finished the ``scribe''
sheet will be only accessible to the author.}

\end{itemize} 

