%%%%%%%%%%%%%%%%%%%%%%%%%%%%%% -*- Mode: Latex -*- %%%%%%%%%%%%%%%%%%%%%%%%%%%%
%% diss-abstract.tex -- 
%% Author          : Carleton Moore
%% Created On      : Mon Oct  5 10:57:35 1998
%% Last Modified By: Carleton Moore
%% Last Modified On: Wed Oct 27 08:02:08 1999
%% RCS: $Id: diss-abstract.tex,v 1.2 1999/10/27 20:31:14 cmoore Exp $
%%%%%%%%%%%%%%%%%%%%%%%%%%%%%%%%%%%%%%%%%%%%%%%%%%%%%%%%%%%%%%%%%%%%%%%%%%%%%%%
%%   Copyright (C) 1998 Carleton Moore
%%%%%%%%%%%%%%%%%%%%%%%%%%%%%%%%%%%%%%%%%%%%%%%%%%%%%%%%%%%%%%%%%%%%%%%%%%%%%%%
%% 

\begin{abstract}
  Software developers work too hard and yet do not get enough done.  Developing
  high quality software efficiently and consistently is a very difficult
  problem.  Developers and managers have tried many different solutions to
  address this problem.  Recently their focus has shifted from the software
  organization to the individual software developer.  The Personal Software
  Process incorporates many of the previous solutions while focusing on the
  individual software developer.
  
  I combined ideas from prior research on the Personal Software Process, Formal
  Technical Review and my experiences building automated support for software
  engineering activities to produce the Leap toolkit.  The Leap toolkit is
  intended to help individuals in their efforts to improve their development
  capabilities.  Since it is a light-weight, flexible, powerful, and private
  tool, it allows individual developers to gain valuable insight into their own
  development process. The Leap toolkit also addresses many measurement and
  data issues involved with recording any software development process.
  
  The main thesis of this work is the Leap toolkit provides a more accurate and
  effective way for developers to collect and analyze their software
  engineering data than manual methods.  To evaluate this thesis I will
  investigate three claims: (1) the Leap toolkit prevents many important errors
  in data collection and analysis; (2) the Leap toolkit supports data
  collection and analyses that are not amenable to manual enactment; and (3)
  the Leap toolkit reduces the level of ``collection stage'' errors.  To
  evaluate the first claim, I will show how the design of the Leap toolkit
  effectively prevents important classes of errors shown to occur in prior
  related research. To evaluate the second claim, I will conduct an experiment
  investigating 14 different quantitative time estimation techniques based upon
  historical size data to show that the Leap toolkit is capable of complex
  analyses not possible in manual methods.  To evaluate the third claim, I will
  analyze software developers data and conduct surveys to investigate the level
  of data collection errors.
  
%  This research will show that the Leap toolkit is an effect tool for
%  individual software developer improvement.

\end{abstract}
