%%%%%%%%%%%%%%%%%%%%%%%%%%%%%% -*- Mode: Latex -*- %%%%%%%%%%%%%%%%%%%%%%%%%%%%
%% diss-eval.tex -- 
%% Author          : Carleton Moore
%% Created On      : Mon Oct  5 11:01:31 1998
%% Last Modified By: Carleton Moore
%% Last Modified On: Wed Oct 27 10:26:49 1999
%% RCS: $Id: diss-eval.tex,v 1.3 1999/10/27 20:26:58 cmoore Exp $
%%%%%%%%%%%%%%%%%%%%%%%%%%%%%%%%%%%%%%%%%%%%%%%%%%%%%%%%%%%%%%%%%%%%%%%%%%%%%%%
%%   Copyright (C) 1998 Carleton Moore
%%%%%%%%%%%%%%%%%%%%%%%%%%%%%%%%%%%%%%%%%%%%%%%%%%%%%%%%%%%%%%%%%%%%%%%%%%%%%%%
%% 

\chapter{Evaluation}
\label{sec:evaluation}
\begin{quote}
  {\em There's a large journey to be taken, of many trials. } - Joseph Campbell
\end{quote}

The main thesis of this work is LEAP provides a more accurate and
effective way for developers to collect and analyze their software engineering
data than methods designed for manual enactment. To evaluate this thesis I will
deconstruct it into three claims based upon the three intended benefits of the
Leap toolkit.

\begin{itemize}
\item{The Leap toolkit is able to prevent many important errors in individual
    software engineering data collection and analysis form occurring.}
\item{The Leap toolkit implements an approach to individual software
    engineering data collection and analysis that requires automated support
    and is not amenable to manual enactment. As a result, it enables more
    sophisticated approaches to data collection and analysis than is possible
    in a manual setting.}
\item{The Leap Toolkit reduces the level of collection stage errors by reducing
    the overhead associated with collection and by mechanisms that support
    privacy.}
\end{itemize}

The next section detail each of these claims.

\section{Claim \#1: Preventing important classes of errors}

To evaluate this claim I will discuss how the design and implementation of the
Leap toolkit addresses each of Disney's error categories: calculation error,
blank fields, inter-project transfer error, entry error, intra-project transfer
error, impossible values, and sequence errors.  Providing suitable automated
support should reduce these errors.  Relaxing some of the constraints on the
user will also remove some of these errors. Section \ref{sec:leap-bene3}
is a brief example of how I intend to address each error category.

\section{Claim \#2: Sophisticated approaches to data analysis}

To evaluate this claim I will use the Leap toolkit to conduct an experiment to
evaluate 14 different quantitative estimation processes and the developer's
estimation process and determine if there is any significant difference between
the estimation methods. This experiment requires the Leap toolkit's automation
to make the data collection and analysis possible.  


\subsection{Experimental environment}

The experiment will be performed in a student environment in the Introduction
to Reflective Software Engineering course at the University of Hawaii Manoa.
The students in the class will be developing 10 (software) projects and
recording their software processes in the Leap toolkit.  By reflecting on their
experiences and the data they collect about their software development
processes they should learn how to improve their development processes.  During
the development process they will be asked to estimate the size and amount of
effort each of each software project.  The Leap toolkit provides an automated
tool for looking at historical development data and deriving effort estimations
based upon historical size and effort data.  The Leap Time estimation tool
provides students with many different effort estimates based upon the students'
historical data.  The student can use these estimates to make their own
estimate of how long the project will be.

\subsection{Experimental variables}

\subsubsection{Independent and Dependent variables}

Since the objective is to evaluate the different quantitative time estimation
methods, the independent variable will be the estimation technique.  This means
that there is one independent variable that can take on 14 different values:
the 12 method values, the PSP value and the student's own estimate.  The
accuracy of a estimation method applies to the actual estimate itself, the mean
value of all the estimates, and to the standard deviation of all the estimates.
Therefore, there are three dependent variables in this experiment.  The first
two dependent variables are for the class as a whole.  The third dependent
variable is for each individual student's estimate.  The relative prediction
error will be calculated for both the mean and the standard deviation for the
entire class.  The following two dependent variables will be calculated for
each of the 14 different estimation methods for the entire class:
\begin{quote}
Mean prediction error$ = | $estimation mean - actual mean$ | / $actual mean

Standard deviation prediction error$ = |$ estimation std - actual std$ | / $actual
std
\end{quote}

For each individual student the following dependent variable will be calculated
for each of the 14 different estimation methods.

\begin{quote}
Relative Predicted Error$ = | $estimated time - actual time$ | / $actual time
\end{quote}

These measures cannot be measured until after the task has been completed.  The
different estimates must be calculated and chosen for each project before the
project is started.  The Leap toolkit will provide me 13 of the 14 estimates
automatically. The students will record their estimate and the method(s) they use
to obtain their estimate.

\subsubsection{Blocking variable}

Since the purpose of the experiment is to determine the effect of the
estimation method on the prediction error and different projects may have an
effect on the prediction error I will use a blocking variable to account for
this effect.  The reason that the different projects may have an effect on the
prediction error is that it may be easier to estimate the size of some projects
than others.  To distinguish between the effects of the different projects and
the effects of the estimation methods I have added a blocking variable
associated with the project number.

\subsection{The Design}

The design for this experiment has two parts, the class as a whole and each
individual student.  At the beginning of every project the student will develop
a planned size and planned effort for the project.  After the third project,
Leap will estimate the effort according to the different alternatives
(treatment, alt 1 - alt 13). The student will produce the 14th estimate.
During the project the students will record the amount of effort in Leap.
After the project is finished, the student will measure the actual size of the
project and total the actual amount of effort in Leap.  After all the projects
are finished, I will calculate the relative prediction error for the mean and
standard deviation for each of the fourteen estimation methods.  I will
calculate these values for the entire class and for each individual student.
We cannot use the data from the first three projects since the quantitative
estimation methods require at least three data points.  With ten projects we
will still have enough data points to evaluate the different estimation
methods.  Since the estimates are independently generated except for the
students' estimate there is no problem with the order of the alternatives.

\subsection{Analysis}
I am using the following relationship to model the experiment $y_{ij} = \mu + t_i + \beta_j
+ e_{ij}$ where: \begin{itemize}
\item $y_{ij} =$ the relative prediction error for alternative i
\item $\mu =$ the overall mean
\item $t_i = $the effect of  the ith treatment (estimation method)
\item $\beta_j =$ the effect of the jth block (project)
\item $e_{ij} = $residual for the ith and jth treatment.
\end{itemize}

This model can be analyzed with standard analysis of variance (ANOVA)
procedures with the null hypothesis:  

       \[H_0:  \mu_1 = \mu_2 = \mu_3 = \ldots = \mu_{14} \qquad\mbox{where}\qquad \mu_i = \mu + t_i; i = \{1, 2,3, \ldots , 14\}.\]
      
The null hypothesis states that there is no effect of estimation methods
on the prediction error.  For the general comparison of the 14 methods
the entire class' data will be compared.  The mean prediction error and
standard deviation prediction error will be calculated for each method
using the entire class' data.
\begin{quote}
Mean prediction error $ = | $ Class' estimation mean  - Class' actual mean $ | /
$ Class' actual mean

Standard deviation prediction error $  = | $ Class' estimation std. dev. -
Class' actual std. dev. $ | / $ Class' actual std. dev. 
\end{quote}

The null hypothesis can be tested for the entire class.  Rejecting the null
hypothesis only means that there is a difference between the accuracy of the
estimation methods it does not indicate which estimation technique is more
accurate.  If the null hypothesis is rejected I will use the Least Significant
Method to distinguish between the different alternative estimation methods.
For each student I will consider the relative predicted error only.  The
equation for relative predicted error is

\begin{quote}
        Relative Predicted error $ = | $ estimated time - actual time $ | / $ actual
        time
\end{quote}

I will perform similar calculations and ANOVA to determine if there is an individual difference in estimation 
methods. 

\subsection{Example Data}

Table \ref{tab:exampledata} shows some example data from my own Java development
experience.  This data is from a pilot study that I conducted this spring and
summer.  I have recorded the planned size and effort for over 20 projects
beginning in December 1997.  This data reflects the ten most recent projects
that I have data for.  I treated these ten projects just like the projects the
students in the class will.  Leap generated the 13 quantitative estimates and I
provided the student's estimate.  Since it is a single subject's data I can
only calculate the relative predicted error for this data.


\begin{table}[htbp]
  \caption{Example Project Data.}  
  \label{tab:exampledata}
  \begin{tabular}{|l|r|r|r|r|r|r|r|r|r|} \hline
    &{\bf Project}&{\bf Project}&{\bf Project}&{\bf Project}&{\bf Project}&{\bf
      Project}&{\bf Project}&& \\ 
    {\bf Method}&{\bf 4}&{\bf 5}&{\bf 6}&{\bf 7}&{\bf 8}&{\bf 9}&{\bf 10}&{\bf
      Mean}&{\bf Std. Dev.} \\ \hline
    APL&263&1150&386&63&48&60&151&303.00&393.84\\ \hline
    AAL&191&763&217&38&30&37&160&205.14&258.35\\ \hline
    LPL&315&717&509&0*&0*&0*&109&235.71&287.33\\ \hline
    LAL&186&268&183&0*&0*&9&72&102.57&109.82\\ \hline
    EPL&394&6634&234&118&116&99&102&1099.57&2242.82\\ \hline
    EAL&176&321&157&139&136&112&99&162.86&74.35\\ \hline
    APM&319&1346&405&65&101&70&174&354.29&456.16\\ \hline
    AAM&329&1336&343&55&72&50&126&330.14&460.87\\ \hline
    LPM&0*&179&484&0*&0*&0*&145&115.43&179.84\\ \hline
    LAM&136&387&318&18&42&14&111&146.57&149.23\\ \hline
    EPM&42&237&227&123&132&103&109&139.00&69.83\\ \hline
    EAM&124&747&186&144&143&114&105&223.29&232.45\\ \hline
    PSP&191&763&509&38&30&37**&109&239.57&286.23\\ \hline
    Student&240&837&265&93&56&37&141&238.42&227.94\\ \hline
    {\bf Actual}&198&3047&380&139&42&26&248&582.86&1093.39\\ \hline
  \end{tabular}
\end{table}


Based upon the above data I calculated the relative predicted error for all 14
estimation methods and all 7 projects.  Table \ref{tab:exampleerror} shows the
values for the relative predicted error.

\begin{table}[htbp]
  \caption{Example Predicted Error.}  
  \label{tab:exampleerror}
  \begin{tabular}{|l|r|r|r|r|r|r|r|} \hline
    &{\bf Project}&{\bf Project}&{\bf Project}&{\bf Project}&{\bf Project}&{\bf
      Project}&{\bf Project} \\ 
    {\bf Method}&{\bf 4}&{\bf 5}&{\bf 6}&{\bf 7}&{\bf 8}&{\bf 9}&{\bf 10}\\ \hline
    APL&0.3282&0.6226&0.0158&0.5468&0.1429&1.3077&0.3911\\ \hline
    AAL&0.0354&0.7496&0.4289&0.7266&0.2857&0.4231&0.3548\\ \hline
    LPL&0.5909&0.7647&0.3395&1.0000&1.0000&1.0000&0.5605\\ \hline
    LAL&0.0606&0.9120&0.5184&1.0000&1.0000&0.6538&0.7097\\ \hline
    EPL&0.9899&1.1772&0.3842&0.1511&1.7619&2.8077&0.5887\\ \hline
    EAL&0.1111&0.8947&0.5868&0.0000&2.2381&3.3077&0.6008\\ \hline
    APM&0.6111&0.5583&0.0658&0.5324&1.4048&1.6923&0.2984\\ \hline
    AAM&0.6616&0.5615&0.0974&0.6043&0.7143&0.9231&0.4919\\ \hline
    LPM&1.0000&0.9413&0.2737&1.0000&1.0000&1.0000&0.4153\\ \hline
    LAM&0.3131&0.8730&0.1632&0.8705&0.0000&0.4615&0.5524\\ \hline
    EPM&0.7879&0.9222&0.4026&0.1151&2.1429&2.9615&0.5605\\ \hline
    EAM&0.3737&0.7548&0.5105&0.0360&2.4048&3.3846&0.5766\\ \hline
    PSP&0.0354&0.7496&0.3395&0.7266&0.2857&0.4231&0.5605\\ \hline
    Student&0.2121&0.7253&0.3026&0.3309&0.3333&0.4231&0.4315\\ \hline
  \end{tabular}
\end{table}


The analysis of variance for the relative predictive error data is shown in
Table \ref{tab:exampleAnovaPred}.

\begin{table}[htbp]
  \caption{Example ANOVA for Predicted Error.}  
  \label{tab:exampleAnovaPred}
  \begin{tabular}{|l|c|c|c|c|c|} \hline
    {\bf Source of}&{\bf SS}&{\bf Df}&{\bf MS}&{\bf $F_0$}&{\bf p-value} \\ 
    {\bf variance}&&&&& \\ \hline
    Treatment& 7.54237769&13&0.58018289&2.03137219&0.07729167 \\
    (estimation&&&&&\\
    method)&&&&& \\ \hline
    Block (project)&14.2348694&6&2.37247824&8.30666732&0.00634850 \\ \hline
    Error&22.2776831&78&0.28561132&&\\ \hline
    Total&44.05493933918&98&&&\\ \hline
  \end{tabular}
\end{table}

Table \ref{tab:exampleAnovaPred} shows that the estimation method does not play
a significant role in the relative prediction error.  I cannot reject the null
hypothesis. The data does suggest that the blocking factor, the project, does
play a significant role in the prediction error.  This implies that the ability
to estimate the size of the project is very important. This result supports the
results from a study on the effects of PSP training\cite{Prechelt99}. Prechelt
found engineers trained in the PSP could accurately estimate their productivity
rate, but not the total size of the project or the amount of time it would take 
them.

\section{Claim \#3: Reduces collection stage errors}

To investigate this claim I will conduct a case study of the students in ICS
613. This is the most difficult part of my evaluation.  Getting at the
collection stage errors is very difficult since direct observation of all the
students is impossible.  The purpose of this case study is to determine the
level of data collection errors for the students in the experiment. I will look
for indications of collection errors.

\subsection{Case Study Method}

To find indications of collection errors I will conduct four surveys and
analyze the students' raw data.  The answers to the surveys and the analysis of
the students' data will allow me to tease out suspicious data.  All the students
in the class will be asked to fill out the four surveys and provide me with
access to their raw data.  Only students that consent will be asked to fill out
the surveys.  If any of the students drop the course or want to quit the study
their surveys will be ignored. 

\subsubsection{Ensuring Anonymity}

To ensure anonymity of the surveys we will have each student place their survey 
in an envelope, seal and sign the envelope.  This will ensure to the students
that we cannot look at the surveys before the end of the class. After the
grades have been turned in I will open the envelopes and randomly give the
students code numbers.  These numbers will be used to match up the survey
results and the student's raw data.  No where in the raw data or the surveys do 
we ask for any identifying data.  Before any results are published I will
ensure that any identifying material is removed.


%\subsubsection{Organization of this Protocol}
\subsection{Data Collection}

There are two main data collection methods of this case study: the student's
raw data analysis and the student surveys.

\subsubsection{Student's raw data}

As a part of ICS613 each student will turn in their Leap data for each of the
class' 10 projects. I have gotten consent from each of the students to have
access to their raw data.  I will use the Leap toolkit to help analyze the data.

\paragraph{Indirect Collection Error Evidence}

I will look for patterns in the data that are suspicious.  Some suspicious
trends are having time data that is in increments of five or ten minutes or
having times that start or end on the hour.  The Leap toolkit records times in
increments of seconds so it is extremely unlikely that anyone could work in
exact increments of five or ten minutes.  Such patterns indicates that the user
is editing their data.  This is just an indirect indication that the data does
not accurately reflect their actual development.

\paragraph{Direct Collection Error Evidence}

A more direct indication of collection errors is discrepancies between the time
and defect data.  If the student records large amounts of test time with very
few defects, this indicates that they are not recording all the defects that
they had to fix during the test phase.  The students are supposed to record the 
amount of time it takes them to fix each error.  If the total fix time for all
the recorded defects is substantially less than or greater than the total time
spent during test then this indicates that they did not collect accurate data.

\subsubsection{Surveys}

Portions of the surveys have been used and validated at other institutions.
However, we were unable to find validated surveys that ask all the questions
that we want answered so portions of the surveys have not been validated
externally.  If these surveys reveal interesting data, we may then start
another research project to validate these instruments for use in other
organizations. 

Each student that is in the class will be asked to fill out the surveys as they 
turn in their assignments.  In the class the students have an interview with
Dr. Johnson when they turn in each programming assignment.  During the
interview the student turns in their completed project and their Leap data.
The focus of these interviews is on the student's performance of the
assignment. I will not be collecting data about the interview, but I will get a 
copy of all the student's Leap data.  After their interview with Dr. Johnson
each student will be asked to fill out the appropriate survey.

The proposed survey schedule is
\begin{center}
  \begin{table}[htbp]
    \caption{Proposed Study Schedule.}  
    \begin{tabular}{|l|l|} \hline
      {\bf Task}&{\bf Milestone} \\ \hline
      Conduct 1st Survey of students & 04 Oct 99 \\ \hline
      Conduct 2nd Survey of students & 01 Nov 99 \\ \hline
      Conduct 3rd Survey of students & 22 Nov 99 \\ \hline
      Conduct 4th Survey of students & 06 Dec 99 \\ \hline
      Conduct student interviews & 06 - 17 Dec 99 \\ \hline
    \end{tabular}
  \end{table}
\end{center}



\subsubsection{Leap Survey \#1}
\label{area1}
This survey is intended to be a baseline survey to gather information about the 
students and gather initial use of the time recording tools of the Leap toolkit.

\paragraph{Topics}
This survey focuses on the student's programming experience, their experiences
with time collection and any issues they have with the Leap toolkit. I will
compare the student's programming experience to their feelings of pressure and
detected data collection error rate.

\paragraph{Summary of Questions for Survey \#1}

The first section of the survey asks about the student's programming
experience. The second section asks about their knowledge of software
engineering principles.  Many of these principles will be briefly covered
during the class. The third section asks the students about their time data
collection experience. I'm trying to find out which data collection tool they
use more often.  The fourth section asks the students to list three negative
aspects about the toolkit and three positive aspects of the toolkit.  See
\ref{sec:questionnaires} for the actual survey that was given to the students.

\subsubsection{Leap Survey \#2}
\label{area2}

The second survey focuses on usability issues of the Leap toolkit, the
student's use of the time recording tools and time estimation. At this point in 
the course they will have been taught how to use the time estimation and will
have used it for a few projects.

\paragraph{Topics}
The usability portion of the survey looks at the Human Computer Interface issue 
in the Leap toolkit.  I will compare their reported easy of use with
indications of collection errors.  The second time survey will allow me to see
if their perception or use of the time recording tools is changing over time.

\paragraph{Summary of Questions for Survey \#2}

The first section of the survey asks about the Leap toolkit's usability. The
second section asks the students about their time data collection experience.
I'm trying to find out which data collection tool they use more often.  The
third section asks them about their time estimation experience.  I'm trying to
find out if making a plan affected their perception of the project. The fourth
section asks the students to list three negative aspects about the toolkit and
three positive aspects of the toolkit.  See \ref{sec:questionnaires} for the
actual survey that will be given to the students.


\subsubsection{Leap Survey \#3}
\label{area3}
Survey \#3 repeats the time collection, and time estimation surveys from \#1
and \#2.  In addition it asks about their defect collection experience.

\paragraph{Topics}

This survey focuses on data collection and analysis.  It looks at time and
defect data collection and time estimation.

\paragraph{Summary of Questions for Survey \#3}
The first section of the survey asks the students about their time data
collection experience.  The second section asks them about their time
estimation experience.  The third section asks the students about their defect
collection experiences.  See \ref{sec:questionnaires} for the actual survey
that will be given to the students.

\subsubsection{Leap Survey \#4}
\label{area4}
In the last survey I again ask the students to fill out a usability survey.  I
want to learn how there perceptions have changed over the semester.

\paragraph{Topics}

This survey focuses on the students' perception of the Leap toolkit. 
\paragraph{Summary of Questions for Survey \#4}
The first section of the survey asks about the Leap toolkit's usability.  The
second section asks them about their perceptions of the Leap toolkit.  The
third section asks the students about any lessons they may have learned from
using the Leap toolkit.  See \ref{sec:questionnaires} for the actual survey
that will be given to the students.

\subsection{Possible Interviews with students}

In addition to the surveys and raw data, I will conduct interviews with any
students who are willing at the end of the course.  During this interview I will 
ask the students about data collection issues and the overhead of using the
Leap toolkit for improvement.  I will also ask them about what they learned
about their own development process.  These interviews will support the data
collected in the surveys and the raw Leap data.
 
%\section{Anticipated Results}

%\section{Limitations}



