%%%%%%%%%%%%%%%%%%%%%%%%%%%%%% -*- Mode: Latex -*- %%%%%%%%%%%%%%%%%%%%%%%%%%%%
%% abstract.tex -- 
%% Author          : Carleton Moore
%% Created On      : Wed Dec 30 09:46:12 1998
%% Last Modified By: Carleton Moore
%% Last Modified On: Fri Feb 26 11:08:20 1999
%% RCS: $Id$
%%%%%%%%%%%%%%%%%%%%%%%%%%%%%%%%%%%%%%%%%%%%%%%%%%%%%%%%%%%%%%%%%%%%%%%%%%%%%%%
%%   Copyright (C) 1998 Carleton Moore
%%%%%%%%%%%%%%%%%%%%%%%%%%%%%%%%%%%%%%%%%%%%%%%%%%%%%%%%%%%%%%%%%%%%%%%%%%%%%%%
%% 
%\documentclass[12pt]{article}
\documentstyle[twocolumn,icse99,times]{article}
%\RequirePackage{times}

%\usepackage{url}

%\oddsidemargin 0in   %   Note that \oddsidemargin = \evensidemargin
%\evensidemargin 0in
%\marginparwidth 0pt
%\marginparsep 10pt        % Horizontal space between outer margin and 
%\textheight = 8.45in
%\textwidth 6.5truein     % Width of text line.
%\topmargin 0.0in        %    Nominal distance from top of page to top of
%                        %    box containing running head.
%\headheight 0pt         %    Height of box containing running head.
%\headsep 0pt            %    Space between running head and text.
%% for some reason setting topskip to 0 creates a extra blank page in front!
%% So we use 1 pt. RSB 9/28/98
%\topskip = 1pt          %    '\baselineskip' for first line of page.

\begin{document}
\title{Project LEAP:\\
Personal Process Improvement for the Differently Disciplined}
\author{
        \hspace*{-2ex}
        \parbox{4.0in} {\begin{center}
        {\authornamefont Carleton A. Moore}\\ 
        Collaborative Software Development Laboratory\\
        Information \& Computer Sciences Department\\
        University of Hawaii, Manoa\\
        Honolulu, Hawaii 96822  USA \\
        (808) 956-6920\\
        cmoore@hawaii.edu
        \end{center} }}
\maketitle
\copyrightspace

% Page number on the initial page can be omitted in both the review and
% final submission (and should be removed in the final submission).  The
% line below does that.

\thispagestyle{empty}  % suppresses page number on first page

% In the review submission, page numbers should appear (they can be omitted
%  from the first page).  The pagestyle command below puts them in.
% In the final submission of accepted papers, page numbers should be
%  omitted; remove or comment out the pagestyle line below to omit them. 

% \pagestyle{plain}

% Use \section* instead of \section to suppress numbering for
% the abstract, acknowledgements, and references.

\section*{ABSTRACT}

\section*{Research Area}
Process Improvement, Measurement, Personal Software Process
\section*{Problem}

Software developers and managers have faced the problem of producing
quality software since the beginning of the computer age.  Many people have
studied the software quality problem and have proposed many solutions.  We
can categorize these different solutions into two groups: (1) ``Top-down''
solutions, that focus on software development as a group effort and (2)
``Bottom-up'' solutions, that focus on the individual software developer.  Some
of the many Top-down solutions include: the Capability Maturity Model,
Clean Room development, software quality assurance groups, and Formal
Technical Review.  These top down methods help improve the quality of the
software, however they may not be enough.

In the past four years, there has arisen a new focus on the individual
software developer.  One such effort is Watts Humphrey's Personal Software
Process (PSP)\cite{Humphrey95}.  In PSP, software engineers record the time
they spend programming, the defects they find in their software and the
size of the software.  Based upon these measurements, engineers can
track their productivity, make better predictions for future projects, gain
insight to what types of errors they make, and learn how to remove defects
earlier in their development process.  The PSP, as described by Humphrey,
is a completely manual process.

After two years of experience with the PSP, we noticed three general
problems.  First, we started to question the quality of the data recorded.
For example, we noticed that it is extremely difficult to accurately record
every defect made during software development, in part because of the
overhead of collection.  Anne Disney and Philip Johnson conducted a study
to look at the data quality of PSP data.  They found that there are
significant data quality issues with manual PSP.\cite{Disney98}

Second, our experiences with industrial partners, management practices, and
Robert Austin's book ``Measuring and Managing Performance in
Organizations''\cite{Austin96} made us think about the issues of
measurement dysfunction in PSP and review data.  There are many subtle
pressures on professionals to provide management with ``good'' results.
While the PSP is a private process, we question whether management directed
PSP training might not induce measurement dysfunction.

Third, after four years, the results with long term adoption of PSP are
mixed.  Pat Ferguson and others report excellent results with PSP adoption
at Advanced Information Services, Motorola and Union Switch and
Signal\cite{Ferguson97}.  However, Barry Shostak and others report poor
adoption of PSP in industry\cite{Shostak96,Emam96}.

%Fourth, we had significant problems with PSP's strict water fall process
%for software development and data collection.  We wanted the flexibility to
%define our own development processes and explore the possibilities of
%collecting different types of data on our processes.

These issues started us thinking about how to design an automated,
empirically based, personal process improvement tool to address these
issues.  Our goals are to reduce the collection and analysis overhead for
the engineer, to reduce the potential for measurement dysfunction in the
collection process, and to allow the engineer to use their own development
style.  We are also incorporating collaborative review support into our
personal process improvement tool.  Adding review support allows the
developer to gain insight from other developers.  This group input is an
important feature lacking in the PSP.  These features are intended to
improve the benefits to the engineer and the long term adoption of
empirically based process improvement.  To pursue this work, we initiated
Project LEAP, {\bf http://csdl.ics.hawaii.edu/Research/LEAP/LEAP.html}, and
began developing the Leap Tool Set as a ``reference implementation'' of
these design goals.

\section*{Research Questions}

We intend to deploy the Leap Tool Set in both academic and industry
settings in order to investigate the following research questions:
\begin{itemize}
  
\item{What are the strengths and weaknesses of the Leap Tool Set and
    empirically based process improvement?}

\item{What are the barriers to adoption of the Leap Tool Set?}
  
%\item{What are the benefits to the users of Leap?}

%\item{How do users improve after using Leap and what are the kinds of
%    improvements we can make using Leap beyond improved estimation?}

%\item{Is the integration of collaborative review and personal data
%    collection appropriate?}
  
\item{Is Leap an appropriate form of automated support for personal process
    improvement?}

\end{itemize}

Based upon these research questions we have developed the following 
testable hypotheses:
\begin{itemize}
\item{Automating data collection and analysis will lead to improved
    adoption of personal software process improvement.}

\item {Reducing the constraints on developers imposed by PSP will lead to
    improved adoption.}
  
\end{itemize}

\section*{Evaluation}

To evaluate these hypotheses we plan on conducting at least two case
studies.  We are conducting a pilot case study on senior level
undergraduates in Spring, 1999.  The undergraduates are learning about
empirically based process improvement and software engineering.  We are
training them on how to use the Leap Tool Set and they are using it to
record development features of several projects.  The students are
submitting reports on their progress and findings.  We will interview the
students to determine their feelings and attitudes toward process
improvement and the Leap Tool Set.  We will also conduct a survey of their
perceptions of the Leap Tool Set and their environment.  The interviews and
the surveys will allow us to predict the students' adoption of the Leap
Tool Set and empirically-based process improvement practices.  A few months
after they finish the class we will contact the students and ask them to
fill out a survey.  This survey will determine if they are still using
Leap, and/or any of the empirically-based process improvement concepts in
their work.  This survey will provide us with some preliminary data on the
level of adoption of Leap and empirically-based process improvement, and on
barriers to adoption of Leap.

We plan on conducting the second case study in an industrial research group
during Summer, 1999.  We will introduce the Leap Tool Set as automated
support for collaborative review and personal process improvement.  We will
train the members on the use of Leap and help them conduct reviews and
analyses.  During the training we will conduct a survey and interviews to
determine their perceptions of Leap and personal process improvement.
Again the surveys and interviews will provide data useful in predicting the 
extent of successful adoption of
Leap.  A few months after the training we will contact the members and
conduct a survey and interviews.  This survey will determine the level of
adoption of Leap, any barriers to adoption, and the member's attitude
toward personal process improvement.


\section*{Contributions}

We expect the following contributions from this research:
\begin{itemize}
  
\item{The Leap Tool Set provides a novel form of automated support for
    empirically based process improvement, including time, defect, size,
    and pattern recording and analysis. It is implemented in Java and runs
    on Windows, Unix, and Macintosh. You may download it at
    {\bf http://csdl.ics.hawaii.edu/Tools/LEAP/LEAP.html}. }
  
\item{Insight into adoption of personal process improvement.  The results
    of this research will provide new insight into barriers to adoption of
    personal process improvement.  We designed the Leap Tool Set to
    overcome several issues we conjecture to be barriers to adoption.  If
    Leap is not adopted, then this may suggest that these barriers are not
    key barriers.}
  
\item{Insight into empirical process improvement.  The use of Leap helps us
    learn what improvements we can make using empirical measurement.  The
    case studies may provide insights into the limits of empirically based
    approaches to process improvement.}
  
\item{Insight into process improvement issues.  The PSP uses the classic
    waterfall software development model, fixed forms, and a single size
    definition. Leap relaxes many of these constraints.  This research
    should help us learn if these constraints are required for effective
    process improvement. }


\end{itemize}

%%% Input file for bibliography
\bibliography{/group/csdl/bib/psp}
%% Use this for an alphabetically organized bibliography
\bibliographystyle{plain}

\end{document}
