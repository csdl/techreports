%%%%%%%%%%%%%%%%%%%%%%%%%%%%%% -*- Mode: Latex -*- %%%%%%%%%%%%%%%%%%%%%%%%%%%%
%% thesis-abstract.tex -- 
%% Author          : Carleton Moore
%% Created On      : Fri Jun  9 09:43:42 1995
%% Last Modified By: Robert Brewer
%% Last Modified On: Thu Mar  2 12:05:29 2000
%% Status          : Unknown
%% RCS: $Id: thesis-abstract.tex,v 1.2 2000/03/02 22:18:43 rbrewer Exp $
%%%%%%%%%%%%%%%%%%%%%%%%%%%%%%%%%%%%%%%%%%%%%%%%%%%%%%%%%%%%%%%%%%%%%%%%%%%%%%%
%%   Copyright (C) 1995 University of Hawaii
%%%%%%%%%%%%%%%%%%%%%%%%%%%%%%%%%%%%%%%%%%%%%%%%%%%%%%%%%%%%%%%%%%%%%%%%%%%%%%%
%% 

\begin{abstract}
%% For publication in UMI, thesis abstract must be 150 words or less.
  
  Searching the archives of electronic product support mailing lists often
  provides unsatisfactory results for users looking for quick solutions to
  their problems. Archives are inconvenient because they are too voluminous,
  lack efficient searching mechanisms, and retain the original thread structure
  which is not relevant to knowledge seekers.
  
  I present MCS, a system which improves mailing list archives through {\em
    condensation}. Condensation involves omitting redundant or useless
  messages, and adding meta-level information to messages to improve searching.
  The condensation process is performed by a human assisted by an editing tool.
  
  I describe the design and implementation of MCS, and compare it to related
  systems. I also present my experiences condensing a 1428 message mailing list
  archive to an archive containing only 177 messages (an 88\% reduction). The
  condensation required only 1.5 minutes of editor effort per message. The
  condensed archive was adopted by the users of the mailing list.

\end{abstract}
