%%%%%%%%%%%%%%%%%%%%%%%%%%%%%% -*- Mode: Latex -*- %%%%%%%%%%%%%%%%%%%%%%%%%%%%
%% conclusions.tex -- 
%% Author          : Philip Johnson
%% Created On      : Wed Apr  8 14:25:13 1998
%% Last Modified By: Philip Johnson
%% Last Modified On: Mon Aug 10 12:21:33 1998
%% RCS: $Id$
%%%%%%%%%%%%%%%%%%%%%%%%%%%%%%%%%%%%%%%%%%%%%%%%%%%%%%%%%%%%%%%%%%%%%%%%%%%%%%%
%%   Copyright (C) 1998 Philip Johnson
%%%%%%%%%%%%%%%%%%%%%%%%%%%%%%%%%%%%%%%%%%%%%%%%%%%%%%%%%%%%%%%%%%%%%%%%%%%%%%%
%% 

\section{CONCLUSIONS}

This paper reports on the results of analysis of the data from a single PSP 
class with only 10 students.  As with any case study, care must be taken
in interpreting these results.  We do not know whether this data is 
representative of PSP courses in general, and if the way we teach the
PSP is representative of the way the PSP is taught by others. 

While we do not claim that these results are representative of all PSP
courses, neither do we believe that they result from some peculiarity
and/or failing of our environment.  First, this case study was performed
after the instructor had taught the PSP for one semester in a graduate
level course, and instituted it within his research group, and adopted it
himself for his own software development activities.  By the time of this
study, we were quite experienced as both teachers and users of the PSP.
Second, as already noted, we were concerned with data quality problems from
the beginning, and instituted curriculum modifications specifically
intended to raise data quality. Our overall error rate of less than 5\%
while quite small, was still not sufficient to prevent significant
differences between original and corrected data. Third, our results cannot
be due to our lack of enthusiasm for the PSP: both of us consider it to be
one of the most powerful software engineering practices we have adopted in
our careers.  The first author, for example, has used her automated PSP
tool to gather data on over 120 of her industrial projects over the past
two years.  Fourth, our results cannot be due to lack of enthusiasm for the
PSP by our students, as the post-course comments reveal, most of the
students indicated that they found the class to be very useful and
interesting.

We believe there are four basic conclusions to be drawn from this case
study.  

First, we believe this study indicates the need to explicitly assess
collection and analysis data errors by others in the PSP community. With
better understanding of these two types of errors and their impact upon the
PSP, the community can better guide the evolution of the PSP toward higher
data quality.

Second, we continue to believe that the PSP has substantial educational
value. It has had a tremendous positive impact on our students for
several semesters, and we do not plan to abandon it due to these results.

Third, we believe that integrated tool support for the PSP is required, not
merely helpful, to obtain high analysis-stage PSP data quality.  We also
believe that integrated tool support will make adoption of the PSP
substantially easier, since the most common complaint made by students
using the manual PSP in our classes is the time and effort required to fill
out the forms.


Finally, we believe that until questions raised by this study with respect
to PSP data quality are resolved, {\em PSP data should not be used to
  evaluate the PSP method itself.} In other words, we believe that it is
not yet appropriate to infer that changes in PSP measures during (or after) a
training course accurately reflect changes in the underlying developer
behavior.  A statement such as ``The improvement in average defect levels
for engineers who complete the course is 58.0\%'', if based upon PSP data
alone, might only reflect a decreasing trend in defect recording, not a
decreased trend in the defects present in the work product.

We are happy to report that not all PSP evaluations are based upon PSP data
alone. For example, in one of the case studies \cite{Ferguson97}, evidence
for the utility of the PSP method was based upon reductions in acceptance
test defect density for products subsequent to the introduction of PSP
practices.  Although alternative explanations for this trend can be
hypothesized (such as the PSP-based projects were more simple than those
before and thus acceptance test defect density would have decreased
anyway), at least the evaluation measure is independent of the PSP data
and not subject to PSP data quality problems.

Unfortunately, integrated tool support is not a ``magic bullet'' that
will solve all PSP data quality problems, and it is useful to 
recall the age-old computing axiom: ``garbage in, garbage out''.
No matter how automated the analysis stage, overall PSP data quality will
still depend completely upon the data quality from the collection stage.
Obtaining high quality from both collection and analysis stages in
the PSP is a challenging goal for future research on personal software
process improvement.


%PJ
\section{ACKNOWLEDGMENTS}

We gratefully acknowledge all of the students in all of the PSP classes at
the University of Hawaii.  Our colleagues in the Collaborative Software
Development Laboratory during the time of this study (Cam Moore, Robert
Brewer, Jennifer Geis, and Russ Tokuyama) provided ongoing support.  We
would like to thank Watts Humphrey, James Over, and Will Hayes of the
Software Engineering Institute, and the anonymous reviewers, whose comments
sharpened the presentation of this research.  This research was sponsored
in part by grants CCR-9403475 and CCR-9804010 from the National Science
Foundation.

