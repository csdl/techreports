\documentstyle[11pt,times,/export/home/csdl/tex/definemargins]{article}
% Psfig/TeX 
\def\PsfigVersion{1.9}
% dvips version
%
% All psfig/tex software, documentation, and related files
% in this distribution of psfig/tex are 
% Copyright 1987, 1988, 1991 Trevor J. Darrell
%
% Permission is granted for use and non-profit distribution of psfig/tex 
% providing that this notice is clearly maintained. The right to
% distribute any portion of psfig/tex for profit or as part of any commercial
% product is specifically reserved for the author(s) of that portion.
%
% *** Feel free to make local modifications of psfig as you wish,
% *** but DO NOT post any changed or modified versions of ``psfig''
% *** directly to the net. Send them to me and I'll try to incorporate
% *** them into future versions. If you want to take the psfig code 
% *** and make a new program (subject to the copyright above), distribute it, 
% *** (and maintain it) that's fine, just don't call it psfig.
%
% Bugs and improvements to trevor@media.mit.edu.
%
% Thanks to Greg Hager (GDH) and Ned Batchelder for their contributions
% to the original version of this project.
%
% Modified by J. Daniel Smith on 9 October 1990 to accept the
% %%BoundingBox: comment with or without a space after the colon.  Stole
% file reading code from Tom Rokicki's EPSF.TEX file (see below).
%
% More modifications by J. Daniel Smith on 29 March 1991 to allow the
% the included PostScript figure to be rotated.  The amount of
% rotation is specified by the "angle=" parameter of the \psfig command.
%
% Modified by Robert Russell on June 25, 1991 to allow users to specify
% .ps filenames which don't yet exist, provided they explicitly provide
% boundingbox information via the \psfig command. Note: This will only work
% if the "file=" parameter follows all four "bb???=" parameters in the
% command. This is due to the order in which psfig interprets these params.
%
%  3 Jul 1991	JDS	check if file already read in once
%  4 Sep 1991	JDS	fixed incorrect computation of rotated
%			bounding box
% 25 Sep 1991	GVR	expanded synopsis of \psfig
% 14 Oct 1991	JDS	\fbox code from LaTeX so \psdraft works with TeX
%			changed \typeout to \ps@typeout
% 17 Oct 1991	JDS	added \psscalefirst and \psrotatefirst
%

% From: gvr@cs.brown.edu (George V. Reilly)
%
% \psdraft	draws an outline box, but doesn't include the figure
%		in the DVI file.  Useful for previewing.
%
% \psfull	includes the figure in the DVI file (default).
%
% \psscalefirst width= or height= specifies the size of the figure
% 		before rotation.
% \psrotatefirst (default) width= or height= specifies the size of the
% 		 figure after rotation.  Asymetric figures will
% 		 appear to shrink.
%
% \psfigurepath#1	sets the path to search for the figure
%
% \psfig
% usage: \psfig{file=, figure=, height=, width=,
%			bbllx=, bblly=, bburx=, bbury=,
%			rheight=, rwidth=, clip=, angle=, silent=}
%
%	"file" is the filename.  If no path name is specified and the
%		file is not found in the current directory,
%		it will be looked for in directory \psfigurepath.
%	"figure" is a synonym for "file".
%	By default, the width and height of the figure are taken from
%		the BoundingBox of the figure.
%	If "width" is specified, the figure is scaled so that it has
%		the specified width.  Its height changes proportionately.
%	If "height" is specified, the figure is scaled so that it has
%		the specified height.  Its width changes proportionately.
%	If both "width" and "height" are specified, the figure is scaled
%		anamorphically.
%	"bbllx", "bblly", "bburx", and "bbury" control the PostScript
%		BoundingBox.  If these four values are specified
%               *before* the "file" option, the PSFIG will not try to
%               open the PostScript file.
%	"rheight" and "rwidth" are the reserved height and width
%		of the figure, i.e., how big TeX actually thinks
%		the figure is.  They default to "width" and "height".
%	The "clip" option ensures that no portion of the figure will
%		appear outside its BoundingBox.  "clip=" is a switch and
%		takes no value, but the `=' must be present.
%	The "angle" option specifies the angle of rotation (degrees, ccw).
%	The "silent" option makes \psfig work silently.
%

% check to see if macros already loaded in (maybe some other file says
% "\input psfig") ...
\ifx\undefined\psfig\else\endinput\fi

%
% from a suggestion by eijkhout@csrd.uiuc.edu to allow
% loading as a style file. Changed to avoid problems
% with amstex per suggestion by jbence@math.ucla.edu

\let\LaTeXAtSign=\@
\let\@=\relax
\edef\psfigRestoreAt{\catcode`\@=\number\catcode`@\relax}
%\edef\psfigRestoreAt{\catcode`@=\number\catcode`@\relax}
\catcode`\@=11\relax
\newwrite\@unused
\def\ps@typeout#1{{\let\protect\string\immediate\write\@unused{#1}}}
\ps@typeout{psfig/tex \PsfigVersion}

%% Here's how you define your figure path.  Should be set up with null
%% default and a user useable definition.

\def\figurepath{./}
\def\psfigurepath#1{\edef\figurepath{#1}}

%
% @psdo control structure -- similar to Latex @for.
% I redefined these with different names so that psfig can
% be used with TeX as well as LaTeX, and so that it will not 
% be vunerable to future changes in LaTeX's internal
% control structure,
%
\def\@nnil{\@nil}
\def\@empty{}
\def\@psdonoop#1\@@#2#3{}
\def\@psdo#1:=#2\do#3{\edef\@psdotmp{#2}\ifx\@psdotmp\@empty \else
    \expandafter\@psdoloop#2,\@nil,\@nil\@@#1{#3}\fi}
\def\@psdoloop#1,#2,#3\@@#4#5{\def#4{#1}\ifx #4\@nnil \else
       #5\def#4{#2}\ifx #4\@nnil \else#5\@ipsdoloop #3\@@#4{#5}\fi\fi}
\def\@ipsdoloop#1,#2\@@#3#4{\def#3{#1}\ifx #3\@nnil 
       \let\@nextwhile=\@psdonoop \else
      #4\relax\let\@nextwhile=\@ipsdoloop\fi\@nextwhile#2\@@#3{#4}}
\def\@tpsdo#1:=#2\do#3{\xdef\@psdotmp{#2}\ifx\@psdotmp\@empty \else
    \@tpsdoloop#2\@nil\@nil\@@#1{#3}\fi}
\def\@tpsdoloop#1#2\@@#3#4{\def#3{#1}\ifx #3\@nnil 
       \let\@nextwhile=\@psdonoop \else
      #4\relax\let\@nextwhile=\@tpsdoloop\fi\@nextwhile#2\@@#3{#4}}
% 
% \fbox is defined in latex.tex; so if \fbox is undefined, assume that
% we are not in LaTeX.
% Perhaps this could be done better???
\ifx\undefined\fbox
% \fbox code from modified slightly from LaTeX
\newdimen\fboxrule
\newdimen\fboxsep
\newdimen\ps@tempdima
\newbox\ps@tempboxa
\fboxsep = 3pt
\fboxrule = .4pt
\long\def\fbox#1{\leavevmode\setbox\ps@tempboxa\hbox{#1}\ps@tempdima\fboxrule
    \advance\ps@tempdima \fboxsep \advance\ps@tempdima \dp\ps@tempboxa
   \hbox{\lower \ps@tempdima\hbox
  {\vbox{\hrule height \fboxrule
          \hbox{\vrule width \fboxrule \hskip\fboxsep
          \vbox{\vskip\fboxsep \box\ps@tempboxa\vskip\fboxsep}\hskip 
                 \fboxsep\vrule width \fboxrule}
                 \hrule height \fboxrule}}}}
\fi
%
%%%%%%%%%%%%%%%%%%%%%%%%%%%%%%%%%%%%%%%%%%%%%%%%%%%%%%%%%%%%%%%%%%%
% file reading stuff from epsf.tex
%   EPSF.TEX macro file:
%   Written by Tomas Rokicki of Radical Eye Software, 29 Mar 1989.
%   Revised by Don Knuth, 3 Jan 1990.
%   Revised by Tomas Rokicki to accept bounding boxes with no
%      space after the colon, 18 Jul 1990.
%   Portions modified/removed for use in PSFIG package by
%      J. Daniel Smith, 9 October 1990.
%
\newread\ps@stream
\newif\ifnot@eof       % continue looking for the bounding box?
\newif\if@noisy        % report what you're making?
\newif\if@atend        % %%BoundingBox: has (at end) specification
\newif\if@psfile       % does this look like a PostScript file?
%
% PostScript files should start with `%!'
%
{\catcode`\%=12\global\gdef\epsf@start{%!}}
\def\epsf@PS{PS}
%
\def\epsf@getbb#1{%
%
%   The first thing we need to do is to open the
%   PostScript file, if possible.
%
\openin\ps@stream=#1
\ifeof\ps@stream\ps@typeout{Error, File #1 not found}\else
%
%   Okay, we got it. Now we'll scan lines until we find one that doesn't
%   start with %. We're looking for the bounding box comment.
%
   {\not@eoftrue \chardef\other=12
    \def\do##1{\catcode`##1=\other}\dospecials \catcode`\ =10
    \loop
       \if@psfile
	  \read\ps@stream to \epsf@fileline
       \else{
	  \obeyspaces
          \read\ps@stream to \epsf@tmp\global\let\epsf@fileline\epsf@tmp}
       \fi
       \ifeof\ps@stream\not@eoffalse\else
%
%   Check the first line for `%!'.  Issue a warning message if its not
%   there, since the file might not be a PostScript file.
%
       \if@psfile\else
       \expandafter\epsf@test\epsf@fileline:. \\%
       \fi
%
%   We check to see if the first character is a % sign;
%   if so, we look further and stop only if the line begins with
%   `%%BoundingBox:' and the `(atend)' specification was not found.
%   That is, the only way to stop is when the end of file is reached,
%   or a `%%BoundingBox: llx lly urx ury' line is found.
%
          \expandafter\epsf@aux\epsf@fileline:. \\%
       \fi
   \ifnot@eof\repeat
   }\closein\ps@stream\fi}%
%
% This tests if the file we are reading looks like a PostScript file.
%
\long\def\epsf@test#1#2#3:#4\\{\def\epsf@testit{#1#2}
			\ifx\epsf@testit\epsf@start\else
\ps@typeout{Warning! File does not start with `\epsf@start'.  It may not be a PostScript file.}
			\fi
			\@psfiletrue} % don't test after 1st line
%
%   We still need to define the tricky \epsf@aux macro. This requires
%   a couple of magic constants for comparison purposes.
%
{\catcode`\%=12\global\let\epsf@percent=%\global\def\epsf@bblit{%BoundingBox}}
%
%
%   So we're ready to check for `%BoundingBox:' and to grab the
%   values if they are found.  We continue searching if `(at end)'
%   was found after the `%BoundingBox:'.
%
\long\def\epsf@aux#1#2:#3\\{\ifx#1\epsf@percent
   \def\epsf@testit{#2}\ifx\epsf@testit\epsf@bblit
	\@atendfalse
        \epsf@atend #3 . \\%
	\if@atend	
	   \if@verbose{
		\ps@typeout{psfig: found `(atend)'; continuing search}
	   }\fi
        \else
        \epsf@grab #3 . . . \\%
        \not@eoffalse
        \global\no@bbfalse
        \fi
   \fi\fi}%
%
%   Here we grab the values and stuff them in the appropriate definitions.
%
\def\epsf@grab #1 #2 #3 #4 #5\\{%
   \global\def\epsf@llx{#1}\ifx\epsf@llx\empty
      \epsf@grab #2 #3 #4 #5 .\\\else
   \global\def\epsf@lly{#2}%
   \global\def\epsf@urx{#3}\global\def\epsf@ury{#4}\fi}%
%
% Determine if the stuff following the %%BoundingBox is `(atend)'
% J. Daniel Smith.  Copied from \epsf@grab above.
%
\def\epsf@atendlit{(atend)} 
\def\epsf@atend #1 #2 #3\\{%
   \def\epsf@tmp{#1}\ifx\epsf@tmp\empty
      \epsf@atend #2 #3 .\\\else
   \ifx\epsf@tmp\epsf@atendlit\@atendtrue\fi\fi}


% End of file reading stuff from epsf.tex
%%%%%%%%%%%%%%%%%%%%%%%%%%%%%%%%%%%%%%%%%%%%%%%%%%%%%%%%%%%%%%%%%%%

%%%%%%%%%%%%%%%%%%%%%%%%%%%%%%%%%%%%%%%%%%%%%%%%%%%%%%%%%%%%%%%%%%%
% trigonometry stuff from "trig.tex"
\chardef\psletter = 11 % won't conflict with \begin{letter} now...
\chardef\other = 12

\newif \ifdebug %%% turn me on to see TeX hard at work ...
\newif\ifc@mpute %%% don't need to compute some values
\c@mputetrue % but assume that we do

\let\then = \relax
\def\r@dian{pt }
\let\r@dians = \r@dian
\let\dimensionless@nit = \r@dian
\let\dimensionless@nits = \dimensionless@nit
\def\internal@nit{sp }
\let\internal@nits = \internal@nit
\newif\ifstillc@nverging
\def \Mess@ge #1{\ifdebug \then \message {#1} \fi}

{ %%% Things that need abnormal catcodes %%%
	\catcode `\@ = \psletter
	\gdef \nodimen {\expandafter \n@dimen \the \dimen}
	\gdef \term #1 #2 #3%
	       {\edef \t@ {\the #1}%%% freeze parameter 1 (count, by value)
		\edef \t@@ {\expandafter \n@dimen \the #2\r@dian}%
				   %%% freeze parameter 2 (dimen, by value)
		\t@rm {\t@} {\t@@} {#3}%
	       }
	\gdef \t@rm #1 #2 #3%
	       {{%
		\count 0 = 0
		\dimen 0 = 1 \dimensionless@nit
		\dimen 2 = #2\relax
		\Mess@ge {Calculating term #1 of \nodimen 2}%
		\loop
		\ifnum	\count 0 < #1
		\then	\advance \count 0 by 1
			\Mess@ge {Iteration \the \count 0 \space}%
			\Multiply \dimen 0 by {\dimen 2}%
			\Mess@ge {After multiplication, term = \nodimen 0}%
			\Divide \dimen 0 by {\count 0}%
			\Mess@ge {After division, term = \nodimen 0}%
		\repeat
		\Mess@ge {Final value for term #1 of 
				\nodimen 2 \space is \nodimen 0}%
		\xdef \Term {#3 = \nodimen 0 \r@dians}%
		\aftergroup \Term
	       }}
	\catcode `\p = \other
	\catcode `\t = \other
	\gdef \n@dimen #1pt{#1} %%% throw away the ``pt''
}

\def \Divide #1by #2{\divide #1 by #2} %%% just a synonym

\def \Multiply #1by #2%%% allows division of a dimen by a dimen
       {{%%% should really freeze parameter 2 (dimen, passed by value)
	\count 0 = #1\relax
	\count 2 = #2\relax
	\count 4 = 65536
	\Mess@ge {Before scaling, count 0 = \the \count 0 \space and
			count 2 = \the \count 2}%
	\ifnum	\count 0 > 32767 %%% do our best to avoid overflow
	\then	\divide \count 0 by 4
		\divide \count 4 by 4
	\else	\ifnum	\count 0 < -32767
		\then	\divide \count 0 by 4
			\divide \count 4 by 4
		\else
		\fi
	\fi
	\ifnum	\count 2 > 32767 %%% while retaining reasonable accuracy
	\then	\divide \count 2 by 4
		\divide \count 4 by 4
	\else	\ifnum	\count 2 < -32767
		\then	\divide \count 2 by 4
			\divide \count 4 by 4
		\else
		\fi
	\fi
	\multiply \count 0 by \count 2
	\divide \count 0 by \count 4
	\xdef \product {#1 = \the \count 0 \internal@nits}%
	\aftergroup \product
       }}

\def\r@duce{\ifdim\dimen0 > 90\r@dian \then   % sin(x+90) = sin(180-x)
		\multiply\dimen0 by -1
		\advance\dimen0 by 180\r@dian
		\r@duce
	    \else \ifdim\dimen0 < -90\r@dian \then  % sin(-x) = sin(360+x)
		\advance\dimen0 by 360\r@dian
		\r@duce
		\fi
	    \fi}

\def\Sine#1%
       {{%
	\dimen 0 = #1 \r@dian
	\r@duce
	\ifdim\dimen0 = -90\r@dian \then
	   \dimen4 = -1\r@dian
	   \c@mputefalse
	\fi
	\ifdim\dimen0 = 90\r@dian \then
	   \dimen4 = 1\r@dian
	   \c@mputefalse
	\fi
	\ifdim\dimen0 = 0\r@dian \then
	   \dimen4 = 0\r@dian
	   \c@mputefalse
	\fi
%
	\ifc@mpute \then
        	% convert degrees to radians
		\divide\dimen0 by 180
		\dimen0=3.141592654\dimen0
%
		\dimen 2 = 3.1415926535897963\r@dian %%% a well-known constant
		\divide\dimen 2 by 2 %%% we only deal with -pi/2 : pi/2
		\Mess@ge {Sin: calculating Sin of \nodimen 0}%
		\count 0 = 1 %%% see power-series expansion for sine
		\dimen 2 = 1 \r@dian %%% ditto
		\dimen 4 = 0 \r@dian %%% ditto
		\loop
			\ifnum	\dimen 2 = 0 %%% then we've done
			\then	\stillc@nvergingfalse 
			\else	\stillc@nvergingtrue
			\fi
			\ifstillc@nverging %%% then calculate next term
			\then	\term {\count 0} {\dimen 0} {\dimen 2}%
				\advance \count 0 by 2
				\count 2 = \count 0
				\divide \count 2 by 2
				\ifodd	\count 2 %%% signs alternate
				\then	\advance \dimen 4 by \dimen 2
				\else	\advance \dimen 4 by -\dimen 2
				\fi
		\repeat
	\fi		
			\xdef \sine {\nodimen 4}%
       }}

% Now the Cosine can be calculated easily by calling \Sine
\def\Cosine#1{\ifx\sine\UnDefined\edef\Savesine{\relax}\else
		             \edef\Savesine{\sine}\fi
	{\dimen0=#1\r@dian\advance\dimen0 by 90\r@dian
	 \Sine{\nodimen 0}
	 \xdef\cosine{\sine}
	 \xdef\sine{\Savesine}}}	      
% end of trig stuff
%%%%%%%%%%%%%%%%%%%%%%%%%%%%%%%%%%%%%%%%%%%%%%%%%%%%%%%%%%%%%%%%%%%%

\def\psdraft{
	\def\@psdraft{0}
	%\ps@typeout{draft level now is \@psdraft \space . }
}
\def\psfull{
	\def\@psdraft{100}
	%\ps@typeout{draft level now is \@psdraft \space . }
}

\psfull

\newif\if@scalefirst
\def\psscalefirst{\@scalefirsttrue}
\def\psrotatefirst{\@scalefirstfalse}
\psrotatefirst

\newif\if@draftbox
\def\psnodraftbox{
	\@draftboxfalse
}
\def\psdraftbox{
	\@draftboxtrue
}
\@draftboxtrue

\newif\if@prologfile
\newif\if@postlogfile
\def\pssilent{
	\@noisyfalse
}
\def\psnoisy{
	\@noisytrue
}
\psnoisy
%%% These are for the option list.
%%% A specification of the form a = b maps to calling \@p@@sa{b}
\newif\if@bbllx
\newif\if@bblly
\newif\if@bburx
\newif\if@bbury
\newif\if@height
\newif\if@width
\newif\if@rheight
\newif\if@rwidth
\newif\if@angle
\newif\if@clip
\newif\if@verbose
\def\@p@@sclip#1{\@cliptrue}


\newif\if@decmpr

%%% GDH 7/26/87 -- changed so that it first looks in the local directory,
%%% then in a specified global directory for the ps file.
%%% RPR 6/25/91 -- changed so that it defaults to user-supplied name if
%%% boundingbox info is specified, assuming graphic will be created by
%%% print time.
%%% TJD 10/19/91 -- added bbfile vs. file distinction, and @decmpr flag

\def\@p@@sfigure#1{\def\@p@sfile{null}\def\@p@sbbfile{null}
	        \openin1=#1.bb
		\ifeof1\closein1
	        	\openin1=\figurepath#1.bb
			\ifeof1\closein1
			        \openin1=#1
				\ifeof1\closein1%
				       \openin1=\figurepath#1
					\ifeof1
					   \ps@typeout{Error, File #1 not found}
						\if@bbllx\if@bblly
				   		\if@bburx\if@bbury
			      				\def\@p@sfile{#1}%
			      				\def\@p@sbbfile{#1}%
							\@decmprfalse
				  	   	\fi\fi\fi\fi
					\else\closein1
				    		\def\@p@sfile{\figurepath#1}%
				    		\def\@p@sbbfile{\figurepath#1}%
						\@decmprfalse
	                       		\fi%
			 	\else\closein1%
					\def\@p@sfile{#1}
					\def\@p@sbbfile{#1}
					\@decmprfalse
			 	\fi
			\else
				\def\@p@sfile{\figurepath#1}
				\def\@p@sbbfile{\figurepath#1.bb}
				\@decmprtrue
			\fi
		\else
			\def\@p@sfile{#1}
			\def\@p@sbbfile{#1.bb}
			\@decmprtrue
		\fi}

\def\@p@@sfile#1{\@p@@sfigure{#1}}

\def\@p@@sbbllx#1{
		%\ps@typeout{bbllx is #1}
		\@bbllxtrue
		\dimen100=#1
		\edef\@p@sbbllx{\number\dimen100}
}
\def\@p@@sbblly#1{
		%\ps@typeout{bblly is #1}
		\@bbllytrue
		\dimen100=#1
		\edef\@p@sbblly{\number\dimen100}
}
\def\@p@@sbburx#1{
		%\ps@typeout{bburx is #1}
		\@bburxtrue
		\dimen100=#1
		\edef\@p@sbburx{\number\dimen100}
}
\def\@p@@sbbury#1{
		%\ps@typeout{bbury is #1}
		\@bburytrue
		\dimen100=#1
		\edef\@p@sbbury{\number\dimen100}
}
\def\@p@@sheight#1{
		\@heighttrue
		\dimen100=#1
   		\edef\@p@sheight{\number\dimen100}
		%\ps@typeout{Height is \@p@sheight}
}
\def\@p@@swidth#1{
		%\ps@typeout{Width is #1}
		\@widthtrue
		\dimen100=#1
		\edef\@p@swidth{\number\dimen100}
}
\def\@p@@srheight#1{
		%\ps@typeout{Reserved height is #1}
		\@rheighttrue
		\dimen100=#1
		\edef\@p@srheight{\number\dimen100}
}
\def\@p@@srwidth#1{
		%\ps@typeout{Reserved width is #1}
		\@rwidthtrue
		\dimen100=#1
		\edef\@p@srwidth{\number\dimen100}
}
\def\@p@@sangle#1{
		%\ps@typeout{Rotation is #1}
		\@angletrue
%		\dimen100=#1
		\edef\@p@sangle{#1} %\number\dimen100}
}
\def\@p@@ssilent#1{ 
		\@verbosefalse
}
\def\@p@@sprolog#1{\@prologfiletrue\def\@prologfileval{#1}}
\def\@p@@spostlog#1{\@postlogfiletrue\def\@postlogfileval{#1}}
\def\@cs@name#1{\csname #1\endcsname}
\def\@setparms#1=#2,{\@cs@name{@p@@s#1}{#2}}
%
% initialize the defaults (size the size of the figure)
%
\def\ps@init@parms{
		\@bbllxfalse \@bbllyfalse
		\@bburxfalse \@bburyfalse
		\@heightfalse \@widthfalse
		\@rheightfalse \@rwidthfalse
		\def\@p@sbbllx{}\def\@p@sbblly{}
		\def\@p@sbburx{}\def\@p@sbbury{}
		\def\@p@sheight{}\def\@p@swidth{}
		\def\@p@srheight{}\def\@p@srwidth{}
		\def\@p@sangle{0}
		\def\@p@sfile{} \def\@p@sbbfile{}
		\def\@p@scost{10}
		\def\@sc{}
		\@prologfilefalse
		\@postlogfilefalse
		\@clipfalse
		\if@noisy
			\@verbosetrue
		\else
			\@verbosefalse
		\fi
}
%
% Go through the options setting things up.
%
\def\parse@ps@parms#1{
	 	\@psdo\@psfiga:=#1\do
		   {\expandafter\@setparms\@psfiga,}}
%
% Compute bb height and width
%
\newif\ifno@bb
\def\bb@missing{
	\if@verbose{
		\ps@typeout{psfig: searching \@p@sbbfile \space  for bounding box}
	}\fi
	\no@bbtrue
	\epsf@getbb{\@p@sbbfile}
        \ifno@bb \else \bb@cull\epsf@llx\epsf@lly\epsf@urx\epsf@ury\fi
}	
\def\bb@cull#1#2#3#4{
	\dimen100=#1 bp\edef\@p@sbbllx{\number\dimen100}
	\dimen100=#2 bp\edef\@p@sbblly{\number\dimen100}
	\dimen100=#3 bp\edef\@p@sbburx{\number\dimen100}
	\dimen100=#4 bp\edef\@p@sbbury{\number\dimen100}
	\no@bbfalse
}
% rotate point (#1,#2) about (0,0).
% The sine and cosine of the angle are already stored in \sine and
% \cosine.  The result is placed in (\p@intvaluex, \p@intvaluey).
\newdimen\p@intvaluex
\newdimen\p@intvaluey
\def\rotate@#1#2{{\dimen0=#1 sp\dimen1=#2 sp
%            	calculate x' = x \cos\theta - y \sin\theta
		  \global\p@intvaluex=\cosine\dimen0
		  \dimen3=\sine\dimen1
		  \global\advance\p@intvaluex by -\dimen3
% 		calculate y' = x \sin\theta + y \cos\theta
		  \global\p@intvaluey=\sine\dimen0
		  \dimen3=\cosine\dimen1
		  \global\advance\p@intvaluey by \dimen3
		  }}
\def\compute@bb{
		\no@bbfalse
		\if@bbllx \else \no@bbtrue \fi
		\if@bblly \else \no@bbtrue \fi
		\if@bburx \else \no@bbtrue \fi
		\if@bbury \else \no@bbtrue \fi
		\ifno@bb \bb@missing \fi
		\ifno@bb \ps@typeout{FATAL ERROR: no bb supplied or found}
			\no-bb-error
		\fi
		%
%\ps@typeout{BB: \@p@sbbllx, \@p@sbblly, \@p@sbburx, \@p@sbbury} 
%
% store height/width of original (unrotated) bounding box
		\count203=\@p@sbburx
		\count204=\@p@sbbury
		\advance\count203 by -\@p@sbbllx
		\advance\count204 by -\@p@sbblly
		\edef\ps@bbw{\number\count203}
		\edef\ps@bbh{\number\count204}
		%\ps@typeout{ psbbh = \ps@bbh, psbbw = \ps@bbw }
		\if@angle 
			\Sine{\@p@sangle}\Cosine{\@p@sangle}
	        	{\dimen100=\maxdimen\xdef\r@p@sbbllx{\number\dimen100}
					    \xdef\r@p@sbblly{\number\dimen100}
			                    \xdef\r@p@sbburx{-\number\dimen100}
					    \xdef\r@p@sbbury{-\number\dimen100}}
%
% Need to rotate all four points and take the X-Y extremes of the new
% points as the new bounding box.
                        \def\minmaxtest{
			   \ifnum\number\p@intvaluex<\r@p@sbbllx
			      \xdef\r@p@sbbllx{\number\p@intvaluex}\fi
			   \ifnum\number\p@intvaluex>\r@p@sbburx
			      \xdef\r@p@sbburx{\number\p@intvaluex}\fi
			   \ifnum\number\p@intvaluey<\r@p@sbblly
			      \xdef\r@p@sbblly{\number\p@intvaluey}\fi
			   \ifnum\number\p@intvaluey>\r@p@sbbury
			      \xdef\r@p@sbbury{\number\p@intvaluey}\fi
			   }
%			lower left
			\rotate@{\@p@sbbllx}{\@p@sbblly}
			\minmaxtest
%			upper left
			\rotate@{\@p@sbbllx}{\@p@sbbury}
			\minmaxtest
%			lower right
			\rotate@{\@p@sbburx}{\@p@sbblly}
			\minmaxtest
%			upper right
			\rotate@{\@p@sbburx}{\@p@sbbury}
			\minmaxtest
			\edef\@p@sbbllx{\r@p@sbbllx}\edef\@p@sbblly{\r@p@sbblly}
			\edef\@p@sbburx{\r@p@sbburx}\edef\@p@sbbury{\r@p@sbbury}
%\ps@typeout{rotated BB: \r@p@sbbllx, \r@p@sbblly, \r@p@sbburx, \r@p@sbbury}
		\fi
		\count203=\@p@sbburx
		\count204=\@p@sbbury
		\advance\count203 by -\@p@sbbllx
		\advance\count204 by -\@p@sbblly
		\edef\@bbw{\number\count203}
		\edef\@bbh{\number\count204}
		%\ps@typeout{ bbh = \@bbh, bbw = \@bbw }
}
%
% \in@hundreds performs #1 * (#2 / #3) correct to the hundreds,
%	then leaves the result in @result
%
\def\in@hundreds#1#2#3{\count240=#2 \count241=#3
		     \count100=\count240	% 100 is first digit #2/#3
		     \divide\count100 by \count241
		     \count101=\count100
		     \multiply\count101 by \count241
		     \advance\count240 by -\count101
		     \multiply\count240 by 10
		     \count101=\count240	%101 is second digit of #2/#3
		     \divide\count101 by \count241
		     \count102=\count101
		     \multiply\count102 by \count241
		     \advance\count240 by -\count102
		     \multiply\count240 by 10
		     \count102=\count240	% 102 is the third digit
		     \divide\count102 by \count241
		     \count200=#1\count205=0
		     \count201=\count200
			\multiply\count201 by \count100
		 	\advance\count205 by \count201
		     \count201=\count200
			\divide\count201 by 10
			\multiply\count201 by \count101
			\advance\count205 by \count201
			%
		     \count201=\count200
			\divide\count201 by 100
			\multiply\count201 by \count102
			\advance\count205 by \count201
			%
		     \edef\@result{\number\count205}
}
\def\compute@wfromh{
		% computing : width = height * (bbw / bbh)
		\in@hundreds{\@p@sheight}{\@bbw}{\@bbh}
		%\ps@typeout{ \@p@sheight * \@bbw / \@bbh, = \@result }
		\edef\@p@swidth{\@result}
		%\ps@typeout{w from h: width is \@p@swidth}
}
\def\compute@hfromw{
		% computing : height = width * (bbh / bbw)
	        \in@hundreds{\@p@swidth}{\@bbh}{\@bbw}
		%\ps@typeout{ \@p@swidth * \@bbh / \@bbw = \@result }
		\edef\@p@sheight{\@result}
		%\ps@typeout{h from w : height is \@p@sheight}
}
\def\compute@handw{
		\if@height 
			\if@width
			\else
				\compute@wfromh
			\fi
		\else 
			\if@width
				\compute@hfromw
			\else
				\edef\@p@sheight{\@bbh}
				\edef\@p@swidth{\@bbw}
			\fi
		\fi
}
\def\compute@resv{
		\if@rheight \else \edef\@p@srheight{\@p@sheight} \fi
		\if@rwidth \else \edef\@p@srwidth{\@p@swidth} \fi
		%\ps@typeout{rheight = \@p@srheight, rwidth = \@p@srwidth}
}
%		
% Compute any missing values
\def\compute@sizes{
	\compute@bb
	\if@scalefirst\if@angle
% at this point the bounding box has been adjsuted correctly for
% rotation.  PSFIG does all of its scaling using \@bbh and \@bbw.  If
% a width= or height= was specified along with \psscalefirst, then the
% width=/height= value needs to be adjusted to match the new (rotated)
% bounding box size (specifed in \@bbw and \@bbh).
%    \ps@bbw       width=
%    -------  =  ---------- 
%    \@bbw       new width=
% so `new width=' = (width= * \@bbw) / \ps@bbw; where \ps@bbw is the
% width of the original (unrotated) bounding box.
	\if@width
	   \in@hundreds{\@p@swidth}{\@bbw}{\ps@bbw}
	   \edef\@p@swidth{\@result}
	\fi
	\if@height
	   \in@hundreds{\@p@sheight}{\@bbh}{\ps@bbh}
	   \edef\@p@sheight{\@result}
	\fi
	\fi\fi
	\compute@handw
	\compute@resv}

%
% \psfig
% usage : \psfig{file=, height=, width=, bbllx=, bblly=, bburx=, bbury=,
%			rheight=, rwidth=, clip=}
%
% "clip=" is a switch and takes no value, but the `=' must be present.
\def\psfig#1{\vbox {
	% do a zero width hard space so that a single
	% \psfig in a centering enviornment will behave nicely
	%{\setbox0=\hbox{\ }\ \hskip-\wd0}
	%
	\ps@init@parms
	\parse@ps@parms{#1}
	\compute@sizes
	%
	\ifnum\@p@scost<\@psdraft{
		%
		\special{ps::[begin] 	\@p@swidth \space \@p@sheight \space
				\@p@sbbllx \space \@p@sbblly \space
				\@p@sbburx \space \@p@sbbury \space
				startTexFig \space }
		\if@angle
			\special {ps:: \@p@sangle \space rotate \space} 
		\fi
		\if@clip{
			\if@verbose{
				\ps@typeout{(clip)}
			}\fi
			\special{ps:: doclip \space }
		}\fi
		\if@prologfile
		    \special{ps: plotfile \@prologfileval \space } \fi
		\if@decmpr{
			\if@verbose{
				\ps@typeout{psfig: including \@p@sfile.Z \space }
			}\fi
			\special{ps: plotfile "`zcat \@p@sfile.Z" \space }
		}\else{
			\if@verbose{
				\ps@typeout{psfig: including \@p@sfile \space }
			}\fi
			\special{ps: plotfile \@p@sfile \space }
		}\fi
		\if@postlogfile
		    \special{ps: plotfile \@postlogfileval \space } \fi
		\special{ps::[end] endTexFig \space }
		% Create the vbox to reserve the space for the figure.
		\vbox to \@p@srheight sp{
		% 1/92 TJD Changed from "true sp" to "sp" for magnification.
			\hbox to \@p@srwidth sp{
				\hss
			}
		\vss
		}
	}\else{
		% draft figure, just reserve the space and print the
		% path name.
		\if@draftbox{		
			% Verbose draft: print file name in box
			\hbox{\frame{\vbox to \@p@srheight sp{
			\vss
			\hbox to \@p@srwidth sp{ \hss \@p@sfile \hss }
			\vss
			}}}
		}\else{
			% Non-verbose draft
			\vbox to \@p@srheight sp{
			\vss
			\hbox to \@p@srwidth sp{\hss}
			\vss
			}
		}\fi	



	}\fi
}}
\psfigRestoreAt
\let\@=\LaTeXAtSign





\definemargins{1.0in}{1.0in}{1.0in}{1.0in}{0.3in}{0.3in}

\begin{document}

\title{A Critical Analysis of PSP Data Quality: Results from a Case Study}


\author{
        Philip M. Johnson\\
        Anne M. Disney\\
        Dept. of Information and Computer Sciences\\
        University of Hawaii\\
        Honolulu, HI  96822 USA\\
        johnson@hawaii.edu\\
        adisney@ilhawaii.net\\
       }

\maketitle

\begin{abstract}
  
  The Personal Software Process (PSP) is used by software engineers to
  gather and analyze data about their work.  Published studies typically
  use data collected using the PSP to draw quantitative conclusions about
  its impact upon programmer behavior and product quality.  However, our
  experience using PSP led us to question the quality of data both during
  collection and its later analysis.  We hypothesized that data quality
  problems can make a significant impact upon the value of PSP
  measures---significant enough to lead to incorrect conclusions regarding
  process improvement.  To test this hypothesis, we built a tool to
  automate the PSP and then examined 89 projects completed by ten subjects
  using the PSP manually in an educational setting.  We discovered 1539
  primary errors and categorized them by type, subtype, severity, and age.
  To examine the collection problem we looked at the 90 errors that
  represented impossible combinations of data and at other less concrete
  anomalies in Time Recording Logs and Defect Recording Logs.  To examine
  the analysis problem we developed a rule set, corrected the errors as far
  as possible, and compared the original and corrected data.  We found
  significant differences for measures such as yield and the
  cost-performance ratio, confirming our hypothesis.  Our results raise
  questions about the accuracy of manually collected and analyzed PSP data,
  indicate that integrated tool support may be required for high quality
  PSP data analysis, and suggest that external measures should be used when
  attempting to evaluate the impact of the PSP upon programmer behavior and
  product quality.

\end{abstract}

\subsection*{Keywords}

\noindent Personal software process, defects, 
empirical software engineering,
measurement dysfunction, automated process support\\

\newpage
\tableofcontents
\newpage

%%%%%%%%%%%%%%%%%%%%%%%%%%%%%% -*- Mode: Latex -*- %%%%%%%%%%%%%%%%%%%%%%%%%%%%
%% 1-intro.tex -- 
%% Author          : Philip Johnson
%% Created On      : Wed Apr  8 14:22:56 1998
%% Last Modified By: Philip Johnson
%% Last Modified On: Tue Feb  2 08:10:27 1999
%% RCS: $Id$
%%%%%%%%%%%%%%%%%%%%%%%%%%%%%%%%%%%%%%%%%%%%%%%%%%%%%%%%%%%%%%%%%%%%%%%%%%%%%%%
%%   Copyright (C) 1998 Philip Johnson
%%%%%%%%%%%%%%%%%%%%%%%%%%%%%%%%%%%%%%%%%%%%%%%%%%%%%%%%%%%%%%%%%%%%%%%%%%%%%%%
%% 

\section{Introduction}
\label{sec:introduction}

\begin{quotation}
\noindent {\em The actual process is what you do, with all its omissions, mistakes, and
oversights. The official process is what the book says you are supposed to
do.} \cite{Humphrey95}
\end{quotation}

The Personal Software Process (PSP) was introduced in 1995 in the book, ``A
Discipline for Software Engineering'' \cite{Humphrey95}.  This text
describes a one-semester curriculum for advanced undergraduates or graduate
students in computer science that teaches concepts in empirically-guided
software process improvement. Since its introduction, experience with the
PSP has been reported on in several case studies
\cite{Ceberio-Verghese96,Ferguson97,Humphrey96,Humphrey97}.
Although empirically-guided software process improvement is a key feature
of other software engineering initiatives, such as the Capability Maturity
Model (CMM) \cite{Paulk95}, ISO-9000, and Inspection \cite{Gilb93}, the PSP
differs from these other approaches in important ways.

The CMM, ISO-9000, and Inspection discuss empirical software process
improvement in the context of a large organization.  Process improvement in
this context requires the gathering and analysis of large amounts of data,
within and across departments, generated by different people at different
times.  Indeed, inevitable personnel turnover means that the data collected
from the working procedures of one set of people tend to generate
measurements leading to process changes that affect the working procedures
of a potentially different set of people.  The substantial effort required
to collect, interpret, and introduce organizational change based upon the
measurements for a large organization leads to the need for an explicit
software engineering process group (SEPG) whose mission is to manage
empirically guided improvement. Although the utility of these approaches
have been repeatedly validated, they leave the unfortunate impression that
empirically-guided software process improvement is the sole province of
large organizations who can dedicate teams of people to its enactment.

The PSP provides an alternative, complementary approach in which
empirically guided software process improvement is ``scaled down'' to the
level of an individual developer.  In the PSP, individuals gather
measurements related to their own work products and the process by which
they were developed, and use these measures to drive changes to their
development behavior.  PSP focuses on defect reduction and
estimation accuracy improvement as the two primary goals of personal
process improvement. Through individual collection and analysis of personal
data, the PSP provides a novel example of how empirically-guided software
process improvement can be implemented by individuals regardless of the
surrounding organizational context and the availability of institutional
infrastructure support.

Since the PSP is a new technique, relatively little data exists on its use
and effectiveness.  Case studies typically report positive results, usually
based upon the data collected during enactment of the PSP curriculum.  One
typical case study conclusion is that ``during the course, productivity
improvements average around 20\% and product quality, as measured by
defects, generally improves by five times or more'' \cite{Ferguson97}.
Similarly, another study states that ``the improvement in average defect
levels for engineers who complete the course is 58.0 percent for total defects
per KLOC and 71.9 percent for defects per KLOC found in test''
\cite{Humphrey96}.  Indeed, our own
PSP data yields similarly positive measurements for process and products.

In this paper, we report on a case study performed to assess the quality of
PSP data---the measurements typically used to evaluate the effectiveness of
the PSP as illustrated above.  Our case study was motivated by our
experiences teaching and using the PSP, which led us to suspect that the
empirical measures gathered by the PSP may not, in all cases, reflect the
true underlying process or products of development.

We hypothesized that problems with the quality of process data collected
with the PSP could significantly change at least some of the measures
produced by the PSP that are commonly used to evaluate its effectiveness.
By ``significantly'', we mean something stronger than just a statistically
significant difference between the recorded measurements and the actual
underlying programmer behavior. We mean that the difference between
recorded measures and actual behavior would be sufficient, at least in some
cases, to lead developers to the wrong conclusion about how to improve
their process.

To test this hypothesis, one of us taught a modified version of the PSP
curriculum to a class of 10 students in the Fall of 1996.  The 
course was augmented
with features designed to improve the data quality of the raw
PSP data.  We then entered the PSP measures into a database and subjected
them to a variety of data quality analyses.  These analyses uncovered over
1500 errors in the PSP data used by the students to track their work and
motivate process improvements.  Additional analysis yielded a seven part
classification scheme for PSP data errors.  Although we were not always
able to generate corrected values for the data errors, partial correction
lead to ``significantly'' different values for certain PSP measurements,
confirming our hypothesis.

The remainder of the paper is organized as follows.  Section
\ref{sec:related} presents a description of related work, including an
overview of the PSP itself, case studies of the PSP, and additions and
enhancements to the basic approach.  Section \ref{sec:model} presents a
model of PSP data quality we devised to guide our investigation.  Section
\ref{sec:case-study} presents the case study; its design, instrumentation,
data collection, analysis, and threats to internal and external validity.
Section \ref{sec:results} presents the quantitative and qualitative results
we obtained from the study.  Section \ref{sec:discussion} presents our
interpretation of these results.  This section also introduces the concept
of ``measurement dysfunction'', which is important to our interpretation of
the results from this study and our recommendations for future research and
practice.










%%%%%%%%%%%%%%%%%%%%%%%%%%%%%% -*- Mode: Latex -*- %%%%%%%%%%%%%%%%%%%%%%%%%%%%
%% 2-related.tex -- 
%% Author          : Philip Johnson
%% Created On      : Wed Apr  8 14:23:39 1998
%% Last Modified By: Philip Johnson
%% Last Modified On: Wed Jul 21 09:12:22 1999
%% RCS: $Id$
%%%%%%%%%%%%%%%%%%%%%%%%%%%%%%%%%%%%%%%%%%%%%%%%%%%%%%%%%%%%%%%%%%%%%%%%%%%%%%%
%%   Copyright (C) 1998 Philip Johnson
%%%%%%%%%%%%%%%%%%%%%%%%%%%%%%%%%%%%%%%%%%%%%%%%%%%%%%%%%%%%%%%%%%%%%%%%%%%%%%%
%% 

\section{Related Work}
\label{sec:related}

There is a small but growing number of research studies describing
experiences with teaching and evaluating the PSP. This section overviews
the relevant literature with special attention paid to three questions.

First, how does the research study evaluate the effectiveness of the PSP?
The most common approach is to compare PSP data collected at the beginning
of the course to data collected at the end of the course. We term these
kinds of evaluations {\em internal measurement evaluation}, because
measures collected using the PSP are used to evaluate the PSP itself.  The
less common approach is to compare some other measure of programmer/program
quality collected before the introduction of PSP training to the same
measure of programmer/program quality after the PSP training.  We term
these kinds of evaluation {\em external measurement evaluation}, because
measures collected independently of the PSP training itself are used to
evaluate PSP effectiveness.

Second, how does the research study verify the accuracy of the measurements
used in evaluation--in other words, verify that the measurements actually
reflect the underlying behavior of the PSP users?  This question addresses
one form of internal validity: that the research is designed
in such a manner that the data can actually be used to answer the questions
posed.  The most common approach to measurement verification in PSP
research is {\em manual inspection} of the data by the instructor. In some
cases, the researcher employed {\em subject exclusion}, i.e. eliminating
all the data from one or more subjects based upon data incompleteness or
some qualitative appraisal of data quality.

Third, what issues related to data quality are raised by the research
study?  Does the research present any experiences related to low data
quality and how they might be overcome?  The most common improvement 
mentioned in the research is automation. 

To begin, we present a brief overview of the PSP itself.

\subsection{The Personal Software Process}

In the PSP curriculum presented in ``A Discipline for Software
Engineering'' \cite{Humphrey95}, each student develops 10 small programs
over the course of a semester using a sequence of seven increasingly
sophisticated software development processes, labeled PSP0 to PSP3.  For
every program, the students record various measurements related to their
personal development activities. Such measures include, for example, the
time spent in each phase of development, the numbers of defects injected
and removed during each phase, and the size of the resulting work product.
Five analysis exercises focus on trends and relationships between 
all of the process data collected to that point in the course.

The initial programs use relatively simple PSP processes that establish a
baseline set of process measures for time, size, and defects. Later
programs employ more advanced PSP processes that extend these baseline
process statistics.  Although there are a myriad of extensions, 
most are of two general types: extensions to planning and extensions
to defect management.

Extensions to the planning phase include estimates of the program's
projected size, the projected time required to complete each of the phases,
and the number of anticipated defects that will be injected and removed
during development.  The process by which these estimates are produced
involves statistical analysis of historical correlations between designs
(i.e. class and method counts) and actual size (in lines of code), between
estimated size and actual time, between actual size and actual time, and
between size and defects injected and removed.  (While lines of code as a
metric of size at the organizational level is almost uniformly condemned
in the measurement literature, it seems to work surprisingly well in the
PSP, since the measure is collected and applied to a single individual working
in a single language in a relatively uniform domain.)

By the middle of the course, each student has typically recorded a hundred
or more defects made during development.  More advanced PSP processes
implement defect management mechanisms to help students understand the
impact of various kinds of defects and to drive process improvements
intended to reduce future occurrence of important defect types. For
example, since students record the phase in which each defect was injected and
removed and the time required to fix it, it is possible to analyze the
relationship between fix time and various characteristics of defects.  One
relationship nearly always present in student data is that the ``longer'' a
defect is present, the more time it takes to remove it.  Thus, defects
injected during design and not removed until testing are nearly always more
expensive to remove than, for example, defects injected during coding and
removed during compiling. This outcome typically motivates students to put
more effort and care into design activities. Later processes support such
behaviors by providing active defect management mechanisms. For example, by
analyzing defect data to determine the types of design defects made on
prior projects, a student can generate a checklist to be used as part of a
personal design review. This checklist can be used to ensure that when
similar defects occur in future projects, they do not escape into code,
compile, or test phases.

The most advanced PSP processes extend the basic PSP paradigm to support
larger projects using a cyclic development method. In addition, PSP
includes a meta-level process for defining personal processes in
non-software domains or for specific software organizational contexts.

Students record their PSP data onto one or more PSP forms provided in the
textbook and made available electronically at the textbook publisher's
website. Students fill out these forms manually and turn them in to the
instructor.  Supplemental PSP spreadsheets automate some of the
calculations, which students manually transfer to the forms. The number of
forms filled out increases from three for the first PSP process to over a
dozen for the most advanced processes.

In this curriculum, PSP data quality is addressed in two ways. First, the
instructor manually reviews all PSP data as it is submitted, and is
instructed to accept only complete and accurate PSP data.  If an error on a
prior assignment is discovered, the student must go back and recompute both
the measurements for the prior assignment and all assignments subsequently
affected by this error. (The provided spreadsheets can ease this process.)
Second, the instructor should exhort the students to approach this course
professionally, and to recognize that only the completeness and accuracy of
the data, not the actual values for defect density, productivity, etc. will
be used in the determination of their grade for the course.

The chosen PSP programming assignments are also important to
understanding issues of data quality, because the standard PSP assignments
require students to build software systems that they later invoke to
produce important PSP measurements and analyses.  For example, program 2A
builds a size counter, a tool necessary for obtaining one of the three
primary measures in the PSP. Other assignments produce statistical
calculations necessary for PSP analyses, including linear regression,
correlation, multiple regression, and prediction intervals.


\subsection{Case Studies of the PSP}

The PSP text contains the original ``case study'' of the PSP. Throughout
the book, Humphrey informally presents data collected from students in PSP
classes at Carnegie Mellon University during pre-publication development of
the text. Utilizing an internal measurement evaluation approach, Humphrey
compares PSP data collected from early in the semester with data collected
at the end of the semester, and finds trends toward decreased defect
density, improved yield, and improved estimation accuracy.  Data
verification appears to consist of manual inspection in conjunction with
some subject exclusion.  

Humphrey provides more information about his experiences with the
PSP in several related articles
\cite{Humphrey94,Humphrey94a,Humphrey95a,Humphrey95b,Humphrey96}.  In
general, he presents results based on PSP data collected from over 100
engineers in both industrial and academic settings.  Humphrey employs
internal measurement evaluation and subject exclusion when engineers
``reported either incomplete or obviously incorrect results.''  Results
include an increase in size and time estimation accuracy, and a
reduction in reported
defects of approximately 50\% over the course of the training.

In his master's thesis, Dellien describes his attempt to introduce a
tailored version of the PSP into an existing industrial organization
\cite{Dellien97}.  He analyzed the PSP, broke it down into components such
as measurement and quality management, compared these components with the
existing processes used by the organization, and rebuilt a modified version
of the PSP that addressed the perceived needs of developers but did not
result in overlapping organizational processes.  His evaluation concludes
that using PSP in an industrial setting is different than using it in an
academic setting due to differences in accountability, group context, lack
of automation, and resistance to change.  In his evaluation, Dellien found
that it is difficult to objectively evaluate whether a PSP introduction has
been successful or not, even when success is measured purely as
cost-effectiveness.

Sherdil and Madhavji use the PSP as the basis for research on
human-oriented improvement in the software process \cite{Sherdil96}.  This
research attempts to measure an individual's ``progress function'', using
such variables as productivity, defect rate, and estimation error.  The
analysis attempts to differentiate progress due to ``first order learning''
(i.e. simple task repetition, unrelated to the PSP) and progress due to
``second order learning'' (i.e. introduction of PSP techniques).  Their
evaluation uses internal measurement evaluation with standard PSP measures
to track the progress function.  Verification involved manual inspection of
PSP data ``... for consistency, accuracy and logical validity.  Automatic
tools were also used to verify the program size values.  We also checked if
two subjects were illegally exchanging code, but never found such an
occurrence.''  


Hayes and Over report on a statistical analysis of a set of PSP data sets
in an attempt to understand the overall impact of PSP education
\cite{Hayes97}.  This ``case study on a set of case studies'' involves 298
engineers who spent more than 15,000 hours writing over 300,000 LOC and
removing about 22,000 defects, during the course of 23 separate PSP
training programs in both academic and industrial settings.  Hayes and Over
used internal measurement evaluation to demonstrate improvement in size
estimation, time estimation, and defect density, with no significant change
in productivity.  The report does not indicate that the authors performed
any independent data verification or assessment of data quality, though the
authors do claim that the data quality is ``exceptional'':
\begin{quotation}
Instructors enter the engineers' data into a spreadsheet provided with the
course materials.  The paper forms completed by the engineers are collected
by the instructor, and the class data are analyzed and used to provide
feedback to the engineers.  During the training, trends in class data
provide insights to the engineers, who may then compare their own data with 
that of the group.  Given this careful focus on data and statistical
analysis, the quality and accuracy of the data used in any given class
tends to be exceptional.
\end{quotation}

Emam, Shostak, and Madhavji report on a study of the implementation of PSP
in a commercial setting, with special emphasis on adoption success
\cite{Emam96}.  Unlike the majority of PSP studies, their evaluation
focussed primarily on the long-term adoption success, rather than
short-term changes in PSP measurements during training.  In addition, they
present several issues that may impact adversely upon PSP data quality and
thus the use of internal measurement evaluation. For example, high levels
of reuse can act as a confounding factor on trends in productivity
and defect density.  Trends in defect density could reflect changes
in defect detection capabilities rather than changes in the underlying
density of defects in the work product. Trends in yield could be 
primarily due to introduction of code reviews and not due to any other
aspects of the PSP.   Finally, they found that the paper intensive nature
of PSP was problematic for professional engineers.


Ferguson, Humphrey, Khajenoori, Macke, and Matvya report on case studies of
the PSP at three industry locations: Advanced Information Services,
Motorola Paging Projects Group, and Union Switch and Signal
\cite{Ferguson97}.  Unlike most other studies, PSP effectiveness was
evaluated primarily using external measures.  For example, post-development
defect report data (either during acceptance test or field use) was used to
compare the quality of PSP-developed projects with non-PSP-developed
projects.  Similarly, one of the studies compared schedule estimation error
before and after PSP training.  These case studies showed substantial 
improvement with respect to both defects and estimation on industry 
projects after introduction of the PSP.   The use of external measures
for evaluation, and the particular external measures chosen (such as 
customer-reported defects) eliminates the issue of PSP measurement
accuracy. The researchers did not present any issues regarding 
data quality. 

Claes Wohlin discusses the use of the Personal Software Process as
a context for empirical experimentation \cite{Wohlin98a}.  He finds that
the PSP has several desirable aspects for experimentation, including a
comprehensive specification of the experimental context (i.e. the PSP
textbook \cite{Humphrey95}), relative ease in replication, and the ready
availability of experimental measures. Among the challenges he cites are
internal validity and external validity.  In the case study used as an
example, data from six students were removed because it was ``regarded as
invalid or at least questionable.''

Our own case study, presented in this paper, is intended to contribute to
this body of knowledge concerning the PSP by demonstrating the importance
of explicit concern for data quality beyond what is covered in the PSP
textbook.  The next several sections present our case study and its
results.  In Section \ref{sec:discussion}, we will revisit several of the
case studies presented above and reinterpret them in the light of our
findings. 

%%%%%%%%%%%%%%%%%%%%%%%%%%%%%% -*- Mode: Latex -*- %%%%%%%%%%%%%%%%%%%%%%%%%%%%
%% 3-model.tex -- 
%% Author          : Philip Johnson
%% Created On      : Wed Apr  8 14:23:59 1998
%% Last Modified By: Philip Johnson
%% Last Modified On: Wed Jul 21 09:14:16 1999
%% RCS: $Id$
%%%%%%%%%%%%%%%%%%%%%%%%%%%%%%%%%%%%%%%%%%%%%%%%%%%%%%%%%%%%%%%%%%%%%%%%%%%%%%%
%%   Copyright (C) 1998 Philip Johnson
%%%%%%%%%%%%%%%%%%%%%%%%%%%%%%%%%%%%%%%%%%%%%%%%%%%%%%%%%%%%%%%%%%%%%%%%%%%%%%%
%% 

\section{Modeling PSP Data Quality}
\label{sec:model}

As we pursued this investigation of data quality problems in the PSP, we
found it necessary to build a model and define some new terminology to 
clarify the approach of this research and its conclusions. This section
presents the model and this terminology.


\subsection{Collection and Analysis Stages}

\begin{figure*}
    {\centerline{\psfig{figure=PSPmodel.eps}}}
    \caption{\label{fig:model} A simple model for PSP data quality. Through 
      a process of {\em collection}, the developer generates an initial
      empirical representation (``Records of Work'') of her personal process
      (``Actual Work'').  Through additional {\em analyses}, the developer
      augments her initial empirical representation with
      derived data (``Analyzed Work'') intended to enable process improvement
      through ``Insights about Work''.
      }
\end{figure*}
     
Figure \ref{fig:model} illustrates a simple two stage model of PSP data
representing an iterative cycle of {\em collection} followed by {\em
  analysis}. The model begins with ``Actual Work''---the actual developer
efforts devoted to a software development project.  As part of these
efforts, the developer enters the collection stage during which she
collects a set of {\em primary} measures of defects, time, and work product
characteristics---the ``Records of Work''.  Given these primary measures,
the developer performs additional analyses during the Analysis Stage, many
of which produce {\em derived} measures which are themselves inputs for
further analyses.  The secondary, derived measures and associated analyses
are presented in various PSP forms---the ``Analyzed Work''---and hopefully
yield ``Insights about Work'' that change and improve future actual
software development work activities.

\subsection{Manual and Integrated Automation}

We have also found it important to distinguish between the different levels of
automation possible in the PSP.  ``A Discipline for Software Engineering''
\cite{Humphrey95} presents an approach to PSP automation that we call {\em
  manual} PSP.  Manual PSP refers to a situation in which PSP forms are
filled out by hand, either by editing a copy of the form on-line, or by
filling out a printed copy with pen or pencil.  Manual PSP does not
preclude the use of tools ``on the side'' to store historical data and to
perform certain computations.  In our classification scheme, if PSP tool
support cannot eliminate manual manipulation and recalculation of derived
measures, and thus guarantee their consistency and accuracy 
(assuming consistent and accurate primary measurements), then we view the level
of automation as manual.

In contrast, we use {\em integrated} PSP to refer to the use of tools in which most
or all of the derived measures during the analysis stage are calculated and
placed into the forms automatically.  Although integrated PSP should
essentially automate all analysis stage calculations, there are
limits to the ability of current technology to automate collection stage
gathering of primary measures. For the foreseeable future, this part of the
PSP will continue to be essentially ``manual'' in nature.

At the time we performed this case study, as was the case at the time of
publication of ``A Discipline for Software Engineering'', there was no
integrated software support for the PSP.  Thus, the case study involved
manual PSP, despite our extensive use of spreadsheets, program size
counting tools, and statistical tools during the course.  Since then, 
integrated support for the PSP has become available, including
the Personal Software Process Studio tool
produced by East Tennessee State University \cite{Henry97}, and 
the Leap toolset at the University of Hawaii \cite{Moore98}.

\subsection{Omission, Addition, Calculation, and Transcription Errors}

There are three basic ways to affect PSP data quality in the collection
stage: errors of omission, errors of addition, and errors of transcription.
Errors of omission occur when the developer does not record a primary
measure related to defects, time, or the work product itself.  If a defect
occurring during ``Actual Work'' does not appear in the ``Records of
Work'', then, for example, the PSP model of that work product's defect
density will be lower than its actual defect density. If time spent on the
work product is not recorded, then the PSP model of that developer's
productivity will be higher than her actual productivity. Errors of
addition occur when the developer augments the ``Records of Work'' with
data not reflecting actual practice. For example, a developer, having made
an error of omission to the point of having no time or defect data, may
recover by simply inventing enough time and defect entries to make his or
her PSP data appear plausible. Finally, errors of transcription occur when
the developer does intend to accurately record his ``Actual Work'' in the
``Records of Work'' but makes a mistake during this process.

The presence of collection stage data quality problems is typically
difficult to ascertain and difficult or impossible to rectify. In the
PSP, primary data collection often feels both time consuming and
psychologically disruptive.  Many students complain that stopping to record
defects disrupts their ``flow'' state, and that the time spent recording a
defect---particularly for compilation stage errors---often exceeds the time
spent correcting the defect.  The PSP requires users to learn to constantly
interleave ``doing work'' with ``recording the work you are doing''.

There are also three basic ways to affect PSP data quality in the analysis
stage of manual PSP: errors of omission, errors of calculation, and errors of
transcription.  Errors of omission occur when the developer does not
perform a required analysis of the primary data. Errors of calculation
occur when the developer attempts to perform an analysis but does so
incorrectly. For example, a developer might use a regression-based
estimation method when the historical data is so uncorrelated that this
method is invalid. Finally, errors of transcription occur when the
developer makes a clerical error when moving data from one form to another.

Analysis stage data quality problems are typically much
easier to ascertain and correct than collection stage data
quality problems, {\em provided that the problems did not originate
  during the collection stage}.  In other words, if one assumes that the
work records accurately reflect the underlying work, then appropriate use
of automated tools can reduce or eliminate analysis errors of omission,
calculation, and transcription.  On the other hand, since the quality of
these analyses are totally dependent upon the quality of the work records
produced in the collection phase,
overall PSP data quality could be quite low even if the analysis stage is
totally automated to eliminate all of its potential data quality errors.







 
%%%%%%%%%%%%%%%%%%%%%%%%%%%%%% -*- Mode: Latex -*- %%%%%%%%%%%%%%%%%%%%%%%%%%%%
%% 4-case-study.tex -- 
%% Author          : Philip Johnson
%% Created On      : Wed Apr  8 14:24:28 1998
%% Last Modified By: Philip Johnson
%% Last Modified On: Wed Aug 11 11:18:14 1999
%% RCS: $Id$
%%%%%%%%%%%%%%%%%%%%%%%%%%%%%%%%%%%%%%%%%%%%%%%%%%%%%%%%%%%%%%%%%%%%%%%%%%%%%%%
%%   Copyright (C) 1998 Philip Johnson
%%%%%%%%%%%%%%%%%%%%%%%%%%%%%%%%%%%%%%%%%%%%%%%%%%%%%%%%%%%%%%%%%%%%%%%%%%%%%%%
%% 3 pages


\section{The Case Study}
\label{sec:case-study}

To gain insight into the occurrence and significance of collection and
analysis data quality problems, we conducted a case study. The case
study was designed to investigate the following hypothesis:

\begin{quotation}

{\em Data quality problems during collection and analysis can distort the 
PSP data's representation of the programmer's actual behavior, leading to 
invalid process improvement changes.}

\end{quotation}


\subsection{Case Study Design}
\label{sec:design}

The case study began by teaching a one semester course on the PSP, modified
in certain ways in an attempt to improve the quality of PSP data.  The 10
students in the course submitted all of their paper forms to the instructor
after each assignment. Some errors required students to correct and
resubmit prior forms.  The set of paper PSP forms collected over the course
of the semester comprises the {\em original} PSP dataset, and can be
thought of as one experimental treatment.

Next, the second author entered each data value from the original PSP dataset into a
database system that she developed. This database system implements
automated calculation of the derived measures, and detects a subset of the
possible errors that can exist in a PSP dataset.

She also developed a second system to support correction of some of the errors
found through the first database system.  This system corrects a subset of
erroneous data values according to a set of correction rules (specified in
Section \ref{sec:rules}).  Application of these rules produced the second,
{\em corrected} PSP dataset, which can be thought of as the second
experimental treatment. Note that we do not claim that this second dataset
is completely correct, merely that it corrects a set of clearly inaccurate
calculations from the first dataset.

Given this approach, the case study design is similar to a 
within-subjects comparison of a ``control'' treatment (the original PSP
dataset) to the ``experimental'' treatment (the corrected PSP dataset).
Our data analysis is designed to determine whether significant differences exist
between these two treatments. 

In addition to the test of our primary hypothesis, we used the database
systems to perform several additional analyses on the observed  errors to
understand their cause and potential significance to PSP data values and
the method itself.

\subsection{The Modified PSP Curriculum}
  
  The projects used for this study were obtained from a software
  engineering class taught by Philip Johnson, in which the PSP was taught
  over the course of a semester using nine project assignments. There were
  ten students in the class, and 89 completed projects.
   
  Because of the concern with data quality from prior experience teaching
  PSP, the instructor made four principal modifications to the standard PSP
  curriculum: increased process repetition,  increased process description,
  technical reviews, and tool support.  For replication purposes, a more
  detailed description of the curriculum used in this couse is available
  at the website: http://www.ics.hawaii.edu/$\sim$johnson/613s98/.
  
  {\bf Increased process repetition.} In the standard PSP curriculum,
  students are assigned 10 programs during the semester (in addition to
  several midterm and final reports). Over the course of these ten
  programs, students practice seven different PSP processes, which means
  that the development process used by the students changes for seven out
  of ten programs.  From our initial experience with the PSP, we found
  that the overhead of this almost constant ``process acquisition'' led to
  data errors and could overwhelm the effort related to actual development.
  To ameliorate this situation, the modified curriculum included only five
  PSP processes, enabling students to practice most
  processes at least twice before moving on to a new one. The modified
  curriculum also included only nine programs instead of ten, providing
  additional time in each program for data collection and analysis.
  
  {\bf Increased process description.} In our initial experiences teaching
  the PSP, the instructor found that students had a great deal of trouble
  learning to do size and time estimation correctly.  For example, PSP time
  estimation requires choosing between three alternative methods for
  estimation depending upon the types of correlations that exist in the
  historical process data from prior programs.  To help resolve this and
  other problems, the instructor added four additional worksheets: (1) a
  Time Estimating Worksheet to provide a guide through the various methods
  of time estimation; (2) a Conceptual Design Worksheet to help in
  developing class names, method names, method parameters, and method
  return values; (3) an Object Size Category Worksheet to help in size
  estimation; and (4) a Size Estimating Template Appendix to provide a
  place to record planned and actual size for prior projects.
  
  {\bf Technical reviews.} At the completion of each project, students
  divided into pairs and carried out a technical review of each other's
  work.  A two-page checklist facilitated this process.  It included such
  questions as ``Did the author follow the PSP Development Phases
  correctly?'' and ``Is the Projected LOC calculated correctly?''  A second
  ``Technical Review Defect Recording Log'' form included columns for
  number, document, severity, location, and description. Students were
  given approximately 60 minutes to do the review.  The technical review
  forms were submitted with the completed projects.  The instructor
  reviewed the projects a second time for grading purposes, using the
  Technical Review Defect Recording Log to record any additional mistakes.
  
  {\bf Tool support.} Finally, the instructor provided four spreadsheets to
  support records of planned and actual data values. In addition, students
  were provided with well-tested tools to count non-comment source lines of
  code for Java programs, to compare two versions of a Java program and
  report non-comment lines of code added and deleted, and to perform
  certain statistical analyses.  (In the textbook PSP curriculum, students
  ``bootstrap'' their environment by implementing these tools themselves.
  While elegant pedagogically, this approach unfortunately introduces a
  potentially significant source of data quality problems, since these
  freshly developed tools with no usage history are used to generate many
  of the measures used in later data analysis.)

  In addition to these curriculum modifications, the instructor
  emphasized data quality throughout the course, as recommended in the
  textbook.  For example, he augmented the lecture notes in the
  Instructor's Guide with fully worked out examples of the PSP process data
  for a fictitious student to show how data is collected and analyzed for
  each assignment and accumulated over the course of the semester.  He
  dedicated lectures to collection and analysis of data periodically
  throughout the semester. He regularly showed the class aggregate
  statistics on class performance.  He met with students individually and
  in groups throughout the semester to go over their assignments and PSP
  data while they were in the midst of planning, design, code, compile,
  test, and/or postmortem; but prior to project turn-in.  He uncovered and
  removed many, many PSP data errors through these meetings which are not
  counted in our results.  He did technical reviews of every assignment's
  PSP data, and circulated problem reports throughout the semester
  summarizing issues discovered from student data.

\subsection{Case Study Instrumentation}
  
We developed a database application to support analysis of PSP data from
PSP0 to PSP2, using the Progress 4GL/RDBMS.  In order to reduce
opportunities for making mistakes, this tool was designed to require a
minimum amount of user input and to provide the user with default values
whenever possible.  Apart from task and scheduling template values, the
application automated all analysis stage calculations, from determining
delta times for Time Recording Log entries to performing linear regression
for size estimation. In addition, the application guides the user through
the appropriate forms and fields in the order most appropriate for the
current process and phase.
 
\subsection{Data Collection}
  
  Once the database application was ready, we entered data from the student
  project PSP forms and compared each student value with the
  value computed by the application.  Although every discrepancy
  between the manually generated data and the application-generated data
  could be considered an error, we only counted an error at its insertion
  point.  For example, in a Time Recording Log entry for the Design phase,
  if {\it Stop} is incorrectly subtracted from {\it Start}, {\it Delta
    Time} will be incorrect.  Even if all other calculations are done
  correctly for the rest of the project, {\it Time in Phase, Design,
    Actual}; {\it Time in Phase, Total, Actual}; {\it Time in Phase,
    Design, To Date}; {\it Time in Phase, Total, To Date}; {\it Time in
    Phase, To Date \%}; and {\it Time in Phase, To Date} values for an
  indefinite number of future projects will all be inaccurate to some
  degree.  And this is just for the most simple process, PSP0!  In more
  advanced processes, {\it LOC/Hour}, time estimation, {\it Cost-Performance
    Index}, and {\it Defect Removal Efficiency} values could all be
  affected for both the current project and future projects.  To eliminate
  this combinatorial explosion in the number of errors, we counted this as a
  single error in {\it Delta Time}. 
  
  Although we analyzed the project data quite carefully, we do not feel
  confident that we have uncovered all or even most of the errors in this
  case study.  While our database application does enable us to determine
  the correctness or incorrectness of values generated during the analysis
  stage of our data quality model, it provides only limited insight into
  collection stage errors.  For example, in the Time Recording Log, it was
  possible to check the {\it Delta Time} computation, but not the accuracy
  of {\it Date}, {\it Start}, {\it Stop}, or {\it Interruption Time}.  Of
  course, the tool could not, in general, detect the absence of entries for
  work that was done but not recorded.  Two other areas that created
  similar problems were the Defect Recording Log and the measured and
  counted {\it Program Size} fields for the Project Plan Summary.

\subsection{Data Analysis}
  
  In order to analyze the 1539 errors uncovered by the PSP data entry tool,
  we developed a second database application, the PSP Error Data Analysis
  Tool.  For each error discovered, we tracked the person who made the
  error, the method by which the error was found (technical review,
  instructor review, or comparison with the PSP tool results), the
  assignment in which the error occurred, the PSP process used for that
  assignment, the PSP phase in which the student was working when the error
  occurred, the general error type, the specific error type, the severity
  of the error, the age of the error (number of assignments since the
  introduction of the PSP operation in which the error occurred), the
  incorrect and correct values (where applicable), and an optional comment
  for noting issues of interest in that error.

  \subsubsection{Error Correction}
  \label{sec:rules}

  Although our initial analysis of our case study data revealed many
  errors, the sheer presence of errors might only lead to imprecision,
  rather than inaccuracy. In other words, it was possible that these errors
  were only ``noise'', similar in magnitude to naturally occurring random
  fluctuations in behavior, but not sufficient to actually change the
  trends or interpretations of PSP data.
 
  To test this hypothesis, we attempted, where possible, to fix errors so
  that original and corrected versions of the data could be compared.  It
  soon became clear that errors fell into three classes.  First, there were
  errors where the correct value could be determined.  This class included
  such values as {\it LOC/Hour} that were wrong simply because of an
  incorrect calculation. These errors were easily fixed by correctly
  performing the calculation in question.  Second, there were errors where
  the correct value could not be determined, such as a blank {\it Phase
    Injected} for a Defect Recording Log entry.  Fortunately, most errors
  in this class occurred in fields that didn't affect other fields, such as
  missing header data or missing dates in the Defect Recording Log.  Third,
  there were errors where the correct value could be guessed.  In a Time
  Recording Log entry with {\it Start} 10:00, {\it Stop} 10:30, {\it
    Interruption Time} 0, and {\it Delta Time} 40; it is clear that there
  is a problem, but not clear which field is incorrect and should be
  corrected.  However we can guess that there was a problem calculating
  {\it Delta Time} and assume that the other values are valid.  To correct
  this third class of errors in an explicit and consistent fashion, we
  developed a set of rules.  Underlying each of our rules is the assumption
  that primary data is more likely to be accurate than calculations
  performed upon it. The following lists each of the rules along with
  the number of times it was used in the case study.
      
  Rule 1 (used 53 times): Defects in Time Recording Log entries should be
  handled by assuming that the start/stop/interruption times are correct
  and that the delta time is wrong, unless two Time Recording Log entries
  overlap.  In that case, the preceding and following entries and the delta
  time for the current entry should be used to formulate plausible
  start/stop times.  Generally this will mean starting the second entry
  where the first one stops.

      
  Rule 2 (used 5 times): If a Time Recording Log is missing an entry for
  an entire phase, but the Project Plan Summary form contains a value for
  the phase under {\it Time in Phase (min.), Actual}, an appropriate Time
  Recording Log entry should be formulated with fabricated date and time
  values.
        
  Rule 3 (used 28 times): For conflicts between a Defect Recording Log and a Project Plan Summary
  it should be assumed that the number of defects and the phases recorded
  in the Defect Recording Log are correct and that the discrepancy occurred
  as a result of incorrectly adding up the numbers of defects
  injected/fixed per phase and/or incorrectly transferring these totals to
  the Project Plan Summary form.
      
  Rule 4 (used 10 times): If, for the Defect Recording Log, the total of
  all fix times for defects removed in a certain phase is more than the
  time recorded for that phase in the Time Recording Log, a Time Recording
  Log entry should be inserted with start and stop times that, combined
  with the existing Time Recording Log entries for the phase, will produce
  a delta time of the total fix times plus one minute for each defect.
  This will represent the minimum amount of time required to find and
  remove the recorded defects.
      
  Rule 5 (used 1 time): To provide a value for a blank {\it Time in Phase
    (min.), Plan} field on the Project Plan Summary form, the value for
  {\it Time in Phase (min.), Actual} for the same phase should be
  used. Note that this rule, if used widely, would itself introduce
  error into the correction process. However, we used it only once 
  on one project and it has negligible impact upon our results.
    
  Rule 6 (used 62 times): Conflicts in {\it Program Size (LOC)} fields on
  the Project Plan Summary form should be handled by assuming that {\it
    Base, Deleted, Modified, Added, and Reused} are correct and that errors
  are the result of incorrect calculations for {\it Total New and Changed}
  and {\it Total LOC}.  Actually, this is not a truly satisfactory
  assumption because {\it Total LOC, Actual} should be a measurement rather
  than a calculation and should therefore be relied upon.  However, given
  correct values for {\it Base, Deleted, Modified, Added,} and {\it
    Reused}, it is possible to calculate {\it Total LOC}, whereas it is
  impossible to even guess at the correct values for the other fields.
  Unfortunately, defects in the {\it Program Size (LOC)} fields were some
  of the most common defects.  


\subsubsection{Data Comparison}

After we partially corrected the project data according 
to the rule set, we investigated which values to compare to best reveal the
effects of errors.  Projects 8 and 9 had the most fields to
compare since they were completed using PSP2, and provided the best
opportunities for observing the cumulative effect of errors made in earlier
projects. Project 9 was the best project for comparison because students
had had the most practice in PSP by the time this project was completed and
because it provided more time for cumulative effects to exhibit their true
characteristics. Unfortunately one student did not complete this project,
resulting in fewer data points for the final project.
   
One of the more interesting areas for comparison would have been size and
time estimation.  This was not possible due to the difficulties in
adequately correcting the {\it Program Size (LOC)} fields. Instead, we
selected a few fields from each of the other major sections of the Project
Plan Summary, including some fields that resulted from fairly simple
calculations but represented to date values from all nine projects, and
other fields that were more local to the current project but were the
result of more difficult operations.

\subsection{Threats}

We tested the hypothesis investigated in this study by comparing two
PSP datasets: an uncorrected PSP dataset obtained from our students, and a
(partially) corrected PSP dataset produced through automated analysis
and implementation of correction rules.  In this section, we discuss
threats to the internal validity (whether the approach used is actually
valid for testing the hypothesis) and to external validity (whether
the results obtained in this study are applicable to external 
industry and academic practice of the PSP). 

\subsubsection{Internal Validity}

One threat to internal validity is an instrumentation effect.  This could
have occurred in two ways.  First, there could exist defects in the design
and/or implementation of the database system used to create the partially
corrected PSP dataset.  To minimize this threat, great care was taken in
the development of this database system to ensure the accuracy of its
computations, and all data entered was re-checked at least once to ensure
that there were no transcription errors.  It is also relevant to note that
Anne Disney, who designed and implemented the database, has worked
professionally for many years doing database development using the same
DBMS employed in this study.  

A second threat to internal validity occurs from our use of correction
rules. It is conceivable that a correction rule, if improperly designed,
could introduce a systematic bias into corrected dataset that produces an
artificial difference between the two datasets not related to underlying
programmer behavior.  To minimize this threat, we evaluated each of our
rules for the potential presence of such systematic bias.  One rule, in
fact, does have the potential to produce this problem, but we used this
rule in only one case in the entire dataset and our results are not sensitive
to the specific value chosen.

\subsubsection{External Validity}

One threat to external validity is the sample size and nature of our
subjects. Our sample size of 10 students leaves open the possibility
that the results could be an artifact of the individuals involved
in the study.  A related threat involves the use of students for the
study. Perhaps professional software engineers would approach the 
learning of the PSP in a different manner than students, given that
the rewards and motivation structure in industry are quite different 
from academia.

Another threat to external validity is the instructor.  Clearly, the level
of PSP data quality during both collection and analysis is influenced by
the quality of instruction.  It may be possible to obtain different
outcomes merely through alternative approaches to instruction. Furthermore,
the instructor in this study has not attended the official SEI-sponsored PSP
instructor training course.  However, as discussed in Section
\ref{sec:education}, the PSP datasets submitted by students in the case
study show precisely the same sorts of trends reported by other
instructors, and as discussed in Section \ref{sec:instruction}, the number
of PSP dataset errors detected in our study is actually quite low, when
viewed as a percentage of the total number of possible errors.  In
addition, the case study semester was the second time the instructor taught
the PSP curriculum, and student evaluations were overwhelming positive. The
case study outcomes do not appear to be the result of lack of instructor
familiarity with the material or the result of simple student apathy
regarding the course.

While we attempted to minimize these threats to both internal and external
validity of this study, they are still real.  The most effective way to
evaluate these threats is through replication of this study in other
environments using different data verification mechanisms, different
subjects, and different instructors.  We hope that this study will
demonstrate the need and importance of such replication efforts in the PSP
community.




 
%%%%%%%%%%%%%%%%%%%%%%%%%%%%%% -*- Mode: Latex -*- %%%%%%%%%%%%%%%%%%%%%%%%%%%%
%% 5-results.tex -- 
%% Author          : Philip Johnson
%% Created On      : Wed Apr  8 14:24:46 1998
%% Last Modified By: Philip Johnson
%% Last Modified On: Wed Aug 11 10:40:42 1999
%% RCS: $Id$
%%%%%%%%%%%%%%%%%%%%%%%%%%%%%%%%%%%%%%%%%%%%%%%%%%%%%%%%%%%%%%%%%%%%%%%%%%%%%%%
%%   Copyright (C) 1998 Philip Johnson
%%%%%%%%%%%%%%%%%%%%%%%%%%%%%%%%%%%%%%%%%%%%%%%%%%%%%%%%%%%%%%%%%%%%%%%%%%%%%%%

\section{Results}
\label{sec:results}

This section presents two types of results from the case study.  First, we
present the educational results, indicating that students did acquire
substantial insight into software engineering during the semester and
viewed the course as valuable.  Second, we present the data quality
results, obtained from a comparison of the original PSP dataset with the
corrected PSP dataset according to our experimental design as discussed in
Section \ref{sec:design}.



\subsection{Educational Results}
\label{sec:education}

Despite the discovery of data quality problems to be reported below, we
still view the case study semester as an unqualified success from an
educational standpoint.  From a quantitative perspective, student data for
the course parallels the positive outcomes from other PSP
case studies:

\begin{itemize}
\item Average defect density showed a downward trend from around 200
    defects/KLOC to around 50 defects/KLOC, a 75\% decrease.

\item Average productivity showed a very slight positive
    trend, from around 15 LOC/hour to around 20 LOC/hour.

\item Time and size estimation showed dramatic improvement. On the
    last program, both size and time estimation error
    dropped below 15\% for half the class, with several
    student estimates within 3-5\% of their actual values. For
    example, one size estimate of 507 LOC was
    off by only 11 LOC. One time estimate of 14.5 hours
    was off by only 25 minutes.

\item Two students out of ten during the case study achieved what we
    consider to be the ``Holy Grail" of PSP: 100\% yield,
    i.e. programs that compiled and ran correctly the first
    time without any syntax or run-time errors.
\end{itemize}

The qualitative outcomes were equally positive. Most students expressed a
very high degree of satisfaction with the course. The following
comments are typical:

\begin{itemize}

\item ``In September, I didn't know anything about software
engineering.  Now I know a great deal thanks to PSP.  I now
know the importance of why a process is used to finish a
task.  Software development is not easy and using a process
helps in development.'' 

\item ``I thought I was a good programmer, but after using
PSP I realize that I was nothing back then.  Now, I can
proudly say that I have gotten much much better than ever
before."

\item ``I must admit, when I started this course, I understood what we were
  supposed to do in good software engineering, but I never really did it.
  Now I understand the reasons behind these practices and the benefits of
  actually following a process instead of just jumping right into coding...
  Teachers who push doing planning and design might actually know what
  they're talking about."

\item ``At the beginning, I just coded to finish the project
or solve the problem. Now I take an in-depth look at the
problem and think about it for a while before trying to
develop a solution.  By executing and learning this process
I know way more about software engineering than when I
started this course."

\end{itemize}


\subsection{Data Quality Results}

Despite these excellent educational outcomes, comparison of the original
PSP dataset with the corrected dataset yielded 1539 errors.  The following
sections provide a breakdown of these defects according to their type,
severity, age, the manner in which they were detected, whether they
occurred during the analysis or collection stage, their ``ripple effect'',
and the overall percentage error rate.

\subsubsection{Error Types}

We found that the errors naturally fell into one of seven general types.
We present each type in descending order of frequency, and include the
number of errors found of that type and the percentage of all errors
represented by this type.

{\bf Calculation Error.} (705 errors, 46\%). This error type applied
to data fields whose values were derived using any sort of calculation from
addition to linear regression.  If the calculation was not done correctly,
an error was counted.  This type was not used for values that were
incorrect because fields used in the calculation contained bad
numbers.
        
{\bf Blank Field.} (275 errors, 18\%). This error type was used when a data field
required to contain a value, such as the {\it Start} field in a Time
Recording Log entry, was left blank.  This type was not used in fields
where a value was optional, such as comment fields.
        
{\bf Transfer of Data Between Projects Incorrect.} (212 errors, 14\%) This
error type was used for incorrect values in fields that involved data from
a prior project.  Typically these fields were ``to date'' fields that
involved adding a to date value from a prior project with a similar value
in the current project.  Unfortunately, it was often impossible to
determine if the error arose from bringing forward a bad number,
or incorrectly adding two good numbers, or bringing forward the correct number
and correctly adding it to the wrong number from the current form.
However, in two important areas, time and size estimation, the forms were
modified so that students were required to fill in the prior values to be
used in the estimation calculations. In these cases we could determine
when incorrect values originated in the transfer.
        
{\bf Entry Error.} (142 errors, 9\%). This error type applied when a
student clearly did not understand the purpose of a field or used an
incorrect method in selecting data.  Examples include the use of a phase
name in the {\it Fix Defect} field of the Defect Recording Log, or having
the {\it Defects Injected, To Date} values in the Project Plan Summary
originate from a different project than the {\it Program Size (LOC), To
  Date} values.
      
{\bf Transfer of Data Within Project Incorrect.} (99 errors, 6\%). This
error type is similar to the error type involving incorrect transfer of
data between projects, except that it applied to values transferred from
one form to another within the current project.  For example, filling in
172 for {\it Estimated New and Changed LOC} on the Size Estimating
Template, but using 290 for {\it Total New and Changed, Plan} on the
Project Plan Summary.
    
{\bf Impossible Values.} (90 errors, 6\%). This error type indicates that
two values were mutually exclusive.  Examples of this error type include
overlapping time log entries, defect fix times for a phase adding up to
more time than the time log entries for the phase, or phases occurring in
the Defect Recording Log in a different order than those in the Time
Recording Log.
      
{\bf Process Sequence not Followed} (16 errors, 1\%).  This error type was
used when the Time Recording Log showed a student moving back and forth
between phases such as Compile and Test instead of sequentially moving
through the phases appropriate for the process.\newline


\subsubsection{Error Severity}

Some PSP data errors have relatively little ``ripple effect'' upon other
data values, while others can have an enormous impact. To gain insight into
the distribution of the ripple effect, we classified the errors into one of
five ``severity'' levels.  We present the levels in increasing order
of ripple effect.  As before, we include the total number of errors
found for a given severity level and its percentage of the total.


{\bf Error has no impact on PSP data.} (104 errors, 7\%). This level
included errors such as missing header data, incorrect dates in the time
recording log, and filling in fields for a more advanced process.
        
{\bf Results in a single bad value, single form.} (674 errors, 44\%).  This
level was used if a significant field which affected no other fields, such
as {\it LOC/Hour, Actual}, was blank or incorrect.
        
{\bf Results in multiple bad values, single form.} (197 errors, 13\%).
This level indicates when an incorrect or blank value was used in the
calculation of values for one or more other fields on the same form, but
when none of these other values were used beyond the current form.  For
example, in PSP1 on the Size Estimating Template, incorrectly calculating a
prediction interval.  This results in a bad prediction interval and a bad
prediction range, but these values are not used anywhere else in the
process.
        
{\bf Results in multiple bad values, multiple forms, single project.} (41
errors, 3\%).  This level indicates when an incorrect or blank value was
used to determine the values for one or more other fields on one or more
different forms in the same project, but when none of these other values
were used beyond the current project.  For example, in PSP1, on the Size
Estimating Template, calculating an incorrect value for {\it Estimated
  Total New Reused (T)}.  This results in an incorrect value for {\it Total
  New Reused, Plan} on the Project Plan Summary form, but this value is not
referenced by future projects.
        
{\bf Results in multiple bad values, multiple forms, multiple pro\-jects.}
(523 errors, 34\%).  This level was used if an incorrect or blank value
affected future projects.  For example, when {\it Defects Injected,
  Planning, Actual} on the Project Plan Summary does not match the number
of defects entered for the planning phase in the Defect Recording Log.

\subsubsection {Age of Errors}

In any learning situation, a certain number of errors are to be expected.
We hypothesized that perhaps the errors we discovered were simply a natural
by-product of the learning process, and would ``go away'' as students
gained experience with the various techniques in the PSP.

To evaluate this hypothesis, we calculated the ``age'' of errors---in other
words, the number of projects since the introduction of the data field in
which the error could be observed. If the errors were simply a by-product
of the learning process, then we would expect a low average ``age'' for
errors.  In other words, people might make an error in a field initially,
but then stop making the error after gaining more experience with the data
field in question.

For example, the calculation of {\it Delta Time} for the Time Recording Log
was introduced in the first project.  If a student made an error in this
field during the first project the error would have an age of zero.  If a
similar error was made during the second project the error would have an
age of one.  By the ninth project this type of error would have an age of
eight.
      
We first analyzed the errors to determine the average error age in each
project.  Figure \ref{errorAgeAll} shows the average age for all errors
in each project.

\begin{figure}[htbp]
  \begin{center} 
  \begin{tabular}{|l|l|r|r|}\hline 
  Project \# & PSP Process & \# of Errors & Average Age \\ \hline\hline 
  1 & PSP0    &  51  &  0.00 \\ \hline
  2 & PSP0.1  &  59  &  0.73 \\ \hline    
  3 & PSP0.1  &  63  &  1.76 \\ \hline
  4 & PSP1    & 150  &  1.27 \\ \hline
  5 & PSP1    & 165  &  2.27 \\ \hline
  6 & PSP1    & 186  &  3.30 \\ \hline
  7 & PSP1.1  & 160  &  3.26 \\ \hline
  8 & PSP2    & 351  &  3.04 \\ \hline
  9 & PSP2    & 354  &  3.84 \\ \hline
  \end{tabular}
  \end{center}
  \caption{\label{errorAgeAll}Average Error Age by Project - All Errors} 
  \end{figure}
      
  We then filtered out the 309 errors with an age of zero.  This
  eliminated errors that could result from students being
  introduced to new fields and/or PSP operations for the first time.  
  Figure  \ref{errorAgeSome} shows the resulting data.

  \begin{figure} [htpb]
  \begin{center} 
  \begin{tabular}{|l|l|r|r|}\hline 
  Project \# & PSP Process & \# of Errors & Average Age \\ \hline\hline 
  1 & PSP0    &   0  &    NA \\ \hline 
  2 & PSP0.1  &  43  &  1.00 \\ \hline 
  3 & PSP0.1  &  63  &  1.76 \\ \hline
  4 & PSP1    &  70  &  2.71 \\ \hline
  5 & PSP1    & 165  &  2.27 \\ \hline
  6 & PSP1    & 186  &  3.30 \\ \hline
  7 & PSP1.1  & 135  &  3.86 \\ \hline
  8 & PSP2    & 214  &  4.99 \\ \hline
  9 & PSP2    & 354  &  3.84 \\ \hline
  \end{tabular}
  \end{center} 
  \caption{\label{errorAgeSome}Average Error Age Where Age is not Zero}
  \end{figure}
      
When combining the 1539 errors from all projects, the average error age
was 2.78 projects.  After removing the 309 errors with an age of zero,
the average error age rose to 3.48 projects.


\subsubsection{Error Detection Methods}

In this study, there were three ways an error could be detected: by another
student during technical review (40 errors), by the instructor during the
grading/evaluation process (32 errors), or through the use of the PSP data
entry tool (1467 errors).  Thus, students were made aware of about 5\% of
the mistakes in their completed projects during the course of the class.

\subsubsection{Analysis Stage Errors}

Our two stage model of PSP data quality indicates that errors can be
introduced during either collection or analysis. 
Most of the errors that we detected occurred during PSP
analysis activities, with 700 errors occurring in the Plan phase and 561
errors in the Postmortem phase. Some of the errors occurring in other
phases, such as errors in {\it Delta Time} calculations, were also analysis
errors.

\paragraph{The Most Severe Errors.} 

34\% of errors found were of the most serious type - persistent errors.
These were the errors resulting in multiple bad values on multiple forms
for multiple projects.  A defect of this type not only causes incorrect
values in the current project, but may still be causing flawed results ten
projects later, even if all subsequent calculations are done correctly.
Figure \ref{errorsCommon} shows the four most common errors of this type.

\begin{figure} [htpb]
 
    \begin{center} 
    \begin{tabular}{|l|r|}\hline 
    Description & \# \\ \hline\hline 
    Time Estimation: historical data  &    \\ 
    not transferred correctly         & 61 \\ \hline
    Size Estimation: historical data  &    \\
    not transferred correctly         & 56 \\ \hline 
    
    Time Log: delta time incorrect    & 48 \\ \hline 
    Project Plan Summary: Total LOC,  &    \\
    actual, not equal to B-D+A+R      & 45 \\ \hline 
    \end{tabular}
    \end{center}  
    \caption{\label{errorsCommon} Most Frequently Occurring Persistent Errors}   
\end{figure}
    
There were two main ways that the error in transferring time estimation
data appeared to occur: incorrectly transferring the value from the correct
field, or accidentally transferring the correct value from an incorrect
field.  For example, instead of transferring {\it Total New and Changed
  (N)} (Plan or Actual), students often transferred {\it Total LOC (T)}.
This could easily occur because the Project Plan Summary form has over 90
fields even at the level of PSP1, and these two values are vertically
adjacent on the form. It is particularly easy to make this mistake with the
Actual values because the fields are separated by one column from the
labels.  Additionally, it appeared that students made spreadsheets to avoid
thumbing through the entire stack of completed projects every time a time
or size estimation was needed for a new project.  We infer this because the
same incorrect value for a particular project would be transferred over and
over again for time and/or size estimation in new projects.
    
Similar factors surround the error in transferring data for size
estimation.  These transfer errors were not insignificant.  Over the 56
errors resulting from incorrect transfer of data used for size estimation,
the sum of the errors was 7753 LOC (lines of code), with an average error
of 138.4 LOC.  The sum of the LOC as they should have been transferred was
10,255, with an average of 183 LOC per field.  Thus, the average
incorrectly transferred number was in error by an amount equaling 75.6\% of
the number that should have been transferred.
    
The error in calculating {\it Delta Time} in the Time Recording Log was
notable in several respects.  First, the errors were not insignificant.
The average mistake was 37.8 minutes, which was an average of 39.9 percent
of the correct value. Second, of 48 occurrences, 16 were in error by one
hour and 4 were in error by two hours, indicating small errors in simple
arithmetic. Third, the distribution of this error across projects is as
shown in Table \ref{deltaErrors}.

\begin{figure}
   \begin{center} 
   \begin{tabular}{|l|r|r|r|}\hline 
   Project \#  & Errors & Time Log Entries & \% in Error \\ \hline\hline 
   1  & 7 &  84 &  8.33 \\ \hline 
   2  & 2 &  88 &  2.27 \\ \hline 
   3  & 8 &  92 &  8.70 \\ \hline 
   4  & 8 & 108 &  7.41 \\ \hline  
   5  & 2 & 102 &  1.96 \\ \hline 
   6  & 9 & 121 &  7.44 \\ \hline 
   7  & 2 &  77 &  2.60 \\ \hline 
   8  & 5 & 122 &  4.10 \\ \hline 
   9  & 5 & 105 &  4.76 \\ \hline 
   \end{tabular} \newline \newline
   \end{center} 
   \caption{\label{deltaErrors}Delta Time Errors by Project}
\end{figure}
   
Despite nine projects worth of experience, this error never ``went away''.
However it did appear to occur less frequently after Project 6.
Interestingly, the assignment for this project was a Time Recording Log
applet, which at least some students seem to have used for subsequent
projects.

 
\subsubsection{Collection Stage Errors}

As noted previously, analysis stage errors are relatively easy to determine
and correct. However, the accuracy of recorded process measures from the
collection stage was much more difficult to examine because the time of
collection had already passed and, unlike the analysis operations, was
impossible to reproduce. However, we found both direct and indirect
evidence for collection errors during the case study.

\paragraph{Direct Collection Error Evidence.}

Direct evidence of collection problems appeared in the 90 errors of
type of ``Impossible Values''.  We classified these errors into
three major subtypes.
      
{\bf Internal Time Log Conflicts.} There were five time logs with
overlapping entries, indicating some sort of problem with accurately
collecting time-related data.

{\bf Internal Defect Log Conflicts.} 51 errors showed problems with
correctly collecting defect data.  48 of these errors were Defect Recording
Log entries showing defects that were injected during the Compile and Test
phases, but these same defects were not noted as being the result of
correcting other defects found during Compile or Test.
      
{\bf Discrepancies Between Time and Defect Logs.} In 22 cases, Defect
Recording Log entries were entered with dates that did not match any Time
Recording Log entries for the given date.  For example, a defect would be
recorded as removed during the Code phase on a Wednesday, but the time log
would show that all coding had been completed by Monday and that the
project was in the Test phase on Wednesday.  For 10 projects, the total
{\it Fix Time} for defects removed during a particular phase added up to
more time than was recorded for that phase in the Time Recording Log.
Finally, in two cases, the Defect Recording Log showed a different phase
order than the Time Recording Log.


\paragraph{Indirect Collection Error Evidence}

Besides the recorded errors, there were other indicators that collection
problems had occurred. Some Time Recording Logs showed a suspicious number
of even-hour (e.g. 6:00 to 7:00, 10:00 to 12:00) entries, even though
students
were required to record times at the minute level.  Others showed
long stretches of consecutive entries with no breaks or interruptions.
Often, the total {\it Fix Time} for the defects in a phase was far less
than the time spent in the phase. For example, the Time Recording Log might
show three hours spent in the Test phase, but the Defect Recording Log
would show two defects that took eight minutes to fix.  Obviously, it is
not impossible that this would occur, but it is much more likely that not
all defects found in test were recorded.  

In a similar vein, some projects had suspiciously few defects overall, such
as seven defects for a project with 284 new lines of code and almost 11 hours 
of development time, (including 40 minutes in compile for two defects requiring 
6 minutes of fix time). Our analysis of the PSP data for that same project 
yielded 27 errors.

Finally, the instructor has anecdotally observed the following trend in
every PSP course he has taught so far: the students turning in the highest
quality projects also tend to record far higher numbers of defects than the
students who turn in average or lower quality projects.  If this trend is
real, then we can provide two possible explanations. It may be the case
that the students turning in lower quality projects tend to make far fewer
errors than those turning in the higher quality projects, although this
seems {\em extremely} unlikely.  What appears more likely is that the
students turning in the highest quality projects also exhibit the lowest
level of collection error, which indicates that substantial but
non-enumerable collection error exists in the PSP data we examined.


\subsubsection{Comparison of Original and Corrected Data}

% \begin{figure}
%    \begin{center} 
%    \begin{tabular}{|l|c|c|}\hline 
%    Student & Original & Corrected  \\ \hline\hline 
%    A & 0.32 &  0.99  \\ \hline 
%    B & 0.89 &  1.05  \\ \hline 
%    C & 1.17 &  1.05  \\ \hline 
%    D & 0.76 & 1.29  \\ \hline  
%    E & 0.67 & 0.78  \\ \hline 
%    F & 0.79 & 1.12  \\ \hline 
%    G & 0.27 &  1.29  \\ \hline 
%    H & 0.91 & 0.67  \\ \hline 
%    I & 0.87 & 1.39  \\ \hline 
%    J & 1.32 & 1.39  \\ \hline 
%    \end{tabular} \newline \newline
%    \end{center} 
%    \caption{\label{compareCPI} A comparison of original and corrected
%    cost-performance index values for the ten students in the case study.}
% \end{figure}

% \begin{figure}
%    \begin{center} 
%    \begin{tabular}{|l|c|c|}\hline 
%    Student & Original & Corrected  \\ \hline\hline 
%    A & 69 &  5.7  \\ \hline 
%    B & 36 &  27  \\ \hline 
%    C & 47 &  19  \\ \hline 
%    D & 48 & 23  \\ \hline  
%    E & 11 & 12  \\ \hline 
%    F & 11 & 13  \\ \hline 
%    G & 8 &  8  \\ \hline 
%    H & 9 & 9  \\ \hline 
%    I & 45 & 7  \\ \hline 
%    J & 57 & 26  \\ \hline 
%    \end{tabular} \newline \newline
%    \end{center} 
%    \caption{\label{compareYield} A comparison of original and corrected Yield values for the ten students in the case study.}
% \end{figure}

%PJ
When we compared the original and corrected data, we found significant
differences (p$<$.05) for the Cost-Performance Index (planned
time-to-date/actual time-to-date) and Yield (percentage of defects injected
before first compile that were also removed before first compile).  We used
the Wilcoxon Signed Rank Test \cite{Ferguson89}, a non-parametric test of
significance which does not make any assumptions regarding the underlying
distribution of the data.  Figure \ref{compareCPI} and Figure
\ref{compareYield} illustrate the differences between these two measures
graphically.  


  \begin{figure} [htbp]
    {\centerline{\psfig{figure=8cpi2.eps}}}
    \caption{\label{compareCPI}Effect of Correction on CPI}
  \end{figure}

  \begin{figure} [htbp]
    {\centerline{\psfig{figure=8yield2.eps}}}
    \caption{\label{compareYield}Effect of Correction on Yield}
  \end{figure}


A CPI value of 1 indicates that planned effort equals actual effort. CPI
values greater than 1 indicate overestimation of resource requirements,
while CPI values less than 1 indicate underestimation of resource
requirements.  In half of the subjects, correction of the CPI value
reversed its interpretation (from underplanning to overplanning, or
vice-versa).  In the remaining cases, several corrected CPI values differed
dramatically from original values.  For example Subject A's original CPI
was 0.32, indicating dramatic underplanning, while the corrected CPI was
0.99, indicating an average planned resource requirements virtually equal
the average actual resource requirements.

Correction of yield values tended to move their values downward, sometimes
dramatically.  In half of the subjects, the corrected yield was less than
half of the original yield values, indicating that subjects were removing a
far fewer proportion of defects from their programs prior to compiling than
indicated by the Yield measurement.

These particular results confirm our hypothesis. In the case of CPI, use of
the uncorrected data would lead half of the subjects to make exactly
the wrong process improvement. In the case of yield, use of the uncorrected
data would lead subjects to not take process improvement actions 
indicated when yield is low. 


\subsubsection{Overall Percentage Error Rate}
\label{sec:instruction}
 
Such a large number of data quality errors calls into question the 
quality of instruction. Perhaps these results are a simple artifact of
poor quality control on the part of the teacher? Unfortunately, 
the very large number of data values to check in the manual PSP suggests
otherwise. 

For example, a time recording log contains six fields (plus a comment
field, but for our purposes, this field is extraneous): {\em Date, Start, Stop, Interrupt time,
Delta Time,} and {\em Phase}. Students typically entered about 10 time log entries
for an assignment.  This results in 60 data values to check for one student
on one assignment, and 600 data values to check for a class of 10 students.
This is for one form and one assignment. Following this approach, one can
arrive at an estimate of almost 32,000 data values to be checked by hand
for this single case study, as illustrated in Figure \ref{overallErrors}.
The 1539 data errors uncovered during this study represents only 4.8\% of
the total possible, which means that the instructor obtained over 95\%
correctness (at least with respect to analysis-stage data quality).

\begin{figure}
   \begin{center} 
   \begin{tabular}{|l|r|r|r|}\hline 
   Process & Approx. Fields & Projects & Total Values \\ \hline\hline 
   PSP0    & 200 &  10 &  2000 \\ \hline 
   PSP0.1  & 220 &  20 &  4400 \\ \hline 
   PSP1.0  & 329 &  20 &  6580 \\ \hline 
   PSP1.1  & 437 &  20 &  8740 \\ \hline  
   PSP2.0  & 528 &  19 &  10,032 \\ \hline 
   \multicolumn{2}{|r|}{\bf Total} &  {\bf 89} &  {\bf 31,752} \\ \hline 
   \end{tabular} \newline \newline
   \end{center} 
   \caption{\label{overallErrors}Data values present in PSP}
\end{figure}






%%%%%%%%%%%%%%%%%%%%%%%%%%%%%% -*- Mode: Latex -*- %%%%%%%%%%%%%%%%%%%%%%%%%%%%
%% 6-discussion.tex -- 
%% Author          : Philip Johnson
%% Created On      : Wed Apr  8 14:25:13 1998
%% Last Modified By: Philip Johnson
%% Last Modified On: Wed Aug 11 10:47:53 1999
%% RCS: $Id$
%%%%%%%%%%%%%%%%%%%%%%%%%%%%%%%%%%%%%%%%%%%%%%%%%%%%%%%%%%%%%%%%%%%%%%%%%%%%%%%
%%   Copyright (C) 1998 Philip Johnson
%%%%%%%%%%%%%%%%%%%%%%%%%%%%%%%%%%%%%%%%%%%%%%%%%%%%%%%%%%%%%%%%%%%%%%%%%%%%%%%
%% 

\section{Discussion}
\label{sec:discussion}

This paper reports on the results of analysis of the data from a single PSP 
class with only 10 students.  As with any case study, care must be taken
in interpreting these results.  We do not know whether this data is 
representative of PSP courses in general, and if the way we teach the
PSP is representative of the way the PSP is taught by others. Data quality
problems might be less prevalent in other PSP courses; on the other hand,
they might just as easily be more prevalent.

While we do not claim that these results are representative of all PSP
courses, neither do we believe that they are an artifact of some peculiarity
and/or failing of our environment.  First, this case study was performed
after the instructor had taught the PSP for one semester in a graduate
level course, and instituted it within his research group, and adopted it
himself for his own software development activities.  By the time of this
study, we were quite experienced as both teachers and users of the PSP.
Second, as already noted, we were concerned with data quality problems from
the beginning, and instituted curriculum modifications specifically
intended to raise data quality. The overall error rate of less than 5\%,
while quite small, was still not sufficiently small to prevent significant
differences between original and corrected data sets. Third, our results cannot
be due to our lack of enthusiasm for the PSP: both of us consider it to be
one of the most powerful software engineering practices we have adopted in
our careers.  The second author, for example, has used her automated PSP
tool to gather data on over 120 of her industrial projects over the past
two years.  Fourth, our results cannot be due to lack of enthusiasm for the
PSP by our students, as the post-course comments reveal, most of the
students indicated that they found the class to be very useful and
interesting.

\subsection{Recommendations for research and practice}

Based upon the results of this study, we have the following recommendations 
for future research and practice of the PSP:


\subsubsection{Replication} 

We believe this study provides strong evidence for the need for more
research on collection and analysis data quality in the PSP.  Current
studies of the PSP appear to take the accuracy of PSP data for granted, or 
else simply assume that tool support can eliminate all sources of 
data quality problems. 
This study is the first to methodically examine the assumptions underlying
data quality in the PSP and subject them to empirical investigation. Our
results indicate that the PSP community may be overly optimistic about the
quality of PSP data, particularly when produced using the traditional, manual
approaches that lack integrated, PSP-specific tool support. Even when such
support is provided, the possibility of measurement dysfunction introduces
substantial threats to the accuracy of the data in the collection phase
as discussed below in Section \ref{measurement-dysfunction}.  Better
understanding of the true extent of PSP data quality problems requires
replication of this study, or at least further PSP research that includes
PSP data quality verification as an explicit design component.  To 
support this endeavor, researchers are invited to peruse a website
containing curriculum materials from this course at
http://www.ics.hawaii.edu/$\sim$johnson/613s98/.  



\subsubsection{Software engineering education}

We continue to believe that the PSP has substantial educational value in
software engineering, despite the issues we have raised with data quality.
Students learn valuable, concrete skills concerning defect management and
planning in the PSP curriculum.  Additionally, the PSP provides students
with a framework for empirically evaluating the usefulness of any other
process improvement techniques or programming methods they come across in
the future.  


\subsubsection{PSP tool support}

We believe that integrated tool support for the PSP is required, not
merely helpful, to obtain high analysis-stage PSP data quality.  We also
believe that integrated tool support will make adoption of the PSP
substantially easier, since the most common complaint made by students
using the manual PSP in our classes is the time and effort required to fill
out the forms.  Currently, we have designed and implemented a Java-based
toolset for integrated empirical software process improvement that
automates many of the analysis stage computations in the PSP, and which
extends the PSP paradigm with support for group review and patterns
\cite{Moore98}. We are currently using this toolset, called Leap, in a
software engineering course and will deploy it in an extensively
redesigned PSP-like curriculum in Fall 1999.  

\subsubsection{PSP research design}

We believe that the results of this case study have a number of
implications for current and future research on the PSP.

First, until questions raised by this study with respect to PSP data
quality are resolved, PSP data should not be used to evaluate the PSP
method itself. In other words, we believe that it is not yet appropriate to
assume that changes in PSP measures during (or after) a training course
accurately reflect changes in the underlying developer behavior.  A
statement such as ``The improvement in average defect levels for engineers
who complete the course is 58.0\%'', if based upon PSP data alone, might
only reflect a decreasing trend in defect recording, not a decreased trend
in the defects present in the work product.

Second, our research on the PSP has demonstrated that high quality pedagogical
design is not equivalent to high quality experimental design.  In other
words, some of the features of the PSP with respect to pedagogy are bugs
with respect to experimental design. 

One problem in the PSP with respect to experimental design concerns
uncontrolled instrumentation. The PSP programming exercises incrementally
build a set of tools for use in gathering and managing PSP data.  This is
elegant pedagogically, since it enables an instructor to use the PSP and
have the students build partial tool support for it as they go along.
Unfortunately, this is disastrous from an experimental design viewpoint,
since it means that crucial primary data measures (size) and derived
measures (size and time estimates) are calculated from a set of student
programs with no experimental control over their quality and accuracy.  We
know from bitter experience that writing a high quality size counting and
differencing tool for Java that handles all aspects of the language and
produces both a meaningful measure of size and differences in size between
two versions of a program is a nontrivial programming project. It
requires extensive design, implementation, and field use far beyond the 10
days available for this program in the PSP curriculum.  For the PSP
curriculum to be useful experimentally, there must be control over and
verification of the instrumentation.

Another problem in the PSP with respect to experimental design concerns the
lack of control over curriculum modifications.  For example, the SEI study
notes that ``there are many cases where instructors tailored the training
course (including selection of assignments, data collection requirements,
and sequence of introduction for process changes.)''  Our course also
deviates from the standard curriculum. 
  
Yet another problem in the manual PSP with respect to experimental design
concerns systematic bias in the data. For example, the PSP curriculum
requires, in an academic setting, a full semester course. In academic
settings, the workload on students tends to be light during the beginning
of the semester, become heavier after midterms, and reach a peak near the
end of the semester.  For PSP measures to be accurate, students must
maintain a consistent level of process data collection throughout the
course of the semester.  From our personal experience, we have observed
that a portion of the students in our PSP classes appear to begin to ``cut
corners'' in their recording of defects and time near the end of the semester,
presumably due to external pressures on their time and energies.  This
``end of semester crunch'' can introduce a systematic bias into PSP data,
leading to, for example, artificial decrease in defect density values
near the end of the course.

Another example of systematic bias can occur from what we term the ``process
overhead ceiling effect''.  Many students complain that the amount of
effort collecting and analyzing PSP process data can equal, exceed, or
interfere with the time and focus required to actually develop the
programs.  Early in the course, process overhead consists almost purely
of time and defect data collection, so students devote a great deal
of time and energy to that task.  By the end of the semester, the
total process overhead of the PSP has risen dramatically, since estimation,
time and schedule planning, and so forth have all been added. If 
at least some portion of the students decide to limit the amount of 
time spent on process collection and analysis, the most likely place
to cut corners is, once again, in defect recording, which would 
once again produce an artificial decrease in defect density values near the 
end of the course.

A final example of systematic bias occurs from the format of the manual PSP
forms themselves. As we note in our results, the case study students
frequently transferred a ``Total'' LOC value from one form to another
instead of the ``New and Changed'' LOC value.  Since the Total value is
always greater than ``New and Changed'', a systematic bias toward inflated
system sizes is present. We found other situations in which the design
of the forms lead to consistent user errors. 

From an experimental design standpoint, uncontrolled instrumentation and
systematic bias are threats to the internal and external validity of any
study which both uses the manual PSP and which draws conclusions about
underlying programmer behavior based purely upon the PSP data.  One example
of research suffering from these threats is the Software Engineering
Institute technical report by Hayes and Over \cite{Hayes97}.  The report
refers to collection of ``paper forms'', indicating the manual PSP.  There
is no mention of any control over the quality and accuracy of PSP
instrumentation, such as the size counter.  There is no mention of any
rigorous validation of the PSP data. Instead, the researchers simply claim
that ``the quality and accuracy of the data used in any given class tend to
be exceptional.''  Unfortunately, our case study shows that even an
accuracy of over 95\% in the PSP dataset is insufficient to obtain data
accurately reflecting underlying programmer behavior. Furthermore, our
original dataset is quite consistent in its outcome with the 
aggregate outcome reported by the SEI.  The research design presented
in their report cannot detect the data quality problems in our original
dataset, and so presumably cannot detect data quality problems present in
any of the datasets actually used in the study. Finally, although the
researchers subjected the PSP data to extensive statistical analysis, these
analyses all assume the absence of systematic bias in the dataset, an
assumption which we believe to be incorrect in the manual PSP.

We are happy to note that not all PSP evaluations are based upon PSP data
alone. For example, in one industrial case study, evidence for the utility
of the PSP method was based upon reductions in acceptance test defect
density for products subsequent to the introduction of PSP practices
\cite{Ferguson97}.  Although alternative explanations for this trend can be
hypothesized (such as the PSP-based projects were more simple than those
before and thus acceptance test defect density would have decreased
anyway), at least the evaluation measure is independent of the PSP data and
not subject to PSP data quality problems.

\subsubsection{Collection data quality and measurement dysfunction}
\label{measurement-dysfunction}

Unfortunately, integrated tool support is not a ``magic bullet'' that will
solve all PSP data quality problems.  As our simple model of PSP data
quality shows, no matter how perfectly we are able to automate the analysis
stage, overall PSP data quality will still depend largely upon the data
quality from the collection stage.

Our case study was able to detect substantial numbers of analysis errors
which could be eliminated through appropriate automation.  Our case study
was also able to detect the potential presence of substantial collection
errors, but the solution to this issue is much more complex.  It is
currently beyond the state of the art to accurately and completely automate
the collection of all primary process measures (time, size, defects) for a
programmer.  For the foreseeable future, we must rely on users of the PSP
to accurately and consistently record primary data values.  

In our research on the collection data quality problem, we have gained
insight from research on ``measurement dysfunction'' \cite{Austin96}.
According to Austin, whenever you measure an attribute of an organization
with the goal of improving the organization's performance, you run the risk
of worsening the organization's performance as a direct result of the
measurement.  This is because there are at least two uses to which a given
measurement can be applied: for information and for motivation.

Informational measurement ``tells about an organizational process... It is
used to learn from and to plan.''  In the PSP, all measures are intended to
be informational.

Motivational measurement, on the other hand, ``is used to quantify the
value of compensation for compliance with objectively verifiable standards
of work.'' In other words, motivational measurement is used to evaluate the
performance of individuals. In the PSP, no measures are meant to be
motivational.

Although this seems straightforward, a principal claim of Austin's research
is that any individual measure is "value-free" with respect to its
application: it can be used for informational purposes, motivational
purposes, or both.  Importantly, it is impossible for an organization to
guarantee that a measure, once collected, will never be used for
motivational purposes.  Thus, individuals in an organization may tend to
operate under the assumption that any measures of individual performance
can be used for motivational purposes, regardless of the stated intention
of the organization with respect to that measure at the time it is taken.

We find the measurement dysfunction perspective quite revealing with
respect to the PSP, because in any PSP academic or industrial training
situation, the ``organization'' collects the PSP measures from the
individual.  Even though competant PSP instructors always inform the
students that they will not be evaluated on the actual values of their PSP
data they collect, measurement dysfunction theory indicates that
individuals may still act under the assumption that they might at some
point become accountable for the values they submit.  As PSP data provides
very revealing and potentially dangerous information about a programmer's
practice, the appropriate PSP data to provide the organization for
motivational measurement may be quite different from the appropriate data
for personal, informational measurement.

We conjecture that collection stage data quality requires, at a minimum, a
combination of low collection overhead along with environmental features
that minimize the potential for measurement dysfunction. Overhead can be
reduced through tool support that makes manual recording of time, defect,
and size data fast and accurate.  Minimizing measurement dysfunction
requires, in essence, the property of privacy for PSP data---in other
words, that the organization does not and cannot have access to an
individual's PSP data.

Measurement dysfunction, unfortunately, introduces yet another obstacle to
the use of PSP data for experimental purposes.  In order to teach a PSP
course effectively, the instructor must inspect the PSP data submitted by
students.  However, this essential educational feature violates the privacy
of an individual's PSP dataset, an essential feature to minimize
measurement dysfunction.  The problem of measurement dysfunction, on top of
the problems cited earlier, lead us to question if collecting PSP data from
an educational setting is a fundamentally unsound approach to assessing
underlying programmer behavior. If this is true, we must redesign our
current paradigms for research using the PSP.







    

\section{Acknowledgments}

We gratefully acknowledge all of the students in all of the PSP classes at
the University of Hawaii.  Our colleagues in the Collaborative Software
Development Laboratory during the time of this research (Cam Moore, Robert
Brewer, Jennifer Geis, Joe Dane, and Russ Tokuyama) provided ongoing
support.  We would like to thank Watts Humphrey, James Over, and Will Hayes
of the Software Engineering Institute, and the anonymous reviewers, whose
comments sharpened the presentation of this research.  This research was
sponsored in part by grants CCR-9403475 and CCR-9804010 from the National
Science Foundation.


\bibliographystyle{plain}
\bibliography{/group/csdl/bib/psp,/group/csdl/bib/ftr} 

\end{document}



