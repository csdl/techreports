%%%%%%%%%%%%%%%%%%%%%%%%%%%%%% -*- Mode: Latex -*- %%%%%%%%%%%%%%%%%%%%%%%%%%%%
%% 3-model.tex -- 
%% Author          : Philip Johnson
%% Created On      : Wed Apr  8 14:23:59 1998
%% Last Modified By: Philip Johnson
%% Last Modified On: Wed Jul 21 09:14:16 1999
%% RCS: $Id$
%%%%%%%%%%%%%%%%%%%%%%%%%%%%%%%%%%%%%%%%%%%%%%%%%%%%%%%%%%%%%%%%%%%%%%%%%%%%%%%
%%   Copyright (C) 1998 Philip Johnson
%%%%%%%%%%%%%%%%%%%%%%%%%%%%%%%%%%%%%%%%%%%%%%%%%%%%%%%%%%%%%%%%%%%%%%%%%%%%%%%
%% 

\section{Modeling PSP Data Quality}
\label{sec:model}

As we pursued this investigation of data quality problems in the PSP, we
found it necessary to build a model and define some new terminology to 
clarify the approach of this research and its conclusions. This section
presents the model and this terminology.


\subsection{Collection and Analysis Stages}

\begin{figure*}
    {\centerline{\psfig{figure=PSPmodel.eps}}}
    \caption{\label{fig:model} A simple model for PSP data quality. Through 
      a process of {\em collection}, the developer generates an initial
      empirical representation (``Records of Work'') of her personal process
      (``Actual Work'').  Through additional {\em analyses}, the developer
      augments her initial empirical representation with
      derived data (``Analyzed Work'') intended to enable process improvement
      through ``Insights about Work''.
      }
\end{figure*}
     
Figure \ref{fig:model} illustrates a simple two stage model of PSP data
representing an iterative cycle of {\em collection} followed by {\em
  analysis}. The model begins with ``Actual Work''---the actual developer
efforts devoted to a software development project.  As part of these
efforts, the developer enters the collection stage during which she
collects a set of {\em primary} measures of defects, time, and work product
characteristics---the ``Records of Work''.  Given these primary measures,
the developer performs additional analyses during the Analysis Stage, many
of which produce {\em derived} measures which are themselves inputs for
further analyses.  The secondary, derived measures and associated analyses
are presented in various PSP forms---the ``Analyzed Work''---and hopefully
yield ``Insights about Work'' that change and improve future actual
software development work activities.

\subsection{Manual and Integrated Automation}

We have also found it important to distinguish between the different levels of
automation possible in the PSP.  ``A Discipline for Software Engineering''
\cite{Humphrey95} presents an approach to PSP automation that we call {\em
  manual} PSP.  Manual PSP refers to a situation in which PSP forms are
filled out by hand, either by editing a copy of the form on-line, or by
filling out a printed copy with pen or pencil.  Manual PSP does not
preclude the use of tools ``on the side'' to store historical data and to
perform certain computations.  In our classification scheme, if PSP tool
support cannot eliminate manual manipulation and recalculation of derived
measures, and thus guarantee their consistency and accuracy 
(assuming consistent and accurate primary measurements), then we view the level
of automation as manual.

In contrast, we use {\em integrated} PSP to refer to the use of tools in which most
or all of the derived measures during the analysis stage are calculated and
placed into the forms automatically.  Although integrated PSP should
essentially automate all analysis stage calculations, there are
limits to the ability of current technology to automate collection stage
gathering of primary measures. For the foreseeable future, this part of the
PSP will continue to be essentially ``manual'' in nature.

At the time we performed this case study, as was the case at the time of
publication of ``A Discipline for Software Engineering'', there was no
integrated software support for the PSP.  Thus, the case study involved
manual PSP, despite our extensive use of spreadsheets, program size
counting tools, and statistical tools during the course.  Since then, 
integrated support for the PSP has become available, including
the Personal Software Process Studio tool
produced by East Tennessee State University \cite{Henry97}, and 
the Leap toolset at the University of Hawaii \cite{Moore98}.

\subsection{Omission, Addition, Calculation, and Transcription Errors}

There are three basic ways to affect PSP data quality in the collection
stage: errors of omission, errors of addition, and errors of transcription.
Errors of omission occur when the developer does not record a primary
measure related to defects, time, or the work product itself.  If a defect
occurring during ``Actual Work'' does not appear in the ``Records of
Work'', then, for example, the PSP model of that work product's defect
density will be lower than its actual defect density. If time spent on the
work product is not recorded, then the PSP model of that developer's
productivity will be higher than her actual productivity. Errors of
addition occur when the developer augments the ``Records of Work'' with
data not reflecting actual practice. For example, a developer, having made
an error of omission to the point of having no time or defect data, may
recover by simply inventing enough time and defect entries to make his or
her PSP data appear plausible. Finally, errors of transcription occur when
the developer does intend to accurately record his ``Actual Work'' in the
``Records of Work'' but makes a mistake during this process.

The presence of collection stage data quality problems is typically
difficult to ascertain and difficult or impossible to rectify. In the
PSP, primary data collection often feels both time consuming and
psychologically disruptive.  Many students complain that stopping to record
defects disrupts their ``flow'' state, and that the time spent recording a
defect---particularly for compilation stage errors---often exceeds the time
spent correcting the defect.  The PSP requires users to learn to constantly
interleave ``doing work'' with ``recording the work you are doing''.

There are also three basic ways to affect PSP data quality in the analysis
stage of manual PSP: errors of omission, errors of calculation, and errors of
transcription.  Errors of omission occur when the developer does not
perform a required analysis of the primary data. Errors of calculation
occur when the developer attempts to perform an analysis but does so
incorrectly. For example, a developer might use a regression-based
estimation method when the historical data is so uncorrelated that this
method is invalid. Finally, errors of transcription occur when the
developer makes a clerical error when moving data from one form to another.

Analysis stage data quality problems are typically much
easier to ascertain and correct than collection stage data
quality problems, {\em provided that the problems did not originate
  during the collection stage}.  In other words, if one assumes that the
work records accurately reflect the underlying work, then appropriate use
of automated tools can reduce or eliminate analysis errors of omission,
calculation, and transcription.  On the other hand, since the quality of
these analyses are totally dependent upon the quality of the work records
produced in the collection phase,
overall PSP data quality could be quite low even if the analysis stage is
totally automated to eliminate all of its potential data quality errors.







