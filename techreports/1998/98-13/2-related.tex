%%%%%%%%%%%%%%%%%%%%%%%%%%%%%% -*- Mode: Latex -*- %%%%%%%%%%%%%%%%%%%%%%%%%%%%
%% 2-related.tex -- 
%% Author          : Philip Johnson
%% Created On      : Wed Apr  8 14:23:39 1998
%% Last Modified By: Philip Johnson
%% Last Modified On: Wed Jul 21 09:12:22 1999
%% RCS: $Id$
%%%%%%%%%%%%%%%%%%%%%%%%%%%%%%%%%%%%%%%%%%%%%%%%%%%%%%%%%%%%%%%%%%%%%%%%%%%%%%%
%%   Copyright (C) 1998 Philip Johnson
%%%%%%%%%%%%%%%%%%%%%%%%%%%%%%%%%%%%%%%%%%%%%%%%%%%%%%%%%%%%%%%%%%%%%%%%%%%%%%%
%% 

\section{Related Work}
\label{sec:related}

There is a small but growing number of research studies describing
experiences with teaching and evaluating the PSP. This section overviews
the relevant literature with special attention paid to three questions.

First, how does the research study evaluate the effectiveness of the PSP?
The most common approach is to compare PSP data collected at the beginning
of the course to data collected at the end of the course. We term these
kinds of evaluations {\em internal measurement evaluation}, because
measures collected using the PSP are used to evaluate the PSP itself.  The
less common approach is to compare some other measure of programmer/program
quality collected before the introduction of PSP training to the same
measure of programmer/program quality after the PSP training.  We term
these kinds of evaluation {\em external measurement evaluation}, because
measures collected independently of the PSP training itself are used to
evaluate PSP effectiveness.

Second, how does the research study verify the accuracy of the measurements
used in evaluation--in other words, verify that the measurements actually
reflect the underlying behavior of the PSP users?  This question addresses
one form of internal validity: that the research is designed
in such a manner that the data can actually be used to answer the questions
posed.  The most common approach to measurement verification in PSP
research is {\em manual inspection} of the data by the instructor. In some
cases, the researcher employed {\em subject exclusion}, i.e. eliminating
all the data from one or more subjects based upon data incompleteness or
some qualitative appraisal of data quality.

Third, what issues related to data quality are raised by the research
study?  Does the research present any experiences related to low data
quality and how they might be overcome?  The most common improvement 
mentioned in the research is automation. 

To begin, we present a brief overview of the PSP itself.

\subsection{The Personal Software Process}

In the PSP curriculum presented in ``A Discipline for Software
Engineering'' \cite{Humphrey95}, each student develops 10 small programs
over the course of a semester using a sequence of seven increasingly
sophisticated software development processes, labeled PSP0 to PSP3.  For
every program, the students record various measurements related to their
personal development activities. Such measures include, for example, the
time spent in each phase of development, the numbers of defects injected
and removed during each phase, and the size of the resulting work product.
Five analysis exercises focus on trends and relationships between 
all of the process data collected to that point in the course.

The initial programs use relatively simple PSP processes that establish a
baseline set of process measures for time, size, and defects. Later
programs employ more advanced PSP processes that extend these baseline
process statistics.  Although there are a myriad of extensions, 
most are of two general types: extensions to planning and extensions
to defect management.

Extensions to the planning phase include estimates of the program's
projected size, the projected time required to complete each of the phases,
and the number of anticipated defects that will be injected and removed
during development.  The process by which these estimates are produced
involves statistical analysis of historical correlations between designs
(i.e. class and method counts) and actual size (in lines of code), between
estimated size and actual time, between actual size and actual time, and
between size and defects injected and removed.  (While lines of code as a
metric of size at the organizational level is almost uniformly condemned
in the measurement literature, it seems to work surprisingly well in the
PSP, since the measure is collected and applied to a single individual working
in a single language in a relatively uniform domain.)

By the middle of the course, each student has typically recorded a hundred
or more defects made during development.  More advanced PSP processes
implement defect management mechanisms to help students understand the
impact of various kinds of defects and to drive process improvements
intended to reduce future occurrence of important defect types. For
example, since students record the phase in which each defect was injected and
removed and the time required to fix it, it is possible to analyze the
relationship between fix time and various characteristics of defects.  One
relationship nearly always present in student data is that the ``longer'' a
defect is present, the more time it takes to remove it.  Thus, defects
injected during design and not removed until testing are nearly always more
expensive to remove than, for example, defects injected during coding and
removed during compiling. This outcome typically motivates students to put
more effort and care into design activities. Later processes support such
behaviors by providing active defect management mechanisms. For example, by
analyzing defect data to determine the types of design defects made on
prior projects, a student can generate a checklist to be used as part of a
personal design review. This checklist can be used to ensure that when
similar defects occur in future projects, they do not escape into code,
compile, or test phases.

The most advanced PSP processes extend the basic PSP paradigm to support
larger projects using a cyclic development method. In addition, PSP
includes a meta-level process for defining personal processes in
non-software domains or for specific software organizational contexts.

Students record their PSP data onto one or more PSP forms provided in the
textbook and made available electronically at the textbook publisher's
website. Students fill out these forms manually and turn them in to the
instructor.  Supplemental PSP spreadsheets automate some of the
calculations, which students manually transfer to the forms. The number of
forms filled out increases from three for the first PSP process to over a
dozen for the most advanced processes.

In this curriculum, PSP data quality is addressed in two ways. First, the
instructor manually reviews all PSP data as it is submitted, and is
instructed to accept only complete and accurate PSP data.  If an error on a
prior assignment is discovered, the student must go back and recompute both
the measurements for the prior assignment and all assignments subsequently
affected by this error. (The provided spreadsheets can ease this process.)
Second, the instructor should exhort the students to approach this course
professionally, and to recognize that only the completeness and accuracy of
the data, not the actual values for defect density, productivity, etc. will
be used in the determination of their grade for the course.

The chosen PSP programming assignments are also important to
understanding issues of data quality, because the standard PSP assignments
require students to build software systems that they later invoke to
produce important PSP measurements and analyses.  For example, program 2A
builds a size counter, a tool necessary for obtaining one of the three
primary measures in the PSP. Other assignments produce statistical
calculations necessary for PSP analyses, including linear regression,
correlation, multiple regression, and prediction intervals.


\subsection{Case Studies of the PSP}

The PSP text contains the original ``case study'' of the PSP. Throughout
the book, Humphrey informally presents data collected from students in PSP
classes at Carnegie Mellon University during pre-publication development of
the text. Utilizing an internal measurement evaluation approach, Humphrey
compares PSP data collected from early in the semester with data collected
at the end of the semester, and finds trends toward decreased defect
density, improved yield, and improved estimation accuracy.  Data
verification appears to consist of manual inspection in conjunction with
some subject exclusion.  

Humphrey provides more information about his experiences with the
PSP in several related articles
\cite{Humphrey94,Humphrey94a,Humphrey95a,Humphrey95b,Humphrey96}.  In
general, he presents results based on PSP data collected from over 100
engineers in both industrial and academic settings.  Humphrey employs
internal measurement evaluation and subject exclusion when engineers
``reported either incomplete or obviously incorrect results.''  Results
include an increase in size and time estimation accuracy, and a
reduction in reported
defects of approximately 50\% over the course of the training.

In his master's thesis, Dellien describes his attempt to introduce a
tailored version of the PSP into an existing industrial organization
\cite{Dellien97}.  He analyzed the PSP, broke it down into components such
as measurement and quality management, compared these components with the
existing processes used by the organization, and rebuilt a modified version
of the PSP that addressed the perceived needs of developers but did not
result in overlapping organizational processes.  His evaluation concludes
that using PSP in an industrial setting is different than using it in an
academic setting due to differences in accountability, group context, lack
of automation, and resistance to change.  In his evaluation, Dellien found
that it is difficult to objectively evaluate whether a PSP introduction has
been successful or not, even when success is measured purely as
cost-effectiveness.

Sherdil and Madhavji use the PSP as the basis for research on
human-oriented improvement in the software process \cite{Sherdil96}.  This
research attempts to measure an individual's ``progress function'', using
such variables as productivity, defect rate, and estimation error.  The
analysis attempts to differentiate progress due to ``first order learning''
(i.e. simple task repetition, unrelated to the PSP) and progress due to
``second order learning'' (i.e. introduction of PSP techniques).  Their
evaluation uses internal measurement evaluation with standard PSP measures
to track the progress function.  Verification involved manual inspection of
PSP data ``... for consistency, accuracy and logical validity.  Automatic
tools were also used to verify the program size values.  We also checked if
two subjects were illegally exchanging code, but never found such an
occurrence.''  


Hayes and Over report on a statistical analysis of a set of PSP data sets
in an attempt to understand the overall impact of PSP education
\cite{Hayes97}.  This ``case study on a set of case studies'' involves 298
engineers who spent more than 15,000 hours writing over 300,000 LOC and
removing about 22,000 defects, during the course of 23 separate PSP
training programs in both academic and industrial settings.  Hayes and Over
used internal measurement evaluation to demonstrate improvement in size
estimation, time estimation, and defect density, with no significant change
in productivity.  The report does not indicate that the authors performed
any independent data verification or assessment of data quality, though the
authors do claim that the data quality is ``exceptional'':
\begin{quotation}
Instructors enter the engineers' data into a spreadsheet provided with the
course materials.  The paper forms completed by the engineers are collected
by the instructor, and the class data are analyzed and used to provide
feedback to the engineers.  During the training, trends in class data
provide insights to the engineers, who may then compare their own data with 
that of the group.  Given this careful focus on data and statistical
analysis, the quality and accuracy of the data used in any given class
tends to be exceptional.
\end{quotation}

Emam, Shostak, and Madhavji report on a study of the implementation of PSP
in a commercial setting, with special emphasis on adoption success
\cite{Emam96}.  Unlike the majority of PSP studies, their evaluation
focussed primarily on the long-term adoption success, rather than
short-term changes in PSP measurements during training.  In addition, they
present several issues that may impact adversely upon PSP data quality and
thus the use of internal measurement evaluation. For example, high levels
of reuse can act as a confounding factor on trends in productivity
and defect density.  Trends in defect density could reflect changes
in defect detection capabilities rather than changes in the underlying
density of defects in the work product. Trends in yield could be 
primarily due to introduction of code reviews and not due to any other
aspects of the PSP.   Finally, they found that the paper intensive nature
of PSP was problematic for professional engineers.


Ferguson, Humphrey, Khajenoori, Macke, and Matvya report on case studies of
the PSP at three industry locations: Advanced Information Services,
Motorola Paging Projects Group, and Union Switch and Signal
\cite{Ferguson97}.  Unlike most other studies, PSP effectiveness was
evaluated primarily using external measures.  For example, post-development
defect report data (either during acceptance test or field use) was used to
compare the quality of PSP-developed projects with non-PSP-developed
projects.  Similarly, one of the studies compared schedule estimation error
before and after PSP training.  These case studies showed substantial 
improvement with respect to both defects and estimation on industry 
projects after introduction of the PSP.   The use of external measures
for evaluation, and the particular external measures chosen (such as 
customer-reported defects) eliminates the issue of PSP measurement
accuracy. The researchers did not present any issues regarding 
data quality. 

Claes Wohlin discusses the use of the Personal Software Process as
a context for empirical experimentation \cite{Wohlin98a}.  He finds that
the PSP has several desirable aspects for experimentation, including a
comprehensive specification of the experimental context (i.e. the PSP
textbook \cite{Humphrey95}), relative ease in replication, and the ready
availability of experimental measures. Among the challenges he cites are
internal validity and external validity.  In the case study used as an
example, data from six students were removed because it was ``regarded as
invalid or at least questionable.''

Our own case study, presented in this paper, is intended to contribute to
this body of knowledge concerning the PSP by demonstrating the importance
of explicit concern for data quality beyond what is covered in the PSP
textbook.  The next several sections present our case study and its
results.  In Section \ref{sec:discussion}, we will revisit several of the
case studies presented above and reinterpret them in the light of our
findings. 
