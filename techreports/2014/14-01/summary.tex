%%%%%%%%%%%%%%%%%%%%%%%%%%%%%% -*- Mode: Latex -*- %%%%%%%%%%%%%%%%%%%%%%%%%%%%
%% summary.tex -- 
%% Author          : Philip Johnson
%% Created On      : Tue Mar 31 11:42:10 2009
%% Last Modified By: Philip Johnson
%% Last Modified On: Wed Dec 16 16:20:25 2009
%% RCS: $Id$
%%%%%%%%%%%%%%%%%%%%%%%%%%%%%%%%%%%%%%%%%%%%%%%%%%%%%%%%%%%%%%%%%%%%%%%%%%%%%%%
%%   Copyright (C) 2009 
%%%%%%%%%%%%%%%%%%%%%%%%%%%%%%%%%%%%%%%%%%%%%%%%%%%%%%%%%%%%%%%%%%%%%%%%%%%%%%%
%% 

\section*{Project Summary}
%\renewcommand{\thepage} {A--\arabic{page}}

% {\em Each proposal must contain a summary of the proposed project not more than one page in length. The Project Summary consists of an overview, a statement on the intellectual merit of the proposed activity, and a statement on the broader impacts of the proposed activity.

% The overview includes a description of the activity that would result if the proposal were funded and a statement of objectives and methods to be employed. The statement on intellectual merit should describe the potential of the proposed activity to advance knowledge. The statement on broader impacts should describe the potential of the proposed activity to benefit society and contribute to the achievement of specific, desired societal outcomes.

% The Project Summary should be written in the third person, informative to other persons working in the same or related fields, and, insofar as possible, understandable to a scientifically or technically literate lay reader. It should not be an abstract of the proposal.}


% This project will investigate the use of crowdsourced power quality data to determine if it can enable active participation by consumers in the Smart Grid.  It will also investigate the hardware, software, and analytic requirements for crowdsourced data that make it effective for prediction and diagnosis of Smart Grid power quality issues.

\noindent {\bf Overview.}  

The ``Smart Grid'' represents a new vision for
the electrical infrastructure of the United States, whose goals include
more active participation by consumers, new generation and storage options
including renewable energy, and new products, services, and markets. 

This project will design, implement, and evaluate a combination of low-cost hardware and cloud-based software that enables consumers to monitor power quality (voltage, frequency, and total harmonic distortion) in their household and upload that data to our Internet service to produce a crowdsourced perspective on power quality.  The approach enables consumers to learn: (a) whether their household is experiencing degraded power quality; (b) whether the observed problems are isolated to their own house or more widespread in their neighborhood; (c) whether the problems are intermittent, frequent, unpredictable, or regularly occurring; and (d) whether the problems are severe enough to warrant calls to the utility and/or purchase of residential UPS systems to protect power quality-sensitive appliances such as computers.

The technical development of hardware and software is combined with a deployment into 150 households in Hawaii in order to determine if crowdsourced power quality data can enable active participation in the Smart Grid, and to assess the ability of the hardware and software to detect, monitor, predict and diagnose Smart Grid problems.

\medskip

\noindent {\bf Intellectual Merit.} 

This project will produce an variety of innovations, including: (1) low cost, open source hardware for residential power quality monitoring of voltage, frequency, and total harmonic distortion; (2) an open source cloud-based repository for sotorage, retrieval, and analysis of this data and its combination with additional environmental data including temperature, humidity, and insolation; (3) the addressing of privacy concerns by allowing consumers to ``coarsen'' their locational information; (4) the use of a pre and post-test experimental design in order to gain insight into the effect that power quality data has upon consumers with respect to their attitudes and behaviors toward the Smart Grid and the utility implementing it.


\medskip 

\noindent{\bf Broader Impacts.}  

Our hardware, software, and data will be released as open source, providing new and useful infrastructure for others interested in investigating power quality and its potential influence on consumer engagement with the smart Grid.This project will serve underrepresented populations, as the University of Hawaii is in an EPSCOR state. Approximately 84\% of undergraduates at the University of Hawaii are minorities.

\medskip

\noindent {\bf Key Words:} smart grid; human behavior; social networks;
energy.




