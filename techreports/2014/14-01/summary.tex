%%%%%%%%%%%%%%%%%%%%%%%%%%%%%% -*- Mode: Latex -*- %%%%%%%%%%%%%%%%%%%%%%%%%%%%
%% summary.tex -- 
%% Author          : Philip Johnson
%% Created On      : Tue Mar 31 11:42:10 2009
%% Last Modified By: Philip Johnson
%% Last Modified On: Wed Dec 16 16:20:25 2009
%% RCS: $Id$
%%%%%%%%%%%%%%%%%%%%%%%%%%%%%%%%%%%%%%%%%%%%%%%%%%%%%%%%%%%%%%%%%%%%%%%%%%%%%%%
%%   Copyright (C) 2009 
%%%%%%%%%%%%%%%%%%%%%%%%%%%%%%%%%%%%%%%%%%%%%%%%%%%%%%%%%%%%%%%%%%%%%%%%%%%%%%%
%% 

\section*{Project Summary}
%\renewcommand{\thepage} {A--\arabic{page}}

% {\em Each proposal must contain a summary of the proposed project not more than one page in length. The Project Summary consists of an overview, a statement on the intellectual merit of the proposed activity, and a statement on the broader impacts of the proposed activity.

% The overview includes a description of the activity that would result if the proposal were funded and a statement of objectives and methods to be employed. The statement on intellectual merit should describe the potential of the proposed activity to advance knowledge. The statement on broader impacts should describe the potential of the proposed activity to benefit society and contribute to the achievement of specific, desired societal outcomes.

% The Project Summary should be written in the third person, informative to other persons working in the same or related fields, and, insofar as possible, understandable to a scientifically or technically literate lay reader. It should not be an abstract of the proposal.}


% This project will investigate the use of crowdsourced power quality data to determine if it can enable active participation by consumers in the Smart Grid.  It will also investigate the hardware, software, and analytic requirements for crowdsourced data that make it effective for prediction and diagnosis of Smart Grid power quality issues.

\noindent {\bf Overview.}  

The ``Smart Grid'' represents a more sustainable vision for the electrical infrastructure of the United States, whose goals include: more active participation by consumers; new generation and storage options including renewable energy; and new products, services, and markets.  Hawaii's nation-leading penetration of distributed photovoltaics has created concerns related to power quality and grid stability with significant technical, social, and economic dimensions.

The objectives of this project are to design, implement, and evaluate a combination of low-cost open source hardware, software, and data that provides new visibility into end-user power quality (voltage, frequency, and total harmonic distortion). The project provides data regarding household and/or neighborhood power quality events, how these events relate to other environmental phenomena, and recommendations regarding appropriate action to take.

The methods include evaluation of the system through a case study experimental design involving deployment into 150 households in Hawaii. It will assess the extent to which crowdsourced power quality data can enable active participation in the Smart Grid, and the ability of the system to detect, monitor, predict and diagnose Smart Grid problems.

\medskip

\noindent {\bf Intellectual Merit.} 

This project will produce innovations including: (a) low cost, open source hardware for residential power quality monitoring of voltage, frequency, and total harmonic distortion; (b) an open source cloud-based repository for storage, retrieval, and analysis of this data along with environmental data including temperature, humidity, lightning, and insolation; (c) the addressing of privacy concerns by allowing consumers to ``coarsen'' locational information; (d) the use of a pre and post-test experimental design in order to gain insight into the effect that power quality data has upon consumers. 

The project will test the following hypotheses: (1) Knowledge of personal power quality problems leads to actions such as contacting the utilities, installing UPS, or unplugging on alerts; (2) Intrinsic motivators (insight into personal and neighborhood power quality) plus a free device will suffice for participation in crowdsourced data collection; (3) Knowledge of neighborhood power quality issues leads to active engagement with neighbors; (4) Consumers find the recommendations provided by the system to be useful; (5) The frequency and severity of events is positively correlated with the degree of penetration of distributed PV on that circuit; (6) Consumers find crowdsourced power quality data to be more useful than their own power quality in isolation; (7) Participation is positively correlated with high monthly bills, installation of rooftop PV, or high numbers of severe PQ events.


\medskip 

\noindent{\bf Broader Impacts.}  

Hawaii leads the nation in penetration of distributed photovoltaics, and so the contributions this project makes now toward understanding the social, technical and economic problems of end-use power quality will benefit other communities across the U.S. later. The project will also aid in understanding the use of crowdsourcing for citizen science in general.  The hardware, software, and data will be released as open source, providing maximally usable infrastructure for others interested in investigating end-use power quality and its potential influence on consumer engagement with the Smart Grid.  The infrastructure will also support workforce training through course work, case studies, and group projects.

Our completed system will enable any community to rapidly deploy a low-cost mechanism for end-use power quality data collection and analysis. Our project will test hypotheses enabling communities in future to make the best use of the data.

This project will serve underrepresented populations, as the University of Hawaii is in an EPSCOR state. Approximately 84 percent of undergraduates at the University of Hawaii are minorities. 

\medskip

\noindent {\bf Key Words:} smart grid; human behavior; social networks;
energy.




