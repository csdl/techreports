%%%%%%%%%%%%%%%%%%%%%%%%%%%%%% -*- Mode: Latex -*- %%%%%%%%%%%%%%%%%%%%%%%%%%%%
%% project-background.tex -- 
%% Author          : Philip Johnson
%% Created On      : Tue Mar 31 11:44:58 2009
%% Last Modified By: Philip Johnson
%% Last Modified On: Wed Dec 16 15:29:18 2009
%% RCS: $Id$
%%%%%%%%%%%%%%%%%%%%%%%%%%%%%%%%%%%%%%%%%%%%%%%%%%%%%%%%%%%%%%%%%%%%%%%%%%%%%%%
%%   Copyright (C) 2009 
%%%%%%%%%%%%%%%%%%%%%%%%%%%%%%%%%%%%%%%%%%%%%%%%%%%%%%%%%%%%%%%%%%%%%%%%%%%%%%%
%% 

\section{Vision Statement}
\label{sec:vision}

% Enabling active participation in the Smart Grid through crowdsourced power quality data

% {\em The project description of a Type 1 proposal must begin with a concise statement of the project vision, situating an integrative, collaborative research project within a clearly articulated context of larger sustainability goals. It is understood that the proposed research may focus only on a piece of this vision, but the relationship of the proposed research to the overall vision must be articulated here. The vision statement must also briefly address important human, social, behavioral, economic, and adoption issues that are relevant, whether or not they are directly addressed within the proposed research.}

Development of the ``Smart Grid'', a modernized power infrastructure, is a key sustainability challenge facing the United States. According to the Department of Energy, the Smart Grid should: (1) Enable active participation by consumers by providing choices and incentives to modify electricity purchasing patterns and behavior; (2) Accommodate all generation and storage options, including wind and solar power.  (3) Enable new products, services, and markets through a flexible market providing cost-benefit trade-offs to consumers and market participants; (4) Provide reliable power that is relatively interruption-free; (5) Optimize asset utilization and maximize operational efficiency; (6) Provide the ability to self-heal by anticipating and responding to system disturbances; (7) Resist attacks on physical infrastructure by natural disasters and attacks on cyber-structure by malware and hackers \cite{NETL:GridCharacteristics}.

The very first goal, enable active participation by consumers, involves a fundamental paradigm shift.
Electrical utilities traditionally focus on achieving the opposite goal: enabling {\em passive} 
consumers whose participation is limited to plugging in appliances and paying a monthly bill. The historical success of utilities at reliably providing high quality power at low cost has led to multiple generations of consumers who know almost nothing about how their homes and workplaces are powered.

Initial efforts to enable active participation have focused on providing consumers with energy consumption data, and have achieved only limited success with respect to participation, savings, and long-term adoption \cite{Darby06,Faruqui09,Foster2012}.  Feedback on consumption does facilitate one-time positive behaviors, such as installing new insulation or purchasing energy efficient appliances. While helpful, such behaviors do not constitute active participation.    Time-of-use pricing is another approach to active participation, but the typical result is installation of controllers and a ``set it and forget it'' behavior.  Unfortunately, feedback on consumption can even facilitate negative behaviors, such as when consumers install grid-tied solar panels and then consume more electricity because it now appears to be ``free''.  In all of these cases, consumer participation is ultimately personal and short-term.

We believe that an exclusive focus on consumption is not the way to active participation---in other words, sustained awareness and engagement at both personal and community levels. There is another, synergistic focus, however. In Hawaii, our nation-leading adoption of distributed, intermittent renewables such as rooftop photovoltaics has created the potential for significant negative impact on power quality, with attendant impact on consumer electronics reliability and overall grid stability \cite{Rodriguez2010,Laskar2012}. Unfortunately, attempts by Hawaii's major electrical utility to address this problem by restricting PV installation have resulted in media and government scrutiny and significant consumer backlash \cite{Yonan2013,Zunin2013,Elston2013,Cocke2013,Cocke2012}.  Hawaii thus serves as an ideal testbed for determining if  consumer-oriented power {\em quality} feedback can lead not only to active participation in the Smart Grid but to public buy-in for the investments needed to make the overall grid more reliable.

To investigate this, we have initiated the Open Power Quality (OPQ) research project \cite{opq-site}, which involves the design and implementation of a combination of low-cost hardware and cloud-based software that enables consumers to monitor power quality (voltage, frequency, and total harmonic distortion) in their household and upload that data to our Internet service to produce a crowdsourced perspective on power quality.  Our approach enables consumers to learn: (a) whether their household is experiencing degraded power quality; (b) whether the observed problems are isolated to their own house or more widespread in their neighborhood;  (c) whether the problems are intermittent, frequent, unpredictable, or regularly occurring;  and (d) whether the problems are severe enough to warrant calls to the utility and/or purchase of residential UPS systems to protect power quality-sensitive appliances such as computers. 

% OPQ also captures data with the potential to support limited forms of prediction (such as whether certain forms of cloud cover create local instabilities) along with limited forms of diagnosis (such as whether an instability was a result of malfunction within a house or due to interactions between multiple houses).

At first glance, shifting the focus from consumption to power quality might seem even less likely to produce active engagement by consumers. For one thing, consumers have a direct economic interest in consumption since it appears on their monthly bill, while power quality does not appear to have a visible cost.  For another thing, power quality seems like a low-level technical issue that should be entrusted to the utility to monitor and maintain. 

We disagree with both of these assertions. First, consumers do have a direct economic interest in power quality. Over the past 20 years, consumers have significantly increased use of electronics creating non-linear loads (i.e. PV inverters, power supplies, photocopiers, computers, laser printers, battery chargers). Such nonlinear loads reduce power quality by injecting harmonics, which have been shown to reduce appliance efficiency, cause overheating, and increase power and air conditioning cost \cite{Rodriguez2010}. Ironically, these are the very same devices that are sensitive to power quality problems. A study by the National Power Laboratory indicates that the average computer site is subject to almost 100 potentially harmful power quality events per year \cite{Dorr1992}.

Second, utilities rarely have equipment in place to monitor power quality at the household level, nor are there regulatory requirements on utilities to monitor or report power quality. The finest granularity reporting required in the U.S. is called MAIFI, which tracks the number of occurrences of {\em outages} lasting three to five minutes. Various groups have called MAIFI inadequate to measure the presence and consumer cost of non-outage power quality events related to voltage, frequency, and harmonic distortion \cite{Rouse2011,LaCommare2004,Eto2008}. Making matters worse, Moreno-Munoz cites an estimate that more than 30\% of the power being drawn from utilities is headed for sensitive equipment, and that this percentage is rising \cite{Moreno-Munoz2007}. 

To summarize: consumers require better power quality than ever before, but at the same time are installing both consumer electronics and distributed generation that can have a negative impact on the performance, cost, reliability, and lifespan of their own electrical appliances. Utilities are not required to monitor and report power quality events, and as we will see in Section \ref{sec:background-hardware}, current utility-scale equipment is too expensive for wide-spread deployment. As a result, active participation by consumers (via power quality monitoring) may not be simply a desirable goal for the Smart Grid, it may in fact be a necessary prerequisite for achieving the other goals of the Smart Grid. 

Our research and development efforts over the past year have demonstrated the basic technical feasibility of our approach.   We now propose an interdisciplinary study to assess whether the potential benefits of crowdsourced power quality data can be achieved in practice, and to better understand the social, behavioral, and economic trade-offs inherent in our approach.  In general, we propose to investigate the following research questions:

\begin{enumerate}

\item {\em Can crowdsourced power quality data enable active participation in the Smart Grid?}

\item {\em What are the technical, social, behavioral, and economic requirements for crowdsourced data that make it effective for detection, monitoring, prediction and diagnosis of selected Smart Grid power quality issues?}

\item {\em How can our project outcomes improve ``citizen science'' in general and the kinds of intrinsic and extrinsic motivators needed for success?}

\end{enumerate}

To investigate the first question, we will manufacture and distribute 150 power quality monitoring devices to volunteer households in three Oahu neighborhoods, producing power quality data to be stored in our public cloud-based service over a period of six months. Through pre and post test questionnaires along with analysis of data collected by our devices, we will assess the extent to which crowdsourced power quality data influences household member attitudes toward the electrical utility and public policy regarding the grid, as well as behaviors including: interaction with neighbors regarding power quality; calls to the utility; unplugging of sensitive consumer electronics during periods shown to correlate with power quality events (such as thunderstorms); and installation of residential UPS systems in response to household power quality events.  We will compare this data with a control group of households who do not receive the monitoring devices. 

To investigate the second question, we will combine the power quality dataset collected above with environmental data (temperature, humidity, wind speed and direction, lightning, and insolation), household consumption data (as available through an opt-in procedure), and household generation through photovoltaics (again, as available through an opt-in procedure).  We will perform exploratory analysis on the resulting dataset to determine how sampling rates, synchronization, precision, and history impact on detection, monitoring, prediction and diagnosis of power problems.

To investigate the third question, we will compare and contrast the design and outcomes of our approach to other citizen science projects such as Frogwatch \cite{frogwatch}, Urban Coyote Sightings \cite{urbancoyote}, and CosmoQuest \cite{cosmoquest}.

To be successful, our approach requires a combination of engineering, human-computer interaction, data analytics, and social science experimentation.  Philip Johnson is a Professor of Computer Science, has a research background in software engineering, and currently leads the NSF-sponsored Kukui Cup project \cite{kukuicup} to investigate the application of gamification techniques to improve energy literacy and behavioral change. Matthias Fripp is an Assistant Professor of Electrical Engineering, has a research background in power systems, and leads the Switch project \cite{switch} that supports deployment of renewable and conventional energy resources in large power systems. Daniele Spirandelli is an Assistant Professor in the Department of Urban and Regional Planning, has a research background in environmental planning and climate change, and leads projects to better understand the feedback systems between environmental change and human behavior. 

To help maximize its benefits to and impact upon the scientific community, the OPQ project is ``triple open source'': the hardware schematics are available under the CERN Open Hardware License, the software and firmware are available under the GNU Public License Version 3, while the power quality data will be available under the Open Data Commons Open Database License.  Our intent is to create a developer and user community around an ecosystem of hardware, software, and data that maximizes forward progress in understanding the relationship between household power quality, grid dynamics, and consumer engagement.

% As a longer range goal, we intend the OPQ project to provide a basis for research into the application of crowdsourced data to address other sustainability-related resources.  For example, water quality and air quality could be investigated using the same basic paradigm. 

% In the remainder of this proposal we will review research related to Open Power Quality, present our project plan, and summarize its intellectual merit and broader impacts.


% Unfortunately, our current fossil fuel-based grid system is unsustainable for a variety of ecological, political, economic, and social reasons, and so a principle goal for the Smart Grid is to integrate distributed, intermittent energy sources such as wind and solar.  Hawaii, due to its high cost of energy, abundant availability of wind and solar, and small population, has become a living laboratory for the social, technical, and political problems that can occur during the transformation to a Smart Grid.

% For example, over the past five years, Hawaii has far exceeded the rest of the nation in the rate of installation of photovoltaics, and now has the highest per capita penetration of PV in the country.  This rapid uptake has been driven by economics: the combination of high electricity rates (roughly triple the mainland) coupled with tax incentives has meant that solar system installation costs can be recovered in 5-7 years. In Hawaii, companies will literally rent your roof to install panels. 
 
% Our recent experience in Hawaii shows that the goal of ``active participation'' is only partially, and sometimes just temporarily, achieved by consumers installing solar panels.  In fact, PV installation can actually work against this goal when consumers decide that the installation of solar panels has ``solved their energy problem'' and thus they no longer need to engage with the energy issue.   Worse, solar installation can increase consumptive behaviors (for example, running air conditioners more frequently or setting thermostats to a more consumptive setting) because PV owners now perceive their electrical energy as ``free''. 

% Hawaii's recent experience yields a great deal of insight into the social, technical, and economic complexities of achieving the various goals for the Smart Grid.  High penetration of a distributed, intermittent, and unpredictable energy source such as PV creates significant new challenges in maintaining grid reliability.  For example, on the island of Oahu, substations currently interpret backflow (i.e. energy flowing the ``wrong way'') as a fault in the grid, and respond by shutting down power distribution. This technical characteristic means that utilities must limit the installation of grid-tied PV below the substation level to an amount whose generation capacity does not exceed the lowest consumption level for the consumers serviced by that substation. Otherwise, the possibility exists that PV generation could exceed consumption, leading to backflow, interpreted as a fault, and subsequent shutdown. 

% This technical issue has had enormous social consequences. 

% Go from here into our hypothesis: that to get active participation, we need to not focus on consumption, which is an individual issue, but rather focus on energy quality, which is a community problem.   That if we can succeed in getting broad consumer engagement with the issue of power quality, then instead of ``withdrawing''  from the grid upon installation of PV, they will be more engaged as they realize that they are now partners with the utilities not only for generation but for reliability as well.  

% So, we will test this through the open power quality project....

% If successful, OPQ will have other benefits: the ability to create low-cost devices that will correlate grid power quality measurements with other factors. 

% (Have Leon Roose read this prior to submission.  Co-PIs?)


