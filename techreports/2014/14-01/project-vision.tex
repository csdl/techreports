%%%%%%%%%%%%%%%%%%%%%%%%%%%%%% -*- Mode: Latex -*- %%%%%%%%%%%%%%%%%%%%%%%%%%%%
%% project-background.tex -- 
%% Author          : Philip Johnson
%% Created On      : Tue Mar 31 11:44:58 2009
%% Last Modified By: Philip Johnson
%% Last Modified On: Wed Dec 16 15:29:18 2009
%% RCS: $Id$
%%%%%%%%%%%%%%%%%%%%%%%%%%%%%%%%%%%%%%%%%%%%%%%%%%%%%%%%%%%%%%%%%%%%%%%%%%%%%%%
%%   Copyright (C) 2009 
%%%%%%%%%%%%%%%%%%%%%%%%%%%%%%%%%%%%%%%%%%%%%%%%%%%%%%%%%%%%%%%%%%%%%%%%%%%%%%%
%% 

\section{Vision Statement}
\label{sec:vision}

% Enabling active participation in the Smart Grid through Open Power Quality.

% {\em The project description of a Type 1 proposal must begin with a concise statement of the project vision, situating an integrative, collaborative research project within a clearly articulated context of larger sustainability goals. It is understood that the proposed research may focus only on a piece of this vision, but the relationship of the proposed research to the overall vision must be articulated here. The vision statement must also briefly address important human, social, behavioral, economic, and adoption issues that are relevant, whether or not they are directly addressed within the proposed research.}

Development of the ``Smart Grid'', a modernized power infrastructure, is a key sustainability challenge facing the United States. According to the Department of Energy, the Smart Grid should: (1) Enable active participation by consumers by providing choices and incentives to modify electricity purchasing patterns and behavior; (2) Accommodate all generation and storage options, including wind and solar power.  (3) Enable new products, services, and markets through a flexible market providing cost-benefit trade-offs to consumers and market participants; (4) Provide reliable power that is relatively interruption-free; (5) Optimize asset utilization and maximize operational efficiency; (6) Provide the ability to self-heal by anticipating and responding to system disturbances; (7) Resist attacks on physical infrastructure by natural disasters and attacks on cyber-structure by malware and hackers \cite{NETL:GridCharacteristics}.

The very first goal, enable active participation by consumers, involves a fundamental paradigm shift.
Electrical utilities traditionally focus on achieving the opposite goal: enabling {\em passive} 
consumers whose participation is limited to plugging in appliances and paying a monthly bill. The historical success of utilities at reliably providing high quality power at low cost has led to multiple generations of consumers who need to know almost nothing about how their homes and workplaces are powered.

Initial efforts to enable active participation have focused on providing consumers with new forms of data regarding their energy consumption, and have achieved only moderate success.  Feedback on consumption does facilitate one-time positive behaviors, such as installing new insulation or purchasing energy efficient appliances. While helpful, such behaviors do not constitute active participation.    Time-of-use pricing is another approach to active participation, but the typical result is installation of controllers and a ``set it and forget it'' behavior.  Unfortunately, feedback on consumption can even facilitate negative behaviors, such as when consumers install grid-tied solar panels and then consume more electricity because it now appears to be ``free''.  In all of these approaches, consumer behavior is individual and independent of others. 

We propose a different path to active participation by consumers in the Smart Grid.  Based upon our experiences as consumers and energy researchers in Hawaii, we have found that high penetration of distributed, intermittent renewables such as rooftop photovoltaics has the potential for negative impact on power quality and overall grid stability \cite{Rodriguez2010,Laskar2012}, and that this potential negative impact has created actual technical, economic, and social problems in Hawaii.  On a technical level, we have observed power quality events at a weekly frequency in one of our households over the past year. On social and economic levels, our electrical utility has sought rate increases to improve infrastructure and limitations on the penetration of residential PV at the circuit level. Both of these actions have produced significant negative reactions from consumers, the media, and the government. 

One possible solution for Hawaii involves a much deeper level of active participation by consumers, in which we as a community take more direct interest in the state of our grid and more political and economic responsibility for its stability.  This is a tall order: how do we get there?

Over the past year, we have initiated the Open Power Quality (OPQ) research project, which involves designing and implementing a combination of low-cost hardware and cloud-based software that enables consumers to monitor the quality (voltage and frequency) of power in their household and upload that data to our Internet service in order to produce a crowdsourced perspective on grid health.  If effectively deployed, the data could enable consumers to learn for themselves whether their household is experiencing degraded power quality, whether this problem is isolated to their own house or widespread in their community, whether the problem is intermittent or frequent, and whether the problem is unpredictable or regularly occurring. The OPQ approach also has potential to support limited forms of prediction (such as whether certain forms of cloud cover create local instabilities) along with limited forms of diagnosis (such as whether an instability was a result of malfunction within a house or due to interactions between multiple houses).   In contrast to consumption-based approaches with their individually-oriented, one-off results, our quality-based approach has the potential to produce community-level engagement and policy-level activism.  

To test whether these potential benefits of crowdsourced power quality data can be achieved in practice, we propose a two year project to gather evidence regarding the following research questions:

\begin{enumerate}

\item {\em Can crowdsourced power quality data enable active participation in the Smart Grid?}

\item {\em What are the hardware, software, and analytic requirements for crowdsourced data that make it effective for prediction and diagnosis of Smart Grid power quality issues?}

\end{enumerate}

To investigate the first question, we will manufacture and distribute 300 power quality monitoring devices to volunteer households in three Oahu neighborhoods (selected to study low, moderate, and high penetration of distributed renewables). Through pre and post test questionnaires augmented with selected face-to-face interviews, we will assess the extent to which crowdsourced power quality data influenced 
consumer attitudes toward the electrical utility and public policy regarding the grid, as well as behaviors including interaction with neighbors regarding power quality and installation of appliance or whole house battery systems to improve household power quality.

To investigate the second question, we will augment the power quality data collected above with environmental data (temperature, humidity, wind speed and direction, and insolation), household consumption data (as available through an opt-in procedure), and household generation through photovoltaics (again, as available through an opt-in procedure).  We will perform exploratory analytics on the resulting dataset in an attempt to determine the necessary sampling rates, granularity, and history for local power quality, consumption, generation, and environment necessary to support prediction and/or diagnosis of power quality and grid stability. 

To be successful, our approach requires a combination of engineering, human-computer interaction, data analytics, and social science experimentation.  Our research team combines expertise from all of these disciplines. 

To maximize its benefits to and impact upon the scientific community, the OPQ project is ``triple open source'': the hardware schematics are available under the CERN Open Hardware License, the software and firmware are available under the GNU Public License Version 3, and (with consumer consent) the data collected will be available under the Open Data Commons Open Database License.  Our goal is to create an ecosystem of hardware, software, and data that maximizes progress in understanding grid dynamics and consumer involvement. 

As a longer range goal, we intend the OPQ project to provide a basis for research into the application of crowdsourced data to address other sustainability-related resources.  For example, water quality and air quality could be investigated using the same basic paradigm. 

In the remainder of this proposal we will review research related to Open Power Quality, present our project plan, and summarize its intellectual merit and broader impacts.


% Unfortunately, our current fossil fuel-based grid system is unsustainable for a variety of ecological, political, economic, and social reasons, and so a principle goal for the Smart Grid is to integrate distributed, intermittent energy sources such as wind and solar.  Hawaii, due to its high cost of energy, abundant availability of wind and solar, and small population, has become a living laboratory for the social, technical, and political problems that can occur during the transformation to a Smart Grid.

% For example, over the past five years, Hawaii has far exceeded the rest of the nation in the rate of installation of photovoltaics, and now has the highest per capita penetration of PV in the country.  This rapid uptake has been driven by economics: the combination of high electricity rates (roughly triple the mainland) coupled with tax incentives has meant that solar system installation costs can be recovered in 5-7 years. In Hawaii, companies will literally rent your roof to install panels. 
 
% Our recent experience in Hawaii shows that the goal of ``active participation'' is only partially, and sometimes just temporarily, achieved by consumers installing solar panels.  In fact, PV installation can actually work against this goal when consumers decide that the installation of solar panels has ``solved their energy problem'' and thus they no longer need to engage with the energy issue.   Worse, solar installation can increase consumptive behaviors (for example, running air conditioners more frequently or setting thermostats to a more consumptive setting) because PV owners now perceive their electrical energy as ``free''. 

% Hawaii's recent experience yields a great deal of insight into the social, technical, and economic complexities of achieving the various goals for the Smart Grid.  High penetration of a distributed, intermittent, and unpredictable energy source such as PV creates significant new challenges in maintaining grid reliability.  For example, on the island of Oahu, substations currently interpret backflow (i.e. energy flowing the ``wrong way'') as a fault in the grid, and respond by shutting down power distribution. This technical characteristic means that utilities must limit the installation of grid-tied PV below the substation level to an amount whose generation capacity does not exceed the lowest consumption level for the consumers serviced by that substation. Otherwise, the possibility exists that PV generation could exceed consumption, leading to backflow, interpreted as a fault, and subsequent shutdown. 

% This technical issue has had enormous social consequences. 

% Go from here into our hypothesis: that to get active participation, we need to not focus on consumption, which is an individual issue, but rather focus on energy quality, which is a community problem.   That if we can succeed in getting broad consumer engagement with the issue of power quality, then instead of ``withdrawing''  from the grid upon installation of PV, they will be more engaged as they realize that they are now partners with the utilities not only for generation but for reliability as well.  

% So, we will test this through the open power quality project....

% If successful, OPQ will have other benefits: the ability to create low-cost devices that will correlate grid power quality measurements with other factors. 

% (Have Leon Roose read this prior to submission.  Co-PIs?)


