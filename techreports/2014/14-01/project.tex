%%%%%%%%%%%%%%%%%%%%%%%%%%%%%% -*- Mode: Latex -*- %%%%%%%%%%%%%%%%%%%%%%%%%%%%
%% project-techreport.tex -- 
%% Author          : Philip Johnson
%% Created On      : Tue Mar 31 11:44:58 2009
%% Last Modified By: Philip Johnson
%% Last Modified On: Wed Dec 16 15:29:18 2009
%% RCS: $Id$
%%%%%%%%%%%%%%%%%%%%%%%%%%%%%%%%%%%%%%%%%%%%%%%%%%%%%%%%%%%%%%%%%%%%%%%%%%%%%%%
%%   Copyright (C) 2009 
%%%%%%%%%%%%%%%%%%%%%%%%%%%%%%%%%%%%%%%%%%%%%%%%%%%%%%%%%%%%%%%%%%%%%%%%%%%%%%%
%% 

\subsection{Introduction}

Development of the ``Smart Grid'', a modernized power infrastructure, is
one of the key technological challenges facing the United States at the
dawn of the 21st century. According to the Department of Energy, the smart
grid should: (1) Enable active participation by consumers by providing
choices and incentives to modify electricity purchasing patterns and
behavior; (2) Accommodate all generation and storage options, including
wind and solar power.  (3) Enable new products, services, and markets
through a flexible market providing cost-benefit trade-offs to consumers and
market participants; (4) Provide reliable power that is relatively
interruption-free; (5) Optimize asset utilization and maximize operational
efficiency; (6) Provide the ability to self-heal by anticipating and
responding to system disturbances; (7) Resist attacks on physical
infrastructure by natural disasters and attacks on cyber-structure by
malware and hackers \cite{NETL:GridCharacteristics}.

In October, 2009, the government awarded approximately \$3.4 billion in
federal stimulus money to approximately 100 organizations in 49 states to
support Smart Grid development.  While substantial, this investment is
almost totally focused on low-level infrastructure including meters,
communication networks, and phasor measurement units.  By analogy to the
Internet, this investment is similar to upgrading a copper wire network to
fiber optic cable, along with installation of high performance routers and
name servers.  Such infrastructure is a necessary requirement for
high-level Internet services such as the World Wide Web, but does
relatively little to determine the nature of those services.

The Smart Grid will produce and consume information about electricity as
well as electricity itself, thus forming a specialized kind of Internet. At
the current time, it is not clear what ecosystem of higher-level services
should be developed to best communicate, analyze and interpret this
low-level electrical data to consumers.  (By ``consumers'', we mean all of
the various customers of utilities: both individuals and businesses
alike. )

One reason why there is little clarity about consumer-facing information
services in the Smart Grid is the nature of the current grid, which (from a
consumer point of view) is a classic ``black box'' technology.  For almost
100 years, customers have plugged appliances into electrical outlets and
expected them to ``just work''.  Put another way, the U.S. power industry
has operated for a century under the assumption that consumers should have
{\em reliable} access to a virtually {\em unlimited} amount of {\em high
  quality} electricity.  Traditionally, residential consumers have been
given extremely little information about electricity beyond a monthly bill.

This ``Electricity As Black Box'' paradigm requires electricity to be cheap,
reliable, and unlimited, but those days appear to be ending for a number of
reasons.  First, the heavy reliance on fossil fuels for power generation is
not sustainable: most economists agree that ``peak oil'' has either already
occurred or will occur in the next two decades. After that point, oil
supply will decrease and prices will increase.  Second, fossil fuels
generate green house gases that contribute to climate change, and so there
is a need to move to less carbon intensive energy sources such as solar,
wind, wave, and geothermal.  Third, reliance on fossil fuels creates a
variety of political problems.

Could the Electricity As Black Box paradigm be maintained by simply
replacing fossil fuel by renewable energy sources? Unfortunately,
unlike fossil-fuel based power generation, most renewable energy sources
depend upon generally unpredictable environmental
factors.  As a result, simply adding a solar panel to every rooftop and
tying them into the grid would do more harm than good at present.
This is because grid stability requires energy production to equal energy demand
on a second to second basis, and utilities generally do not have a way to
monitor and balance a grid that incorporates high levels of widespread,
variable, and distributed generation. 

Another candidate for preserving the Electricity As Black Box paradigm
is ``demand-response'' technology.  With demand response,
utilities obtain control over major appliances in the home such as the hot
water heater and the heating/cooling system. At peak times, utilities can
shut off the water heater or change thermostat settings in order to reduce
the load on the grid.  While demand response can be effective and is
already deployed in a limited manner, it is not clear that
consumers will universally accept utility-based control of their home
systems.  For example, a recent study of Boulder residents found that
approximately 50\% of those surveyed would not want demand-response
installed in their house \cite{Farhar09}.

Even if electricity could remain a black box to consumers, there is
compelling evidence suggesting that it should not.  According to the July
2009 report ``Unlocking energy efficiency in the U.S. Economy''
\cite{Granade09}, there is the potential to reduce annual
non-transportation energy consumption by roughly 23 percent by 2020,
eliminating more than \$1.2 trillion in waste.  Such a reduction in energy
use would result in abatement of 1.1 gigatons of greenhouse gas emissions
annually. ``Unlocking'' this untapped potential requires, in part, improved
access by consumers to energy information, along with changes in behavior
based on this information that produce the desired efficiencies.

The view that electricity cannot and should not remain a black box to
consumers motivates the central question of this research: just what kind of
``white box'' should it become?  More specifically, {\em What kinds of
  information, provided in what ways and at what times, enables consumers
  to make positive, sustained changes to their energy consumption
  behaviors?}

In this research, we propose to develop technologies and experimental
methods that support more efficient and effective scientific study of the
ways in which access to information about energy usage impacts consumer
behavior.  Our work will build upon current energy user interface successes
and failures; leverage emerging Smart Grid standards; incorporate open
source and component-oriented development techniques; and incorporate findings
from behavioral research. Based upon current research and our prior
experience, we will implement a novel open source framework for energy data
collection, storage, analysis, and presentation called WattBlocks, along
with a novel social networking application for energy behavioral change
called eSpheres.  To evaluate these innovations, we will carry out a series of case studies:
one focusing on university campus dormitory competitions and one focusing
on community residential energy challenges. Our studies will help establish standardized technology
and methodological support that foster replication, meta-analysis of data,
and quicker convergence to scientifically-based understanding of energy
information needs and behaviors in the Smart Grid.

\subsection{Related work}
\label{sec:related-work}

% Our proposed research combines insights from behavioral research on energy
% efficiency along with prior approaches to energy data collection and analysis.

\subsubsection{Energy and behavior}

For utilities that are attempting to move beyond the Electricity As Black
Box paradigm, the traditional approach involves providing financial
incentives to consumers to adopt energy efficient devices and products.
Adoption can occur when a consumer buys something new (like a new house or
light bulb) or when they replace an existing product.  Energy efficiency
improvements come from changing the efficiency of the technology consumers
are using, not from changing their use of that technology.  This approach
has been termed the Physical, Technical, and Economic Model (PTEM)
\cite{Lutzenhiser93}.  PTEM takes an engineering-centric approach to energy
efficiency, assuming that energy use is only affected by new technology,
whose adoption is only affected by price subsidies. As the approach adopted
by most utilities, it is the reason why the vast majority of residential
energy efficiency programs rely on product rebates to incentivize purchases
of energy efficient products and services. PTEM is an attractive model
because it doesn't require consumers to change the way they use energy, and
it doesn't require any effort to ensure that any change in behavior
persists.

Unfortunately, a variety of research indicates that human behavior with
respect to energy cannot be modeled in terms of cost-effectiveness alone,
as is assumed by the PTEM model.  For example, Granade \cite{Granade09}
shows that the PTEM model fails to model US society accurately on a large
scale, as U.S. energy consumption could be reduced by almost 25\% if
consumers actually responded ``rationally'' to the cost-effective energy
saving measures already available to them.

Problems with PTEM can also be demonstrated on a smaller scale.  For
example, Geller \cite{Geller81} performed an experiment in which 40
consumers attended a three hour workshop on energy conservation.  A pre and
post workshop questionnaire determined that all participants gained greater
awareness of energy issues, more appreciation for what could be done in
their homes to reduce energy use and save money, and a willingness to
implement the changes that were advocated in the workshop. However, a one
month followup indicated very little actual change in behavior. One person
lowered the temperature on the hot water heater. Two additional people had
installed insulating blankets around their hot water heaters, but they had
already done this before the workshop. Finally, eight people installed
low-flow shower heads---after all 40 participants had been given the
low-flow shower heads at the workshop.

A final issue with the PTEM model is that it fails to account for consumers who
reduce their energy usage even when they do not gain any financial benefit
from doing so.  For example, an increasingly popular and successful
university activity is the Dormitory Energy Challenge, in which residents
of one or more dormitories compete to reduce their dorm's energy
consumption (even though their housing bill will not change as a result).
Challenges at Brandeis University, Carleton College, Harvard University,
MIT, Mount Holyoke College, Ohio University, and Williams College have
reported energy savings, generally in the range of 7\% to 16\%.  John
Peterson leads what is perhaps the longest running dorm energy challenge
research project at Oberlin College \cite{Peterson07,Peterson07a}.  He is
currently investigating not only ambient devices, but also expanding to
other resources beyond energy, such as water.

Based upon these and other studies, a more nuanced model is beginning to
emerge in which at least the following factors appear to influence energy
behavior.  First, to influence behavior, provide {\em personalized
  information} that reflects the consumer's unique circumstances.  For
example, a dorm resident will not respond well to energy tips involving
improved insulation.  Second, provide both {\em general and specific
  commitments}, especially when they can be tied to a broader issue. For
example, pledging to use a clothesline rather than a dryer because it
reduces green house gas emissions.  Third, provide {\em achievable goals}
that can be objectively measured.  An example might be to reduce energy
consumption by 10\% over the previous month.  Fourth, elicit {\em social
  reinforcement} which can be manifested in both overt and subtle ways.  For
example, in dorm settings, as more and more residents publicly
participate, it implicitly becomes ``the thing to do''.  Fifth, provide
{\em constant and contextual feedback} which helps verify progress toward
goals and can reinvigorate commitments, as long as the feedback is provided
in the right way at the right time.  Sixth, {\em financial incentives} can
be a powerful motivator for energy conservation. 

The first research on behavioral approaches to modifying energy consumption
occurred during the first energy crisis of the late 1970s. Becker found
significant effects from combining energy saving goals and feedback
\cite{Becker78} .  Households committed to goals of either 2 percent or 20
percent energy savings, and then a subset of each group were given feedback
on how well they were doing. The group that received the higher goal and
feedback reduced their actual usage by an average of 15 percent, while the
groups that received either the higher goal or the feedback reduced energy
use by about 5 percent.  Similarly, Houwelingen found that feedback and
goal setting produced a reduction of 12\% in energy use, and that usage
returned to prior levels if the feedback device was removed
\cite{Houwelingen89}.

Vollink and Meertens combined information, feedback, and goal-setting to
produce very significant savings: 23 percent natural gas reductions, 15
percent electricity reductions and 18 percent less water \cite{Vollink99}.
Staats et al. developed a program involving ``Eco-Teams'' of 6-10 people
who met once a month to discuss energy conservation and related
topics. These teams achieved sustained energy reductions averaging 7.5\% over
three years by combining information, social reinforcement, and feedback
\cite{Staats04}.

Faruqui et al \cite{Faruqui09} performed a review of over a dozen pilot projects
performed by utilities involving in-home displays which provide near-real
time feedback.  The authors found evidence that the introduction of this
single motivational factor (feedback) did appear to influence behavior:
consumers self-reported average reductions of around 7\%.   When paired with an
additional motivator (cost savings), average reductions increased to around
15\%.  However, these studies do not provide much evidence concerning the
sustainability of the changes; it could be that consumers will eventually
acclimate to the real-time display and cease conservation behaviors. 

As noted by Darby \cite{Darby06}, feedback is a necessary but
not always sufficient condition for savings and awareness among
consumers. Darby maintains that  the condition of housing, personal contact
with a trustworthy adviser when needed, and the availability of technology, 
training, and social infrastructure for learning may also be required 
in order for long-term change to occur.  

Recent behavioral research has made great strides in moving beyond the
traditional PTEM model and has started to reveal important factors in
changing consumer energy behavior. But the connection to the Smart Grid is
still tenuous, the most appropriate motivational factors to apply to
particular circumstances is still unknown, and the effective use of modern
information technology such as social networks is just beginning
\cite{StepGreen}.  The next section focuses on information technology.

\subsubsection{Energy data interfaces}

Utilities, of course, can have a primary role in consumer-facing energy
data in the Smart Grid. Unfortunately, many of the current ``smart meters''
(such as those designed by Enel and Telvent) do not even provide a
consumer-facing interface; instead, they simply report on electrical usage
directly to the utility.  This simplifies billing operations for the
utility, but provides no direct feedback to the consumer.

Whether using a smart meter with a consumer-facing interface, or a home
energy monitoring system, the most common form of interface to energy data
provides a simple line chart or histogram of energy usage over various time
intervals: minutes, hours, days, or weeks.  Examples include home energy
monitoring systems such as the TED 5000; Google's PowerMeter application,
and building information management systems such as those by Obvius, Lucid
Design Group, Small Energy Group, and AgileWaves. In some cases, these
systems can display energy usage in alternative units, such as carbon
emitted or cost.  Figure \ref{fig:LucidDesignGroup} shows a representative
interface from Lucid Design Group.



At the utility-level, there are attempts to show grid-level information
through sites such as Ecotricity \cite{Ecotricity} or CurrentEnergy
\cite{CurrentEnergy}.  While most adhere to the standard line chart style
of visualization, the Ecotricity site also provides an interpretation of
the data with a  traffic light that shows red, yellow, or green depending upon
the amount of carbon being produced by the grid at the current time.  The
goal of the stop light is to provide consumers with the information
necessary to enable them to shift energy-intensive tasks (such as running
their dryer) away from times when carbon intensity is high.

An emerging approach to energy interfaces is ``Eco-Visualization'',
defined by Pierce as ``Any kind of interactive device targeted at
revealing energy use in order to promote more sustainable behaviors or
foster positive attitudes towards sustainable practices \cite{Pierce08}.''
The seminal eco-visualization is called ``7000 oaks and counting'' by
Tiffany Holmes \cite{Holmes07}.  In the visualization, rings of spinning
oak trees represent carbon loads for a given environment from a single
building to a city.  If the loads are low, the rings are composed of green
trees. As loads increase, trees are removed and replaced by appliances such
as light bulbs or refrigerators.  Another eco-visualization involves a
virtual polar bear floating on an ice floe. The ice floe grows when users
commit to environmental actions and decreases when users choose not to
commit to change \cite{Dillahunt08}.

There is also movement toward interfaces apart from traditional websites.
For example, Lucid Design Group and Oberlin College are planning to
experiment with the use of ambient orbs to indicate energy consumption
\cite{Peterson09}.  The StepGreen system attempts to leverage social
networking sites like FaceBook and MySpace to enable sustainable behavior
\cite{Mankoff07}.

Not all interfaces focus exclusively on energy data.  For example, the
Energy Pledge prototype was designed for public display of personal
commitments \cite{Pierce09}.  The StepGreen system also provides a display
of commitments \cite{StepGreen}.  

While these various interface directions are promising and many have
yielded positive initial results, they do not address some issues and leave
other questions unanswered. First, it is often not clear whether or not and
how a given technology could be re-used in a new context.  Many of the
systems are either stand-alone (such as StepGreen) or appear highly
customized to a specific context (Oberlin's Campus Resource
System). Second, there tends to be a single focus to the interface: either
very specific (one's dorm), or very general (the overall grid).  

\subsubsection{Energy data design issues}

The growing popularity of energy data related information systems is
heartening, but also demonstrates that much more can be done. 

First, the current state of consumer-facing energy information technology
is reminiscent of web site technology prior to the development of content
management systems such as Plone or Drupal, where lack of infrastructure
meant that web site developers were repeatedly re-implementing the same
basic functionality for navigation, page templates, file uploads, and so
forth, or paying a commercial provider to do it for them.  There are no
open source energy information technologies equivalent to Plone that
enables, for example, a university to quickly build a new dorm energy
challenge system from generic components.  No current system has an
architecture that facilitates ``mash-ups'' or ``plug-ins'', such that a
developer could quickly build an interface to a new technology such as
Google Wave, or integrate a new approach to commitments. This slows the
pace of innovation and progress toward understanding how to best improve
energy behaviors.

Second, energy information technologies are not designed to be amenable to
scientific research.  To gauge the effectiveness of an information
capability, it is helpful to gain a detailed understanding of how it is
accessed, by whom, and under what circumstances, and how that connects
to outcomes.  Such instrumentation support appears to be largely lacking
from current approaches.

In addition to instrumentation, it is often scientifically useful to
support treatments: in other words, providing different forms of
information to different populations of users in order to gain insight into
the effect of a treatment upon an outcome of interest. For example, one way
to assess the effect of commitments on energy conservation would be to
provide a ``Pledge Wall'' to only half the floors in a dorm, then use a
combination of qualitative (post-study interviews) and quantitative (actual
changes in energy) to assess the impact of that behavioral
treatment. Current systems take an ``all or nothing'' approach to
features. 

Finally, no current information system spans energy information from the ``micro''
(i.e. single building) to the ``macro'' (i.e. utility-level grid).  For
example, current in-home energy meters provide no insight into the carbon
intensity of the grid currently supplying the power to the home appliances.
Grid-level information systems like Ecotricity or CurrentEnergy do not
allow drill down to individual cities or homes.  As a result, interesting
opportunities to influence energy behavior are lost, such as a commitment
like ``I will use my dorm's dryer only when the carbon intensity of the
grid is low.''

\subsubsection{Results from prior NSF research}

\small
\begin{tabular}{p{1in}p{5in}}
Award: & CCF02-34568 \\ 
Program: & Highly Dependable Computing and Communication Systems Research\\ 
Amount: & \$638,000 \\ 
Period: & September 2002 to September 2007 \\ 
Title: & Supporting development of highly dependable software through
continuous, automated, in-process, and individualized software measurement validation \\ 
PI: & Philip M. Johnson \\ 
Selected Pubs: & \cite{csdl2-04-22,csdl2-04-13,csdl2-04-11,csdl2-03-12,
csdl2-02-07,csdl2-03-07,csdl2-04-02,csdl2-04-04,csdl2-04-11,csdl2-06-07,csdl2-06-08,csdl2-06-13,csdl2-06-06,csdl2-09-01}
\end{tabular} \\ %[3mm]
\normalsize

The objective of this research was to design, implement, and validate
software measures within a development infrastructure that supports the
development of highly dependable software systems.  Contributions include:
(a) development of Hackystat configuration to gather build and workflow
data from the configuration management system for the Mission Data System
(MDS) project at Jet Propulsion Laboratory; (b) development of MDS build
and workflow analyses to identify potential process bottlenecks; (c)
identification of previous unknown variation within the MDS development
process; (d) development of a generalized approach to in-process,
continuous measurement validation called Software Project Telemetry, (e)
development of undergraduate and graduate software engineering curriculum
using Hackystat; (f) support for 3 Ph.D., 6 M.S., and 3 B.S. degree
students.


\subsection{Intellectual Merit and Broader Impact}
\label{sec:merit}

A report to the U.S. Department of Energy in September 2009 on the
Principle Characteristics of the Smart Grid echoes our research orientation
when it states: {\em ``Achieving consumer participation means
  making participation easy and understandable.  And essential to this will
  be providing a user interface that successfully motivates and supports
  consumer action. [...] Today's communications and electronic technologies
  create options that were just not viable in the past.''}
\cite{NETL:EnablesActiveParticipation}

We believe our research proposal defines an ambitious, aggressive, yet
feasible approach to obtaining significant insight into an important
societal question: What kinds of information, provided in what ways and at
what times, enables consumers to make positive, sustained changes to their
energy consumption behaviors?  We will gain new insights into this question
in a number of ways.  First, we will develop open source, component-based
infrastructure called WattBlocks that will lower the technological barriers
to scientific research on energy and human behavior.  Second, we will
develop a novel social networking technology called eSpheres that supports
a variety of motivating factors for behavioral change.  Third, we will
conduct six case studies, in both dormitory and residential settings, that
will provide new forms of empirical data on energy behaviors and how they
change (or don't change) given various motivating factors.

To achieve these goals, we will build upon our prior experiences with open
source software development and empirical software engineering research. We
will also build upon established relationships with University of Hawaii
organizations (REIS, Sustainable UH) and community organizations (Kanu
Hawaii, Blue Planet Foundation, Hawaiian Electric Company).  Hawaii
is an EPSCOR state and approximately 84\% of University of Hawaii
undergraduates are minorities, so this research will benefit
under-represented populations. 

This research creates a replicable mechanism for assessing and improving
energy knowledge, attitudes, and behaviors of university students and
residential community members. We expect the outcome will not only be
immediate energy conservation with all of the attending positive benefits,
but also an empowering of these individuals who have gained first-hand
knowledge that their actions can make a difference, and the ways that they
can engage with their communities and at many different scales of
interaction.  It will create publicly available, open source technology
useful not only for research but for education as well.  For example, we
have used WattDepot in software engineering class projects in Fall of 2009.

Our research provides an evidence-driven approach to understanding what
kinds of information should be provided by the Smart Grid to its users in
order to effect sustained, positive changes in behavior.  For the purposes
of this research, energy data will be gathered through installation of
third-party metering systems and analyzed through WattBlocks services. In
the long run, we hope the insights from this research will migrate into the Smart
Grid infrastructure itself, using standard utility meters to collect data
which is then distributed to information services (either inside or outside
of the utility industry) for analysis, interpretation, and presentation.
