%%%%%%%%%%%%%%%%%%%%%%%%%%%%%% -*- Mode: Latex -*- %%%%%%%%%%%%%%%%%%%%%%%%%%%%
%% supplemental.tex -- 
%% Author          : Philip Johnson
%% Created On      : Tue Mar 31 11:42:10 2009
%% Last Modified By: Philip Johnson
%% Last Modified On: Thu Dec 10 09:09:23 2009
%% RCS: $Id$
%%%%%%%%%%%%%%%%%%%%%%%%%%%%%%%%%%%%%%%%%%%%%%%%%%%%%%%%%%%%%%%%%%%%%%%%%%%%%%%
%%   Copyright (C) 2009 
%%%%%%%%%%%%%%%%%%%%%%%%%%%%%%%%%%%%%%%%%%%%%%%%%%%%%%%%%%%%%%%%%%%%%%%%%%%%%%%
%% 

\section*{Data Management Plan}

\subsubsection*{Types of data}

As part of this project, many types of data will be collected. This includes environmental resource data gathered from sensors including temperature, humidity, wind direction and speed, and solar irradiance. It also includes electrical grid data including voltage, current, power, and reactance. It also includes power and energy data from distributed generation devices such as photovoltaic systems.

All of the above data will be experimental measurement data obtained through deployed weather boxes. The data will be captured using standard environmental sensors such as anemometers, power meters,  and pyrometers that are housed in the weather box. 

We will also collect data from experiments conducted on the RTDS system.  This will be from experiments conducted on components and subsystems of the electrical grid and also includes distributed generation and loads.

Data will also be collected from the electrical grid using existing power meters and also deployed devices such as AMI to get time stamped data at different points on feeder lines and buses of the University of Hawaii microgrid.

Finally data will be collected from buildings where we will monitoring energy usage in different parts of the building to see time and seasonal patterns in commercial, industrial, and residential buildings. 
  
These different data types will be transmitted through wireless and wired networks to a data server and storage system that is located in the Smart Campus Energy Lab (SCEL).  

\subsubsection*{Data and metadata standards}

While there are many systems that have been developed to store environmental sensor data and electrical grid data, such as Geo- CENS, Pachube, and the Berkeley Sensor Database, there are no commonly recognized standards for the formatting, storage, or transmission of the data we will be collecting, storing, and analyzing. Because of this, we plan to store our data using a custom, but publicly documented set of database schemas in an internet-accessible server running a standard open source such as LAMP (Linux, Apache, MySQL, Python) stack. 

\subsubsection*{Policies for access and sharing and provisions for appropriate protection and privacy}

As noted above, we plan to design and implement a web service that will enable external access to the data by interested researchers (subject to IRB approval in the case of survey data). The environmental data that we will collect, store, analyze, and publicize will not reveal personal characteristics of users. We will restrict real-time access to power and/or energy consumption data, or aggregate this data as required, in order to prevent users from being able to gain behavioral insight from patterns of energy usage.  
There is also discussion of getting access to facilities and data in the main project description under the management plan.  Access to the Smart Campus Energy Lab (SCEL) facilities and data will be coordinated with the Principal Investigators and the IT administrator who will be responsible for operation and maintenance of the RTDS system.

\subsubsection*{Policies and provisions for re-use and re-distribution}

We plan to make the data collected in this research freely available under a Creative Commons Attribution CC BY license. This license will let others use, distribute, and analyze the data we collect without restriction as long as they credit us as the original creators of the data.

We believe that the data we collect will be of interest to others developing ``smart microgrid'' systems as it will provide, at a minimum, baseline data for environmental conditions we experienced during the course of our research.

\subsubsection*{Plans for presentation, archiving, and preservation of data}

During the course of this research, we will be storing the data in a web accessible database as discussed above. Archiving the data past the conclusion of the study will be done via data archiving services provided by the University of Hawaii. All research reports and collateral documents created in response to the data will also be permanently archived through the University of Hawaii technical report services.   Research results will also be presented through journal publications, presentations at conferences, and seminars to students, faculty, and industry collaborators at the University of Hawaii.

The University of Hawaii implements standard best practices for data storage backup and retrieval, including off-site storage, redundant power supplies, RAID disk storage, and so forth.  Sites for storage will be at the UHM Smart Campus Energy Lab (SCEL)  in 493 Holmes Hall which has a data server and storage system.

\subsubsection*{Dissemination of research and educational work}

Research and educational results will be disseminated through journals, dissertations, conferences (oral and poster presentations), and website demonstrations.   Our Renewable energy and Island Sustainability  (REIS) group website will also contain information about courses, and technical reports (which will be turned into journal and conference publications).  In conjunction with the REIS group we will also give short energy and smart grid courses to the community.  These courses will be funded through REIS, the city of Honolulu, state of Hawaii and UHM and will provide another method where knowledge and material can be disseminated to the community.  A wind energy course was offered in Spring, 2011 and courses in smart grids and solar thermal energy were offered in summer, 2012.  




