%%%%%%%%%%%%%%%%%%%%%%%%%%%%%% -*- Mode: Latex -*- %%%%%%%%%%%%%%%%%%%%%%%%%%%%
%% project-background.tex -- 
%% Author          : Philip Johnson
%% Created On      : Tue Mar 31 11:44:58 2009
%% Last Modified By: Philip Johnson
%% Last Modified On: Wed Dec 16 15:29:18 2009
%% RCS: $Id$
%%%%%%%%%%%%%%%%%%%%%%%%%%%%%%%%%%%%%%%%%%%%%%%%%%%%%%%%%%%%%%%%%%%%%%%%%%%%%%%
%%   Copyright (C) 2009 
%%%%%%%%%%%%%%%%%%%%%%%%%%%%%%%%%%%%%%%%%%%%%%%%%%%%%%%%%%%%%%%%%%%%%%%%%%%%%%%
%% 

\section{Results from prior NSF support}

% {\em If any PI or co-PI identified on the project has received NSF funding (including any current funding) in the past five years, information on the award(s) is required, irrespective of whether the support was directly related to the proposal or not. In cases where the PI or co-PI has received more than one award (excluding amendments), they need only report on the one award most closely related to the proposal. Funding includes not just salary support, but any funding awarded by NSF. The following information must be provided:

% \begin{itemize}

% \item the NSF award number, amount and period of support;

% \item the title of the project;

% \item a summary of the results of the completed work, including accomplishments, supported by the award. The results must be separately described under two distinct headings, Intellectual Merit and Broader Impacts;

% \item the publications resulting from the NSF award;

% \item evidence of research products and their availability, including, but not limited to: data, publications, samples, physical collections, software, and models, as described in any Data Management Plan; and

% \item if the proposal is for renewed support, a description of the relation of the completed work to the proposed work.

% \end{itemize}

% Reviewers will be asked to comment on the quality of the prior work described in this section of the proposal. Please note that the proposal may contain up to five pages to describe the results. Results may be summarized in fewer than five pages, which would give the balance of the 15 pages for the Project Description.

% }


P. Johnson, {\em Human centered information integration for the Smart Grid}, NSF Grant IIS-1017126, 8/15/10 to 7/31/14, \$413,467. The objective of this research is to design information technology and associated experimental methods to help understand what information, provided in what ways and at what times, enables consumers to make positive, sustained changes to their energy consumption behaviors. 

{\em Intellectual Merit:} The research provided novel insight into: the inadequacy of baseline data for energy competition research, the design of experimental studies for assessing energy behaviors, the design of energy competitions incorporating educational activities. 

{\em Broader Impacts:} The broader impacts of this research include: the creation and distribution of two open source systems, WattDepot and Makahiki, that can be used for collection and analysis of energy data and the design and implementation of sustainability games; the publication of data regarding the impact of energy
education and gamification techniques on energy literacy and behavior; the training of approximately 9 undergraduate students, 3 M.S. students, and 3 Ph.D. students in research techniques, sustainability concepts, and software design and development.

Selected publications include:
  \cite{csdl2-10-05,csdl2-10-07,csdl2-10-08,csdl2-11-02,csdl2-11-03,csdl2-12-06,csdl2-11-07, csdl2-12-12,csdl2-13-10,csdl2-13-05,csdl2-13-03}.

Research products are available at the Kukui Cup site \cite{kukuicup}, the WattDepot site \cite{wattdepot}, and the Makahiki site \cite{makahiki}.
  
% \vspace{-.1in}

% \item A. Kuh, {\em Incremental and Distributed Learning in Nonstationary
%     Environments with Applications to Wind Forecasting}, NSF Grant ECCS-098344,
%   9/01/09 - 8/31/13, \$150,251.  The objective of this research is to
%   design novel nonlinear kernel online and distributed learning algorithms
%   for applications including wind forecasting.   Research was also conducted
%   to model the microgrid using a factor-graph framework. 
%   Selected publications for
%   this project include \cite{kuh-etal10,kowahl-kuh,hu-etal10,hu-kuh-yang-kavcic,hu-kuh-kavcic-nakafuji,ji-wei-kuh,uddin12,kuh-isess,carland,navid}.
  
% %\vspace{-.1in}

% \item A. Kuh, {\em US-Japan Joint Seminar Information Theory}, NSF Grant 0508025
%   \$35,750.  Funds used to support graduate students for conferences and
%   for visit to Japan.


%\vspace{-.1in}



