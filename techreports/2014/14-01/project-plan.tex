%%%%%%%%%%%%%%%%%%%%%%%%%%%%%% -*- Mode: Latex -*- %%%%%%%%%%%%%%%%%%%%%%%%%%%%
%% project-plan.tex -- 
%% Author          : Philip Johnson
%% Created On      : Tue Mar 31 11:44:58 2009
%% Last Modified By: Philip Johnson
%% Last Modified On: Wed Dec 16 15:29:18 2009
%% RCS: $Id$
%%%%%%%%%%%%%%%%%%%%%%%%%%%%%%%%%%%%%%%%%%%%%%%%%%%%%%%%%%%%%%%%%%%%%%%%%%%%%%%
%%   Copyright (C) 2009 
%%%%%%%%%%%%%%%%%%%%%%%%%%%%%%%%%%%%%%%%%%%%%%%%%%%%%%%%%%%%%%%%%%%%%%%%%%%%%%%
%% 

\section{Research plan}

{\em The proposed research must approach fundamental research from an interdisciplinary perspective that integrates computing and communications with domain sciences and engineering to address the sustainability issues of interest. The proposal must describe a synergistic approach by which the team addresses scientific challenges in sustainability.}

\subsection{Intellectual Merit and Broader Impact}
\label{sec:merit}

A report to the U.S. Department of Energy in September 2009 on the
Principle Characteristics of the Smart Grid echoes our research orientation
when it states: {\em ``Achieving consumer participation means
  making participation easy and understandable.  And essential to this will
  be providing a user interface that successfully motivates and supports
  consumer action. [...] Today's communications and electronic technologies
  create options that were just not viable in the past.''}
\cite{NETL:EnablesActiveParticipation}

We believe our research proposal defines an ambitious, aggressive, yet
feasible approach to obtaining significant insight into an important
societal question: What kinds of information, provided in what ways and at
what times, enables consumers to make positive, sustained changes to their
energy consumption behaviors?  We will gain new insights into this question
in a number of ways.  First, we will develop open source, component-based
infrastructure called WattBlocks that will lower the technological barriers
to scientific research on energy and human behavior.  Second, we will
develop a novel social networking technology called eSpheres that supports
a variety of motivating factors for behavioral change.  Third, we will
conduct six case studies, in both dormitory and residential settings, that
will provide new forms of empirical data on energy behaviors and how they
change (or don't change) given various motivating factors.

To achieve these goals, we will build upon our prior experiences with open
source software development and empirical software engineering research. We
will also build upon established relationships with University of Hawaii
organizations (REIS, Sustainable UH) and community organizations (Kanu
Hawaii, Blue Planet Foundation, Hawaiian Electric Company).  Hawaii
is an EPSCOR state and approximately 84\% of University of Hawaii
undergraduates are minorities, so this research will benefit
under-represented populations. 

This research creates a replicable mechanism for assessing and improving
energy knowledge, attitudes, and behaviors of university students and
residential community members. We expect the outcome will not only be
immediate energy conservation with all of the attending positive benefits,
but also an empowering of these individuals who have gained first-hand
knowledge that their actions can make a difference, and the ways that they
can engage with their communities and at many different scales of
interaction.  It will create publicly available, open source technology
useful not only for research but for education as well.  For example, we
have used WattDepot in software engineering class projects in Fall of 2009.

Our research provides an evidence-driven approach to understanding what
kinds of information should be provided by the Smart Grid to its users in
order to effect sustained, positive changes in behavior.  For the purposes
of this research, energy data will be gathered through installation of
third-party metering systems and analyzed through WattBlocks services. In
the long run, we hope the insights from this research will migrate into the Smart
Grid infrastructure itself, using standard utility meters to collect data
which is then distributed to information services (either inside or outside
of the utility industry) for analysis, interpretation, and presentation.

