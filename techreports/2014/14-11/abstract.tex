%%%%%%%%%%%%%%%%%%%%%%%%%%%%%% -*- Mode: Latex -*- %%%%%%%%%%%%%%%%%%%%%%%%%%%%
%% uhtest-abstract.tex -- 
%% Author          : Robert Brewer
%% Created On      : Fri Oct  2 16:30:18 1998
%% Last Modified By: Robert Brewer
%% Last Modified On: Fri Oct  2 16:30:25 1998
%% RCS: $Id: uhtest-abstract.tex,v 1.1 1998/10/06 02:06:30 rbrewer Exp $
%%%%%%%%%%%%%%%%%%%%%%%%%%%%%%%%%%%%%%%%%%%%%%%%%%%%%%%%%%%%%%%%%%%%%%%%%%%%%%%
%%   Copyright (C) 1998 Robert Brewer
%%%%%%%%%%%%%%%%%%%%%%%%%%%%%%%%%%%%%%%%%%%%%%%%%%%%%%%%%%%%%%%%%%%%%%%%%%%%%%%
%% 

\begin{abstract}
The face of power distribution has changed rapidly over the last several decades.
Modern grids are evolving to accommodate distributed power generation, and  highly variable loads.
Furthermore as the devices we use every day become more electronically complex,
they become increasingly more sensitive to power quality problems. Distributed 
power quality monitoring systems have been shown to provide real-time insight on the
status of the power grid and even pinpoint the origin of power disturbances. \cite{fnet_defence}
Oahu's isolated power grid combined with high penetration of distributed renewable energy generators create perfect
conditions to assess the feasibility and utility of such a network. Over the last three months we have been collecting power
quality data from several locations on Oahu as a pilot study for a larger monitoring system.
This papers describes our methodology, hardware and software design and presents a preliminary analysis
of the data we collected so far. Lastly this paper presents a design for an improved power quality monitor
based upon the pilot study experiences.
\end{abstract}
