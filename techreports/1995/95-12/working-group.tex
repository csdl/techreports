%%%%%%%%%%%%%%%%%%%%%%%%%%%%%% -*- Mode: Latex -*- %%%%%%%%%%%%%%%%%%%%%%%%%%%%
%% working-group.tex -- 
%% Author          : Carleton Moore
%% Created On      : Thu Apr 13 09:21:45 1995
%% Last Modified By: Carleton Moore
%% Last Modified On: Mon May 15 12:12:46 1995
%% Status          : Unknown
%% RCS: $Id: working-group.tex,v 1.1 1995/05/15 22:03:58 cmoore Exp cmoore $
%%%%%%%%%%%%%%%%%%%%%%%%%%%%%%%%%%%%%%%%%%%%%%%%%%%%%%%%%%%%%%%%%%%%%%%%%%%%%%%
%%   Copyright (C) 1995 University of Hawaii
%%%%%%%%%%%%%%%%%%%%%%%%%%%%%%%%%%%%%%%%%%%%%%%%%%%%%%%%%%%%%%%%%%%%%%%%%%%%%%%
%% 


\documentstyle [nftimes,twocolumn,/group/csdl/tex/ieee,
               /group/csdl/tex/lmacros]{article}
\pagestyle{empty}
\input{/group/csdl/tex/psfig/psfig}
\begin{document}

\makeieeetitle
{WET ICE Tools Working Group Report}
{Carleton Moore\\
Department of Information and Computer Sciences\\ 
University of Hawaii\\
Honolulu, Hawaii 96822\\ 
{\tt cmoore@uhics.ics.hawaii.edu}}


\section{Introduction}

The tools working group report is organized as follows.  We'll discuss:
\begin{enumerate} 
\item Three scenarios that represent different situations for collaboration,
\item A framework that can be used to describe collaborative tools and,
\item Some concerns, issues, and holes we found.
\end{enumerate}

The members of the tools working group were:\footnote{Any omissions or
errors are the author's.} K. Joseph Cleetus, Jack Hong, Giuseppe Iazeolla,
Uwe Jasnoch, Vinay Kumar, Kichie Matsuzaki, Elliot A. Shefrin, Sri
Sridharan, and Sankar Virdhagriswaran.


\section{Scenarios}

The tools working group came up with three scenarios that represent a wide
range of situation in which collaborative tools are needed.  

\subsection{Scenario 1}

The first scenario is in the domain of Software Product Engineering.  In
the conventional waterfall life cycle model, software goes through several stages:
Requirements, Design, Implementation, and Testing.  At each stage of the
process, tools are used to aid the developer in creating an artifact that
represents the outcome of the phase.  The final product is produced 
as a result of testing. 

There are several drawbacks to this method.  First, there are
few interactions between the stages, which can result in low quality.  Since
there are many iterations, this method can lead to schedule overruns.  We
suggest a new life cycle model.  In this model there is only one artifact,
the product master model.  This artifact has many different views depending
on the phase of software development.  One of the advantages of using this
model are that there are many interactions with the artifact leading to better
quality.  Another advantage to this model is that there are very few iterations
which can lead to on time delivery of the software product.  Having only
one artifact forces raises new requirements for tools:

\begin{itemize}
\item {\em Composibility of tools.}  Each of the tools that interact with
  this artifact must work together and be able to support each other.
  The tools should be thought of as object that can be used to build
  new tools.
\item {\em Model decomposability.}  The common model representing the project
  must be decomposable into its constituent parts.
\item {\em Model sharing.}  Different tools will have to share the same
  model without corrupting that model.
\item {\em Multifaceted.}  The model must support multiple views.  Each
  phase of software development should see a different view of the product
  master model.
\item {\em Model abstraction.}  The product master model and the tools
  that interact with it must support multiple levels of abstraction.
\end{itemize}

This new life cycle model requires a collaborative toolkit that must be
able to support distributed users across the planet.

\subsection{Scenario 2}

The second scenario we came up with is a telecommunications planning
system.  The State of Hawaii is coming up with their strategic
telecommunications infrastructure plan.  They want public, private and
government input into this plan.  The process of coming up with this plan
is iterative.  The public will provide input and review to the current
state of the plan through several different input methods.  The public will
be able to make comments and suggestions through email, World Wide Web
forms or by hard copy.  We expect that there will be thousands of people
reviewing and commenting on the plan.  The system will collect these
comments and suggestions and update the telecommunications plan.  The
government policy makers will be able to update the plan and help the
discussion of issues raised.  As the plan is updated new releases will be
presented to the public for further comment and review.  This plan will be
evolving as the situation changes.

\subsection{Scenario 3}

The third scenario is a factory equipment planning situation.
Approximately 150 factory planners develop their factory needs.  These
needs are organized by factory, equipment type and time-line.
Approximately 20 capital acquisition agents contact the suppliers of
factory equipment and based upon the needs of the factories purchase new
equipment.  The factory planners and capital acquisitions personnel use an
Excel spreadsheet, connected through a WAN to a relational database that
maintains all of the factory needs and capital acquired.  Since the users
are spread out across the world, new users are connecting to the network
and logging off.  When the users are connected they can use audio and video
to communicate and coordinate in real time.  When users are disconnected
they receive updates to inform them what is happening to the database.
These updates allow the users to keep track of the changes so that when
they connect they understand how the database changed.  A requirement of
the system is that users must be able to detach and link back up with the
system easily.

\section{Framework}

Based upon the scenarios we discussed we came up with a generic framework
for collaborative tools.  This framework is based upon three pillars,
presentation, communication, persistence.  All collaborative tools must
have these three pillars.  We have come up with features for each pillar.
These features should be supported by good collaborative tools.
\begin{itemize}
\item{\em Presentation:}  The presentation pillar supports the provision of 
  information to the user.  It is the interface portion of the tool.
  Some characteristics of collaborative presentation are:
  \begin{itemize}
  \item Presentation Sharing: the ability for tools to coordinate the
    presentations to the users.
  \item Event synchronization: synchronizing the events from the users.
  \item Localization: presenting the information in a format that is best
    for the user.
  \item User Profiles: determining who the user is and customizing the software to
    the profile.
  \item Levels of Abstraction: the ability to represent information
    in different ways.
  \item Notification: update the user when things change.
  \item Views: the ability to view information in different ways.
  \item Information Capture: collecting information from the user/environment.
  \item Rich Dense Information: provide a rich, dense, and understandable
    information.
  \end{itemize}

\item{\em Communication:}  The communications pillar supports users/tools
  exchanging information. Some important characteristics of communication are:
  \begin{itemize}
  \item Media Negotiation: determining the type of media based upon the
    quality and capability of the communications media.
  \item Planet Earth: communicate across the entire planet.
  \item Protocols: WWW/HTTP: deal with the interconnectability issues.
  \item Data synchronization: synchronizing different types of data so that
    they are presented correctly.
  \item Linking and Detaching: support the issues of connecting to
    the network and disconnecting.
  \item Notification: communicating changes to the different users/tools.
  \item User Profiles: determine what communications should be provided.
  \end{itemize}

\item{\em Persistence:} The persistence pillar supports data storage, with
the following important characteristics:
  \begin{itemize}
  \item Planet Earth: the requirement to support data storage across the
    globe leads to the following issues.
    \begin{itemize}
    \item Checkouts - handling check outs of data across different networks.
    \item transactions - rollback of transactions across distributed data.
    \item locking - guaranteeing locking across distributed data.
    \end{itemize}
  \item Caching: caching data in the best manner.
  \item Localization: storing the data in local format.
  \item Data Capture: capturing the correct data.
  \item Notification: updating the different storage locations.
  \item Version Control: maintaining good version control over distributed data.
  \item Views: maintaining different views in the data.
  \item User Profiles: maintaining user profiles.
  \item Concurrency Control: maintaining correct data access.
  \item Information Sharing: ensuring access do common data.
  \end{itemize}
\end{itemize}

Resting upon the three pillars is the coordination level.  This level
supports the coordination of the three pillars.  At this level are the
information brokers and moderators that control the flow of information
from the persistence pillar through the communications pillar and to the
presentation pillar.  Other issues at this level are workflow, user
synchronization, and enterprise API.

Above the coordination level is the application level.  This level is where
the cooperation and problem solving takes place.  This level is where the
tools are put together to solve problems.

\section{Concerns, Issues, Holes}

We came up with the following issues.
\begin{itemize}
\item Our group did not have enough ``vision''.  Above analysis is very
  close to the state of the art.
\item The state of the art in tools is close to the state of practice.
\item There is a need for cross pollination with DAI (distributed AI), DB
(relational, object and distributed databases), and CAD (Design tools that
support group work).
\item The semantics of collaboration, dialog, etc. were not presented at
  this workshop.
\item There are social consequences and cultural issues that should be
  looked into.
\item We did not see any new process models.
\item Composibilty and performance tunability are requirement on the
  environment.
\item Tools differ by environment.
\end{itemize}
\end{document}

