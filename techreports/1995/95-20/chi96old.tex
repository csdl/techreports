%%%%%%%%%%%%%%%%%%%%%%%%%%%%% -*- Mode: Latex -*- %%%%%%%%%%%%%%%%%%%%%%%%%%%%
%% chi96old.tex -- 
%% Author          : Rosemary Andrada
%% Created On      : Sat Oct 14 15:02:39 1995
%% Last Modified By: Rosemary Andrada
%% Last Modified On: Thu Nov 30 07:49:07 1995
%% RCS: $Id$
%%%%%%%%%%%%%%%%%%%%%%%%%%%%%%%%%%%%%%%%%%%%%%%%%%%%%%%%%%%%%%%%%%%%%%%%%%%%%%%
%%   Copyright (C) 1995 Rosemary Andrada
%%%%%%%%%%%%%%%%%%%%%%%%%%%%%%%%%%%%%%%%%%%%%%%%%%%%%%%%%%%%%%%%%%%%%%%%%%%%%%%
%% 

\documentstyle[nftimes,/group/csdl/tex/CHI95]{article}

\begin{document}

\title{The Effect of a Virtual WWW Community on its Physical Counterpart}

\author{
Rosemary Andrada\\
Collaborative Software Development Laboratory\\
2565 The Mall\\
Honolulu, Hawaii 96822\\
(808) 845-9291\\
rosea@hcc.Hawaii.Edu}

\maketitle

\abstract 

This paper overviews the results from a study that assessed the strengths and
weaknesses of a computer-based approach to improving the sense of community
within one organization, the Department of Information and Computer Sciences at
the University of Hawaii.  The case study used a pretest-posttest design.
First, several measures of the sense of community within the department were
obtained via a questionnaire.  Second, a World Wide Web information system was
introduced in an effort to affect the level of community within the department.
Third, a similar questionnaire was administered after a period of four months.
Analysis of the survey responses and system logs showed that the information
system designed to promote community had instead polarized some of its members.
In addition, the system served as a valuable diagnostic tool for discovering
what factors may help promote or inhibit community building.

%I don't think *everyone* was polarized; I think many people felt it improved
%the sense of community.  However, the polarization was the surprising result,
%which is one part of why this research was so interesting.  You need to focus
%on that, without making it seem like that was the only thing that happened.

\paragraph{KEYWORDS:} WWW, community

\section{INTRODUCTION}
%Here You Want To Make Sure You Get To The "Punch".  What Was The Input And
%What Was The Output From Your Research?  Allow People To Stop Reading After
%The Introduction (Many Will, Unfortunately) With A Decent Sense For The 
%Entire Study.

The sense of community in an organization is not a binary entity.  One cannot
say that it either exists or it does not.  As with many other difficult to
measure concepts, community exists in varying degrees.  Clark \cite{Clark73}
defines community as

\begin{quote}
  ``a sentiment which people have about themselves in relation to others and
  others in relation to themselves;...there are two essentials for the
  existence of community: a sense of significance and a sense of solidarity.
  The strength of community within any given group is determined by the
  degree to which its member's experience both a sense of solidarity and a
  sense of significance within it.''
\end{quote}

More importantly, it is the perception of those involved that determines the
relative strength of a community.  Taking this a step further, the strength of
a community is also based on the community's sense of self-awareness.  For how
can one feel important or bonded with others without first knowing who they
are, what they do and how they contribute to the community?

In this paper, a ``sense of community'' is viewed as a combination of several
ingredients, including: each group member's sense of significance, their sense
of solidarity, and their collective sense of self-awareness.  From this
definition, it is possible to characterize the level of community in an
organization by evaluating the following measures:

\begin{enumerate}
\item{Do members feel they play an important role?}
\item{Do members feel a sense of belonging?}
\item{Can members associate names with the faces of others in the organization?}
\item{Do members know each other personally?}
\item{Can members correctly identify a `resident expert,' if there is one, on
  subjects of importance to the organization?}
\item{Are members aware of the different projects or issues in the
  organization?}
\end{enumerate}

The basic premise of this paper is that an organization's sense of community
can be positively affected through computer mediation.  An information system
designed with features that support these measures of community was the testbed
for this study.

This system is similar to several others, Edunet \cite{Wolpert91}, the Well
\cite{Rheingold93}, and the Davis Community Network ({\tt
http://www.dcn.davis.ca.us:80/}).

The thesis of this paper is that an appropriately designed information
system can improve the sense of community in an organization.  To evaluate this
thesis, an information system was adopted and its design extended with the
above measures in mind.  Pre-test and post-test questionnaires were
administered to the organization to assess the sense of community.  Logfiles
kept by the information system were analyzed to evaluate how the system was
actually used.

\section{RESEARCH METHOD}
%I investigated the strengths and weaknesses of a computer-based approach to
%improving the sense of community within one organization, the Department of
%Information and Computer Sciences at the University of Hawaii.  The World Wide
%Web's communication infrastructure was used to support community building in
%the department.  This seemed to be the best choice as the related software is
%free, popular, and easy to use.  So, I set up a Web server for the ICS
%department with features designed specifically for improving the sense of
%community.

To investigate the strengths and weaknesses of a computer-based approach to
improving sense of community, I designed an interactive WWW server for the
Department of Information and Computer Sciences at the University of Hawaii.

Based on these measures of community, I came up with requirements for a
community-enhancing information system and incorporated them.  

%What measures of community? This must have gotten lost in the cutting and
%pasting.

%One requirement was that it should help the community members feel important in
%the group.  This was accomplished by providing the users with an unedited voice
%in their publication effort.  The system was also extended to foster a sense of
%belonging.  I built mechanisms for facilitating user involvement and
%contributions to the information system.  Several additional features were
%incorporated to increase a group's collective self-awareness.  I included a
%photoboard of organization members.  There were sections identifying the
%subgroups of the organization.  These subgroups could be formed through common
%personal or professional interests.  Other sections highlighted the various
%projects underway in the department.

%Needs to be much more succinct:
Several mechanisms were designed to foster importance and belonging:
  \begin{itemize}
     \item {\bf Unrestricted contributions.} ...
     \item {\bf Photoboard} ..
       :
       :
  \end{itemize}

The goal was to increase the sense of community in the department with the
support of computer-based mechanisms.  Armed with a tool, I set out to study
this approach at community building over the course of one semester.  Before
starting, department members anonymously and voluntarily filled out a pre-test
questionnaire.  This questionnaire was designed to evaluate the ``baseline''
level of community.  Next, I introduced the Web site to the department,
offering training sessions on how to access information as well as how to
contribute to it.  I conducted introductory classes throughout the semester and
held some advanced ones in the latter half.  At the end of the semester,
department members filled out a post-test questionnaire to assess the level of
community at that time.

%Where's the \section{RESULTS}???

\section{CONTRIBUTIONS OF THIS RESEARCH}
%I would itemize this and make it more succint.
The results of this research offers many insights into the strengths and
weaknesses of computer based mechanisms to improving the sense of community in
an organization.  One contribution is that this research showed how inequities
between department members inhibited community development.  This was clear
between the undergraduate and graduate students.  Graduate students received
support from the department in the form of computing facilities whereas the
undergraduates did not.  This did more than prevent community development.  In
fact, it further divided the community.

Another contribution concerns the collective self-awareness of a group.  Public
disclosure of information increased communication in one direction.  This can
be detrimental to an organization if the information continues to flow one way.
There also needs to be a way for the receivers of this information to become
involved.  However, the collective self-awareness of a group can be improved
through computer mediation.

The design of the information is another contribution.  While this system
design did not improve the overall sense of community in the department, people
found it useful and quickly adopted it.  The study lasted only 4 months but
people quickly began using the system in the first month with relatively little
training.

The fourth contribution this research has made is that it has defined another
purpose for the World Wide Web.  It is no longer just a universal network of
information; it now also contributes to community building.

The final contribution is that the Web information system also served as a
diagnostic tool for factors promoting or inhibiting community development.  The
system draws out the visible differences between department members to reveal
new issues important to the group.

%\section{RECOMMENDATIONS FOR WEB SITE BUILDERS}
%Fold this into the CONTRIBUTIONS section.
Based upon these experiences, I recommend the following to encourage community
building through participation in a Web site: look inward as well as outward,
anticipate the impact of knowledge - provide mechanisms for user involvement,
repeatedly offer training sessions and scale to skill level, publicly release
tools and assess the impact of the system on community at introduction.

The first recommendation is to look inward as well as outward.  Many Web sites
focus on an outward design.  Their goal is to increase their visibility to the
world.  This is important and should not be discouraged.  However, where
community is concerned, a design addressing the needs of the members and the
organization should be emphasized.

The second recommendation is to anticipate the impact of knowledge.  Recognize
that information will be flowing in new directions.  People will learn more
about their organization.  By providing a way for users to react to this
information or becoming involved with it somehow can help alleviate alienation.

The next recommendation is to repeatedly offer training sessions.  Once users
have achieved a certain level of knowledge with the system, more advanced
training sessions should be offered to both maintain their interest as well as
allow them to better utilize the system capabilities.

The fourth recommendation is to make a public release of all tools associated
with the system.  For example, a unix shell script, such as the one I used for
creating home pages, should be made public and advertised online so that other
users have a way to use these tools also.

The last recommendation is to assess the impact of the system on the sense of
community when it is first introduced.  This could be in the form of an online
questionnaire.  The goal here is to determine the users' reaction to the
system.  Do they like it and find it useful, or can they think of features they
would like to see incorporated?  Since the system is for their benefit, it
should certainly be tailored to their needs.

The above recommendations may help to encourage community building.  The system
itself is not as important as the process by which it is introduced.  The
members should be involved in its development and determine how it evolves.

%\section{RELATED WORK}
%I would have one sentence in the INTRODUCTION section that says, 

%  This system is similar to several others, Edunet \cite{..}, the Well
%  \cite{...}, and the Davis Community Network \cite{...}.  

%Then you can delete this whole section. 

%A few systems have been developed for similar purposes.  I first discuss GC
%EduNET, an information system designed for supporting educators in the Georgia
%education system.  I then discuss the Davis Community Network, a virtual
%community created to enhance the physical one composed of its residents.  I
%also mention the Well, a global virtual community popularized and recounted by
%Rheingold \cite{Rheingold93}.

%%talk about how these systems have a recipe for community - internet access

%GC EduNET \cite{Wolpert91} is an electronic information service developed to
%bring together Georgia teachers and school administrators and to create to a
%pool of resources important to that group.  The system provides capabilities
%for conferencing, email, file sharing and database searching.  It is unique in
%its design to alleviate teacher and administrator isolation.  The system
%services more than 1500 members covering 101 counties in Georgia.  In addition
%to supporting email and conferencing, the system was organized into six major
%areas: Libraries, Questions and Answers, Electronic Marketing, Center for
%Adolescent Needs, Software and File Exchange and Organizations.  The major
%difference between GC EduNET's online community and that of the ICS department
%is their absence of a physical community.  They designed a system to create a
%community of interest {\em online}.  They did not have a membership; they
%created one for the purpose of sharing experiences and resources.  The group
%exists only in a virtual community.

%The Davis Community Network ({\tt http://www.dcn.davis.ca.us:80/}) is a
%non-profit organization created for residents of Davis, California.  Its
%purpose is to strengthen the community of Davis by providing Internet-based
%communications and information services and support.  More specifically, they
%provide newsgroups pertinent to the Davis community, mailings lists, and a WWW
%server.  The Web site offers information on local businesses, city government,
%education, entertainment and tourist and new resident information.  Like the
%ICS department, this online group was created to enhance their physical bond.
%However, theirs is a community of place servicing a much larger membership
%where the roles of the citizens are not as clearly defined as in communities of
%interest.

%The Well (Whole Earth 'Lectronic Link) is an online service based in San
%Francisco providing connection to USENET newsgroups, the World Wide Web, Gopher
%and its own MUSE (multi user simulation environment).  Users from all over the
%world may join and more than half of its members are from outside of
%California.  The Well is also associated with physical communities in the San
%Francisco Bay and New York City areas. 

%There are many other community-enhancing systems\footnote{A list is available
%at http://www.well.com/user/hlr/vircom/index.html} for various types of groups.
%They often incorporate the same technology and software: email, WWW, Gopher,
%etc.  However, few have modified these systems to address the particular needs
%of their organization.

\bibliography{/group/csdl/bib/www-ics,/group/csdl/bib/csdl-trs}
%\bibliographystyle{/group/csdl/tex/named-citations}
\bibliographystyle{plain}

\end{document}
