\documentstyle [12pt,twoside,epsf,ecscw95]{article}

%%\input{/group/csdl/tex/psfig/psfig}

\begin{document}
\title {Building Software Review Systems\\
 Using CSRS}
\author{ Danu Tjahjono\\
Department of Information and Computer Sciences \\
University of Hawaii, U.S.A.}

\maketitle

\thispagestyle{empty}
\begin{abstract}
The importance of Software Review and its benefits have been well
documented. However, there are many variations of the method in
practice. This paper discusses a new approach to how organizations can 
build their own review systems that are most suitable to them.
Our basic approach is to use CSRS modeling languages
to characterize the review method descriptively. 
The language descriptions are then compiled to generate the
corresponding review systems.
CSRS modeling languages are developed based on FTR framework, that is,
a classification of current FTR practices.
\end{abstract}

\section{Introduction}

Software Reviews or Formal Technical Reviews (FTR) play
an important role in software development as a means to
improve the quality of software products. Unfortunately, many
software organizations have not employed the full potential of
this technique.  One problem that hinders its adoption to an
organization is the labor intensive nature of the technique,
especially when practiced manually. Another serious problem is related
to  the current proliferation of FTR methods and ambiguities in their 
practices. The same review method is often practiced differently by
different organizations. 
%Furthermore, there are many conflicting theories as to
%what factors can be attributed to the success or failure of a
%particular review method. 

Our CSRS (Collaborative Software Review System) version 3 addresses
these problems. Instead of implementing a specific review 
method, we developed a generic review system with enactable review
methods. The review method is first described using the CSRS modeling 
languages. The language descriptions are then compiled to generate the
review system that runs the prescribed review method. 

This approach allows the organizations to implement their own computer
based review systems using the review methods that most suitable to their
organizations. Furthermore, we provide 
automatic metrics collection that allows the organization to
evaluate their review methods and improve them over time.


\section {Research Problems}
Formal Technical Review is a well-defined and well-executed activity
where a group of people meet together to evaluate software products.
Although there is a well published literature on how to
effectively conduct this review \cite{Fagan76}, in
practice there are many variations in performing it.
For example, some researchers
\cite{Fagan76,Russell91} explicitly advocate the use of paraphrasing
for effective review, while others \cite{Humphrey90} consider it
optional. Still others prefer to include no group process (i.e.,
primarily the collection of individual efforts)
\cite{Votta93}, or discourage active participation of the producer
during group meeting \cite{Ackerman89,Russell91},  or to advocate the
use of checklists \cite{Fagan76,Humphrey90,Freedman90} or selective
test cases \cite{Ackerman89,Dunn84}. 

Similarly, the same review method is often 
implemented differently by different review systems. For example, both
ICICLE \cite{Brothers90} and Scrutiny \cite{Gintell93} implement
Fagan's inspection.  As a result, the users of these systems are no
longer practicing the same method.

To address these problems, we developed a framework that
characterizes a FTR process. As such, it will not only expose the
similarities and differences in current FTR practices, but also remove
any ambiguities concerning how to conduct the review.


\section {FTR Framework}

The framework looks into the underlying review factors or review
dimensions that characterize a review process. In general, a review
process consists of a set of review phases, each has its own
objective(s), interaction mode and technique or strategy.

\begin{enumerate}
\item {\bf Review Objective.} This dimension describes the goal
of a given review phase. This normally includes comprehension,
examination and/or collection/consolidation. Comprehension is the
review activity that focuses on understanding review materials;
examination is the activity that focuses on finding errors or other
anomalies in the software products; and collection is the activity
with the goal of collecting or consolidating issues
raised by individual reviewers (e.g., from previous phase). 
For example, the meeting phase in Fagan's
inspection \cite{Fagan76} has the objective of examination, while the
meeting phase in Humphrey's inspection \cite{Humphrey90} has the
objective of collection. 
In practice, a review phase can include more than one objective (e.g.,
the preparation phase in Humphrey's inspection has the
objectives of both comprehension and examination).

\item {\bf Interaction Mode.} This dimension describes how the
participants collaborate to realize the objective. This includes three
sub-factors: degree of collaboration, synchronicity, and role. The
degree or type of collaboration refers to whether the participants
perform the 
assigned task as a group, or as  individuals (i.e., collection of
individual efforts).  
Synchronicity refers to whether the group meeting takes place
synchronously (at the same time) or asynchronously (can be at
different time). Role refers to 
individual participant roles (e.g., Moderator, Reviewer, Reader,
Recorder, Producer, etc.). 
All of these factors have been shown to shape a review method.
For example, in Fagan's code inspection, the preparation phase is
basically individual mode, where the participants perform the task
individually and asynchronously; the meeting phase is
group-synchronous mode where the participants meet face to face at the
same time.
Likewise, the group meeting normally includes different participant
roles (e.g., the Moderator who moderates group discussion, the
Reader who presents 
review materials, the Recorder who records meeting results, etc).


\item {\bf Technique.}
This dimension describes a specific technique or strategy that the
participants use to accomplish the objective of a given phase.
For example, the use of checklist during individual examination
activity, or paraphrasing during group examination activity
\cite{Fagan76}, or stepwise abstraction technique during individual
examination \cite{Basili85}.

\end{enumerate}

With this framework, a review process can now be described in terms
of phases, where each phase has a specific objective, interaction mode
and technique. Using this approach, we are able to characterize and
classify most major current review methods \cite{Tjahjono94}.

\section {CSRS (Collaborative Software Review System)}

CSRS version 3 is a review system that implement this framework. As
such, it can model and run various review methods. The basic approach
is to provide generic modeling languages to implement the framework,
specifically, the data and the process modeling languages.
The data modeling language allows one to describe the artifacts
manipulated by the review method, and the process modeling language
allows one to construct a review method by 
prescribing the review phases, the objective, the interaction mode and
the review strategy to be used within each phase.
These language descriptions are then compiled to generate the
corresponding review system. 


\subsection{CSRS Data Modeling Language}

The data modeling language describes what artifacts are manipulated by
the review method. It defines various types of review
artifacts, their field structures, and the relationships among the
artifacts. In CSRS terminology, review artifacts are called nodes.
Each node may contain multiple fields, and each field may store
various types of object, such as 
textual string, function definitions, and link definitions.
Relationships among the nodes are represented by links.   
Nodes and links are typed, that is, different review artifacts and
their relationships can be represented by different node and link types. 
A set of computational supports can also be provided for different
artifact types.

One may also define a set of attributes to node instances (with or
without specific node types), such as node names, creation-date,
creator, etc. 

A node can also have {\it state}. This state allows one to
keep track various review status, such as whether the user has read
and reviewed specific source nodes during specific phases, or has
reviewed all the checklist, etc.

CSRS data language defines four {\it base} node
types: Source, Commentary, Checklist and Administrative. The
applications can inherit from these types to implement their own
specific sub-types.

Source nodes contain input artifacts to be reviewed. For code review,
the users typically want to differentiate source nodes into function,
macro, procedure, etc.  CSRS also provides built-in parsing mechanisms
that allow input files to be automatically parsed into source nodes. 

Commentary nodes contain output artifacts generated during the
review. Similarly, one can define different commentary types,
such as, {\it issue} nodes to record defects in the source,
{\it action} nodes to record proposal in resolving the issue, 
{\it consolidated-issue} nodes to consolidate similar issue nodes, etc.

Checklist nodes contain verification aids. In manual review practices,
such as Fagan's method \cite{Fagan76}, the reviewers are often required
to consult this checklist when examining source nodes. CSRS language
system allows checklist items to be defined for a specific source
node, and/or a specific participant role.
This checklist can also be stated as {\it required} or {\it optional}. 
For the required checklist, the reviewers
must explicitly check the item off when examining the corresponding
source. 

Administrative nodes contain administrative information about the
review process and project related information, such as participant names
and roles, the starting and ending review date, etc.

Figure \ref{fig:data-language} 
shows an example of the data modeling language for describing
{\sl Issue} node, its fields (Subject, Category, etc),
and the link ({\sl derives}) originated from {\sl Source} node to this
Issue node. 

\begin{figure}[tp]
  \footnotesize
  \begin{verbatim}
(l*data*define-node-schema
  :name "Issue"
  :superschema "Commentary"
  :fields (("Subject")
           ("Category")
           ("Criticality")
           ("Source-node" :init-fn e*issue*set-source)
           ("Lines")
           ("Description")
           ("Consensus" :init-fn (e*node*set-consensus "confirm"))
           ("Related-issues")
           ("Proposed-actions")
           )
  :make-link-fn e*derives*make
  )

(l*data*define-field-schema
  :name "Subject"
  :type text
  )

(l*data*define-field-schema
  :name "Source-node"
  :type lisp
  :display-fn e*node*name
  )

(l*data*define-link-schema
 :name "derives"
 :documentation "Derives issue node from any source nodes"
 :from-nodes ("Source")
 :to-nodes ("Issue" "Consolidated-issue")
 :label "csrs-link-label"
 :at-fields  '((:from-node "Source" :to-node "Commentary" :field "Issues")
              )
 )

  \end{verbatim}
  \normalsize
  \caption{Example of CSRS data language}
  \label{fig:data-language}
\end{figure}



\subsection{CSRS Process Modeling Language}

The process modeling language describes specific phases that
review participants must go through in order to complete the review
process. 
Each phase  may involve one or more objectives to be
achieved, specific interaction mode, and specific strategy or
technique to be used in order to achieve the objective. 

To describe the objective of a phase using this language, the user
defines a predicate function that specifies the entry and/or the exit
condition of a phase. The system will check these conditions to
determine whether the current phase can be ended (i.e., exit
condition of current phase) and the next phase can be started (i.e.,
entry condition of the next phase). 

The interaction mode of a phase is specified by defining the
synchronicity mode, the roles and the type of collaboration within the
phase. 

CSRS provides two built-in synchronicity modes: asynchronous
and synchronous review mode. In the asynchronous mode, the
participants manipulate review artifacts (nodes) asynchronously.
Group discussion (if it is allowed by the review method) is done
through the creation of follow-up links to the appropriate nodes. 
Additional computational supports can
also be provided to automatically consolidate the discussion by 
traversing the proper links. For example, to display all arguments that
support a proposal mentioned in a node, one can write the operation
that traverses all links of type {\it confirm} originating from the
node. 

The synchronous mode allows all participants' screens to be
synchronized or locked with the presenter.
Thus, when the presenter displays a node, the node will also be
displayed in all participants' screens. 
The language construct allows one to define what node types to be
synchronized, and with whom they are to be synchronized. In
synchronous mode, one needs also to define a meeting leader (for
example, Moderator). The meeting leader will be granted additional 
operations not available to other participants during the
corresponding phase. One of them includes the operation to 
temporarily reassign the presenter to other participants during the
meeting. 

Participant roles are defined using the {\it define-role} construct. 
This construct itself does not specify the participants who will
assume the corresponding roles. Roles assignment is  done during system
initialization when all participants have been given specific
roles by the meeting administrator.

The type of collaboration among participants is defined through access
control privileges of the artifacts. This access control specifies
what artifacts can be 
accessed (read/write) by whom (what role) during what phase.

The review technique or strategy to be used in a phase is review
method specific. They have to be implemented
separately and later to be included into the system using {\it
define-operations} construct. 
CSRS only provides one built-in review technique, i.e., the
checklist based review technique.

CSRS implements a sequential process workflow, that is, a
review process consists of a sequential ordering of review phases.
CSRS provides both manual and automatic
activation of these phases. In the manual mode, a designated participant
(e.g., moderator) will check the status of the review periodically (or
program the system to notify him/her when the status changes) and activate
the next phase accordingly. In the automatic mode, an autonomous agent
(background process) will monitor the status of the review
periodically by checking the exit condition of the current phase, and/or
the entry conditions of the next phase, and activate the next phase
accordingly when the conditions have been met.

Within a phase, the process workflow is enforced through menu
selections. The operations installed in the menu are
not only sensitive to the current phase, but also to the current role that
the participant is assuming and the current node type and field type.
When installing the user-defined operations (i.e., using the
define-operations construct), one must specify
whether the operations are applied to specific 
phases, roles, node types and/or field types. The system can then
figure out whether to install the operations as pop-up menu,
or pull-down menu.

The operations can also be installed to be automatically invoked by the
system right after the participants successfully connect to a phase.
This feature allows the process work flow to be initiated. It can
guide the participants to what they have to do when they first
start the phase.

Figure \ref{fig:process-language} shows a partial  example of process
language for describing the review process of FTArm
\cite{Johnson93,CSDL-93-17}.  The method defines 6 review phases:
Orientation, Private, Public, Consolidation, Meeting, and Conclusion.
The Private phase defines Reviewer and Producer roles, that is, only
participants with these roles can login to Private phase.

The define-access-privilege construct specifies
that the participants with the Reviewer role 
may create Issue, Action and Comment nodes during 
Private review phase. They may also read Source
and Checklist nodes, but can only read their own Issue, Action and
Comment nodes (i.e., commentary nodes created by other reviewers are
not visible).

Finally, the last language construct installs the operation
to list all source nodes. This operation is intended for the
participants with the roles of Producer, Reviewer, and/or Moderator
and to be used in Orientation phase. Furthermore, it will be
automatically invoked when the participants first connect or login to
this phase.  


\begin{figure}[tp]
  \footnotesize
  \begin{verbatim}
(l*process*define-method       ;define process model of FTR method
  :name "FTArm"                
  :phases ("Orientation" "Private" "Public" "Consolidation" "Meeting" "Conclusion")
  )

(l*process*define-phase
 :name "Private"
 :documentation "Individual reviewers examine source nodes"
 :objective "Finding issues in the source nodes."
 :roles ("Reviewer" "Producer")
 :synchronicity "Asynchronous"
 :technique "Free Review"
 :exit-condition-fn e*private*sources-reviewed-p
 :display-status-fn e*private*display-status
 :metrics-collection-p t
 )

(l*process*define-role         ;define valid role
 :name "Reviewer"
 :documentation "Participants who review source nodes."
 )

(l*process*define-access-privilege 
  :roles ("Reviewer")
  :phases ("Private")
  :privilege ((:operation create-node 
               :access private 
               :node-schemas ("Issue" "Action" "Comment"))
              (:operation read-node 
               :access public 
               :node-schemas ("Source" "Checklist"))
              (:operation read-node
               :access private
               :node-schemas ("Issue" "Action" "Comment"))
              )
  )

(l*process*define-operations
 :operation-specs 
 (("List all source nodes" (i*sbuff*sources-orientation 'i*sbuff*sort-sources))
  )
 :roles ("Producer" "Reviewer" "Moderator")
 :phases ("Orientation")
 :connect-p t
 )

  \end{verbatim}
  \normalsize
  \caption{Example of CSRS process language}
  \label{fig:process-language}
\end{figure}


\section{User Interface}

CSRS user interface is built on top of Lucid Emacs, an X-Window
version of Emacs editor that supports pull-down menu, pop-up menu, and
scroll bar. A node is displayed as an instance of Emacs
buffer, with its fields as distinguished regions in the buffer. A link
will appear as link-label in the buffer. To follow the link, the user
simply clicks the {\it mouse} on the corresponding label. The
user may also annotate a 
specific text region in a node with another node.

Different node buffers can also be displayed on different Emacs screens. 
CSRS also provides language constructs to define screen real estate,
such as where to put the Emacs screen on the workstation display and
what node types should be displayed on it.
The interface language also allows one to define
{\it summary-buffers} to display summarized views of node instances. 
The specific items to be displayed on this buffer and their display
format can be specified in advance. 

Figure \ref{fig:csrs-screen} shows typical CSRS screens: the left
side displays source nodes, the upper right side displays summary
nodes and the lower right side displays commentary nodes.

\begin{figure}[h]
%  {\centerline{\psfig{figure=csrs-screen.ps}}}
  \begin{center}
  \epsffile{csrs-screen.ps}
  \end{center}
  \caption{CSRS Screen}
  \label{fig:csrs-screen}
\end{figure}

\section{Review Metrics}
Metrics collection is always part of a well executed review activity.
CSRS collect these metrics automatically and at a
fine-grain level. There are two types of metrics collected by
the system: the ones associated with review artifacts and their
attributes, and the ones associated with review efforts (time spent by
participants).  The first type can be obtained directly
from the corresponding nodes/links generated during the review
(e.g., counting the number of issue nodes generated by a
particular participant). To obtain the second type of metrics, we
instrument the primitive CSRS commands with time-stamp
mechanisms. When the user 
invokes the command, the system will record the starting time of the
command, the artifact type upon which the command is operated on, and the
current selected screen. 
The time-stamp mechanism allows one to obtain various kinds of
review metrics that relate to the user's efforts, such as, how long the
user reads particular nodes, the order of nodes traversal,
or even the order of command invocations from the beginning to
the end of the review session.

\section{System Architecture}
CSRS implements a client-server architecture. Multiple clients send
commands to a shared server. 
The server implements basic hypertext operations,
such as creating, retrieving and deleting nodes and links.
The server also implements event mechanisms
that allow different clients to synchronize their states (e.g., when one
client updates the node, the updates will be propagated to other clients
currently displaying the nodes). This  feature is used to
implement synchronous meeting mode.

The clients are implemented on top of Egret, a generic hypertext system
that provides session management to the server as well as typing
mechanism \cite{csdl-92-01,csdl-93-09}. The next layer is CSRS engine
that provides internal 
operations of the system, and the language subsystem that provides
enactable review methods.

One may also add special client running as a background process,
called {\it Agent}, that performs specific 
tasks by listening to specific events from the server.
Some examples of agents include GAgent (which updates internal global
tables of the system), Magent (which listens to review state
and informs the participants through electronic mail messages
accordingly), etc.  
The system architecture is shown in Figure \ref{fig:csrs-architecture}

\begin{figure}[h]
%  {\centerline{\psfig{figure=csrs-arch.epsi}}}
  \begin{center}
  \epsffile{csrs-arch.epsi}
  \end{center}
  \caption{CSRS Architecture}
  \label{fig:csrs-architecture}
\end{figure}


\section{Experiences with CSRS and Current Status}

Our first prototype of CSRS (version 1)
has been around for a while. It implements an asynchronous review
method \cite{Johnson93}. 
The experience we gained from this system was generally positive.
However, it made us realize the need
for generalizing CSRS to implement other review methods.  
This leads to the development of CSRS version 3 that support generic
modeling capability as discussed above. This version has just recently
been completed. 

Since its completion, we have been experimenting with several different
review methods/review systems. The first review method we implemented
in CSRS version 3 is called {\it Hello-World} (i.e., famous term to 
describe simplicity or trivial). It is based on
one phase review process, which we call {\it Private-Review}
where individual review participants inspect the code privately. The
phase is concluded when all participants have declare the review state
of all artifacts to {\it reviewed}. The latter is also defined as the exit
condition of the phase. The 
synchronicity mode being implemented is asynchronous (i.e.,
participants can do their 
review at their own time). Furthermore, all participants assume the
role of {\it Reviewer}, and no review technique is used (i.e.,
free technique: the participants use their knowledge and intuition to
find errors). 

The data model includes primarily
source artifacts of type {\it Function}, commentary artifacts of
type {\it Issue}, and links of type {\it derives}.
Thus, errors discovered in the {\it Function} nodes are recorded in
{\it Issue} nodes. 

Soon after the participants used the system, they
felt that they needed to see
the issues raised by others and be able to comment on them. We then
implemented the 
second review method, called {\it Csdl-Method} (named after our
research group). This review method is almost the same as {\it
Hello-World} except that it uses {\it Public-Review} instead of {\it
Private-Review}, where
individual participants can see the issues raised by other
reviewers. They can also type their comments into the corresponding 
issue nodes. 

As we started to use Csdl-Method,
some participants felt that they had difficulty understanding the
artifacts. They wanted the producer to explain the code beforehand.
We then added a new role  {\it Producer}, and a new 
phase called {\it Orientation} where the producer must present the
code before the reviewers to Csdl-Method.
The Orientation phase implements synchronous
mode, in which the Producer assumes the role of presenter. As such, all
reviewers' screen are locked to the Producer. We also added the role
{\it Moderator} to lead this synchronous phase.
The objective of this phase is solely to understand review
artifacts. To indicate whether we have achieved this objective,
we defined the exit condition to ensure all artifacts have been 
presented by the presenter.

After completing one review session with this new method, we noticed
that during Orientation 
phase, some reviewers did not pay close attention to the
presenter. They looked for errors and created issue nodes on their own.
We then deinstalled the operation to create issue nodes during this
phase. 
During Public-Review phase, some participants also felt the need to
express their opinions and argue about their 
positions using separate commentary nodes instead of putting their
comments in the issue nodes. 
We then extended the data model to include a new node of type {\it
Comment} and  and a new link of type {\it is-commented-by/follow-up}. 
With this extension, the system can also display
unread comment nodes per participant separately.
We also noticed that during Public-Review phase, one
individual contributed a large number of issues ($>$90\%), while the
rest of the group contributed very few issues. We suspect that the
Public-Review phase promotes free-riding. The participants tend to
agree with what others have said. They spend more time reading
existing issues than finding new issues. We felt
that we needed to add a private review phase before proceeding to the
public review phase. 

As a result of this experience, we decided
to implement the final Csdl-Method which is similar to
the FTArm review method \cite{Johnson93,CSDL-93-17}.
This method not only addresses all our concerns above, but also
has rich data and process models. Unlike in the earlier FTArm version,
the new Csdl-Method provides computational supports to all
review phases. 

Currently, we are using CSRS as an experimental testbed to investigate
various Formal Technical Review factors. 
Specifically, we are planning to compare the review effectiveness and
review cost of group based review method and individual based review
method. The former method is basically a face-to-face synchronous
review, while the latter is a private asynchronous review.
Two review methods are developed for this purpose: EGSM (Experimental
Group Synchronous Method) and EIAM (Experimental Individual
Asynchronous Method). Both methods include only one phase where the 
participants search for program errors. In EGSM, this task is done as
a group with one participant assumes the role of presenter and uses
paraphrasing technique, while  in EIAM, the review is done by
individuals privately with all participants assume the role of reviewer.
We also provide electronic voting supports to record the degree of
consensus among group members in EGSM.
We expect to complete the study at the end of Spring 1995.

%We also plan to use the system as an experimental testbed to
%investigate various Formal Technical Review factors \cite{CSDL-94-07}. 


\section{Conclusions}

We have described CSRS (version 3), a computer assisted review that
provides generic modeling languages to implement many variations of
review methods. This capability is not only useful for organizations
to implement their own review methods, but also to incrementally
improve their methods as demonstrated by our experience in using the
system. 

The development of a new review method using these languages 
can take several hours to several days depending upon how extensive new
operations need to be added. 
In general, the language constructs  will
expand into a set 
of standard operations when they are compiled. The user
then needs to install these operations appropriately (for example, for
a particular phase, a particular role, etc). However, in case
of defining a new strategy, the user needs to implement the
operations from scratch.

In the future, we plan to maintain a database of reusable operations
to implement specific review techniques used by different review
methods. As such, it can take much shorter 
time to develop a new review method that use similar techniques
already present in other review methods.

Finally, CSRS can also be used as an experimental testbed for
empirically comparing different FTR methods, and/or evaluating the
underlying review factors. One of our future goals is to remodel
existing review methods using CSRS and perform empirical studies on
them. 


\section{Acknowledgments}

I would like to thank my advisor, Dr Philip Johnson, for his
continuing supports and guidance in this research,
and Carleton Moore for his assistance in preparing this manuscript.

\bibliography{/group/csdl/bib/csdl-trs,/group/csdl/bib/ftr}
\bibliographystyle{plain}

\end{document}

