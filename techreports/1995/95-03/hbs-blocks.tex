%%%%%%%%%%%%%%%%%%%%%%%%%%%%%% -*- Mode: Latex -*- %%%%%%%%%%%%%%%%%%%%%%%%%%%%
%% hbs-blocks.tex -- 
%% Author          : Carleton Moore
%% Created On      : Fri Jan 20 17:15:19 1995
%% Last Modified By: Carleton Moore
%% Last Modified On: Thu Jul 20 14:50:12 1995
%% Status          : Unknown
%% RCS: $Id: hbs-blocks.tex,v 1.3 1995/07/21 01:11:36 cmoore Exp $
%%%%%%%%%%%%%%%%%%%%%%%%%%%%%%%%%%%%%%%%%%%%%%%%%%%%%%%%%%%%%%%%%%%%%%%%%%%%%%%
%%   Copyright (C) 1995 University of Hawaii
%%%%%%%%%%%%%%%%%%%%%%%%%%%%%%%%%%%%%%%%%%%%%%%%%%%%%%%%%%%%%%%%%%%%%%%%%%%%%%%
%% 

\section{Blocks}
\label{sec:blocks}

In the previous section we briefly introduced the four blocks or layers of
HBS.  This section explores each of the blocks in turn, discussing the
requirements, classes, global variables and any other important design issues.

\subsection{Block1: File Operations}

Block 1 provides an interface between the Unix file system and the
rest of HBS.  It provides a set of low level functions that operate on
the basic entities of data-nodes and link-nodes \footnote{See \cite{Wiil90a} for a
detailed description of data-nodes and link-nodes}.  It handles all of the conversions
between the node and link entities and the actual storage on the Unix file
system.  

% Figure \ref{fig:block1} shows the
%internal structure of both the DataFileHandler and the LinkFileHandler.
%
%\begin{figure}[htb]
%  \centerline{\psfig{figure=block1.eps}}
%  \caption{{\bf DataFileHandler and LinkFileHandler}}
%  \label{fig:block1}
%\end{figure}

There are two main classes in this block: DataFileHandler and
LinkFileHandler.  These two classes manage five different files, one for
data-nodes {\em datanodes.hb}, one for link-nodes {\em linknodes.hb}, one
for data blocks {\em datablocks.hb}, one for the last
data-node number {\em maxNode.hb} and one for the last link-node number
{\em maxLink.hb}.  The last data-node
and last link-node files are kept so that the HBS does not reuse data-node
and link-node IDs.  Both classes manage their respective nodes in the same
manner. Both classes have similar in memory data structures.

\begin{itemize}

\item{A balanced binary tree holds the node number and the file address for
  valid entities.  When a node is opened the binary tree is searched to find
  the file address of the entity.  The file is then read to get the
  information.  A balanced binary tree is used to reduce the search time for
  large databases.}

\item{A linked list holds the free file addresses.  When an entity is
  deleted its file address is added to the linked list.  When a new entity
  is created the linked list is checked to reuse disk addresses.}

\item{Bitmaps store the used data-node, and link-node IDs.  In HBS once the
  bit is set for the data-node or link-node ID it is never reset.  This stops
  the reuse of data-node or link-node IDs.}

\item{A Bitmap keeps track of the used data blocks.  When a data-node is
  deleted the bit for each of its data blocks is reset.}

\item{Flags to keep track if there is an entity stored in memory.}

\end{itemize}

We will now discuss each of the classes in Block1.

\subsubsection*{Classes}
\begin{itemize}

\item The TreeEntity class represents a file address that will be stored in
  the balanced binary tree.  This class encapsulates the file address so the
  user does not have to worry about the internal representation.  This class
  provides two accessor functions to get and set the file address.

\item The UseList class represents a linked list that holds the file
  addresses.  It is used to hold the free file addresses. It provides the
  standard linked list operations.  UseList is used by the LinkFileHandler and
  the DataFileHandler classes to store free file addresses.
%
%
% The following
%  operations are provided by  UseList:\begin{itemize}
%    \item{\em  removeFront} removes and returns head element of the list.
%    \item{\em  append} appends element at tail of the list.
%    \item{\em  count} returns number of elements in the list.
%    \item{\em  clear} deletes all elements in the list.
%    \item{\em  useListCopy} copy function.
%    \item{\em  useListPrint} prints the contents of the list.
%    \end{itemize}


\item The DataFileHandler class is one of the two main classes in block1.  It
provides the data-node interface to block2.  The following operations are
defined in DataFileHandler: 

\begin{itemize}
  \item{\em openEntity} retrieves data-node from file.
  \item{\em closeEntity} writes data-node fields to file.
  \item{\em openNewEntity} creates a data-node.
  \item{\em deleteEntity} deletes a data-node.
  \item{\em openData} opens datablock area.
  \item{\em closeData} writes to datablock area.
  \item{\em openLink} opens departing links.
  \item{\em closeLink} writes to departing links.
  \item{\em browse} returns list of data-nodes.
  \item{\em isAvailable} returns 1 if the node exists.
\end{itemize}

\item The LinkFileHandler class is the other main class in block1.  It
provides the link-node interface to block2.  The following operations are
defined in LinkFileHandler:

\begin{itemize}

  \item{\em openEntity} retrieves link-node from file.
  \item{\em closeEntity} writes link-node fields to file.
  \item{\em openNewEntity} creates a link-node.
  \item{\em deleteEntity} deletes a link-node.
  \item{\em browse} returns list of link-nodes.
  \item{\em isAvailable} returns 1 if the node exists.
  \end{itemize}

\item The File class represents the basic Unix file.  It provides the basic
  file operation of open, close, and setting the read point.  It does not
  support any reading and writing operations.

\item The NodeFile class represents the file that hold the data-nodes.  It
is publicly inherited from the File class.  It adds the readEntity and
writeEntity operations that read and write a data-node to the file.  NodeFile
directly stores and retrieves the data-nodes.

\item The LinkFile class represents the file that hold the link-nodes.  It
is publicly inherited from the File class.  It adds the readEntity and
writeEntity operations that read and write a link-node to the file.  LinkFile
directly stores and retrieves the link-nodes.

\item The DataFile class represents the file that hold the data.  It
is publicly inherited from the File class.  It adds the readEntity,
writeEntity, readLong, and writeLong operations that read and write
characters or long integers to the file. DataFile directly stores and retrieves
all of the data stored in the HBS.

\end{itemize}

\subsubsection*{Global Variables}

There are no global variables in block1.  However, the files created by HBS
are hard coded into the system. 

\subsubsection*{Issues}
\label{sec:block1Issues}

Since the names of the files are hard coded there can only be one instance
of DataFileHandler and LinkFileHandler.  Creating more than one instance
can corrupt the files.  

\subsection{Block2: Basic HyperText Operations}

Block 2 handles the basic hypertext operations.  It ensures that all of the
internal linknodes and datanodes are kept consistent.  Block2 acts as a
buffer between block3 and the low level file operations.  

Requests from block3 are passed on to a single instance of the Operations
class which has a single instace of both the Link and Data classes.  The
operations class provides the interface to block3.  When it receives a
request to perform, the operations class decides what combination of link
and node operations must be performed to fulfill the request.  It then
requests Link and Data to perform those individual operations in the
correct order.  Link and Data both have a single instance of the respective
FileHandler class.  Link and Data make requests of the FileHandlers to
handle file operations.  Link and Data are able to hold one link or node in
memory, this way they may be able to service the request without accessing
the disk.  All writes are automatically sent to disk.

\subsubsection*{Classes}
\begin{itemize}
\item The Operations class represents the interface between block 2 and
  3.  It provides all of the Hypertext operations in HBS. \begin{itemize}
    \item{\em entityDelete} deletes a link or node.
    \item{\em createNode} creates a new node.
    \item{\em link} creates links or connects nodes to links.
    \item{\em moveLink} moves a link.
    \item{\em removeLink} removes the connection from a node to a link.
    \item{\em read} reads a key value.
    \item{\em write} writes a key value.
    \item{\em entityRead} reads an entire entity.
    \item{\em entityWrite} writes an entire entity.
    \item{\em browse} returns a list of nodes or links.
    \item{\em appendToData} appends information to nodes.
    \item{\em appendToLink} appends information to links.
    \item{\em isAvailable} returns 1 if the entity exists.
\end{itemize}

\item The Data class represents the nodes in the hypertext database.  It
  interacts with block 1 to perform its operations.  It receives the
  requests from the Operations class that deal with nodes. The Data class
  caches data-node in memory.  This is a write through cache. The Data
  class calls the DataFileHandler operations to fulfill requests that the
  cache cannot.

%
%It provides the
%  following operations:  \begin{itemize}
%    \item{\em createData} creates a new node.
%    \item{\em deleteData} deletes a node.
%    \item{\em openData} retrieves info from file.
%    \item{\em closeData} writes node info to file.
%    \item{\em incrementLinksToMe} increments the number of links pointing to the node.
%    \item{\em decrementLinksToMe} decrements the number of links pointing to the node.
%    \item{\em dataNumber} accessor function.
%    \item{\em size} accessor function.
%    \item{\em linksToMe} accessor function.
%    \item{\em readOneUserDefinedKey} reads key value.
%    \item{\em writeOneUserDefinedKey} writes key value.
%    \item{\em readDataNode} reads entire node info.
%    \item{\em writeAllUserDefinedKeys} writes user defined keys to node
%    \item{\em browse} returns list of nodes.
%    \item{\em readDataBlockArea} reads the data associated with this node.
%    \item{\em writeDataBlockArea} writes the data associated with this node. 
%    \item{\em openDepartingLinks} opens departing links for retrieval.
%    \item{\em closeDepartingLinks} updates departing links.
%    \item{\em readDepartingLinks} returns link array.
%    \item{\em isAvailable} returns 1 if the node exists.
%
%\end{itemize}

\item The Link class represents the links in the hypertext database.  It also
  interacts with block1 to perform its operations.  Like the Data class the
  Link class caches one link-node.  File operations are made though the
  LinkFileHandler class. 
%
%
% It provides the following
%  operations: \begin{itemize}
%    \item{\em createLink} creates a new linknode.
%    \item{\em deleteLink} deletes a linknode.
%    \item{\em openLink} retrieves link from file.
%    \item{\em closeLink} writes link to file.
%    \item{\em setDataNodeNum} sets the target of the link. 
%    \item{\em incrementUseCount} increments the number of nodes pointing to the link.
%    \item{\em decrementUseCount} decrements the number of nodes pointing to the link.
%    \item{\em linkNumber} accessor function.
%    \item{\em useCount} accessor function.
%    \item{\em toDataNodeNo} accessor function.
%    \item{\em readOneUserDefinedKey} reads key value.
%    \item{\em writeOneUserDefinedKey} writes key value.
%    \item{\em readLinkNode} reads entire linknode info.
%    \item{\em writeAllUserDefinedKeys} writes user defined keys to link.
%    \item{\em browse} returns list of links.
%    \item{\em isAvailable} returns 1 if the link exists.
%\end{itemize}

\end{itemize}
\subsubsection*{Global Variables}
There are no global variables in this block.  Each instance of Operations
has a single instance of the Link and Data classes.

\subsubsection*{Issues}

There can only be one instance of the Operations class since it will create
an instance of both Link and Data.  Link and Data will create a single
instance of their FileHandlers.  See section \ref{sec:block1Issues} for issues with
multiple FileHandlers.

\subsection{Block3}

Block3 is responsible for ensuring that locks and events occur.  It
``listens'' to all of the operations that the user requests and performs
the appropriate locking and eventing mechanisms.  In order to perform
locking and events there are several data structures maintained in
memory. 

\begin{itemize} 

\item A balanced binary tree which stores LockEntities.  Each node in this
  tree is indexed based upon it entity ID in the HBS.  When a lock is
  acquired the tree is searched to see if there is a LockEntity for the node
  being locked.  If not a new LockEntity is created and inserted into the
  binary tree.  The LockEntity has an array of integers.  Each element of
  this array represents a key of the entity\footnote{See \cite{Wiil90a} for a
  description of the keys.}.  The value of each element is the userID of the
  owner of the lock. 

  When a lock is released the LockEntity's array is updated.  If there are
  no more locks for that node then the LockEntity is removed from the
  binary tree.

\item An array of EventEntities, one for each operation in the HBS.  This
  array is used to allow users to subscribe to all events on an operation.
  When a user want to know about all createNode operations their bit will be
  set in the EventEntity for the createNode operation.  Each time an
  operation is successful this array is accessed and the users that have
  subscribed to the operation are informed.

\item An array of binary balanced trees, one for each operation in the HBS,
  store EventEntities.  Each tree in the array is indexed by the entity ID.
  When a user subscribes to an operation for a particular entity ID and key the
  appropriate tree is searched for that entity ID.  If no EventEntity is
  found a new one is created and inserted into the tree.  The bit for the
  user and key is set.  When an operation is successful, the appropriate tree is
  searched to see if any user has subscribed to the operation, entity ID
  and key.  If there are any users subscribed they are informed of the operation.

\end{itemize}

These data structures are stored in memory so if the HBS crashes all event
and lock information is lost. When a user logs out all of their locks and
events are cleared.

\subsubsection*{Classes}

\begin{itemize}

  \item The BalanceTree class is a template class that represents a
  generic balanced binary tree.  There are several instantiations of
  the balanced binary tree.  LinkFileHandler and DataFileHandler both
  have balanced binary trees to store file addresses.  There are two
  balanced binary trees in this block to hold the Event and Lock
  information. 

  \item The EventLock class represents all of the operations that a user is
  able to perform in HBS.  It provides the following operation to the user:
  \begin {itemize}
    \item {\em write} writes data to entity.
    \item {\em read} read from entity.
    \item {\em entityWrite} write whole entity structure.
    \item {\em entityRead} read whole entity structure.
    \item {\em createNode} create a node.
    \item {\em deleteNode} delete a node.
    \item {\em link} creates links or connects nodes to links.
    \item {\em moveLink} 
    \item {\em removeLink} 
    \item {\em event} subscribe to an event.
    \item {\em unEvent} unsubscribe to an event.
    \item {\em showEvent} show users subscribed to event.
    \item {\em lock} lock an entity.
    \item {\em unLock} unlock an entity.
    \item {\em showLock} show who owns the lock.
    \item {\em showUsers} show all the users connected to HBS.
    \item {\em connect} used to inform other users that a new user has connected.
    \item {\em disconnect} disconnect from the HBS.
    \item {\em browse} get a list of nodes or links.
    \item {\em append} append data to a node.
    \item {\em createWithName} create a new node with the supplied name.
    \item {\em goodLink} create a link between the supplied nodes.
    \item {\em message} send a message to all users.
    \item {\em connected} tell the HBS to start listening to other users.
    \item {\em createWithNameData} create a new node with supplied name and data.
    \item {\em createWithNameDataLock} create a new node with supplied name,
    data and lock. 
    \item {\em linkNewNode} create a link to a new node with supplied name
    and data.
    \item {\em linkNewLockNode} create a link to a new node with supplied
    name, data and lock.
    \item {\em writeUnlock} write data to entity and unlock.
    \item {\em abnormalDisconnect} used to inform other user when a
    connection is broken.
    \item {\em deleteLink} delete the supplied link.
    \item {\em pointToPoint} send a message to a particular user.
    \item {\em changePassword} change the user's password.
\end{itemize}

\item The EventEntity class provides an encapsulated array of BitSets one
  for each user in HBS.  The BitSet represents a key defined in HBS.
  When a user subscribes to an operation on an entity ID and key, the key
  bit is set in the EventEntity for the entity ID in the binary tree for
  the operation. event the appropriate bit is set for that user.

%  Figure
%  \ref{fig:eventEntity} shows the structure of an instance of Evententity.
%
%\begin{figure}[htb]
%  \centerline{\psfig{figure=eventEntity.eps}}
%  \caption{{\bf Structure of an EventEntity}}
%  \label{fig:eventEntity}
%\end{figure}

  
\item The LockEntity class provides an encapsulated array of integers,
  each element of the array represents a key defined in HBS.  When a user
  locks a key of an entity the element of the array is set to the user's
  ID.  When the user unlocks the key the element is set to -1.

\item The TreeLink class is a template class used as the nodes of the
  BalanceTree class.  Since it is a template it can be instantiated to hold
  any type necessary.  In HBS it is instantiated with LockEntities,
  EventEntities, and TreeEntities from block1.

\end{itemize}

\subsubsection*{Global Variables}

There is an array of balanced binary trees called Tree.  The index into
this array is the operation number.  There is an array of EventEntities
called AllNodes that stores event subscriptions for all nodes.  There is an
array of integers called ShowEventVar that holds a list of users who
are subscribed to the event.  The array of integer SendEventVar is used to
hold a list of users that are going to receive the event.

\subsubsection*{Issues}

Locks and Events are not persistent.  When a user disconnects from the HBS
all of their lock and event information is cleared.  Applications that want
persistent locks and events must use some other mechanism to ensure that
that information is maintained.

\subsection{Block4}

Block4 is responsible for communicating with multiple users.  It handles
all of the socket based communications and handling multiple users.  Block4
makes HBS a multiuser database.  To do this block4 has two major data
structures: \begin{itemize}

\item An instance of the EventLock class to handle all requests from the
  users.

\item An array of hb\_client instances.  Each instance represents a
  potential user.  Each user has 3 TCP/IP sockets associated with them.  This
  is how the HBS communicates with the users.

\end{itemize}
The server goes to sleep until there is a request from a connected user or
a new user requesting a connection.  When a new user want to connect to the
HBS they connect to a unique socket determined from the command line when
HBS is started.  When Block4 detects a connection on this unique socket it
creates 3 new sockets and assigns them to the next available hb\_client
instance.  When Block4 detects a request from a connected user it reads in
the operation number and then the parameters for the operation.  It then
passes on the request to Block3.  When Block3 returns Block4 passes on the
results to the user.

\subsubsection*{Classes}
\begin{itemize}
  \item The hb\_client class represents the users in HBS.  Each instance of
  hb\_client has three socket to communicate with the HBS, a string for the
  user's name, and some other administrative variable.

\end{itemize}

\subsubsection*{Global Variables}

One instance of EventLock class, called block3, to handle all of the
requests.  An array of hb\_client, called client, instances to represent the users.

\subsubsection*{Issues}

The maximum number of users that can connect to is fixed by a global
constant.  Currently this number is 32.  To change the maximum number of
users HBS must be recompiled.

\subsection{Config}

The Config block contains the definitions of the error codes, the operation
numbers, the maximum number of users, the node structure and the link
structure.  This block is included in all other blocks.  There are no class
declarations just basic structures and values.

\subsection{Intra Block Issues}

There are several intra block issues.  The first is the flow of control is
complicated since each block can detect an error and return an error
result.  Also block3 uses block4 to send out the events before it returns
the result to block4.  Another issue is often data structures are
created in block1 and have to be destroyed in block4.  This can lead to
memory leaks.
