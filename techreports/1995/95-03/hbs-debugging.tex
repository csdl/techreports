%%%%%%%%%%%%%%%%%%%%%%%%%%%%%% -*- Mode: Latex -*- %%%%%%%%%%%%%%%%%%%%%%%%%%%%
%% hbs-debugging.tex -- 
%% Author          : Carleton Moore
%% Created On      : Fri Jan 20 17:16:00 1995
%% Last Modified By: Carleton Moore
%% Last Modified On: Thu Jul 20 14:57:59 1995
%% Status          : Unknown
%% RCS: $Id: hbs-debugging.tex,v 1.3 1995/07/21 01:12:10 cmoore Exp $
%%%%%%%%%%%%%%%%%%%%%%%%%%%%%%%%%%%%%%%%%%%%%%%%%%%%%%%%%%%%%%%%%%%%%%%%%%%%%%%
%%   Copyright (C) 1995 University of Hawaii
%%%%%%%%%%%%%%%%%%%%%%%%%%%%%%%%%%%%%%%%%%%%%%%%%%%%%%%%%%%%%%%%%%%%%%%%%%%%%%%
%% 
\section{Debugging Tools}
\label{sec:debugging}

The HBS code has two types of debugging tools built into it: memory leak
detection and control flow tracing.  For efficiency, the memory
leak detection code is commented out of the source.  To enable the memory
leak detection HBS will have to be recompiled.  The control flow tool
is active in the current release of HBS.  Each debugging tool is discussed
in turn.

\subsection{Memory Leak Detection.}

There are two different Memory Leak packages available in the HyperBase
{\bf HeapStats} and {\bf MemoryCheck}.  Both packages output their messages
to standard error.

\subsubsection{HeapStats Package.}

The simplest package is {\bf HeapStats}\cite{Cargill92}.  {\bf HeapStats}
provides a count of all calls to {\tt new} and {\tt delete}.  To initialize
the {\bf HeapStats} counters call the function {\tt HeapStats::reset()}.
To get a display of all the {\tt new}s and {\tt delete}s use the function
{\tt HeapStats::report()}.

To use the {\bf HeapStats} package do the following:

\begin{enumerate}
\item{Un-comment the {\tt HeapStats::reset();} and {\tt HeapStats::report();}
lines in the files you are interested in.}
\item{Add {\tt HeapStats::reset();} and {\tt HeapStats::report();} to the
files you are interested in.}
\item{Re-compile HBS with the {\tt -D HEAPSTATS} option in the command line.}
\end{enumerate}
\subsubsection{MemoryCheck Package.}

The second Memory Leak package is {\bf MemoryCheck}\cite{Kanze94}  {\bf
MemoryCheck} is similar to {\bf HeapStats} in that it keeps track of all
{\tt new}s and {\tt delete}s.  {\bf MemoryCheck} is more sophisticated than
{\bf HeapStats}.  It keeps track of the number of bytes allocated and
deallocated.  It can tell you if there is an memory leaks and the number of
bytes that were leaked.  It tells you the location in the code where the
memory was allocated.  To initialize the {\bf MemoryCheck} routines call
the function {\tt MemoryCheck::mark()}.  To check for any memory leaks call
{\tt MemoryCheck::check()}.

To use the {\bf MemoryCheck} package do the following:

\begin{enumerate}
\item{Un-comment the {\tt MemoryCheck::mark();} and \\
{\tt MemoryCheck::check();} lines in the files you are interested in.}
\item{Add {\tt MemoryCheck::mark();} and \\
{\tt MemoryCheck::check();} to the files you are interested in.}
\item{Re-compile HBS with the {\tt -D MEMORYCHECK} option in the command line.}
\end{enumerate}


\subsection{Control Flow Tracing.}

Control Flow is built in to HBS. The control flow instrumentation sends its
output to standard error.  There are four levels of control flow

\begin{enumerate}
\item{Client Interface Commands.}
\item{Event Level Functions.}
%\item{File Level Function Calls.*}
%\item{Low Level System Calls.*}
\item{Strings received by/sent from HBS.}
\end{enumerate}

You can select any of the control flow tracings you wish.  You may select
more than one level of control flow tracing just by adding the switch to
the command line.

\subsubsection{Client Interface Commands.}

This level of control flow tracing reports which client requested an
operation, what the operation was and the return value for the operation.
The format for the message is: \\ ``{\tt $<userName> `<'  <operation>
<parameters> `>' <return value> at <time>$}''.

To enable this level of control flow instrumentation just use the command
switch {\tt -d1} or {\tt -debug1}  when you start HBS. (i.e. {\tt \% HBS
-d1} or {\tt \% HBS -debug1})
\subsection{Event Level Functions.}
This level of control flow tracing reports all the events the HBS sends
out.  The format for the message is: \\  ``{\tt Event: $<userName> <nodeNumber>,
<operation>, <key>, <concatenated message>$}''

To enable this level of control flow instrumentation just use the command
switch {\tt -d2} or {\tt -debug2}  when you start HBS. (i.e. {\tt \% HBS
-d2} or {\tt \% HBS -debug2})
%\subsection{File Level Function Calls.*}
%
%To enable this level of control flow instrumentation just use the command
%switch {\tt -d3} or {\tt -debug3}  when you start HBS. (i.e. {\tt \% HBS
%-d3} or {\tt \% HBS -debug3})
%\subsection{Low Level System Calls.*}
%
%To enable this level of control flow instrumentation just use the command
%switch {\tt -d4} or {\tt -debug4}  when you start HBS. (i.e. {\tt \% HBS
%-d4} or {\tt \% HBS -debug4})
\subsubsection{Strings received by/sent from HBS.}
This level of control flow tracing reports all the Strings the HBS sends
out or receives.  The format for the message is: \\ ``{\tt Sending $<length> bytes
. done! at <time> \\  Sent: <String>$}'' or \\ ``{\tt Receiving $<length>
bytes . done! at <time> \\ Received : <String>$}''

To enable this level of control flow instrumentation just use the command
switch {\tt -d5} or {\tt -debug5}  when you start HBS. (i.e. {\tt \% HBS
-d5} or {\tt \% HBS -debug5})


Now that the debugging tools have been described, the next section
discusses how to extend the HBS.







