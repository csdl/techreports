%%%%%%%%%%%%%%%%%%%%%%%%%%%%%% -*- Mode: Latex -*- %%%%%%%%%%%%%%%%%%%%%%%%%%%%
%% hbs-extentions.tex -- 
%% Author          : Carleton Moore
%% Created On      : Fri Jan 20 17:16:19 1995
%% Last Modified By: Philip Johnson
%% Last Modified On: Thu Feb  9 12:10:32 1995
%% Status          : Unknown
%% RCS: $Id: hbs-extentions.tex,v 1.2 1995/02/09 22:25:02 johnson Exp $
%%%%%%%%%%%%%%%%%%%%%%%%%%%%%%%%%%%%%%%%%%%%%%%%%%%%%%%%%%%%%%%%%%%%%%%%%%%%%%%
%%   Copyright (C) 1995 University of Hawaii
%%%%%%%%%%%%%%%%%%%%%%%%%%%%%%%%%%%%%%%%%%%%%%%%%%%%%%%%%%%%%%%%%%%%%%%%%%%%%%%
%% 

\section{Popular Extentions}
\label{sec:extentions}
\subsection{Adding new operations}
There are five steps to adding new operations to
HBS. 

\begin{enumerate}
\item Edit the {\tt config/hb\_config.hh} file and include the new operation
  name in the following statement.\small 
  \begin{verbatim}typedef enum { ALLOP, ... , NEW_OPERATION,  NO_OPERATION} operations; 
  \end{verbatim} \normalsize 

\item Edit the {\tt block4/block4a.cc} file and add a new case to the
switch statement in the {\tt hb\_client::hb\_read\_message} function.  The
case should look something like this: \small \begin{verbatim}
    case NEW_OPERATION:
      // Does wonderful things
      read(readfd, &larg, LONGSIZE); // first parameter
      ...
      readsok(&nodeName, &nodeNameLength);   // string parameter
      ...
      readsok(&message, &messageLength);  // user message
      ret = block3.newOperation(clientnum, username, eventfd, larg,
                                   ...
                                   nodeName, nodeNameLength,
                                   ...
                                   message, messageLength);
      delete [] message;
      write(writefd, &ret, INTSIZE);
      if(ret == OK)
        {
          write(writefd, &nodeNum, INTSIZE); // any more information that
	                                     // needs to be sent to the user

        }	  
      if (debug == 1) // level one Control Flow Tracing 
        { 
	  cerr << username << "\t< NEW\_OPERATION " << form("%5d", arg1) << 
	    form("%5d", arg2) << ">" << form("%5d",ret) << " at " << currentTime(); 
        }  
      returnvalue = ret;      
      break;
\end{verbatim}
\normalsize

\item Edit the {\tt block3/MainEventLock.hh} file and add a new member
  function declaration to the EventLock class declaration.  The operation
  would look something like: \small
\begin{verbatim}  int newOperation(int userid,char* username, int eventfd,
                                   long parameterOne,...,char* string1, 
                                   long string1Length,..., char* message,
                                   long messageLength);
\end{verbatim}
\normalsize

\item Edit the {\tt block3/MainEventLock.cc} file and add the new member
function to the EventLock class.

\item Edit {\tt block4/main.cc} to change the version number of HBS.

\end{enumerate}

\subsection{Changing max node size}

The maximum size of any node is determined by the number of direct data
pointers and the data block size.  The formula to determine the maximum data
size is: $(b*s) + ((s/4) * s) + ((s/4) * (s/4) * s)$.  Where b is the
number of direct links, s is the size of the data block in bytes, and 4 is
the {\em sizeof(long)} on the machine running the HBS. See \cite{Wiil90a}
page 27 for a detailed description. (e.g. if the number of direct links is
8 and the data block size is 256 bytes then the maximum amount of data that
can be stored in a node is $(8 * 256) + ((256/4) * 256) + ((256/4) *
(256/4) * 256) = 1,067,008$ bytes)

If you wish to change the maximum datasize edit the {\tt
config/hb\_config.hh} file and change the following two lines:\small
\begin{verbatim}#define NUMBER_OF_BLOCK_PTR_IN_DATANODE 8
#define DATABLOCKSIZE 256
\end{verbatim}
\normalsize
You will have to recompile the entire HBS once you have changed the values.

\subsection{Changing number of users}

Changing the maximum number of users in the HBS is very similar to changing
the maximum datasize.  Edit the {\tt config/hb\_config.hh} file and change
the following line:

\small 
\begin{verbatim}
#define MAXUSERS 32
\end{verbatim}
\normalsize 

You will have to recompile the entire HBS once you have change the {\tt
MAXUSERS} value.

