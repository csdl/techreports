%%%%%%%%%%%%%%%%%%%%%%%%%%%%%% -*- Mode: Latex -*- %%%%%%%%%%%%%%%%%%%%%%%%%%%%
%% ftarm-guide.tex -- 
%% Author          : Danu Tjahjono
%% Created On      : Wed Sep 20 16:21:34 1995
%% Last Modified By: Danu Tjahjono
%% Last Modified On: Wed Jun 12 14:54:07 1996
%% RCS: $Id$
%%%%%%%%%%%%%%%%%%%%%%%%%%%%%%%%%%%%%%%%%%%%%%%%%%%%%%%%%%%%%%%%%%%%%%%%%%%%%%%
%%   Copyright (C) 1995 Danu Tjahjono
%%%%%%%%%%%%%%%%%%%%%%%%%%%%%%%%%%%%%%%%%%%%%%%%%%%%%%%%%%%%%%%%%%%%%%%%%%%%%%%
%% 

\documentstyle[nftimes,11pt,titlepage,/group/csdl/tex/lmacros,/group/csdl/tex/definemargins]{article}
% Psfig/TeX 
\def\PsfigVersion{1.9}
% dvips version
%
% All psfig/tex software, documentation, and related files
% in this distribution of psfig/tex are 
% Copyright 1987, 1988, 1991 Trevor J. Darrell
%
% Permission is granted for use and non-profit distribution of psfig/tex 
% providing that this notice is clearly maintained. The right to
% distribute any portion of psfig/tex for profit or as part of any commercial
% product is specifically reserved for the author(s) of that portion.
%
% *** Feel free to make local modifications of psfig as you wish,
% *** but DO NOT post any changed or modified versions of ``psfig''
% *** directly to the net. Send them to me and I'll try to incorporate
% *** them into future versions. If you want to take the psfig code 
% *** and make a new program (subject to the copyright above), distribute it, 
% *** (and maintain it) that's fine, just don't call it psfig.
%
% Bugs and improvements to trevor@media.mit.edu.
%
% Thanks to Greg Hager (GDH) and Ned Batchelder for their contributions
% to the original version of this project.
%
% Modified by J. Daniel Smith on 9 October 1990 to accept the
% %%BoundingBox: comment with or without a space after the colon.  Stole
% file reading code from Tom Rokicki's EPSF.TEX file (see below).
%
% More modifications by J. Daniel Smith on 29 March 1991 to allow the
% the included PostScript figure to be rotated.  The amount of
% rotation is specified by the "angle=" parameter of the \psfig command.
%
% Modified by Robert Russell on June 25, 1991 to allow users to specify
% .ps filenames which don't yet exist, provided they explicitly provide
% boundingbox information via the \psfig command. Note: This will only work
% if the "file=" parameter follows all four "bb???=" parameters in the
% command. This is due to the order in which psfig interprets these params.
%
%  3 Jul 1991	JDS	check if file already read in once
%  4 Sep 1991	JDS	fixed incorrect computation of rotated
%			bounding box
% 25 Sep 1991	GVR	expanded synopsis of \psfig
% 14 Oct 1991	JDS	\fbox code from LaTeX so \psdraft works with TeX
%			changed \typeout to \ps@typeout
% 17 Oct 1991	JDS	added \psscalefirst and \psrotatefirst
%

% From: gvr@cs.brown.edu (George V. Reilly)
%
% \psdraft	draws an outline box, but doesn't include the figure
%		in the DVI file.  Useful for previewing.
%
% \psfull	includes the figure in the DVI file (default).
%
% \psscalefirst width= or height= specifies the size of the figure
% 		before rotation.
% \psrotatefirst (default) width= or height= specifies the size of the
% 		 figure after rotation.  Asymetric figures will
% 		 appear to shrink.
%
% \psfigurepath#1	sets the path to search for the figure
%
% \psfig
% usage: \psfig{file=, figure=, height=, width=,
%			bbllx=, bblly=, bburx=, bbury=,
%			rheight=, rwidth=, clip=, angle=, silent=}
%
%	"file" is the filename.  If no path name is specified and the
%		file is not found in the current directory,
%		it will be looked for in directory \psfigurepath.
%	"figure" is a synonym for "file".
%	By default, the width and height of the figure are taken from
%		the BoundingBox of the figure.
%	If "width" is specified, the figure is scaled so that it has
%		the specified width.  Its height changes proportionately.
%	If "height" is specified, the figure is scaled so that it has
%		the specified height.  Its width changes proportionately.
%	If both "width" and "height" are specified, the figure is scaled
%		anamorphically.
%	"bbllx", "bblly", "bburx", and "bbury" control the PostScript
%		BoundingBox.  If these four values are specified
%               *before* the "file" option, the PSFIG will not try to
%               open the PostScript file.
%	"rheight" and "rwidth" are the reserved height and width
%		of the figure, i.e., how big TeX actually thinks
%		the figure is.  They default to "width" and "height".
%	The "clip" option ensures that no portion of the figure will
%		appear outside its BoundingBox.  "clip=" is a switch and
%		takes no value, but the `=' must be present.
%	The "angle" option specifies the angle of rotation (degrees, ccw).
%	The "silent" option makes \psfig work silently.
%

% check to see if macros already loaded in (maybe some other file says
% "\input psfig") ...
\ifx\undefined\psfig\else\endinput\fi

%
% from a suggestion by eijkhout@csrd.uiuc.edu to allow
% loading as a style file. Changed to avoid problems
% with amstex per suggestion by jbence@math.ucla.edu

\let\LaTeXAtSign=\@
\let\@=\relax
\edef\psfigRestoreAt{\catcode`\@=\number\catcode`@\relax}
%\edef\psfigRestoreAt{\catcode`@=\number\catcode`@\relax}
\catcode`\@=11\relax
\newwrite\@unused
\def\ps@typeout#1{{\let\protect\string\immediate\write\@unused{#1}}}
\ps@typeout{psfig/tex \PsfigVersion}

%% Here's how you define your figure path.  Should be set up with null
%% default and a user useable definition.

\def\figurepath{./}
\def\psfigurepath#1{\edef\figurepath{#1}}

%
% @psdo control structure -- similar to Latex @for.
% I redefined these with different names so that psfig can
% be used with TeX as well as LaTeX, and so that it will not 
% be vunerable to future changes in LaTeX's internal
% control structure,
%
\def\@nnil{\@nil}
\def\@empty{}
\def\@psdonoop#1\@@#2#3{}
\def\@psdo#1:=#2\do#3{\edef\@psdotmp{#2}\ifx\@psdotmp\@empty \else
    \expandafter\@psdoloop#2,\@nil,\@nil\@@#1{#3}\fi}
\def\@psdoloop#1,#2,#3\@@#4#5{\def#4{#1}\ifx #4\@nnil \else
       #5\def#4{#2}\ifx #4\@nnil \else#5\@ipsdoloop #3\@@#4{#5}\fi\fi}
\def\@ipsdoloop#1,#2\@@#3#4{\def#3{#1}\ifx #3\@nnil 
       \let\@nextwhile=\@psdonoop \else
      #4\relax\let\@nextwhile=\@ipsdoloop\fi\@nextwhile#2\@@#3{#4}}
\def\@tpsdo#1:=#2\do#3{\xdef\@psdotmp{#2}\ifx\@psdotmp\@empty \else
    \@tpsdoloop#2\@nil\@nil\@@#1{#3}\fi}
\def\@tpsdoloop#1#2\@@#3#4{\def#3{#1}\ifx #3\@nnil 
       \let\@nextwhile=\@psdonoop \else
      #4\relax\let\@nextwhile=\@tpsdoloop\fi\@nextwhile#2\@@#3{#4}}
% 
% \fbox is defined in latex.tex; so if \fbox is undefined, assume that
% we are not in LaTeX.
% Perhaps this could be done better???
\ifx\undefined\fbox
% \fbox code from modified slightly from LaTeX
\newdimen\fboxrule
\newdimen\fboxsep
\newdimen\ps@tempdima
\newbox\ps@tempboxa
\fboxsep = 3pt
\fboxrule = .4pt
\long\def\fbox#1{\leavevmode\setbox\ps@tempboxa\hbox{#1}\ps@tempdima\fboxrule
    \advance\ps@tempdima \fboxsep \advance\ps@tempdima \dp\ps@tempboxa
   \hbox{\lower \ps@tempdima\hbox
  {\vbox{\hrule height \fboxrule
          \hbox{\vrule width \fboxrule \hskip\fboxsep
          \vbox{\vskip\fboxsep \box\ps@tempboxa\vskip\fboxsep}\hskip 
                 \fboxsep\vrule width \fboxrule}
                 \hrule height \fboxrule}}}}
\fi
%
%%%%%%%%%%%%%%%%%%%%%%%%%%%%%%%%%%%%%%%%%%%%%%%%%%%%%%%%%%%%%%%%%%%
% file reading stuff from epsf.tex
%   EPSF.TEX macro file:
%   Written by Tomas Rokicki of Radical Eye Software, 29 Mar 1989.
%   Revised by Don Knuth, 3 Jan 1990.
%   Revised by Tomas Rokicki to accept bounding boxes with no
%      space after the colon, 18 Jul 1990.
%   Portions modified/removed for use in PSFIG package by
%      J. Daniel Smith, 9 October 1990.
%
\newread\ps@stream
\newif\ifnot@eof       % continue looking for the bounding box?
\newif\if@noisy        % report what you're making?
\newif\if@atend        % %%BoundingBox: has (at end) specification
\newif\if@psfile       % does this look like a PostScript file?
%
% PostScript files should start with `%!'
%
{\catcode`\%=12\global\gdef\epsf@start{%!}}
\def\epsf@PS{PS}
%
\def\epsf@getbb#1{%
%
%   The first thing we need to do is to open the
%   PostScript file, if possible.
%
\openin\ps@stream=#1
\ifeof\ps@stream\ps@typeout{Error, File #1 not found}\else
%
%   Okay, we got it. Now we'll scan lines until we find one that doesn't
%   start with %. We're looking for the bounding box comment.
%
   {\not@eoftrue \chardef\other=12
    \def\do##1{\catcode`##1=\other}\dospecials \catcode`\ =10
    \loop
       \if@psfile
	  \read\ps@stream to \epsf@fileline
       \else{
	  \obeyspaces
          \read\ps@stream to \epsf@tmp\global\let\epsf@fileline\epsf@tmp}
       \fi
       \ifeof\ps@stream\not@eoffalse\else
%
%   Check the first line for `%!'.  Issue a warning message if its not
%   there, since the file might not be a PostScript file.
%
       \if@psfile\else
       \expandafter\epsf@test\epsf@fileline:. \\%
       \fi
%
%   We check to see if the first character is a % sign;
%   if so, we look further and stop only if the line begins with
%   `%%BoundingBox:' and the `(atend)' specification was not found.
%   That is, the only way to stop is when the end of file is reached,
%   or a `%%BoundingBox: llx lly urx ury' line is found.
%
          \expandafter\epsf@aux\epsf@fileline:. \\%
       \fi
   \ifnot@eof\repeat
   }\closein\ps@stream\fi}%
%
% This tests if the file we are reading looks like a PostScript file.
%
\long\def\epsf@test#1#2#3:#4\\{\def\epsf@testit{#1#2}
			\ifx\epsf@testit\epsf@start\else
\ps@typeout{Warning! File does not start with `\epsf@start'.  It may not be a PostScript file.}
			\fi
			\@psfiletrue} % don't test after 1st line
%
%   We still need to define the tricky \epsf@aux macro. This requires
%   a couple of magic constants for comparison purposes.
%
{\catcode`\%=12\global\let\epsf@percent=%\global\def\epsf@bblit{%BoundingBox}}
%
%
%   So we're ready to check for `%BoundingBox:' and to grab the
%   values if they are found.  We continue searching if `(at end)'
%   was found after the `%BoundingBox:'.
%
\long\def\epsf@aux#1#2:#3\\{\ifx#1\epsf@percent
   \def\epsf@testit{#2}\ifx\epsf@testit\epsf@bblit
	\@atendfalse
        \epsf@atend #3 . \\%
	\if@atend	
	   \if@verbose{
		\ps@typeout{psfig: found `(atend)'; continuing search}
	   }\fi
        \else
        \epsf@grab #3 . . . \\%
        \not@eoffalse
        \global\no@bbfalse
        \fi
   \fi\fi}%
%
%   Here we grab the values and stuff them in the appropriate definitions.
%
\def\epsf@grab #1 #2 #3 #4 #5\\{%
   \global\def\epsf@llx{#1}\ifx\epsf@llx\empty
      \epsf@grab #2 #3 #4 #5 .\\\else
   \global\def\epsf@lly{#2}%
   \global\def\epsf@urx{#3}\global\def\epsf@ury{#4}\fi}%
%
% Determine if the stuff following the %%BoundingBox is `(atend)'
% J. Daniel Smith.  Copied from \epsf@grab above.
%
\def\epsf@atendlit{(atend)} 
\def\epsf@atend #1 #2 #3\\{%
   \def\epsf@tmp{#1}\ifx\epsf@tmp\empty
      \epsf@atend #2 #3 .\\\else
   \ifx\epsf@tmp\epsf@atendlit\@atendtrue\fi\fi}


% End of file reading stuff from epsf.tex
%%%%%%%%%%%%%%%%%%%%%%%%%%%%%%%%%%%%%%%%%%%%%%%%%%%%%%%%%%%%%%%%%%%

%%%%%%%%%%%%%%%%%%%%%%%%%%%%%%%%%%%%%%%%%%%%%%%%%%%%%%%%%%%%%%%%%%%
% trigonometry stuff from "trig.tex"
\chardef\psletter = 11 % won't conflict with \begin{letter} now...
\chardef\other = 12

\newif \ifdebug %%% turn me on to see TeX hard at work ...
\newif\ifc@mpute %%% don't need to compute some values
\c@mputetrue % but assume that we do

\let\then = \relax
\def\r@dian{pt }
\let\r@dians = \r@dian
\let\dimensionless@nit = \r@dian
\let\dimensionless@nits = \dimensionless@nit
\def\internal@nit{sp }
\let\internal@nits = \internal@nit
\newif\ifstillc@nverging
\def \Mess@ge #1{\ifdebug \then \message {#1} \fi}

{ %%% Things that need abnormal catcodes %%%
	\catcode `\@ = \psletter
	\gdef \nodimen {\expandafter \n@dimen \the \dimen}
	\gdef \term #1 #2 #3%
	       {\edef \t@ {\the #1}%%% freeze parameter 1 (count, by value)
		\edef \t@@ {\expandafter \n@dimen \the #2\r@dian}%
				   %%% freeze parameter 2 (dimen, by value)
		\t@rm {\t@} {\t@@} {#3}%
	       }
	\gdef \t@rm #1 #2 #3%
	       {{%
		\count 0 = 0
		\dimen 0 = 1 \dimensionless@nit
		\dimen 2 = #2\relax
		\Mess@ge {Calculating term #1 of \nodimen 2}%
		\loop
		\ifnum	\count 0 < #1
		\then	\advance \count 0 by 1
			\Mess@ge {Iteration \the \count 0 \space}%
			\Multiply \dimen 0 by {\dimen 2}%
			\Mess@ge {After multiplication, term = \nodimen 0}%
			\Divide \dimen 0 by {\count 0}%
			\Mess@ge {After division, term = \nodimen 0}%
		\repeat
		\Mess@ge {Final value for term #1 of 
				\nodimen 2 \space is \nodimen 0}%
		\xdef \Term {#3 = \nodimen 0 \r@dians}%
		\aftergroup \Term
	       }}
	\catcode `\p = \other
	\catcode `\t = \other
	\gdef \n@dimen #1pt{#1} %%% throw away the ``pt''
}

\def \Divide #1by #2{\divide #1 by #2} %%% just a synonym

\def \Multiply #1by #2%%% allows division of a dimen by a dimen
       {{%%% should really freeze parameter 2 (dimen, passed by value)
	\count 0 = #1\relax
	\count 2 = #2\relax
	\count 4 = 65536
	\Mess@ge {Before scaling, count 0 = \the \count 0 \space and
			count 2 = \the \count 2}%
	\ifnum	\count 0 > 32767 %%% do our best to avoid overflow
	\then	\divide \count 0 by 4
		\divide \count 4 by 4
	\else	\ifnum	\count 0 < -32767
		\then	\divide \count 0 by 4
			\divide \count 4 by 4
		\else
		\fi
	\fi
	\ifnum	\count 2 > 32767 %%% while retaining reasonable accuracy
	\then	\divide \count 2 by 4
		\divide \count 4 by 4
	\else	\ifnum	\count 2 < -32767
		\then	\divide \count 2 by 4
			\divide \count 4 by 4
		\else
		\fi
	\fi
	\multiply \count 0 by \count 2
	\divide \count 0 by \count 4
	\xdef \product {#1 = \the \count 0 \internal@nits}%
	\aftergroup \product
       }}

\def\r@duce{\ifdim\dimen0 > 90\r@dian \then   % sin(x+90) = sin(180-x)
		\multiply\dimen0 by -1
		\advance\dimen0 by 180\r@dian
		\r@duce
	    \else \ifdim\dimen0 < -90\r@dian \then  % sin(-x) = sin(360+x)
		\advance\dimen0 by 360\r@dian
		\r@duce
		\fi
	    \fi}

\def\Sine#1%
       {{%
	\dimen 0 = #1 \r@dian
	\r@duce
	\ifdim\dimen0 = -90\r@dian \then
	   \dimen4 = -1\r@dian
	   \c@mputefalse
	\fi
	\ifdim\dimen0 = 90\r@dian \then
	   \dimen4 = 1\r@dian
	   \c@mputefalse
	\fi
	\ifdim\dimen0 = 0\r@dian \then
	   \dimen4 = 0\r@dian
	   \c@mputefalse
	\fi
%
	\ifc@mpute \then
        	% convert degrees to radians
		\divide\dimen0 by 180
		\dimen0=3.141592654\dimen0
%
		\dimen 2 = 3.1415926535897963\r@dian %%% a well-known constant
		\divide\dimen 2 by 2 %%% we only deal with -pi/2 : pi/2
		\Mess@ge {Sin: calculating Sin of \nodimen 0}%
		\count 0 = 1 %%% see power-series expansion for sine
		\dimen 2 = 1 \r@dian %%% ditto
		\dimen 4 = 0 \r@dian %%% ditto
		\loop
			\ifnum	\dimen 2 = 0 %%% then we've done
			\then	\stillc@nvergingfalse 
			\else	\stillc@nvergingtrue
			\fi
			\ifstillc@nverging %%% then calculate next term
			\then	\term {\count 0} {\dimen 0} {\dimen 2}%
				\advance \count 0 by 2
				\count 2 = \count 0
				\divide \count 2 by 2
				\ifodd	\count 2 %%% signs alternate
				\then	\advance \dimen 4 by \dimen 2
				\else	\advance \dimen 4 by -\dimen 2
				\fi
		\repeat
	\fi		
			\xdef \sine {\nodimen 4}%
       }}

% Now the Cosine can be calculated easily by calling \Sine
\def\Cosine#1{\ifx\sine\UnDefined\edef\Savesine{\relax}\else
		             \edef\Savesine{\sine}\fi
	{\dimen0=#1\r@dian\advance\dimen0 by 90\r@dian
	 \Sine{\nodimen 0}
	 \xdef\cosine{\sine}
	 \xdef\sine{\Savesine}}}	      
% end of trig stuff
%%%%%%%%%%%%%%%%%%%%%%%%%%%%%%%%%%%%%%%%%%%%%%%%%%%%%%%%%%%%%%%%%%%%

\def\psdraft{
	\def\@psdraft{0}
	%\ps@typeout{draft level now is \@psdraft \space . }
}
\def\psfull{
	\def\@psdraft{100}
	%\ps@typeout{draft level now is \@psdraft \space . }
}

\psfull

\newif\if@scalefirst
\def\psscalefirst{\@scalefirsttrue}
\def\psrotatefirst{\@scalefirstfalse}
\psrotatefirst

\newif\if@draftbox
\def\psnodraftbox{
	\@draftboxfalse
}
\def\psdraftbox{
	\@draftboxtrue
}
\@draftboxtrue

\newif\if@prologfile
\newif\if@postlogfile
\def\pssilent{
	\@noisyfalse
}
\def\psnoisy{
	\@noisytrue
}
\psnoisy
%%% These are for the option list.
%%% A specification of the form a = b maps to calling \@p@@sa{b}
\newif\if@bbllx
\newif\if@bblly
\newif\if@bburx
\newif\if@bbury
\newif\if@height
\newif\if@width
\newif\if@rheight
\newif\if@rwidth
\newif\if@angle
\newif\if@clip
\newif\if@verbose
\def\@p@@sclip#1{\@cliptrue}


\newif\if@decmpr

%%% GDH 7/26/87 -- changed so that it first looks in the local directory,
%%% then in a specified global directory for the ps file.
%%% RPR 6/25/91 -- changed so that it defaults to user-supplied name if
%%% boundingbox info is specified, assuming graphic will be created by
%%% print time.
%%% TJD 10/19/91 -- added bbfile vs. file distinction, and @decmpr flag

\def\@p@@sfigure#1{\def\@p@sfile{null}\def\@p@sbbfile{null}
	        \openin1=#1.bb
		\ifeof1\closein1
	        	\openin1=\figurepath#1.bb
			\ifeof1\closein1
			        \openin1=#1
				\ifeof1\closein1%
				       \openin1=\figurepath#1
					\ifeof1
					   \ps@typeout{Error, File #1 not found}
						\if@bbllx\if@bblly
				   		\if@bburx\if@bbury
			      				\def\@p@sfile{#1}%
			      				\def\@p@sbbfile{#1}%
							\@decmprfalse
				  	   	\fi\fi\fi\fi
					\else\closein1
				    		\def\@p@sfile{\figurepath#1}%
				    		\def\@p@sbbfile{\figurepath#1}%
						\@decmprfalse
	                       		\fi%
			 	\else\closein1%
					\def\@p@sfile{#1}
					\def\@p@sbbfile{#1}
					\@decmprfalse
			 	\fi
			\else
				\def\@p@sfile{\figurepath#1}
				\def\@p@sbbfile{\figurepath#1.bb}
				\@decmprtrue
			\fi
		\else
			\def\@p@sfile{#1}
			\def\@p@sbbfile{#1.bb}
			\@decmprtrue
		\fi}

\def\@p@@sfile#1{\@p@@sfigure{#1}}

\def\@p@@sbbllx#1{
		%\ps@typeout{bbllx is #1}
		\@bbllxtrue
		\dimen100=#1
		\edef\@p@sbbllx{\number\dimen100}
}
\def\@p@@sbblly#1{
		%\ps@typeout{bblly is #1}
		\@bbllytrue
		\dimen100=#1
		\edef\@p@sbblly{\number\dimen100}
}
\def\@p@@sbburx#1{
		%\ps@typeout{bburx is #1}
		\@bburxtrue
		\dimen100=#1
		\edef\@p@sbburx{\number\dimen100}
}
\def\@p@@sbbury#1{
		%\ps@typeout{bbury is #1}
		\@bburytrue
		\dimen100=#1
		\edef\@p@sbbury{\number\dimen100}
}
\def\@p@@sheight#1{
		\@heighttrue
		\dimen100=#1
   		\edef\@p@sheight{\number\dimen100}
		%\ps@typeout{Height is \@p@sheight}
}
\def\@p@@swidth#1{
		%\ps@typeout{Width is #1}
		\@widthtrue
		\dimen100=#1
		\edef\@p@swidth{\number\dimen100}
}
\def\@p@@srheight#1{
		%\ps@typeout{Reserved height is #1}
		\@rheighttrue
		\dimen100=#1
		\edef\@p@srheight{\number\dimen100}
}
\def\@p@@srwidth#1{
		%\ps@typeout{Reserved width is #1}
		\@rwidthtrue
		\dimen100=#1
		\edef\@p@srwidth{\number\dimen100}
}
\def\@p@@sangle#1{
		%\ps@typeout{Rotation is #1}
		\@angletrue
%		\dimen100=#1
		\edef\@p@sangle{#1} %\number\dimen100}
}
\def\@p@@ssilent#1{ 
		\@verbosefalse
}
\def\@p@@sprolog#1{\@prologfiletrue\def\@prologfileval{#1}}
\def\@p@@spostlog#1{\@postlogfiletrue\def\@postlogfileval{#1}}
\def\@cs@name#1{\csname #1\endcsname}
\def\@setparms#1=#2,{\@cs@name{@p@@s#1}{#2}}
%
% initialize the defaults (size the size of the figure)
%
\def\ps@init@parms{
		\@bbllxfalse \@bbllyfalse
		\@bburxfalse \@bburyfalse
		\@heightfalse \@widthfalse
		\@rheightfalse \@rwidthfalse
		\def\@p@sbbllx{}\def\@p@sbblly{}
		\def\@p@sbburx{}\def\@p@sbbury{}
		\def\@p@sheight{}\def\@p@swidth{}
		\def\@p@srheight{}\def\@p@srwidth{}
		\def\@p@sangle{0}
		\def\@p@sfile{} \def\@p@sbbfile{}
		\def\@p@scost{10}
		\def\@sc{}
		\@prologfilefalse
		\@postlogfilefalse
		\@clipfalse
		\if@noisy
			\@verbosetrue
		\else
			\@verbosefalse
		\fi
}
%
% Go through the options setting things up.
%
\def\parse@ps@parms#1{
	 	\@psdo\@psfiga:=#1\do
		   {\expandafter\@setparms\@psfiga,}}
%
% Compute bb height and width
%
\newif\ifno@bb
\def\bb@missing{
	\if@verbose{
		\ps@typeout{psfig: searching \@p@sbbfile \space  for bounding box}
	}\fi
	\no@bbtrue
	\epsf@getbb{\@p@sbbfile}
        \ifno@bb \else \bb@cull\epsf@llx\epsf@lly\epsf@urx\epsf@ury\fi
}	
\def\bb@cull#1#2#3#4{
	\dimen100=#1 bp\edef\@p@sbbllx{\number\dimen100}
	\dimen100=#2 bp\edef\@p@sbblly{\number\dimen100}
	\dimen100=#3 bp\edef\@p@sbburx{\number\dimen100}
	\dimen100=#4 bp\edef\@p@sbbury{\number\dimen100}
	\no@bbfalse
}
% rotate point (#1,#2) about (0,0).
% The sine and cosine of the angle are already stored in \sine and
% \cosine.  The result is placed in (\p@intvaluex, \p@intvaluey).
\newdimen\p@intvaluex
\newdimen\p@intvaluey
\def\rotate@#1#2{{\dimen0=#1 sp\dimen1=#2 sp
%            	calculate x' = x \cos\theta - y \sin\theta
		  \global\p@intvaluex=\cosine\dimen0
		  \dimen3=\sine\dimen1
		  \global\advance\p@intvaluex by -\dimen3
% 		calculate y' = x \sin\theta + y \cos\theta
		  \global\p@intvaluey=\sine\dimen0
		  \dimen3=\cosine\dimen1
		  \global\advance\p@intvaluey by \dimen3
		  }}
\def\compute@bb{
		\no@bbfalse
		\if@bbllx \else \no@bbtrue \fi
		\if@bblly \else \no@bbtrue \fi
		\if@bburx \else \no@bbtrue \fi
		\if@bbury \else \no@bbtrue \fi
		\ifno@bb \bb@missing \fi
		\ifno@bb \ps@typeout{FATAL ERROR: no bb supplied or found}
			\no-bb-error
		\fi
		%
%\ps@typeout{BB: \@p@sbbllx, \@p@sbblly, \@p@sbburx, \@p@sbbury} 
%
% store height/width of original (unrotated) bounding box
		\count203=\@p@sbburx
		\count204=\@p@sbbury
		\advance\count203 by -\@p@sbbllx
		\advance\count204 by -\@p@sbblly
		\edef\ps@bbw{\number\count203}
		\edef\ps@bbh{\number\count204}
		%\ps@typeout{ psbbh = \ps@bbh, psbbw = \ps@bbw }
		\if@angle 
			\Sine{\@p@sangle}\Cosine{\@p@sangle}
	        	{\dimen100=\maxdimen\xdef\r@p@sbbllx{\number\dimen100}
					    \xdef\r@p@sbblly{\number\dimen100}
			                    \xdef\r@p@sbburx{-\number\dimen100}
					    \xdef\r@p@sbbury{-\number\dimen100}}
%
% Need to rotate all four points and take the X-Y extremes of the new
% points as the new bounding box.
                        \def\minmaxtest{
			   \ifnum\number\p@intvaluex<\r@p@sbbllx
			      \xdef\r@p@sbbllx{\number\p@intvaluex}\fi
			   \ifnum\number\p@intvaluex>\r@p@sbburx
			      \xdef\r@p@sbburx{\number\p@intvaluex}\fi
			   \ifnum\number\p@intvaluey<\r@p@sbblly
			      \xdef\r@p@sbblly{\number\p@intvaluey}\fi
			   \ifnum\number\p@intvaluey>\r@p@sbbury
			      \xdef\r@p@sbbury{\number\p@intvaluey}\fi
			   }
%			lower left
			\rotate@{\@p@sbbllx}{\@p@sbblly}
			\minmaxtest
%			upper left
			\rotate@{\@p@sbbllx}{\@p@sbbury}
			\minmaxtest
%			lower right
			\rotate@{\@p@sbburx}{\@p@sbblly}
			\minmaxtest
%			upper right
			\rotate@{\@p@sbburx}{\@p@sbbury}
			\minmaxtest
			\edef\@p@sbbllx{\r@p@sbbllx}\edef\@p@sbblly{\r@p@sbblly}
			\edef\@p@sbburx{\r@p@sbburx}\edef\@p@sbbury{\r@p@sbbury}
%\ps@typeout{rotated BB: \r@p@sbbllx, \r@p@sbblly, \r@p@sbburx, \r@p@sbbury}
		\fi
		\count203=\@p@sbburx
		\count204=\@p@sbbury
		\advance\count203 by -\@p@sbbllx
		\advance\count204 by -\@p@sbblly
		\edef\@bbw{\number\count203}
		\edef\@bbh{\number\count204}
		%\ps@typeout{ bbh = \@bbh, bbw = \@bbw }
}
%
% \in@hundreds performs #1 * (#2 / #3) correct to the hundreds,
%	then leaves the result in @result
%
\def\in@hundreds#1#2#3{\count240=#2 \count241=#3
		     \count100=\count240	% 100 is first digit #2/#3
		     \divide\count100 by \count241
		     \count101=\count100
		     \multiply\count101 by \count241
		     \advance\count240 by -\count101
		     \multiply\count240 by 10
		     \count101=\count240	%101 is second digit of #2/#3
		     \divide\count101 by \count241
		     \count102=\count101
		     \multiply\count102 by \count241
		     \advance\count240 by -\count102
		     \multiply\count240 by 10
		     \count102=\count240	% 102 is the third digit
		     \divide\count102 by \count241
		     \count200=#1\count205=0
		     \count201=\count200
			\multiply\count201 by \count100
		 	\advance\count205 by \count201
		     \count201=\count200
			\divide\count201 by 10
			\multiply\count201 by \count101
			\advance\count205 by \count201
			%
		     \count201=\count200
			\divide\count201 by 100
			\multiply\count201 by \count102
			\advance\count205 by \count201
			%
		     \edef\@result{\number\count205}
}
\def\compute@wfromh{
		% computing : width = height * (bbw / bbh)
		\in@hundreds{\@p@sheight}{\@bbw}{\@bbh}
		%\ps@typeout{ \@p@sheight * \@bbw / \@bbh, = \@result }
		\edef\@p@swidth{\@result}
		%\ps@typeout{w from h: width is \@p@swidth}
}
\def\compute@hfromw{
		% computing : height = width * (bbh / bbw)
	        \in@hundreds{\@p@swidth}{\@bbh}{\@bbw}
		%\ps@typeout{ \@p@swidth * \@bbh / \@bbw = \@result }
		\edef\@p@sheight{\@result}
		%\ps@typeout{h from w : height is \@p@sheight}
}
\def\compute@handw{
		\if@height 
			\if@width
			\else
				\compute@wfromh
			\fi
		\else 
			\if@width
				\compute@hfromw
			\else
				\edef\@p@sheight{\@bbh}
				\edef\@p@swidth{\@bbw}
			\fi
		\fi
}
\def\compute@resv{
		\if@rheight \else \edef\@p@srheight{\@p@sheight} \fi
		\if@rwidth \else \edef\@p@srwidth{\@p@swidth} \fi
		%\ps@typeout{rheight = \@p@srheight, rwidth = \@p@srwidth}
}
%		
% Compute any missing values
\def\compute@sizes{
	\compute@bb
	\if@scalefirst\if@angle
% at this point the bounding box has been adjsuted correctly for
% rotation.  PSFIG does all of its scaling using \@bbh and \@bbw.  If
% a width= or height= was specified along with \psscalefirst, then the
% width=/height= value needs to be adjusted to match the new (rotated)
% bounding box size (specifed in \@bbw and \@bbh).
%    \ps@bbw       width=
%    -------  =  ---------- 
%    \@bbw       new width=
% so `new width=' = (width= * \@bbw) / \ps@bbw; where \ps@bbw is the
% width of the original (unrotated) bounding box.
	\if@width
	   \in@hundreds{\@p@swidth}{\@bbw}{\ps@bbw}
	   \edef\@p@swidth{\@result}
	\fi
	\if@height
	   \in@hundreds{\@p@sheight}{\@bbh}{\ps@bbh}
	   \edef\@p@sheight{\@result}
	\fi
	\fi\fi
	\compute@handw
	\compute@resv}

%
% \psfig
% usage : \psfig{file=, height=, width=, bbllx=, bblly=, bburx=, bbury=,
%			rheight=, rwidth=, clip=}
%
% "clip=" is a switch and takes no value, but the `=' must be present.
\def\psfig#1{\vbox {
	% do a zero width hard space so that a single
	% \psfig in a centering enviornment will behave nicely
	%{\setbox0=\hbox{\ }\ \hskip-\wd0}
	%
	\ps@init@parms
	\parse@ps@parms{#1}
	\compute@sizes
	%
	\ifnum\@p@scost<\@psdraft{
		%
		\special{ps::[begin] 	\@p@swidth \space \@p@sheight \space
				\@p@sbbllx \space \@p@sbblly \space
				\@p@sbburx \space \@p@sbbury \space
				startTexFig \space }
		\if@angle
			\special {ps:: \@p@sangle \space rotate \space} 
		\fi
		\if@clip{
			\if@verbose{
				\ps@typeout{(clip)}
			}\fi
			\special{ps:: doclip \space }
		}\fi
		\if@prologfile
		    \special{ps: plotfile \@prologfileval \space } \fi
		\if@decmpr{
			\if@verbose{
				\ps@typeout{psfig: including \@p@sfile.Z \space }
			}\fi
			\special{ps: plotfile "`zcat \@p@sfile.Z" \space }
		}\else{
			\if@verbose{
				\ps@typeout{psfig: including \@p@sfile \space }
			}\fi
			\special{ps: plotfile \@p@sfile \space }
		}\fi
		\if@postlogfile
		    \special{ps: plotfile \@postlogfileval \space } \fi
		\special{ps::[end] endTexFig \space }
		% Create the vbox to reserve the space for the figure.
		\vbox to \@p@srheight sp{
		% 1/92 TJD Changed from "true sp" to "sp" for magnification.
			\hbox to \@p@srwidth sp{
				\hss
			}
		\vss
		}
	}\else{
		% draft figure, just reserve the space and print the
		% path name.
		\if@draftbox{		
			% Verbose draft: print file name in box
			\hbox{\frame{\vbox to \@p@srheight sp{
			\vss
			\hbox to \@p@srwidth sp{ \hss \@p@sfile \hss }
			\vss
			}}}
		}\else{
			% Non-verbose draft
			\vbox to \@p@srheight sp{
			\vss
			\hbox to \@p@srwidth sp{\hss}
			\vss
			}
		}\fi	



	}\fi
}}
\psfigRestoreAt
\let\@=\LaTeXAtSign





\begin{document}
\title{FTArm User's Guide \\
{\normalsize (Version 1.2.0)}}
\author 
{Danu Tjahjono \\
 (dat@hawaii.edu) \\
 \\
Philip Johnson \\
(johnson@hawaii.edu)\\
\\
Collaborative Software Development Laboratory\\
Department of Information and Computer Sciences\\
University of Hawaii at Manoa\\
\\
CSDL-TR-95-18}
\date {October, 1995}  %%\today
\maketitle
\newpage
\tableofcontents
\newpage

\section {Introduction}

\subsection {Motivation for FTArm}
The primary goal of formal technical review (FTR) is to improve
software quality. FTR has been shown to have unique advantages for
improving software quality, since it can be applied early on in the
development process, it can be applied to all artifacts of
development, it improves the quality of both the developed product and
the developer, and it disperses knowledge of the application domain
and development practices across the organization.

FTR is typically manual in nature: participants mark up hard copies of
the development artifact by hand, then attend a group meeting where
the artifact is reviewed and issues are raised. The ultimate product
of FTR is a written document that notes the issues raised and
specifies the result of review, such as an evaluation of the quality
of the artifact, and/or a specification of rework tasks. All manual
FTR approaches emphasize one point: the purpose of FTR is to raise
issues and discover defects, not to resolve them.

Such approaches to FTR have well documented problems, such as
insufficient review preparation, failure of the reviewer or scribe to
adequately document the issues, failure of the moderator to run the
group meeting effectively, and digression away from the stated purpose
of review. These problems are significant when the cost of review is
taken into account: studies have shown that review of a 20 KLOC
program requires an entire man-year of effort by skilled technical
staff. 

FTArm (Formal, Technical, Asynchronous review method)
implements a computer-supported approach to FTR that differs from
most manual review approaches in the following major ways. First,
virtually all review activities are done on-line with automated
support: from initial document preparation to generation of final
report. Second, practically all review is done asynchronously: review
participants work individually at whatever times and places they
choose. The importance of the group meeting is de-emphasized and can
often be eliminated entirely. Third, fine-grained measurements of the
process and products of review are made by the environment, which can
support improvement in the quality of the review process
itself. Fourth, FTArm liberates the review process from the limitations
of the ``raise issues, don't resolve them'' constraint: stylistic
issues, requirement issues, design rationale, and rework approaches
can all be efficiently addressed with the FTArm framework. 

Despite these changes, FTArm attempts to retain and even enhance the
principle power of FTR, which derives from the unique capabilities of
a group to analyze, understand, and improve the quality of software
artifacts. 


\subsection {Historical background}

The current version of FTArm is generated from a generic review system
called CSRS (Collaborative Software Review System) version 3. 

FTArm was initially released as CSRS (Collaborative Software Review
System) version 1 in 1993. The feedback we received from the users
was generally positive. However, we also realized that the system was
constrained to a specific review method (FTArm). Customization to the
system cannot be made easily.
In 1994, we redesigned CSRS to resolve this problem.  

The current version of CSRS (version 3.x) no longer supports one specific
review method. 
Instead, it serves as a generic application program for building FTR systems.
It provides a set of language constructs for defining a review method
along with computer support.
The application developers simply
describe the method using the language constructs, and then compile
the constructs to generate the running system.
With this approach, the programming efforts associated with
implementing a new FTR system can be kept minimal.

The current version of FTArm was generated from CSRS with 2,500 lines
of code. However, only half of them involves real programming code, the
rest are written as language descriptions. 

\subsection {Scope of this manual}
This manual provides a description of FTArm system for
review participants and the administrator. It includes the
descriptions of 
how to execute the review process, what review
artifacts are involved, and the associated user commands to
manipulate the artifacts and the process. 

In this manual, we only describe the typical users commands or
operations to run the system. Many other commands can be discovered
throughout the use of the system. 

Finally, this manual does not describe customization of the
system. The developer should refer to CSRS design document to make
any customization. 

\subsection {Feedback on FTArm}
We would greatly appreciated any feedback you care to give us about
the FTArm system, this manual, and the FTArm process. Please send it
to dat@uhics.ics.hawaii.edu, or johnson@hawaii.edu.

\subsection {Organization of this manual}
The next section of this manual overviews the FTArm review
process: the sequence of activities performed during review, and the
outcomes of each phase.
The following section presents general conventions followed in the
FTArm/CSRS user interface. The following sections present detailed
instructions for review participants for their principle tasks during
each of the FTArm phases. Detailed descriptions of FTArm nodes
and their fields structure are presented next.
Finally, we conclude with the descriptions of other FTArm commands. 

\section {Overview of the FTArm Review Process}
FTR using FTArm involves seven phases.
Each phase constrains review by specifying the focus of
review activities, the visibility of nodes and links, and the kinds of
node and link manipulations. 
The next paragraphs overview these phases. Upcoming sections provide
detailed instructions on how to carry out reviewer-related activities
in each of these phases.
\begin{enumerate}
\item {\bf Setup.} In this phase the administrator/moderator decides
upon the composition of the review team and the artifacts to be reviewed.
The administrator then sets up the review database with these
information. 

\item {\bf Orientation.} 
This phase prepares the reviewers for private review. 
The producer explains the source artifacts to be reviewed in a
face-to-face group meeting.
The moderator may also use this phase to introduce the review process
itself to the participants.
The moderator may also decide to skip this phase when all
participants have been familiar with the source artifacts as well as
the review process.
\item {\bf Private-review.}
In this phase, reviewers inspect source artifacts privately and create
issue, action and/or comment nodes. Issue and action nodes are not
publicly available to other reviewers at this time, but comment nodes
are publicly available. Comment nodes allow reviewers to request
clarification about the source nodes, and also contain answers to
these questions by other reviewers, the moderator, and/or the
producer. 

Reviewers explicitly mark each source node as reviewed when
finished.  While reviewers do not have access to each other's state
during private review, the moderator does. This allows the moderator
to monitor the progress of private review. 

Private review normally terminates when all reviewers have marked all
source nodes as reviewed. In the unlikely event that no issues have
been raised, the review would terminate at this point.
If some issues have been raised, public review begins.

\item{\bf Public-review.}
In this phase, all nodes are made public, and all review participants
(including the producer) react to the issues and actions by voting and
creating new nodes. 
Participants can create new issue, action, or comment nodes based upon
existing nodes. Voting is used to determine the degree of agreement
within the group about the validity and implications of issues and
actions. This phase normally concludes when all issue, action, and
evidence nodes have been marked as reviewed by all reviewers.

\item {\bf Consolidation.} 
In this phase, the moderator consolidates and analyzes the results of
public review. Similar issues and related actions and comments are
grouped together into consolidated issues. All differences in opinion
are highlighted and summarized. Rework decisions are issued for the
issues that get the majority consensus. Other issues are declared
unresolved, and will be brought to group meeting.

\item {\bf Group review meeting.}
If the consolidation phase identifies issues or actions that were not
successfully resolved via public and private review, a group meeting
is required.
In the meeting, the moderator presents the unresolved issues or
actions and summarizes the differences of opinion. After discussion,
the group may vote to decide them, or the moderator may unilaterally
make the decision. 

\item {\bf Conclusion/External Development.}
In this phase, the moderator makes final updates to CSRS database,
noting the decision reached during the group meeting.
The moderator also generates a final hardcopy document/report
representing the product of review as well as the report of review
metrics.  These reports are then used to enhance or correct the
software product.

\end{enumerate}

\section {Concepts of the FTArm user interface}
This section introduces concepts and features of the FTArm/CSRS user
interface. The underlying interface is based on XEmacs user interface.

\subsection{General Terminology}
Several terms with varying interpretations are used repeatedly within
this document, and to avoid confusion, we provide definitions here.

\begin{description}
\item [Screen.] We use ``screen'' to refer to a single XEmacs display
unit or frame. The workstation monitor can display multiple screens at any
particular time. All CSRS screens for a single user are owned by a
single XEmacs process. 
 
The term ``screen'' is the same as ``window'' in conventional
sense. However, we use ``window'' for different object.

\item [Window.] We use ``window'' to refer to display unit within a
screen. A screen may be split into multiple windows.
In most cases, however, CSRS screen displays only one window.

\item [Buffer.] A ``buffer'' is an XEmacs data structure to display
and modify a CSRS node. A CSRS buffer may be hidden, but once 
displayed, it is always associated with one window and one screen.

\item [Node.] A ``node'' is the review artifact stored in the
database. A node consists of a set of fields. Each field starts with a
field label followed by colon(:). A field can be read-only or editable. 
An editable field is filled out by simply typing the text.
A read-only field, however, cannot be typed. It can only be filled out
by invoking 
the corresponding commands (normally through popup menu selection) .

A node also has different types, such as, Source, Comment, Issue, etc. 
When displayed, it is always appeared in a buffer. 
In this manual we use ``buffer'' and ``node'' interchangeably.

\item [Minibuffer.] The ``minibuffer'' is a distinguished buffer in
XEmacs. The minibuffer appears as the bottom line in all XEmacs
screens. CSRS uses the minibuffer to display messages as it manipulates
the review database. Also, CSRS occasionally uses the minibuffer to
prompt reviewers for information (for example, user's password).

\end{description}

\subsection {Three screen mode}

Much of review takes place in what we call ``three screen mode'',
where the display monitor is split vertically in half, with one screen
occupying the entire left half of the monitor and with two screens
stacking vertically on the right half of the monitor (see Figure
\ref{csrs-screen}). 

\begin{figure}[htpb]
  {\centerline{\psfig{figure=csrs-screen.ps}}}
  \caption{FTArm Screens}
  \label{csrs-screen}
\end{figure}

The left screen almost always contains a buffer displaying a
Source node. The right screens contains buffers displaying commentary
nodes (e.g., Issue, Action, and Comment nodes). The upper right screens
consist of two overlapping screens. One is to display commentary nodes,
another one is to display summary information about sets of nodes.

In general, overlapping screens in CSRS need not be raised manually; the
corresponding command will do it automatically.
However, one can also raise a hidden screen explicitly by using
the ``Screen'' pulldown menu.

\subsection {Pulldown and popup menus}
FTArm/CSRS provides both pulldown menus, which appear by pressing any
mouse button in the menubar at the top of each screen, and popup
menus, 
which appear by depressing the right mouse button. All FTArm commands
available to review participants can be initiated through mouse clicks
and these menus; there is never any need to use meta-X or control key
combinations (some commands are bound to control key combinations for
the comfort of wizards).

Popup menus are context-sensitive in two ways. First, the menu
displayed by pressing the mouse-right button varies according to the
window under the mouse. For example, if the mouse is over an Issue
node, the mouser-right will popup a menu with actions appropriate to
an Issue node appear. Second, popup menus are also context-sensitive
to the field under the mouse. For example, if the mouse is over the
Lines field of an Issue node, then mouse-right will popup a menu with
actions specific to the Lines field.

{\it Hint:} If the popup menu displayed does not provide the action
you wish to take, make sure that the mouse is over both the correct
screen, and the correct buffer, or field within the screen. The
popup menu header also shows the node name (type), or the the field
name. 

\subsection {Mouse button conventions}

Each mouse button has a standard function in FTArm/CSRS. Mouse-left
selects a screen and a point within the buffer. Mouse-middle retrieves
a node, either by following a link when the mouse is over a link
label, or by following a reference to a node when the mouse is over a
highlighted text. Mouse-right displays a popup menu of commands. 


\section {Setup phase}
During the Setup phase, the review participants and the artifacts
to be reviewed are selected. Review roles are also assigned to
individual participants. The Administrator will then enter the
information into the database.

FTArm defines three review roles: Moderator, Producer, 
and Reviewer. The moderator leads the review process, and ensure that
the participants execute each review phases properly. Specifically, the 
moderator moderates discussions during group meeting, analyze review 
outcomes, generates reports and metrics, and 
decides on rework decision based on review results.

The Producer is the person who produces the source artifacts to be 
reviewed, and the Reviewer is the one who inspect and evaluate the
review artifacts.

In addition, there is one built-in role ``Administrator'' defined by 
CSRS/FTArm. This role is intended for the
database maintainer, specifically to initiate and conclude the
review phases. The participants holding this role also have
unlimited access to any nodes in the database.

Admin privilege can be given to any participants regardless of their
roles. It is specified in the participants file (i.e., participants.el).
However, we recommend that only the moderator be given this
privilege. 

In addition to built-in role, there is also a built-in participant
named ``Administrator'' holding the role of ``Administrator''. This
special user is responsible 
for initializing the database for the first time. In other words, when
initializing the database, the user connects as ``Administrator''. 

The user ``Administrator'' can also be activated as ``Agent'' similar
to ``Gagent'' in Egret. The exact procedures to activate this agent
are not described in this manual.


\subsection {Procedures}
In general, setting up a database for review involves the following
procedures: 
\begin{enumerate}
\item {\bf Preparing Project directory.}  Create a directory to store
  database files for the  project.
\item {\bf Creating project files.}  In the project directory, create
  the following files:
\begin {itemize}
\item {\it project-name.el:} containing project name.
\item {\it participants.el:} containing participant names and their
  roles.
\item {\it clients.hb:} containing participant names and their
  passwords. All names specified in participants.el should be listed
  in this file along with their passwords. In addition, one should
  include two special participants: Gagent and  Administrator. Gagent 
  is a special user/client required by {\it Egret} system. Administrator
  is a special participant handling administrative matters discussed
  earlier. 
\end{itemize}

\item {\bf Encrypting password file (clients.hb).}  Invoke the command
  {\it hbs-encrypt} to encrypt the users passwords specified in clients.hb.

\item {\bf Preparing source files.}  Create a directory to store
  source files to be reviewed, then copy the files into this directory.
  In addition, a source definition file (source-defns.el) containing
  source node types definitions as well as parsing rules must be
  created in this directory.  The source files are then delimited with
  the characters described in the parsing rules (see: demo project for
  sample of source-defns.el and the corresponding source file).

  During initialization, these source files will be parsed and entered
  into the database according to the specified parsing rules.

\item {\bf Initializing database}
The next step after the above files have been created is to initialize
the database. Use the shell script {\bf initialize-ftarm} to
initialize the database.
First, go to the project directory, and invoke the script {\bf
initialize-ftarm} 
with three arguments {\it Database-ID}, {\it Port-No} and {\it
source-files-directory}. 
{\it Database-ID} is a symbolic
name for the database (Do not include space in the
name). {\it Port-No} is the port number of the connection in which the
server will be listening to (Use any integer greater or equal
than 10000). 
{\it source-files-directory} is the name of the directory where the
source files to be reviewed can be found. This argument needs to
be supplied when initializing the database for the first time. It
is omitted when merely initializing the server.

This command will also prompt the user to enter the Gagent password,
and the Administrator password if {\it source-files-directory} is
supplied.  

When the initialization is successful, two processes will be run in the 
background: hbs process and Gagent process (See Egret
documentation for detailed explanation of these processes).

When the initialization failed, diagnose the problem, remove all
*.hb files in the project directory except clients.hb, and then rerun
the script.

To initialize/rerun the server after the database has been
initialized, simply invoke the above script without supplying the
third argument.

\end{enumerate}

When the initialization is successful, the review process will be
automatically put into {\it Orientation} phase. In other words, the
orientation phase can now begin.

\section{Admin phase}
The system defines a built-in phase called Admin phase, where the
Administrator performs administrative procedures.

Unlike FTArm phases, Admin phase can be entered at any time during the
review process. Furthermore, only the participants with the
Administrator role can enter this phase.
The two most common administrative procedures are
ending current phase and starting next phase.

\subsection* {Ending current phase}
To end the current phase, perform the following steps:
\begin{enumerate}
\item Connect as Administrator.
\item Select the menu item ``List projects'' from the menubar
``Administrator/Projects''.
\item Select the project to be retrieved using middle-mouse.
\item Move the mouse to ``Review Phases'' field of the project node,
and select the highlighted phase name using middle-mouse button.
\item Popup a menu on this Phase node and select the item
``End this phase''. 
\item When the exit conditions have not been satisfied, the system will
prompt the user through a dialog box to overwrite the
conditions. Simply press button ``yes'' to overwrite this condition if
desired. 
\item Save and close phase node by pressing ``Close button'' on the
menubar. 
\item Similarly, close the project node by pressing  ``Close button''
on the menubar.
\item Disconnect or quit from the database.
\end{enumerate}

\subsection* {Starting next phase}
To start the next phase, follow the similar procedures as in the
``Ending current phase'', with the exception of selecting the command
``Start this phase'' in step 5.

%One may also activate or deactivate review phases automatically
%without manual intervention. This is done through setting up a special 
%agent process running Administrator mode. This agent connects as user
%'Administrator. The script {\it run-ftarm-admin} is used for this
%purpose. 

\section {Orientation phase}
The primary goal of the orientation phase is to familiarize the
review participants with the source artifacts under review. 

In addition, the review leader (i.e., the Moderator) may use this
phase to familiarize the review team with the review objectives and
process, and the functions and usage of the FTArm system.
This is especially important when 
the review team is using the system for the first time.
The benefits of participating in the review should be made
explicit for each participant, and make sure each reviewer is
committed to putting effort into the review process.

During the orientation phase, the review team meet face-to-face in one room,
and individual participants login to the database with their
assigned roles. 
The system also runs synchronous interaction mode. Any nodes presented
by the producer will be displayed synchronously in all participants
screens. 

When presenting program code, the producer should not read the code line
by line. Instead, the producer should provide the context necessary
for reviewers to understand how to read it. Any ``clever tricks'' in
the code, and all important dependencies between the review code and the
rest of the system should be described in detail. 
 
Any questions and answers are encouraged during this phase. However,
the questions should be directed mostly to understand the source
artifacts, and not to find or address any defects. The moderator
should ensure this policy is followed.

The moderator may also make a note regarding a particular question
raised during this orientation (for example, to better prepare reviewers
understanding the source nodes under review).

The orientation phase is completed when all source nodes have been
presented by the producer. 

\subsection {Procedures}

\subsubsection*{Moderator}
\begin{enumerate}
\item Connect to the database as Moderator.
\item Wait until all participants have connected to the database, then
instruct the producer to start presenting the source nodes.
\item Monitor the discussion. If necessary, record the questions
raised in a comment node: 
   \begin{itemize}
   \item Go to Source buffer, popup a menu by pressing mouse-right
   button. 
   \item Select the item ``Make a note'', and fill out the fields.
   \item Save the content of the node by pressing ``Save'' button on
   the menubar. Allow some time for all participants to read the node.
   \item Close the node by pressing ``Close'' button on the menubar.
    \end{itemize}
\item Quit from database.
\end{enumerate}

\subsubsection*{Producer}
\begin{enumerate}
\item Connect to the database as Producer.
\item Wait the moderator's instruction to start presentation.
\item Present each source node displayed in the summary buffer:
 \begin{itemize}
  \item Move the mouse to the summary buffer.
  \item Select the desired source node by clicking mouse-middle button
   on the highlighted item.
  \item Present the source node.
 \end{itemize}
\item Repeat the above step until all source nodes have been presented.
\item Quit from database.
\end{enumerate}

\subsubsection*{Reviewer}
\begin{enumerate}
\item Connect to the database as Reviewer.
\item Focus attention to both the Moderator and Producer.
\item Focus attention to the screen that displays the nodes.
\item Raise any questions to better understand the source nodes.
Interrupt the producer if necessary.
\item Quit when producer has completed his/her presentation.
\end{enumerate}


\section {Private review phase}
Typical goals for private review phase are: 
(1) to raise issues about the correctness,
maintainability, efficiency or other attributes of the source nodes;
(2) to propose corresponding actions; (3) to learn the source
    artifacts from other people efforts.

The private review phase is conducted asynchronously. Individual
reviewers connect to the database at their convenient time. Issues
raised by reviewers during this phase are kept
private, but questions or comments are made public.

This phase is completed when all reviewers have marked all source
nodes ``reviewed''.

\subsection {Procedures}

\subsubsection*{Reviewer}
\begin{enumerate}
\item Connect to the database as Reviewer. Upon successful connection,
all unreviewed source nodes will be displayed on the summary screen.
\item Select the source node to be reviewed from the summary screen.
\item Review the source node.
\item When noticing any errors in the source node, create an issue
  node as follow:
  \begin{itemize}
    \item Move the mouse to the source node buffer, then
   popup a menu to select the command ``Raise an issue''.
    \item  Fill out the fields in Issue node.
    \item Save the issue node by pressing  ``Save'' or ``Close''
    button in the   menubar (Note: The command ``Close'' automatically
    saves the node). 
  \end{itemize}
\item To raise a general issue (without referring to a particular
source node), select the command ``Raise a general issue'' from the
pulldown menu ``Reviewer''.
\item When proposing an action to this issue, first, make sure the
mouse in the Proposed-action field of the Issue node, then popup a
menu to select the command ``Propose an action'', and fill out the
Action node. Close the node when done.
\item To ask questions regarding the source node to the producer or
other reviewers: first, select the item ``Raise a question'' from the
popup menu in the source 
node, then fill out the question/comment node. Close the node when done.
\item When reviewing the source node, observe the presence of
star (*) in the left margin. This indicates that someone has posted
a question/comment regarding this particular line. To find out more
about this question/comment, press mouse-left on the *, and the
corresponding lines in 
the source node will be highlighted. Follow the highlighted lines to
retrieve the question node.
The question node can also be retrieved by following the link located
in the Comments field of the source node.
To get a summary of all comment nodes (questions and responses),
pulldown the Reviewer menubar and select ``List all comments''.

\item Repeat the above steps as necessary until no more issues,
actions or questions remained for the current source node.
\item Mark the source node as reviewed by selecting the command ``Set
status to reviewed'' from the popup menu in the source node.
\item Review the remaining source nodes, and mark the source nodes as
``Reviewed'' when done.
\item Disconnect or quit from the database.

\end{enumerate}


\subsubsection*{Producer}
The primary task of Producer during private review is to answer any
questions from the reviewers.
\begin{enumerate}
\item Connect to the database as Reviewer. Upon successful connection,
all comment nodes raised by the reviewers will be displayed on the
summary screen. 
\item Select the unread comment nodes from the summary screen.
\item Read the corresponding source node described in the comment node
by following the reference in the Source-node field or the Lines
field. 
\item To respond or further question this comment node, select the
command ``Respond to this comment'' or ``Raise a question'' from the
popup menu in the comment node.
\item Repeat the above steps until no more unread comment nodes in the
summary buffer remained. 
Entries with star (*) in the summary buffer indicate the producer
responses. 
\item Disconnect or quit from the database.
\end{enumerate}

\subsubsection*{Moderator}
During the private review phase, the primary task of the moderator is to
monitor the status of participants, in order to decide whether the
private review phase can be ended and the next phase (public review)
can be started.
The private review phase is completed when all reviewers have marked
every source node as Reviewed.

To perform this task, the moderator should assume the role of
Administrator and periodically check the content of Current-status
field in the Private-Review node. Refer to section {\it Admin phase}
for detailed explanation on how to read Private-Review node.


\section {Public review phase}
Similar to private review phase, public review phase is conducted
asynchronously at the participants convenient time.

The goals of public review are: (1) to read every issue and proposed
action created by other reviewers, (2) to respond to each issue by
confirming or agreeing with the issue, disconfirming or rejecting the
issue, or expressing no opinion (neutral), with or without
providing comments.  

Upon reading the comments, other reviewers may
disagree and create a follow-up comment/rebuttal. They may also agree
with the responses and change their votes.

When reading issue nodes, the reviewers should also indicate
whether the issues raised by others are similar to theirs by creating
``similar links''.

Finally, the public review completed when all reviewers have voted on
every issue nodes.


\subsection{Procedures}
\subsubsection*{Reviewer and Producer}
\begin{enumerate}
\item Connect to the database as Reviewer or Producer. Upon successful
connection all Issue nodes will be displayed on summary screen.
\item Select an issue to be read from the summary screen.
\item Follow the Source-node or Lines field in the Issue
node to get more information concerning the corresponding source node.
\item After reading the issue, vote to confirm, disconfirm, or express no
opinion (neutral) by selecting one of the item in the popup menu
``Consensus''. 
\item You may also make a confirming, disconfirming, or neutral
argument by selecting the command ``Make a confirming, disconfirming,
or neutral argument'' from the popup menu of Issue node. 
\item You may also question the issue if you do not understand by
selecting the command ``Raise a question'' from the popup menu of Issue node.

\item Similarly, when reading someone's comment, you may raise further
question, make a confirming, disconfirming or neutral argument by 
by selecting the corresponding command from the popup menu of Comment node.

\item You may also propose an action to this issue by selecting the
command ``Propose an action'' from the popup menu of Issue node.

\item Finally, you should indicate whether the issue you're currently
reading is similar to the issue you encountered before. If you notice
similarity, do the following:
\begin{itemize}
\item Select or mark this issue by selecting the command ``Select this
issue'' from the popup menu of Issue node.
\item From the summary screen, retrieve the similar issue.
\item Select the command ``Declare similarity to selected issue''.
 When successful, a link of type ``is-similar-to'' will be inserted
 into Related-issues field in  both nodes.
\end{itemize}
\item Repeat the above steps until you have read and voted on all
issue nodes. Refresh the summary screen if necessary by pressing the
``Refresh'' button. 
\item Also make sure that you have read all commentary nodes (i.e., Issue,
Action and Comment nodes). To display these nodes in the summary
screen do the following: 
 \begin{itemize}
   \item Pulldown ``Reviewer'' menu.
   \item Select the command ``List unread commentary nodes''.
 \end{itemize}
\item Disconnect or quit from the database when done.

\end{enumerate}

\subsubsection*{Moderator}
Similar to private review, the task of the moderator during public review
is to monitor the status of participants in order to decide when to
move to the next phase (Consolidation).
The public review phase is completed when all reviewers and producer
have read and voted on all issues.

To perform this task, the moderator assumes the role of
Administrator and periodically check the content of Current-status
field in the Private-Review node. Refer to section {\it Admin phase}
for detailed explanation on how to read the Private-Review node.

\section {Consolidation phase}

The Consolidation phase is carried out by the Moderator only. Other
participants will be denied access to the database.

The goals of consolidation phase are: (1) to consolidate similar
issues, (2) to consolidate supporting and opposing arguments
concerning each issue,
(3) to decide on final action regarding resolveable issues, 
(4) to prepare
    unresolveable issues for meeting.

The first and second goals can be carried out by the system
automatically when the appropriate command is invoked.
The system will generate
consolidated-issue nodes for every non similar issue nodes and their
related action and comment nodes. The moderator then creates a new or
attaches an existing consolidated-action node for every resolveable
consolidated-issue node. In other words, the moderator makes the final
decision about whether the issue is considered resolved or not
(typically by evaluating the majority votes and the arguments
presented by the participants).  

The consolidated-issue that do not have consolidated-action
nodes are considered unresolveable. These nodes are brought to the
meeting phase for final resolution.

\subsection {Procedures}


\subsection*{Moderator}
\begin{enumerate}
\item Connect to the database as Moderator.
\item Pulldown the menu ``Consolidation'', and select the command
``Consolidate all issues''. Wait until this command is completed.
When done, all Consolidated-issue nodes will appear on the summary
screen. 
%%\item Again, pulldown the menu ``Consolidation'', and select the command
%%``List all consolidation issues'' to view
%%the resulting consolidated issues nodes in the summary screen.
\item For each consolidated-issue node, evaluate whether a
consolidated-action node can be created. If so, create a consolidated-
action for this node by selecting the command ``Create consolidated
action'' from the popup menu of consolidated-issue node
\item If the resolution of the issue is already available from
existing consolidated-action node, then simply create a link to the
consolidated action as follow:
\begin {itemize}
\item Move the mouse to the summary screen containing consolidated
issue and action nodes.
\item Move the mouse to the existing consolidated action, popup a menu,
and then select the command ``Select this consolidated action''.
\item Move the mouse back to consolidated-issue node, and popup the
menu to select the command ``Make a link to selected consolidated
action''. 
\end{itemize}
\item If some issues remain unresolved, quit from the database and
proceed to Meeting phase:
  \begin{itemize}
    \item Connect to the database as Administrator.
    \item End the consolidation phase.
    \item Start the meeting phase.
   \end {itemize}
\item If all issues have been resolved, generate reports and metrics
to end the review process. The commands can be found in the 
pulldown menu of ``Moderator''.

\end{enumerate}


\section {Meeting phase}
The goals of meeting phase  is primarily to resolve outstanding issues.
The participants meet face to face in a meeting room. The moderator
presents the outstanding issues, discuss them with the 
participants, and decide on final action.

The system runs synchronous interaction mode. Any nodes presented by
the moderator will be displayed synchronously in all participants
screens. 

\subsection {Procedures}
\subsubsection*{Moderator}
\begin{enumerate}
\item Connect to the database as Moderator. All unresolved issues will
be displayed on summary screen.
Wait until all participants ready. 
\item Select and present unresolved issue from the summary screen.
\item Open for discussion and take verbal vote.
\item Create consolidated action once decision has been made.
\item Repeat the above steps until no more unresolved issues remained.
\item Quit from the database.
\item Connect as Administrator and activate the Conclusion phase.
\end{enumerate}

\subsubsection*{Reviewer and Producer}
\begin{enumerate}
\item Connect to the database as Reviewer or Producer.
\item At all time, pay attention to the moderator and the screen where
the consolidated issues are displayed.
\item Participate in the discussion and voting.
\item Quit from the database.
\end{enumerate}

\section {Conclusion phase}

The purpose of conclusion phase is to generate report and related
metrics data. 
The moderator may also use this phase to follow up the results from
the meeting phase.
Only the moderator can carry out this task.

\subsection {Procedures}
\subsubsection*{Moderator}
\begin{enumerate}
\item Connect to the database as Moderator.
\item First, ensure that all issues have been resolved. Read through
all consolidated-issue nodes and the corresponding consolidated-action
nodes if necessary. Make final editing as necessary.
\item Generate reports and metrics by selecting appropriate commands
from the pulldown menu of Moderator.
\item Quit from the database.
\item Connect as Administrator to conclude the review process. 
\end{enumerate}

\section {FTArm Nodes}
\subsection {Source nodes}
Source nodes express source artifacts to be reviewed. They are created
during initialization of the system. Source nodes can only be read;
they cannot be modified, except by the Producer during the Orientation
phase.
However, once the private review has started, we strongly recommend to
keep the source nodes intact since issue or comment nodes may refer to
specific lines in these nodes. 

Source nodes can have several different types depending on the type of
source artifacts being submitted for review.
FTArm defines two super-types of source nodes:
``Code'' containing program code, and ``Document'' containing 
textual description. The Code can be further subtyped into specific
program unit, such as Function, Declaration, etc.

\noindent The fields structure of source Code are as follow (see
Figure \ref{source-code}):
\begin{description}
\item [Name:] The name of the source node.
\item [Specification:] The program code specification or documentation string.
\item [Source-code:] The actual program code.
\item [Issues:] Links to issue nodes. Created by the system
automatically when the issue nodes are created.
\item [Comments:] Links to comment nodes of type 'note or
'question. Created by the system automatically when the comment nodes
are created. 
\item [Related-documents:] Links to source nodes of type 'Document.
\end{description}

\begin{figure}[htpb]
  {\centerline{\psfig{figure=source-code.ps}}}
  \caption{Source code node}
  \label{source-code}
\end{figure}
The fields structure of Document is similar to Code, except there
are no Specification, and Source-code fields. Instead, it has
Description field containing the document article.

\subsection {Issue nodes}
Issue nodes express an opinion, such as a potential problem, error,
inefficiency. Issue nodes cannot be seen by other reviewers during
private review. 

\noindent The fields structure of Issue node are as follow (see Figure
\ref{issue}) :
\begin{description}
\item [Subject:] One line summary of the issue.
\item [Category:] The issue category. To fill out this field, simply
type in the value. Typical value includes design
error, interface error, coding style, etc.
\item [Criticality:] The issue criticality. Typical value includes
Hi, Med or Low selectable from the popup menu in this field. Other
value can be typed in as desired.
\item [Source-node:] Reference to the source node concerning this issue.
Filled out by the system automatically.
\item [Lines:]  The lines or regions in the source node concerning
this issue. To fill out this field, see the section how to
select source node lines. 
\item [Description:] A detailed description of the issue.
\item [Consensus:]  This field contains the current
votes concerning this issue. It shows the number of
participants who vote {\it Confirm}, {\it Disconfirm}, or {\it
Neutral}. The votes are completely anonymous, however, character star
(*) will appear on your vote.

When an issue is created (for example, during private review), your
vote will be automatically recorded as 'Confirm.
During public review, others will have an opportunity to discuss this
issue and you may change your vote if desired.

To fill out this field, simply the select the desired value (confirm,
disconfirm or neutral) from the popup menu of this field.

\item [Related-issues:] This field contains links to
related or similar issues. The links are created by the system
automatically during public review when two issues are declared similar.

\item [Proposed-actions:] Contains links to action
nodes that propose to resolve this issue. They are created when the
reviewers or producer proposes action nodes during private or public
review phases. 

\item [Comments:] Contains links to comment nodes that further elaborate
this issue (such as, confirming, disconfirming, or questioning this
issue). These links are created by the system automatically
(during the public review phase) when the corresponding comment nodes are
created. 
\end{description}

\begin{figure}[htpb]
  {\centerline{\psfig{figure=issue.ps}}}
  \caption{Issue node}
  \label{issue}
\end{figure}

\subsection {Action nodes}
Action nodes express a proposal for action to be taken regarding an
Issue. Each action node and its corresponding issue node contain links
to each other. Action nodes are always created after the creation of
an Issue node. 
Similar to issue node, action nodes cannot be seen by other reviewers
during private review.

\noindent The action node consists of the following fields (see Figure
\ref{action}) :
\begin{description}
\item [Subject:] One line summary of the action.
\item [Action-type:] The type of  action. Typical value includes 
Fix, Ignore, or Unknown selectable from the popup menu in this field.
Other value can be typed in as desired. 
\item [Source-node:] Reference to the corresponding issue
node. Created by the system automatically when the action node is
created. 
\item [Description:] A detailed description of the proposed action.
\end{description}

\begin{figure}[htpb]
  {\centerline{\psfig{figure=action.ps}}}
  \caption{Action node}
  \label{action}
\end{figure}


\subsection {Comment nodes}

During the Orientation phase, comment nodes (of type 'note) provide information
to the reviewers to help them better understand the source artifacts.
Only the moderator can create comment nodes during the orientation phase.

During the private review phase, comment nodes 
may question (type 'question), or provide  
answers (type 'respond) about particular source nodes. 
Unlike Issue or Action nodes, 
comment nodes can be seen by other reviewers during private review,
and are answered by the creation of additional comment nodes by the
producer or other reviewers.

During the public review phase, comment nodes may express one's opinions
or arguments concerning issues raised by others. 
Comment node types  during public review include
'confirm, 'disconfirm, 'neutral, 'question or 'respond.

\noindent The comment node consists of the following fields (see Figure
\ref{comment}):
\begin{description}
\item [Subject:] One line summary of the comment.
\item [Source-node:] Reference to the source node concerning this comment.
To be filled out by the system automatically.
\item [Lines:]  The lines or regions in the source node concerning
this comment. (To fill out this field, see the section how to
select source node lines). 
\item [Comment-type:] note, question, respond, confirm, disconfirm, or
neutral. To be filled out by the system automatically when the node is
created by the corresponding command.
\item [Description:] A detailed description of this comment.
\item [Follow-up:] Contains links to any comment nodes generated in
reply to this comment.
\end{description}

\begin{figure}[htpb]
  {\centerline{\psfig{figure=comment.ps}}}
  \caption{Comment node}
  \label{comment}
\end{figure}

\subsection {Consolidated-Issue nodes}
Consolidated-Issue nodes express consolidated opinions of review participants
about the issues concerning the source nodes. The nodes contain not only
the issues raised by individual participants, but also the arguments 
supporting or against the issues, related comments, and the proposed
actions to be taken to resolve the issues.

Consolidated-Issue nodes are generated automatically by the system during
the consolidation phase. However, the moderator may edit their contents as 
desired for presentation during the meeting, or for final report.

\noindent The consolidated-issue node consists of the following fields
(see Figure \ref{consolidated-issue}):
\begin{description}
\item [Subject:] One line summary of the issue. It is copied from the
corresponding issue node by the system.

\item [Category:] Issue category. Copy from the corresponding issue node.

\item [Criticality:] Issue criticality. Copy from the corresponding
issue node.

\item [Source-nodes:] References to all source nodes and the line
numbers concerning this consolidated issue. Copy
from the corresponding issue nodes.

\item [Issues-consolidated:] References to the original issue nodes
that were consolidated.

\item [Description:] Detailed description of the issue. Copy from the
corresponding issue node, but editable by the Moderator.

\item [Supporting-arguments:] 
Contains discussion that support or confirm this issue, and the 
arguments that follow up the discussion.
Generated by the system automatically by following the links
related to the discussion. This field is editable by the Moderator.

\item [NonSupporting-arguments:] 
Contains discussion  against or disconfirm this issue, and the
related arguments following this. This field is editable by the Moderator.

\item [Other-arguments:]
Contains discussions that do not take particular position pro
or against the issue. It may contain questions as well as the follow
up response. This field is editable by the Moderator.

\item [Proposals:] 
Contains actions proposal to resolve this issue.
Taken from individual action nodes. This field is editable by the Moderator.

\item [Final-consensus:]
Contains the final votes regarding this issue. Cannot be modified
by the moderator.

\item [Final-action:] Links to consolidated action nodes, or
final resolution of the issue (only one link). 

\end{description}

\begin{figure}[htpb]
  {\centerline{\psfig{figure=consolidated-issue.ps}}}
  \caption{Consolidated issue node}
  \label{consolidated-issue}
\end{figure}

\subsection {Consolidated-action nodes}
Consolidated action nodes express final action to be taken to resolve
the consolidated issue. Only the moderator can create consolidated action 
nodes. Furthermore, two or more consolidated issues may have one
consolidated action. 

\noindent The consolidated-action node consists of the following
fields (see Figure \ref{consolidated-action}):
\begin{description}
\item [Subject:] One line summary of the action.

\item [Consolidated-issues:] References to consolidated issues.

\item [Action-type:] Type of action to be taken. Typical values include
Fix, Ignore, or Unknown. Modifiable by the moderator.

\item [Rework-decision:]  Detail description on how to do the
 rework.
\end{description}

\begin{figure}[htpb]
  {\centerline{\psfig{figure=consolidated-action.ps}}}
  \caption{Consolidated action node}
  \label{consolidated-action}
\end{figure}

\section {Other FTArm commands}
\subsection {Connecting to database}
To connect to a database, invoke the shell script {\bf run-ftarm}
with optional arguments {\it-r review-role} and {\it -d database}. 
{\it review-role} is the participant's role, which is, 
{\it Moderator}, {\it Producer}, {\it Reviewer}, or {\it
Administrator} (Note: review-role is case sensitive).
{\it database} is the database to connect to; it defaults to the first
database found in the file databases.el.

This command connects the current login user to the specified database
with the specified review-role. The system will also prompt the user to
enter the password for authentication.
When successful, a XEmacs process will be created
and the connection will be established to the specified database.

If no arguments are supplied, a XEmacs process will be created, but no
connection will be established. To connect to a database,
the user must first select a review role from pulldown menu
``Session'', and then select the database from the pulldown
menu ``Session/Connect''. In this mode, one may also change the
default user using the command {\it M-x
k*session*set-connect-participant} (see FTArm demo documentation).


\subsection {Disconnecting from database}
To disconnect from database, simply select the command ``Disconnect''
or ``Quit'' from ``CSRS'' pulldown menu. Please allow the system some
time to update its internal nodes before disconnected.

\subsection {Selecting source node lines}
When creating an issue or comment node concerning the source node, you
may wish to explicitly indicate the specific lines in the source node
that give rise to the issue/comment.

\noindent To specify these lines, follow these steps:
\begin{enumerate}
\item Move the mouse over the source node.
\item Drag mouse-left over the lines to be selected. These lines will
then be highlighted.
\item Without moving the mouse, popup a menu and select the command
``Select this region''. Now, the highlighted lines will be permanent
(i.e., when you move the mouse to other positions, the lines remained
highlighted).
\item Repeat the above steps as many times as you wish to select
 different lines/regions in the source node.
\item  Now move the mouse to the ``Lines'' field in the Issue or
Comment node, then popup a menu to select the command ``Select current
highlighted lines''
\item The corresponding lines in the source node are now underlined.
A star (*) will also appear on the left margin to indicate a reference
has been created.
\item To hide/unhide the underlines, simply click on the character *.
When the underline appears, the text will be highlighted when the
mouse is moved over it. 
This indicates that the text can be followed by using 
mouse-middle button.

\end{enumerate}


\subsection {Deleting a node}
Sometimes you create a new issue, action, or comment node by accident,
and want to delete it from database. You can do this by simply
pulldown the menu ``CSRS'' and select the command ``Delete node''.
You will be prompted through Dialog box to confirm that you wish to
delete the node; press the button ``yes'' on the dialog box to carry
out the deletion. 

It is also possible to delete a node even after you have created it
and closed it. First, you must retrieve and display the node and then
follow the delete node steps described above.

\subsection {Editing a node that was previously created}
To edit the textual contents of a node you have previously created,
you must first obtain a lock on the node. To get the lock, pulldown
``CSRS'' menu and select the command ``Lock node''.

Note that you do not need to lock the node in order to add links to it
during creation of other nodes.

\subsection {Refreshing nodes or summary buffer}

When a node changes its status or attributes, the status may not
appear in the summary buffer. You may need to refresh the summary buffer
manually by pressing the ``Refresh'' button on the 
menubar of the summary screen.

Similarly, the content of a node can be refreshed by selecting the
command ``Refresh'' on the pulldown menu ``CSRS''. 
For example, after an Issue node has been created and saved, this command 
refreshes the link labels shown on the Source nodes 


\subsection {Running synchronous mode}
When the system is running synchronous mode, the Moderator may control
(enable or disable) who should do the presentation, namely, whether the
nodes being read by the presenters should be broadcasted to other
participants' workstations.

In general, there can be more than one active presenters at one 
time, but usually they are responsible for different types of nodes.
For example, during the Orientation phase, the Producer is responsible for
presenting the Source nodes, while the Moderator is responsible for
presenting the Comment nodes.
What types of nodes can be presented by whom are decided by the review
method. The moderator cannot change this behavior. However, the
Moderator can disable or enable the presenters.

In FTArm, the Moderator may even temporarily assign a reviewer to
be the presenter, for example, in order to better express his/her
question to the entire team.
To enable or disable a presenter, the Moderator can simply
select/deselect (toggle) the participant name from the current
presenters list in pulldown menu ``Moderator/Presenters''.

\noindent Furthermore when running synchronous mode, the following behaviors
should be observed:
\begin{itemize}
\item The presenter should not use the mouse to point to the exact
lines in the source node. The presenter mouse is not synchronized with
other participants.
Instead, the presenter should refer to the line number displayed
on the mode-line (right above the minibuffer). Note that this line
number is referring to buffer line number, which is different from the
line number shown on the Lines field (which is Field line number).
Nevertheless, this line number is the same for all participants.
Thus, the participants should move their mice to the line numbers
accordingly .


\item When filling out a synchronized node (such as Comment
node in Orientation phase), the characters just typed are not
synchronously appeared on everybody workstation. Instead, the node
needs to be saved first in order its content to appear on everybody
workstation. 

\item When closing a synchronized node, the buffer will be substituted
by a node currently present but previously hidden in the
screen. However, this node is not necessarily the same as the node
on other participant screens. In other words, the participant screens
are not always in sync with the leader when a node is closed.
If in the previous session,
the participants were displaying nodes other than those initiated by
the leader, then their screens may be out of sync with the leader when
a node is closed synchronously.
In any cases, the leader can re-synchronize all participant screens by
simply re-reading the substitute node (i.e., re-selecting the node from
the summary buffer). 
\end{itemize}


\end{document}


