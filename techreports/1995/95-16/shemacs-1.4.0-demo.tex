%%%%%%%%%%%%%%%%%%%%%%%%%%%%%% -*- Mode: Latex -*- %%%%%%%%%%%%%%%%%%%%%%%%%%%%
%% shemacs-1.4.0-demo.tex -- 
%% Author          : Julio C. Polo
%% Created On      : Mon Sep 18 21:46:34 1995
%% Last Modified By: Julio C. Polo
%% Last Modified On: Wed Sep 27 22:59:51 1995
%% RCS: $Id$
%%%%%%%%%%%%%%%%%%%%%%%%%%%%%%%%%%%%%%%%%%%%%%%%%%%%%%%%%%%%%%%%%%%%%%%%%%%%%%%
%%   Copyright (C) 1995 Julio C. Polo
%%%%%%%%%%%%%%%%%%%%%%%%%%%%%%%%%%%%%%%%%%%%%%%%%%%%%%%%%%%%%%%%%%%%%%%%%%%%%%%
%% 

\documentstyle [nftimes,11pt,/group/csdl/tex/definemargins]{article}
\input{/group/csdl/tex/psfig/psfig}

\begin{document}

\title{A Quick Guided Tour of Shemacs}

\author{Julio Polo\\ 
        University of Hawaii\\
        2565 The Mall\\ 
        Honolulu, Hawaii 96822\\ 
        (808) 956-2405\\
        {\tt julio@hawaii.edu}}

\date{ICS/CSDL-TR-95-16\\\today}

\maketitle

\newpage

\section{Prerequisites}

This is a quick guided tour through the main features of Shemacs.  It
assumes you have already installed Shemacs following the instructions from
the provided INSTALLATION file.

Unless otherwise specified, an instruction to select a menu item refers to
a menu item under the Shemacs menu.

\section{Bring up 3 XEmacs processes}

Start three XEmacs processes:

\small\begin{verbatim}
  xemacs &; xemacs &; xemacs &
\end{verbatim}\normalsize

\begin{figure}[htbp]
\centerline{\psfig{figure=/group/csdl/techreports/95-16/shemacs-demo-session.ps}}
\caption{Shemacs Demo Session.}
\label{fig:shemacs-demo-session}
\end{figure}

After all three XEmacs processes have come up, lay them out on your screen
such that all three of them are visible as in figure
\ref{fig:shemacs-demo-session}.

\section{Bring Up a Shemacs Server}

Shemacs is a client-server system.  A Shemacs server is sometimes referred
to as a {\it sherver\/}.  In this demo, we started three XEmacs processes
because we will start a sherver on one of them and a client on each of the
remaining two.

You need to start a sherver because it will hold the list of documents that
you want to make available for shared editing with others, but most
importantly, the sherver will perform concurrency control on the changes
performed by users simultaneously editing a sherver-shared document.

Let's start a sherver on the first XEmacs process.  Hereafter, we wil refer
to this XEmacs process as the {\it sherver\/}.  To start the server, enter

\small\begin{verbatim}
  M-x shm*cmd*start-sherver
\end{verbatim}\normalsize

You will now be asked a series of questions at the minibuffer.  Here are
the questions you will be asked along with the answers you should provide:

\small
%% \begin{figure}[htpb]
\begin{center}
\begin{tabular} {|l|l|l|} \hline
{\em Step} & {\em Prompt}  & {\em Your Response} \\ \hline
1 & host name: & (press return for default) \\
2 & port number: & (press return for default) \\
3 & directory for this sherver: & \~ /shemacs-demo \\
4 & Is this a brand new sherver? & yes \\
5 & Building clients.hb: password for Gagent?  & kalanianaole \\
6 & Building clients.hb: user and password (if no more, type "exit")? &
steve kaaawa \\
7 & Building clients.hb: user and password (if no more, type "exit")? &
dano malaekahana \\
8 & Building clients.hb: user and password (if no more, type "exit")? &
exit \\
9 & Enter HBS password (db-ID: host.domain.15000, client: Gagent): &
kalanianaole \\
\hline
\end{tabular}
\end{center}
%% \end{figure}
\normalsize

It is encouraged that you choose a password of your own in steps 5 through
7 and to substitute them in these instructions as needed. This is to
prevent other people reading this demo to access your sherver with the
sample passwords used in here.

The sherver is identified by a db-ID which is normally your host's fully
qualified domain name followed by a period and the port number used.

Steps 5 through 8 are performed only once: when a brand new sherver is
initialized.  These steps allow you to create client names and passwords
for connecting to the sherver.  Please refer to the Egret documentation on
hbs-encrypt for additional information.

Several things will flash on the screen until the sherver's minibuffer says
``sherver is up'', and a Shemacs menu has been inserted in front of all other
XEmacs menus.

\section{Bring Up Two Shemacs Clients}

For each of the two remaining XEmacs processes, start a shemacs client
as follows:
	
\small\begin{verbatim}
  M-x shm*cmd*start-client
\end{verbatim}\normalsize

The ``shemacs ready'' message will then appear at the minibuffer, and a Shemacs 
menu appears in front of all other XEmacs menus.  Note that these 
clients have not connected to any sherver yet.  When you connect to a sherver, 
you will be prompted for your user and password on that sherver.

\section{Submit the Document to be Shared}

We have started a sherver but it is not serving any documents yet.  Here we
will use the second XEmacs process to connect as client ``steve'' to our
sherver and then submit the /usr/local/shemacs/demo.doc file so that it may
be available for shared editing through our sherver.

Select the ``Submit Document...'' menu item (M-x shm*cmd*submit-document)
and answer the prompts as follows:

\small
%% \begin{figure}[htpb]
\begin{center}
\begin{tabular} {|l|l|l|} \hline
{\em Step} & {\em Prompt}  & {\em Your Response} \\ \hline
1 & host name: & (press return for default) \\
2 & port number: & (press return for default) \\
3 & file to submit: & /usr/local/shemacs/demo.doc \\
4 & share it with the name of?  & (press return for demo.doc) \\
5 & host.domain.15000 login: & steve \\
6 & Enter HBS password (db-ID: host.domain.15000, client: steve): & kaaawa \\
\hline
\end{tabular}
\end{center}
%% \end{figure}
\normalsize

You will see several messages showing the connection to the sherver in 
progress.  If you keep your eyes on the sherver's minibuffer, you will see 
that a file named demo.doc was written out, and that the buffer for this file 
is in the sherver process as well.

The demo.doc document is now ready to be edited by more than one user.

\section{Open demo.doc From Two Shemacs Clients}

We will now open the demo.doc document from the second and third XEmacs
processes.  Select ``Open Document...'' (C-c f or M-x
shm*cmd*open-document) and answer the prompts as follows:

\small
%% \begin{figure}[htpb]
\begin{center}
\begin{tabular} {|l|l|l|} \hline
{\em Step} & {\em Prompt}  & {\em Your Response} \\ \hline
1 & host name: & (press return for default) \\
2 & port number: & (press return for default) \\
3 & document name: & demo.doc \\
4 & host.domain.15000 login: & dano \\
5 & Enter HBS password (db-ID: host.domain.15000, client: steve): & malaekahana \\
\hline
\end{tabular}
\end{center}
%% \end{figure}
\normalsize

Note that steps 4 and 5 are only needed on the third XEmacs process since we 
have already logged in as ``steve'' in the second XEmacs process when we 
submitted the demo.doc document to the sherver.

The buffer for demo.doc should now be on both clients.

\section{Edit demo.doc From Both Shemacs Clients}

Switch to the demo.doc buffer on the sherver so that you can also see what 
happens on the sherver when shared editing is happening on demo.doc

Now go to the second XEmacs process (the one you logged in as ``steve''),
highlight the first paragraph, and then select the ``Edit Region'' menu
item (C-c e or M-x shm*cmd*edit-region).  What you have just done is claim
the first paragraph for steve to edit.  In Shemacs lingo, you have just
claimed a region for editing.  No other client can touch this paragraph
until steve does a ``Close Region'' on it.

Notice how the claimed region of text is now in blue boldface and has ``steve''
next to it.  If you look at the other XEmacs processes, the region steve has 
claimed is in red italic text and also shows ``steve'' as the person editing 
that region. 

Fix the mispelled word ``sornkeling'' to ``snorkeling''.  Notice that the
changes you have made are NOT automatically updated on the other XEmacs
processes.  You need to select ``Synchronize Region'' (C-c z or M-x
shm*cmd*synchronize-region) or ``Synchronize All''(shm*cmd*synchronize-all)
for that to happen.  Shemacs will automatically update other clients during
crucial events such as closing a region, closing a document, or
disconnecting from a sherver.

Repeat the above on the third XEmacs process.  Edit the second paragraph
and fix the mispelled word ``beech'' to ``beach''.  Feel free to experiment
by trying to edit regions that have been claimed by someone else and seeing
what happens, or by adding and deleting text and observing how fast changes
propagate among clients.

Note that the sherver should never be used for shared editing.  Only shemacs 
clients can safely be used for editing a sherver-shared document.

\section{Saving in Shemacs}

There are actually two demo.doc documents. The original /usr/shemacs/demo.doc 
and the demo.doc inside the sherver's directory. The latter is the one that is 
edited by shemacs clients.

To save the shared demo.doc, select the ``Save Document'' menu item (C-c s
or M-x shm*cmd*save-document).  You will notice that the usual minibuffer
message you get after saving a buffer will appear at the sherver's
minibuffer.  This is because, true to a client-server system, the saving
was actually performed at the sherver.

\section{Goodies}

If you plan to access the demo.doc document often, you can add it to your
``Bookmarks'' menu by selecting the menu item ``Bookmarks -> Add Bookmark''.  
Thereafter, demo.doc can be quickly opened from the ``Bookmarks'' menu, but you 
will still need to enter your login and password if you have not connected to 
the sherver.

You can also add commonly accessed Shemacs servers to your Shervers menu.
The ``Shervers -> Add host.domain.port'' menu item will add the sherver for
the current document to your Shervers menu.  After that, you can easily
connect to the sherver by simply selecting it from the Shervers menu.
After connecting, the sherver's documents will be available under that
sherver's menu item.

All of these settings are automatically saved to your .shemacs file in your 
home directory.

\section{Closing up}

When you are done editing a region you have claimed, you can relinquish it
by selecting the ``Close Region'' menu item (C-c c or M-x
shm*cmd*close-region).  Make sure your cursor is in the region you want to
close.

If you are completely done editing a shared document, select the ``Close 
Document'' menu item (C-c k or M-x shm*cmd*close-document).  

If you wish to close all documents for a particular sherver, go to any
document from that sherver and select the ``Shervers -> Disconnect from
host.domain.port'' menu item.  Note that this will disconnect your
connection to that sherver and that you will need to login again if you
want to access any document from that sherver.

\section{Quitting Shemacs}

To quit a Shemacs client, select the ``Exit Shemacs'' menu item (M-x
shm*cmd*exit-shemacs).  This will perform the client updates, closing,
saving and disconnection necessary to preserve your editing changes.

To quit the Shemacs server, select the ``Shutdown Sherver'' menu item 
(M-x shm*cmd*shutdown-sherver) on the sherver.  Make sure that all shemacs
clients have logged out or their changes will be lost.

\section{Exercise}

Bringing up several shervers (in different hosts if possible), submit 
documents to each one of them, and open one document from each sherver using 
the same shemacs client.

\end{document}



