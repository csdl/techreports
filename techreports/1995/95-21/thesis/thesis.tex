\documentstyle[nftimes,11pt,makeidx, /group/csdl/tex/definemargins,
/group/csdl/tex/lmacros, /group/csdl/tex/functiondoc]{article}
% Psfig/TeX 
\def\PsfigVersion{1.9}
% dvips version
%
% All psfig/tex software, documentation, and related files
% in this distribution of psfig/tex are 
% Copyright 1987, 1988, 1991 Trevor J. Darrell
%
% Permission is granted for use and non-profit distribution of psfig/tex 
% providing that this notice is clearly maintained. The right to
% distribute any portion of psfig/tex for profit or as part of any commercial
% product is specifically reserved for the author(s) of that portion.
%
% *** Feel free to make local modifications of psfig as you wish,
% *** but DO NOT post any changed or modified versions of ``psfig''
% *** directly to the net. Send them to me and I'll try to incorporate
% *** them into future versions. If you want to take the psfig code 
% *** and make a new program (subject to the copyright above), distribute it, 
% *** (and maintain it) that's fine, just don't call it psfig.
%
% Bugs and improvements to trevor@media.mit.edu.
%
% Thanks to Greg Hager (GDH) and Ned Batchelder for their contributions
% to the original version of this project.
%
% Modified by J. Daniel Smith on 9 October 1990 to accept the
% %%BoundingBox: comment with or without a space after the colon.  Stole
% file reading code from Tom Rokicki's EPSF.TEX file (see below).
%
% More modifications by J. Daniel Smith on 29 March 1991 to allow the
% the included PostScript figure to be rotated.  The amount of
% rotation is specified by the "angle=" parameter of the \psfig command.
%
% Modified by Robert Russell on June 25, 1991 to allow users to specify
% .ps filenames which don't yet exist, provided they explicitly provide
% boundingbox information via the \psfig command. Note: This will only work
% if the "file=" parameter follows all four "bb???=" parameters in the
% command. This is due to the order in which psfig interprets these params.
%
%  3 Jul 1991	JDS	check if file already read in once
%  4 Sep 1991	JDS	fixed incorrect computation of rotated
%			bounding box
% 25 Sep 1991	GVR	expanded synopsis of \psfig
% 14 Oct 1991	JDS	\fbox code from LaTeX so \psdraft works with TeX
%			changed \typeout to \ps@typeout
% 17 Oct 1991	JDS	added \psscalefirst and \psrotatefirst
%

% From: gvr@cs.brown.edu (George V. Reilly)
%
% \psdraft	draws an outline box, but doesn't include the figure
%		in the DVI file.  Useful for previewing.
%
% \psfull	includes the figure in the DVI file (default).
%
% \psscalefirst width= or height= specifies the size of the figure
% 		before rotation.
% \psrotatefirst (default) width= or height= specifies the size of the
% 		 figure after rotation.  Asymetric figures will
% 		 appear to shrink.
%
% \psfigurepath#1	sets the path to search for the figure
%
% \psfig
% usage: \psfig{file=, figure=, height=, width=,
%			bbllx=, bblly=, bburx=, bbury=,
%			rheight=, rwidth=, clip=, angle=, silent=}
%
%	"file" is the filename.  If no path name is specified and the
%		file is not found in the current directory,
%		it will be looked for in directory \psfigurepath.
%	"figure" is a synonym for "file".
%	By default, the width and height of the figure are taken from
%		the BoundingBox of the figure.
%	If "width" is specified, the figure is scaled so that it has
%		the specified width.  Its height changes proportionately.
%	If "height" is specified, the figure is scaled so that it has
%		the specified height.  Its width changes proportionately.
%	If both "width" and "height" are specified, the figure is scaled
%		anamorphically.
%	"bbllx", "bblly", "bburx", and "bbury" control the PostScript
%		BoundingBox.  If these four values are specified
%               *before* the "file" option, the PSFIG will not try to
%               open the PostScript file.
%	"rheight" and "rwidth" are the reserved height and width
%		of the figure, i.e., how big TeX actually thinks
%		the figure is.  They default to "width" and "height".
%	The "clip" option ensures that no portion of the figure will
%		appear outside its BoundingBox.  "clip=" is a switch and
%		takes no value, but the `=' must be present.
%	The "angle" option specifies the angle of rotation (degrees, ccw).
%	The "silent" option makes \psfig work silently.
%

% check to see if macros already loaded in (maybe some other file says
% "\input psfig") ...
\ifx\undefined\psfig\else\endinput\fi

%
% from a suggestion by eijkhout@csrd.uiuc.edu to allow
% loading as a style file. Changed to avoid problems
% with amstex per suggestion by jbence@math.ucla.edu

\let\LaTeXAtSign=\@
\let\@=\relax
\edef\psfigRestoreAt{\catcode`\@=\number\catcode`@\relax}
%\edef\psfigRestoreAt{\catcode`@=\number\catcode`@\relax}
\catcode`\@=11\relax
\newwrite\@unused
\def\ps@typeout#1{{\let\protect\string\immediate\write\@unused{#1}}}
\ps@typeout{psfig/tex \PsfigVersion}

%% Here's how you define your figure path.  Should be set up with null
%% default and a user useable definition.

\def\figurepath{./}
\def\psfigurepath#1{\edef\figurepath{#1}}

%
% @psdo control structure -- similar to Latex @for.
% I redefined these with different names so that psfig can
% be used with TeX as well as LaTeX, and so that it will not 
% be vunerable to future changes in LaTeX's internal
% control structure,
%
\def\@nnil{\@nil}
\def\@empty{}
\def\@psdonoop#1\@@#2#3{}
\def\@psdo#1:=#2\do#3{\edef\@psdotmp{#2}\ifx\@psdotmp\@empty \else
    \expandafter\@psdoloop#2,\@nil,\@nil\@@#1{#3}\fi}
\def\@psdoloop#1,#2,#3\@@#4#5{\def#4{#1}\ifx #4\@nnil \else
       #5\def#4{#2}\ifx #4\@nnil \else#5\@ipsdoloop #3\@@#4{#5}\fi\fi}
\def\@ipsdoloop#1,#2\@@#3#4{\def#3{#1}\ifx #3\@nnil 
       \let\@nextwhile=\@psdonoop \else
      #4\relax\let\@nextwhile=\@ipsdoloop\fi\@nextwhile#2\@@#3{#4}}
\def\@tpsdo#1:=#2\do#3{\xdef\@psdotmp{#2}\ifx\@psdotmp\@empty \else
    \@tpsdoloop#2\@nil\@nil\@@#1{#3}\fi}
\def\@tpsdoloop#1#2\@@#3#4{\def#3{#1}\ifx #3\@nnil 
       \let\@nextwhile=\@psdonoop \else
      #4\relax\let\@nextwhile=\@tpsdoloop\fi\@nextwhile#2\@@#3{#4}}
% 
% \fbox is defined in latex.tex; so if \fbox is undefined, assume that
% we are not in LaTeX.
% Perhaps this could be done better???
\ifx\undefined\fbox
% \fbox code from modified slightly from LaTeX
\newdimen\fboxrule
\newdimen\fboxsep
\newdimen\ps@tempdima
\newbox\ps@tempboxa
\fboxsep = 3pt
\fboxrule = .4pt
\long\def\fbox#1{\leavevmode\setbox\ps@tempboxa\hbox{#1}\ps@tempdima\fboxrule
    \advance\ps@tempdima \fboxsep \advance\ps@tempdima \dp\ps@tempboxa
   \hbox{\lower \ps@tempdima\hbox
  {\vbox{\hrule height \fboxrule
          \hbox{\vrule width \fboxrule \hskip\fboxsep
          \vbox{\vskip\fboxsep \box\ps@tempboxa\vskip\fboxsep}\hskip 
                 \fboxsep\vrule width \fboxrule}
                 \hrule height \fboxrule}}}}
\fi
%
%%%%%%%%%%%%%%%%%%%%%%%%%%%%%%%%%%%%%%%%%%%%%%%%%%%%%%%%%%%%%%%%%%%
% file reading stuff from epsf.tex
%   EPSF.TEX macro file:
%   Written by Tomas Rokicki of Radical Eye Software, 29 Mar 1989.
%   Revised by Don Knuth, 3 Jan 1990.
%   Revised by Tomas Rokicki to accept bounding boxes with no
%      space after the colon, 18 Jul 1990.
%   Portions modified/removed for use in PSFIG package by
%      J. Daniel Smith, 9 October 1990.
%
\newread\ps@stream
\newif\ifnot@eof       % continue looking for the bounding box?
\newif\if@noisy        % report what you're making?
\newif\if@atend        % %%BoundingBox: has (at end) specification
\newif\if@psfile       % does this look like a PostScript file?
%
% PostScript files should start with `%!'
%
{\catcode`\%=12\global\gdef\epsf@start{%!}}
\def\epsf@PS{PS}
%
\def\epsf@getbb#1{%
%
%   The first thing we need to do is to open the
%   PostScript file, if possible.
%
\openin\ps@stream=#1
\ifeof\ps@stream\ps@typeout{Error, File #1 not found}\else
%
%   Okay, we got it. Now we'll scan lines until we find one that doesn't
%   start with %. We're looking for the bounding box comment.
%
   {\not@eoftrue \chardef\other=12
    \def\do##1{\catcode`##1=\other}\dospecials \catcode`\ =10
    \loop
       \if@psfile
	  \read\ps@stream to \epsf@fileline
       \else{
	  \obeyspaces
          \read\ps@stream to \epsf@tmp\global\let\epsf@fileline\epsf@tmp}
       \fi
       \ifeof\ps@stream\not@eoffalse\else
%
%   Check the first line for `%!'.  Issue a warning message if its not
%   there, since the file might not be a PostScript file.
%
       \if@psfile\else
       \expandafter\epsf@test\epsf@fileline:. \\%
       \fi
%
%   We check to see if the first character is a % sign;
%   if so, we look further and stop only if the line begins with
%   `%%BoundingBox:' and the `(atend)' specification was not found.
%   That is, the only way to stop is when the end of file is reached,
%   or a `%%BoundingBox: llx lly urx ury' line is found.
%
          \expandafter\epsf@aux\epsf@fileline:. \\%
       \fi
   \ifnot@eof\repeat
   }\closein\ps@stream\fi}%
%
% This tests if the file we are reading looks like a PostScript file.
%
\long\def\epsf@test#1#2#3:#4\\{\def\epsf@testit{#1#2}
			\ifx\epsf@testit\epsf@start\else
\ps@typeout{Warning! File does not start with `\epsf@start'.  It may not be a PostScript file.}
			\fi
			\@psfiletrue} % don't test after 1st line
%
%   We still need to define the tricky \epsf@aux macro. This requires
%   a couple of magic constants for comparison purposes.
%
{\catcode`\%=12\global\let\epsf@percent=%\global\def\epsf@bblit{%BoundingBox}}
%
%
%   So we're ready to check for `%BoundingBox:' and to grab the
%   values if they are found.  We continue searching if `(at end)'
%   was found after the `%BoundingBox:'.
%
\long\def\epsf@aux#1#2:#3\\{\ifx#1\epsf@percent
   \def\epsf@testit{#2}\ifx\epsf@testit\epsf@bblit
	\@atendfalse
        \epsf@atend #3 . \\%
	\if@atend	
	   \if@verbose{
		\ps@typeout{psfig: found `(atend)'; continuing search}
	   }\fi
        \else
        \epsf@grab #3 . . . \\%
        \not@eoffalse
        \global\no@bbfalse
        \fi
   \fi\fi}%
%
%   Here we grab the values and stuff them in the appropriate definitions.
%
\def\epsf@grab #1 #2 #3 #4 #5\\{%
   \global\def\epsf@llx{#1}\ifx\epsf@llx\empty
      \epsf@grab #2 #3 #4 #5 .\\\else
   \global\def\epsf@lly{#2}%
   \global\def\epsf@urx{#3}\global\def\epsf@ury{#4}\fi}%
%
% Determine if the stuff following the %%BoundingBox is `(atend)'
% J. Daniel Smith.  Copied from \epsf@grab above.
%
\def\epsf@atendlit{(atend)} 
\def\epsf@atend #1 #2 #3\\{%
   \def\epsf@tmp{#1}\ifx\epsf@tmp\empty
      \epsf@atend #2 #3 .\\\else
   \ifx\epsf@tmp\epsf@atendlit\@atendtrue\fi\fi}


% End of file reading stuff from epsf.tex
%%%%%%%%%%%%%%%%%%%%%%%%%%%%%%%%%%%%%%%%%%%%%%%%%%%%%%%%%%%%%%%%%%%

%%%%%%%%%%%%%%%%%%%%%%%%%%%%%%%%%%%%%%%%%%%%%%%%%%%%%%%%%%%%%%%%%%%
% trigonometry stuff from "trig.tex"
\chardef\psletter = 11 % won't conflict with \begin{letter} now...
\chardef\other = 12

\newif \ifdebug %%% turn me on to see TeX hard at work ...
\newif\ifc@mpute %%% don't need to compute some values
\c@mputetrue % but assume that we do

\let\then = \relax
\def\r@dian{pt }
\let\r@dians = \r@dian
\let\dimensionless@nit = \r@dian
\let\dimensionless@nits = \dimensionless@nit
\def\internal@nit{sp }
\let\internal@nits = \internal@nit
\newif\ifstillc@nverging
\def \Mess@ge #1{\ifdebug \then \message {#1} \fi}

{ %%% Things that need abnormal catcodes %%%
	\catcode `\@ = \psletter
	\gdef \nodimen {\expandafter \n@dimen \the \dimen}
	\gdef \term #1 #2 #3%
	       {\edef \t@ {\the #1}%%% freeze parameter 1 (count, by value)
		\edef \t@@ {\expandafter \n@dimen \the #2\r@dian}%
				   %%% freeze parameter 2 (dimen, by value)
		\t@rm {\t@} {\t@@} {#3}%
	       }
	\gdef \t@rm #1 #2 #3%
	       {{%
		\count 0 = 0
		\dimen 0 = 1 \dimensionless@nit
		\dimen 2 = #2\relax
		\Mess@ge {Calculating term #1 of \nodimen 2}%
		\loop
		\ifnum	\count 0 < #1
		\then	\advance \count 0 by 1
			\Mess@ge {Iteration \the \count 0 \space}%
			\Multiply \dimen 0 by {\dimen 2}%
			\Mess@ge {After multiplication, term = \nodimen 0}%
			\Divide \dimen 0 by {\count 0}%
			\Mess@ge {After division, term = \nodimen 0}%
		\repeat
		\Mess@ge {Final value for term #1 of 
				\nodimen 2 \space is \nodimen 0}%
		\xdef \Term {#3 = \nodimen 0 \r@dians}%
		\aftergroup \Term
	       }}
	\catcode `\p = \other
	\catcode `\t = \other
	\gdef \n@dimen #1pt{#1} %%% throw away the ``pt''
}

\def \Divide #1by #2{\divide #1 by #2} %%% just a synonym

\def \Multiply #1by #2%%% allows division of a dimen by a dimen
       {{%%% should really freeze parameter 2 (dimen, passed by value)
	\count 0 = #1\relax
	\count 2 = #2\relax
	\count 4 = 65536
	\Mess@ge {Before scaling, count 0 = \the \count 0 \space and
			count 2 = \the \count 2}%
	\ifnum	\count 0 > 32767 %%% do our best to avoid overflow
	\then	\divide \count 0 by 4
		\divide \count 4 by 4
	\else	\ifnum	\count 0 < -32767
		\then	\divide \count 0 by 4
			\divide \count 4 by 4
		\else
		\fi
	\fi
	\ifnum	\count 2 > 32767 %%% while retaining reasonable accuracy
	\then	\divide \count 2 by 4
		\divide \count 4 by 4
	\else	\ifnum	\count 2 < -32767
		\then	\divide \count 2 by 4
			\divide \count 4 by 4
		\else
		\fi
	\fi
	\multiply \count 0 by \count 2
	\divide \count 0 by \count 4
	\xdef \product {#1 = \the \count 0 \internal@nits}%
	\aftergroup \product
       }}

\def\r@duce{\ifdim\dimen0 > 90\r@dian \then   % sin(x+90) = sin(180-x)
		\multiply\dimen0 by -1
		\advance\dimen0 by 180\r@dian
		\r@duce
	    \else \ifdim\dimen0 < -90\r@dian \then  % sin(-x) = sin(360+x)
		\advance\dimen0 by 360\r@dian
		\r@duce
		\fi
	    \fi}

\def\Sine#1%
       {{%
	\dimen 0 = #1 \r@dian
	\r@duce
	\ifdim\dimen0 = -90\r@dian \then
	   \dimen4 = -1\r@dian
	   \c@mputefalse
	\fi
	\ifdim\dimen0 = 90\r@dian \then
	   \dimen4 = 1\r@dian
	   \c@mputefalse
	\fi
	\ifdim\dimen0 = 0\r@dian \then
	   \dimen4 = 0\r@dian
	   \c@mputefalse
	\fi
%
	\ifc@mpute \then
        	% convert degrees to radians
		\divide\dimen0 by 180
		\dimen0=3.141592654\dimen0
%
		\dimen 2 = 3.1415926535897963\r@dian %%% a well-known constant
		\divide\dimen 2 by 2 %%% we only deal with -pi/2 : pi/2
		\Mess@ge {Sin: calculating Sin of \nodimen 0}%
		\count 0 = 1 %%% see power-series expansion for sine
		\dimen 2 = 1 \r@dian %%% ditto
		\dimen 4 = 0 \r@dian %%% ditto
		\loop
			\ifnum	\dimen 2 = 0 %%% then we've done
			\then	\stillc@nvergingfalse 
			\else	\stillc@nvergingtrue
			\fi
			\ifstillc@nverging %%% then calculate next term
			\then	\term {\count 0} {\dimen 0} {\dimen 2}%
				\advance \count 0 by 2
				\count 2 = \count 0
				\divide \count 2 by 2
				\ifodd	\count 2 %%% signs alternate
				\then	\advance \dimen 4 by \dimen 2
				\else	\advance \dimen 4 by -\dimen 2
				\fi
		\repeat
	\fi		
			\xdef \sine {\nodimen 4}%
       }}

% Now the Cosine can be calculated easily by calling \Sine
\def\Cosine#1{\ifx\sine\UnDefined\edef\Savesine{\relax}\else
		             \edef\Savesine{\sine}\fi
	{\dimen0=#1\r@dian\advance\dimen0 by 90\r@dian
	 \Sine{\nodimen 0}
	 \xdef\cosine{\sine}
	 \xdef\sine{\Savesine}}}	      
% end of trig stuff
%%%%%%%%%%%%%%%%%%%%%%%%%%%%%%%%%%%%%%%%%%%%%%%%%%%%%%%%%%%%%%%%%%%%

\def\psdraft{
	\def\@psdraft{0}
	%\ps@typeout{draft level now is \@psdraft \space . }
}
\def\psfull{
	\def\@psdraft{100}
	%\ps@typeout{draft level now is \@psdraft \space . }
}

\psfull

\newif\if@scalefirst
\def\psscalefirst{\@scalefirsttrue}
\def\psrotatefirst{\@scalefirstfalse}
\psrotatefirst

\newif\if@draftbox
\def\psnodraftbox{
	\@draftboxfalse
}
\def\psdraftbox{
	\@draftboxtrue
}
\@draftboxtrue

\newif\if@prologfile
\newif\if@postlogfile
\def\pssilent{
	\@noisyfalse
}
\def\psnoisy{
	\@noisytrue
}
\psnoisy
%%% These are for the option list.
%%% A specification of the form a = b maps to calling \@p@@sa{b}
\newif\if@bbllx
\newif\if@bblly
\newif\if@bburx
\newif\if@bbury
\newif\if@height
\newif\if@width
\newif\if@rheight
\newif\if@rwidth
\newif\if@angle
\newif\if@clip
\newif\if@verbose
\def\@p@@sclip#1{\@cliptrue}


\newif\if@decmpr

%%% GDH 7/26/87 -- changed so that it first looks in the local directory,
%%% then in a specified global directory for the ps file.
%%% RPR 6/25/91 -- changed so that it defaults to user-supplied name if
%%% boundingbox info is specified, assuming graphic will be created by
%%% print time.
%%% TJD 10/19/91 -- added bbfile vs. file distinction, and @decmpr flag

\def\@p@@sfigure#1{\def\@p@sfile{null}\def\@p@sbbfile{null}
	        \openin1=#1.bb
		\ifeof1\closein1
	        	\openin1=\figurepath#1.bb
			\ifeof1\closein1
			        \openin1=#1
				\ifeof1\closein1%
				       \openin1=\figurepath#1
					\ifeof1
					   \ps@typeout{Error, File #1 not found}
						\if@bbllx\if@bblly
				   		\if@bburx\if@bbury
			      				\def\@p@sfile{#1}%
			      				\def\@p@sbbfile{#1}%
							\@decmprfalse
				  	   	\fi\fi\fi\fi
					\else\closein1
				    		\def\@p@sfile{\figurepath#1}%
				    		\def\@p@sbbfile{\figurepath#1}%
						\@decmprfalse
	                       		\fi%
			 	\else\closein1%
					\def\@p@sfile{#1}
					\def\@p@sbbfile{#1}
					\@decmprfalse
			 	\fi
			\else
				\def\@p@sfile{\figurepath#1}
				\def\@p@sbbfile{\figurepath#1.bb}
				\@decmprtrue
			\fi
		\else
			\def\@p@sfile{#1}
			\def\@p@sbbfile{#1.bb}
			\@decmprtrue
		\fi}

\def\@p@@sfile#1{\@p@@sfigure{#1}}

\def\@p@@sbbllx#1{
		%\ps@typeout{bbllx is #1}
		\@bbllxtrue
		\dimen100=#1
		\edef\@p@sbbllx{\number\dimen100}
}
\def\@p@@sbblly#1{
		%\ps@typeout{bblly is #1}
		\@bbllytrue
		\dimen100=#1
		\edef\@p@sbblly{\number\dimen100}
}
\def\@p@@sbburx#1{
		%\ps@typeout{bburx is #1}
		\@bburxtrue
		\dimen100=#1
		\edef\@p@sbburx{\number\dimen100}
}
\def\@p@@sbbury#1{
		%\ps@typeout{bbury is #1}
		\@bburytrue
		\dimen100=#1
		\edef\@p@sbbury{\number\dimen100}
}
\def\@p@@sheight#1{
		\@heighttrue
		\dimen100=#1
   		\edef\@p@sheight{\number\dimen100}
		%\ps@typeout{Height is \@p@sheight}
}
\def\@p@@swidth#1{
		%\ps@typeout{Width is #1}
		\@widthtrue
		\dimen100=#1
		\edef\@p@swidth{\number\dimen100}
}
\def\@p@@srheight#1{
		%\ps@typeout{Reserved height is #1}
		\@rheighttrue
		\dimen100=#1
		\edef\@p@srheight{\number\dimen100}
}
\def\@p@@srwidth#1{
		%\ps@typeout{Reserved width is #1}
		\@rwidthtrue
		\dimen100=#1
		\edef\@p@srwidth{\number\dimen100}
}
\def\@p@@sangle#1{
		%\ps@typeout{Rotation is #1}
		\@angletrue
%		\dimen100=#1
		\edef\@p@sangle{#1} %\number\dimen100}
}
\def\@p@@ssilent#1{ 
		\@verbosefalse
}
\def\@p@@sprolog#1{\@prologfiletrue\def\@prologfileval{#1}}
\def\@p@@spostlog#1{\@postlogfiletrue\def\@postlogfileval{#1}}
\def\@cs@name#1{\csname #1\endcsname}
\def\@setparms#1=#2,{\@cs@name{@p@@s#1}{#2}}
%
% initialize the defaults (size the size of the figure)
%
\def\ps@init@parms{
		\@bbllxfalse \@bbllyfalse
		\@bburxfalse \@bburyfalse
		\@heightfalse \@widthfalse
		\@rheightfalse \@rwidthfalse
		\def\@p@sbbllx{}\def\@p@sbblly{}
		\def\@p@sbburx{}\def\@p@sbbury{}
		\def\@p@sheight{}\def\@p@swidth{}
		\def\@p@srheight{}\def\@p@srwidth{}
		\def\@p@sangle{0}
		\def\@p@sfile{} \def\@p@sbbfile{}
		\def\@p@scost{10}
		\def\@sc{}
		\@prologfilefalse
		\@postlogfilefalse
		\@clipfalse
		\if@noisy
			\@verbosetrue
		\else
			\@verbosefalse
		\fi
}
%
% Go through the options setting things up.
%
\def\parse@ps@parms#1{
	 	\@psdo\@psfiga:=#1\do
		   {\expandafter\@setparms\@psfiga,}}
%
% Compute bb height and width
%
\newif\ifno@bb
\def\bb@missing{
	\if@verbose{
		\ps@typeout{psfig: searching \@p@sbbfile \space  for bounding box}
	}\fi
	\no@bbtrue
	\epsf@getbb{\@p@sbbfile}
        \ifno@bb \else \bb@cull\epsf@llx\epsf@lly\epsf@urx\epsf@ury\fi
}	
\def\bb@cull#1#2#3#4{
	\dimen100=#1 bp\edef\@p@sbbllx{\number\dimen100}
	\dimen100=#2 bp\edef\@p@sbblly{\number\dimen100}
	\dimen100=#3 bp\edef\@p@sbburx{\number\dimen100}
	\dimen100=#4 bp\edef\@p@sbbury{\number\dimen100}
	\no@bbfalse
}
% rotate point (#1,#2) about (0,0).
% The sine and cosine of the angle are already stored in \sine and
% \cosine.  The result is placed in (\p@intvaluex, \p@intvaluey).
\newdimen\p@intvaluex
\newdimen\p@intvaluey
\def\rotate@#1#2{{\dimen0=#1 sp\dimen1=#2 sp
%            	calculate x' = x \cos\theta - y \sin\theta
		  \global\p@intvaluex=\cosine\dimen0
		  \dimen3=\sine\dimen1
		  \global\advance\p@intvaluex by -\dimen3
% 		calculate y' = x \sin\theta + y \cos\theta
		  \global\p@intvaluey=\sine\dimen0
		  \dimen3=\cosine\dimen1
		  \global\advance\p@intvaluey by \dimen3
		  }}
\def\compute@bb{
		\no@bbfalse
		\if@bbllx \else \no@bbtrue \fi
		\if@bblly \else \no@bbtrue \fi
		\if@bburx \else \no@bbtrue \fi
		\if@bbury \else \no@bbtrue \fi
		\ifno@bb \bb@missing \fi
		\ifno@bb \ps@typeout{FATAL ERROR: no bb supplied or found}
			\no-bb-error
		\fi
		%
%\ps@typeout{BB: \@p@sbbllx, \@p@sbblly, \@p@sbburx, \@p@sbbury} 
%
% store height/width of original (unrotated) bounding box
		\count203=\@p@sbburx
		\count204=\@p@sbbury
		\advance\count203 by -\@p@sbbllx
		\advance\count204 by -\@p@sbblly
		\edef\ps@bbw{\number\count203}
		\edef\ps@bbh{\number\count204}
		%\ps@typeout{ psbbh = \ps@bbh, psbbw = \ps@bbw }
		\if@angle 
			\Sine{\@p@sangle}\Cosine{\@p@sangle}
	        	{\dimen100=\maxdimen\xdef\r@p@sbbllx{\number\dimen100}
					    \xdef\r@p@sbblly{\number\dimen100}
			                    \xdef\r@p@sbburx{-\number\dimen100}
					    \xdef\r@p@sbbury{-\number\dimen100}}
%
% Need to rotate all four points and take the X-Y extremes of the new
% points as the new bounding box.
                        \def\minmaxtest{
			   \ifnum\number\p@intvaluex<\r@p@sbbllx
			      \xdef\r@p@sbbllx{\number\p@intvaluex}\fi
			   \ifnum\number\p@intvaluex>\r@p@sbburx
			      \xdef\r@p@sbburx{\number\p@intvaluex}\fi
			   \ifnum\number\p@intvaluey<\r@p@sbblly
			      \xdef\r@p@sbblly{\number\p@intvaluey}\fi
			   \ifnum\number\p@intvaluey>\r@p@sbbury
			      \xdef\r@p@sbbury{\number\p@intvaluey}\fi
			   }
%			lower left
			\rotate@{\@p@sbbllx}{\@p@sbblly}
			\minmaxtest
%			upper left
			\rotate@{\@p@sbbllx}{\@p@sbbury}
			\minmaxtest
%			lower right
			\rotate@{\@p@sbburx}{\@p@sbblly}
			\minmaxtest
%			upper right
			\rotate@{\@p@sbburx}{\@p@sbbury}
			\minmaxtest
			\edef\@p@sbbllx{\r@p@sbbllx}\edef\@p@sbblly{\r@p@sbblly}
			\edef\@p@sbburx{\r@p@sbburx}\edef\@p@sbbury{\r@p@sbbury}
%\ps@typeout{rotated BB: \r@p@sbbllx, \r@p@sbblly, \r@p@sbburx, \r@p@sbbury}
		\fi
		\count203=\@p@sbburx
		\count204=\@p@sbbury
		\advance\count203 by -\@p@sbbllx
		\advance\count204 by -\@p@sbblly
		\edef\@bbw{\number\count203}
		\edef\@bbh{\number\count204}
		%\ps@typeout{ bbh = \@bbh, bbw = \@bbw }
}
%
% \in@hundreds performs #1 * (#2 / #3) correct to the hundreds,
%	then leaves the result in @result
%
\def\in@hundreds#1#2#3{\count240=#2 \count241=#3
		     \count100=\count240	% 100 is first digit #2/#3
		     \divide\count100 by \count241
		     \count101=\count100
		     \multiply\count101 by \count241
		     \advance\count240 by -\count101
		     \multiply\count240 by 10
		     \count101=\count240	%101 is second digit of #2/#3
		     \divide\count101 by \count241
		     \count102=\count101
		     \multiply\count102 by \count241
		     \advance\count240 by -\count102
		     \multiply\count240 by 10
		     \count102=\count240	% 102 is the third digit
		     \divide\count102 by \count241
		     \count200=#1\count205=0
		     \count201=\count200
			\multiply\count201 by \count100
		 	\advance\count205 by \count201
		     \count201=\count200
			\divide\count201 by 10
			\multiply\count201 by \count101
			\advance\count205 by \count201
			%
		     \count201=\count200
			\divide\count201 by 100
			\multiply\count201 by \count102
			\advance\count205 by \count201
			%
		     \edef\@result{\number\count205}
}
\def\compute@wfromh{
		% computing : width = height * (bbw / bbh)
		\in@hundreds{\@p@sheight}{\@bbw}{\@bbh}
		%\ps@typeout{ \@p@sheight * \@bbw / \@bbh, = \@result }
		\edef\@p@swidth{\@result}
		%\ps@typeout{w from h: width is \@p@swidth}
}
\def\compute@hfromw{
		% computing : height = width * (bbh / bbw)
	        \in@hundreds{\@p@swidth}{\@bbh}{\@bbw}
		%\ps@typeout{ \@p@swidth * \@bbh / \@bbw = \@result }
		\edef\@p@sheight{\@result}
		%\ps@typeout{h from w : height is \@p@sheight}
}
\def\compute@handw{
		\if@height 
			\if@width
			\else
				\compute@wfromh
			\fi
		\else 
			\if@width
				\compute@hfromw
			\else
				\edef\@p@sheight{\@bbh}
				\edef\@p@swidth{\@bbw}
			\fi
		\fi
}
\def\compute@resv{
		\if@rheight \else \edef\@p@srheight{\@p@sheight} \fi
		\if@rwidth \else \edef\@p@srwidth{\@p@swidth} \fi
		%\ps@typeout{rheight = \@p@srheight, rwidth = \@p@srwidth}
}
%		
% Compute any missing values
\def\compute@sizes{
	\compute@bb
	\if@scalefirst\if@angle
% at this point the bounding box has been adjsuted correctly for
% rotation.  PSFIG does all of its scaling using \@bbh and \@bbw.  If
% a width= or height= was specified along with \psscalefirst, then the
% width=/height= value needs to be adjusted to match the new (rotated)
% bounding box size (specifed in \@bbw and \@bbh).
%    \ps@bbw       width=
%    -------  =  ---------- 
%    \@bbw       new width=
% so `new width=' = (width= * \@bbw) / \ps@bbw; where \ps@bbw is the
% width of the original (unrotated) bounding box.
	\if@width
	   \in@hundreds{\@p@swidth}{\@bbw}{\ps@bbw}
	   \edef\@p@swidth{\@result}
	\fi
	\if@height
	   \in@hundreds{\@p@sheight}{\@bbh}{\ps@bbh}
	   \edef\@p@sheight{\@result}
	\fi
	\fi\fi
	\compute@handw
	\compute@resv}

%
% \psfig
% usage : \psfig{file=, height=, width=, bbllx=, bblly=, bburx=, bbury=,
%			rheight=, rwidth=, clip=}
%
% "clip=" is a switch and takes no value, but the `=' must be present.
\def\psfig#1{\vbox {
	% do a zero width hard space so that a single
	% \psfig in a centering enviornment will behave nicely
	%{\setbox0=\hbox{\ }\ \hskip-\wd0}
	%
	\ps@init@parms
	\parse@ps@parms{#1}
	\compute@sizes
	%
	\ifnum\@p@scost<\@psdraft{
		%
		\special{ps::[begin] 	\@p@swidth \space \@p@sheight \space
				\@p@sbbllx \space \@p@sbblly \space
				\@p@sbburx \space \@p@sbbury \space
				startTexFig \space }
		\if@angle
			\special {ps:: \@p@sangle \space rotate \space} 
		\fi
		\if@clip{
			\if@verbose{
				\ps@typeout{(clip)}
			}\fi
			\special{ps:: doclip \space }
		}\fi
		\if@prologfile
		    \special{ps: plotfile \@prologfileval \space } \fi
		\if@decmpr{
			\if@verbose{
				\ps@typeout{psfig: including \@p@sfile.Z \space }
			}\fi
			\special{ps: plotfile "`zcat \@p@sfile.Z" \space }
		}\else{
			\if@verbose{
				\ps@typeout{psfig: including \@p@sfile \space }
			}\fi
			\special{ps: plotfile \@p@sfile \space }
		}\fi
		\if@postlogfile
		    \special{ps: plotfile \@postlogfileval \space } \fi
		\special{ps::[end] endTexFig \space }
		% Create the vbox to reserve the space for the figure.
		\vbox to \@p@srheight sp{
		% 1/92 TJD Changed from "true sp" to "sp" for magnification.
			\hbox to \@p@srwidth sp{
				\hss
			}
		\vss
		}
	}\else{
		% draft figure, just reserve the space and print the
		% path name.
		\if@draftbox{		
			% Verbose draft: print file name in box
			\hbox{\frame{\vbox to \@p@srheight sp{
			\vss
			\hbox to \@p@srwidth sp{ \hss \@p@sfile \hss }
			\vss
			}}}
		}\else{
			% Non-verbose draft
			\vbox to \@p@srheight sp{
			\vss
			\hbox to \@p@srwidth sp{\hss}
			\vss
			}
		}\fi	



	}\fi
}}
\psfigRestoreAt
\let\@=\LaTeXAtSign



 \pagestyle{headings} \makeindex
\begin{document}


\title{An Evaluation of Flashmail: \\ a computer-mediated communication
  tool} \author{Jennifer Geis\\ \\ Collaborative Software Development
  Laboratory\\ Department of Information and Computer Sciences\\ University
  of Hawaii at Manoa\\ Honolulu, HI 96822\\ (808) 956-6920\\ {\sf
    jgeis@uhics.ics.hawaii.edu}\\ {\sf ICS-TR-95-21}}\maketitle

\begin{abstract}
  This paper presents the results from an analysis of a new
  computer-mediated communication tool called Flashmail.  I investigated
  how people used Flashmail as well as Flashmail's relationship to
  conventional electronic mail.  Participants in the experiment loaded
  extensions that gathered data regarding the characteristics of all
  messages sent through E-mail and Flashmail.  This data was used 
  to analyze the conditions under which each system was used.  I found that
  Flashmail seems to be preferred whenever the message is short, needs to
  be communicated in a short period of time, and when both the recipient
  and the sender are logged into the system and active at the time of
  sending.  In contrast, I found that E-mail was preferred for messages
  that were large (over 400 characters) and non-urgent, or when the
  receiver was either not logged into Flashmail or had been idle for longer
  than 7 minutes.  These results indicate that Flashmail is generally used
  as a rapid, synchronous messaging method.
\end{abstract}
\newpage
\tableofcontents
\newpage \ls{1.0}


\section{Introduction}
\subsection{Computer-mediated communication} 
This paper presents the results of an evaluation of a new computer-mediated
communication system called Flashmail. Computer-mediated communication
(CMC) is the use of computers to facilitate human-human interaction.
Systems of this type use computers to structure, store, and process
communications.  

Computer-mediated communication changes the way we interact with each
other.  The following quote by Robert Sproull draws a parallel between CMC
and a non-computer-mediated communication tool, the telephone:

\begin{quote}
  The telephone had extensive and unanticipated effects in part because it
  routinely extended attention, social contacts, and interdependencies
  beyond patterns determined by physical proximity. Reducing the
  constraints of physical proximity increased people's choice of
  interactions, whether with family members who had migrated from the farm
  to the city, or distant employees, or the boss. Amplification occurred
  because communication networks have a mutually causal, spiraling
  relationship with information networks, close relationships, conformity,
  and cultural change.... Our research demonstrates that, as the telephone
  did, new computer-based communication technology in some organizations is
  changing attention, social contact patterns, and interdependencies.
  \cite{Sproull93}
\end{quote}

Similar to the telephone, CMC allows people to communicate more information
in a shorter amount of time, regardless of physical settings.  For example,
suppose a student wishes to ask a question in regards to an assignment.  If
the student cannot locate the professor, the student can send the question
to the professor through electronic mail (E-mail).  The professor can then
respond to the question through the same medium.

With the use of a computer and a modem an individual can conduct many
activities without leaving their home.  Although the common use of CMC is
to further communication attempts within a business setting, many more
possibilities are becoming available.  Currently, people conduct bank
transactions, do research, read journals and articles, reserve books at the
library, get the daily news, look at movie listings, and even order pizza
using CMC.

In this paper, I focus on two CMC tools: Electronic mail and Flashmail.

\subsection{Electronic mail}
Electronic mail is ``the electronic, one-directional transfer of
information in the form of a message, via an intermediate
(tele-)communication system, from an identified sending party to one or
more identified receiving parties.''  \cite{Vervest85}

E-mail allows an individual to send an electronic message to one or more
receivers.  The recipient(s) of the message have the options of replying to
the message, deleting it, saving it, forwarding it, or simply doing nothing
with it and leaving it in their mailbox.

E-mail is characterized by several behaviors.  First, E-mail is an
asynchronous messaging system.  What this means is that messages that are
sent are not received and read in the same time frame. E-mail messages are
stored in the recipient's ``in-box.'' Regardless of when the message was
originally sent, the receiver will not see the contents of the message
until they make a conscious decision to ``open'' their in-box and read
their mail.  Hence, E-mail messages do not interrupt the receiver and the
receiver has total control of when they read their messages.  As a result,
the sender has no way of knowing when they will 
get a reply to their message. The sender just has to wait until the
receiver gets around to reading their mail and issuing any needed replies. The time
lapsed between when a message is sent and when it is replied to can vary
from seconds to days.

Another feature of E-mail is that it allows saving and categorizing of
messages.  This is useful for messages that the receiver wants to
save for future reference, or when the message contains an attachment.
Attachments are files that are appended onto
an E-mail message.  For example, if you want to send someone a copy of a
particularly large file for them to use later, you don't really want to
include it in the message itself.  By utilizing the attachment feature, the
file is kept separate from the message itself, but can be saved by the
receiver.

A final relevant feature of E-mail is the ability to
address a single E-mail message to several people at the same time, and
reply to all the receivers of an E-mail.  This creates a form of transient,
virtual ``community'' or discussion group.


\subsection{Flashmail: an alternative to E-mail}
Although E-mail has many strengths as a CMC tool, it also has some
weaknesses. One problem with E-mail is lack of support for time-dependent
messages.  Time-dependent messages are those whose usefulness expires after
a short period of time when they are not read.  For example, say that you
want to hold a meeting with colleagues in one hour. You can send them an
E-mail, but you have no guarantee that everyone will read their E-mail
within the hour; furthermore, if one of the recipients does not read their
mail before the meeting time, the message no longer serves any purpose and
simply clutters up their mailbox.  

The Flashmail CMC system is designed to handle this type of situation in
which E-mail is no longer adequate.  Flashmail transfers messages between a
small, pre-selected group of people, and can virtually guarantee that a
message sent to group members will be seen immediately in certain
situations. 

Flashmail is a useful computer-mediated communication tool as it is a
mechanism to facilitate transferring of messages, especially those that are
time-dependent in nature.  The hypothesis of this research is that people
will prefer Flashmail over E-mail for this type of communication whenever
certain criteria are met: that the message is short, and the intended
recipient has a very low idle-time. 

Flashmail displays the idle time associated with every user in the group,
as well as the machine they are working on. This enables everyone to know
who is logged in, where they are, and whether or not they are in front of
their terminal. For instance, if the idle time associated with an
individual is ``0 seconds,'' you can be reasonably sure that if you send
them a Flashmail message, they will see it immediately.  However, if the
displayed idle time reads ``1 hour 20 seconds,'' they are probably not in
front of their computer screen. 

\subsubsection{Some Flashmail features and their comparison to E-mail}

In addition to displaying all connected group member's idle-times,
Flashmail has a few other features to facilitate rapid messaging.

\begin{itemize}
\item Flashmail is deliberately intrusive and interrupts whatever the
  recipient is doing.  This is a significant difference between E-mail and
  Flashmail.  E-mail stores the messages for the receiver, and does not
  necessarily inform the receiver that a message has arrived.

\item Flashmail mandates that the receiver read their message immediately,
  while E-mail lets the receiver decide when they will read their mail.
  When a Flashmail message is sent, a screen pops up on the receiver's
  terminal containing the message, so the recipient is forced to view the
  message immediately.  In E-mail, messages collect in what is called an
  ``inbox'' and remain there until the recipient chooses to ``open'' the
  box and read their mail.

\item Flashmail helps the sender shorten the latency of response.
  Flashmail messages are displayed in less that a second after being sent.
  Once displayed, the window takes up desktop space, so usually the most convenient
  thing for the recipient to do is issue a response immediately and
  delete the window.  As a result, Flashmail messages tend to receive any
  needed replies fairly quickly.  Since an E-mail recipient
  reads their messages at their leisure, the sender has to wait until the
  receiver opens their mailbox, reads the message and decides to respond to
  it.

\item Flashmail does not support the saving and categorizing of
  messages.  Flashmail has the ability to save messages, but it is not as
  convenient as saving E-mail messages. In fact, by default Flashmail messages
  are deleted.  Most E-mail systems allow a message recipient to
  save a message and categorize it for easy future reference.

\item Attachments of other files is not provided in Flashmail. To send
  files using Flashmail, one has to cut and paste the file contents into
  the Flashmail message buffer. E-mail allows the attachment of many files
  to a message through an easy to use mechanism.

\item Both Flashmail and E-mail messages can be addressed to multiple
  receivers. The ability to send the same message to several receivers at
  once saves a sender the hassle of having to re-address and send out the
  same message several times, and creates virtual discussion groups.

\item Flashmail is a synchronous messaging system.  What this means is that
  the time between when a Flashmail message is sent and when it is displayed is
  a second or less.  E-mail is an asynchronous messaging system, so the
  time lapsed between when an E-mail message is sent and when it is read
  can be seconds, minutes, or even days. 

\item In order for someone to be part of a Flashmail group, they must be
  explicitly invited and added to the system.  Anyone with an E-mail
  address can send anyone else a message.  Flashmail is a much more private
  tool and assumes that you personally know the people with which you are
  going to be communicating.
\end{itemize}

\begin{figure}[htpb]
 \caption{Summary of the system characteristics (Flashmail vs E-mail)}
  \begin{center}
  \begin{tabular}{|l|c|c|}
    \hline Features & E-mail & Flashmail \\ \hline Intrusiveness of
    messages & non-intrusive & Deliberately intrusive \\ Control over when
    receiver reads messages & Receiver controls & Sender controls \\ 
    Control over immediacy of response & No control & Sender has influence
    \\ Saving and categorizing of messages & Available & Not available
    \\ Attachment of other files & Available & Not available \\ Multiple
    receivers & Facilitated & Facilitated \\ Asynchronous/Synchronous &
    Asynchronous & Synchronous \\ Membership & Anyone & Must be explicitly
    invited \\ \hline
   \end{tabular}
  \end{center}
 \label{characteristics}
\end{figure}


\subsection{Experiment}
The purpose of this research was to test the conditions under which one
system was used to communicate a message instead of the other.  The time
period in which the experiment was conducted was from March 11 through
April 15, and the participants were the members of the Collaborative
Software Development Laboratory (CSDL).

All experimental data was collected through the Flashmail and VM systems
automatically. The data collected included the time and date at which a
Flashmail or E-mail message was sent, who the senders and receivers were,
the time and date at which a reply to a previous message was sent, and the
buffer sizes of these messages. The data was used to determine the
circumstances under
which one system was utilized over the other.  It was found that Flashmail
messages tend to be short (under 300 characters), and that Flashmail
messages receive replies much more rapidly than E-mail.

Section Two of this paper describes the history and design of the Flashmail
system.  This section provides an illustration of the Flashmail user
interface along with some of the more important features of Flashmail.  

The next section discusses the Flashmail evaluation experience.  Here, I show
sample data from the Flashmail and E-mail systems, experimental results,
and possible explanations for these results.  I also provide a brief
summary comparing the two systems based upon the experimental results. 

The conclusion examines the results of this study and their implications. 

\section{Flashmail}
\subsection{Design}
Flashmail is a system designed to compliment E-mail and was
implemented in 1995 at the University of Hawaii at Manoa by Professor
Philip Johnson.  Flashmail allows synchronous electronic messaging through
a pop-up window mechanism and was developed to assist inter-group
communications.  It was designed using Egret, a framework for
computer-supported cooperative work (CSCW) applications. Egret implements a
multi-client, multi-server, multi-agent architecture. Egret clients and
agents are implemented by a 15 KLOC extension to XEmacs, the X-window Emacs
editor.  Egret servers are implemented by a 15 KLOC system written in C++.
\cite{Egret96}

Flashmail is a real-time communication method complimentary to E-mail.
Flashmail's main feature is that it is deliberately intrusive upon the
receiver.  This feature allows the sender to assume the receiver has seen
the message, as the transmission will be the first thing the receiver sees
when they look at their computer screen.  In contrast to E-mail, this allows
an efficient method of transmitting time-dependent messages.

\subsection{Features}
\begin{itemize}
\item Flashmail indicates which group members are currently logged in to a
  workstation. This determines to whom a user can and cannot send a Flashmail
  message.  If a user is not logged in, the system will not allow a
  message to be sent to them.

\item If a Flashmail group member is logged in, the idle-time associated
  with their workstation is given. This is extremely useful for knowing if
  a Flashmail member will see your message immediately.  If the idle time
  displayed reads ``0 seconds,'' then it is most likely that the recipient
  will see the message as soon as it is sent.  On the other hand, if the
  idle-time reads ``2 hours,'' the individual has not been near their
  machine in quite a while and will not see the message until they return.

\item The Flashmail system gives real-time information on which workstation
  each group member is using.  This feature allows other group members to
  know where everybody else is located.  Every machine has a name, and if
  you know where that particular machine is located, you know where to find
  that group member.  The utility of this is that it gives group members
  who work in different parts of a building the ability to locate other
  members without having to physically track them down.

\item Flashmail is deliberately intrusive, which distinguishes it from many
  other CMC tools.  Flashmail has a mechanism that pops up a window
  containing the message on the receiver's screen.  This window will appear
  over whatever the receiver is doing.  In this respect, it is intrusive,
  but it assures that the message will be seen immediately.  Should a group
  member be particularly busy and wish to ensure that they will not be
  interrupted, Flashmail can be turned off, thereby preventing anyone from
  sending them a Flashmail message.

\item Senders of Flashmail messages can assume their receivers have seen
  the message. As mentioned above, the idle-time associated with each user
  is displayed. If the idle time reads ``0 seconds'' at the time you
  send a user a Flashmail message, you can assume they will see the
  message immediately.   The reason for this is Flashmail has told you that
  the receiver is currently active on their machine and you
  also know that the message appears over whatever they are working on.

\item Flashmail messages are delivered and received practically
  instantaneously. In contrast to E-mail, where the delivery is sometimes
  unpredictable, Flashmail messages appear on the recipients screen almost
  immediately after being sent.  A sender can rest assured that once a
  Flashmail message is sent, it appears on the receiver's screen almost
  instantaneously. 

\end{itemize}

To be able to use Flashmail, you must be added to the Flashmail group by
whoever is acting as the Flashmail system administrator.  You need to be
given a password and added into the Flashmail menu.  If you want to create
a new Flashmail group, you first establish who all the group members are
going to be and then have the system administrator set up a new Flashmail
group instance with your group name. You can be part of multiple Flashmail
groups simultaneously.

Once you are a member of a Flashmail group, to start using Flashmail, you
must be using XEmacs.  There is a menu item that can be added to your
XEmacs sessions which allows you to start up Flashmail.  By selecting this
menu button, you will be connected to the Flashmail server.  While you are
connecting, all other connected Flashmail users will be notified of your
messaging availability via a mini-buffer message.  At this time the
Flashmail main-menu will appear in your menu-bar. If you decide that you do
not want to be available to receive Flashmail messages, you can simply
disconnect and all connected group members will receive notification that
you have quit the Flashmail session.

\newpage
\subsection{User Interface}
An example Flashmail main menu is shown in Figure 2.  In this case, the
users are \textit{dat}, \textit{cmoore}, \textit{johnson}, \textit{jgeis},
\textit{rosea}, \textit{julio}, and \textit{russ}. \textit{Cmoore}, \textit{rosea}, \textit{julio}, and
\textit{russ} are dimmed because they are either not logged on or they have
not activated Flashmail.  User \textit{jgeis} is dimmed because she is the
current flashmail user. The current user is always dimmed as one cannot
send a flashmail to oneself. \textit{Dat}, \textit{johnson}, and
\textit{jgeis} are all logged in and can be sent flashmail messages.  For
these three users, Flashmail displays the workstations they are using and
their respective idle-times.  In this instance, user \textit{dat} is using
the workstation named bianca and has been idle for 25 seconds.

To illustrate how Flashmail is utilized, suppose \textit{jgeis} wants to send user
\textit{dat} a Flashmail message.  First, the sender selects \textit{dat} from the
available users in the Flashmail menu, notice that the box next to \textit{dat's}
name is darkened, indicating the selection.  One can also send the message
to multiple recipients if so desired.

\begin{figure}[htb]



  {\centerline{\psfig{figure=/group/csdl/techreports/95-21/figures/mainmenu.ps}}}
  \caption{The Flashmail Main Menu}
  \label{Main-menu}
\end{figure}


A window as in Figure 3 then pops up on the sender's screen and the
message to be sent is typed in. A click on the ``send'' button delivers the
message.

\begin{figure}[htb]



  {\centerline{\psfig{figure=/group/csdl/techreports/95-21/figures/compose.ps}}}
  \caption{Composing a Flashmail Message}
  \label{Compose}
\end{figure}

At this time, a window appears on \textit{dat's} screen containing the original
message.  The recipient has the options of either deleting the message or
replying to it.  In this instance, the message necessitates a reply.
Figure 4 is \textit{dat's} response, where he included the original message by
means of a ``cite'' mechanism shown in the reply-menu of Figure 5.

\begin{figure}[htb]



  {\centerline{\psfig{figure=/group/csdl/techreports/95-21/figures/receive.ps}}}
  \caption{A Reply to the Original Flashmail Message}
  \label{Reply}
\end{figure}


\begin{figure}[htb]



  {\centerline{\psfig{figure=/group/csdl/techreports/95-21/figures/receive-menu.ps}}}
  \caption{The Reply Menu Options}
  \label{Reply-menu}
\end{figure}

Although Flashmail and E-mail are both systems designed to facilitate
messaging, they are quite different in functionality.  Note that the
Flashmail main menu lists the individuals to whom you can send a message.
With Flashmail, a person must be explicitly added to the Flashmail menu and
given a password. Unless you are a member of the group, you can not send or
receive Flashmail messages.

In the example above,
the message asks ``Do you want to meet for lunch.''  This type of message
exemplifies several of the common characteristics of a typical Flashmail
message.  First, the message is short.  In my study, the average Flashmail message was
roughly 200 characters.  Next, the message needs a quick reply, and got
one.  Many Flashmail messages are sent with the aim of obtaining a response
in a very short period of time.  The message in this example
received a rapid response.  In the study, the average time for a user to
receive a response to a Flashmail message they sent was under five minutes.

Notice that there is no ``save'' button on any of the Flashmail buffers.
Since Flashmail messages tend to be short and involve messages that expire,
to implement a ``save'' mechanism would be contrary to the purpose of
sending a Flashmail message.

One final difference between Flashmail and E-mail that is illustrated in
this example is the ability to know where everyone is and whether or not
you can reach them.  Since the Flashmail main menu lists the machines that
people are using and their associated idle-times, you know where those
people are physically and whether or not they would immediately see a
Flashmail message that you send.  If someone was to send \textit{jgeis} a message
while her idle-time reads ``0 sec,'' they could be reasonably sure that she
has seen the message within a second after it was sent.

\section{Evaluation of Flashmail}

Flashmail was evaluated using data I collected through the Flashmail and
E-mail systems.  I collected data from both systems which consisted of
whether or not the message was an original message or a reply to a previous
message, the time lapse between the sending of the original message and the
sending of its reply (if any), and the message's buffer size.  For E-mail
messages I also collected data on whether or not the sender was logged into
Flashmail, and whether or not the recipient was logged into Flashmail.  If
the E-mail recipient was logged into Flashmail, I also recorded their
idle-time.  If a Flashmail message was sent, I recorded the receiver's
idle-time in addition to the data gathered from both systems.


This is an example of some of the E-mail data that I collected:
\begin{verbatim}              

date/time    sender  Fmail  date/time     receiver   Fmail   size
    
3/29 12:45   dat     t      nil           russ       nil     634

4/1  14:51   russ    t      3/29 12:45    dat        22m     1519


\end{verbatim}

The first row means that user \textit{dat} sent an original E-mail message
to \textit{russ} on March 29th at 12:45 in the afternoon.  \textit{Dat} was
logged into Flashmail as indicated by the 't' under the first 'Fmail'
column.  \textit{Russ} was not logged into Flashmail as shown by the 'nil'
in the second 'Fmail' column.  The message was 634 characters in length.
The 'nil' in the second 'date/time' column indicates that the message is an
original. 

The second row is \textit{russ'} reply to \textit{dat's} message.  \textit{Russ'} message was sent on
April 1st at 14:51 (2:51 P.M.).  At the time of sending \textit{russ} and \textit{dat} were
both logged into Flashmail. As \textit{Dat} was the recipient of the Flashmail message
and was logged into Flashmail, his idle-time is listed which in this case
was 22 minutes.  \textit{Russ'} message was 1519 characters in length.  The
second 'date/time'
column shows when the original message was sent that \textit{russ} is replying to.
\newpage For Flashmail, below is an example of some of the data that I
gathered.

\begin{verbatim}

date/time   sender    recipient   idle     reply?   size   delay

3/12 15:42  rosea     jgeis       1m 56s   nil      37     nil 
                      johnson     35m 
                      cmoore      1h 19m 
                      dat         0s

3/12 15:43  dat       rosea       0s       t        53     1m 
                      jgeis       0s 
                      johnson     36m
                      cmoore      1h 20m

\end{verbatim}

This example show the sending of two sequential Flashmail messages.  The
first message was sent on March 12th at 15:42 (3:42 P.M.) from user \textit{rosea}
to \textit{jgeis}, \textit{johnson}, \textit{cmoore}, and \textit{dat}.  Each recipient's idle-time is listed
with \textit{dat} having the lowest time of zero seconds and \textit{cmoore} having the
highest at over an hour.  There is a nil under the reply column which means
that the message is an original and is not a result of a previous message.
The buffer size is 37 characters.  The delay column shows the time
difference between when an original message was sent and when the reply was
sent, since this message is the original, there is no delay associated with
it.

The second item shows that \textit{dat} responded to rosea's message at 15:43.  As a
result, the delay column shows a delay of one minute, meaning that is how
much time elapsed between when the original was sent and when the reply was
sent. \textit{Dat} chose to send his reply to all receivers of the original message.
In this case the idle-times of
\textit{johnson} and \textit{cmoore} have increased by one minute, and \textit{jgeis} has become
active resulting in an idle-time of zero seconds.  The size of the message
was 53 characters.

\begin{figure}[htpb]
 \caption{Summary of the experimental results (E-mail vs. Flashmail)}
  \begin{center}
  \begin{tabular}{|l|c|c|}
    \hline Categories & E-mail & Flashmail \\ \hline Average buffer size
    (characters)& 1428.2 & 210.4 \\ Avg. time between sent message and its'
    reply & 15.9 hrs.  & 4.6 min.  \\ \% of time sender logged into
    Flashmail & 81.6 & n/a \\ \% of time recipient logged into Flashmail &
    22.4 & n/a \\ Avg. Flashmail idle-time of recipient & 16.7 min.  & 6.7
    min.  \\ \% of messages sent that were original & 28.3 & 42.1 \\ \% of
    messages sent that were replies & 71.7 & 57.9 \\ \hline
   \end{tabular}
  \end{center}
 \label{results}
\end{figure}

The reason that Flashmail data for ``percent of time the sender/receiver
was logged into Flashmail'' is blank, is that both the sender and receiver
must be logged into Flashmail for a message to be sent.

\subsection{Analysis}

My original hypothesis was that once individuals were introduced to
Flashmail, they would prefer it over E-mail for the communication of
time-dependent messages within their Flashmail group.  After attempting to
implement a method through which I would justify my hypothesis, I realized
that the statement was far too vague.  Eventually, I settled on learning
about the conditions under which Flashmail was utilized over E-mail.  I
hypothesized that Flashmail is used primarily for messages that have a greater
dependency on time than most electronic communications, and I set about
attempting to evaluate this.  Although I don't believe that I have ``proved''
this hypothesis, I believe the results of my research lend it credence.

According to my data, the circumstances under which an individual utilized
one system over another varied.  However, the primary characteristics of a
Flashmail message were these:
\begin{itemize}
\item \textsl{Flashmail messages are short.}  On average, a VM message was seven
  times larger than a Flashmail message.

  The most probable reason for the difference in buffer size is the nature
  of the two mediums.  A Flashmail message, intrusive by nature, demands the
  immediate attention of the recipient.  To send a rather lengthy message
  via Flashmail would be equivalent to dropping a several-page long paper
  on someone's desk and ordering them to read it immediately. It can be
  perceived as inconsiderate and irritating depending on the importance of
  the message being conveyed.

  As for VM messages, they are stored in the receiver's ``inbox'' for them
  to read later at their own convenience.  If a message is particularly
  lengthy, the reader has the option of saving the message and reading it
  in portions as time permits.  Although this ``save to read later'' option
  is possible in Flashmail, it is extremely inconvenient.

\item \textsl{Recipients of Flashmail messages have low idle-times.}  The average
  idle-time for a recipient of a Flashmail message was 6.7 minutes, while
  the average Flashmail idle-time for a E-mail recipient (where both the
  sender and receiver were logged into Flashmail) was 16.7 minutes.

  I had actually expected the difference in idle-times to be somewhat
  greater, but I think the results are reasonable.  However, I must mention
  that the calculated idle-time varies depending on what method of
  computation you use.  When a Flashmail message was sent to more than one
  person, The idle times sometimes varied quite drastically.  For example a
  message could be sent to two users, one whose idle-time is 0 seconds, and
  another whose idle-time reads 6 hours.  In general, most Flashmail
  messages were sent to individuals with idle-times of 0 seconds, or a few
  minutes at most. In calculating the data for this, I included all
  idle-times from every recipient.  If I just included just the idle-times
  from messages to one individual only, the calculated idle-time average
  would be 1.7 minutes.

\item \textsl{Flashmail messages are responded to much more quickly than E-mail
  messages.}  The time elapsed between when a VM message is sent and when
  its reply is issued is over 200 times that for a Flashmail message.
  (Time elapsed for VM = 15.9 hours, Flashmail = 4.6 minutes)

  A Flashmail message appears immediately on the recipient's screen,
  practically requiring an immediate response.  If the receiver is not
  directly in front of their computer at the time the message is sent, the
  message will still be the first thing they see when they return to their
  computer.  In addition, Flashmail beeps when a message is received, so if
  the receiver is close by, but not looking at their computer, they are
  audibly notified of the incoming message.

  As for getting the reply quickly, since a flashmail message only has two
  real options, ``delete'' and ``reply'', the receiver either has to send
  their reply immediately, or push the buffer aside and have it take up
  window space until they do reply.  In this respect, it is much more
  convenient to issue the reply, delete the window, and go on with whatever
  you were doing than to wait until later to respond.

  Since VM messages wait until the recipient opens their inbox to read
  their mail, it could be hours or days before the message is read.  Even
  when the message is read, there is no guarantee that the recipient will
  issue a reply to it immediately. The receiver of the message is hardly
  inconvenienced by deciding to wait until later to send a response.

\end{itemize}

There two other interesting pieces of quantitative data that I would like
to mention.  I found that Flashmail has 13.8 percent more original messages
sent than E-mail. One possibility for this is that VM messages tended to
have somewhat longer threads (series of messages originating from one
original message) than Flashmail messages.  The average Flashmail thread
(including original) was 2.9 messages, while the average E-mail thread was
just over 4.0 messages. 
   
I also found that some individuals used the Flashmail menus to ``just see
who's around.''  I recorded data on how often Flashmail group members
pulled down the menu.  Although I did not do any statistical analysis on
this data, I noted that most individuals pulled down the menu roughly 8 or
more times a day, and occasionally did this as many as 30 or more times in
a single day.  Only a small fraction of these ``menu-pull-downs'' resulted
in actual Flashmail messages being sent, so obviously this system is
serving another purpose instead of being strictly a messaging tool.

Like the unanticipated effects of the telephone, Flashmail resulted in some
unexpected behaviors.  The individuals of the Flashmail group in the
experiment reported some interesting incidents that would not have occurred
without Flashmail.  For example, most of the Flashmail group members
involved in this experiment can be found on the University campus, however
two of the users graduated before the experiment took place and work
off-campus. It turns
out that Flashmail allowed these two group members to maintain a ``virtual
presence.''  Although the other group members did not see these individuals
in person, by seeing their names in the Flashmail menu, it seemed as if
they were as readily accessible as the colleague down the hall.  As a
result, these outside group members report still feeling involved and
included in the group's activities.

Other incidents reported include members checking to see if an individual
is in their office through the Flashmail menu and then calling them on the
phone, or even going to see them in person.  Apparently this checking
before making a personal visit happens fairly frequently to avoid making
unnecessary trips.  One incident involved a student who had finished a
draft of a paper rather late in the day.  She noticed through the Flashmail
menu that the professor was still in his office and took the paper over.
She knew that the professor usually left earlier and would not have
bothered to take the paper over had it not been for his idle-time being
displayed through Flashmail.

The ``Flashmail conference'' was another interesting side-effect.  On
several occasions, one group member would send a Flashmail to several 
individuals at once.  The recipients began to reply to the message and
sent their replies to everyone involved instead of just to the sender.
Eventually, a type of conference evolved with all messages going to
everyone who received the original Flashmail message along with the
original sender.  After one particularly heated discussion, a lunch meeting
was called to discuss the issue in person and stop the torrent of Flashmail
messages. 

The biggest effect of Flashmail is that it allowed the members of the group
to interact in a synchronous fashion with minimal inconvenience and effort.

\section{Conclusion}
My goal was to investigate whether or not Flashmail is a viable CMC tool
and under what conditions it is normally used.  I found that it is a useful
CMC tool which is generally used for short messages to individuals who were
currently active at their machine at the time of sending.

Although E-mail generated much more messaging traffic than Flashmail,
Flashmail was used periodically by all members of the study.  It appears
that Flashmail is being utilized as an ``add-on'' to E-mail, in other
words, Flashmail seems to be complimenting E-mail rather than competing
with it.

My justification for this statement is that almost all Flashmail messages
fit a certain pattern, the messages were primarily under 400 characters in
length and sent to individuals with idle-times usually well under 7
minutes.  I do not believe that all messages that fit this criteria will be
sent via Flashmail, but I do think that if a message fits this pattern, it
is more likely to be communicated through this route than E-mail.  If the
message fits this criteria and necessitates a rapid response, I believe
that it is almost guaranteed that the Flashmail messaging system will be
used instead of E-mail.

As I mentioned above, Flashmail will not replace E-mail. Instead, Flashmail
makes up for the areas in which E-mail is not sufficient.  E-mail is weak
in the transmission of messages which the sender wants to be seen fairly
quickly.  This is Flashmail's specialty, and it appears that Flashmail is
living up to its name.  \nocite{*}
\bibliographystyle{plain} \bibliography{fmail} \printindex
\end{document}



















