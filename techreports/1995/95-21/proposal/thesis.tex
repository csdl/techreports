\documentstyle[nftimes,11pt,makeidx, /group/csdl/tex/definemargins,
/group/csdl/tex/lmacros, /group/csdl/tex/functiondoc]{report}
% Psfig/TeX 
\def\PsfigVersion{1.9}
% dvips version
%
% All psfig/tex software, documentation, and related files
% in this distribution of psfig/tex are 
% Copyright 1987, 1988, 1991 Trevor J. Darrell
%
% Permission is granted for use and non-profit distribution of psfig/tex 
% providing that this notice is clearly maintained. The right to
% distribute any portion of psfig/tex for profit or as part of any commercial
% product is specifically reserved for the author(s) of that portion.
%
% *** Feel free to make local modifications of psfig as you wish,
% *** but DO NOT post any changed or modified versions of ``psfig''
% *** directly to the net. Send them to me and I'll try to incorporate
% *** them into future versions. If you want to take the psfig code 
% *** and make a new program (subject to the copyright above), distribute it, 
% *** (and maintain it) that's fine, just don't call it psfig.
%
% Bugs and improvements to trevor@media.mit.edu.
%
% Thanks to Greg Hager (GDH) and Ned Batchelder for their contributions
% to the original version of this project.
%
% Modified by J. Daniel Smith on 9 October 1990 to accept the
% %%BoundingBox: comment with or without a space after the colon.  Stole
% file reading code from Tom Rokicki's EPSF.TEX file (see below).
%
% More modifications by J. Daniel Smith on 29 March 1991 to allow the
% the included PostScript figure to be rotated.  The amount of
% rotation is specified by the "angle=" parameter of the \psfig command.
%
% Modified by Robert Russell on June 25, 1991 to allow users to specify
% .ps filenames which don't yet exist, provided they explicitly provide
% boundingbox information via the \psfig command. Note: This will only work
% if the "file=" parameter follows all four "bb???=" parameters in the
% command. This is due to the order in which psfig interprets these params.
%
%  3 Jul 1991	JDS	check if file already read in once
%  4 Sep 1991	JDS	fixed incorrect computation of rotated
%			bounding box
% 25 Sep 1991	GVR	expanded synopsis of \psfig
% 14 Oct 1991	JDS	\fbox code from LaTeX so \psdraft works with TeX
%			changed \typeout to \ps@typeout
% 17 Oct 1991	JDS	added \psscalefirst and \psrotatefirst
%

% From: gvr@cs.brown.edu (George V. Reilly)
%
% \psdraft	draws an outline box, but doesn't include the figure
%		in the DVI file.  Useful for previewing.
%
% \psfull	includes the figure in the DVI file (default).
%
% \psscalefirst width= or height= specifies the size of the figure
% 		before rotation.
% \psrotatefirst (default) width= or height= specifies the size of the
% 		 figure after rotation.  Asymetric figures will
% 		 appear to shrink.
%
% \psfigurepath#1	sets the path to search for the figure
%
% \psfig
% usage: \psfig{file=, figure=, height=, width=,
%			bbllx=, bblly=, bburx=, bbury=,
%			rheight=, rwidth=, clip=, angle=, silent=}
%
%	"file" is the filename.  If no path name is specified and the
%		file is not found in the current directory,
%		it will be looked for in directory \psfigurepath.
%	"figure" is a synonym for "file".
%	By default, the width and height of the figure are taken from
%		the BoundingBox of the figure.
%	If "width" is specified, the figure is scaled so that it has
%		the specified width.  Its height changes proportionately.
%	If "height" is specified, the figure is scaled so that it has
%		the specified height.  Its width changes proportionately.
%	If both "width" and "height" are specified, the figure is scaled
%		anamorphically.
%	"bbllx", "bblly", "bburx", and "bbury" control the PostScript
%		BoundingBox.  If these four values are specified
%               *before* the "file" option, the PSFIG will not try to
%               open the PostScript file.
%	"rheight" and "rwidth" are the reserved height and width
%		of the figure, i.e., how big TeX actually thinks
%		the figure is.  They default to "width" and "height".
%	The "clip" option ensures that no portion of the figure will
%		appear outside its BoundingBox.  "clip=" is a switch and
%		takes no value, but the `=' must be present.
%	The "angle" option specifies the angle of rotation (degrees, ccw).
%	The "silent" option makes \psfig work silently.
%

% check to see if macros already loaded in (maybe some other file says
% "\input psfig") ...
\ifx\undefined\psfig\else\endinput\fi

%
% from a suggestion by eijkhout@csrd.uiuc.edu to allow
% loading as a style file. Changed to avoid problems
% with amstex per suggestion by jbence@math.ucla.edu

\let\LaTeXAtSign=\@
\let\@=\relax
\edef\psfigRestoreAt{\catcode`\@=\number\catcode`@\relax}
%\edef\psfigRestoreAt{\catcode`@=\number\catcode`@\relax}
\catcode`\@=11\relax
\newwrite\@unused
\def\ps@typeout#1{{\let\protect\string\immediate\write\@unused{#1}}}
\ps@typeout{psfig/tex \PsfigVersion}

%% Here's how you define your figure path.  Should be set up with null
%% default and a user useable definition.

\def\figurepath{./}
\def\psfigurepath#1{\edef\figurepath{#1}}

%
% @psdo control structure -- similar to Latex @for.
% I redefined these with different names so that psfig can
% be used with TeX as well as LaTeX, and so that it will not 
% be vunerable to future changes in LaTeX's internal
% control structure,
%
\def\@nnil{\@nil}
\def\@empty{}
\def\@psdonoop#1\@@#2#3{}
\def\@psdo#1:=#2\do#3{\edef\@psdotmp{#2}\ifx\@psdotmp\@empty \else
    \expandafter\@psdoloop#2,\@nil,\@nil\@@#1{#3}\fi}
\def\@psdoloop#1,#2,#3\@@#4#5{\def#4{#1}\ifx #4\@nnil \else
       #5\def#4{#2}\ifx #4\@nnil \else#5\@ipsdoloop #3\@@#4{#5}\fi\fi}
\def\@ipsdoloop#1,#2\@@#3#4{\def#3{#1}\ifx #3\@nnil 
       \let\@nextwhile=\@psdonoop \else
      #4\relax\let\@nextwhile=\@ipsdoloop\fi\@nextwhile#2\@@#3{#4}}
\def\@tpsdo#1:=#2\do#3{\xdef\@psdotmp{#2}\ifx\@psdotmp\@empty \else
    \@tpsdoloop#2\@nil\@nil\@@#1{#3}\fi}
\def\@tpsdoloop#1#2\@@#3#4{\def#3{#1}\ifx #3\@nnil 
       \let\@nextwhile=\@psdonoop \else
      #4\relax\let\@nextwhile=\@tpsdoloop\fi\@nextwhile#2\@@#3{#4}}
% 
% \fbox is defined in latex.tex; so if \fbox is undefined, assume that
% we are not in LaTeX.
% Perhaps this could be done better???
\ifx\undefined\fbox
% \fbox code from modified slightly from LaTeX
\newdimen\fboxrule
\newdimen\fboxsep
\newdimen\ps@tempdima
\newbox\ps@tempboxa
\fboxsep = 3pt
\fboxrule = .4pt
\long\def\fbox#1{\leavevmode\setbox\ps@tempboxa\hbox{#1}\ps@tempdima\fboxrule
    \advance\ps@tempdima \fboxsep \advance\ps@tempdima \dp\ps@tempboxa
   \hbox{\lower \ps@tempdima\hbox
  {\vbox{\hrule height \fboxrule
          \hbox{\vrule width \fboxrule \hskip\fboxsep
          \vbox{\vskip\fboxsep \box\ps@tempboxa\vskip\fboxsep}\hskip 
                 \fboxsep\vrule width \fboxrule}
                 \hrule height \fboxrule}}}}
\fi
%
%%%%%%%%%%%%%%%%%%%%%%%%%%%%%%%%%%%%%%%%%%%%%%%%%%%%%%%%%%%%%%%%%%%
% file reading stuff from epsf.tex
%   EPSF.TEX macro file:
%   Written by Tomas Rokicki of Radical Eye Software, 29 Mar 1989.
%   Revised by Don Knuth, 3 Jan 1990.
%   Revised by Tomas Rokicki to accept bounding boxes with no
%      space after the colon, 18 Jul 1990.
%   Portions modified/removed for use in PSFIG package by
%      J. Daniel Smith, 9 October 1990.
%
\newread\ps@stream
\newif\ifnot@eof       % continue looking for the bounding box?
\newif\if@noisy        % report what you're making?
\newif\if@atend        % %%BoundingBox: has (at end) specification
\newif\if@psfile       % does this look like a PostScript file?
%
% PostScript files should start with `%!'
%
{\catcode`\%=12\global\gdef\epsf@start{%!}}
\def\epsf@PS{PS}
%
\def\epsf@getbb#1{%
%
%   The first thing we need to do is to open the
%   PostScript file, if possible.
%
\openin\ps@stream=#1
\ifeof\ps@stream\ps@typeout{Error, File #1 not found}\else
%
%   Okay, we got it. Now we'll scan lines until we find one that doesn't
%   start with %. We're looking for the bounding box comment.
%
   {\not@eoftrue \chardef\other=12
    \def\do##1{\catcode`##1=\other}\dospecials \catcode`\ =10
    \loop
       \if@psfile
	  \read\ps@stream to \epsf@fileline
       \else{
	  \obeyspaces
          \read\ps@stream to \epsf@tmp\global\let\epsf@fileline\epsf@tmp}
       \fi
       \ifeof\ps@stream\not@eoffalse\else
%
%   Check the first line for `%!'.  Issue a warning message if its not
%   there, since the file might not be a PostScript file.
%
       \if@psfile\else
       \expandafter\epsf@test\epsf@fileline:. \\%
       \fi
%
%   We check to see if the first character is a % sign;
%   if so, we look further and stop only if the line begins with
%   `%%BoundingBox:' and the `(atend)' specification was not found.
%   That is, the only way to stop is when the end of file is reached,
%   or a `%%BoundingBox: llx lly urx ury' line is found.
%
          \expandafter\epsf@aux\epsf@fileline:. \\%
       \fi
   \ifnot@eof\repeat
   }\closein\ps@stream\fi}%
%
% This tests if the file we are reading looks like a PostScript file.
%
\long\def\epsf@test#1#2#3:#4\\{\def\epsf@testit{#1#2}
			\ifx\epsf@testit\epsf@start\else
\ps@typeout{Warning! File does not start with `\epsf@start'.  It may not be a PostScript file.}
			\fi
			\@psfiletrue} % don't test after 1st line
%
%   We still need to define the tricky \epsf@aux macro. This requires
%   a couple of magic constants for comparison purposes.
%
{\catcode`\%=12\global\let\epsf@percent=%\global\def\epsf@bblit{%BoundingBox}}
%
%
%   So we're ready to check for `%BoundingBox:' and to grab the
%   values if they are found.  We continue searching if `(at end)'
%   was found after the `%BoundingBox:'.
%
\long\def\epsf@aux#1#2:#3\\{\ifx#1\epsf@percent
   \def\epsf@testit{#2}\ifx\epsf@testit\epsf@bblit
	\@atendfalse
        \epsf@atend #3 . \\%
	\if@atend	
	   \if@verbose{
		\ps@typeout{psfig: found `(atend)'; continuing search}
	   }\fi
        \else
        \epsf@grab #3 . . . \\%
        \not@eoffalse
        \global\no@bbfalse
        \fi
   \fi\fi}%
%
%   Here we grab the values and stuff them in the appropriate definitions.
%
\def\epsf@grab #1 #2 #3 #4 #5\\{%
   \global\def\epsf@llx{#1}\ifx\epsf@llx\empty
      \epsf@grab #2 #3 #4 #5 .\\\else
   \global\def\epsf@lly{#2}%
   \global\def\epsf@urx{#3}\global\def\epsf@ury{#4}\fi}%
%
% Determine if the stuff following the %%BoundingBox is `(atend)'
% J. Daniel Smith.  Copied from \epsf@grab above.
%
\def\epsf@atendlit{(atend)} 
\def\epsf@atend #1 #2 #3\\{%
   \def\epsf@tmp{#1}\ifx\epsf@tmp\empty
      \epsf@atend #2 #3 .\\\else
   \ifx\epsf@tmp\epsf@atendlit\@atendtrue\fi\fi}


% End of file reading stuff from epsf.tex
%%%%%%%%%%%%%%%%%%%%%%%%%%%%%%%%%%%%%%%%%%%%%%%%%%%%%%%%%%%%%%%%%%%

%%%%%%%%%%%%%%%%%%%%%%%%%%%%%%%%%%%%%%%%%%%%%%%%%%%%%%%%%%%%%%%%%%%
% trigonometry stuff from "trig.tex"
\chardef\psletter = 11 % won't conflict with \begin{letter} now...
\chardef\other = 12

\newif \ifdebug %%% turn me on to see TeX hard at work ...
\newif\ifc@mpute %%% don't need to compute some values
\c@mputetrue % but assume that we do

\let\then = \relax
\def\r@dian{pt }
\let\r@dians = \r@dian
\let\dimensionless@nit = \r@dian
\let\dimensionless@nits = \dimensionless@nit
\def\internal@nit{sp }
\let\internal@nits = \internal@nit
\newif\ifstillc@nverging
\def \Mess@ge #1{\ifdebug \then \message {#1} \fi}

{ %%% Things that need abnormal catcodes %%%
	\catcode `\@ = \psletter
	\gdef \nodimen {\expandafter \n@dimen \the \dimen}
	\gdef \term #1 #2 #3%
	       {\edef \t@ {\the #1}%%% freeze parameter 1 (count, by value)
		\edef \t@@ {\expandafter \n@dimen \the #2\r@dian}%
				   %%% freeze parameter 2 (dimen, by value)
		\t@rm {\t@} {\t@@} {#3}%
	       }
	\gdef \t@rm #1 #2 #3%
	       {{%
		\count 0 = 0
		\dimen 0 = 1 \dimensionless@nit
		\dimen 2 = #2\relax
		\Mess@ge {Calculating term #1 of \nodimen 2}%
		\loop
		\ifnum	\count 0 < #1
		\then	\advance \count 0 by 1
			\Mess@ge {Iteration \the \count 0 \space}%
			\Multiply \dimen 0 by {\dimen 2}%
			\Mess@ge {After multiplication, term = \nodimen 0}%
			\Divide \dimen 0 by {\count 0}%
			\Mess@ge {After division, term = \nodimen 0}%
		\repeat
		\Mess@ge {Final value for term #1 of 
				\nodimen 2 \space is \nodimen 0}%
		\xdef \Term {#3 = \nodimen 0 \r@dians}%
		\aftergroup \Term
	       }}
	\catcode `\p = \other
	\catcode `\t = \other
	\gdef \n@dimen #1pt{#1} %%% throw away the ``pt''
}

\def \Divide #1by #2{\divide #1 by #2} %%% just a synonym

\def \Multiply #1by #2%%% allows division of a dimen by a dimen
       {{%%% should really freeze parameter 2 (dimen, passed by value)
	\count 0 = #1\relax
	\count 2 = #2\relax
	\count 4 = 65536
	\Mess@ge {Before scaling, count 0 = \the \count 0 \space and
			count 2 = \the \count 2}%
	\ifnum	\count 0 > 32767 %%% do our best to avoid overflow
	\then	\divide \count 0 by 4
		\divide \count 4 by 4
	\else	\ifnum	\count 0 < -32767
		\then	\divide \count 0 by 4
			\divide \count 4 by 4
		\else
		\fi
	\fi
	\ifnum	\count 2 > 32767 %%% while retaining reasonable accuracy
	\then	\divide \count 2 by 4
		\divide \count 4 by 4
	\else	\ifnum	\count 2 < -32767
		\then	\divide \count 2 by 4
			\divide \count 4 by 4
		\else
		\fi
	\fi
	\multiply \count 0 by \count 2
	\divide \count 0 by \count 4
	\xdef \product {#1 = \the \count 0 \internal@nits}%
	\aftergroup \product
       }}

\def\r@duce{\ifdim\dimen0 > 90\r@dian \then   % sin(x+90) = sin(180-x)
		\multiply\dimen0 by -1
		\advance\dimen0 by 180\r@dian
		\r@duce
	    \else \ifdim\dimen0 < -90\r@dian \then  % sin(-x) = sin(360+x)
		\advance\dimen0 by 360\r@dian
		\r@duce
		\fi
	    \fi}

\def\Sine#1%
       {{%
	\dimen 0 = #1 \r@dian
	\r@duce
	\ifdim\dimen0 = -90\r@dian \then
	   \dimen4 = -1\r@dian
	   \c@mputefalse
	\fi
	\ifdim\dimen0 = 90\r@dian \then
	   \dimen4 = 1\r@dian
	   \c@mputefalse
	\fi
	\ifdim\dimen0 = 0\r@dian \then
	   \dimen4 = 0\r@dian
	   \c@mputefalse
	\fi
%
	\ifc@mpute \then
        	% convert degrees to radians
		\divide\dimen0 by 180
		\dimen0=3.141592654\dimen0
%
		\dimen 2 = 3.1415926535897963\r@dian %%% a well-known constant
		\divide\dimen 2 by 2 %%% we only deal with -pi/2 : pi/2
		\Mess@ge {Sin: calculating Sin of \nodimen 0}%
		\count 0 = 1 %%% see power-series expansion for sine
		\dimen 2 = 1 \r@dian %%% ditto
		\dimen 4 = 0 \r@dian %%% ditto
		\loop
			\ifnum	\dimen 2 = 0 %%% then we've done
			\then	\stillc@nvergingfalse 
			\else	\stillc@nvergingtrue
			\fi
			\ifstillc@nverging %%% then calculate next term
			\then	\term {\count 0} {\dimen 0} {\dimen 2}%
				\advance \count 0 by 2
				\count 2 = \count 0
				\divide \count 2 by 2
				\ifodd	\count 2 %%% signs alternate
				\then	\advance \dimen 4 by \dimen 2
				\else	\advance \dimen 4 by -\dimen 2
				\fi
		\repeat
	\fi		
			\xdef \sine {\nodimen 4}%
       }}

% Now the Cosine can be calculated easily by calling \Sine
\def\Cosine#1{\ifx\sine\UnDefined\edef\Savesine{\relax}\else
		             \edef\Savesine{\sine}\fi
	{\dimen0=#1\r@dian\advance\dimen0 by 90\r@dian
	 \Sine{\nodimen 0}
	 \xdef\cosine{\sine}
	 \xdef\sine{\Savesine}}}	      
% end of trig stuff
%%%%%%%%%%%%%%%%%%%%%%%%%%%%%%%%%%%%%%%%%%%%%%%%%%%%%%%%%%%%%%%%%%%%

\def\psdraft{
	\def\@psdraft{0}
	%\ps@typeout{draft level now is \@psdraft \space . }
}
\def\psfull{
	\def\@psdraft{100}
	%\ps@typeout{draft level now is \@psdraft \space . }
}

\psfull

\newif\if@scalefirst
\def\psscalefirst{\@scalefirsttrue}
\def\psrotatefirst{\@scalefirstfalse}
\psrotatefirst

\newif\if@draftbox
\def\psnodraftbox{
	\@draftboxfalse
}
\def\psdraftbox{
	\@draftboxtrue
}
\@draftboxtrue

\newif\if@prologfile
\newif\if@postlogfile
\def\pssilent{
	\@noisyfalse
}
\def\psnoisy{
	\@noisytrue
}
\psnoisy
%%% These are for the option list.
%%% A specification of the form a = b maps to calling \@p@@sa{b}
\newif\if@bbllx
\newif\if@bblly
\newif\if@bburx
\newif\if@bbury
\newif\if@height
\newif\if@width
\newif\if@rheight
\newif\if@rwidth
\newif\if@angle
\newif\if@clip
\newif\if@verbose
\def\@p@@sclip#1{\@cliptrue}


\newif\if@decmpr

%%% GDH 7/26/87 -- changed so that it first looks in the local directory,
%%% then in a specified global directory for the ps file.
%%% RPR 6/25/91 -- changed so that it defaults to user-supplied name if
%%% boundingbox info is specified, assuming graphic will be created by
%%% print time.
%%% TJD 10/19/91 -- added bbfile vs. file distinction, and @decmpr flag

\def\@p@@sfigure#1{\def\@p@sfile{null}\def\@p@sbbfile{null}
	        \openin1=#1.bb
		\ifeof1\closein1
	        	\openin1=\figurepath#1.bb
			\ifeof1\closein1
			        \openin1=#1
				\ifeof1\closein1%
				       \openin1=\figurepath#1
					\ifeof1
					   \ps@typeout{Error, File #1 not found}
						\if@bbllx\if@bblly
				   		\if@bburx\if@bbury
			      				\def\@p@sfile{#1}%
			      				\def\@p@sbbfile{#1}%
							\@decmprfalse
				  	   	\fi\fi\fi\fi
					\else\closein1
				    		\def\@p@sfile{\figurepath#1}%
				    		\def\@p@sbbfile{\figurepath#1}%
						\@decmprfalse
	                       		\fi%
			 	\else\closein1%
					\def\@p@sfile{#1}
					\def\@p@sbbfile{#1}
					\@decmprfalse
			 	\fi
			\else
				\def\@p@sfile{\figurepath#1}
				\def\@p@sbbfile{\figurepath#1.bb}
				\@decmprtrue
			\fi
		\else
			\def\@p@sfile{#1}
			\def\@p@sbbfile{#1.bb}
			\@decmprtrue
		\fi}

\def\@p@@sfile#1{\@p@@sfigure{#1}}

\def\@p@@sbbllx#1{
		%\ps@typeout{bbllx is #1}
		\@bbllxtrue
		\dimen100=#1
		\edef\@p@sbbllx{\number\dimen100}
}
\def\@p@@sbblly#1{
		%\ps@typeout{bblly is #1}
		\@bbllytrue
		\dimen100=#1
		\edef\@p@sbblly{\number\dimen100}
}
\def\@p@@sbburx#1{
		%\ps@typeout{bburx is #1}
		\@bburxtrue
		\dimen100=#1
		\edef\@p@sbburx{\number\dimen100}
}
\def\@p@@sbbury#1{
		%\ps@typeout{bbury is #1}
		\@bburytrue
		\dimen100=#1
		\edef\@p@sbbury{\number\dimen100}
}
\def\@p@@sheight#1{
		\@heighttrue
		\dimen100=#1
   		\edef\@p@sheight{\number\dimen100}
		%\ps@typeout{Height is \@p@sheight}
}
\def\@p@@swidth#1{
		%\ps@typeout{Width is #1}
		\@widthtrue
		\dimen100=#1
		\edef\@p@swidth{\number\dimen100}
}
\def\@p@@srheight#1{
		%\ps@typeout{Reserved height is #1}
		\@rheighttrue
		\dimen100=#1
		\edef\@p@srheight{\number\dimen100}
}
\def\@p@@srwidth#1{
		%\ps@typeout{Reserved width is #1}
		\@rwidthtrue
		\dimen100=#1
		\edef\@p@srwidth{\number\dimen100}
}
\def\@p@@sangle#1{
		%\ps@typeout{Rotation is #1}
		\@angletrue
%		\dimen100=#1
		\edef\@p@sangle{#1} %\number\dimen100}
}
\def\@p@@ssilent#1{ 
		\@verbosefalse
}
\def\@p@@sprolog#1{\@prologfiletrue\def\@prologfileval{#1}}
\def\@p@@spostlog#1{\@postlogfiletrue\def\@postlogfileval{#1}}
\def\@cs@name#1{\csname #1\endcsname}
\def\@setparms#1=#2,{\@cs@name{@p@@s#1}{#2}}
%
% initialize the defaults (size the size of the figure)
%
\def\ps@init@parms{
		\@bbllxfalse \@bbllyfalse
		\@bburxfalse \@bburyfalse
		\@heightfalse \@widthfalse
		\@rheightfalse \@rwidthfalse
		\def\@p@sbbllx{}\def\@p@sbblly{}
		\def\@p@sbburx{}\def\@p@sbbury{}
		\def\@p@sheight{}\def\@p@swidth{}
		\def\@p@srheight{}\def\@p@srwidth{}
		\def\@p@sangle{0}
		\def\@p@sfile{} \def\@p@sbbfile{}
		\def\@p@scost{10}
		\def\@sc{}
		\@prologfilefalse
		\@postlogfilefalse
		\@clipfalse
		\if@noisy
			\@verbosetrue
		\else
			\@verbosefalse
		\fi
}
%
% Go through the options setting things up.
%
\def\parse@ps@parms#1{
	 	\@psdo\@psfiga:=#1\do
		   {\expandafter\@setparms\@psfiga,}}
%
% Compute bb height and width
%
\newif\ifno@bb
\def\bb@missing{
	\if@verbose{
		\ps@typeout{psfig: searching \@p@sbbfile \space  for bounding box}
	}\fi
	\no@bbtrue
	\epsf@getbb{\@p@sbbfile}
        \ifno@bb \else \bb@cull\epsf@llx\epsf@lly\epsf@urx\epsf@ury\fi
}	
\def\bb@cull#1#2#3#4{
	\dimen100=#1 bp\edef\@p@sbbllx{\number\dimen100}
	\dimen100=#2 bp\edef\@p@sbblly{\number\dimen100}
	\dimen100=#3 bp\edef\@p@sbburx{\number\dimen100}
	\dimen100=#4 bp\edef\@p@sbbury{\number\dimen100}
	\no@bbfalse
}
% rotate point (#1,#2) about (0,0).
% The sine and cosine of the angle are already stored in \sine and
% \cosine.  The result is placed in (\p@intvaluex, \p@intvaluey).
\newdimen\p@intvaluex
\newdimen\p@intvaluey
\def\rotate@#1#2{{\dimen0=#1 sp\dimen1=#2 sp
%            	calculate x' = x \cos\theta - y \sin\theta
		  \global\p@intvaluex=\cosine\dimen0
		  \dimen3=\sine\dimen1
		  \global\advance\p@intvaluex by -\dimen3
% 		calculate y' = x \sin\theta + y \cos\theta
		  \global\p@intvaluey=\sine\dimen0
		  \dimen3=\cosine\dimen1
		  \global\advance\p@intvaluey by \dimen3
		  }}
\def\compute@bb{
		\no@bbfalse
		\if@bbllx \else \no@bbtrue \fi
		\if@bblly \else \no@bbtrue \fi
		\if@bburx \else \no@bbtrue \fi
		\if@bbury \else \no@bbtrue \fi
		\ifno@bb \bb@missing \fi
		\ifno@bb \ps@typeout{FATAL ERROR: no bb supplied or found}
			\no-bb-error
		\fi
		%
%\ps@typeout{BB: \@p@sbbllx, \@p@sbblly, \@p@sbburx, \@p@sbbury} 
%
% store height/width of original (unrotated) bounding box
		\count203=\@p@sbburx
		\count204=\@p@sbbury
		\advance\count203 by -\@p@sbbllx
		\advance\count204 by -\@p@sbblly
		\edef\ps@bbw{\number\count203}
		\edef\ps@bbh{\number\count204}
		%\ps@typeout{ psbbh = \ps@bbh, psbbw = \ps@bbw }
		\if@angle 
			\Sine{\@p@sangle}\Cosine{\@p@sangle}
	        	{\dimen100=\maxdimen\xdef\r@p@sbbllx{\number\dimen100}
					    \xdef\r@p@sbblly{\number\dimen100}
			                    \xdef\r@p@sbburx{-\number\dimen100}
					    \xdef\r@p@sbbury{-\number\dimen100}}
%
% Need to rotate all four points and take the X-Y extremes of the new
% points as the new bounding box.
                        \def\minmaxtest{
			   \ifnum\number\p@intvaluex<\r@p@sbbllx
			      \xdef\r@p@sbbllx{\number\p@intvaluex}\fi
			   \ifnum\number\p@intvaluex>\r@p@sbburx
			      \xdef\r@p@sbburx{\number\p@intvaluex}\fi
			   \ifnum\number\p@intvaluey<\r@p@sbblly
			      \xdef\r@p@sbblly{\number\p@intvaluey}\fi
			   \ifnum\number\p@intvaluey>\r@p@sbbury
			      \xdef\r@p@sbbury{\number\p@intvaluey}\fi
			   }
%			lower left
			\rotate@{\@p@sbbllx}{\@p@sbblly}
			\minmaxtest
%			upper left
			\rotate@{\@p@sbbllx}{\@p@sbbury}
			\minmaxtest
%			lower right
			\rotate@{\@p@sbburx}{\@p@sbblly}
			\minmaxtest
%			upper right
			\rotate@{\@p@sbburx}{\@p@sbbury}
			\minmaxtest
			\edef\@p@sbbllx{\r@p@sbbllx}\edef\@p@sbblly{\r@p@sbblly}
			\edef\@p@sbburx{\r@p@sbburx}\edef\@p@sbbury{\r@p@sbbury}
%\ps@typeout{rotated BB: \r@p@sbbllx, \r@p@sbblly, \r@p@sbburx, \r@p@sbbury}
		\fi
		\count203=\@p@sbburx
		\count204=\@p@sbbury
		\advance\count203 by -\@p@sbbllx
		\advance\count204 by -\@p@sbblly
		\edef\@bbw{\number\count203}
		\edef\@bbh{\number\count204}
		%\ps@typeout{ bbh = \@bbh, bbw = \@bbw }
}
%
% \in@hundreds performs #1 * (#2 / #3) correct to the hundreds,
%	then leaves the result in @result
%
\def\in@hundreds#1#2#3{\count240=#2 \count241=#3
		     \count100=\count240	% 100 is first digit #2/#3
		     \divide\count100 by \count241
		     \count101=\count100
		     \multiply\count101 by \count241
		     \advance\count240 by -\count101
		     \multiply\count240 by 10
		     \count101=\count240	%101 is second digit of #2/#3
		     \divide\count101 by \count241
		     \count102=\count101
		     \multiply\count102 by \count241
		     \advance\count240 by -\count102
		     \multiply\count240 by 10
		     \count102=\count240	% 102 is the third digit
		     \divide\count102 by \count241
		     \count200=#1\count205=0
		     \count201=\count200
			\multiply\count201 by \count100
		 	\advance\count205 by \count201
		     \count201=\count200
			\divide\count201 by 10
			\multiply\count201 by \count101
			\advance\count205 by \count201
			%
		     \count201=\count200
			\divide\count201 by 100
			\multiply\count201 by \count102
			\advance\count205 by \count201
			%
		     \edef\@result{\number\count205}
}
\def\compute@wfromh{
		% computing : width = height * (bbw / bbh)
		\in@hundreds{\@p@sheight}{\@bbw}{\@bbh}
		%\ps@typeout{ \@p@sheight * \@bbw / \@bbh, = \@result }
		\edef\@p@swidth{\@result}
		%\ps@typeout{w from h: width is \@p@swidth}
}
\def\compute@hfromw{
		% computing : height = width * (bbh / bbw)
	        \in@hundreds{\@p@swidth}{\@bbh}{\@bbw}
		%\ps@typeout{ \@p@swidth * \@bbh / \@bbw = \@result }
		\edef\@p@sheight{\@result}
		%\ps@typeout{h from w : height is \@p@sheight}
}
\def\compute@handw{
		\if@height 
			\if@width
			\else
				\compute@wfromh
			\fi
		\else 
			\if@width
				\compute@hfromw
			\else
				\edef\@p@sheight{\@bbh}
				\edef\@p@swidth{\@bbw}
			\fi
		\fi
}
\def\compute@resv{
		\if@rheight \else \edef\@p@srheight{\@p@sheight} \fi
		\if@rwidth \else \edef\@p@srwidth{\@p@swidth} \fi
		%\ps@typeout{rheight = \@p@srheight, rwidth = \@p@srwidth}
}
%		
% Compute any missing values
\def\compute@sizes{
	\compute@bb
	\if@scalefirst\if@angle
% at this point the bounding box has been adjsuted correctly for
% rotation.  PSFIG does all of its scaling using \@bbh and \@bbw.  If
% a width= or height= was specified along with \psscalefirst, then the
% width=/height= value needs to be adjusted to match the new (rotated)
% bounding box size (specifed in \@bbw and \@bbh).
%    \ps@bbw       width=
%    -------  =  ---------- 
%    \@bbw       new width=
% so `new width=' = (width= * \@bbw) / \ps@bbw; where \ps@bbw is the
% width of the original (unrotated) bounding box.
	\if@width
	   \in@hundreds{\@p@swidth}{\@bbw}{\ps@bbw}
	   \edef\@p@swidth{\@result}
	\fi
	\if@height
	   \in@hundreds{\@p@sheight}{\@bbh}{\ps@bbh}
	   \edef\@p@sheight{\@result}
	\fi
	\fi\fi
	\compute@handw
	\compute@resv}

%
% \psfig
% usage : \psfig{file=, height=, width=, bbllx=, bblly=, bburx=, bbury=,
%			rheight=, rwidth=, clip=}
%
% "clip=" is a switch and takes no value, but the `=' must be present.
\def\psfig#1{\vbox {
	% do a zero width hard space so that a single
	% \psfig in a centering enviornment will behave nicely
	%{\setbox0=\hbox{\ }\ \hskip-\wd0}
	%
	\ps@init@parms
	\parse@ps@parms{#1}
	\compute@sizes
	%
	\ifnum\@p@scost<\@psdraft{
		%
		\special{ps::[begin] 	\@p@swidth \space \@p@sheight \space
				\@p@sbbllx \space \@p@sbblly \space
				\@p@sbburx \space \@p@sbbury \space
				startTexFig \space }
		\if@angle
			\special {ps:: \@p@sangle \space rotate \space} 
		\fi
		\if@clip{
			\if@verbose{
				\ps@typeout{(clip)}
			}\fi
			\special{ps:: doclip \space }
		}\fi
		\if@prologfile
		    \special{ps: plotfile \@prologfileval \space } \fi
		\if@decmpr{
			\if@verbose{
				\ps@typeout{psfig: including \@p@sfile.Z \space }
			}\fi
			\special{ps: plotfile "`zcat \@p@sfile.Z" \space }
		}\else{
			\if@verbose{
				\ps@typeout{psfig: including \@p@sfile \space }
			}\fi
			\special{ps: plotfile \@p@sfile \space }
		}\fi
		\if@postlogfile
		    \special{ps: plotfile \@postlogfileval \space } \fi
		\special{ps::[end] endTexFig \space }
		% Create the vbox to reserve the space for the figure.
		\vbox to \@p@srheight sp{
		% 1/92 TJD Changed from "true sp" to "sp" for magnification.
			\hbox to \@p@srwidth sp{
				\hss
			}
		\vss
		}
	}\else{
		% draft figure, just reserve the space and print the
		% path name.
		\if@draftbox{		
			% Verbose draft: print file name in box
			\hbox{\frame{\vbox to \@p@srheight sp{
			\vss
			\hbox to \@p@srwidth sp{ \hss \@p@sfile \hss }
			\vss
			}}}
		}\else{
			% Non-verbose draft
			\vbox to \@p@srheight sp{
			\vss
			\hbox to \@p@srwidth sp{\hss}
			\vss
			}
		}\fi	



	}\fi
}}
\psfigRestoreAt
\let\@=\LaTeXAtSign



 \pagestyle{headings} \makeindex
\begin{document}


\title{An Evaluation of Flashmail: \\ a computer-mediated communication
tool} \author{Jennifer Geis\\ \\ Collaborative Software Development
Laboratory\\ Department of Information and Computer Sciences\\ University
of Hawaii at Manoa\\ Honolulu, HI 96822\\ (808) 956-6920\\ {\sf
jgeis@uhics.ics.hawaii.edu}\\ {\sf ICS-TR-95-21}}\maketitle

\begin{abstract}
  This research will analyze a new computer-mediated communication tool
  called Flashmail.  I will investigate how people use Flashmail and its
  relationship to conventional electronic mail.  Both systems will be
  equipped with a ``contents characteristics checklist,'' in which a user
  of either system will indicate what purpose was served by the message
  being transmitted.  I will use this information along with statistics
  gathered by the systems to assess the differences in utility of Flashmail
  and electronic mail.  The thesis of this research is that users, once
  introduced to Flashmail, will prefer it over electronic mail for all
  time-dependent communications within their flashmail group.
\end{abstract}
\newpage
\tableofcontents
\newpage \ls{1.0}


\chapter{Introduction}
\section{Computer-mediated communication}
Computer-mediated communication (CMC) systems use computers to structure,
store, and process communications. CMC is the use of computers to
facilitate human-human interaction.  This thesis will evaluate a new CMC
system called Flashmail.

Computer-mediated communication changes the way we interact with each
other.  The following quote by Robert Sproull draws a parallel between CMC
and a different type of communication tool, the telephone.

\begin{quote}
  The telephone had extensive and unanticipated effects in part because it
  routinely extended attention, social contacts, and interdependencies
  beyond patterns determined by physical proximity. Reducing the
  constraints of physical proximity increased people's choice of
  interactions, whether with family members who had migrated from the farm
  to the city, or distant employees, or the boss. Amplification occurred
  because communication networks have a mutually causal, spiraling
  relationship with information networks, close relationships, conformity,
  and cultural change.... Our research demonstrates that, as the telephone
  did, new computer-based communication technology in some organizations is
  changing attention, social contact patterns, and interdependencies.
  \cite{Sproull93}
\end{quote}

Similar to the telephone, CMC allows people to communicate more information
in a shorter amount of time, regardless of physical settings.  For example,
suppose a student wishes to ask a question in regards to an assignment.  If
the student cannot locate the professor, the student can send the question
to the professor through electronic mail (E-mail).  The professor can then
respond to the question through the same medium.

With the use of a computer and a modem an individual can conduct many
activities without leaving their home.  Although the common use of CMC is
to further communication attempts within a business setting, many more
possibilities are becoming available.  Currently, people conduct bank
transactions, do research, read journals and articles, reserve books at the
library, get the daily news, look at movie listings, and even order pizza
using CMC. 

In this thesis, I focus on two CMC tools: Electronic mail and Flashmail.
Electronic mail is ``the electronic, one-directional transfer of
information in the form of a message, via an intermediate
(tele-)communication system, from an identified sending party to one or
more identified receiving parties.''  \cite{Vervest85}

E-mail allows an individual to send an electronic message to one or more
receivers.  The recipients of the message have the options of replying to
the message, deleting it, saving it, forwarding it, or simply doing nothing
with it and just leaving it in their mailbox.

Some characteristics of E-mail and their comparisons to Flashmail are:
\begin{itemize}
\item Messages do not interrupt the receiver: This is a significant
  difference between E-mail and Flashmail.  Flashmail is deliberately
  intrusive and interrupts whatever the recipient is doing.  E-mail stores
  the messages for the recipient to read at their leisure.

\item The receiver controls when they read their mail: In E-mail, messages
  collect in what is called an ``inbox'' and remain there until the
  recipient chooses to open the box and read their mail.  When a Flashmail
  message is sent, a screen pops up on the receivers terminal containing
  the message, so the recipient is forced to view the message immediately.

\item There is no way for the sender to control the immediacy of response:
  Since the recipient reads their E-mail at their leisure, the sender has
  to wait until the receiver opens their mailbox, reads the message and
  decides to respond to it.  Since a Flashmail message is viewed within
  seconds of being sent, if a response is warranted, it is usually issued
  immediately.

\item Allows saving and categorizing of messages: This is useful for
  messages which need to be saved for future reference.  Flashmail has the
  ability to save messages, but it is not as convenient as saving E-mail
  messages. In fact, by default the messages are deleted.
  
\item Allows attachments of other files: This feature allows an individual
  to send several files at once easily.  Flashmail does not have this
  mechanism.  To send files using Flashmail, one would have to cut and
  paste the file contents into the Flashmail message buffer.

\item Can be addressed to multiple receivers: The ability to send the same
  message to several receivers at once saves a sender the hassle of having
  to re-address and send out the same message several times.  Flashmail
  also has this ability.

\item Asynchronous: E-mail is asynchronous meaning messages that are sent
  are not received and viewed in the same time frame.  The time lapsed
  between when a message is sent and when it is read can vary from minutes
  to days.  Flashmail is a synchronous messaging system in that the time
  between when a message is sent and when it is read is a matter of
  seconds.

\end{itemize}

\section{Flashmail: an alternative approach to E-mail}
One problem with E-mail is the lack of a tool which facilitates
time-dependent messages.  Time-dependent messages are those whose
usefulness expires after a short period of time if they are not read.  For
example, say that you want to hold a meeting with colleagues in an hour's
time.  You can send them an E-mail, but if a recipient does not read their
mail before the meeting time, the message is useless to them.

Flashmail fills this gap.  Flashmail is a CMC tool which transfers
time-dependent messages in a manner which greatly increases their
effectiveness.  Flashmail can almost guarantee that the message will be
seen.  The sender can rest assured that once sent, the message will be the
first thing the recipient sees when they next look at their computer
screen.

Flashmail is a useful computer-mediated communication tool as it is a
mechanism to facilitate the transfers of time-dependent messages.  I
believe that people will use flashmail for all communications of this type.

One other feature of Flashmail is that it displays the idle time associated
with every user, as well as the machine they are working on.  The is an
important feature in that it enables everyone to know who is logged in,
where they are, and whether or not they are in front of their terminal.
For instance, if the idle time associated with an individual is ``0
seconds,'' you can be reasonably sure that if you send them a Flashmail
message, they will see it immediately.  However, if the displayed idle time
reads ``1 hour 20 seconds,'' they are probably nowhere near their
machine.

\section{Experiment}
This research is concerned with how people use Flashmail and E-mail to
assist their communications. I intend to find out if my hypothesis (that
people will use Flashmail over E-mail for all time-dependent messages) is
correct.
\subsection{Duration}
The time period of the experiment will be one week in length and occur
during the Spring of 1996.

\subsection{Method}
The participants in this experiment will be three groups within the
Information and Computer Sciences (ICS) department at the University of
Hawaii at Manoa. These groups are the classes ICS413, ICS613, and the
members of the Collaborative Software Development Laboratory (CSDL).

A pre-test questionnaire will be administered to all participants.  This
questionnaire will be used to assess which CMC tools currently are in use,
and the subjects' experiences with these tools.  

Both Flashmail and E-mail will be altered to include a ``content characteristics''
questionnaire. When a group member sends any Flashmail or E-mail message, a
box will immediately appear with a checklist.  The member will then 
click on the boxes that accurately describe the contents of their
message.  There will also be a field to allow the subject to list any
additional comments they may have.  The system will record who the sender
was as well as the receiver(s) of the message and at what date and time the
message was sent. 

I am using this checklist approach in order to avoid altering the group's
normal communication behavior.  If I was to actually read the contents of
their messages with the intent of gathering information, the group members
might ``edit'' their messages and the results would not be accurate.  With
this method, I will still be able to gain the information needed, but the
participants do not need to be concerned about anyone reading their
messages.  

After the completion of the study, a questionnaire will be administered to
the group members in regards to whether or not they modified their behavior
as a result of the study or if they feel the communication patterns they
exhibited were normal behavior. I will be using this questionnaire as
an assurance of the validity of the information collected during the study.

\section{Remainder of Proposal}
\begin{itemize}
\item Chapter 2: Flashmail

Description of the history and design of the Flashmail system.  Includes an 
illustration of the user interface and some of the features of Flashmail
\item Chapter 3: Evaluation of Flashmail

Discussion of the Flashmail evaluation experience and the results.
\item Chapter 4: Conclusion
\end{itemize}

\chapter{Flashmail}
\section{Design}
Flashmail is a system designed to be complimentary to E-mail and was
implemented in 1995 at the University of Hawaii at Manoa by Professor
Philip Johnson.  Flashmail allows synchronous electronic messaging through
a pop-up window mechanism and was developed to assist inter-group
communications.  It was designed using Egret, a framework for
computer-supported cooperative work (CSCW) applications. Egret implements a
multi-client, multi-server, multi-agent architecture. Egret clients and
agents are implemented by a 15 KLOC extension to XEmacs, the X-window Emacs
editor.  Egret servers are implemented by a 15 KLOC system written in C++.

Flashmail is a real-time communication method
complimentary to E-mail.  Flashmail's main feature is that it is
deliberately intrusive upon the receiver.  This feature allows the sender
to assume the receiver has seen the message, as the transmission will be
the first thing the receiver sees when they look at their computer screen.
In contrast to email, this allows an efficient method of transmitting
time-dependent messages.

\section{Features}
\begin{itemize}
\item Indicates which group members are currently logged in to a
  workstation: This determines who a user can and cannot send a Flashmail
  message to.  If a user is not logged in, the system will not allow a
  message to be sent to them.

\item If a group member is logged in, the idle-time associated with their
  workstation is given: This is extremely useful for knowing if a Flashmail
  member will see your message immediately.  If the idle time displayed
  reads ``0 seconds,'' then it is most likely that the recipient will see
  the message as soon as it is sent.  On the other hand, if the idle-time
  reads ``2 hours,'' the individual has not been near their machine in
  quite a while and will not see the message until they return.

\item Gives real-time information on which workstation each group member is
  using: This allows other group members to know where everybody else is
  located.  Every machine has a name, and if you know where that particular
  machine is located, you know where to find that group member.  The
  utility of this is that it gives group members who work in different
  parts of a building the ability to locate other members without having to
  physically track them down.

\item Deliberately intrusive: This is what differs Flashmail from many
  other CMC tools.  Flashmail has a mechanism that pops up a window
  containing the message on the receiver's screen.  This window will appear
  over whatever the receiver is doing.  In this respect, it is intrusive,
  but it assures that the message will be seen immediately.  Should a group
  member be particularly busy and wish to ensure that they will not be
  interrupted, Flashmail can be turned off, thereby preventing anyone from
  sending them a Flashmail message.

\item The sender can assume the receiver has seen the message: As mentioned
  above, the idle-time associated with each user is displayed, if the idle
  time reads ``0 seconds'' at the time you send that individual a Flashmail
  message, you can assume they will see the message immediately as they are
  currently active on their machine and the message appears over whatever
  they are working on.

\item Messages are delivered and received practically instantaneously: In
  contrast to E-mail, where the delivery is sometimes unpredictable,
  Flashmail messages appear on the recipients screen almost immediately
  after being sent.  A sender can rest assured that once a Flashmail
  message is sent, it appears on the receiver's screen within seconds.

\item Can be addressed to multiple receivers: Like E-mail, Flashmail
  provides the capability to send several individuals the same message at
  once.
\end{itemize}

\newpage
\section{User Interface}
The Flashmail main menu is shown in Figure 2.1.  The users are dat, cmoore,
johnson, jgeis, rosea, julio, and russ.  Cmoore, rosea, julio, and russ are
all dimmed because they are either not logged on, or they have not
activated Flashmail.  Although user jgeis is logged in, she is dimmed
because she is the current flashmail user. The current user is always
dimmed as one cannot send a flashmail to oneself. Dat, johnson, and jgeis
are all currently logged in and can be sent flashmail messages.  For these
three users, Flashmail displays the workstation they are currently using
and their idle time.  In this instance, user dat is using the workstation
named bianca and has been idle for 25 seconds.

To illustrate how Flashmail is utilized, suppose jgeis wants to
send user ``dat'' a Flashmail message.  First, the sender selects dat from
the available users in the Flashmail menu, notice that the box next to
dat's name is darkened, indicating the selection.  One can also send the
message to multiple recipients if so desired.

\begin{figure}[htb]
  {\centerline{\psfig{figure=/group/csdl/techreports/95-21/figures/mainmenu.ps}}}
  \caption{The Flashmail Main Menu}
  \label{Main-menu}
\end{figure}


A window as in Figure 2.2 then pops up on the senders screen and the
message to be sent is typed in. A click on the ``send'' button delivers the
message.

\begin{figure}[htb]
  {\centerline{\psfig{figure=/group/csdl/techreports/95-21/figures/compose.ps}}}
  \caption{Composing a Flashmail Message}
  \label{Compose}
\end{figure}

At this time, a window appears on dat's screen containing the original
message.  The recipient has the options of either deleting the message or
replying to it.  In this instance, the message necessitates a reply.
Figure 2.3 is dat's response, where he included the original message by
means of a ``cite'' mechanism shown in the reply-menu of Figure 2.4.

\begin{figure}[htb]
  {\centerline{\psfig{figure=/group/csdl/techreports/95-21/figures/receive.ps}}}
  \caption{A reply to the original Flashmail message}
  \label{Reply}
\end{figure}


\begin{figure}[htb]
  {\centerline{\psfig{figure=/group/csdl/techreports/95-21/figures/receive-menu.ps}}}
  \caption{The Reply Menu Options}
  \label{Reply-menu}
\end{figure}


\chapter{Evaluation of Flashmail}
First, we need to know the initial state of users in regards to their
Flashmail and E-mail experiences.  The pre-test questionnaire will answer
the questions of whether or not the subjects have already used Flashmail,
and if they feel that E-mail satisfies their current CMC needs.  Other
issues addressed include the frequency of E-mail use along with what other
CMC tools are being utilized.

I expect to find that a large majority of subjects have never used
Flashmail, and that most use E-mail as their primary means of
computer-mediated communication. I  also expect that most will feel that
E-mail is not a sufficient means of communicating time-dependent messages.
If these assumptions are true, then it is likely that Flashmail will be
utilized by the subjects for the majority of their time dependent
communications.  

The data collected during the experiment will consist of information
containing whether the messages sent are personal or business related in
nature, and if the messages require immediate responses, responses with
little time criteria, or no response at all. 

Aside from the business or personal aspect, I expect this data to be
different for each of the two systems being analyzed, E-mail and
Flashmail. For E-mail, the results will probably indicate that most E-mail
messages do not require immediate responses, although whether or not they
require a response at all will likely vary. For Flashmail, I believe the
data will show that this particular CMC tool is utilized primarily for
messages requiring an immediate response.

Other information being collected during the experiment include who sent
messages and when the messages were sent.  The purpose of this information
is to assess the frequency of each user's CMC use.  If it is shown that a
subject utilized E-mail 20 times during the experiment and used Flashmail only
once, they probably did not find Flashmail to be a useful tool, although
the possibility exists that they had only one time-dependent message to
send during the duration of the experiment.  The data mentioned earlier
regarding time dependency will be used to analyze if this is actually the
case.

After the experiment the post-test questionnaire will be administered,
collected, and evaluated.  This data will consist of how the users
perceived their E-mail and Flashmail use, whether or not they found
Flashmail to be a useful tool, the utility of Flashmail in conjunction with
E-mail, and their general messaging behavior.

The data regarding how users perceive their E-mail and Flashmail use will
be used to see if the subjects perception of the tools match their actual
behavior.  For example, if a subject believes that they used E-mail
primarily for time-dependent messages, when the data collected previously
indicates that this is false, it might make for an interesting future
study. 

The information about the usefulness of Flashmail and its relationship to
E-mail is to assess if Flashmail is indeed utilized as an ``add-on'' system
to E-mail as it was designed to be.

In regards to the subject's communication behavior, this information is
especially important.  The purpose of this information is to analyze the
validity of the information collected.  The aim of this data is to see if
the subject's behavior was typical of their normal messaging patterns.  If
the subjects indicate that their behavior was typical of their normal
messaging behavior, then the data is most likely an accurate representation
of such.  On the other hand, if the subjects indicate that they changed
their behavior on account of the study, the information is rendered invalid
and the study must be redesigned.


\chapter{Conclusion}
My goal is to investigate whether or not Flashmail is a viable CMC tool and
under what conditions it normally is used.

The pre-test questionnaire is needed to determine the participants previous
experiences with CMC.  I will learn whether or not students have used
Flashmail before.  I will also determine what sort of experiences the
subjects have had with the CMC tools that they are currently using.

Having a mini-questionnaire filled out each time a Flashmail or
E-mail message is sent is necessary to determine what is accomplished
through these tools.  The results of these questionnaires will indicate
whether or not Flashmail is indeed a useful CMC tool and for what types of
messages it is being used for in contrast with the types of messages E-mail
is utilized for.

After the one week long experiment, a post-test questionnaire is to be given
to assess the validity of the data collected and evaluate whether or not
participants found Flashmail to be a useful tool.  I am expecting to find that
users will utilize Flashmail as a complimentary system to E-mail.  I do not
believe that Flashmail will replace E-mail.  Instead, I think that Flashmail
will make up for the areas in which E-mail is not sufficient, primarily the
communication of time-dependent messages.  Flashmail is not designed to be
a ``stand-alone'' program, and I do not think it will be utilized as such.

\appendix
\newpage
\chapter{Pre-Test Questionnaire}
\begin{enumerate}
\item Are you currently familiar with Flashmail?  If so, how often do you
  use it?
  \begin{itemize}
  \item Daily \item Weekly \item Monthly \item Never
  \end{itemize}
\item Do you have any suggestions on how to improve Flashmail?
\item How often do you use E-mail?
  \begin{itemize}
  \item Daily \item Weekly \item Monthly \item Never
  \end{itemize}
\item Which E-mail system do you normally use? (e.g. vm, pine, etc.)
\item Are there any other computer-mediated communication (CMC) tools that
  you use on a regular basis?  If so, what are they?
\item Do you find that E-mail is sufficient for time-dependent
  communications?  (Time-dependent communications are messages whose value
  expires after a short period of time.  For example, the message ``Do you
  want to meet for lunch?'' requires a response before lunchtime, otherwise
  the message is useless.)
\item Do you have any suggestions on how to improve E-mail?
\end{enumerate}

\newpage
\chapter{Content-Characteristics Checklist}
The content-characteristics checklist shown in Figure B.1 will need to be
filled out before any message is sent.  In this instance, the items
``Personal'' and ``Immediate reply needed'' are checked to indicate that the
message being sent is of a personal nature and is a time-dependent
message. A click on the ``Comments'' button result in a new window popping
up called Comments, as illustrated in Figure B.2.


\begin{figure}[htb]
  {\centerline{\psfig{figure=/group/csdl/techreports/95-21/figures/experiment.ps}}}
  \caption{The Flashmail Content-Characteristics Checklist}
  \label{Checklist}
\end{figure}


\begin{figure}[htb]
  {\centerline{\psfig{figure=/group/csdl/techreports/95-21/figures/comments.ps}}}
  \caption{The Content-Characteristics Comments Section }
  \label{Comments}
\end{figure}



\newpage
\chapter{Post-Test Questionnaire}
\begin{enumerate}
\item How often do you use E-mail?
  \begin{itemize}
  \item Daily \item Weekly \item Monthly \item Never
  \end{itemize}
\item How often do you use Flashmail?
  \begin{itemize}
  \item Daily \item Weekly \item Monthly \item Never
  \end{itemize}
\item For what types of messages did you use Flashmail primarily for?
  \begin{itemize}
  \item Personal \item Business \item Immediate reply needed \item
    Reply needed-delayed O.K. \item Reply not needed
  \end{itemize}
\item For what types of messages did you use E-mail primarily for?
  \begin{itemize}
  \item Personal \item Business \item Immediate reply needed \item
    Reply needed-delayed O.K \item Reply not needed
  \end{itemize}
\item Do you think Flashmail is a useful computer-mediated communication
  tool?
\item Did you find Flashmail to be a system that is complimentary to
  E-mail? (meaning Flashmail made up for some problems in E-mail, but did
  not replace E-mail.)
\item Do you feel that your communication behavior during the study was
  typical of your normal messaging pattern, or do you think that you
  altered your behavior in any way?
\item If you did alter your behavior, could you please explain in what way
  you did so and why?
\item Do you have any suggestions on how to improve Flashmail?
\end{enumerate}
Comments: \nocite{*} \bibliographystyle{plain} \bibliography{fmail}
\printindex
\end{document}



















