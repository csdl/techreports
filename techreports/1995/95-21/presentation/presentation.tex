\documentstyle[11pt,slidesonly]{/group/csdl/tex/seminar}
%%% Use: dvitps -L -O file.ps file.dvi
%%% or Use: dvips file.dvi -t landscape -o file.ps
\newcommand{\horizontalline} {\rule{\textwidth}{.02in}} \slideframe{none}
\slidesmag{0}        % integer value ranging from -5 to 9
\special{landscape}  %comment out this line for notes
\pagestyle{empty}
\begin{document}      

\begin{slide} \Huge 
  \begin{center} 
    {\bf Flashmail:\\ a new computer-mediated communication tool}\\

    \vspace{0.5in} by\\ Jennifer Geis\\ jgeis@uhics.ics.hawaii.edu
    \vspace{0.5in}
    
    Collaborative Software Development Laboratory \\ Department of
    Information \& Computer Sciences\\ University of Hawaii at Manoa
  \end{center} 
\end{slide} \Huge   


\begin{slide}\Huge 
  {\bf Research area} \horizontalline
  \begin{itemize}\huge
  \item {\bf What is Computer-Mediated Communication (CMC)? \ }
    \begin {itemize}
  \item The use of computers to facilitate human communications.
  \end{itemize}
  \begin {itemize}
\item Examples of CMC:
  \begin {itemize}
\item WWW \item IRC \item E-mail
  \end {itemize}
  \end {itemize}
\item {\bf Research goal:}
  \begin{itemize}
  \item To explore an alternative to E-mail.
  \end{itemize}
\end{itemize}
\end{slide}



\begin{slide}\Huge 
  {\bf What's wrong with E-mail?}  \horizontalline
  \begin{itemize}\huge
  \item Insufficient for the communication of time dependent messages.
    \begin{itemize}
    \item For the sender:
      \begin{itemize}
      \item E-mail does not let you know when or if the person has
        received your message.  \item If the receiver of the E-mail
        does not read and respond to it within seconds or minutes,
        then the value of the E-mail is lost.
      \end{itemize}
    \item For the receiver:
      \begin{itemize}
      \item Some experience pressure to read their E-mails as soon
        as they come in so they won't miss these types of messages.
      \item Such E-mails have no value once read, they simply
        clutter one's mailbox unless deleted immediately.
      \end{itemize}
    \end{itemize}
  \end{itemize}
\end{slide}


\begin{slide}\Huge 
  {\bf The Solution?  Flashmail} \horizontalline
  \begin{itemize}\huge
  \item {\bf What is Flashmail?}
    \begin{itemize}
    \item Flashmail is a simple group-ware tool providing two
      facilities:
      \begin{itemize}
      \item Real-time information on which workstation each group
        member is using and the current idle-time associated with
        that workstation.  \item A mechanism to pop up a window
        containing a message on individual group members who are
        logged in to a workstation.
      \end{itemize}
    \end{itemize}
  \item {\bf The features of Flashmail}
    \begin{itemize}
    \item Flashmail lets you know if a member of the group is
      electronically accessible or not and their location.  \item Once
      a Flashmail message is sent, it pops up as a window on the
      receiver(s) screen(s) immediately.  \item Flashmail messages can
      be cut and pasted if the user wants to save them, but by default
      they are deleted.
    \end{itemize}
  \end{itemize} 
\end{slide}


\begin{slide}\Huge 
  {\bf Experimental procedures} \horizontalline
  \begin{itemize}\huge
  \item Subjects: CSDL, ICS-413, and ICS-613.  \item Qualifications:
    \begin{itemize}
    \item Willing to participate \item Currently using Xemacs
    \end{itemize}  
  \item Design:
    \begin{itemize}
    \item Pre-test questionnaire \item Content characteristics checklist
    \item Post-test questionnaire
    \end{itemize}
    
  \end{itemize} 
\end{slide}

\end{document} 




