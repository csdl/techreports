%%%%%%%%%%%%%%%%%%%%%%%%%%%%%% -*- Mode: Latex -*- %%%%%%%%%%%%%%%%%%%%%%%%%%%%
%% primer-tools.tex -- 
%% Author          : Philip Johnson
%% Created On      : Mon May 15 13:56:44 1995
%% Last Modified By: Philip Johnson
%% Last Modified On: Tue Sep 26 09:25:39 1995
%% Status          : Unknown
%% RCS: $Id$
%%%%%%%%%%%%%%%%%%%%%%%%%%%%%%%%%%%%%%%%%%%%%%%%%%%%%%%%%%%%%%%%%%%%%%%%%%%%%%%
%%   Copyright (C) 1995 University of Hawaii
%%%%%%%%%%%%%%%%%%%%%%%%%%%%%%%%%%%%%%%%%%%%%%%%%%%%%%%%%%%%%%%%%%%%%%%%%%%%%%%
%% 

\section{The Primer Tools Menu}
\label{sec:primer-tools}

\subsection{Motivation}
To work through an exercise in this Primer, you must first bring up an HBS
server, a Gagent, and one or more client processes.  In most cases, you
must then load the implementation of one or more predefined primer classes
which serve as the foundation for the exercises.

Next, you often gain a better feel for the example by invoking some or all
of the forms in the ``Example Usage'' section, perhaps varying the
arguments slightly from those provided. 

Simply working through the mechanics of these actions can take quite a bit
of time.  To eliminate the tedium associated with each of these ``startup''
activities, we've developed an XEmacs menu called ``Primer'' containing
some simple, but useful tools to facilitate:
\begin{itemize}
\item Bringing an HBS server up and down.
\item Bringing the Gagent and clients up and down.
\item Loading and viewing Primer example class implementations.
\item Viewing your own primer working files.
\item Viewing the example session log files. 
\end{itemize}

\subsection{Loading the Primer Tools Menu}

To load the Primer tools menu, you must load the primer-tools menu 
implementation and invoke the p*env*make form.  Here is an example:

\begin{code}
(load "/group/csdl/ecs/primer-examples/dot.emacs-primer.el")
(p*env*make :hbs-directory "/home/4/johnson/HBS/Primer/" 
            :hbs-socketID 10003 
            :password "kdk87bhda"
            :examples-directory "/group/csdl/ecs/primer-examples/"
            :working-directory "/home/4/johnson/Primer-workspace/")
\end{code}

The first form loads the implementation. Edit its argument to the correct
path for the dot.emacs-primer.el file in the primer-examples subdirectory.

The second form calls p*env*make with some arguments that simplify Primer
activities. 
\begin{itemize}
\item The :hbs-directory keyword argument specifies the directory in
  which *.hb files will be created when the Primer menu item for creating
  a new HBS server is invoked.  Replace the example
  ``/home/4/johnson/HBS/Primer'' subdirectory by one accessable to you.
\item The :hbs-socketID argument indicates the socket number to be
  used. Replace ``10003'' by some other integer if you prefer your Primer
  HBS to listen on a different socket for connect requests.
\item The :password argument supplies a password to be used by the Gagent
  and all user clients for your personal primer sessions.  For security's
  sake, make sure that the file containing this call to p*env*make is not
  readable by others!  In CSDL, we typically create a file called
  ``.primer'' in our home directory containing these two forms, set the
  permissions to rw owner only, then call {\tt (load ".primer")} within 
  our .emacs file.  
\item The :examples-directory argument specifies where the example files
  are located.  Change this path only if the ecs/primer-examples subdirectory
  is located elsewhere on your system.
\item The :working-directory argument specifies a path to a subdirectory
  where you keep your personal files created while working on Primer
  exercises.  Replace the example ``/home/4/johnson/Primer-workspace/'' by a
  subdirectory in your own home directory.  (This argument is optional but 
  highly recommended.)
\end{itemize} 

Once this form is edited appropriately and placed in your .emacs file, 
you should find that a menu called ``Primer'' appears when you bring up 
XEmacs.   An example screen dump of this Primer tools menu appears in 
Figure \ref{fig:primer-tools}.

\begin{figure}[htbp]
\centerline{\psfig{figure=/group/csdl/techreports/95-10/primer-tools.ps}}
\caption{The Primer Tools menu.  The window behind displays the output
from starting up a Primer HBS process.}  
\label{fig:primer-tools}
\end{figure}

\subsection{Using the Primer Tools Menu}

\subsubsection{Starting up an HBS Server}


The first menu selection in the Primer menu brings up a new HBS server.
This server will place its *.hb files in the subdirectory specified in the
call to p*env*make, and use the connect socket number specified in
p*env*make.  It creates a db-ID for this database by concatenating together
``Primer-'' with the current username, and this db-ID appears in this 
first menu selection. 

When the ``Start HBS''  menu item is selected, the window splits in half
and a new Emacs subprocess is started to run the server.  Here is an
example of the output obtained from a successful Start HBS command:

\begin{code}
--- working directory: /home/4/johnson/HBS/Primer/
% /group/csdl/bin/hbs -socketid 10003 -dbID Primer-johnson

HBS version 3.2.8 (Feb 26, 1995)
Collaborative Software Development Laboratory 
The University of Hawaii, 1994
Mon May 15 15:21:18 1995
Socketid is: 10003
\end{code}

You do not have to keep this buffer displayed, of course, but do not delete
it. 

There are three possibilities for error during starting: the chosen
socketid could be in use, the hbs executable may not be found, or 
the *.hb files could be corrupted. 

\paragraph{Startup error: socket in use.}  If the HBS fails to start up
because the socket is already in use, there are two possibilities.  

First, the most likely reason is because an HBS server is already running
on that socket.  You can check this via ``ps -ef'' (on Solaris) or ``ps
-aux'' (on SunOS).  If there is an HBS on that socket, you normally connect
to that one instead of starting up a new one.

Second, the socket may be in use even though there is {\em no} HBS running.
This can happen because XEmacs and/or Unix are fairly conservative about
cleaning up killed subprocesses and allocated sockets.  Thus, if you (a)
start up HBS within XEmacs, then kill the HBS, then try to startup HBS
again within the {\em same} XEmacs, you will often get this ``socket in
use'' error, even though there is no HBS server running.  Here are the
workarounds: 

\begin{enumerate}

\item Start up HBS manually at the command line, outside of XEmacs.  Unix
  seems to clean up the sockets almost immediately in this case.

\item Start up HBS within XEmacs using the Primer menu, but if you need to
  kill HBS {\em and} restart it again, you must first exit XEmacs, wait until
  the socket is deallocated, then start up a fresh XEmacs to restart HBS.  You
  can check to see if the socket is still allocated via:
  \begin{code}
    netstat -a | grep <socketid>
  \end{code}
\end{enumerate} 

Since it is typically rare to startup HBS, kill it, and restart it within a
single XEmacs session, you will typically run into this problem only
rarely. However, it's very confusing if you don't understand what's going
on when it does happen, thus this brief explanation.

\paragraph{Startup error: HBS executable not found.}  If you are running
this primer outside the UH ICS environment, then the default pathname for
the HBS executable in the Primer Tools implementation will need to be
changed.  The pathname used is the value of the variable 
p*env*hbs-command, which defaults to ``/group/csdl/bin/hbs''.  You can 
setq this variable to the appropriate value for your site. 

\paragraph{Startup error: hb file opening problems.}  Under rare
circumstances, one of the *.hb files containing the database can get
corrupted, which prevents HBS from starting up correctly. In this case, you
need to go to the Primer hbs-directory and delete the files manually, 
then try again. 



\subsubsection{Starting a Gagent}

The next step in bringing up Egret is to start up a Gagent.  The second
menu item in the Primer menu allows you to ``Connect as Gagent''.
Selecting that menu item connects this process as the Gagent, and resets
the screen name to ``PRIMER: GAgent ({\em db-ID})'', where {\em db-ID} is
replaced by the actual db-ID used by your primer HBS.

If the Gagent does not connect successfully when invoked from the Primer
menu, it is most likely due to corrupted *.hb files.  In this case, you
must bring down the HBS server, delete the *.hb files manually, and bring
up the HBS server and the Gagent again. 

After successfully connecting this XEmacs process as a Gagent, you will 
notice that all of the ``Connect'' menu items have been deselected, and 
the menu item ``Disconnect (Gagent or Client)'' is now selectable.  This
is done, of course, since you cannot connect to the same db-ID twice from 
one XEmacs process.  

At this point, you typically iconify this screen and bring up a new 
XEmacs process to connect as a user. 

\subsubsection{Connecting as a user}

First, bring up a fresh XEmacs process. Second, pull down the Primer menu
and select one of the four user ``Connect As'' menu items.  The Primer
provides you with four automatically defined user names, created by
concatenating your username with the numbers 1, 2, 3, and 4. As with the ``Connect
as Gagent'' menu item, once you have connected as a user, the connect menu
items are deselected and the disconnect menu item becomes selectable.

\subsubsection{Loading and viewing files}

At this point, you should have one XEmacs process connected as the Gagent
(with the HBS server running as a subprocess of this process), and another
XEmacs process connected as a user.   You will use this latter user process to 
load, view, and complete your Primer exercises.  

The ``Load Primer Implementations'' menu cascades into a set of file names 
that, when selected, automatically load the corresponding files.  Each of
these files corresponds to the implementation of a class or system from
one section of the Primer.  In addition to these files, there appears a file
called dot.emacs-primer.el, which contains the source for this Primer tools
implementation.  You don't normally need to load this file (since it is 
already loaded by your .emacs file), but reloading it will do no harm. 

The ``View Primer Implementations'' menu cascades into the same set of file
names as the the ``Load Primer Implementations'' menu, but when these are
selected, the corresponding file is brought into a buffer.  These files
should all be read-only, since you should not edit the primer example
files.  Again, the dot.emacs-primer.el file appears in this list, which
allows you to easily peruse the implementation of these tools. 

The ``View Primer Sessions'' menu cascades into a set of file
names with the same prefix as the implementations, but with a suffix of
``.log''.  These files contain the example sessions found in the Primer, 
and allow you to easily replicate their results.  Again, these files are
read-only, so you cannot edit them, but you can still place the point
after any of the forms and type c-x c-e to evaluate them.  

The ``View Personal Work Files'' cascades into a set of all
the files currently located in the subdirectory specified in the
:working-directory argument to p*env*make, as documented above.  

When files are added and subtracted from your personal working directory (or
from the Primer subdirectories), the cascading menu items above do not
automatically update themselves.  (Sorry, I'm not an XEmacs Menu Wizard
yet.)  The ``Refresh Directory Entries'' menu item rebuilds these 
cascading menus with the current contents of the subdirectories. 


\subsubsection{Disconnecting clients and bringing down the server}

The final set of menu items allows you to easily disconnect the current
client from the HBS server, or bring down the HBS server entirely.  Since
the HBS server runs as a subprocess of an XEmacs process, bringing down
that XEmacs process automatically kills the associated HBS process. 


















