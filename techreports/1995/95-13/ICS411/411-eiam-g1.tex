%%% \documentstyle[11pt,/group/csdl/tex/definemargins,
%%%                        /group/csdl/tex/lmacros]{article} 
%%% 
%%%           \begin{document}
%%%           \begin{center}
%%%           {\large\bf CSRS Experiment Results}\\
%%%           \end{center}
%%%           \small 
%%% 	  
\chapter {CSRS Experiment Results (ICS411): Group1 (EIAM)}
\small

\begin{description}
\item [Method:] EIAM
\item [Group:] Group1
\item [Source:] Pass1
\item [Participants:] mjuan (Reviewer), jflores (Reviewer), jedwards (Reviewer)
\end{description}
\section{Issue Lists}
\begin{enumerate}
\item {\it Issue\#198 (jedwards)}
\begin{description}
\item [Subject:] Spelling/Syntax
\item [Criticality:] Low
\item [Confidence-level:] Low
\item [Source-node:] Type and var declarations

\item [Lines:] 18

\item [Description:] This statement should be deleted since the following line is the correct one.
\end{description}
\item {\it Issue\#202 (jedwards)}
\begin{description}
\item [Subject:] Syntax/Space
\item [Criticality:] Med
\item [Confidence-level:] Med
\item [Source-node:] Type and var declarations

\item [Lines:] 19-20

\item [Description:] Need to remove the spaces in between the \# and
define.
\end{description}
\item {\it Issue\#206 (jedwards)}
\begin{description}
\item [Subject:] Array size
\item [Criticality:] Hi
\item [Confidence-level:] Med
\item [Source-node:] Type and var declarations

\item [Lines:] 67

\item [Description:] Will possibly lead to a constraint error.  The
array SYMTAM's limit should be SYMTABLIMIT only.  Do not add 1.
\end{description}
\item {\it Issue\#208 (mjuan)}
\begin{description}
\item [Subject:] Accessing the SYMTAB or symbol table
\item [Criticality:] Low
\item [Confidence-level:] Hi
\item [Source-node:] Access\_Symtab

\item [Lines:] 39-40

\item [Description:] Whether the function is used to STORE or
SEARCH a symbol into the symbol table, it will always store it.  Therefore we
could have redundant data, or an overflow.  It will still run, but there is
no conservation of memory allocation and lost of integrity.
\end{description}
\item {\it Issue\#214 (jedwards)}
\begin{description}
\item [Subject:] Array
\item [Criticality:] Med
\item [Confidence-level:] Med
\item [Source-node:] Type and var declarations

\item [Lines:] 75

\item [Description:] I'm rather unsure about this, but I believe
that the array is incorrect.
\end{description}
\item {\it Issue\#220 (jedwards)}
\begin{description}
\item [Subject:] Conjunction Condition
\item [Criticality:] Med
\item [Confidence-level:] Med
\item [Source-node:] hextonum

\item [Lines:] 15

\item [Description:] This appears to be incorrect in that it will
not do what the programmer intends it to do.
\end{description}
\item {\it Issue\#226 (jedwards)}
\begin{description}
\item [Subject:] Error in statement
\item [Criticality:] Med
\item [Confidence-level:] Low
\item [Source-node:] hextonum

\item [Lines:] 16

\item [Description:] Adding the value 10 to this statement is
incorrect cause it will not return the desired result.
\end{description}
\item {\it Issue\#230 (mjuan)}
\begin{description}
\item [Subject:] When the current source is a comment line.
\item [Criticality:] Med
\item [Confidence-level:] Hi
\item [Source-node:] Write\_Int\_File

\item [Lines:] 15-17

\item [Description:] Somewhat minor when writing the INTFILE, since
if it is a comment line, it should write the comment AND write at least a
newline, allowing it to be reset to the next line when accessed again.  If
there should come a time of reading the INTFILE, problems could occur and may
crash the program.  But it depends how the program reads it.  But, if one
were to view it, it would look very ugly.
\end{description}
\item {\it Issue\#234 (jedwards)}
\begin{description}
\item [Subject:] While Loop Constraint
\item [Criticality:] Hi
\item [Confidence-level:] Med
\item [Source-node:] hextonum

\item [Lines:] 21-22

\item [Description:] The condition of having i {\tt >} first + 3 is in
correct cause for the first 3 values the loop will not terminate no matter
what.
\end{description}
\item {\it Issue\#244 (jedwards)}
\begin{description}
\item [Subject:] Unclear statement
\item [Criticality:] Med
\item [Confidence-level:] Low
\item [Source-node:] hextonum

\item [Lines:] 25

\item [Description:] Not quit sure what the programmer is doing by
setting *last = first + i.  Since the values of first and i could be return
an incorrect value.
\end{description}
\item {\it Issue\#250 (jflores)}
\begin{description}
\item [Subject:] Unknown assignment result
\item [Criticality:] Hi
\item [Confidence-level:] Hi
\item [Source-node:] Access\_Symtab

\item [Lines:] 11

\item [Description:] Hash variable not declared.
\end{description}
\item {\it Issue\#254 (mjuan)}
\begin{description}
\item [Subject:] Checking to see if the label field is a digit
\item [Criticality:] Med
\item [Confidence-level:] Med
\item [Source-node:] P1\_Read\_Source

\item [Lines:] 44-45

\item [Description:] Since it is checking to see if the the label
is numeric, this will work, but if any case should an operator refer to it as
an operand, it might interpret it as a numeric rather than a symbol.
Therefore ALL labels should be ALPHA.
\end{description}
\item {\it Issue\#258 (jedwards)}
\begin{description}
\item [Subject:] hash calculation statement
\item [Criticality:] Med
\item [Confidence-level:] Low
\item [Source-node:] Access\_Symtab

\item [Lines:] 12

\item [Description:] This statement is unclear, especially in
consideration of if it will do as required
\end{description}
\item {\it Issue\#262 (jflores)}
\begin{description}
\item [Subject:] While condition not working
\item [Criticality:] Hi
\item [Confidence-level:] Hi
\item [Source-node:] Pass\_1

\item [Lines:] 15

\item [Description:] Need to assign endofinput boolean
value first.
\end{description}
\item {\it Issue\#266 (jedwards)}
\begin{description}
\item [Subject:] prt assignment
\item [Criticality:] Med
\item [Confidence-level:] Med
\item [Source-node:] Access\_Symtab

\item [Lines:] 28

\item [Description:] This will not achieve the desired result.
\end{description}
\item {\it Issue\#270 (jedwards)}
\begin{description}
\item [Subject:] address variable is unclear
\item [Criticality:] Med
\item [Confidence-level:] Low
\item [Source-node:] Access\_Symtab

\item [Lines:] 19

\item [Description:] The purpose of setting the address to this
term is incorrect
\end{description}
\item {\it Issue\#274 (mjuan)}
\begin{description}
\item [Subject:] Checking the operand for blanks.
\item [Criticality:] Low
\item [Confidence-level:] Hi
\item [Source-node:] P1\_Proc\_START

\item [Lines:] 21-23

\item [Description:] Operand values is from location 18 to 34, not
from 4 to 17 which is the operator values.
\end{description}
\item {\it Issue\#278 (jedwards)}
\begin{description}
\item [Subject:] Else condition
\item [Criticality:] Low
\item [Confidence-level:] Med
\item [Source-node:] Access\_Symtab

\item [Lines:] 53-55

\item [Description:] These statements following the ELSE condition
appear to be incorrect.
\end{description}
\item {\it Issue\#288 (mjuan)}
\begin{description}
\item [Subject:] Retrieving the new location counter
\item [Criticality:] Low
\item [Confidence-level:] Hi
\item [Source-node:] P1\_Assign\_Loc

\item [Lines:] 11-12

\item [Description:] No matter what newlocctr returns, at the end
of the routine, it will always be replaced with locctr+3.  Therefore we get
redundant values no matter how oftern we call it.
\end{description}
\item {\it Issue\#290 (jflores)}
\begin{description}
\item [Subject:] Wrong blank assign for labl.
\item [Criticality:] Med
\item [Confidence-level:] Med
\item [Source-node:] P1\_Read\_Source

\item [Lines:] 36

\item [Description:] Blank assignment should be BLANK7.
\end{description}
\item {\it Issue\#294 (jedwards)}
\begin{description}
\item [Subject:] FIle Print Statement
\item [Criticality:] Med
\item [Confidence-level:] Low
\item [Source-node:] Write\_Int\_File

\item [Lines:] 14

\item [Description:] This will not acheive the apparent desired goal
\end{description}
\item {\it Issue\#302 (mjuan)}
\begin{description}
\item [Subject:] Retriveing the location counter.
\item [Criticality:] Low
\item [Confidence-level:] Hi
\item [Source-node:] P1\_Assign\_Loc

\item [Lines:] 14-15

\item [Description:] Again, no matter what newlocctr returns, it
will always be replace by locctr + 3.
\end{description}
\item {\it Issue\#308 (mjuan)}
\begin{description}
\item [Subject:] Retrieving the location counter.
\item [Criticality:] Low
\item [Confidence-level:] Hi
\item [Source-node:] P1\_Assign\_Loc

\item [Lines:] 17-18

\item [Description:] Again, no matter what it returns, it is
replace by locctr + 3.
\end{description}
\item {\it Issue\#312 (jedwards)}
\begin{description}
\item [Subject:] Conditional Error
\item [Criticality:] Med
\item [Confidence-level:] Med
\item [Source-node:] Write\_Int\_File

\item [Lines:] 25-26

\item [Description:] THis does not make any sense.
\end{description}
\item {\it Issue\#316 (mjuan)}
\begin{description}
\item [Subject:] Retrieving the new location counter.
\item [Criticality:] Low
\item [Confidence-level:] Hi
\item [Source-node:] P1\_Assign\_Loc

\item [Lines:] 20-21

\item [Description:] Again, routine will always return locctr + 3
no matter what newlocctr returns.
\end{description}
\item {\it Issue\#320 (mjuan)}
\begin{description}
\item [Subject:] Retrieving the new location counter.
\item [Criticality:] Low
\item [Confidence-level:] Hi
\item [Source-node:] P1\_Assign\_Loc

\item [Lines:] 23-24

\item [Description:] It will always return locctr + 3 regardless
what newlocctr returns.
\end{description}
\item {\it Issue\#322 (jflores)}
\begin{description}
\item [Subject:] else never executed because if stmt always true.
\item [Criticality:] Low
\item [Confidence-level:] Hi
\item [Source-node:] P1\_Read\_Source

\item [Lines:] 62

\item [Description:] Need to fix test condition in if stmt.
\end{description}
\item {\it Issue\#328 (jedwards)}
\begin{description}
\item [Subject:] Uncertainty
\item [Criticality:] Med
\item [Confidence-level:] Med
\item [Source-node:] P1\_Read\_Source

\item [Lines:] 16

\item [Description:] THis appears to be decieving to the reader.
Won't this result in an error?
\end{description}
\item {\it Issue\#330 (mjuan)}
\begin{description}
\item [Subject:] Writing a symbol to SYMTAB.
\item [Criticality:] Low
\item [Confidence-level:] Hi
\item [Source-node:] P1\_Assign\_Sym

\item [Lines:] 16-17

\item [Description:] It is suppose to write to the SYMTAB, not
search twice.
\end{description}
\item {\it Issue\#336 (jedwards)}
\begin{description}
\item [Subject:] Incorrect Conditional
\item [Criticality:] Med
\item [Confidence-level:] Hi
\item [Source-node:] P1\_Read\_Source

\item [Lines:] 32

\item [Description:] This conditional has an error in it.
\end{description}
\item {\it Issue\#340 (jedwards)}
\begin{description}
\item [Subject:] Incorrect condition
\item [Criticality:] Hi
\item [Confidence-level:] Low
\item [Source-node:] P1\_Read\_Source

\item [Lines:] 24

\item [Description:] The FOR conditional is incorrect and will
result in a program error
\end{description}
\item {\it Issue\#344 (jedwards)}
\begin{description}
\item [Subject:] While loop is wrong
\item [Criticality:] Med
\item [Confidence-level:] Med
\item [Source-node:] P1\_Read\_Source

\item [Lines:] 43-44

\item [Description:] The while conditional will not be met, thus
producing a program error
\end{description}
\item {\it Issue\#346 (mjuan)}
\begin{description}
\item [Subject:] LOCCTR being assigned the newlocctr.
\item [Criticality:] Low
\item [Confidence-level:] Med
\item [Source-node:] Pass\_1

\item [Lines:] 22-23

\item [Description:] LOCCTR should be assigned newlocctr whether or
not the operation of the source is "START" or not.  Otherwise while assigning
the symbol to the table, values are redundant.  So creation of INTFILE would
be no problem, but if it were to be executed, a crash may occur, but then
again, that's something else.
\end{description}
\item {\it Issue\#354 (jedwards)}
\begin{description}
\item [Subject:] WHile Loop
\item [Criticality:] Hi
\item [Confidence-level:] Med
\item [Source-node:] P1\_Read\_Source

\item [Lines:] 57-59

\item [Description:] THis won't work correctly.
\end{description}
\item {\it Issue\#358 (jedwards)}
\begin{description}
\item [Subject:] Statement
\item [Criticality:] Med
\item [Confidence-level:] Low
\item [Source-node:] P1\_Read\_Source

\item [Lines:] 70

\item [Description:] Incorrect index for array
\end{description}
\item {\it Issue\#362 (jedwards)}
\begin{description}
\item [Subject:] Array index
\item [Criticality:] Med
\item [Confidence-level:] Low
\item [Source-node:] P1\_Read\_Source

\item [Lines:] 75

\item [Description:] This is incorrect cause of the comment array
index
\end{description}
\item {\it Issue\#366 (jflores)}
\begin{description}
\item [Subject:] i counter not inialized.
\item [Criticality:] Med
\item [Confidence-level:] Hi
\item [Source-node:] P1\_Proc\_RESW

\item [Lines:] 23

\item [Description:] While loop is unpredictable.
\end{description}
\item {\it Issue\#370 (jedwards)}
\begin{description}
\item [Subject:] Statement group
\item [Criticality:] Hi
\item [Confidence-level:] Low
\item [Source-node:] P1\_Proc\_START

\item [Lines:] 21-22

\item [Description:] Absolutely unsure as to why these statements
are structured as they are.  This will not execute properly
\end{description}
\item {\it Issue\#374 (jedwards)}
\begin{description}
\item [Subject:] statement
\item [Criticality:] Med
\item [Confidence-level:] Med
\item [Source-node:] P1\_Proc\_START

\item [Lines:] 29

\item [Description:] Setting newlocctr to temploc will not work
because of the possible values that can result.
\end{description}
\item {\it Issue\#378 (jedwards)}
\begin{description}
\item [Subject:] Loop and Conditional
\item [Criticality:] Med
\item [Confidence-level:] Med
\item [Source-node:] P1\_Proc\_RESW

\item [Lines:] 23-24

\item [Description:] Unsure about the operation of these
statements.  I believe that an error will result in the execution of these
statements.
\end{description}
\item {\it Issue\#382 (jedwards)}
\begin{description}
\item [Subject:] Last Conditional
\item [Criticality:] Hi
\item [Confidence-level:] Hi
\item [Source-node:] P1\_Proc\_RESW

\item [Lines:] 29-30

\item [Description:] THis conditional and the following assignment
do not make any sense.  Why add nwords?  This seems like an error would
result.
\end{description}
\item {\it Issue\#384 (jflores)}
\begin{description}
\item [Subject:] i not initialized.
\item [Criticality:] Low
\item [Confidence-level:] Hi
\item [Source-node:] P1\_Read\_Source

\item [Lines:] 43

\item [Description:] Need to start from 0-7 for Label. Initialize
i to 0.
\end{description}
\item {\it Issue\#390 (jedwards)}
\begin{description}
\item [Subject:] THis else if
\item [Criticality:] Med
\item [Confidence-level:] Med
\item [Source-node:] P1\_Assign\_Loc

\item [Lines:] 25

\item [Description:] The END should not be in this line.
\end{description}
\item {\it Issue\#392 (jflores)}
\begin{description}
\item [Subject:] While test condition is wrong.
\item [Criticality:] Low
\item [Confidence-level:] Hi
\item [Source-node:] P1\_Read\_Source

\item [Lines:] 57

\item [Description:] Should be i{\tt <}=14 to complete operation code.
\end{description}
\item {\it Issue\#398 (jedwards)}
\begin{description}
\item [Subject:] Conditional
\item [Criticality:] Med
\item [Confidence-level:] Med
\item [Source-node:] P1\_Assign\_Sym

\item [Lines:] 14-15

\item [Description:] The conditional and following assignment will
not work
\end{description}
\item {\it Issue\#402 (jedwards)}
\begin{description}
\item [Subject:] Uncertain
\item [Criticality:] Hi
\item [Confidence-level:] Med
\item [Source-node:] Pass\_1

\item [Lines:] 11

\item [Description:] Not sure as to what this command will do?  May
result in possible errors.
\end{description}
\item {\it Issue\#406 (jedwards)}
\begin{description}
\item [Subject:] LABL
\item [Criticality:] Hi
\item [Confidence-level:] Hi
\item [Source-node:] Pass\_1

\item [Lines:] 24-25

\item [Description:] Label size is incorrect.  Error.
\end{description}
\item {\it Issue\#410 (jflores)}
\begin{description}
\item [Subject:] Label should only have 7 spaces.
\item [Criticality:] Low
\item [Confidence-level:] Hi
\item [Source-node:] Type and var declarations

\item [Lines:] 47

\item [Description:] should be CHAR7 labl;
\end{description}
\item {\it Issue\#414 (jedwards)}
\begin{description}
\item [Subject:] COnditional
\item [Criticality:] Med
\item [Confidence-level:] Med
\item [Source-node:] Pass\_1

\item [Lines:] 21

\item [Description:] Error in the if conditional
\end{description}
\end{enumerate}
\section{Review Metrics}
\begin{table}[hb]
\begin{center}
\begin{tabular}{|l|l|l|l|l|}
\hline
Participant & Start-time & End-time & Elapsed-time & Busy-time \\
\hline
jedwards & May 05, 1995 09:13:44 & May 05, 1995 10:48:23 & 1:34:39 & 1:34:39 \\
jflores & May 05, 1995 09:14:57 & May 05, 1995 10:58:29 & 1:43:32 & 1:40:32 \\
mjuan & May 05, 1995 09:15:01 & May 05, 1995 10:21:10 & 1:6:9 & 1:6:9 \\
\hline
 & & Total & 4:24:20 & \\
\hline
\end{tabular}
\end{center}
\caption{Review Session}
\end{table}


\begin{table}[hb]
\begin{center}
\begin{tabular}{|l|l|l|l|}
\hline
Source & jedwards & jflores & mjuan\\
\hline
(172)Type and var declarations & 858 & 2145 & 139\\
(174)hextonum & 811 & 406 & 569\\
(176)Access\_Symtab & 864 & 1307 & 602\\
(178)Write\_Int\_File & 541 & 347 & 296\\
(180)P1\_Read\_Source & 775 & 2035 & 513\\
(182)P1\_Proc\_START & 713 & 187 & 431\\
(184)P1\_Proc\_RESW & 432 & 223 & 446\\
(186)P1\_Assign\_Loc & 239 & 205 & 460\\
(188)P1\_Assign\_Sym & 178 & 326 & 146\\
(190)Pass\_1 & 215 & 376 & 335\\
\hline
\end{tabular}
\end{center}
\caption{Review Time}
\end{table}

\begin{table}[hb]
\begin{center}
\begin{tabular}{|l|l|l|l|}
\hline
Source & jedwards & jflores & mjuan\\
\hline
hextonum & 119.9 & 239.4 & 170.8\\
Access\_Symtab & 241.7 & 159.8 & 346.8\\
Write\_Int\_File & 239.6 & 373.5 & 437.8\\
P1\_Read\_Source & 353.0 & 134.4 & 533.3\\
P1\_Proc\_START & 151.5 & 577.5 & 250.6\\
P1\_Proc\_RESW & 258.3 & 500.4 & 250.2\\
P1\_Assign\_Loc & 436.8 & 509.3 & 227.0\\
P1\_Assign\_Sym & 546.1 & 298.2 & 665.8\\
Pass\_1 & 602.8 & 344.7 & 386.9\\
\hline
\end{tabular}
\end{center}
\caption{Paraphrasing Rate (lines/hour)}
\end{table}

\begin{table}[hb]
\begin{center}
\begin{tabular}{|l|l|l|l|l|}
\hline
Source & jedwards & jflores & mjuan & OK\\
\hline
Type and var.. & 198,202,206,214 (=4) & 410 (=1) &  & \\
hextonum & 220,226,234,244 (=4) &  &  & 234\\
Access\_Symta.. & 258,266,270,278 (=4) & 250 (=1) & 208 (=1) & 258=250\\
Write\_Int\_Fi.. & 294,312 (=2) &  & 230 (=1) & 312\\
P1\_Read\_Sour.. & 328,336,340,344 & 290,322,384,392 (=4) & 254 (=1) & 344=384\\
 & 354,358,362 (=7) &  &  & 354,392\\
P1\_Proc\_STAR.. & 370,374 (=2) &  & 274 (=1) & \\
P1\_Proc\_RESW & 378,382 (=2) & 366 (=1) &  & \\
P1\_Assign\_Lo.. & 390 (=1) &  & 288,302,308,316 & \\
 &  &  & 320 (=5) & \\
P1\_Assign\_Sy.. & 398 (=1) &  & 330 (=1) & 330\\
Pass\_1 & 402,406,414 (=3) & 262 (=1) & 346 (=1) & 262\\
\hline
\end{tabular}
\caption{Source node v.s Issue node}
\end{center}
\end{table}

%%\end{document}
