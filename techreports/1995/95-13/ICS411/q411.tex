
\chapter {Questionnaire For Groups (EGSM) -- ICS411}


\begin{enumerate}
\item My understanding of the source code before the review was: 
\\
Very low \hfill 1 \dotfill  2 \dotfill 3 \dotfill 4 \dotfill 5 \hfill Very high\\
(Score: 3.3)


\item My understanding of the source code after the review was: 
\\
Very low \hfill 1 \dotfill  2 \dotfill 3 \dotfill 4 \dotfill 5 \hfill Very high\\
(Score: 3.8)

\item My understanding of C programming language was improved after
this review.  
\\
Not at all true \hfill 1 \dotfill  2 \dotfill 3 \dotfill 4 \dotfill 5 \hfill Very true\\
(Score: 3.1)


\item In general, our group found it easy to understand the logic of
the code. 
\\
Not at all true \hfill 1 \dotfill  2 \dotfill 3 \dotfill 4 \dotfill 5 \hfill Very true\\
(Score: 3.6)

\item During the review, I follow or pay attention to the presenter.
\\
Not at all \hfill 1 \dotfill  2 \dotfill 3 \dotfill 4 \dotfill 5
\hfill At all time\\
(Score: 4.1)

\item For this review, I would have preferred working in a group rather 
than working alone (if I have had a choice).
\\
Not at all true \hfill 1 \dotfill  2 \dotfill 3 \dotfill 4 \dotfill 5 \hfill Very true\\
(Score: 4.0)

\item For this review, I felt more confidence about the issues I raised when 
working in a group that working alone.
\\
Not at all true \hfill 1 \dotfill  2 \dotfill 3 \dotfill 4 \dotfill 5 \hfill Very true\\
(Score: 4.0)

\item In general, I felt working in this group increases my ability
in finding errors. 
\\
Not at all true \hfill 1 \dotfill  2 \dotfill 3 \dotfill 4 \dotfill 5 \hfill Very true\\
(Score: 4.1)

\item The training session for group review was sufficient.
\\
Not at all true \hfill 1 \dotfill  2 \dotfill 3 \dotfill 4 \dotfill 5 \hfill Very true\\
(Score: 4.0)

\item There was sufficient time to work on this review.
\\
Not at all true \hfill 1 \dotfill  2 \dotfill 3 \dotfill 4 \dotfill 5 \hfill Very true\\
(Score: 3.6)

\item Our group was motivated to do this review project.
\\
Not at all true \hfill 1 \dotfill  2 \dotfill 3 \dotfill 4 \dotfill 5 \hfill Very true\\
(Score: 3.8)

\item Our group worked seriously on this review.
\\
Not at all true \hfill 1 \dotfill  2 \dotfill 3 \dotfill 4 \dotfill 5 \hfill Very true\\
(Score: 4.2)

\item My overall confidence in the quality of our review was
\\
Very low \hfill 1 \dotfill  2 \dotfill 3 \dotfill 4 \dotfill 5 \hfill Very high\\
(Score: 3.9)

\item My overall satisfaction with the discussion among my group
 members was:
\\
Very low \hfill 1 \dotfill  2 \dotfill 3 \dotfill 4 \dotfill 5 \hfill Very high\\
(Score: 4.1)

\item This group was too small (in number of members) for best results
in the task it was trying to do.
\\
Not at all true \hfill 1 \dotfill  2 \dotfill 3 \dotfill 4 \dotfill 5 \hfill Very true\\
(Score: 2.0)

\item I felt comfortable doing this review with the group.
\\
Not at all true \hfill 1 \dotfill  2 \dotfill 3 \dotfill 4 \dotfill 5 \hfill Very true\\
(Score: 4.0)

\item There was much disagreement among the members of the group on
this task.
\\
Not at all true \hfill 1 \dotfill  2 \dotfill 3 \dotfill 4 \dotfill 5 \hfill Very true\\
(Score: 2.1)

\item Some people in the group dominated the discussion.
\\
Not at all true \hfill 1 \dotfill  2 \dotfill 3 \dotfill 4 \dotfill 5 \hfill Very true\\
(Score: 2.7)

\item My opinion was given adequate consideration by the other group members.
\\
Not at all true \hfill 1 \dotfill  2 \dotfill 3 \dotfill 4 \dotfill 5 \hfill Very true\\
(Score: 4.0)


\item I felt that I could express my opinion freely during group discussion. 
\\
Not at all true \hfill 1 \dotfill  2 \dotfill 3 \dotfill 4 \dotfill 5 \hfill Very true\\
(Score: 4.3)

\item I felt that I could express disagreement freely.
\\
Not at all true \hfill 1 \dotfill  2 \dotfill 3 \dotfill 4 \dotfill 5 \hfill Very true\\
(Score: 4.3)

\item I felt that I participated a great deal in the group discussion.
\\
Not at all true \hfill 1 \dotfill  2 \dotfill 3 \dotfill 4 \dotfill 5 \hfill Very true\\
(Score: 4.0)

\item I felt our group wasted too much time on unproductive
discussion.
\\
Not at all true \hfill 1 \dotfill  2 \dotfill 3 \dotfill 4 \dotfill 5 \hfill Very true\\
(Score: 2.2)

\item I felt I contributed a great deal to the discovery of issues.
\\
Not at all true \hfill 1 \dotfill  2 \dotfill 3 \dotfill 4 \dotfill 5 \hfill Very true\\
(Score: 3.7)

\item I felt the group made a great deal of influence on my decision
about what would be an issue, the criticality of an issue, and/or confidence-level.
\\
Not at all true \hfill 1 \dotfill  2 \dotfill 3 \dotfill 4 \dotfill 5 \hfill Very true\\
(Score: 3.3)

\item Overall, I was satified with the group interaction
\\
Not at all true \hfill 1 \dotfill  2 \dotfill 3 \dotfill 4 \dotfill 5 \hfill Very true\\
(Score: 4.2)

\item I would enjoy working with members of this group again.
\\
Not at all true \hfill 1 \dotfill  2 \dotfill 3 \dotfill 4 \dotfill 5 \hfill Very true\\
(Score: 4.3)

\item There was some open hostility in the group.
\\
Not at all true \hfill 1 \dotfill  2 \dotfill 3 \dotfill 4 \dotfill 5 \hfill Very true\\
(Score: 1.8)

\item Before this experiment, I knew all members of the group fairly well. 
\\
Not at all true \hfill 1 \dotfill  2 \dotfill 3 \dotfill 4 \dotfill 5 \hfill Very true\\
(Score: 3.5)

\item Our group took too long to come to an agreement.
\\
Not at all true \hfill 1 \dotfill  2 \dotfill 3 \dotfill 4 \dotfill 5 \hfill Very true\\
(Score: 2.4)

\item Some individual(s) in the group wanted to change things after
the group had come to a decision.
\\
Not at all true \hfill 1 \dotfill  2 \dotfill 3 \dotfill 4 \dotfill 5 \hfill Very true\\
(Score: 2.1)

\item I felt the presenter was too fast in presenting the code.
\\
Not at all true \hfill 1 \dotfill  2 \dotfill 3 \dotfill 4 \dotfill 5 \hfill Very true\\
(Score: 2.0)

\item The paraphrasing technique inspired me in finding errors.
\\
Not at all true \hfill 1 \dotfill  2 \dotfill 3 \dotfill 4 \dotfill 5 \hfill Very true\\
(Score: 3.3)

\item I felt the paraphrasing technique in general was useful.
\\
Not at all true \hfill 1 \dotfill  2 \dotfill 3 \dotfill 4 \dotfill 5 \hfill Very true\\
(Score: 3.5)

\item Overall, I felt the presenter did a good job in presenting/
paraphrasing the code.
\\
Not at all true \hfill 1 \dotfill  2 \dotfill 3 \dotfill 4 \dotfill 5 \hfill Very true\\
(Score: 4.0)

\item My overall satisfaction with my group was:
\\
Very low \hfill 1 \dotfill  2 \dotfill 3 \dotfill 4 \dotfill 5 \hfill Very high\\
(Score: 4.3)

\item My overall satisfaction with group review (EGSM) process as
outlined in the FTR guideline was:
\\
Very low \hfill 1 \dotfill  2 \dotfill 3 \dotfill 4 \dotfill 5 \hfill Very high\\
(Score: 4.2)

\item I believe EGSM system made my review more productive (i.e., find
lots of issues) 
\\
Not at all true \hfill 1 \dotfill  2 \dotfill 3 \dotfill 4 \dotfill 5 \hfill Very true\\
(Score: 4.0)

\item I believe EGSM system made my review more effective
(i.e., find lots of ``good'' issues in a relatively short time)
\\
Not at all true \hfill 1 \dotfill  2 \dotfill 3 \dotfill 4 \dotfill 5 \hfill Very true\\
(Score: 4.1)

\item EGSM system is easy to use.
\\
Not at all true \hfill 1 \dotfill  2 \dotfill 3 \dotfill 4 \dotfill 5 \hfill Very true\\
(Score: 4.3)


\item EGSM system is useful.
\\
Not at all true \hfill 1 \dotfill  2 \dotfill 3 \dotfill 4 \dotfill 5 \hfill Very true\\
(Score: 4.2)

\item My overall satisfaction with EGSM system was
\\
Very low \hfill 1 \dotfill  2 \dotfill 3 \dotfill 4 \dotfill 5 \hfill Very high\\
(Score: 4.2)

\item Problems that I had with paraphrasing technique (please
explain):
\begin{itemize}
\item we need more time for this kind of going
\item explaining things 


\item sometimes it is hard to say what the code does before you understand
it well
also, some c source is not easy to paraphrase

\item None, I found it very useful.
\item What do I say
\item I felt that the presenter could have used more paraphrasing when going
through the code.
But I thought that paraphrasing was good.

\item NO PROBLEMS
\item None
\item The paraphrasing was effective, but could be improved by paraphrasing
the code in laymans term.





\item no problems
\item Do you mean paraphrasing what each line means aloud?  I like that idea.

\item It would probably have been code to describe it in psuedocode rather
than in direct C for clarity.
\item No problem
\item Understanding some of the code
\end{itemize}

\item Problems that I had with the presenter (please explain):

\begin{itemize}
\item He wasn't that comfortable doing what he was doing
We need to get more used to it
\item He/she wants to finish all the modules
\item I was the presenter.
\item Presenter could have spoke a little louder and clearer.

\item NONE
\item He was great!
\item presenter did a good job of presenting all source nodes
\item none.
\item Not at all
\item Not really.
\item Kind of slow
\item The presenter did not clearly explain the code due to illness
\end{itemize}

\item Problems that I had with the moderator (please explain):
\begin{itemize}
\item Our group did everythin perfectly so we did not need a
moderator

\item None.
\item none.

\item NONE
\item None
\item none

\item none.
\item No problems.
\item Was very helpful especially in time of trouble and/or when certain
things were a bit unclear.
\item No problem
\item No problmes
\end{itemize}


\item Problems that I had with EGSM system (please explain): 
\begin{itemize}
\item None

\item I felt that certain areas of this review system were unstable.
\item none
\item The mouse is not responsive.  It is hard to point and drag
\item there should be an easy way of tracing or searching variables,
functions, etc.
sometimes it is hard to find stuffs
maybe adding a watch window would help
or if I can open two or more source windows
\item None.  I was just not as familiar with sparcstations as I am with PC's.
\item none
\item none but I may have had if I was the presenter.

\item JUST GETTTING USED TO THE COMMANDS
\item same as previous questionnaire for the individual review.

\item In my individual review, I did not like it, as I had forgotten how it
worked, and I had not reviewed for it.  It was much much more useful
this time around.
\item I didn't have any
\item I wish we can bring our probgram along
\item No problems
\end{itemize}


\item Problems that I had with EGSM process (please explain): 
\begin{itemize}
\item NOne
\item none
\item None.
\item well, I wasn't quit sure if I was presenting the material OK or not
\item none.

\item WORKING TOGETHER AS A GROUP WITHOUT MUCH EXPERIENCE OR TIME TO GET TO
KNOW ALL OF THE MEMBERS WELL.

\item same as previous answer.

\item none ( now )
\item if one of the participants does not know much, he/she can slow down
the entire process.  However he/she would learn a lot from the session
because of interaction with members more knowledgeable in the subject.

\item No.
\item I wish we could do the group review first before the individule review.
\end{itemize}

\item Suggestions on how to improve EGSM system (please explain):
\begin{itemize}
\item None
\item please see \#46
\item More emphasis on procedures
\item none.
\item THE SYSTEM WAS VERY INTERESTING TO WORK WITH.  THE COMMANDS WEREN'T
THAT HARD.
\item same answer as previous questionnaire for the individual review.

\item I don't really have any.
\item If we can bring in our program, we can do more corrections
\item None, well for now, everything worked out fine
so maybe if I worked more on this, I would `probably find
something to improve.
\end{itemize}


\item Suggestions on how to improve EGSM process (please explain):
\begin{itemize}
\item None
\item Explainations for the code to review should be done ahead of time, and
better descriptions should be given.
\item allow user to write something to another user

\item Would be interesting to see how this system would hold out when placed
over a long distance network with maybe a box for communication chat
instead of the audio from the system being conducted person to person.
Maybe test it by having people in different rooms of Keller form a group.
\item none.

\item NONE
\item same as previous answer.


\item Sometimes, to achieve optimum performance, it is essential that all
participants are knowledgeable about the subject at hand
\item It is really cool now, can't think of any suggestions.
\item I cannot think any at this moment.
\end{itemize}

\item Other comments:
\begin{itemize}
\item NOne
\item In general I felt this system was ok.  It allowed for group
involvement, certain explainations for code and evaluation were
unclear at times.  I felt you needed a better evaluation of code.
\item Good idea for FRT!!
\item The group had trouble with (!strncmp. . . .) condition that occurred
frequently in the code.  We were not exactly sure what was meant, if
the strings are supposed to be the same or not the same.
\item Overall I felt that this was an excellent way to conduct reviews on
source code.

\item code is too long.  I feel sleepy after a while.  Productivity was
greatly reduce as time goes on.
\item THE GROUP PROCESS WENT VERY WELL.  WE DIDN'T RUN INTO MANY PROBLEMS.
OUR PRESENTER HAS USED THIS SYSTEM BEFORE SO WE NOT HAVE THAT MUCH
DIFFICULTY IN USING THE SYSTEM.  THE MODERATOR WAS VERY HELPFUL WHEN
WE NEEDED HELP.  THE AMOUNT OF FUNCTIONS WAS QUITE ALOT.  IT WAS
TIRING AFTER A WHILE BUT NEEDED GREAR ATTENTION FROM ALL MEMBERS.
\item none

\item I would like to see results of our review.  Dont' know if
we picked out all errors in the code.  In other words, feedback
would be nice.
\item I cannot think any more comment at this moment.  I just want to go
home to study.
\item I feel this EGSM is great.
But this would have been more productive if we had
more trainning and better understaning about the code.

\end{itemize}

\end{enumerate}


=======================EIAM=========================

\chapter {Questionnaire For Individuals (EIAM) --ICS411}

\begin{enumerate}

\item  My understanding of the source code before the review was: 
\\
Very low \hfill 1 \dotfill  2 \dotfill 3 \dotfill 4 \dotfill 5 \hfill Very high\\
(Score: 3.1)

\item  My understanding of the source code after the review was: 
\\
Very low \hfill 1 \dotfill  2 \dotfill 3 \dotfill 4 \dotfill 5 \hfill
Very high\\
(Score: 3.6)

\item  My understanding of C programming language was improved after
this review.  
\\
Not at all true \hfill 1 \dotfill  2 \dotfill 3 \dotfill 4 \dotfill 5 \hfill Very true\\
(Score: 3.1)

\item  In general, I found it easy to understand the logic of
the code. 
\\
Not at all true \hfill 1 \dotfill  2 \dotfill 3 \dotfill 4 \dotfill 5 \hfill Very true\\
(Score: 3.2)

\item  For this review, I would have preferred working alone rather 
than working in a group (if I have had a choice).
\\
Not at all true \hfill 1 \dotfill  2 \dotfill 3 \dotfill 4 \dotfill 5 \hfill Very true\\
(Score: 2.4)

\item  For this review, I felt more confidence about the issues I raised when 
working alone rather than working in a group.
\\
Not at all true \hfill 1 \dotfill  2 \dotfill 3 \dotfill 4 \dotfill 5 \hfill Very true\\
(Score: 3.0)

\item  The training session for individual review was sufficient.
\\
Not at all true \hfill 1 \dotfill  2 \dotfill 3 \dotfill 4 \dotfill 5 \hfill Very true\\
(Score: 4.0)

\item   There was sufficient time to work on this review.
\\
Not at all true \hfill 1 \dotfill  2 \dotfill 3 \dotfill 4 \dotfill 5 \hfill Very true\\
(Score: 4.1)

\item   I was motivated to do this review project.
\\
Not at all true \hfill 1 \dotfill  2 \dotfill 3 \dotfill 4 \dotfill 5 \hfill Very true\\
(Score: 3.6)

\item  My overall confidence in the quality of my review was
\\
Very low \hfill 1 \dotfill  2 \dotfill 3 \dotfill 4 \dotfill 5 \hfill Very high\\
(Score: 3.3)

\item  I felt comfortable doing this review.
\\
Not at all true \hfill 1 \dotfill  2 \dotfill 3 \dotfill 4 \dotfill 5 \hfill Very true\\
(Score: 3.5)

\item  It took me a good while to decide whether a particular program
segment contained valid issues.
\\
Not at all true \hfill 1 \dotfill  2 \dotfill 3 \dotfill 4 \dotfill 5 \hfill Very true\\
(Score: 3.7)

\item  I often wanted to change or delete issues I just created.
\\
Not at all true \hfill 1 \dotfill  2 \dotfill 3 \dotfill 4 \dotfill 5 \hfill Very true\\
(Score: 2.7)

\item  My overall satisfaction with individual review (EIAM) process as
outlined in the FTR guideline was:
\\
Very low \hfill 1 \dotfill  2 \dotfill 3 \dotfill 4 \dotfill 5 \hfill Very high\\
(Score: 3.8)

\item  I believe EIAM system made my review more productive (i.e., find
lots of issues) 
\\
Not at all true \hfill 1 \dotfill  2 \dotfill 3 \dotfill 4 \dotfill 5 \hfill Very true\\
(Score: 3.4)

\item  I believe EIAM system made my review more effective
(i.e., find lots of ``good'' issues in a relatively short time)
\\
Not at all true \hfill 1 \dotfill  2 \dotfill 3 \dotfill 4 \dotfill 5 \hfill Very true\\
(Score: 3.3)

\item  EIAM system is easy to use.
\\
Not at all true \hfill 1 \dotfill  2 \dotfill 3 \dotfill 4 \dotfill 5 \hfill Very true\\
(Score: 4.3)

\item  EIAM system is useful. 
\\
Not at all true \hfill 1 \dotfill  2 \dotfill 3 \dotfill 4 \dotfill 5 \hfill Very true\\
(Score: 3.8)

\item  My overall satisfaction with EIAM system was:
\\
Very low \hfill 1 \dotfill  2 \dotfill 3 \dotfill 4 \dotfill 5 \hfill Very high\\
(Score: 3.9)

\item  Problems that I had with EIAM system (please explain): 
\begin{itemize}
\item I wanted to search but I didn't know what key combinations to do.
Maybe a menu item for that would be nice.
\item none
\item Was that there were no manuals to consult in aiding the review
process.  It is often difficult to memorize an entire C/C++
reference manual, and without the luxury of knowing the full extent of
the C/C++ language, the review process was affected.  The entire
purpose of the review process, as I see it, was to raise certain
issues concerning the program.  This I feel was more or less
accomplished, though not very precisely.

When writing a program, programmers often refer to programming manuals
for language syntax and command usage.  During the review process,
this was a luxury that weren't available.  Without knowing the
language, the issues that arose may not be correct simply because the
reviewer did not understand what the command in the procedure was for
or if it was the proper syntax.  This is unfortunate and affects the
reliability of the review process.


\item Trying to remember which mouse button to press.
\item  not familiar to use at the first time, after several trials, it's
 wonderful!

\item Not familiar with openwin...
\item I never really had any problems with the EIAM system, since it was
pretty general, and also it was quite efficient to make me work
faster.

\item Looking at the screen for long periods of time.
\item It takes too long to do this experiment.  The functions are not in any
logical orders which makes it very diffcult to view.  Maybe we should
have a printout of all the codes went we doing this experiment. 
\item The problems with this  system was the environemnt
\item controlling the environment
pointing here and there

\item     I don't debug programs entirely from memory without doing test
compiles to find the errors.  I open my K \& R C book and compile the
code to catch all the obvious errors, such as spelling, forgotten
semi-colons, etc.  The "true" programming way may be to understand the
code 100% without compiling like they did in the olden days, but I
cannot adapt to that system.  I know that if you took away my C book
and compiler, I would NEVER be able to turn in any working programs longer
than "hello world."  
    As such, I think that this process could be much improved
if it could compile code ( just to find obvious spelling and structure
errors ).  Doesn't it make sense to use a spell checker on a term
paper?  Even the brightest English student won't find all typos in his
essays.  Why should computer users be held to higher standard?
\item I prefer to work in groups
\item I did not have many problems with EIAM.  I like the format and it does
make the review process much easier.
\item The scrolling bar for the review screen should be on the right not
left because the mouse is on the right.
\item a little confusing at the start knowing which button on the mouse to
use to do a certain function 
\item none
\item sometimes it gets stuck and had to start over
\item I was expecting to do a group review first and I have not prepared for
doing individual review psychologically.  It scared me almost freezed.
\item none
\item NOne
\end{itemize}

\item  Problems that I had with EIAM process (please explain): 
\begin{itemize}
\item my limited knowledge of 'C' was an impediment in the review process.
sometimes I was not sure what the function was doing
\item none
\item none
\item Read Above

\item none
\item 1. the confidence part is not that important at all. 
2. not enough on line help for the first time users.

\item 

\item Not really sure if it's any more efficient than pouring thru code
that's printed or jus' using a debugger...
\item There were no problems with the process
\item 	     .......................
I think that I summed it up in the last comment.  I've just been
spoiled by the online compilers in Borland 4.0 and Microsoft Visual
C++.  How can EIAM compete will compilers that contain online manuals?


\item None
\item The mouse is not very responsive.  The system is slow.
\item referring back to other source nodes was a little confusing at times
\item none
\item I understood the C++ codes that I turned in, but not the C codes that
you gave me to do.  I arrange the codes differently.  I bid I loose
lots of points of this review by programing in C++ instead of C 
\item none
\item It was opening some scratch file.
I did not know what it was.
\end{itemize}

\item  Suggestions on how to improve EIAM system (please explain):
\begin{itemize}
\item suggested in item 20.
\item Perhaps incorporate some sort of reference concerning the language
being used for the program.  After all the review process was to bring
up issues concerning the program's functionality.  Without knowing how
the language, in this case C/C++, works.  The review process has
become one which dealt with the syntax of the language rather than the
program it was used to to create.

\item Need the users practice using EIAM system, to get use to the buttons
and mouse controls.
\item run it on pc and make it more user friendly.
\item It doesn't really replace an active debugging program.
\item This is just minor, but rather than scrolling page by page, how about
line by line, meaning the top line of the page disappears, and the
bottom line incrents up by one, and a new line in the bottom
appears. This would be easier on my part in reviewing the code, rather
than having to flip page by page.
\item More days to debug
\item it is good
very easy to use
maybe let the user see more! or have a debugger 
\item I know that EIAM is NOT designed to compete directly against Borland
or Microsoft, but I can't see how it would improve my coding over
using such a compiler.  Personally, I know that I find a hell of a lot more
errors with gcc than with this, because gcc tells me where the errors are.
Yes, it is sloppy to do things that way, but if I didn't, I would
never be able to turn in any assignments.

As this is the age of hypermedia, I think that the system could be
improved if it offered hypertext links to help files.  For example,
clicking on the word "int" in the code being reviewed would take the
user to a help file on the "int" type.  This would be helpful for
people like me who don't know 100% of the C++ commands by heart.

\item maybe a hot key to automatically move hightlight to issue form.  Mouse
are a bit difficult to use.
\item include sample input and output for the functions and the program.
\item it is very good now.  I wish I could've spent more time using it for
all SE documents
\item I would prefer to have more windows for me to put more related
function on the screen at the same time.  Moreover, the room was full
of noise while I was taking the review.  The noising enviornment really distracted me
\item none
\item 1. Short demo before the actual thing.
2. Dynamic loading of the code for review
3. Better graphical user interface(Certainly better than emacs)

\end{itemize}


\item  Suggestions on how to improve EIAM process (please explain):
\begin{itemize}
\item online-help for 'C' functions do would be a great help.  The
reviewers would be able to avoid making issues which are not based on
clear understanding of the actual process going on in the program.
\item none

\item Read Above

\item 	     .......................
if there is an on line help, that's great!

\item Not really sure but I preferred the group process.
\item If we were allowed to take a short break.
\item No suggestions
\item I think I covered it in the last comment...
\item So far so good!
\item provide a set of procedures and suggestions on how to review more productively
\item ......................
I would prefer to do the group review first.  I quess I am pretty much
a chicken.  
\item none
\item It is fine as it is
\end{itemize}


\item  Other comments:
\begin{itemize}
\item none
\item The ics 411 project due date is very close to this CRCS which is very
good because I got to read my code in trying to prepare for this
review.  For future reviews, I think that we should start the review
after the project due date, but do the review as soon as possible.
\item 1.  the handout that Danu gave us is very detail and is a really good
    guideline for our first time users.
2.  the training session is good too. I gave me a rough idea of what
    this program is and how to use it.
3.  for the sample training program, if there is clues showing the
    users where the errors are, it will be more helpful for the users
	 to understand how to found out errors.
 
\item None.
\item This was really interesting!
\item The assignment should have been base on a different program, not the
one already done in class.
\item I don't mean to cut the system down out of spite.  It's just that I
cannot learn to adapt to this style of debugging.  I don't think that
it's safe to program entirely from memory, as I cannot remember 100%
of the possible C++ commands.  I need a manual, online help, and as
much assistance as possible.  
\item 
FIRST FAMILIAR WITH THE PROGRAM

\item The producer of the EIAM is very friendly and helpful!
\item I like how this system works, although i think it can be more
effective if there was a book of C programming language nearby
which would allow me to confirm my errors.
\item it would help to be prepared before coming in
\item it was well worth the time.  its a good system
\item The reviewing schedules are not flexible.  I bid you next time I will
do a better job.  Think you for reviewing my comments and have a
wonderful day!
\item this was an interesting reivew.  i needed to be dependant on only
myself and not anyone else.  this really made me concentrate harder and try
to understand the code a lot better.  i wish i was more prepared so i
would feel more confident in creating the issues.
\item None
\end{itemize}

\end{enumerate}

====================POST=====================

\chapter{Final Questionnaire (ICS411)}

\begin{enumerate}

\item  I would rather review source code manually using pencil and paper than
using CSRS (on-line Collaborative Software Review System).
\\
Not at all true \hfill 1 \dotfill  2 \dotfill 3 \dotfill 4 \dotfill 5 \hfill Very true\\
(Score: 1.8)

\item  In general, CSRS is useful.
\\
Not at all true \hfill 1 \dotfill  2 \dotfill 3 \dotfill 4 \dotfill 5 \hfill Very true\\
(Score: 4.2)

\item  Overall, I enjoyed using CSRS.
\\
Not at all true \hfill 1 \dotfill  2 \dotfill 3 \dotfill 4 \dotfill 5
\hfill Very true\\
(Score: 4.1)

\item I would:
   \begin{enumerate}
   \item[(1)]Strongly prefer using EGSM
   \item[(2)]Somewhat prefer using EGSM
   \item[(3)]Equally prefer using EGSM or EIAM
   \item[(4)]Somewhat prefer using EIAM
   \item[(5)]Strongly prefer using EIAM
   \end{enumerate}
(Score: 2.1)

\item I am:
   \begin{enumerate}
     \item[(1)] Much more productive using EGSM
     \item[(2)] Somewhat more productive using EGSM
     \item[(3)] Equally productive with EGSM or EIAM
     \item[(4)] Somewhat more productive using EIAM
     \item[(5)] Much more productive using EIAM
    \end{enumerate}   
(Score: 2.0)

\item  I prefer working:
  \begin{enumerate}
     \item[(1)] Alone 
     \item[(2)] Group
  \end{enumerate} 
(Score: 1.8)

\item  Reasons for preferring working in a group: 
\begin{itemize}
\item have more input as a group and we could talk.
Also, more heads are better than one

\item Comparison of my performance in each situation should speak for itself.
I don`t like programming alone.
\item Because there is no output, it is very hard to look for specific
errors.  I'd rather trace a program then review it's code any day.

\item It's more easy to find errors in group than individal.
\item If I have to work on code myself then I'd rather use a debugger or
simply look through the code via hardcopy but if working in a group I
find that the group discussion and consensus process allows me to find
more errors and be more critical because I can ask questions that
hopefully someone will have answers to.
\item I'm a group oriented type of person however at times I do like to work
on my own.

\item In group work even though you are not confident,
when you say and issue others may recognize and they will offer
support to the issue
\item In a group situation, there are things that others may see that I will
not see, and vice versa.  Also, the issues that a group member raises
may not be an issue and it can be determined in a group discussion.  I
think a group setting, with a group that works, is better and more
reliable than individual work.
\item collaborative review leads to higher quality and quality is job \#1
\item We can discuss the error before create a issue.  That's the good way
to eliminate unnecessary time to find errors.

\item The CSRS is a very useful system.  It would be a lot harder or
inefficient to review source code manually using pencil and paper. The
system was a neat way for reviewing code and I think improves the
amount and types of errors we find in a program.  Although it takes a
lot more effort and reliability on the individual effort, I would
prefer working in a group.  Groups are good because I get to see how
other people work out problems and working together, to me is more
efficient in finding errors. 
\item Working in group helped me to find more number of errors than the
individual review.

\item sometimes you get ideas from others that you never could thought of
you feel less stressed out


\item group is better than individual
because the presenter's explanation was very helpful to think about
the questions



\item I am a pepole person
\item Can discuss issues with other people and use them as a sounding board.

\item I think whether working in a group or as an individual depends upon
the assignment, group members, and the overall understanding of the
issues at hand.
\item Because if I did not know or understand some code then I could get
help and support from the other members of my group.
\item feel more confident about making decisions;

\item more chance of finding errors.
\item We can discuss and support by others with suggestionsIssue
\item it is much easier to 'bounce' ideas off of each other and see things
i never could see before. This makes it much more efficient to debug code.

\item It has brought our own individually ideas as to how things work and
how they should work.  The presentation of ideas allows us to
personally evaluate the issue, and see if it is true or not.
Sometimes a person may think of an issue, and another may not, and
vice versa.
\item I think this is a terrific system.  I do have a good fun in doing
the review.  I prefer to do the review in group instead of individual.
I wish you have a wonderful summer.
\item working in a group can stimulate more ideas and is far more effective
than doing alone. Also, if one gets an error, other group member can
point it out immediately to avoid unneccessary mistakes. moreover, I
found that working in a group can help solving some of the C syntax
questions too if one is not that familiar with certain functions in C.


\item It was not easy to look at all the codes and gets kind of boring after
an while.
\item Interaction for code is good !
\item Three eyes looking at one code will definitely
find errors, one way or the other.  Also we have
known each other for a long time and we worked
on other projects together.
\item my current limited knowledge of 'C' makes me less confident about the
issues I make.  I am sure I am missing out many issues.

\end{itemize}

\item  Reasons for preferring working alone:
\begin{itemize}
\item It's hard to work efficiently as a group.
I think all members should well familiar with each other, and
have good understanding about program.
\item Better concentration.
\item In a group, more time is spent for discussion and less time to
concentrate to find errors.  However, if we have a good group; it will
be better to do in the group because you can learn more from other findings.
\item Don't need to debate the issue.  It easier to work alone.
\item Group working always involves lots of disscutions and
debates. Sometimes, I feel so sick of those timeless arguments.
But, working in a group gives more productivity when working on a
large scale of project. That is why we software developers need an
efficient program like CSCR. Thanks!

\end{itemize}

\end{enumerate}

