%%% \documentstyle[11pt,/group/csdl/tex/definemargins,
%%%                        /group/csdl/tex/lmacros]{article} 
%%% 
%%%           \begin{document}
%%%           \begin{center}
%%%           {\large\bf CSRS Experiment Results}\\
%%%           \end{center}
%%%           \small 
%%% 	  
\chapter {CSRS Experiment Results (ICS411): Group11 (EIAM)}
\small

\begin{description}
\item [Method:] EIAM
\item [Group:] Group11
\item [Source:] Pass2
\item [Participants:] msamson (Reviewer), kiyuna (Reviewer), chasek (Reviewer)
\end{description}
\section{Issue Lists}
\begin{enumerate}
\item {\it Issue\#194 (msamson)}
\begin{description}
\item [Subject:] variable i uninitialized
\item [Criticality:] Hi
\item [Confidence-level:] Hi
\item [Source-node:] dectonum

\item [Lines:] 11

\item [Description:] i is not initialized before it is used
\end{description}
\item {\it Issue\#198 (kiyuna)}
\begin{description}
\item [Subject:] i is unknown
\item [Criticality:] Hi
\item [Confidence-level:] Hi
\item [Source-node:] dectonum

\item [Lines:] 11

\item [Description:] i is not initialized before it is used so the
'if' clause is checking something undefined.
\end{description}
\item {\it Issue\#204 (msamson)}
\begin{description}
\item [Subject:] for loop iterartion one to little
\item [Criticality:] Hi
\item [Confidence-level:] Hi
\item [Source-node:] Read\_Int\_File

\item [Lines:] 31

\item [Description:] operation is 6 bytes long but loop iteration
only accounts for 5 bytes for the operation code.
\end{description}
\item {\it Issue\#210 (kiyuna)}
\begin{description}
\item [Subject:] this line always executes
\item [Criticality:] Hi
\item [Confidence-level:] Hi
\item [Source-node:] Read\_Int\_File

\item [Lines:] 18

\item [Description:] this line is supposed to execute only is ch ==
'F' and not otherwise.
\end{description}
\item {\it Issue\#214 (kiyuna)}
\begin{description}
\item [Subject:] reading the newline character which is wrong.
\item [Criticality:] Hi
\item [Confidence-level:] Hi
\item [Source-node:] Read\_Int\_File

\item [Lines:] 29-30

\item [Description:] You are storing the newline character in the
first position of source-{\tt >}labl and from that everything after that will have
the wrong information such as (operation, operand, and comment).
\end{description}
\item {\it Issue\#218 (msamson)}
\begin{description}
\item [Subject:] strncmp doesn`t check if not equal to 0
\item [Criticality:] Hi
\item [Confidence-level:] Med
\item [Source-node:] P2\_Search\_Optab

\item [Lines:] 18

\item [Description:] strncmp does not if is the size of the
OPTAB[mid] is not equal to 0.
\end{description}
\item {\it Issue\#222 (msamson)}
\begin{description}
\item [Subject:] missing else
\item [Criticality:] Hi
\item [Confidence-level:] Hi
\item [Source-node:] Read\_Int\_File

\item [Lines:] 16-18

\item [Description:] if ch equals 'T' it sets source-{\tt >}comline to
true. then on the next line sets source-{\tt >}comline to false. an else command is
missing so it doesn't set source-{\tt >}comline to false no matter what ch is.
\end{description}
\item {\it Issue\#226 (kiyuna)}
\begin{description}
\item [Subject:] comparing error
\item [Criticality:] Med
\item [Confidence-level:] Hi
\item [Source-node:] P2\_Search\_Optab

\item [Lines:] 12-13

\item [Description:] since high and low include the 'mid' mnemonic
in the statements following this line, this line should check if the
'mnemonic {\tt <}= OPTAB[mid].mnemonic' not only '{\tt <}'
\end{description}
\item {\it Issue\#232 (kiyuna)}
\begin{description}
\item [Subject:] checking for wrong comparison
\item [Criticality:] Hi
\item [Confidence-level:] Hi
\item [Source-node:] P2\_Search\_Optab

\item [Lines:] 16

\item [Description:] this part should check if 'mnemonic !=
OPTAB[mid].mnemonic.  This loop will end prematurely.
\end{description}
\item {\it Issue\#236 (kiyuna)}
\begin{description}
\item [Subject:] should return if error is found.
\item [Criticality:] Hi
\item [Confidence-level:] Hi
\item [Source-node:] P2\_Proc\_START

\item [Lines:] 9-13

\item [Description:] Even if the errorflags were set the procedure
would still return the some kind objcode in structure objct and rectype will
still be set to HEADREC.
\end{description}
\item {\it Issue\#240 (msamson)}
\begin{description}
\item [Subject:] no return statement
\item [Criticality:] Hi
\item [Confidence-level:] Hi
\item [Source-node:] P2\_Proc\_START

\item [Lines:] 19

\item [Description:] procedure does not return anyting to where it
was called.  procedure processes the start statement but does not return the
object code.
\end{description}
\item {\it Issue\#244 (kiyuna)}
\begin{description}
\item [Subject:] 'i' is not set correctly
\item [Criticality:] Hi
\item [Confidence-level:] Hi
\item [Source-node:] P2\_Proc\_BYTE

\item [Lines:] 35-36

\item [Description:] 'i' is not set as in the first part when it
checks for 'C' so this will produce an unpredictable answer.
\end{description}
\item {\it Issue\#248 (msamson)}
\begin{description}
\item [Subject:] checking the second space in operand first
\item [Criticality:] Hi
\item [Confidence-level:] Hi
\item [Source-node:] P2\_Proc\_BYTE

\item [Lines:] 20-21

\item [Description:] i is first set to one.  then the while loop
condition checks if the second position in the operand is a quote. the second
position will never be a quote.  it should check if the first position is a
quote.
\end{description}
\item {\it Issue\#252 (kiyuna)}
\begin{description}
\item [Subject:] should return out of the procedure.
\item [Criticality:] Hi
\item [Confidence-level:] Hi
\item [Source-node:] P2\_Assemble\_Inst

\item [Lines:] 10-14

\item [Description:] A return statement should be the last
statement in the 'if' statement to exit the procedure before any other codes
are processed.
\end{description}
\item {\it Issue\#254 (msamson)}
\begin{description}
\item [Subject:] strncmp does not check if == 0.
\item [Criticality:] Hi
\item [Confidence-level:] Hi
\item [Source-node:] P2\_Assemble\_Inst

\item [Lines:] 15-25

\item [Description:] strncmp not checking if == 0.
\end{description}
\item {\it Issue\#260 (kiyuna)}
\begin{description}
\item [Subject:] should be checking if strncmp is == 0
\item [Criticality:] Hi
\item [Confidence-level:] Hi
\item [Source-node:] P2\_Assemble\_Inst

\item [Lines:] 15-26

\item [Description:] All of these if and else if statements check
if the first string is not equal to the second string when it should be
checking if the first string is equal to the second string.
\end{description}
\item {\it Issue\#266 (msamson)}
\begin{description}
\item [Subject:] endfound is not initialized
\item [Criticality:] Hi
\item [Confidence-level:] Hi
\item [Source-node:] P2\_Assemble\_Inst

\item [Lines:] 10-12

\item [Description:] variable endfound is not initialize as true or
false.
\end{description}
\item {\it Issue\#270 (kiyuna)}
\begin{description}
\item [Subject:] should be checking is strncmp == 0
\item [Criticality:] Hi
\item [Confidence-level:] Hi
\item [Source-node:] P2\_Assemble\_Inst

\item [Lines:] 31-33

\item [Description:] This is the same as the previous issue.  It
should be checking if strncmp is equal to 0 not 'not' equal to 0.
\end{description}
\item {\it Issue\#274 (msamson)}
\begin{description}
\item [Subject:] strncmp does not check if ==0.
\item [Criticality:] Hi
\item [Confidence-level:] Hi
\item [Source-node:] P2\_Assemble\_Inst

\item [Lines:] 31

\item [Description:] strncmp not checking if == 0.
\end{description}
\item {\it Issue\#278 (kiyuna)}
\begin{description}
\item [Subject:] Should return if TEXTLENGTH == 0
\item [Criticality:] Hi
\item [Confidence-level:] Med
\item [Source-node:] P2\_Write\_Obj

\item [Lines:] 28-32

\item [Description:] If i'm not mistaken this was supposed to exit
the procedure if the TEXTLENGTH was found to be 0 otherwise it will continue
to try to write to the objfile nothing or what ever was stored in temp.
\end{description}
\item {\it Issue\#282 (kiyuna)}
\begin{description}
\item [Subject:] temp was not initialized.
\item [Criticality:] Hi
\item [Confidence-level:] Hi
\item [Source-node:] P2\_Write\_Obj

\item [Lines:] 14-18

\item [Description:] temp was not initialized so when a call to
numtohex is made it will return some weird answer or unknown answer.
\end{description}
\item {\it Issue\#290 (msamson)}
\begin{description}
\item [Subject:] 
\item [Criticality:] 
\item [Confidence-level:] 
\item [Source-node:] 

\item [Lines:] 

\item [Description:] 
\end{description}
\item {\it Issue\#292 (msamson)}
\begin{description}
\item [Subject:] no retrun statement
\item [Criticality:] Hi
\item [Confidence-level:] Hi
\item [Source-node:] P2\_Write\_Obj

\item [Lines:] 28-31

\item [Description:] if TEXTLENGTH == 0 then it should return.
\end{description}
\item {\it Issue\#296 (kiyuna)}
\begin{description}
\item [Subject:] Not writing the proglength to objfile.
\item [Criticality:] Hi
\item [Confidence-level:] Hi
\item [Source-node:] P2\_Write\_Obj

\item [Lines:] 

\item [Description:] The Header part is not writing the program
length 'proglength' to the objfile.
\end{description}
\item {\it Issue\#300 (chasek)}
\begin{description}
\item [Subject:] i used without being initialized
\item [Criticality:] Hi
\item [Confidence-level:] Hi
\item [Source-node:] dectonum

\item [Lines:] 11-12

\item [Description:] i not initialized but used in a computation
\end{description}
\item {\it Issue\#304 (chasek)}
\begin{description}
\item [Subject:] Error found but doesn't exit loop
\item [Criticality:] Hi
\item [Confidence-level:] Hi
\item [Source-node:] dectonum

\item [Lines:] 16-17

\item [Description:] Even though an error is found, the loop
continues to execute.
\end{description}
\item {\it Issue\#308 (chasek)}
\begin{description}
\item [Subject:] Loop missing 4 digit condition
\item [Criticality:] Hi
\item [Confidence-level:] Hi
\item [Source-node:] dectonum

\item [Lines:] 10

\item [Description:] Although the maximum value to be scanned is 4
digits, there is no such condition to insure this rule in the loop.
\end{description}
\item {\it Issue\#312 (chasek)}
\begin{description}
\item [Subject:] Missing Else statement
\item [Criticality:] Med
\item [Confidence-level:] Hi
\item [Source-node:] Read\_Int\_File

\item [Lines:] 18-19

\item [Description:] Without the Else clause, this line will always
be exected and thereby cancel out the previous if clause.  Source-{\tt >}comline
will always be false.
\end{description}
\item {\it Issue\#316 (chasek)}
\begin{description}
\item [Subject:] condition should only be looking for zero
\item [Criticality:] Med
\item [Confidence-level:] Hi
\item [Source-node:] P2\_Search\_Optab

\item [Lines:] 18-19

\item [Description:] Strncmp returns zero for an exact match, which
is what we want but not what the program will currently accept.
\end{description}
\item {\it Issue\#320 (chasek)}
\begin{description}
\item [Subject:] Missing else statement
\item [Criticality:] Med
\item [Confidence-level:] Hi
\item [Source-node:] P2\_Proc\_START

\item [Lines:] 13-19

\item [Description:] If it is not the first statement it will still
generate a HEADREC due to a missing else statement.
\end{description}
\item {\it Issue\#324 (chasek)}
\begin{description}
\item [Subject:] i not initialized
\item [Criticality:] Hi
\item [Confidence-level:] Hi
\item [Source-node:] P2\_Proc\_BYTE

\item [Lines:] 35-36

\item [Description:] i is used in a calculation but is not initialized
\end{description}
\item {\it Issue\#328 (chasek)}
\begin{description}
\item [Subject:] should only accept zero from strncmp
\item [Criticality:] Hi
\item [Confidence-level:] Hi
\item [Source-node:] P2\_Assemble\_Inst

\item [Lines:] 15-38

\item [Description:] should only accept a perfect match from the
strncmp, which involves only accpeting a zero
\end{description}
\item {\it Issue\#332 (chasek)}
\begin{description}
\item [Subject:] Missing else statement
\item [Criticality:] Med
\item [Confidence-level:] Hi
\item [Source-node:] P2\_Assemble\_Inst

\item [Lines:] 10-14

\item [Description:] After an error is found (statement after the
end) the program still tries to produce object code.
\end{description}
\end{enumerate}
\section{Review Metrics}
\begin{table}[hb]
\begin{center}
\begin{tabular}{|l|l|l|l|l|}
\hline
Participant & Start-time & End-time & Elapsed-time & Busy-time \\
\hline
chasek & May 09, 1995 14:12:17 & May 09, 1995 14:58:29 & 0:46:12 & 0:46:12 \\
kiyuna & May 05, 1995 11:38:20 & May 05, 1995 13:26:13 & 1:47:53 & 1:40:29 \\
msamson & May 05, 1995 11:38:27 & May 05, 1995 13:18:45 & 1:40:18 & 1:40:18 \\
\hline
 & & Total & 4:14:23 & \\
\hline
\end{tabular}
\end{center}
\caption{Review Session}
\end{table}


\begin{table}[hb]
\begin{center}
\begin{tabular}{|l|l|l|l|}
\hline
Source & chasek & kiyuna & msamson\\
\hline
(172)Type and var declarations & 79 & 163 & 69\\
(174)dectonum & 405 & 716 & 365\\
(176)Read\_Int\_File & 286 & 939 & 1250\\
(178)P2\_Search\_Optab & 445 & 751 & 805\\
(180)P2\_Proc\_START & 276 & 340 & 696\\
(182)P2\_Proc\_BYTE & 422 & 665 & 553\\
(184)P2\_Assemble\_Inst & 309 & 680 & 647\\
(186)P2\_Write\_Obj & 395 & 1475 & 1130\\
(188)Pass\_2 & 131 & 279 & 483\\
\hline
\end{tabular}
\end{center}
\caption{Review Time}
\end{table}

\begin{table}[hb]
\begin{center}
\begin{tabular}{|l|l|l|l|}
\hline
Source & chasek & kiyuna & msamson\\
\hline
dectonum & 186.7 & 105.6 & 207.1\\
Read\_Int\_File & 742.7 & 226.2 & 169.9\\
P2\_Search\_Optab & 226.5 & 134.2 & 125.2\\
P2\_Proc\_START & 247.8 & 201.2 & 98.3\\
P2\_Proc\_BYTE & 426.5 & 270.7 & 325.5\\
P2\_Assemble\_Inst & 489.3 & 222.4 & 233.7\\
P2\_Write\_Obj & 656.2 & 175.7 & 229.4\\
Pass\_2 & 1044.3 & 490.3 & 283.2\\
\hline
\end{tabular}
\end{center}
\caption{Paraphrasing Rate (lines/hour)}
\end{table}


\begin{table}[hb]
\begin{center}
\begin{tabular}{|l|l|l|l|l|}
\hline
Source & chasek & kiyuna & msamson & OK\\
\hline
Type and var.. &  &  &  & \\
dectonum & 300,304,308 (=3) & 198 (=1) & 194 (=1) & 300=198=194\\
         &  & &  & 304,308 \\
Read\_Int\_Fil.. & 312 (=1) & 210,214 (=2) & 204,222 (=2) & 312=210=222\\
         &  & &  & 214,204 \\
P2\_Search\_Op.. & 316 (=1) & 226,232 (=2) & 218 (=1) & 316=218\\
P2\_Proc\_STAR.. & 320 (=1) & 236 (=1) & 240 (=1) & \\
P2\_Proc\_BYTE & 324 (=1) & 244 (=1) & 248 (=1) & 324=244\\
P2\_Assemble\_.. & 328,332 (=2) & 252,260,270 (=3) & 254,266,274 (=3) & \\
P2\_Write\_Obj &  & 278,282,296 (=3) & 292 (=1) & 296\\
Pass\_2 &  &  &  & \\
\hline
\end{tabular}
\caption{Source node v.s Issue node}
\end{center}
\end{table}

%%\end{document}
