%%% \documentstyle[11pt,/group/csdl/tex/definemargins,
%%%                        /group/csdl/tex/lmacros]{article} 
%%% 
%%%           \begin{document}
%%%           \begin{center}
%%%           {\large\bf CSRS Experiment Results}\\
%%%           \end{center}
%%%           \small 
\chapter {CSRS Experiment Results (ICS411): Group10 (EIAM)}
\small
	  

\begin{description}
\item [Method:] EIAM
\item [Group:] Group10
\item [Source:] Pass2
\item [Participants:] cokumoto (Reviewer), briang (Reviewer), casem (Reviewer)
\end{description}
\section{Issue Lists}
\begin{enumerate}
\item {\it Issue\#196 (cokumoto)}
\begin{description}
\item [Subject:] Each condition should be contained in parenthesis
\item [Criticality:] Med
\item [Confidence-level:] Med
\item [Source-node:] dectonum

\item [Lines:] 11

\item [Description:] 
\end{description}
\item {\it Issue\#200 (briang)}
\begin{description}
\item [Subject:] variable i not initialized.
\item [Criticality:] Med
\item [Confidence-level:] Hi
\item [Source-node:] dectonum

\item [Lines:] 11

\item [Description:] the variable i has not been initialized so if
on the first case i is not zero str[first + i ] will not point to where it
should such as str[first] because i must be zero
\end{description}
\item {\it Issue\#206 (cokumoto)}
\begin{description}
\item [Subject:] do/while is nonstandard
\item [Criticality:] Low
\item [Confidence-level:] Med
\item [Source-node:] P2\_Search\_Optab

\item [Lines:] 10

\item [Description:] 
\end{description}
\item {\it Issue\#214 (cokumoto)}
\begin{description}
\item [Subject:] ors and ands testing a condition without ()s makes
it difficult to debug
\item [Criticality:] Low
\item [Confidence-level:] Med
\item [Source-node:] P2\_Proc\_BYTE

\item [Lines:] 37

\item [Description:] 
\end{description}
\item {\it Issue\#216 (briang)}
\begin{description}
\item [Subject:] Loop has only one exit condition
\item [Criticality:] Med
\item [Confidence-level:] Hi
\item [Source-node:] dectonum

\item [Lines:] 10

\item [Description:] This While loop should continue until
a scanning error is found or until i {\tt >} 3. the function should only read in 4
decimal digits.

should be while(scanning \&\& i {\tt <} 4)
\end{description}
\item {\it Issue\#222 (casem)}
\begin{description}
\item [Subject:] unfamiliar to ungetc
\item [Criticality:] Low
\item [Confidence-level:] Med
\item [Source-node:] Read\_Int\_File

\item [Lines:] 12

\item [Description:] I don't know what is ungetc means.  Shouldn`t it be a getc?
\end{description}
\item {\it Issue\#226 (briang)}
\begin{description}
\item [Subject:] else condition needs another statement
\item [Criticality:] Med
\item [Confidence-level:] Med
\item [Source-node:] dectonum

\item [Lines:] 15-16

\item [Description:] since an error has been found scanning needs
to be set to false. This way the loop will terminate properly.
\end{description}
\item {\it Issue\#234 (briang)}
\begin{description}
\item [Subject:] Needs else!
\item [Criticality:] Med
\item [Confidence-level:] Hi
\item [Source-node:] Read\_Int\_File

\item [Lines:] 18

\item [Description:] If ch = 'T' then ths comline of SOURCETYPE
will be set to true as it should be, but the since there is no else the
comline param will be set to false again.
\end{description}
\item {\it Issue\#238 (casem)}
\begin{description}
\item [Subject:] syntax error
\item [Criticality:] Hi
\item [Confidence-level:] Hi
\item [Source-node:] P2\_Search\_Optab

\item [Lines:] 16-17

\item [Description:] In a while statement the syntax is while(true)\{do something \}.  In a while
loop, the argument is evaluated first, then the action is taken place.
\end{description}
\item {\it Issue\#242 (briang)}
\begin{description}
\item [Subject:] Loop has improper length
\item [Criticality:] Med
\item [Confidence-level:] Hi
\item [Source-node:] Read\_Int\_File

\item [Lines:] 31

\item [Description:] sice the operation is defined as a char6 the
for loop needs to be for(i=0;i{\tt <}=5;i++)
\end{description}
\item {\it Issue\#246 (briang)}
\begin{description}
\item [Subject:] Need to check OPTAB[mid].mnemonic before loop
\item [Criticality:] Med
\item [Confidence-level:] Hi
\item [Source-node:] P2\_Search\_Optab

\item [Lines:] 10-17

\item [Description:] The mid position is not checked for
equivalence on the first try so if OPTAB[mid].mnemonic is the optab then it
wont be found the loop sould be a while loop not a do while.
\end{description}
\item {\it Issue\#254 (casem)}
\begin{description}
\item [Subject:] temp assigned incorrect value
\item [Criticality:] Med
\item [Confidence-level:] Med
\item [Source-node:] P2\_Write\_Obj

\item [Lines:] 16

\item [Description:] in the header record of the object file, it contains, from what i remember
the H, then the starting position then the size.  the second attempt to
calling numtohex may give an incorrect value for the size of the object file.
\end{description}
\item {\it Issue\#258 (briang)}
\begin{description}
\item [Subject:] Needs an else statement block
\item [Criticality:] Med
\item [Confidence-level:] Hi
\item [Source-node:] P2\_Proc\_START

\item [Lines:] 12

\item [Description:] If it is the first statement the there is no
problem, but if it is not the first statment the the error flag will be set
and will not exit the function so the the rest of the function should be in
an else statement.
\end{description}
\item {\it Issue\#262 (briang)}
\begin{description}
\item [Subject:] Wrong length for loop
\item [Criticality:] Med
\item [Confidence-level:] Hi
\item [Source-node:] P2\_Proc\_START

\item [Lines:] 15-16

\item [Description:] the source.labl field is defined as a CHAR8
therefore the for loop should be for(i=0;i{\tt <}=7; i++)
\end{description}
\item {\it Issue\#266 (casem)}
\begin{description}
\item [Subject:] Error in calling p2writeobj
\item [Criticality:] Hi
\item [Confidence-level:] Low
\item [Source-node:] Pass\_2

\item [Lines:] 25

\item [Description:] in the specification, it says that p2writeobj is called
 only if genoobject is true.  In that conditional statement, it looks if
!errorsfound is true then call p2writeobj, which was not specified in the
specifications.
\end{description}
\item {\it Issue\#270 (briang)}
\begin{description}
\item [Subject:] Reset the start for this loop
\item [Criticality:] Med
\item [Confidence-level:] Hi
\item [Source-node:] P2\_Proc\_START

\item [Lines:] 17

\item [Description:] Since the issue above changes the length of
the loop0 this loop will also change, I.e for (i=8; i{\tt <}=14; i++)
\end{description}
\item {\it Issue\#274 (cokumoto)}
\begin{description}
\item [Subject:] There is no function P2\_WRITE\_LIST.  It is in the
comments, so it may have been intentionally left out.
\item [Criticality:] Low
\item [Confidence-level:] Med
\item [Source-node:] Pass\_2

\item [Lines:] 27

\item [Description:] 
\end{description}
\item {\it Issue\#278 (cokumoto)}
\begin{description}
\item [Subject:] the integer n cannot store the value of '0'
\item [Criticality:] Hi
\item [Confidence-level:] Med
\item [Source-node:] dectonum

\item [Lines:] 12

\item [Description:] 
\end{description}
\item {\it Issue\#280 (casem)}
\begin{description}
\item [Subject:] not in the correct loop
\item [Criticality:] Med
\item [Confidence-level:] Low
\item [Source-node:] Pass\_2

\item [Lines:] 29-37

\item [Description:] These statements compares if the files are not null, then closes it, then
assign it a Null value.
\end{description}
\item {\it Issue\#286 (casem)}
\begin{description}
\item [Subject:] error in passing parameters
\item [Criticality:] Hi
\item [Confidence-level:] Hi
\item [Source-node:] Pass\_2

\item [Lines:] 18

\item [Description:] In calling Read\_Int\_File, you must pass three arguments, all of which should
pass in an address of the argument.
\end{description}
\item {\it Issue\#290 (briang)}
\begin{description}
\item [Subject:] i has not been initialized!
\item [Criticality:] Med
\item [Confidence-level:] Hi
\item [Source-node:] P2\_Proc\_BYTE

\item [Lines:] 35

\item [Description:] i need to be initialized to one at the start
of the loop so as to skip over the first quote.
\end{description}
\item {\it Issue\#294 (casem)}
\begin{description}
\item [Subject:] Error in passing parameters
\item [Criticality:] Hi
\item [Confidence-level:] Hi
\item [Source-node:] Pass\_2

\item [Lines:] 23

\item [Description:] An address of the parameter should be passed in since the procedure takes in
a pointer.
\end{description}
\item {\it Issue\#298 (casem)}
\begin{description}
\item [Subject:] maybe incorrect values
\item [Criticality:] Low
\item [Confidence-level:] Hi
\item [Source-node:] Type and var declarations

\item [Lines:] 23-25

\item [Description:] Usually, Success or True is equal to 1.  And False or Failure = 0.  This is a
minor error, but the programmer may get confused in other procedures when
using these two arguments.
\end{description}
\item {\it Issue\#302 (casem)}
\begin{description}
\item [Subject:] no initialization of i
\item [Criticality:] Hi
\item [Confidence-level:] Hi
\item [Source-node:] dectonum

\item [Lines:] 11

\item [Description:] The int i is not initialized to any value, so when the if conditional takes
in the i value, it will be of an unpredictable number.  That goes for the
same for other values that need the integer i.  Also when i++ is evaluated,
it will be of an unknown number +1.
\end{description}
\item {\it Issue\#304 (briang)}
\begin{description}
\item [Subject:] Needs a return if ENDFOUND
\item [Criticality:] Med
\item [Confidence-level:] Low
\item [Source-node:] P2\_Assemble\_Inst

\item [Lines:] 9-13

\item [Description:] If the ENDFOUND flag is set the function shoul return?
\end{description}
\item {\it Issue\#310 (casem)}
\begin{description}
\item [Subject:] should follow an else statment
\item [Criticality:] Hi
\item [Confidence-level:] Hi
\item [Source-node:] Read\_Int\_File

\item [Lines:] 18

\item [Description:] This line will be evaluated no matter if ch is = to T.  Meaning that
source-{\tt >}comline will be evaluated to false all the time.
\end{description}
\item {\it Issue\#314 (casem)}
\begin{description}
\item [Subject:] never be evaluated
\item [Criticality:] Hi
\item [Confidence-level:] Hi
\item [Source-node:] Read\_Int\_File

\item [Lines:] 19-28

\item [Description:] This if body will never be evaluated because source-{\tt >}comline will always be
false into entering this if conditional.
\end{description}
\item {\it Issue\#318 (briang)}
\begin{description}
\item [Subject:] Need a if statement
\item [Criticality:] Med
\item [Confidence-level:] Low
\item [Source-node:] P2\_Write\_Obj

\item [Lines:] 68-71

\item [Description:] There needs to be an if statement that checks
for the ENDREC enum
\end{description}
\end{enumerate}
\section{Review Metrics}
\begin{table}[hb]
\begin{center}
\begin{tabular}{|l|l|l|l|l|}
\hline
Participant & Start-time & End-time & Elapsed-time & Busy-time \\
\hline
casem & May 02, 1995 11:24:33 & May 02, 1995 13:20:10 & 1:55:37 & 1:55:37 \\
briang & May 02, 1995 11:24:14 & May 02, 1995 13:20:04 & 1:55:50 & 1:49:21 \\
cokumoto & May 02, 1995 11:21:53 & May 02, 1995 12:51:58 & 1:30:5 & 1:30:5 \\
\hline
 & & Total & 5:21:32 & \\
\hline
\end{tabular}
\end{center}
\caption{Review Session}
\end{table}


\begin{table}[hb]
\begin{center}
\begin{tabular}{|l|l|l|l|}
\hline
Source & casem & briang & cokumoto\\
\hline
(172)Type and var declarations & 2803 & 693 & 346\\
(174)dectonum & 550 & 867 & 530\\
(176)Read\_Int\_File & 2656 & 1165 & 769\\
(178)P2\_Search\_Optab & 463 & 653 & 444\\
(180)P2\_Proc\_START & 420 & 1041 & 334\\
(182)P2\_Proc\_BYTE & 380 & 778 & 621\\
(184)P2\_Assemble\_Inst & 947 & 557 & 624\\
(186)P2\_Write\_Obj & 895 & 633 & 653\\
(188)Pass\_2 & 1758 & 148 & 1051\\
\hline
\end{tabular}
\end{center}
\caption{Review Time}
\end{table}

\begin{table}[hb]
\begin{center}
\begin{tabular}{|l|l|l|l|}
\hline
Source & casem & briang & cokumoto\\
\hline
dectonum & 137.5 & 87.2 & 142.6\\
Read\_Int\_File & 80.0 & 182.3 & 276.2\\
P2\_Search\_Optab & 217.7 & 154.4 & 227.0\\
P2\_Proc\_START & 162.9 & 65.7 & 204.8\\
P2\_Proc\_BYTE & 473.7 & 231.4 & 289.9\\
P2\_Assemble\_Inst & 159.7 & 271.5 & 242.3\\
P2\_Write\_Obj & 289.6 & 409.5 & 396.9\\
Pass\_2 & 77.8 & 924.3 & 130.2\\
\hline
\end{tabular}
\end{center}
\caption{Paraphrasing Rate (lines/hour)}
\end{table}


\begin{table}[hb]
\begin{center}
\begin{tabular}{|l|l|l|l|l|}
\hline
Source & casem & briang & cokumoto & OK\\
\hline
Type and var.. & 298 (=1) &  &  & \\
dectonum & 302 (=1) & 200,216,226 (=3) & 196,278 (=2) & 302=200,216,226\\
Read\_Int\_Fil.. & 222,310,314 (=3) & 234,242 (=2) &  & 310=234,242\\
P2\_Search\_Op.. & 238 (=1) & 246 (=1) & 206 (=1) & \\
P2\_Proc\_STAR.. &  & 258,262,270 (=3) &  & \\
P2\_Proc\_BYTE &  & 290 (=1) & 214 (=1) & 290\\
P2\_Assemble\_.. &  & 304 (=1) &  & \\
P2\_Write\_Obj & 254 (=1) & 318 (=1) &  & 254\\
Pass\_2 & 266,280,286,294 (=4) &  & 274 (=1) & \\
\hline
\end{tabular}
\caption{Source node v.s Issue node}
\end{center}
\end{table}

%%\end{document}
