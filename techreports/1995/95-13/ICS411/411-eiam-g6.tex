%%% \documentstyle[11pt,/group/csdl/tex/definemargins,
%%%                        /group/csdl/tex/lmacros]{article} 
%%% 
%%%           \begin{document}
%%%           \begin{center}
%%%           {\large\bf CSRS Experiment Results}\\
%%%           \end{center}
%%%           \small 
\chapter {CSRS Experiment Results (ICS411): Group6 (EIAM)}
\small
	  

\begin{description}
\item [Method:] EIAM
\item [Group:] Group6
\item [Source:] Pass1
\item [Participants:] rshen (Reviewer), giang (Reviewer), hhwang (Reviewer)
\end{description}
\section{Issue Lists}
\begin{enumerate}
\item {\it Issue\#198 (giang)}
\begin{description}
\item [Subject:] Unsigned for Char type
\item [Criticality:] Med
\item [Confidence-level:] Med
\item [Source-node:] Type and var declarations

\item [Lines:] 34

\item [Description:] unsigned use for char type
\end{description}
\item {\it Issue\#204 (giang)}
\begin{description}
\item [Subject:] Static variable
\item [Criticality:] Med
\item [Confidence-level:] Med
\item [Source-node:] Type and var declarations

\item [Lines:] 69-72

\item [Description:] This can be a problem since source files are
compiled separately; other source files can't not access these variables if
they are declared as static
\end{description}
\item {\it Issue\#208 (hhwang)}
\begin{description}
\item [Subject:] open INTFILE to write
\item [Criticality:] Hi
\item [Confidence-level:] Hi
\item [Source-node:] Pass\_1

\item [Lines:] 11

\item [Description:] after rewind(srcfile) line, there should be "intfile=fopen("intfile","w+")".
\end{description}
\item {\it Issue\#212 (rshen)}
\begin{description}
\item [Subject:] parameter passing mistake
\item [Criticality:] Hi
\item [Confidence-level:] Hi
\item [Source-node:] P1\_Assign\_Loc

\item [Lines:] 11

\item [Description:] newlocctr and errorsfound should be passed by reference \&newlocctr,
\&errosfound in function p1-proc-start and pc\_proc\_resw
\end{description}
\item {\it Issue\#216 (rshen)}
\begin{description}
\item [Subject:] loccrt incrementation
\item [Criticality:] Hi
\item [Confidence-level:] Med
\item [Source-node:] P1\_Assign\_Loc

\item [Lines:] 28

\item [Description:] locctr is incremented by 3 not multiply by 3 therefore should have be
*newlocctr = locctr + 3;
\end{description}
\item {\it Issue\#218 (giang)}
\begin{description}
\item [Subject:] Missing one condition
\item [Criticality:] Hi
\item [Confidence-level:] Hi
\item [Source-node:] hextonum

\item [Lines:] 17-19

\item [Description:] There should be one condition before this
condition to check the end of the hex number.  If the end is a space then the
loop will be stop.  Rather produce error for anything after the end of the
hex string.
\end{description}
\item {\it Issue\#220 (hhwang)}
\begin{description}
\item [Subject:] passing parameters
\item [Criticality:] Hi
\item [Confidence-level:] Hi
\item [Source-node:] Pass\_1

\item [Lines:] 19-20

\item [Description:] errorfound should be \&errorfound
\end{description}
\item {\it Issue\#228 (rshen)}
\begin{description}
\item [Subject:] initialization
\item [Criticality:] Hi
\item [Confidence-level:] Med
\item [Source-node:] P1\_Read\_Source

\item [Lines:] 43

\item [Description:] i should be initialized to 0 before the while loop
\end{description}
\item {\it Issue\#230 (hhwang)}
\begin{description}
\item [Subject:] passing parameter
\item [Criticality:] Hi
\item [Confidence-level:] Med
\item [Source-node:] Pass\_1

\item [Lines:] 27

\item [Description:] errorflags shoud be \&errorflags
\end{description}
\item {\it Issue\#236 (hhwang)}
\begin{description}
\item [Subject:] parameter passing
\item [Criticality:] Hi
\item [Confidence-level:] Hi
\item [Source-node:] Pass\_1

\item [Lines:] 29

\item [Description:] errorflags should be \&errorflags
\end{description}
\item {\it Issue\#242 (hhwang)}
\begin{description}
\item [Subject:] parameter passing
\item [Criticality:] Hi
\item [Confidence-level:] Hi
\item [Source-node:] Pass\_1

\item [Lines:] 31

\item [Description:] error flags should be \&errorflags
\end{description}
\item {\it Issue\#246 (rshen)}
\begin{description}
\item [Subject:] source.operation field
\item [Criticality:] Hi
\item [Confidence-level:] Med
\item [Source-node:] P1\_Read\_Source

\item [Lines:] 57

\item [Description:] operation field takes 6 space therfore ipboubd for i should be 14 iso 13
\end{description}
\item {\it Issue\#250 (hhwang)}
\begin{description}
\item [Subject:] indention
\item [Criticality:] Low
\item [Confidence-level:] Hi
\item [Source-node:] P1\_Read\_Source

\item [Lines:] 22

\item [Description:] sould be indented since it should be inside for loop.
\end{description}
\item {\it Issue\#254 (giang)}
\begin{description}
\item [Subject:] missing index
\item [Criticality:] Hi
\item [Confidence-level:] Hi
\item [Source-node:] Access\_Symtab

\item [Lines:] 12

\item [Description:] index symbol[100] will be never used.  this
way it can only store up to sumbol[99] although we have 101 indexes.
\end{description}
\item {\it Issue\#258 (rshen)}
\begin{description}
\item [Subject:] function not redinf oprand field
\item [Criticality:] Hi
\item [Confidence-level:] Hi
\item [Source-node:] P1\_Read\_Source

\item [Lines:] 73-74

\item [Description:] after reading operation field, function jump to read comment field, no
operand field is read into source
\end{description}
\item {\it Issue\#262 (rshen)}
\begin{description}
\item [Subject:] wrong array index
\item [Criticality:] Hi
\item [Confidence-level:] Hi
\item [Source-node:] P1\_Read\_Source

\item [Lines:] 45

\item [Description:] source-{\tt >}labl array indec should start from 0 therfore should have been
source-{\tt <}labl[i-7]
\end{description}
\item {\it Issue\#266 (hhwang)}
\begin{description}
\item [Subject:] for loop
\item [Criticality:] Hi
\item [Confidence-level:] Hi
\item [Source-node:] P1\_Read\_Source

\item [Lines:] 29-30

\item [Description:] this while loop should be indented since it should be inside for loop.
\end{description}
\item {\it Issue\#270 (giang)}
\begin{description}
\item [Subject:] condition problem
\item [Criticality:] Hi
\item [Confidence-level:] Hi
\item [Source-node:] Access\_Symtab

\item [Lines:] 22

\item [Description:] This condition will be executed when the value
of the expression is evaluated to nonzero.  If this the case the else state
followed will be never executed. And we never check for the symbol in other
indexes; which is wrong.
\end{description}
\item {\it Issue\#274 (rshen)}
\begin{description}
\item [Subject:] wrong requestcode
\item [Criticality:] Hi
\item [Confidence-level:] Hi
\item [Source-node:] P1\_Assign\_Sym

\item [Lines:] 16

\item [Description:] when symbol not found in symtab, next step should be store therefore should
been Access\_Symtab(STORE.....)
\end{description}
\item {\it Issue\#282 (giang)}
\begin{description}
\item [Subject:] logic is wrong
\item [Criticality:] Hi
\item [Confidence-level:] Hi
\item [Source-node:] Write\_Int\_File

\item [Lines:] 25-27

\item [Description:] This is wrong in the case of when there is no
error the both F and T are written to the INFILE
\end{description}
\item {\it Issue\#288 (hhwang)}
\begin{description}
\item [Subject:] i condition
\item [Criticality:] Hi
\item [Confidence-level:] Hi
\item [Source-node:] P1\_Read\_Source

\item [Lines:] 57

\item [Description:] i {\tt <}= 13 should be i {\tt <}= 14.
\end{description}
\item {\it Issue\#294 (hhwang)}
\begin{description}
\item [Subject:] missing one more loop for operand field
\item [Criticality:] Hi
\item [Confidence-level:] Hi
\item [Source-node:] P1\_Read\_Source

\item [Lines:] 73-74

\item [Description:] befor we start this for loop, we need one more loop for operand field whiich
checks from byte 17 to byte 34.
\end{description}
\item {\it Issue\#298 (rshen)}
\begin{description}
\item [Subject:] 
\item [Criticality:] Hi
\item [Confidence-level:] Hi
\item [Source-node:] P1\_Proc\_RESW

\item [Lines:] 18

\item [Description:] should use hextonum iso. dectonum
\end{description}
\item {\it Issue\#304 (rshen)}
\begin{description}
\item [Subject:] if condition
\item [Criticality:] Med
\item [Confidence-level:] Med
\item [Source-node:] P1\_Proc\_RESW

\item [Lines:] 29

\item [Description:] should also consider errorflag[8] should be if
(!errorflags[9]\&\&!errorflags[8])
\end{description}
\item {\it Issue\#310 (giang)}
\begin{description}
\item [Subject:] initialization of i
\item [Criticality:] Hi
\item [Confidence-level:] Hi
\item [Source-node:] P1\_Read\_Source

\item [Lines:] 43-44

\item [Description:] i should be initialized
\end{description}
\item {\it Issue\#314 (rshen)}
\begin{description}
\item [Subject:] not follow specs
\item [Criticality:] Hi
\item [Confidence-level:] Hi
\item [Source-node:] Write\_Int\_File

\item [Lines:] 29

\item [Description:] there is no statement to check the value of errorsfound according to specs,
errorflas should only be written when errorfound is true.
\end{description}
\item {\it Issue\#318 (giang)}
\begin{description}
\item [Subject:] forget to read operand
\item [Criticality:] Hi
\item [Confidence-level:] Hi
\item [Source-node:] P1\_Read\_Source

\item [Lines:] 74-76

\item [Description:] before these lines there should be statements
to read the operand.
\end{description}
\item {\it Issue\#322 (rshen)}
\begin{description}
\item [Subject:] missing !
\item [Criticality:] Hi
\item [Confidence-level:] Med
\item [Source-node:] Access\_Symtab

\item [Lines:] 22

\item [Description:] missing !infront of strncmp(...)
\end{description}
\item {\it Issue\#326 (giang)}
\begin{description}
\item [Subject:] Read one extra hex digit
\item [Criticality:] Hi
\item [Confidence-level:] Hi
\item [Source-node:] hextonum

\item [Lines:] 22-24

\item [Description:] Since this allow the while to read one more
character (more than 4) ELSE in line 17 will assume that a space after the
last Hex digit will be considered an error.  Which is wrong.
\end{description}
\item {\it Issue\#334 (rshen)}
\begin{description}
\item [Subject:] file i/o
\item [Criticality:] Hi
\item [Confidence-level:] Med
\item [Source-node:] Write\_Int\_File

\item [Lines:] 9

\item [Description:] intfile is not opened before writing
\end{description}
\item {\it Issue\#336 (hhwang)}
\begin{description}
\item [Subject:] wrong parameter
\item [Criticality:] Hi
\item [Confidence-level:] Hi
\item [Source-node:] P1\_Assign\_Sym

\item [Lines:] 16

\item [Description:] SEARCH should be STORE.
\end{description}
\item {\it Issue\#346 (hhwang)}
\begin{description}
\item [Subject:] if condition is wrong
\item [Criticality:] Med
\item [Confidence-level:] Med
\item [Source-node:] P1\_Assign\_Sym

\item [Lines:] 12

\item [Description:] !strncmp() should be (!(!strncmp())).
\end{description}
\item {\it Issue\#350 (rshen)}
\begin{description}
\item [Subject:] 
\item [Criticality:] Low
\item [Confidence-level:] Low
\item [Source-node:] Pass\_1

\item [Lines:] 29

\item [Description:] when call write\_int\_file, source should not be changed therefore should not
include \& in parameter
\end{description}
\item {\it Issue\#354 (rshen)}
\begin{description}
\item [Subject:] passing locctr
\item [Criticality:] Hi
\item [Confidence-level:] Med
\item [Source-node:] P1\_Assign\_Sym

\item [Lines:] 16

\item [Description:] address should be initialized to loccrd before it can` be passed itto Symtab
otherwise locctr will not be stored/
\end{description}
\item {\it Issue\#356 (hhwang)}
\begin{description}
\item [Subject:] if and else
\item [Criticality:] Med
\item [Confidence-level:] Low
\item [Source-node:] Write\_Int\_File

\item [Lines:] 27-28

\item [Description:] should be
	ifv (!errorsfound)
	   ...
	else
	    fprintf();
             =========
\end{description}
\item {\it Issue\#362 (giang)}
\begin{description}
\item [Subject:] Store instead of Search
\item [Criticality:] Hi
\item [Confidence-level:] Hi
\item [Source-node:] P1\_Assign\_Sym

\item [Lines:] 16

\item [Description:] The first parameter should be STORE (NOT SEARCH)
\end{description}
\item {\it Issue\#368 (hhwang)}
\begin{description}
\item [Subject:] missing condition
\item [Criticality:] Hi
\item [Confidence-level:] Hi
\item [Source-node:] P1\_Proc\_RESW

\item [Lines:] 29

\item [Description:] should be if (!errorflags[8] \&\& !errorflags[9])
\end{description}
\item {\it Issue\#372 (hhwang)}
\begin{description}
\item [Subject:] passing parameter
\item [Criticality:] Hi
\item [Confidence-level:] Hi
\item [Source-node:] P1\_Assign\_Loc

\item [Lines:] 11

\item [Description:] P1\_Proc\_START(source, \&newlocctr,\&errorfound,\&errorflags)
\end{description}
\item {\it Issue\#380 (hhwang)}
\begin{description}
\item [Subject:] passing parameters
\item [Criticality:] Hi
\item [Confidence-level:] Hi
\item [Source-node:] P1\_Assign\_Loc

\item [Lines:] 20

\item [Description:] P1\_Proc\_RESW(source,locctr,\&errorsfound,\&errorflags)
\end{description}
\item {\it Issue\#384 (rshen)}
\begin{description}
\item [Subject:] maximal hex digits
\item [Criticality:] Med
\item [Confidence-level:] Med
\item [Source-node:] hextonum

\item [Lines:] 22

\item [Description:] the maximal hex digit should not be depend upon first.  if first is 2 then
i{\tt >}first+3 --{\tt >} i{\tt >}5 against spec therfore, should be i{\tt >}4
\end{description}
\item {\it Issue\#386 (giang)}
\begin{description}
\item [Subject:] locctr +3
\item [Criticality:] Hi
\item [Confidence-level:] Hi
\item [Source-node:] P1\_Assign\_Loc

\item [Lines:] 28-29

\item [Description:] It should be locctr + 3 instead of locctr * 3
\end{description}
\item {\it Issue\#392 (hhwang)}
\begin{description}
\item [Subject:] fprintf reading condition
\item [Criticality:] Hi
\item [Confidence-level:] Hi
\item [Source-node:] Write\_Int\_File

\item [Lines:] 14

\item [Description:] fprintf should skip reading of '\\n' and spaces.
\end{description}
\item {\it Issue\#396 (hhwang)}
\begin{description}
\item [Subject:] if cond is wrong
\item [Criticality:] Hi
\item [Confidence-level:] Hi
\item [Source-node:] Access\_Symtab

\item [Lines:] 22

\item [Description:] should be "else if (!strncmp())".
\end{description}
\end{enumerate}
\section{Review Metrics}
\begin{table}[hb]
\begin{center}
\begin{tabular}{|l|l|l|l|l|}
\hline
Participant & Start-time & End-time & Elapsed-time & Busy-time \\
\hline
hhwang & May 04, 1995 15:15:06 & May 04, 1995 17:09:34 & 1:54:28 & 1:48:15 \\
giang & May 04, 1995 15:12:49 & May 04, 1995 17:07:37 & 1:54:48 & 1:38:53 \\
rshen & May 04, 1995 15:12:49 & May 04, 1995 17:05:29 & 1:52:40 & 1:52:40 \\
\hline
 & & Total & 5:41:56 & \\
\hline
\end{tabular}
\end{center}
\caption{Review Session}
\end{table}


\begin{table}[hb]
\begin{center}
\begin{tabular}{|l|l|l|l|}
\hline
Source & hhwang & giang & rshen\\
\hline
(172)Type and var declarations & 288 & 886 & 4547\\
(174)hextonum & 219 & 778 & 585\\
(176)Access\_Symtab & 1032 & 874 & 1042\\
(178)Write\_Int\_File & 542 & 422 & 385\\
(180)P1\_Read\_Source & 1637 & 986 & 1034\\
(182)P1\_Proc\_START & 325 & 389 & 505\\
(184)P1\_Proc\_RESW & 249 & 340 & 643\\
(186)P1\_Assign\_Loc & 415 & 395 & 396\\
(188)P1\_Assign\_Sym & 751 & 437 & 1003\\
(190)Pass\_1 & 946 & 364 & 859\\
\hline
\end{tabular}
\end{center}
\caption{Review Time}
\end{table}

\begin{table}[hb]
\begin{center}
\begin{tabular}{|l|l|l|l|}
\hline
Source & hhwang & giang & rshen\\
\hline
hextonum & 443.8 & 124.9 & 166.2\\
Access\_Symtab & 202.3 & 238.9 & 200.4\\
Write\_Int\_File & 239.1 & 307.1 & 336.6\\
P1\_Read\_Source & 167.1 & 277.5 & 264.6\\
P1\_Proc\_START & 332.3 & 277.6 & 213.9\\
P1\_Proc\_RESW & 448.2 & 328.2 & 173.6\\
P1\_Assign\_Loc & 251.6 & 264.3 & 263.6\\
P1\_Assign\_Sym & 129.4 & 222.4 & 96.9\\
Pass\_1 & 137.0 & 356.0 & 150.9\\
\hline
\end{tabular}
\end{center}
\caption{Paraphrasing Rate (lines/hour)}
\end{table}

\begin{table}[hb]
\begin{center}
\begin{tabular}{|l|l|l|l|l|}
\hline
Source & hhwang & giang & rshen & OK\\
\hline
Type and var.. &  & 198,204 (=2) &  & \\
hextonum &  & 218,326 (=2) & 384 (=1) & 218,326,384\\
Access\_Symta.. & 396 (=1) & 254,270 (=2) & 322 (=1) & 396=270=322,254\\
Write\_Int\_Fi.. & 356,392 (=2) & 282 (=1) & 314,334 (=2) & 356=282=314\\
P1\_Read\_Sour.. & 250,266,288, & 310,318 (=2) & 228,246,258, & 288=246,310=228 \\
                 &  294 (=4) &                & 262 (=4) & 294=318=258 \\
P1\_Proc\_STAR.. &  &  &  & \\
P1\_Proc\_RESW & 368 (=1) &  & 298,304 (=2) & \\
P1\_Assign\_Lo.. & 372,380 (=2) & 386 (=1) & 212,216 (=2) & 386=216\\
P1\_Assign\_Sy.. & 336,346 (=2) & 362 (=1) & 274,354 (=2) & 336=362=274,346\\
Pass\_1 & 208,220,230, &  & 350 (=1) & \\
 & 236,242 (=5) &  &  & \\
\hline
\end{tabular}
\caption{Source node v.s Issue node}
\end{center}
\end{table}

%%\end{document}
