%%% \documentstyle[11pt,/group/csdl/tex/definemargins,
%%%                        /group/csdl/tex/lmacros]{article} 
%%% 
%%%           \begin{document}
%%%           \begin{center}
%%%           {\large\bf CSRS Experiment Results}\\
%%%           \end{center}
%%%           \small 
\chapter {CSRS Experiment Results (ICS411): Group15 (EIAM)}
\small
	  

\begin{description}
\item [Method:] EIAM
\item [Group:] Group15
\item [Source:] Pass1
\item [Participants:] ysiou (Reviewer), pli (Reviewer), cwchan (Reviewer)
\end{description}
\section{Issue Lists}
\begin{enumerate}
\item {\it Issue\#198 (pli)}
\begin{description}
\item [Subject:] Array is incremented before value is passed to a
         variable.
\item [Criticality:] Med
\item [Confidence-level:] Med
\item [Source-node:] Access\_Symtab

\item [Lines:] 11

\item [Description:] The symbol table array is incremented before
a value is passed to a variable.  Since the symbol table is an array, we
should assume that data stored in the symbol table is stored incrementally
begining with 0 and not one as shown in the program.  The first position of
the array should be symbol[0] and not symbol[1].
\end{description}
\item {\it Issue\#200 (cwchan)}
\begin{description}
\item [Subject:] useless struct element
\item [Criticality:] Med
\item [Confidence-level:] Med
\item [Source-node:] Type and var declarations

\item [Lines:] 45

\item [Description:] we don't need this struct element because it
is not required in the source program format.
\end{description}
\item {\it Issue\#202 (ysiou)}
\begin{description}
\item [Subject:] LOCCTR set wrong value
\item [Criticality:] Hi
\item [Confidence-level:] Med
\item [Source-node:] Pass\_1

\item [Lines:] 22

\item [Description:] newlocctr has been given value yet if first line?
\end{description}
\item {\it Issue\#210 (ysiou)}
\begin{description}
\item [Subject:] wrong PROGSTART value due to LOCCTR
\item [Criticality:] 
\item [Confidence-level:] Med
\item [Source-node:] Pass\_1

\item [Lines:] 23

\item [Description:] 
\end{description}
\item {\it Issue\#214 (pli)}
\begin{description}
\item [Subject:] Printing of T to the intfile.
\item [Criticality:] Med
\item [Confidence-level:] Hi
\item [Source-node:] Write\_Int\_File

\item [Lines:] 25-27

\item [Description:] fprintf will always print "T" to the intfile
regardless of whether there was an error or not ecountered.
\end{description}
\item {\it Issue\#218 (cwchan)}
\begin{description}
\item [Subject:] cannot be static type
\item [Criticality:] Hi
\item [Confidence-level:] Med
\item [Source-node:] Type and var declarations

\item [Lines:] 67

\item [Description:] the REC\_SYMTABTYPE cannot be static type.
\end{description}
\item {\it Issue\#222 (cwchan)}
\begin{description}
\item [Subject:] type mismatch
\item [Criticality:] Hi
\item [Confidence-level:] Hi
\item [Source-node:] Type and var declarations

\item [Lines:] 67

\item [Description:] REC\_SYMTABTYPE is struct type while SYMTAB is
an array. Also, SYMTAB is not defined.
\end{description}
\item {\it Issue\#224 (ysiou)}
\begin{description}
\item [Subject:] i is not correct
\item [Criticality:] Hi
\item [Confidence-level:] Hi
\item [Source-node:] P1\_Read\_Source

\item [Lines:] 43-44

\item [Description:] 
\end{description}
\item {\it Issue\#230 (ysiou)}
\begin{description}
\item [Subject:] labl field not get set because i is not correct
and jumps out
\item [Criticality:] Hi
\item [Confidence-level:] Hi
\item [Source-node:] P1\_Read\_Source

\item [Lines:] 45

\item [Description:] 
\end{description}
\item {\it Issue\#234 (ysiou)}
\begin{description}
\item [Subject:] please reinitilize the variable i
\item [Criticality:] Hi
\item [Confidence-level:] Hi
\item [Source-node:] P1\_Read\_Source

\item [Lines:] 40-41

\item [Description:] 
\end{description}
\item {\it Issue\#238 (ysiou)}
\begin{description}
\item [Subject:] what if source-{\tt >}line starts like this?  12LOOP
do we detect the error?
\item [Criticality:] Med
\item [Confidence-level:] Med
\item [Source-node:] P1\_Read\_Source

\item [Lines:] 43-44

\item [Description:] 
\end{description}
\item {\it Issue\#242 (ysiou)}
\begin{description}
\item [Subject:] operation can not start with numeric: eg., 1ADD
\item [Criticality:] Hi
\item [Confidence-level:] Med
\item [Source-node:] P1\_Read\_Source

\item [Lines:] 57-58

\item [Description:] over looked! maybe it is ok
\end{description}
\item {\it Issue\#246 (cwchan)}
\begin{description}
\item [Subject:] nothing to compare
\item [Criticality:] Hi
\item [Confidence-level:] Hi
\item [Source-node:] hextonum

\item [Lines:] 22

\item [Description:] if (i {\tt >} first + 3) is wrong because the string
doesn't have that many characters.
\end{description}
\item {\it Issue\#250 (ysiou)}
\begin{description}
\item [Subject:] wrong value of i and therefore wrong value of j too
\item [Criticality:] Hi
\item [Confidence-level:] Hi
\item [Source-node:] P1\_Read\_Source

\item [Lines:] 67

\item [Description:] note that we need to reinitlize i
what if i is still 9 that we did not get the privous loop and set
errorflags[1] = true, but i did not get incremented to 13 suppose to be.
\end{description}
\item {\it Issue\#256 (cwchan)}
\begin{description}
\item [Subject:] calcualtion not correct
\item [Criticality:] Hi
\item [Confidence-level:] Hi
\item [Source-node:] hextonum

\item [Lines:] 13-16

\item [Description:] the conversion here is not correct.
\end{description}
\item {\it Issue\#258 (ysiou)}
\begin{description}
\item [Subject:] wrong formula for n
\item [Criticality:] Hi
\item [Confidence-level:] Hi
\item [Source-node:] hextonum

\item [Lines:] 14 14

\item [Description:] the formula is wrong 
we should do scanning backward consider this hex A2 the result of this
function is 10 + (1 * 16 + 2) = 28
\end{description}
\item {\it Issue\#264 (cwchan)}
\begin{description}
\item [Subject:] hash not initialized
\item [Criticality:] Med
\item [Confidence-level:] Med
\item [Source-node:] Access\_Symtab

\item [Lines:] 11

\item [Description:] integer hash is not initialized.
\end{description}
\item {\it Issue\#266 (pli)}
\begin{description}
\item [Subject:] Undefined variable
\item [Criticality:] Hi
\item [Confidence-level:] Hi
\item [Source-node:] Pass\_1

\item [Lines:] 12

\item [Description:] The variable LOCCTR is not defined.
\end{description}
\item {\it Issue\#270 (pli)}
\begin{description}
\item [Subject:] Type mismatch.
\item [Criticality:] Hi
\item [Confidence-level:] Med
\item [Source-node:] Pass\_1

\item [Lines:] 19-20

\item [Description:] errorflags is defined as an array of BOOLEAN.
This is a mismatch from what is being used within the P1\_Assign\_Loc.
\end{description}
\item {\it Issue\#276 (ysiou)}
\begin{description}
\item [Subject:] scan backward, so make it i-- instead
\item [Criticality:] Hi
\item [Confidence-level:] Hi
\item [Source-node:] hextonum

\item [Lines:] 21

\item [Description:] 
\end{description}
\item {\it Issue\#278 (cwchan)}
\begin{description}
\item [Subject:] nothing to return
\item [Criticality:] Hi
\item [Confidence-level:] Med
\item [Source-node:] Access\_Symtab

\item [Lines:] 1-5

\item [Description:] there is nothing to return since it is void.
\end{description}
\item {\it Issue\#284 (ysiou)}
\begin{description}
\item [Subject:] n is miss used as the variable for total
\item [Criticality:] Hi
\item [Confidence-level:] Hi
\item [Source-node:] hextonum

\item [Lines:] 13-16

\item [Description:] 
\end{description}
\item {\it Issue\#288 (pli)}
\begin{description}
\item [Subject:] Wrong addressing for errorsfound.
\item [Criticality:] Med
\item [Confidence-level:] Med
\item [Source-node:] Pass\_1

\item [Lines:] 29

\item [Description:] The errorsfound on this line should be
\&errorsfound as with the other instances that it has been used.
\end{description}
\item {\it Issue\#290 (ysiou)}
\begin{description}
\item [Subject:] wrong n as result
\item [Criticality:] Hi
\item [Confidence-level:] Hi
\item [Source-node:] hextonum

\item [Lines:] 26

\item [Description:] 
\end{description}
\item {\it Issue\#292 (cwchan)}
\begin{description}
\item [Subject:] ptr is increased before the if statement
\item [Criticality:] Hi
\item [Confidence-level:] Hi
\item [Source-node:] Access\_Symtab

\item [Lines:] 53

\item [Description:] ptr should not increase before executing the
if statement.
\end{description}
\item {\it Issue\#300 (pli)}
\begin{description}
\item [Subject:] Should errorflags be an array?
\item [Criticality:] Hi
\item [Confidence-level:] Med
\item [Source-node:] Pass\_1

\item [Lines:] 6

\item [Description:] Should errorflags be defined as an array
of BOOLEAN?  In instances where the variable is used, it is not addressed as
an array.
\end{description}
\item {\it Issue\#304 (cwchan)}
\begin{description}
\item [Subject:] value in ERRORSFLAGS are not written
\item [Criticality:] Hi
\item [Confidence-level:] Hi
\item [Source-node:] Write\_Int\_File

\item [Lines:] 25-27

\item [Description:] eventhough ERRORSFOUND was true, values in
ERRORFLAGS are not written.
\end{description}
\item {\it Issue\#306 (ysiou)}
\begin{description}
\item [Subject:] 
\item [Criticality:] 
\item [Confidence-level:] 
\item [Source-node:] Access\_Symtab

\item [Lines:] 

\item [Description:] 
\end{description}
\item {\it Issue\#310 (ysiou)}
\begin{description}
\item [Subject:] hash value
\item [Criticality:] Med
\item [Confidence-level:] Low
\item [Source-node:] Access\_Symtab

\item [Lines:] 12

\item [Description:] 
\end{description}
\item {\it Issue\#314 (cwchan)}
\begin{description}
\item [Subject:] invalid return
\item [Criticality:] Hi
\item [Confidence-level:] Hi
\item [Source-node:] Write\_Int\_File

\item [Lines:] 15-16

\item [Description:] this line may exit the program without writing
everything to theintfile.
\end{description}
\item {\it Issue\#316 (pli)}
\begin{description}
\item [Subject:] Wrong definition for variable errorflags.
\item [Criticality:] Hi
\item [Confidence-level:] Hi
\item [Source-node:] P1\_Assign\_Sym

\item [Lines:] 4-5

\item [Description:] The variable should be defined as an array
of BOOLEAN since that is how it is used within this procedure.
\end{description}
\item {\it Issue\#322 (pli)}
\begin{description}
\item [Subject:] Un-initialized variable.
\item [Criticality:] Hi
\item [Confidence-level:] Hi
\item [Source-node:] P1\_Assign\_Sym

\item [Lines:] 11

\item [Description:] The errorflags[0] variable has not been
initialized.
\end{description}
\item {\it Issue\#326 (cwchan)}
\begin{description}
\item [Subject:] no exit back to the main
\item [Criticality:] Med
\item [Confidence-level:] Med
\item [Source-node:] Write\_Int\_File

\item [Lines:] 36

\item [Description:] there is no return to go back to the main.
\end{description}
\item {\it Issue\#330 (cwchan)}
\begin{description}
\item [Subject:] 
\item [Criticality:] 
\item [Confidence-level:] 
\item [Source-node:] P1\_Read\_Source

\item [Lines:] 

\item [Description:] 
\end{description}
\item {\it Issue\#332 (ysiou)}
\begin{description}
\item [Subject:] it always put "T\\n" to file
\item [Criticality:] Hi
\item [Confidence-level:] Hi
\item [Source-node:] Write\_Int\_File

\item [Lines:] 25-27

\item [Description:] need to give an else for the if statement
\end{description}
\item {\it Issue\#336 (pli)}
\begin{description}
\item [Subject:] Un-initialized variable.
\item [Criticality:] Hi
\item [Confidence-level:] Hi
\item [Source-node:] P1\_Proc\_RESW

\item [Lines:] 23

\item [Description:] The variables j and i have not been
initialized.
\end{description}
\item {\it Issue\#340 (ysiou)}
\begin{description}
\item [Subject:] no need to put another  "\\n" to file
\item [Criticality:] Med
\item [Confidence-level:] Hi
\item [Source-node:] Write\_Int\_File

\item [Lines:] 35

\item [Description:] 
\end{description}
\item {\it Issue\#344 (cwchan)}
\begin{description}
\item [Subject:] i \& j is not initialized
\item [Criticality:] Hi
\item [Confidence-level:] Hi
\item [Source-node:] P1\_Proc\_START

\item [Lines:] 20-21

\item [Description:] int i \& j is not initialized.
\end{description}
\item {\it Issue\#346 (pli)}
\begin{description}
\item [Subject:] uninitialized variables
\item [Criticality:] Hi
\item [Confidence-level:] Hi
\item [Source-node:] P1\_Proc\_START

\item [Lines:] 21

\item [Description:] The variables i and j have not been
initialized.
\end{description}
\item {\it Issue\#352 (cwchan)}
\begin{description}
\item [Subject:] boolean converror is not initialized
\item [Criticality:] Hi
\item [Confidence-level:] Hi
\item [Source-node:] P1\_Proc\_START

\item [Lines:] 17

\item [Description:] Boolean converror is not initialised.
\end{description}
\item {\it Issue\#356 (cwchan)}
\begin{description}
\item [Subject:] int i \& j are not initialized
\item [Criticality:] Hi
\item [Confidence-level:] Hi
\item [Source-node:] P1\_Proc\_RESW

\item [Lines:] 22-23

\item [Description:] integers i \& j are not initialized.
\end{description}
\item {\it Issue\#358 (ysiou)}
\begin{description}
\item [Subject:] uninitialized variable i
\item [Criticality:] Hi
\item [Confidence-level:] Hi
\item [Source-node:] P1\_Proc\_START

\item [Lines:] 16

\item [Description:] i could be some junk
\end{description}
\item {\it Issue\#362 (pli)}
\begin{description}
\item [Subject:] Un-initialized variable.
\item [Criticality:] Hi
\item [Confidence-level:] Hi
\item [Source-node:] P1\_Read\_Source

\item [Lines:] 43-47

\item [Description:] The variable i in the while loop has not
been initialized.
\end{description}
\item {\it Issue\#364 (cwchan)}
\begin{description}
\item [Subject:] boolean converror is not initialized
\item [Criticality:] Hi
\item [Confidence-level:] Hi
\item [Source-node:] P1\_Proc\_RESW

\item [Lines:] 19

\item [Description:] boolean converror is not initialized.
\end{description}
\item {\it Issue\#372 (pli)}
\begin{description}
\item [Subject:] Uninitialized variable.
\item [Criticality:] Hi
\item [Confidence-level:] Med
\item [Source-node:] P1\_Read\_Source

\item [Lines:] 48-49

\item [Description:] The variables i and j has not been
initialized.
\end{description}
\item {\it Issue\#374 (ysiou)}
\begin{description}
\item [Subject:] wrong j due to wrong i
\item [Criticality:] Hi
\item [Confidence-level:] Hi
\item [Source-node:] P1\_Proc\_START

\item [Lines:] 21

\item [Description:] 
\end{description}
\item {\it Issue\#376 (cwchan)}
\begin{description}
\item [Subject:] this line  should not put inside else statement
\item [Criticality:] Hi
\item [Confidence-level:] Hi
\item [Source-node:] P1\_Assign\_Loc

\item [Lines:] 28

\item [Description:] newlocctr can never be updated.
\end{description}
\item {\it Issue\#382 (pli)}
\begin{description}
\item [Subject:] Un-initialized variable.
\item [Criticality:] Hi
\item [Confidence-level:] Hi
\item [Source-node:] P1\_Read\_Source

\item [Lines:] 67

\item [Description:] The variables i and j have not been
initialized.
\end{description}
\item {\it Issue\#388 (cwchan)}
\begin{description}
\item [Subject:] wrong condition setting
\item [Criticality:] Hi
\item [Confidence-level:] Hi
\item [Source-node:] P1\_Assign\_Sym

\item [Lines:] 14

\item [Description:] the condition here is not correct, should
change to FOUND.
\end{description}
\item {\it Issue\#390 (ysiou)}
\begin{description}
\item [Subject:] need to initialize variable i
\item [Criticality:] Hi
\item [Confidence-level:] Med
\item [Source-node:] P1\_Proc\_RESW

\item [Lines:] 18

\item [Description:] 
\end{description}
\item {\it Issue\#394 (ysiou)}
\begin{description}
\item [Subject:] what if we have error 8 ?
\item [Criticality:] Med
\item [Confidence-level:] Med
\item [Source-node:] P1\_Proc\_RESW

\item [Lines:] 29-30

\item [Description:] 
\end{description}
\item {\it Issue\#398 (pli)}
\begin{description}
\item [Subject:] Negative values.
\item [Criticality:] Med
\item [Confidence-level:] Med
\item [Source-node:] hextonum

\item [Lines:] 16

\item [Description:] n may be a negative value.
\end{description}
\item {\it Issue\#404 (ysiou)}
\begin{description}
\item [Subject:] should we use STORE ?
\item [Criticality:] Hi
\item [Confidence-level:] Hi
\item [Source-node:] P1\_Assign\_Sym

\item [Lines:] 16

\item [Description:] 
\end{description}
\item {\it Issue\#406 (pli)}
\begin{description}
\item [Subject:] Possible problem with HEX value length being scanned.
\item [Criticality:] Hi
\item [Confidence-level:] Med
\item [Source-node:] hextonum

\item [Lines:] 21-23

\item [Description:] The length of the HEX value being scanned may
be off.
\end{description}
\item {\it Issue\#410 (cwchan)}
\begin{description}
\item [Subject:] syntax error
\item [Criticality:] Hi
\item [Confidence-level:] Hi
\item [Source-node:] P1\_Assign\_Sym

\item [Lines:] 17

\item [Description:] though there should not have any syntax error,
I still find one. Should use "==" instead of single "=".
\end{description}
\item {\it Issue\#416 (cwchan)}
\begin{description}
\item [Subject:] srcfile cannot never set to NULL
\item [Criticality:] Hi
\item [Confidence-level:] Hi
\item [Source-node:] Pass\_1

\item [Lines:] 33 34-35

\item [Description:] the srcfile is close before it is set to NULL.
\end{description}
\item {\it Issue\#418 (ysiou)}
\begin{description}
\item [Subject:] wrong j
\item [Criticality:] Med
\item [Confidence-level:] Hi
\item [Source-node:] P1\_Read\_Source

\item [Lines:] 49

\item [Description:] 
\end{description}
\item {\it Issue\#424 (ysiou)}
\begin{description}
\item [Subject:] wrong i, we need to reinitialize it
\item [Criticality:] Hi
\item [Confidence-level:] Hi
\item [Source-node:] P1\_Read\_Source

\item [Lines:] 40-48

\item [Description:] 
\end{description}
\item {\it Issue\#428 (cwchan)}
\begin{description}
\item [Subject:] while loop incorrect
\item [Criticality:] Hi
\item [Confidence-level:] Hi
\item [Source-node:] P1\_Read\_Source

\item [Lines:] 43-47

\item [Description:] infinite loop
\end{description}
\item {\it Issue\#430 (ysiou)}
\begin{description}
\item [Subject:] is it the error if source-{\tt >}line[j] != ' '
\item [Criticality:] Med
\item [Confidence-level:] Med
\item [Source-node:] P1\_Read\_Source

\item [Lines:] 68-71

\item [Description:] 
\end{description}
\item {\it Issue\#436 (ysiou)}
\begin{description}
\item [Subject:] 
\item [Criticality:] Med
\item [Confidence-level:] Med
\item [Source-node:] Type and var declarations

\item [Lines:] 53-56

\item [Description:] 
\end{description}
\end{enumerate}
\section{Review Metrics}
\begin{table}[hb]
\begin{center}
\begin{tabular}{|l|l|l|l|l|}
\hline
Participant & Start-time & End-time & Elapsed-time & Busy-time \\
\hline
cwchan & May 04, 1995 15:13:21 & May 04, 1995 17:09:36 & 1:56:15 & 1:36:35 \\
pli & May 04, 1995 15:12:40 & May 04, 1995 17:07:27 & 1:54:47 & 1:43:45 \\
ysiou & May 04, 1995 15:15:22 & May 04, 1995 17:10:30 & 1:55:8 & 1:48:19 \\
\hline
 & & Total & 5:46:10 & \\
\hline
\end{tabular}
\end{center}
\caption{Review Session}
\end{table}


\begin{table}[hb]
\begin{center}
\begin{tabular}{|l|l|l|l|}
\hline
Source & cwchan & pli & ysiou\\
\hline
(172)Type and var declarations & 1435 & 342 & 722\\
(174)hextonum & 926 & 651 & 951\\
(176)Access\_Symtab & 711 & 698 & 271\\
(178)Write\_Int\_File & 694 & 357 & 389\\
(180)P1\_Read\_Source & 484 & 810 & 1686\\
(182)P1\_Proc\_START & 202 & 483 & 437\\
(184)P1\_Proc\_RESW & 186 & 518 & 307\\
(186)P1\_Assign\_Loc & 299 & 798 & 174\\
(188)P1\_Assign\_Sym & 565 & 592 & 261\\
(190)Pass\_1 & 265 & 886 & 1208\\
\hline
\end{tabular}
\end{center}
\caption{Review Time}
\end{table}

\begin{table}[hb]
\begin{center}
\begin{tabular}{|l|l|l|l|}
\hline
Source & cwchan & pli & ysiou\\
\hline
hextonum & 105.0 & 149.3 & 102.2\\
Access\_Symtab & 293.7 & 299.1 & 770.5\\
Write\_Int\_File & 186.7 & 363.0 & 333.2\\
P1\_Read\_Source & 565.3 & 337.8 & 162.3\\
P1\_Proc\_START & 534.7 & 223.6 & 247.1\\
P1\_Proc\_RESW & 600.0 & 215.4 & 363.5\\
P1\_Assign\_Loc & 349.2 & 130.8 & 600.0\\
P1\_Assign\_Sym & 172.0 & 164.2 & 372.4\\
Pass\_1 & 489.1 & 146.3 & 107.3\\
\hline
\end{tabular}
\end{center}
\caption{Paraphrasing Rate (lines/hour)}
\end{table}


\begin{table}[hb]
\begin{center}
\begin{tabular}{|l|l|l|l|l|}
\hline
Source & cwchan & pli & ysiou & OK\\
\hline
Type and var.. & 200,218,222 (=3) &  & 436 (=1) & \\
hextonum & 246,256 (=2) & 398,406 (=2) & 258,276,284,290(=4) & 246,290\\
Access\_Symta.. & 264,278,292 (=3) & 198 (=1) & 306,310 (=2) & 264,292\\
Write\_Int\_Fi.. & 304,314,326 (=3) & 214 (=1) & 332,340 (=2) & 304=214=332\\
P1\_Read\_Sour.. & 330,428 (=2) & 362,372,382 (=3) & 224,230,234,238 & 362=234\\
 &  &  & 242,250,418,424 &\\
 &  &  & 430 (=9) & \\
P1\_Proc\_STAR.. & 344,352 (=2) & 346 (=1) & 358,374 (=2) & \\
P1\_Proc\_RESW & 356,364 (=2) & 336 (=1) & 390,394 (=2) & \\
P1\_Assign\_Lo.. & 376 (=1) &  &  & \\
P1\_Assign\_Sy.. & 388,410 (=2) & 316,322 (=2) & 404 (=1) & 410,404\\
Pass\_1 & 416 (=1) & 266,270,288,300 (=4) & 202,210 (=2) & \\
\hline
\end{tabular}
\caption{Source node v.s Issue node}
\end{center}
\end{table}

%%\end{document}
