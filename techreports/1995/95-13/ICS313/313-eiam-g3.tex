%\documentstyle[11pt,/group/csdl/tex/definemargins,
%                       /group/csdl/tex/lmacros]{article} 

%          \begin{document}
%          \begin{center}
%          {\large\bf CSRS Experiment Results}\\
%          \end{center}
%          \small 
\chapter {CSRS Experiment Results (ICS313): Group3 (EIAM)}
\small
	  

\begin{description}
\item [Method:] EIAM
\item [Group:] Group3
\item [Source:] Employee1
\item [Participants:] an (Reviewer), mscruz (Reviewer), tkoyama (Reviewer)
\end{description}
\section{Issue Lists}
\begin{enumerate}
\item {\it Issue\#210 (tkoyama)}
\begin{description}
\item [Subject:] missing semicolon
\item [Criticality:] Low
\item [Confidence-level:] Hi
\item [Source-node:] Constant

\item [Lines:] 2

\item [Description:] a semicolon is missing at end of the line
\end{description}
\item {\it Issue\#214 (mscruz)}
\begin{description}
\item [Subject:] no allocated space
\item [Criticality:] Hi
\item [Confidence-level:] Hi
\item [Source-node:] Employee::\~Employee

\item [Lines:] 7-8

\item [Description:] the allocated space for name and employee has
been deleted, yet a value has been assigned.  Re-initialization after
deletion is not necessary since this is garbage collecting.
\end{description}
\item {\it Issue\#218 (an)}
\begin{description}
\item [Subject:] make sure the condition
\item [Criticality:] Med
\item [Confidence-level:] Hi
\item [Source-node:] Employee::setName

\item [Lines:] 10

\item [Description:] the source code should set initial condition,
which is if(name !=0) delete[] name.
\end{description}
\item {\it Issue\#226 (an)}
\begin{description}
\item [Subject:] segmentation fault
\item [Criticality:] Hi
\item [Confidence-level:] Hi
\item [Source-node:] Employee::setName

\item [Lines:] 11

\item [Description:] since the name[len] was set to null, so the
character length should be len+1,which means the select line should be writen
as name=new char[len+1].
\end{description}
\item {\it Issue\#230 (an)}
\begin{description}
\item [Subject:] unnecessary code
\item [Criticality:] Low
\item [Confidence-level:] Hi
\item [Source-node:] Employee::\~Employee

\item [Lines:] 7-8

\item [Description:] this code are not necessary. if these codes
are deleted, the result will be the same as the program which include these
codes.
\end{description}
\item {\it Issue\#232 (tkoyama)}
\begin{description}
\item [Subject:] array allocation
\item [Criticality:] Med
\item [Confidence-level:] Hi
\item [Source-node:] Employee::setName

\item [Lines:] 11

\item [Description:] No allocation for the null character at the
end of the name.
\end{description}
\item {\it Issue\#242 (an)}
\begin{description}
\item [Subject:] no significiant assignment
\item [Criticality:] Low
\item [Confidence-level:] Hi
\item [Source-node:] Employee::Employee

\item [Lines:] 4

\item [Description:] since whenever assign a new to the name, the
initial space which will be occupied by the new will be checked first and
then cleared based on the checking condition, there is no point to assign 0
to the char[1].
\end{description}
\item {\it Issue\#246 (tkoyama)}
\begin{description}
\item [Subject:] initialzing name
\item [Criticality:] Med
\item [Confidence-level:] Med
\item [Source-node:] Employee::Employee

\item [Lines:] 4

\item [Description:] don't need to initialize name with 0 since it
is already going to be pointing to the allocated space for the array.
\end{description}
\item {\it Issue\#252 (mscruz)}
\begin{description}
\item [Subject:] Checking in wrong area
\item [Criticality:] Hi
\item [Confidence-level:] Hi
\item [Source-node:] Employee::setSocSecurity

\item [Lines:] 22 22

\item [Description:] This should be at the beginning of the
function.  If the testing of the format proceeds without checking if space
has been allocated, the program may produce unpredictable results.
\end{description}
\item {\it Issue\#258 (mscruz)}
\begin{description}
\item [Subject:] function does not fit specification
\item [Criticality:] Low
\item [Confidence-level:] Hi
\item [Source-node:] Employee::setNumDependents

\item [Lines:] 4

\item [Description:] According to the specs, the number of
dependents is an integer between 0 and 10.  However, this condition will
never allow a 0 to be a valid value.  It should be numDependents {\tt >} -1 or
numDependents {\tt >}= 0.
\end{description}
\item {\it Issue\#264 (an)}
\begin{description}
\item [Subject:] no null tenimator setting
\item [Criticality:] Hi
\item [Confidence-level:] Hi
\item [Source-node:] Employee::setSocSecurity

\item [Lines:] 29

\item [Description:] there is no null terminator setting in tbe
newNum before return,this will cause the segamenation fault in the program
running.
\end{description}
\item {\it Issue\#272 (mscruz)}
\begin{description}
\item [Subject:] No allocated space for pointers to objects
\item [Criticality:] Hi
\item [Confidence-level:] Med
\item [Source-node:] Company1::Company1

\item [Lines:] 7

\item [Description:] The array of Employees has not been allocated
space to contain the 0.  Thus, a segmentation error results (in a Unix
system). Since it is an array of pointers to Employees, each element of the
array should be allocated space for the object and then nulled out to
indicate it is empty.  Otherwise, the array could just hold the objects
themselves.
\end{description}
\item {\it Issue\#276 (tkoyama)}
\begin{description}
\item [Subject:] printing name
\item [Criticality:] Low
\item [Confidence-level:] Med
\item [Source-node:] Employee::print

\item [Lines:] 4

\item [Description:] Since "name" is a pointer to a character, you
need to have a loop to display the contents of the entire name in the array
which "name" points to.
\end{description}
\item {\it Issue\#282 (tkoyama)}
\begin{description}
\item [Subject:] printing SSN
\item [Criticality:] Low
\item [Confidence-level:] Med
\item [Source-node:] Employee::print

\item [Lines:] 5-6

\item [Description:] Since socSecurity is a pointer to a char,
which is actually a pointer to an array which holds the social security
number, you need to have a loop to go print every character.
\end{description}
\item {\it Issue\#286 (mscruz)}
\begin{description}
\item [Subject:] Re-initialization not needed
\item [Criticality:] Low
\item [Confidence-level:] Hi
\item [Source-node:] Company1::\~Company1

\item [Lines:] 5

\item [Description:] Since the object is not needed, initialization
is not necessary.  This line could be left out.
\end{description}
\item {\it Issue\#290 (tkoyama)}
\begin{description}
\item [Subject:] for loop
\item [Criticality:] Med
\item [Confidence-level:] Hi
\item [Source-node:] Company1::Company1

\item [Lines:] 6

\item [Description:] i will decrement till the second cell in the
array then stop.  the for loop should continue until "i{\tt >}=0".
\end{description}
\item {\it Issue\#296 (mscruz)}
\begin{description}
\item [Subject:] derived class should be declared public to base class
\item [Criticality:] Hi
\item [Confidence-level:] Hi
\item [Source-node:] Employee

\item [Lines:] 1

\item [Description:] Since this is a derived class, it should be
declared public to the base class.  Without it, the class will not be visible
to the base class.  A " : public" should be appended.
\end{description}
\item {\it Issue\#300 (tkoyama)}
\begin{description}
\item [Subject:] array deletion
\item [Criticality:] Hi
\item [Confidence-level:] Med
\item [Source-node:] Company1::\~Company1

\item [Lines:] 4

\item [Description:] Workers array does not have to be deallocated
since it was no allocated by the keyword new.  The program automatically
takes care of that.
\end{description}
\item {\it Issue\#308 (mscruz)}
\begin{description}
\item [Subject:] Loop will check out of bounds
\item [Criticality:] Med
\item [Confidence-level:] Hi
\item [Source-node:] Company1::addEmployee

\item [Lines:] 28 27-28

\item [Description:] when j=0 and then decremented to -1, the next
line will access Workers[-1], which does not exist here.  The two lines
should be interchanged.
\end{description}
\item {\it Issue\#316 (an)}
\begin{description}
\item [Subject:] infinite loop
\item [Criticality:] Hi
\item [Confidence-level:] Med
\item [Source-node:] Company1::print

\item [Lines:] 6

\item [Description:] the initial value of i is set to 0, so there
is no
\end{description}
\item {\it Issue\#320 (mscruz)}
\begin{description}
\item [Subject:] Inappropriate deletion
\item [Criticality:] Med
\item [Confidence-level:] Hi
\item [Source-node:] Company1::\~Company1

\item [Lines:] 4

\item [Description:] This line only deletes the pointers to each
object in the array.  Thus, the space allocated for each object is not made
unavailable.  A for-loop should be used to delete each Employee object being
pointed to before the array itself is deleted.
\end{description}
\item {\it Issue\#324 (mscruz)}
\begin{description}
\item [Subject:] no allocated space to store new value
\item [Criticality:] Hi
\item [Confidence-level:] Med
\item [Source-node:] Company1::deleteEmployee

\item [Lines:] 9

\item [Description:] The space that the pointer was pointing to has
been deallocated.  Therefore, in a Unix system, a segmentation fault will
result.  This line could be deleted since the space will be used anyway.
\end{description}
\item {\it Issue\#328 (mscruz)}
\begin{description}
\item [Subject:] Condition will never be satisfied
\item [Criticality:] Hi
\item [Confidence-level:] Hi
\item [Source-node:] Company1::deleteEmployee

\item [Lines:] 14

\item [Description:] i will never reach zero since the initial
value of i is positive, and i will always be incremented.  The line should be
"i--" instead.
\end{description}
\item {\it Issue\#332 (an)}
\begin{description}
\item [Subject:] wrong position
\item [Criticality:] Hi
\item [Confidence-level:] Hi
\item [Source-node:] Company1::addEmployee

\item [Lines:] 28

\item [Description:] currentSSN is put in the wrong positon in the
Worker[].
\end{description}
\item {\it Issue\#336 (tkoyama)}
\begin{description}
\item [Subject:] allocation size
\item [Criticality:] Med
\item [Confidence-level:] Hi
\item [Source-node:] Employee::Employee

\item [Lines:] 4

\item [Description:] Allocated size of name array is too small.
That's only enough room for null.
\end{description}
\item {\it Issue\#340 (an)}
\begin{description}
\item [Subject:] wrong position
\item [Criticality:] Hi
\item [Confidence-level:] Hi
\item [Source-node:] Company1::addEmployee

\item [Lines:] 28

\item [Description:] currentSSN is put into the wrong position in
the workers[].
\end{description}
\item {\it Issue\#344 (an)}
\begin{description}
\item [Subject:] wrong position
\item [Criticality:] Hi
\item [Confidence-level:] Hi
\item [Source-node:] Company1::addEmployee

\item [Lines:] 35

\item [Description:] the newWorker is put into a wrong position in
the Workers[].
\end{description}
\item {\it Issue\#348 (tkoyama)}
\begin{description}
\item [Subject:] copying pointers to characters
\item [Criticality:] Hi
\item [Confidence-level:] Low
\item [Source-node:] Employee::setName

\item [Lines:] 15

\item [Description:] Need to have stars in front of both name and
newName to dereference the values in each.
\end{description}
\item {\it Issue\#352 (tkoyama)}
\begin{description}
\item [Subject:] delete employee
\item [Criticality:] Med
\item [Confidence-level:] Low
\item [Source-node:] Company1::deleteEmployee

\item [Lines:] 8

\item [Description:] Since Workers contains pointers to Employees,
you don't want to delete the pointers.  Instead, there should be a star in
front of Workers to call the destructor of that particular Employee.
\end{description}
\item {\it Issue\#356 (tkoyama)}
\begin{description}
\item [Subject:] endless while loop
\item [Criticality:] Med
\item [Confidence-level:] Med
\item [Source-node:] Company1::addEmployee

\item [Lines:] 30-31

\item [Description:] In while loop, there is no case that exits the
loop for when newSSN {\tt >} currentSSN.  This last else statement supposedly takes
care this statement but in actuallity, there should be another else if
statement for this last case.  Since stilllooking is still assigned the value
of 0, the loop never ends.

In other words, a new employee is never added to the end of the arrary of
workers.
\end{description}
\item {\it Issue\#360 (mscruz)}
\begin{description}
\item [Subject:] Private function not defined
\item [Criticality:] Low
\item [Confidence-level:] Hi
\item [Source-node:] Company1

\item [Lines:] 7

\item [Description:] There is no implementation of this function.
Thus, this line could be eliminated.
\end{description}
\item {\it Issue\#364 (mscruz)}
\begin{description}
\item [Subject:] non-virtual
\item [Criticality:] Hi
\item [Confidence-level:] Hi
\item [Source-node:] Company1

\item [Lines:] 14

\item [Description:] Without the virtual declaration, the print
implementation of Employee will not be used.  Instead, the implementation of
Company1 is used.  It should be declared virtual.
\end{description}
\item {\it Issue\#368 (mscruz)}
\begin{description}
\item [Subject:] non-virtual
\item [Criticality:] Low
\item [Confidence-level:] Hi
\item [Source-node:] Company1

\item [Lines:] 10

\item [Description:] The implementation of Employee's delete will
not be called.  Thus, garbage collection is improperly done.  Should be
declared virtual.
\end{description}
\end{enumerate}
\section{Review Metrics}
\begin{table}[hb]
\begin{center}
\begin{tabular}{|l|l|l|l|l|}
\hline
Participant & Start-time & End-time & Elapsed-time & Total Busy-time \\
\hline
tkoyama & Mar 21, 1995 18:18:12 & Mar 21, 1995 20:21:38 & 2:3:26 & 1:51:9 \\
mscruz & Mar 21, 1995 18:16:42 & Mar 21, 1995 20:23:46 & 2:7:4 & 2:7:4 \\
an & Mar 21, 1995 18:16:38 & Mar 21, 1995 20:07:30 & 1:50:52 & 1:32:52 \\
\hline
\end{tabular}
\end{center}
\caption{Review Session}
\end{table}


\begin{table}[hb]
\begin{center}
\begin{tabular}{|l|l|l|l|}
\hline
Source & tkoyama & mscruz & an\\
\hline
(192)Company1::Company1 & 319 & 972 & 475\\
(176)Employee::Employee & 1803 & 1828 & 431\\
(194)Company1::\~Company1 & 193 & 393 & 130\\
(178)Employee::\~Employee & 151 & 292 & 310\\
(196)Company1::addEmployee & 793 & 1655 & 618\\
(180)Employee::setName & 1588 & 1183 & 588\\
(198)Company1::findEmployee & 206 & 522 & 117\\
(182)Employee::setSocSecurity & 273 & 1030 & 688\\
(200)Company1::deleteEmployee & 370 & 1105 & 293\\
(184)Employee::setAge & 169 & 96 & 193\\
(202)Company1::print & 258 & 498 & 750\\
(186)Employee::setNumDependents & 186 & 189 & 111\\
(170)Constant & 498 & 664 & 76\\
(188)Employee::print & 475 & 87 & 195\\
(172)Employee & 2379 & 1230 & 240\\
(190)Employee::getSocSecurity & 59 & 47 & 47\\
(174)Company1 & 1771 & 3887 & 259\\
\hline
\end{tabular}
\end{center}
\caption{Review Time}
\end{table}


\begin{table}[hb]
\begin{center}
\begin{tabular}{|l|l|l|l|l|}
\hline
Source & tkoyama & mscruz & an & OK\\
\hline
Company1::Company1 & 290 (=1) & 272 (=1) &  & 290\\
Employee::Employee & 246,336 (=2) &  & 242 (=1) & 246=242 \\
Company1::\~Company1 & 300 (=1) & 286,320 (=2) &  & 320\\
Employee::\~Employee &  & 214 (=1) & 230 (=1) & \\
Company1::addEmployee & 356 (=1) & 308 (=1) & 332,340,344 (=3) & 308\\
Employee::setName & 232,348 (=2) &  & 218,226 (=2) & 232=226\\
Company1::findEmployee &  &  &  & \\
Employee::setSocSecurity &  & 252 (=1) & 264 (=1) & 264\\
Company1::deleteEmployee & 352 (=1) & 324,328 (=2) &  & \\
Employee::setAge &  &  &  & \\
Company1::print &  &  & 316 (=1) & \\
Employee::setNumDependents &  & 258 (=1) &  & 258\\
Constant & 210 (=1) &  &  & \\
Employee::print & 276,282 (=2) &  &  & \\
Employee &  & 296 (=1) &  & \\
Employee::getSocSecurity &  &  &  & \\
Company1 &  & 360,364,368 (=3) &  & \\
\hline
\end{tabular}
\caption{Source node v.s Issue node}
\end{center}
\end{table}

%\end{document}
