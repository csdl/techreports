%%% \documentstyle[11pt,/group/csdl/tex/definemargins,
%%%                        /group/csdl/tex/lmacros]{article} 
%%% 
%%%           \begin{document}
%%%           \begin{center}
%%%           {\large\bf CSRS Experiment -- Source Listing \\
%%%             Source: Employee1 }
%%% \end{center}

\chapter {CSRS Experiment (ICS313): Source Listing -- Employee1}
\small
	  

\section{Constant}
\subsection*{Specification:}

\subsection*{Source-code:}
\begin{verbatim}
001:const int MAXLEN  256;     // Max no of chars of string input
002:const int MAX_EMPLOYEES 100 // Max no of employees in the company  
003:
004:
\end{verbatim}
\section{Employee}
\subsection*{Specification:}
Employee class: each instance of this class represents a single employee. Each Employee
has the following attributes: name (a string), socSecurity (a string with format
"XXX-XX-XXXX", where each X is a digit), age (an integer between 17-55), numDependents
(an integer between 0 and 10).
\subsection*{Source-code:}
\begin{verbatim}
001:class Employee
002:
003:{
004:  protected:    // Protected for use in derived classes
005:    char *name;
006:    char *socSecurity;
007:    int age;
008:    int numDependents;
009:
010:  public:    // public member functions
011:    Employee();
012:    virtual ~Employee();
013:    int setName(char* newName);
014:    int setSocSecurity(char* newNum);
015:    int setAge(int newAge);
016:    int setNumDependents(int newNumDependents);
017:    char* getSocSecurity();
018:    virtual void print();
019:  };
020:
021:
\end{verbatim}
\section{Company1}
\subsection*{Specification:}
Company1 class: Each instance of company1 holds instances of employees.  Instances of
employee must be stored as a sorted list ordered by social security number.  The list
must be implemented by an array.
\subsection*{Source-code:}
\begin{verbatim}
001:class Company1
002:
003:{
004:private:
005:    Employee* Workers[MAX_EMPLOYEES];
006:    int numEmployees;
007:    void insertion_sort();
008:public:
009:  Company1();
010:  ~Company1();
011:  int addEmployee(Employee* newWorker);
012:  int deleteEmployee(char* SSN);
013:  Employee* findEmployee(char* SSN);
014:  void print();
015:};
016:
\end{verbatim}
\section{Employee::Employee}
\subsection*{Specification:}
Constructor to init Employee
\subsection*{Source-code:}
\begin{verbatim}
001:Employee::Employee()
002:
003:{
004:    name = new char[1] = 0;     //initialize name string to one space
005:    socSecurity = new char [12];//initialize spaces for SSN
006:    age = 17;                   // init age
007:    numDependents = 0;          // init number of dependents
008:}
009:
010:
\end{verbatim}
\section{Employee::\~Employee}
\subsection*{Specification:}
Destructor for Employee
\subsection*{Source-code:}
\begin{verbatim}
001:Employee::~Employee()
002:
003:{
004:    delete[] name;                      // delete name array
005:    delete[] socSecurity;               // delete SSN array
006:
007:    name = 0;
008:    socSecurity = 0;
009:}
010:
011:
\end{verbatim}
\section{Employee::setName}
\subsection*{Specification:}
Set the name of the Employee to newName
  Takes a char pointer to newName.
  Returns an int 0 if name set, 1 if failed.
\subsection*{Source-code:}
\begin{verbatim}
001:int Employee::setName( char* newName )
002:  
003:{
004:   int len = strlen( newName );
005:
006:   //error check.
007:   if( len < 1 ) 
008:       return 1; //Error: name not read
009:
010:   delete[] name;
011:   name = new char[len];
012:
013:   //copy the new name into name
014:   for( int i = 0; i < len; i++ ) 
015:       name[i] =  newName[i];
016:   name[len] = 0; //NULL terminator.
017:
018:   return 0;
019:}
020:
021:
\end{verbatim}
\section{Employee::setSocSecurity}
\subsection*{Specification:}
Set SSN for the Employee to newNum.
  Takes a char pointer to newNum.
  Returns an int 0 if SSN set 1 if failed.
  SSN must be a string with format "XXX-XX-XXXX",
  where each X is a digit.
\subsection*{Source-code:}
\begin{verbatim}
001:int Employee::setSocSecurity(char* newNum)
002:
003:     
004:{
005:  int i;
006:  
007:  //Test for SSN format
008:  for (i = 0; i < 11; ++i){
009:    if ( i != 3 && i != 6)
010:      {  
011:    if (newNum[i] < '0' && newNum[i] > '9')
012:      cout << "Invalid SSN." << endl;
013:    return 1;
014:      }
015:    else if (newNum[i] != '-')
016:      { 
017:    cout << "Invalid SSN." << endl; 
018:    return 1;
019:      }
020:  }
021:  //Copy the SSN
022:  if (socSecurity)          // is space allocated?
023:    {
024:      int n = strlen(newNum);
025:      char *p = socSecurity;        // pointer to SSN
026:      char *q = newNum;             // pointer to new SSN
027:      while(n--)
028:         *p++ = *q++  ;          // Copy the SSN
029:      return 0;
030:    }
031:  else
032:    return 1;                       // if not send error
033:}
034:
035:
\end{verbatim}
\section{Employee::setAge}
\subsection*{Specification:}
Set the age of the Employee to newAge
  Takes an integer newAge.
  Returns an int 0 if age set 1 if failed.
  Age is an integer between 17 - 55.
\subsection*{Source-code:}
\begin{verbatim}
001:int Employee::setAge(int newAge)    
002:  
003:{
004:  if ((newAge < 17) || (newAge > 55))       // is outside the range?
005:    return 1;                   
006:  else
007:    {
008:      age = newAge;         
009:      return 0;
010:    }
011:}
012:
013:
\end{verbatim}
\section{Employee::setNumDependents}
\subsection*{Specification:}
Set the number of employee dependents to newNumDependents.
  Takes an integer newNumDependents.
  Returns an int 0 if the number of dependents set 1 if failed.
  The number of dependents is an integer between 0 and 10.
\subsection*{Source-code:}
\begin{verbatim}
001:int Employee::setNumDependents(int newNumDependents) 
002:  
003:{
004:  if((newNumDependents>0) && (newNumDependents<11))
005:    {
006:      numDependents=newNumDependents;
007:      return 0;
008:    }
009:  else
010:    return 1;
011:}
012:
013:
\end{verbatim}
\section{Employee::print}
\subsection*{Specification:}
Prints the Employee
\subsection*{Source-code:}
\begin{verbatim}
001:void Employee::print()
002:
003:{
004:    cout << "Name: "<< name;
005:    cout << "  SSN: "<< socSecurity;
006:    cout << "  Age: "<< age;
007:    cout << "  #Dep: "<< numDependents;
008:}
009:
010:
\end{verbatim}
\section{Employee::getSocSecurity}
\subsection*{Specification:}
Returns a pointer to the SSN of the Employee
\subsection*{Source-code:}
\begin{verbatim}
001:char* Employee::getSocSecurity() 
002:
003:{
004:    return socSecurity;
005:}
006:
007:
\end{verbatim}
\section{Company1::Company1}
\subsection*{Specification:}
Constructor for Company1 class
\subsection*{Source-code:}
\begin{verbatim}
001:Company1::Company1()
002:
003:{
004:  int i;                    // counter for init loop
005:    
006:  for(i=MAX_EMPLOYEES-1; i>0 ; i--)
007:    Workers[i] = 0;         // Init the array
008:
009:  numEmployees = 0;         // No Employees yet
010:}
011:
012:
\end{verbatim}
\section{Company1::\~Company1}
\subsection*{Specification:}
Destructor for Company1 class
\subsection*{Source-code:}
\begin{verbatim}
001:Company1::~Company1()
002:
003:{
004:  delete [] Workers; //Delete the array of workers
005:  numEmployees = 0;
006:}
007:
008:
\end{verbatim}
\section{Company1::addEmployee}
\subsection*{Specification:}
Adds a new Employee to Company1.
  Employee records are maintained in ascending order of SSN.
  Takes an Employee pointer newWorker.
  Returns an integer 0 if newWorker is successfully added,
  returns 1 otherwise.
\subsection*{Source-code:}
\begin{verbatim}
001:int Company1::addEmployee(Employee* newWorker) 
002:  
003:{
004:  int j, stillLooking, stringcomparison;
005:  char *newSSN, *currentSSN; 
006:  
007:  if (numEmployees == MAX_EMPLOYEES)
008:    return 1;
009:  else
010:    {
011:      j = numEmployees-1;
012:      stillLooking = 1;     //set flag to true
013:      newSSN = newWorker->getSocSecurity();
014:      currentSSN = Workers[j]->getSocSecurity();
015:
016:      //use insertion sort, starting from the last array index and
017:      //move down the array looking for the correct place to insert
018:      //newSSN
019:      while ((j>=0) && (stillLooking))
020:    {
021:      stringcomparison = strcmp(newSSN,currentSSN);
022:      if (stringcomparison == 0) 
023:        return 1;           //same ssn is illegal
024:      else if (stringcomparison < 0) 
025:        {
026:          Workers[j+1] = Workers[j]; //shift
027:          j--;  
028:          currentSSN = Workers[j]->getSocSecurity();
029:        }
030:      else
031:        stillLooking = 0;   //set flag to false
032:    }
033:      //upon leaving loop j+1 is the index where newWorker belongs
034:      
035:      Workers[j+1] = newWorker;
036:      ++numEmployees;
037:      return 0;
038:    }
039:}
040:
041:
\end{verbatim}
\section{Company1::findEmployee}
\subsection*{Specification:}
Finds an Employee in the Company1.
   Takes a char pointer SSN.
   Returns a pointer to Employee if found, NULL pointer otherwise.
\subsection*{Source-code:}
\begin{verbatim}
001:Employee* Company1::findEmployee(char* SSN)
002:    
003:{
004:    
005:  for(int i=0; Workers[i] != 0; i++)
006:    if(strcmp(Workers[i]->getSocSecurity(), SSN) == 0)
007:      return Workers[i];
008:
009:  return 0; //NULL pointer
010:}
011:
012:
\end{verbatim}
\section{Company1::deleteEmployee}
\subsection*{Specification:}
Deletes an Employee in the Company1.
  Takes a char pointer SSN.
  Returns an integer 0 if deleted successful, 1 if failed.
\subsection*{Source-code:}
\begin{verbatim}
001:int Company1::deleteEmployee(char* SSN)
002:  
003:{
004:  for (int i = 0; Workers[i] != 0; i++)
005:    { 
006:      if (strcmp(SSN, Workers[i]->getSocSecurity()) == 0)  
007:    { 
008:      delete Workers[i];
009:      Workers[i] = 0;
010:      i++;
011:      while (Workers[i] != 0)     
012:        { 
013:          Workers[i-1] = Workers[i];          
014:          i++;
015:        }
016:      return 0; 
017:    }
018:    }   
019:  return 1;                 /*not found*/
020:}
021:
022:
\end{verbatim}
\section{Company1::print}
\subsection*{Specification:}
Print instances of company1
\subsection*{Source-code:}
\begin{verbatim}
001:void Company1::print() 
002:
003:{
004:  int i;
005:
006:  for(i=0; Workers[i]!= 0; i++)
007:    // print each Employee from first to last
008:    {
009:      cout << "Employee " << i + 1 << "\n";
010:      Workers[i]->print();  // call employee print
011:    }
012:}
013:
014:
\end{verbatim}
%%\end{document}
