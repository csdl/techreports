%\documentstyle[11pt,/group/csdl/tex/definemargins,
%                       /group/csdl/tex/lmacros]{article} 

%          \begin{document}
%          \begin{center}
%          {\large\bf CSRS Experiment Results}\\
%          \end{center}
%          \small 
\chapter {CSRS Experiment Results (ICS313): Group7 (EIAM)}
\small
	  

\begin{description}
\item [Method:] EIAM
\item [Group:] Group7
\item [Source:] Employee1
\item [Participants:] jungacta (Reviewer), vjender (Reviewer), atrang (Reviewer)
\end{description}
\section{Issue Lists}
\begin{enumerate}
\item {\it Issue\#210 (vjender)}
\begin{description}
\item [Subject:] macro declration
\item [Criticality:] Med
\item [Confidence-level:] Med
\item [Source-node:] Constant

\item [Lines:] 1-3

\item [Description:] The declaration should have been as follows
	     
		const int MAXLEN = 256;

		const int MAX\_EMPLOYEES = 100;
\end{description}
\item {\it Issue\#222 (atrang)}
\begin{description}
\item [Subject:] need to be Null terminated
\item [Criticality:] Hi
\item [Confidence-level:] Hi
\item [Source-node:] Employee::Employee

\item [Lines:] 5

\item [Description:] *socSecurity need to be NULL terminated or else when you use the print
function before using the setSocSecurity function, it will prints a whole
bunch of garbage.  eg. socSecurity[0] = 0;
\end{description}
\item {\it Issue\#230 (vjender)}
\begin{description}
\item [Subject:] incorrect function declaration
\item [Criticality:] Hi
\item [Confidence-level:] Hi
\item [Source-node:] Employee

\item [Lines:] 12

\item [Description:] the distructor function shold not be declared
as a virtual function.
\end{description}
\item {\it Issue\#234 (atrang)}
\begin{description}
\item [Subject:] bad range
\item [Criticality:] Hi
\item [Confidence-level:] Hi
\item [Source-node:] Employee::setName

\item [Lines:] 11

\item [Description:] the allocated memory should have an extra slot to hold the NULL char.
eg. name=new char[len+1];
\end{description}
\item {\it Issue\#238 (vjender)}
\begin{description}
\item [Subject:] unnecessary code in employee destructor
\item [Criticality:] Low
\item [Confidence-level:] Hi
\item [Source-node:] Employee::\~Employee

\item [Lines:] 7-8

\item [Description:] the assignment statements in this function
 are unnecessary. since the destructor function of a calss deletes all the
variables associated with it, it makes no sense to assign something to a
variable which is going to be deleted after the execution of this function.
\end{description}
\item {\it Issue\#242 (jungacta)}
\begin{description}
\item [Subject:] while condition
\item [Criticality:] Hi
\item [Confidence-level:] Med
\item [Source-node:] Employee::setSocSecurity

\item [Lines:] 27

\item [Description:] while condition n-- provides no escape from
the while loop.
\end{description}
\item {\it Issue\#246 (atrang)}
\begin{description}
\item [Subject:] extra not needed
\item [Criticality:] Med
\item [Confidence-level:] Med
\item [Source-node:] Employee::\~Employee

\item [Lines:] 7

\item [Description:] when a destructor is called, the instant is usually removed so name=0 is not
require.
\end{description}
\item {\it Issue\#250 (atrang)}
\begin{description}
\item [Subject:] extra not needed
\item [Criticality:] Med
\item [Confidence-level:] Med
\item [Source-node:] Employee::\~Employee

\item [Lines:] 8

\item [Description:] same as issue \# 246 or line 7
\end{description}
\item {\it Issue\#254 (atrang)}
\begin{description}
\item [Subject:] wrong test case
\item [Criticality:] Hi
\item [Confidence-level:] Hi
\item [Source-node:] Employee::setNumDependents

\item [Lines:] 4

\item [Description:] if ((newNumDependents {\tt >} 0) \&\& (newNumDependents {\tt <} 11)

should be

if ((newNumDependents {\tt >}= 0) \&\& (newNumDependents {\tt <} 11)

since 0 is an accepted integer
\end{description}
\item {\it Issue\#258 (jungacta)}
\begin{description}
\item [Subject:] buffer allocation
\item [Criticality:] Hi
\item [Confidence-level:] Hi
\item [Source-node:] Employee::\~Employee

\item [Lines:] 7

\item [Description:] delete[] name de-allocates memory that would
be used for name array. There is no memory location initialize name to 0.
\end{description}
\item {\it Issue\#262 (vjender)}
\begin{description}
\item [Subject:] constants not used, insted arbitrary numbers are
used
\item [Criticality:] Med
\item [Confidence-level:] Hi
\item [Source-node:] Employee::Employee

\item [Lines:] 4-5

\item [Description:] In declration of the buffer for the name and 
social security number, Constants defined in the the constant function must
be used.

other wise this program beomes difficult ot modify.  a hard code and the
program will not work for the name length greater than 1.
\end{description}
\item {\it Issue\#266 (jungacta)}
\begin{description}
\item [Subject:] buffer allocation
\item [Criticality:] Hi
\item [Confidence-level:] Hi
\item [Source-node:] Employee::\~Employee

\item [Lines:] 8

\item [Description:] delete[] socSecurity de-allocates memory thta
would be used for socSecurity. There is no memory location to initialize
socSecurity to 0.
\end{description}
\item {\it Issue\#270 (atrang)}
\begin{description}
\item [Subject:] wrong encompass
\item [Criticality:] Hi
\item [Confidence-level:] Hi
\item [Source-node:] Employee::setSocSecurity

\item [Lines:] 11-13

\item [Description:] The if on line 11 should encompass the return 1 on line 13 also or else, 1 if
return everytime i != 3 \&\& i != 6
\end{description}
\item {\it Issue\#278 (vjender)}
\begin{description}
\item [Subject:] illegal deletion and creation of buffer.
\item [Criticality:] Med
\item [Confidence-level:] Hi
\item [Source-node:] Employee::setName

\item [Lines:] 10-11

\item [Description:] the deletion and creation of the buffer for 
private (or protected ) members must be done by the constructor and
destructor functions respectively.

other member functions must not do the job which is expected to be done by
constructors and destructors.


CONSEQUENCES: I cant think of any situation where this code can bomb.
\end{description}
\item {\it Issue\#284 (atrang)}
\begin{description}
\item [Subject:] wrong test case
\item [Criticality:] Hi
\item [Confidence-level:] Hi
\item [Source-node:] Employee::setSocSecurity

\item [Lines:] 11

\item [Description:] the \&\& should be || in line 11
\end{description}
\item {\it Issue\#288 (vjender)}
\begin{description}
\item [Subject:] improper null ternmination.
\item [Criticality:] Hi
\item [Confidence-level:] Hi
\item [Source-node:] Employee::setName

\item [Lines:] 16

\item [Description:] the assignment is wrong as the memory location
name[len] is outside the boundry of the array name[].

this statement is not required if the parameter passed to this function is
terminated by a NULL terminator, then strcpy function can be used instead of
the for loop.
\end{description}
\item {\it Issue\#296 (atrang)}
\begin{description}
\item [Subject:] need NULL terminated
\item [Criticality:] Hi
\item [Confidence-level:] Hi
\item [Source-node:] Employee::setSocSecurity

\item [Lines:] 24-29

\item [Description:] string socSecurity needs to be NULL terminated.
\end{description}
\item {\it Issue\#300 (jungacta)}
\begin{description}
\item [Subject:] new mem allocator
\item [Criticality:] Hi
\item [Confidence-level:] Med
\item [Source-node:] Company1

\item [Lines:] 5

\item [Description:] need to allocate memory for Workers array with
new.
\end{description}
\item {\it Issue\#304 (atrang)}
\begin{description}
\item [Subject:] wrong range
\item [Criticality:] Hi
\item [Confidence-level:] Hi
\item [Source-node:] Company1::Company1

\item [Lines:] 6

\item [Description:] should be: for (i = MAX\_EMPLOYEES-1; i {\tt >}= 0; i--) since workers[0] is also
used.
\end{description}
\item {\it Issue\#308 (jungacta)}
\begin{description}
\item [Subject:] private or public?
\item [Criticality:] Hi
\item [Confidence-level:] Med
\item [Source-node:] Company1

\item [Lines:] 7

\item [Description:] sort function should be made public so that
it can be accessed by the derived classes.
\end{description}
\item {\it Issue\#312 (atrang)}
\begin{description}
\item [Subject:] wrong coding
\item [Criticality:] Hi
\item [Confidence-level:] Hi
\item [Source-node:] Company1::\~Company1

\item [Lines:] 4

\item [Description:] VERY WRONG !!  workers is a static array of ptrs must delete each index
separately.
\end{description}
\item {\it Issue\#314 (jungacta)}
\begin{description}
\item [Subject:] virtual
\item [Criticality:] Hi
\item [Confidence-level:] Hi
\item [Source-node:] Company1

\item [Lines:] 14

\item [Description:] print function should be made virtual in order
for the correct implementation of the function to be executed.
\end{description}
\item {\it Issue\#320 (atrang)}
\begin{description}
\item [Subject:] extra not needed
\item [Criticality:] Med
\item [Confidence-level:] Med
\item [Source-node:] Company1::\~Company1

\item [Lines:] 5

\item [Description:] since instance are destroy after destrutor are call, static var will be
destroy anyway.
\end{description}
\item {\it Issue\#324 (jungacta)}
\begin{description}
\item [Subject:] Pointer
\item [Criticality:] Hi
\item [Confidence-level:] Low
\item [Source-node:] Employee::setNumDependents

\item [Lines:] 6

\item [Description:] numDependants should be a pointer so that
value of numDependants will be updated when function is exited.
\end{description}
\item {\it Issue\#330 (vjender)}
\begin{description}
\item [Subject:] incorrect logical operation used in if statement
\item [Criticality:] Hi
\item [Confidence-level:] Hi
\item [Source-node:] Employee::setAge

\item [Lines:] 4

\item [Description:] the correct statment must be as follows
if((newAge {\tt >}=17) \&\& (newAge {\tt <}= 55)
\end{description}
\item {\it Issue\#338 (vjender)}
\begin{description}
\item [Subject:] improper if statement
\item [Criticality:] Hi
\item [Confidence-level:] Hi
\item [Source-node:] Employee::setNumDependents

\item [Lines:] 4

\item [Description:] the correct if statement must be as follows
if( (newNumDependents {\tt >}= 0) \&\& ( newNumDependets {\tt <} 11))
\end{description}
\item {\it Issue\#342 (atrang)}
\begin{description}
\item [Subject:] one left standing
\item [Criticality:] Hi
\item [Confidence-level:] Hi
\item [Source-node:] Company1::deleteEmployee

\item [Lines:] 6-17

\item [Description:] when found, the item is delete and the rest are move down, BUT the last one
is not deleted, that is, there are two copies of the last one now.
\end{description}
\item {\it Issue\#346 (vjender)}
\begin{description}
\item [Subject:] output not properly formatted
\item [Criticality:] Low
\item [Confidence-level:] Hi
\item [Source-node:] Employee::print

\item [Lines:] 4-7

\item [Description:] all the output appears in one continuous
line, there must be newline charecter provided at the end of every cout
statement

eg: cout {\tt <}{\tt <} "Name : " {\tt <}{\tt <} name {\tt <}{\tt <} '\\n' ;
\end{description}
\item {\it Issue\#350 (vjender)}
\begin{description}
\item [Subject:] private member violation.
\item [Criticality:] Med
\item [Confidence-level:] Hi
\item [Source-node:] Employee::getSocSecurity

\item [Lines:] 4

\item [Description:] this function is retuing a pointer which is a
private member of the class this is clearly a violation as it gives access of
private members(or in this case protected members) to the outside world.
\end{description}
\item {\it Issue\#354 (vjender)}
\begin{description}
\item [Subject:] incorrect termination condtion in for loop
\item [Criticality:] Hi
\item [Confidence-level:] Hi
\item [Source-node:] Company1::Company1

\item [Lines:] 6

\item [Description:] the correct for loop must be as follows

for(i=MAX\_EMPLOYEES-1; i {\tt >}=0; i--)
\end{description}
\item {\it Issue\#358 (vjender)}
\begin{description}
\item [Subject:] extra unnecessay code.
\item [Criticality:] Low
\item [Confidence-level:] Hi
\item [Source-node:] Company1::\~Company1

\item [Lines:] 5

\item [Description:] the assignment here is nunecessay and serves
no purpose.
\end{description}
\item {\it Issue\#362 (vjender)}
\begin{description}
\item [Subject:] boundry of the array violated
\item [Criticality:] Hi
\item [Confidence-level:] Hi
\item [Source-node:] Company1::addEmployee

\item [Lines:] 27-28

\item [Description:] when the j=0 case is being tested in the loop
 decrementing of j makes the value of j = -1; in the next
 statement the Workers[j]-{\tt >}getSocSecurity will crash as
 value j will be -1 in this case
\end{description}
\item {\it Issue\#366 (vjender)}
\begin{description}
\item [Subject:] incorrect  for loop
\item [Criticality:] Hi
\item [Confidence-level:] Hi
\item [Source-node:] Company1::findEmployee

\item [Lines:] 5

\item [Description:] there  is no testing for the i value in the
for statement, the correct statement must be as follows.

for ( i = 0;(( Workeres[i] !=0) \&\&( i {\tt <} MAX\_EMPLOYEES)); i++)
\end{description}
\item {\it Issue\#370 (jungacta)}
\begin{description}
\item [Subject:] purpose of destructors
\item [Criticality:] Low
\item [Confidence-level:] Med
\item [Source-node:] Company1::\~Company1

\item [Lines:] 5

\item [Description:] Destructors are not used for initializations.
\end{description}
\item {\it Issue\#374 (vjender)}
\begin{description}
\item [Subject:] incorrect for loop
\item [Criticality:] Hi
\item [Confidence-level:] Hi
\item [Source-node:] Company1::deleteEmployee

\item [Lines:] 4

\item [Description:] the correct for loop must be as follows,

for(int i = 0;((Workers[i] !=0) \&\&( i{\tt <} MAX\_EMPLOYEES); i++)
\end{description}
\item {\it Issue\#378 (vjender)}
\begin{description}
\item [Subject:] incorrect while loop
\item [Criticality:] Hi
\item [Confidence-level:] Hi
\item [Source-node:] Company1::deleteEmployee

\item [Lines:] 11

\item [Description:] the correct while loop must be as follows

while((Workers[i] !=0) \&\&( i {\tt <} MAX\_EMPLOYEES))
\end{description}
\item {\it Issue\#382 (vjender)}
\begin{description}
\item [Subject:] incorrect for loop
\item [Criticality:] Hi
\item [Confidence-level:] Hi
\item [Source-node:] Company1::print

\item [Lines:] 6

\item [Description:] the correct for loop must be as follows

for(i=0;( (Workers[i] !=0) \&\& ( i {\tt <} MAX\_EMPLOYEES)); i++)
\end{description}
\end{enumerate}
\section{Review Metrics}
\begin{table}[hb]
\begin{center}
\begin{tabular}{|l|l|l|l|l|}
\hline
Participant & Start-time & End-time & Elapsed-time & Total Busy-time \\
\hline
atrang & Mar 22, 1995 18:40:18 & Mar 22, 1995 20:17:30 & 1:37:12 & 1:30:12 \\
vjender & Mar 22, 1995 18:38:41 & Mar 22, 1995 21:04:03 & 2:25:22 & 2:19:29 \\
jungacta & Mar 22, 1995 18:43:55 & Mar 22, 1995 20:55:30 & 2:11:35 & 1:56:20 \\
\hline
\end{tabular}
\end{center}
\caption{Review Session}
\end{table}


\begin{table}[hb]
\begin{center}
\begin{tabular}{|l|l|l|l|}
\hline
Source & atrang & vjender & jungacta\\
\hline
(192)Company1::Company1 & 257 & 254 & 547\\
(176)Employee::Employee & 627 & 1884 & 173\\
(194)Company1::\~Company1 & 340 & 141 & 351\\
(178)Employee::\~Employee & 278 & 462 & 411\\
(196)Company1::addEmployee & 585 & 474 & 265\\
(180)Employee::setName & 287 & 2034 & 333\\
(198)Company1::findEmployee & 102 & 379 & 154\\
(182)Employee::setSocSecurity & 1074 & 647 & 950\\
(200)Company1::deleteEmployee & 379 & 556 & 435\\
(184)Employee::setAge & 100 & 376 & 439\\
(202)Company1::print & 190 & 295 & 339\\
(186)Employee::setNumDependents & 270 & 229 & 254\\
(170)Constant & 455 & 595 & 324\\
(188)Employee::print & 70 & 349 & 165\\
(172)Employee & 180 & 2260 & 763\\
(190)Employee::getSocSecurity & 43 & 369 & 204\\
(174)Company1 & 145 & 827 & 699\\
\hline
\end{tabular}
\end{center}
\caption{Review Time}
\end{table}


\begin{table}[hb]
\begin{center}
\begin{tabular}{|l|l|l|l|l|}
\hline
Source & atrang & vjender & jungacta & OK \\
\hline
Company1::Company1 & 304 (=1) & 354 (=1) &  & 304=354\\
Employee::Employee & 222 (=1) & 262 (=1) &  & 222\\
Company1::\~Company1 & 312,320 (=2) & 358 (=1) & 370 (=1) & 312 \\
Employee::\~Employee & 246,250 (=2) & 238 (=1) & 258,266 (=2) & \\
Company1::addEmployee &  & 362 (=1) &  & 362\\
Employee::setName & 234 (=1) & 278,288 (=2) &  & 234=288 \\
Company1::findEmployee &  & 366 (=1) &  & 366\\
Employee::setSocSecurity & 270,284,296 (=3) &  & 242 (=1) & 270,284,296\\
Company1::deleteEmployee & 342 (=1) & 374,378 (=2) &  & 342,374,378\\
Employee::setAge &  & 330 (=1) &  & \\
Company1::print &  & 382 (=1) &  & 382\\
Employee::setNumDependents & 254 (=1) & 338 (=1) & 324 (=1) & 254=338\\
Constant &  & 210 (=1) &  & \\
Employee::print &  & 346 (=1) &  & \\
Employee &  & 230 (=1) &  & \\
Employee::getSocSecurity &  & 350 (=1) &  & 350\\
Company1 &  &  & 300,308,314 (=3) & \\
\hline
\end{tabular}
\caption{Source node v.s Issue node}
\end{center}
\end{table}

%\end{document}
