%\documentstyle[11pt,/group/csdl/tex/definemargins,
%                       /group/csdl/tex/lmacros]{article} 

%          \begin{document}
%          \begin{center}
%          {\large\bf CSRS Experiment Results}\\
%          \end{center}
%          \small 
\chapter {CSRS Experiment Results (ICS313): Group4 (EIAM)}
\small
	  

\begin{description}
\item [Method:] EIAM
\item [Group:] Group4
\item [Source:] Employee1
\item [Participants:] eugenio (Reviewer), rshadian (Reviewer), das (Reviewer)
\end{description}
\section{Issue Lists}
\begin{enumerate}
\item {\it Issue\#210 (das)}
\begin{description}
\item [Subject:] age type should be unsigned
\item [Criticality:] Low
\item [Confidence-level:] Hi
\item [Source-node:] Employee

\item [Lines:] 7

\item [Description:] age can never be a negative number so 
unsigned int would be better.
\end{description}
\item {\it Issue\#214 (eugenio)}
\begin{description}
\item [Subject:] space allocation
\item [Criticality:] Hi
\item [Confidence-level:] Hi
\item [Source-node:] Employee::setName

\item [Lines:] 11

\item [Description:] space allocation does not include cell for
NULL terminator.
\end{description}
\item {\it Issue\#218 (rshadian)}
\begin{description}
\item [Subject:] Assigning value to deleted pointer
\item [Criticality:] Hi
\item [Confidence-level:] Med
\item [Source-node:] Employee::\~Employee

\item [Lines:] 7-8

\item [Description:] Attempt to assign null to a pointer that has
already been deleted
\end{description}
\item {\it Issue\#220 (eugenio)}
\begin{description}
\item [Subject:] use of private types
\item [Criticality:] Low
\item [Confidence-level:] Med
\item [Source-node:] Company1

\item [Lines:] 7

\item [Description:] this function does not need to be private
\end{description}
\item {\it Issue\#226 (eugenio)}
\begin{description}
\item [Subject:] terminology
\item [Criticality:] Hi
\item [Confidence-level:] Med
\item [Source-node:] Employee

\item [Lines:] 4-5

\item [Description:] should this term be called 'private`?
\end{description}
\item {\it Issue\#230 (rshadian)}
\begin{description}
\item [Subject:] allocation of space for variable "name"
\item [Criticality:] Hi
\item [Confidence-level:] Hi
\item [Source-node:] Employee::setName

\item [Lines:] 11

\item [Description:] Not enough space allocated for name. Strlen
only returns the length of the string, not including the /0 character at the
end.  Should be new char[len+1].
\end{description}
\item {\it Issue\#234 (eugenio)}
\begin{description}
\item [Subject:] setting pointer to NULL
\item [Criticality:] Hi
\item [Confidence-level:] Med
\item [Source-node:] Employee::Employee

\item [Lines:] 5

\item [Description:] the pointer should be assigned to NULL
\end{description}
\item {\it Issue\#238 (eugenio)}
\begin{description}
\item [Subject:] while loop ends early
\item [Criticality:] Hi
\item [Confidence-level:] Hi
\item [Source-node:] Employee::setSocSecurity

\item [Lines:] 27

\item [Description:] the loop ends before the null is copied into
socsecurity
\end{description}
\item {\it Issue\#242 (rshadian)}
\begin{description}
\item [Subject:] Invalid condition in if statement
\item [Criticality:] Hi
\item [Confidence-level:] Hi
\item [Source-node:] Employee::setSocSecurity

\item [Lines:] 11

\item [Description:] newNum[i] will never be BOTH less than 0 and
greater than 9 since 9 is greater than 0.  The statement will never be
evaluated. Should be || instead of \&\&
\end{description}
\item {\it Issue\#244 (eugenio)}
\begin{description}
\item [Subject:] boundaries
\item [Criticality:] Med
\item [Confidence-level:] Hi
\item [Source-node:] Employee::setNumDependents

\item [Lines:] 4

\item [Description:] newNumdependents should include 0 and be up to 11
\end{description}
\item {\it Issue\#250 (eugenio)}
\begin{description}
\item [Subject:] cell[0] does not get init
\item [Criticality:] Hi
\item [Confidence-level:] Hi
\item [Source-node:] Company1::Company1

\item [Lines:] 6

\item [Description:] i goes down to 1, never 0
\end{description}
\item {\it Issue\#254 (das)}
\begin{description}
\item [Subject:] loop index  decrement done in wrong place.
\item [Criticality:] Hi
\item [Confidence-level:] Hi
\item [Source-node:] Company1::addEmployee

\item [Lines:] 27

\item [Description:] the value should change just before leaving
the loop
\end{description}
\item {\it Issue\#256 (rshadian)}
\begin{description}
\item [Subject:] null allocation not considered in loop
\item [Criticality:] Hi
\item [Confidence-level:] Hi
\item [Source-node:] Employee::setSocSecurity

\item [Lines:] 27

\item [Description:] strlen returns length of string but does not
include the terminating null.  The loop will end before the null is copied
and leaves garbage in the last array slot.
\end{description}
\item {\it Issue\#262 (rshadian)}
\begin{description}
\item [Subject:] invalid condition in if statement
\item [Criticality:] Med
\item [Confidence-level:] Hi
\item [Source-node:] Employee::setNumDependents

\item [Lines:] 4

\item [Description:] According to the specification,
newNumDependents should be between 0 and 10 but the if statement will not be
evaluated if the variable is equal to zero.  should read newNumDependents{\tt >}=0
\end{description}
\item {\it Issue\#266 (eugenio)}
\begin{description}
\item [Subject:] boundaries
\item [Criticality:] Hi
\item [Confidence-level:] Med
\item [Source-node:] Company1::findEmployee

\item [Lines:] 5

\item [Description:] the loop should go from 0 to numemployees
because this is implemented as an array not a linked list.
\end{description}
\item {\it Issue\#270 (eugenio)}
\begin{description}
\item [Subject:] loop boundaries
\item [Criticality:] Hi
\item [Confidence-level:] Med
\item [Source-node:] Company1::deleteEmployee

\item [Lines:] 4

\item [Description:] the loop should go from 0 to numemployees
\end{description}
\item {\it Issue\#274 (rshadian)}
\begin{description}
\item [Subject:] output format
\item [Criticality:] Low
\item [Confidence-level:] Low
\item [Source-node:] Employee::print

\item [Lines:] 4-7

\item [Description:] no end of line in cout statements.
\end{description}
\item {\it Issue\#276 (eugenio)}
\begin{description}
\item [Subject:] uaw of delete function
\item [Criticality:] Hi
\item [Confidence-level:] Hi
\item [Source-node:] Company1::deleteEmployee

\item [Lines:] 8

\item [Description:] this call should be deleting an employee
instance not a worker.
\end{description}
\item {\it Issue\#282 (eugenio)}
\begin{description}
\item [Subject:] loop boundaries
\item [Criticality:] Hi
\item [Confidence-level:] Med
\item [Source-node:] Company1::deleteEmployee

\item [Lines:] 11

\item [Description:] again, the boundaries should go up till numemployees
\end{description}
\item {\it Issue\#286 (eugenio)}
\begin{description}
\item [Subject:] loop boundaries
\item [Criticality:] Hi
\item [Confidence-level:] Med
\item [Source-node:] Company1::print

\item [Lines:] 6

\item [Description:] again, the boundaries go from 0 up till numemployees-1
\end{description}
\item {\it Issue\#290 (das)}
\begin{description}
\item [Subject:] 
\item [Criticality:] 
\item [Confidence-level:] 
\item [Source-node:] Employee::setSocSecurity

\item [Lines:] 9

\item [Description:] 
\end{description}
\item {\it Issue\#292 (rshadian)}
\begin{description}
\item [Subject:] invalid test in for loop
\item [Criticality:] Hi
\item [Confidence-level:] Hi
\item [Source-node:] Company1::Company1

\item [Lines:] 6

\item [Description:] the loop wil exit before reaching the first
array slot and leave garbage there.  Test should be i{\tt >}=0.
\end{description}
\item {\it Issue\#294 (eugenio)}
\begin{description}
\item [Subject:] other private members should be reset
\item [Criticality:] Med
\item [Confidence-level:] Hi
\item [Source-node:] Employee::\~Employee

\item [Lines:] 

\item [Description:] age and numdependents should be set to their
default values as well
\end{description}
\item {\it Issue\#300 (eugenio)}
\begin{description}
\item [Subject:] boundaries of newage
\item [Criticality:] Low
\item [Confidence-level:] Hi
\item [Source-node:] Employee::setAge

\item [Lines:] 4

\item [Description:] if statement does not include 17 and 55 as
valid values
\end{description}
\item {\it Issue\#304 (rshadian)}
\begin{description}
\item [Subject:] if conditon
\item [Criticality:] Med
\item [Confidence-level:] Hi
\item [Source-node:] Company1::addEmployee

\item [Lines:] 7

\item [Description:] if statement does not error check for values
greater than MAX\_EMPLOYEES.
\end{description}
\item {\it Issue\#308 (eugenio)}
\begin{description}
\item [Subject:] use of print()
\item [Criticality:] Hi
\item [Confidence-level:] Low
\item [Source-node:] Company1::print

\item [Lines:] 10

\item [Description:] I don't think that this is the correct way to
call the employee class print function.  I think the arrow should be a dot.
\end{description}
\item {\it Issue\#312 (das)}
\begin{description}
\item [Subject:] loop index increment not required
\item [Criticality:] Hi
\item [Confidence-level:] Hi
\item [Source-node:] Company1::deleteEmployee

\item [Lines:] 14

\item [Description:] loop index increment not required
\end{description}
\item {\it Issue\#316 (rshadian)}
\begin{description}
\item [Subject:] improper call to delete
\item [Criticality:] Hi
\item [Confidence-level:] Med
\item [Source-node:] Company1::deleteEmployee

\item [Lines:] 8

\item [Description:] the call to delete will send the current node
back to memory and cut it off from the rest of array leaving a break in the
memory block.
\end{description}
\item {\it Issue\#320 (rshadian)}
\begin{description}
\item [Subject:] while loop condition is invalid
\item [Criticality:] Hi
\item [Confidence-level:] Hi
\item [Source-node:] Company1::deleteEmployee

\item [Lines:] 9 11

\item [Description:] while loop checks to see if Workers[i] is not
equal to null, but becuase of previous assignment statement, Workers[i] will
always be null and while loop won't be executed.
\end{description}
\end{enumerate}
\section{Review Metrics}
\begin{table}[hb]
\begin{center}
\begin{tabular}{|l|l|l|l|l|}
\hline
Participant & Start-time & End-time & Elapsed-time & Total Busy-time \\
\hline
das & Apr 08, 1995 15:04:31 & Apr 08, 1995 16:41:50 & 1:37:19 & 1:21:3 \\
rshadian & Apr 08, 1995 15:04:32 & Apr 08, 1995 16:51:27 & 1:46:55 & 1:39:3 \\
eugenio & Apr 08, 1995 15:03:42 & Apr 08, 1995 16:19:10 & 1:15:28 & 1:7:17 \\
\hline
\end{tabular}
\end{center}
\caption{Review Session}
\end{table}


\begin{table}[hb]
\begin{center}
\begin{tabular}{|l|l|l|l|}
\hline
Source & das & rshadian & eugenio\\
\hline
(176)Employee::Employee & 140 & 99 & 224\\
(192)Company1::Company1 & 99 & 287 & 165\\
(178)Employee::\~Employee & 19 & 272 & 226\\
(194)Company1::\~Company1 & 13 & 66 & 52\\
(180)Employee::setName & 546 & 674 & 413\\
(196)Company1::addEmployee & 782 & 885 & 484\\
(182)Employee::setSocSecurity & 1119 & 939 & 331\\
(198)Company1::findEmployee & 185 & 159 & 210\\
(184)Employee::setAge & 111 & 99 & 219\\
(200)Company1::deleteEmployee & 580 & 909 & 386\\
(170)Constant & 138 & 48 & 77\\
(186)Employee::setNumDependents & 237 & 291 & 171\\
(202)Company1::print & 204 & 179 & 314\\
(172)Employee & 334 & 256 & 337\\
(188)Employee::print & 38 & 235 & 49\\
(174)Company1 & 116 & 410 & 240\\
(190)Employee::getSocSecurity & 54 & 114 & 44\\
\hline
\end{tabular}
\end{center}
\caption{Review Time}
\end{table}


\begin{table}[hb]
\begin{center}
\begin{tabular}{|l|l|l|l|l|}
\hline
Source & das & rshadian & eugenio & OK\\
\hline
Employee::Employee &  &  & 234 (=1) & \\
Company1::Company1 &  & 292 (=1) & 250 (=1) & 292=250\\
Employee::\~Employee &  & 218 (=1) & 294 (=1) & \\
Company1::\~Company1 &  &  &  & \\
Employee::setName &  & 230 (=1) & 214 (=1) & 230=214\\
Company1::addEmployee & 254 (=1) & 304 (=1) &  & \\
Employee::setSocSecurity & 290 (=1) & 242,256 (=2) & 238 (=1) & 242,256=238\\
Company1::findEmployee &  &  & 266 (=1) & \\
Employee::setAge &  &  & 300 (=1) & \\
Company1::deleteEmployee & 312 (=1) & 316,320 (=2) & 270,276,282 (=3) & \\
Constant &  &  &  & \\
Employee::setNumDependents &  & 262 (=1) & 244 (=1) & 262=244\\
Company1::print &  &  & 286,308 (=2) & \\
Employee & 210 (=1) &  & 226 (=1) & \\
Employee::print &  & 274 (=1) &  & \\
Company1 &  &  & 220 (=1) & \\
Employee::getSocSecurity &  &  &  & \\
\hline
\end{tabular}
\caption{Source node v.s Issue node}
\end{center}
\end{table}

%%\end{document}
