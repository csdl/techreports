%\documentstyle[11pt,/group/csdl/tex/definemargins,
%                       /group/csdl/tex/lmacros]{article} 

%          \begin{document}
%          \begin{center}
%          {\large\bf CSRS Experiment Results}\\
%          \end{center}
%          \small 
	  
\chapter {CSRS Experiment Results (ICS313): Group2 (EIAM)}
\small

\begin{description}
\item [Method:] EIAM
\item [Group:] Group2
\item [Source:] Employee1
\item [Participants:] ishisaka (Reviewer), hkim (Reviewer), jackfen (Reviewer)
\end{description}
\section{Issue Lists}
\begin{enumerate}
\item {\it Issue\#210 (jackfen)}
\begin{description}
\item [Subject:] redundent
\item [Criticality:] Hi
\item [Confidence-level:] Hi
\item [Source-node:] Employee::\~Employee

\item [Lines:] 7-8

\item [Description:] initalize unwanted memory space
\end{description}
\item {\it Issue\#214 (ishisaka)}
\begin{description}
\item [Subject:] for loop tests too much
\item [Criticality:] Med
\item [Confidence-level:] Hi
\item [Source-node:] Employee::setSocSecurity

\item [Lines:] 8

\item [Description:] for loop will go through 1 too many array
elements, might return falso value if end element is between 1-9.
\end{description}
\item {\it Issue\#216 (jackfen)}
\begin{description}
\item [Subject:] insufficient allocation of memory space for socSecurity
\item [Criticality:] Hi
\item [Confidence-level:] Hi
\item [Source-node:] Employee::setName

\item [Lines:] 11-12

\item [Description:] it lacks the space for terminating character null
\end{description}
\item {\it Issue\#226 (ishisaka)}
\begin{description}
\item [Subject:] while loop will go forever
\item [Criticality:] Hi
\item [Confidence-level:] Med
\item [Source-node:] Employee::setSocSecurity

\item [Lines:] 27

\item [Description:] while loop has no end condition! will keep on copying characters characters!
\end{description}
\item {\it Issue\#230 (hkim)}
\begin{description}
\item [Subject:] The range of numDependents
\item [Criticality:] Hi
\item [Confidence-level:] Hi
\item [Source-node:] Employee::setNumDependents

\item [Lines:] 4

\item [Description:] 0 is not included in the range of newNumDependents.
\end{description}
\item {\it default (jackfen)}
\begin{description}
\item [Subject:] 
\item [Criticality:] 
\item [Confidence-level:] 
\item [Source-node:] 

\item [Lines:] 

\item [Description:] 
\end{description}
\item {\it Issue\#242 (jackfen)}
\begin{description}
\item [Subject:] accesse to unallocated memory
\item [Criticality:] Hi
\item [Confidence-level:] Hi
\item [Source-node:] Employee::setName

\item [Lines:] 16

\item [Description:] name [len] is not allocated; only 0th to
(n-1)th are allocated
\end{description}
\item {\it Issue\#246 (hkim)}
\begin{description}
\item [Subject:] 
\item [Criticality:] 
\item [Confidence-level:] 
\item [Source-node:] Employee::setNumDependents

\item [Lines:] 

\item [Description:] 
\end{description}
\item {\it Issue\#252 (hkim)}
\begin{description}
\item [Subject:] Destructor of class
\item [Criticality:] Low
\item [Confidence-level:] Hi
\item [Source-node:] Company1::\~Company1

\item [Lines:] 5-6

\item [Description:] numEmployees is set to 0 to remove from memory.
\end{description}
\item {\it Issue\#256 (jackfen)}
\begin{description}
\item [Subject:] not take care of string delimitor
\item [Criticality:] Hi
\item [Confidence-level:] Hi
\item [Source-node:] Employee::setSocSecurity

\item [Lines:] 30

\item [Description:] socSecurity will not be a string because
missing string delimitor.  This will cause problem when use socSecurity especially when
applying strlen to socSecurity
\end{description}
\item {\it default (jackfen)}
\begin{description}
\item [Subject:] 
\item [Criticality:] 
\item [Confidence-level:] 
\item [Source-node:] 

\item [Lines:] 

\item [Description:] 
\end{description}
\item {\it Issue\#260 (hkim)}
\begin{description}
\item [Subject:] The range of array.
\item [Criticality:] Hi
\item [Confidence-level:] Hi
\item [Source-node:] Company1::Company1

\item [Lines:] 6

\item [Description:] i is greater than 0. i must include 0 in order
to store the information of the number of MAX-EMOLOYEES.
\end{description}
\item {\it Issue\#274 (jackfen)}
\begin{description}
\item [Subject:] missing 11th character
\item [Criticality:] Hi
\item [Confidence-level:] Hi
\item [Source-node:] Employee::setSocSecurity

\item [Lines:] 8

\item [Description:] the inequality restrict the access of 11th
character. So the 11th character is not inspected
\end{description}
\item {\it Issue\#278 (jackfen)}
\begin{description}
\item [Subject:] length of newNum is no checked: potential access
to outside of socSecurity array
\item [Criticality:] Hi
\item [Confidence-level:] Hi
\item [Source-node:] Employee::setSocSecurity

\item [Lines:] 27

\item [Description:] Since the length is newNum is not checked
before applied here, there may be access to memory outside of allocated array.  try to
have newNum of 15 character!  Not to mention that it did not meet the specification of
11 character of socSecurity.
\end{description}
\item {\it Issue\#280 (ishisaka)}
\begin{description}
\item [Subject:] name not copied correctly
\item [Criticality:] Hi
\item [Confidence-level:] Low
\item [Source-node:] Employee::setName

\item [Lines:] 15

\item [Description:] name and newname pointers not referenced or incremented.  newname not being copied to
name.
\end{description}
\item {\it Issue\#288 (jackfen)}
\begin{description}
\item [Subject:] redundency
\item [Criticality:] Hi
\item [Confidence-level:] Hi
\item [Source-node:] Employee::setSocSecurity

\item [Lines:] 22

\item [Description:] this is always TRUE! because of the class
constructor already allocate space for socSecurity
\end{description}
\item {\it Issue\#292 (hkim)}
\begin{description}
\item [Subject:] Unused function.
\item [Criticality:] Low
\item [Confidence-level:] Hi
\item [Source-node:] Company1

\item [Lines:] 7

\item [Description:] This functions is not defined and unused.
\end{description}
\item {\it Issue\#304 (ishisaka)}
\begin{description}
\item [Subject:] invalid range checking
\item [Criticality:] Low
\item [Confidence-level:] Hi
\item [Source-node:] Employee::setNumDependents

\item [Lines:] 4

\item [Description:] will not accept values of 11 or 0.
\end{description}
\item {\it Issue\#310 (hkim)}
\begin{description}
\item [Subject:] The use of destructor.
\item [Criticality:] Low
\item [Confidence-level:] Hi
\item [Source-node:] Employee::\~Employee

\item [Lines:] 7-8

\item [Description:] name and socSecurity are set to 0.
\end{description}
\item {\it Issue\#316 (jackfen)}
\begin{description}
\item [Subject:] allows access to private variable
\item [Criticality:] Hi
\item [Confidence-level:] Hi
\item [Source-node:] Employee::getSocSecurity

\item [Lines:] 4

\item [Description:] by returning pointer to an object of a private
variable, it allows user to alternate the content of the private variable. An Absolute
NO NO !
\end{description}
\item {\it Issue\#320 (hkim)}
\begin{description}
\item [Subject:] The initialization of name
\item [Criticality:] Low
\item [Confidence-level:] Hi
\item [Source-node:] Employee::Employee

\item [Lines:] 4

\item [Description:] One array space is allocated and initialized
to 0.  It is not necessary to allocate one array space.
\end{description}
\item {\it Issue\#324 (jackfen)}
\begin{description}
\item [Subject:] no deletion of employee objects to which the array
of Employee pointers points
\item [Criticality:] Hi
\item [Confidence-level:] Hi
\item [Source-node:] Company1::\~Company1

\item [Lines:] 4

\item [Description:] it need to delete all Employee objects that
the array points before it can delete the array
\end{description}
\item {\it Issue\#328 (ishisaka)}
\begin{description}
\item [Subject:] possible infinte loop
\item [Criticality:] Med
\item [Confidence-level:] Med
\item [Source-node:] Company1::findEmployee

\item [Lines:] 5

\item [Description:] for loop will end if ssn is found however if
ssn is not found by the time the array is traversed totaly, for loop will look for ssn
in other memory space that is not within the array and keep looking.
\end{description}
\item {\it Issue\#332 (hkim)}
\begin{description}
\item [Subject:] The initialization of the constructor.
\item [Criticality:] Low
\item [Confidence-level:] Hi
\item [Source-node:] Employee::Employee

\item [Lines:] 5

\item [Description:] socSecurity is not initialized.
\end{description}
\item {\it Issue\#336 (ishisaka)}
\begin{description}
\item [Subject:] for loop will create 2 less than max amount of
worker slots
\item [Criticality:] Low
\item [Confidence-level:] Med
\item [Source-node:] Company1::Company1

\item [Lines:] 6

\item [Description:] for loop will only create 1-255 worker instances (254 worker instances) instead of
amount defined in constant
\end{description}
\item {\it Issue\#342 (ishisaka)}
\begin{description}
\item [Subject:] memory space not allocated
\item [Criticality:] Hi
\item [Confidence-level:] Low
\item [Source-node:] Company1::Company1

\item [Lines:] 7

\item [Description:] memory space not allocated for the worker.
\end{description}
\item {\it Issue\#350 (jackfen)}
\begin{description}
\item [Subject:] allows duplication of employee in company list
\item [Criticality:] Hi
\item [Confidence-level:] Hi
\item [Source-node:] Company1::addEmployee

\item [Lines:] 30-31

\item [Description:] try the case if currentSSN == newSSN.  It will
stop the loop the allow insertion on the lines followed
\end{description}
\item {\it Issue\#356 (hkim)}
\begin{description}
\item [Subject:] 
\item [Criticality:] 
\item [Confidence-level:] 
\item [Source-node:] Company1::addEmployee

\item [Lines:] 

\item [Description:] 
\end{description}
\item {\it Issue\#362 (jackfen)}
\begin{description}
\item [Subject:] has potential of fever loop
\item [Criticality:] Hi
\item [Confidence-level:] Hi
\item [Source-node:] Company1::findEmployee

\item [Lines:] 5

\item [Description:] when the worker array is full, there will be
NO null pointer at the end of the array.  So the loop will never stop
\end{description}
\item {\it Issue\#366 (jackfen)}
\begin{description}
\item [Subject:] has potential to access to unallocated memory space
\item [Criticality:] Hi
\item [Confidence-level:] Hi
\item [Source-node:] Company1::findEmployee

\item [Lines:] 6

\item [Description:] When the Workers array is full, there will be
NO null pointers the end of array.  So the loop continue.  As consequence, line 6 will
access this memory space pointed by the the unallocated Workers[i] where i is greater
or equal to MAX\_EMPLOYEES
\end{description}
\item {\it Issue\#374 (jackfen)}
\begin{description}
\item [Subject:] has potential to have fever loop
\item [Criticality:] Hi
\item [Confidence-level:] Hi
\item [Source-node:] Company1::deleteEmployee

\item [Lines:] 4

\item [Description:] When Workers array is full, there will be No
null pointer at the end of the array to stop the loop
\end{description}
\item {\it Issue\#378 (jackfen)}
\begin{description}
\item [Subject:] has potential to access unallocated memory
\item [Criticality:] Hi
\item [Confidence-level:] Hi
\item [Source-node:] Company1::deleteEmployee

\item [Lines:] 6

\item [Description:] When the Workers array is full, the loop will
never be ended.  So it will access memory out the array of supposed length.
\end{description}
\item {\it Issue\#386 (jackfen)}
\begin{description}
\item [Subject:] may have fever loop
\item [Criticality:] Hi
\item [Confidence-level:] Hi
\item [Source-node:] Company1::deleteEmployee

\item [Lines:] 11

\item [Description:] When Workers array is full, there will be NO
null pointer to stop the loop
\end{description}
\item {\it Issue\#392 (jackfen)}
\begin{description}
\item [Subject:] May access to unallocated memory
\item [Criticality:] 
\item [Confidence-level:] Hi
\item [Source-node:] Company1::deleteEmployee

\item [Lines:] 13

\item [Description:] When Workers array is full, the loop will go
forever.  So it access memory outside array of supposed length.
\end{description}
\item {\it Issue\#398 (jackfen)}
\begin{description}
\item [Subject:] May have forever loop
\item [Criticality:] Hi
\item [Confidence-level:] Hi
\item [Source-node:] Company1::print

\item [Lines:] 6

\item [Description:] When Workers array is full, there will be no
Null pointers to stop the loop.
\end{description}
\item {\it Issue\#402 (jackfen)}
\begin{description}
\item [Subject:] May access to unallocated memory
\item [Criticality:] Hi
\item [Confidence-level:] Hi
\item [Source-node:] Company1::print

\item [Lines:] 10

\item [Description:] When Workers array is full, the loop will
never stop. Consequently, this line and the line in the for loop condition expression
accesses memory spaces out the array of supposed length.
\end{description}
\item {\it Issue\#406 (hkim)}
\begin{description}
\item [Subject:] The range of j
\item [Criticality:] Hi
\item [Confidence-level:] Hi
\item [Source-node:] Company1::addEmployee

\item [Lines:] 19

\item [Description:] j is set to 0. It should not include 0.
\end{description}
\end{enumerate}
\section{Review Metrics}
\begin{table}[hb]
\begin{center}
\begin{tabular}{|l|l|l|l|l|}
\hline
Participant & Start-time & End-time & Elapsed-time & Total Busy-time \\
\hline
jackfen & Mar 22, 1995 18:24:51 & Mar 22, 1995 20:52:30 & 2:27:39 & 2:10:39 \\
hkim & Mar 22, 1995 18:25:40 & Mar 22, 1995 20:55:18 & 2:29:38 & 2:18:25 \\
ishisaka & Mar 22, 1995 18:24:44 & Mar 22, 1995 20:53:41 & 2:28:57 & 2:28:57 \\
\hline
\end{tabular}
\end{center}
\caption{Review Session}
\end{table}


\begin{table}[hb]
\begin{center}
\begin{tabular}{|l|l|l|l|}
\hline
Source & jackfen & hkim & ishisaka\\
\hline
(176)Employee::Employee & 1123 & 587 & 218\\
(192)Company1::Company1 & 3121 & 356 & 917\\
(178)Employee::\~Employee & 277 & 222 & 49\\
(194)Company1::\~Company1 & 308 & 357 & 42\\
(180)Employee::setName & 759 & 552 & 1470\\
(196)Company1::addEmployee & 902 & 1728 & 1273\\
(182)Employee::setSocSecurity & 2149 & 1051 & 1363\\
(198)Company1::findEmployee & 855 & 483 & 591\\
(184)Employee::setAge & 411 & 117 & 590\\
(200)Company1::deleteEmployee & 776 & 971 & 961\\
(170)Constant & 27 & 4 & 755\\
(186)Employee::setNumDependents & 105 & 656 & 308\\
(202)Company1::print & 454 & 132 & 432\\
(172)Employee & 2570 & 379 & 7200\\
(188)Employee::print & 85 & 90 & 178\\
(174)Company1 & 1258 & 513 & 890\\
(190)Employee::getSocSecurity & 226 & 44 & 126\\
\hline
\end{tabular}
\end{center}
\caption{Review Time}
\end{table}


\begin{table}[hb]
\begin{center}
\begin{tabular}{|l|l|l|l|l|}
\hline
Source & jackfen & hkim & ishisaka & OK\\
\hline
Employee::Employee &  & 320,332 (=2) &  & 320\\
Company1::Company1 &  & 260 (=1) & 336,342 (=2) & 260=336\\
Employee::\~Employee & 210 (=1) & 310 (=1) &  & \\
Company1::\~Company1 & 324 (=1) & 252 (=1) &  & 324\\
Employee::setName & 216,242 (=2) &  & 280 (=1) & 216,242\\
Company1::addEmployee & 350 (=1) & 356,406 (=2) &  & 350\\
Employee::setSocSecurity & 256,274,278, &  & 214,226 (=2) & 256,278\\
                         & 288 (=4)    &   &              & \\            
Company1::findEmployee & 362,366 (=2) &  & 328 (=1) & 362=328,366 \\
Employee::setAge &  &  &  & \\
Company1::deleteEmployee & 374,378,386, &  &  & 374,386\\
                         & 392 (=4)    &   &   & \\            
Constant &  &  &  & \\
Employee::setNumDependents &  & 230,246 (=2) & 304 (=1) & 230=304\\
Company1::print & 398,402 (=2) &  &  & 398\\
Employee &  &  &  & \\
Employee::print &  &  &  & \\
Company1 &  & 292 (=1) &  & \\
Employee::getSocSecurity & 316 (=1) &  &  & 316 \\
\hline
\end{tabular}
\caption{Source node v.s Issue node}
\end{center}
\end{table}

%\end{document}
