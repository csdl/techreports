%\documentstyle[11pt,/group/csdl/tex/definemargins,
%                       /group/csdl/tex/lmacros]{article} 

%          \begin{document}
%          \begin{center}
%          {\large\bf CSRS Experiment Results}\\
%          \end{center}
%          \small 
	  
\chapter {CSRS Experiment Results (ICS313): Group5 (EIAM)}
\small

\begin{description}
\item [Method:] EIAM
\item [Group:] Group5
\item [Source:] Employee1
\item [Participants:] subin (Reviewer), ktakaki (Reviewer), ilee (Reviewer)
\end{description}
\section{Issue Lists}
\begin{enumerate}
\item {\it Issue\#214 (ilee)}
\begin{description}
\item [Subject:] invalid counter length
\item [Criticality:] Hi
\item [Confidence-level:] Hi
\item [Source-node:] Employee::setSocSecurity

\item [Lines:] 8

\item [Description:] if i {\tt <} 11, then only 10 characters will be
copy.  so it should be i {\tt <} 12.
\end{description}
\item {\it Issue\#218 (ktakaki)}
\begin{description}
\item [Subject:] wrong use of NULL termination
\item [Criticality:] Med
\item [Confidence-level:] Hi
\item [Source-node:] Employee::setName

\item [Lines:] 16

\item [Description:] the null terminated symbol is '\\0', not simply

0.
\end{description}
\item {\it Issue\#220 (subin)}
\begin{description}
\item [Subject:] incorrect class declaration
\item [Criticality:] Hi
\item [Confidence-level:] Hi
\item [Source-node:] Employee

\item [Lines:] 4

\item [Description:] this line should be private.  Company1 should
call this class as derived class.
\end{description}
\item {\it Issue\#226 (ktakaki)}
\begin{description}
\item [Subject:] not enough space is allocated to include the null 
terminator
\item [Criticality:] Hi
\item [Confidence-level:] Hi
\item [Source-node:] Employee::setName

\item [Lines:] 14-17

\item [Description:] the for loop copies the string correctly, but
the assignment of the null terminator erases the last string character.
\end{description}
\item {\it Issue\#228 (ilee)}
\begin{description}
\item [Subject:] initial value for i
\item [Criticality:] Hi
\item [Confidence-level:] Hi
\item [Source-node:] Company1::Company1

\item [Lines:] 6

\item [Description:] since  the for loop is checking for every {\tt >} 0,
the MAX\_EMPLOYEE should not be minus 1.  it should become i = MAX\_EMPLOYEE.
\end{description}
\item {\it Issue\#234 (subin)}
\begin{description}
\item [Subject:] incorrect class delaration
\item [Criticality:] Hi
\item [Confidence-level:] Hi
\item [Source-node:] Company1

\item [Lines:] 4

\item [Description:] this should be protected instead of private so
that these member variables become private to public function declaration.
\end{description}
\item {\it Issue\#236 (ktakaki)}
\begin{description}
\item [Subject:] ambiguous use of \&\&
\item [Criticality:] Hi
\item [Confidence-level:] Hi
\item [Source-node:] Employee::setSocSecurity

\item [Lines:] 11

\item [Description:] the expression is not parsed correctly, and
will lead to errors in scope.
\end{description}
\item {\it Issue\#242 (ktakaki)}
\begin{description}
\item [Subject:] the if condition is wrong
\item [Criticality:] Hi
\item [Confidence-level:] Hi
\item [Source-node:] Employee::setSocSecurity

\item [Lines:] 9

\item [Description:] the if condition is wrong, since it won't
allow valid ssn's to be entered.  The second and conidtion should be i!=7.
Also, the parsing for \&\& is ambiguous.
\end{description}
\item {\it Issue\#246 (ilee)}
\begin{description}
\item [Subject:] invalid array copy
\item [Criticality:] Hi
\item [Confidence-level:] Hi
\item [Source-node:] Company1::addEmployee

\item [Lines:] 26

\item [Description:] only one value will be changed in array and 
the value for Workers[j+1] will be recovered by Workers[j].  What happened to
the value of [j+2], [j+3]....
\end{description}
\item {\it Issue\#250 (ktakaki)}
\begin{description}
\item [Subject:] poor logic in if statement
\item [Criticality:] Hi
\item [Confidence-level:] Hi
\item [Source-node:] Employee::setSocSecurity

\item [Lines:] 15

\item [Description:] the if statement will be entered when it
shouldn't, leading to the rejection of valid ssn's entered by the user.
\end{description}
\item {\it Issue\#252 (subin)}
\begin{description}
\item [Subject:] max num of characters exceeds the limit
\item [Criticality:] Hi
\item [Confidence-level:] Hi
\item [Source-node:] Constant

\item [Lines:] 1

\item [Description:] max number of characters cannot execeeds the
limit of 255 characters by default.
\end{description}
\item {\it Issue\#260 (subin)}
\begin{description}
\item [Subject:] 
\item [Criticality:] 
\item [Confidence-level:] 
\item [Source-node:] Employee::Employee

\item [Lines:] 

\item [Description:] 
\end{description}
\item {\it Issue\#262 (ilee)}
\begin{description}
\item [Subject:] wrong position
\item [Criticality:] Hi
\item [Confidence-level:] Hi
\item [Source-node:] Company1::addEmployee

\item [Lines:] 35

\item [Description:] since j is started from the last array index, 
it should be Workers[j-1] instead of [j+1]
\end{description}
\item {\it Issue\#272 (ktakaki)}
\begin{description}
\item [Subject:] questionable if statement
\item [Criticality:] Med
\item [Confidence-level:] Low
\item [Source-node:] Employee::setSocSecurity

\item [Lines:] 22

\item [Description:] what is the value of socSecurity? When is it 
changed, if at all?  It seems to me that the body of the if statement won't
be entered, causing failures.
\end{description}
\item {\it Issue\#274 (subin)}
\begin{description}
\item [Subject:] unnecessary init
\item [Criticality:] Med
\item [Confidence-level:] Med
\item [Source-node:] Employee::\~Employee

\item [Lines:] 7-8

\item [Description:] name and socSecurity is not needed to be
initialized here.
\end{description}
\item {\it Issue\#276 (ilee)}
\begin{description}
\item [Subject:] pointing to a wrong place
\item [Criticality:] Hi
\item [Confidence-level:] Hi
\item [Source-node:] Company1::findEmployee

\item [Lines:] 6

\item [Description:] Workers[i] should pointed to socSecurity. If
not, the compare should always become 0
\end{description}
\item {\it Issue\#284 (ktakaki)}
\begin{description}
\item [Subject:] need dynamic allocation
\item [Criticality:] Hi
\item [Confidence-level:] Hi
\item [Source-node:] Employee::setSocSecurity

\item [Lines:] 25

\item [Description:] space for socSecuirty is not allocated,
causing a failure to copy one string into another array.
\end{description}
\item {\it Issue\#288 (ilee)}
\begin{description}
\item [Subject:] pointing to a wrong place
\item [Criticality:] Hi
\item [Confidence-level:] Hi
\item [Source-node:] Company1::deleteEmployee

\item [Lines:] 6

\item [Description:] it should be Workers[i]-{\tt >}socSecurity
\end{description}
\item {\it Issue\#292 (subin)}
\begin{description}
\item [Subject:] char[len] should be char[len+1]
\item [Criticality:] Hi
\item [Confidence-level:] Hi
\item [Source-node:] Employee::setName

\item [Lines:] 11

\item [Description:] need the length of array is to be len+1
because we need one more space for eof
\end{description}
\item {\it Issue\#294 (ktakaki)}
\begin{description}
\item [Subject:] wrong specification
\item [Criticality:] Med
\item [Confidence-level:] Med
\item [Source-node:] Employee::setNumDependents

\item [Lines:] 4

\item [Description:] the specification states that 0 dependents is 
ok. this if statement does not allow for that.
\end{description}
\item {\it Issue\#300 (ktakaki)}
\begin{description}
\item [Subject:] forgot to format printing
\item [Criticality:] Low
\item [Confidence-level:] Med
\item [Source-node:] Employee::print

\item [Lines:] 4-8

\item [Description:] after each cout, there needs to be a endl,
since formatting is part of the specification.
\end{description}
\item {\it Issue\#304 (ktakaki)}
\begin{description}
\item [Subject:] what if socSecurity is 0?
\item [Criticality:] Low
\item [Confidence-level:] Low
\item [Source-node:] Employee::getSocSecurity

\item [Lines:] 4

\item [Description:] if socSecuirty is 0, then an error should be 
returned, since it isn't valid.
\end{description}
\item {\it Issue\#308 (ilee)}
\begin{description}
\item [Subject:] only one space is initial for use
\item [Criticality:] Hi
\item [Confidence-level:] Hi
\item [Source-node:] Employee::Employee

\item [Lines:] 4

\item [Description:] the length of a string should be used MAX\_LEN.
It is only allowed one space for use.
\end{description}
\item {\it Issue\#310 (ktakaki)}
\begin{description}
\item [Subject:] segmentation fault and other errors
\item [Criticality:] Hi
\item [Confidence-level:] Hi
\item [Source-node:] Company1::Company1

\item [Lines:] 6

\item [Description:] the for loop will result in a segmentation
fault if the user decides to set MAX\_EMPLOYEES to 1.  Otherwise, the for loop
will fail to initialize the first array slot.
\end{description}
\item {\it Issue\#318 (ilee)}
\begin{description}
\item [Subject:] miss last char pointer
\item [Criticality:] Hi
\item [Confidence-level:] Hi
\item [Source-node:] Employee::setName

\item [Lines:] 14

\item [Description:] the last char will not be copy.  it should 
be changed to i {\tt <}= len instead of i {\tt <} len
\end{description}
\item {\it Issue\#322 (ktakaki)}
\begin{description}
\item [Subject:] delete is wrong
\item [Criticality:] Med
\item [Confidence-level:] Low
\item [Source-node:] Company1::\~Company1

\item [Lines:] 4

\item [Description:] I think that what is needed is a deletion of 
nodes, and not pointers to nodes.  Hence, Workers should be dereferenced.
\end{description}
\item {\it Issue\#326 (ilee)}
\begin{description}
\item [Subject:] invalid null terminator
\item [Criticality:] Hi
\item [Confidence-level:] Hi
\item [Source-node:] Employee::setName

\item [Lines:] 16

\item [Description:] the last char pointer will become zero. Null 
terminator should place at name[len+1].
\end{description}
\item {\it Issue\#328 (subin)}
\begin{description}
\item [Subject:] incorrect loop decrement
\item [Criticality:] Hi
\item [Confidence-level:] Hi
\item [Source-node:] Employee::setSocSecurity

\item [Lines:] 27

\item [Description:] n-- should be --n.
\end{description}
\item {\it Issue\#334 (subin)}
\begin{description}
\item [Subject:] this is ok.
\item [Criticality:] 
\item [Confidence-level:] 
\item [Source-node:] Employee::setAge

\item [Lines:] 4

\item [Description:] ok.
\end{description}
\item {\it Issue\#336 (ktakaki)}
\begin{description}
\item [Subject:] this is my code, and I know it doesn't work
\item [Criticality:] Hi
\item [Confidence-level:] Hi
\item [Source-node:] Company1::addEmployee

\item [Lines:] 19-35

\item [Description:] This insertion sort doesn't work on UNIX,
though I am pretty damn sure that it works with Borland C++.  The program
crashes due to a segmentation fault that occurs when j goes below zero.  I
don't know why this happens--it just does.  It crashed and burned.
\end{description}
\item {\it Issue\#340 (ilee)}
\begin{description}
\item [Subject:] missing first char of string
\item [Criticality:] Hi
\item [Confidence-level:] Hi
\item [Source-node:] Employee::setSocSecurity

\item [Lines:] 8

\item [Description:] Using ++i, i will be equal to i+1 before the
next process.  Therefore, the first place of the string will be missing at
this point.  instead of using ++1, it should become i++.
\end{description}
\item {\it Issue\#346 (ktakaki)}
\begin{description}
\item [Subject:] 0 is an integer, not an employee node
\item [Criticality:] Hi
\item [Confidence-level:] Med
\item [Source-node:] Company1::findEmployee

\item [Lines:] 9

\item [Description:] the return value must be a pointer to an 
Employee.
\end{description}
\item {\it Issue\#350 (subin)}
\begin{description}
\item [Subject:] invalid if statement
\item [Criticality:] Hi
\item [Confidence-level:] Hi
\item [Source-node:] Employee::setNumDependents

\item [Lines:] 4

\item [Description:] it should be newNumDependents {\tt >}= 0.
\end{description}
\item {\it Issue\#354 (ilee)}
\begin{description}
\item [Subject:] length of newNum
\item [Criticality:] Hi
\item [Confidence-level:] Hi
\item [Source-node:] Employee::setSocSecurity

\item [Lines:] 24 24 24

\item [Description:] what happened if a newNum looks like:
111-11-1111***?  so n should be equal 11, since we have checked the first 12
characters are legal.
\end{description}
\item {\it Issue\#358 (ktakaki)}
\begin{description}
\item [Subject:] w3hat if there are no workers in the array?
\item [Criticality:] Med
\item [Confidence-level:] Med
\item [Source-node:] Company1::deleteEmployee

\item [Lines:] 3

\item [Description:] if there are no workers in the array, then
there will be no ssn to return when the for loop is entered. This is a
potential problem.
\end{description}
\item {\it Issue\#362 (ktakaki)}
\begin{description}
\item [Subject:] no null pointer for last worker
\item [Criticality:] Med
\item [Confidence-level:] Hi
\item [Source-node:] Company1::deleteEmployee

\item [Lines:] 11-15

\item [Description:] if an employee is deleted, then when the
employees are moved over a slot, the last employee will be pointing to two
different things.  This can potentially crash the program.  One of the
pointers has to be a null pointer.
\end{description}
\item {\it Issue\#366 (ilee)}
\begin{description}
\item [Subject:] non-used function
\item [Criticality:] Low
\item [Confidence-level:] Low
\item [Source-node:] Company1

\item [Lines:] 7

\item [Description:] function is not used or not found in any other
function.
\end{description}
\item {\it Issue\#370 (ktakaki)}
\begin{description}
\item [Subject:] what if there are no employees?
\item [Criticality:] Med
\item [Confidence-level:] Hi
\item [Source-node:] Company1::findEmployee

\item [Lines:] 3

\item [Description:] if there are no employees, then there will 
be problems in the for loop (perhaps segmentation faults) and with the
getSocSecurity() function.
\end{description}
\item {\it Issue\#376 (ktakaki)}
\begin{description}
\item [Subject:] wrong condition
\item [Criticality:] Med
\item [Confidence-level:] Med
\item [Source-node:] Employee::setName

\item [Lines:] 6-7

\item [Description:] what if the user enteres more than the
required 11 characters? This is not tested for.  Also, it seems to me that
this condition is specious, though I don't know exactly why.  It seems to me
to be safer to check if len==0 rather than len{\tt <}1.
\end{description}
\item {\it Issue\#380 (subin)}
\begin{description}
\item [Subject:] ok.
\item [Criticality:] Hi
\item [Confidence-level:] Hi
\item [Source-node:] Company1::addEmployee

\item [Lines:] 26

\item [Description:] this is ok.
\end{description}
\item {\it Issue\#384 (ilee)}
\begin{description}
\item [Subject:] pointing to a wrong place
\item [Criticality:] Hi
\item [Confidence-level:] Hi
\item [Source-node:] Company1::addEmployee

\item [Lines:] 13-14

\item [Description:] newWorker and Workers should be pointed to
socSecurity instead of function getSocSecurity().
\end{description}
\item {\it Issue\#388 (subin)}
\begin{description}
\item [Subject:] illegal array.
\item [Criticality:] Hi
\item [Confidence-level:] Hi
\item [Source-node:] Company1::deleteEmployee

\item [Lines:] 9

\item [Description:] this line should not be there.  after deleting
Workers[1], there is no existence of Workers[1]. therefore the assignment
Workers[1]=0 is illegal.
\end{description}
\item {\it Issue\#392 (subin)}
\begin{description}
\item [Subject:] why i+1?
\item [Criticality:] Low
\item [Confidence-level:] Hi
\item [Source-node:] Company1::print

\item [Lines:] 9

\item [Description:] why is it i+1? I think it should be i to print
all the employees from beginning of array (0) to end.
\end{description}
\item {\it Issue\#396 (subin)}
\begin{description}
\item [Subject:] numEmployees
\item [Criticality:] Med
\item [Confidence-level:] Hi
\item [Source-node:] Company1

\item [Lines:] 6

\item [Description:] this member variable is not used by any of the
functions declared.
\end{description}
\end{enumerate}
\section{Review Metrics}
\begin{table}[hb]
\begin{center}
\begin{tabular}{|l|l|l|l|l|}
\hline
Participant & Start-time & End-time & Elapsed-time & Total Busy-time \\
\hline
ilee & Mar 21, 1995 18:25:19 & Mar 21, 1995 20:09:12 & 1:43:53 & 1:40:17 \\
ktakaki & Mar 21, 1995 18:25:13 & Mar 21, 1995 19:57:35 & 1:32:22 & 1:32:22 \\
subin & Mar 21, 1995 18:27:21 & Mar 21, 1995 20:19:08 & 1:51:47 & 1:34:50 \\
\hline
\end{tabular}
\end{center}
\caption{Review Session}
\end{table}


\begin{table}[hb]
\begin{center}
\begin{tabular}{|l|l|l|l|}
\hline
Source & ilee & ktakaki & subin\\
\hline
(176)Employee::Employee & 321 & 194 & 490\\
(192)Company1::Company1 & 641 & 409 & 343\\
(178)Employee::\~Employee & 63 & 256 & 288\\
(194)Company1::\~Company1 & 34 & 149 & 33\\
(180)Employee::setName & 735 & 854 & 561\\
(196)Company1::addEmployee & 987 & 371 & 466\\
(182)Employee::setSocSecurity & 1463 & 1002 & 205\\
(198)Company1::findEmployee & 447 & 499 & 145\\
(184)Employee::setAge & 76 & 120 & 301\\
(200)Company1::deleteEmployee & 435 & 507 & 419\\
(170)Constant & 44 & 180 & 188\\
(186)Employee::setNumDependents & 71 & 111 & 152\\
(202)Company1::print & 108 & 126 & 382\\
(172)Employee & 154 & 176 & 723\\
(188)Employee::print & 51 & 83 & 86\\
(174)Company1 & 477 & 165 & 670\\
(190)Employee::getSocSecurity & 59 & 121 & 87\\
\hline
\end{tabular}
\end{center}
\caption{Review Time}
\end{table}


\begin{table}[hb]
\begin{center}
\begin{tabular}{|l|l|l|l|l|}
\hline
Source & ilee & ktakaki & subin & OK\\
\hline
Employee::Employee & 308 (=1) &  & 260 (=1) & \\
Company1::Company1 & 228 (=1) & 310 (=1) &  & 228=310\\
Employee::\~Employee &  &  & 274 (=1) & \\
Company1::\~Company1 &  & 322 (=1) &  & 322\\
Employee::setName & 318,326 (=2) & 218,226,376 (=3) & 292 (=1) & 326=226=292\\
                  &              &                  &         & 376=354* \\ 
Company1::addEmployee & 246,262,384 (=3) & 336 (=1) & 380 (=1) & \\
Employee::setSocSecurity & 214,340,354 (=3) & 236,242,250, & 328 (=1) & 236 \\
                         &                 &  272,284 (=5) &   & \\
Company1::findEmployee & 276 (=1) & 346,370 (=2) &  & \\
Employee::setAge &  &  & 334 (=1) & \\
Company1::deleteEmployee & 288 (=1) & 358,362 (=2) & 388 (=1) & 362\\
Constant &  &  & 252 (=1) & \\
Employee::setNumDependents &  & 294 (=1) & 350 (=1) & 294=350\\
Company1::print &  &  & 392 (=1) & \\
Employee &  &  & 220 (=1) & \\
Employee::print &  & 300 (=1) &  & \\
Company1 & 366 (=1) &  & 234,396 (=2) & 396\\
Employee::getSocSecurity &  & 304 (=1) &  & \\
\hline
\end{tabular}
\caption{Source node v.s Issue node}
\end{center}
\end{table}

%\end{document}
