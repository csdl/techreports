%\documentstyle[11pt,/group/csdl/tex/definemargins,
%                       /group/csdl/tex/lmacros]{article} 

%          \begin{document}
%          \begin{center}
%          {\large\bf CSRS Experiment Results}\\
%          \end{center}
\chapter {CSRS Experiment Results (ICS313): Group5 (EGSM)}
\small
	  

\begin{description}
\item [Method:] EGSM
\item [Group:] Group5
\item [Source:] Employee2
\item [Participants:] ilee (Reviewer), dat (Moderator), subin (Reviewer), ktakaki (Presenter)
\end{description}
\section{Issue Lists}
\begin{enumerate}
\item {\it Issue\#228 (ktakaki)}
\begin{description}
\item [Subject:] questionable use of initialization
\item [Criticality:] Hi:{\it } Med:{\it } Low:{\it subin,ilee,ktakaki} None:{\it }
\item [Suggested-by:] Me:{\it subin} Me-other:{\it } Other-and-me:{\it ilee,ktakaki} Other:{\it }
\item [Confidence-level:] Hi:{\it } Med:{\it subin,ktakaki} Low:{\it ilee} None:{\it }
\item [Source-node:] Employee::\~Employee

\item [Lines:] 7-8

\item [Description:] after the destructor is invoked, it seems 
pointless to assign 0 to name and socSecurity.
\end{description}
\item {\it Issue\#232 (ktakaki)}
\begin{description}
\item [Subject:] the specfication is not satisfied
\item [Criticality:] Hi:{\it subin,ilee} Med:{\it ktakaki} Low:{\it } None:{\it }
\item [Suggested-by:] Me:{\it subin} Me-other:{\it } Other-and-me:{\it ilee,ktakaki} Other:{\it }
\item [Confidence-level:] Hi:{\it ilee,ktakaki} Med:{\it subin} Low:{\it } None:{\it }
\item [Source-node:] Employee::setName

\item [Lines:] 

\item [Description:] the node does not return 1.  The only value 
returned is 0.
\end{description}
\item {\it Issue\#234 (ktakaki)}
\begin{description}
\item [Subject:] no null terminated symbol
\item [Criticality:] Hi:{\it ilee,subin,ktakaki} Med:{\it } Low:{\it } None:{\it }
\item [Suggested-by:] Me:{\it } Me-other:{\it ktakaki} Other-and-me:{\it ilee,subin} Other:{\it }
\item [Confidence-level:] Hi:{\it subin,ilee,ktakaki} Med:{\it } Low:{\it } None:{\it }
\item [Source-node:] Employee::setName

\item [Lines:] 7-8

\item [Description:] when invoking strlen, name needs to be null 
terminated.
\end{description}
\item {\it Issue\#238 (ktakaki)}
\begin{description}
\item [Subject:] name is not deleted
\item [Criticality:] Hi:{\it ilee,subin} Med:{\it ktakaki} Low:{\it } None:{\it }
\item [Suggested-by:] Me:{\it } Me-other:{\it ktakaki} Other-and-me:{\it ilee} Other:{\it subin}
\item [Confidence-level:] Hi:{\it ilee,subin,ktakaki} Med:{\it } Low:{\it } None:{\it }
\item [Source-node:] Employee::setName

\item [Lines:] 

\item [Description:] name is not deleted.  If the user enters
many names, then space will be continually allocated, and memory may run out.
\end{description}
\item {\it Issue\#240 (ktakaki)}
\begin{description}
\item [Subject:] else condition wrong
\item [Criticality:] Hi:{\it ilee,subin,ktakaki} Med:{\it } Low:{\it } None:{\it }
\item [Suggested-by:] Me:{\it subin} Me-other:{\it } Other-and-me:{\it ilee,ktakaki} Other:{\it }
\item [Confidence-level:] Hi:{\it ilee,subin,ktakaki} Med:{\it } Low:{\it } None:{\it }
\item [Source-node:] Employee::setSocSecurity

\item [Lines:] 17-18

\item [Description:] the else condition does not copy the '-' 
into socSecurity.  Hence validDigits will be wrong and the program will
return 1 when a valid ssn has been entered
\end{description}
\item {\it Issue\#244 (ktakaki)}
\begin{description}
\item [Subject:] wrong logic in if condition.
\item [Criticality:] Hi:{\it subin,ilee} Med:{\it ktakaki} Low:{\it } None:{\it }
\item [Suggested-by:] Me:{\it } Me-other:{\it } Other-and-me:{\it ilee,subin} Other:{\it ktakaki}
\item [Confidence-level:] Hi:{\it subin} Med:{\it ilee,ktakaki} Low:{\it } None:{\it }
\item [Source-node:] Employee::setSocSecurity

\item [Lines:] 

\item [Description:] if the user enters a valid ssn (up to 11
chars), and then a buch of junk chars, the update will be successful, though
it shouldn't be.
\end{description}
\item {\it Issue\#246 (ktakaki)}
\begin{description}
\item [Subject:] last node will lead to errors
\item [Criticality:] Hi:{\it ilee,subin,ktakaki} Med:{\it } Low:{\it } None:{\it }
\item [Suggested-by:] Me:{\it } Me-other:{\it ktakaki} Other-and-me:{\it subin,ilee} Other:{\it }
\item [Confidence-level:] Hi:{\it ilee,subin,ktakaki} Med:{\it } Low:{\it } None:{\it }
\item [Source-node:] Company2::findEmployee

\item [Lines:] 13-16

\item [Description:] if the while loop is entered, and we are 
currently at the last node of the list, then current will be updated and will
be given the value null.  But, when current is dereferenced, there will be an
error
\end{description}
\item {\it Issue\#250 (ktakaki)}
\begin{description}
\item [Subject:] Assuming that the while loop is correct, the if
statement is faulty
\item [Criticality:] Hi:{\it ktakaki,ilee} Med:{\it subin} Low:{\it } None:{\it }
\item [Suggested-by:] Me:{\it } Me-other:{\it ilee} Other-and-me:{\it } Other:{\it subin,ktakaki}
\item [Confidence-level:] Hi:{\it ilee} Med:{\it ktakaki} Low:{\it subin} None:{\it }
\item [Source-node:] Company2::findEmployee

\item [Lines:] 17-20

\item [Description:] in the if statement, if current==0 and
result{\tt <}0, then we will go to the else conidtion.  But, what is returned is
wrong, since the matching employee has not been found.
\end{description}
\item {\it Issue\#254 (ktakaki)}
\begin{description}
\item [Subject:] wrong comparison statement
\item [Criticality:] Hi:{\it ilee} Med:{\it subin} Low:{\it ktakaki} None:{\it }
\item [Suggested-by:] Me:{\it } Me-other:{\it ilee} Other-and-me:{\it subin} Other:{\it ktakaki}
\item [Confidence-level:] Hi:{\it subin,ilee} Med:{\it ktakaki} Low:{\it } None:{\it }
\item [Source-node:] Company2::addEmployee

\item [Lines:] 11

\item [Description:] the if statement should not be an assignment,
since it will always be false.
\end{description}
\item {\it Issue\#258 (ktakaki)}
\begin{description}
\item [Subject:] the first node's ssn isn't checked
\item [Criticality:] Hi:{\it subin} Med:{\it ilee,ktakaki} Low:{\it } None:{\it }
\item [Suggested-by:] Me:{\it } Me-other:{\it } Other-and-me:{\it subin,ilee} Other:{\it ktakaki}
\item [Confidence-level:] Hi:{\it ilee,subin,ktakaki} Med:{\it } Low:{\it } None:{\it }
\item [Source-node:] Company2::deleteEmployee

\item [Lines:] 

\item [Description:] if there is only one node in the list, the 
program returns 1.  But, the ssn in the first node isn't checked, and so we
don't know if the employee entered by the user is there or not (i.e., no
evaluation is made.
\end{description}
\item {\it Issue\#260 (ktakaki)}
\begin{description}
\item [Subject:] there needs to be a conidtion for result {\tt >} 0
\item [Criticality:] Hi:{\it subin,ilee} Med:{\it ktakaki} Low:{\it } None:{\it }
\item [Suggested-by:] Me:{\it } Me-other:{\it ilee} Other-and-me:{\it subin,ktakaki} Other:{\it }
\item [Confidence-level:] Hi:{\it ktakaki,ilee} Med:{\it subin} Low:{\it } None:{\it }
\item [Source-node:] Company2::deleteEmployee

\item [Lines:] 12

\item [Description:] if result is greater than 0, then then
we should go on to the next node.  But, the node isn't updated, and so we go
into an infinite loop.
\end{description}
\item {\it Issue\#264 (ktakaki)}
\begin{description}
\item [Subject:] the last node won't be printed
\item [Criticality:] Hi:{\it subin,ilee} Med:{\it ktakaki} Low:{\it } None:{\it }
\item [Suggested-by:] Me:{\it } Me-other:{\it ilee} Other-and-me:{\it subin} Other:{\it ktakaki}
\item [Confidence-level:] Hi:{\it subin,ilee,ktakaki} Med:{\it } Low:{\it } None:{\it }
\item [Source-node:] Company2::print

\item [Lines:] 13

\item [Description:] in the while condition, we point to the  
next node instead of the current one.  The correct condition should be: while
(current!=0)
\end{description}
\end{enumerate}
\section{Review Metrics}
\begin{table}[hb]
\begin{center}
\begin{tabular}{|l|l|l|l|l|}
\hline
Participant & Start-time & End-time & Elapsed-time & Total Busy-time \\
\hline
dat & Mar 24, 1995 18:46:21 & Mar 24, 1995 21:01:23 & 2:15:2 & 1:31:45 \\
subin & Mar 24, 1995 18:45:37 & Mar 24, 1995 21:01:03 & 2:15:26 & 1:20:45 \\
ilee & Mar 24, 1995 18:46:08 & Mar 24, 1995 21:00:53 & 2:14:45 & 1:24:28 \\
ktakaki & Mar 24, 1995 18:46:17 & Mar 24, 1995 21:00:16 & 2:13:59 & 1:34:59 \\
\hline
\end{tabular}
\end{center}
\caption{Review Session}
\end{table}


\begin{table}[hb]
\begin{center}
\begin{tabular}{|l|l|l|l|l|}
\hline
Source & dat & subin & ilee & ktakaki\\
\hline
(208)Company2::\~Company2 & 126 & 127 & 126 & 125\\
(192)Employee::setSocSecurity & 801 & 312 & 472 & 767\\
(210)Company2::findEmployee & 819 & 547 & 489 & 569\\
(194)Employee::setAge & 31 & 32 & 31 & 32\\
(212)Company2::addEmployee & 906 & 402 & 467 & 696\\
(196)Employee::setNumDependents & 45 & 46 & 45 & 46\\
(180)Employee & 136 & 137 & 136 & 137\\
(214)Company2::deleteEmployee & 452 & 814 & 608 & 816\\
(198)Employee::print & 28 & 28 & 28 & 29\\
(182)EmployeeNode & 178 & 178 & 178 & 176\\
(216)Company2::print & 305 & 290 & 470 & 270\\
(200)Employee::getSocSecurity & 34 & 33 & 33 & 32\\
(184)Company2 & 406 & 409 & 408 & 409\\
(202)EmployeeNode::EmployeeNode & 22 & 22 & 22 & 22\\
(186)Employee::Employee & 56 & 57 & 56 & 57\\
(204)EmployeeNode::\~EmployeeNode & 71 & 73 & 72 & 74\\
(188)Employee::\~Employee & 444 & 444 & 443 & 442\\
(206)Company2::Company2 & 56 & 73 & 224 & 224\\
(190)Employee::setName & 473 & 691 & 661 & 727\\
\hline
\end{tabular}
\end{center}
\caption{Review Time}
\end{table}


\begin{table}[hb]
\begin{center}
\begin{tabular}{|l|l|l|}
\hline
Source node & Issue node  & OK\\
\hline
(208)Company2::\~Company2 &  & \\
(192)Employee::setSocSecurity & \#240,\#244 (=2) & 240,244\\
(210)Company2::findEmployee & \#246,\#250 (=2) & 246,250\\
(194)Employee::setAge &  & \\
(212)Company2::addEmployee & \#254 (=1) & 254\\
(196)Employee::setNumDependents &  & \\
(180)Employee &  & \\
(214)Company2::deleteEmployee & \#258,\#260 (=2) & 258,260\\
(198)Employee::print &  & \\
(182)EmployeeNode &  & \\
(216)Company2::print & \#264 (=1) & 264\\
(200)Employee::getSocSecurity &  & \\
(184)Company2 &  & \\
(202)EmployeeNode::EmployeeNode &  & \\
(186)Employee::Employee &  & \\
(204)EmployeeNode::\~EmployeeNode &  & \\
(188)Employee::\~Employee & \#228 (=1) & \\
(206)Company2::Company2 &  & \\
(190)Employee::setName & \#232,\#234,\#238 (=3) & 234\\
\hline
\end{tabular}
\caption{Source node v.s Issue node}
\end{center}
\end{table}

%\end{document}
