%%%%%%%%%%%%%%%%%%%%%%%%%%%%%% -*- Mode: Latex -*- %%%%%%%%%%%%%%%%%%%%%%%%%%%%
%% introduction.tex -- 
%% Author          : Rosemary Andrada
%% Created On      : Sun Mar 12 18:24:26 1995
%% Last Modified By: Rosemary Andrada
%% Last Modified On: Mon Jul 24 17:10:11 1995
%% Status          : Unknown
%% RCS: $Id: introduction.tex,v 1.3 1995/07/05 00:32:54 rosea Exp rosea $
%%%%%%%%%%%%%%%%%%%%%%%%%%%%%%%%%%%%%%%%%%%%%%%%%%%%%%%%%%%%%%%%%%%%%%%%%%%%%%%
%%   Copyright (C) 1995 University of Hawaii
%%%%%%%%%%%%%%%%%%%%%%%%%%%%%%%%%%%%%%%%%%%%%%%%%%%%%%%%%%%%%%%%%%%%%%%%%%%%%%%
%% 

\ls{1.5}
\chapter{Computers and Community}
\label{chap:introduction}

%should be 5-10 page expansion of abstract
%should include a roadmap to thesis
%  ``In chapter 2...''
%spend time on leading the reader into your domain

%  What is the purpose of this research?  What is the central problem being
%  addressed?  Why is this problem important?

\section{What is Community and Why Bother With it?}
This research concerns computer-based mechanisms for improving the
sense of community within an organization.  It is often easy for a group to
lose sight of its goal or for its members to feel alienated, thereby
contributing to the breakdown of sense of community.  Building community is
important because it builds morale amongst its members and helps the group to
function better as a whole.  It is important enough that the Rockefeller
Foundation commissioned a report \cite{Sharp94} from the Millennium Group to
investigate methods of revitalizing communities.  This falls under the
Foundation's Common Enterprise Initiative, which is viewed as ``an advocate for
the whole''.  They recognize the significance of community and their report is
a first attempt to rebuild and revitalize our communities.

How does one even begin trying to build community in an organization?  First,
the term `community' must be clearly defined.  Webster's dictionary
\cite{Webster84} lists the following definition:

\ls{1}
\begin{verbatim}
1 : a unified body of individuals : as
     1  a : STATE, COMMONWEALTH
        b : the people with common interests living in a particular 
            area; broadly : the area itself
        c : an interacting population of various kinds of individuals
            (as species) in a common location
        d : a group of people with a common characteristic or
            interest living together within a larger society
        e : a group linked by a common policy
        f : a body of persons or nations having a common history or
            common social, economic, and political interests
        g : a body of persons of common and esp. professional 
            interests scattered through a larger society
     2 : society at large
     3  a : joint ownership or participation
        b : common character : LIKENESS
        c : social activity : FELLOWSHIP
        d : a social state or condition 
\end{verbatim}
\ls{1.5}

The broad meaning of the word community is a ``unified body of individuals.''
But in what ways are they unified?  According to Clark \cite{Clark77}, there
are communities of place and of interest.  Man has historically built his
community on the basis of place.  The place in which he was born and raised
became the place he lived and raised his family, thereby continuing the cycle
of life.  However, fundamental changes in transportation, communication and
other aspects of society produced a new form of community based on interest.
Man's increase in spatial mobility allowed him to more easily move his
residence.  His social mobility allowed him to move easily through the social
ladder.  Lastly, his cognitive mobility allowed him to absorb and comprehend
other ideas and experiences from far across the globe \cite{Clark77}.  Through
these changes, community is no longer synonymous with one's neighborhood.
Where communities of place are typically concerned with the maintenance of
tradition, communities of interest are typically concerned with action and
innovation.  Interests and issues are what gathers people together with great
inspiration and motivation.  The focus of this study, the Department of
Information and Computer Sciences (ICS), is a community based on interest.

The sense of community in an organization is not a binary entity.  One cannot
say that it either exists or it does not.  As with many other difficult to
measure concepts, community exists in varying degrees.  Clark \cite{Clark73}
defines community as

\ls{1}
\begin{quote}
  ``a sentiment which people have about themselves in relation to others and
  others in relation to themselves;...there are two essentials for the
  existence of community: a sense of significance and a sense of solidarity.
  The strength of community within any given group is determined by the
  degree to which its member's experience both a sense of solidarity and a
  sense of significance within it.''
\end{quote}
\ls{1.5}

More importantly, it is the perception of those involved that determines the
relative strength of a community.  Taking this a step further, the strength of
a community is also based on the community's sense of self-awareness.  For how
can one feel important or bonded with others without first knowing who they
are, what they do and how they contribute to the community?

In this thesis, a ``sense of community'' is viewed as a combination of several
ingredients, including: each group member's sense of significance, their sense
of solidarity, and their collective sense of self-awareness.  From this
definition, it is possible to characterize the level of community in an
organization by evaluating the following measures:

\ls{1}
\begin{enumerate}
\item{Do members feel they play an important role?}
\item{Do members feel a sense of belonging?}
\item{Can members associate names with the faces of others in the
  organization?}
\item{Do members know each other personally?}
\item{Can members correctly identify a `resident expert,' if there is one, on
  subjects of importance to the organization?}
\item{Are members aware of the different projects or issues in the
  organization?}
\end{enumerate}
\ls{1.5}

\section{Research Thesis}
The basic premise of this research is that an organization's sense of community
can be positively affected through computer mediation.  An information system
designed with features that support these measures of community was the testbed
for this study.

The thesis of this research is that an appropriately designed information
system can improve the sense of community in an organization.  To evaluate this
thesis, an information system was adopted and its design extended with the
above measures in mind.  Pre-test and post-test questionnaires were
administered to the organization to assess the sense of community.  Logfiles
kept by the information system were analyzed to evaluate how the system was
actually used.

%\section{Computer mediated support for building community}
\section{Research Method and Results}
I investigated the strengths and weaknesses of a computer-based approach to
improving the sense of community within one organization, the Department of
Information and Computer Sciences at the University of Hawaii.  The World Wide
Web's communication infrastructure was used to support community building in
the department.  This seemed to be the best choice as the related software is
free, popular, and easy to use.  So, I set up a Web server for the ICS
department with features designed specifically for improving the sense of
community.

Based on these measures of community, I came up with requirements for a
community-enhancing information system and incorporated them.  One requirement
was that it should help the community members feel important in the group.
This was accomplished by providing the users with an unedited voice in their
publication effort.  The system was also extended to foster a sense of
belonging.  I built mechanisms for facilitating user involvement and
contributions to the information system.  Several additional features were
incorporated to increase a group's collective self-awareness.  I included a
photoboard of organization members.  There were sections identifying the
subgroups of the organization.  These subgroups could be formed through common
personal or professional interests.  Other sections highlighted the various
projects underway in the department.

%\section{Research Method}
The goal was to increase the sense of community in the department with the
support of computer-based mechanisms.  Armed with a tool, I set out to study
this approach at community building over the course of one semester.  Before
starting, department members anonymously and voluntarily filled out a pre-test
questionnaire.  This questionnaire was designed to evaluate the ``baseline''
level of community.  Next, I introduced the Web site to the department,
offering training sessions on how to access information as well as how to
contribute to it.  I conducted introductory classes throughout the semester and
held some advanced ones in the latter half.  At the end of the semester,
department members filled out a post-test questionnaire to assess the level of
community at that time.

The results of this research are as follows:

\ls{1}
\begin{itemize}
\item{Department members did not view their group as one community, but
  envisioned themselves as three communities: faculty \& staff, graduate
  students and undergraduate students.}
\item{The faculty \& staff felt importance and belonging and had a high sense of
  collective self-awareness both before and after the introduction of the
  information system.}
\item{Initially, the graduate and undergraduate students did not feel important
  in the department but did feel a sense of belonging.  Their collective
  self-awareness was poor.}
\item{At the end of the study, graduate students did not feel important but
  still felt belonging in the department.  However, their collective
  self-awareness seemed to increase.}
\item{At the end of the study, undergraduate students neither felt importance
  nor felt belonging in the department.  But their collective self-awareness
  seemed to increase.}
\end{itemize}
\ls{1.5}

\section{Research Contributions}
The results of this research gives many insights into computing systems and
communities.  These are the contributions this research has made to those
fields.

\ls{1}
\begin{itemize}
\item{Physical inequities, such as equipment or laboratory resources, can
  inhibit community development.}
\item{The collective self-awareness of a group can be a double-edged sword.
  Information dissemination flowing in one direction can reveal inequities
  and polarize a community.}
\item{The design of the information system is a successful one.  Department
  members quickly adopted the system and are still using it.}
\item{The World Wide Web is typically promoted as a means for universal
  readership of hypertext documents.  This research articulates an
  alternative purpose for the Web.}
\item{Collective self-awareness can improve through computer mediation.}
\end{itemize}
\ls{1.5}

\section{Related Work}
A few systems have been developed for similar purposes.  I first discuss GC
EduNET, an information system designed for supporting educators in the Georgia
education system.  I then discuss the Davis Community Network, a virtual
community created to enhance the physical one composed of its residents.  I
also mention the Well, a global virtual community popularized and recounted by
Rheingold \cite{Rheingold93}.

%talk about how these systems have a recipe for community - internet access

GC EduNET \cite{Wolpert91} is an electronic information service developed to
bring together Georgia teachers and school administrators and to create to a
pool of resources important to that group.  The system provides capabilities
for conferencing, email, file sharing and database searching.  It is unique in
its design to alleviate teacher and administrator isolation.  The system
services more than 1500 members covering 101 counties in Georgia.  In addition
to supporting email and conferencing, the system was organized into six major
areas: Libraries, Questions and Answers, Electronic Marketing, Center for
Adolescent Needs, Software and File Exchange and Organizations.  The major
difference between GC EduNET's online community and that of the ICS department
is their absence of a physical community.  They designed a system to create a
community of interest {\em online}.  They did not have a membership; they
created one for the purpose of sharing experiences and resources.  The group
exists only in a virtual community.

The Davis Community Network ({\tt http://www.dcn.davis.ca.us:80/}) is a
non-profit organization created for residents of Davis, California.  Its
purpose is to strengthen the community of Davis by providing Internet-based
communications and information services and support.  More specifically, they
provide newsgroups pertinent to the Davis community, mailings lists, and a WWW
server.  The Web site offers information on local businesses, city government,
education, entertainment and tourist and new resident information.  Like the
ICS department, this online group was created to enhance their physical bond.
However, theirs is a community of place servicing a much larger membership
where the roles of the citizens are not as clearly defined as in communities of
interest.

The Well (Whole Earth 'Lectronic Link) is an online service based in San
Francisco providing connection to USENET newsgroups, the World Wide Web, Gopher
and its own MUSE (multi user simulation environment).  Users from all over the
world may join and more than half of its members are from outside of
California.  The Well is also associated with physical communities in the San
Francisco Bay and New York City areas. 

There are many other community-enhancing systems\footnote{A list is available
at http://www.well.com/user/hlr/vircom/index.html} for various types of groups.
They often incorporate the same technology and software: email, WWW, Gopher,
etc.  However, few have modified these systems to address the particular needs
of their organization.

\section{Organization of this Thesis}
Chapter \ref{chap:www} discusses the use of the World Wide Web, which provides
the infrastructure for the information system used in launching the experiment.
The requirements for a community-oriented Web site and the implementation of
these requirements are detailed.

Chapter \ref{chap:experiment} explains how the experiment was conducted.
Questionnaires were used to assess the the sense of community in the
department.  There is discussion of the introduction of the system and the
training involved to get department members interested in using it.  The
chapter also outlines the possible outcomes of the experiment.

Chapter \ref{chap:results} presents the results of the experiment.
Questionnaire responses are discussed in detail.  Analysis of database access
logs reveals how much of the department participated as well as who
participated.  It also gives some insight on access and publication patterns of
its users.

Finally, Chapter \ref{chap:conclusion} shows analysis of the results.
Implications of these results and contributions of this research are given.
Guidance for potential Web site builders and future research in this area is
provided.


