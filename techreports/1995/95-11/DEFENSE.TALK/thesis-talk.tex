%%%%%%%%%%%%%%%%%%%%%%%%%%%%%% -*- Mode: Latex -*- %%%%%%%%%%%%%%%%%%%%%%%%%%%%
%% thesis-talk.tex -- 
%% RCS:            : $Id: thesis-talk.tex,v 1.1 93/11/19 13:28:45 dxw Exp $
%% Author          : Dadong Wan
%% Created On      : Fri Nov 12 16:05:48 1993
%% Last Modified By: Dadong Wan
%% Last Modified On: Tue Nov 16 17:59:03 1993
%% Status          : Unknown
%%%%%%%%%%%%%%%%%%%%%%%%%%%%%%%%%%%%%%%%%%%%%%%%%%%%%%%%%%%%%%%%%%%%%%%%%%%%%%%
%%   Copyright (C) 1993 University of Hawaii
%%%%%%%%%%%%%%%%%%%%%%%%%%%%%%%%%%%%%%%%%%%%%%%%%%%%%%%%%%%%%%%%%%%%%%%%%%%%%%%
%% 
%% History
%% 12-Nov-1993		Dadong Wan	
%%    created.
\documentstyle[11pt,slidesonly]
{/group/csdl/tex/seminar}
\input{/group/csdl/tex/psfig}
\special{header=/group/csdl/tex/psfig/lprep71.pro}
%\rotateheaderstrue               % Try this out if using rotation macros.
%\articlemag{-1}
\newcommand{\horizontalline} {\rule{\textwidth}{.05in}}
\slideframe{none}
\slidesmag{0}        % integer value ranging from -5 to 9
\special{landscape}  %comment out this line for notes
\pagestyle{empty}
%\twoup[-2]          %uncomment this line for notes
\begin{document}
\sloppy

\ls{1.2}

\begin{slide} \Huge 
  \begin{center}
    CLARE: a Computer-Supported Collaborative Learning

    Environment Based on the Thematic

    Structure of Scientific Text

   {\rule{2.0in}{.02in}}
 
   A Dissertation Presented 

    
    by

    DADONG WAN

    \vspace{0.5in}

   Interdisciplinary Program in Communication

   and Information Sciences

   and
     
    Department of Information \& Computer Sciences
    
    University of Hawaii
  \end{center}
\end{slide} \Huge  


\begin{slide} \Huge 
  {\bf Outline:}
  \horizontalline
  
  \begin{itemize}
  \item The problems definition
      
  \item Overview of research contributions

  \item Three main components of CLARE

  \item Empirical evaluation

  \item Conclusions \& future directions
    \end{itemize}
\end{slide} \Huge


\begin{slide} \Huge 
  {\bf Problems-in-the-large:}
  \horizontalline    
  \begin{itemize}
  \item What is wrong in virtual-classroom \& hypermedia systems?
    \begin{itemize}
      \item shared Access = collaborative learning

      \item Technology-driven
    \end{itemize}      
    \end{itemize}
\end{slide} \Huge


\begin{slide} \Huge
  {\bf Problems-in-the-small:}
  \horizontalline  

  \begin{itemize}
  \item Difficulties in learning from scientific text:
   \begin{itemize}
   \item Ambiguity in purpose
       
   \item Haphazard process
    
   \item Inconsistent frames of reference
    
   \item Face-to-face barriers
   \end{itemize}
   \end{itemize}
\end{slide} \Huge


\begin{slide} \Huge
  {\bf CLARE's answers to the ``problems-in-the-large''}
  \horizontalline
    
  \begin{center}
   \begin{tabular} {||p{3.5in}|p{5.0in}||} \hline   
     {\bf Problems} & {\bf CLARE's Answers} \\ \hline \hline
    
     Access-driven & Focus on collaborative knowledge-building \\
     \hline
     
    Technology-driven & 1) ``Theory-pull:'' theory of
    cognitive learning; 2) ``Pedagogy-push:'' problems-in-the-small \\
    \hline \hline
   \end{tabular}
  \end{center}
\end{slide} \Huge


\begin{slide} \Huge
  {\bf CLARE's answers to the ``problems-in-the-small:''}
  \horizontalline
    
  \begin{center}
    \begin{tabular} {||p{3.3in}|p{5.2in}||} \hline   
    {\bf Problems} &   {\bf CLARE's Answers} \\ \hline \hline 
    
    Ambiguity in purpose & Focus on themes, relationships, and points
    of views \\ \hline
     
    Haphazard process & SECAI \\ \hline
     
    Inconsistent frames of reference & RESRA \\
    \hline
    
    Face-to-face barriers & Computer-mediated environment\\
    \hline \hline
   \end{tabular}
  \end{center}
\end{slide} \Huge


\begin{slide} \Huge 
  {\bf What is CLARE:} 
  \horizontalline
  \bigskip

  \centerline{\psfig{figure=Figures/clare-definition.eps,width=8.0in,height=6.0in}}
\end{slide} \Huge


\begin{slide} \Huge 
  {\bf Research contributions: an overview}
  \horizontalline

  \begin{itemize}
  \item SECAI: the process model
      
 \item RESRA: the representation language
 
  \item CLARE: the system
      
  \item Empirical evaluation
  \end{itemize}
\end{slide} \Huge


\begin{slide} \Huge
  {\bf Experiments:}
  \horizontalline
  \begin{itemize}
  \item Tasks:
    \begin{itemize}
    \item Analysis \& discussion of research papers
    \end{itemize}
    
  \item Subjects:
    \begin{itemize}
    \item ICS414: 16 students (4 groups)
      
    \item ICS613: 8 students (2 groups)
    \end{itemize}
    
  \item Over 300 hours of CLARE usage
    \end{itemize}    
\end{slide} \Huge    


\begin{slide} \Huge 
  {\bf Contribution \#1: RESRA}
  \horizontalline

  \begin{itemize}
    \item To characterize thematic features of scientific text
      
    \item To highlight different points of views

    \item A shared framework for collaborative deliberation
      
    \item A KR scheme for the group knowledge base
  \end{itemize}
\end{slide} \Huge


\begin{slide} \Huge 
  {\bf Contribution \#1: RESRA results}
  \horizontalline
  \begin{itemize}

  \item The node primitive is the most useful feature

  \item 80\%: node primitives are extremely useful or very useful        

  \item 84\%: useful for characterizing the content of scientific text

  \item 90\%: effective in exposing different points of view
  \end{itemize}    
\end{slide} \Huge


\begin{slide} \Huge 
  {\bf Contribution \#2: SECAI}
  \horizontalline
  
  \centerline{\psfig{figure=Figures/2-learning-community.eps,width=9.0in,height=7.0in}}
\end{slide} \Huge


\begin{slide} \Huge 
  {\bf Contribution \#2: SECAI results}
  \horizontalline

  \begin{itemize}
  \item Second most useful feature of CLARE
    
  \item 77\%: extremely useful or very useful
  \end{itemize}
\end{slide} \Huge


\begin{slide} \Huge 
  {\bf Contribution \#3: CLARE system}
  \horizontalline
  \begin{itemize}
  \item A theory-based system: constructionism and theory of cognitive
    learning
    
  \item An environment for empirical experimentation: fine-grained,
    non-obtrusive instrumentation
    
  \item An extensible system: object-oriented design + Emacs-lisp
    environment
  \end{itemize}
\end{slide} \Huge


\begin{slide} \Huge 
  {\bf CLARE system: an evolutionary view}
  \horizontalline
  \bigskip

  \centerline{\psfig{figure=Figures/history.eps,width=8.0in}}
\end{slide} \Huge


\begin{slide} \Huge 
  {\bf Contribution \#3: CLARE results}
  \horizontalline

  \begin{itemize}
   \item 70\%: A novel way of understanding research papers
     
   \item 80\%: A novel way of understanding peers' points of views

   \item 70\%: CLARE should be used at least once a semester
     
   \item 65\%: Would recommend using CLARE for studying research
     papers
  \end{itemize}   
\end{slide} \Huge
   

\begin{slide} \Huge 
{\bf Contribution \#4: Empirical evaluation}
  \horizontalline
    
  \begin{itemize}
  \item Hypotheses

  \begin{itemize}
  \item RESRA provides a viable framework for collaborative construction
    of knowledge
    
  \item CLARE is a viable platform to support collaborative learning
  \end{itemize}    
  \end{itemize}
\end{slide} \Huge


\begin{slide} \Huge
{\bf Contribution \#4: empirical evaluation}
  \horizontalline

  \begin{itemize}
  \item Summarization is difficult and labor-intensive
     
  \item Major themes were not always ``reconstructed,'' even by a group
    of learners
    
  \item Summarization \(\Rightarrow\) argumentation

  \item Learners adopted four general strategies during summarization
    
  \item Collaboration took place at both content and representation
    levels
  \end{itemize}
\end{slide} \Huge


\begin{slide} \Huge
  {\bf Comments from a CLARE user:}

  \horizontalline 
  \begin{verbatim}
  ``...  Before I used CLARE I just read the
  artifacts.  Now using CLARE I look for
  the meaning of the artifact and
  learn more about the subject...'' 
  \end{verbatim}  

\end{slide} \Huge


\begin{slide} \Huge 
  {\bf Conclusions}
  \horizontalline

  \begin{itemize}
    \item SECAI: a useful process for collaborative learning
      
    \item RESRA: useful in characterizing the content of
      scientific text and in exposing different points of view
    
    \item CLARE: a viable environment
    
    \item Empirical evaluation: confirmed primary hypotheses;
      suggests areas for improvement
    
  \end{itemize}
\end{slide} \Huge


\begin{slide} \Huge
  {\bf Future directions:}
  \horizontalline

  \begin{itemize}
    \item Domain-specific RESRAs 
      
    \item Advanced interface support
      
    \item Longitudinal \& comparative studies
  \end{itemize}
\end{slide} \Huge


\begin{slide} \Huge 
  \centerline{\psfig{figure=Figures/resra-of-clare.eps,width=8.0in,height=6.5in}}
\end{slide} \Huge


\end{document}









