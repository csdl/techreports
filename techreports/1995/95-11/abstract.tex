%%%%%%%%%%%%%%%%%%%%%%%%%%%%%% -*- Mode: Latex -*- %%%%%%%%%%%%%%%%%%%%%%%%%%%%
%% abstract.tex -- 
%% Author          : Rosemary Andrada
%% Created On      : Thu Jun 15 14:58:21 1995
%% Last Modified By: Rosemary Andrada
%% Last Modified On: Thu Jul  6 09:31:50 1995
%% RCS: $Id: abstract.tex,v 1.3 1995/07/05 00:32:42 rosea Exp rosea $
%%%%%%%%%%%%%%%%%%%%%%%%%%%%%%%%%%%%%%%%%%%%%%%%%%%%%%%%%%%%%%%%%%%%%%%%%%%%%%%
%%   Copyright (C) 1995 Rosemary Andrada
%%%%%%%%%%%%%%%%%%%%%%%%%%%%%%%%%%%%%%%%%%%%%%%%%%%%%%%%%%%%%%%%%%%%%%%%%%%%%%%
%% 

\ls{1.5}

\abstract{

This thesis presents a case study designed to assess the strengths and
weaknesses of a computer-based approach to improving the sense of community
within one organization, the Department of Information and Computer Sciences at
the University of Hawaii.  The case study used a pretest-posttest design.
First, several measures of the sense of community within the department were
obtained via a questionnaire.  Second, a World Wide Web information system was
introduced in an effort to affect the level of community within the department.
Third, a similar questionnaire was administered after a period of four months.
Analysis of the survey responses and system logs showed that the information
system designed to promote community in this organization had instead polarized
it.  However, these systems can also serve as a diagnostic tool for discovering
what factors may help promote or inhibit community building.

}