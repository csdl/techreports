%%%%%%%%%%%%%%%%%%%%%%%%%%%%%% -*- Mode: Latex -*- %%%%%%%%%%%%%%%%%%%%%%%%%%%%
%% www.tex -- 
%% Author          : Rosemary Andrada
%% Created On      : Sun Mar 12 18:24:26 1995
%% Last Modified By: Rosemary Andrada
%% Last Modified On: Mon Jul 24 17:10:22 1995
%% Status          : Unknown
%% RCS: $Id: www.tex,v 1.3 1995/07/05 00:33:03 rosea Exp rosea $
%%%%%%%%%%%%%%%%%%%%%%%%%%%%%%%%%%%%%%%%%%%%%%%%%%%%%%%%%%%%%%%%%%%%%%%%%%%%%%%
%%   Copyright (C) 1995 University of Hawaii
%%%%%%%%%%%%%%%%%%%%%%%%%%%%%%%%%%%%%%%%%%%%%%%%%%%%%%%%%%%%%%%%%%%%%%%%%%%%%%%
%% 

\ls{1.5}
\chapter{The World Wide Web as a Tool for Community Building}
\label{chap:www}
This chapter describes how the World Wide Web was used as a tool for community
building.  Section \ref{sec:background} discusses how it came into existence
and then goes into some detail about how it works.  Section \ref{sec:design}
enumerates the community-oriented features built into the system and explains
some design decisions.

%The last section discusses other features important to any information system.

\section{What is the World Wide Web and Why Use it?}
\label{sec:background}
The World Wide Web (WWW or Web, for short) is a ``distributed, heterogeneous,
collaborative, multimedia information system'' \cite{Berners-Lee94}.  It is a
seamless network of information, all of which is accessible simply and
consistently.  It supports universal readership in a hypertext environment and
uses the client-server model.  Designed with no centralized control, the Web is
easy to scale.  Anyone wishing to publish information need only run a server
while those who wish to access information need only run a client application.
This ``web'' of clients and servers runs on top of the networking
infrastructure provided by the Internet.

A Web Server acts as the interface between a database and a client program
using the WWW protocol specification, HTTP (HyperText Transfer Protocol).  It
is also possible for a Web Server to interface with WAIS, GOPHER and ftp
servers.  However, Web Servers primarily serve information from a local
database, which is structured as a directed graph of linked files.  There is
one main file, or Home Page, which serves as a default entry point to
information in a particular database.  However, users may access any file in
the database, given its Uniform Resource Locator (URL) or ``address.''

Since the Web uses HTTP, files are usually written in HyperText Markup Language
(HTML) format.  An HTML file consists of text and tags.  The text is displayed
to the reader of an HTML document while the tags specify how the text should be
formatted.  Documents written in HTML can be linked to other documents.
Browsers highlight regions of text to indicate that they are hypertext links.

Documents may contain links to other files in the database as well as other
files residing on other Web sites.  Selecting these links retrieves the
document associated with it.  Thus, users may navigate through the Web by
simply following links in various documents.

The idea for the Web was conceived in 1989 but has began rapidly gaining in
popularity in 1993.  That was when the National Center for Supercomputing
Applications (NCSA) released its first alpha version of Marc Andreessen's
``Mosaic for X.''  Since then, Web servers have emerged all over the world in
universities, companies and government agencies in an effort to quickly
disseminate information to as large an audience as possible.

In this research, I chose to use the communication infrastructure provided by
WWW for three reasons:

\ls{1}
\begin{enumerate}
\item{It is simple to publish and retrieve information on it.}
\item{Its software is in the public domain.}
\item{Training would be less difficult as some people were already using it.}
\end{enumerate}
\ls{1.5}

Having decided upon a tool for community building, the next task was to extend
it to test the thesis of this research.

\section{A Community-enhancing Web Site}
\label{sec:design}
A community-enhancing information system should be designed to improve the
measures of community described in Chapter \ref{chap:introduction}.  The design
should enhance a person's feelings of importance and belonging in the
organization.  It should provide a way for users to become more self-aware of
the people and projects in their organization.  

Figure \ref{fig:UHPage} shows the ICS Home page, which overviews how department
information is organized.  There are 5 main areas of interest: research,
education, facilities, people and Web services.  Different features were
implemented in each area of the system in response to the design requirements.

\begin{figure}[htbp]
\centerline{\psfig{figure=UHPage.eps}}
\caption{A screen shot of the ICS Home page.}
\label{fig:UHPage}
\end{figure}

\subsection{Getting People Involved}
\label{sec:involved}
Part of feeling a sense of community has to do with feeling a sense of
importance and belonging in the organization.  Two major areas of the
information system were designed with this in mind.  Both of these areas
involve user participation.

The first area falls under the category of ``People''.  It contains two
subsections which were created to augment feelings of importance and belonging
in department members.  I included a listing of all home pages created by
department members where each item is a link to the respective page.
Participation in the creation of home pages was strictly voluntary and
unregulated.  That is, I did not disallow users from publishing on the Web nor
did I require approval of their work.  This feature supports importance because
it implies trust in giving users ``carte blanche'' on what they published and
offered them a choice of participation.  The second item is a Photoboard of
department members.  This part of the database is dynamically created on a
daily basis.  Users voluntarily posted their pictures on this board.  These
pages support feelings of belonging because only members of the ICS department
were listed.

The second area that contributes to importance and belonging appears under the
category of ``Web Services.''  Of those listed, the one that contributes to
importance is the feedback page.  This page was provided for users to ask for
help on system usage and to receive comments and suggestions on the system.
This supports importance because users were acknowledged and their comments
taken seriously.  During this study, all department members who requested help
got it or were referred to a more appropriate place.

\subsection{Staying Informed}

Becoming aware and maintaining awareness of people and their projects in an
organization contributes to its members' sense of community.  Four major areas
of the information system contribute to this purpose.

The first one comes under ``Research.''  In this area, readers learn about the
different research groups and their related projects.  There is also a
repository of publications by department members.  This information can help
members learn about resident experts within the department.

The second one falls under the ``Facilities'' category.  This includes a map of
the department.  It is an interactive map, allowing users to click on a portion
of the map and thus retrieve data about that place.  In the department map,
faculty and department offices are included.  Pictures of faculty and staff
members are on these Web pages and allow people to associate names with faces.

The third area is called ``People.''  Section \ref{sec:involved} mentioned how
the home page listing and photoboard contributed to a person's sense of
importance and belonging.  Here, it also contributes to self-awareness of
people in the organization.  There is a complete listing of the faculty with
links to their pages or, if they do not have one, a page describing their work
and interests.  A listing of students could not be compiled and published
without their expressed permission.  However, students who published home pages
implicitly gave their consent, so I have a listing of a subset of ICS students.
Like the department map, the photoboard helps users identify names with faces.
Also included is a page called ``Personal Interests.''  The home pages of
people sharing the same hobby are listed together.  This page is created by a
unix shell script executed daily which searches user home pages for listings of
their hobbies.  Hobbies must be listed in a certain format in a home page for
the program to process it correctly.  Instructions for this procedure are
posted online.  This feature does not address the measure of people knowing
each other personally, but it provides a way for people to make that attempt.
Upon learning that people share the same hobbies, they may optionally go to the
home page of those people with similar interests to learn more about them and
how to contact them.

The last area is ``Web Services.''  It includes a page that lists documents
that were recently changed in the database.  The database may be accessed
anytime at a user's leisure.  They are free to choose what information to
access.  As new items are added to the database, users have no way of knowing
what information has been added.  New information differs between users.  Users
are kept abreast of changes in the database by consulting the ``what's new''
page.  This document lists items changed in the last few days, week and month.
Unfortunately, Web servers do not automatically detect changes in the structure
of the database.  That is, they are unable to detect changes in the way
documents are linked.  Aside from parsing all HTML documents in the database,
the only automated way to discover changes in content is to check the date a
document was last modified.  However, this has the disadvantage that changes
made to a file due to a typographical error are indistinguishable from
significant changes of content.  In any case, until a better mechanism for
detecting modifications is developed, listing recently modified documents that
may include those with insignificant changes is better than not listing them at
all.  Change notification is important because it keeps users up to date on the
state of information.

Also included in the ``Web Services'' area is a page that gives statistics on
database accesses.  Users may want to utilize this feature in designing their
Web pages.  They can get statistics on when, from where and how often their
pages were accessed.  This is useful because users may publish information and
upon reading its statistics, find that people are not accessing those items.
They could thus modify their pages in response to these statistics.  This
contributes to self-awareness in the organization.

I implemented the above mentioned features and included other information
about the department such as degree programs, course descriptions, computing
policies, etc.  The ICS home page is accessible at {\tt http://www.ics.hawaii.edu/}.
