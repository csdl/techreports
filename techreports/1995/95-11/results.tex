%%%%%%%%%%%%%%%%%%%%%%%%%%%%%% -*- Mode: Latex -*- %%%%%%%%%%%%%%%%%%%%%%%%%%%%
%% results.tex -- 
%% Author          : Rosemary Andrada
%% Created On      : Sun Mar 12 18:24:26 1995
%% Last Modified By: Rosemary Andrada
%% Last Modified On: Fri Jul  7 11:30:13 1995
%% Status          : Unknown
%% RCS: $Id: results.tex,v 1.3 1995/07/05 00:33:23 rosea Exp rosea $
%%%%%%%%%%%%%%%%%%%%%%%%%%%%%%%%%%%%%%%%%%%%%%%%%%%%%%%%%%%%%%%%%%%%%%%%%%%%%%%
%%   Copyright (C) 1995 University of Hawaii
%%%%%%%%%%%%%%%%%%%%%%%%%%%%%%%%%%%%%%%%%%%%%%%%%%%%%%%%%%%%%%%%%%%%%%%%%%%%%%%
%% 

\ls{1.5}
\chapter{Results of the Case Study}
\label{chap:results}

%\subsection{Data analysis}
% What are the range of outcomes from this research?  What is the
% significance of each outcome?

% What data will be collected?  What are example instances of this data?  How
% will the data be analyzed?  How will I know how well I've addressed it?

This chapter reviews results of the case study.  Analysis of this data may
reveal how the Web server might have influenced changes in community.  Data was
collected in two ways.  The first way was through the responses received from
the pre- and post-test questionnaires.  The second way was by collecting data
directly from the Web server using its log of all requests made.  The next
chapter discusses the implications of these results.

% How will questionnaire data be analyzed?  What constitutes success?  Will
% any data from server statistics be collected/used/how?

\section{Overview of Questionnaire Responses}
This section assesses the change in level of community in the department by
comparing the results of the pre-test and post-test questionnaires.  I first
discuss how each questionnaire is structured and what the various responses
indicate about the sense of community.  Slightly different versions of the
questionnaire were given to two groups, the faculty \& staff and the students.
They only differ in two questions which ask how well the respondents know
people from their department.  These two questions are meant to gather
information about inter-group relationships.  That is, in the questionnaire
given to the faculty \& staff, they are asked how well they know members of the
other group, the students.  Similarly, the questionnaire given to the students
asked how well they know members of the faculty \& staff.  This difference in
the questionnaires came about because people are inclined to be more familiar
with people they work with on a regular basis and I wanted to analyze
cross-group interactions as they pertain to community.  I will now discuss the
results for each group in turn.

\subsection{Questionnaire Structure}
The pre-test (see Appendix \ref{sec:pre-test}) and post-test questionnaires
(see Appendix \ref{sec:post-test}) for each group differ in that the post-test
questionnaire has seven questions regarding the World Wide Web that replace the
single question on the WWW in the pre-test version.  The pre-test questionnaire
was administered to assess the pre-existing, or baseline sense of community.
The post-test questionnaire was administered to see how the various elements of
community had changed in the department over the course of a semester with the
introduction of the system.  The questionnaire is structured as follows:

\ls{1}
\begin{itemize}
\item{Questions 1, 2 and 3 request numeric answers.  The first question
  determines which group the respondent works with the most.  The second and
  third questions establish how many members of the department the respondent
  knows personally and how many they could identify by face.  Answers to
  these questions indicate how many other members of the department people
  are familiar with or know on a personal basis.}
%It also tells us how far-reaching community is for a single person.
\item{Questions 5 and 6 address the respondent's sense of significance and
  solidarity in the department.}
\item{Questions 4, 7 and 8 are fact-finding questions.  They have right and
  wrong answers.  These responses reveal how self-aware the respondents are
  of people in their own organization.  They show if people know how many
  members there are in the department as well as who the resident experts are
  and what different projects are being conducted by various people.}
\item{Question 9 suggests how different people like to communicate with each
  other.}
\item{Question 10 assesses how many people needed to be introduced
  to the software needed for Web browsing.}
\item{Question 11 is open-ended to provide feedback on what people think about
  the sense of community within the department.}
\end{itemize}
\ls{1.5}

In addition to these 11 questions, the post-test questionnaire asked seven
other questions, most of which relate to the World Wide Web.  They replace question
10 as mentioned above.  The purpose and structure of these questions follow:

\ls{1}
\begin{itemize}
\item{Question 10 asks how people discovered the ICS Web site, if at all.  This
  is important in determining if a lack of advertising inhibited usage of the
  information system.}
\item{Question 11 is useful for determining whether a new user had adequate
  resources to get training in using the Web information system.}
\item{Questions 12-15 relate to what extent people were using the system, how
  comfortable they were with it, whether they found it useful and why they
  bothered to use it at all.}
\item{Question 16 was asked to see if the respondents had filled out the
  Pre-Test questionnaire as well.}
\end{itemize}
\ls{1.5}

\subsection{The Faculty \& Staff}

Table \ref{tab:numResponses} shows 10 pre-test responses out of a possible 21
from the faculty and staff.  However, there were only 7 post-test responses
from them.  There are 19 faculty and 3 staff members.  With such few responses,
it is difficult to tell how representative this data is of the whole group.

\begin{table}[ht]
\caption{Number of Questionnaire Responses}
\begin{center}
\begin{tabular}{|c|c|c|c|}\hline
  & {\bf Pre-Test} & {\bf Post-Test} & {\bf Max Possible} \\ \hline 
  {\bf Faculty/Staff} & 10 & 7 & 21 \\ \hline 
  {\bf Students} & 43 & 84 & 278 \\ \hline 
\end{tabular}
\end{center}
\label{tab:numResponses}
\end{table}

\subsubsection{Questions 1, 2 and 3}

Question 1 in Table \ref{tab:question1a} shows that these respondents appear to
mix well with other subgroups in the department, as there were responses for
every subgroup, (i.e. faculty, staff, graduate students and undergraduate
students) in the pre-test and in the post-test.

\begin{table}[htb]
\caption{Faculty \& Staff: Question 1 results}
\begin{center}
{Which group do you work with most?}\\[1ex]
\begin{tabular}{|l|c|c|} \hline
% \multicolumn{3}{l}{\rule[-5mm]{0mm}{10mm}\bf Which group do you work with
% most?}\\ \hline 
 \multicolumn{1}{|c|}{\bf Group} & 
 \multicolumn{1}{|c|}{\bf Pre-Test} &
 \multicolumn{1}{|c|}{\bf Post-Test} \\ \hline
 Faculty & \multicolumn{1}{|c|}{4} & \multicolumn{1}{|c|}{3} \\ \hline 
 Staff & \multicolumn{1}{|c|}{3} & \multicolumn{1}{|c|}{1} \\ \hline 
 Graduate students & \multicolumn{1}{|c|}{2} & \multicolumn{1}{|c|}{2} \\ \hline 
 Undergraduate students & \multicolumn{1}{|c|}{1} & \multicolumn{1}{|c|}{1} \\ \hline
\end{tabular}
\end{center}
\label{tab:question1a}
\end{table}

Since it is more likely for faculty and staff to know people in their own
subgroups, I asked if they knew people from a different subgroup, students.
The pre-test results for questions 2 and 3 in Tables \ref{tab:question2a} and
\ref{tab:question3a} shows that few faculty and staff members initially knew
many of the students while most knew only a handful.  Most of them felt they
could identify less than 20 of their students.  In the post-test, they
collectively seemed to improve in getting to know their students personally and
being able to identify them by face.  This can be inferred because they all but
one of the respondents indicated they also filled out the pre-test
questionnaire (see Table \ref{tab:question16a}).

\begin{table}[htb]
\caption{Faculty \& Staff: Question 2 results}
{How many ICS students do you feel you know personally?}
\begin{center}
\begin{tabular}{|l|l|l|} \hline
% \multicolumn{3}{l}{\rule[-0mm]{0mm}{10mm}\bf How many ICS students do you 
% feel}  \\
% \multicolumn{3}{l}{\rule[-5mm]{0mm}{10mm}\bf you know personally?}\\ \hline
 \multicolumn{1}{|c|}{\bf \# Students} &
 \multicolumn{1}{|c|}{\bf Pre-Test} &
 \multicolumn{1}{|c|}{\bf Post-Test} \\ \hline
 0-10    & \multicolumn{1}{|c|}{6} & \multicolumn{1}{|c|}{2} \\ \hline 
 10-20   & \multicolumn{1}{|c|}{2} & \multicolumn{1}{|c|}{3} \\ \hline 
 20-30   & \multicolumn{1}{|c|}{0} & \multicolumn{1}{|c|}{2} \\ \hline 
 over 30 & \multicolumn{1}{|c|}{2} & \multicolumn{1}{|c|}{0} \\ \hline
\end{tabular}
\end{center}
\label{tab:question2a}
\end{table}

\begin{table}[htb]
\caption{Faculty \& Staff: Question 3 results}
{How many ICS students do you think you could name, given their face?}
\begin{center}
\begin{tabular}{|l|l|l|} \hline
%%%  \multicolumn{3}{l}{\bf How many ICS students do you think} \\
%%%  \multicolumn{3}{l}{\rule[-5mm]{0mm}{10mm}\bf  you could name, 
%%%  given their face?}\\ \hline
 \multicolumn{1}{|c|}{\bf \# Students} &
 \multicolumn{1}{|c|}{\bf Pre-Test} &
 \multicolumn{1}{|c|}{\bf Post-Test} \\ \hline
 0-10    & \multicolumn{1}{|c|}{2} & \multicolumn{1}{|c|}{0}\\ \hline 
 10-20   & \multicolumn{1}{|c|}{4} & \multicolumn{1}{|c|}{4}\\ \hline 
 20-30   & \multicolumn{1}{|c|}{1} & \multicolumn{1}{|c|}{3}\\ \hline 
 over 30 & \multicolumn{1}{|c|}{3} & \multicolumn{1}{|c|}{0}\\ \hline
\end{tabular}
\end{center}
\label{tab:question3a}
\end{table}

\subsubsection{Questions 5 and 6}

Question 5 (Table \ref{tab:question5a}) asked if people felt they played an
important role in the department.  Nine responded affirmatively and there was
one with no response.  Their positive responses were based on their positions
in the department.  They each had a job which they felt was important.  The
post-test shows the same sentiment as six members felt they played an important
role and one gave an unclear answer.  Two of them felt important through their
teaching and research.  One felt ``not any more significant than any other
faculty member'', which equates to a ``yes'' since the faculty generally tended
to feel a sense of significance in the department.  The last respondent felt
less important in that he ``just'' handled classes.  But he felt some
importance in that he had some leeway in determining the class content.

\ls{1}
\begin{table}[htb]
\caption{Faculty \& Staff: Question 5 results}
{Do you feel you play a significant role in the ICS department?  Why
  or why not?} 
\begin{center}
\begin{tabular}{|c|c|c|} \hline
% \multicolumn{3}{l}{\rule[-5mm]{0mm}{10mm}\bf Do you feel you play a
% significant role in the ICS department?}\\ \hline
  & {\bf Pre-Test} & {\bf Post-Test} \\ \hline 
 Yes & 9 & 6 \\ \hline 
 No & 0 & 0 \\ \hline 
\end{tabular}
\end{center}
\label{tab:question5a}
\end{table}
\ls{1.5}

%!!! double-check numbers
In question 6 (see Table \ref{tab:question6a}), I asked if people felt a sense
of belonging in the department.  As with the previous question, nine said they
did while one only felt ``so-so''.  Their sense of belonging seemed to be less
related to their jobs.  Some felt they belonged just because they had a
``position'' in the department.  One felt that he or she belonged because of
participation in departmental affairs while the person that responded ``so-so''
said that the department ``does not do much at all anyway.''  Yet another
person felt belonging through ``longevity, if nothing else.''  In the
post-test, all seven people said they felt a sense of belonging.  One response
was that the person felt that he ``belonged here more than anywhere else''.

\ls{1}
\begin{table}[htb]
\caption{Faculty \& Staff: Question 6 results}
{Do you feel a sense of belonging in the ICS department?  Why or why not?}
\begin{center}
\begin{tabular}{|c|c|c|} \hline
% \multicolumn{3}{l}{\rule[-5mm]{0mm}{10mm}\bf Do you feel a sense of belonging
% in the ICS department?}\\ \hline
  & {\bf Pre-Test} & {\bf Post-Test} \\ \hline 
 Yes & 9 & 7 \\ \hline 
 No & 0 & 0 \\ \hline 
\end{tabular}
\end{center}
\label{tab:question6a}
\end{table}
\ls{1.5}

\subsubsection{Questions 4, 7 and 8}

\ls{1}
\begin{table}[htb]
\caption{Faculty \& Staff: Question 4 results}
  {Please provide three numeric estimates of how many faculty, graduate
  students and undergraduates you think are in the department:} 
\begin{center}
\begin{tabular}{|l|c|c|c|c|c|} \hline
%  \multicolumn{10}{l}{\rule[-5mm]{0mm}{10mm}\bf Please provide numeric
%  estimates for the following groups:}\\ \hline
  \multicolumn{6}{|c|}{\bf Pre-Test} \\ \hline
  \multicolumn{1}{|c} {}& \multicolumn{1}{c|}{Actual} &
  \multicolumn{4}{|c|}{Responses} \\   \hline 
  Faculty        &  18  & 0-10  &  11-15 & 16-20 & 21$+$  \\ \cline{3-6}
                 &      & 0     &  3     & 7     & 0      \\   \hline
  Graduates      &  38  & 0-20  & 21-30 & 31-40 & 41$+$ \\   \cline{3-6}
                 &      & 0     & 2     & 6     & 2     \\   \hline
  Undergraduates & 240 & 0-100 & 101-150 & 151-200 & 201$+$ \\ \cline{3-6}
                 &     &  0    & 1       & 5       & 4      \\ \hline \hline
  \multicolumn{6}{|c|}{\bf Post-Test} \\ \hline
  \multicolumn{1}{|c} {}& \multicolumn{1}{c|}{Actual} &
  \multicolumn{4}{|c|}{Responses} \\   \hline 
  Faculty        &  18  & 0-10 & 11-15 & 16-20 & 21$+$ \\ \cline{3-6}
                 &      & 0    &  3    & 3     & 0     \\ \hline
  Graduates      &  38  & 0-20 & 21-30 & 31-40 & 41$+$ \\ \cline{3-6}
                 &      & 0    &  1    & 3     &  2     \\ \hline
  Undergraduates & 240 & 0-100 & 101-150 & 151-200 & 201$+$ \\ \cline{3-6}
                 &     & 2     & 3       & 1       & 0      \\ \hline
\end{tabular}
\end{center}
\label{tab:question4a}
\end{table}
\ls{1.5}

Question 4 (Table \ref{tab:question4a}) asked respondents to give numeric
estimates of how many faculty, graduate and undergraduate students they think
are in the department.  The faculty and staff did well in the pre-test
identifying within a reasonable range the size of each subgroup.  In the
post-test, they did not do as well, where half guessed incorrectly on the
faculty and all guessed too low on the undergraduates.

%\ls{1}
%\begin{table}[htbp]
%\caption{Faculty \& Staff: Question 7 results}
%{What ICS research projects are you aware of?  Please briefly list the
%  projects you can think of immediately and any faculty, staff, or students
%  you know who are involved in them. (If you know many people involved in a
%  particular project, then simply estimate the number of involved people
%  and provide that number.)}
%\begin{center}
%\begin{tabular}{|l|c|c|} \hline
% \multicolumn{3}{l}{\rule[-5mm]{0mm}{10mm}\bf What ICS research projects are
% you aware of?}\\ \hline
% {\bf Research Group/Project Leader} & {\bf Pre-Test} & {\bf Post-Test} \\ \hline 
% {CSDL: Johnson}     & 6 & 6 \\ \hline 
% {J. Corbett}        & 5 & 3 \\ \hline 
% {PACCOM: Nielsen}   & 5 & 1 \\ \hline 
% {SERL: Miyamoto}    & 5 & 3 \\ \hline 
% {S. Itoga}          & 4 &  \\ \hline 
% {Crosby/Peterson}   & 4 & 2 \\ \hline 
% {J. Lee}            & 3 & 1 \\ \hline 
% {J. Stelovsky}      & 3 &  \\ \hline 
% {ICL:Lew}           & 2 & 2 \\ \hline 
% {D. Chin}           & 1 & 1\\ \hline 
% {D. Watanabe}       & 1 &  \\ \hline 
% {W. Gersch}         &   & 2 \\ \hline 
% {R. Andrada}        & 1 &  \\ \hline 
%\end{tabular}
%\end{center}
%\label{tab:question7a}
%\end{table}
%\ls{1.5}

%\ls{1}
%\begin{table}[htbp]
%\caption{Faculty \& Staff: Question 8 results}
%{Assume you had a question about one of the following topics.  For
%  each of the topics, name one or two people in the department who you
%  would want to ``just stop by'' to talk with about it, or leave it blank
%  if you can't think of anybody.}
%\begin{center}
%\begin{tabular}{|l|l|c|c|} \hline
% \multicolumn{6}{l}{\rule[-5mm]{0mm}{10mm}\bf Name the resident expert for each
% subject matter:} \\ \hline
% \multicolumn{1}{|c|}{\bf Field}  & \multicolumn{1}{|c|}{\bf Expert} 
% & \multicolumn{1}{|c|}{\bf Pre-Test} 
% & \multicolumn{1}{|c|}{\bf Post-Test} \\ \hline 
% {Departmental rules}        & D. Watanabe  & 5 & 4 \\ \cline{2-4}
%                             & Others       & 8 & 5 \\ \hline 
% {Artificial intelligence}   & D. Chin      & 8 & 6 \\ \cline{2-4}
%                             & Others       & 3 & 1 \\ \hline
% {Software Engineering}      & P. Johnson   & 8 & 6 \\ \cline{2-4}
%                             & Others       & 6 & 2 \\ \hline
% {Computer networks}         & T. Nielsen   & 9 & 6 \\ \cline{2-4}
%                             & Others       & 1 & 3 \\ \hline
% {Hypertext and multimedia}  & J. Stelovsky & 8 & 6 \\ \cline{2-4}
%                             & Others       & 2 & 1 \\ \hline
% {Cognitive Science}         & M. Crosby    & 8 & 5 \\ \cline{2-4}
%                             & Others       & 3 & 0 \\ \hline
% {Human Computer Interaction}& M. Crosby    & 8 & 6 \\ \cline{2-4}
%                             & Others       & 3 & 1 \\ \hline
% {Computer games}            & D. Chin      & 1 &   \\ \cline{2-4}
%                             & Others       & 4 & 4 \\ \hline
% {Employment opportunities}  &              &   &   \\ \cline{2-4}
%                             & Others       & 6 & 3 \\ \hline
%\end{tabular}
%\end{center}
%\label{tab:question8a}
%\end{table}
%\ls{1.5}

Responses to question 7 showed that about half of the respondents could name
the prominent research groups such as CSDL, SERL and PACCOM and some of the
individual research projects going on in the department.  They mostly listed
projects of their colleagues' and only one person mentioned this graduate
student's work.  The post-test responses yielded the same results but
identified fewer projects.  

The faculty and staff did well in question 8 where 80\% - 90\% of them
identified a resident expert in a particular subject.  The two exceptions are
that they did not identify a games expert or an employment opportunities
expert.  As in the pre-test questionnaire, the post-test results were also
accurate except for identifying the games expert and the employment
opportunities expert.  

%In Table \ref{tab:question8a}, an expert is listed followed by the count of
%``votes'' he or she received.  The row ``Others'' refers to the total count of
%votes for other people who garnered just two or less votes.

\subsubsection{Questions 9 and 10}

Question 9 (see Table \ref{tab:question9a}) suggests how these respondents
prefer to communicate.  Most people are daily users of email and telephone.
Informal meetings ran the gamut from daily use to never doing it at all.
Formal meetings were either done monthly or rarely to never.  The results are
the same in the post-test which shows again that most people are daily users of
email and the telephone.  Informal meetings take place with varying frequencies
and formal meetings are held mostly monthly.

Question 10 (see Table \ref{tab:Prequestion10a}) revealed that 7 of the
respondents already used the World Wide Web while 3 did not.

\ls{1}
\begin{table}[htb] 
\caption{Faculty \& Staff: Question 9 results}
{For each of the following types of communication, indicate whether
  you use it Daily, Weekly, Monthly, or Never, to communicate with other
  people in the department.}
\begin{center}
\begin{tabular}{|l|c|c|c|c|} \hline
% \multicolumn{9}{l}{\rule[-5mm]{0mm}{10mm}\bf How often do you use the
% following types of communication?} \\ \hline
 & \multicolumn{4}{|c|}{\bf Pre-Test} \\ \hline 
 & {\bf Daily} & {\bf Weekly} & {\bf Monthly} & {\bf Never} \\ \hline
 {Email}            & 9 & 0 & 0 & 1 \\ \hline 
 {Telephone}        & 8 & 2 & 0 & 0 \\ \hline 
 {Fax}              & 0 & 2 & 3 & 5 \\ \hline 
 {Informal Meetings}& 2 & 3 & 2 & 3 \\ \hline 
 {Formal Meetings}  & 0 & 0 & 6 & 4 \\ \hline \hline
 & \multicolumn{4}{|c|}{\bf Post-Test} \\ \hline 
 & {\bf Daily} & {\bf Weekly} & {\bf Monthly} & {\bf Never} \\ \hline
 {Email}            & 6 & 0 & 0 & 0 \\ \hline 
 {Telephone}        & 6 & 0 & 1 & 0 \\ \hline 
 {Fax}              & 0 & 0 & 4 & 2 \\ \hline 
 {Informal Meetings}& 1 & 4 & 2 & 0 \\ \hline 
 {Formal Meetings}  & 0 & 1 & 4 & 1 \\ \hline 
\end{tabular}
\end{center}
\label{tab:question9a}
\end{table}
\ls{1.5}

\ls{1}
\begin{table}[htb]
\caption{Faculty \& Staff: Pre-Test question 10 results}
{Do you use Mosaic, or any other Web browser to access the World Wide
  Web?}
\begin{center}
\begin{tabular}{|c|c|} \hline
% \multicolumn{2}{l}{\rule[-5mm]{0mm}{10mm}\bf Do you use any Web browser to
% access the World Wide Web?}\\ \hline
  & {\bf Pre-Test} \\ \hline 
 Yes & 7 \\ \hline 
 No  & 3 \\ \hline 
\end{tabular}
\end{center}
\label{tab:Prequestion10a}
\end{table}
\ls{1.5}

\subsubsection{Post-Test Questions 10-16}

In questions 10-15 (see Tables
\ref{tab:Postquestion10a}-\ref{tab:question15a}), it seems that all seven
people knew of the ICS Web site, while two of them had no knowledge of training
sessions offered throughout the semester.  Most used it with relatively low
frequency.  Five of them felt competent in accessing information on the Web and
three felt they could publish information on it.  Five people felt that the ICS
departmental information was useful.  One comment was that it was not kept up
to date.  Others thought it was good to know ``who's who in the department and
what they do.''  These people found the Web to be fun and exciting and an easy
way to navigate the Internet.

\ls{1}
\begin{table}[htbp]
\caption{Faculty \& Staff: Post-Test question 10 results}
{Did you know the ICS Department has a Web site? If so, how did you
  discover this?}
\begin{center}
\begin{tabular}{|l|c|} \hline
% \multicolumn{2}{l}{\rule[-5mm]{0mm}{10mm}\bf How did you discover the ICS Web
% site?} \\ \hline
  & {\bf Post-Test} \\ \hline 
 {Flyers} & 0 \\ \hline 
 {Email announcements} & 2 \\ \hline 
 {Colleague or friend} & 5 \\ \hline 
 {I stumbled on it by accident} & 0 \\ \hline 
\end{tabular}
\end{center}
\label{tab:Postquestion10a}
\end{table}
\ls{1.5}

\ls{1}
\begin{table}[htbp]
\caption{Faculty \& Staff: Question 11 results}
{The WWW training sessions offered by the department were...}
\begin{center}
\begin{tabular}{|l|c|} \hline
% \multicolumn{2}{l}{\rule[-5mm]{0mm}{10mm}\bf The WWW training sessions offered
% by the department were...} \\ \hline
  & {\bf Post-Test} \\ \hline 
 {Useful} & 4 \\ \hline 
 {Not very useful} & 0 \\ \hline 
 {Frequently and flexibly scheduled} & 0 \\ \hline 
 {Not easily accessible} & 0 \\ \hline 
 {What training session?} & 2 \\ \hline 
\end{tabular}
\end{center}
\label{tab:question11a}
\end{table}
\ls{1.5}

\ls{1}
\begin{table}[htbp]
\caption{Faculty \& Staff: Question 12 results}
{How often have you been accessing information from the ICS
  Department's Web site?}
\begin{center}
\begin{tabular}{|l|c|} \hline
% \multicolumn{2}{l}{\rule[-5mm]{0mm}{10mm}\bf How often have you accessed the
% ICS Web site?} \\ \hline
  & {\bf Post-Test} \\ \hline 
 {Rarely or never} & 3 \\ \hline 
 {A few times a month} & 2 \\ \hline 
 {A few times a week} & 1 \\ \hline 
 {Almost everyday} & 1 \\ \hline 
\end{tabular}
\end{center}
\label{tab:question12a}
\end{table}
\ls{1.5}

\ls{1}
\begin{table}[htbp]
\caption{Faculty \& Staff: Question 13 results}
{I feel I am a competent...}
\begin{center}
\begin{tabular}{|l|c|} \hline
% \multicolumn{2}{l}{\rule[-5mm]{0mm}{10mm}\bf I feel I am a competent...} \\ \hline
%  & {\bf Post-Test} \\ \hline 
 {Web surfer} & 5 \\ \hline 
 {Web publisher} & 3 \\ \hline 
 {neither} & 2 \\ \hline 
\end{tabular}
\end{center}
\label{tab:question13a}
\end{table}
\ls{1.5}

\ls{1}
\begin{table}[htbp]
\caption{Faculty \& Staff: Question 14 results}
{Did you find the information presented about the department useful?
  If so, in what ways?  If not, why not and how would you improve it?}
\begin{center}
\begin{tabular}{|c|c|} \hline
% \multicolumn{2}{l}{\rule[-5mm]{0mm}{10mm}\bf Did you find the ICS Web site
% useful?} \\ \hline
  & {\bf Post-Test} \\ \hline 
 Yes & 5 \\ \hline 
 No  & 0 \\ \hline 
\end{tabular}
\end{center}
\label{tab:question14a}
\end{table}
\ls{1.5}

\ls{1}
\begin{table}[htbp]
\caption{Faculty \& Staff: Question 15 results}
{I use the World Wide Web because...}
\begin{center}
\begin{tabular}{|l|c|} \hline
% \multicolumn{2}{l}{\rule[-5mm]{0mm}{10mm}\bf I use the World Wide Web
% because...}  \\ \hline
  & {\bf Post-Test} \\ \hline 
 {The Web is fun and exciting} & 3 \\ \hline 
 {Everyone else seems to be using it} & 1 \\ \hline 
 {It makes it easy for me to navigate the Internet} & 5 \\ \hline 
 {I don't use it} & 1 \\ \hline 
 {Other: Information is available} & 1 \\ \hline 
\end{tabular}
\end{center}
\label{tab:question15a}
\end{table}
\ls{1.5}

\ls{1}
\begin{table}[htbp]
\caption{Faculty \& Staff: Question 16 results}
{Did you complete the Pre-Test questionnaire?}
\begin{center}
\begin{tabular}{|c|c|} \hline
% \multicolumn{2}{l}{\rule[-5mm]{0mm}{10mm}\bf Did you complete the Pre-Test
% questionnaire?}  \\ \hline
  & {\bf Post-Test} \\ \hline 
 Yes & 6 \\ \hline 
 No  & 1 \\ \hline 
\end{tabular}
\end{center}
\label{tab:question16a}
\end{table}
\ls{1.5}

\subsubsection{Question 17}

The last question asked people to comment on the sense of community in the
department.  There were only 3 in the pre-test, none of which were very
optimistic.  

\ls{1}
\begin{quote}
``Given the message traffic, I sense a very low morale within the
department.  Since I only lecture and do not work on site, I have not
personally observed this condition.''
\end{quote}
\ls{1.5}

\ls{1}
\begin{quote}
``There doesn't seem to be any.  Probably due to the varying research interests
and an apparent lack of goal and direction for the department.''
\end{quote}
\ls{1.5}

\ls{1}
\begin{quote}
``It's more of a sense of communities - alternate realities.''
\end{quote}
\ls{1.5}

Here is the only comment from the post-test:

\ls{1}
\begin{quote}
``Faculty: somewhat divisive\\
Chinese students: a strong community\\
Grad students who use K304: a strong sense of community\\
Undergrads: There are community pockets!''
\end{quote}
\ls{1.5}

\subsubsection{Summary}

%summing up - pre
In summary, the pre-test results for the faculty and staff showed:

\ls{1}
\begin{itemize}
\item{Most worked with other faculty members.}
\item{Most knew less than 10 students personally and could identify less than
20 of them by face.}
\item{They accurately estimated the number of faculty but not the students.}
\item{They felt a sense of significance and belonging in the department.}
\item{They were able to accurately identify research projects and subject experts.}
\end{itemize}
\ls{1.5}

From the pre-test responses, the faculty and staff seem to be a group that
knows each other well if not all of the students.  For the most part, they all
felt a sense of significance and solidarity in this community.  However, there
were a few responses that seem to indicate some lack of a common goal or
direction.

%summing up - post
Here are the results from the post-test:

\ls{1}
\begin{itemize}
\item{Most worked with other faculty members.}
\item{Most knew less than 20 students personally and could identify them by
face.}
\item{They only accurately estimated the size of graduate students.}
\item{They felt a sense of significance and belonging in the department.}
\item{They were able to accurately identify research projects and subject
experts.}
\item{Most knew of the ICS Web site and found it useful.}
\item{They accessed the ICS Web site with low frequency.}
\end{itemize}
\ls{1.5}

With such few responses, it is impossible to speculate on any changes within
the faculty.  Although they could not identify the sizes of each subgroup, they
were aware of the various projects and resident experts on different subjects.
They seemed to have a strong sense of importance and belonging.  All but two
felt they were competent users of the Web.  To what extent can community
building be attributed to the introduction of the ICS information system on the
Web?  It is unlikely that the ICS Web site greatly affected their sense of
community since they did not use it with great frequency.  On a brighter side,
they did use the system to both get and give information and noted its
usefulness.

\subsection{The Students}

I received 43 pre-test responses and 84 post-test responses out of a possible
278 from graduate and undergraduate students (see Table
\ref{tab:numResponses}).  There are approximately 38 graduate students and 240
undergraduates.

\subsubsection{Questions 1, 2 and 3}

When asked which group they work with the most (Table \ref{tab:question1s}),
about 70\% said they worked most with undergraduates while 26\% said they
worked most with graduate students.  Only 3 people said they worked most with
the faculty.  The post-test yielded the same results where 73\% said they
worked most with undergraduates while 20\% said they worked most with graduate
students.  Only 1\% worked most with the faculty and 6\% worked most with the
staff.

\begin{table}[htb]
\caption{Students: Question 1 results}
\begin{center}
{Which group do you work with most?}\\[1ex]
\begin{tabular}{|l|c|c|} \hline
% \multicolumn{3}{l}{\rule[-5mm]{0mm}{10mm}\bf Which group do you work with
% most?}\\ \hline 
 \multicolumn{1}{|c|}{\rule[-2mm]{0mm}{6mm}\bf Group} & 
 \multicolumn{1}{|c|}{\rule[-2mm]{0mm}{6mm}\bf Pre-Test} &
 \multicolumn{1}{|c|}{\rule[-2mm]{0mm}{6mm}\bf Post-Test} \\ \hline
 Faculty & \multicolumn{1}{|c|}{3} & \multicolumn{1}{|c|}{1} \\ \hline 
 Staff   & \multicolumn{1}{|c|}{0} & \multicolumn{1}{|c|}{5} \\ \hline 
 Graduate students & \multicolumn{1}{|c|}{11} & \multicolumn{1}{|c|}{17} \\ \hline 
 Undergraduate students & \multicolumn{1}{|c|}{30} & \multicolumn{1}{|c|}{61} \\ \hline
\end{tabular}
\end{center}
\label{tab:question1s}
\end{table}

In question 2 (Table \ref{tab:question2s}), I asked how many faculty members
the students knew personally and drew a near unanimous ``less than 10'' in the
pre-test.  In the post-test, 85\% knew less than 5, 14\% knew less than 10 and
1\% knew less than 15.  Note the different scaling in the pre-test and
post-test questionnaires.

\begin{table}[htb]
\caption{Students: Question 2 results}
{How many ICS faculty members do you feel you know personally?}
\begin{center}
\begin{tabular}{|l|l|l|l|} \hline
% \multicolumn{4}{l}{\rule[-0mm]{0mm}{10mm}\bf How many faculty members do you 
% feel}  \\
% \multicolumn{4}{l}{\rule[-5mm]{0mm}{10mm}\bf you know personally?}\\ \hline
 \multicolumn{2}{|c|}{\rule[-2mm]{0mm}{6mm}\bf Pre-Test} &
 \multicolumn{2}{|c|}{\rule[-2mm]{0mm}{6mm}\bf Post-Test} \\ \hline
 \# Faculty & \# Responses & \# Faculty & \# Responses \\ \hline
 0-10    & \multicolumn{1}{|c|}{42} & 0-5 &\multicolumn{1}{|c|}{71} \\ \hline 
 10-20   & \multicolumn{1}{|c|}{1} & 6-10 & \multicolumn{1}{|c|}{12} \\ \hline 
 20-30   & \multicolumn{1}{|c|}{0} & 11-15 & \multicolumn{1}{|c|}{1} \\ \hline 
 over 30 & \multicolumn{1}{|c|}{0} & 16-20 & \multicolumn{1}{|c|}{0} \\ \hline
\end{tabular}
\end{center}
\label{tab:question2s}
\end{table}

In question 3 (Table \ref{tab:question3s}), most students felt they could
identify less than 10 ICS faculty members by face.  About 16\% felt they could
name less than 20 ICS faculty.  Strangely, one person said they could do it for
over 30 ICS faculty.  This is an impossible feat since there are only 18 ICS
faculty members.  The post-test results show that 45\% could identify less than
a handful of the faculty while 40\% felt they could identify less than 10 of
them.  The scaling was a little different in the pre-test questionnaire.  But
there was some improvement in that now 12\% can identify less than 15 faculty
members.

\ls{1}
\begin{table}[htb]
\caption{Students: Question 3 results}
{How many ICS faculty members do you think you could name,
  given their face?}
\begin{center}
\begin{tabular}{|l|l|l|l|} \hline
% \multicolumn{4}{l}{\rule[-0mm]{0mm}{10mm}\bf How many faculty members do you 
% think}  \\
% \multicolumn{4}{l}{\rule[-5mm]{0mm}{10mm}\bf you could name, 
% given their face?}\\ \hline
 \multicolumn{2}{|c|}{\rule[-2mm]{0mm}{6mm}\bf Pre-Test} &
 \multicolumn{2}{|c|}{\rule[-2mm]{0mm}{6mm}\bf Post-Test} \\ \hline
 \# Faculty & \# Responses & \# Faculty & \# Responses \\ \hline
 0-10    & \multicolumn{1}{|c|}{35} & 0-5 &\multicolumn{1}{|c|}{38} \\ \hline 
 10-20   & \multicolumn{1}{|c|}{7} & 6-10 & \multicolumn{1}{|c|}{34} \\ \hline 
 20-30   & \multicolumn{1}{|c|}{0} & 11-15 & \multicolumn{1}{|c|}{10} \\ \hline 
 over 30 & \multicolumn{1}{|c|}{1} & 16-20 & \multicolumn{1}{|c|}{1} \\ \hline
\end{tabular}
\end{center}
\label{tab:question3s}
\end{table}
\ls{1.5}

\subsubsection{Questions 5 and 6}

Question 5 (Table \ref{tab:question5s}) asked if people felt they played an
important role in the department.  About 14\% responded ``yes'' while an
overwhelming 81\% said ``no''.  As with the faculty and staff, those that
responded positively related their response to their job in the department as
being a teaching assistant or lab monitor.  These respondents seemed to think
that being paid to do something affirms one's importance.  Others felt less
important because they were not active in ICS related projects.  Some felt
unimportant because of a lack of communication between faculty, staff and
students.  One person commented on his unimportance being related to a lack of
regard for the course/instructor evaluations.  He felt students were not taken
seriously in this respect.  A large number of responses indicated that they did
not feel they played an important role because they were ``just'' students who
only attended classes while a few said they were important for the very reason
that ``students are the most important part in the university.''

\ls{1}
\begin{table}[htb]
\caption{Students: Question 5 results}
{Do you feel you play a significant role in the ICS department?  Why
  or why not?} 
\begin{center}
\begin{tabular}{|c|c|c|} \hline
% \multicolumn{3}{l}{\rule[-5mm]{0mm}{10mm}\bf Do you feel you play a
% significant role in the ICS department?}\\ \hline
  & {\bf Pre-Test} & {\bf Post-Test} \\ \hline 
 Yes & 6 & 12 \\ \hline 
 No & 35 & 60 \\ \hline 
\end{tabular}
\end{center}
\label{tab:question5s}
\end{table}
\ls{1.5}

The post-test results of question 5 were the same as in the pre-test.
Approximately 14\% responded that they felt they played a significant role in
the department while 71\% felt they did not.  Somehow students felt important
if there was some tangible measure of that importance.  For example, the TAs
and monitors again were some of the few that felt important because of their
positions.  Others felt they were important because they were working with some
professor on a specific project.  But there were still those few that did not
need anything tangible and felt they were important just because they are
students representing the department.  The naysayers felt less important for
many of the same reasons as before.  They were just ``peon'' students who only
took classes and did little else to further the department.  The lack of
communication between the faculty and students also prevented students from
feeling like they were important enough to be kept informed.  Just as they did
not feel they were receiving information from the faculty, the students neither
felt they had any means of expressing their views to them.  They had ``no real
venue to express opinions about policy...''.  They also felt their course
evaluations were ignored as some professors continued in the same teaching
style found to be difficult for some.

Perhaps the biggest complaint was a feeling of unworthiness among the
undergraduate students.  They felt doubly alienated in the fact that they do
not have a dedicated lab for computing resources and rarely participate in
important research projects.  But it seems that as this experiment with the Web
information system wore on, students learned there was a whole lot of
interesting work being done in the department, none of which they were a part
of.  As one student puts it, ``no one really encourages undergrads to pursue
projects as much as they do the graduate students.  And I think that this
attitude is the main reason why most undergrad students feel as if they're `in
the way' in the ICS department and that they're not enthusiastic about becoming
a graduate student at UH.''

\ls{1}
\begin{table}[htb]
\caption{Students: Question 6 results}
{Do you feel a sense of belonging in the ICS department?  Why or why not?}
\begin{center}
\begin{tabular}{|c|c|c|} \hline
% \multicolumn{3}{l}{\rule[-5mm]{0mm}{10mm}\bf Do you feel a sense of belonging
% in the ICS department?}\\ \hline
  & {\bf Pre-Test} & {\bf Post-Test} \\ \hline 
 Yes & 24 & 37 \\ \hline 
 No & 16 & 38 \\ \hline 
\end{tabular}
\end{center}
\label{tab:question6s}
\end{table}
\ls{1.5}

In question 6 (Table \ref{tab:question6s}), I asked if people felt a sense of
belonging in the department.  In the pre-test, 56\% said ``yes'', 37\% said
``no'' and 2\% said ``so-so''.  Those who felt they belonged did so because
they were acquainted with some faculty members and many other students.
Sharing many of the same classes fostered this sense of belonging for them.
Spending a lot of time in the building that houses the department also seemed
to foster some belonging.  Some even said they belonged here in the ICS
department more than anywhere else on campus.  Those that did not feel they
belonged said so because they had little involvement in the department or that
the department did not hold any outside activities such as seminars or
workshops.

The post-test results were split down the middle where 44\% felt they belonged
and 45\% did not feel belonging in the department.  The sentiment of those who
said ``yes'' was summed up well by one student, ``All of us are like some kind
of extended family.  We all see each other before, during, and after class
almost every school day.  We spend so much time in the labs that sometimes I
actually expect the others to be there when I'm there.''  Others felt that the
people were generally friendly and helpful, making them feel a sense of
belonging.  Those who said ``no'' felt that way because they either hardly
spent time in school activities or just kept to themselves, going to classes
and leaving once they were done.  Many did not like the fact that they once had
an ICS club which is no longer active.

\subsubsection{Questions 4, 7, and 8}

When asked to give numeric estimates of how many faculty, graduate and
undergraduate students they think are in the department (Table
\ref{tab:question4s}), I got a wide range of answers with relatively equal
amounts in each range, indicating most did not know the answer and were
probably just guessing.  The post-test yielded the same results.  The estimate
of undergraduate students ran mainly under 100, which is just about how many
are listed under the Web site.

\ls{1}
\begin{table}[htb]
\caption{Students: Question 4 results}
  {Please provide three numeric estimates of how many faculty, graduate
  students and undergraduates you think are in the department:} 
\begin{center}
\begin{tabular}{|l|c|c|c|c|c|} \hline
%  \multicolumn{10}{l}{\rule[-5mm]{0mm}{10mm}\bf Please provide numeric
%  estimates for the following groups:}\\ \hline
  \multicolumn{6}{|c|}{\bf Pre-Test} \\ \hline
  \multicolumn{1}{|c} {}& \multicolumn{1}{c|}{Actual} &
  \multicolumn{4}{|c|}{Responses} \\   \hline 
  Faculty        &  18  & 0-10  &  11-15 & 16-20 & 21$+$  \\ \cline{3-6}
                 &      & 5     &  13    & 11    & 11      \\   \hline
  Graduates      &  38  & 0-20  & 21-30  & 31-40 & 41$+$ \\ \cline{3-6}
                 &      & 11    & 6      & 8     & 15     \\   \hline
  Undergraduates & 240 & 0-100 & 101-150 & 151-200 & 201$+$ \\ \cline{3-6}
                 &     &  20   & 11      & 8       & 1      \\ \hline \hline
  \multicolumn{6}{|c|}{\bf Post-Test} \\ \hline
  \multicolumn{1}{|c} {}& \multicolumn{1}{c|}{Actual} &
  \multicolumn{4}{|c|}{Responses} \\   \hline 
  Faculty        &  18  & 0-10 & 11-15 & 16-20 & 21$+$ \\ \cline{3-6}
                 &      & 19   &  24   & 22    & 10    \\ \hline
  Graduates      &  38  & 0-20 & 21-30 & 31-40 & 41$+$ \\ \cline{3-6}
                 &      & 21   &  17   & 16    &  22     \\ \hline
  Undergraduates & 240 & 0-100 & 101-150 & 151-200 & 201$+$ \\ \cline{3-6}
                 &     & 42    & 10      & 15      & 10      \\ \hline
\end{tabular}
\end{center}
\label{tab:question4s}
\end{table}
\ls{1.5}

%\ls{1}
%\begin{table}[htbp]
%\caption{Students: Question 7 results}
%{What ICS research projects are you aware of?  Please briefly list the
%  projects you can think of immediately and any faculty, staff, or students
%  you know who are involved in them. (If you know many people involved in a
%  particular project, then simply estimate the number of involved people
%  and provide that number.)}
%\begin{center}
%\begin{tabular}{|l|c|c|} \hline
% \multicolumn{3}{l}{\rule[-5mm]{0mm}{10mm}\bf What ICS research projects are
% you aware of?}\\ \hline
% \multicolumn{1}{|c|}{\bf Research Group/Project Leader} & 
% {\bf Pre-Test} & {\bf Post-Test} \\ \hline 
% {CSDL: Johnson}     & 6 & 22 \\ \hline 
% {J. Corbett}        & 1 & 5 \\ \hline 
% {PACCOM: Nielsen}   & 3 & 2 \\ \hline 
% {SERL: Miyamoto}    & 1 & 6 \\ \hline 
% {S. Itoga}          &   & 4 \\ \hline 
% {Crosby/Peterson}   &   & 1 \\ \hline 
% {J. Lee}            & 1 & 2 \\ \hline 
% {J. Stelovsky}      & 1 &  \\ \hline 
% {ICL:Lew/Halverson} &   & 2 \\ \hline 
% {D. Chin}           &   & 4 \\ \hline 
% {K. Sugihara}       & 1 & 1 \\ \hline 
% {D. Pager}          &   & 1 \\ \hline 
% {R. Halverson}      &   & 1 \\ \hline 
% {R. Andrada}        & 3 & 19 \\ \hline 
% {D. Tjahjono}       & 3 & 17 \\ \hline 
% {C. Moore}          & 3 & 8 \\ \hline 
% {R. Brewer}         &   & 1 \\ \hline 
% {L. Weldon}         & 1 &  \\ \hline 
% {I. Miyamoto w/CIA} & 1 &  \\ \hline 
% {P. Lundstrom}      & 1 &  \\ \hline \hline
% {No attempt made}   & 32 & 11 \\ \hline 
% {Feeble attempt}    & 6 & 12 \\ \hline 
%\end{tabular}
%\end{center}
%\label{tab:question7s}
%\end{table}
%\ls{1.5}

Responses to question 7 were very dismal.  A large percentage (74\%) did not
even attempt to answer this question.  About 14\% made feeble attempts at
answers meaning they listed only 1 or 2 projects.  The few that were listed
were mostly correct.  There were 2 projects listed which did not exist.  The
post-test responses to question 7 were very encouraging.  This time, only 13\%
did not attempt to answer this question and 14\% made feeble attempts.  The
rest gave a good try and were successful in naming the different research
groups.  They were also able to list individual projects by many of the
professors.

%\ls{1}
%\begin{table}[htbp]
%\caption{Students: Question 8 results}
%{Assume you had a question about one of the following topics.  For
%  each of the topics, name one or two people in the department who you
%  would want to ``just stop by'' to talk with about it, or leave it blank
%  if you can't think of anybody.}
%\begin{center}
%\begin{tabular}{|l|l|c|c|} \hline
% \multicolumn{6}{l}{\rule[-5mm]{0mm}{10mm}\bf Name the resident expert for each
% subject matter:} \\ \hline
% \multicolumn{1}{|c|}{\bf Field} & \multicolumn{1}{|c|}{\bf Expert}  & 
% \multicolumn{1}{|c|}{\bf Pre-Test}  & 
% \multicolumn{1}{|c|}{\bf Post-Test} \\ \hline 
% {Departmental rules}        & D. Watanabe  & 5  & 9  \\ \cline{2-4}
%                             & K. Sugihara  & 5  & 15 \\ \cline{2-4}
%                             & W. Peterson  & 11 & 22 \\ \cline{2-4}
%                             & Secretary    & 3  & 9  \\ \cline{2-4}
%                             & Others       & 5  & 9 \\ \hline 
% {Artificial intelligence}   & D. Chin      & 12 & 26 \\ \cline{2-4}
%                             & Others       & 8  & 22 \\ \hline 
% {Software Engineering}      & P. Johnson   & 16 & 44 \\ \cline{2-4}
%                             & Others       & 12 & 23 \\ \hline 
% {Computer networks}         & W. Peterson  & 8  & 12 \\ \cline{2-4}
%                             & W. Gersch    & 3  & 15 \\ \cline{2-4}
%                             & T. Nielsen   & 2  & 1  \\ \cline{2-4}
%                             & Others       & 8  & 14 \\ \hline 
% {Hypertext and multimedia}  & J. Stelovsky & 16 & 26 \\ \cline{2-4}
%                             & Others       & 8  & 28 \\ \hline 
% {Cognitive Science}         & M. Crosby    & 8  & 23 \\ \cline{2-4}
%                             & Others       & 5  & 17 \\ \hline 
% {Human Computer Interaction}& M. Crosby    & 11 & 19 \\ \cline{2-4}
%                             & Others       & 5  & 17 \\ \hline 
% {Computer games}            & D. Chin      & 1  & 9  \\ \cline{2-4}
%                             & Others       & 8  & 21 \\ \hline 
% {Employment opportunities}  & K. Sugihara  & 3  & 8  \\ \cline{2-4}
%                             & Others       & 11 & 24 \\ \hline \hline
% {No attempt made to answer} &              & 12 & 8  \\ \hline 
% {Feeble attempt at answer}  &              & 13 & 7  \\ \hline 
%\end{tabular} 
%\end{center}
%\label{tab:question8s}
%\end{table}
%\ls{1.5}

The students did not fare well in question 8 where 28\% made no attempt to
answer the question and 30\% made feeble attempts at answers.  For those that
did respond, they could only identify a resident expert in half the categories.
The other half had a range of responses where nearly every faculty member was
listed at least once.  It is clear that most students did not know who was the
expert in what field.  The post-test yielded some interesting insights.  About
10\% made no attempt to answer the question and 8\% made feeble attempts at
answers.  They did a little better this time since a larger percentage
accurately named a resident expert in nearly all of the categories.  For the
expert in departmental rules, students generally listed the undergraduate and
graduate advisors.  Obviously, this is who they turn to for advice whereas the
faculty named the chairman, their boss, as the expert in this category.  The
category of computer network expert also yielded interesting answers.  Students
listed professors who taught that course in the last few semesters.  Only one
student listed the person which the faculty felt was the network expert.  Since
that particular faculty member rarely teaches, most students do not know him
nor what he does.  Unlike the professors, students knew of the computer games
expert.  As with the department rules, students listed their advisors for
knowledge in employment opportunities.

\subsubsection{Questions 9 and 10}

\ls{1}
\begin{table}[htbp]
\caption{Students: Question 9 results}
{For each of the following types of communication, indicate whether
  you use it Daily, Weekly, Monthly, or Never, to communicate with other
  people in the department.}
\begin{center}
\begin{tabular}{|l|c|c|c|c|} \hline
% \multicolumn{9}{l}{\rule[-5mm]{0mm}{10mm}\bf How often do you use the
% following types of communication?} \\ \hline
 & \multicolumn{4}{|c|}{\bf Pre-Test} \\ \hline 
 & {\bf Daily} & {\bf Weekly} & {\bf Monthly} & {\bf Never} \\ \hline
 {Email}            & 29 &  8 & 4 &  0 \\ \hline 
 {Telephone}        & 18 &  8 & 7 &  6 \\ \hline 
 {Fax}              &  1 &  4 & 5 & 26 \\ \hline 
 {Informal Meetings}&  6 & 10 & 7 & 14 \\ \hline 
 {Formal Meetings}  &  0 &  7 & 7 & 22 \\ \hline 
 {Other: Notes}     &  1 &    &   &    \\ \hline \hline
 & \multicolumn{4}{|c|}{\bf Post-Test} \\ \hline 
 & {\bf Daily} & {\bf Weekly} & {\bf Monthly} & {\bf Never} \\ \hline
 {Email}            & 61 & 13 &  4 &  1 \\ \hline 
 {Telephone}        & 39 & 10 & 11 & 17 \\ \hline 
 {Fax}              &  1 &  8 &  6 & 56 \\ \hline 
 {Informal Meetings}& 11 & 22 & 15 & 21 \\ \hline 
 {Formal Meetings}  &  1 & 13 & 14 & 40 \\ \hline 
 {Other: Notes}     &  1 &    &    &    \\ \hline 
 {Other: WWW}       &  2 &    &    &    \\ \hline 
 {Other: Teleconference} &    &    & 1   &    \\ \hline 
 {Other: IRC}       &  1 &    &    &    \\ \hline 
\end{tabular}
\end{center}
\label{tab:question9s}
\end{table}
\ls{1.5}

In question 9 (Table \ref{tab:question9s}), 67\% of students use email on a
daily basis and 42\% also use the telephone daily.  Almost no one uses a fax
machine or holds formal meetings to communicate.  Informal meetings are held
with varying frequencies.  The post-test shows the same results where 73\% of
students use email on a daily basis and 46\% also use the telephone daily.
Informal meetings are held with varying frequencies, but mostly on a weekly
basis.

\ls{1}
\begin{table}[htbp]
\caption{Students: Pre-Test question 10 results}
{Do you use Mosaic, or any other Web browser to access the World Wide
  Web?}
\begin{center}
\begin{tabular}{|c|c|} \hline
% \multicolumn{2}{l}{\rule[-5mm]{0mm}{10mm}\bf Do you use any Web browser to
% access the World Wide Web?}\\ \hline
  & {\bf Pre-Test} \\ \hline 
 Yes & 31 \\ \hline 
 No  & 10 \\ \hline 
\end{tabular}
\end{center}
\label{tab:Prequestion10s}
\end{table}
\ls{1.5}

Question 10 revealed that 72\% of the respondents already used the World Wide Web
while 23\% did not.

\subsubsection{Post-Test Questions 10-16}

\ls{1}
\begin{table}[htbp]
\caption{Students: Post-Test question 10 results}
{Did you know the ICS Department has a Web site? If so, how did you
  discover this?}
\begin{center}
\begin{tabular}{|l|c|} \hline
% \multicolumn{2}{l}{\rule[-5mm]{0mm}{10mm}\bf How did you discover the ICS Web
% site?} \\ \hline
  & {\bf Post-Test} \\ \hline 
 {Flyers} & 7 \\ \hline 
 {Email announcements} & 20 \\ \hline 
 {Colleague or friend} & 33 \\ \hline 
 {My Professor} & 24 \\ \hline 
 {I stumbled on it by accident} & 11 \\ \hline 
 {Other: I looked for it} & 2 \\ \hline 
 {Other: Training session} & 3 \\ \hline 
 {Other: Didn't know about it} & 4 \\ \hline 
\end{tabular}
\end{center}
\label{tab:Postquestion10s}
\end{table}
\ls{1.5}

The following pertains to questions 10-15 (Tables \ref{tab:Postquestion10s} -
\ref{tab:question15s}).  Only 4 people (5\%) of the respondents said they did
not know the department had a Web site.  Most found out either through email, a
friend or their professor.  About 57\% found the training sessions useful and
20\% thought they were frequently and flexibly scheduled while 14\% did not
bother to attend them.  Most of the students use the ICS Department's web site
between a few times a month to a few times a week.  69\% felt they were
competent Web surfers but only 19\% were confident in Web publishing.  However,
27\% felt that they were not competent in using the World Wide Web.  The main
reason (67\%) students are using the Web is that it is fun and exciting.  Other
reasons were that it makes navigation of the Internet simple (49\%) and that it
was required by their professor(42\%).  Only 5\% still did not use the Web.

\ls{1}
\begin{table}[htbp]
\caption{Students: Question 11 results}
{The WWW training sessions offered by the department were...}
\begin{center}
\begin{tabular}{|l|c|} \hline
% \multicolumn{2}{l}{\rule[-5mm]{0mm}{10mm}\bf The WWW training sessions offered
% by the department were...} \\ \hline
  & {\bf Post-Test} \\ \hline 
 {Useful} & 48 \\ \hline 
 {Not very useful} & 2 \\ \hline 
 {Frequently and flexibly scheduled} & 17 \\ \hline 
 {Not easily accessible} & 3 \\ \hline 
 {What training session?} & 3 \\ \hline 
 {Other: Did not attend} & 12 \\ \hline 
 {Other: Should cover more material} & 1 \\ \hline 
 {Other: Frequent but offered at a bad time} & 1 \\ \hline 
\end{tabular}
\end{center}
\label{tab:question11s}
\end{table}
\ls{1.5}

\ls{1}
\begin{table}[htbp]
\caption{Students: Question 12 results}
{How often have you been accessing information from the ICS
  Department's Web site?}
\begin{center}
\begin{tabular}{|l|c|} \hline
% \multicolumn{2}{l}{\rule[-5mm]{0mm}{10mm}\bf How often have you accessed the
% ICS Web site?} \\ \hline
  & {\bf Post-Test} \\ \hline 
 {Rarely or never} & 18 \\ \hline 
 {A few times a month} & 33 \\ \hline 
 {A few times a week} & 22 \\ \hline 
 {Almost everyday} & 9 \\ \hline 
\end{tabular}
\end{center}
\label{tab:question12s}
\end{table}
\ls{1.5}

\ls{1}
\begin{table}[htbp]
\caption{Students: Question 13 results}
{I feel I am a competent...}
\begin{center}
\begin{tabular}{|l|c|} \hline
% \multicolumn{2}{l}{\rule[-5mm]{0mm}{10mm}\bf I feel I am a competent...} \\ \hline
  & {\bf Post-Test} \\ \hline 
 {Web surfer} & 58 \\ \hline 
 {Web publisher} & 16 \\ \hline 
 {neither} & 23 \\ \hline 
\end{tabular}
\end{center}
\label{tab:question13s}
\end{table}
\ls{1.5}

\ls{1}
\begin{table}[htbp]
\caption{Students: Question 14 results}
{Did you find the information presented about the department useful?
  If so, in what ways?  If not, why not and how would you improve it?}
\begin{center}
\begin{tabular}{|c|c|} \hline
% \multicolumn{2}{l}{\rule[-5mm]{0mm}{10mm}\bf Did you find the ICS Web site
% useful?} \\ \hline
  & {\bf Post-Test} \\ \hline 
 Yes & 14 \\ \hline 
 No  & 2 \\ \hline 
\end{tabular}
\end{center}
\label{tab:question14s}
\end{table}
\ls{1.5}

\ls{1}
\begin{table}[htbp]
\caption{Students: Question 15 results}
{I use the World Wide Web because...}
\begin{center}
\begin{tabular}{|l|c|} \hline
% \multicolumn{2}{l}{\rule[-5mm]{0mm}{10mm}\bf I use the World Wide Web
% because...}  \\ \hline
  & {\bf Post-Test} \\ \hline 
 {It was required by my professor} & 35 \\ \hline 
 {The Web is fun and exciting} & 56 \\ \hline 
 {Everyone else seems to be using it} & 18 \\ \hline 
 {It makes it easy for me to navigate the Internet} & 41 \\ \hline 
 {I don't use it} & 4 \\ \hline 
 {Other: I must teach it} & 1 \\ \hline 
 {Other: Try something new} & 1 \\ \hline 
 {Other: For my research} & 1 \\ \hline 
\end{tabular}
\end{center}
\label{tab:question15s}
\end{table}
\ls{1.5}

When asked whether the Web information presented on the department was useful
(Table \ref{tab:question14s}), 17\% thought that it was and 2\% did not.  The
rest made no comment.  One thought it was entertaining and one thought it was
``so-so.''  Most students found it useful in getting course information,
assignments and lecture notes.  The rest liked that it helped them learn more
about other people in the department.  One commented, ``It was interesting to
see the interests of others through their home pages.  I think it would be
better if ALL people in the department were on the page.''  Another said ``I
liked the fact that everyone has a home page and most with their picture on it.
That way I can put a name with a face and their work and interests.''

About 33\% of the students said they had completed the pre-test questionnaire
and 61\% said they did not (see Table \ref{tab:question16s}).

\ls{1}
\begin{table}[htbp]
\caption{Students: Question 16 results}
{Did you complete the Pre-Test questionnaire?}
\begin{center}
\begin{tabular}{|c|c|} \hline
% \multicolumn{2}{l}{\rule[-5mm]{0mm}{10mm}\bf Did you complete the Pre-Test
% questionnaire?}  \\ \hline
  & {\bf Post-Test} \\ \hline 
 Yes & 28 \\ \hline 
 No  & 51 \\ \hline 
\end{tabular}
\end{center}
\label{tab:question16s}
\end{table}
\ls{1.5}

\subsubsection{Question 17}

The last question asked people to comment on the sense of community in the
department.  Here are some of those comments:

\ls{1}
\begin{quote}
  ``It's there, but it needs to be brought out - some type of club or other
  activities.''
\end{quote}
\ls{1.5}

\ls{1}
\begin{quote}
  ``Since I do almost all my work on my home computer, I don't know how accurate
  this statement is, but from my personal point of view, I can walk through the
  labs, classes, etc and maybe recognize 7 people total, and know the names of
  3-4 of them.  Also, although I do talk to friends in the department, I could
  walk past most of them without even knowing it, since I have never seen them
  before.''
\end{quote}
\ls{1.5}

\ls{1}
\begin{quote}
  ``I feel that equipment in the department are pretty out-of-date.  Slow
  computers, old junky software, lack of computers and software,...  These are
  the facts of our ICS Dept.  The workstation is always crowded with people.  The
  UHUNIX system is disappointingly difficult to access through modems until
  1am.''
\end{quote}
\ls{1.5}

\ls{1}
\begin{quote}
``The ICS dept is obviously lacking in communication with other CS students.
If Rose did not come here today, we definitely don't know if ICS dept has any
activities.  This is very important since if there is no announcement, then
students will have no interest at all for this major...Finally, where is the ICS
club?''
\end{quote}
\ls{1.5}

\ls{1}
\begin{quote}
  ``There is no list of other people in the ICS major.  No information about
  anything concerning ICS.  To me there is no sense of community in the ICS
  dept.''
\end{quote}
\ls{1.5}

\ls{1}
\begin{quote}
  ``I feel a sense of belonging in the ICS dept.  But also sense very high
  competition among students.  Being a grad student I sometimes see a lack of
  willingness of people to work together.  I was always encouraged to work in
  groups.  I also noticed ethnic divisions in the ICS grad department and see
  some resentment with some students.  I have also heard from some that certain
  students get TAs in the lower division courses.  Some upper division TAs go to
  mainly foreign students.  If this is true, it's not very fair.''
\end{quote}
\ls{1.5}

There were many different post-test comments on the sense of community in the
department.  Some are listed below:

\ls{1}
\begin{quote}
``I think that the ICS Web Site has been the closest to a community that we've
had among the undergrads.  We really don't have the inter-department
association found in other majors ( ex. business ) through social activities
(ex. dances, picnics ).  Of course, this could be because we're all geeks and
don't go to those things, anyway.  This is why the Web is the best means of
bringing the department together.  We all can learn about about each other in a
safe environment.  The Web allows us to face the world as we see ourselves.
For example, [name withheld] wishes the community to view him as a crazy, nutty
guy, and has consistently portrayed himself as such in his Web page.  I do
believe that our Web pages are a reflection of our own self-images, and as
such, they allow others to get to know us without even meeting us.  This may
seem cold and impersonal, but it seems to be the best way to meet neighbors who
we may never cross paths with.''
\end{quote}
\ls{1.5}

\ls{1}
\begin{quote}
``There's far too little Americans in the classes.  The other nationalities
often take over the classes and the labs.  I feel that the ICS department
should be doing something to recruit more American students into the program,
instead of scaring them off in the beginning levels.''
\end{quote}
\ls{1.5}

\ls{1}
\begin{quote}
``It would be nice if there is a common room for everyone in our department to
sit down, talk, and relax.''
\end{quote}
\ls{1.5}

\ls{1}
\begin{quote}
``Undergrads don't have a place in the `community', and because of that it may
cause students to lose interest in cs, and/or not perform as well in class.
Also there is no opportunity to apply theories that we learn in class, other
than assignments because we don't have the chance to go to the ``graduate'' lab
or even know about the research projects going on in the department.  I think
that if a professor is working on some research, or on a project, he/she should
enhance his/her class with exposure to the kinds of research/projects, perhaps
give incentive to students who are willing to participate and integrate
classroom experience with some practical applications.  Basically the
professors do not have much enthusiasm and motivation in teaching.  If they do
have any enthusiasm or motivation much of it in my guess is in their research,
holed up in their offices/labs, and only the biggies knowing about it....of
course they have to worry about tenure and being published and writing papers
for consortiums/conferences all over the world, doing some kind of
world-reknown research with huge grants--matters more important to them than
being concerned about the quality of teaching peons like us.''
\end{quote}
\ls{1.5}

\ls{1}
\begin{quote}
``I think it has improved at least among some of us graduate students.  I'm not
sure how much the WWW had to do with this.  The biggest gain was going out
drinking on Thursdays.''
\end{quote}
\ls{1.5}

\ls{1}
\begin{quote}
``I know that much of the motivation should come from the individual (student),
but morale goes down and enthusiasm goes down when the quality of teaching just
isn't there--who wants to go to class, who wants to participate in class, in
that case.  There are lots of good professors here, if only they'd bring it in
to their teaching.''
\end{quote}
\ls{1.5}

\ls{1}
\begin{quote}
  ``The instructors and TA and grad students seem to be a more tightly-knit
  community.''
\end{quote}
\ls{1.5}

\ls{1}
\begin{quote}
``There is none.''
\end{quote}
\ls{1.5}

\ls{1}
\begin{quote}
  ``There might be a sense of community among graduates, but if there is some
  in the undergrad community, I have yet to see it.''
\end{quote}
\ls{1.5}

\subsubsection{Summary}

%summing up - pre
In summary, the major points identified in the student responses during the
pre-test are:

\ls{1}
\begin{itemize}
\item{Most worked with other undergraduate students.}
\item{Most knew less than 10 faculty members personally and could identify
  less than 10 of them by face.}
\item{They had no idea the size of faculty, graduate and undergraduate students.}
\item{They did not feel a sense of significance in the department.}
\item{More than half felt a sense of belonging in the department.}
\item{They did not accurately identify research projects and subject experts.}
\end{itemize}
\ls{1.5}

Unlike the faculty and staff, students were not quite as self-aware of the
people and projects in their department.  They did not know many faculty
members well nor could identify them by face.  They could not cite the size of
each subgroup, the major research projects nor the resident experts.  They
generally did not feel they played a significant role in the department.  The
one bright spot is that more than half did feel a sense of belonging in the
department.  It is not all that is needed in a community, but it is a start.

%summing up - post
Here is a brief summary of the post-test questionnaire results:

\ls{1}
\begin{itemize}
\item{Most worked with other undergraduate students.}
\item{Most knew less than 10 faculty members personally and could identify
  less than 10 of them by face.}
\item{They did not know the size of faculty, graduate and undergraduate
  students.}
\item{They did not feel a sense of significance in the department.}
\item{About half felt a sense of belonging in the department.}
\item{They accurately identified research projects and subject experts.}
\item{Most knew of the ICS Web site and some found it useful.}
\item{They accessed the ICS Web site with some frequently.}
\end{itemize}
\ls{1.5}

By the end of the semester, students appeared to increase their sense of
self-awareness in the department.  They may not know many people on a personal
level but were able to identify them by face, and can name their interests,
research projects and their expertise.  Their feelings of importance in the
department were still rather low.  It seemed that the students who had paid
positions felt any sense of significance.  These were typically the graduate
students employed as teaching assistants or lab monitors.  The students' sense
of belonging seemed to be divided.  Those that spent any amount of time on
campus outside of classes were more apt to feel belonging than those who went
to class and participated in little else concerning the department.  Many of
the respondents were competent users of the Web, finding it fun, exciting,
useful and using it with some frequency.  Students seemed to be less in the
dark about departmental projects, activities and events.

\section{Statistics of Database Accesses}
The Web server keeps a log of all incoming requests.  Information about which
documents are being requested, when, how often, etc. is all noted in a logfile.
Unfortunately, it does not collect data on which specific user requested a
document.  Instead, it only records the machine name from which the document
request was made.

To assess how the Web server affected the level of community in the ICS
department, I investigated how users accessed the system.

\subsection{User Contribution}
Any user may directly contribute to the ICS Web information system.  By
``directly,'' I mean that there is no middle man involved in this contribution.
Unlike most publications, the data published on the Web site is not cleared by
an editor before everything goes public.  Users are given free reign regarding
this issue.  I was not trying to be overly democratic in wanting to increase
the sense of community here.  This simply happens to be an outcome of the way
the server was designed.

Basically, a Web site can serve documents to requesting clients from either of
two places within the file system.  The first is a distinguished directory
which serves as ``Home Base'' for any Web site.  In a networked file system
which supports multi-user access, such as the one we have in the ICS
department, there is usually only one person in charge of these files.  It may
be the system administrator and/or the webmaster.  This person generally acts
as the editor for all information published at their site.  The other area of
the file system where a Web server may serve documents is in a user's
directory.

All users need to do to publish information is to create a subdirectory with a
specified name and store their related files under that subdirectory hierarchy.
The files must adhere to some formatting conventions and must also be
world-readable.  Just through the creation of a directory and some files, a
user may publish on the World Wide Web.

Given the simplicity in publishing, I worked out three ways for users to
contribute to the Web site.  The first and most popular way is through the
creation of a home page.  It is simply a text file written in HTML.  This
process was facilitated by a unix shell script written by me.  The script was a
program such that when executed, it would create a personalized home page for
that user.  Instructions on this procedure were also given online.  Users who
liked to learn things on their own were also free to look up numerous
references on the creation of an HTML document.  Having an accessible home page
is only useful if people know its URL, or address.  To simplify matters for
people browsing the system, I automatically created a listing of all users who
have home pages in their directories.

The second way to contribute to the Web site was by publishing your picture on
the network.  All people had to do was to get a digitized picture of themselves
and to store it under a specific name in the above mentioned subdirectory.  As
with the home pages, another listing of people with pictures was also created
for ease in browsing.  

The last way for users to directly contribute to this Web site was through our
Personal Interests Page.  Users need only modify a line in their home page to
list their hobbies.  Again, I created yet another page, which listed different
hobbies and the home pages of the people associated with those hobbies.

One measure of user participation is the creation of a personal home page.
There are 18 faculty members, 3 staff, approximately 38 Master's students and
240 undergraduates.  Due to a shortage of resources, the department can only
supply unix accounts to the faculty, staff and graduate students.  PhD
candidates who are part of an interdisciplinary program also have accounts.
There are less than 10 of these students.  Undergraduates are accommodated by
the University's computing center.  The ICS Web server is housed on a unix
platform maintained by this department.  Only users with accounts on this
network may publish home pages.  This posed the problem of accommodating the
undergraduates, who make up the vast majority of the department.  To alleviate
this problem, I coordinated with the University's computing center to allow
undergraduates full authorship on the Web.  They are publishing on a different
network, but lists and links to undergraduate pages are still maintained by the
ICS department.

\begin{figure}[htb]
\centerline{\psfig{figure=HomePages.eps}}
\caption{Percentage of home pages created by members of the ICS Department.}
\label{fig:HomePages}
\end{figure}

Analysis of the logfiles for the ICS network reveals user participation of
faculty, staff and graduate students.  When the server was initially installed
in August, I ran a pilot test of the system on my research group of 9 people.
Before the public release in January five months later, some of the faculty and
graduate students somehow caught wind of the Web server and took the initiative
in publishing their own pages without any training.  In fact, a few days before
the first training session, 37\% of the graduate students already had a home
page.  One week after the server was publicly released, 29\% of the faculty \&
staff and 55\% of the graduate students created home pages.  This coincided
with the introductory training sessions that month to an audience of
approximately 140 people.  See Table \ref{tab:chronology} which marks important
dates and events regarding the ICS Department's Web server public release and
training schedule.  In addition to the training, the questionnaire was
administered shortly after the public release.  I also gave a talk to a
graduate level class on my research.

Unfortunately, since the undergraduates are housed on a different network, I do
not maintain the logfiles for their system.  The administrators of that system
were only able to provide logfiles beginning January 22, 1995.  They were also
missing data for the week of Feb 5-12.  During these ``missing'' dates, in
Figure \ref{fig:HomePages}, there is a lull in the creation of home pages for
undergraduates.  However, the general trend is that there is a steady increase
in home pages until mid-February when training had slowed down.  Advanced
classes were offered, but this did not increase the number of new users.  As
for the graduate students, their increase in activity picked up at the start of
the semester and leveled off starting in February.  There is also a lull in
home page creation during the week of January 22 when there were some problems
with the logfiles.  Data for that week was not recorded.

By the end of the semester, there was 97\% participation by the graduate
students, 38\% participation by the undergraduates and about 62\% from the
faculty and staff.  It seems that the most active users of the system were the
graduate students.  Even though the faculty \& staff was half as large as the
graduate students, only a little more than half of them actively participated
through the creation of a home page.  Although the total number of
undergraduates who participated amounted to more than the combined total of
graduates and faculty \& staff, the overall participation of that group was
low.

\begin{table}[htb]
\caption{Chronology of Events}
\begin{center}
\begin{tabular}{|l|l|c|} \hline
{\bf Date}    & {\bf Event}   & {\bf \# Attendees} \\ \hline
04 Aug 94     & ICS Web Server installed  & NA\\
10 Jan 95     & Training: Intro to WWW  & 40\\
18 Jan 95     & Administered Pre-Test Questionnaires & NA\\
18 Jan 95     & Training: Intro to WWW  & 50\\
20 Jan 95     & Training: Intro to WWW  & 8\\
25 Jan 95     & Training: Intro to WWW  & 20\\
27 Jan 95     & Seminar: Redefining the Web  & 20\\
09 Feb 95     & Training: Intro to WWW  & 15\\
10 Feb 95     & Training: How to design good HTML & 25\\
10 Mar 95     & Training: WWW Advanced Topics & 10\\
17 Mar 95     & Training: WWW Advanced Topics & 20\\
24 Mar 95     & Training: Intro to WWW  & 20\\
24 Mar 95     & Training: WWW Advanced Topics  & 10\\
07 Apr 95     & Training: WWW Advanced Topics  & 10\\
27 Apr 95     & Training: Intro to WWW  & 5\\
28 Apr 95     & Training: Intro to WWW  & 5\\
28 Apr 95     & Training: WWW Advanced Topics  & 10\\ 
01 May 95     & Administered Post-Test Questionnaires & NA\\ \hline 
\end{tabular}
\end{center}
\label{tab:chronology}
\end{table}

\subsection{Access Patterns}
In order to better design a community-oriented information system, it would
help to know how users actually use the database.  I studied access patterns of
the database by analyzing the logfiles.  Although I do not have an explicit
measure for knowing which user accessed what information, this can be gathered
implicitly.  Figure \ref{fig:logitem} shows a sample log entry item.  The four
items noted in a log entry include: the machine name of the requesting client,
the date and time of request, the document request and the return codes.  The
return code 200 indicates a successful request.  Since the logfiles do record
the machine names of requesting clients, I can selectively study accesses from
machines located within the department.  To learn how department members access
information from the Web site, I chose to study logs originating from 10
workstations in the lab, all of which get an average of 12 hours of daily use
primarily by graduate students.  I can heuristically identify different
``sessions'' as request sequences with less than 5 minutes between successive
requests.  It is possible that one user may be physically logged on to a
machine while another is remotely logged on to the same one.  If both users ran
a Web browser and accessed information from the Web site simultaneously, then
the session logs would show their accesses interleaved.  Careful analysis of
the log files can almost always reveal this type of situation.  Access patterns
such as these were ignored.

\small
\ls{1}
\begin{figure}[htb]
\begin{center}
\begin{verbatim}
zip.ics.Hawaii.Edu [12/Jan/1995:13:58:11 -1000] "GET / HTTP/1.0" 200 2016
\end{verbatim}
\end{center}
\caption{A sample logfile entry.}
\label{fig:logitem}
\end{figure}
\ls{1.5}
\normalsize

From this data, I found four main access patterns.  The first pattern concerns
which document from the database a user accessed first in any given session.  I
found two main entry points from which users accessed the system.  Many chose
to start at the department's home page or their own personal home page.  The
unix shell script used in the training sessions had the additional feature of
configuring a user's system such that it would retrieve the ICS home page when
starting a particular Web browser.  This could account for people beginning
their navigation through the database at that entry point.  URLs can be very
long and appear cryptic.  For this reason, users typically do not explicitly
type them in to retrieve a document.  They can set a ``bookmark'' in their
browser such that clicking on this bookmark retrieves the associated document.
People often set bookmarks for pages they find interesting or important enough
that they plan to retrieve it again in the future.  It appears that users are
configuring their browsers to start at these entry points or setting bookmarks
for these pages.

The second pattern discovered concerns the general departmental information
available on the system.  Requests for these documents occurred throughout the
semester.  However, there were substantially less requests for these documents
a month after the system was introduced.  It appears that users will initially
navigate through the system to get a feel for the types of information
available.  This type of information is usually static.  So users would not
need to repeatedly request it.  Changes to these documents would be reflected
in the ``What's new'' page mentioned in Chapter \ref{chap:experiment}.

The third pattern concerns how users navigate through the database.  They often
navigated by going through the hierarchy in which the information was
presented.  Users typically started at the ICS home page and traversed a path
to a specific point.  For example, a user may wish to see a fellow student's
home page.  Its URL is very simple to remember since it differs from one's URL
by the login name.  However, the logs often showed users to first consult the
ICS home page, followed by the home page listing, and finally the user's page.
An exception to this pattern was course-related information.  Many users did go
through the typical path: ICS home page, home page listing, professor's home
page, class home page.  However, there were many sessions showing users
directly accessing their class home page.  Since it is very unlikely they would
type the URL for every access, I assumed they set a bookmark to this page.

\ls{1}
\begin{figure}[htbp]
\begin{center}
\begin{verbatim}
[13/Feb/1995:09:48:53 -1000] "GET / HTTP/1.0" 200 2162
[13/Feb/1995:09:54:22 -1000] "GET /homePages.html HTTP/1.0" 200 8596
[13/Feb/1995:09:54:29 -1000] "GET /~professor1/ HTTP/1.0" 200 3283
[13/Feb/1995:09:54:42 -1000] "GET /~professor2/ HTTP/1.0" 200 2360
[13/Feb/1995:09:55:04 -1000] "GET /~student1/ HTTP/1.0" 200 1580
[13/Feb/1995:09:55:12 -1000] "GET /~professor3/ HTTP/1.0" 200 4623
[13/Feb/1995:09:56:11 -1000] "GET /~professor4/ HTTP/1.0" 200 1412
[13/Feb/1995:09:56:23 -1000] "GET /~professor5/ HTTP/1.0" 200 2969
[13/Feb/1995:09:56:57 -1000] "GET /~professor6/ HTTP/1.0" 200 1132
[13/Feb/1995:09:57:05 -1000] "GET /~professor7/ HTTP/1.0" 200 1235
[13/Feb/1995:09:57:13 -1000] "GET /~professor7/ HTTP/1.0" 200 1235
[13/Feb/1995:09:57:36 -1000] "GET /~student2/ HTTP/1.0" 200 2091
[13/Feb/1995:10:00:02 -1000] "GET /homePages.html HTTP/1.0" 200 8596
[13/Feb/1995:10:00:04 -1000] "GET /~student1/ HTTP/1.0" 200 1580
\end{verbatim}
\end{center}
The machine name is omitted from the logfile.  Login names are replaced by
generic subject labels.
\caption{Sample session log of home page perusal.}
\label{fig:loghp}
\end{figure}
\ls{1.5}

Finally, the last pattern concerned how much time users spent on particular
documents.  Figure \ref{fig:loghp} shows a session where a user perused several
other home pages.  The ``/'' document refers to the ICS home page.  A document
whose pattern begins with ``/$\sim$'' and is followed by a login name refers to
Web files for that user.  Requests for graphics were omitted from this log
since they are automatically downloaded on a document request and do not
contribute to navigational data.  Notice that it took this user 12 minutes to
peruse 9 home pages by various faculty and students.  In fact, this user often
spent less than a minute on some pages.

This is in stark contrast to how much time a user will spend on his or her own
Web pages.  Why would someone be accessing their own home page?  They will
typically be redesigning it or adding new information to it.  Some will
incorporate advanced HTML features by making their pages interactive.  This
often involves writing programs and testing them.  These sessions are
characterized as a stream of many mini-sessions.  Users often start at their
home page and stop.  Some minutes later, they will access their page again.  It
may or may not return successfully.  The return codes would indicate this one
way or the other.  The session will end and restart again later.  These
debugging and redesigning sessions can last for a few minutes or they sometimes
drag on for hours.  Where accessing the information system is concerned, users
spent much more time designing and tailoring their Web pages than they did in
wandering through the database.  This does not imply they did not read other
information.  Once a user successfully retrieves a document, there is no way of
knowing what they did with it.  They could save it, mail it to someone, or
print it for later use.  All that can be said is that they spent much time
repeatedly accessing and refining their own pages.

In summary, users tend to access the information system with the following
patterns:

\ls{1}
\begin{itemize}
\item{They begin navigation through the database at either the ICS home page
  or their own home page.}
\item{They traverse the general information area of the database upon first
  introduction to the system.  These documents get substantially less
  requests thereafer.}
\item{They set bookmarks to areas they found important, such as a class home
  page.  However, they otherwise navigated through the database starting at
  the ICS home page and traversed the system using the hierarchy presented to
  them.}
\item{They labored over the creation of their own Web documents, but spent
  anywhere from seconds to unknown amounts of time reviewing other Web
  pages.}
\end{itemize}
\ls{1.5}

%!!!relate to community
How does this information help us in designing systems that enhance an
organization's sense of community?  Since the first pattern tells us that users
typically start at their home page or the department's home page, it is likely
they will start at the main home page when they need to gather departmental
information.  So this page should be logically organized such that any
departmental information can be accessed quickly and easily.  The third pattern
reinforces this suggestion.  Different users will consider different documents
as important to them.  Instead of attempting to list all such important
documents, information should be logically organized and easily accessible.

Since users are less likely to repeatedly request general information as
suggested by the second access pattern, there should be some mechanism for
informing users when this data changes.  The ``What's new'' page is an example
of how this can be accomplished.  The last pattern suggests that users who
modify their home pages probably want their pages viewed again.  The ``What's
new'' page also included recently modified home pages.  Since users spent much
time on modifying their home page, online support for home page creation and
design might be helpful.  The information system included online help and was
listed under ``Web Services.''  These patterns offer suggestions that address
the collective self-awareness of an organization.  Fortunately, the design of
the system include features, explained in Chapter \ref{chap:www}, which
directly address these issues.
