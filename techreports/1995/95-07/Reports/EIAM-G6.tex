%%% \documentstyle[11pt,/group/csdl/tex/definemargins,
%%%                        /group/csdl/tex/lmacros]{article} 
%%% 
%%%           \begin{document}
%%%           \begin{center}
%%%           {\large\bf CSRS Experiment Results}\\
%%%           \end{center}
%%%           \small 
\chapter {CSRS Experiment Results: Group6(EIAM)}	  
\small
	  

\begin{description}
\item [Method:] EIAM
\item [Group:] Group6
\item [Source:] driver
\item [Participants:] awong (Reviewer), ccheung (Reviewer), gnakamur (Reviewer)
\end{description}
\section{Issue Lists}
\begin{enumerate}
\item {\it Issue\#192 (gnakamur)}
\begin{description}
\item [Subject:] Illegal assignment of name member variable.
\item [Criticality:] Hi
\item [Confidence-level:] Hi
\item [Source-node:] Driver::Driver()

\item [Lines:] 10

\item [Description:] The statement assigns a pointer to an empty string to name.  
This replaces the pointer to the memory allocated for name and results in a
memory leak.  This may also crash the program if name is modified.
\end{description}
\item {\it Issue\#196 (ccheung)}
\begin{description}
\item [Subject:] Variable assignment error
\item [Criticality:] Med
\item [Confidence-level:] Hi
\item [Source-node:] Driver::change

\item [Lines:] 5

\item [Description:] Varible 'name' is a pointer to a char array, Just assign name = 
temp\_name would not change the variable 'name'.
\end{description}
\item {\it Issue\#198 (gnakamur)}
\begin{description}
\item [Subject:] Destructor memory leak
\item [Criticality:] Low
\item [Confidence-level:] Hi
\item [Source-node:] Driver::\~Driver()

\item [Lines:] 3-6

\item [Description:] The destructor currently does nothing, but memory was allocated 
for the name member variable in the constructor.  This memory should be
deallocated before the object is destroyed or it will result in a memory
leak.
\end{description}
\item {\it Issue\#204 (gnakamur)}
\begin{description}
\item [Subject:] Illegal assignment of name member variable
\item [Criticality:] Hi
\item [Confidence-level:] Hi
\item [Source-node:] Driver::change

\item [Lines:] 5

\item [Description:] This statement assigns the pointer to the temp\_name string to 
name.  The pointer to the memory allocated for name is lost resulting in a
memory leak, and the temp\_name variable may be deallocated elsewhere causing
ill results if the name variable is referenced.
\end{description}
\item {\it Issue\#208 (ccheung)}
\begin{description}
\item [Subject:] Variable assignment error
\item [Criticality:] Med
\item [Confidence-level:] Hi
\item [Source-node:] Driver::Driver()

\item [Lines:] 10-11

\item [Description:] Variable 'name' is a pointer to a char array.  Just assigning 
name = "" would not change the variable 'name'.
\end{description}
\item {\it Issue\#212 (ccheung)}
\begin{description}
\item [Subject:] Deallocation
\item [Criticality:] Hi
\item [Confidence-level:] Hi
\item [Source-node:] Driver::\~Driver()

\item [Lines:] 

\item [Description:] Did not explictly deallocate unused pointer space.
\end{description}
\item {\it Issue\#214 (gnakamur)}
\begin{description}
\item [Subject:] Incorrect while condition
\item [Criticality:] Hi
\item [Confidence-level:] Hi
\item [Source-node:] main

\item [Lines:] 33

\item [Description:] The while statement executes when userinput\_shift is the valid 
values of 1, 2, or 3, when it should execute when an illegal value is inputed
for userinput\_shift (eg. not 1, 2, or 3).
\end{description}
\item {\it Issue\#218 (gnakamur)}
\begin{description}
\item [Subject:] Incorrect allocation of memory for userinput\_driver
\item [Criticality:] Med
\item [Confidence-level:] Hi
\item [Source-node:] main

\item [Lines:] 26

\item [Description:] Memory is allocated for userinput\_driver within the main while 
loop, but is deallocated outside of the while loop at the end of the program.
If the user changes the driver information more than once, the result is a
memory leak.  If the user just exits the program without changing the driver,
the userinput\_driver variable is undefined when passed to the delete[]
function and the results are unpredictable.
\end{description}
\item {\it Issue\#222 (ccheung)}
\begin{description}
\item [Subject:] Input value assignment error
\item [Criticality:] Med
\item [Confidence-level:] Med
\item [Source-node:] main

\item [Lines:] 25-27

\item [Description:] 'userinput\_driver' is not getting the user input.
\end{description}
\item {\it Issue\#226 (ccheung)}
\begin{description}
\item [Subject:] Space allocation error
\item [Criticality:] Med
\item [Confidence-level:] Low
\item [Source-node:] main

\item [Lines:] 9-10

\item [Description:] 'tempchar' is a single character.  It would not hold the user 
input.
\end{description}
\item {\it Issue\#234 (awong)}
\begin{description}
\item [Subject:] memory leak
\item [Criticality:] Low
\item [Confidence-level:] Med
\item [Source-node:] Driver::\~Driver()

\item [Lines:] 1-5

\item [Description:] destructor does not provide a way to free the memory allocated 
by the name character pointer
.
\end{description}
\item {\it Issue\#238 (awong)}
\begin{description}
\item [Subject:] improper space allocated
\item [Criticality:] Med
\item [Confidence-level:] Hi
\item [Source-node:] Driver::change

\item [Lines:] 5

\item [Description:] the program does not allocated space for the name character 
pointer.
\end{description}
\item {\it Issue\#242 (awong)}
\begin{description}
\item [Subject:] improper uses of "{\tt <}" "{\tt >}" symbol in while loop
\item [Criticality:] Low
\item [Confidence-level:] Hi
\item [Source-node:] main

\item [Lines:] 33-37

\item [Description:] shift is 1,2,3 and loop does not allow user to enter these 
values to insert to shift
\end{description}
\item {\it Issue\#248 (awong)}
\begin{description}
\item [Subject:] improper assignment to pointer
\item [Criticality:] Med
\item [Confidence-level:] Low
\item [Source-node:] Driver::Driver()

\item [Lines:] 10

\item [Description:] the assignment of a "" to pointer name is not correct
\end{description}
\item {\it Issue\#254 (awong)}
\begin{description}
\item [Subject:] not enough space allocated
\item [Criticality:] Low
\item [Confidence-level:] Low
\item [Source-node:] main

\item [Lines:] 26

\item [Description:] space allocated should be maxlength + 1 to allow for the '/0' 
character just in case userinput\_driver equals maxlength.
\end{description}
\item {\it Issue\#258 (awong)}
\begin{description}
\item [Subject:] not creating instances of drivers
\item [Criticality:] Low
\item [Confidence-level:] Hi
\item [Source-node:] main

\item [Lines:] 54-57

\item [Description:] The program is not really creating new instances of drivers. 
Basically it is just changing the old instance with new values
\end{description}
\end{enumerate}
\section{Review Metrics}
\begin{table}[hb]
\begin{center}
\begin{tabular}{|l|l|l|l|}
\hline
Participant & Start-time & End-time & Total Busy-time \\
\hline
gnakamur & Nov 28, 1994 10:30:38 & Nov 28, 1994 11:26:36 & 0:47:51 \\
ccheung & Nov 28, 1994 10:30:18 & Nov 28, 1994 11:32:47 & 0:53:50 \\
awong & Nov 28, 1994 15:31:23 & Nov 28, 1994 16:14:40 & 0:43:17 \\
\hline
\end{tabular}
\end{center}
\caption{Review Session}
\end{table}


\begin{table}[hb]
\begin{center}
\begin{tabular}{|l|l|l|l|}
\hline
Source & gnakamur & ccheung & awong\\
\hline
(176)Driver::\~Driver() & 232 & 211 & 290\\
(178)Driver::change & 430 & 384 & 295\\
(180)Driver::print & 61 & 106 & 92\\
(182)print\_error & 20 & 22 & 22\\
(184)print\_menu & 45 & 58 & 76\\
(170)Constant & 39 & 168 & 74\\
(186)main & 1292 & 1515 & 1311\\
(172)Driver & 207 & 131 & 176\\
(174)Driver::Driver() & 515 & 600 & 240\\
\hline
\end{tabular}
\end{center}
\caption{Review Time}
\end{table}


\begin{table}[hb]
\begin{center}
\begin{tabular}{|l|l|l|l|l|}
\hline
Source & gnakamur & ccheung & awong & OK\\
\hline
(176)Driver::\~Driver() & \#198 & \#212 & \#234 & 198=212=234 \\
(178)Driver::change & \#204 & \#196 & \#238 & 204=196=238 \\
(180)Driver::print &  &  & & \\
(182)print\_error &  &  & & \\
(184)print\_menu &  &  & & \\
(170)Constant &  &  & & \\
(186)main & \#214,\#218 & \#222,\#226 & \#242,\#254, & 214=242,218\\
          & (=2)        &  (=2) & \#258 (=3) & 254 \\
(172)Driver &  &  & & \\
(174)Driver::Driver() & \#192& \#208 & \#248 & 192=208=248 \\
\hline
\end{tabular}
\caption{Source node v.s Issue node}
\end{center}
\end{table}

%%%\end{document}
