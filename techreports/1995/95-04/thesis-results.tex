%%%%%%%%%%%%%%%%%%%%%%%%%%%%%% -*- Mode: Latex -*- %%%%%%%%%%%%%%%%%%%%%%%%%%%%
%% thesis-results.tex -- 
%% Author          : Carleton Moore
%% Created On      : Tue Jan 10 12:02:31 1995
%% Last Modified By: Carleton Moore
%% Last Modified On: Thu Jul  6 19:41:06 1995
%% Status          : Unknown
%% RCS: $Id: thesis-results.tex,v 1.11 1995/07/07 03:58:32 cmoore Exp cmoore $
%%%%%%%%%%%%%%%%%%%%%%%%%%%%%%%%%%%%%%%%%%%%%%%%%%%%%%%%%%%%%%%%%%%%%%%%%%%%%%%
%%   Copyright (C) 1995 University of Hawaii
%%%%%%%%%%%%%%%%%%%%%%%%%%%%%%%%%%%%%%%%%%%%%%%%%%%%%%%%%%%%%%%%%%%%%%%%%%%%%%%
%% 

%for review purposes
%\ls{1}

\newpage
\chapter{Study Results}
\label{sec:analysis}

This chapter presents the results from the evaluation of AEN during Spring,
1995.  Section \ref{sec:data-overview} briefly overviews the results of the
groups' usage of AEN.  Section \ref{sec:AEN-supports} presents the
evaluation of AEN's support of strong collaboration.  Section
\ref{sec:Tools} discusses the major features of AEN that were designed to
support strong collaboration.  Section \ref{sec:eval-AEN} presents the
evaluation of AEN as a collaborative tool.  Section \ref{sec:bottom-up}
closes this chapter with a brief description of the three groups'
procedures for collaboration.


\section{Overview of the Collected Data}
\label{sec:data-overview}

\small
\begin{table}[htbp]
  \begin{center}
    \caption{Statistics Summary of AEN usage.}
    \begin{tabular}{|l|l|l|l|}
      \hline
      {\bf Group}&{\bf No.}&{\bf Usage Time}&{\bf Node}\\
      &{\bf Sessions}&{\bf (hrs)}&{\bf Counts}\\ \hline
      A&109&44.4&108\\ \hline
      B&120&70.4&66\\ \hline
      C&90&36.0&35\\\hline
      \hline
      Indiv. Avg.&20&9.8& ---\\ \hline
      \hline
      Indiv. Std.&12&5.4& ---\\ \hline
      \hline
      Group Avg.&106&49&70\\ \hline
      \hline
      Group Std.&12&15&30\\ \hline
    \end{tabular}
  \end{center}
  \label{tab:summary}
\end{table}  
\normalsize

At the end of the study, I analyzed the database, metrics, bug reports,
user suggestions and user surveys.  Table \ref{tab:summary} provides a
summary of the statistics of the three groups.  Over the course of the
study, the three groups spent over 180 hours using AEN. They created 100
document nodes, 111 comment nodes, and 5 figure nodes.  There were large
differences in how the different groups and members used AEN.  Thirteen of
the fifteen users returned their post-test surveys.

The bug reports for Fall, 1994 were compared with the bug reports for
Spring, 1995.  I classified the bug reports into three categories, system
bugs, data corruption bugs, and user misunderstanding.  The system bugs
category describes bug reports on errors caused by AEN doing something
wrong. For example, the following is a system error bug report from the
Spring, 1995 case study. 

\ls{1.0}

\small
\begin{quote}
  \begin{verbatim}
Description of what happened: 

I tried to quit(AEN menu) after closing the Partyline 
window but I couldn't. 

I got error message :
Error signalled by u*error system: gi*partyline, 
invalid-db-ID, recoverable
  \end{verbatim}
\end{quote}
\normalsize

\ls{1.5}

The data corruption category is for bug reports describing an error in
the HTML data stored in a node. For example, the following is a data
corruption bug report from the
Fall, 1994 case study.

\ls{1.0}

\small 
\begin{quote}
  \begin{verbatim}
Description of what happened: 

Inaba's "Help by QUESTION" node under the AEN tutorial
cannot be retrieved.  (invalid link-ID).  Can you
please repair this corruption?
  \end{verbatim}
\end{quote}
\normalsize

\ls{1.5}

The user misunderstanding category is for bug reports that describe an
error that is due to the user's misunderstanding of AEN.  For example, the
following is a data corruption bug report from the Spring, 1995 case study.

\ls{1.0}

\small
\begin{quote}
  \begin{verbatim}
Description of what happened: 

I am just curious why when I type in the "less than sign" 
is gives me '&lt;' instead.  I am not able to make some html 
statements without the sign.
  \end{verbatim}
\end{quote}
\normalsize

\ls{1.5}

AEN's has a menu item that allows the user to insert a $<$ character instead
of the HTML code, ``\&lt;''.

\small
\begin{table}
  \begin{center}
    \caption{Classification of Bug Reports.}
    \begin{tabular}{|l|c|c|c||c|}
      \hline
      \multicolumn{5}{|c|}{\rule[-3mm]{0mm}{8mm}\bf Bug Reports}\\ \hline
      Semester&System Errors&Data Corruption&User Misunderstanding&Total\\ \hline
      \hline
      Fall, 1994&12&3&1&16\\ \hline
      Spring, 1995&5&1&3&9\\\hline
    \end{tabular}
    \label{tab:bugs}
  \end{center}
\end{table}
\normalsize
    
Table \ref{tab:bugs} shows the numbers and classification of bug reports
submitted during the two semesters.  There were seven fewer bug reports
during the Spring, 1995 use of AEN than the Fall, 1994 use.  Of the Spring,
1995 bug reports, 56\% were due to system errors.  In Fall, 1994 75\% of
the bug reports were due to system errors.  Interestingly, the number of
user misunderstandings is larger in the Spring, 1995 usage.  This may be
due to the experience difference in subjects.  In Fall, 1994, AEN was used
by graduate students and in Spring, 1995, AEN was used
by undergraduates.  %This could be due to??

This study investigated two questions. First, does AEN support strong
collaboration? Second, does AEN's set of tools promote collaboration?  The
next section discusses how well AEN supports strong collaboration.

\section{Does AEN support strong collaboration?}
\label{sec:AEN-supports}

To answer this question, each of the metrics for each component of the
operationalized definition are examined in turn.  The analysis of each
metric presents two sets of numbers, one relating to all of the nodes owned
by the group, the other focusing on the only the nodes that appear in the
group's final requirements document.  For the purpose of the this
evaluation, I will look only at the groups' document nodes, since the
comment nodes are for feedback purposes.  I do not expect collaborative
editing on a comment intended as feedback.  In order to say that AEN does
support the strongest collaboration possible, a single group must
demonstrate all four components of the operationalized definition of strong
collaboration presented in Section \ref{sec:operationalized-definition}.

\subsection{RPN: Readers per Node}

The first metric I will look at, is the readers per node (RPN) metric. For
this study, the lowest RPN value is one, indicating that no member of the
group read any nodes other than those he created.  The highest possible RPN
value is five, indicating that every member read every node.  Each group
in the study behaved differently with respect to RPN.

\subsubsection{Group A}

\small
\begin{table}[htb]
  \caption{Group A RPN Metric Breakdown.}
  \begin{center}
    \begin{tabular}{|c|c|c|c|c|}
      \hline
      \multicolumn{5}{|c|}{\rule[-3mm]{0mm}{8mm}\bf RPN metric of Group A
      Members}\\ \hline 
      Readers/Node&All Nodes&{\%}&Final Doc Nodes&{\%}\\ \hline 
      \hline
      1&14&41.2&0&0.0\\ \hline
      2&6&17.6&4&25.0\\ \hline
      3&4&11.8&4&25.0\\ \hline
      4&3&8.8&2&12.5\\\hline 
      5&7&20.6&6&37.5\\\hline 
      \hline
      &\multicolumn{2}{|l|}{All Nodes}&\multicolumn{2}{|c|}{2.5
      Readers/Node}\\ \cline{2-5} 
      \raisebox{1.5ex}[0pt]{\bf RPN}&\multicolumn{2}{|l|}{Final
      Document}&\multicolumn{2}{|c|}{3.6 Readers/Node}\\\hline
    \end{tabular}
  \end{center}
  \label{tab:A-reading}
\end{table}
\normalsize

Table \ref{tab:A-reading} shows the RPN breakdown for Group A.  Group
A had the full range of readers per node values.  They had a fairly even
distribution of nodes with different readers per node values.  Even in the
final document, they still had a wide range of readers per node values,
unlike the other groups.  One quarter of their final document had 2 readers
per node, while 37.5\% of the nodes had five readers.  Group A's RPN
value for their final document is 3.6, the lowest RPN value for the three
groups.  However, this value is still in the top half of the RPN range.

\subsubsection{Group B}

\small
\begin{table}[htbp]
  \caption{Group B RPN Breakdown.}
  \begin{center}
    \begin{tabular}{|c|c|c|c|c|}
      \hline
      \multicolumn{5}{|c|}{\rule[-3mm]{0mm}{8mm}\bf RPN metric of Group B
      Members}\\  
      \hline
      Readers/Node&All Nodes&\%&Final Doc Nodes&\%\\ \hline
      \hline
      1&19&46.3&0&0.0\\ \hline
      2&1&2.4&0&0.0\\ \hline
      3&0&0.0&0&0.0\\ \hline
      4&6&14.6&3&21.4\\\hline 
      5&15&36.6&11&78.6\\\hline
      \hline
      &\multicolumn{2}{|l|}{All Nodes}&\multicolumn{2}{|c|}{2.9
      Readers/Node}\\ \cline{2-5} 
      \raisebox{1.5ex}[0pt]{\bf RPN}&\multicolumn{2}{|l|}{Final
      Document}&\multicolumn{2}{|c|}{4.8 Readers/Node}\\\hline
    \end{tabular}
  \end{center}
  \label{tab:Ku-reading}
\end{table}
\normalsize

Group B had the same range for their readers per node values as
Group A, but their distribution was very different than Group A's.  Figure
\ref{tab:Ku-reading} shows their readers per node values and RPN values.
Over 80\% of all their nodes had one or five readers per node.  In their
final document, they did not use any nodes with fewer than four readers per
node.  Their RPN value, for their final document, was 4.8.  Their RPN value
is much higher than Group A's. It is in the top 4\% of the RPN range.

\subsubsection{Group C}

\small
\begin{table}[htb]
  \caption{Group C RPN Breakdown.}
  \begin{center}  
    \begin{tabular}{|c|c|c|c|c|}
      \hline
      \multicolumn{5}{|c|}{\rule[-3mm]{0mm}{8mm}\bf RPN metric of Group C
      Members}\\ 
      \hline
      Readers/Node&All Nodes&\%&Final Doc Nodes&\%\\ \hline
      \hline
      1&11&44.0&0&0.0\\ \hline
      2&2&8.0&0&0.0\\ \hline
      3&0&0.0&0&0.0\\ \hline
      4&0&0.0&0&0.0\\\hline 
      5&12&48.0&10&100.0\\\hline
      \hline
      &\multicolumn{2}{|l|}{All Nodes}&\multicolumn{2}{|c|}{3.0
      Readers/Node}\\ \cline{2-5} 
      \raisebox{1.5ex}[0pt]{\bf RPN}&\multicolumn{2}{|l|}{Final
      Document}&\multicolumn{2}{|c|}{5.0 Readers/Node}\\\hline
    \end{tabular}
  \end{center}  
  \label{tab:Na-reading}
\end{table}
\normalsize


Group C had the largest clustering of reader per node values near the
ends of the scale.  Table \ref{tab:Na-reading} shows the RPN breakdown for
Group C.  All but two of their nodes had one or five readers per node.
In their final document they only chose nodes that had five readers per
node.  They demonstrated the highest possible RPN value (5) in their final
document.


All three groups had RPN values in the top half of the RPN scale.
The next metric looks at collaborative editing of nodes. 



%\subsection{MCE: Member Co-editing}
%
%This metric focusses on the members and whether they work together.  To do
%this it looks at the collaboration from a ``group dynamic point'' of view.
%How does the group collaborate to construct the document?  The range of
%possible collaboration modes in this study range from, members only editing
%their own nodes, a score of one, to each member editing with every other
%member, a score of five.  In the strongest editing each member would have a
%MCE score of five.
%
%\small
%\begin{table}[htb]
%  \begin{center}  
%    \begin{tabular}{|c|c|c|}
%      \hline
%      \multicolumn{3}{|c|}{\rule[-3mm]{0mm}{8mm}\bf MCE metric for Group A}\\ 
%      \hline
%      Member&Ave. All Nodes&Ave. Final Doc\\ \hline
%      \hline
%      clam&2.8&4.0\\ \hline
%      eugenio&3.7&3.8\\ \hline
%      jedwards&2.1&3.2\\ \hline
%      mching&2.1&2.4\\\hline 
%      rogawa&2.8&3.2\\\hline
%      \hline
%      &All Nodes&2.7 co-editors\\ \cline{2-3} 
%      \raisebox{1.5ex}[0pt]{\bf MCE}&Final Document&3.2 co-editors\\\hline
%    \end{tabular}
%  \end{center}  
%  \caption{Group A MCE Results.}
%  \label{tab:A-MCE}
%\end{table}
%\normalsize
%
%
%\small
%\begin{table}[htb]
%  \begin{center}  
%    \begin{tabular}{|c|c|c|}
%      \hline
%      \multicolumn{3}{|c|}{\rule[-3mm]{0mm}{8mm}\bf MCE metric for Group B}\\ 
%      \hline
%      Member&Ave. All Nodes&Ave. Final Doc\\ \hline
%      \hline
%      hhwang&4.5&4.5\\ \hline
%      jimh&2.8&4.13\\ \hline
%      rsato&3.3&4.13\\ \hline
%      rshen&3.2&4.7\\\hline 
%      shuang&4.1&4.4\\\hline
%      \hline
%      &All Nodes&3.6 co-editors\\ \cline{2-3} 
%      \raisebox{1.5ex}[0pt]{\bf MCE}&Final
%      Document&4.4 co-editors\\\hline
%    \end{tabular}
%  \end{center}  
%  \caption{Group B MCE Results.}
%  \label{tab:Ku-MCE}
%\end{table}
%\normalsize
%
%\small
%\begin{table}[htb]
%  \begin{center}  
%    \begin{tabular}{|c|c|c|}
%      \hline
%      \multicolumn{3}{|c|}{\rule[-3mm]{0mm}{8mm}\bf MCE metric for Group C}\\
%      \hline
%      Member&Ave. All Nodes&Ave. Final Doc\\ \hline
%      \hline
%      bacani&3.4&4.7\\ \hline
%      cokomoto&2.9&4.1\\ \hline
%      kiyuna&3.3&3.8\\ \hline
%      msamson&3.0&3.8\\\hline 
%      shan&4.1&4.1\\\hline
%      \hline
%      &All Nodes&3.3 co-editors\\ \cline{2-3} 
%      \raisebox{1.5ex}[0pt]{\bf MCE}&Final Document&4.1 co-editors\\\hline
%    \end{tabular}
%  \end{center}  
%  \caption{Group C MCE Results.}
%  \label{tab:Na-MCE}
%\end{table}
%\normalsize
%
%\small
%\begin{table}[htb]
%  \begin{center}
%    \begin{tabular}{|c|c||c|c||c|c|}
%      \hline \multicolumn{6}{|c|}{\rule[-3mm]{0mm}{8mm}\bf MCE metrics}\\
%      \hline 
%      \multicolumn{2}{|c||}{\rule[-3mm]{0mm}{8mm}\bf A}&
%      \multicolumn{2}{|c||}{\rule[-3mm]{0mm}{8mm}\bf B}&
%      \multicolumn{2}{|c|}{\rule[-3mm]{0mm}{8mm}\bf C}\\ \hline
%      Member&Co-Edits&Member&Co-Edits&Member&Co-Edits\\ \hline 
%      \hline 
%      clam&5&hhwang&5&bacani&5\\\hline 
%      eugenio&5&jimh&5&cokumoto&5\\ \hline
%      jedwards&5&rsato&5&kiyuna&5\\ \hline
%      mching&5&rshen&5&msamson&5\\ \hline
%      rogawa&5&shuang&5&shan&5\\\hline
%    \end{tabular}
%  \end{center}
%  \caption{MCE metrics for all groups.}
%  \label{tab:MCE}
%\end{table}
%\normalsize
%
%
%Groups B and C both had nodes that all five members worked
%on.  Therefore, each member gets an MCE value of five.  Group A was the
%only group that did not have any nodes where all five members edited.
%However, each member worked with four other members at least once in
%the other nodes.  All the groups demonstrated the strongest MCE values
%possible.
%
%After looking at the members collaborating, the next step is to look at the
%individual nodes. 


\subsection{EPN: Editors per Node}

The Editors per node (EPN) metric calculates the average number of editors
per node.  The lowest EPN value possible for this study is one, indicating
that each node was only edited by one member.  The highest possible EPN
value is five, indicating that every node was edited by all five group
members.

\subsubsection{Group A}

\small
\begin{table}[htbp]
  \caption{Group A EPN Breakdown.}
  \begin{center}
    \begin{tabular}{|c|c|c|c|c|}
      \hline
      \multicolumn{5}{|c|}{\rule[-3mm]{0mm}{8mm}\bf EPN metric for Group A}\\ \hline
      Editors/Node&All Nodes&\%&Final Doc Nodes&\%\\ \hline
      \hline
      1&21&61.8&3&18.8\\ \hline
      2&6&17.6&6&37.5\\ \hline
      3&2&5.9&1&6.2\\ \hline
      4&5&14.7&5&31.3\\\hline 
      5&0&0.0&0&0.0\\\hline 
      \hline
      &\multicolumn{2}{|l|}{All Nodes}&\multicolumn{2}{|c|}{1.7
      Editors/Node}\\ \cline{2-5} 
      \raisebox{1.5ex}[0pt]{\bf EPN}&\multicolumn{2}{|l|}{Final
      Document}&\multicolumn{2}{|c|}{2.7 Editors/Node}\\\hline
    \end{tabular}
  \end{center}
  \label{tab:A-editing}
\end{table}
\normalsize

Group A was the only group to not have a single node that had all five
members edit it.  Figure \ref{tab:A-editing} shows the EPN breakdown for
Group A.  For all the nodes that Group A created, 79\% of the nodes had
one or two editors per node.  In their final document, only 37\% of the
nodes had three or more editors per node.  Group A's EPN value of 2.7 for
their final document, is in the lower half of the EPN range.  Group A's
EPN values are the lowest of the three groups.

\subsubsection{Group B}

\small
\begin{table}[htb]
  \caption{Group B EPN Breakdown.}
  \begin{center}
    \begin{tabular}{|c|c|c|c|c|}
      \hline
      \multicolumn{5}{|c|}{\rule[-3mm]{0mm}{8mm}\bf EPN metric for Group B}\\ \hline
      Editors/Node&All Nodes&\%&Final Doc Nodes&\%\\ \hline
      \hline
      1&23&56.1&0&0.0\\ \hline
      2&3&7.3&0&0.0\\ \hline
      3&3&7.3&2&14.3\\ \hline
      4&6&14.6&6&42.9\\\hline 
      5&6&14.6&6&42.9\\\hline
      \hline
      &\multicolumn{2}{|l|}{All Nodes}&\multicolumn{2}{|c|}{2.2
      Editors/Node}\\ \cline{2-5} 
      \raisebox{1.5ex}[0pt]{\bf EPN}&\multicolumn{2}{|l|}{Final
      Document}&\multicolumn{2}{|c|}{4.3 Editors/Node}\\\hline
    \end{tabular}
  \end{center}
  \label{tab:Ku-editing}
\end{table}
\normalsize

Group B's distribution of editors per node is very different than Group A.
Table \ref{tab:Ku-editing} shows the EPN breakdown for Group B.  Almost a
third, 29\%, of all their nodes had four or five editors per node.  In
their final requirements document, they had an EPN value of 4.3.  This
value is in the top 20\% of the EPN range.  Group B's EPN value is the
highest EPN value of all three groups.

\subsubsection{Group C}

\small
\begin{table}[htb]
  \caption{Group C EPN Breakdown.}
  \begin{center}
    \begin{tabular}{|c|c|c|c|c|}
      \hline
      \multicolumn{5}{|c|}{\rule[-3mm]{0mm}{8mm}\bf EPN metric for Group C }\\ \hline
      Editors/Node&All Nodes&\%&Final Doc Nodes&\%\\ \hline
      \hline
      1&13&52.0&0&0.0\\ \hline
      2&2&8.0&1&10.0\\ \hline
      3&2&8.0&2&20.0\\ \hline
      4&6&24.0&5&50.0\\\hline 
      5&2&8.0&2&20.0\\\hline
      \hline
      &\multicolumn{2}{|l|}{All Nodes}&\multicolumn{2}{|c|}{2.3
      Editors/Node}\\ \cline{2-5} 
      \raisebox{1.5ex}[0pt]{\bf EPN}&\multicolumn{2}{|l|}{Final
      Document}&\multicolumn{2}{|c|}{3.6 Editors/Node}\\\hline
    \end{tabular}
  \end{center}
  \label{tab:Na-editing}
\end{table}
\normalsize

Group C also had high values for their editors per node.  Table
\ref{tab:Na-editing} shows their breakdown.  Group C's distribution of
editors per node values was very similar to Group B's.  However, for the
final document, only 70\% of their nodes had four or five editors per node.
In contrast, 86\% of Group B's nodes had four or five editors per node.
Group C's EPN value for their final document was 3.8, which is in the
upper half of the EPN range.


This section discussed one way to provide input on a node of the document ---
direct editing.  Another way to provide input is to create feedback or
comment nodes.  The next section discusses the feedback creation metric of strong
collaboration, FNC.

\subsection{FNC: Feedback Node Creation}

The Feedback Node Creation (FNC) metric in combination with EPN helps define
how much collaborative input occurred. The creation of feedback nodes gives
an indication how much discussion is going on among the authors of the
document.  FNC calculates the percentage of document nodes that have been
commented on. The lowest possible FNC value is 0\%, indicating that no
document nodes were commented on.  The highest FNC value is 100\%,
indicating that all document nodes were commented on.  
\small
\begin{table}[htb]
  \caption{Feedback Nodes.}
  \begin{center}
    \begin{tabular}{|c|c|c||c|c|}
      \hline
      \multicolumn{5}{|c|}{\rule[-3mm]{0mm}{8mm}\bf FNC metric for
      all groups}\\ \hline
      Group&All Nodes&FNC&Final Doc Nodes&FNC\\ \hline
      \hline
      A&10&29.4\%&8&50.0\%\\ \hline
      B&8&19.5\%&6&42.9\%\\ \hline
      C&4&16.0\%&2&20.0\%\\ \hline
    \end{tabular}
  \end{center}
  \label{tab:feedback}
\end{table}
\normalsize

Table \ref{tab:feedback} shows the number of comment nodes and the FNC
values for each group.  Different collaboration styles can easily be seen
in the different FNC values.  Group A, with an FNC of 50\%, used comments
widely, while Group C, with an FNC of 20\%, hardly used any comments
at all. 
% These higher FNC values might explain group A's lower EPN scores

The next component of the definition is how the users control the access to
the nodes.

\subsection{NAC: Non-default Access Control}

The Non-default Access Control (NAC) metric is the first of two access
control metrics.  It calculates the percentage of nodes that have had their
access control changed from the default.  The lowest NAC value possible is
0\%, indicating that no nodes had their access control changed from the
default.  This also indicates that there was no interaction among the
members since the default access control is no access for other members.
The highest possible NAC value is 100\%, indicating that every node has had
its access control modified.

\small
\begin{table}[htb]
  \caption{NAC Breakdown.}
  \begin{center}
    \begin{tabular}{|c|c|c|c|c|}
      \hline
      \multicolumn{5}{|c|}{\rule[-3mm]{0mm}{8mm}\bf NAC metric by group}\\ \hline
      Group&All Nodes&NAC&Final Doc Nodes&NAC\\ \hline
      \hline
      A&26&76.5\%&16&100.0\%\\ \hline
      B&23&56.1\%&14&100.0\%\\ \hline
      C&16&64.0\%&10&100.0\%\\ \hline
    \end{tabular}
  \end{center}
  \label{tab:access-control}
\end{table}
\normalsize

Table \ref{tab:access-control} shows the NAC values for each group.  In the
final document, 100\% of the nodes for all groups had their access control
changed.  This is not surprising.  Since all groups had RPN values for their
final documents greater than one, the access control for all those nodes
must have been changed to allow other members access to the node.  The
relatively high NAC value for A's total document implies that they
created fewer ``private'' nodes.  Private nodes are nodes whose access
control does not let any one else read or change them.

\subsection{EAC: Evolving Access Control}

The second metric dealing with access control is Evolving Access Control
(EAC). It calculates the percentage of document nodes that have had their
access control changed more than once. This metric measures the degree to
which the groups' collaboration changed. The lowest EAC value possible is
0\%, indicating that no nodes had their access control changed more than
once.  The highest possible EAC value is 100\%, indicating that all nodes
had their access control changed more than once.

\small
\begin{table}[htb]
  \caption{EAC Breakdown.}
  \begin{center}
    \begin{tabular}{|c|c|c||c|c|}
      \hline
      \multicolumn{5}{|c|}{\rule[-3mm]{0mm}{8mm}\bf EAC metric by group}\\ 
      \hline
      Group&All Nodes&NAC&Final Doc Nodes&NAC\\ \hline
      \hline
      A&7&21\%&7&44\%\\ \hline
      B&2&5\%&1&7\%\\ \hline
      C&2&8\%&1&10\%\\ \hline
    \end{tabular}
  \end{center}
  \label{tab:d-access-control}
\end{table}
\normalsize

Table \ref{tab:d-access-control} shows the results for the EAC metric.  All
three groups evolved the access control for a few of their nodes.  Group
A, which so far has had the lower metric scores in RPN and EPN, greatly
out scores the other two groups in EAC.  The absolute values for EAC are
low, 44\%, 7\%, and 10\%.  However, a high score in EAC represents a
dynamic collaborative process.  I did not expect the three groups to have
very dynamic collaborative process, because of their low experience with
group collaboration and because they had never collaborated together.

\subsection{Summary}


Does AEN support strong collaboration?  In order to answer positively, at
least one group must demonstrate strong collaboration in their hypertext
document. Figure \ref{fig:op-def} shows a view of all the metrics for each
group.

\begin{figure}[htbp]
  \centerline{\psfig{figure=metrics.eps,width=6in}}
 \caption{Summary of Groups' Collaboration Metrics.}
 \label{fig:op-def}
\end{figure}

Since no one group had uniformly high values for their five metrics, I
cannot say that any group collaborated in a ``purely'' strong manner on their
document.  This does not mean that AEN does not support strong
collaboration. There could be some obscure interaction between the five
major features of AEN that prevents a single group from having high scores
in all five metrics, but I do not believe this is true.

The one or more of the three groups showed high values for the RPN, EPN and
NAC metrics.  For the FNC and EAC metrics, group A scored 50\% and 44\%,
indicating that they used AEN to comment on and change the collaborative
style of their document.  Based upon the high scores in RPN, EPN and NAC
and Group A's ability to demonstrate the other two components of the
operationalized definition of strong collaboration, I claim that AEN is
capable of supporting strong collaboration.

Let us look at the specific features that AEN uses to support strong
collaboration.

\section{AEN's Features}
\label{sec:Tools}

AEN's set of features was designed to support and encourage collaboration
between members.  The features were developed during the Fall, 1994 semester
and evaluated in the Spring, 1995 semester.  This section discusses the
five major features provided by AEN in support of collaboration: access
control, tables of contents, node lists, Snoopy and Partyline.


\subsection{Access Control}

As discussed above, each group used the access control mechanisms during
the case study.  I hoped the design of access control would promote
evolving access control.  In other words, the access to a node would change
over the lifetime of the node reflecting the different states of
completeness and collaboration.  AEN's access control model was a partial
success.  All the groups did use the control mechanism to restrict access
to the node, but they did not use its evolutionary nature to significantly
change their mode of collaboration.


For each of the remaining four major features, I will present the usage
data and the survey results.  The usage data will compare the number of
sessions each user had and the number of instances they used the tool (when
this information is available).  The survey results indicate the tools'
helpfulness and their frequency of use.  The survey questions (see Appendix
\ref{app:questionnaire1} and \ref{app:questionnaire2}) ask the user to rate
different aspects of AEN from a scale of one to nine, with one being poor
and nine being good.

The first two features, tables of contents and node lists, are
navigation aids that help the users find their way around the hypertext
document.  The tools should help the users get to the interesting nodes of
their documents.  The last two tools, Snoopy and Partyline, are designed to
provide a sense of physical presence in the hypertext.  Each will be
discussed in turn.

\newpage
\subsection{Context-Sensitive Table of Contents}
\footnotesize
\begin{table}[htb]
  \caption{TOC Creations.}
  \begin{center}
    \begin{tabular}{|c|c|c||c|c|c||c|c|c|}
      \hline
      \multicolumn{9}{|c|}{\rule[-3mm]{0mm}{8mm}\bf Tables of Contents}\\
      \hline
      \multicolumn{3}{|c||}{\rule[-3mm]{0mm}{8mm}\bf Group A}&
      \multicolumn{3}{|c||}{\rule[-3mm]{0mm}{8mm}\bf Group B}&
      \multicolumn{3}{|c|}{\rule[-3mm]{0mm}{8mm}\bf Group C}\\ \hline
      Member&TOCs&Sessions&Member&TOCs&Sessions&Member&TOCs&Sessions\\ \hline
      \hline
      A1&10&7&B1&12&9&C1&11&16\\\hline 
      A2&10&10&B2&36&36&C2&38&17\\ \hline
      A3&66&49&B3&49&40&C3&17&17\\ \hline
      A4&20&18&B4&28&18&C4&14&15\\ \hline
      A5&9&9&B5&16&17&C5&25&25\\\hline
    \end{tabular}
  \end{center}
  \label{tab:A-TOC}
\end{table}
\normalsize

The table of contents (TOC) was the most used tool provided by AEN.  Table
\ref{tab:A-TOC} shows the number of times each user created a TOC vs. the
number of their sessions.  Since AEN does not automatically create a TOC,
the metrics show exactly how many times the users created tables of
contents.  Each group usage pattern of the TOCs is different.  Group A
created TOCs more often than the other groups.  Group C and B
are very close in their numbers.


\small
\begin{table}
  \caption{Survey Questions about Table of Contents. (Scale 1 to 9)}
  \begin{center}
    \begin{tabular}{|l|c|c|}
      \hline
      \multicolumn{3}{|c|}{\rule[-3mm]{0mm}{8mm}\bf Survey Questions:
      Table of Contents}\\ \hline
      &Average&Std. Dev.\\ \hline
      I used the Table of Contents:&8.7&0.7\\ \hline
      The Table of Contents is helpful:&8.3&1.3\\ \hline
    \end{tabular}
  \end{center}
  \label{tab:survey-TOC}
\end{table}
\normalsize

The survey results in Table \ref{tab:survey-TOC} show that the users felt
they used the TOC very often (8.7 out of 9, with a very small standard
deviation).  The users also thought that the TOC was a very helpful tool,
average score of 8.3.  

The next feature, node lists, complements tables of contents as a navigation
aid.

\newpage
\subsection{Node Lists}

\footnotesize
\begin{table}[htb]
  \caption{Node List Creations.}
  \begin{center}
    \begin{tabular}{|c|c|c||c|c|c||c|c|c|}
      \hline
      \multicolumn{9}{|c|}{\rule[-3mm]{0mm}{8mm}\bf Node Lists}\\
      \hline
      \multicolumn{3}{|c||}{\rule[-3mm]{0mm}{8mm}\bf Group A}&
      \multicolumn{3}{|c||}{\rule[-3mm]{0mm}{8mm}\bf Group B}&
      \multicolumn{3}{|c|}{\rule[-3mm]{0mm}{8mm}\bf Group C}\\ \hline
      Member&Lists&Sessions&Member&Lists&Sessions&Member&Lists&Sessions\\ \hline
      \hline
      A1&0&7&B1&2&9&C1&36&16\\\hline 
      A2&4&10&B2&5&36&C2&30&17\\ \hline
      A3&4&49&B3&5&40&C3&7&17\\ \hline
      A4&16&18&B4&20&18&C4m&28&15\\ \hline
      A5&1&9&B5&44&17&C5&0&25\\\hline
    \end{tabular}
  \end{center}
  \label{tab:A-Node List}
\end{table}
\normalsize

The groups' use and feelings towards node lists vary greatly.  Some group
members did not create any node lists, while others created more node lists
than tables of contents.  Table \ref{tab:A-Node List} shows the number of
node lists each member created.


\small
\begin{table}[htbp]
  \caption{Survey Questions about Node Lists. (Scale 1 to 9)}
  \begin{center}
    \begin{tabular}{|l|c|c|}
      \hline
      \multicolumn{3}{|c|}{\rule[-3mm]{0mm}{8mm}\bf Survey Questions:
      Node Lists}\\ \hline
      &Average&Std. Dev.\\ \hline
      I used the Unread Nodes:&5.6&2.8\\\hline
      I used the Nodes by type:&4.9&2.6\\\hline
      I used the Owned Nodes:&4.1&2.2\\\hline
      Unread Nodes is helpful helpful:&6.8&2.2\\\hline
      Nodes by type is helpful:&5.3&2.2\\\hline
      Owned Nodes is helpful:&5.1&2.2\\\hline
    \end{tabular}
  \end{center}
  \label{tab:survey-Nodelist}
\end{table}
\normalsize


The survey results (see Table \ref{tab:survey-Nodelist}) reflect the wide
variety in the metrics data.  The users did not feel that they had used the
node lists as much as the table of contents (Avg. 5.6 vs. Avg. 8.7).  The
metrics data shows that some users created more node lists than TOCs.  The
standard deviations of the node list surveys show this large variation.
The users thought that the Unread Nodes list was almost as helpful as the
TOC (Avg. 6.8 vs. Avg. 8.3).



The last two features, Snoopy and Partyline were designed to promote
collaboration in AEN.  They are discussed next.

\subsection{Snoopy}

\small
\begin{table}[htbp]
  \caption{Survey Questions about Snoopy. (Scale 1 to 9)}
  \begin{center}
    \begin{tabular}{|l|c|c|}
      \hline
      \multicolumn{3}{|c|}{\rule[-3mm]{0mm}{8mm}\bf Survey Questions:
      Snoopy}\\ \hline
      &Average&Std. Dev.\\ \hline
      I used Snoopy:&6.3&2.6\\\hline
      Snoopy is helpful:&6.8&2.4\\\hline
    \end{tabular}
  \end{center}
  \label{tab:survey-Snoopy}
\end{table}
\normalsize

The Snoopy feature is started with each AEN session.  To use the tool, the
user simply looks at a window that displays the usage information of all
users.  Thus there were no meaningful metrics collected about Snoopy.  The
survey results (see Table \ref{tab:survey-Snoopy}) show that the users
generally used and liked Snoopy.  The large standard deviation might be
attributed to the different user's method of collaboration.  Snoopy's
information is not very interesting if there are no other users logged in
or if the other members are present and accessible in the room.


The last feature evaluated in AEN is Partyline.
\subsection{Partyline} 

\footnotesize
\begin{table}[htbp]
  \caption{Partyline Messages.}
  \begin{center}
    \begin{tabular}{|c|c|c||c|c|c||c|c|c|}
      \hline
      \multicolumn{9}{|c|}{\rule[-3mm]{0mm}{8mm}\bf Partyline Messages}\\
      \hline
      \multicolumn{3}{|c||}{\rule[-3mm]{0mm}{8mm}\bf Group A}&
      \multicolumn{3}{|c||}{\rule[-3mm]{0mm}{8mm}\bf Group B}&
      \multicolumn{3}{|c|}{\rule[-3mm]{0mm}{8mm}\bf Group C}\\ \hline
      Member&Msgs.&Sessions&Member&Msgs.&Sessions&Member&Msgs.&Sessions\\ \hline
      \hline
      A1&0&7&B1&0&9&C1&1&16\\\hline 
      A2&0&10&B2&47&36&C2&0&17\\ \hline
      A3&53&49&B3&154&40&C3&10&17\\ \hline
      A4&158&18&B4&0&18&C4&22&15\\ \hline
      A5&41&9&B5&0&17&C5&15&25\\\hline
    \end{tabular}
  \end{center}
  \label{tab:A-Partyline}
\end{table}
\normalsize


The metrics for Partyline use (see Table \ref{tab:A-Partyline}) show
that there is a wide variety in the number of Partyline messages sent.
Two users, A4 and B3, have 62.3\% of all the Partyline messages
sent.  Comparing the times Partyline messages were sent and when users
were simultaneously editing node, it does not appear that Partyline was
used more than 15 times to coordinate editing.

\small
\begin{table}
  \caption{Survey Questions about Partyline. (Scale 1 to 9)}
  \begin{center}
    \begin{tabular}{|l|c|c|}
      \hline
      \multicolumn{3}{|c|}{\rule[-3mm]{0mm}{8mm}\bf Survey Questions:
      Partyline}\\ \hline
      &Average&Std. Dev.\\ \hline
      I used Partyline:&5.2&2.6\\\hline
      Partyline is helpful helpful:&6.2&2.4\\\hline
    \end{tabular}
  \end{center}
  \label{tab:survey-Partyline}
\end{table}
\normalsize

One possible reason for this low use of Partyline is referenced in the
limitations section of the experimental design.  The users could only
access AEN from two rooms.  This meant that they were frequently co-located
in the same room, and thus able to speak directly to each other instead of
using Partyline.  Since it is easier talk than type, I believe they
collaborated verbally instead of using Partyline.  This could lead to the
low score for Partyline use in the survey.  Table
\ref{tab:survey-Partyline} shows the results of the survey for Partyline.
The users felt that Partyline was fairly helpful even though the did not
use it very often.


\small
\begin{table}[htbp]
  \caption{Survey Question about Sense of Presence. (Scale 1 to 9)}
  \begin{center}
    \begin{tabular}{|l|c|c|}
      \hline
      \multicolumn{3}{|c|}{\rule[-3mm]{0mm}{8mm}\bf Survey Questions:
      Sense of Presence}\\ \hline
      &Average&Std. Dev.\\ \hline
      AEN provides a sense of:&6.5&1.9\\
      physical presence for&&\\
      the other users.&&\\\hline
    \end{tabular}
  \end{center}
  \label{tab:survey-Presence}
\end{table}
\normalsize

The above two features are designed to promote a sense of physical presence
for each of the users.  The survey asked the users about their sense of
physical presence.  Table \ref{tab:survey-Presence} shows the results of
the sense of presence survey question.  They felt that AEN does provide a
sense of physical presence for the other users.

\section{Evaluation of AEN}
\label{sec:eval-AEN}

The evaluation of the set of features provided by AEN shows that they
generally were used and liked by the group members.  There was a wide
variation in the feelings of the group members and different features were
used to different degrees.  A goal of this research is to evaluate AEN as a
collaborative editing tool.

There are three different sets of evidence that lead me to believe AEN is a
successful collaborative tool. One, all three groups collaboratively
created quality requirements documents. Two groups received an A on their
document and one received a B.  Two, all three requirements documents show
high values in at least three of their collaborative metrics.  Three, the
survey results show that the users liked AEN.  Table \ref{tab:survey-AEN}
shows the results of the survey questions about AEN.

\small
\begin{table}[htb]
  \caption{Survey Questions about AEN. (Scale 1 to 9)}
  \begin{center}
    \begin{tabular}{|l|c|c|}
      \hline
      \multicolumn{3}{|c|}{\rule[-3mm]{0mm}{8mm}\bf Survey Questions:
      Overall reactions to AEN:}\\ \hline
      &Average&Std. Dev.\\ \hline
      terrible---wonderful&5.5&1.6\\\hline
      frustrating---satisfying&5.0&1.8\\\hline
      dull---stimulating&5.0&1.7\\\hline
      difficult---easy&6.2&1.9\\\hline
      rigid---flexible&5.2&1.9\\\hline\hline
      AEN promotes collaboration:&7.2&2.0\\\hline
    \end{tabular}
  \end{center}
  \label{tab:survey-AEN}
\end{table}
\normalsize

The users felt that AEN promoted collaboration, but there was disagreement
about that.  One user did not think AEN promoted collaboration.  That user
rated AEN a two out of nine.

AEN as a system supports collaboration, but what kinds of collaboration
occurred during the study?  The next section discusses some of the styles
each group displayed during the study.

\section{Three Examples of Strong Collaboration}
\label{sec:bottom-up}

Each of the three groups had their own collaborative style.  Two groups
defined their processes in their AEN workspace.  One group divided up the
work and used a style with lower EPN and RPN metrics.  A second group's
style has high EPN and RPN metrics.  The next three sections briefly
discusses each of the three group's collaborative styles.

\subsection{Group A}

Group A used the style with the lowest EPN and RPN metrics of the three groups.
They produced their own custom process model for collaboration.  One node
discusses this process and is reproduced here:

\ls{1.0}
\begin{quote}
  {\em
  We need to develop some sort of an operating instruction to more
  effectively use AEN.  some difficulties that we have experienced so far
  have included:
  \begin{enumerate}
  \item{Multiple users at one time - confusion exists, collaboration
    breaks down.}
  \item{Review of documents and updating - how and who}
  \item{Do we give write access to everyone in the group? then, who controls
    the node?}
  \end{enumerate}
    
  Suggestions to improve:

  \begin{enumerate}
  \item{Every node created should be given write access by two individuals.
  The person who created the node and another.  Thus, there should be a
  primary author and a secondary author.}
  \item{Every member must be given read and annotate access!}
  \end{enumerate}}
\end{quote}
\ls{1.5}

I believe this policy led to their low number of readers and authors per
node as compared to the other groups.  Their policy tends to be a divide
and conquer process.  Not allowing all five member write access
automatically reduces the number of authors.  This low number of authors
also explains why their FNC metric was so high.  Since each node only has
two authors, the other three members must use commenting to provide input
to the node.

Group A was not the only group with a defined process.  The other group
that defined a process in their workspace was Group B.

\subsection{Group B}

Group B created a node where they would put any comments to the other
members.  In this node they outlined their process for using AEN:
\ls{1.0}
\begin{quote}
  {\em
  Please make nodes that you have created, not only readable, but also
  writable to other group members, at least annotation rights so that we
  comment on what you have done.  Due to the fact that we have a lot of
  work that should be done on AEN, I would like us to eventually put most
  of our questions, comments and work surrounding the project in AEN.
  }
\end{quote}
\ls{1.5}

The instructions to make all nodes readable and writable to the other group
members leads to their high number of nodes with five editors.

\subsection{Group C}

The last group, C did not publish their process.  I believe that their lack
of a published process leads to their median results between Group A and
Group B with respect to EPN.  They did not have a process that restricted
the number of authors nor promoted having all five members author the
nodes.




