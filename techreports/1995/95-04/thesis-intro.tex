%%%%%%%%%%%%%%%%%%%%%%%%%%%%%% -*- Mode: Latex -*- %%%%%%%%%%%%%%%%%%%%%%%%%%%%
%% thesis-intro.tex -- 
%% Author          : Carleton Moore
%% Created On      : Tue Jan 10 11:58:40 1995
%% Last Modified By: Carleton Moore
%% Last Modified On: Thu Jul  6 17:28:13 1995
%% Status          : Unknown
%% RCS: $Id: thesis-intro.tex,v 1.15 1995/07/07 03:28:26 cmoore Exp $
%%%%%%%%%%%%%%%%%%%%%%%%%%%%%%%%%%%%%%%%%%%%%%%%%%%%%%%%%%%%%%%%%%%%%%%%%%%%%%%
%%   Copyright (C) 1995 University of Hawaii
%%%%%%%%%%%%%%%%%%%%%%%%%%%%%%%%%%%%%%%%%%%%%%%%%%%%%%%%%%%%%%%%%%%%%%%%%%%%%%%
%% 

%for review purposes
%\ls{1}

\chapter{Introduction}
\label{sec:introduction}

\section{Motivation}

With the rapid expansion of the World Wide Web (WWW) \cite{Berners-Lee94},
the use of hypertext as a structuring mechanism for richly interdependent
information is becoming widespread.  Indeed, hypertext is almost certain to
be the `lingua franca' of the information superhighway.  The WWW allows
authors to collaborate by including sections of other authors' documents.
Unfortunately, WWW provides little support for collaborative authoring
beyond including someone else's work.

The issues associated with collaborative authoring have been investigated
by many researchers.  Baecker et. al. \cite{Baecker93} provide several
requirements for collaborative writing systems.  They include: preservation
of identities, enhanced communications, enhanced collaborator awareness,
annotations, undo, session control, explicit roles, variety of activities,
transition between activities, several document access methods, separate
document segments, version control, one or several writers, and synchronous
and asynchronous writing.

This thesis investigates the issues related to collaborative hypertext
document construction.  Specifically, it investigates and characterizes a
form of collaboration termed {\em strong collaboration}.  In strong
collaboration, each participant contributes to the construction of the
document, and gains new knowledge as a direct result of this construction.
The constructed document is not simply a patchwork of individual
contributions, but instead, an incremental, emergent synthesis that
reflects the knowledge created by the group as a whole.  In general,
strong collaboration most often (though not always) occurs in non-computer
mediated, face-to-face contexts such as a group software design session or
an interactive classroom setting.  A distinguishing characteristic of strong
collaboration is a sense of collective authorship over most, if not all of
the components of the document.  Documents produced in this way should be of
higher quality, because more perspectives have combined to produce, review,
and correct errors in it.

Strong collaboration is at one end of a collaborative spectrum.  I call
the other end of the spectrum {\em weak collaboration}.  Weak collaboration
uses a divide and conquer process --- participants divide up the document
and work independently on their section.  There is very little interaction
between the participants.  When all of the sections are finished, the
document is created by combining the sections into a whole.  A
distinguishing characteristic of weak collaboration is distinct authorship
of components of the document.  

Computational support for weak collaboration is well established.  For
example, word processors or editors allow different authors to create
subsections of a document and combine them together.  E-mail can deliver
different sections of a document to an editor who compiles them.  In
contrast, the computational needs of strong collaboration are not well
established.  For example, very few editors allow more than one author to
edit the same document at the same time.  In order to understand strong
collaboration, I implemented a collaborative hypertext authoring
environment called the Annotated Egret\footnote{Exploratory Group Research
EnvironmenT} Navigator (AEN).  This environment provides tools designed to
encourage strong collaboration and measurement facilities designed to
detect strong collaboration, if it were to occur.

AEN is built upon Egret \cite{csdl-93-09}, a client-server hypertext
database manager.  Authors/learners use AEN's typed nodes and links to give
the hypertext document structure.  They also use an access control model to
dynamically restrict the type of access to the individual nodes in the
database.  Several autonomous agents help AEN manage all of the information
needed to support strong collaboration.

\section{Research Thesis}

The basic premise of this research is that strong collaboration can be
enabled and encouraged through computer mediation.  The process of strong
collaboration can be supported by tools that assist the group in their
collaborative process.

The thesis of this research is that AEN provides a viable computer mediated
environment for strongly collaborative hypertext document creation.  AEN's
main features were designed with collaboration in mind.

To evaluate this thesis, I designed an operationalized definition that
can be used to empirically assess the degree to which AEN supports strong
collaboration.  I developed questionnaires to evaluate the user's feelings
about AEN's support for collaboration and AEN itself.

\section{Research Design and Results}

The ability of AEN to support strong collaboration was evaluated through a
case study during the Spring of 1995.  The study looked for strong
collaboration, recorded the tools used, and asked the users for their
opinions of the tools provided by AEN.

\small
\begin{table}[htb]
  \caption{Summary of Operationalized Definitions and Metrics.}
  \begin{center}
    \begin{tabular}{|l|l|}
      \hline
      {\rule[-3mm]{0mm}{8mm}{\bf Operationalized
      Definition }}& {\rule[-3mm]{0mm}{8mm}{\bf Corresponding Metric(s)}}\\ \hline
      \hline
      Members read each other's nodes&Readers per node (RPN)\\ \hline
%      Members edit nodes that were also edited by others&Member Co-editing
%      (MCE)\\ \hline
      Document nodes are edited by more&Editors per node
      (EPN)\\
      than one person&\\\hline
      Members create feedback nodes&Feedback Node Creation (FNC)\\ \hline
      Members manipulate access control &Non-default
      Access Control (NAC) \&\\
      to publish/protect documents &Evolving Access Control (EAC)\\ 
      under development&\\\hline
    \end{tabular}
    \label{tab:op-metrics}
  \end{center}
\end{table}
\normalsize

In order to detect strong collaboration, I developed a metric for each
component of the operationalized definition of strong collaboration. Table
\ref{tab:op-metrics} summarizes the operationalized definitions and their
corresponding metrics.

I used these collaborative metrics to detect occurrences of 
collaborative behaviors.  In order to evaluate AEN's set of tools, tool use
was monitored and a post study survey asked the participants their
impressions of the tools.

The data analysis reveals that a high score for each metric was recorded by
at least one of the groups studied.  However, no single group scored high
on all five metrics.  These results indicate that AEN does support the
various components of strong collaboration, but the results do not
demonstrate that AEN allows a single group to collaborate strongly with
respect to all metrics simultaneously.  Chapter \ref{sec:analysis} explores
the results of the case study in detail.

\section{Contributions of this research}

This research addresses an important area of research --- the use of
computer mediated tools as a means to facilitate collaborative authoring of
hypertext documents.  It raises a number of significant questions.  For
example, what are characteristics of strong collaboration?  What types of
computational augmentation are required to encourage strong collaboration?
Though this research may not lead to definite answers to these and many
other questions, it does represent a first step toward understanding of
these important issues.

Conceptually, AEN provides an operationalized definition of strong
collaboration.  This definition and the associated metrics allow the
detection and measurement of collaborative behaviors.

Technically, AEN is both a research and authoring tool.  The former is
evidenced by the built-in instrumentation that allows fine-grained process
data to be collected.  The latter is evidenced by the set of features
designed to support collaborative authoring.

Empirically, the data from the Fall, 1994 and Spring, 1995 use of AEN
confirmed that AEN is a viable approach to supporting collaborative
authoring of hypertext documents.  The results also suggest a number of
interesting directions in which AEN can be extended and empirical
investigations which can be conducted.

\section{Organization of this document}

The remainder of this thesis is organized as follows. Chapter
\ref{sec:related-work} examines other collaborative learning and authoring
systems.  Chapter \ref{sec:AEN} introduces AEN.  Chapter
\ref{sec:evaluation} is a discussion of how I evaluated AEN's design and
its support for strong collaboration.  Chapter \ref{sec:analysis} discusses
the results of my evaluation.  Chapter \ref{sec:conclusions} presents some
conclusions about strong collaboration.  Chapter \ref{sec:future} discusses
future directions for AEN.




