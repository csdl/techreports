%%%%%%%%%%%%%%%%%%%%%%%%%%%%%% -*- Mode: Latex -*- %%%%%%%%%%%%%%%%%%%%%%%%%%%%
%% thesis-future.tex -- 
%% Author          : Carleton Moore
%% Created On      : Tue Jan 10 12:03:50 1995
%% Last Modified By: Carleton Moore
%% Last Modified On: Thu Jul  6 18:08:39 1995
%% Status          : Unknown
%% RCS: $Id: thesis-future.tex,v 1.7 1995/07/07 04:08:48 cmoore Exp $
%%%%%%%%%%%%%%%%%%%%%%%%%%%%%%%%%%%%%%%%%%%%%%%%%%%%%%%%%%%%%%%%%%%%%%%%%%%%%%%
%%   Copyright (C) 1995 University of Hawaii
%%%%%%%%%%%%%%%%%%%%%%%%%%%%%%%%%%%%%%%%%%%%%%%%%%%%%%%%%%%%%%%%%%%%%%%%%%%%%%%
%% 

%for review purposes
%\ls{1}

\chapter{Future Directions}
\label{sec:future}

This chapter presents, briefly, some future directions for AEN.  Section
\ref{sec:exp} discusses a possible experiment to explore the affects of
AEN's features on collaboration.  Section \ref{sec:research} presents some
general directions for further investigation into strong collaboration.
Section \ref{sec:development} concludes this chapter with a discussion
about some future developments for AEN.  AEN will be released to the public
at the end of August, 1995.



\section{Future experiment on AEN}
\label{sec:exp}

This research was intended to evaluate AEN and provide a basis for future
experiments on strong collaboration.  A logical next step would be to
conduct an empirical experiment on the effects of each feature in AEN on
the five collaboration metrics.

%\subsection{Method}

The basic method for the experiment will be to have several groups create
two documents.  For the first document, half of the groups use an unaltered
version of AEN.  The other half use AEN with the feature being investigated
disabled.  In the second half of experiment, the two halves switch AEN
versions.  Thus, data is collected on each group for using AEN with the
feature and using AEN with the feature disabled.

%\subsection{Data Collection}

Three user surveys will be used, one pre-test and two post-test.  The
pre-test survey will collect information on the experience of the users.  A
post-test survey will be given after each experiment.  They will ask how
the users felt about their collaboration and the features AEN provided.
The surveys will be administered on line so I can correlate the user's
responses with the metrics AEN collects.

As a part of their documents, each group will describe their process for
collaboration. 

%\subsection{Data Analysis}

The five collaborative metrics will be calculated for each group.  The
values of the five metrics for the groups that used the complete version of
AEN will be compared to the metrics of the disabled version of AEN.  The
differences between the two metrics can be attributed to the disabled
feature.

The survey results will be correlated with the usage metrics for the
features to give us an idea of how the feature affected the users' feelings
toward AEN and collaboration.

\section{Future Research on Strong Collaboration}
\label{sec:research}

Beyond the above specific experiment, some general directions for research
into strong collaboration are: determining the costs and benefits of
strong collaboration and developing a set of rules or heuristics for
determining when strong collaboration should be used.  

The first future direction could be to investigate the costs and benefits
of strong collaboration vs. weak collaboration.  Quantified costs and
benefits to using strong collaboration should be determined to help
determine the suitability of using strong collaboration.  

The second future direction could be to develop a set of rules or
heuristics to determine when strong collaboration would be more appropriate
than weak collaboration.  This would help group decide on which mode of
collaboration to use.  For example, the following two situations require
entirely different styles of collaboration.

In the first situation, five C++ programmers with similar experience are
required to develop a set of five classes.  The interfaces for the classes
have already been defined.  In this situation, weak collaboration is the
correct style of collaboration.  Each of the programmers can develop a
single class by themselves and combine the five implementations together to
form the set of classes.  This seems like the most effective way to handle
this situation.  Adding the overhead of keeping track of the other four
classes is a waste of the programmers time and effort.

In the second situation, a small software company is creating a
requirements document for a client.  In this situation, strong
collaboration seems to be the answer.  The team to create the requirements
document should include a manager, software developer, client
representative, an end user.  Each of these people will contribute their
perspective to the requirements document, improving its quality.  Through
strong collaboration, they should detect more errors and omissions before
the document is finished.

\section{Future Development}
\label{sec:development}

Currently, AEN does not support the concept of roles for the different
participants.  A future development for AEN is to support different roles.
This will allow more controlled group processes and will move some of the
burden of process management from the groups to AEN.

Many users complained about the lack of a spell checking mechanism and
other word processing tools.  Another extension to AEN could be a more
robust word processing environment with helpful tools like a spell checker,
thesaurus, and print manager.  Another user suggestion was to provide a
tool that allows the importing of WWW pages directly into the AEN hypertext
document.  This leads to the general extension for AEN of being WWW aware.


