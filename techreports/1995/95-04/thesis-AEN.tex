%%%%%%%%%%%%%%%%%%%%%%%%%%%%%% -*- Mode: Latex -*- %%%%%%%%%%%%%%%%%%%%%%%%%%%%
%% thesis-AEN.tex -- 
%% Author          : Carleton Moore
%% Created On      : Tue Jan 10 11:59:25 1995
%% Last Modified By: Carleton Moore
%% Last Modified On: Thu Jul  6 17:37:53 1995
%% Status          : Unknown
%% RCS: $Id: thesis-AEN.tex,v 1.15 1995/07/07 03:38:20 cmoore Exp $
%%%%%%%%%%%%%%%%%%%%%%%%%%%%%%%%%%%%%%%%%%%%%%%%%%%%%%%%%%%%%%%%%%%%%%%%%%%%%%%
%%   Copyright (C) 1995 University of Hawaii
%%%%%%%%%%%%%%%%%%%%%%%%%%%%%%%%%%%%%%%%%%%%%%%%%%%%%%%%%%%%%%%%%%%%%%%%%%%%%%%
%% 

%for review purposes
%\ls{1}

\newpage

\chapter{AEN: A Collaborative Hypertext Authoring Tool}
\label{sec:AEN}

AEN is a collaborative hypertext authoring tool designed to support and
explore the nature of strong collaboration.  This chapter provides an
overview of the design and implementation of AEN.  Section
\ref{sec:architectureAEN} describes the general architecture of AEN.
Section \ref{sec:designAEN} discusses the design of AEN and how its tools
help support strong collaboration.  The final section provides a brief
discussion of the development of AEN and its current status.

\section{Architecture of AEN}
\label{sec:architectureAEN}

AEN is a collaborative, instrumented learning/authoring tool implemented
with the Egret collaborative system development
environment\cite{csdl-93-09}.  As with all Egret domain-specific
applications, AEN is a client-server system running over TCP/IP, where
clients (implemented as customized versions of XEmacs) interact with a
central server called HBS (implemented in C++) to store and retrieve
information or communicate with other clients.  AEN makes substantial use
of Egret's facilities for agents, or autonomous client processes that
extend the system's information processing capabilities beyond those
provided by HBS or the clients.  Figure \ref{fig:aen-architecture}
illustrates AEN's basic client-server architecture.

\begin{figure}[htb]
  \centerline{\psfig{figure=client-server.eps,width=5in}} 
  {\ls{1.0}{\em AEN consists of a central server called HBS that provides basic
  storage and concurrency control mechanisms, augmented with several agent
  processes to support strong collaboration.}}
 \caption{AEN's architecture from an OS process perspective.}
\label{fig:aen-architecture}
\end{figure}

\section{The design of AEN}
\label{sec:designAEN}
\subsection{Structure of Data in AEN}

AEN provides facilities for storage, retrieval, and analysis of a hypertext
network of node and links. The hypertext network represents the text from
which the students learn and the product of collaboration among the
participants.  AEN provides a static, predefined set of node and link
types.

\subsubsection{Node Types}

At the user-level, AEN provides two type hierarchies of nodes: artifact
nodes and figure nodes.  Nodes of type {\em artifact} represent textual
information in the hypertext document.  Nodes of type {\em figure}
represent graphical information.

Artifact nodes contain ETML\footnote{A primary difference between the Egret
Text Markup Language (ETML) and HTML is that ETML supports {\tt egret::}
URL specifier for link and node objects within an HBS server.} code that
provides formating directives and inter-node links.  There are four
subclasses of artifacts:

\begin{itemize}

\item{\bf Document:} Nodes that contain the text of the document created by
the participants.

\item{\bf Comment:} Nodes containing the reactions of the participants to
the contents of other nodes.

\item{\bf Quicky Quiz:} Nodes that contain exercises for the participants
to complete.

\item{\bf Quicky Quiz Answer:} Nodes that contain the participant's
solutions to the quicky quiz nodes.

\end{itemize}

Quicky Quiz and Quicky Quiz Answer nodes are used in AEN to support
learning.  They can be ignored for collaborative authoring.

\begin{figure}[htb]
  \centerline{\psfig{figure=Data.eps,width=5in}}
  \caption{Data Relationships in AEN}
  \label{fig:Data}
\end{figure}

Unlike artifact nodes, figure nodes contain the graphics data used by Xview
to display the figures.  Users can create links from artifact nodes to
figure nodes or other artifact nodes by a simple menu-based command.  Such
an operation results in both the creation of a new link object within the
server and the creation of an ETML link anchor as part of the contents of
the artifact node.  However, since figure nodes contain graphical
information, compatible with Xview, rather than ETML code, users cannot
create links from figure nodes.  Figure \ref{fig:Data} shows the
relationships between the node types.

\subsubsection{Link Types}

AEN defines six types of links: Include, Xref,
see\_Quicky\_\-Quiz, see\_\-Quicky\_\-Quiz\_\-Answer, see\_\-Comment, and
see\_Figure.  AEN restricts the type of the source and destination nodes
for each type of link. Table 2.1 describes the different
links.  The different types of links allows the authors to create different
structures in the hypertext document. 

\begin{table}
  \begin{center}
    \caption{Link Types}
    \begin{tabular}{|l|l|l|l|}
      \hline
      Link Type&Source Node&Destination Node&Description\\ \hline \hline&&&\\ 
      Include&document&document&Backbone structure \\
      &&&of hypertext document\\ \hline
      Xref&document&document&Cross reference\\ \hline
      See\_Comment&Any&comment&Points to a Comment\\ \hline
      See\_Quicky\_Quiz&Any&Quicky Quiz&Points to a Quicky Quiz\\ \hline
      See\_Quicky\_Quiz\_&Quicky Quiz&Quicky Quiz&Points to a\\
      Answer&&Answer&Quicky Quiz Answer\\\hline
      See\_Figure&Any&Figure&Points to a Figure\\ \hline
    \end{tabular}
    \label{tab:types}
  \end{center}
\end{table}


\subsubsection{The AEN Backbone}
\label{sec:aen-backbone}

Given the set of node and link types in AEN, there are many possible ways
to organize the resulting hypertext network.  However, given the intended
usage of AEN, it is natural and appropriate to provide a fundamental
organization of the hypertext as an ordered sequence of interlinked,
hypertext ``chapters'', each hypertext chapter having as its root a single
document node.

Each document node with a link to it from a chapter node represents a
``section'' within that chapter. In the same manner, each document node
with a link to it from a section node represents a subsection within that
section, and so forth ad infinitum.

Any document node can, of course, have a link to it from more than one
other document node in the hypertext, with the result that the same node
can be a ``section'' with respect to one node, but a ``sub-subsection''
with respect to another.  As a result, the AEN hypertext is organized as a
directed, hierarchical, but potentially cyclic graph structure.

So far, this description seems to indicate that there is a one-to-one
correspondence between each document node and each chapter, section,
subsection, and so forth in the document.  The reality is slightly more
complex.  Each document node does contribute at least one new chapter, or
section, or subsection, etc. to the hypertext.  However, if a document
node contains the HTML {\em heading} formatting directive, then the label for
this heading indicates a new section (or subsection, etc. depending upon
context) within the node.

To bootstrap this organization, some distinguished ``root'' document node
is required, which I call the AEN Backbone.  This node contains a set of
links to document nodes which define the set of top-level chapters in AEN.

The AEN Backbone and the nested structure of document nodes defines the
major organizing principle for the AEN hypertext, and leads to the first
data manipulation tool: the table of contents screen, as shown in Figure
\ref{fig:Screen1}.  The next section discusses the table of contents screen
and the other data manipulation tools. 



\begin{figure}[htb]
 \centerline{\psfig{figure=screen1.ps,width=6in}}
 {\ls{1.0}{\em This screen shot shows a portion of this 
 user's table of contents in the upper left screen, and a retrieved
 document node called ``Design'' in the upper right screen.  The Design
 node contains within it both a link to a figure called ``Figure 1'' and a
 comment called ``Regions'' immediately after it (represented in the
 Design node by the textual anchor ``[??]''.  Both the figure and the
 comment are retrieved and displayed in the screens in the lower left and
 lower right hand corners, respectively.}}
 \caption{Screen Shot of AEN's Main display, a comment node, a figure node,
 and TOC.}
 \label{fig:Screen1}
\end{figure}


\subsection{Mechanisms for Data Manipulation in AEN}
\label{sec:mechanisms}

For browsing, AEN emulates many of the facilities of WWW browsers such as
Mosaic or Netscape.  Each node is displayed in a separate screen, and
multiple nodes can be displayed simultaneously.  The contents of nodes are
automatically reflowed to conform to the size of the screen.
Middle-clicking over a link anchor retrieves the corresponding node from
the server.  As noted above, AEN supports storage and retrieval of both
text and graphics, and although audio is not currently available in AEN, it
has been implemented in other Egret-based applications, and could be easily
added.  Due to a limitation in the current release of XEmacs, in-line
graphics are not currently available in AEN.

Where AEN begins to depart from WWW systems is in its support for
authoring and collaboration.  First, AEN provides an authoring mode
for nodes, including menu and command key-based formatting directives
for headings, lists, and fontification, link anchor creation, and so
forth.  Second, all artifact nodes must be locked before they can be
modified.  Locking preserves the integrity of information when
multiple authors are working on a document, but provides only basic
support for collaboration.  AEN augments locking with an access
control mechanism, discussed next.

\subsubsection{Access Control}

AEN implements three forms of access to individual nodes: read access,
write access, and annotation access.  The administrator(s) of a node can set each
form of access independently for each user of AEN.

When no access is provided to a node, then that node is essentially
private and can be retrieved and edited only by the administrator(s) of the
node. (The creator of a node is given the status of ``administrator'', which
has the effect of irrevocably providing all forms of access. Only the
administrator[s] of a node can delete the node.)

When a user has read access for a node, they can retrieve it but they
cannot add a link to it or edit its contents. Providing only read access to
a node to others indicates that the information is available for use, but
not for review or feedback.

When a user has both read and annotation access for a node, they can both
retrieve it and add a link from the node to a newly created comment
node. Providing both read and annotation access indicates that the
information is available for use and that comments from others about the
content of the node are welcomed.  Annotation access allows the administrator of
the node to solicit suggestions for changes to the contents of the node,
while still retaining control over whether those changes are actually
applied to the document.

When a user has both read and write access for a node, they can both
retrieve it and lock it for editing.  Providing both read and write access
signifies the desire for the closest form of collaborative development, in
which those with write access can make arbitrary changes to the contents of
a node.

Of course, the administrator of a node may grant all three access rights to another
user, which allows them to select which form of feedback to provide.  It is
also interesting to note that without read access, neither annotation nor
write access is useful, since one cannot annotate or lock a node without
first retrieving it.  This is actually quite useful in practice, since it
allows the administrator of a node to temporarily ``hide'' a node simply by
disabling read access to the group.  By re-enabling read access, the
previous annotation, and write access are all reinstated.


\begin{figure}%[htb]
 \centerline{\psfig{figure=screen2.ps,width=6in}}
 {\em This screen shot illustrates some of the real-time support
for strong collaboration in AEN.  The Snoopy window in the lower right hand
corner shows, for each user of the system, whether or not they are logged
in currently and if so, where they are working in the document.  Both user
cmoore and user jeremyt are working on the document node called ``Shared
Emacs''.  The Partyline screen, in the lower right hand corner, allows these
users (as well as all others) to talk to each other while they work.  This
figure also shows a local TOC for the Shared Emacs chapter, and a screen
displaying the current set of unread document nodes for this user.}
 \caption{Screen Shot of AEN's Main display, Snoopy, Partyline,  Node List
 and TOC.}
 \label{fig:Screen2}
\end{figure}

\subsubsection{Context-Sensitive Table of Contents}

The access control mechanism determines which nodes a user may read and
what they may do to the node once they have read it, but it does not help
the user determine which node they want to read.

The primary navigation mechanism  provided by AEN is {\em tables of
contents}.  Each table of contents (TOC) is a dynamically generated, linear
traversal of document nodes in the database, organized according to the AEN
Backbone structure described in Section \ref{sec:aen-backbone}.  Figures
\ref{fig:Screen1} and \ref{fig:Screen2} each contain an example of a table
of contents.  Each line in a TOC is mouse-sensitive, and middle-clicking on
it retrieves the corresponding node (or section within a node) from the
server.  The indentation of headings provides a visual cue to the
hierarchical structure of the hypertext.  Once the TOC is displayed the
user can collapse or expand the listing to reduce or enlarge the number of
listings present.


AEN generates a TOC based upon a starting node that the user chooses.  The
TOC algorithm only analyzes document nodes for which the user has read
access, thus, the TOC is both location and user-specific.  The algorithm
only visits the children of a node once, which breaks cycles in the
hypertext. Finally, each time a document node is written to the server,
structural information relevant to the TOC is extracted, cached, and
propagated to all clients if it has changed. Such a mechanism is crucial to
the usability of the system; one cannot retrieve every document node from
the database each time a TOC is generated.


\subsubsection{Node Lists}

The tables of contents provide the most important perspective on the
hypertext---that of the document-level structure, but other viewpoints are
also useful.  Users can make queries for nodes satisfying various criteria
with the node-list mechanism. In AEN, useful criteria so far include the
type of nodes, whether or not the nodes have been retrieved by this user
yet, and the nodes owned by the user.

Each node-list returned from a query consists of a mouse-sensitive
list of node names appearing in a single screen. Middle-clicking
on the node retrieves it in the standard fashion for AEN navigation.

The Hyperstar Bulletin \foot{The name of this mechanism is a pun on the
name of Honolulu's evening newspaper: the Star-Bulletin.} is an agent-based
mechanism intended to support collaboration via a daily ``snapshot'' of
what has changed in the database since each user last visited it.  The
Hyperstar Bulletin consists of an agent process that wakes up at
approximately 4 a.m. each morning and sends a daily electronic
``newspaper'' to each participant containing a listing of which nodes (for
which the user has read access) have been created or changed since the last
time the user logged in. 


\subsubsection{Snoopy and Partyline}

The mechanisms described above support the construction of a hypertext by
multiple authors, but falls short of support for strong collaboration in
one crucial way.  With these above mechanisms alone, the actual people
involved in the collaborative activity are ``invisible'' within the system:
their effects upon hypertext document are visible, but they themselves are
not.  To create a physical presence for participants within the system, AEN
provides two additional facilities: Snoopy and Partyline, as illustrated in
Figure \ref{fig:Screen2}.

The Snoopy mechanism allows the user to see who else is at work and have an
idea of what they are working on.  It displays basic information about the
connection status and last read node of the users of AEN. A Snoopy screen
is created automatically each time a user connects to AEN, and is updated
every 30 seconds.  Through Snoopy, each user knows who else is currently in
AEN and where they are in the hypertext.

Partyline is a mechanism for passing real time messages to other users of
AEN.  Partyline allows the user to send a textual message to all the users
currently connected to AEN or send a private message to one other user
currently connected to Partyline.  Partyline is similar to an Internet
``chat'' mechanism.  Typically, users greet the other people when they
log in to AEN with Partyline messages.

Both facilities provide access to the users of AEN.  In order to help
support strong collaboration, users of AEN need access to the other users.
Snoopy and Partyline provide this service.

\section{Development and Status}

\subsection{Brief History}

Research on AEN has been underway for almost one year.  The initial
requirements document for AEN \cite{csdl-94-06} was developed in June,
1994.  In this document, the stated design goal of AEN was to support
virtually all lecture material and participant interactions for a seminar
in collaborative systems to be held in Fall, 1994.  An alpha Version 1.0 of
AEN conforming to these requirements was released for classroom use in
September, 1994.  

Participants in the seminar used AEN to read and react to material posted
by students and the instructor.  Based upon their reactions, the material
was revised and improved, or new material was added. On-line quizzes, as
well as semester projects, assessed the learning process. Face-to-face
classroom meetings provided supplementary forums to discuss things learned
or questioned through on-line activities.  Through AEN, groups of students:
learned how to develop software in Emacs Lisp; used Egret to
implement a collaborative system; designed and presented a project proposal for
their collaborative system; and created a well-structured report on the results
of their effort.

During the Fall, 1995 seminar, the notion of AEN as a testbed for the
concept of strong collaboration emerged.  However, during this initial
usage, strong collaboration occurred only rarely due to system frailty and
incompleteness.  After redesign and rework, I used AEN in a case study
during Spring, 1995, to evaluate the system and better understand its
support for strong collaboration.


\subsection{Current Status}

Since its initial release, AEN has gone through over 30 minor version and
three major versions.  AEN consists of over 12,000 lines of lisp code. I
have authored or co-authored three documents about AEN, CSDL technical
report 94-06 \cite{csdl-94-06}, my Masters thesis proposal
\cite{csdl-94-16}, and a paper published in the Fourth
Workshop on Enabling Technologies: Infrastructure for Collaborative
Enterprises (WET-ICE '95) \cite{csdl-94-20}.

I am cleaning up the code and documentation and writing an on-line tutorial
for AEN, in preparation for AEN's public release in early Fall, 1995.



