%%%%%%%%%%%%%%%%%%%%%%%%%%%%%% -*- Mode: Latex -*- %%%%%%%%%%%%%%%%%%%%%%%%%%%%
%% thesis-title.tex -- 
%% Author          : Carleton Moore
%% Created On      : Tue Jan 10 11:58:20 1995
%% Last Modified By: Carleton Moore
%% Last Modified On: Wed Jun  7 09:59:46 1995
%% Status          : Unknown
%% RCS: $Id: thesis-title.tex,v 1.6 1995/06/07 20:00:34 cmoore Exp cmoore $
%%%%%%%%%%%%%%%%%%%%%%%%%%%%%%%%%%%%%%%%%%%%%%%%%%%%%%%%%%%%%%%%%%%%%%%%%%%%%%%
%%   Copyright (C) 1995 University of Hawaii
%%%%%%%%%%%%%%%%%%%%%%%%%%%%%%%%%%%%%%%%%%%%%%%%%%%%%%%%%%%%%%%%%%%%%%%%%%%%%%%
%% 


\title{Supporting strong collaboration with a hypertext system:\\ 
The Annotated Egret Navigator.}

\author {Carleton Moore\\
\\ Collaborative Software Development Laboratory\\ Department of
Information and Computer Sciences\\ 2565 The Mall\\ University of Hawaii\\
Honolulu, Hawaii 96822\\ (808) 956-6920\\ {\tt
cmoore@uhics.ics.Hawaii.edu}} \date{Techreport CSDL-TR-95-04\\ \today}

\maketitle

\begin{abstract}
  The Annotated Egret Navigator (AEN) is a system designed to support
  {\em strong collaboration} among a group as they cooperatively build,
  review, revise, improve and learn from a structured hypertext document.
  AEN was used as the central instructional and research system for a
  graduate seminar on collaborative systems at the University of Hawaii
  during Fall, 1994.  In Spring, 1995 we conducted an experimental case
  study to evaluate AEN's support of {\em strong collaboration}.  The
  study helped define strong collaboration and showed that AEN does
  support strong collaboration.

         %AEN was used for over 285
%         hours during the second half of the semester alone, and
%         users generated over 800 nodes and 800 links.  Lessons
%         learned about strong collaboration include: (1) Users as
%         well as artifacts should be visible; (2) Provide direct
%         and indirect authoring mechanisms; (3) Provide
%         context-sensitive change information; (4) Provide access
%         to intermediate work products; (5) Maintain database
%         integrity; (6) An agent-based architecture may be
%         necessary for systems supporting strong collaboration;
%         and (7) The WWW is not effective for strong
%         collaboration.  

\end{abstract}

