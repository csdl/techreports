%%%%%%%%%%%%%%%%%%%%%%%%%%%%%% -*- Mode: Latex -*- %%%%%%%%%%%%%%%%%%%%%%%%%%%%
%% slides.tex -- 
%% Author          : Carleton Moore
%% Created On      : Wed Apr  5 13:32:21 1995
%% Last Modified By: Carleton Moore
%% Last Modified On: Mon Jul  3 15:02:36 1995
%% Status          : Unknown
%% RCS: $Id$
%%%%%%%%%%%%%%%%%%%%%%%%%%%%%%%%%%%%%%%%%%%%%%%%%%%%%%%%%%%%%%%%%%%%%%%%%%%%%%%
%%   Copyright (C) 1995 University of Hawaii
%%%%%%%%%%%%%%%%%%%%%%%%%%%%%%%%%%%%%%%%%%%%%%%%%%%%%%%%%%%%%%%%%%%%%%%%%%%%%%%
%% 


\documentstyle[11pt,slidesonly]{/group/csdl/tex/seminar}
\input{/group/csdl/tex/psfig/psfig}

%%% Use: dvitps -L -O file.ps file.dvi
%%% or Use: dvips file.dvi -t landscape -o file.ps
\newcommand{\horizontalline} {\rule{\textwidth}{.02in}} 
\slideframe{none}
\slidesmag{0}        % integer value ranging from -5 to 9
\special{landscape}  %comment out this line for notes
\pagestyle{empty}


%% Define a new global counter to keep track of reference numbers.
\newcounter{tenurecite}

%% Define a new environment that resets the local counter within 
%% the enumerate environment to the global counter, and locally redefines 
%% the way the label is printed.

\newenvironment{tenurelist}{\begin{enumerate}%
                            \setcounter{enumi}{\thetenurecite}%
                            \def\labelenumi{\theenumi. }}
                           {\setcounter{tenurecite}{\theenumi}%
                            \end{enumerate}}


\begin{document}      

\begin{slide} \Huge 
 \begin{center} 
   {\bf Supporting Strong Collaboration with\\
   a Hypertext System:\\ 
   The Annotated Egret Navigator}\\

   \vspace{0.5in} 
   Thesis presented\\
   by\\ 
   Carleton Moore\\
   \vspace{0.5in} 
    
   Collaborative Software Development Laboratory \\ 
   Department of Information \& Computer Sciences\\ 
   University of Hawaii at Manoa \\
   \vspace{0.1in}
   \LARGE
   {\tt file://ftp.ics.hawaii.edu/pub/tr/ics-tr-95-04.ps.Z}
 \end{center} 
\end{slide} \Huge   

\begin{slide} \Huge 
  {\bf Synopsis} \horizontalline
  \begin{itemize}\huge
  \item{Definition of strong collaboration}
  \item{Overview of research contributions}
  \item{Main features of  AEN}
%  \item{Supporting Strong Collaboration.}
%    \begin{enumerate}
%    \item{AEN's features that reduce the cost of strong collaboration.}
%    \item{AEN's features that encourage strong collaboration.}
%    \end{enumerate}
  \item{Empirical evaluation of AEN}
%  \item{Related Work.}
  \item{Conclusions and future directions}
  \end{itemize}

\end{slide} \Huge   


\begin{slide}\Huge 
  {\bf Weak Collaboration:}
  \horizontalline\\
  \begin{itemize}\huge
  \item {\bf Characteristic:} Distinct authorship of components of the
    document.  
  \item {\bf Pro:} Divides the work.

  \item {\bf Con:}  Lack of perspectives.

  \item {\bf Example:} Set of C++ classes from an interface document.
  \end{itemize}

\end{slide}

\begin{slide}\Huge 
  {\bf Strong Collaboration:}
  \horizontalline
  \begin{itemize}\huge
  \item {\bf Characteristic:} A sense of collective authorship over most of the
    document.
  \item {\bf Pros:}  Many perspectives, higher quality, consistency.

  \item {\bf Cons:}  Information overload, effort.

  \item {\bf Example:} Requirements document.
  \end{itemize}

\end{slide}

%
%\begin{slide}\Huge 
%  {\bf Features that reduce the cost of Strong Collaboration}
%  \horizontalline\\
%  \huge
%  \begin{itemize}\huge
%  \item Agent Based daily E-mail reminder mechanism.
%  \item Context sensitive tables of contents.
%  \item Commonly needed queries.
%  \end{itemize}
%\end{slide}
%
%\begin{slide}\Huge 
%  {\bf Features to encourage Strong Collaboration}
%  \horizontalline\\
%  \huge
%  \begin{itemize}\huge
%  \item Access control solicits or inhibits Strong Collaboration.
%  \item Users and Data have physical presence in the database.
%  \item Real time communication is supported.
%  \end{itemize}
%\end{slide}


\begin{slide}\Huge 
  {\bf Research contributions: an overview}
  \horizontalline\\
  \begin{itemize}\huge
  \item The Annotated Egret Navigator (AEN)
    \begin{itemize}
    \item Collaborative Authoring Environment
    \item Support for Strong Collaboration
    \end{itemize}
  \item Operationalized Definition of Strong Collaboration
    \begin{itemize}
    \item Detects instances of collaborative behavior
    \item Metric to measure behavior
    \end{itemize}
  \item Empirical evaluation
    \begin{itemize}
    \item Strong Collaboration
    \item AEN's features
    \end{itemize}
  \end{itemize}
\end{slide}

\begin{slide}\Huge 
  {\bf Contribution \# 1:\\AEN's history}
  \horizontalline\\
  \begin{description}\huge
  \item[June, 1994:] Initial requirements document
  \item[September, 1994:] Alpha version 1.0 released for classroom use
  \item[Fall, 1994:] Development, redesign, reimplementation
  \item[Spring, 1995:] Empirical evaluation
  \end{description}
\end{slide}


\begin{slide}\Huge 
  {\bf Contribution \# 1:\\AEN}
  \horizontalline\\
  \begin{itemize}\huge
  \item A 12 KLOC object-oriented, extensible system written in Lisp 
  \item A successful collaborative authoring environment
  \item An environment for empirical experimentation: fine-grained,
  non-obtrusive instrumentation
  \end{itemize}
\end{slide}


\begin{slide}\Huge 
  {\bf Contribution \# 2:\\Operationalized definition of Strong Collaboration.}
  \horizontalline\\
  \begin{itemize}
  \item Members read each other's nodes.\\
   -- Metric: Readers Per Node (RPN).
  \item{Document nodes are edited by more than one person.}\\
    -- Metric: Editors Per Node (EPN).
  \end{itemize}
\end{slide}


\begin{slide}\Huge 
  {\bf Contribution \# 2:\\Operationalized definition of Strong
  Collaboration cont.}
  \horizontalline\\
  \begin{itemize}\huge
  \item{Members create feedback nodes.}\\
    -- Metric: Feedback Node Creation (FNC).
  \item{Members manipulate access control to publish/protect documents
  under development.}\\
    -- Metric: Non-default Access Control (NAC).\\
    -- Metric: Evolutionary Access Control (EAC).
  \end{itemize}
\end{slide}


\begin{slide}\Huge 
  {\bf AEN Use}
  \horizontalline\\
  \begin{itemize}\huge
  \item Fall 1994, 
    \begin{itemize}
    \item Design, Implementation, Evaluation
    \item 10 Users, 285 hours, 800 nodes, over 12 weeks.
    \item Concepts of support for strong collaboration were developed.
    \end{itemize}
  \item Spring 1995,
    \begin{itemize}
    \item Case Study
    \item 15 Users, 180 hours, 216 nodes, over 9 weeks.
    \item Instances of strong collaborative behaviors.
    \end{itemize}
  \end{itemize}
\end{slide}


\begin{slide}\Huge 
  {\bf Contribution \# 3:\\Evaluation of AEN.}
  \horizontalline\\
  \begin{itemize}\huge
  \item Overall reaction to AEN: 5.4
  \item AEN promotes collaboration: 7.2
  \end{itemize}
   \begin{center}
    Average Score on a scale from 1 to 9.
  \end{center}
\end{slide}

\begin{slide}\Huge 
  {\bf Contribution \# 3:\\Empirical evaluation: Operationalized definition}
  \horizontalline\\
  \ \\
  \centerline{\psfig{figure=metrics2.eps,width=5.5in}}
\end{slide}

\begin{slide}\Huge 
  {\bf Contribution \# 3:\\Empirical evaluation of AEN's major Features.}
  \horizontalline\\
    \ \\\huge
  \begin{tabular}{l|c|c}
    Feature&Used&Helpful\\ \hline
    %  \begin{itemize}
    %  \item Access Control:\\
    Context Sensitive Tables of Context&8.7&8.3 \\
    Node Lists&4.9&5.7\\
    Snoopy&6.3&6.8\\
    Partyline&5.2&6.2\\
  \end{tabular}
  \begin{center}
    Average Score on a scale from 1 to 9.
  \end{center}
\end{slide}

\begin{slide}\Huge 
  {\bf Conclusions about AEN}
  \horizontalline\\
  \begin{itemize}\huge
  \item AEN is a unique authoring environment.
  \item AEN is capable of supporting strong collaboration.
  \item AEN is a successful collaborative authoring environment.
  \item AEN's lack of a predefined process allows it to support many
  different processes. 
  \item AEN is a valuable tool in researching collaborative construction of
  hypertext documents.
  \end{itemize}
\end{slide}


\begin{slide}\Huge 
  {\bf Recommendations to encourage Strong Collaboration.}
  \horizontalline\\
  \begin{tenurelist}\huge
  \item Provide direct and indirect authoring mechanisms
  \item Provide access to intermediate work products
  \item Provide context-sensitive ``what's new''
  \item Provide mechanisms to allow users as well as documents to be visible
  \end{tenurelist}
\end{slide}



\begin{slide}\Huge 
  {\bf Future Directions:}
  \horizontalline\\
  \begin{itemize}\huge
  \item Empirical Experiment on effects of AEN's features on collaboration
  \item Determine costs and benefits of strong collaboration
  \item Develop set of rules for determining when to use strong collaboration
  \item Extend AEN to support roles
  \item Extend AEN to understand the WWW
  \end{itemize}
\end{slide}





 
%%%%%%%%%%%%%%%%%%%%%%%%%%%%%%%%%%%%%%%%%%%%%%%%%%%%%%%%%%%%%%%%%%%%%%%%%%%



%\begin{slide}\Huge 
%  {\bf Metrics collected in AEN.}
%  \horizontalline\\
%  \huge
%  \begin{itemize}
%  \item Node creation.
%  \item Reading the contents of a node.
%  \item Changing the contents of a node.
%  \item Locking of nodes.
%  \item Changing the access privileges for a node.
%  \end{itemize}
%\end{slide}
%\begin{slide}\Huge 
%  {\bf Metrics collected in AEN.}
%  \horizontalline\\
%  \huge
%  \begin{itemize}
%  \item Creating a Table of Contents.
%  \item Creating a Node list.
%  \item Creating an Unread Nodes list.
%  \item Starting and stopping Partyline and Snoopy.
%  \item Sending a Partyline message.
%  \end{itemize}
%\end{slide}


\end{document} 


