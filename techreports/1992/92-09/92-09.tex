\documentstyle[]{article}


%%%%%%%%%%%%%%%magic definemargins command follows%%%%%%%%%%%%%
\makeatletter

\newlength{\@marpageheight}
\newlength{\@marpagewidth}
\newlength{\@computedmargin}

%%
%% 92-09.tex -- Defines the physical page size.
%%

\newcommand{\pagesize}[2]{%
  \setlength{\@marpageheight}{#1}%
  \setlength{\@marpagewidth}{#2}}

%%
%% \definemargins{topmargin}{bottommargin}{leftmargin}{rightmargin}
%%               {headup}{footdown}
%%
%% Note: - parameters for marginal notes are not handled here;
%%       - all dimensions will be stretched about 2 percent.
%%

\newcommand{\definemargins}[6]{%
  \setlength{\textwidth}{\@marpagewidth}%  Determine \textwidth by
  \addtolength{\textwidth}{-#3}%           subtracting each side
  \addtolength{\textwidth}{-#4}%           margin.
  \if@twoside %                            Values for two-sided printing:
    \setlength{\@computedmargin}{\@marpagewidth}
    \addtolength{\@computedmargin}{-#3}
    \addtolength{\@computedmargin}{-\textwidth}
    \oddsidemargin #3 %                    Left margin on odd-numbered pages.
    \addtolength{\oddsidemargin}{-1.0in}
    \evensidemargin \@computedmargin%      Left margin on even-numbered pages.
    \addtolength{\evensidemargin}{-1.0in}
  \else %                                  Values for one-sided printing:
    \oddsidemargin #3 %                    \oddsidemargin = \evensidemargin.
    \addtolength{\oddsidemargin}{-1.0in}
    \evensidemargin #3
    \addtolength{\evensidemargin}{-1.0in}
  \fi
  \topmargin #1 %                          Nominal distance from top of page
                %                          to top of text.
  \addtolength{\topmargin}{-#5}
  \addtolength{\topmargin}{-1.0in}
  \setlength{\textheight}{\@marpageheight}
  \addtolength{\textheight}{-#1}
  \addtolength{\textheight}{-#2}%          Determine \textheight from #1 and #2.
  \headsep #5 %                            Space between running head and text.
  \footskip #6 %                           Distance from top of box containing 
               %                           foot to baseline of last line of text.
  \addtolength{\footskip}{\footheight}
}

\pagesize{11in}{8.5in}
\definemargins{0.75in}{0.75in}{0.75in}{0.75in}{0.3in}{0.3in}
\makeatother

\pagestyle{empty}
\begin{document}
\columnsep .2in
\twocolumn
\title{{\bf Architectural Concerns in EGRET}}
\author{Philip Johnson\\
             Department of Information and Computer Sciences\\
             University of Hawaii, Honolulu, HI 96822\\
             {\tt johnson@hawaii.edu}}

\maketitle
\sloppy


EGRET \cite{Johnson92a} is an environment for exploratory
collaboration: its basic purpose is to allow groups to dynamically
explore the structural nature of collaborative artifacts.  This
fundamental requirement has two immediate implications. First, EGRET
must provide support for groups as they collaborate together upon a
common task.  Second, EGRET must support a dynamic and extensible
structure definition facility, along with supporting analysis and
environment tools.  

The requirements for EGRET motivate two specific architectural
idiosyncracies--- ``heavy-weight'' clients and ``gtables.''  This
position page briefly overviews these architectural decisions.
Details about the EGRET architecture can be found in \cite{Johnson92b}.

From an extremely abstract viewpoint, EGRET's architecture is a
classic client-server database.  There is a central database process
that stores the contents of the database, and client processes connect
to it in order to obtain information from it or add information to it.
Clients communicate with each other via changes to the database; no
direct inter-client communication is currently supported.

In classic client-server database architectures, most information
processing capabilities are contained in the server, while the
capabilities of the clients are rather simple.  For example, the
server process will implement a highly structured database and a
sophisticated query language, relieving the client of all information
processing responsibilities except those needed to generate queries.

In contrast, the trend in EGRET has been toward increasingly
simplified information processing within the database server process
and increasingly powerful client-side processing.  We call this the
{\em heavy-weight client} orientation.  For example, we now have a
mixture of human and non-human (autonomous) clients connecting to 
our centralized server database process. The autonomous clients 
give the appearance of a more powerful, expressive, and responsive
database server to the human clients.   

Second, a classic dynamic tension in hypertext systems exists between
the requirements of hypertext applications for dynamic and relatively
unpredictable structuring of information, and the performance
requirements for fast access, retrieval, and querying.  Relational
database models typically fulfill performance requirements, but at the
cost of statically defined and rigidly constrained structuring of
information.  Many hypertext systems, such as resolve this tension by
implementing a hypertext model on top of an underlying relational
database system.  Such systems must trade-off generality in their
hypertext model in order to accomodate the constraints imposed by the
underlying relational database mechanism.

In contrast, in EGRET the underlying platform is a pure hypertext
database system, supporting dynamic creation of arbitrarily-sized
nodes with unconstrained relationships between them via links.  For
performance and query processing, EGRET provides a mechanism called
{\em gtables} to build application-specific, locally cached, high
performance tables.  Thus, EGRET allows a developer to build in just
those relational capabilities required for the particular application
area, without suffering the restrictions imposed by the entire
relational model. This architectural decision has its own set of
trade-offs that we are just beginning to elucidate.

\begin{thebibliography}{9}

\bibitem{Johnson92a}
Philip Johnson. {\em Supporting Exploratory CSCW with the EGRET
Framework.} In Proceedings of the 1992 Conference on Computer
Supported Cooperative Work, Toronto, Canada. November, 1992.

\bibitem{Johnson92b}
Philip Johnson and Dadong Wan and Danu Tjahjono. {\em 
EGRET Version 2 Design Specification.} Collaborative
Software Development Laboratory Technical Report 91-02, 
University of Hawaii Department of Information and Computer Sciences,
Honolulu, HI. Last revision: July, 1992.

\end{thebibliography}

{\it Support for this research was provided in part by the
National Science Foundation Research Initiation Award CCR-9110861 and
by the University of Hawaii Research Council Seed Money Award
R-93-868-F-728-B-430}

\end{document}