%%%%%%%%%%%%%%%%%%%%%%%%%%%%%% -*- Mode: Latex -*- %%%%%%%%%%%%%%%%%%%%%%%%%%%%
%% csrs.tex -- 
%% RCS:            : $Id$
%% Author          : Danu Tjahjono
%% Created On      : Tue May  4 09:11:06 1993
%% Last Modified By: Danu Tjahjono
%% Last Modified On: Wed May 19 19:45:31 1993
%% Status          : Unknown
%%%%%%%%%%%%%%%%%%%%%%%%%%%%%%%%%%%%%%%%%%%%%%%%%%%%%%%%%%%%%%%%%%%%%%%%%%%%%%%
%%   Copyright (C) 1993 University of Hawaii
%%%%%%%%%%%%%%%%%%%%%%%%%%%%%%%%%%%%%%%%%%%%%%%%%%%%%%%%%%%%%%%%%%%%%%%%%%%%%%%
%% 
%% History
%% 4-May-1993		Danu Tjahjono	


\documentstyle [11pt,
/group/csdl/tex/named-citations,
/group/csdl/tex/definemargins,
/group/csdl/tex/lmacros]{article}
% Psfig/TeX 
\def\PsfigVersion{1.9}
% dvips version
%
% All psfig/tex software, documentation, and related files
% in this distribution of psfig/tex are 
% Copyright 1987, 1988, 1991 Trevor J. Darrell
%
% Permission is granted for use and non-profit distribution of psfig/tex 
% providing that this notice is clearly maintained. The right to
% distribute any portion of psfig/tex for profit or as part of any commercial
% product is specifically reserved for the author(s) of that portion.
%
% *** Feel free to make local modifications of psfig as you wish,
% *** but DO NOT post any changed or modified versions of ``psfig''
% *** directly to the net. Send them to me and I'll try to incorporate
% *** them into future versions. If you want to take the psfig code 
% *** and make a new program (subject to the copyright above), distribute it, 
% *** (and maintain it) that's fine, just don't call it psfig.
%
% Bugs and improvements to trevor@media.mit.edu.
%
% Thanks to Greg Hager (GDH) and Ned Batchelder for their contributions
% to the original version of this project.
%
% Modified by J. Daniel Smith on 9 October 1990 to accept the
% %%BoundingBox: comment with or without a space after the colon.  Stole
% file reading code from Tom Rokicki's EPSF.TEX file (see below).
%
% More modifications by J. Daniel Smith on 29 March 1991 to allow the
% the included PostScript figure to be rotated.  The amount of
% rotation is specified by the "angle=" parameter of the \psfig command.
%
% Modified by Robert Russell on June 25, 1991 to allow users to specify
% .ps filenames which don't yet exist, provided they explicitly provide
% boundingbox information via the \psfig command. Note: This will only work
% if the "file=" parameter follows all four "bb???=" parameters in the
% command. This is due to the order in which psfig interprets these params.
%
%  3 Jul 1991	JDS	check if file already read in once
%  4 Sep 1991	JDS	fixed incorrect computation of rotated
%			bounding box
% 25 Sep 1991	GVR	expanded synopsis of \psfig
% 14 Oct 1991	JDS	\fbox code from LaTeX so \psdraft works with TeX
%			changed \typeout to \ps@typeout
% 17 Oct 1991	JDS	added \psscalefirst and \psrotatefirst
%

% From: gvr@cs.brown.edu (George V. Reilly)
%
% \psdraft	draws an outline box, but doesn't include the figure
%		in the DVI file.  Useful for previewing.
%
% \psfull	includes the figure in the DVI file (default).
%
% \psscalefirst width= or height= specifies the size of the figure
% 		before rotation.
% \psrotatefirst (default) width= or height= specifies the size of the
% 		 figure after rotation.  Asymetric figures will
% 		 appear to shrink.
%
% \psfigurepath#1	sets the path to search for the figure
%
% \psfig
% usage: \psfig{file=, figure=, height=, width=,
%			bbllx=, bblly=, bburx=, bbury=,
%			rheight=, rwidth=, clip=, angle=, silent=}
%
%	"file" is the filename.  If no path name is specified and the
%		file is not found in the current directory,
%		it will be looked for in directory \psfigurepath.
%	"figure" is a synonym for "file".
%	By default, the width and height of the figure are taken from
%		the BoundingBox of the figure.
%	If "width" is specified, the figure is scaled so that it has
%		the specified width.  Its height changes proportionately.
%	If "height" is specified, the figure is scaled so that it has
%		the specified height.  Its width changes proportionately.
%	If both "width" and "height" are specified, the figure is scaled
%		anamorphically.
%	"bbllx", "bblly", "bburx", and "bbury" control the PostScript
%		BoundingBox.  If these four values are specified
%               *before* the "file" option, the PSFIG will not try to
%               open the PostScript file.
%	"rheight" and "rwidth" are the reserved height and width
%		of the figure, i.e., how big TeX actually thinks
%		the figure is.  They default to "width" and "height".
%	The "clip" option ensures that no portion of the figure will
%		appear outside its BoundingBox.  "clip=" is a switch and
%		takes no value, but the `=' must be present.
%	The "angle" option specifies the angle of rotation (degrees, ccw).
%	The "silent" option makes \psfig work silently.
%

% check to see if macros already loaded in (maybe some other file says
% "\input psfig") ...
\ifx\undefined\psfig\else\endinput\fi

%
% from a suggestion by eijkhout@csrd.uiuc.edu to allow
% loading as a style file. Changed to avoid problems
% with amstex per suggestion by jbence@math.ucla.edu

\let\LaTeXAtSign=\@
\let\@=\relax
\edef\psfigRestoreAt{\catcode`\@=\number\catcode`@\relax}
%\edef\psfigRestoreAt{\catcode`@=\number\catcode`@\relax}
\catcode`\@=11\relax
\newwrite\@unused
\def\ps@typeout#1{{\let\protect\string\immediate\write\@unused{#1}}}
\ps@typeout{psfig/tex \PsfigVersion}

%% Here's how you define your figure path.  Should be set up with null
%% default and a user useable definition.

\def\figurepath{./}
\def\psfigurepath#1{\edef\figurepath{#1}}

%
% @psdo control structure -- similar to Latex @for.
% I redefined these with different names so that psfig can
% be used with TeX as well as LaTeX, and so that it will not 
% be vunerable to future changes in LaTeX's internal
% control structure,
%
\def\@nnil{\@nil}
\def\@empty{}
\def\@psdonoop#1\@@#2#3{}
\def\@psdo#1:=#2\do#3{\edef\@psdotmp{#2}\ifx\@psdotmp\@empty \else
    \expandafter\@psdoloop#2,\@nil,\@nil\@@#1{#3}\fi}
\def\@psdoloop#1,#2,#3\@@#4#5{\def#4{#1}\ifx #4\@nnil \else
       #5\def#4{#2}\ifx #4\@nnil \else#5\@ipsdoloop #3\@@#4{#5}\fi\fi}
\def\@ipsdoloop#1,#2\@@#3#4{\def#3{#1}\ifx #3\@nnil 
       \let\@nextwhile=\@psdonoop \else
      #4\relax\let\@nextwhile=\@ipsdoloop\fi\@nextwhile#2\@@#3{#4}}
\def\@tpsdo#1:=#2\do#3{\xdef\@psdotmp{#2}\ifx\@psdotmp\@empty \else
    \@tpsdoloop#2\@nil\@nil\@@#1{#3}\fi}
\def\@tpsdoloop#1#2\@@#3#4{\def#3{#1}\ifx #3\@nnil 
       \let\@nextwhile=\@psdonoop \else
      #4\relax\let\@nextwhile=\@tpsdoloop\fi\@nextwhile#2\@@#3{#4}}
% 
% \fbox is defined in latex.tex; so if \fbox is undefined, assume that
% we are not in LaTeX.
% Perhaps this could be done better???
\ifx\undefined\fbox
% \fbox code from modified slightly from LaTeX
\newdimen\fboxrule
\newdimen\fboxsep
\newdimen\ps@tempdima
\newbox\ps@tempboxa
\fboxsep = 3pt
\fboxrule = .4pt
\long\def\fbox#1{\leavevmode\setbox\ps@tempboxa\hbox{#1}\ps@tempdima\fboxrule
    \advance\ps@tempdima \fboxsep \advance\ps@tempdima \dp\ps@tempboxa
   \hbox{\lower \ps@tempdima\hbox
  {\vbox{\hrule height \fboxrule
          \hbox{\vrule width \fboxrule \hskip\fboxsep
          \vbox{\vskip\fboxsep \box\ps@tempboxa\vskip\fboxsep}\hskip 
                 \fboxsep\vrule width \fboxrule}
                 \hrule height \fboxrule}}}}
\fi
%
%%%%%%%%%%%%%%%%%%%%%%%%%%%%%%%%%%%%%%%%%%%%%%%%%%%%%%%%%%%%%%%%%%%
% file reading stuff from epsf.tex
%   EPSF.TEX macro file:
%   Written by Tomas Rokicki of Radical Eye Software, 29 Mar 1989.
%   Revised by Don Knuth, 3 Jan 1990.
%   Revised by Tomas Rokicki to accept bounding boxes with no
%      space after the colon, 18 Jul 1990.
%   Portions modified/removed for use in PSFIG package by
%      J. Daniel Smith, 9 October 1990.
%
\newread\ps@stream
\newif\ifnot@eof       % continue looking for the bounding box?
\newif\if@noisy        % report what you're making?
\newif\if@atend        % %%BoundingBox: has (at end) specification
\newif\if@psfile       % does this look like a PostScript file?
%
% PostScript files should start with `%!'
%
{\catcode`\%=12\global\gdef\epsf@start{%!}}
\def\epsf@PS{PS}
%
\def\epsf@getbb#1{%
%
%   The first thing we need to do is to open the
%   PostScript file, if possible.
%
\openin\ps@stream=#1
\ifeof\ps@stream\ps@typeout{Error, File #1 not found}\else
%
%   Okay, we got it. Now we'll scan lines until we find one that doesn't
%   start with %. We're looking for the bounding box comment.
%
   {\not@eoftrue \chardef\other=12
    \def\do##1{\catcode`##1=\other}\dospecials \catcode`\ =10
    \loop
       \if@psfile
	  \read\ps@stream to \epsf@fileline
       \else{
	  \obeyspaces
          \read\ps@stream to \epsf@tmp\global\let\epsf@fileline\epsf@tmp}
       \fi
       \ifeof\ps@stream\not@eoffalse\else
%
%   Check the first line for `%!'.  Issue a warning message if its not
%   there, since the file might not be a PostScript file.
%
       \if@psfile\else
       \expandafter\epsf@test\epsf@fileline:. \\%
       \fi
%
%   We check to see if the first character is a % sign;
%   if so, we look further and stop only if the line begins with
%   `%%BoundingBox:' and the `(atend)' specification was not found.
%   That is, the only way to stop is when the end of file is reached,
%   or a `%%BoundingBox: llx lly urx ury' line is found.
%
          \expandafter\epsf@aux\epsf@fileline:. \\%
       \fi
   \ifnot@eof\repeat
   }\closein\ps@stream\fi}%
%
% This tests if the file we are reading looks like a PostScript file.
%
\long\def\epsf@test#1#2#3:#4\\{\def\epsf@testit{#1#2}
			\ifx\epsf@testit\epsf@start\else
\ps@typeout{Warning! File does not start with `\epsf@start'.  It may not be a PostScript file.}
			\fi
			\@psfiletrue} % don't test after 1st line
%
%   We still need to define the tricky \epsf@aux macro. This requires
%   a couple of magic constants for comparison purposes.
%
{\catcode`\%=12\global\let\epsf@percent=%\global\def\epsf@bblit{%BoundingBox}}
%
%
%   So we're ready to check for `%BoundingBox:' and to grab the
%   values if they are found.  We continue searching if `(at end)'
%   was found after the `%BoundingBox:'.
%
\long\def\epsf@aux#1#2:#3\\{\ifx#1\epsf@percent
   \def\epsf@testit{#2}\ifx\epsf@testit\epsf@bblit
	\@atendfalse
        \epsf@atend #3 . \\%
	\if@atend	
	   \if@verbose{
		\ps@typeout{psfig: found `(atend)'; continuing search}
	   }\fi
        \else
        \epsf@grab #3 . . . \\%
        \not@eoffalse
        \global\no@bbfalse
        \fi
   \fi\fi}%
%
%   Here we grab the values and stuff them in the appropriate definitions.
%
\def\epsf@grab #1 #2 #3 #4 #5\\{%
   \global\def\epsf@llx{#1}\ifx\epsf@llx\empty
      \epsf@grab #2 #3 #4 #5 .\\\else
   \global\def\epsf@lly{#2}%
   \global\def\epsf@urx{#3}\global\def\epsf@ury{#4}\fi}%
%
% Determine if the stuff following the %%BoundingBox is `(atend)'
% J. Daniel Smith.  Copied from \epsf@grab above.
%
\def\epsf@atendlit{(atend)} 
\def\epsf@atend #1 #2 #3\\{%
   \def\epsf@tmp{#1}\ifx\epsf@tmp\empty
      \epsf@atend #2 #3 .\\\else
   \ifx\epsf@tmp\epsf@atendlit\@atendtrue\fi\fi}


% End of file reading stuff from epsf.tex
%%%%%%%%%%%%%%%%%%%%%%%%%%%%%%%%%%%%%%%%%%%%%%%%%%%%%%%%%%%%%%%%%%%

%%%%%%%%%%%%%%%%%%%%%%%%%%%%%%%%%%%%%%%%%%%%%%%%%%%%%%%%%%%%%%%%%%%
% trigonometry stuff from "trig.tex"
\chardef\psletter = 11 % won't conflict with \begin{letter} now...
\chardef\other = 12

\newif \ifdebug %%% turn me on to see TeX hard at work ...
\newif\ifc@mpute %%% don't need to compute some values
\c@mputetrue % but assume that we do

\let\then = \relax
\def\r@dian{pt }
\let\r@dians = \r@dian
\let\dimensionless@nit = \r@dian
\let\dimensionless@nits = \dimensionless@nit
\def\internal@nit{sp }
\let\internal@nits = \internal@nit
\newif\ifstillc@nverging
\def \Mess@ge #1{\ifdebug \then \message {#1} \fi}

{ %%% Things that need abnormal catcodes %%%
	\catcode `\@ = \psletter
	\gdef \nodimen {\expandafter \n@dimen \the \dimen}
	\gdef \term #1 #2 #3%
	       {\edef \t@ {\the #1}%%% freeze parameter 1 (count, by value)
		\edef \t@@ {\expandafter \n@dimen \the #2\r@dian}%
				   %%% freeze parameter 2 (dimen, by value)
		\t@rm {\t@} {\t@@} {#3}%
	       }
	\gdef \t@rm #1 #2 #3%
	       {{%
		\count 0 = 0
		\dimen 0 = 1 \dimensionless@nit
		\dimen 2 = #2\relax
		\Mess@ge {Calculating term #1 of \nodimen 2}%
		\loop
		\ifnum	\count 0 < #1
		\then	\advance \count 0 by 1
			\Mess@ge {Iteration \the \count 0 \space}%
			\Multiply \dimen 0 by {\dimen 2}%
			\Mess@ge {After multiplication, term = \nodimen 0}%
			\Divide \dimen 0 by {\count 0}%
			\Mess@ge {After division, term = \nodimen 0}%
		\repeat
		\Mess@ge {Final value for term #1 of 
				\nodimen 2 \space is \nodimen 0}%
		\xdef \Term {#3 = \nodimen 0 \r@dians}%
		\aftergroup \Term
	       }}
	\catcode `\p = \other
	\catcode `\t = \other
	\gdef \n@dimen #1pt{#1} %%% throw away the ``pt''
}

\def \Divide #1by #2{\divide #1 by #2} %%% just a synonym

\def \Multiply #1by #2%%% allows division of a dimen by a dimen
       {{%%% should really freeze parameter 2 (dimen, passed by value)
	\count 0 = #1\relax
	\count 2 = #2\relax
	\count 4 = 65536
	\Mess@ge {Before scaling, count 0 = \the \count 0 \space and
			count 2 = \the \count 2}%
	\ifnum	\count 0 > 32767 %%% do our best to avoid overflow
	\then	\divide \count 0 by 4
		\divide \count 4 by 4
	\else	\ifnum	\count 0 < -32767
		\then	\divide \count 0 by 4
			\divide \count 4 by 4
		\else
		\fi
	\fi
	\ifnum	\count 2 > 32767 %%% while retaining reasonable accuracy
	\then	\divide \count 2 by 4
		\divide \count 4 by 4
	\else	\ifnum	\count 2 < -32767
		\then	\divide \count 2 by 4
			\divide \count 4 by 4
		\else
		\fi
	\fi
	\multiply \count 0 by \count 2
	\divide \count 0 by \count 4
	\xdef \product {#1 = \the \count 0 \internal@nits}%
	\aftergroup \product
       }}

\def\r@duce{\ifdim\dimen0 > 90\r@dian \then   % sin(x+90) = sin(180-x)
		\multiply\dimen0 by -1
		\advance\dimen0 by 180\r@dian
		\r@duce
	    \else \ifdim\dimen0 < -90\r@dian \then  % sin(-x) = sin(360+x)
		\advance\dimen0 by 360\r@dian
		\r@duce
		\fi
	    \fi}

\def\Sine#1%
       {{%
	\dimen 0 = #1 \r@dian
	\r@duce
	\ifdim\dimen0 = -90\r@dian \then
	   \dimen4 = -1\r@dian
	   \c@mputefalse
	\fi
	\ifdim\dimen0 = 90\r@dian \then
	   \dimen4 = 1\r@dian
	   \c@mputefalse
	\fi
	\ifdim\dimen0 = 0\r@dian \then
	   \dimen4 = 0\r@dian
	   \c@mputefalse
	\fi
%
	\ifc@mpute \then
        	% convert degrees to radians
		\divide\dimen0 by 180
		\dimen0=3.141592654\dimen0
%
		\dimen 2 = 3.1415926535897963\r@dian %%% a well-known constant
		\divide\dimen 2 by 2 %%% we only deal with -pi/2 : pi/2
		\Mess@ge {Sin: calculating Sin of \nodimen 0}%
		\count 0 = 1 %%% see power-series expansion for sine
		\dimen 2 = 1 \r@dian %%% ditto
		\dimen 4 = 0 \r@dian %%% ditto
		\loop
			\ifnum	\dimen 2 = 0 %%% then we've done
			\then	\stillc@nvergingfalse 
			\else	\stillc@nvergingtrue
			\fi
			\ifstillc@nverging %%% then calculate next term
			\then	\term {\count 0} {\dimen 0} {\dimen 2}%
				\advance \count 0 by 2
				\count 2 = \count 0
				\divide \count 2 by 2
				\ifodd	\count 2 %%% signs alternate
				\then	\advance \dimen 4 by \dimen 2
				\else	\advance \dimen 4 by -\dimen 2
				\fi
		\repeat
	\fi		
			\xdef \sine {\nodimen 4}%
       }}

% Now the Cosine can be calculated easily by calling \Sine
\def\Cosine#1{\ifx\sine\UnDefined\edef\Savesine{\relax}\else
		             \edef\Savesine{\sine}\fi
	{\dimen0=#1\r@dian\advance\dimen0 by 90\r@dian
	 \Sine{\nodimen 0}
	 \xdef\cosine{\sine}
	 \xdef\sine{\Savesine}}}	      
% end of trig stuff
%%%%%%%%%%%%%%%%%%%%%%%%%%%%%%%%%%%%%%%%%%%%%%%%%%%%%%%%%%%%%%%%%%%%

\def\psdraft{
	\def\@psdraft{0}
	%\ps@typeout{draft level now is \@psdraft \space . }
}
\def\psfull{
	\def\@psdraft{100}
	%\ps@typeout{draft level now is \@psdraft \space . }
}

\psfull

\newif\if@scalefirst
\def\psscalefirst{\@scalefirsttrue}
\def\psrotatefirst{\@scalefirstfalse}
\psrotatefirst

\newif\if@draftbox
\def\psnodraftbox{
	\@draftboxfalse
}
\def\psdraftbox{
	\@draftboxtrue
}
\@draftboxtrue

\newif\if@prologfile
\newif\if@postlogfile
\def\pssilent{
	\@noisyfalse
}
\def\psnoisy{
	\@noisytrue
}
\psnoisy
%%% These are for the option list.
%%% A specification of the form a = b maps to calling \@p@@sa{b}
\newif\if@bbllx
\newif\if@bblly
\newif\if@bburx
\newif\if@bbury
\newif\if@height
\newif\if@width
\newif\if@rheight
\newif\if@rwidth
\newif\if@angle
\newif\if@clip
\newif\if@verbose
\def\@p@@sclip#1{\@cliptrue}


\newif\if@decmpr

%%% GDH 7/26/87 -- changed so that it first looks in the local directory,
%%% then in a specified global directory for the ps file.
%%% RPR 6/25/91 -- changed so that it defaults to user-supplied name if
%%% boundingbox info is specified, assuming graphic will be created by
%%% print time.
%%% TJD 10/19/91 -- added bbfile vs. file distinction, and @decmpr flag

\def\@p@@sfigure#1{\def\@p@sfile{null}\def\@p@sbbfile{null}
	        \openin1=#1.bb
		\ifeof1\closein1
	        	\openin1=\figurepath#1.bb
			\ifeof1\closein1
			        \openin1=#1
				\ifeof1\closein1%
				       \openin1=\figurepath#1
					\ifeof1
					   \ps@typeout{Error, File #1 not found}
						\if@bbllx\if@bblly
				   		\if@bburx\if@bbury
			      				\def\@p@sfile{#1}%
			      				\def\@p@sbbfile{#1}%
							\@decmprfalse
				  	   	\fi\fi\fi\fi
					\else\closein1
				    		\def\@p@sfile{\figurepath#1}%
				    		\def\@p@sbbfile{\figurepath#1}%
						\@decmprfalse
	                       		\fi%
			 	\else\closein1%
					\def\@p@sfile{#1}
					\def\@p@sbbfile{#1}
					\@decmprfalse
			 	\fi
			\else
				\def\@p@sfile{\figurepath#1}
				\def\@p@sbbfile{\figurepath#1.bb}
				\@decmprtrue
			\fi
		\else
			\def\@p@sfile{#1}
			\def\@p@sbbfile{#1.bb}
			\@decmprtrue
		\fi}

\def\@p@@sfile#1{\@p@@sfigure{#1}}

\def\@p@@sbbllx#1{
		%\ps@typeout{bbllx is #1}
		\@bbllxtrue
		\dimen100=#1
		\edef\@p@sbbllx{\number\dimen100}
}
\def\@p@@sbblly#1{
		%\ps@typeout{bblly is #1}
		\@bbllytrue
		\dimen100=#1
		\edef\@p@sbblly{\number\dimen100}
}
\def\@p@@sbburx#1{
		%\ps@typeout{bburx is #1}
		\@bburxtrue
		\dimen100=#1
		\edef\@p@sbburx{\number\dimen100}
}
\def\@p@@sbbury#1{
		%\ps@typeout{bbury is #1}
		\@bburytrue
		\dimen100=#1
		\edef\@p@sbbury{\number\dimen100}
}
\def\@p@@sheight#1{
		\@heighttrue
		\dimen100=#1
   		\edef\@p@sheight{\number\dimen100}
		%\ps@typeout{Height is \@p@sheight}
}
\def\@p@@swidth#1{
		%\ps@typeout{Width is #1}
		\@widthtrue
		\dimen100=#1
		\edef\@p@swidth{\number\dimen100}
}
\def\@p@@srheight#1{
		%\ps@typeout{Reserved height is #1}
		\@rheighttrue
		\dimen100=#1
		\edef\@p@srheight{\number\dimen100}
}
\def\@p@@srwidth#1{
		%\ps@typeout{Reserved width is #1}
		\@rwidthtrue
		\dimen100=#1
		\edef\@p@srwidth{\number\dimen100}
}
\def\@p@@sangle#1{
		%\ps@typeout{Rotation is #1}
		\@angletrue
%		\dimen100=#1
		\edef\@p@sangle{#1} %\number\dimen100}
}
\def\@p@@ssilent#1{ 
		\@verbosefalse
}
\def\@p@@sprolog#1{\@prologfiletrue\def\@prologfileval{#1}}
\def\@p@@spostlog#1{\@postlogfiletrue\def\@postlogfileval{#1}}
\def\@cs@name#1{\csname #1\endcsname}
\def\@setparms#1=#2,{\@cs@name{@p@@s#1}{#2}}
%
% initialize the defaults (size the size of the figure)
%
\def\ps@init@parms{
		\@bbllxfalse \@bbllyfalse
		\@bburxfalse \@bburyfalse
		\@heightfalse \@widthfalse
		\@rheightfalse \@rwidthfalse
		\def\@p@sbbllx{}\def\@p@sbblly{}
		\def\@p@sbburx{}\def\@p@sbbury{}
		\def\@p@sheight{}\def\@p@swidth{}
		\def\@p@srheight{}\def\@p@srwidth{}
		\def\@p@sangle{0}
		\def\@p@sfile{} \def\@p@sbbfile{}
		\def\@p@scost{10}
		\def\@sc{}
		\@prologfilefalse
		\@postlogfilefalse
		\@clipfalse
		\if@noisy
			\@verbosetrue
		\else
			\@verbosefalse
		\fi
}
%
% Go through the options setting things up.
%
\def\parse@ps@parms#1{
	 	\@psdo\@psfiga:=#1\do
		   {\expandafter\@setparms\@psfiga,}}
%
% Compute bb height and width
%
\newif\ifno@bb
\def\bb@missing{
	\if@verbose{
		\ps@typeout{psfig: searching \@p@sbbfile \space  for bounding box}
	}\fi
	\no@bbtrue
	\epsf@getbb{\@p@sbbfile}
        \ifno@bb \else \bb@cull\epsf@llx\epsf@lly\epsf@urx\epsf@ury\fi
}	
\def\bb@cull#1#2#3#4{
	\dimen100=#1 bp\edef\@p@sbbllx{\number\dimen100}
	\dimen100=#2 bp\edef\@p@sbblly{\number\dimen100}
	\dimen100=#3 bp\edef\@p@sbburx{\number\dimen100}
	\dimen100=#4 bp\edef\@p@sbbury{\number\dimen100}
	\no@bbfalse
}
% rotate point (#1,#2) about (0,0).
% The sine and cosine of the angle are already stored in \sine and
% \cosine.  The result is placed in (\p@intvaluex, \p@intvaluey).
\newdimen\p@intvaluex
\newdimen\p@intvaluey
\def\rotate@#1#2{{\dimen0=#1 sp\dimen1=#2 sp
%            	calculate x' = x \cos\theta - y \sin\theta
		  \global\p@intvaluex=\cosine\dimen0
		  \dimen3=\sine\dimen1
		  \global\advance\p@intvaluex by -\dimen3
% 		calculate y' = x \sin\theta + y \cos\theta
		  \global\p@intvaluey=\sine\dimen0
		  \dimen3=\cosine\dimen1
		  \global\advance\p@intvaluey by \dimen3
		  }}
\def\compute@bb{
		\no@bbfalse
		\if@bbllx \else \no@bbtrue \fi
		\if@bblly \else \no@bbtrue \fi
		\if@bburx \else \no@bbtrue \fi
		\if@bbury \else \no@bbtrue \fi
		\ifno@bb \bb@missing \fi
		\ifno@bb \ps@typeout{FATAL ERROR: no bb supplied or found}
			\no-bb-error
		\fi
		%
%\ps@typeout{BB: \@p@sbbllx, \@p@sbblly, \@p@sbburx, \@p@sbbury} 
%
% store height/width of original (unrotated) bounding box
		\count203=\@p@sbburx
		\count204=\@p@sbbury
		\advance\count203 by -\@p@sbbllx
		\advance\count204 by -\@p@sbblly
		\edef\ps@bbw{\number\count203}
		\edef\ps@bbh{\number\count204}
		%\ps@typeout{ psbbh = \ps@bbh, psbbw = \ps@bbw }
		\if@angle 
			\Sine{\@p@sangle}\Cosine{\@p@sangle}
	        	{\dimen100=\maxdimen\xdef\r@p@sbbllx{\number\dimen100}
					    \xdef\r@p@sbblly{\number\dimen100}
			                    \xdef\r@p@sbburx{-\number\dimen100}
					    \xdef\r@p@sbbury{-\number\dimen100}}
%
% Need to rotate all four points and take the X-Y extremes of the new
% points as the new bounding box.
                        \def\minmaxtest{
			   \ifnum\number\p@intvaluex<\r@p@sbbllx
			      \xdef\r@p@sbbllx{\number\p@intvaluex}\fi
			   \ifnum\number\p@intvaluex>\r@p@sbburx
			      \xdef\r@p@sbburx{\number\p@intvaluex}\fi
			   \ifnum\number\p@intvaluey<\r@p@sbblly
			      \xdef\r@p@sbblly{\number\p@intvaluey}\fi
			   \ifnum\number\p@intvaluey>\r@p@sbbury
			      \xdef\r@p@sbbury{\number\p@intvaluey}\fi
			   }
%			lower left
			\rotate@{\@p@sbbllx}{\@p@sbblly}
			\minmaxtest
%			upper left
			\rotate@{\@p@sbbllx}{\@p@sbbury}
			\minmaxtest
%			lower right
			\rotate@{\@p@sbburx}{\@p@sbblly}
			\minmaxtest
%			upper right
			\rotate@{\@p@sbburx}{\@p@sbbury}
			\minmaxtest
			\edef\@p@sbbllx{\r@p@sbbllx}\edef\@p@sbblly{\r@p@sbblly}
			\edef\@p@sbburx{\r@p@sbburx}\edef\@p@sbbury{\r@p@sbbury}
%\ps@typeout{rotated BB: \r@p@sbbllx, \r@p@sbblly, \r@p@sbburx, \r@p@sbbury}
		\fi
		\count203=\@p@sbburx
		\count204=\@p@sbbury
		\advance\count203 by -\@p@sbbllx
		\advance\count204 by -\@p@sbblly
		\edef\@bbw{\number\count203}
		\edef\@bbh{\number\count204}
		%\ps@typeout{ bbh = \@bbh, bbw = \@bbw }
}
%
% \in@hundreds performs #1 * (#2 / #3) correct to the hundreds,
%	then leaves the result in @result
%
\def\in@hundreds#1#2#3{\count240=#2 \count241=#3
		     \count100=\count240	% 100 is first digit #2/#3
		     \divide\count100 by \count241
		     \count101=\count100
		     \multiply\count101 by \count241
		     \advance\count240 by -\count101
		     \multiply\count240 by 10
		     \count101=\count240	%101 is second digit of #2/#3
		     \divide\count101 by \count241
		     \count102=\count101
		     \multiply\count102 by \count241
		     \advance\count240 by -\count102
		     \multiply\count240 by 10
		     \count102=\count240	% 102 is the third digit
		     \divide\count102 by \count241
		     \count200=#1\count205=0
		     \count201=\count200
			\multiply\count201 by \count100
		 	\advance\count205 by \count201
		     \count201=\count200
			\divide\count201 by 10
			\multiply\count201 by \count101
			\advance\count205 by \count201
			%
		     \count201=\count200
			\divide\count201 by 100
			\multiply\count201 by \count102
			\advance\count205 by \count201
			%
		     \edef\@result{\number\count205}
}
\def\compute@wfromh{
		% computing : width = height * (bbw / bbh)
		\in@hundreds{\@p@sheight}{\@bbw}{\@bbh}
		%\ps@typeout{ \@p@sheight * \@bbw / \@bbh, = \@result }
		\edef\@p@swidth{\@result}
		%\ps@typeout{w from h: width is \@p@swidth}
}
\def\compute@hfromw{
		% computing : height = width * (bbh / bbw)
	        \in@hundreds{\@p@swidth}{\@bbh}{\@bbw}
		%\ps@typeout{ \@p@swidth * \@bbh / \@bbw = \@result }
		\edef\@p@sheight{\@result}
		%\ps@typeout{h from w : height is \@p@sheight}
}
\def\compute@handw{
		\if@height 
			\if@width
			\else
				\compute@wfromh
			\fi
		\else 
			\if@width
				\compute@hfromw
			\else
				\edef\@p@sheight{\@bbh}
				\edef\@p@swidth{\@bbw}
			\fi
		\fi
}
\def\compute@resv{
		\if@rheight \else \edef\@p@srheight{\@p@sheight} \fi
		\if@rwidth \else \edef\@p@srwidth{\@p@swidth} \fi
		%\ps@typeout{rheight = \@p@srheight, rwidth = \@p@srwidth}
}
%		
% Compute any missing values
\def\compute@sizes{
	\compute@bb
	\if@scalefirst\if@angle
% at this point the bounding box has been adjsuted correctly for
% rotation.  PSFIG does all of its scaling using \@bbh and \@bbw.  If
% a width= or height= was specified along with \psscalefirst, then the
% width=/height= value needs to be adjusted to match the new (rotated)
% bounding box size (specifed in \@bbw and \@bbh).
%    \ps@bbw       width=
%    -------  =  ---------- 
%    \@bbw       new width=
% so `new width=' = (width= * \@bbw) / \ps@bbw; where \ps@bbw is the
% width of the original (unrotated) bounding box.
	\if@width
	   \in@hundreds{\@p@swidth}{\@bbw}{\ps@bbw}
	   \edef\@p@swidth{\@result}
	\fi
	\if@height
	   \in@hundreds{\@p@sheight}{\@bbh}{\ps@bbh}
	   \edef\@p@sheight{\@result}
	\fi
	\fi\fi
	\compute@handw
	\compute@resv}

%
% \psfig
% usage : \psfig{file=, height=, width=, bbllx=, bblly=, bburx=, bbury=,
%			rheight=, rwidth=, clip=}
%
% "clip=" is a switch and takes no value, but the `=' must be present.
\def\psfig#1{\vbox {
	% do a zero width hard space so that a single
	% \psfig in a centering enviornment will behave nicely
	%{\setbox0=\hbox{\ }\ \hskip-\wd0}
	%
	\ps@init@parms
	\parse@ps@parms{#1}
	\compute@sizes
	%
	\ifnum\@p@scost<\@psdraft{
		%
		\special{ps::[begin] 	\@p@swidth \space \@p@sheight \space
				\@p@sbbllx \space \@p@sbblly \space
				\@p@sbburx \space \@p@sbbury \space
				startTexFig \space }
		\if@angle
			\special {ps:: \@p@sangle \space rotate \space} 
		\fi
		\if@clip{
			\if@verbose{
				\ps@typeout{(clip)}
			}\fi
			\special{ps:: doclip \space }
		}\fi
		\if@prologfile
		    \special{ps: plotfile \@prologfileval \space } \fi
		\if@decmpr{
			\if@verbose{
				\ps@typeout{psfig: including \@p@sfile.Z \space }
			}\fi
			\special{ps: plotfile "`zcat \@p@sfile.Z" \space }
		}\else{
			\if@verbose{
				\ps@typeout{psfig: including \@p@sfile \space }
			}\fi
			\special{ps: plotfile \@p@sfile \space }
		}\fi
		\if@postlogfile
		    \special{ps: plotfile \@postlogfileval \space } \fi
		\special{ps::[end] endTexFig \space }
		% Create the vbox to reserve the space for the figure.
		\vbox to \@p@srheight sp{
		% 1/92 TJD Changed from "true sp" to "sp" for magnification.
			\hbox to \@p@srwidth sp{
				\hss
			}
		\vss
		}
	}\else{
		% draft figure, just reserve the space and print the
		% path name.
		\if@draftbox{		
			% Verbose draft: print file name in box
			\hbox{\frame{\vbox to \@p@srheight sp{
			\vss
			\hbox to \@p@srwidth sp{ \hss \@p@sfile \hss }
			\vss
			}}}
		}\else{
			% Non-verbose draft
			\vbox to \@p@srheight sp{
			\vss
			\hbox to \@p@srwidth sp{\hss}
			\vss
			}
		}\fi	



	}\fi
}}
\psfigRestoreAt
\let\@=\LaTeXAtSign




\special{header=/group/csdl/tex/psfig/lprep71.pro}
%\definemargins{0.75in}{0.75in}{0.75in}{0.75in}{0.3in}{0.3in}

\begin{document}

\begin{titlepage}
\vspace*{1in}

\begin{center}
  {\Large\bf CSRS: a New Approach to Software Review}\\
  \bigskip\par
  
  \silentfootnote{Support for this research was provided in part by the
  National Science Foundation Research Initiation Award CCR-9110861 and
  the University of Hawaii Research Council Seed Money Awards
  R-91-867-F-728-B-270 and F-92-868-F-728-B-430.}
  
  Danu Tjahjono                    \bigskip\par
  
  Collaborative Software Development Laboratory\\
  Department of Information and Computer Sciences\\
  University of Hawaii\\
  Honolulu, HI 96822\\
  (808) 956-6920\\
  {\tt dat@uhics.ics.hawaii.edu}           \bigskip\par
  
  {\bf CSDL-TR-93-04}                \bigskip\par
  
  Last Revised: \today
  
\end{center}
\end{titlepage}
\newpage

%\ls{1.0}

\pagenumbering{roman}
\tableofcontents
\newpage

\pagenumbering{arabic}

\section {Overview}

The importance of software review has been well accepted by
software engineers. It is a powerful way to improve the quality
and productivity of the software process \cite{Humphrey90}. It is quite
effective in finding errors that one or more of these should be
employed in every programming project \cite{Myers79}. It is even more
effective than testing \cite{Basili85}. In cases where no automatic support is
currently feasible, it is the only technique available (e.g.,
for assessing the readability and understandability of software
product). 
When  applied early in the development cycle, it will save a
significant amount of development resources as errors are detected
early and closer to the originator and are therefore prevented
from propagating to the next phases of the cycle \cite{Fagan76}.
In the field of software quality assurance, it plays
the central role of managing and ensuring the quality of software
product. 

Despite its overwhelming documented benefits, there are many
problems and questions associated with current review practices. 

This research attempts to investigate these problems and
provide empirical evidence to the questions related to the practices of
software review.
We also introduce a review system called CSRS (Collaborative Software
Review System) that allows one to
experiment with  various aspects of review process, to compare
different review methods,  and at the same 
time to use it as practical tools for supporting review activities.
The system also allows the user's organization to analyze and 
to improve their review process based upon metrics data collected by
the system automatically. 

In the next section, we outline our motivation and the research
problems, then we describe the approaches to solve the problems.
After that we present the system in details and
finally, we conclude with the discussion on the current status
and the future direction  of the system.

\section {Research Problems}
Our research is primarily motivated by the following problems apparent
from the current practices of software review:

\begin {itemize}
\item {\sl No or very  few computational supports currently available for
software review.}
Most current practices of software review/ inspection are entirely
manual and labor-intensive. Reviewers flip through a stack of printout
paper back and forth, record the issues on a separate sheet/form. The
moderator then goes through each of the forms, combines similar
issues and rewrites them (if any) so that they can be presented
properly during the meeting. During and after the meeting, the
moderator must recheck the forms to ensure
all issues have been discussed and actions have been specified.
Finally, he/she must rewrite the results to be included in the final
report.
Current practices also lack of computational supports for human
activity, especially supports for group process. 

The computer-based review systems such as
ICICLE \cite{Brothers90}, INSPEQ \cite{Knight91}, provides very limited
computational supports. They are also influenced strongly by the
manual method. 

\item {\sl Ill defined process.}
    Although there is a well published literature on how to
    effectively conduct software review/inspection \cite{Fagan76}, in
    practice there
    are many variations on how to carry this out; the
    review results also vary considerably, some even show conflicting
    results.  For example, some practices
    \cite{Fagan76,Russel91} explicitly 
    advocate the use of parapharasing for effective review, others
    do not \cite{Humphrey90}.

\item {\sl Lack of understanding of review factors and high quality
measurements.}
As stated by Myers \cite{Myers79}, no one currently knows or has done any
investigation of what factors are actually contributed to the
effectiveness of software review, or whether some review methods are
better than the others. This type of study is currently very
difficult to setup and measure  since different methods use different
environment and/or incompatible review process. 

\item {\sl Lack of understanding of review metrics and support for
process improvement.} 
  Although metrics collection is always part of a review process, 
  there are few studies that investigate the contribution of various
  metrics data to the effectiveness of review process and to use it
  for process improvement. In other words, the data are not carried
  over to the subsequent review processes.
 Worse yet current practices collect metrics manually
  which often reduces the accuracy of the data.
  For example, the literature \cite{Gilb88,Humphrey90} suggest that the time the reviewers
  spent in preparing review meeting correlates positively with the
  effectiveness of review.  However, there is no empirical support for
  this yet. Similarly, inspection rate correlates strongly with the
  number of errors found; reducing this rate will drop review
  effectiveness significantly \cite{Gilb88} Unfortunately, no study
  has investigated this optimum rate. And most of all, no one has
  studied how these performance change over time.
  Other interesting metrics that has never been investigated
  include the effect of program size, program complexity, etc. to
  the review effectiveness (average number of errors found per unit
  time  per participant),etc. 
\end {itemize}


\section {Approach}
Our basic approach to software review is to implement computer
supported review system  called CSRS (Collaborative Software Review
System) that addresses the above problems as follows:
\begin {itemize}
\item {\sl Supports human activity and automates clerical procedures.}
In principle, software review  is a human activity, any computational
tools must therefore support this activity.  
CSRS provides this support through automation of clerical procedures
such as forms preparation and consolidation, and by improving both the
individual and the group process associated with the review activity.
The later is realized by means of data and process models.
The data model characterizes review products (review artifacts),
and the process model characterizes review steps or phases.
From the user's perspective, the data model guides review participants in
what to look for and how to express it during review process, and
the process model guides them in when to look for.


\item {\sl Provides well defined data and process model.}
CSRS is a generic review method that operates on a well defined data and
process model described earlier. The data model captures  wide
varieties of review artifacts 
independent of review methods. The process model, however, is
review method specific. In other words, to follow a specific review
method, such as Fagan \cite{Fagan76}, only the process model needs to be changed.
The current system, however, does not support various process models
yet. It operates on our own process model described shortly.


\item {\sl Provides a framework for investigating review factors.}
Any software review activities involve two basic activities, namely,
reading comprehension and verification/error checking. Each
can be further distinguished by the degree of efforts involved,
whether it is individual or group, and also by the specific technique
to carry out this activity (for example, stepwise abstraction technique
for program reading, checklist or program correctness for verification,
etc). These factors can be used (as control variable) to compare the
effectiveness of different review methods currently available.
CSRS can be setup to perform this experiment by running it with
different process models corresponding to the different review
methods under investigation. Some interesting experiments include
whether checklist based verification is more effective than {\it free}
verification, whether paraphrasing (group verification) is more effective
than individual verification, etc.

\item {\sl Collect review metrics automatically.}
Review metrics is used to assess the effectiveness of a review
process. It can also tell the quality of the product by
which the process is operated on.
CSRS collects metrics associated with the product and the process of
review unobtrusively.  When taken repeatedly for a long period of
time using the same process model, this data can tell and predict the
effectiveness of the process and how to improve it.

CSRS metrics data currently include
time spent in reviewing program objects, in
raising issues, in proposing actions, and/or engaging in 
discussion; the number of errors found (under various categories); the
number of participants; and the characteristics of
the work product, such as program size and type (i.e., function, macro,
design) for code review, etc.

\end{itemize}

\section {CSRS}

CSRS is a collaborative software review system that assists review
participants in carrying out formal review activity. The system
follows a specific process and data model corresponding to the product
and the process of the activity.

\subsection {Data Model}
The CSRS data model describes what artifacts are produced and
consumed by review activity. 
The  model also facilitates reasoning about the artifacts both by
human and machine.  For example, it is easy to compare and contrast
various issues raised by different reviewers, semi-automatically
consolidate them and finally propose a single consentual action.
The model also imposes a more structured group communication. In fact,
CSRS can also be considered as group support system. The later
has been shown effective in reducing group barriers
typically found in face-to-face meeting \cite{Nunamaker91}.  The system allows
reviewers to express his opinions and argue about one's position in a
controlled and productive manner. It also implements
consensus building mechanism to further speed up the group process.

In general, CSRS data model consists of a set of typed nodes and links
where the nodes correspond to review artifacts and the links
correspond to relationship among these artifacts. \ref{fig:data-model} shows the
details of the model. This model has been constantly improved since the
initial design of the system \cite{csdl-92-02}.
Source nodes are input materials  to be reviewed.  They may include
requirement document, design document, source code, and/or other software
artifacts.
Source nodes can be annotated with other source nodes. For example, source
code is annotated with its detailed design for code review.
Annotation can be defined on a specific region in the source 
and substitute-able (i.e., the region can be
replaced with its annotation).  This later feature allows, for example,
source code to be systematically 
abstracted into its detailed design section by section similar to the
stepwise abstraction technique for code reading \cite{Linger79}.
For code review, the source nodes are classified into
program objects such as function, macro, structure, and variable
definition. 

Standard-issue nodes contain issues that are identified by the
organization (from previous review process or the literatures) and are
checked off by reviewers against the source nodes during review session.
Standard issues are similar to checklist in code inspection
\cite{Fagan76}.  CSRS's checklist, however, may include other artifacts such as
special requests from the producer to the reviewers to check specific
properties in the source nodes. It may also include a set of questions
prepared by the producer 
to assess reviewers' familiarity with specific concepts in the source.
This type of requests and questions  has been shown effective in
the design review process \cite{Parnas85}.  
Unlike standard issues which are generally  applied to
all source nodes, questions based checklist are source
nodes specific; when the source is displayed, the question
will be displayed as well.  In figure 1,
this question appears as comment node that annotates source node.


The rest of the nodes in the data model are commentary nodes; they are
generated as a result of review process. Individual reviewers record
any suspected problems or defects in the source in issue nodes.
He/she may then propose one or more actions to resolve this issue in
action nodes. 

Comment nodes are to question, confirm, or disconfirm  issue or action
nodes. They may also question source nodes in order to seek
clarification about the source. Responses to questions are
also recorded in comment nodes. 

Case nodes are to record test cases or test data that are raised by
reviewers in particular source nodes. These test cases are 
similar to the ones used in ordinary testing method, which are derived from
the corresponding source, except that case nodes are created only when the
reviewers find a specific test case in the source very difficult to
exercise mentally, and wants to delegates this task to the producer or the
group. Unlike questions (in comment node) that inquire the logic
of the source, cases inquire verification of the source. 
Mentally simulating test cases are very common in code inspection
practices; it is conducted as part of paraphrasing activity during
inspection meeting \cite{Russel91,Ackerman89}. 
Case and comment nodes may suggest new issues, and comment
suggests new actions.

Consolidated-issue and consolidated-action nodes are to consolidate
issues/ actions raised by different reviewers. These nodes represent
the group consensus on the issues/ action.
Consolidated-action nodes are further classified into type {\it Fix, Ignore}
and {\it Undecided}.
Consolidated-action also modifies the source. 
This {\it modify} link is especially important during re-review phase. It
facilitates reviewers to check whether the suggested changes described
in the consolidated-action has been properly implemented by the
producer. It is even more important since the study shows that error
rate found in modified code is considerably higher than is found in
new code \cite{Fagan76}. 
Finally, consolidated-issue may derives new standard-issue node. 
The procedure of incrementally refines standard-issues is similar to
the defect prevention method \cite{Jones85}.

Finally, despite many different types of nodes and links, reviewers do
not need to memorize  them, they are all transparent from the
CSRS's user interface that implements pull-down menu and
context-sensitive pop-up menu. 

\begin{figure}[htb]
  \fbox{\centerline{\psfig{figure=data-model.eps}}}
  \caption{CSRS Data Model}
  \label{fig:data-model}
\end{figure}



\subsection {Process model}
The CSRS process model describes specific steps or procedures that
review participants must follow during review session.
In manual code inspection, the process model is well
documented \cite{Fagan76}. However, there is no evidence yet that shows this
model is the most effective one, especially when it involves
computer supported review environment such as CSRS.
In fact, one of our research goal is to experiment with
different review phases (i.e., process model) and finds the most
productive one (if any). 

In general, the process model constrains what review artifacts can be
manipulated during a given phase, or what nodes and links defined in
the data model are legal for a given phase.
The CSRS process model is shown in \ref{fig:process-model}. 

\begin{figure}[htb]
  \fbox{\centerline{\psfig{figure=process-model.eps}}}
  \caption{CSRS Process Model}
  \label{fig:process-model}
\end{figure}


This process model is 
somewhat similar to classic code inspection except that it gears toward
computer based review environment. The model consists of the
following review phases:

\begin {itemize}
\item {\bf Review Initialization.}
This initial review phase starts with selecting review participants
and assigning review roles to the them.
The roles are similar to Fagan's inspection which
include moderator, producer, reviewer, and recorder. 
These information are then entered into the system along with
administrative data such as starting review date, project description,
the nature of this review (new or re-review), etc.
Afterwards, the producer with the assistance of the moderator inputs
review source nodes from the prepared source files (assuming the files
have met all input conditions decided by the management). The system
will automatically parse the files into program objects (for source
codes) and generate appropriate types of source nodes. The producer
then prepares annotation nodes, comment nodes for special request
checklist if any, and finally  a short description of the top level
design of the source 
nodes to be reviewed along with its guided map (for code review). This
map suggests the ordering of source nodes to be followed by the reviewers 
in order to facilitate better comprehension of the source. 
Some code inspection practices uses top-down techniques
when reviewing source codes \cite{Hart82}, others prefer bottom-up
techniques \cite{Linger79}. At present, we use the top down technique.
When done, the moderator initiates the next phase of the review process.

\item {\bf Orientation.}  This is a short group meeting where the
producer briefs  reviewers the work products, usually the top-level
description of the products. This phase, however, can be skipped if the
moderator feels that the on-line documentation prepared in the
previous phase is sufficient, or the participants are already familiar
with the product. CSRS currently does not provide computational
services for this phase, other than to generate hard-copy briefing
materials.

\item {\bf Private review.}
In manual code inspection, this phase is called
preparation: individual reviewers familiarize themselves with the
software product in order to attend the upcoming inspection
meeting; they normally focus their efforts more on understanding the
products than searching for errors. 

In computer based review such as CSRS, where one of the objective is 
to minimize meeting time, this phase is used for both understanding
and error hunting. Errors (i.e., issues in CSRS terminology)
discovered during  this time are not made public in order to reduce
free riding \cite{Nunamaker91} and to prevent unnecessary discussion/debate.

Program understanding is the most difficult, mentally draining but
essential activity in code review and yet many computer-based review 
systems (e.g., ICICLE \cite{Brothers90} ) explicitly avoided this issue.
In manual code inspection, this activity  may take up more than half of
the entire review time \cite{Humphrey90}.  Our approach in supporting
program understanding is to allow 
reviewers to post questions (comment node of type question) directly
to the producer. The question can be as specific as referring to
a small block in the source code.  Questions and answers of this type
are made public to all reviewers. 
Additional tools to facilitate program reading and understanding in
CSRS include manual abstraction (replacement of a block of code with
its informal description),  hiding or showing a selective block of
code, and cross-references. 

Similar to code inspection, reviewers are provided with a set of
checklist items. The checklist includes both standard-issues and
special requests from the producer described in the
previous section if any.
Individual reviewers simply
go through this checklist item by item and mark v when done. When all
these items are marked v  the corresponding source node is
considered reviewed. 

When the reviewer suspects any defects in the source, he/she creates
an issue node describing the problem. One or more action nodes can be
proposed to 
resolve this issue. Many code inspection literatures
\cite{Fagan76,Humphrey90} disallow actions to be decided during the review.
However, our experience with manual
inspection shows that reviewers often suggest actions when identifying
issues. In fact, these actions often give clarification to the issues
both during the review and/or the rework phase. In
structured-walkthrough \cite{Yourdon79}, action is part of the review process.
In light of this evidence, we decided to 
makes action generation optional; in other words, reviewers are free to
create or not create action nodes. 

The reviewer may also create a case node when he/she has difficulty
verifying specific property of the source as described in the
previous section.

The private review phase concludes when all reviewers have marked the
source nodes reviewed. The moderator (running the system in Admin
mode) then deactivates this phase and activates the next phase, Public
Review. 


\item {\bf Public review.} In this phase,  review participants
including producer react to the 
issues, actions and comments generated by other participants during
private review phase.  In classic code inspection this phase takes
place in a 
face-to-face meeting. Our public review is conducted entirely in a non
face-to-face meeting (computer-mediated). This computer-mediated
meeting mode has been consistently shown \cite{Nunamaker91} to yield  more productive
meeting outcomes. When the reviewer disagrees to a particular issue or
comment, he/she can create a comment node and attaches disconfirm-link.
Similarly, he/she may further confirm the issue using a comment node (
and confirm-link). We also require that every participants take one
position of either confirm, disconfirm or neutral on every issue
node. This information is then used to build group consensus by the
moderator. The reviewers are also required to create similarity link
when noticing that his/her issue or action is similar to the one
posted by others. This link allows the system to consolidate similar
issues/actions semi-automatically. They may also create additional
issue nodes, especially after reading comments or issues posted by
others. 

In general, during public review phase, all reviewers must read and
response to every issues created in 
the system. Other commentary nodes do not need to be responded but
read. 
The reviewers may start this phase by reading all unseen issues,
actions or other commentary nodes, or prefer to go through each source
and read the issues listed in it one by one. The reviewers can also read 
consolidated-issue node. This node merges two or more similar issues
into one and also presents the discussion/ argument regarding the
issue by the group 
members so far (the discussion is extracted from comment nodes and is
grouped according to its confirmation or disconfirmation links.)
Unlike individual issue, which can only be modified by the creator of the 
issue, consolidated issue can only be modified by the moderator (during
consolidation phase). During public review phase, the system will
refresh the content of this consolidated node every time a new 
issue or comment is added to the discussion thread. 

This phase concludes when reviewers have responded to all commentary
nodes.

\item {\bf Consolidation.}
This phase, which is carried out by the moderator,
is to consolidate issues raised by individual reviewers during public
review. In manual review, this task is  non-trivial since the moderator
needs to read and go through all issues, group them and finally write one
consolidated one. In CSRS, all issues are grouped naturally as part of
the public review activity.
The moderator simply edits this consolidated issue if necessary and
then creates one consolidated action if consensus has been reached.

At the conclusion, some issues may remain unresolved (no consensus). 
The moderator then calls for a meeting to discuss and resolve these
issues.  A report summarizing the consolidated and unconsolidated
issues will be distributed to the participants prior to the meeting.

\item {\bf Meeting.}
This optional phase is to discuss issues that have not been
resolved or agreed upon during previous phase. In addition, it may be
used to walkthrough test cases (recorded in Case node) requested by
some participants. The procedure of doing this task is similar to
paraphrasing technique used in Fagan's method; the producer/reader
goes to the source in question mentally simulating the test cases.
Additional issues discovered during this phase will be recorded in
consolidated-issue nodes, and proposed resolutions in consolidated
actions respectively. This face-to-face meeting is currently done
manually without computer support.

This phase concludes when all consolidated issues have consolidated
actions. 

\item {\bf Conclusion.}
This phase, which is also carried out by the moderator, is to generate
metrics and final report of the entire review process.  The metrics
include every 
aspect of review activity. The number of errors is obtained by
simply counting the number of consolidated-issues that have
consolidated-actions.
The errors per user can be obtained by counting the number of  user's
issue nodes that lead (link) to consolidated action.
Time related metrics  is collected by the system automatically.
Specifically, for every CSRS command invoked by the user, the system
will generate a metrics data consisting of {\sl User-ID, Operation, Entity,
Time, Screen, Info},  where {\sl User-ID} is the ID of the user who
invokes the operation, {\sl Operation} is the name of the operation,
{\sl Entity} is the name of the node or link on which the operation
operates on, {\sl Time} is the time when this operation takes place,
{\sl Screen} is the name of the screen (visible window), and {\sl Info}
is additional information to be added if any. These raw metrics are then passed
to metrics analyzer to generate review statistics.
For example, to find out how many hours spent by a particular
reviewer in reading a particular node, we subtract the
time this node is read (operation = read-node) by this user 
until it is closed (operation = close-node).

This phase is also used to generate or refine standard issues
by analyzing the identified errors as described in the previous section.

\item {\bf External development.}
This phase is to carry out the rework task. The assigned programmers can
simply look up consolidated action nodes (generated as final hard-copy
report) to determine what need to be done with the source nodes.

\end {itemize}

\section {Review experiences}

We have conducted two rounds of code review using CSRS.  The
first review was actually intended to debug the system. The
source program codes are trivial (small and easy to comprehend).  Based
on the feedback from the user, the user's interface of the system
was then improved. The second review involves 435 lines of 
codes and 18 program objects (functions and global variables) of
generic-interface subsystem of EGRET called nbuff (short for
node-buffer). The participants include 4 reviewers and 1 producer
(5 people).  Two of the participants also assume the
role of moderator; one played the role during review
initialization and another performed issues consolidation and
moderated the group meeting.

The review process was conducted according to the process model
described earlier which involved mostly no meeting.  The
orientation meeting where the producer explained the top-level
design of the nbuff took only 15 minutes. The handout from this
meeting was also provided on-line in CSRS.

In general, we imposed no time restriction to reviewers. They
basically worked at their leisure time.  The entire review
process took 3 weeks to complete. 
The private review lasted for about 10 days, but the average time
spent by reviewers is only 3.5 hours.  The review rate ranges
from 100 lines/hour to 250 lines/hour.
  
The public review completed in about a week, and the average
review time was 2 hours, or 3 minutes/issue.  The consolidation
phase performed by the moderator took less than 30 minutes to complete,
or 0.75 minutes/issue.
There were 42 issues identified, 37 of them were resolved during
public review based upon the reviewers' consensus.  The last 5
issues were brought to the meeting for discussion.  In the
meeting, the moderator also presented the remaining 37 issues (in
consolidated form) and their corresponding actions.  The overall meeting
lasted for about 50 minutes, or the average time for issue
discussion is 1.2 minutes/issue.

Informal questionnaires after the review indicate that the
reviewers are generally satisfied with the outcomes, the review
process and the CSRS itself. They also do not have problem reading the
code from computer display without having the hardcopy on hand.
Some reviewers who have been exposed to manual review even prefer
CSRS to hardcopy program code.

Prior to implementing CSRS, we have also conducted manual code
review on 2500 lines of codes of server subsystem of Egret and
identified 174 issues. The review process was more or less
similar to CSRS's process model except that the public review
was carried out in a face-to-face meeting. After the meeting, the
moderator performed issues and actions consolidation off-line.
The review rate was about 350 lines/hour on the average, and during the
meeting issues were discussed at the average rate of 2.5 minutes/issue.

The data clearly shows that the meeting for manual review
takes twice longer to conclude than for CSRS.  The CSRS's public
review, however, is slightly longer (3 minutes/issue) than the
public review in manual code review (2.5 minutes/issue). In the
later review, however, the moderator needs to spend additional
1.5-2 minutes/issue after the meeting for consolidation. Worse
yet, the moderator had difficult time carrying out the later
consolidation; the actions decided during the meeting were not
properly 
recorded, they were often implied from the group discussion. 
In CSRS arguments are presented as comment nodes and attached to
the corresponding issues.  This facilitates consensus building
and at the same time serves as supporting evidence for the action
taken by the group.

Taken into consideration the public review, meeting and consolidation 
phase, the data does not show significant time saving using
CSRS. However, participants uses review time more productively in CSRS
(i.e., doing review rather than doing clerical tasks) and the process
also yields more satisfactory outcomes as issues, comments and actions
are effectively captured.

Despite the many benefits of asynchronous mode, we also observed that
some reviewers feel 
irritated during public review phase as they have to constantly login
to the system to check for new issues or responses from other
reviewers/ producer. We plan to provide automatic electronic mail
notification to resolve this problem.

As for review rate during private review, the data shows that
CSRS takes longer (100-250 lines/hour) than manual review (350
lines/hour).  On the other hand, this may indicate that CSRS
yields more comprehensive review than manual one.
The literatures \cite{Russel91} suggest than 150 lines/hour is the
optimum review rate for effective review.

As for error detection effectiveness, our data shows that CSRS
identified more errors or issues (9.4 issues/100LOC) than manual
review (7 issues/100LOC). In both reviews the error are mostly
design or specification incorrect; coding or algorithm errors are
rarely identified. However, one issue indirectly identified severe
errors in the program and others accurately predicted the problem that
may occur in the future based upon the current coding. For example, 
one reviewer identified the use of current buffer may pose
brittleness to the system, and this was later confirmed.


\section {Current status}

The current version of CSRS (version CSRS-1.2) implements most of the
functions described in the previous sections. Specifically, it
provides basic hypertext functionalities to create nodes and links as
described in the data model. It supports private review, public
review, consolidation phase and administrative mode during review
initialization and for metrics generation. The review metrics 
also includes raw data collection as described in the previous
section. 

Checklist support is partially completed; reviewers must explicitly
invoke the command {\sl set status to reviewed} when finish reviewing
the source. Re-review 
cycle which takes into account the {\sl modify} relationship between
the consolidated-action node and the new source node is not yet
supported.

\section {Future Direction}
We intend to extend the system for different programming environment,
such as C, or C++. We also plan to use the system for design review and
requirement review in the future. And finally, we plan to perform
several formal experiments to compare different review methods
currently available.

\newpage
\bibliography{csrs}
\bibliographystyle{/group/csdl/tex/named-citations}

\end{document}

