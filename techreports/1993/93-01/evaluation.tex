\section{Evaluation}
\label{sec:evaluation}

The effectiveness of CLARE as a collaborative learning tool will be
evaluated through two experiments. The basic claim motivating both is that
CLARE can help the learner improve their performance in selected learning
tasks. The experiment subjects are advanced learners, i.e., upper-level
undergraduate and graduate students. The experiments involve two types of
tasks: generation of study questions and joint writing of research review
papers, both of which are done based on a set of research papers.

\subsection{Experiment I}

\subsubsection{Tasks} 

This experiment involves a joint generation of study questions from a
selected set of papers in software verification and validation (V \& V).
The subjects are graduate students enrolled in ICS 613 (Advanced software
engineering) at the University of Hawaii. The subjects will be given 2-3
papers on a specific topic in software V \& V. They are expected to read
them and, based on the reading, generate and turn in a set of important
questions. The purpose of this experiment is to assess the effect of CLARE
on the the quality of resulted questions and on the process through which
those questions are generated.

\subsubsection{Procedure}

The experiment involves 12 subjects, which are randomly assigned into two
groups: the treatment group which uses CLARE, and the control group which
relies solely on the face-to-face and pencil-and-paper based learning. Each
group is randomly divided into two (2) study groups, with three (3)
students in each. The experiment will start at about one-month after the
semester begins, and will last for two weeks. During these two weeks, two
experiment sessions will be conducted each week. Each experiment session is
divided into stages: private question generation and group consolidation.
For both groups, the first stage is done outside classrooms: the treatment
group creates questions online, while the control group records their
questions on the paper.

The second phase is conducted in the face-to-face setting. During this
phase, the treatment group will be gathered in front of a workstation,
reviewing, discussing, and integrating questions generated by the
individual members from the previous phase. A print function will be
invoked at the end of this session to print a hardcopy of all the questions
generated by both individual and groups, which will be turned in for
evaluation. Meanwhile, the control group will consolidate their questions
through face-to-face discussions. One member in the group will be assigned
to record questions agreed upon by the group. Like the treatment group, at
the end of the session the control group must turn in both individual and
group questions. Although both groups may use as much time as they need for
the first phase, the time for the second phase is fixed, i.e., a
fifty-minute class period.

Since there will be four (4) experiment sessions during the 2-week
experiment period, each study group will have opportunity to alternate
between the treatment and the control group. By the end of the experiment,
each subject will have two sessions with CLARE, and two sessions without.
A written questionnaire will be administered at the end of each experiment
session.


\subsubsection{Measures}

This experiment will be evaluated on the basis of three grounds: {\it
outcome\/}, {\it process\/}, and {\it learner satisfaction\/}. The outcome
measure is the number and the quality of questions raised by the individual
study group. A printed copy of all questions from each group will be
collected and distributed to three experts in the field, including the
course instructor. They will be graded on the scale of 1-100 based on the
clarity of questions, the level of understanding of the materials shown,
and the level of integration with related literature. The group identity of
those questions will be hidden from the evaluators.

The face-to-face sessions for both treatment and control groups will
be videotaped. The transcript will be analyzed at the end of the
experiment. For the CLARE group, additional process data will be captured
by the system through its instrumentation facilities. Examples include the
type and the number of instances generated, the time they were created and
modified, the time used in creating those instances, the usage frequency of
selected functions, e.g., example and domain instances, the density of
links, and the similarity scores.

The learner satisfaction measure for both treatment and control groups will
be evaluated through written feedback. At the end of each experiment
session, a questionnaire, which contains scaled and open-ended questions,
will be administered to each group. The data will be analyzed with
reference to the printed copy of questions and the process data gathered.


\subsection{Experiment II}

\subsubsection{Tasks}

This experiment involves a joint writing of a research review papers on
software requirement analysis. The subjects are upper-level undergraduates,
i.e., juniors and seniors, enrolled in ICS 413 (Software engineering) at
the University of Hawaii in the Spring Semester, 1993. The subjects will be
assigned 3-5 papers on software requirement analysis from current journals
and conference proceedings. They are expected to read those papers, discuss
them, and write a critical review that integrates issues/problems
addressed, compare and contrast approaches proposed, and generate a list of
further questions. The purpose of this experiment is to assess the effect
of CLARE on the quality of resulted review paper and on the process by
which the review paper is produced.


\subsubsection{Procedure}

This experiment involves 18 subjects, which are randomly assigned to two
groups: the treatment group which uses CLARE, and the control group which
does not. These two groups are further divided into three (3) study groups,
with three (3) students in each. The experiment will start at around the
half way into the semester and will last for two weeks. It is composed of
four (4) phases: private outlining, private consolidation, group
consolidation, and writing-up. Before the experiment starts, the subjects
will be instructed about how the final output should be like.

\begin{itemize}
  
\item Private outlining: During this phase, both groups are required to
  read the assigned papers. As they do so, they write down key problems,
  ideas, comments, et al. The difference between what the two groups are
  doing is that the treatment group uses CLARE for this activity, while the
  control group does not.
  
\item Private consolidation: At the beginning of this phase, the subjects
  within each study group will distribute to other group members what they
  have generated from the previous phase. Next, they will compare and
  contrast the ideas/questions from their fellow members with those of
  their own, and revise their individual review outlines accordingly.
  Again, the key difference between the two groups is that one does it
  through CLARE, and the other doesn't.
  
\item Group consolidation: At this phase, subjects in each study group
  come together face-to-face to discuss major problems they have
  encountered while reading each other's work. In the control group, one
  member in each study group is assigned to record the discussion result.
  At the end, the discussion log, along with the revised reviews are given
  to a designated member, who is responsible for the final document. For
  the treatment group, this phase is conducted in front of a workstation;
  the discussion log will be recorded online. This session is fixed time
  for both groups, i.e., one class period.
  
\item Writing-up: For both groups, the designated individual integrates
  the discussion results with individual reports into a coherent printed
  document, which, along with all intermediate artifacts from all group
  members, will be turned in to the researcher for evaluation.
\end{itemize}


\subsubsection{Measures} 

Like the previous experiment, three types of measures will be collected on
this experiment: outcome, process, and user satisfaction. The quality of
the review paper will be graded by the instructor and two other experts on
the subject. The evaluation criteria will include completeness,
consistency, integration, level of support evidence for critiques, and
quality of research questions raised. The support documents will also be
taken into consideration during evaluation.


\subsection{Comparison of the Two Experiments}
\label{sec:comparison}

The experiments described above illustrate two example learning tasks CLARE
can support (See Section ~\ref{sec:usage} for other potential CLARE
usages). Despite their similarities, e.g., evaluation measures, these
experiments differ in two important ways: subject and task. First, the two
subject groups consist of two different levels of learners, i.e., graduate
students versus undergraduates. Second, writing a review paper (task II) is
a much more elaborate activity than generating a set of study questions
(task I). The former, for example, requires not only longer time but also
a higher level of integration of the learning materials on hand and what
the learners already know.

The above two differences allow us to answer different empirical questions
regarding the usefulness of CLARE. First, since CLARE was designed to
support advanced learners, we expect that the graduate CLARE users have
significantly more positive experience with the system than the
undergraduate users do, whether or not their task performance reflects this
difference. Second, one of CLARE's main features is its integrated support
for the five levels of learning (see Section ~\ref{sec:activity}). The more
sophisticated a learning task is, the more integration it requires of the
five levels, and the more likely it benefits from using CLARE. Since
writing a review paper is a much more sophisticated task than generating a
set of study questions, we expect that the experiment involving the former
be more likely to have a significant difference in the task performance.


\section{Other Related Work}
\label{sec:related-work}

CLARE represents a confluence of several streams of research.
Pedagogically, our work is based on the theory of meaningful learning
\cite{Ausubel78,Novak84}. The illustrator function of CLARE is built on
such theoretical principles as prior knowledge, subsumption, progressive
differentiation, and integrative reconciliation. CLARE, like CSILE
\cite{Scardamalia91}, follows the constructivist tradition, which asserts
the primacy of the social nature of knowledge, and therefore, of
collaborative learning \cite{Slavin90}. However, it deviates from CSILE,
Intermedia \cite{Yankelovich88}, and other similar tools in its emphasis
and explicit representation of metaknowledge, and using it to help the
learner organize and make sense of specific knowledge.

Representation is fundamental to design and computer science, in
particular, artificial intelligence (AI) \cite{Winograd87}. The AI view of
knowledge representation, however, is extremely varied, covering a broad
range of ontological and epistemological schemes \cite{Swaminathan90}, most
of which have little to do human learning in classroom settings.  The
representational issues CLARE attempts to tackle bear more resemblance with
what has come to be called ``semi-structured'' representations
\cite{Lee90}, for instance, IBIS \cite{Kunz70,Conklin88Gibis}, Toulmin's
rhetorical model \cite{Toulmin84}. In recent years, quite a few such
schemes have been developed in the area of design rationale, e.g.,
\cite{Lee91What,MacLean89Design,Conklin91Process}. More recently, we have
begun to see similar approaches being used learning support environments
(e.g., \cite{Cavalli-Sforza92}). RESRA, in essence, is an extension to
these approaches; its most important novelty is its treatment of such a
scheme not merely as a medium for representing ``other'' things but itself
as a subject of learning and exploration (i.e., metaknowledge).

Since CLARE is a hypertext-based collaborative learning environment, its
design and implementation is shaped by previous work and experience in both
of these areas, in particular, works such as
\cite{Conklin88Gibis,Halasz88Reflections}, and systems like NoteCards
\cite{Halasz87NOTECARDS}, gIBIS \cite{Conklin88Gibis}, and more recently,
Aquanet \cite{MARSHALL91Aquanet}.  CLARE's exploratory functions are a
direct instantiation of the collaborative model provided by EGRET
\cite{csdl-91-02,csdl-92-01,csdl-92-08}.

Empirically, our work was inspired by consistent encouraging findings from
studies on the use of concept maps in science learning \cite{Cliburn90},
and computer-mediated communication in augmenting traditional
classroom-based learning \cite{HILTZ88Collaborative}. Instead of providing
the learner merely with an information access and sharing mechanism, as
many CMC and hypertext systems do, we believe that the introduction of
domain and/or meta structures and facilities to manipulate them can
significantly enhance the usability of those learning support environments.
CLARE represents our first attempt to testing this claim.

\section{Research Plan}
\label{sec:plan}

This research is divided into two phases: system development and
evaluation.  The first phase involves the design, implementation, and
testing of CLARE prototype. The second phase requires conducting two 
experiments planned for evaluating the effectiveness of the CLARE
system. Below are important milestones:

\begin{table}[ht]
  \ls{0.8}
  \begin{center}
    \begin{tabular} {|p{1.5in}|p{4.0in}|} \hline   
      {\bf Dates } & {\bf Milestones}\\ \hline

      January 25 & Proposal Defense. \\ \hline
      
      February 14. & Completing implementation of new functions on top of
      current CLARE prototype. \\ \hline
      
      February 26. & Completing inhouse testing and pilot experiments. \\
      \hline
      
      March 20. & Completion of Phase I experiments. \\ \hline
      
      April 16. & Completion of Phase II experiments. \\ \hline
      
      June 30.  & Completion of draft dissertation. \\ \hline July 5.  &
      Tentative date for dissertation defense. \\ \hline
    \end{tabular}
    \caption{Important Research Milestones}
    \label{tab:milestones}
    \ls{0.9}
  \end{center}
\end{table}

