\section{CLARE: the Computational Environment}
\label{sec:clare}

\subsection{Functional Overview}
\label{sec:design-overview}

CLARE is a computer-based collaborative learning environment based on the
RESRA representational framework. Though the latter forms the conceptual
core of the former, CLARE contains several novelties of its own, including
aggregates (i.e., thread and perspective), a comparator, a hierarchical
illustrator, and a number of exploratory support facilities.  At the user
level, CLARE is a hypertext-based system that provides the following
functions:

\begin{itemize}
\item Supporting distributed collaboration, both synchronous and
  asynchronous;
  
\item Ability to create, update, navigate, and summarize RESRA instances;
  
\item Ability to extend, adapt, and explore RESRA primitives;
  
\item Hierarchy of examples, templates, and reusable domain-specific
  instances for facilitating example-based learning;
  
\item Aggregation in terms of user, artifact (i.e., thread), and
  perspective.
  
\item Ability to compare representations from different individuals about
  the same artifact;
  
\item Incrementally organized knowledge base capturing group learning
  experience; and
  
\item Graphical navigator for overviewing the structure of collaborative
  knowledge base.
\end{itemize}


\subsection{Main Features}
\label{sec:features}

The design of CLARE was driven by the following assumptions:

\begin{itemize}
\item RESRA is the structural basis of the system; in other words, CLARE
  should reflect the five-level model of learning and allow manipulations of
  RESRA objects at both representational and instance levels;
  
\item Representational exploration is an integral part of any metalearning
  tool, including CLARE;
  
\item The ability to create higher-level entities, i.e., the ability to
  aggregate, generalize, and abstract, is essential to meaningful learning,
  and therefore must be supported by CLARE.
  
\item Collaborative learning requires explicit mechanisms for representing
  individual viewpoints and means of comparing, contrasting, and integrating
  them; and
  
\item Examples and templates are important in reducing structural
  uncertainty and are a basis for generating group consensus.
\end{itemize}

Based on these assumptions, CLARE incorporates the following features: a
mode manager, aggregation, an illustrator, a comparator, an explorer, and a
navigator.

\subsubsection{Mode Manager}

At the base level, CLARE provides five interaction modes, which correspond
to the five levels of learning identified in Section ~\ref{sec:activity},
i.e., {\it summarization\/}, {\it evaluation\/}, {\it integration\/}, {\it
argumentation\/}, and {\it construction\/}. In each of the five modes, the
user is allowed to create, update and browse RESRA instances appropriate to
the mode. One may toggle between those modes but, at a given time, only one
mode is active and visible to the user (i.e., via pulldown menu). The
intent of this design was to give the learner a framework to organize their
activities. If the learner needs to, for example, create an {\it
evaluation\/} instance while in the {\it summarization\/} mode, they can do
so either by temporarily switching to the {\it evaluation\/} mode, or using
instance creation function from the generic mode, in which case they are
responsible for explicitly specifying the type of instances to be created.
Alternatively, if such combinations are often called for, the user should
consider using the explorer to add the function to the current mode (see
Section ~\ref{sec:explorer} for more details).

\subsubsection{Aggregation}

As mentioned in Section ~\ref{sec:resra1}, RESRA in itself does not come
with any aggregate. To overcome this deficiency, CLARE defines two of its
own aggregates: {\it thread\/} and {\it perspective\/}. A {\it thread\/} is
a set of RESRA instances that are related, directly or indirectly, to a
{\it base\/} entity, while a {\it base\/} entity is a RESRA instance that
serves as a {\it center\/} of discussions. By default, all {\it source\/}
and {\it problem\/} instances are base entities. The user, however, is free
to add or remove instances from the default set by specifying either a
primitive identifier, such as ``concept'' (i.e., meaning all ``concept''
instances) or an instance identifier, such as ``metalearning''. In CLARE,
threads are indexed by their base entities. They are used not only for
querying and browsing but also as the basic grain for examples. See Section
~\ref{sec:illustrator} for more details.

A {\it perspective \/} defines a set of RESRA instances that share a
consistent pattern of viewing a given artifact, problem, and so forth. It
is generally not as fluid as {\it thread\/}. Unlike threads, which are
dynamic and automatically defined once the {\it base\/} entities are
identified, a perspective must be explicitly described by the learner, and
a name is normally required at the time when it is defined. The perspective
provides a structural basis for facilitating group consensus building.
Typically, a learner holds one perspective with regard to a given {\it
thread\/}, e.g., the implementor or designer perspective. Different
learners, however, may share the same perspective.


\subsubsection{Illustrator}
\label{sec:illustrator}

The illustrator is designed to provide CLARE users with examples and
templates to facilitate the understanding and consistent use of RESRA
primitives. It consists of three libraries: domain, templates, and
examples.

\begin{itemize}
\item Domain-specific instance library. The library contains a set of
  standard RESRA instances, such as concepts, theories, claims, et al, and
  relationships between them. Those instances may not be related to any
  particular artifacts. Rather, they are probably created by the expert on
  the subject (e.g., the course instructor), or extracted from RESRA
  instances generated by previous learners, or a combination of both. Such
  a library may be viewed as the RESRA version of the core knowledge in a
  given domain.  It serves as a point of reference for exploring less
  understood aspects of the subject.
  
\item Template library. As described in Section ~\ref{sec:resra1}, for
  each well-defined artifact type such as survey, conceptual paper,
  empirical report, a RESRA ``template'' can be defined. Such templates
  consist of a set of primitives. For example, an empirical study contains
  instances of such primitives as {\it problem\/}, {\it claim\/}, {\it
  method\/}, {\it evidence\/}, while a survey may contain only {\it
  artifact\/} and {\it claim\/}. Such templates, of course, do not always
  exactly match with the artifacts under concern.  Nevertheless, they are
  ``ideal types'' which provide useful thematic heuristics in orienting the
  learner's attention, guiding the proper use of RESRA, and leading to
  unusual discoveries, such as uncovering implicit {\it problem\/}(s) or
  {\it claim\/}(s) in a research paper.
  
\item Example library. Compared to the preceding two, the example library
  is less formal; instances from this library are typically created by the
  learners themselves. They differ from other instances in their
  typicality, good or bad. Examples are important in enabling the learner
  to learn from other other people's experiences. They also illuminate how
  abstract constructs (e.g., RESRA primitives) are used.  Examples are
  typically identified by threads and indexed by respective base entities.
\end{itemize}

What is common to all three libraries is their dynamic nature: CLARE users
are allowed to add, delete, and modify instances from those libraries; of
course, they can always lookup, view, and navigate them as well.


\subsubsection{Comparator}

To better understand individual differences and similarities and to
facilitate consensus building among group members, CLARE provides a
comparator function which computes a similarity metric, called {\it
similarity score\/}, between any two individual learners, and between that
learner and the group as a whole. The similarity score, whose value range
between 1 and 100, with 100 indicating the most similar, is based on a
number of factors, including the number of instances created, the type and
the size (for entities only) of instances, and the number of references to
the domain, template, and example libraries. The similarity function, i.e.,
the number of factors and the weight of each factor, is customizable.
Typically, similarity scores are computed on per thread basis, though it is
trivial to aggregate multi-thread scores.

In addition to the similarity score, which serves as a high-level
quantitative index to the group view of an artifact, the comparator
also reports the number of RESRA instances by category. From that point,
the learner may ``zoom in'' to individual instances that are of interest.
See Section ~\ref{sec:session} for an example use of the comparator.


\subsubsection{Explorer}
\label{sec:explorer}

CLARE is primarily a metalearning tool for using RESRA metaknowledge
primitives to structure specific knowledge under concern.  Since RESRA is
inherently heuristic (see Section ~\ref{sec:resra1}), its structure and
semantics, which depend heavily on the characteristics of the user group
and the learning task on hand, will inevitably undergo change.  CLARE
provides full support for evolving not only RESRA but also for CLARE's own
extensions, such as aggregates, similarity scores. Its exploratory
functions include:

\begin{itemize}
\item Extending, i.e., adding, deleting, and modifying, predefined RESRA
  primitives;
  
\item Adding, deleting, and modifying the field structure of RESRA
  entities;
  
\item Changing relationships between the five interaction modes and RESRA
  primitives;
  
\item Defining new aggregates;
  
\item Customizing the similarity score function, i.e., adjusting the number
  of factors or their relative weights, and
  
\item Allowing online conversations on above structures.
\end{itemize}

Most of the above functions are directly implemented using the exploratory
type system of EGRET, the platform on which CLARE is built (see
\cite{csdl-92-01} for details).


\subsubsection{Navigator}

CLARE is equipped with a graphical browser, called {\it navigator\/}, which
allows the learner to overview RESRA network structures and ``zoom in'' to
individual objects. This capability is available to both ordinary RESRA
instances, (domain, template, example) library instances, aggregates, and
RESRA primitives. Though providing an advanced graphical interface is not
our primary design goal, we consider the ease of use as an important part
of the system functionality. The navigator, in particular, is important in
easing the ``lost in hyperspace'' problem.


\subsection{Architectural Components}

Architecturally, CLARE is composed of six components: hyperbase server,
agent, EGRET, libraries, interface, and navigator. The relationships
between these components are depicted in Figure ~\ref{fig:arch}. The major
functions provided by each are briefly summarized below:

\begin{itemize}
\item Hyperbase: A database engine which provides persistent stores for
  node and link data. It ensures the data consistency through the
  field-level locking/unlocking, and the state consistency between multiple
  clients through the event mechanism.
  
\item Agent: A distributed, specialized background process(es) that
  maintains necessary global data for realizing high overhead client
  functions. The agent is driven by events generated by the Hyperbase server
  based on client actions.
  
\item EGRET: A generic platform for supporting distributed collaboration.
  It provides infrastructure support (i.e., gtables) and interfacing
  machinery with the hyperbase. EGRET's exploratory type system is the basis
  on which RESRA objects and CLARE's explorer are implemented.
  
\item Libraries: It includes RESRA primitives, domain instance, template,
  and example libraries. They are implemented using the gtable mechanism
  provided by EGRET. An agent is responsible for maintaining the consistency
  of these libraries.
  
\item Interface: It consists of a large set of functions that are organized
  into these groups: the mode manager, aggregation, the illustrator, the
  comparator, and the explorer. It interacts directly with the navigator.
  
\item Navigator: A graphical browser that shows the structure of RESRA
  networks and allows ``zoom-in'' capabilities. It works with all levels of
  RESRA objects, from primitives, templates, examples, to ordinary instances.
\end{itemize}

\begin{figure}[htb]
  \fbox{\centerline{\psfig{figure=Figures/arch.eps,height=3.5in}}}
  \caption{CLARE's Architecture Components}
  \label{fig:arch}
\end{figure}


\subsection{Implementation Environment}
\label{sec:implementation}

The prototype of CLARE is implemented on top of Lucid Emacs, an X Window
based version of the popular GNU Emacs editing environment
\cite{Stallman85}. EGRET, which is runnable on vanilla Emacs, provides
CLARE necessary low-level support. The database server is HyperBase,
developed at University of Aalborg \cite{Wiil90}. The graphical navigator
is implemented using the XView library.


\subsection{A Sample Session with CLARE}
\label{sec:session}

John, Doug, and Lynn were taking a seminar in computer-supported
cooperative work (CSCW) together. Since I had mentioned to them one time
about the greatness of CLARE, they later came back to me and asked whether
they could give the system a try; they were curious about what CLARE could
give them which they couldn't get from other systems. After a 15-minute
demo and showing them a few examples, I asked them to (1) select a paper
from their required reading list; (2) privately (i.e., without looking over
each other's shoulders) summarize and evaluate the paper using CLARE; and
(3) come together to look through what they have created individually, ask
each other questions if necessary, and see whether they can use CLARE to
link together their individual representations to form a consensual view of
the paper.

Two days later, my three friends and I gathered in front of a workstation.
We brought up CLARE, and invoked the comparator on \cite{Kaplan92} -- the
artifact under scrutiny. Figure ~\ref{fig:clare-compare} provides a
high-level index to the difference between what the three had done. The
group similarity score, shown near the bottom of the screen, is 28.4\%,
implying that the three had little in common. The comparator table reports
the number of summarative and evaluative RESRA instances created by each
user; each cell contains two numbers: one is the number of entities and the
other, of relations (inside parentheses). For example, Doug has created 8
summarative entities, 3 summarative links, 3 evaluative entities, and only
2 evaluative links. Note that the similarity scores listed in the table are
two-way scores: since Doug is the current user, he has a similarity score
of 32.1 with Lynn, and 24.7 with John. To illustrate why the scores are so
low, Figure ~\ref{fig:clare-eg1} and ~\ref{fig:clare-eg2} provide two
example {\it problem\/} instances, which are ``problem'' representations of
the paper by John and Doug, respectively. At the first glance of the two
instances, I can hardly tell they were talking about the same artifact!


\begin{figure}[htb]
  \centerline{\psfig{figure=Figures/scr0.ps,height=3.2in}}
  \caption{An Example Similarity Report}
  \label{fig:clare-compare}
\end{figure}


\begin{figure}[htb]
  \centerline{\psfig{figure=Figures/scr1.ps,height=3.2in}}
  \caption{Doug's ``Problem'' Representation of [Kaplan92]}
  \label{fig:clare-eg1}
\end{figure}


\begin{figure}[htb]
  \centerline{\psfig{figure=Figures/scr2.ps,height=3.2in}}
  \caption{John's ``Problem'' Representation of [Kaplan92]}
  \label{fig:clare-eg2}
\end{figure}

Given their differences, my friends spent almost two hours navigating
through each other's work, asking themselves questions, and explaining to
each other why they came up with what did. At the end, they were able to
link their individual views and ideas together and generate something close
to a consensual representation (See Figure ~\ref{fig:cbx} for a skeleton of
the final representation). By going through this process, John came to know
why Doug had treated the example in the paper as an object of study, i.e.,
{\it thing \/}, instead of as {\it evidence\/} to support the authors'
claims.  Similarly, Doug finally realized how Lynn had come up with four
{\it claims\/} instead of just one, as he did.


\subsection{Potential Usages of CLARE}
\label{sec:usage}

CLARE was originally conceived as a collaborative research review system to
be used in advanced learning settings such as graduate seminars
\cite{csdl-92-03}. During the course of our research, however, it has
evolved into a much more generic environment that can support a wide range
of learning tasks, both collaborative and individual-based ones. In this
section, we focus on the collaborative aspect and describe three key areas
in which CLARE might be applied: reviewing, writing, and discussion. One
unique feature of CLARE is that it offers fully integrated support for all
the three activities.

\subsubsection{A Collaborative Review Environment}
\label{sec:coreview}

Reviews not only provide a way to discover defects in products, as used in
engineering design \cite{Freedman90}, but also is a good learning tool
which allows the student to develop critical skills, learn to differentiate
good work from bad ones, learn to relate existing work and from other
people's experience. As such, reviews are commonly employed in classrooms,
for example, book reviews, literature reviews, project reviews, and so
forth. 

In the CLARE environment, reviews are conducted at two levels: {\it
summarization\/} and {\it evaluation\/} (see Section ~\ref{sec:activity}),
with the former typically preceding the latter. To some learners, this
distinction may seem arbitrary and difficult to put into practice. The
assumption behind the CLARE's approach is that whether or not such a
separation lead to more better learning is a question requiring empirical
investigation. CLARE provides a support environment for conducting such
studies. For example, one may hypothesize that the separation of evaluation
from summarization reduce the level of ``free riding'' in collaborative
learning settings. To test this hypothesis, one can design a CLARE-based
experiment which requires that, in one group, learners be allowed to access
and make reference to each other's ongoing work, while in the other,
individual work is kept private until all members finish their
summarization and then, they proceed to the evaluation phase. One might
expect that the second group generate higher-level quality evaluations
because they are barred from taking ``shortcuts'' via merely linking to or
``signing off'' other leaners' work.


\subsubsection{A Collaborative Writing Tool}
\label{sec:co-write}

As a learning activity, collaborative writing (CW) consist of not a
singular task but a number of interrelated tasks, which include
brainstorming, idea organizing, planning, composing, editing, revising, et
al \cite{Baecker93}. Through its {\it construction\/} mode (see Section
~\ref{sec:activity}), CLARE supports primarily the first three, or the early
phase of CW. In a sense, CLARE is an advanced group outliner, which differs
from ordinary outline editors in the following three ways:

\begin{itemize}
\item It is hypertextual instead of linear;

\item It is primarily for collaborative idea exploration, organizing, and
  planning; and most important of all, 

\item It uses RESRA primitives and its own aggregates as a knowledge
  representation scheme.
\end{itemize}

By using a linearization utility which we plan to incorporate into CLARE,
the learner will be able to obtain a linear representation of ``idea webs''
from the system, which serves as a basis for later stages of CW, namely,
composing, editing, and revising.


\subsubsection{A Structured Bulletin-board System}
\label{sec:bbs}

The concept of ``virtual classroom'' is supported mainly through
electronic mail and bulletin-board systems \cite{HILTZ88Collaborative}.
CLARE augments those environments by providing a number of structuring
mechanisms, including RESRA, which allows fine-grained representation of
conversations and discussions, and aggregates (i.e., threads and
perspectives), which allow individual differences and similarities to be
highlighted, compared, contrasted, and integrated. CLARE's argumentation
mode, in particular, is designed to facilitates focused discussions and
deliberations within tightly coupled learning groups.










