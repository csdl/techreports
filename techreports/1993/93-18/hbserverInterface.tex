%%%%%%%%%%%%%%%%%%%%%%%%%%%%%% -*- Mode: Latex -*- %%%%%%%%%%%%%%%%%%%%%%%%%%%%
%% hbserverInterface.tex -- 
%% RCS:            : $Id: hbserverInterface.tex,v 1.20 1995/01/17 20:21:45 cmoore Exp $
%% Author          : Carleton Moore
%% Created On      : Tue Aug 31 10:28:14 1993
%% Last Modified By: Carleton Moore
%% Last Modified On: Tue Jan 17 10:21:08 1995
%% Status          : Unknown
%%%%%%%%%%%%%%%%%%%%%%%%%%%%%%%%%%%%%%%%%%%%%%%%%%%%%%%%%%%%%%%%%%%%%%%%%%%%%%%
%%   Copyright (C) 1993 University of Hawaii
%%%%%%%%%%%%%%%%%%%%%%%%%%%%%%%%%%%%%%%%%%%%%%%%%%%%%%%%%%%%%%%%%%%%%%%%%%%%%%%
%% 
%% History
%% 31-Aug-1993		Carleton Moore	
%%    

\documentstyle [12pt,/group/csdl/tex/definemargins]{article}
      % Unix
% \documentstyle [12pt,definemargins]{article}            % Macintosh
\input{/home/3/dxw/c/tex/psfig}

\begin{document}

\title{HBS Interface} 
\author {Cam Moore, Rosemary Andrada\\
Collaborative Software Development Laboratory\\ 
Department of Information and Computer Sciences\\
University of Hawaii\\
Honolulu, HI 96822\\
(808) 956-3489\\
CSDL-TR-93-18}

\maketitle

\tableofcontents

\newpage 
\section{Motivation}
\label{sec:motivation}

This interface document is intended to aid in our understanding of the
relationship between the hyperbase server (HBS) and the client program
Egret Client System (ECS).  It is to act as the base for discussing
improvements to HBS.  This document is broken up into several sections.
The first gives an overview of HBS.  Then we discuss how to connect to HBS
followed by information about operations available in version 3.2.4.  The
event mechanism is explained.  Then we present tables on all possible
function return values, node field values and operation numbers. We end
with some proposed changes.

\section{Overview of the Hyperbase Server}
\label{sec:overview}
HBS is a multiuser, database server for HyperText applications.  It manages
the nodes and links of the HyperText artifact.  HBS provides a small set of
generic operations.  Clients can use these generic operations for their own
HyperText applications.  The small set of operations also reduces the
bottleneck at the server.

We have significantly expanded the original set of operations provided by
Wiil et al \cite{Wiil90}.  We have added eighteen new operations.  Many of
them are Egret specific operations.  Section \ref{sec:operations} describes
all of the operations available from HBS.
\newpage
\section{Connecting to HBS}
\label{sec:connecting}

The HBS connects to new clients through a TCP/IP socket, a `unique socket'
whose default value is 10008.  The request to connect is made by calling
the Unix system call connect().  The HBS sends three, four-byte integers
through the unique socket.  These numbers are necessary to establish more
pipes for specific types of communication.  Herein, these pipes will be
referred to as the server's write, read and event sockets.  After sending
each four-byte integer the server waits for the client to connect on each
socket.  The first four-byte integer is the server's write socket.
Information that the HBS wants to send to the client will be sent through
this socket.  The second four-byte integer is the HBS's read socket.  All
requests and information must be sent through this socket.  The third
four-byte integer is the event socket.  Event information is sent out by
the HBS on this socket.  Once these three sockets are set up the HBS reads
in the length of the user's name and then the name itself.  The user's name
must be sent on the HBS's read socket and the user's name must end with a
`null' character \footnote{Note the only time a string must be terminated
with a `null' is when that string is the user's name.  All user names in
HBS are `null' terminated.}.  Next it reads the length of the message to be
sent with the event (a four-byte integer) from the read socket.  Next it
reads that many characters from the read socket.  

When a client connects to HBS, HBS will queue all requests from all other
connected clients except the client with the name ``Gagent''.  this is done
to ensure the clients get the correct global state in Egret.  Once the
client has the correct global state they must send the connected op


\begin{figure}[htb]
  \centerline{\psfig{figure=Figures/HB_Connect1.eps,width=2.25in}}
  \caption{{\bf Packet Sent Out by HBS Connect}}
  \label{fig:hb_connect1}
\end{figure}

\begin{figure}[htb]
  \centerline{\psfig{figure=Figures/HB_Connect2.eps,width=2.25in}}
  \caption{{\bf Packet Expected by HBS Connect}}
  \label{fig:hb_connect2}
\end{figure}


\newpage
\section{Available Operations: ver 3.0}
\label{sec:operations}

\subsection{Read}

This operation reads the specified key value of a given entity number.  If
successful, it returns the key value through its write socket.  

The HBS reads the operation READ (the four-byte integer 1) from its read
socket.  Next it reads the entity number (a four-byte integer) from its
read socket.  Next it reads the key number (a four-byte integer).  
Next it reads the length of the message to be sent with the event (a
four-byte integer).  Next it reads that many characters from the read
socket.  The READ operation then sends out the return value (a four-byte
integer) on the HBS's write socket.  Then if the return value is equal to
0(OK) or 351(Locked by other), it sends the length of the data (a four-byte
integer) to its write socket.  Next it sends that many characters to its
write socket.

\begin{figure}[htb]
  \centerline{\psfig{figure=Figures/read1.eps,width=4.5in}}
  \caption{{\bf Packet Expected by Read}}
  \label{fig:read1}
\end{figure}

\begin{figure}[htb]
  \centerline{\psfig{figure=Figures/read2.eps,width=3.0in}}
  \caption{{\bf Packet Sent Out by Read}}
  \label{fig:read2}
\end{figure}


\newpage
\subsection{Write}

This operation writes information to a specific key of a given entity
number.  

The HBS reads the operation WRITE (the four-byte integer 2) from its read
socket.  Next it reads the entity number (a four-byte integer) from its
read socket.  Next it reads the key number (a four-byte integer).  Next it
reads from the read socket the length of the data to be written in the node
(a four-byte integer).  Next it reads that many characters from the read
socket.  Next it reads the length of the message to be sent with the
event (a four-byte integer).  Next it reads that many characters from the
read socket.  The WRITE operation then sends out the return value (a
four-byte integer) on the HBS's write socket.

\begin{figure}[htb]
  \centerline{\psfig{figure=Figures/write1.eps,width=4.5in}}
  \caption{{\bf Packet Expected by Write}}
  \label{fig:write1}
\end{figure}

\begin{figure}[htb]
  \centerline{\psfig{figure=Figures/write2.eps,width=1.25in}}
  \caption{{\bf Packet Returned by Write}}
  \label{fig:write2}
\end{figure}


\newpage
\subsection{Create Node}

This operation creates a new node.  If successful, it returns the ID
number of the newly created node.  These node IDs are always even numbered
and are reused after deletion.  

The HBS reads the operation CREATE NODE (the four-byte integer 5) from its
read socket.  Next it reads the length of the message to be sent with
the event (a four-byte integer).  Next it reads that many characters from
the read socket.  The CREATE NODE operation then sends out the return
value (a four-byte integer) on the HBS's write socket.  If the return value
is equal to 0 it sends the new entity number (a four-byte integer) to its
write socket.

\begin{figure}[htb]
  \centerline{\psfig{figure=Figures/createnode1.eps,width=3.0in}}
  \caption{{\bf Packet Expected by Create Node}}
  \label{fig:CreateNode1}
\end{figure}

\begin{figure}[htb]
  \centerline{\psfig{figure=Figures/createnode2.eps,width=1.5in}}
  \caption{{\bf Packet Sent Out by CreateNode}}
  \label{fig:CreateNode2}
\end{figure}

\newpage
\subsection{Delete}

This operation deletes the given entity (node or link).  Deletion will fail
for nodes if it is linked with another node.  Deletion will fail for links
if its `From' entity still points to it.  This operation ignores the fact that a
node or link may be subscribed to using the event mechanism or that a client
may own a lock on it.  

The HBS reads the operation DELETE (the four-byte integer 6) from its read
socket.  Next it reads the entity number (a four-byte integer) from its
read socket.  Next it reads the length of the message to be sent with
the event (a four-byte integer).  Next it reads that many characters from
the read socket.  The DELETE operation then sends out the return value
(a four-byte integer) on the HBS's write socket.

\begin{figure}[htb]
  \centerline{\psfig{figure=Figures/delete1.eps,width=3.75in}}
  \caption{{\bf Packet Expected by Delete}}
  \label{fig:Delete1}
\end{figure}

\begin{figure}[htb]
  \centerline{\psfig{figure=Figures/write2.eps,width=1.25in}}
  \caption{{\bf Packet Returned by Delete}}
  \label{fig:Delete2}
\end{figure}

\newpage
\subsection{Link}

This operation is a remnant from the original hyperbase.  A more complete
version will be available in HBS 2.1.x.  This operation will do one of two
things.  First, if the `To' entity number is a node, then a link will be
created and the ID number of the newly created link will be returned.  Note
that this link points to the given node, but no node points to this link
yet.  To establish this connection, one would need to call this operation
again in the following context.  If the `From' entity number is a node and
the `To' entity number is a link, then this operation establishes a
connection between the two.  That is, the node has as one of its outgoing
links the given existent link.

The HBS reads the operation LINK (the four-byte integer 7) from its read
socket.  Next it reads the `From' entity number (a four-byte integer) from
its read socket.  Next it reads the `To' entity number (a four-byte
integer) from its read socket.  Next it reads the length of the
message to be sent with the event (a four-byte integer).  Next it reads
that many characters from the read socket.  The LINK operation then
sends out the return value (a four-byte integer) on the HBS's write socket.
Then if the return value is equal to 0 it sends the entity number of `From'
entity.  If the `To' entity is a {\bf Node} then a new {\bf Link} is
created and the {\bf Link's} entity number, which is always odd, is
returned.

\begin{figure}[htb]
  \centerline{\psfig{figure=Figures/link1.eps,width=4.5in}}
  \caption{{\bf Packet Expected by Link}}
  \label{fig:Link1}
\end{figure}

\begin{figure}[htb]
  \centerline{\psfig{figure=Figures/link2.eps,width=1.5in}}
  \caption{{\bf Packet Sent Out by Link, If a new Link is created}}
  \label{fig:Link2}
\end{figure}

\newpage
\subsection{Move Link}

This operation changes the destination node for an existing link.  

The HBS reads the operation MOVE LINK (the four-byte integer 8) from its
read socket.  Next it reads the {\bf Link} number (a four-byte integer)
from its read socket.  Next it reads the `To' {\bf Node} number (a
four-byte integer) from its read socket.  Next it reads the length of
the message to be sent with the event (a four-byte integer).  Next it reads
that many characters from the read socket.  The MOVE LINK operation then
sends out the return value (a four-byte integer) on the HBS's write socket.

\begin{figure}[htb]
  \centerline{\psfig{figure=Figures/movelink1.eps,width=4.5in}}
  \caption{{\bf Packet Expected by MoveLink}}
  \label{fig:MoveLink1}
\end{figure}


\begin{figure}[htb]
  \centerline{\psfig{figure=Figures/write2.eps,width=1.25in}}
  \caption{{\bf Packet Returned by MoveLink}}
  \label{fig:MoveLink2}
\end{figure}

\newpage
\subsection{Remove Link}

This operation destroys the conceptual connection between a given node and
one of its outgoing links.  Note that no nodes or links are deleted.  Use
Delete for deleting nodes and links.  

The HBS reads the operation REMOVE LINK (the four-byte integer 9) from its
read socket.  Next it reads the {\bf Node} number (a four-byte integer)
from its read socket.  Next it reads the {\bf Link} number (a four-byte
integer) from its read socket.  Next it reads the length of the
message to be sent with the event (a four-byte integer).  Next it reads
that many characters from the read socket.  The REMOVE LINK operation
then sends out the return value (a four-byte integer) on the HBS's write
socket.

\begin{figure}[htb]
  \centerline{\psfig{figure=Figures/removelink1.eps,width=4.5in}}
  \caption{{\bf Packet Expected by ReMoveLink}}
  \label{fig:ReMoveLink1}
\end{figure}


\begin{figure}[htb]
  \centerline{\psfig{figure=Figures/write2.eps,width=1.25in}}
  \caption{{\bf Packet Returned by RemoveLink}}
  \label{fig:ReMoveLink2}
\end{figure}

\newpage
\subsection{Event}

This operation allows the client to subscribe to a particular event
involving an entity and the operation on that entity at a given key.  Each
combination of an event must be subscribed to separately.  However, it is
possible to subscribe to mass entities, operations or keys using the `ALL'
code represented by the number 0.  

The HBS reads the operation EVENT (the four-byte integer 10) from its read
socket.  Next it reads the entity number (a four-byte integer) from its
read socket.  Next it reads the operation number (a four-byte integer).
Next it reads the key number (an integer).  Next it reads the length
of the message to be sent with the event (a four-byte integer).  Next it
reads that many characters from the read socket.  The EVENT operation
then sends out the return value (a four-byte integer) on the HBS's write
socket.

\begin{figure}[htb]
  \centerline{\psfig{figure=Figures/event1.eps,width=5.25in}}
  \caption{{\bf Packet Expected by Event}}
  \label{fig:Event1}
\end{figure}

\begin{figure}[htb]
  \centerline{\psfig{figure=Figures/write2.eps,width=1.25in}}
  \caption{{\bf Packet Returned by Event}}
  \label{fig:Event2}
\end{figure}

\newpage
\subsection{UnEvent}

This operation allows the client to unsubscribe himself from a particular
event involving an entity and the operation on that entity at a given key.
The one restriction here is that once a client subscribes to all nodes for
any combination of operation or key, he can only unsubscribe himself from
all nodes.  Unsubscribing in this fashion will not cause an error.  It will
simply be ignored.  

The HBS reads the operation UNEVENT (the four-byte integer 11) from its
read socket.  Next it reads the entity number (a four-byte integer) from
its read socket.  Next it reads the operation number (a four-byte integer).
Next it reads the key number (an integer).  Next it reads the length
of the message to be sent with the event (a four-byte integer).  Next it
reads that many characters from the read socket.  The UNEVENT operation
then sends out the return value (a four-byte integer) on the HBS's write
socket.

\begin{figure}[htb]
  \centerline{\psfig{figure=Figures/unevent1.eps,width=5.25in}}
  \caption{{\bf Packet Expected by UnEvent}}
  \label{fig:UnEvent1}
\end{figure}

\begin{figure}[htb]
  \centerline{\psfig{figure=Figures/write2.eps,width=1.25in}}
  \caption{{\bf Packet Returned by UnEvent}}
  \label{fig:UnEvent2}
\end{figure}

\newpage
\subsection{Show Event}

This operation displays a list of clients subscribing to a particular event
involving an entity and the operation on that entity at a given key.  

The HBS reads the operation SHOW EVENT (the four-byte integer 12) from its
read socket.  Next it reads the entity number (a four-byte integer) from
its read socket.  Next it reads the operation number (a four-byte integer).
Next it reads the key number (an integer).  Next it reads the length
of the message to be sent with the event (a four-byte integer).  Next it
reads that many characters from the read socket.  The SHOW EVENT
operation then sends out the return value (an integer) on the HBS's write
socket.  Then if the return value is equal to 0 it sends the length of the
string containing a list of users subscribed to the event (a four-byte
integer).  Next it sends that many characters to the write socket.  Note
that the string contains a list of one or more names delimited by a newline
character.

\begin{figure}[htb]
  \centerline{\psfig{figure=Figures/showevent1.eps,width=5.25in}}
  \caption{{\bf Packet Expected by ShowEvent}}
  \label{fig:ShowEvent1}
\end{figure}

\begin{figure}[htb]
  \centerline{\psfig{figure=Figures/read2.eps,width=3.0in}}
  \caption{{\bf Packet Sent Out by ShowEvent}}
  \label{fig:ShowEvent2}
\end{figure}


\newpage
\subsection{Lock}

This operation allows the client to lock any writable key at the given
entity.  Once a client obtains a lock, no other client may perform the
write operation on the locked key.  

The HBS reads the operation LOCK (the four-byte integer 13) from its read
socket.  Next it reads the entity number (a four-byte integer) from its
read socket.  Next it reads the key number (a four-byte integer).  
Next it reads the length of the message to be sent with the event (a
four-byte integer).  Next it reads that many characters from the read
socket.  The LOCK operation then sends out the return value (a four-byte
integer) on the HBS's write socket.

\begin{figure}[htb]
  \centerline{\psfig{figure=Figures/lock1.eps,width=5.25in}}
  \caption{{\bf Packet Expected by Lock}}
  \label{fig:Lock1}
\end{figure}

\begin{figure}[htb]
  \centerline{\psfig{figure=Figures/write2.eps,width=1.25in}}
  \caption{{\bf Packet Returned by Lock}}
  \label{fig:Lock2}
\end{figure}

\newpage
\subsection{UnLock}

This operation allows the client to unlock a particular key at the given
entity.  

The HBS reads the operation UNLOCK (the four-byte integer 14) from its read
socket.  Next it reads the entity number (a four-byte integer) from its
read socket.  Next it reads the key number (a four-byte integer).  Next it
reads the length of the message to be sent with the event (a 
four-byte integer).  Next it reads that many characters from the read
socket.  The UNLOCK operation then sends out the return value (a
four-byte integer) on the HBS's write socket.

\begin{figure}[htb]
  \centerline{\psfig{figure=Figures/unlock1.eps,width=5.25in}}
  \caption{{\bf Packet Expected by UnLock}}
  \label{fig:UnLock1}
\end{figure}

\begin{figure}[htb]
  \centerline{\psfig{figure=Figures/write2.eps,width=1.25in}}
  \caption{{\bf Packet Returned by UnLock}}
  \label{fig:UnLock2}
\end{figure}


\newpage
\subsection{Show Lock}

This operation displays the client that currently holds a lock on a
particular key at a given entity.  

The HBS reads the operation SHOW LOCK (the four-byte integer 15) from its
read socket.  Next it reads the entity number (a four-byte integer) from
its read socket.  Next it reads the key number (a four-byte integer).  Next
it reads the length of the message to be sent with the event (a four-byte
integer).  Next it reads that many characters from the read socket.  The
SHOW LOCK operation then sends out the return value (a four-byte integer)
on the HBS's write socket.  Then it sends the length of the user's name or
an error message (a four-byte integer) to the write socket.  Next it sends
that many characters to the write socket.

\begin{figure}[htb]
  \centerline{\psfig{figure=Figures/showlock1.eps,width=5.25in}}
  \caption{{\bf Packet Expected by ShowLock}}
  \label{fig:ShowLock1}
\end{figure}

\begin{figure}[htb]
  \centerline{\psfig{figure=Figures/read2.eps,width=2.25in}}
  \caption{{\bf Packet Sent Out by ShowLock}}
  \label{fig:ShowLock2}
\end{figure}


\newpage
\subsection{Disconnect}

This operation disconnects the client from the server process.  In doing so,
the client is automatically unsubscribed from all events and all the client's
locks are released.  

The HBS reads the operation DISCONNECT (the four-byte integer 17) from its
read socket.  Next it reads the length of the message to be sent with
the event (a four-byte integer).  Next it reads that many characters from
the read socket.  The DISCONNECT operation then sends out the return
value (a four-byte integer) on the HBS's write socket.

\begin{figure}[htb]
  \centerline{\psfig{figure=Figures/disconnect1.eps,width=3.0in}}
  \caption{{\bf Packet Expected by Disconnect}}
  \label{fig:Disconnect1}
\end{figure}


\begin{figure}[htb]
  \centerline{\psfig{figure=Figures/write2.eps,width=1.25in}}
  \caption{{\bf Packet Returned by Disconnect}}
  \label{fig:Disconnect2}
\end{figure}


\newpage
\subsection{Browse}

This operation allows the client to obtain a list of all node or link IDs
currently in the database.  The client specifies the node type (data or
link) and a list of nodes or links is returned respectively.  

The HBS reads the operation BROWSE (the four-byte integer 18) from its read
socket.  Next it reads the type of entity (a four-byte integer, 0 for {\bf
Node}, 1 for {\bf Link}).  Next it reads the length of the message to
be sent with the event (a four-byte integer).  Next it reads that many
characters from the read socket.  The BROWSE operation then sends out
the return value (a four-byte integer) on the HBS's write socket.  Then if
the return value is equal to 0 it sends the length of the string of
entities (a four-byte integer).  Next it sends that many four-byte integers
followed by a four-byte zero to the write socket.

\begin{figure}[htb]
  \centerline{\psfig{figure=Figures/browse1.eps,width=3.75in}}
  \caption{{\bf Packet Expected by Browse}}
  \label{fig:Browse1}
\end{figure}

\begin{figure}[htb]
  \centerline{\psfig{figure=Figures/browse2.eps,width=3.0in}}
  \caption{{\bf Packet Sent Out by Browse}}
  \label{fig:Browse2}
\end{figure}


\newpage
\subsection{Shut Down Server}

This operation allows a client to shut down the server.  However, the
server will only shut down if the requesting client is the last client
connected to the server or send an error code if unsuccessful.  

The HBS reads the operation SHUT DOWN (the four-byte integer 19) from its
read socket.  The SHUT DOWN operation then sends out the return value on
the HBS's write socket.

\begin{itemize}
\item{This operation is a new addition of version 2.0}
\end{itemize}

\begin{figure}[htb]
  \centerline{\psfig{figure=Figures/shutdown1.eps,width=1.25in}}
  \caption{{\bf Packet Expected by ShutDown}}
  \label{fig:ShutDown1}
\end{figure}


\begin{figure}[htb]
  \centerline{\psfig{figure=Figures/write2.eps,width=1.25in}}
  \caption{{\bf Packet Returned by ShutDown}}
  \label{fig:ShutDown2}
\end{figure}

\newpage
\subsection{Show Users}

This operation returns the number of clients currently connected followed
by a list of names of those clients.  

The HBS reads the operation SHOW USERS (the four-byte integer 20) from its
read socket. Next it reads the length of the message to be sent with
the event (a four-byte integer).  Next it reads that many characters from
the read socket.  The SHOW USERS operation then sends out the return
value (a four-byte integer) on the HBS's write socket.  Next HBS sends the
user count (a four-byte integer) to its write socket.  For each user, it
sends the length of the user name (a four-byte integer) and that many
characters to its write socket.

\begin{itemize}
\item{This operation is a new addition of version 2.0}
\end{itemize}

\begin{figure}[htb]
  \centerline{\psfig{figure=Figures/showusers1.eps,width=3.0in}}
  \caption{{\bf Packet Expected by ShowUsers}}
  \label{fig:ShowUsers1}
\end{figure}

\begin{figure}[htb]
  \centerline{\psfig{figure=Figures/showusers2.eps,width=4.5in}}
  \caption{{\bf Packet Sent Out by ShowUsers}}
  \label{fig:ShowUsers2}
\end{figure}

\newpage
\subsection{Append}

This operation appends text to the data field in a given entity.  

The HBS reads the operation APPEND (the four-byte integer 21) from its read
socket.  Next it reads the entity number (a four-byte integer) from its
read socket.  Next it reads from the read socket the length of the data to
be appended to the node (a four-byte integer).  Next it reads that many
characters from the read socket.  Next it reads the length of the
message to be sent with the event (a four-byte integer).  Next it reads
that many characters from the read socket.  The APPEND operation then
sends out the return value (a four-byte integer) on the HBS's write socket.

\begin{itemize}
\item{This operation is a new addition of version 2.0}
\end{itemize}

\begin{figure}[htb]
  \centerline{\psfig{figure=Figures/append1.eps,width=3.75in}}
  \caption{{\bf Packet Expected by Append}}
  \label{fig:append1}
\end{figure}


\begin{figure}[htb]
  \centerline{\psfig{figure=Figures/write2.eps,width=1.25in}}
  \caption{{\bf Packet Returned by Append}}
  \label{fig:Append2}
\end{figure}

\newpage
\subsection{Create Node with Name}

This operation creates a new node and writes a given name to the name
field.  If successful, it returns the even-numbered ID of the newly
created node.  

The HBS reads the operation CREATE\_NODE\_W\_NAME (the four-byte integer
24) from its read socket. Next it reads the length of the name (a four-byte
integer) to be written to the name field.  Then it reads in that many
characters from the read socket.  Next it reads the length of the
message to be sent with the event (a four-byte integer).  Next it read that
many characters from the read socket.  The CREATE\_NODE\_W\_NAME
operation then sends out the return value (a four-byte integer) on the
HBS's write socket.  If the return value is equal to 0 it sends the new
entity number (a four-byte integer) to its write socket.

\begin{figure}[htb]
  \centerline{\psfig{figure=Figures/createnodewithname1.eps,width=5.25in}}
  \caption{{\bf Packet Expected by CreateNodeWithName}}
  \label{fig:CreateNodeWithName1}
\end{figure}

\begin{figure}[htb]
  \centerline{\psfig{figure=Figures/createnode2.eps,width=1.5in}}
  \caption{{\bf Packet Returned by CreateNodeWithName}}
  \label{fig:CreateNodeWithName2}
\end{figure}


\newpage
\subsection{Create Link with Name}

This operation creates a new link, writes a given name to its name field
and establishes a connection with its `From' node and its `To' node.  If
successful, it returns the odd-numbered ID of the newly created link.  

The HBS reads the operation CREATE\_LINK\_W\_NAME (the four-byte integer
25) from its read socket.  Next it reads the `From' entity number (a
four-byte integer) from its read socket.  Next it reads the `To' entity
number (a four-byte integer) from its read socket. Next it reads the length
of the name (a four-byte integer) to be written to the name field.  Next it
reads that many characters from the read socket.  Next it reads the
length of the message to be sent with the event (a four-byte integer).
Next it reads that many characters from the read socket.  The
CREATE\_LINK\_W\_\_NAME operation then sends out the return value (a
four-byte integer) on the HBS's write socket.  Then if the return value is
equal to 0 it sends the entity number of the newly created link.

\begin{figure}[htb]
  \centerline{\psfig{figure=Figures/completelink1.eps,width=4.5in}}
  \caption{{\bf Packet Expected by Create Link with Name}}
  \label{fig:completelink1}
\end{figure}


\begin{figure}[htb]
  \centerline{\psfig{figure=Figures/link2.eps,width=1.5in}}
  \caption{{\bf Packet Returned by Create Link with Name}}
  \label{fig:completelink2}
\end{figure}

\subsection{Message}

This operation allows a client to send a message to all clients that have
subscribed to the events for the particular node and key.

The HBS reads the operation MESSAGE (the four-byte integer 26) from its
read socket.  Next it reads the entity number (a four-byte integer), the
key (a four-byte integer), the length of the message (a four-byte integer)
and finally message length bytes (the message) from the read socket.  The
HBS creates a MESSAGE event and sends it to all clients that have
subscribed to events for the given node and key. The HBS returns OK through
the write socket.

\subsection{Connected}

This operation tells the HBS to listen to other clients and end the special
connect mode.

The HBS reads the operation CONNECTED (the four-byte integer 27) from the
read socket.  It then reads in the length of the message (a four-byte
integer) and that many bytes (the message) from the read socket.  The HBS
returns OK on the write socket.

\subsection{Create Node with Name and Data}

This operation creates a new node, writes a given name to the name field
and writes the given data to the data field.  If successful, it returns the
even-numbered ID of the newly created node.

The HBS reads the operation CREATE\_NAME\_DATA (the four-byte integer 28)
from its read socket. Next it reads the length of the name (a four-byte
integer) to be written to the name field.  Then it reads in that many
characters from the read socket.  Next it reads the length of the data to
be placed in the node (a four-byte integer).  Next it read that many
characters from the read socket.  Next it reads the length of the message
to be sent with the event (a four-byte integer).  Next it read that many
characters from the read socket.  The CREATE\_NAME\_DATA operation then
sends out the return value (a four-byte integer) on the HBS's write socket.
If the return value is equal to 0 it sends the new entity number (a
four-byte integer) to its write socket.

\subsection{Create Node with Name, Data and Lock}

This operation creates a new node, writes a given name to the name
field, writes the given data to the data field and acquires the lock for
the data field.  If successful, it returns the even-numbered ID of the newly
created node.  

The HBS reads the operation CREATE\_NAME\_DATA\_LOCK (the four-byte integer
29) from its read socket. Next it reads the length of the name (a four-byte
integer) to be written to the name field.  Then it reads in that many
characters from the read socket.  Next it reads the length of the data to
be placed in the node (a four-byte integer).  Next it read that many
characters from the read socket.  Next it reads the length of the message
to be sent with the event (a four-byte integer).  Next it read that many
characters from the read socket.  The CREATE\_NAME\_DATA\_LOCK operation
then sends out the return value (a four-byte integer) on the HBS's write
socket.  It acquires the lock on the data field.  If the return value is
equal to 0 it sends the new entity number (a four-byte integer) to its
write socket.

\subsection{Delete Link}

This operation deletes a link and removes the reference to that link from
the data node that points to the link.

The HBS reads in the operation DELETE\_LINK (the four-byte integer 31) from
the read socket.  It then reads in the source node number (a four-byte
integer), the link node number (a four-byte integer), the length of the
message (a four-byte integer), then message length bytes (the message) from
the read socket.  The HBS sends out the return value (a four-byte integer)
out the write socket.


\subsection{Create a Link to a New Node}

This operation creates a new node, with the supplied name and data, then
creates a new link, with the supplied name, to that new node from the
supplied source node.

The HBS reads in the operation LINK\_NEW\_NODE (the four-byte integer 32)
from the read socket.  It then read in the source node number (a four-byte
integer), the length of the new node's name (a four-byte integer), that
many bytes (the new node's name), the length of the new link's name (a
four-byte integer), that many bytes (the new link's name), the length of
the data for the new node (a four-byte integer), that many bytes (the new
node's data), the length of the message (a four-byte integer), then that
many bytes (the message) all from the read socket. The HBS sends out the
return value (a four-byte integer) out the write socket.  If the return
value is zero (every thing was successful) then the HBS sends out the new
link ID (a four-byte integer) and the new node ID (a four-byte integer) to
the write socket.


\subsection{Create a Link to a New Locked Node}

This operation creates a new node, with the supplied name and data, then
creates a new link, with the supplied name, to that new node from the
supplied source node, then it locks the new node.

The HBS reads in the operation LINK\_NEW\_NODE (the four-byte integer 33)
from the read socket.  It then read in the source node number (a four-byte
integer), the length of the new node's name (a four-byte integer), that
many bytes (the new node's name), the length of the new link's name (a
four-byte integer), that many bytes (the new link's name), the length of
the data for the new node (a four-byte integer), that many bytes (the new
node's data), the length of the message (a four-byte integer), then that
many bytes (the message) all from the read socket. The HBS sends out the
return value (a four-byte integer) out the write socket.  If the return
value is zero (every thing was successful) then the HBS sends out the new
link ID (a four-byte integer) and the new node ID (a four-byte integer) to
the write socket.

\subsection{Write and Unlock the Node}

This operation writes out to disk the field of the node determined by the
key.  It then unlocks that field of the node.

The HBS reads the operation WRITE\_UNLOCK (the four-byte integer 34) from
its read socket.  Next it reads the entity number (a four-byte integer)
from its read socket.  Next it reads the key number (a four-byte integer).
Next it reads from the read socket the length of the data to be written in
the node (a four-byte integer).  Next it reads that many characters from
the read socket.  Next it reads the length of the message to be sent with
the event (a four-byte integer).  Next it reads that many characters from
the read socket.  The WRITE operation then sends out the return value (a
four-byte integer) on the HBS's write socket.

\newpage
\section{Events}

Once a client subscribes to Events, by using the EVENT operation, the HBS
will automatically send the client events.  The HBS screens all activity in
the Hyperbase and sends the client all events that the client has
subscribed to.  The HBS sends the events to the event socket for that
client.  The client then just has to read from the socket to get the event.
If the HBS sends more than one event to the client, the events just form a
queue in the socket.  The format of each event is as follows: the user's
name (a `null' terminated string), the entity number on which the event
occurred (a four-byte integer), the operation number (a four-byte integer),
the key on which the operation occurred (a four-byte integer), the length of
a message (a four-byte integer), the message from the user who performed
the operation.



\begin{figure}[htb]
  \centerline{\psfig{figure=Figures/event2.eps,width=4.5in}}
  \caption{{\bf Event Packet sent out HBS}}
  \label{fig:Event3}
\end{figure}


\section{Operations}
Figure \ref{ops} list all possible Operations
\small

\begin{figure}[htpb]
  \begin{center}
    \begin{tabular} {|c|l|} \hline
      \multicolumn{2}{|c|}{{\bf Operations}} \\ \hline {\em Operation
      Number} & {Operation}\\  \hline \hline
      0 & All Operations (for use with Lock/Event only)\\ \hline
      1 & Read \\ \hline
      2 & Write \\ \hline
      3 & Data Node Read (Not fully implemented) \\ \hline
      4 & Data Node Write (Not fully implemented) \\ \hline
      5 & Create Node \\ \hline
      6 & Delete \\ \hline
      7 & Link \\ \hline
      8 & Move Link \\ \hline
      9 & Remove Link \\ \hline
      10 & Subscribe to Events \\ \hline
      11 & Unsubscribe from Events \\ \hline
      12 & Show Event subscriptions \\ \hline
      13 & Lock a Key(s) of a Node \\ \hline
      14 & Unlock a Key(s) of a Node \\ \hline
      15 & Show the locks on a Key(s) of a Node \\ \hline
      16 & Connect to the Server \\ \hline
      17 & Disconnect from the Server \\ \hline
      18 & Browse the Nodes or Links \\ \hline
      19 & Shut down the Server \\ \hline
      20 & Show Users \\ \hline
      21 & Append data to a Data Node \\ \hline
      22 & Link Node Read (Not fully implemented) \\ \hline
      23 & Link Node Write (Not fully implemented) \\ \hline
      24 & Create Node With Name \\ \hline
      25 & Create Link With Name \\ \hline
      26 & Message \\ \hline
      27 & Connected \\ \hline
      28 & Create with Name and Data \\ \hline
      29 & Create with Name, Data and Lock \\ \hline
      30 & Abnomal Disconnect \\ \hline
      31 & Delete Link \\ \hline
      32 & Create Link to New Node \\ \hline
      33 & Create Link to New Locked Node \\ \hline
      34 & Write Contents and Unlock \\ \hline
    \end{tabular}
  \end{center}
  \caption{\label{ops}Operations. }
\end{figure}
\normalsize

\section{Key Values}

Figure \ref{keyVal} list all possible key values
\small

\begin{figure}[htpb]
  \begin{center}
    \begin{tabular} {|c|l|} \hline
      \multicolumn{2}{|c|}{{\bf Key Values}} \\ \hline {\em Key
      Value} & {\em Meaning}\\  \hline 
      0 & All Keys (for use with Lock/Event only) \\ \hline \hline
      \multicolumn{2}{|c|}{{User Defined Keys for Data Nodes}} \\ \hline
      1 & Geometry  \\ \hline
      2 & Font  \\ \hline
      3 & Node Last Modified Date  \\ \hline
      4 & Node Last Modified By  \\ \hline
      5 & Node Created Date  \\ \hline
      6 & Node Created By  \\ \hline
      7 & Node Name  \\ \hline
      8 & Node Version  \\ \hline
      9 & Node Reference Number  \\ \hline \hline
      \multicolumn{2}{|c|}{{Server Defined Keys for Data Nodes}} \\ \hline
      512 & All Data Node elements  \\ \hline
      513 & Data Node Number  \\ \hline
      514 & Data Node Data Size  \\ \hline
      515 & Number of Links to the Data Node  \\ \hline
      516 & Array of Links leaving the Data Node \\ \hline
      517 & The Data stored in the Data Node \\ \hline \hline
      \multicolumn{2}{|c|}{{User Defined Keys for Link Nodes}} \\ \hline
      256 & Link Last Modified Date  \\ \hline
      257 & Link Last Modified By  \\ \hline
      258 & Link Created Date  \\ \hline
      259 & Link Created By  \\ \hline
      260 & Link Name  \\ \hline
      261 & Link Reference Number  \\ \hline \hline
      \multicolumn{2}{|c|}{{Server Defined Keys for Link Nodes}} \\ \hline
      1024 & All Link Node elements \\ \hline
      1025 & Link Node Number \\ \hline
      1026 & Number of Data Nodes that point to the Link Node \\ \hline
      1027 & The Data Node Number the Link points to \\ \hline
    \end{tabular}
  \end{center}
  \caption{\label{keyVal}Possible Key Values. }
\end{figure}
\normalsize


\section{Return Values}

Figure \ref{returnVal} lists all possible return values.
\small

\begin{figure}[htpb]
  \begin{center}
    \begin{tabular} {|c|l|} \hline
      \multicolumn{2}{|c|}{{\bf Return Values}} \\ \hline {\em Return
      Value} & {\em Meaning}\\ \hline 
      0 & Okay \\ \hline 
      208 & Direct write not allowed \\ \hline 
      207 & Link not found \\ \hline 
      206 & Can't use data keys in link node \\ \hline 
      205 & Can't use link keys in data node \\ \hline 
      204 & Wrong arguments in call \\ \hline 
      203 & Unable to open entity \\ \hline 
      202 & Pointed entity not found \\ \hline 
      201 & key not recognized \\ \hline
      -201 & Could not open/close entity \\ \hline 
      -202 & Entity not found \\ \hline 
      -203 & Unable to write links \\ \hline 
      -204 & There are still pointers to entity \\ \hline 
      301 & No event \\ \hline 
      302 & No send event \\ \hline 
      350 & Locked \\ \hline 
      351 & Locked by other \\ \hline 
      352 & Not locked \\ \hline 
      353 & Not all nodes \\ \hline 
      356 & Invalid Key Number \\ \hline 
      357 & Invalid Lock Key \\ \hline 
      401 & HBS is already connected \\ \hline 
      -402 & There is no machine by that name, gethostbyname failed \\
      \hline 
      -403 & The socket call failed, could not make a file descriptor \\
      \hline 
      -404 & Could not connect to the host with that port, or no more \\
           & room for clients at server \\ \hline 
      -405 & Can't disconnect from server \\ \hline 
      -406 & fcntl() can't change readfd \\ \hline
      -407 & The bind call failed \\ \hline 
      -408 & The call getsockname failed \\ \hline 
      -411 & There is no event in event buffer or an error \\ \hline 
      -412 & Can't shut down server \\ \hline 
      -502 & No such key or wrong key \\ 
      -503 & \\ \hline 
      -504 & No such operation \\ \hline 
      -505 & No such type or wrong type \\ 
      -507 & \\ \hline 
      -506 & Already connected \\ \hline 
      -508 & Not disconnected \\ \hline
    \end{tabular}
  \end{center}
  \caption{\label{returnVal}Possible Return Values. }
\end{figure}
\normalsize


\newpage
\section{Proposed Changes}




\bibliography{hbserverInterface}
\bibliographystyle{plain}



\end{document}
