%%%%%%%%%%%%%%%%%%%%%%%%%%%%%% -*- Mode: Latex -*- %%%%%%%%%%%%%%%%%%%%%%%%%%%%
%% ieee93-conclusions.tex -- 
%% RCS:            : $Id: ieee93-conclusions.tex,v 1.1 1993/12/27 10:43:56 johnson Exp johnson $
%% Author          : Philip Johnson
%% Created On      : Thu Dec 16 08:43:32 1993
%% Last Modified By: Philip Johnson
%% Last Modified On: Fri Nov 25 09:25:18 1994
%% Status          : Unknown
%%%%%%%%%%%%%%%%%%%%%%%%%%%%%%%%%%%%%%%%%%%%%%%%%%%%%%%%%%%%%%%%%%%%%%%%%%%%%%%
%%   Copyright (C) 1993 University of Hawaii
%%%%%%%%%%%%%%%%%%%%%%%%%%%%%%%%%%%%%%%%%%%%%%%%%%%%%%%%%%%%%%%%%%%%%%%%%%%%%%%
%% 
%% History
%% 16-Dec-1993		Philip Johnson	
%%    

\section{Design for Instrumentation}
\label{sec:design}

In conclusion, perhaps the most important lesson learned from our
experiences with CSRS is this: one must design with instrumentation
explicitly in mind, if one is to gather useful empirical data from a
computer-mediated process.  For example, while the use of hypertext
provides certain representational benefits, it adds most value by revealing
and tracking the user's focus of attention.  In future, we hope to extend
our instrumentation for this crucial measure through the use of
eye-tracking devices.  This instrumentation will reveal which individual
symbol the user fixates upon with a precision of milliseconds.  This will
provide an entirely new, qualitatively different representation of reviewer
activity, and enable correspondingly new experimentation and insight into
the review analysis process.

Our design experiences have also convinced us that traditional FTR methods,
such as Fagan's Code Inspection, are designed for and inextricably bound to
a non-computer supported context.  As formal technical review becomes
increasingly computer-mediated, it is essential to devise new methods that
exploit the strengths and minimize the weaknesses of this entirely
different context for collaboration. For example, virtually all manual
review methods embrace the edict to ``raise issues, don't resolve them.''
We believe that this may only be appropriate within a synchronous,
face-to-face group process. In FTArm, entwining
issue generation, discussion, and resolution is natural and appears to
provide significant advantages.

On the other hand, our experiences have also convinced us that while FTArm
is a powerful method, it has weaknesses as well as strengths.  For example,
its asynchronous nature is helpful in reducing the cost intrinsic to
synchronous meetings, but some organizations may require more
``synchronicity'' in their reviews than FTArm provides.  FTArm is also a
complex method that is not well-suited to initial adoption of
computer-supported FTR by an organization.  CSRS includes an FTR method
definition language so that organizations can design methods with tradeoffs
appropriate to their own context.


By eliminating the cost of FTR measurement, CSRS can serve as
infrastructure for a new paradigm of research and collaboration in the
software engineering community.  We are currently negotiating technology
transfer arrangements with several industry organizations.  A goal of this
process is to validate our design and instrumentation within several
organizational contexts, and to learn the requirements for unaided adoption
of this technology.  If you are interested in participating in our research
on computer-supported formal technical review, please visit our World Wide
Web home page at ``http://www.ics.hawaii.edu/$\sim$csdl/csrs''.  


 



