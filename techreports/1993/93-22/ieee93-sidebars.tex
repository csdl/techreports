%%%%%%%%%%%%%%%%%%%%%%%%%%%%%% -*- Mode: Latex -*- %%%%%%%%%%%%%%%%%%%%%%%%%%%%
%% ieee93-sidebars.tex -- 
%% RCS:            : $Id: ieee93-sidebars.tex,v 1.5 93/12/27 08:44:54 johnson Exp Locker: johnson $
%% Author          : Philip Johnson
%% Created On      : Wed Dec  8 10:04:36 1993
%% Last Modified By: Philip Johnson
%% Last Modified On: Wed Nov 23 09:03:57 1994
%% Status          : Unknown
%%%%%%%%%%%%%%%%%%%%%%%%%%%%%%%%%%%%%%%%%%%%%%%%%%%%%%%%%%%%%%%%%%%%%%%%%%%%%%%
%%   Copyright (C) 1993 University of Hawaii
%%%%%%%%%%%%%%%%%%%%%%%%%%%%%%%%%%%%%%%%%%%%%%%%%%%%%%%%%%%%%%%%%%%%%%%%%%%%%%%
%% 
%% History
%% 8-Dec-1993		Philip Johnson	
%%    

%% Include CSRS in each FTR Flavor and System, and do more of a 
%% compare/contrast.


\newpage
\section{The FTArm Method}
\label{sidebar:csrs}

\begin{figure}
  {\centerline{\psfig{figure=process.ps}}}
\caption{The CSRS process model.}
\label{fig:process-model}
\end{figure}

The FTArm method for CSRS involves seven well-defined phases,
as illustrated by the diagram in Figure \ref{fig:process-model}. It
also involves a {\em meta-phase} for FTR process improvement as a
result of measurements analysis.  An informal description of each of
the seven phases and the meta-phase are provided below.

\paragraph{Setup Phase.} 

In general, each FTArm review begins with the downloading of the
software review artifacts, such as requirements specifications,
designs, source code, test plans, and so forth from their ASCII files
into the CSRS database.  CSRS semi-automatically re-structures the
artifacts as a fine-grained hypertext network consisting of typed
nodes and links. For example, for source code artifacts, each function
and variable definition would be stored within its own node, with
hypertext links to related nodes.  For a requirements specification,
CSRS would re-structure the document hierarchically into a set of
nodes for each major section, each containing subnodes for individual
specifications typed according to their focus.  During the Setup
Phase, the moderator and/or the producer also decide upon the
composition of the review team and the artifacts to be reviewed.

\paragraph{Orientation Phase.} 

This phase prepares the participants for private review by introducing
them to the artifacts under review.  The exact nature of this phase
depends upon the kind of review and the review group.  It may range
from a formal overview meeting with a presentation by the producer
about the review artifact structure and behavior, to an informal
notification through e-mail noting the presence of new artifacts to
review.  The Orientation phase assumes that the participants are
familiar with CSRS; if not, a separate CSRS training session must be
provided.

\paragraph {Private Review Phase.} 

In this phase, reviewers inspect artifacts privately and create issue,
action and/or comment nodes.  Issue and action nodes are not publicly
available to other reviewers at this time, but comment nodes are
publicly available.  Comment nodes allow reviewers to request
clarification about the review artifact, or review process, and may
also contain answers to these questions by other participants.  CSRS
supports three common techniques for analysis of review artifacts:
free review, checklist-based review, and verification-based review.

\paragraph{Public Review Phase.} 

In this phase, all nodes are made public, and all review participants
(including the producer) react to the issues and actions by voting and
creating new nodes.  Participants can create new issue, action or
comment nodes based upon existing nodes.  Voting is used to determine
the degree of agreement within the group about the validity and
implications of issues and actions.  This phase normally concludes
when all nodes have been read by all reviewers, and when voting has
stabilized on all issues.  During both public and private review, CSRS
automatically sends a daily e-mail message to any participants who
have nodes for review or voting.


\paragraph{Consolidation Phase.} 

In this phase, the moderator analyzes the results of public and
private review, and produces a condensed written report of the review
thus far.  In contrast to traditional FTR reports, which typically
contain only a checklist of raised issues with brief comments about
the general quality of the source, consolidation reports contain a
re-organized and condensed presentation of the analyses provided by
reviewers in issue, action, and comment nodes, thus providing
contrasting opinions, the degree of consensus, and proposals for
changes.

\paragraph {Group Review Phase.} 

If the consolidated report identifies issues or actions that were not
successfully resolved via public and private review, a group meeting
is the next step.  In the meeting, the moderator presents the
unresolved issues or actions and summarizes the differences of
opinion.  After discussion, the group may vote to decide them, or the
moderator may unilaterally make the decision. The moderator then
updates the CSRS database, noting the decisions reached during the
group meeting and then generating a final hardcopy document
representing the product of review.  CSRS-basd support for this 
face-to-face meeting is currently under way.

\paragraph{External Development Phase.}  

During this phase the software is enhanced or corrected in response to
the issues raised during review or due to other development processes.

\paragraph{Process Improvement Meta-Phase.}

The seven phases above provide a framework for the FTR process, but
also allow for evolution in response to the measurements supplied
by CSRS.  Many review factors can and should be methodically tested
within the CSRS framework to discover their optimum values within
a particular organization and application context.  Some of these
factors include: artifact size and complexity, review team size
and composition, private review analysis technique, review checklist
composition (if checklists are used), public review scope and
duration, individual and team effort, and scope and duration of 
group review. 

