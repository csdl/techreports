%%%%%%%%%%%%%%%%%%%%%%%%%%%%%% -*- Mode: Latex -*- %%%%%%%%%%%%%%%%%%%%%%%%%%%%
%% ieee93-title.tex -- 
%% RCS:            : $Id: ieee93-title.tex,v 1.5 93/12/27 08:18:12 johnson Exp Locker: johnson $
%% Author          : Philip Johnson
%% Created On      : Fri Dec  3 12:05:03 1993
%% Last Modified By: Philip Johnson
%% Last Modified On: Wed Nov 23 08:40:28 1994
%% Status          : Unknown
%%%%%%%%%%%%%%%%%%%%%%%%%%%%%%%%%%%%%%%%%%%%%%%%%%%%%%%%%%%%%%%%%%%%%%%%%%%%%%%
%%   Copyright (C) 1993 University of Hawaii
%%%%%%%%%%%%%%%%%%%%%%%%%%%%%%%%%%%%%%%%%%%%%%%%%%%%%%%%%%%%%%%%%%%%%%%%%%%%%%%
%% 
%% History
%% 3-Dec-1993		Philip Johnson	
%%    

\title {{\bf Design for Instrumentation:\\
             High Quality Measurement\\
             of Formal Technical Review}}

\author{Philip Johnson\\
        Department of Information and Computer Sciences\\ 
        University of Hawaii\\ 
        Honolulu, HI 96822\\                       
       (808) 956-3489\\
       (808) 956-3548 (fax)\\
       {\tt johnson@hawaii.edu}}

\maketitle

\noindent {\bf Keywords:} Formal technical review, computer-supported 
cooperative work, software quality assurance.

\begin{abstract}
  
  All current software quality assurance methods incorporate some form of
  formal technical review (FTR), because structured analysis of software
  artifacts by a team of skilled technical personnel has demonstrated
  ability to improve quality.  However, FTR methods come in a wide
  variety of forms with varying effectiveness, incur significant
  overhead on technical staff, and have little computer support.
  Measurements of these FTR methods are coarse-grained, frequently low
  quality, and expensive to obtain.
  
  This paper describes CSRS, a highly instrumented, computer-supported
  system for formal technical review, and shows how it is designed to
  collect high quality, fine-grained measures of FTR process and products
  automatically.  The paper also discusses some results from over one year
  of experimentation with CSRS; describes how CSRS improves current
  process improvement approaches to FTR; and overviews several novel
  research projects on FTR that are made possible by this system.
  
\end{abstract}

\silentfootnote{To appear in Software Quality Journal.}



