\abstract{

This dissertation presents a computer-based collaborative learning
environment, called CLARE, that is based on the theory of learning as
collaborative knowledge building. It addresses the question, ``what can a
computer do for a group of learners beyond helping them share
information?'' CLARE differs from virtual classrooms and hypermedia systems
in three ways. First, CLARE is grounded on the theory of meaningful
learning, which focuses the role of meta-knowledge in human
learning. Instead of merely allowing learners to share information, CLARE
provides an explicit meta-cognitive framework, called RESRA, to help
learners interpret information and build knowledge. Second, CLARE defines a
new group process, called SECAI, that guides learners to systematically
analyze, relate, and discuss scientific text through a set of structured
steps: {\it summarization\/}, {\it evaluation\/}, {\it comparison\/}, {\it
argumentation\/}, and {\it integration\/}. Third, CLARE provides a
fine-grained, non-obtrusive instrumentation mechanism that keeps track of
the usage process of its users. Such data forms an important source of
feedback for enhancing the system and a basis for rigorously studying
collaboration learning behaviors of CLARE users.

CLARE was evaluated through sixteen usage sessions involving six groups of
students from two classes. The experiments consist of a total of about 300
hours of usage and over 80,000 timestamps. The survey shows that about 70\%
of learners think that CLARE provides a novel way of understanding
scientific text, and about 80\% of learners think that CLARE provides a
novel way of understanding their peers' perspectives. The analysis of the
CLARE database and the process data also reveals that learners differ
greatly in their interpretations of RESRA, strategies for comprehending the
online text, and understanding of the selected artifact. It is also found
that, despite the large amount of time spent on summarization (up to 66\%),
these learners often fail to correctly represent important features of
scientific text and the relationships between those features. Implications
of these findings at the design, empirical, and pedagogical levels are
discussed.

}


