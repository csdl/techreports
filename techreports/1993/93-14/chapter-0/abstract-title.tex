\documentstyle [11pt,/group/csdl/tex/definemargins,
                     /group/csdl/tex/functiondoc,
		     /group/csdl/tex/lmacros]{article}

\begin{document}

\thispagestyle{empty}

\begin{center}
  {\large\bf CLARE: a Computer-Supported Collaborative  Learning Environment\\
  Based on the Thematic Structure of Scientific Text}\\  \bigskip
  
  Dadong Wan\\ \medskip
  
  Interdisciplinary Ph.D. Program in Communication and Information Sciences\\
  \&\\
  Department of Information \& Computer Sciences\\
  University of Hawaii at Manoa\\
  Honolulu, HI 96822\\
  Voice: (808) 956-6920 $\diamond$ Fax: (808) 956-3548\\
  E-mail: {\tt dxw@uhics.ics.hawaii.edu}
\end{center}

\begin{abstract}
  
  This dissertation presents a computer-supported collaborative
  learning environment, called CLARE, that is based on the theory of
  learning as collaborative knowledge-building. It addresses the
  question, ``what can the computer do for a group of learners beyond
  helping them share information?''  CLARE differs from other systems
  such as virtual classrooms and hypermedia environments in three
  important ways. First, CLARE is grounded on the theory of meaningful
  learning, which focuses the essential role of meta-knowledge in human
  learning; instead of merely enabling learners to share information,
  CLARE provides an explicit meta-cognitive framework, called RESRA,
  that helps learners interpret information or build knowledge. Second,
  CLARE introduces a new group process, called SECAI, that defines
  collaborative learning from scientific text; SECAI enables learners
  to systematically analyze, relate, and discuss scientific text
  through structured steps such as summarization, evaluation,
  comparison, argumentation, and integration.  Third, CLARE provides a
  fine-grained, non-obtrusive instrumentation mechanism that keeps
  track of the usage process of its users. The data captured through
  this mechanism forms not only an important source of feedback for
  incrementally improving the system but more importantly, an empirical
  basis for rigorously studying the collaboration learning behavior of
  CLARE users.
  
  CLARE was evaluated through sixteen experiment sessions participated by
  six groups of students from two different classes.  These experiments
  resulted in a total of over 300 hours of usage and about 80,000
  timestamps. The post-session survey shows that about 70\% of learners
  think that CLARE provides a novel way of understanding scientific text,
  and about 80\% of learners think that CLARE provides a novel way of
  understanding their peers' perspectives. The analysis of the CLARE
  database and the detailed process data also reveals that learners
  differ greatly in their interpretations of RESRA, the strategies they
  use to comprehend the text, and their ultimate understanding of the
  text.  It is also found that, despite the large portion of the usage
  time they spend on summarization (up to 70\%), learners from these
  experiments often mis-represent and/or miss major themes of the text;
  in addition, they also often fail to identify and represent the
  relationships between those major themes.  Implications of these
  findings at the design, empirical, and pedagogical levels are
  discussed.
\end{abstract}

\end{document}

