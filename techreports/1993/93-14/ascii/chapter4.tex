%%%%%%%%%%%%%%%%%%%%%%%%%%%%%% -*- Mode: Latex -*- %%%%%%%%%%%%%%%%%%%%%%%%%%%%
%% chapter4.tex -- 
%% RCS:            : $Id: chapter4.tex,v 1.9 94/04/07 21:20:09 dxw Exp $
%% Author          : Dadong Wan
%% Created On      : Mon Jul 26 20:56:26 1993
%% Last Modified By: Dadong Wan
%% Last Modified On: Thu Apr  7 21:19:23 1994
%% Status          : Unknown
%%%%%%%%%%%%%%%%%%%%%%%%%%%%%%%%%%%%%%%%%%%%%%%%%%%%%%%%%%%%%%%%%%%%%%%%%%%%%%%
%%   Copyright (C) 1993 University of Hawaii
%%%%%%%%%%%%%%%%%%%%%%%%%%%%%%%%%%%%%%%%%%%%%%%%%%%%%%%%%%%%%%%%%%%%%%%%%%%%%%%
%% 
%% History
%% 26-Jul-1993		Dadong Wan	
%%    created. CLARE system: design and implementation
%%% \documentstyle [12pt,/group/csdl/tex/definemargins,
%%% /group/csdl/tex/lmacros]{report}
%%% % Psfig/TeX 
\def\PsfigVersion{1.9}
% dvips version
%
% All psfig/tex software, documentation, and related files
% in this distribution of psfig/tex are 
% Copyright 1987, 1988, 1991 Trevor J. Darrell
%
% Permission is granted for use and non-profit distribution of psfig/tex 
% providing that this notice is clearly maintained. The right to
% distribute any portion of psfig/tex for profit or as part of any commercial
% product is specifically reserved for the author(s) of that portion.
%
% *** Feel free to make local modifications of psfig as you wish,
% *** but DO NOT post any changed or modified versions of ``psfig''
% *** directly to the net. Send them to me and I'll try to incorporate
% *** them into future versions. If you want to take the psfig code 
% *** and make a new program (subject to the copyright above), distribute it, 
% *** (and maintain it) that's fine, just don't call it psfig.
%
% Bugs and improvements to trevor@media.mit.edu.
%
% Thanks to Greg Hager (GDH) and Ned Batchelder for their contributions
% to the original version of this project.
%
% Modified by J. Daniel Smith on 9 October 1990 to accept the
% %%BoundingBox: comment with or without a space after the colon.  Stole
% file reading code from Tom Rokicki's EPSF.TEX file (see below).
%
% More modifications by J. Daniel Smith on 29 March 1991 to allow the
% the included PostScript figure to be rotated.  The amount of
% rotation is specified by the "angle=" parameter of the \psfig command.
%
% Modified by Robert Russell on June 25, 1991 to allow users to specify
% .ps filenames which don't yet exist, provided they explicitly provide
% boundingbox information via the \psfig command. Note: This will only work
% if the "file=" parameter follows all four "bb???=" parameters in the
% command. This is due to the order in which psfig interprets these params.
%
%  3 Jul 1991	JDS	check if file already read in once
%  4 Sep 1991	JDS	fixed incorrect computation of rotated
%			bounding box
% 25 Sep 1991	GVR	expanded synopsis of \psfig
% 14 Oct 1991	JDS	\fbox code from LaTeX so \psdraft works with TeX
%			changed \typeout to \ps@typeout
% 17 Oct 1991	JDS	added \psscalefirst and \psrotatefirst
%

% From: gvr@cs.brown.edu (George V. Reilly)
%
% \psdraft	draws an outline box, but doesn't include the figure
%		in the DVI file.  Useful for previewing.
%
% \psfull	includes the figure in the DVI file (default).
%
% \psscalefirst width= or height= specifies the size of the figure
% 		before rotation.
% \psrotatefirst (default) width= or height= specifies the size of the
% 		 figure after rotation.  Asymetric figures will
% 		 appear to shrink.
%
% \psfigurepath#1	sets the path to search for the figure
%
% \psfig
% usage: \psfig{file=, figure=, height=, width=,
%			bbllx=, bblly=, bburx=, bbury=,
%			rheight=, rwidth=, clip=, angle=, silent=}
%
%	"file" is the filename.  If no path name is specified and the
%		file is not found in the current directory,
%		it will be looked for in directory \psfigurepath.
%	"figure" is a synonym for "file".
%	By default, the width and height of the figure are taken from
%		the BoundingBox of the figure.
%	If "width" is specified, the figure is scaled so that it has
%		the specified width.  Its height changes proportionately.
%	If "height" is specified, the figure is scaled so that it has
%		the specified height.  Its width changes proportionately.
%	If both "width" and "height" are specified, the figure is scaled
%		anamorphically.
%	"bbllx", "bblly", "bburx", and "bbury" control the PostScript
%		BoundingBox.  If these four values are specified
%               *before* the "file" option, the PSFIG will not try to
%               open the PostScript file.
%	"rheight" and "rwidth" are the reserved height and width
%		of the figure, i.e., how big TeX actually thinks
%		the figure is.  They default to "width" and "height".
%	The "clip" option ensures that no portion of the figure will
%		appear outside its BoundingBox.  "clip=" is a switch and
%		takes no value, but the `=' must be present.
%	The "angle" option specifies the angle of rotation (degrees, ccw).
%	The "silent" option makes \psfig work silently.
%

% check to see if macros already loaded in (maybe some other file says
% "\input psfig") ...
\ifx\undefined\psfig\else\endinput\fi

%
% from a suggestion by eijkhout@csrd.uiuc.edu to allow
% loading as a style file. Changed to avoid problems
% with amstex per suggestion by jbence@math.ucla.edu

\let\LaTeXAtSign=\@
\let\@=\relax
\edef\psfigRestoreAt{\catcode`\@=\number\catcode`@\relax}
%\edef\psfigRestoreAt{\catcode`@=\number\catcode`@\relax}
\catcode`\@=11\relax
\newwrite\@unused
\def\ps@typeout#1{{\let\protect\string\immediate\write\@unused{#1}}}
\ps@typeout{psfig/tex \PsfigVersion}

%% Here's how you define your figure path.  Should be set up with null
%% default and a user useable definition.

\def\figurepath{./}
\def\psfigurepath#1{\edef\figurepath{#1}}

%
% @psdo control structure -- similar to Latex @for.
% I redefined these with different names so that psfig can
% be used with TeX as well as LaTeX, and so that it will not 
% be vunerable to future changes in LaTeX's internal
% control structure,
%
\def\@nnil{\@nil}
\def\@empty{}
\def\@psdonoop#1\@@#2#3{}
\def\@psdo#1:=#2\do#3{\edef\@psdotmp{#2}\ifx\@psdotmp\@empty \else
    \expandafter\@psdoloop#2,\@nil,\@nil\@@#1{#3}\fi}
\def\@psdoloop#1,#2,#3\@@#4#5{\def#4{#1}\ifx #4\@nnil \else
       #5\def#4{#2}\ifx #4\@nnil \else#5\@ipsdoloop #3\@@#4{#5}\fi\fi}
\def\@ipsdoloop#1,#2\@@#3#4{\def#3{#1}\ifx #3\@nnil 
       \let\@nextwhile=\@psdonoop \else
      #4\relax\let\@nextwhile=\@ipsdoloop\fi\@nextwhile#2\@@#3{#4}}
\def\@tpsdo#1:=#2\do#3{\xdef\@psdotmp{#2}\ifx\@psdotmp\@empty \else
    \@tpsdoloop#2\@nil\@nil\@@#1{#3}\fi}
\def\@tpsdoloop#1#2\@@#3#4{\def#3{#1}\ifx #3\@nnil 
       \let\@nextwhile=\@psdonoop \else
      #4\relax\let\@nextwhile=\@tpsdoloop\fi\@nextwhile#2\@@#3{#4}}
% 
% \fbox is defined in latex.tex; so if \fbox is undefined, assume that
% we are not in LaTeX.
% Perhaps this could be done better???
\ifx\undefined\fbox
% \fbox code from modified slightly from LaTeX
\newdimen\fboxrule
\newdimen\fboxsep
\newdimen\ps@tempdima
\newbox\ps@tempboxa
\fboxsep = 3pt
\fboxrule = .4pt
\long\def\fbox#1{\leavevmode\setbox\ps@tempboxa\hbox{#1}\ps@tempdima\fboxrule
    \advance\ps@tempdima \fboxsep \advance\ps@tempdima \dp\ps@tempboxa
   \hbox{\lower \ps@tempdima\hbox
  {\vbox{\hrule height \fboxrule
          \hbox{\vrule width \fboxrule \hskip\fboxsep
          \vbox{\vskip\fboxsep \box\ps@tempboxa\vskip\fboxsep}\hskip 
                 \fboxsep\vrule width \fboxrule}
                 \hrule height \fboxrule}}}}
\fi
%
%%%%%%%%%%%%%%%%%%%%%%%%%%%%%%%%%%%%%%%%%%%%%%%%%%%%%%%%%%%%%%%%%%%
% file reading stuff from epsf.tex
%   EPSF.TEX macro file:
%   Written by Tomas Rokicki of Radical Eye Software, 29 Mar 1989.
%   Revised by Don Knuth, 3 Jan 1990.
%   Revised by Tomas Rokicki to accept bounding boxes with no
%      space after the colon, 18 Jul 1990.
%   Portions modified/removed for use in PSFIG package by
%      J. Daniel Smith, 9 October 1990.
%
\newread\ps@stream
\newif\ifnot@eof       % continue looking for the bounding box?
\newif\if@noisy        % report what you're making?
\newif\if@atend        % %%BoundingBox: has (at end) specification
\newif\if@psfile       % does this look like a PostScript file?
%
% PostScript files should start with `%!'
%
{\catcode`\%=12\global\gdef\epsf@start{%!}}
\def\epsf@PS{PS}
%
\def\epsf@getbb#1{%
%
%   The first thing we need to do is to open the
%   PostScript file, if possible.
%
\openin\ps@stream=#1
\ifeof\ps@stream\ps@typeout{Error, File #1 not found}\else
%
%   Okay, we got it. Now we'll scan lines until we find one that doesn't
%   start with %. We're looking for the bounding box comment.
%
   {\not@eoftrue \chardef\other=12
    \def\do##1{\catcode`##1=\other}\dospecials \catcode`\ =10
    \loop
       \if@psfile
	  \read\ps@stream to \epsf@fileline
       \else{
	  \obeyspaces
          \read\ps@stream to \epsf@tmp\global\let\epsf@fileline\epsf@tmp}
       \fi
       \ifeof\ps@stream\not@eoffalse\else
%
%   Check the first line for `%!'.  Issue a warning message if its not
%   there, since the file might not be a PostScript file.
%
       \if@psfile\else
       \expandafter\epsf@test\epsf@fileline:. \\%
       \fi
%
%   We check to see if the first character is a % sign;
%   if so, we look further and stop only if the line begins with
%   `%%BoundingBox:' and the `(atend)' specification was not found.
%   That is, the only way to stop is when the end of file is reached,
%   or a `%%BoundingBox: llx lly urx ury' line is found.
%
          \expandafter\epsf@aux\epsf@fileline:. \\%
       \fi
   \ifnot@eof\repeat
   }\closein\ps@stream\fi}%
%
% This tests if the file we are reading looks like a PostScript file.
%
\long\def\epsf@test#1#2#3:#4\\{\def\epsf@testit{#1#2}
			\ifx\epsf@testit\epsf@start\else
\ps@typeout{Warning! File does not start with `\epsf@start'.  It may not be a PostScript file.}
			\fi
			\@psfiletrue} % don't test after 1st line
%
%   We still need to define the tricky \epsf@aux macro. This requires
%   a couple of magic constants for comparison purposes.
%
{\catcode`\%=12\global\let\epsf@percent=%\global\def\epsf@bblit{%BoundingBox}}
%
%
%   So we're ready to check for `%BoundingBox:' and to grab the
%   values if they are found.  We continue searching if `(at end)'
%   was found after the `%BoundingBox:'.
%
\long\def\epsf@aux#1#2:#3\\{\ifx#1\epsf@percent
   \def\epsf@testit{#2}\ifx\epsf@testit\epsf@bblit
	\@atendfalse
        \epsf@atend #3 . \\%
	\if@atend	
	   \if@verbose{
		\ps@typeout{psfig: found `(atend)'; continuing search}
	   }\fi
        \else
        \epsf@grab #3 . . . \\%
        \not@eoffalse
        \global\no@bbfalse
        \fi
   \fi\fi}%
%
%   Here we grab the values and stuff them in the appropriate definitions.
%
\def\epsf@grab #1 #2 #3 #4 #5\\{%
   \global\def\epsf@llx{#1}\ifx\epsf@llx\empty
      \epsf@grab #2 #3 #4 #5 .\\\else
   \global\def\epsf@lly{#2}%
   \global\def\epsf@urx{#3}\global\def\epsf@ury{#4}\fi}%
%
% Determine if the stuff following the %%BoundingBox is `(atend)'
% J. Daniel Smith.  Copied from \epsf@grab above.
%
\def\epsf@atendlit{(atend)} 
\def\epsf@atend #1 #2 #3\\{%
   \def\epsf@tmp{#1}\ifx\epsf@tmp\empty
      \epsf@atend #2 #3 .\\\else
   \ifx\epsf@tmp\epsf@atendlit\@atendtrue\fi\fi}


% End of file reading stuff from epsf.tex
%%%%%%%%%%%%%%%%%%%%%%%%%%%%%%%%%%%%%%%%%%%%%%%%%%%%%%%%%%%%%%%%%%%

%%%%%%%%%%%%%%%%%%%%%%%%%%%%%%%%%%%%%%%%%%%%%%%%%%%%%%%%%%%%%%%%%%%
% trigonometry stuff from "trig.tex"
\chardef\psletter = 11 % won't conflict with \begin{letter} now...
\chardef\other = 12

\newif \ifdebug %%% turn me on to see TeX hard at work ...
\newif\ifc@mpute %%% don't need to compute some values
\c@mputetrue % but assume that we do

\let\then = \relax
\def\r@dian{pt }
\let\r@dians = \r@dian
\let\dimensionless@nit = \r@dian
\let\dimensionless@nits = \dimensionless@nit
\def\internal@nit{sp }
\let\internal@nits = \internal@nit
\newif\ifstillc@nverging
\def \Mess@ge #1{\ifdebug \then \message {#1} \fi}

{ %%% Things that need abnormal catcodes %%%
	\catcode `\@ = \psletter
	\gdef \nodimen {\expandafter \n@dimen \the \dimen}
	\gdef \term #1 #2 #3%
	       {\edef \t@ {\the #1}%%% freeze parameter 1 (count, by value)
		\edef \t@@ {\expandafter \n@dimen \the #2\r@dian}%
				   %%% freeze parameter 2 (dimen, by value)
		\t@rm {\t@} {\t@@} {#3}%
	       }
	\gdef \t@rm #1 #2 #3%
	       {{%
		\count 0 = 0
		\dimen 0 = 1 \dimensionless@nit
		\dimen 2 = #2\relax
		\Mess@ge {Calculating term #1 of \nodimen 2}%
		\loop
		\ifnum	\count 0 < #1
		\then	\advance \count 0 by 1
			\Mess@ge {Iteration \the \count 0 \space}%
			\Multiply \dimen 0 by {\dimen 2}%
			\Mess@ge {After multiplication, term = \nodimen 0}%
			\Divide \dimen 0 by {\count 0}%
			\Mess@ge {After division, term = \nodimen 0}%
		\repeat
		\Mess@ge {Final value for term #1 of 
				\nodimen 2 \space is \nodimen 0}%
		\xdef \Term {#3 = \nodimen 0 \r@dians}%
		\aftergroup \Term
	       }}
	\catcode `\p = \other
	\catcode `\t = \other
	\gdef \n@dimen #1pt{#1} %%% throw away the ``pt''
}

\def \Divide #1by #2{\divide #1 by #2} %%% just a synonym

\def \Multiply #1by #2%%% allows division of a dimen by a dimen
       {{%%% should really freeze parameter 2 (dimen, passed by value)
	\count 0 = #1\relax
	\count 2 = #2\relax
	\count 4 = 65536
	\Mess@ge {Before scaling, count 0 = \the \count 0 \space and
			count 2 = \the \count 2}%
	\ifnum	\count 0 > 32767 %%% do our best to avoid overflow
	\then	\divide \count 0 by 4
		\divide \count 4 by 4
	\else	\ifnum	\count 0 < -32767
		\then	\divide \count 0 by 4
			\divide \count 4 by 4
		\else
		\fi
	\fi
	\ifnum	\count 2 > 32767 %%% while retaining reasonable accuracy
	\then	\divide \count 2 by 4
		\divide \count 4 by 4
	\else	\ifnum	\count 2 < -32767
		\then	\divide \count 2 by 4
			\divide \count 4 by 4
		\else
		\fi
	\fi
	\multiply \count 0 by \count 2
	\divide \count 0 by \count 4
	\xdef \product {#1 = \the \count 0 \internal@nits}%
	\aftergroup \product
       }}

\def\r@duce{\ifdim\dimen0 > 90\r@dian \then   % sin(x+90) = sin(180-x)
		\multiply\dimen0 by -1
		\advance\dimen0 by 180\r@dian
		\r@duce
	    \else \ifdim\dimen0 < -90\r@dian \then  % sin(-x) = sin(360+x)
		\advance\dimen0 by 360\r@dian
		\r@duce
		\fi
	    \fi}

\def\Sine#1%
       {{%
	\dimen 0 = #1 \r@dian
	\r@duce
	\ifdim\dimen0 = -90\r@dian \then
	   \dimen4 = -1\r@dian
	   \c@mputefalse
	\fi
	\ifdim\dimen0 = 90\r@dian \then
	   \dimen4 = 1\r@dian
	   \c@mputefalse
	\fi
	\ifdim\dimen0 = 0\r@dian \then
	   \dimen4 = 0\r@dian
	   \c@mputefalse
	\fi
%
	\ifc@mpute \then
        	% convert degrees to radians
		\divide\dimen0 by 180
		\dimen0=3.141592654\dimen0
%
		\dimen 2 = 3.1415926535897963\r@dian %%% a well-known constant
		\divide\dimen 2 by 2 %%% we only deal with -pi/2 : pi/2
		\Mess@ge {Sin: calculating Sin of \nodimen 0}%
		\count 0 = 1 %%% see power-series expansion for sine
		\dimen 2 = 1 \r@dian %%% ditto
		\dimen 4 = 0 \r@dian %%% ditto
		\loop
			\ifnum	\dimen 2 = 0 %%% then we've done
			\then	\stillc@nvergingfalse 
			\else	\stillc@nvergingtrue
			\fi
			\ifstillc@nverging %%% then calculate next term
			\then	\term {\count 0} {\dimen 0} {\dimen 2}%
				\advance \count 0 by 2
				\count 2 = \count 0
				\divide \count 2 by 2
				\ifodd	\count 2 %%% signs alternate
				\then	\advance \dimen 4 by \dimen 2
				\else	\advance \dimen 4 by -\dimen 2
				\fi
		\repeat
	\fi		
			\xdef \sine {\nodimen 4}%
       }}

% Now the Cosine can be calculated easily by calling \Sine
\def\Cosine#1{\ifx\sine\UnDefined\edef\Savesine{\relax}\else
		             \edef\Savesine{\sine}\fi
	{\dimen0=#1\r@dian\advance\dimen0 by 90\r@dian
	 \Sine{\nodimen 0}
	 \xdef\cosine{\sine}
	 \xdef\sine{\Savesine}}}	      
% end of trig stuff
%%%%%%%%%%%%%%%%%%%%%%%%%%%%%%%%%%%%%%%%%%%%%%%%%%%%%%%%%%%%%%%%%%%%

\def\psdraft{
	\def\@psdraft{0}
	%\ps@typeout{draft level now is \@psdraft \space . }
}
\def\psfull{
	\def\@psdraft{100}
	%\ps@typeout{draft level now is \@psdraft \space . }
}

\psfull

\newif\if@scalefirst
\def\psscalefirst{\@scalefirsttrue}
\def\psrotatefirst{\@scalefirstfalse}
\psrotatefirst

\newif\if@draftbox
\def\psnodraftbox{
	\@draftboxfalse
}
\def\psdraftbox{
	\@draftboxtrue
}
\@draftboxtrue

\newif\if@prologfile
\newif\if@postlogfile
\def\pssilent{
	\@noisyfalse
}
\def\psnoisy{
	\@noisytrue
}
\psnoisy
%%% These are for the option list.
%%% A specification of the form a = b maps to calling \@p@@sa{b}
\newif\if@bbllx
\newif\if@bblly
\newif\if@bburx
\newif\if@bbury
\newif\if@height
\newif\if@width
\newif\if@rheight
\newif\if@rwidth
\newif\if@angle
\newif\if@clip
\newif\if@verbose
\def\@p@@sclip#1{\@cliptrue}


\newif\if@decmpr

%%% GDH 7/26/87 -- changed so that it first looks in the local directory,
%%% then in a specified global directory for the ps file.
%%% RPR 6/25/91 -- changed so that it defaults to user-supplied name if
%%% boundingbox info is specified, assuming graphic will be created by
%%% print time.
%%% TJD 10/19/91 -- added bbfile vs. file distinction, and @decmpr flag

\def\@p@@sfigure#1{\def\@p@sfile{null}\def\@p@sbbfile{null}
	        \openin1=#1.bb
		\ifeof1\closein1
	        	\openin1=\figurepath#1.bb
			\ifeof1\closein1
			        \openin1=#1
				\ifeof1\closein1%
				       \openin1=\figurepath#1
					\ifeof1
					   \ps@typeout{Error, File #1 not found}
						\if@bbllx\if@bblly
				   		\if@bburx\if@bbury
			      				\def\@p@sfile{#1}%
			      				\def\@p@sbbfile{#1}%
							\@decmprfalse
				  	   	\fi\fi\fi\fi
					\else\closein1
				    		\def\@p@sfile{\figurepath#1}%
				    		\def\@p@sbbfile{\figurepath#1}%
						\@decmprfalse
	                       		\fi%
			 	\else\closein1%
					\def\@p@sfile{#1}
					\def\@p@sbbfile{#1}
					\@decmprfalse
			 	\fi
			\else
				\def\@p@sfile{\figurepath#1}
				\def\@p@sbbfile{\figurepath#1.bb}
				\@decmprtrue
			\fi
		\else
			\def\@p@sfile{#1}
			\def\@p@sbbfile{#1.bb}
			\@decmprtrue
		\fi}

\def\@p@@sfile#1{\@p@@sfigure{#1}}

\def\@p@@sbbllx#1{
		%\ps@typeout{bbllx is #1}
		\@bbllxtrue
		\dimen100=#1
		\edef\@p@sbbllx{\number\dimen100}
}
\def\@p@@sbblly#1{
		%\ps@typeout{bblly is #1}
		\@bbllytrue
		\dimen100=#1
		\edef\@p@sbblly{\number\dimen100}
}
\def\@p@@sbburx#1{
		%\ps@typeout{bburx is #1}
		\@bburxtrue
		\dimen100=#1
		\edef\@p@sbburx{\number\dimen100}
}
\def\@p@@sbbury#1{
		%\ps@typeout{bbury is #1}
		\@bburytrue
		\dimen100=#1
		\edef\@p@sbbury{\number\dimen100}
}
\def\@p@@sheight#1{
		\@heighttrue
		\dimen100=#1
   		\edef\@p@sheight{\number\dimen100}
		%\ps@typeout{Height is \@p@sheight}
}
\def\@p@@swidth#1{
		%\ps@typeout{Width is #1}
		\@widthtrue
		\dimen100=#1
		\edef\@p@swidth{\number\dimen100}
}
\def\@p@@srheight#1{
		%\ps@typeout{Reserved height is #1}
		\@rheighttrue
		\dimen100=#1
		\edef\@p@srheight{\number\dimen100}
}
\def\@p@@srwidth#1{
		%\ps@typeout{Reserved width is #1}
		\@rwidthtrue
		\dimen100=#1
		\edef\@p@srwidth{\number\dimen100}
}
\def\@p@@sangle#1{
		%\ps@typeout{Rotation is #1}
		\@angletrue
%		\dimen100=#1
		\edef\@p@sangle{#1} %\number\dimen100}
}
\def\@p@@ssilent#1{ 
		\@verbosefalse
}
\def\@p@@sprolog#1{\@prologfiletrue\def\@prologfileval{#1}}
\def\@p@@spostlog#1{\@postlogfiletrue\def\@postlogfileval{#1}}
\def\@cs@name#1{\csname #1\endcsname}
\def\@setparms#1=#2,{\@cs@name{@p@@s#1}{#2}}
%
% initialize the defaults (size the size of the figure)
%
\def\ps@init@parms{
		\@bbllxfalse \@bbllyfalse
		\@bburxfalse \@bburyfalse
		\@heightfalse \@widthfalse
		\@rheightfalse \@rwidthfalse
		\def\@p@sbbllx{}\def\@p@sbblly{}
		\def\@p@sbburx{}\def\@p@sbbury{}
		\def\@p@sheight{}\def\@p@swidth{}
		\def\@p@srheight{}\def\@p@srwidth{}
		\def\@p@sangle{0}
		\def\@p@sfile{} \def\@p@sbbfile{}
		\def\@p@scost{10}
		\def\@sc{}
		\@prologfilefalse
		\@postlogfilefalse
		\@clipfalse
		\if@noisy
			\@verbosetrue
		\else
			\@verbosefalse
		\fi
}
%
% Go through the options setting things up.
%
\def\parse@ps@parms#1{
	 	\@psdo\@psfiga:=#1\do
		   {\expandafter\@setparms\@psfiga,}}
%
% Compute bb height and width
%
\newif\ifno@bb
\def\bb@missing{
	\if@verbose{
		\ps@typeout{psfig: searching \@p@sbbfile \space  for bounding box}
	}\fi
	\no@bbtrue
	\epsf@getbb{\@p@sbbfile}
        \ifno@bb \else \bb@cull\epsf@llx\epsf@lly\epsf@urx\epsf@ury\fi
}	
\def\bb@cull#1#2#3#4{
	\dimen100=#1 bp\edef\@p@sbbllx{\number\dimen100}
	\dimen100=#2 bp\edef\@p@sbblly{\number\dimen100}
	\dimen100=#3 bp\edef\@p@sbburx{\number\dimen100}
	\dimen100=#4 bp\edef\@p@sbbury{\number\dimen100}
	\no@bbfalse
}
% rotate point (#1,#2) about (0,0).
% The sine and cosine of the angle are already stored in \sine and
% \cosine.  The result is placed in (\p@intvaluex, \p@intvaluey).
\newdimen\p@intvaluex
\newdimen\p@intvaluey
\def\rotate@#1#2{{\dimen0=#1 sp\dimen1=#2 sp
%            	calculate x' = x \cos\theta - y \sin\theta
		  \global\p@intvaluex=\cosine\dimen0
		  \dimen3=\sine\dimen1
		  \global\advance\p@intvaluex by -\dimen3
% 		calculate y' = x \sin\theta + y \cos\theta
		  \global\p@intvaluey=\sine\dimen0
		  \dimen3=\cosine\dimen1
		  \global\advance\p@intvaluey by \dimen3
		  }}
\def\compute@bb{
		\no@bbfalse
		\if@bbllx \else \no@bbtrue \fi
		\if@bblly \else \no@bbtrue \fi
		\if@bburx \else \no@bbtrue \fi
		\if@bbury \else \no@bbtrue \fi
		\ifno@bb \bb@missing \fi
		\ifno@bb \ps@typeout{FATAL ERROR: no bb supplied or found}
			\no-bb-error
		\fi
		%
%\ps@typeout{BB: \@p@sbbllx, \@p@sbblly, \@p@sbburx, \@p@sbbury} 
%
% store height/width of original (unrotated) bounding box
		\count203=\@p@sbburx
		\count204=\@p@sbbury
		\advance\count203 by -\@p@sbbllx
		\advance\count204 by -\@p@sbblly
		\edef\ps@bbw{\number\count203}
		\edef\ps@bbh{\number\count204}
		%\ps@typeout{ psbbh = \ps@bbh, psbbw = \ps@bbw }
		\if@angle 
			\Sine{\@p@sangle}\Cosine{\@p@sangle}
	        	{\dimen100=\maxdimen\xdef\r@p@sbbllx{\number\dimen100}
					    \xdef\r@p@sbblly{\number\dimen100}
			                    \xdef\r@p@sbburx{-\number\dimen100}
					    \xdef\r@p@sbbury{-\number\dimen100}}
%
% Need to rotate all four points and take the X-Y extremes of the new
% points as the new bounding box.
                        \def\minmaxtest{
			   \ifnum\number\p@intvaluex<\r@p@sbbllx
			      \xdef\r@p@sbbllx{\number\p@intvaluex}\fi
			   \ifnum\number\p@intvaluex>\r@p@sbburx
			      \xdef\r@p@sbburx{\number\p@intvaluex}\fi
			   \ifnum\number\p@intvaluey<\r@p@sbblly
			      \xdef\r@p@sbblly{\number\p@intvaluey}\fi
			   \ifnum\number\p@intvaluey>\r@p@sbbury
			      \xdef\r@p@sbbury{\number\p@intvaluey}\fi
			   }
%			lower left
			\rotate@{\@p@sbbllx}{\@p@sbblly}
			\minmaxtest
%			upper left
			\rotate@{\@p@sbbllx}{\@p@sbbury}
			\minmaxtest
%			lower right
			\rotate@{\@p@sbburx}{\@p@sbblly}
			\minmaxtest
%			upper right
			\rotate@{\@p@sbburx}{\@p@sbbury}
			\minmaxtest
			\edef\@p@sbbllx{\r@p@sbbllx}\edef\@p@sbblly{\r@p@sbblly}
			\edef\@p@sbburx{\r@p@sbburx}\edef\@p@sbbury{\r@p@sbbury}
%\ps@typeout{rotated BB: \r@p@sbbllx, \r@p@sbblly, \r@p@sbburx, \r@p@sbbury}
		\fi
		\count203=\@p@sbburx
		\count204=\@p@sbbury
		\advance\count203 by -\@p@sbbllx
		\advance\count204 by -\@p@sbblly
		\edef\@bbw{\number\count203}
		\edef\@bbh{\number\count204}
		%\ps@typeout{ bbh = \@bbh, bbw = \@bbw }
}
%
% \in@hundreds performs #1 * (#2 / #3) correct to the hundreds,
%	then leaves the result in @result
%
\def\in@hundreds#1#2#3{\count240=#2 \count241=#3
		     \count100=\count240	% 100 is first digit #2/#3
		     \divide\count100 by \count241
		     \count101=\count100
		     \multiply\count101 by \count241
		     \advance\count240 by -\count101
		     \multiply\count240 by 10
		     \count101=\count240	%101 is second digit of #2/#3
		     \divide\count101 by \count241
		     \count102=\count101
		     \multiply\count102 by \count241
		     \advance\count240 by -\count102
		     \multiply\count240 by 10
		     \count102=\count240	% 102 is the third digit
		     \divide\count102 by \count241
		     \count200=#1\count205=0
		     \count201=\count200
			\multiply\count201 by \count100
		 	\advance\count205 by \count201
		     \count201=\count200
			\divide\count201 by 10
			\multiply\count201 by \count101
			\advance\count205 by \count201
			%
		     \count201=\count200
			\divide\count201 by 100
			\multiply\count201 by \count102
			\advance\count205 by \count201
			%
		     \edef\@result{\number\count205}
}
\def\compute@wfromh{
		% computing : width = height * (bbw / bbh)
		\in@hundreds{\@p@sheight}{\@bbw}{\@bbh}
		%\ps@typeout{ \@p@sheight * \@bbw / \@bbh, = \@result }
		\edef\@p@swidth{\@result}
		%\ps@typeout{w from h: width is \@p@swidth}
}
\def\compute@hfromw{
		% computing : height = width * (bbh / bbw)
	        \in@hundreds{\@p@swidth}{\@bbh}{\@bbw}
		%\ps@typeout{ \@p@swidth * \@bbh / \@bbw = \@result }
		\edef\@p@sheight{\@result}
		%\ps@typeout{h from w : height is \@p@sheight}
}
\def\compute@handw{
		\if@height 
			\if@width
			\else
				\compute@wfromh
			\fi
		\else 
			\if@width
				\compute@hfromw
			\else
				\edef\@p@sheight{\@bbh}
				\edef\@p@swidth{\@bbw}
			\fi
		\fi
}
\def\compute@resv{
		\if@rheight \else \edef\@p@srheight{\@p@sheight} \fi
		\if@rwidth \else \edef\@p@srwidth{\@p@swidth} \fi
		%\ps@typeout{rheight = \@p@srheight, rwidth = \@p@srwidth}
}
%		
% Compute any missing values
\def\compute@sizes{
	\compute@bb
	\if@scalefirst\if@angle
% at this point the bounding box has been adjsuted correctly for
% rotation.  PSFIG does all of its scaling using \@bbh and \@bbw.  If
% a width= or height= was specified along with \psscalefirst, then the
% width=/height= value needs to be adjusted to match the new (rotated)
% bounding box size (specifed in \@bbw and \@bbh).
%    \ps@bbw       width=
%    -------  =  ---------- 
%    \@bbw       new width=
% so `new width=' = (width= * \@bbw) / \ps@bbw; where \ps@bbw is the
% width of the original (unrotated) bounding box.
	\if@width
	   \in@hundreds{\@p@swidth}{\@bbw}{\ps@bbw}
	   \edef\@p@swidth{\@result}
	\fi
	\if@height
	   \in@hundreds{\@p@sheight}{\@bbh}{\ps@bbh}
	   \edef\@p@sheight{\@result}
	\fi
	\fi\fi
	\compute@handw
	\compute@resv}

%
% \psfig
% usage : \psfig{file=, height=, width=, bbllx=, bblly=, bburx=, bbury=,
%			rheight=, rwidth=, clip=}
%
% "clip=" is a switch and takes no value, but the `=' must be present.
\def\psfig#1{\vbox {
	% do a zero width hard space so that a single
	% \psfig in a centering enviornment will behave nicely
	%{\setbox0=\hbox{\ }\ \hskip-\wd0}
	%
	\ps@init@parms
	\parse@ps@parms{#1}
	\compute@sizes
	%
	\ifnum\@p@scost<\@psdraft{
		%
		\special{ps::[begin] 	\@p@swidth \space \@p@sheight \space
				\@p@sbbllx \space \@p@sbblly \space
				\@p@sbburx \space \@p@sbbury \space
				startTexFig \space }
		\if@angle
			\special {ps:: \@p@sangle \space rotate \space} 
		\fi
		\if@clip{
			\if@verbose{
				\ps@typeout{(clip)}
			}\fi
			\special{ps:: doclip \space }
		}\fi
		\if@prologfile
		    \special{ps: plotfile \@prologfileval \space } \fi
		\if@decmpr{
			\if@verbose{
				\ps@typeout{psfig: including \@p@sfile.Z \space }
			}\fi
			\special{ps: plotfile "`zcat \@p@sfile.Z" \space }
		}\else{
			\if@verbose{
				\ps@typeout{psfig: including \@p@sfile \space }
			}\fi
			\special{ps: plotfile \@p@sfile \space }
		}\fi
		\if@postlogfile
		    \special{ps: plotfile \@postlogfileval \space } \fi
		\special{ps::[end] endTexFig \space }
		% Create the vbox to reserve the space for the figure.
		\vbox to \@p@srheight sp{
		% 1/92 TJD Changed from "true sp" to "sp" for magnification.
			\hbox to \@p@srwidth sp{
				\hss
			}
		\vss
		}
	}\else{
		% draft figure, just reserve the space and print the
		% path name.
		\if@draftbox{		
			% Verbose draft: print file name in box
			\hbox{\frame{\vbox to \@p@srheight sp{
			\vss
			\hbox to \@p@srwidth sp{ \hss \@p@sfile \hss }
			\vss
			}}}
		}\else{
			% Non-verbose draft
			\vbox to \@p@srheight sp{
			\vss
			\hbox to \@p@srwidth sp{\hss}
			\vss
			}
		}\fi	



	}\fi
}}
\psfigRestoreAt
\let\@=\LaTeXAtSign




%%% \special{header=/group/csdl/tex/psfig/lprep71.pro}
%%% \begin{document}

%%% \tableofcontents
%%% \newpage
%%% \pagenumbering{arabic}

\setcounter{chapter}{3}
\chapter{CLARE: the System}
\label{sec:clare}

CLARE is a computerized collaborative learning environment that implements
the conceptual framework described in Chapter 2. This chapter provides an
overview of the design and implementation of CLARE. Section
\ref{sec:requirements} describes the requirements for the system. Section
\ref{sec:c4-design} discusses two major design considerations: a combined
object-oriented and layered approach, and the initial choice of services
over interface. Section \ref{sec:architecture} describes the five
architectural components: {\it kernel\/}, {\it exploration\/}, {\it
consolidation\/}, {\it initialization\/}, and {\it utilities\/}. Section
\ref{sec:interface} depicts the user interface. Section \ref{sec:roadmap}
shows a road map that outlines major process steps that the user goes
through in a typical session. The chapter concludes with a brief review of
the development history of CLARE, and a report of its current status.

This chapter is supplemented by a detailed design specification
\cite{csdl-93-24}, and a user's guide to CLARE \cite{csdl-93-15}. Readers
whose primary interest is in using the system should read the latter.
Readers with the system designer and implementor perspectives might find it
worthwhile to browse the former.  Readers with a research focus on CLARE
should start with this chapter, and consult the other two documents as
necessary.


\section{Basic requirements}
\label{sec:requirements}

CLARE is a research tool. As such, it has three primary objectives: 1) to
demonstrate the implementational viability of the proposed approach, that
is, whether or not the approach can be implemented within the constraint of
available resources; 2) to provide an instrumentation mechanism to capture
the data necessary for gaining new insights on the nature of
computer-augmented collaborative learning; and 3) to show that CLARE is a
usable system to the potential user population. The first objective is
bounded by the conceptual framework of the current research; it is called
the {\it conceptual\/} requirement. The second aims at yielding new
empirical understanding of the underlying process; it is the {\it data}
requirement of the system. And, finally, it is the {\it usability
requirement,\/} which ensures the resulting system is usable by the
intended user. The following sections address each of the three
requirements.


\subsection{Conceptual requirements}
\label{sec:conceptual-requirement}

Conceptually, there are three key requirements for CLARE: (1) multi-user,
distributed environment; (2) support for the RESRA representation language;
and (3) support for the proposed collaborative learning model.

\paragraph{As a collaborative system.}

As a collaborative system, CLARE must support the following five essential
functions:

\begin{itemize}
\item To allow multiple simultaneously connected users from physically
  dispersed locations;
  
\item To ensure data consistency and integrity via concurrency control
  mechanisms;
  
\item To maintain {\it What-You-See-Is-What-I-See\/} (WYSIWIS) among
  distributed clients, i.e., synchronization of the client's views of the
  shared data;
  
\item To provide capabilities to identify individual users and track
  their status; and
  
\item To manage access and actions on the shared data, for example, not
  allowing deletion of a node created by another user.
\end{itemize}

CLARE is primarily an asynchronous system. In other words, the {\it virtual
co-presence\/} of multiple users in the system is not required for engaging
online conversations. However, the effect from any user action ought to be
communicated to all connected users in a consistent and transparent manner.


\paragraph{As a representation-based learning system.}

One key feature that differentiates CLARE from other collaborative learning
systems is the embedded representation language, i.e., RESRA. In essence,
CLARE is a RESRA-centered environment. Operations for manipulating RESRA
primitives, templates, and instances constitute a core of that system.
Specifically, CLARE must support the following capabilities:

\begin{itemize}
\item To define, update, annotate, instantiate, and view RESRA node and
  link primitives;
  
\item To define, update, annotate, view, and associate RESRA templates with
  specific artifacts;
  
\item To create, update, select, and navigate RESRA node instances;
  
\item To support context-sensitive creation and deletion of link instances;
  
\item To define, update, select, consult, and browse RESRA template
  instances;
  
\item To define, update, and browse example RESRA node, link, tuple, and
  template instances;
  
\item To support artifact-based comparison of node, tuple, and template
  instances among different users; and
  
\item To generate hardcopy summary report of session activities.
\end{itemize}

The ability to extend RESRA by adding new node primitives, link primitives,
and templates, and by amending existing ones is deemed essential to the
long-term viability of RESRA. However, during an initial phase, such a
feature does not have to be directly available to the user via interactive
invocation. This restriction gives the user opportunity to gain a full
grasp of the existing primitives and templates before attempting to invent
new ones.


\paragraph{As a collaborative learning system.}

The term {\it collaborative learning \/} has a well-defined meaning in
CLARE: the use of RESRA to represent, evaluate, deliberate, and integrate
the thematic features of scientific text according to the SECAI learning
model (see Section \ref{sec:secai} for details on SECAI). Below are a set
of specific requirements derived from this model:

\begin{itemize}
\item RESRA-based operations should be structured into five groups, which
  correspond to the five components of SECAI: {\it summarization\/}, {\it
  evaluation\/}, {\it comparison\/}, {\it argumentation\/}, and {\it
  integration\/};
  
\item CLARE's main functions should be partitioned into at least two main
  modules, corresponding to the two phases of the SECAI model, i.e., {\it
  exploration\/} and {\it consolidation\/};
  
\item Since exploration is a {\it private} activity, CLARE needs to provide
  access control to ensure that a user at this phase is not allowed to view
  or act on RESRA instances created by other users; and
  
\item The {\it anchor point} for exploration is the source nodes that
  constitute the selected artifact; the anchor point for consolidation is
  the RESRA representations of the contents of the selected artifact and
  the points of view on the content from individual users.
\end{itemize}

The SECAI learning model is adaptable to support a variety of learning
tasks. For example, with the omission of {\it summarization\/} and {\it
integration\/}, the system can be used to critique peer's term papers or
project proposals. Similarly, by importing an interesting thread from any
newsgroup into CLARE, one may only need the {\it argumentation\/} functions
to continue the discussion in CLARE. Such variations, however, go beyond
the initial scope of this prototype.


\subsection{Data requirements}
\label{sec:data requirements}

As a research tool, CLARE needs to be equipped with built-in
instrumentation mechanism which gathers fine-grained process data.
Specifically, such a mechanism must meet the following four requirements:

\begin{itemize}
\item Instrumented data points should be available for all important CLARE
  operations;
  
\item Each data point should include the name of the operation, the order
  and exact time of invocation, and the name of the user who invokes it;
  
\item Data capturing should be done in a non-obtrusive manner, and should
  not incur substantial runtime overhead; and
  
\item Certain measures should be taken to maximize the validity of
  the process data gathered, e.g., eliminating spurious elapsed time.
\end{itemize}

Ultimately, CLARE should have an analysis back-end which
automatically extracts instrumented process and outcome data and exports
them into a format importable to plot programs (e.g., {\sf gnuplot\/}), and
statistical packages (e.g., {\sf SAS}, {\sf S Plus\/}). Currently, it is
adequate that analyses be done on an ad hoc basis.


\subsection{Usability requirements}
\label{sec:usability requirements}

Despite its research orientation, CLARE was designed to be used by users in
real classroom settings. As a result, it needs to meet some minimal
usability requirements, including:

\begin{itemize}
\item Abiding by established interface design standards, for example,
  \cite{Apple87};
  
\item Response time for interactive commands should be acceptable; and
  
\item Examples should be provided whenever possible to show the user the
  proper way of using the system constructs and functions.
\end{itemize}

In the initial prototype, no attempt is made to offer graphical means of
browsing and manipulating RESRA instances. Hence, the usability or
user-friendliness of CLARE is ultimately constrained by the textual nature
of its interface.


\section{Main design considerations}
\label{sec:c4-design}

CLARE is not a stand-alone system. Rather, it is a tool built upon a
generic collaborative application platform called EGRET\footnote{EGRET,
which stands for {\bf E}xploratory {\bf GR}oup {\bf E}nvironmen{\bf T}, is
hypertext-based distributed environment for supporting collaborative
applications. See \cite{csdl-93-09} for details on EGRET. Besides CLARE,
there are also other applications built on top of EGRET, including CSRS ---
a collaborative software review system (\cite{csdl-93-04}), and URN --- a
collaborative knowledge structuring tool for USENET
(\cite{csdl-93-06}).}. The two principal design features of CLARE, namely,
a layered, object-oriented design, and the emphasis of services over
interface, are both inherited from the underlying platform.


\subsection{Layered + object-oriented design}
\label{sec:oo design}

EGRET --- the platform on which CLARE is built --- was designed with the
special consideration given to the extensibility and verifiability. To
reach the two goals, it blends a layered design with the object-oriented
method (see \cite{csdl-93-02} for more details). CLARE has adopted the same
design principle for two main reasons:

\begin{itemize}
\item CLARE is a domain-specific instantiation of EGRET.  It relies
  heavily on EGRET for infrastructure support. As a result, the more
  consistent CLARE is with EGRET at the design level, the more it can
  leverage on the services EGRET offers, for example, the automatic
  generator of design documentation.
  
\item Like in EGRET, extensibility is one of the primary concerns for
  CLARE.  Because of the ill-structured nature of the problem domain, the
  initial implementation was purposely kept small, containing only
  functions that are well-understood and deemed as essential. As the usage
  experience of the system increases, and as the underlying process becomes
  better understood, new features, such as support for domain-specific
  RESRA, will be likely to find their way into the system.  The combination
  of the layered and the object-oriented paradigm provides an effective
  means of addressing changing requirements.
\end{itemize}

See \cite{csdl-93-24} for details of the object-oriented design
specification of the system.


\subsection{``Services'' versus ``interface''}
\label{sec:service versus interface}

Although having a fancy graphical user interface would probably enhance the
usability of CLARE, there are two other competing design factors to
consider as well.  The first concerns the accessibility of the chosen
platform.  The state-of-the-art interface is typically available only from
commercial vendors, some of which run only on specialized hardware and
software platforms, for example, SGI or NextStep, which are not widely
accessible.  In contrast, publicly available environments, such as the ones
from the Free Software Foundation, are often less up-to-date in terms of
interface. At the very early stage, it was decided that both EGRET and
CLARE should be implemented on a widely accessible platform. Hence, we have
chosen what we consider as the best of publicly available environments,
i.e., {\sf Lucid Emacs}, as the initial delivery mechanism.

Second, CLARE is still an early stage of its evolution. It will be some
time before it reaches the point of maturation. Until then, the primary
purpose of this project remains as to explore the exact requirements for
system through successive instrumented usage and revisions. The principal
requirement at this phase is flexibility --- the ability to accommodate new
functionalities and to adapt the existing ones. The interface, on the other
hand, occupies a secondary importance.


\section{Architecture}
\label{sec:architecture}

Architecturally, the CLARE environment consists of two main components: the
{\it platform\/} and the {\it system proper\/}. The former includes the
EGRET client and the hypertext database server. Together, the two provide
basic infrastructure-level services such as storage, retrieval, updating,
and caching of typed nodes, links, and fields, data locking and
synchronization, and so forth. The CLARE system proper is composed of five
modules: {\it kernel\/}, {\it preparation\/}, {\it exploration\/}, {\it
consolidation\/}, and {\it utilities\/}. The relationships between these
five modules are depicted in Figure \ref{fig:arch}.

\begin{figure}[htb]
 \fbox{\centerline{\psfig{figure=Figures/arch.eps,width=5.0in}}}
  \caption{CLARE architecture}
  \label{fig:arch}
\end{figure}


\subsection{Kernel}
\label{sec:kernel}

The {\it kernel\/} performs two essential functions: it is an interface to
EGRET, and it provides support for RESRA. As an interface, it encapsulates
all primitive services provided by EGRET, including node and link creation,
update, retrieval, packing/unpacking, display, attribute data caching, and
etc. Hence, it forms the foundation of the entire system. All other CLARE
modules need to go through the kernel to access necessary EGRET
functions. This design provides a single point of entry to EGRET, which can
minimize potential ripple effects due to changes in the underlying
platform. In addition, it also enables flexibilities such as customized
default parameters, additional error checking, etc.

The other principal service provided by the kernel is the support for RESRA
representational language (see Chapter 3).  Specifically, it include three
types of operations: {\it primitives\/}, {\it CRFs\/}, and {\it
examples\/}. Typical operations on RESRA primitives are initialization,
instantiation, validity checking, type-based instance listing, and status
tracking. Important CRF functions are CRF creation, updating, annotation,
listing, consulting, and importing/exporting. The example mechanism allows
the user to tag and untag the current node as an example, browse predefined
examples in the current database, and import/export example
instances. Examples are intended to help the user clarify the semantics of
RESRA primitives and guide them toward the proper use of the representation
language.


\subsection{Preparation}

This module implements a set of supervisory facilities accessible only to
the designated individuals, e.g., the instructor, researcher, or
administrator. Its functions fall into three categories: {\it artifact
conversion\/}, {\it session setup\/}, and {\it session tracking\/}. First,
learning artifacts such as research papers are imported into the CLARE
database. If an artifact is available only from the printed source, it is
first converted into the electronic format. The online document is then
marked up in terms of {\it semantic units,\/} with each corresponding to a
section or subsection. Any cross-reference, such as footnotes,
bibliographic references, is also identified explicitly at this time. The
marked-up document is converted into CLARE's internal hypertext format.
Necessary links are added automatically by CLARE  between various nodes.

The {\it session setup\/} involves defining parameters specific to a given
CLARE session, which typically include participants for the current
session, leading questions, CRF, and perspectives for the current artifact.
The {\it session tracking\/} includes periodic checking of the status of
individual participants, answering online feedback questions, and toggling
phases.


\subsection{Exploration}
\label{sec:exploration}

This module implements functions that constitute the {\it exploration\/}
component of the SECAI learning model. It consists of two parts: {\it
summarization\/} and {\it evaluation\/}. The former includes the creation,
updating, and navigation of RESRA summarative node and link instances;
consultation of the predefined CRFs for guidance on what to do next,
viewing examples selected from previous sessions, and so on.  The
evaluation includes the creation, updating, and navigation of evaluative
nodes, and answering the leading questions defined for the current
artifact. At the invocation level, the former is very similar to that of
summarization. Leading questions are a set of standard questions
pre-prepared by the instructor or researcher for the current artifact. They
are much less structured than the CRFs, and are used to address global
aspects of different types of artifacts, e.g., major contributions.

One main feature of the exploration module is that most of its functions
operate on data that is private to individual users. For example, {\bf User
A} is not allowed to read a RESRA node instance created by {\bf User B}.
Similarly, {\bf User B} cannot follow a RESRA link generated by {\bf User
A}. However, both users are allowed to read each other's online feedback
nodes and, follow-up them as necessary. CLARE provides an access control
mechanism that supports this feature.


\subsection{Consolidation}
\label{sec:consolidation}

This module implements functions that constitute the {\it consolidation\/}
component of the CLARE learning model. Specifically, it consists of three
parts: {\it comparison\/}, {\it argumentation\/}, and {\it integration\/}.
The purpose of comparison is to highlight the differences and similarities
between different users' representations of the selected artifact. CLARE
supports four types of comparison: {\it summarization\/}, {\it template\/},
{\it evaluation\/}, and {\it leading questions\/}. It allows the user to
easily toggle between different types of comparison, and to move back and
forth between one RESRA type to another (e.g., \fbox{{\sf problem\/}},
\fbox{{\sf claim\/}}), and one leading question and another.

CLARE's argumentative functions are similar to their summarative and
evaluative counterparts in the exploration module. There are two main
differences, however. First, while the summarative and evaluative functions
are applied primarily to the selected artifact itself, the argumentative
functions focus on the user's representation and evaluation of the
artifact. Second, summarization and evaluation are private activities,
argumentation, on the other hand, is a direct form of {\it discussion\/} among
the participants. This public nature of argumentation requires CLARE to
maintain a synchronized view of the shared database.

One key feature of CLARE's support for argumentation is
context-sensitivity, namely, the actions available to a user at a given
point is constrained by the type of the current node. Such constraints are
reflected in the options available from the popup menu.

CLARE's integrative functions are divided into three categories: (1) adding
integrative links between existing nodes; (2) identifying perspectives
related to the current discussion, and associate existing nodes to those
perspectives; and (3) endorsing existing positions and relationships
between those positions. The user can also find out the current state of
consensus among users by invoking consensus report functions.


\subsection{Utilities}
\label{sec:utilities}

The {\it utility\/} module consists of two types of functions: generic,
infrastructure-level utilities not supplied by EGRET, and miscellaneous
functions that do not fit to other modules of CLARE. The former, which
constitutes the majority of this module, include support for regions,
screens, modes, menus, styles, and so on. Most of these functions are
application independent; they are implemented in a way such that,
ultimately, they might be smoothly incorporated into EGRET.

The second group includes reporting facilities, which parse the CLARE
database content to produce structured, typeset hardcopy summary of session
activities, and CLARE-specific instrumentation functions, e.g., extracting
metric data from the database.


\section{Interface}
\label{sec:interface}

CLARE is implemented on top of Lucid Emacs, an X Window based version of
the GNU Emacs editing environment. Most of its interface facilities, such
as multi-screens, pulldown and popup menus, multi-fonts, active-regions,
are inherited from the underlying platform.

Figure \ref{fig:explore} represents a typical user view of CLARE during the
{\it exploration\/} phase. The screen consists of three windows: one
occupies the entire left half of the screen, and the other two equally
divide up the remaining portion of the screen. The left window is for
displaying source or artifact nodes. The upper-right window is used to list
a group of related RESRA instances (e.g., all unseen \fbox{{\sf claim\/}})
so that the user may selectively view their detailed content through mouse
clicking. The lower-right window is used to show the detailed content of a
node.  All user-level commands are available from popup and pulldown menus,
both of which are sensitive to such contexts as the node type, the current
phase, and so forth.

\begin{figure}[htb]
  \centerline{\psfig{figure=Figures/explore.eps,width=6.0in}}
  \caption{Example view of CLARE during exploration}
  \label{fig:explore}
\end{figure}

Figure \ref{fig:consolidate} is a typical view of CLARE during the
consolidation phase. The upper left window contains a comparative view of
the RESRA instances generated from the previous phase. In the example
shown, it is a listing of \fbox{{\sf Problems\/}} by Peter, Cam, and Rose
 --- the three participants of the current session.  The highlighted text
(i.e., with the bold italic font) represents the links to the corresponding
node that contains more detailed information, e.g., the source nodes from
which the RESRA node instances are derived. In the example shown, when the
user mouse clicks the {\bf Problem 1}, i.e., ``Software discipline,'' the
corresponding node will be shown in the lower right window. When the user
follows the link under the {\bf Summarization\/}, the related source node
is shown in the lower left window. The highlighted text (in {\bf bold}
font) is the place from which Peter derives his \fbox{{\sf problem}}.

The upper-right window shows the summary statistics of each user has done
during the exploration phase. Entries in this window are
mouse-sensitive. When an entry is selected using the middle mouse button,
the detailed listing of the corresponding user will be displayed. For
example, if the first entry is chosen, the list of all the 19 nodes created
by Peter will the be shown in that window. The user may then select any
node to view.

\begin{figure}[htb]
  \centerline{\psfig{figure=Figures/consolidate.eps,width=6.0in}}
  \caption{Example view of CLARE during consolidation}
  \label{fig:consolidate}
\end{figure}


\section{The road map}
\label{sec:roadmap}

Figure \ref{fig:roadmap} is an overview map of process steps a user follows
in a typical CLARE session. The shaded and numbered boxes on the left
represent main activity steps. The unidirectional link from {\it exploration\/}
to {\it consolidation\/} indicate the sequential order of the process, i.e.,
once the group enters the consolidation phase, they are normally not
allowed to revert to the exploration phase, nor is it permissible to have
the two going on at the same time.  The bidirectional links connected
shaded boxes within each phase show that the associated activities might be
engaged simultaneously or intermixed in any order. For example, although it
is typical that {\it evaluation\/} comes after {\it summarization\/}, it is
also not uncommon to see the two unfolding simultaneously, or even {\em
evaluation\/} invoked before {\it summarization\/}. The boxes on the right
side list major sub-activities of the corresponding entries on the
left. The one-line question/statements in the middle capture the basic
question to be answered by the corresponding process step.

\begin{figure}
 \fbox{\centerline{\psfig{figure=Figures/road-map.eps,width=5.5in}}}
  \caption{CLARE functional road map}
  \label{fig:roadmap}
\end{figure}


\section{History and status}
\label{sec:c4-history}

\subsection{Brief history}
\label{sec:history1}

CLARE, in its various incarnations, has been in existence for almost three
years. During this time, it has undergone substantial changes at both
conceptual and implementation levels.  Because CLARE is the first system
built on top of EGRET, it is not surprising that the evolutionary path of
CLARE is interleaved with that of EGRET. Figure \ref{fig:history} depicts a
few important milestones in the development history of CLARE.

\begin{figure}
  \fbox{\centerline{\psfig{figure=Figures/history.eps,width=5.5in}}}
  \caption{Evolutionary path of CLARE}
  \label{fig:history}
\end{figure}


\paragraph{Plover.}
\label{sec:plover}

The origin of CLARE traces back to {\sf Plover\/}, an 800-LOC
implementation of a rudimentary hypertext system based on {\sf Info\/}, the
Emacs' online documentation browsing system. Plover uses the same internal
storage format as that of Info. However, it differs from Info in an
important way: unlike Info, which is a read-only, online browsing tool,
Plover is a collaborative system that supports semi-structured group
discussions about design and research. As such, it emphasizes interactive
creation of nodes and links. However, as a group tool, Plover suffers from
several vital flaws. First, it provides very limited distributed support.
Even though the user is allowed to access and modify the shared document on
the remote machine via anonymous FTP, Plover has no locking mechanism to
prevent concurrent updates.  Furthermore, because Plover was implemented on
top of the vanilla Emacs, it does not have support for multi-screens,
popup/pulldown menus, and active regions.  The latter constraint forces
links and field labels to be treated as plain text which are modifiable by
anyone and, consequently, can lead to in corrupted data.  Plover had only a
brief existence; it was used by the development group a few times, and was
quickly succeeded by the initial implementation of COREVIEW/EGRET.


\paragraph{COREVIEW: prototype and experience.}
\label{sec:coreview}

Following the public availability of Hyperbase in May, 1991, a hypertext
database server from the University of Aalborg, Denmark, an initial
prototype of COREVIEW was implemented \cite{csdl-92-03}. This prototype was
largely derived from EHTS, the Emacs interface to the hyperbase that comes
with the distribution, which runs under Epoch, an X-window based version of
Emacs from University of Illinois. The primary purpose of this system is to
support collaborative literature review in seminar settings using IBIS-like
nodes and link types. In fact, COREVIEW was used, albeit briefly, in a
graduate seminar on object-oriented design in the Fall, 1991, at the
Information and Computer Sciences Department of the University of Hawaii.
The attempt, however, was unsuccessful; frustrations and overwhelming
negative feedback about the system from the students forced us to withdraw
the original plan of using it in throughout the semester. The experience
reveals three major problems with our initial system:

\begin{itemize}
\item {\it Fragility of COREVIEW.} The initial prototype, which was
  implemented from bottom-up, i.e., restructured from an existing system,
  rather than from top-down, i.e., based upon well-defined specifications,
  was not sufficiently robust and reliable for public usage. At the
  interface level, the lack of context-sensitive popup/pulldown menus of
  available commands also presented quite a barrier to the uninitialized
  users.
  
\item {\it Lack of understanding of the problem domain.\/} At the
  conceptual level, COREVIEW provides no definite answer to such basic
  questions as, ``what is the precise problem the system attempts to
  address?'' ``What approach it embodies?'' ``How does the approach
  addresses specific aspects of the problem?'' Such ambiguities were
  visible from the absence of an explicit process model and a coherent
  representation in COREVIEW.
  
\item {\it Lack of specification in experimental procedures.\/} Although,
  empirically, the initial usage of COREVIEW was guided by some preliminary
  research hypotheses, detailed procedures and the range of outcomes from
  the experiment, however, were not specified.
\end{itemize}

The above factors had led to the birth of EGRET and a complete
re-conceptualization of CLARE.


\paragraph{CLARE and the new EGRET.}

Following the first unsuccessful use of COREVIEW and, for that matter,
EGRET, much effort in the next 10 months was devoted to the redesign,
implementation, and optimization of the platform using more rigorous
development methodology, i.e., a combination of layered and object-oriented
methods. A separate regression testing module was also developed for
assuring the reliability of the system.  Meanwhile, work on CLARE itself
took place mainly at the conceptual level, i.e., characterizing the problem
domain and developing a new representation-based approach to solving the
problem. During the summer, 1992, six rounds of paper-and-pencil based
usage experiments within the CSDL were conducted in an attempt to test the
viability of the solution and refine the representation, i.e., RESRA.

As EGRET became stable and the conceptual framework for CLARE gradually
came into form, efforts began to shift toward implementing the system.
During next 10-months, CLARE underwent several cycles of ``prototype \(
{\Rightarrow} \) shakedown use \( {\Rightarrow} \) re-design/refinements.''
The initial prototype, for example, does not support online browsing and
marking of the full-text of the selected artifact; the user was required to
read the printed artifact and create necessary RESRA instances online. The
next release provides such a feature but only supports a {\it flat\/} view
of the artifact, i.e., the entire artifact was treated as a single entity,
which, among other things, makes it difficult to collect fine-grained
process data. The subsequent prototype introduces the concept of {\it semantic
units,\/} and provides functions for semi-automatically converting the
linear text into a hypertext network.


\subsection{Current status}
\label{sec:status}

The current implementation of CLARE consists of approximately 15 KLOC of
Emacs Lisp code. Although it is still a research prototype, CLARE has been
functionally stable. The system is being used on a regular basis both
within the Collaborative Software Development Laboratory and in real
classroom settings. The main pending change to the system is to upgrade it
to use the most recent version of EGRET with improved reliability,
efficiency, and functionality. Another important extension that is
currently under consideration is a graphical front-end which, based on our
experience and user feedback thus far, will enhance the usability of the
system by allowing direct manipulation of RESRA instances and better
visualization of the relationships between various RESRA nodes.


%%% 
%%% \newpage
%%% \singlespace
%%% \bibliography{../bib/clare,../bib/csdl-trs}
%%% \bibliographystyle{alpha}
%%% 
%%% 
%%% 
%%% \end{document}





