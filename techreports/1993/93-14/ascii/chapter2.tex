%%%%%%%%%%%%%%%%%%%%%%%%%%%%%% -*- Mode: Latex -*- %%%%%%%%%%%%%%%%%%%%%%%%%%%%
%% chapter2.tex -- 
%% RCS:            : $Id: chapter2.tex,v 1.8 94/04/07 21:05:54 dxw Exp $
%% Author          : Dadong Wan
%% Created On      : Mon Jul 26 21:09:59 1993
%% Last Modified By: Dadong Wan
%% Last Modified On: Thu Apr  7 21:05:48 1994
%% Status          : Unknown
%%%%%%%%%%%%%%%%%%%%%%%%%%%%%%%%%%%%%%%%%%%%%%%%%%%%%%%%%%%%%%%%%%%%%%%%%%%%%%%
%%   Copyright (C) 1993 University of Hawaii
%%%%%%%%%%%%%%%%%%%%%%%%%%%%%%%%%%%%%%%%%%%%%%%%%%%%%%%%%%%%%%%%%%%%%%%%%%%%%%%
%% 
%% History
%% 26-Jul-1993		Dadong Wan	
%%    created
%%% \documentstyle [12pt,/group/csdl/tex/definemargins,
%%% /group/csdl/tex/lmacros]{report}
%%% \input{/home/3/dxw/c/tex/psfig}
%%% \special{header=/group/csdl/tex/psfig/lprep71.pro}
%%% \begin{document}
%%% \ls{1.6}

%%% \pagenumbering{roman}
%%%  \tableofcontents
%%%  \newpage
%%% \pagenumbering{arabic}

\setcounter{chapter}{1}
\chapter{Toward a Representation-Based Approach to Collaborative
Learning from Scientific Text}
\label{sec:approach}

The term {\it collaborative learning\/} has many connotations, ranging from
peer-tutoring to computer conferencing. This research concerns a specific
type of collaborative learning, called {\it learning from scientific
text\/}, which is centered on scientific literature, such as research
papers, journal articles, monographs, and so on. The approach focuses on
the thematic structure of scientific discourse embodied in the written
artifact. More specifically, collaborative learning in this context
involves the following key activities:

\begin{itemize}
\item To summarize the content of a scientific artifact by identifying
  and representing its key thematic features and the relationships
  between these features;
  
\item To evaluate the content of an artifact through making critiques,
  raising questions, and suggesting improvements;
  
\item To compare individual summarizations and evaluations to uncover
  ambiguities, inconsistencies, and differences and similarities;
  
\item To clarify and resolve those ambiguities, inconsistencies, and
  differences through constructive argumentation; and
  
\item To integrate similar and related points of view to form a coherent
  corpse of shared knowledge.
\end{itemize}

This chapter describes the conceptual basis of the above approach. It draws
from several streams of research: knowledge representation in AI, cognitive
learning theory, constructionism, and structural analysis of scientific
text. In doing so, it formulates a theoretical framework for the entire
project.

The chapter begins by relating the AI concept of {\it knowledge
representation\/} to human learning. It points out how knowledge
representation may be viewed as a means of understanding the high-order
structure of knowledge. Second, it discusses the characteristics of
scientific text, followed by an overview of RESRA --- the new
representation language for characterizing the thematic features of
scientific text and for serving as a shared framework for collaborative
learning. The chapter concludes by describing the representation-based
model of collaborative learning from scientific text, and shows how it
integrates knowledge representation, scientific text, and human learning
into a single learning support environment.


\section{From semantic nets to concept maps: knowledge representation in
human learning} 

Knowledge representation (KR) is a central concern of artificial
intelligence (AI). In essence, it is the core of all intelligent systems,
including machine learning, intelligent tutoring, and expert
systems. However, the concept of KR is rarely discussed with respect to
human learning, especially collaborative learning. The purpose of this
section is to examine human learning in terms of knowledge
representation. In doing so, it highlights several important differences
and similarities of KR in these two contexts.  The emphasis is on the
representation language instead of the information processing details by
either human or machine. Since KR concerns the deep-structure of human
knowledge, the view of human learning in terms of KR highlights the
importance of meta-knowledge. In addition, it also forms a basis for the
proposed representation-based approach to collaborative learning among
human learners.


\subsection{Two types of knowledge representation}

In AI, the term {\it knowledge representation\/} (KR) is used to refer to
the process of encoding various types of knowledge into a form with which a
computer program can reason; the embodiment of such knowledge enables the
computer to perform certain tasks that normally require human intelligence.
At the center of this transformation lies the representation formalism,
ranging from formal logic and production rules to semantic networks and
frames. These schemes define what knowledge to represent and how it is
represented and manipulated inside the computer. When viewed from the
context of human learning, however, knowledge representation denotes
something quite different : it concerns {\it meaning extraction\/} from
external knowledge sources, which include both written artifacts and living
sources, e.g., researchers, teachers, peers. At the center of this process
also lies the representation scheme, which determines what and how human
knowledge is represented. The default human representation scheme is the
natural languages, e.g., English or Chinese. There are also other
special-purpose representations proposed to overcome certain deficiencies
of the natural language. Concept maps, which will be introduced below, is a
example KR scheme that is intended to facilitate human learning.

Figure \ref{fig:kr} depicts knowledge representation in AI and human
learning settings\footnote{The analogy between the computer and the human mind
has profound philosophical implications which go far beyond the scope of
the current discussion. For an example discussion, see \cite{Harnad89}.}.
At the process-level, the two possess a number of similarities. First, KR
in both contexts consists of the same components: {\it knowledge source\/},
{\it agent\/} and {\it knowledge engineer\/} or {\it teacher\/}.  The two
also share the same goals: both aim at improving the level of knowledge and
the ability to perform selected tasks, although one concerns a
computational agent, while the other concerns a human learner. Third, they
both acquire knowledge from the same sources, i.e., either codified,
written artifacts such as research papers, or living sources such as
experts in a given field\footnote{The focus of the current research is on the
codified, public knowledge, as recorded in the written artifacts, in
particular, {\it primary\/} artifacts, such as research reports. See
Section \ref{sec:research artifacts} for details.}, or both.  Fourth,
knowledge acquisition is accomplished using a selected KR scheme, and
facilitated by the knowledge engineer or the teacher.

\begin{figure}[htb]
  \fbox{\centerline{\psfig{figure=Figures/kr.eps,height=3.5in}}}
  \caption{Knowledge representation in human learning and AI contexts}
  \label{fig:kr}
\end{figure}

Despite these similarities, KR schemes used by human and computational
agents differ in some fundamental ways. The dividing line between the two
is the distinct roles of the computational agent and human learner.  A
typical AI program is merely a passive recipient of knowledge.  In
contrast, human learners are {\it actors\/} who {\it interpret \/} incoming
information in light of their prior knowledge and experience, and give it
context-specific meanings. In other words, it is the learner who plays a
decisive role in determining what and how knowledge is acquired.  The
teacher, on the other hand, serves mostly a facilitating role in the
knowledge acquisition process. In an AI system, the knowledge engineer is
instrumental in deciding what knowledge gets transferred to the target
system. This role difference is shown in Figure \ref{fig:kr} by the
presence of the link with big arrows on both ends connecting human learner
and the knowledge source.  The small arrow pointing from the AI program to
the knowledge source indicates that few such systems actually contribute to
the public knowledge source, though some machine learning systems can
improve their performance automatically by incorporating new knowledge from
external sources.

At the representation level, the KR schemes used by computational agents
and human learners differ in a number of ways. The specific differences
between them are summarized in Table \ref{tab:kr-schemes}. In general, the
representation used by the computer program is formal, that is, its syntax
and semantics are precisely defined and enforced (e.g., semantic nets);
restrictive in terms of what can be expressed and how it can be expressed;
fine-grained, e.g., phrase or sentence levels as opposed to paragraph or
artifact levels; and finally, a good AI representation is parsimonious and
computationally efficient. In contrast, the KR scheme used by humans, as
exemplified in the natural language, is informal, expressive, emphasizing
on coarse-grained structures, likely to be redundant, and computationally
inefficient and, in many cases, computationally intractable. A good example
of the latter is natural language understanding.

\ls{1.0}
\small
\begin{table}[hbt]
  \caption{KR schemes used by human learners and AI programs}
  \begin{center}
    \begin{tabular} {||l|p{1.8in}|p{1.8in}||} \hline   
      {\bf Criteria} & {\bf AI Systems} & {\bf Human Learning} \\
      \hline \hline
      
      Formality & Formal & Informal \\ \hline
      
      Expressiveness & Restrictive in scope & Expressive \\ \hline
      
      Granularity & Fine-grained & Coarse-grained \\ \hline
      
      Parsimony & Concise & Redundant \\ \hline
      
      Computational efficiency & Efficient & Inefficient \\ \hline
    \end{tabular}
   \end{center}
    \label{tab:kr-schemes}
\end{table}
\normalsize
\ls{1.6}

The above discussion of KR and the comparison of KR requirements for human
learning and machine reasoning is relevant to the current context in two
significant ways. First, representation is important not only in AI systems
but also in human contexts, especially in ill-structured domains such as
learning and design, and in information-overloaded environments such as
hypermedia and virtual classrooms. AI researchers have developed a large
number of techniques and representation schemes which can be readily
applied to other contexts, including human learning. In fact, this approach
has already been applied to the domains such as design rationale management
\cite{Lee91What,Conklin91Process}. Second, KR is either ontological or
epistemological \cite{Swaminathan90}. In other words, KR concerns the
higher-order structure of human knowledge, or meta-knowledge.  Many KR
schemes, such as frames, semantics nets, Schank's conceptual dependency
theory, are themselves examples of meta-structures.  As described in the
following section, the importance of higher-order knowledge in human
learning is increasingly recognized by educational researchers. To
facilitate the use of such knowledge, they have in fact proposed their
version of KR schemes, which they call {\it meta-cognitive tools.\/}
Concept mapping, to be described in Section \ref{sec:concept-map}, is such
an example.


\subsection{Content learning, meta-learning, and knowledge representation}

Human learning can be viewed at two levels: {\it specific\/} and {\it
generic\/}. The former involves the understanding of a specific subject
matter; hence, it is also referred to as {\it content learning\/}. The
latter, which is commonly called {\it meta-learning\/}, concerns the nature
and the structure of what is being learned, as well as the process through
which knowledge is acquired. For example, at a content level, when a
student is first exposed a programming language such as C, he learns the
syntax and semantics of that language. At a meta-level, he might attempt to
draw an analogy between a programming language and the natural language, or
be interested in general strategies in acquiring language skills.  More
specific distinctions can be made between the two at four different levels:
{\it semantic orientation\/}, {\it relational knowledge\/}, {\it process
knowledge\/}, and {\it learning strategies\/}, as summarized in Table
\ref{tab:meta-learning}.  In general, content learning emphasizes the
direct meanings from the snapshot of isolated learning artifacts.
Meta-learning, on the hand, views each artifact as an episodic component of
the overall knowledge of a given field. Its focus, therefore, is on the
deep structure and semantics, as well as the inter-connections among
various knowledge chunks. In addition, it also attaches a central
importance to knowledge acquisition, i.e., how learners make sense of
artifacts presented to them.  In terms of learning strategies,
meta-learning is often associated with {\it meaningful learning\/} (to be
described below), while content learning is often realized through {\it rote
learning.\/}  In typical classroom settings, content and meta-level
learnings are often intertwined.  At a content level, for example,
participants in a research seminar are expected to understand the
particular subject matter that is under concern (e.g., AI). At a
meta-level, they need to learn how to collaborate, how to research
literature, how to present and evaluate research artifacts, how to identify
interesting problems and develop novel solutions, and so forth.

\ls{1.0}
\small
\begin{table}[hbt]
    \caption{Characteristics of content and meta-learning}
    \begin{center}
    \begin{tabular} {||l|p{1.9in}|p{1.9in}||} \hline   
      {\bf Category} & {\bf Content learning} & {\bf Meta-learning}
      \\ \hline \hline
      
      Semantic orientation & Surface, direct meanings & Deep,
      embedded meanings \\ \hline
      
      Relational knowledge & Isolated pieces of knowledge &
      Inter-connections among various knowledge chunks \\ \hline
      
      Process knowledge & Static, final version of knowledge &
      Process of knowledge acquisition and changes of meanings over
      time \\ \hline
      
      Learning strategies & Mostly, rote learning & Meaningful
      learning \\ \hline
    \end{tabular}
    \end{center}
    \label{tab:meta-learning}
\end{table}
\normalsize
\ls{1.6}

The distinction between content learning and meta-learning is significant
for three reasons. First, although content learning tools exist and are
improving, there are few tools to support meta-learning. For example,
concept mapping is still mostly done manually.  Second, meta-learning has
become increasingly important because of the accelerating rate of knowledge
production and dissemination, students may find the subject content they
learn in school quickly becoming obsolete, but any meta-knowledge and
skills they acquire will enable them to better adapt and cope with the
changing state of human knowledge. Third, the differentiation between
content and meta-learning clarifies the role of knowledge representation
in human learning. As shown in Figure \ref{fig:triangle}, the links from
knowledge representation to meta-learning, and from meta-learning to
content learning are {\it direct\/} ones, as expressed the solid line,
while the link from knowledge representation to content learning is an {\it
indirect\/} one, as shown in a dashed line. For example, at a content
level, when a student from a software engineering class is asked to read a
research paper on that subject, he may simply browse through it and learn
nothing. However, he can probably get a much deeper understanding of the
paper content by using a representation such as RESRA. On the other hand,
if he is asked to merely study RESRA independent from his learning context,
he may not find RESRA helpful or relevant.

\begin{figure}[htb]
  \fbox{\centerline{\psfig{figure=Figures/triangle.eps,height=2.5in}}}
  \caption{Relationships between knowledge representation, meta-learning,
  and content learning} 
  \label{fig:triangle}
\end{figure}


\subsection{Cognitive learning theory and concept mapping}
\label{sec:meaningful learning}

One major development in educational psychology is the {\it theory of
meaningful learning\/}, also known as the {\it assimilation theory of
cognitive learning\/} or simply, theory of cognitive learning. It has
evolved at Cornell University over the past three decades
\cite{Ausubel63,Novak84}.  The thrust of the theory is its emphasis on the
importance of meta-learning, that is, {\it learning how to learn\/}, and
the role of the meta-knowledge in human learning. The basic tenets of this
theory include:

\begin{enumerate}
\item The single most important factor influencing human learning is what
  the learner already knows, i.e., prior knowledge;
  
\item Learning is evidenced by a change in the meaning of experience
  rather than a change in behavior, in contrast to the view long held by
  behavioral psychologists; and
  
\item The key role of the educator is to help students reflect on their
  experience (and hence give it new meanings), and construct meanings from
  the artifact in light of the changing experience.
\end{enumerate}

The view that learning is {\it meaning-making\/} places knowledge
representation at the focal point of the human learning process. KR defines
not only the form in which certain type of knowledge is highlighted to the
learner, but also the process by which such a form is derived.  In
addition, KR schemes, which are called {\it meta-cognitive tools\/} by
educational researchers, are the standard language for characterizing both
knowledge structures and corresponding cognitive structures. They help the
learner differentiate and organize newly acquired meanings. As part of the
theoretic formulation, Novak and Gowin \cite{Novak84} propose two knowledge
representation schemes: {\it concept maps\/} and {\it Vee diagrams\/}.
Figure \ref{fig:concept-map} shows an example of the concept map.  The next
section describes concept mapping --- a scheme that has been found quite
effective in enhancing science teaching \cite{Cliburn90,Novak90,Roth92} --- 
from a KR perspective, and identifies the problems that concept mapping
shares with semantic networks.

\begin{figure}
  \fbox{\centerline{\psfig{figure=Figures/concept-map.eps,width=5.5in}}}
  \caption{An example concept map on knowledge construction and
  acquisition from the perspective of the assimilation theory of
  cognitive learning (from [NG84])}
  \label{fig:concept-map}
\end{figure}


\subsection{Critiques on concept maps: a KR perspective}
\label{sec:concept-map}

Concept maps are similar to semantic networks --- a widely used KR scheme
first proposed by \cite{Quilian67} --- in that both represent knowledge as a
network of inter-connected nodes, where nodes correspond to concepts, and
links correspond to various types of relationships between these concepts.
While semantic nets are constructed by trained knowledge engineers solely
for computational manipulations, concept maps are built by learners
themselves as a means of understanding knowledge. The two also differ in a
more profound way. Figure \ref{fig:kr-example} illustrates a concept map
and a semantic net representation of ``the car is red.''  Although the two
express the same proposition, they use two very distinct link labels: the
semantic net uses ``color'' to indicate that ``red'' is the value of the
attribute {\it color\/} (instead of, for example, {\it make\/}); the
concept map, on the other hand, uses the generic verb {\it is\/} to express
the same semantics.  The latter expression, albeit more readable, is also
more ambiguous, which is a clear remnant of the natural language.

\begin{figure}
  \fbox{\centerline{\psfig{figure=Figures/kr-example.eps,height=2.0in}}}
  \caption{Concept maps versus semantic nets: an example}
  \label{fig:kr-example}
\end{figure}

Despite the above differences, concept maps share certain problems with
semantic nets. One such a criticism concerns their semantics: although
different types of nodes and links are used in semantic networks, their
exact semantics are often not specified or left ambiguous \cite{Woods85}.
The simple network in Figure \ref{fig:kr-example}, for example, may mean
either the definition of the concept of a blue car, or the assertion that
some car is blue. Like semantic nets, the concept map does not restrict the
type of nodes and links to be used, and hence gives the learner complete
freedom of deciding what to represent and how to represent it. This
flexibility, while making concept maps extremely expressive, adds little
structure that can be used as a basis of computation, and/or as an aid to
human learners in making sense of the map.  The latter is especially
significant in a collaborative setting, for the lack of shared semantics
will invariably make it difficult to compare, contrast, and integrate
concept maps generated by different learners.

The other major limitation of the concept map is {\it granularity\/}: all
knowledge must ultimately be reduced to concepts and propositional links
between them.  Though such constraint is acceptable for introductory
knowledge, it is inadequate for advanced learning (e.g., graduate
students). The latter often requires analysis and synthesis to be done at a
much higher levels, such as ``claims,'' ``problems,'' ``questions,''
``goals.''

The above deficiencies of concept maps call for an alternative scheme that
addresses issues specific to collaborative learning from scientific text.
The next section discusses the role and structure of scientific text. It
also introduces RESRA, the successor to concept maps.


\section{Representation of scientific text}
\label{sec:research artifacts}

Learning and research are traditionally viewed as very distinct activities:
one concerns the production of knowledge and the other, the acquisition or
reproduction of knowledge. This view, however, has been challenged by
constructivism, a currently dominant paradigm in both the sociology of
science and educational research \cite{Berger66,Knorr-Cetina81}. From the
constructionist point of view, learning is knowledge-building. Like
science, learning involves such activities as problem identification,
theorizing, hypothesis formulation and testing, refutation, and so on. Also
like science which is primarily a social activity involving a community of
scientists who share the same disciplinary knowledge, learning takes place
in the context of a community of learners who interact with each other in
an attempt to deepen their understanding of a particular subject
domain.

This section is based on the premise that scientific text is not only a
primary source of human knowledge but also an embodiment of the norms
governing scientific knowledge-building. It first discusses the role of
scientific text in scientific knowledge-building and in human
learning. Then, it describes three types of structures: {\it
presentational\/}, {\it rhetorical\/}, and {\it thematic\/}. It concludes
by introducing RESRA --- the representation that is based on the thematic
structure of scientific text, and intended to serve as a basis of
collaborative learning.


\subsection{The role of scientific text}
\label{sec:role}

The term {\it text\/} is used here in a broad sense to refer to any type of
written record of human knowledge and experience, including letters,
working papers, technical reports, journal articles, monographs, and so
forth. It comes with any form of media, printed, audio, and video.  Written
text plays a vital role in scientific knowledge-building. Specifically,
they serve the following four purposes:

\begin{itemize}
\item {\it As a formal channel of scientific communication.\/}
  Communication among researchers is done at two levels: {\it informal\/}
  and {\it formal\/}. The former includes direct interactions among
  scientists such as face-to-face conversations taking place in
  laboratories, hallways, conference rooms, or via telephones. The latter
  includes indirect exchanges through writings, e.g., letters, workshop
  and conference submissions, journal articles, monographs. The formality
  of the written text is based on the fact that it can exist, and thus be
  evaluated independent of the originator. The latter may lead to a
  higher-level of objectivity. The boundary between the two, however, is
  increasingly blurred by the wide use of electronic media.  E-mail, for
  example, is used for both informal and formal purposes. Similarly,
  digital journals are gradually acquiring an equal level of formality as
  their printed counterparts \cite{Harnad91,Gaines92}.
  
\item {\it As a measure of professional accomplishments.\/} The quantity
  and quality of written publications are often considered as a key
  indicator of research productivity and, hence, the basis of promotion,
  recognition, and prestige within the scientific community.
    
\item {\it As a primary repository of human knowledge.\/} Unlike
  textbooks, which contain pre-digested, often carefully-filtered
  knowledge, written text from the research front provides {\it
  knowledge-in-progress\/}, which may range from very preliminary ideas
  to coherent, well-established theories. They also allow the student to
  see how conflicts among competing scientific explanations are resolved,
  and how early explanations succeeded by more recent and, presumably,
  more valid ones.
  
\item {\it As an important data source for studying scientific
  discourse.\/} Scientific text is seldom a verbatim record of what
  actually takes place in the laboratory or on the field. Rather, the
  decision on what to report (and not to report), and how to report it is
  often the result of a complex social process that involves the interplay
  of many factors. Scientific publications, like other types of writings,
  are rhetorical artifacts whose structures are shaped by the then-dominant
  paradigm of a particular field. Since it is not always possible to study
  scientists in their working place, scientific text is increasingly being
  used as a source to study the process of knowledge-building among
  scientists and the evolution of human knowledge over time
  \cite{Selzer93,Bazerman88}
\end{itemize}

From the learner's perspective, the differentiation of the above
functions is significant in two ways. First, when learning in a classroom
setting is viewed as knowledge-building, written text may be used for the
same purpose as they are used in the real research context, for example, as
a formal means of sharing knowledge among learners, and a primary data
source for studying ``classroom discourse.'' Second, the last two functions
described above are especially relevant to the learning context. Written
text from the research community contains ``knowledge-in-progress'' and
embedded discourse structures of the scientific knowledge-building that are
often absent from standard textbooks or lectures. By studying and analyzing
them, the student can gain a better understanding of the nature of human
knowledge as well as the process through which such knowledge is
constructed and evolved.


\subsection{The structure of scientific text}

The structure of scientific text, that is, the ways in which scientific
theories, findings, and evidence are presented in the form of written
artifact, often varies from one discipline to another, and from one type of
artifact to another (e.g., laboratory experiments versus field studies).
Although it is not plausible to enumerate all possible structural types,
scientific text can be analyzed at three levels: {\it presentational\/},
{\it rhetorical\/}, and {\it thematic\/}. At the presentational level, a
research artifact is broken down into a hierarchy of chapters, sections,
subsections, et al. Corresponding to each is a header, such as
``abstract,'' ``introduction,'' ``experimental design,'' ``related work,''
``conclusions.'' Such structures provide useful pointers to the {\it
type\/} of content that is immediately followed. However, they do not
represent the content itself. Presentational structures are standardized
through stylistic guidelines established for a given discipline or journal.

The second type of structure is {\it rhetorical\/}, which concerns the way
in which scientific arguments are presented, supported, and refuted.
Rhetorical models, such as the one by \cite{Toulmin84}, provide a viable
means of understanding competing formulations or explanations about a given
phenomenon. In particular, they are useful for representing
inter-relationships among different artifacts, and the evolution of
scientific formulations over time. For example, a journal might publish a
special issue on a selected topic, that consists of a single ``feature''
article and a series of ``reaction'' articles written by researchers from
different ``schools of thought.'' The structural pattern of these papers
can be captured using a rhetorical model. Interest in the rhetorical
structure of scientific text has been mounting in recent years
\cite{Bazerman88,Selzer93,Simons90}. Research efforts have also started in
applying the same approach to human learning (e.g.,
\cite{Cavalli-Sforza92}).

The third type of structure is {\it thematic\/}. As the name implies, the
thematic structure characterizes the important features or {\it themes\/}
of scientific text, and the relationships between these themes.  Two
example {\it themes\/} are {\sf problem\/} and {\sf claim\/}, and an
example relationship between them might be {\sf claim} \( \stackrel{
responds-to}{\longrightarrow} \) {\sf problem}. Like the rhetorical
structure, the thematic structure is content-oriented. Compared to the
rhetorical structure, the thematic structure is more general and flexible:
thematic models can be developed to capture both discursive and domain
structures, both intra-artifact and inter-artifact relationships.  For
example, a thematic model might include primitives such as {\sf concept\/},
{\sf claim\/}, which, when instantiated into the domain of software
engineering, can include ``software complexity'' ({\sf concept\/}),
``object-oriented-design'' ({\sf concept\/}), and ``Object-oriented design
offers a viable solution to software complexity'' ({\sf claim\/}). RESRA --
the representation to be introduced next and described in details in
Chapter 3 --- is based on the thematic model.


\subsection{RESRA: a KR scheme for representing scientific text}
\label{sec:resra1}

Given the importance of knowledge representation and scientific text in
human learning, and the limitations of concept mapping (see Section
\ref{sec:concept-map}), a new representation scheme is proposed. This new
scheme, called RESRA\footnote{RESRA, which stands for ``REpresentational Schema
of Research Artifacts,'' is a specialized language for representing the
thematic structure of research and learning artifacts generated from both
within and without classrooms. A detailed description of RESRA constructs
is provided in Chapter 3.}, is intended for modeling the thematic structure
of scientific text, and for serving as a shared framework for collaborative
learning.

RESRA provides 11 node and 20 link primitive types (see Table
\ref{tab:resra} for a summary of RESRA node types, and Figure
\ref{fig:sum-resra} and \ref{fig:eval-resra} for link types). The
derivation of these primitives is based on several sources, both theoretic
and empirical. The main theoretical sources are two: Vee heuristic
\cite{Novak84} and Bloom's taxonomy \cite{Bloom56}. The empirical sources
include case analyses of the structure of various artifacts and a number of
rounds of experimental use of the representation.


\ls{1.0}
\small
\begin{table}
  \caption{A synopsis of RESRA primitive node types}
  \begin{center}
    \begin{tabular} {||l|p{2.25in}|p{2.25in}||} \hline   
      {\bf Node Type} & {\bf Description} & {\bf Example} \\ \hline \hline
      
      Problem  & A phenomenon, event, or process whose
      understanding requires further inquiry; & Meta-learning is not
      adequately supported by existing tools. \\ \hline
      
      Claim & A position or proposition about a given problem
      situation.  & Cleanroom engineering provides a viable solution
      in producing zero defect software. \\ \hline
      
      Evidence & Data gathered for the purpose of supporting or
      refuting a given claim. & The use of cleanroom techniques has
      yielded a 10-fold reduction of defects in the project Alpha. \\
      \hline \hline

      Theory & A systemic formulation about a particular problem
      domain, derivable through deductive or inductive procedures. &
      Ausubel's theory of meaningful learning. \\ \hline
      
      Method & Procedures or techniques used for generating evidence
      for a particular claim. & the Delphi study, nominal grouping
      techniques, the Waterfall system development model. \\ \hline
      
      Concept & A primitive construct used in formulating theory,
      claim, or method. & meta-learning; knowledge representation.
      \\ \hline  \hline
      
      Thing & A natural or man-made object that is under study.  &
      Atom, NoteCards.  \\ \hline
      
      Source & An identifiable written artifact, either artifact
      itself or the pointer to it, i.e., surrogate or reference. & An
      article by Ashton; the notes from Kyle's talk. \\ \hline \hline
      
      Critique & Critical remarks or comments about a particular
      claim, evidence, method, source, et al., or relationships
      between them. & The example applications of cleanroom
      engineering so far have been limited to well-defined domains.
      \\ \hline
      
      Question & Aspects of a claim, theory, concept, etc., about
      which the learner is still in doubt. & How does box-structured
      design differ from object-oriented approach? \\ \hline
      
      Suggestion & Ideas, recommendations, or feedbacks on how to
      improve an existing problem statement, claim, method, et al.  &
      I would like to see cleanroom engineering used in some
      non-conventional domains, such as groupware. \\ \hline
    \end{tabular}    
  \end{center}
    \label{tab:resra}
\end{table}
\normalsize
\ls{1.6}


The relationship between RESRA and concept mapping is similar to the
relationship between semantic networks and Schank's conceptual dependency
theory (CD) \cite{schank75}: CD responds to the problem of the lack of
semantics in semantic networks by proposing a small set of node and link
primitives that can be used to represent basic conceptual categories and
the relationships between them.  Similarly, RESRA overcomes the problem
related to the unconstrained form of expression in concept mapping with its
own set of node and link primitives. Furthermore, RESRA also addresses the
fine-granularity problem of concept mapping: RESRA nodes are not limited to
atomic constructs, such as ``concepts''; they may be used to represent any
type of complex propositions.

RESRA belongs to the content theory of knowledge representation; it
concerns the type of knowledge that needs to be represented but not how
this knowledge is used by human learners\cite{Swaminathan90}. Two major
criticisms have been raised on the content theory: one is related to the
large amount of efforts often required to translate, for example, a
sentence into the underlying representation. Furthermore, the mapping from
the text under study into the primitives is usually not unique. The latter,
however, does not constitute a problem in RESRA since it is expected that
individual learners construct different representations of the same
artifact.  The existence of these individual differences is a prerequisite
for group synergy. The first problem (i.e., time-consuming), however, still
exists in RESRA.

The second criticism is that primitives supplied by the content theory is
usually incomplete with respect to the world it attempts to model
\cite{Swaminathan90}. For example, it is likely that one may find thematic
features that do not neatly fit in the existing node categories of RESRA.
However, RESRA, unlike CD, does not claim that the primitive set is
exhaustive. In contrary, it recognizes that, in human learning, it is
neither possible nor desirable to identify all potential primitive types at
the outset. The undesirability is related to the usability of a
representation language: the larger the set of primitives, the less usable
the representation language, because the more efforts the learner has to
devote to learn and use that language. RESRA allows the learner to define
their own primitives if necessary.

One of the main characteristics of RESRA is that it is designed in
conjunction with the SECAI learning process. RESRA primitives, for example,
can be partitioned into {\it layers\/} corresponding to the steps in the
SECAI process, e.g., summarization, evaluation. The next section defines a
model of collaborative learning and highlights the importance of
representation in that model.


\section{Toward a representation-based model of collaborative learning}
\label{sec:model}

While RESRA provides a conceptual framework for representing the structural
features of scientific text, the process-level question remains open: how
do learners go about learning collaboratively from scientific text? This
section describes SECAI, a two-phase, five-step process model of
collaborative learning\footnote{SECAI stands for the five key components of the
collaborative learning model, i.e., Summarization, Evaluation, Comparison,
Argumentation, and Integration.}.


\subsection{Collaborative learning from scientific text}
\label{sec:artifact-based}

Although it may take place in individual, isolated contexts, learning from
scientific text can be most profitably done in a collaborative setting.
The reason is twofold. First, since scientific text is undigested and hence
less systematic compared to textbooks and lectures, interpreting and
understanding its content is not always an easy task.  Because of the
differences among learners in their backgrounds, skills, perspectives, and
experiences, they are likely to have distinct views on the artifact. A
group setting permits the learners to share their views with each other,
and thus enables all of them to gain a better understanding of the
artifact.  Second, like scientific research, learning is a process
involving social construction of knowledge: learners should not only learn
from scientific text but more importantly, from each other through
perspective and knowledge sharing. In addition, since scientific text is
the product of the collaboration among scientists, it sets an example for
the learners on how to form their ``knowledge-building'' communities, how
to jointly construct new knowledge among themselves, and so forth.


\subsection{SECAI: the collaborative learning model}
\label{sec:secai}

The SECAI model defines a process for collaboratively learning from
scientific text. The process consists of two phases: {\it exploration\/}
and {\it consolidation\/}. Exploration requires interpretations and
evaluations of the content of a scientific artifact by individual
learners. It in turn is composed of two steps: {\it summarization\/} and
{\it evaluation\/}. Exploration is done privately; each learner must
independently derive his own representation and assessment. As a result, a
learner cannot either be swayed by, nor free-ride off another learner's
points of view.  The result from this step, namely, the summarative and
evaluative representations by individual learners, forms a basis for the
subsequent consolidation phase.

The consolidation phase is the {\it public\/} phase of the SECAI
process. It consists of three components: {\it comparison\/}, {\it
argumentation\/}, and {\it integration\/}. Consolidation brings together
individual interpretations and views on a given artifact. It allows
learners to compare, question, critique, defend, relate, and integrate
these interpretations and evaluations. As a result of these interactions,
learners can develop a deeper and more complete understanding of the
underlying artifact, and the perspectives of other learners.  As a side
benefit, they also create a group knowledge base that captures the above
deliberation process.

The five process steps of SECAI, namely, {\it summarization\/}, {\it
evaluation\/}, {\it comparison\/}, {\it argumentation\/}, and {\it
integration\/}, are summarized in Table \ref{tab:secai}. Figure
\ref{fig:secai-2} shows the relationships between the five components. It
illustrates SECAI as a process of collaborative knowledge-building based on
the scientific artifact. The world outside the outmost circle contains
scientific text that forms the starting point of learning. The big shaded
arrows indicate the direction of the group process, which begins with {\it
summarization \& evaluation\/}\footnote{There are two main reasons for
showing summarization and evaluation as one combined step in this diagram:
(1) the emphasis of this diagram is on the group collaboration, while both
summarization and evaluation are individual activities as defined in SECAI;
and (2) Both activities are viewed as bootstrapping in the current view,
and they are often invoked in an intertwined rather than sequential
manner.}. It also shows that, as the process moves inward, the amount of
interactions among the group members increases and, concomitantly, the
group knowledge converges.  The ultimate result is a dynamic group
knowledge base that integrates different interpretations, deliberations,
and extensions of the subject content of the artifact.

\ls{1.0}
\small
\begin{table}
  \caption{Five principal activities of collaborative learning from
  scientific text}
  \begin{center}
    \begin{tabular} {||l|p{4.5in}||} \hline   
      
      {\bf Activity} & {\bf Description} \\ \hline
      
      Summarization & Extracting, condensing, and relating important
      themes from an artifact. \\ \hline
      
      Evaluation & Subjective assessment of merits and soundness of the
      author's work. \\ \hline
      
      Comparison & Finding out and highlighting differences and
      similarities among different points of view. \\ \hline

      Argumentation & Challenging others' positions; defending one's
      own positions; proposing alternative solutions. \\ \hline
      
      Integration & Declaring similarity, subsuming, same-perspective
      relationships between nodes; endorsing nodes and links. \\
      \hline
    \end{tabular}    
  \end{center}
    \label{tab:secai}
\end{table}
\normalsize
\ls{1.6}

\begin{figure}
  \fbox{\centerline{\psfig{figure=Figures/2-learning-community.eps,width=4.5in}}}
  \caption{Collaborative learning from scientific text}
  \label{fig:secai-2}
\end{figure}


\subsection{The role of the RESRA representation}

Figure \ref{fig:kr-role} presents another view of SECAI. This view
emphasizes an indispensable role of RESRA in the collaborative learning
process. It shows that none of the SECAI activities is unbound, open-ended.
Rather, they are all guided and constrained by RESRA, which serves as a
{\it glue\/} that ties together:

\begin{itemize}
\item {\it Exploration\/} and {\it consolidation\/} phases; and
  
\item Different interpretations and points of view from individual
  learners.
\end{itemize}

In addition, since the thematic features of scientific text are {\it
summarized\/} and {\it evaluated\/} in terms of RESRA, and that the group
deliberation is also done within the RESRA framework, the representation
also forms a bridge that connects together knowledge-building in the
scientific community and knowledge-building in CLARE-mediated classroom
settings.

\begin{figure}
  \fbox{\centerline{\psfig{figure=Figures/secai.eps,width=4.5in}}}
  \caption{The role of RESRA in collaborative learning}
  \label{fig:kr-role}
\end{figure}


\section{Summary and conclusions}
\label{sec:summary}

This chapter focuses on knowledge representation in human learning, in
particular, human meta-learning. It attempts to integrate the technological
and representational developments of KR into the conceptual framework of
the cognitive learning theory to form a theory-based approach to supporting
human learning. In doing so, it identifies several problems with concept
mapping --- the theorists' proposal to support meta-learning. The importance
of scientific text in human learning is highlighted, and the structure of
scientific text is described. A new representation language called RESRA is
proposed to overcome certain weaknesses of existing KR schemes, and to
provide unique support for collaborative learning from scientific text.
This representation, along with the SECAI process model, forms the
conceptual basis of the CLARE system. The next chapter describes the RESRA
representation language, and Chapter 4 describes the CLARE system.


%%% \begin{itemize}
%%% \item As a mapping tool that highlights essential elements, i.e.,
%%%   ``thematic components,'' and the relationships between them;
%%%   
%%% \item As an organizational tool that enables the learner to dynamically and
%%%   incrementally integrate various types of artifacts at a fine-grain
%%%   level;
%%%   
%%% \item As a communication tool, i.e., a ``shared frame of reference'' in a
%%%   collaborative setting; by comparing and contrasting representations of an
%%%   artifact by individual learners, it can highlight differences and
%%%   similarities between them; and
%%%   
%%% \item As a tool for learning about the process of scientific
%%%   knowledge-building, and the norms governing the written presentation of
%%%   scientific knowledge.
%%% \end{itemize}

%%% \newpage
%%% \singlespace
%%% \bibliography{../bib/clare}
%%% \bibliographystyle{alpha}
%%% 
%%% 
%%%\end{document}








