%%% \documentstyle [12pt,/group/csdl/tex/definemargins,
%%% /group/csdl/tex/lmacros]{report}
%%% % Psfig/TeX 
\def\PsfigVersion{1.9}
% dvips version
%
% All psfig/tex software, documentation, and related files
% in this distribution of psfig/tex are 
% Copyright 1987, 1988, 1991 Trevor J. Darrell
%
% Permission is granted for use and non-profit distribution of psfig/tex 
% providing that this notice is clearly maintained. The right to
% distribute any portion of psfig/tex for profit or as part of any commercial
% product is specifically reserved for the author(s) of that portion.
%
% *** Feel free to make local modifications of psfig as you wish,
% *** but DO NOT post any changed or modified versions of ``psfig''
% *** directly to the net. Send them to me and I'll try to incorporate
% *** them into future versions. If you want to take the psfig code 
% *** and make a new program (subject to the copyright above), distribute it, 
% *** (and maintain it) that's fine, just don't call it psfig.
%
% Bugs and improvements to trevor@media.mit.edu.
%
% Thanks to Greg Hager (GDH) and Ned Batchelder for their contributions
% to the original version of this project.
%
% Modified by J. Daniel Smith on 9 October 1990 to accept the
% %%BoundingBox: comment with or without a space after the colon.  Stole
% file reading code from Tom Rokicki's EPSF.TEX file (see below).
%
% More modifications by J. Daniel Smith on 29 March 1991 to allow the
% the included PostScript figure to be rotated.  The amount of
% rotation is specified by the "angle=" parameter of the \psfig command.
%
% Modified by Robert Russell on June 25, 1991 to allow users to specify
% .ps filenames which don't yet exist, provided they explicitly provide
% boundingbox information via the \psfig command. Note: This will only work
% if the "file=" parameter follows all four "bb???=" parameters in the
% command. This is due to the order in which psfig interprets these params.
%
%  3 Jul 1991	JDS	check if file already read in once
%  4 Sep 1991	JDS	fixed incorrect computation of rotated
%			bounding box
% 25 Sep 1991	GVR	expanded synopsis of \psfig
% 14 Oct 1991	JDS	\fbox code from LaTeX so \psdraft works with TeX
%			changed \typeout to \ps@typeout
% 17 Oct 1991	JDS	added \psscalefirst and \psrotatefirst
%

% From: gvr@cs.brown.edu (George V. Reilly)
%
% \psdraft	draws an outline box, but doesn't include the figure
%		in the DVI file.  Useful for previewing.
%
% \psfull	includes the figure in the DVI file (default).
%
% \psscalefirst width= or height= specifies the size of the figure
% 		before rotation.
% \psrotatefirst (default) width= or height= specifies the size of the
% 		 figure after rotation.  Asymetric figures will
% 		 appear to shrink.
%
% \psfigurepath#1	sets the path to search for the figure
%
% \psfig
% usage: \psfig{file=, figure=, height=, width=,
%			bbllx=, bblly=, bburx=, bbury=,
%			rheight=, rwidth=, clip=, angle=, silent=}
%
%	"file" is the filename.  If no path name is specified and the
%		file is not found in the current directory,
%		it will be looked for in directory \psfigurepath.
%	"figure" is a synonym for "file".
%	By default, the width and height of the figure are taken from
%		the BoundingBox of the figure.
%	If "width" is specified, the figure is scaled so that it has
%		the specified width.  Its height changes proportionately.
%	If "height" is specified, the figure is scaled so that it has
%		the specified height.  Its width changes proportionately.
%	If both "width" and "height" are specified, the figure is scaled
%		anamorphically.
%	"bbllx", "bblly", "bburx", and "bbury" control the PostScript
%		BoundingBox.  If these four values are specified
%               *before* the "file" option, the PSFIG will not try to
%               open the PostScript file.
%	"rheight" and "rwidth" are the reserved height and width
%		of the figure, i.e., how big TeX actually thinks
%		the figure is.  They default to "width" and "height".
%	The "clip" option ensures that no portion of the figure will
%		appear outside its BoundingBox.  "clip=" is a switch and
%		takes no value, but the `=' must be present.
%	The "angle" option specifies the angle of rotation (degrees, ccw).
%	The "silent" option makes \psfig work silently.
%

% check to see if macros already loaded in (maybe some other file says
% "\input psfig") ...
\ifx\undefined\psfig\else\endinput\fi

%
% from a suggestion by eijkhout@csrd.uiuc.edu to allow
% loading as a style file. Changed to avoid problems
% with amstex per suggestion by jbence@math.ucla.edu

\let\LaTeXAtSign=\@
\let\@=\relax
\edef\psfigRestoreAt{\catcode`\@=\number\catcode`@\relax}
%\edef\psfigRestoreAt{\catcode`@=\number\catcode`@\relax}
\catcode`\@=11\relax
\newwrite\@unused
\def\ps@typeout#1{{\let\protect\string\immediate\write\@unused{#1}}}
\ps@typeout{psfig/tex \PsfigVersion}

%% Here's how you define your figure path.  Should be set up with null
%% default and a user useable definition.

\def\figurepath{./}
\def\psfigurepath#1{\edef\figurepath{#1}}

%
% @psdo control structure -- similar to Latex @for.
% I redefined these with different names so that psfig can
% be used with TeX as well as LaTeX, and so that it will not 
% be vunerable to future changes in LaTeX's internal
% control structure,
%
\def\@nnil{\@nil}
\def\@empty{}
\def\@psdonoop#1\@@#2#3{}
\def\@psdo#1:=#2\do#3{\edef\@psdotmp{#2}\ifx\@psdotmp\@empty \else
    \expandafter\@psdoloop#2,\@nil,\@nil\@@#1{#3}\fi}
\def\@psdoloop#1,#2,#3\@@#4#5{\def#4{#1}\ifx #4\@nnil \else
       #5\def#4{#2}\ifx #4\@nnil \else#5\@ipsdoloop #3\@@#4{#5}\fi\fi}
\def\@ipsdoloop#1,#2\@@#3#4{\def#3{#1}\ifx #3\@nnil 
       \let\@nextwhile=\@psdonoop \else
      #4\relax\let\@nextwhile=\@ipsdoloop\fi\@nextwhile#2\@@#3{#4}}
\def\@tpsdo#1:=#2\do#3{\xdef\@psdotmp{#2}\ifx\@psdotmp\@empty \else
    \@tpsdoloop#2\@nil\@nil\@@#1{#3}\fi}
\def\@tpsdoloop#1#2\@@#3#4{\def#3{#1}\ifx #3\@nnil 
       \let\@nextwhile=\@psdonoop \else
      #4\relax\let\@nextwhile=\@tpsdoloop\fi\@nextwhile#2\@@#3{#4}}
% 
% \fbox is defined in latex.tex; so if \fbox is undefined, assume that
% we are not in LaTeX.
% Perhaps this could be done better???
\ifx\undefined\fbox
% \fbox code from modified slightly from LaTeX
\newdimen\fboxrule
\newdimen\fboxsep
\newdimen\ps@tempdima
\newbox\ps@tempboxa
\fboxsep = 3pt
\fboxrule = .4pt
\long\def\fbox#1{\leavevmode\setbox\ps@tempboxa\hbox{#1}\ps@tempdima\fboxrule
    \advance\ps@tempdima \fboxsep \advance\ps@tempdima \dp\ps@tempboxa
   \hbox{\lower \ps@tempdima\hbox
  {\vbox{\hrule height \fboxrule
          \hbox{\vrule width \fboxrule \hskip\fboxsep
          \vbox{\vskip\fboxsep \box\ps@tempboxa\vskip\fboxsep}\hskip 
                 \fboxsep\vrule width \fboxrule}
                 \hrule height \fboxrule}}}}
\fi
%
%%%%%%%%%%%%%%%%%%%%%%%%%%%%%%%%%%%%%%%%%%%%%%%%%%%%%%%%%%%%%%%%%%%
% file reading stuff from epsf.tex
%   EPSF.TEX macro file:
%   Written by Tomas Rokicki of Radical Eye Software, 29 Mar 1989.
%   Revised by Don Knuth, 3 Jan 1990.
%   Revised by Tomas Rokicki to accept bounding boxes with no
%      space after the colon, 18 Jul 1990.
%   Portions modified/removed for use in PSFIG package by
%      J. Daniel Smith, 9 October 1990.
%
\newread\ps@stream
\newif\ifnot@eof       % continue looking for the bounding box?
\newif\if@noisy        % report what you're making?
\newif\if@atend        % %%BoundingBox: has (at end) specification
\newif\if@psfile       % does this look like a PostScript file?
%
% PostScript files should start with `%!'
%
{\catcode`\%=12\global\gdef\epsf@start{%!}}
\def\epsf@PS{PS}
%
\def\epsf@getbb#1{%
%
%   The first thing we need to do is to open the
%   PostScript file, if possible.
%
\openin\ps@stream=#1
\ifeof\ps@stream\ps@typeout{Error, File #1 not found}\else
%
%   Okay, we got it. Now we'll scan lines until we find one that doesn't
%   start with %. We're looking for the bounding box comment.
%
   {\not@eoftrue \chardef\other=12
    \def\do##1{\catcode`##1=\other}\dospecials \catcode`\ =10
    \loop
       \if@psfile
	  \read\ps@stream to \epsf@fileline
       \else{
	  \obeyspaces
          \read\ps@stream to \epsf@tmp\global\let\epsf@fileline\epsf@tmp}
       \fi
       \ifeof\ps@stream\not@eoffalse\else
%
%   Check the first line for `%!'.  Issue a warning message if its not
%   there, since the file might not be a PostScript file.
%
       \if@psfile\else
       \expandafter\epsf@test\epsf@fileline:. \\%
       \fi
%
%   We check to see if the first character is a % sign;
%   if so, we look further and stop only if the line begins with
%   `%%BoundingBox:' and the `(atend)' specification was not found.
%   That is, the only way to stop is when the end of file is reached,
%   or a `%%BoundingBox: llx lly urx ury' line is found.
%
          \expandafter\epsf@aux\epsf@fileline:. \\%
       \fi
   \ifnot@eof\repeat
   }\closein\ps@stream\fi}%
%
% This tests if the file we are reading looks like a PostScript file.
%
\long\def\epsf@test#1#2#3:#4\\{\def\epsf@testit{#1#2}
			\ifx\epsf@testit\epsf@start\else
\ps@typeout{Warning! File does not start with `\epsf@start'.  It may not be a PostScript file.}
			\fi
			\@psfiletrue} % don't test after 1st line
%
%   We still need to define the tricky \epsf@aux macro. This requires
%   a couple of magic constants for comparison purposes.
%
{\catcode`\%=12\global\let\epsf@percent=%\global\def\epsf@bblit{%BoundingBox}}
%
%
%   So we're ready to check for `%BoundingBox:' and to grab the
%   values if they are found.  We continue searching if `(at end)'
%   was found after the `%BoundingBox:'.
%
\long\def\epsf@aux#1#2:#3\\{\ifx#1\epsf@percent
   \def\epsf@testit{#2}\ifx\epsf@testit\epsf@bblit
	\@atendfalse
        \epsf@atend #3 . \\%
	\if@atend	
	   \if@verbose{
		\ps@typeout{psfig: found `(atend)'; continuing search}
	   }\fi
        \else
        \epsf@grab #3 . . . \\%
        \not@eoffalse
        \global\no@bbfalse
        \fi
   \fi\fi}%
%
%   Here we grab the values and stuff them in the appropriate definitions.
%
\def\epsf@grab #1 #2 #3 #4 #5\\{%
   \global\def\epsf@llx{#1}\ifx\epsf@llx\empty
      \epsf@grab #2 #3 #4 #5 .\\\else
   \global\def\epsf@lly{#2}%
   \global\def\epsf@urx{#3}\global\def\epsf@ury{#4}\fi}%
%
% Determine if the stuff following the %%BoundingBox is `(atend)'
% J. Daniel Smith.  Copied from \epsf@grab above.
%
\def\epsf@atendlit{(atend)} 
\def\epsf@atend #1 #2 #3\\{%
   \def\epsf@tmp{#1}\ifx\epsf@tmp\empty
      \epsf@atend #2 #3 .\\\else
   \ifx\epsf@tmp\epsf@atendlit\@atendtrue\fi\fi}


% End of file reading stuff from epsf.tex
%%%%%%%%%%%%%%%%%%%%%%%%%%%%%%%%%%%%%%%%%%%%%%%%%%%%%%%%%%%%%%%%%%%

%%%%%%%%%%%%%%%%%%%%%%%%%%%%%%%%%%%%%%%%%%%%%%%%%%%%%%%%%%%%%%%%%%%
% trigonometry stuff from "trig.tex"
\chardef\psletter = 11 % won't conflict with \begin{letter} now...
\chardef\other = 12

\newif \ifdebug %%% turn me on to see TeX hard at work ...
\newif\ifc@mpute %%% don't need to compute some values
\c@mputetrue % but assume that we do

\let\then = \relax
\def\r@dian{pt }
\let\r@dians = \r@dian
\let\dimensionless@nit = \r@dian
\let\dimensionless@nits = \dimensionless@nit
\def\internal@nit{sp }
\let\internal@nits = \internal@nit
\newif\ifstillc@nverging
\def \Mess@ge #1{\ifdebug \then \message {#1} \fi}

{ %%% Things that need abnormal catcodes %%%
	\catcode `\@ = \psletter
	\gdef \nodimen {\expandafter \n@dimen \the \dimen}
	\gdef \term #1 #2 #3%
	       {\edef \t@ {\the #1}%%% freeze parameter 1 (count, by value)
		\edef \t@@ {\expandafter \n@dimen \the #2\r@dian}%
				   %%% freeze parameter 2 (dimen, by value)
		\t@rm {\t@} {\t@@} {#3}%
	       }
	\gdef \t@rm #1 #2 #3%
	       {{%
		\count 0 = 0
		\dimen 0 = 1 \dimensionless@nit
		\dimen 2 = #2\relax
		\Mess@ge {Calculating term #1 of \nodimen 2}%
		\loop
		\ifnum	\count 0 < #1
		\then	\advance \count 0 by 1
			\Mess@ge {Iteration \the \count 0 \space}%
			\Multiply \dimen 0 by {\dimen 2}%
			\Mess@ge {After multiplication, term = \nodimen 0}%
			\Divide \dimen 0 by {\count 0}%
			\Mess@ge {After division, term = \nodimen 0}%
		\repeat
		\Mess@ge {Final value for term #1 of 
				\nodimen 2 \space is \nodimen 0}%
		\xdef \Term {#3 = \nodimen 0 \r@dians}%
		\aftergroup \Term
	       }}
	\catcode `\p = \other
	\catcode `\t = \other
	\gdef \n@dimen #1pt{#1} %%% throw away the ``pt''
}

\def \Divide #1by #2{\divide #1 by #2} %%% just a synonym

\def \Multiply #1by #2%%% allows division of a dimen by a dimen
       {{%%% should really freeze parameter 2 (dimen, passed by value)
	\count 0 = #1\relax
	\count 2 = #2\relax
	\count 4 = 65536
	\Mess@ge {Before scaling, count 0 = \the \count 0 \space and
			count 2 = \the \count 2}%
	\ifnum	\count 0 > 32767 %%% do our best to avoid overflow
	\then	\divide \count 0 by 4
		\divide \count 4 by 4
	\else	\ifnum	\count 0 < -32767
		\then	\divide \count 0 by 4
			\divide \count 4 by 4
		\else
		\fi
	\fi
	\ifnum	\count 2 > 32767 %%% while retaining reasonable accuracy
	\then	\divide \count 2 by 4
		\divide \count 4 by 4
	\else	\ifnum	\count 2 < -32767
		\then	\divide \count 2 by 4
			\divide \count 4 by 4
		\else
		\fi
	\fi
	\multiply \count 0 by \count 2
	\divide \count 0 by \count 4
	\xdef \product {#1 = \the \count 0 \internal@nits}%
	\aftergroup \product
       }}

\def\r@duce{\ifdim\dimen0 > 90\r@dian \then   % sin(x+90) = sin(180-x)
		\multiply\dimen0 by -1
		\advance\dimen0 by 180\r@dian
		\r@duce
	    \else \ifdim\dimen0 < -90\r@dian \then  % sin(-x) = sin(360+x)
		\advance\dimen0 by 360\r@dian
		\r@duce
		\fi
	    \fi}

\def\Sine#1%
       {{%
	\dimen 0 = #1 \r@dian
	\r@duce
	\ifdim\dimen0 = -90\r@dian \then
	   \dimen4 = -1\r@dian
	   \c@mputefalse
	\fi
	\ifdim\dimen0 = 90\r@dian \then
	   \dimen4 = 1\r@dian
	   \c@mputefalse
	\fi
	\ifdim\dimen0 = 0\r@dian \then
	   \dimen4 = 0\r@dian
	   \c@mputefalse
	\fi
%
	\ifc@mpute \then
        	% convert degrees to radians
		\divide\dimen0 by 180
		\dimen0=3.141592654\dimen0
%
		\dimen 2 = 3.1415926535897963\r@dian %%% a well-known constant
		\divide\dimen 2 by 2 %%% we only deal with -pi/2 : pi/2
		\Mess@ge {Sin: calculating Sin of \nodimen 0}%
		\count 0 = 1 %%% see power-series expansion for sine
		\dimen 2 = 1 \r@dian %%% ditto
		\dimen 4 = 0 \r@dian %%% ditto
		\loop
			\ifnum	\dimen 2 = 0 %%% then we've done
			\then	\stillc@nvergingfalse 
			\else	\stillc@nvergingtrue
			\fi
			\ifstillc@nverging %%% then calculate next term
			\then	\term {\count 0} {\dimen 0} {\dimen 2}%
				\advance \count 0 by 2
				\count 2 = \count 0
				\divide \count 2 by 2
				\ifodd	\count 2 %%% signs alternate
				\then	\advance \dimen 4 by \dimen 2
				\else	\advance \dimen 4 by -\dimen 2
				\fi
		\repeat
	\fi		
			\xdef \sine {\nodimen 4}%
       }}

% Now the Cosine can be calculated easily by calling \Sine
\def\Cosine#1{\ifx\sine\UnDefined\edef\Savesine{\relax}\else
		             \edef\Savesine{\sine}\fi
	{\dimen0=#1\r@dian\advance\dimen0 by 90\r@dian
	 \Sine{\nodimen 0}
	 \xdef\cosine{\sine}
	 \xdef\sine{\Savesine}}}	      
% end of trig stuff
%%%%%%%%%%%%%%%%%%%%%%%%%%%%%%%%%%%%%%%%%%%%%%%%%%%%%%%%%%%%%%%%%%%%

\def\psdraft{
	\def\@psdraft{0}
	%\ps@typeout{draft level now is \@psdraft \space . }
}
\def\psfull{
	\def\@psdraft{100}
	%\ps@typeout{draft level now is \@psdraft \space . }
}

\psfull

\newif\if@scalefirst
\def\psscalefirst{\@scalefirsttrue}
\def\psrotatefirst{\@scalefirstfalse}
\psrotatefirst

\newif\if@draftbox
\def\psnodraftbox{
	\@draftboxfalse
}
\def\psdraftbox{
	\@draftboxtrue
}
\@draftboxtrue

\newif\if@prologfile
\newif\if@postlogfile
\def\pssilent{
	\@noisyfalse
}
\def\psnoisy{
	\@noisytrue
}
\psnoisy
%%% These are for the option list.
%%% A specification of the form a = b maps to calling \@p@@sa{b}
\newif\if@bbllx
\newif\if@bblly
\newif\if@bburx
\newif\if@bbury
\newif\if@height
\newif\if@width
\newif\if@rheight
\newif\if@rwidth
\newif\if@angle
\newif\if@clip
\newif\if@verbose
\def\@p@@sclip#1{\@cliptrue}


\newif\if@decmpr

%%% GDH 7/26/87 -- changed so that it first looks in the local directory,
%%% then in a specified global directory for the ps file.
%%% RPR 6/25/91 -- changed so that it defaults to user-supplied name if
%%% boundingbox info is specified, assuming graphic will be created by
%%% print time.
%%% TJD 10/19/91 -- added bbfile vs. file distinction, and @decmpr flag

\def\@p@@sfigure#1{\def\@p@sfile{null}\def\@p@sbbfile{null}
	        \openin1=#1.bb
		\ifeof1\closein1
	        	\openin1=\figurepath#1.bb
			\ifeof1\closein1
			        \openin1=#1
				\ifeof1\closein1%
				       \openin1=\figurepath#1
					\ifeof1
					   \ps@typeout{Error, File #1 not found}
						\if@bbllx\if@bblly
				   		\if@bburx\if@bbury
			      				\def\@p@sfile{#1}%
			      				\def\@p@sbbfile{#1}%
							\@decmprfalse
				  	   	\fi\fi\fi\fi
					\else\closein1
				    		\def\@p@sfile{\figurepath#1}%
				    		\def\@p@sbbfile{\figurepath#1}%
						\@decmprfalse
	                       		\fi%
			 	\else\closein1%
					\def\@p@sfile{#1}
					\def\@p@sbbfile{#1}
					\@decmprfalse
			 	\fi
			\else
				\def\@p@sfile{\figurepath#1}
				\def\@p@sbbfile{\figurepath#1.bb}
				\@decmprtrue
			\fi
		\else
			\def\@p@sfile{#1}
			\def\@p@sbbfile{#1.bb}
			\@decmprtrue
		\fi}

\def\@p@@sfile#1{\@p@@sfigure{#1}}

\def\@p@@sbbllx#1{
		%\ps@typeout{bbllx is #1}
		\@bbllxtrue
		\dimen100=#1
		\edef\@p@sbbllx{\number\dimen100}
}
\def\@p@@sbblly#1{
		%\ps@typeout{bblly is #1}
		\@bbllytrue
		\dimen100=#1
		\edef\@p@sbblly{\number\dimen100}
}
\def\@p@@sbburx#1{
		%\ps@typeout{bburx is #1}
		\@bburxtrue
		\dimen100=#1
		\edef\@p@sbburx{\number\dimen100}
}
\def\@p@@sbbury#1{
		%\ps@typeout{bbury is #1}
		\@bburytrue
		\dimen100=#1
		\edef\@p@sbbury{\number\dimen100}
}
\def\@p@@sheight#1{
		\@heighttrue
		\dimen100=#1
   		\edef\@p@sheight{\number\dimen100}
		%\ps@typeout{Height is \@p@sheight}
}
\def\@p@@swidth#1{
		%\ps@typeout{Width is #1}
		\@widthtrue
		\dimen100=#1
		\edef\@p@swidth{\number\dimen100}
}
\def\@p@@srheight#1{
		%\ps@typeout{Reserved height is #1}
		\@rheighttrue
		\dimen100=#1
		\edef\@p@srheight{\number\dimen100}
}
\def\@p@@srwidth#1{
		%\ps@typeout{Reserved width is #1}
		\@rwidthtrue
		\dimen100=#1
		\edef\@p@srwidth{\number\dimen100}
}
\def\@p@@sangle#1{
		%\ps@typeout{Rotation is #1}
		\@angletrue
%		\dimen100=#1
		\edef\@p@sangle{#1} %\number\dimen100}
}
\def\@p@@ssilent#1{ 
		\@verbosefalse
}
\def\@p@@sprolog#1{\@prologfiletrue\def\@prologfileval{#1}}
\def\@p@@spostlog#1{\@postlogfiletrue\def\@postlogfileval{#1}}
\def\@cs@name#1{\csname #1\endcsname}
\def\@setparms#1=#2,{\@cs@name{@p@@s#1}{#2}}
%
% initialize the defaults (size the size of the figure)
%
\def\ps@init@parms{
		\@bbllxfalse \@bbllyfalse
		\@bburxfalse \@bburyfalse
		\@heightfalse \@widthfalse
		\@rheightfalse \@rwidthfalse
		\def\@p@sbbllx{}\def\@p@sbblly{}
		\def\@p@sbburx{}\def\@p@sbbury{}
		\def\@p@sheight{}\def\@p@swidth{}
		\def\@p@srheight{}\def\@p@srwidth{}
		\def\@p@sangle{0}
		\def\@p@sfile{} \def\@p@sbbfile{}
		\def\@p@scost{10}
		\def\@sc{}
		\@prologfilefalse
		\@postlogfilefalse
		\@clipfalse
		\if@noisy
			\@verbosetrue
		\else
			\@verbosefalse
		\fi
}
%
% Go through the options setting things up.
%
\def\parse@ps@parms#1{
	 	\@psdo\@psfiga:=#1\do
		   {\expandafter\@setparms\@psfiga,}}
%
% Compute bb height and width
%
\newif\ifno@bb
\def\bb@missing{
	\if@verbose{
		\ps@typeout{psfig: searching \@p@sbbfile \space  for bounding box}
	}\fi
	\no@bbtrue
	\epsf@getbb{\@p@sbbfile}
        \ifno@bb \else \bb@cull\epsf@llx\epsf@lly\epsf@urx\epsf@ury\fi
}	
\def\bb@cull#1#2#3#4{
	\dimen100=#1 bp\edef\@p@sbbllx{\number\dimen100}
	\dimen100=#2 bp\edef\@p@sbblly{\number\dimen100}
	\dimen100=#3 bp\edef\@p@sbburx{\number\dimen100}
	\dimen100=#4 bp\edef\@p@sbbury{\number\dimen100}
	\no@bbfalse
}
% rotate point (#1,#2) about (0,0).
% The sine and cosine of the angle are already stored in \sine and
% \cosine.  The result is placed in (\p@intvaluex, \p@intvaluey).
\newdimen\p@intvaluex
\newdimen\p@intvaluey
\def\rotate@#1#2{{\dimen0=#1 sp\dimen1=#2 sp
%            	calculate x' = x \cos\theta - y \sin\theta
		  \global\p@intvaluex=\cosine\dimen0
		  \dimen3=\sine\dimen1
		  \global\advance\p@intvaluex by -\dimen3
% 		calculate y' = x \sin\theta + y \cos\theta
		  \global\p@intvaluey=\sine\dimen0
		  \dimen3=\cosine\dimen1
		  \global\advance\p@intvaluey by \dimen3
		  }}
\def\compute@bb{
		\no@bbfalse
		\if@bbllx \else \no@bbtrue \fi
		\if@bblly \else \no@bbtrue \fi
		\if@bburx \else \no@bbtrue \fi
		\if@bbury \else \no@bbtrue \fi
		\ifno@bb \bb@missing \fi
		\ifno@bb \ps@typeout{FATAL ERROR: no bb supplied or found}
			\no-bb-error
		\fi
		%
%\ps@typeout{BB: \@p@sbbllx, \@p@sbblly, \@p@sbburx, \@p@sbbury} 
%
% store height/width of original (unrotated) bounding box
		\count203=\@p@sbburx
		\count204=\@p@sbbury
		\advance\count203 by -\@p@sbbllx
		\advance\count204 by -\@p@sbblly
		\edef\ps@bbw{\number\count203}
		\edef\ps@bbh{\number\count204}
		%\ps@typeout{ psbbh = \ps@bbh, psbbw = \ps@bbw }
		\if@angle 
			\Sine{\@p@sangle}\Cosine{\@p@sangle}
	        	{\dimen100=\maxdimen\xdef\r@p@sbbllx{\number\dimen100}
					    \xdef\r@p@sbblly{\number\dimen100}
			                    \xdef\r@p@sbburx{-\number\dimen100}
					    \xdef\r@p@sbbury{-\number\dimen100}}
%
% Need to rotate all four points and take the X-Y extremes of the new
% points as the new bounding box.
                        \def\minmaxtest{
			   \ifnum\number\p@intvaluex<\r@p@sbbllx
			      \xdef\r@p@sbbllx{\number\p@intvaluex}\fi
			   \ifnum\number\p@intvaluex>\r@p@sbburx
			      \xdef\r@p@sbburx{\number\p@intvaluex}\fi
			   \ifnum\number\p@intvaluey<\r@p@sbblly
			      \xdef\r@p@sbblly{\number\p@intvaluey}\fi
			   \ifnum\number\p@intvaluey>\r@p@sbbury
			      \xdef\r@p@sbbury{\number\p@intvaluey}\fi
			   }
%			lower left
			\rotate@{\@p@sbbllx}{\@p@sbblly}
			\minmaxtest
%			upper left
			\rotate@{\@p@sbbllx}{\@p@sbbury}
			\minmaxtest
%			lower right
			\rotate@{\@p@sbburx}{\@p@sbblly}
			\minmaxtest
%			upper right
			\rotate@{\@p@sbburx}{\@p@sbbury}
			\minmaxtest
			\edef\@p@sbbllx{\r@p@sbbllx}\edef\@p@sbblly{\r@p@sbblly}
			\edef\@p@sbburx{\r@p@sbburx}\edef\@p@sbbury{\r@p@sbbury}
%\ps@typeout{rotated BB: \r@p@sbbllx, \r@p@sbblly, \r@p@sbburx, \r@p@sbbury}
		\fi
		\count203=\@p@sbburx
		\count204=\@p@sbbury
		\advance\count203 by -\@p@sbbllx
		\advance\count204 by -\@p@sbblly
		\edef\@bbw{\number\count203}
		\edef\@bbh{\number\count204}
		%\ps@typeout{ bbh = \@bbh, bbw = \@bbw }
}
%
% \in@hundreds performs #1 * (#2 / #3) correct to the hundreds,
%	then leaves the result in @result
%
\def\in@hundreds#1#2#3{\count240=#2 \count241=#3
		     \count100=\count240	% 100 is first digit #2/#3
		     \divide\count100 by \count241
		     \count101=\count100
		     \multiply\count101 by \count241
		     \advance\count240 by -\count101
		     \multiply\count240 by 10
		     \count101=\count240	%101 is second digit of #2/#3
		     \divide\count101 by \count241
		     \count102=\count101
		     \multiply\count102 by \count241
		     \advance\count240 by -\count102
		     \multiply\count240 by 10
		     \count102=\count240	% 102 is the third digit
		     \divide\count102 by \count241
		     \count200=#1\count205=0
		     \count201=\count200
			\multiply\count201 by \count100
		 	\advance\count205 by \count201
		     \count201=\count200
			\divide\count201 by 10
			\multiply\count201 by \count101
			\advance\count205 by \count201
			%
		     \count201=\count200
			\divide\count201 by 100
			\multiply\count201 by \count102
			\advance\count205 by \count201
			%
		     \edef\@result{\number\count205}
}
\def\compute@wfromh{
		% computing : width = height * (bbw / bbh)
		\in@hundreds{\@p@sheight}{\@bbw}{\@bbh}
		%\ps@typeout{ \@p@sheight * \@bbw / \@bbh, = \@result }
		\edef\@p@swidth{\@result}
		%\ps@typeout{w from h: width is \@p@swidth}
}
\def\compute@hfromw{
		% computing : height = width * (bbh / bbw)
	        \in@hundreds{\@p@swidth}{\@bbh}{\@bbw}
		%\ps@typeout{ \@p@swidth * \@bbh / \@bbw = \@result }
		\edef\@p@sheight{\@result}
		%\ps@typeout{h from w : height is \@p@sheight}
}
\def\compute@handw{
		\if@height 
			\if@width
			\else
				\compute@wfromh
			\fi
		\else 
			\if@width
				\compute@hfromw
			\else
				\edef\@p@sheight{\@bbh}
				\edef\@p@swidth{\@bbw}
			\fi
		\fi
}
\def\compute@resv{
		\if@rheight \else \edef\@p@srheight{\@p@sheight} \fi
		\if@rwidth \else \edef\@p@srwidth{\@p@swidth} \fi
		%\ps@typeout{rheight = \@p@srheight, rwidth = \@p@srwidth}
}
%		
% Compute any missing values
\def\compute@sizes{
	\compute@bb
	\if@scalefirst\if@angle
% at this point the bounding box has been adjsuted correctly for
% rotation.  PSFIG does all of its scaling using \@bbh and \@bbw.  If
% a width= or height= was specified along with \psscalefirst, then the
% width=/height= value needs to be adjusted to match the new (rotated)
% bounding box size (specifed in \@bbw and \@bbh).
%    \ps@bbw       width=
%    -------  =  ---------- 
%    \@bbw       new width=
% so `new width=' = (width= * \@bbw) / \ps@bbw; where \ps@bbw is the
% width of the original (unrotated) bounding box.
	\if@width
	   \in@hundreds{\@p@swidth}{\@bbw}{\ps@bbw}
	   \edef\@p@swidth{\@result}
	\fi
	\if@height
	   \in@hundreds{\@p@sheight}{\@bbh}{\ps@bbh}
	   \edef\@p@sheight{\@result}
	\fi
	\fi\fi
	\compute@handw
	\compute@resv}

%
% \psfig
% usage : \psfig{file=, height=, width=, bbllx=, bblly=, bburx=, bbury=,
%			rheight=, rwidth=, clip=}
%
% "clip=" is a switch and takes no value, but the `=' must be present.
\def\psfig#1{\vbox {
	% do a zero width hard space so that a single
	% \psfig in a centering enviornment will behave nicely
	%{\setbox0=\hbox{\ }\ \hskip-\wd0}
	%
	\ps@init@parms
	\parse@ps@parms{#1}
	\compute@sizes
	%
	\ifnum\@p@scost<\@psdraft{
		%
		\special{ps::[begin] 	\@p@swidth \space \@p@sheight \space
				\@p@sbbllx \space \@p@sbblly \space
				\@p@sbburx \space \@p@sbbury \space
				startTexFig \space }
		\if@angle
			\special {ps:: \@p@sangle \space rotate \space} 
		\fi
		\if@clip{
			\if@verbose{
				\ps@typeout{(clip)}
			}\fi
			\special{ps:: doclip \space }
		}\fi
		\if@prologfile
		    \special{ps: plotfile \@prologfileval \space } \fi
		\if@decmpr{
			\if@verbose{
				\ps@typeout{psfig: including \@p@sfile.Z \space }
			}\fi
			\special{ps: plotfile "`zcat \@p@sfile.Z" \space }
		}\else{
			\if@verbose{
				\ps@typeout{psfig: including \@p@sfile \space }
			}\fi
			\special{ps: plotfile \@p@sfile \space }
		}\fi
		\if@postlogfile
		    \special{ps: plotfile \@postlogfileval \space } \fi
		\special{ps::[end] endTexFig \space }
		% Create the vbox to reserve the space for the figure.
		\vbox to \@p@srheight sp{
		% 1/92 TJD Changed from "true sp" to "sp" for magnification.
			\hbox to \@p@srwidth sp{
				\hss
			}
		\vss
		}
	}\else{
		% draft figure, just reserve the space and print the
		% path name.
		\if@draftbox{		
			% Verbose draft: print file name in box
			\hbox{\frame{\vbox to \@p@srheight sp{
			\vss
			\hbox to \@p@srwidth sp{ \hss \@p@sfile \hss }
			\vss
			}}}
		}\else{
			% Non-verbose draft
			\vbox to \@p@srheight sp{
			\vss
			\hbox to \@p@srwidth sp{\hss}
			\vss
			}
		}\fi	



	}\fi
}}
\psfigRestoreAt
\let\@=\LaTeXAtSign




%%% \special{header=/group/csdl/tex/psfig/lprep71.pro}
%%% \begin{document}
%%% \ls{1.2}
%%% 
%%% \tableofcontents
%%% \newpage
%%% \pagenumbering{arabic}


\setcounter{chapter}{6}
\chapter{Related Work}
\label{sec:related work}


CLARE represents a confluence of several streams of research
spanning across a number of intellectual disciplines: computer-supported
cooperative work (CSCW), human learning, knowledge representation,
sociology of knowledge, and hypertext.  While this dissertation falls under
the broad umbrella of CSCW, the theoretical motivation and practical
application reside in human learning.

This chapter is organized into four sections. Section \ref{sec:theory}
reviews the theoretical work on which CLARE is based.  More specifically,
it covers constructionism and the assimilation theory of cognitive
learning. Section \ref{sec:representation} describes schema theory and
related knowledge representation schemes.  Section \ref{sec:cscl-systems}
surveys major existing collaborative learning systems and related empirical
findings.  This chapter concludes with a summary of the relationships
between CLARE and the work being reviewed.



\section{Theoretical underpinnings}
\label{sec:theory}

CLARE is grounded in two main theoretical tenets: constructionism and
assimilation theory of cognitive learning. The former provides CLARE a
philosophical foundation. The latter serves as an overarching pedagogical
framework. The following sections provide an overview of both.

\subsection{Constructionism}

Constructivism, succinctly put, views that knowledge is {\it constructed\/}
rather than merely {\it acquired\/} by the learner. It is a mode of
learning in which the student plays an active, contributing part rather
than being a passive target of knowledge transmission. To a constructivist,
the learning environment is more of a give-and-take, with the teacher being
just one of many resources upon which the student can call. Constructivism
is often associated with the theories of Jean Piaget, who contends that
restructuring of prior knowledge (learning) requires challenging existing
views and coordinating old with new knowledge \cite{Piaget77}, and that
these conditions is present when learners interact with peers of differing
but also inadequate views \cite{Piaget32}.

Constructionism, which represents one variation of constructivism, holds
that knowledge, especially scientific knowledge, is socially constructed
\cite{Berger66,Knorr-Cetina81}. Such knowledge is not the same for
individuals but is {\it taken-to-be-shared\/} \cite{Roth92} with
communities of learners. To become a member of such a community, students
need to actively engage in interactions and undergo learning situations
which allow them to immerse into the discourse practice of a field. In
order to to form classroom communities which function like those of
scientists, for example, students need to have opportunity engaging in
authentic practice of scientists.

CLARE embodies the constructionist view in two ways. First, it treats
learning not merely as reading or understanding what a research artifact
says but as knowledge construction which benefits from explicit process-
and representation-level support.  Second, CLARE is built on the premise
that knowledge-building among learners bears much resemblance with
knowledge-building in the scientific community. Learners can gain
understanding of how scientists construct knowledge by systematically
studying the artifacts they generate. Subsequently, they may apply that
knowledge to improve their own knowledge-building practice.  RESRA, SECAI,
and CLARE were designed to guide and facilitate this process.


\subsection{Assimilation theory of cognitive learning}

The {\it theory of meaningful learning\/}, also known as the {\it
assimilation theory of cognitive learning\/}, is an important theoretical
framework in educational psychology. It has evolved and been tested at
Cornell University over the past three decades \cite{Ausubel63,Novak84}.
The thrust of this theory is the emphasis on meta-learning, that is, {\it
learning how to learn\/}, and the role of the meta-knowledge in human
learning.  The basic tenets of this theory include:

\begin{itemize}
\item The single most important factor influencing human learning is what
  the learner already knows, i.e., prior knowledge;
  
\item Learning is evidenced by a change in the meaning of experience
  rather than a change in behavior --- a view that is long held by
  behavioral psychologists; and
  
\item The key role of the educator is to help students reflect on their
  experience (and hence give it new meanings), and construct meanings from
  the artifact in light of the changing experience.
\end{itemize}

The theory introduces two important concepts: {\it progressive
differentiation\/} and {\it integrative consolidation\/}. The former states
that, since meaningful learning is a continuous process wherein new
concepts gain greater meaning as new relationships are acquired, concepts
are never finally learned, but their meanings are constantly revised, and
made more explicit as they become progressively more differentiated.  The
latter refers to the fact that meaningful learning is enhanced when the
learner recognizes new relationships between related concepts and 
propositions. To assess what a learner already knows and how it changes
over time, Novak and Gowin \cite{Novak84} propose two meta-cognitive tools,
called {\it concept maps\/} and {\it Vee diagrams\/}, both of which are
described in the following section.

The above view that human learning is {\it meaning-making} places knowledge
representation at the focal point of the human learning process, since
knowledge representation defines not only the form in which a certain type
of knowledge is highlighted to the learner, but also the process by which
such a form is derived. Moreover, knowledge representation languages, which
are called {\it meta-cognitive tools\/} by educational researchers, are the
standard language for characterizing both knowledge structures and
corresponding cognitive structures. They help the learner differentiate and
organize newly acquired meanings. The next section describes major related
work in this area.


\section{Representation and human learning}
\label{sec:representation}

This section briefly describes what {\it schema theory\/} is and how it is
related to CLARE. In addition, it also compares RESRA with a number of
alternative representation languages, namely, IBIS, Toulmin's model of
argumentation, concept maps, and Vee diagrams.


\subsection{Schema theory, representation, and learning}

One important conceptual framework in cognitive psychology is {\it schema
theory\/}, which posits that human minds store and retrieve knowledge about
the external world in terms of abstract categories called {\it schemas\/}
\cite{Stillings87}. A {\it schema\/} is defined as ``a data structure for
representing generic concepts in memory'' \cite{Rumelhart80}.  As such, a
schema forms ``a building block of cognition'' that affects how new
information is absorbed as well as how old information is retrieved
from memory. Alba and Hasher \cite{Alba83} identify four main functions of
schemas:

\begin{itemize}
\item {\it Selection:\/} Filtering out certain types of information before
  passing on to memory representation.
  
\item {\it Abstraction:\/} Information reduction by omitting certain
  types of details.
  
\item {\it Interpretation:\/} Inferring missing information based on
  previous instances of the schema.
  
\item {\it Integration:\/} Grouping together related or similar
  information.
\end{itemize}

Although schema theory is a psychological framework for explaining how
humans construct meanings from written or spoken words, it has been heavily
influenced by work in artificial intelligence \cite{Abelson81,Schank77}. In
fact, {\it script\/} --- the schema about activities and processes --- is
used both as a knowledge representation scheme for AI programs and in
psychological research.

Schema-based approaches have been used to study expert-novice differences
in reading and understanding text. It is found that novice readers tend to
skim the text and retain isolated facts. Skilled readers, on the other
hand, recognize patterns/schemas that relate different parts of the text
into a coherent whole. Their strategies include searching the text for its
underlying structure, identifying the major text schema, and formulating
relational links between major and subordinate ideas \cite{Dijk83,Voss83}.

Schema theory and the above empirical findings are directly relevant to
CLARE: RESRA, for example, might be viewed as a set of schemas for
characterizing the deep structure of scientific text. In particular, RESRA
tuples and CRFs serve similar purposes as other types of schemas, such as
{\it scripts\/}, in helping learners understand the content of scientific
text through {\it selection\/}, {\it abstraction\/}, {\it
interpretation\/}, and {\it integration\/}. The expert-novice differences
in their use of schemas invite similar studies to be conducted in the CLARE
environment (see Section \ref{sec:future-directions}).


\subsection{RESRA and other representation schemes}
\label{sec:kr-schemes}

Although RESRA is unique as a conceptual framework for characterizing
thematic features of scientific text and for facilitating collaborative
learning, the use of semi-structured representation in ill-defined tasks is
not new. A number of such schemes have been invented and used in domains
such as software design \cite{Lee91What}. This section reviews two of such
representations: {\it IBIS\/} and {\it Toulmin's model\/}. In addition, it
also discusses two similar approaches that are proposed to specifically
support human learning:{\it concept maps\/} and {\it Vee diagrams.\/}


\paragraph{IBIS.}

IBIS, which stands for {\it Issue-based information systems\/}, was
originally proposed by \cite{Kunz70} for deliberating design decisions in
information systems.  There are several variations of this representation,
one of which, called gIBIS (\cite{Conklin88}), is shown in Figure
\ref{fig:ibis}.

\begin{figure}[htbp]
 \fbox{\centerline{\psfig{figure=Figures/ibis.eps,width=4.0in}}}
 \caption{IBIS model of argumentation (based on [Conklin87])}
  \label{fig:ibis}
\end{figure}

The main feature of IBIS is that it is parsimonious: the three node and
seven link types can be easily learned. However, the limitation of this
representation is also evident: the small set of primitives are not
sufficiently expressive for many task domains.  Furthermore, the model is
biased toward controversies. For example, it omits questions that are not
deliberated in favor of those questions with which debate and controversy
are likely to be associated \cite{MacLean91Questions}.  RESRA was
originally built on the gIBIS representation. In fact, features
shown in Figure \ref{fig:ibis} are also found in RESRA.  For example, {\it
issue,\/} {\it position,\/} and {\it argument\/} are subsumed by
\fbox{{problem}}, \fbox{{claim}}, and \fbox{{evidence}} in RESRA,
respectively.


\paragraph{Toulmin's rhetorical model.}

Figure \ref{fig:toulmin} shows another widely used argumentation model
proposed by philosopher Stephen Toulmin \cite{Toulmin58}. The original
purpose of this model was for delineating logical structure of an argument,
although it is also useful for analyzing scientific controversies (e.g.,
\cite{Cavalli-Sforza92}).

\begin{figure}[htbp]
 \fbox{\centerline{\psfig{figure=Figures/toulmin.eps,width=4.0in}}}
 \caption{Toulmin's model of argumentation (based on [Toulmin52])}
  \label{fig:toulmin}
\end{figure}

Toulmin's model suffers from similar problems as IBIS when placed in the
CLARE application domain, that is, it is overly coarse-grained. The
structure of scientific text varies widely, as shown the example
CRFs. Using a rhetorically-based representation to characterize such
structure is not often possible. Elements of Toulmin's model, however, can
also be found in RESRA, for example, \fbox{{\sf evidence}} ({\it Datum,\/}
{\it Backing\/}), and \fbox{{\sf claim}} ({\it Claim,\/} {\it Rebuttal\/}).


\paragraph{Concept maps and Vee diagrams.}

Concept maps and Vee diagrams are two meta-cognitive tools proposed by
educational theorists to (1) assess what the learner already knows; (2)
discern changes over time in the learner's knowledge structure; and (3)
facilitate {\it meaningful learning\/} \cite{Novak84}. The effectiveness of
concept maps in accomplishing these goals seems well supported from field
studies \cite{Cliburn90,Novak90,Roth92,Arnaudin84}.  Nevertheless, as a
representation scheme for supporting collaborative learning, concept maps
are not adequate (see Section \ref{sec:concept-map}).  RESRA is a direct
response to such inadequacies.

\begin{figure}[htbp]
 \fbox{\centerline{\psfig{figure=Figures/vee.eps,width=4.0in}}}
 \caption{Novak and Gowin's Vee Diagram (from [NG84])}
  \label{fig:vee}
\end{figure}

Figure \ref{fig:vee} depicts a simplified version of the Vee diagram for
understanding the nature of knowledge and knowledge production. The left
side of the Vee is the {\it thinking-side\/}, while the right side is
called {\it doing-side\/}. The two sides are linked together by the
event/object that is under study. To answer the focus question in the
middle requires an integration of issues raised from both sides. At a
conceptually level, the Vee diagram offers an elegant and powerful means of
exposing deep-level structure of knowledge: by constructing a Vee diagram
for each problem situation, a learner can see inter-relationships among
different knowledge components and gaps between them, if any.

The Vee diagram does not meet all important requirements of CLARE, however.
First, Vee was originally designed to help students and teachers better
understand the nature of science laboratory work. In that setting, it is
possible to adopt a {\it bottom-up\/} approach to knowledge, starting from
the actual event/object under study. This approach, however, is not
applicable to CLARE, since CLARE treats scientific text as the primary
source of knowledge.  Because scientific artifacts are {\it episodic\/} in
nature \cite{Swaminathan90}, artifact {\bf A} may cover only the left side
of the Vee, or even only the upper left portion of the Vee, while artifact
{\bf B} may cover the right side of the Vee. As a result, it is not always
possible, nor necessary, to construct the entire Vee, in order to
understand the content of the selected artifact.

Second, the Vee diagram does not lend itself to computerized support.
Unlike RESRA, whose structure is consistent with a hypertext data model and
is easily amenable to automated support, the Vee's symmetric, graphical
approach makes it difficult to leverage through computerization.

Despite the above incompatibilities, the Vee diagram provides some useful
heuristics for the refinement of RESRA. For example, the distinction
between {\it knowledge claim\/} and {\it value claim\/} seems applicable to
RESRA as well. Moreover, RESRA currently does not support the increasing
level of {\it abstractness,\/} as one moves from the bottom to the top of
Vee. 

%%% \subsection{SECAI and Bloom's taxonomy}
%%% 
%%% Bloom's taxonomy of educational objectives the major purpose of
%%% constructing a taxonomy of educational objective is to facilitate
%%% communication: educational research, corriculum development
%%% 
%%% \ls{1.0}
%%% \small
%%% \begin{table}[hbtp]
%%%   \begin{center}
%%%     \begin{tabular} {||p{1.0in}|p{2.40in}|p{2.25in}||} \hline   
%%%       {\bf Educational Objectives} & {\bf Description} & {\bf Examples} \\ \hline \hline
%%%       
%%%       Knowledge  &  &  \\ \hline
%%% 
%%%       Comprehension &  &    \\ \hline
%%% 
%%%       Application &  &  \\ \hline
%%% 
%%%       Analysis &  &  \\ \hline
%%% 
%%%       Synthesis &  &  \\ \hline
%%% 
%%%       Evaluation &  &  \\ \hline \hline
%%%     \end{tabular}
%%%     \caption{{\bf Bloom's Taxnonomy of Educational Objectives}}
%%%     \label{tab:bloom}
%%%   \end{center}
%%% \end{table}
%%% \normalsize
%%% \ls{1.2}
%%% 
%%% 
%%% individual learning rather than collaborative learning.
%%% 
%%% SECAI: summarization encompasses the first five levels, and evaluation


\section{Collaborative learning systems: technologies and outcomes}
\label{sec:cscl-systems}

This section surveys a number of important collaborative learning systems.
For each system, it describes its major features and related empirical
evaluation, if any. The section is organized into four parts, corresponding
to four major types of collaborative learning systems: {\it virtual
classroom,\/} {\it collaborative writing,\/} {\it hypermedia\/}, and {\it
collaborative knowledge-building.\/}


\subsection{Virtual classroom systems}

The term {\it virtual classrooms,\/} or VCs, is used in a broad sense to
encompass both general-purpose computer-mediated communication (CMC), such
as e-mail, electronic bulletin-board systems, and specialized
communication-based learning systems, such as EIES.  The latter also
provides such functions as instruction management tools (assignment
tracking, grading, etc.) \cite{Hiltz88}. VC systems possess the following
main features:

 \begin{itemize}
 \item {\it Access orientation.\/} VC systems enable learners to
   transcend the geographical and temporal limitations of face-to-face
   meetings and allow them interact with one another asynchronously.
   
 \item {\it Affordable technology.\/} VC systems do not require
   sophisticated hardware and software technology. An inexpensive home
   computer equipped with a modem and communication software is often 
   sufficient. 
\end{itemize}

The combination of the above two features has made VC one of the
most successfully and pervasively used collaborative learning systems. The
remaining section summarizes empirical findings on one virtual classroom
system called EIES \cite{Hiltz88}.


\paragraph{Findings on EIES.}

EIES (Electronic Information Exchange System), developed at New Jersey
Institute of Technology, represents one of the earliest large-scale
studies of the impact of computer conferencing and computer-mediated
communication on student learning. The study involves a series of
college-level courses, with subject matters ranging from introductory
sociology, statistics, to computer science. The objective is to describe
the learning experience and outcomes of the VC delivery mode in relation to
the traditional classroom and to determine conditions associated with good
and poor outcomes. Some major findings from this study are:

\begin{itemize}
\item No consistent differences were found in scores measuring mastery of
  material taught in the virtual and traditional classrooms.
  
\item Students in virtual classrooms showed more active participation.
  
\item For those who participated regularly in VC, the level of interest
  tended to be high.
  
\item Virtual classrooms provide more convenient access to educational
  experiences.

\item Students also found virtual classrooms more time-consuming and more
  demanding, for they were required to play a more active part in the
  class.
\end{itemize}

While the quantitative results in most cases are inconclusive, the
qualitative outcomes from this study show that, among well-motivated
students, virtual classrooms provide a new opportunity to participate in
different kinds of learning experience that is based on a community of
learners working together to explore the subject content of a course.


\paragraph{Virtual classrooms and CLARE.}

Unlike virtual classroom systems which in many cases are often used {\it in
place of\/} traditional classrooms, CLARE is designed to {\it complement\/}
the face-to-face mode of learning. First, CLARE supports only one specific
type of learning activity --- collaborative study of scientific text. Other
activities, such as lectures, are still conducted in the traditional mode.
Second, the level of services provided by CLARE goes beyond access by
overcoming the {\it representational\/} constraint of traditional mode of
paper studying. CLARE makes the use of such a representation not only
viable but also measurable. The latter can lead to a continuous improvement
of both the process and the representation. It is interesting to note that,
even given the relatively simple functionality of VC systems, learners
still find it time-consuming and demanding. Thus, it is not surprising that
CLARE users experience similar problems (see Section
\ref{sec:c6-clare-hypothesis}).


\subsection{Collaborative writing systems}

Writing is an integral part of learning that often requires collaboration
among different learners. The use of computers to support collaborative
writing is quite pervasive. Most CMC and virtual classroom systems, for
example, support parts of this process, such as information gathering,
brainstorming, and collaborative commenting. Some systems, notably ENFI
\cite{Bruce93}, are designed with the goal of creating of a {\it writing
community.\/} Hypermedia systems, which are described below, are also used
for such purposes (called {\it authoring systems\/}), especially
brainstorming, collaborative commenting. Most of existing collaborative
writing systems, for instance, PREP
\cite{Neuwirth90issues,Neuwirth92Flexible}, SASE \cite{Baecker93User}, WE
\cite{Smith87Hypertext}, are intended for professional rather than student
writers. It is unclear whether the two are different and, if so, what their
differences are.

CLARE was not designed to explicitly support collaborative writing.
Nevertheless, certain aspects of collaborative writing can benefit from the
CLARE approach. First, RESRA can be used as a framework for brainstorming,
in particular, when writing is based on the reading conducted in
CLARE. Second, a set of related RESRA tuples may serve as an advanced,
non-linear {\it outline\/} for a new research paper. Learners can even
compare these tuples with the ones specified in selected canonical forms to
determine whether necessary features and relationships are present. One
main characteristic of such an outline is that it defines the {\it
deep-level\/} rather than {\it presentational\/} structure of the paper, as
in the traditional outline. Despite this potential, CLARE needs to be
extended at the computational level to provide explicit support for
collaborative writing tasks.


\subsection{Hypermedia systems}

Hypermedia systems represent an important category of collaborative
learning environments. The combination of multi-media, dynamic linking
capabilities, and the distributed nature of such a system makes it a
powerful tool for:

\begin{itemize}
\item Presenting and sharing information;
  
\item Navigating and browsing a complex
network of nodes and links; and

\item Flexible and collaborative annotation and commenting.
\end{itemize}

This section reviews three important hypermedia systems/projects:
Intermedia, NoteCards, and CoVis. Since there are few empirical studies
done on these systems, the focus will be on describing major features of
these systems and their support for collaborative learning.


\paragraph{Intermedia.}

Intermedia, developed at Brown University, is perhaps one of the largest
and oldest hypermedia systems designed specifically to support learning.
It provides a number of commonly-used tools, such as text editor, graphic
editor, timeline editor, and 3-D object viewer. Together, they allow
authors to create links to documents of various media. Intermedia is used
for a number of learning purposes. Instructors and teachers use Intermedia
as an instructional delivery mechanism by organizing and presenting their
lecture materials online. Students browse such networks by using
Intermedia's built-in graphical browser. More importantly, students can add
their own notes and annotations to existing networks of online
artifacts. Since all users have access to notes created by other users,
they can collaboratively comment on one another's notes. Over years, a
number of courses have been taught using Intermedia.  However, very few
empirical reports on the system can be found in the published literature.

\paragraph{NoteCards.}

NoteCards, developed at Xerox PARC, is perhaps one of the most widely
acclaimed hypermedia systems
\cite{HALASZ87Reflections,Trigg88Guided,Trigg87Hypertext,Marshall89Guided}.
The system was originally intended to support tasks related to design of
information systems, such as information gathering and organization. It was
later extended to provide multi-user features, such as allowing more than
one user working on the same notefile at the same time. Although NoteCards
is not a true collaborative learning system, its generic and flexible
design allows it to be easily instantiated to support specific learning
tasks.  Most reports on NoteCards thus far are case studies on the use of
this system for various tasks, such as writing, brainstorming.


\paragraph{CoVis Project.}

CoVis (Collaborative Visualization), a recent project initialized at
Northwestern University, aims at providing a distributed multi-media
learning environments (DMLE) that support {\it learning-in-doing\/}
\cite{Pea93}. The proposed system is intended to integrate high-speed
networks (ISDN), multi-media, and scientific visualization technology into
a media-rich environment that allows students from dispersed locations to
engage in authentic science projects with teachers and practicing
scientists. CoVis is similar to other hypermedia systems in that they are
{\it infrastructure\/} technology that can be used to support a wide
variety of learning activities. However, CoVis differs from the above
systems in that it emphasizes on {\it multi-media\/} rather than {\it
hyper-media.\/} In that sense, CoVis is not truly a hypermedia system.


\paragraph{Hypermedia systems and CLARE.}

Despite its hypertext-based data model and its support for non-linear
navigation, CLARE is not a hypermedia system. Unlike most hypermedia
systems which focus on the presentation of information, CLARE emphasizes
{\it representation\/}, in particular, the process by which such a
representation is derived. However, CLARE can benefit from hypermedia
systems by incorporating certain interface features from them, such as
graphical visualization of network structures, link creation through direct
manipulation. Section \ref{sec:future-directions} provides several specific
proposals on how CLARE might be extended in this direction.


\subsection{Collaborative knowledge-building tools}
\label{kb-tools}

One other type of collaborative learning systems that are of increasingly
importance is called {\it collaborative knowledge-building tools\/}.
Compared to other CSCL systems described above, these system possess the
following characteristics:

\begin{itemize}
\item {\it Learning is knowledge-building.\/} Unlike many other CSCL
  systems that view learning as consisting of such activities as
  information sharing, reading, writing, the design of knowledge-building
  systems is based on the view that learning is knowledge-building, and
  that learning is inherently a collaborative activity. The primary
  purpose of these systems is to help learners make sense of existing
  knowledge and construct new knowledge.
  
\item {\it Integration of technology and pedagogy.\/} Most of these
  systems are grounded in one or more established learning theories.
  
\item {\it Familiar technologies.\/} Most of these systems provide such
  capabilities as shared databases, distributed, asynchronous access,
  hypertext-based navigation, notification, automated activity logs and
  sometimes, graphical interface.
\end{itemize}

CSILE, which is described below, is a general-purpose collaborative
knowledge-building system based on the theory of {\it intentional
learning\/}. CLARE is another example of such systems; it is based on the
assimilation theory of cognitive learning, and provides specific
representational and process-level support for collaborative learning from
scientific text.


\paragraph{System description.}

CSILE ({\it Computer supported intentional learning Environments\/}) is an
integrated learning system developed at the Ontario Institute for Studies
in Education at University of Toronto. At a software level, CSILE consists
of a shared or communal database to which all students have access via a
local area network. Students create text or graphical notes using built-in
editors. They comment on each other's notes and search the database of
notes using keywords, author, and other attributes. When a student create a
note, he or she associates that note to one of the four predefined types
called {\it thinking types\/}: {\it I know,\/} {\it high-level
questions,\/} {\it plan,\/} and {\it problem.\/} When a note is commented
on by other student, the note author is notified. These functionalities,
however, are hardly unique to CSILE; they are also available in many
virtual classroom systems. What differentiates CSILE from other learning
systems is its integration of software and a learning approach that grows
out of over a decade of research on intentional learning, knowledge-telling,
and transformation in writing \cite{Bereiter87}. CSILE represents a joint
effort by cognitive scientists, computer scientists, teachers, and
students.


\paragraph{Empirical findings.}

CSILE has been used at primary, elementary, and graduate school levels.
Preliminary results from these studies show that CSILE users consistently
outperform their non-CSILE counterparts in a number of areas:

\begin{itemize}
\item Standardized test scores in reading and language;
  
\item Depth of explanation and knowledge quality in student writing;
  
\item Comprehension of difficult text and transfer of learning to novel problems;
  
\item Identifying knowledge gaps;

\item Collaboration among students; and
  
\item Beliefs about learning consistent with a progressive view of
  knowledge advancement.
\end{itemize}

Empirical data also demonstrates that, in most of the above
areas, each additional year students spend working with CSILE yield
additional advantage in results. 


\paragraph{CSILE and CLARE.}

CLARE is similar to CSILE in that they are based on the constructionist
paradigm on learning and aim at providing an environment conducive to
collaborative construction of knowledge instead of merely information
sharing.  They both provide detailed, automatic tracking data about the
learner's behavior. However, the two are fundamentally different in their
approaches:

\begin{itemize}
\item {\it Explicit representation support.\/} CLARE provides an explicit
  representation language (RESRA) that serves as a meta-cognitive
  framework for collaborative learning, while CSILE's four thinking types
  ({\it I know,\/} {\it high-level questions,\/} {\it plan,\/} and {\it
  problem\/}) allow learners to categorize their intentions.
  
\item {\it Explicit process-level support.\/} CLARE defines a process
  model called SECAI which dichotomizes collaborative learning into two
  distinct phases: {\it private\/} and {\it public\/}. CSILE, however,
  does not provide any process-level guidance.
  
\item {\it Learning from scientific text.\/} CLARE is designed to support
  a specific type of learning --- learning from scientific text.  CSILE
  is a more general-purpose environment and, as a result, provides less
  task-specific services.

\end{itemize}

One major area in which CLARE can benefit from CSILE's experience is
longitudinal studies for assessing specific impact of the system on
students' learning outcomes, as measured by quantitative tests and
qualitative evaluation. Section \ref{sec:future-directions} identifies a
few specific directions in which future empirical work on CLARE might be
extended.


\section{Summary}
\label{sec:c7-summary}

One key feature that distinguishes CLARE from other collaborative learning
systems is its theory-driven approach: CLARE's treatment of learning from
scientific text as collaborative knowledge construction is based on the
constructionist view of science and learning. CLARE's emphasis on the
meta-cognitive structure and its role in human learning is guided by the
the assimilation theory of cognitive learning, which posits that learning
{\it how to learn\/} (meta-learning) is more important than learning {\it
what it is\/} (content learning). This theoretical proposition has been
supported by empirical findings on concept maps --- one of the two
meta-cognitive tools proposed by learning theorists --- which show that
concept maps are effective in promoting meaningful learning and long-term
retention. CLARE extends concept maps and Vee diagrams by proposing a new
set of meta-cognitive primitives and canonical forms for characterizing the
thematic structure of scientific text and learning activities that are
centered on them.

%%%% to be removed?
%CLARE was designed to support collaborative learning from scientific text.
%There are also other types of learning tasks which CLARE does not support,
%such as, project-based learning, intentional learning. From a user's
%perspective, CLARE is likely to become part of an integrated learning
%environment.

%%% %%%=========================================================
%%% \newpage
%%% \singlespace
%%% \bibliography{../bib/clare,../bib/cscl-systems}
%%% \bibliographystyle{alpha}
%%% \end{document}

