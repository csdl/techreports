\documentstyle[11pt,
/group/csdl/tex/named-citations,
/group/csdl/tex/definemargins,
/group/csdl/tex/lmacros,
/group/csdl/tex/functiondoc]{article}
%\pagestyle{headings}
\begin{document}
\begin{titlepage}
  

\vspace*{1in}
\begin{center}
  {\Large\bf DSB: The Next Generation Tool for Software Engineers} \bigskip\par

  Kavoori Kiran Ram                       \bigskip\par

  Collaborative Software Development Laboratory\\
  Department of Information and Computer Sciences\\
  University of Hawaii\\
  Honolulu, HI 96822\\
  (808) 956-6920\\
  {\tt kavoori@uhics.ics.hawaii.edu}           \bigskip\par
  
  {\bf CSDL-TR-93-05}                \bigskip\par
  
  Last Revised: \today
  
\end{center}
\end{titlepage}
\newpage

%\ls{1.0}

\pagenumbering{roman}
\tableofcontents
\newpage
          \section{ Introduction}

          \subsection{ Overview}
          
          During the development of software projects, there
          always exists the problem of design specification
          maintenance.  As the project team surges ahead
          with the development process, there is a strong
          need to maintain an up-to-date documentation of
          the current system.  This requires an additional
          effort on each team member to maintain a
          consistent report of the modifications and
          additions they make on the system.

This Designbase project attempts to reduce the overhead involved in
the maintenance of ever changing design specifications, by generating
automatically, a design documentation from the source code and the
overview files that are maintained along with the system.



\subsection{ Motivation}


The Designbase (DSB) system satisfies two primary goals: First, it
provides automated extraction of design-level documentation about the
public interface to any Emacs-Lisp system under development or use
within CSDL.  Automated extraction of this information significantly
decreases the overhead to developers of publicizing how to use or
modify a recently developed system.  Rather than requiring developers
to create and maintain a parallel design or user-level description of
their systems, the Designbase simply extracts it from the source code.
Moreover, eliminating this redundant description enormously decreases
the overhead of maintaining the documentation as the underlying system
evolves.  The Designbase also has a simple critique section, where a
series of checks are run on all the public functions to point out
omission of certain standard programming conventions used in software
development.  In an exploratory yet collaborative development
environment, automated design documentation generation and critique
will facilitates shared use of evolving systems.

Of course, the Designbase can't reverse engineer this information from
just any old source code.  Rather, the Designbase can only fulfill its
purpose effectively on code that has been designed and implemented
according to CSDL object oriented Emacs Lisp standards.  This leads to
the second major purpose of the Designbase: to enforce and facilitate
the use of object oriented design techniques within the group by
providing a significant incentive to those who use them.  


\section{ Approach}

\subsection{Problem Definition}
No software system is permanent or immune from change.  Often the need
to change the software begins as soon as it is accepted as
operational.  This occurs for a number of reasons including
requirements and design omissions, or a better understanding by the
user as to what is needed vs what is produced.  Regardless of how well
the software is designed, the need to make changes (enhancements,
extensions and corrections) will arise.  Therefore, it is essential to
anticipate future use and maintainability of the software during the
requirements and design phase.\cite{WMB84}

So there should a software framework to address the maintainability
of software during it's life cycle.  Such a framework would take into
consideration some important requirements as:
\begin{itemize}
\item the ease of understanding the software.  A significant portion
  of the cost of software maintenance can be attributed to an inadequate
  understanding of the software, its intended purpose, and how best to
  make the changes.  Normally, software is maintained by someone other
  than the developer.  The work of the maintainer is drastically reduced
  if the developer provides an accurate documentation for the code he is
  developing.
\item the ease of making software changes.  Maintainability is
  improved if the system is designed in an Object-Oriented style with
  hierarchical, modular units that perform one principal function with
  minimal interaction between the modules.
\end{itemize}


\subsection{Proposed Solution}
This project focuses attention on ``the ease of understanding the
software'' through proper documentation. Software projects are outcome
of combined efforts of programmers, software engineers and managers.
Coordinating individual member of the team and the final integration
of work plays an important role in producing good software systems.
Thus the work done by each member of the team should be easy to
understand, reuse and modify if required, not only by other members
within the group, but also by people reusing code for further
extensions.  In order to attain this higher level of understandability
of code, development teams adopt certain norms and conventions to be
followed by each member of the team.  These norms concern to issues in
programming techniques, like standards in naming conventions,
documentation for each function, public and private functions and so
on.

Given such standardized conventions, an automatic system can be
developed to parse the source code files, and extract a design level
description of the system that encompasses the module level structure
only, and abstracts away the underlying implementation.  As an
additional feature, the designbase can check for the deviations of
these standard conventions and can report to the developer, who then
can make the required corrections, before a public release.  Such a
system would not only be able to improve the understandability of the
software system simultaneously reducing the burden on the developer,
but also help him/her in defining his/her functions precisely thereby
reducing the error percentage from the system.


\section{Current Status}

\subsection{Operational Status}

The Designbase application has been conceived and implemented about
four months back and has been in operation since then.  This is a
masters level project started in Feb '93.  Over this period of four
months, dsb has been fine tuned to incorporate a lot of functionality
and bug fixtures.  From the initial version of parsing source files to
generate a LaTeX file of the public interface of an application, dsb
has been extended to generate LaTeX report/article style formats.  Dsb
has been structured using Object-Oriented concepts making it an easily
extendible system.

Dsb has been in regular use in the CSDL research environment.  The
following statistics of each application within CSDL illustrates the
extent of usage of dsb.
\begin{itemize}
\item {\bf EGRET}\\ 2.0MB code, 4 subsystem modules, 32 classes and
300 public operations. \cite{csdl-93-02}
\item {\bf CLARE}\\ 1.7MB code, 4 subsystem modules, 33 classes and
174 public operations. \cite{csdl-clare-doc}
  
\item {\bf CSRS}\\ 1.0MB code, 2 subsystem modules, 18 classes and 100
public operations. \cite{csdl-csrs-doc}
  
\item {\bf URN}\\ 0.14MB code, 3 subsytems, 5 classes and 31 public
operations. \cite{csdl-urn-doc}
  
\item {\bf DSB}\\ 0.5MB code, 2 subsystem modules, 9 classes and
 48 public operations. \cite{csdl-dsb-doc}

\end{itemize}


\subsection{Object Orientation Conventions}

The applications in CSDL environment are designed in Object-Oriented
fashion.  The major entities in Object Oriented design are classes,
objects, attributes, operations, and collections.  Attributes,
operations and collections can be private or public.  Public nature is
indicated notationally by the use of {\bf *} as the visibility token;
{\bf !} as the visibility token indicates a private.  If an attribute
is private, then functions outside the implementation of the class
should not access or set the attribute's value.  This, of course, is
not enforced in Emacs Lisp, but is rather a way of helping programmers
to understand what parts of a subsystem are internal details and can
thus be safely ignored. Designing and programming in an object
oriented fashion will greatly ease understanding of the system
architecture, as well as supporting migration to other languages
providing direct support for these concepts.

\subsubsection{Classes}  

A {\em class}\/ is a collection of objects with related structure and
behavior.  Each class has associated with it a set of superclasses, a
set of subclasses, a set of attributes, and a set of operations.

Classes frequently have constructor and destructor operations for the objects
associated with them.  These operations are always called {\sf make}
and {\sf delete}. Classes are, in some sense, purely a design-level
representation. No explicit Emacs Lisp support for defining classes
exists.  

\subsubsection{Objects} 

Classes are an organizational entity--what actually exists during
execution are particular instances of classes, or {\em objects}.  Each
object has a unique identifier associated with it, and this unique
identifier (or the object itself) is virtually always the first
argument to the operations associated with a class.

\subsubsection{Attributes}  

Each class instance has a set of characteristics, or {\em
attributes}\/ associated with it.  For example, each screen has a
geometry, color, name, and so forth.  These are attributes of the
class.  Some attributes are {\em set-able}.  This means that an
operation corresponding to the attribute is defined to change its
value. This operation is named by prefixing the attribute name with
{\bf set-}.

\subsubsection{Operations}

While attributes refer to characteristics of classes, {\em operations}
refer to the behaviors of the instances of the class. Operations may
take any number of arguments, but the first argument is, by
convention, an object representing the class instance.  Operations
must always document what form their first (as well as the other)
argument must take. The operation may take additional required or
optional arguments depending upon the nature of the behavior.

\subsubsection{Collections}

Operations, as defined above, operate on individual instances of the
class. There is another kind of operation, however, that operates on
the set of all objects that exist in the class. These are called {\em
collection operations}.  For example, an operation that returns a list
of all nodes of a specified type is a collection operation.  These
operations, taken together, represent many of the classical database
query and retrieval operations supported in Egret .

Collection operations are always defined relative to a class, and are
named by wrapping {\bf \{} and {\bf \}} around the class name. 

\subsubsection{Administrator}

These are special operations that are defined for a few classes.  They
typically provide system initializing operations and recovery
operations.  These administrator functions are used by the system
administrator while bring up the application and in times of system
crash.


\subsection{Notational Conventions}
\label{conventions}

\subsubsection{Identifier Syntax}

Function and variable names in CSDL have a standard
format.\cite{csdl-93-02} Each name must adhere to the following template:

\small\begin{verbatim}
<sys-name><sys-vis><class-name><class-vis><name>
\end{verbatim}\normalsize

\noindent where:
\begin{itemize}
  
\item {\tt <sys-name>} is a set of characters that identifies the
  subsystem membership of the object.

  
\item {\tt <sys-vis>} and {\tt <class-vis>} are single characters
  indicating the external visibility of the object. The character {\bf
  *} indicates that the object is public, the character {\bf !}
  indicates that the object is private, and the character {\bf @}
  indicates that the object is a system administration function to be
  manipulated only by distinguished users at special times (such as
  database initialization, recovery, and so forth).
  
\item {\tt <class-name>} is typically the full class name, but may
  sometimes be an abbreviation.
  
\item {\tt <name>} is the actual name of the operation or attribute.

\end{itemize}

For example, the function {\bf i*cmd*find-node} is the interface
subsystem public operation {\sf find-node} from the class {\sf
command}, which is abbreviated as {\bf cmd}.  By prefixing operations
with their class name, the operations class membership is
automatically documented, as well as allowing the same operation name
to be defined for different classes (for example, {\bf
node*find-node}.)

\subsubsection{Common Name Components}

Beyond naming syntax, Egret also has conventions for naming certain
kinds of semantically related operations.  These conventions are

\begin{itemize}
  
\item Constructor and instantiation operations are prefixed by {\bf
  make-}.
  
\item Deletion operations are prefixed by {\bf delete-}.
  
\item Operations that set the value of an attribute are prefixed by
  {\bf set-}.
  
\item Recovery operation names are by convention prefixed with {\bf
  reset-}, and whenever possible suffixed with the name of the attribute
  or operation whose functioning they repair.
\end{itemize}


\subsection{Parsing Process}

The parsing process of dsb relies heavily on the conventions followed
in the CSDL environment. DSB is designed to carry out the following
procedure:

\begin{itemize}
\item Read in each lisp source file.
\item Process each lisp form in each file selecting only public
  functions and macros.  Macros are expanded and then processed.
  Generate an entity for each operation.  Classify them into attributes,
  operations, collections, variables, and administrator functions.
\item Sort the entities into operations under each class, and then the
  classes under each subsystem.  Critique each operation and class for
  dis-conforming standards.
\item Generate table of operations for each class and for each
  subsystem.  Format the output in an appealing way to the user,
  presenting the public interface of the system.
\end{itemize}



\subsection{Software Quality Assurance Process}

The Designbase has a set of critique instances that can be applied on
each class.  The list of current critique instances are
\begin{itemize}
\item{\bf Missing Constructor}\\ Every class that is defined should
  have an operation that creates an object of that class, typically
  called the constructor function.  Absence of this operation shows a
  serious flaw in the design.
\item{\bf Missing attributes}\\ Each class has an associated set of
  attributes, which need to be documented in an appropriate manner,
  understandable by dsb.  The lack of attributes for a class would
  indicate that some operations might have been wrongly classified or
  the system design does not provide much functionality to that class.
  This would draw attention of the designer to think better ways of
  providing functionality.
\item{\bf Missing Operations} \\ A class exists when it has a defined
  functionality which is exhibited by the set of operations available
  for that class.  Absence of at least a single operation for a class
  indicates that the existence of the class is not well justified.
\item{\bf Missing Documentation}\\ Undocumented functions are the
  worst nightmare any software engineer trying to decode and reuse
  existing systems.  Missing documentation needs immediate attention.
\item{\bf Missing Overview files}\\ Overview files for each class and
  subsystem provide necessary information to understand the modules from
  the developer's perspective.
\item{\bf Undocumented Parameters}\\ When ever a function gets reused,
  it is obvious that it should take the same type of arguments every
  time.  Unless each argument is documented properly, it would be
  difficult to use the function.
\item{\bf Undocumented Return Value}\\ It is important to know what to
  expect as a return value when a function is called, so that the
  calling function can handle the return value properly.
\end{itemize}

The critique operations returns a list of failed instances which form
a part of the documentation.  These glaring shortcomings in the system
implementation would initiate appropriate action from the developer.
Elimination of these defects would make the system more easy to
understand by the other members of the group and by the people
involved in maintaining and extending the system.


\section{Future Directions}


\subsection{Short comings of DSB}
\begin{itemize}
\item The designbase parses only Lisp code.
  
\item Emacs Lisp does not have Object-Oriented constructs and the
  concepts of Object Orientation is incorporated only by convention
  followed in the CSDL environment.
  
\item Information that is extracted from the source code is pretty
  much existing within the source files and not much information is
  derivable from the designbase hardcopy.
\end{itemize}

\subsection{Future additions}

\begin{itemize}
\item The parsing technique can be modified to recognize constructs of
  different Object-Oriented languages like C++ and Common Lisp.  Once
  the basic dsb entities are created, then it can do similar processing.
  
\item A higher level of information can be derived from the current
  dsb system by means of different representations.  For example, an
  Egret system can be used to represent dsb entities in the form of
  nodes and links.  This network can be manipulated from a higher level
  to record changes made to a system over a time period, in a sense the
  evolution of a software system itself.  In Software Engineering
  circles, this kind of representation has a great potential in the
  research of software life cycles.
\end{itemize}



\begin{thebibliography}{}

  \bibitem[\protect\citename{Wilma, }1984]{WMB84}
Wilma M. Osborne.
\newblock A Framework for improving
  Software maintenance throughtout the Software Lifecycle IEEE 1984
  3rd Software Engineering Standards Application Workshop.

\bibitem[\protect\citename{Johnson, }1993]{csdl-93-02}
Philip Johnson.
\newblock EGRET  Requirements Specification
\newblock Also published as Technical Report CSDL-93-02, University of Hawaii
  Department of Information and Computer Sciences.


\bibitem[\protect\citename{Dadong, }1993]{csdl-clare-doc}
Dadong Wan.
\newblock CLARE-1.2d Design Document
\newblock CSDL, University of Hawaii, Department of
  Information and Computer Sciences, May 1993.

\bibitem[\protect\citename{Tjahjono, }1993]{csdl-csrs-doc}
 Danu Tjahjono.
\newblock  CSRS-1.2.1 Design Document
\newblock CSDL, University of Hawaii, Department of
  Information and Computer Sciences, May 1993.

\bibitem[\protect\citename{Brewer, }1993]{csdl-urn-doc}
Robert Brewer.
\newblock  URN-1.0.2 Design Document
\newblock CSDL, University of Hawaii, Department of
  Information and Computer Sciences, May 1993.

\bibitem[\protect\citename{Kavoori, }1993]{csdl-dsb-doc}
Kiran Kavoori.
\newblock  Designbase-1.3.5 Design Document
\newblock CSDL, University of Hawaii, Department of
  Information and Computer Sciences, May 1993.

\end{thebibliography}

\end{document}






