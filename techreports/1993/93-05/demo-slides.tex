%%%%%%%%%%%%%%%%%%%%%%%%%%%%% -*- Mode: Latex -*- %%%%%%%%%%%%%%%%%%%%%%%%%%%%
%% demo-slides.tex -- 
%% RCS:            : $Id: demos-presentation.tex,v 1.1 93/04/25 13:18:18 dxw Exp Locker: dxw $
%% Author          : Dadong Wan
%% Created On      : Sun Apr 25 11:59:13 1993
%% Last Modified By: Kiran Ram Kavoori
%% Last Modified On: Mon May  3 16:25:21 1993
%% Status          : Unknown
%%%%%%%%%%%%%%%%%%%%%%%%%%%%%%%%%%%%%%%%%%%%%%%%%%%%%%%%%%%%%%%%%%%%%%%%%%%%%%%
%%   Copyright (C) 1993 University of Hawaii
%%%%%%%%%%%%%%%%%%%%%%%%%%%%%%%%%%%%%%%%%%%%%%%%%%%%%%%%%%%%%%%%%%%%%%%%%%%%%%%
%% 
%% History
%% 25-Apr-1993		Dadong Wan	
%%    created

\documentstyle[11pt,slidesonly]
{/group/csdl/tex/seminar}
%\input{/group/csdl/tex/psfig}
%\special{header=/home/13/csdl/tex/psfig/lprep71.pro}
%\rotateheaderstrue               % Try this out if using rotation macros.
%\articlemag{-1}
\newcommand{\horizontalline} {\rule{\textwidth}{.02in}} 
\slideframe{none}
\slidesmag{0}        % integer value ranging from -5 to 9
\special{landscape}  %comment out this line for notes
\pagestyle{empty}
%\twoup[-2]          %uncomment this line for notes

\begin{document}      

\begin{slide} \Huge
    \begin{center}
    {\bf DSB: The Next Generation Tool for Software Engineers}
    
    \vspace{0.5in}
    
    by\\
    Kavoori Kiran Ram\\
    {\sf kavoori@uhics.ics.hawaii.edu}
    
    \vspace{0.5in}
    
    Collaborative Software Development Laboratory \\
    Department of Information \& Computer Sciences\\
    University of Hawaii at Manoa
  \end{center}
\end{slide}  \Huge


\begin{slide} \Huge 
  {\bf Problems in Reverse Engineering }
  \horizontalline
  
  \begin{itemize}
  \item Typical Software projects are products of team work.
    
  \item Unless properly documented, code written by others is
  difficult to understand.
    
  \item An up-to-date documentation is an overhead.
  \end{itemize}
\end{slide} \Huge 



\begin{slide} \Huge 
  {\bf The DSB Approach}
  \horizontalline

  \begin{itemize}    
  \item Automatically extract the design level specifications.

   \item Present the public interface in a meaningful way.
    
  \item Critique the application to elucidate short comings.

  \end{itemize}
\end{slide} \Huge 


\begin{slide} \Huge 
  {\bf Current Status}
  \horizontalline

  \begin{itemize}    
  \item Generates a LaTeX document of the design specifications.

  \item Critiques the application.
  \item Successfully tested on the following systems

    EGRET: 4 modules, 32 classes and 300 operations.

    CLARE: 4 modules, 33 classes and 174 operations.
    
    CSRS: 2 modules, 18 classes and 100 operations.
    
    URN: 3 modules, 5 classes and 31 operations. 
    
    DSB: 2 modules, 9 classes and 48 operations.


  \end{itemize}
\end{slide} \Huge 


\begin{slide} \Huge 
  {\bf Short Comings and Future Directions}
  \horizontalline
  
  \begin{itemize}
  \item DSB can only parse Lisp source files.\\
    Can be extended to languages like Common Lisp and C++.
    
  \item DSB works only on applications developed under CSDL. \\
    Can be modified to incorporate varying conventions.
    
  \end{itemize}
\end{slide} \Huge 



\end{document}








