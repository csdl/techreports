%%%%%%%%%%%%%%%%%%%%%%%%%%%%%% -*- Mode: Latex -*- %%%%%%%%%%%%%%%%%%%%%%%%%%%%
%% 93-20.tex -- 
%% Author          : Philip M Johnson
%% Created On      : Thu Nov  7 09:52:47 1991
%% Last Modified By: Philip Johnson
%% Last Modified On: Tue Nov 30 14:33:41 1993
%%%%%%%%%%%%%%%%%%%%%%%%%%%%%%%%%%%%%%%%%%%%%%%%%%%%%%%%%%%%%%%%%%%%%%%%%%%%%%%

\documentstyle [11pt,/group/csdl/tex/definemargins,
                /group/csdl/tex/lmacros]{article}

\begin{document}
\title{{\bf Gtables: From Egret 2.x.x to Egret 3.0.x}}
\author {Philip Johnson\\
         Danu Tjahjono\\
         Dadong Wan\\
         Robert Brewer\\
         Collaborative Software Development Laboratory\\
         University of Hawaii}
\date{CSDL TR 93-20\\November, 1993}

\maketitle
\newpage
\tableofcontents
\newpage

\section{Overview}

This document describes the current design of the server subsystem class
s*gtable, which provides an efficient and high-level mechanism for the
implementation and maintenance of global tables.  The purpose of this
documentation is to serve as a basis for the redesign and reimplementation
of gtables in Egret-3.0.x. 

This document combines bits and pieces of information about gtables from
the original gtable design document created in November of 1991, and its
redesign document for agents in November of 1992, along with miscellaneous
supplements from e-mail and other text files.  It appears that gtables are
consistently redesigned on an annual basis in November!

Conceptually, gtables are important in Egret because they overlay a
``relational'' structure on top of the hypertext structure provided by the
database server.  Gtables thus allow developers to design Egret
applications that can answer certain kinds of queries without traversing
the entire hypertext database.  (In fact, since gtables are cached locally,
the applications typically can answer the query without querying the
database server process at all.)

Functionally, s*gtables can be viewed as a hybrid of the system node class
s*snode\foot{The current relationship between the s*snode class and gtables
is fairly muddy. Gtables turn out to use the internal implementation
mechanisms of s*snodes, but do not obey their external interface.} and the
table class u*table.  In general, an s*gtable is defined and manipulated in
manner similar to a u*table, but unlike a u*table, the s*gtable is both
shared and persistent.  It is shared because updates to an s*gtable by one
client are immediately seen by all other clients.  It is persistent because
an update to the s*gtable during one session will be preserved until the
next session.

Implementationally, gtables are maintained in part by an independent Egret
process called the gagent\foot{Gagent is pronounced {\em gee-agent},
although frustrated Egret developers may have been tempted to pronounce it
{\em GAG-ent}.}.  Use of a separate process has the advantage of vastly
increasing the responsiveness of the system, by offloading many gtable
maintenance duties from the user's client process.  However, the current
gagent design and implementation was vastly complicated by the fact that
the database server process was a defect-ridden black box.  We will return
to a discussion of the gagent after introducing gtables from a user-level,
functional perspective.

Consider the following example definition of an s*gtable:

\small\begin{verbatim}
(s*gtable*define :table-name t*nschemas
                 :fn-prefix "t*nschemas!"
                 :hashp 'yes
                 :rlookup-fn t*nschema-name
                 :comp-fn t*nschema-name

                 :node-entry-table t
                 :recovery-fn :automatic

                 :new-entry-fn t*nschemas!process-create
                 :update-entry-fn t*nschemas!process-update
                 :delete-entry-fn t*nschemas!process-delete
                 )
\end{verbatim}\normalsize

The first five parameters are analogous to their counterparts in
u*table, and specify the representational characteristics of the local
table and its invocation functions.  (The other optional u*table
parameters for table size, initial value, and completion function can
be supplied to s*gtable, but are omitted here for brevity.)

The sixth and seventh parameters allow the user to specify certain
representational characteristics of the global table, and are discussed
in more detail below.

The final three parameters specify functions that will be called whenever
entries in the table are created, updated, or deleted. These functions are
passed the actual key value pair involved, and allow each client to perform
some action whenever any client makes a change to the table.

The result of this call is the definition of the following 
public functions\foot{The appearance of the ``!''  in the 
function names may give the appearance that these are private,
not public functions.  In fact, they are private to the subsystem
in which the gtable is defined, which is why the client used the 
prefix ``t*nschemas!'' rather than ``t*nschemas*''.  However, they
are public functions from the point of view of the gtable implementation.
In other words, the gtable implementation defines many other functions
in addition to these, but these functions define the public, user-level 
interface to the gtable.} for use by the client:
\small\begin{verbatim}
t*nschemas!put
t*nschemas!get
t*nschemas!delete
t*nschemas!get-key
t*nschemas!recover
t*nschemas!add-node-entry
\end{verbatim}\normalsize

Clients add and modify entries in the table using PUT, retrieve
entries using GET, perform reverse-lookup using GET-KEY, and recover
the table using RECOVER. The ADD-NODE-ENTRY function will be described
below.

The next few sections provide additional perspectives on gtables
by comparing them to other EGRET mechanisms, such as u*tables and
s*snodes.

\subsection{s*gtables vs. u*tables}

One major difference between s*gtables and u*tables is that u*tables
are local while s*gtables are global. This means that when one client
adds, deletes, or updates an entry to a s*gtable, the change is
automatically and immediately visible to all other currently connected
clients.  

To allow all clients to perform some action whenever a change is made
to the table by one client, s*gtables have their own simple event
processing facility.  This allows the user to specify functions to be
called after table entry addition, deletion, and modification.  In the
first two cases, the function is passed the key and value pair being
added or deleted, while in the third case, the function is passed both
the old and the new key and value pairs.

Finally, unlike u*tables, s*gtables are persistent.  While u*tables
must be explicitly rebuilt by clients each time they connect, the
state of an s*gtable is automatically preserved between sessions.

\subsection{s*gtables vs. s*snodes}

S*gtables and s*snodes differ primarily in the fact that s*gtables provide a
structured, high-level interface.  While both s*gtables and s*snodes
implement global, persistent data, the characteristics and
manipulation of this data differ greatly in the two implementations.

First, snodes store a single, variable length string, whose contents
must be interpreted manually by clients.  Client manipulation of an
snode involves reading this string into a buffer, parsing its
contents, and write the buffer back out to the snode.  In contrast,
s*gtables implement a table with high-level operations such as put,
get, and delete---the client interacts with the table directly and is
insulated from low-level buffer and string manipulation activities.

A second, more important difference between snodes and s*gtables is
that s*gtables provide an optimized, robust implementation of shared,
persistent tables.  The straightforward implementation of tables using
snodes (such as the implementation of s*nodes in the server level)
suffers from both performance and reliability problems.

Note, again, that gtables are {\em not} implemented in terms of the public
operations on the s*snode class.  It does, however, exploit the internal
mechanisms (such as the defstruct) used for s*snodes. This is not
necessarily a good design decision.


\paragrs is enormously complicated by several
factors:
\begin{itemize}
\item When a new client connects, it must be possible to bring the client
  ``up to speed'' with a consistent, local gtable representation, even
  though other connected clients may be simultaneously changing the
  database and the gtable.
\item When a client changes a gtable, other clients should access the
  database only to get the changes, not to get the entire table.
\item The current implementation was performed in a situation where the
  server process was a defect-ridden black box, which led to amazing
  work-arounds. 
\end{itemize}


\subsection{Representation of gtables}

\subsubsection{Client-side representation}

On the client side, the representation is each individual gtable is simple:
each s*gtable is represented using a u*table instance.

In addition to the u*table for each gtable instance, there is an additional
u*table which has a single entry for each gtable containing specialized
structural information.  This is the s*snodes table.  The s*snodes table is
always stored in the {\em boot} node (with node-ID 2), and is the first
thing that each client downloads on connecting to the hyperbase. Once the
client has downloaded the s*snodes table, then it ``knows'' about all the
currently defined gtables and where to get them. The structure of each
entry in the s*snodes table is provided in Appendix \ref{app:snode}.


\subsubsection{Server-side representation}

On the server side, the representation is more complicated.  There are at
least two s*snodes used to store the global table.  

\paragraph{Primary nodes.}

The first snode stores the current set of entries, or key-value pairs, with
one entry per line.  Each key and corresponding value is retrieved by
calling READ twice per line. The final line of this {\em primary} snode
contains a list containing a single integer---an update counter.
(S*gtables share the u*table constraint that NIL may not be stored as a
value.) These primary snodes are named {\tt msnode-<table-name>}.

\paragraph{Auxiliary nodes.}

The second, {\em auxiliary} snode contains a record of the last update made
to the s*gtable, and is named {\tt asnode-<table-name>}.  This snode
contains one line in the following format:

\small\begin{verbatim}
<update counter> (<operation> <key> <data> <user-name>)
\end{verbatim} \normalsize

{\tt <Operation>} is one of: recover, add, delete, or update.  {\tt <Update
counter>} is an integer which is set to 0 when a recovery operation occurs,
and otherwise incremented each time the auxiliary node stores a new update
to the s*gtable.

\paragraph{Inodes.}

There are more than two snodes used to represent a gtable, if the gtable is
a {\tt :automatic}.  In this case, there is an individual ``backup'' snode
created for each entry in the gtable.  These are called {\em inodes} in the
implementation.

\subsection {Processing Algorithms}

Processing algorithms sometimes depend upon whether
the client is a normal user client or the gagent.  Differences are noted
where they occur.

\subsubsection{Initialization}

Initialization is only performed by the gagent, and consists of the two
steps: (1) add update-counter = 0 to all primary gtable nodes; and (2)
ensure that the gtable is empty.

Note that new entries are always appended to the end of buffer, instead of
inserting at the beginning of buffer (node).  This applies for both the aux
and primary tables.

The last entry of the primary table is the update counter (currently stored
as list object, so that I can use this to indicate the end of buffer as
well).

\subsubsection{Connection}

Connection involves an intricate dance between the client and the gagent.
This is required in order to bring the client up to speed, as discussed
above, and as discussed further below in Appendix
\ref{app:synchronization}.

\paragraph{Gagent connection.}

When a gagent connects, the following occurs:
\begin{enumerate}
\item Read and process all primary tables.
\item For each primary table, do until update-counter is seen:
  \begin{enumerate}
    \item  Read entry (key and data).
    \item  Install it to the local gtable (hashtable).
  \end{enumerate}
  
\item Set this update-counter to the local-variable-update-counter of the
  corresponding auxiliary table.)  Thus, for example the update counter found
  in \&\&gtable-msnode-t*nodes\&\& is set to variable
  \&\&gtable-asnode-t*nodes\&\&-counter (I shall call the later the
  local-update-counter)
	
\item Keep all primary tables in each corresponding buffer, so that
  subsequent update to the primary node involves only write operation.
\end{enumerate}

Gagent connection is relatively simple because it assumes that no other
clients are currently connected, and thus that no changes are being made
to the database. User connections cannot make this assumption, and the
result is the following.

\paragraph{User connection.}
\begin{enumerate}
  
\item When a client is attempting for connection, it first
  subscribes to all events, then locks node-ID 2 (bootnode).  Note:
  events in the event buffer are always in the same order as seen
  by agent and other connected clients.
  
\item The agent operates as usual until it detects/processes this lock
  event by the connecting client. In this case, it stops doing anything: it
  writes the content of all primary nodes into hyperbase and waits until the
  client releases the lock.
  
\item The client then embarks upon the following activities: 
\begin{enumerate}
\item The client resets the filter function associated with the read
  socket and attaches process buffer to this socket.  The process buffer
  implies that all data/command read from the hyperbase will be inserted
  into this buffer. The emacs automatically updates the process-marker
  when inserting data into this buffer, and the main routine uses variable
  s!sp!process-header to read the data/any values returned from hyperbase.
  
\item The client downloads node-ID 2 (in s*snode!bootstrap), and installs
  the content of this node in table s*snodes.
  
\item The client downloads each primary node after detecting that the
  agent has written it to the hyperbase. To detect this, it searches the
  event buffer for this write data event.
  
\item Next, the client processes the content of the downloaded primary
  nodes by calling the corresponding recovery-boot-function. This involves:
  (a) installing the (key data) pair into corresponding table; and (b)
  setting the local counter to the value read from primary node.

\item Next, it resets process buffer.
\item Next, it reactivates the filter function associated with this read
  socket.  
\item Finally, it sets s*sp!event-header to the entry right after the lock
  event.
\end{enumerate}
  
\item The agent resumes its operation upon detecting that the lock on the
  boot node by the connecting client has been released.
  
\item The client processes its pending events beginning from the one right
  after its lock event.

\item The client starts its normal operation.

\end{enumerate}



\subsubsection{Client perspective: Add, Update, or Delete Operations}

When these functions are invoked by the client process,
only the following occurs:

\begin{enumerate}
  
\item Client creates a temporary node and stores the new entry in it.

\item The temporary node has a special name:  \&\&gtable-isnode-$<$table$>$\&\&

\end{enumerate}

\subsubsection{Gtable perspective: Add, Update, or Delete Operations}

When the gagent receives write data event on an is-node, it does
the following:

\begin{enumerate}

\item The gagent reads the contents of the is-node.
  
\item The gagent determines which gtable this is-node refers to by parsing
the name.

\item The gagent updates the corresponding aux node.  This requires
  getting the last update-counter from this aux node. If the aux node is
  empty use local-counter stored in the update-counter slot.  This slot is
  initialized during connection with the value read from the corresponding
  primary node.
        
\item The gagent appends the contents of the is-node to the end of the
  aux node and increments the value of update-counter.
  
\item It then cleans up entries in the aux node if upper-water-mark has
  been reached.

\item Finally, the gagent writes the aux node to the hyperbase.

\end{enumerate}



\subsubsection{Recover}

Only the agent can currently invoke the recover operation.  However, this
operation is not yet tested.

It runs only during s*sp@initialization.  There are two cases: one for the
primary node and one for the aux node.

For the primary node:
\begin{enumerate}
\item If :automatic recovery is defined, calls s*gtable!auto-recover-fn
           otherwise calls user-recovery-fn with argument
           s*gtable!put-entry-to-buffer
\item Append update-counter 0 at the end of the buffer.
\item Save the buffer to hyperbase.
\end{enumerate}

For the aux node, simply clear the node.


\subsubsection{Data Event processing}

All clients, including the gagent, perform data event processing in the
following fashion.  Note that normal clients never process their own 
events. 

When a data event is triggered by s*sp!parse-event, the following occurs:

\begin{enumerate}
\item Get entry from local-cache.
\item If nil, read from hyperbase:
  \begin{enumerate}
   \item Search the entry where update-counter = local-counter + 1.
     
    \item If not found, and if buffer is empty then simply ignore.
     Otherwise signal error if no previous counter or previous-counter
     is not equal to current local counter
     
    \item If found, then get first entry (counter + information), and cache
     remaining entries into local-cache.
   \end{enumerate}
     
 \item Process the entry (if non-nil) according to its operation type:
   add, update, delete, or recover.  (If the client is the agent, then the
   entry is written to the backup node if inode-p is specified for this
   table.)
     
 \item If this is agent, then write to the corresponding primary node.
   Since, the agent keeps the primary-node buffer local, the entry read from
   the aux will be inserted into this buffer, and won't be written to
   hyperbase until the local-cache is empty.

\end{enumerate}

\subsubsection{Recovery Event Boot}

This function is used during connect time to initialize table with the
content of primary node.  This function reads data from the process-buffer
activated during bootstrapping.  It processes things in the following
manner:

\begin{enumerate}
\item Read through all entries in the node and install them to
  corresponding table.

\item Set local-update-counter to the value found in last entry.

\item If agent, keep the buffer (i.e., create primary-node buffer).
\end{enumerate}


\newpage
\section{Gtables in Egret-3.0.x}

\subsection{General Assumptions}

Let me start with a few general assumptions along with the rationale for
making the design decision.  This initial section provides a broad overview
of how things could work in 3.0.x.  Hopefully it will stimulate the reader to
discover errors or improved mechanisms.

\subsubsection*{Scope of Gagent Duties} 

In 3.0, the event message field will be engineered in such a fashion as to
provide all necessary information for non-connect time gtable maintenance.
For example, each node creation event must include a message field that
provides all the details about the node necessary to update all gtables.
(Note: this might be somewhat interesting to engineer from the EGRET side.
More about this below.)

An implication of this assumption is that client-gagent interaction occurs
only at connect time.

{\em Rationale:} This assumption means that the scope of gagent duties are
restricted to one goal: to support efficient and reliable connection by
clients.

\subsubsection*{Server-side representation of gtables}

Each non-{\tt :automatic} gtable will be represented in the HBS by a single
primary node.  Each {\tt :automatic} gtable will be represented in the HBS
by a single primary node and by backup nodes for each entry. 

The implication of this is the elimination of all current aux nodes and
is-nodes, as well as all current aux node and is-node processing.

{\em Rationale: } Since gtables are updated through events after
connect-time, rather than through gagent communication, these
representations and their associated processing are no longer necessary.

\subsubsection*{Locking, or the lack thereof}

Surprisingly, in this redesign there is no locking-related issues, since
clients and the gagent have a strictly read-only relationship with each other.

First, the gagent can keep a permanent lock on the gtable primary nodes,
since it will be the only process writing to them.  And gtable primary
nodes (and backup nodes) are the {\em only} nodes that the gtable is
concerned with.  The gagent never writes to a user-node, and in fact never
even reads one except when doing gtable recovery.  (It uses event messages
under normal processing circumstances.)

Similarly, clients can lock or unlock any of their own nodes independent of
any gagent considerations, since the gagent will never desire a lock on
them.
 
\subsubsection*{Changes for HBS 3.x}

The HBS will be extended by changing the behavior of the CONNECT 
operation and by adding several new operations.

\begin{itemize}

\item CONNECT\_AS\_GAGENT (client)
  
  This is a special form of connect to be invoked only by the gagent.  It
  tells the HBS that CLIENT is a special client whose commands will not be
  queued during regular connect time processing. Otherwise, this command does
  the same thing that CONNECT does in HBS 2.x.

\item CONNECT (client)
  
  In HBS 3.x, the CONNECT command will be changed so that immediately
  upon receiving it, operation requests from all clients {\em except} the
  gagent and this client will be queued.  This queuing will continue until
  this client sends a CONNECTED event.
  
  However, all events generated by the operations of the client and the
  gagent during this {\em connecting time} are sent out to all clients as
  usual.

\item CONNECTED (client-name)
  
  When the HBS processes this command, it starts processing command
  requests from all clients in a normal fashion.

  When clients receive the CONNECTED event, they update their s*users
  gtable to indicate that the client connected successfully.

\end{itemize}

{\em Rationale:} These new operations are intended to support simple and
reliable client-gtable communication during connect-time.

\subsubsection*{Event and Message Processing}

This section contains several miscellaneous assumptions concerning events
and messages.

\begin{itemize}
  
\item All clients (including the gagent) always provide their s*user*ID as
  part of the message field when they invoke HBS operations.  
  
  {\em Rationale: } The identity of the client initiating an operation is
  always known in a uniform manner using EGRET-level representations. This
  is also needed since the user-name of the gagent or other agent might be
  the same as a client (i.e. johnson, dat, etc.)
  
\item All clients (including the gagent) always ignore any event for which
  they were responsible (as evidenced by their s*user*ID in the message
  field.) 
  
  One implication of this is that clients are always responsible for
  updating their own local gtables with the result of their own actions.
  This also means that chronological ordering of events among clients
  will not be preserved---I might update my own gtable before processing
  an event from another client that represents an action that occurred
  before mine. See Appendix \ref{app:synchronization} for an in-depth
  discussion of this issue. In particular, this may have implications for
  face-to-face mode.

  {\em Rationale: } This establishes a uniform processing invariant that is
  intended to simplify event processing algorithms. 

\item The gagent is the only client that reacts to CONNECT events.
  
  {\em Rationale: } This is intended to simplify the gagent implementation,
  by allowing the gagent to be implemented as a ``normal'' EGRET client that
  simply does some additional processing under certain circumstances.
    
\item All clients always ignore gagent-initiated events, except when they
  are connecting.
  
  {\em Rationale: } This is intended to raise a ``firewall'' between the
  clients and the gagent, and further ensure that the scope of their
  interaction is limited to connect-time only.  It also allows the gagent
  to periodically make changes to the gtable primary nodes without worrying
  that clients will react to these events.

\end{itemize}


\subsection{Connect time processing}

Now, let's look at the sequence of actions during connect time, first from
the gagent's perspective, and next from the client's perspective.

\subsubsection{The gagent perspective}

Normally, the gagent is sitting around processing events just like
everybody else.  However, something special happens when clients connect, 
as described next.

\begin{enumerate}
  
\item The HBS receives a CONNECT request.  
  
  All command processing is immediately suspended except for those from
  the gagent and this connecting client. It sends out the CONNECT event,
  which is the last ``regular'' event.
  
\item The gagent receives the CONNECT event.  
  
  It knows that no other events can follow this event, and thus that no
  other changes can be made to the database. By definition, the gagent
  does not have any pending events.
    
  In response to the CONNECT event, the gagent then writes out an {\em
  accurate} (as defined in Appendix \ref{app:synchronization} version of
  each gtable to the gtable's primary node.  (Each of these writes will
  generate an event that will be ignored by all clients except for the
  connecting client. Why?  Because only connecting clients listen to
  gtable events.)

\end{enumerate}

First, note that this single, special behavior, triggered by the receipt of
a CONNECT event (that is always ignored by all other clients), is the {\em
only} major difference between a gagent and a normal client in EGRET.
Otherwise, gagents are exactly similar to normal clients: they maintain
their own local copies of gtables, and update them in response to HBS
events just like everyone else.

Actually, there is one other (minor) difference: the connect sequence for
gagents is also different: the gagent uses CONNECT\_AS\_GAGENT and builds
the gtables by itself.


\subsubsection{The client perspective}

Now let's look at connecting time from the client's perspective:

\begin{enumerate}

\item The client sends a CONNECT command to the HBS.  
  
\item It downloads the boot node to obtain the information needed to
  determine how many primary nodes there are and other relevent
  bootstrapping information.
  
\item The client begins processing events, ignoring all events until it
  sees its CONNECT event.
  
\item Next, the client reads each gtable primary node in response to a
  write data event by the gagent on that node.
    
\item Once all gtables have been initialized, the client sends a
CONNECTED command to the HBS and begins processing events normally.

\end{enumerate}

\subsubsection{Concluding Design Rationale Commentary}

\paragraph {The good news.}
This connect time algorithm is extremely simple, which should make it
easier to verify and debug. Note the following pleasant features.

First, gagents only support client connection.  This vastly simplifies the
representation of gtables in HBS and the gagent's processing.  In fact,
gagents can now be implemented as ``normal'' Egret clients, with only two
special cases: connection of the gagent itself, and processing of CONNECT
events.

Second, clients have a well-defined, and strictly delimited relationship
with the gagent. After connect time, clients can function perfectly
normally even if the gagent crashes. Locks are not even mentioned in this
design, since they are essentially irrelevent.

Third, the handshake between client and gagent is very simple: the client
says, ``I'm connecting'', the gagent responds by writing out an accurate
version of all the gtables, and the client responds to those writes by
reading them the gtables. 

In conclusion, a simple algorithm, will well-defined roles and strictly
limited responsibilities, is typically easier to understand, easier to
implement, and easier to make reliable.

\paragraph {The bad news.}  
This connect time algorithm may incur significant performance penalties 
over our current (unreliable, and currently nonfunctional :-)
implementation for the following three reasons.

First, at connect time the entire contents of all gtables must be written
to the HBS by the gagent.  In addition, the entire contents of all gtables
must be read from the HBS by the connecting client. In general, the time to
connect will be a minimum of the sum of times to both read and write each
gtable. This is the {\em best case} scenario.

Second, connect time may be much worse than the best case, in the event
that multiple clients connect simultaneously.  In this case, the effective
connect time is the best case time {\em multiplied} by the number of
simultaneously connecting clients.  I refer to ``effective'' connect time
since even after successful connection, the clients will be blocked from
accomplishing any work until after the last client connects. 

Three, clients will suffer momentary pauses during processing each time a
new client connects. This pause will be equal at a minimum to the best case
scenerio, or potentially worse if simultaneous connections occur.  

The usability impact of this performance degradation is a function of (1)
how long it takes to read and write the tables, which is itself a function
of their size, and (2) the frequency of co-occuring connections.

Fortunately, I believe that a simple change on the client side can
substantially ameliorate this problem.  Remember that the client has two
things to do to set up a gtable.  First, it must download the gtable node
from the HBS.  Second, it must process the node contents, and build an
internal hash table.  I believe that well over half of our current connect
time processing involves the second activity, which is purely local
processing.  Thus, consider client-side processing as follows:
\begin{enumerate}
\item Download {\em all} gtable nodes into separate buffers without any
local table building.

\item Send the CONNECTED command.

\item Build the local tables. 

\item Begin processing events. 
\end{enumerate}

In this case, other clients will not be waiting for a client to finish
building its own local tables.  Just as a guess, I bet that it takes no
more than a couple of seconds to read or write most gtables if no local
post-processing is going on.  I would think that a half-dozen gtables could
be processed in about 10 seconds. 

\subsection{Non-connect time processing}

As mentioned above, non-connect time gtable processing occurs through
message passing.  The principle engineering problem here are: (1) how to
specify in the s*gtable*define form what needs to go in a message, (2) to
deal with the fact that different gtables may want to add their own
messages to the same HBS event, and (3) to allow each gtable access to its
own message once an event comes in.

These issues aren't hard, but they do require some work. I leave it as an
exercise to the reader. :-)



\appendix

\newpage
\section{Problems in connect-time processing}

Danu notes that these problems have theoretically been fixed in 
Egret-2.10, but that they may still have value in stimulating our
thinking about the current event processing problems. 

\begin{enumerate}
  
\item One problem with connect-time processing is that s*sp!parse-event is
  sometimes run even when another s*sp!parse-event hasn't yet finished
  processing.
  
  For example, when the agent processes the lock event, it is supposed to
  cease any processing but writing the content of primary nodes to
  hyperbase and wait until the bootnode is unlock.  Unfortunately, the
  event buffer shows that after sync-ing primary nodes, the agent continue
  doing some processing instead of waiting for unlock event.  This can only
  happen if another s*sp!parse-event is run while the other one waits for
  unlock event.
  
  In general, s*sp!parse-event is run either by s*sp!event-process-filter
  or s*sp*with-hb-call-protected depending on the value of "flags":
  
\item In the original code, the calls to s*sp!parse-event is only
  controlled by the flag s*sp!in-command.  Thus, when hb-sys-call in
  session s*sp!event-process-filter should never call parse-event.
  Unfortunately, when s*sp!parse-event is currently running and then blocks
  for a while (for example, waiting for unlock node), then another
  s*sp!event-process-filter will run because the flag s*sp!in-command is
  nil.
  
\item The current implementation uses two flags: s*sp!in-command and
  s*sp!pending-event to control the parse event.  If s*sp!in-command or
  s*sp!pending-event is true than don't calls s*sp!parse-event.  This,
  however, still has problem as shown by the example in (1).
\end{enumerate}
 
\newpage
\section{Current Gtable Documentation String}
\label{app:gtable-doc-string}

\small\begin{verbatim}
s*gtable*define: a Lisp macro.

Defines gtable called TABLE-NAME with specified properties.
Gtables are shared, persistant objects containing a set of key/value pairs.
Keys may be any symbol. Data may be any NON-NIL lisp object.

s*gtable*define generates the following operations (with FN-PREFIX)

  PUT (key data) => data     Put key/data pair to the table and shared store.
  DELETE (key)  => key       Deletes key/data from the table and shared store.
  GET (key) => data          Returns data corresponding to key, or NIL.
  MAP (fn) => NIL            Maps through table, calling FN with each key and value.
  RECOVER () => NIL          Rebuilds table from database contents.

It may also generate the following functions, depending upon its arguments:

  GET-COMPLETION-LIST () => LIST   Returns the table's completion list.
  GET-KEY (data) => KEY  => Returns key that matches reverse-lookup data.

Keyword TABLE-NAME is a required symbol indicating the name of table.

Keyword FN-PREFIX is a required string used as prefix of table operations.

Keyword HASHP, if non-nil, specifies hashtable as the internal representation.
  Defaults to nil, specifying an association list internal representation.

Keyword TABLE-SIZE is an optional prime number indicating the size of hashtable. 
 Meaningful only when HASHP is non-nil.

Keyword RLOOKUP-FN is an optional arg indicating DATA's component on which reverse
 lookup is done.  This argument generates the GET-KEY function.
Keyword COMPLETION-FN is an optional arg indicating DATA slot on which 
 completion list will be maintained. This generates the GET-COMPLETION-LIST.

Keyword RECOVERY-FN is a required argument specifying a function that will returns
 a buffer with is its contents initialized from the current state of the database.

Keyword AGENTP is optional Boolean indicating agent is used. When t, the main 
snode is done by the agent instead of individual clients (OBSELETE).

Keyword NEW-ENTRY-FN is an optional arg specifying additional event fns
  in the case of new entry being added. These fns might be used to change 
  other elements in local state. 
  Each NEW-ENTRY-FN fn is called with two args: KEY and DATA.

Keyword UPDATE-ENTRY-FN is an optional arg similar to 
  NEW-ENTRY-FN except it is called upon updating an extant entry.
  Each UPDATE-FN is passed three args: KEY NEW-DATA OLD-DATA

Keyword DELETE-ENTRY-FN is an optional arg similar to 
  NEW-ENTRY-FN except it is called upon deleting an extant entry.
  Each DELETE-ENTRY-FN fn is passed two args: KEY and DATA.

Keyword BOOT-ENTRY-FN is an optional arg similar to
  NEW-ENTRY-FN except it is called during booting process.
  Each NEW-ENTRY-FN fn is passed two args: KEY and DATA.

Keyword NODE-ENTRY-TABLE is Boolean value indicating whether gtable is 
  responsible for creating primary nodes for gtable entries.

Keyword ACCESS-PRIMARY-TABLE-P is Boolean value when specified will generate
two functions (with FN-PREFIX):
  PRIMARY-PUT (key data)  Put key/data pair to the primary table
  PRIMARY-DELETE (key)    Delete key from the primary table
Only AGENT can invoke these functions.

Returns TABLE-NAME.
\end{verbatim}\normalsize



\newpage
\section{Current snode structure}
\label{app:snode}

\small\begin{verbatim}
(defstruct (s*snode (:type vector)
                    (:named))
  "structure holding snode information.
NAME slot holds snode name for uninstantiated node, or
NODE-ID if snode has been instantiated.
Acccess key to snode is unique NODE-NAME, which is different from
other tables. Use S*SNODE!GET-KEY to get NODE-ID of a given snode."
  ID                         ; unique snode ID
  connect-fn                 ; connect-hook function
  data-event-fn              ; data event hook function
  disconnect-fn              ; disconnect hook function
  recovery-fn                ; content recovery function
  gtablep                    ; whether it is GTABLE
  table-name                 ; corresponding ltable name
  msnode-name                ; main snode name
  prefix                     ; inode and table fn prefix
  (update-counter 0)         ; counter since last processed entry 
                             ; in the aux node
  local-cache                ; cache containing remaining entries to be 
                             ; processed in aux node. When processing 
                             ; the aux node (in response to data-event), 
                             ; the program first reads from this local
                             ; cache. Only when this cache is nil is the 
                             ; read operation to the hyperbase performed.

  )

\end{verbatim}\normalsize

\newpage
\section{Synchronization in Egret}
\label{app:synchronization}

{\em Editor's note: The following is a reformatted e-mail message from 
November of 1992 containing additional information about synchronization
in gtables.}


\paragraph{Motivation.}

The introduction of agents into EGRET has brought to the surface the
general issue of synchronization, or, more specifically, how can a
common representation of global state be maintained in a distributed
environment. This document will attempt to show two things: (a) that
maintaining ``perfect" synchronization between clients in EGRET
requires too much overhead to be practical, and (b) the real issue
we are confronting is how much less than ``perfect" synchronization 
is acceptable in our domain?

This document is structured in terms of the following claims, 
loosely summarized as follows:

\begin{itemize}
\item  Gtables have a conceptual global state.
\item  Gtables are semantically inter-dependent.
\item  Client state consistency depends upon order of gtable update.
\item  Client state consistency doesn't depend upon order of gtable update :-)
\item  Our current design allows client state inconsistency to arise in 
    several ways. 
\item  One kind of client state inconsistency is acceptable.
\item  Another kind of client state inconsistency is unacceptable.
\item  Small changes to the design can lead to tolerable inconsistency.
\end{itemize}

This should give you a feel for where I'm going in the paragraphs
below.  I will also introduce a bunch of new terms which will help us
in our discussion.  Let's begin.

\paragraph{Global State.}

There is the concept of a ``global state" of a gtable.  A physical
instance of a gtable that corresponds precisely to the current global
state will be called an ``accurate" gtable.

The current global state of a gtable can be derived by taking the
state of the primary node and invoking upon it the set of ``unapplied"
operations in its auxiliary node.

The global state can also be derived from a locally cached copy of a 
gtable by invoking the unapplied operations in the auxiliary node. 

To create a physical instance of the current global state of a gtable,
one must first lock its auxiliary node to prevent any further updates,
then invoke any unapplied operations.  This results in an ``accurate"
representation of a gtable, since it corresponds perfectly to the
current global state.  As soon as the aux node is unlocked, it can no
longer be assumed that the representation is accurate, since another
client may immediately lock the aux node and insert a new operation,
thus changing the global state. Certain kinds of EGRET operations will
require one or more ``accurate" gtables, as will be shown below.

\paragraph{Gtables are semantically interdependent.}

The information contained in different gtables is not logically
independent. For example, the node instance gtable contains
node-schema-IDs, and each node-schema-ID in the node instance gtable
should be defined in the node-schema gtable.

I will say that a client state is ``inter-gtable consistent" when all
such data dependencies between gtables are satisfied.  I will say that
a client process state is ``inter-gtable inconsistent" when such data
dependencies are not satisfied.  The central issue to be solved in
EGRET is how to ensure that inter-gtable consistency is always
possible, even if it is not currently present.

\paragraph {The order of event processing is significant in maintaining
clients in a state of inter-gtable consistency.}

Each time a client invokes an unapplied aux node operation, it updates
its local state.  In general, this new client state may be either
inter-gtable consistent or inter-gtable inconsistent.  For example, if
the operation adds a node instance I with schema value S to the node
instance gtable, but the schema S does not currently exist in the node
schema gtable, then applying this operation will put the client state
into inter-gtable inconsistency.

Assume for the moment that we want clients to always be in a state of
inter-gtable consistency after every application of an aux node update
operation.  One way to do that is to guarantee the following two
conditions:

(1) Clients must always invoke EGRET functions in an order that 
maintains inter-gtable consistency.  For example, an operation that 
creates a new node schema S must be invoked before an operation 
that creates a  new node instance I with schema S.  

(2) Events signalled by EGRET functions must be processed in the 
order that they are generated, even when the events relate to 
different gtables.   In the above case, if the event associated with the 
node instance creation is processed before the event associated with 
the node schema creation, then the client process state will be inter-
gtable inconsistent until the node schema creation event is processed.

We will see below that satisfying the constraint of continuous
inter-gtable consistency is expensive.

\paragraph {Incorrectly ordered event processing does not necessarily preclude
eventual inter-gtable consistency.}

A desirable property of gtables and the EGRET implementation is that
if a client has ``accurate" versions of all gtables, then it also is in
a state of inter-gtable consistency.  Let's refer to this as the
``accuracy implies consistency" claim.

If this property is true, then a client can create an inter-gtable
consistent state whenever it wants to by simply locking all auxiliary
nodes (to prevent any further updating), then invoking all unapplied
updates to all of its local gtables.  During this process, the client
may transition through intermediate states of inter-gtable
inconsistency on the way to consistency.

\paragraph {Our current design allows inter-gtable inconsistency to arise
during connect time.}

Assume that the current design merely requires clients to download 
the global table at connect time, then begin responding to events. In 
this case, a client state containing a node instance I with undefined 
schema S could arise if either: (a) the set of primary nodes are
inter-gtable inconsistent because the Agent did not respond to 
events in the strict order they were received, or (b) the newly 
connected client did not receive all events generated by the 
unapplied operations in the aux node.  Since the order in which 
events are received dictate the order in which aux node operations 
must be applied, the lack of these events means that information
necessary to maintain consistency is simply not available.

\paragraph {Our current design allows inter-gtable inconsistency to occur 
during a client session.}

A simple illustration of this phenomenon is the following form:

\begin{verbatim}
(t*node-schema*ID-p (t*node-schema*make))
\end{verbatim}

There is no reason to assume that this form will return T, due to the
fact that the local gtable state is updated sometime after the make
form returns, and the ID-p operation uses the local gtable
representation to test for validity.

The designbase update procedure illustrates another example of inter-
gtable inconsistency. One situation is the following: the client
creates a new node schema.  This results in the client updating the
node schema aux node with a schema creation update operation.
However, the presence of this node schema is not reflected in this
clients local state until it receives the associated event and
retrieves this aux node from the database.  However, before this event
is processed, the client has gone on and attempted to create a node
instance using this schema, which signals an undefined schema error,
since the schema does not yet exist in the client's local gtable.

Many other examples can easily be constructed where delays in
receiving events or updating gtables in an improper order can lead to
inter-gtable inconsistency.  However, as long as the ``accuracy implies
consistency" claim is true, then these inconsistent states are
guaranteed to be temporary.

\paragraph {Eliminating the possibility of even temporary inter-gtable
inconsistency in a client is expensive.}

The only approach I can think of involves satisfying the following
two conditions:

First, it must eliminate the ID-p phenomenon illustrated above,
since it is a fundamental form of inconsistency (the integer value
returned by the make for is semantically a legal node-schema-ID, but
this semantic fact may not be reflected in the local state).

Second, it must guarantee that no application of any aux node 
update operation ever results in inter-gtable inconsistency.  A 
reasonable approach to this involves guaranteeing that both EGRET 
functions are invoked in a manner that maintains consistency and 
that clients process events in the order in which they were 
generated, as noted above.

Interestingly, it is non-trivial to satisfy both of these conditions
simultaneously.  The obvious way to eliminate the ID-p phenomenon is
to update the local state of the client that invoked the operation
immediately, and ignore the event that occurs later.  Unfortunately,
in order to also satisfy the second criterion, the client must process
all other events that occurred before he generates this new event.
Thus, any client invoking an EGRET function that results in a gtable
update operation must first lock all of the aux nodes, process all of
the events in the correct order (to create accurate versions of all
gtables in his local state without creating any inconsistent
intermediate states), and finally carry out the operation that will
result in the gtable update.  Since all prior events have been
processed, the client is free to immediately update his local state
without violating the second condition.

Again, this solution requires multiple locks, which at the very least
will require some simple deadlock prevention design measures.  It will
also create significant processing overhead, as now a simple node
creation operation (for example) will require N hyperbase accesses and
locks, where N is the current number of defined gtables in the system
(N = 6 in the current system).

This analysis leads me to the conclusion that guaranteeing continuous
consistency is impractical (unless a smarter way to accomplish it is
thought of).  A plausible alternative is the one hinted at several
times in this document: if the accuracy implies consistency claim is
true, then we can guarantee that ``eventually" the client state will be
consistent, even if it is temporarily inconsistent.  The next section
investigates the issues in satisfying this claim.

\paragraph {Implementing the ``accuracy implies consistency" design invariant 
requires some thought.}

Another way of viewing this invariant might be the following:
temporary inconsistency may result in the system preventing you from
doing something that you really should be allowed to do, but it cannot
allow you to do something that you shouldn't be able to.

For example, temporary inconsistency is permitted to result in a 
situation such as where the person sitting at the workstation next to 
you creates a new node schema with ID= 42, but you get an ``invalid 
node schema ID" error when you immediately try to create a new 
node instance with that node-ID.  However, if you wait sufficiently 
long, you will eventually be able to create that node instance. 

On the other hand, temporary inconsistency is not permitted to result in a
situation where the person sitting at the workstation next to you
successfully deletes the field-schema with ID = 12, and then you also
successfully perform a field-schema deletion operation with ID = 12.
(Temporary inconsistency would allow you to successfully create a node
instance with ID = 12 even after another client has deleted it, as long as
it is guaranteed that this use of the deleted field-ID would eventually be
propogated to your local state, thus preventing you from doing that
operation.

The distinction between legal and illegal situations can also be viewed as
follows: if two events can be re-ordered in time so as to make them
semantically legal, then temporary inconsistency allows them to occur. On
the other hand, if two events cannot be re-ordered in time so as to make
them legal, then they are also not legal under temporary inconsistency.
Multiple deletes of the same field-schema-ID fall into this latter
category.

Under this interpretation, it appears that only delete operations 
require special handling to ensure the accuracy of the local gtable 
before carrying out the operation.  Addition and modification 
operations are legal, although they may result in strange behavior 
(such as adding a field ID to a node instance that was just deleted by 
some other client).  

Here is a hypothesis which I suspect to be true, but for which
everyone should try to find a counterexample. Let's refer to this as
the ``update only accurate gtables" hypothesis:

The current use of gtables permits the ``accuracy implies consistency"
criterion to be satisfied when:
\begin{itemize}
  \item An update to any individual gtable is only performed after 
      assessing its validity with respect to an ``accurate" version
      of the gtable.
  \item Only an accurate version of the gtable being updated is required,
      all others may be inaccurate. 
\end{itemize}

Note that the ``update only accurate gtables" assumption not only
permits ``chunked" updates by clients, but actually demands it in the 
case where a client is itself creating the update. 


\paragraph{ ``Accuracy implies consistency" does not imply ``EGRET is usable".}

The accuracy implies consistency claim means that the client will 
eventually have a consistent state, but says nothing about the 
proportion of time spent in an inconsistent state.  It is conceivable 
that under certain usage patterns, the client state is nearly always 
inconsistent, even though it is guaranteed to eventually become 
inconsistent.  This is not good from a usability standpoint. 

The ID-p inconsistency presented above is perhaps representative of
the most typical user-level problem resulting from temporary
inter-gtable inconsistency: a client performs one operation, then 
attempts to perform a second operation that depends upon the first,
but is prevented because the local state has not yet been updated
to reflect the results of the first operation.

It appears much rarer for a client to be prevented from performing an
operation because they were not provided an update from another
client's operation quickly enough.

\paragraph{ ``Update only accurate gtables" allows elimination of the
common user-level inconsistency problem.}

One simple way to correct this type of problem is to have the
initiating client update his local state during the operation, rather
than returning from the operation with its local state inconsistent
and waiting for the event processing mechanism to ``catch it up" to an
accurate gtable. This happens automatically under the hypothesis,
since the initiating client must validate the operation with respect
to an accurate gtable, which involves performing all the updates up to
the actual operation being invoked.  There is thus little point in not
going one step further and updating the local state with the actual
operation.

\paragraph {Consistency at connect time can be achieved.}

By changing the design to connect and create ``accurate" versions of 
all gtables simultaneously, the connect time inconsistency problem 
can be eliminated.  The cost is that all aux nodes must be 
simultaneously locked.  Note that by property (4) it does not matter 
what order the gtables are updated during the connect time process, 
since the result is defined to be inter-gtable consistent.  (Note that 
locking more than one node at a time requires deadlock 
prevention/recovery procedures.)

\paragraph{Acceptable consistency during a session can be achieved.}

The most important insight that this discussion has given me is that
*temporary* inconsistency is acceptable in our system, as long as
*permanent* inconsistency is prevented. It is not absolutely clear to
me that the ``update only accurHere is (hopefully) my last e-mail before the meeting today.  It contains
an explanation for the reason why Egret is hanging, at least some of the time, and 
a proposal for a *simplified* event processing mechanism that should guarantee
that such hangs cannot occur in the future.  These changes can be implemented into
2.10.x very easily and tested. 

Despite Danu's claims to the contrary, when Egret is accepting process output from 
the read socket (such as when it is getting a return value from the server), Emacs will
run the filter function associated with either the read or the event socket. (I'm completely convinced of this
now, for the simple reason that if this were not the case, then the EHTS system wouldn't
have needed the in-command variable and neither would we.)

Why is this a problem?  It's because if the processing of an event *also* leads to an hb-sys-call,
then Egret can quickly get confused. Consider the following sequence of interactions:
     Command X is waiting for a return value from the read socket.
     Egret receives an event Y.  (The filter function is guaranteed to run.)
     The filter function (erroneously) executes the action associated with the event Y.
     This Action Y calls hb-sys-call command Y.
     Command Y calls accept-process-output from the read socket.
     The return value from X finally comes in, except it is read as Command Y's return value.

Obviously, at this point things are totally unpredictable, but a hang is a pretty likely eventuality.

So, since we can't prevent Emacs from running the event filter function whenever 
we're waiting for a return value from the read socket, both we and EHTS do the next
best thing: we save these events and only process them once we're guaranteed that any
currently executing hb-sys-call has gotten the data it needs. 

The most reliable way I know of to reason about this kind of situation is to introduce 
"design invariants": things that we ensure are true and that help us informally "prove" the 
correctness of the implementation.  In this case, the principle design invariant we need 
is the following:

  HB operations occur in a strictly serial fashion.

In other words, HB operations cannot be nested.  If HB operations cannot be nested, no matter
where or when they occur, then there's no way for one operation to get the other's return value.
Now for the hard part: how do we guarantee that HB operations cannot be nested?

I propose the following, which I believe is sufficient.  It may not be the only way, and it may 
be slightly over-restrictive, but I do not know how to relax any of these assumptions and still
convince myself that operations are serial:

  (1) All calls to hb-sys-call must be wrapped in a with-hb-sys-call-protected form.  (Cam found that
        even this rule is violated in the current implementation.)

  (2) Calls to with-hb-sys-call-protected can never be nested. (This is also currently violated,
        as I showed yesterday.)

  (3) With-hb-sys-call correctly defers event processing until after the hb-sys-call completes.
//                              -*- Mode: Java -*- 
// DocType.java -- 
// Author          : Philip Johnson
// Created On      : Fri Oct 10 07:43:35 1997
// Last Modified By: Carleton Moore
// Last Modified On: Wed Oct 15 15:59:50 1997
// RCS: $Id: DocType.java,v 1.2 1997/10/10 23:32:30 johnson Exp johnson $
// 
//   Copyright (C) 1997 Philip Johnson
// 
// 


package csdl.java.leap.util;

import java.util.*;


/**
 * DocType --- A data structure containing info about Leap document types.
 * @author         Philip Johnson
 * @version        Wed Oct  8 09:43:44 1997
 */

public class DocType extends LeapData {

  private DocType() {} // hide default constructor.
  private Vector docIDs = new Vector(); 
  private Vector defTypes = new Vector ();

  /**
   * Defines a new document type instance.
   * @param name The document type name.
   * @param comment A string containing documentation on the document type.
   */
  public DocType(String name, String comment) {
    super(name, comment);
  }


  /**
   * Add a new documentID as one of these document types.
   * @param docID The new documentID.
   */  
  public void addDocID (DocID docID) {
    docIDs.addElement(docID);
  }

  /**
   * @return A vector of docIDs.
   */
  public Vector getDocIDs () {
    return docIDs;
  }


  /**
   * Add a new defect type as associated with this document type.
   * @param defType The new defect type instance.
   */  
  public void addDefType (DefType defType) {
    defTypes.addElement(defType);
  }

  /**
   * @return A vector of defect types.
   */
  public Vector getDefTypes () {
    return defTypes;
  }

}



//                              -*- Mode: Java -*- 
// LeapData.java -- 
// Author          : Philip Johnson
// Created On      : Fri Oct 10 08:28:59 1997
// Last Modified By: Carleton Moore
// Last Modified On: Wed Oct 15 15:59:53 1997
// RCS: $Id: LeapData.java,v 1.1 1997/10/10 19:17:37 johnson Exp $
// 
//   Copyright (C) 1997 Philip Johnson
// 
// 

package csdl.java.leap.util;

import java.io.*;
import java.util.*;

/**
 * LeapData --- A data structure providing common slots and methods for all leap data objects (Projects, DocTypes, DocIDs, and DefTypes.)
 * @author         Philip Johnson
 * @version        Wed Oct  8 09:43:44 1997
 */

public class LeapData {
  protected String name;
  protected String comment;

  protected LeapData () {};  // forbid this.

  protected LeapData(String name, String comment) {
    this.name = name;
    this.comment = comment;
  }

  /**
   * @return The name.
   */
  public String getName () {
    return name;
  }

  /**
   * @return The comment.
   */
  public String getComment () {
    return comment;
  }

  public String toString() {
    return "[" + name + "]";
  }
}


%%%%%%%%%%%%%%%%%%%%%%%%%%%%% -*- Mode: Latex -*- %%%%%%%%%%%%%%%%%%%%%%%%%%%%
%% primer-precis.tex -- 
%% Author          : Philip Johnson
%% Created On      : Thu Mar  9 14:21:21 1995
%% Last Modified By: Philip Johnson
%% Last Modified On: Thu Jun  8 13:26:25 1995
%% Status          : Unknown
%% RCS: $Id$
%%%%%%%%%%%%%%%%%%%%%%%%%%%%%%%%%%%%%%%%%%%%%%%%%%%%%%%%%%%%%%%%%%%%%%%%%%%%%%%
%%   Copyright (C) 1995 University of Hawaii
%%%%%%%%%%%%%%%%%%%%%%%%%%%%%%%%%%%%%%%%%%%%%%%%%%%%%%%%%%%%%%%%%%%%%%%%%%%%%%%
%% 

\documentstyle [nftimes,11pt,/group/csdl/tex/definemargins,
                /group/csdl/tex/lmacros,/group/csdl/tex/functiondoc]{article}
\definemargins{1in}{1in}{1in}{1in}{0.3in}{0.3in}
% Psfig/TeX 
\def\PsfigVersion{1.9}
% dvips version
%
% All psfig/tex software, documentation, and related files
% in this distribution of psfig/tex are 
% Copyright 1987, 1988, 1991 Trevor J. Darrell
%
% Permission is granted for use and non-profit distribution of psfig/tex 
% providing that this notice is clearly maintained. The right to
% distribute any portion of psfig/tex for profit or as part of any commercial
% product is specifically reserved for the author(s) of that portion.
%
% *** Feel free to make local modifications of psfig as you wish,
% *** but DO NOT post any changed or modified versions of ``psfig''
% *** directly to the net. Send them to me and I'll try to incorporate
% *** them into future versions. If you want to take the psfig code 
% *** and make a new program (subject to the copyright above), distribute it, 
% *** (and maintain it) that's fine, just don't call it psfig.
%
% Bugs and improvements to trevor@media.mit.edu.
%
% Thanks to Greg Hager (GDH) and Ned Batchelder for their contributions
% to the original version of this project.
%
% Modified by J. Daniel Smith on 9 October 1990 to accept the
% %%BoundingBox: comment with or without a space after the colon.  Stole
% file reading code from Tom Rokicki's EPSF.TEX file (see below).
%
% More modifications by J. Daniel Smith on 29 March 1991 to allow the
% the included PostScript figure to be rotated.  The amount of
% rotation is specified by the "angle=" parameter of the \psfig command.
%
% Modified by Robert Russell on June 25, 1991 to allow users to specify
% .ps filenames which don't yet exist, provided they explicitly provide
% boundingbox information via the \psfig command. Note: This will only work
% if the "file=" parameter follows all four "bb???=" parameters in the
% command. This is due to the order in which psfig interprets these params.
%
%  3 Jul 1991	JDS	check if file already read in once
%  4 Sep 1991	JDS	fixed incorrect computation of rotated
%			bounding box
% 25 Sep 1991	GVR	expanded synopsis of \psfig
% 14 Oct 1991	JDS	\fbox code from LaTeX so \psdraft works with TeX
%			changed \typeout to \ps@typeout
% 17 Oct 1991	JDS	added \psscalefirst and \psrotatefirst
%

% From: gvr@cs.brown.edu (George V. Reilly)
%
% \psdraft	draws an outline box, but doesn't include the figure
%		in the DVI file.  Useful for previewing.
%
% \psfull	includes the figure in the DVI file (default).
%
% \psscalefirst width= or height= specifies the size of the figure
% 		before rotation.
% \psrotatefirst (default) width= or height= specifies the size of the
% 		 figure after rotation.  Asymetric figures will
% 		 appear to shrink.
%
% \psfigurepath#1	sets the path to search for the figure
%
% \psfig
% usage: \psfig{file=, figure=, height=, width=,
%			bbllx=, bblly=, bburx=, bbury=,
%			rheight=, rwidth=, clip=, angle=, silent=}
%
%	"file" is the filename.  If no path name is specified and the
%		file is not found in the current directory,
%		it will be looked for in directory \psfigurepath.
%	"figure" is a synonym for "file".
%	By default, the width and height of the figure are taken from
%		the BoundingBox of the figure.
%	If "width" is specified, the figure is scaled so that it has
%		the specified width.  Its height changes proportionately.
%	If "height" is specified, the figure is scaled so that it has
%		the specified height.  Its width changes proportionately.
%	If both "width" and "height" are specified, the figure is scaled
%		anamorphically.
%	"bbllx", "bblly", "bburx", and "bbury" control the PostScript
%		BoundingBox.  If these four values are specified
%               *before* the "file" option, the PSFIG will not try to
%               open the PostScript file.
%	"rheight" and "rwidth" are the reserved height and width
%		of the figure, i.e., how big TeX actually thinks
%		the figure is.  They default to "width" and "height".
%	The "clip" option ensures that no portion of the figure will
%		appear outside its BoundingBox.  "clip=" is a switch and
%		takes no value, but the `=' must be present.
%	The "angle" option specifies the angle of rotation (degrees, ccw).
%	The "silent" option makes \psfig work silently.
%

% check to see if macros already loaded in (maybe some other file says
% "\input psfig") ...
\ifx\undefined\psfig\else\endinput\fi

%
% from a suggestion by eijkhout@csrd.uiuc.edu to allow
% loading as a style file. Changed to avoid problems
% with amstex per suggestion by jbence@math.ucla.edu

\let\LaTeXAtSign=\@
\let\@=\relax
\edef\psfigRestoreAt{\catcode`\@=\number\catcode`@\relax}
%\edef\psfigRestoreAt{\catcode`@=\number\catcode`@\relax}
\catcode`\@=11\relax
\newwrite\@unused
\def\ps@typeout#1{{\let\protect\string\immediate\write\@unused{#1}}}
\ps@typeout{psfig/tex \PsfigVersion}

%% Here's how you define your figure path.  Should be set up with null
%% default and a user useable definition.

\def\figurepath{./}
\def\psfigurepath#1{\edef\figurepath{#1}}

%
% @psdo control structure -- similar to Latex @for.
% I redefined these with different names so that psfig can
% be used with TeX as well as LaTeX, and so that it will not 
% be vunerable to future changes in LaTeX's internal
% control structure,
%
\def\@nnil{\@nil}
\def\@empty{}
\def\@psdonoop#1\@@#2#3{}
\def\@psdo#1:=#2\do#3{\edef\@psdotmp{#2}\ifx\@psdotmp\@empty \else
    \expandafter\@psdoloop#2,\@nil,\@nil\@@#1{#3}\fi}
\def\@psdoloop#1,#2,#3\@@#4#5{\def#4{#1}\ifx #4\@nnil \else
       #5\def#4{#2}\ifx #4\@nnil \else#5\@ipsdoloop #3\@@#4{#5}\fi\fi}
\def\@ipsdoloop#1,#2\@@#3#4{\def#3{#1}\ifx #3\@nnil 
       \let\@nextwhile=\@psdonoop \else
      #4\relax\let\@nextwhile=\@ipsdoloop\fi\@nextwhile#2\@@#3{#4}}
\def\@tpsdo#1:=#2\do#3{\xdef\@psdotmp{#2}\ifx\@psdotmp\@empty \else
    \@tpsdoloop#2\@nil\@nil\@@#1{#3}\fi}
\def\@tpsdoloop#1#2\@@#3#4{\def#3{#1}\ifx #3\@nnil 
       \let\@nextwhile=\@psdonoop \else
      #4\relax\let\@nextwhile=\@tpsdoloop\fi\@nextwhile#2\@@#3{#4}}
% 
% \fbox is defined in latex.tex; so if \fbox is undefined, assume that
% we are not in LaTeX.
% Perhaps this could be done better???
\ifx\undefined\fbox
% \fbox code from modified slightly from LaTeX
\newdimen\fboxrule
\newdimen\fboxsep
\newdimen\ps@tempdima
\newbox\ps@tempboxa
\fboxsep = 3pt
\fboxrule = .4pt
\long\def\fbox#1{\leavevmode\setbox\ps@tempboxa\hbox{#1}\ps@tempdima\fboxrule
    \advance\ps@tempdima \fboxsep \advance\ps@tempdima \dp\ps@tempboxa
   \hbox{\lower \ps@tempdima\hbox
  {\vbox{\hrule height \fboxrule
          \hbox{\vrule width \fboxrule \hskip\fboxsep
          \vbox{\vskip\fboxsep \box\ps@tempboxa\vskip\fboxsep}\hskip 
                 \fboxsep\vrule width \fboxrule}
                 \hrule height \fboxrule}}}}
\fi
%
%%%%%%%%%%%%%%%%%%%%%%%%%%%%%%%%%%%%%%%%%%%%%%%%%%%%%%%%%%%%%%%%%%%
% file reading stuff from epsf.tex
%   EPSF.TEX macro file:
%   Written by Tomas Rokicki of Radical Eye Software, 29 Mar 1989.
%   Revised by Don Knuth, 3 Jan 1990.
%   Revised by Tomas Rokicki to accept bounding boxes with no
%      space after the colon, 18 Jul 1990.
%   Portions modified/removed for use in PSFIG package by
%      J. Daniel Smith, 9 October 1990.
%
\newread\ps@stream
\newif\ifnot@eof       % continue looking for the bounding box?
\newif\if@noisy        % report what you're making?
\newif\if@atend        % %%BoundingBox: has (at end) specification
\newif\if@psfile       % does this look like a PostScript file?
%
% PostScript files should start with `%!'
%
{\catcode`\%=12\global\gdef\epsf@start{%!}}
\def\epsf@PS{PS}
%
\def\epsf@getbb#1{%
%
%   The first thing we need to do is to open the
%   PostScript file, if possible.
%
\openin\ps@stream=#1
\ifeof\ps@stream\ps@typeout{Error, File #1 not found}\else
%
%   Okay, we got it. Now we'll scan lines until we find one that doesn't
%   start with %. We're looking for the bounding box comment.
%
   {\not@eoftrue \chardef\other=12
    \def\do##1{\catcode`##1=\other}\dospecials \catcode`\ =10
    \loop
       \if@psfile
	  \read\ps@stream to \epsf@fileline
       \else{
	  \obeyspaces
          \read\ps@stream to \epsf@tmp\global\let\epsf@fileline\epsf@tmp}
       \fi
       \ifeof\ps@stream\not@eoffalse\else
%
%   Check the first line for `%!'.  Issue a warning message if its not
%   there, since the file might not be a PostScript file.
%
       \if@psfile\else
       \expandafter\epsf@test\epsf@fileline:. \\%
       \fi
%
%   We check to see if the first character is a % sign;
%   if so, we look further and stop only if the line begins with
%   `%%BoundingBox:' and the `(atend)' specification was not found.
%   That is, the only way to stop is when the end of file is reached,
%   or a `%%BoundingBox: llx lly urx ury' line is found.
%
          \expandafter\epsf@aux\epsf@fileline:. \\%
       \fi
   \ifnot@eof\repeat
   }\closein\ps@stream\fi}%
%
% This tests if the file we are reading looks like a PostScript file.
%
\long\def\epsf@test#1#2#3:#4\\{\def\epsf@testit{#1#2}
			\ifx\epsf@testit\epsf@start\else
\ps@typeout{Warning! File does not start with `\epsf@start'.  It may not be a PostScript file.}
			\fi
			\@psfiletrue} % don't test after 1st line
%
%   We still need to define the tricky \epsf@aux macro. This requires
%   a couple of magic constants for comparison purposes.
%
{\catcode`\%=12\global\let\epsf@percent=%\global\def\epsf@bblit{%BoundingBox}}
%
%
%   So we're ready to check for `%BoundingBox:' and to grab the
%   values if they are found.  We continue searching if `(at end)'
%   was found after the `%BoundingBox:'.
%
\long\def\epsf@aux#1#2:#3\\{\ifx#1\epsf@percent
   \def\epsf@testit{#2}\ifx\epsf@testit\epsf@bblit
	\@atendfalse
        \epsf@atend #3 . \\%
	\if@atend	
	   \if@verbose{
		\ps@typeout{psfig: found `(atend)'; continuing search}
	   }\fi
        \else
        \epsf@grab #3 . . . \\%
        \not@eoffalse
        \global\no@bbfalse
        \fi
   \fi\fi}%
%
%   Here we grab the values and stuff them in the appropriate definitions.
%
\def\epsf@grab #1 #2 #3 #4 #5\\{%
   \global\def\epsf@llx{#1}\ifx\epsf@llx\empty
      \epsf@grab #2 #3 #4 #5 .\\\else
   \global\def\epsf@lly{#2}%
   \global\def\epsf@urx{#3}\global\def\epsf@ury{#4}\fi}%
%
% Determine if the stuff following the %%BoundingBox is `(atend)'
% J. Daniel Smith.  Copied from \epsf@grab above.
%
\def\epsf@atendlit{(atend)} 
\def\epsf@atend #1 #2 #3\\{%
   \def\epsf@tmp{#1}\ifx\epsf@tmp\empty
      \epsf@atend #2 #3 .\\\else
   \ifx\epsf@tmp\epsf@atendlit\@atendtrue\fi\fi}


% End of file reading stuff from epsf.tex
%%%%%%%%%%%%%%%%%%%%%%%%%%%%%%%%%%%%%%%%%%%%%%%%%%%%%%%%%%%%%%%%%%%

%%%%%%%%%%%%%%%%%%%%%%%%%%%%%%%%%%%%%%%%%%%%%%%%%%%%%%%%%%%%%%%%%%%
% trigonometry stuff from "trig.tex"
\chardef\psletter = 11 % won't conflict with \begin{letter} now...
\chardef\other = 12

\newif \ifdebug %%% turn me on to see TeX hard at work ...
\newif\ifc@mpute %%% don't need to compute some values
\c@mputetrue % but assume that we do

\let\then = \relax
\def\r@dian{pt }
\let\r@dians = \r@dian
\let\dimensionless@nit = \r@dian
\let\dimensionless@nits = \dimensionless@nit
\def\internal@nit{sp }
\let\internal@nits = \internal@nit
\newif\ifstillc@nverging
\def \Mess@ge #1{\ifdebug \then \message {#1} \fi}

{ %%% Things that need abnormal catcodes %%%
	\catcode `\@ = \psletter
	\gdef \nodimen {\expandafter \n@dimen \the \dimen}
	\gdef \term #1 #2 #3%
	       {\edef \t@ {\the #1}%%% freeze parameter 1 (count, by value)
		\edef \t@@ {\expandafter \n@dimen \the #2\r@dian}%
				   %%% freeze parameter 2 (dimen, by value)
		\t@rm {\t@} {\t@@} {#3}%
	       }
	\gdef \t@rm #1 #2 #3%
	       {{%
		\count 0 = 0
		\dimen 0 = 1 \dimensionless@nit
		\dimen 2 = #2\relax
		\Mess@ge {Calculating term #1 of \nodimen 2}%
		\loop
		\ifnum	\count 0 < #1
		\then	\advance \count 0 by 1
			\Mess@ge {Iteration \the \count 0 \space}%
			\Multiply \dimen 0 by {\dimen 2}%
			\Mess@ge {After multiplication, term = \nodimen 0}%
			\Divide \dimen 0 by {\count 0}%
			\Mess@ge {After division, term = \nodimen 0}%
		\repeat
		\Mess@ge {Final value for term #1 of 
				\nodimen 2 \space is \nodimen 0}%
		\xdef \Term {#3 = \nodimen 0 \r@dians}%
		\aftergroup \Term
	       }}
	\catcode `\p = \other
	\catcode `\t = \other
	\gdef \n@dimen #1pt{#1} %%% throw away the ``pt''
}

\def \Divide #1by #2{\divide #1 by #2} %%% just a synonym

\def \Multiply #1by #2%%% allows division of a dimen by a dimen
       {{%%% should really freeze parameter 2 (dimen, passed by value)
	\count 0 = #1\relax
	\count 2 = #2\relax
	\count 4 = 65536
	\Mess@ge {Before scaling, count 0 = \the \count 0 \space and
			count 2 = \the \count 2}%
	\ifnum	\count 0 > 32767 %%% do our best to avoid overflow
	\then	\divide \count 0 by 4
		\divide \count 4 by 4
	\else	\ifnum	\count 0 < -32767
		\then	\divide \count 0 by 4
			\divide \count 4 by 4
		\else
		\fi
	\fi
	\ifnum	\count 2 > 32767 %%% while retaining reasonable accuracy
	\then	\divide \count 2 by 4
		\divide \count 4 by 4
	\else	\ifnum	\count 2 < -32767
		\then	\divide \count 2 by 4
			\divide \count 4 by 4
		\else
		\fi
	\fi
	\multiply \count 0 by \count 2
	\divide \count 0 by \count 4
	\xdef \product {#1 = \the \count 0 \internal@nits}%
	\aftergroup \product
       }}

\def\r@duce{\ifdim\dimen0 > 90\r@dian \then   % sin(x+90) = sin(180-x)
		\multiply\dimen0 by -1
		\advance\dimen0 by 180\r@dian
		\r@duce
	    \else \ifdim\dimen0 < -90\r@dian \then  % sin(-x) = sin(360+x)
		\advance\dimen0 by 360\r@dian
		\r@duce
		\fi
	    \fi}

\def\Sine#1%
       {{%
	\dimen 0 = #1 \r@dian
	\r@duce
	\ifdim\dimen0 = -90\r@dian \then
	   \dimen4 = -1\r@dian
	   \c@mputefalse
	\fi
	\ifdim\dimen0 = 90\r@dian \then
	   \dimen4 = 1\r@dian
	   \c@mputefalse
	\fi
	\ifdim\dimen0 = 0\r@dian \then
	   \dimen4 = 0\r@dian
	   \c@mputefalse
	\fi
%
	\ifc@mpute \then
        	% convert degrees to radians
		\divide\dimen0 by 180
		\dimen0=3.141592654\dimen0
%
		\dimen 2 = 3.1415926535897963\r@dian %%% a well-known constant
		\divide\dimen 2 by 2 %%% we only deal with -pi/2 : pi/2
		\Mess@ge {Sin: calculating Sin of \nodimen 0}%
		\count 0 = 1 %%% see power-series expansion for sine
		\dimen 2 = 1 \r@dian %%% ditto
		\dimen 4 = 0 \r@dian %%% ditto
		\loop
			\ifnum	\dimen 2 = 0 %%% then we've done
			\then	\stillc@nvergingfalse 
			\else	\stillc@nvergingtrue
			\fi
			\ifstillc@nverging %%% then calculate next term
			\then	\term {\count 0} {\dimen 0} {\dimen 2}%
				\advance \count 0 by 2
				\count 2 = \count 0
				\divide \count 2 by 2
				\ifodd	\count 2 %%% signs alternate
				\then	\advance \dimen 4 by \dimen 2
				\else	\advance \dimen 4 by -\dimen 2
				\fi
		\repeat
	\fi		
			\xdef \sine {\nodimen 4}%
       }}

% Now the Cosine can be calculated easily by calling \Sine
\def\Cosine#1{\ifx\sine\UnDefined\edef\Savesine{\relax}\else
		             \edef\Savesine{\sine}\fi
	{\dimen0=#1\r@dian\advance\dimen0 by 90\r@dian
	 \Sine{\nodimen 0}
	 \xdef\cosine{\sine}
	 \xdef\sine{\Savesine}}}	      
% end of trig stuff
%%%%%%%%%%%%%%%%%%%%%%%%%%%%%%%%%%%%%%%%%%%%%%%%%%%%%%%%%%%%%%%%%%%%

\def\psdraft{
	\def\@psdraft{0}
	%\ps@typeout{draft level now is \@psdraft \space . }
}
\def\psfull{
	\def\@psdraft{100}
	%\ps@typeout{draft level now is \@psdraft \space . }
}

\psfull

\newif\if@scalefirst
\def\psscalefirst{\@scalefirsttrue}
\def\psrotatefirst{\@scalefirstfalse}
\psrotatefirst

\newif\if@draftbox
\def\psnodraftbox{
	\@draftboxfalse
}
\def\psdraftbox{
	\@draftboxtrue
}
\@draftboxtrue

\newif\if@prologfile
\newif\if@postlogfile
\def\pssilent{
	\@noisyfalse
}
\def\psnoisy{
	\@noisytrue
}
\psnoisy
%%% These are for the option list.
%%% A specification of the form a = b maps to calling \@p@@sa{b}
\newif\if@bbllx
\newif\if@bblly
\newif\if@bburx
\newif\if@bbury
\newif\if@height
\newif\if@width
\newif\if@rheight
\newif\if@rwidth
\newif\if@angle
\newif\if@clip
\newif\if@verbose
\def\@p@@sclip#1{\@cliptrue}


\newif\if@decmpr

%%% GDH 7/26/87 -- changed so that it first looks in the local directory,
%%% then in a specified global directory for the ps file.
%%% RPR 6/25/91 -- changed so that it defaults to user-supplied name if
%%% boundingbox info is specified, assuming graphic will be created by
%%% print time.
%%% TJD 10/19/91 -- added bbfile vs. file distinction, and @decmpr flag

\def\@p@@sfigure#1{\def\@p@sfile{null}\def\@p@sbbfile{null}
	        \openin1=#1.bb
		\ifeof1\closein1
	        	\openin1=\figurepath#1.bb
			\ifeof1\closein1
			        \openin1=#1
				\ifeof1\closein1%
				       \openin1=\figurepath#1
					\ifeof1
					   \ps@typeout{Error, File #1 not found}
						\if@bbllx\if@bblly
				   		\if@bburx\if@bbury
			      				\def\@p@sfile{#1}%
			      				\def\@p@sbbfile{#1}%
							\@decmprfalse
				  	   	\fi\fi\fi\fi
					\else\closein1
				    		\def\@p@sfile{\figurepath#1}%
				    		\def\@p@sbbfile{\figurepath#1}%
						\@decmprfalse
	                       		\fi%
			 	\else\closein1%
					\def\@p@sfile{#1}
					\def\@p@sbbfile{#1}
					\@decmprfalse
			 	\fi
			\else
				\def\@p@sfile{\figurepath#1}
				\def\@p@sbbfile{\figurepath#1.bb}
				\@decmprtrue
			\fi
		\else
			\def\@p@sfile{#1}
			\def\@p@sbbfile{#1.bb}
			\@decmprtrue
		\fi}

\def\@p@@sfile#1{\@p@@sfigure{#1}}

\def\@p@@sbbllx#1{
		%\ps@typeout{bbllx is #1}
		\@bbllxtrue
		\dimen100=#1
		\edef\@p@sbbllx{\number\dimen100}
}
\def\@p@@sbblly#1{
		%\ps@typeout{bblly is #1}
		\@bbllytrue
		\dimen100=#1
		\edef\@p@sbblly{\number\dimen100}
}
\def\@p@@sbburx#1{
		%\ps@typeout{bburx is #1}
		\@bburxtrue
		\dimen100=#1
		\edef\@p@sbburx{\number\dimen100}
}
\def\@p@@sbbury#1{
		%\ps@typeout{bbury is #1}
		\@bburytrue
		\dimen100=#1
		\edef\@p@sbbury{\number\dimen100}
}
\def\@p@@sheight#1{
		\@heighttrue
		\dimen100=#1
   		\edef\@p@sheight{\number\dimen100}
		%\ps@typeout{Height is \@p@sheight}
}
\def\@p@@swidth#1{
		%\ps@typeout{Width is #1}
		\@widthtrue
		\dimen100=#1
		\edef\@p@swidth{\number\dimen100}
}
\def\@p@@srheight#1{
		%\ps@typeout{Reserved height is #1}
		\@rheighttrue
		\dimen100=#1
		\edef\@p@srheight{\number\dimen100}
}
\def\@p@@srwidth#1{
		%\ps@typeout{Reserved width is #1}
		\@rwidthtrue
		\dimen100=#1
		\edef\@p@srwidth{\number\dimen100}
}
\def\@p@@sangle#1{
		%\ps@typeout{Rotation is #1}
		\@angletrue
%		\dimen100=#1
		\edef\@p@sangle{#1} %\number\dimen100}
}
\def\@p@@ssilent#1{ 
		\@verbosefalse
}
\def\@p@@sprolog#1{\@prologfiletrue\def\@prologfileval{#1}}
\def\@p@@spostlog#1{\@postlogfiletrue\def\@postlogfileval{#1}}
\def\@cs@name#1{\csname #1\endcsname}
\def\@setparms#1=#2,{\@cs@name{@p@@s#1}{#2}}
%
% initialize the defaults (size the size of the figure)
%
\def\ps@init@parms{
		\@bbllxfalse \@bbllyfalse
		\@bburxfalse \@bburyfalse
		\@heightfalse \@widthfalse
		\@rheightfalse \@rwidthfalse
		\def\@p@sbbllx{}\def\@p@sbblly{}
		\def\@p@sbburx{}\def\@p@sbbury{}
		\def\@p@sheight{}\def\@p@swidth{}
		\def\@p@srheight{}\def\@p@srwidth{}
		\def\@p@sangle{0}
		\def\@p@sfile{} \def\@p@sbbfile{}
		\def\@p@scost{10}
		\def\@sc{}
		\@prologfilefalse
		\@postlogfilefalse
		\@clipfalse
		\if@noisy
			\@verbosetrue
		\else
			\@verbosefalse
		\fi
}
%
% Go through the options setting things up.
%
\def\parse@ps@parms#1{
	 	\@psdo\@psfiga:=#1\do
		   {\expandafter\@setparms\@psfiga,}}
%
% Compute bb height and width
%
\newif\ifno@bb
\def\bb@missing{
	\if@verbose{
		\ps@typeout{psfig: searching \@p@sbbfile \space  for bounding box}
	}\fi
	\no@bbtrue
	\epsf@getbb{\@p@sbbfile}
        \ifno@bb \else \bb@cull\epsf@llx\epsf@lly\epsf@urx\epsf@ury\fi
}	
\def\bb@cull#1#2#3#4{
	\dimen100=#1 bp\edef\@p@sbbllx{\number\dimen100}
	\dimen100=#2 bp\edef\@p@sbblly{\number\dimen100}
	\dimen100=#3 bp\edef\@p@sbburx{\number\dimen100}
	\dimen100=#4 bp\edef\@p@sbbury{\number\dimen100}
	\no@bbfalse
}
% rotate point (#1,#2) about (0,0).
% The sine and cosine of the angle are already stored in \sine and
% \cosine.  The result is placed in (\p@intvaluex, \p@intvaluey).
\newdimen\p@intvaluex
\newdimen\p@intvaluey
\def\rotate@#1#2{{\dimen0=#1 sp\dimen1=#2 sp
%            	calculate x' = x \cos\theta - y \sin\theta
		  \global\p@intvaluex=\cosine\dimen0
		  \dimen3=\sine\dimen1
		  \global\advance\p@intvaluex by -\dimen3
% 		calculate y' = x \sin\theta + y \cos\theta
		  \global\p@intvaluey=\sine\dimen0
		  \dimen3=\cosine\dimen1
		  \global\advance\p@intvaluey by \dimen3
		  }}
\def\compute@bb{
		\no@bbfalse
		\if@bbllx \else \no@bbtrue \fi
		\if@bblly \else \no@bbtrue \fi
		\if@bburx \else \no@bbtrue \fi
		\if@bbury \else \no@bbtrue \fi
		\ifno@bb \bb@missing \fi
		\ifno@bb \ps@typeout{FATAL ERROR: no bb supplied or found}
			\no-bb-error
		\fi
		%
%\ps@typeout{BB: \@p@sbbllx, \@p@sbblly, \@p@sbburx, \@p@sbbury} 
%
% store height/width of original (unrotated) bounding box
		\count203=\@p@sbburx
		\count204=\@p@sbbury
		\advance\count203 by -\@p@sbbllx
		\advance\count204 by -\@p@sbblly
		\edef\ps@bbw{\number\count203}
		\edef\ps@bbh{\number\count204}
		%\ps@typeout{ psbbh = \ps@bbh, psbbw = \ps@bbw }
		\if@angle 
			\Sine{\@p@sangle}\Cosine{\@p@sangle}
	        	{\dimen100=\maxdimen\xdef\r@p@sbbllx{\number\dimen100}
					    \xdef\r@p@sbblly{\number\dimen100}
			                    \xdef\r@p@sbburx{-\number\dimen100}
					    \xdef\r@p@sbbury{-\number\dimen100}}
%
% Need to rotate all four points and take the X-Y extremes of the new
% points as the new bounding box.
                        \def\minmaxtest{
			   \ifnum\number\p@intvaluex<\r@p@sbbllx
			      \xdef\r@p@sbbllx{\number\p@intvaluex}\fi
			   \ifnum\number\p@intvaluex>\r@p@sbburx
			      \xdef\r@p@sbburx{\number\p@intvaluex}\fi
			   \ifnum\number\p@intvaluey<\r@p@sbblly
			      \xdef\r@p@sbblly{\number\p@intvaluey}\fi
			   \ifnum\number\p@intvaluey>\r@p@sbbury
			      \xdef\r@p@sbbury{\number\p@intvaluey}\fi
			   }
%			lower left
			\rotate@{\@p@sbbllx}{\@p@sbblly}
			\minmaxtest
%			upper left
			\rotate@{\@p@sbbllx}{\@p@sbbury}
			\minmaxtest
%			lower right
			\rotate@{\@p@sbburx}{\@p@sbblly}
			\minmaxtest
%			upper right
			\rotate@{\@p@sbburx}{\@p@sbbury}
			\minmaxtest
			\edef\@p@sbbllx{\r@p@sbbllx}\edef\@p@sbblly{\r@p@sbblly}
			\edef\@p@sbburx{\r@p@sbburx}\edef\@p@sbbury{\r@p@sbbury}
%\ps@typeout{rotated BB: \r@p@sbbllx, \r@p@sbblly, \r@p@sbburx, \r@p@sbbury}
		\fi
		\count203=\@p@sbburx
		\count204=\@p@sbbury
		\advance\count203 by -\@p@sbbllx
		\advance\count204 by -\@p@sbblly
		\edef\@bbw{\number\count203}
		\edef\@bbh{\number\count204}
		%\ps@typeout{ bbh = \@bbh, bbw = \@bbw }
}
%
% \in@hundreds performs #1 * (#2 / #3) correct to the hundreds,
%	then leaves the result in @result
%
\def\in@hundreds#1#2#3{\count240=#2 \count241=#3
		     \count100=\count240	% 100 is first digit #2/#3
		     \divide\count100 by \count241
		     \count101=\count100
		     \multiply\count101 by \count241
		     \advance\count240 by -\count101
		     \multiply\count240 by 10
		     \count101=\count240	%101 is second digit of #2/#3
		     \divide\count101 by \count241
		     \count102=\count101
		     \multiply\count102 by \count241
		     \advance\count240 by -\count102
		     \multiply\count240 by 10
		     \count102=\count240	% 102 is the third digit
		     \divide\count102 by \count241
		     \count200=#1\count205=0
		     \count201=\count200
			\multiply\count201 by \count100
		 	\advance\count205 by \count201
		     \count201=\count200
			\divide\count201 by 10
			\multiply\count201 by \count101
			\advance\count205 by \count201
			%
		     \count201=\count200
			\divide\count201 by 100
			\multiply\count201 by \count102
			\advance\count205 by \count201
			%
		     \edef\@result{\number\count205}
}
\def\compute@wfromh{
		% computing : width = height * (bbw / bbh)
		\in@hundreds{\@p@sheight}{\@bbw}{\@bbh}
		%\ps@typeout{ \@p@sheight * \@bbw / \@bbh, = \@result }
		\edef\@p@swidth{\@result}
		%\ps@typeout{w from h: width is \@p@swidth}
}
\def\compute@hfromw{
		% computing : height = width * (bbh / bbw)
	        \in@hundreds{\@p@swidth}{\@bbh}{\@bbw}
		%\ps@typeout{ \@p@swidth * \@bbh / \@bbw = \@result }
		\edef\@p@sheight{\@result}
		%\ps@typeout{h from w : height is \@p@sheight}
}
\def\compute@handw{
		\if@height 
			\if@width
			\else
				\compute@wfromh
			\fi
		\else 
			\if@width
				\compute@hfromw
			\else
				\edef\@p@sheight{\@bbh}
				\edef\@p@swidth{\@bbw}
			\fi
		\fi
}
\def\compute@resv{
		\if@rheight \else \edef\@p@srheight{\@p@sheight} \fi
		\if@rwidth \else \edef\@p@srwidth{\@p@swidth} \fi
		%\ps@typeout{rheight = \@p@srheight, rwidth = \@p@srwidth}
}
%		
% Compute any missing values
\def\compute@sizes{
	\compute@bb
	\if@scalefirst\if@angle
% at this point the bounding box has been adjsuted correctly for
% rotation.  PSFIG does all of its scaling using \@bbh and \@bbw.  If
% a width= or height= was specified along with \psscalefirst, then the
% width=/height= value needs to be adjusted to match the new (rotated)
% bounding box size (specifed in \@bbw and \@bbh).
%    \ps@bbw       width=
%    -------  =  ---------- 
%    \@bbw       new width=
% so `new width=' = (width= * \@bbw) / \ps@bbw; where \ps@bbw is the
% width of the original (unrotated) bounding box.
	\if@width
	   \in@hundreds{\@p@swidth}{\@bbw}{\ps@bbw}
	   \edef\@p@swidth{\@result}
	\fi
	\if@height
	   \in@hundreds{\@p@sheight}{\@bbh}{\ps@bbh}
	   \edef\@p@sheight{\@result}
	\fi
	\fi\fi
	\compute@handw
	\compute@resv}

%
% \psfig
% usage : \psfig{file=, height=, width=, bbllx=, bblly=, bburx=, bbury=,
%			rheight=, rwidth=, clip=}
%
% "clip=" is a switch and takes no value, but the `=' must be present.
\def\psfig#1{\vbox {
	% do a zero width hard space so that a single
	% \psfig in a centering enviornment will behave nicely
	%{\setbox0=\hbox{\ }\ \hskip-\wd0}
	%
	\ps@init@parms
	\parse@ps@parms{#1}
	\compute@sizes
	%
	\ifnum\@p@scost<\@psdraft{
		%
		\special{ps::[begin] 	\@p@swidth \space \@p@sheight \space
				\@p@sbbllx \space \@p@sbblly \space
				\@p@sbburx \space \@p@sbbury \space
				startTexFig \space }
		\if@angle
			\special {ps:: \@p@sangle \space rotate \space} 
		\fi
		\if@clip{
			\if@verbose{
				\ps@typeout{(clip)}
			}\fi
			\special{ps:: doclip \space }
		}\fi
		\if@prologfile
		    \special{ps: plotfile \@prologfileval \space } \fi
		\if@decmpr{
			\if@verbose{
				\ps@typeout{psfig: including \@p@sfile.Z \space }
			}\fi
			\special{ps: plotfile "`zcat \@p@sfile.Z" \space }
		}\else{
			\if@verbose{
				\ps@typeout{psfig: including \@p@sfile \space }
			}\fi
			\special{ps: plotfile \@p@sfile \space }
		}\fi
		\if@postlogfile
		    \special{ps: plotfile \@postlogfileval \space } \fi
		\special{ps::[end] endTexFig \space }
		% Create the vbox to reserve the space for the figure.
		\vbox to \@p@srheight sp{
		% 1/92 TJD Changed from "true sp" to "sp" for magnification.
			\hbox to \@p@srwidth sp{
				\hss
			}
		\vss
		}
	}\else{
		% draft figure, just reserve the space and print the
		% path name.
		\if@draftbox{		
			% Verbose draft: print file name in box
			\hbox{\frame{\vbox to \@p@srheight sp{
			\vss
			\hbox to \@p@srwidth sp{ \hss \@p@sfile \hss }
			\vss
			}}}
		}\else{
			% Non-verbose draft
			\vbox to \@p@srheight sp{
			\vss
			\hbox to \@p@srwidth sp{\hss}
			\vss
			}
		}\fi	



	}\fi
}}
\psfigRestoreAt
\let\@=\LaTeXAtSign





\newcounter{exercise}
\newcommand{\exercise}[4]
   {\addtocounter{exercise}{1}\par\medskip\noindent{\bf Exercise \theexercise: #1}%
   \hfill{\em Time/Size Averages: #2 min.},{\em #3 LOC}\hfill #4 \par\noindent}

\newcommand{\primersummary}[4]
   {\begin{figure}[h]
    \horizontalline \newline
    \noindent {\em Class Name:} #1 \newline
    \noindent {\em Purpose:} #2 \newline
    \noindent {\em Focal ECS facilities:} #3 \newline
    \noindent {\em Related Primer Material:} #4 \newline
    \horizontalline
    \end{figure}
    }


\begin{document}

\title{{\bf {\rm\small A Publisher's Preview of:}\\Constructing Collaborative Systems:\\{\Large Coordination and Control
in Interactive Client-Server-Agent Applications}}}

\author{Philip Johnson, Ph.D.\\ 
        Director, Collaborative Software Development Laboratory\\
        Assistant Professor, Department of Information and Computer Sciences\\ 
        2565 The Mall\\ 
        University of Hawaii\\ 
        Honolulu, Hawaii 96822\\ 
        (808) 956-3489\\
        {\tt johnson@hawaii.edu}}

\date{}

\maketitle

\section{Why collaborative systems are important}

If the 1980's were the era of the ``personal computer'', the 1990's are the
era of the ``collaborative network.''  In just a few short years, the
Internet and the World Wide Web have begun to fundamentally change the
public's perception of computers: from a mechanism used to automate tasks
for individuals, computers are now becoming perceived as a mechanism for 
communication and coordination of people spread across offices,
towns, countries, and the earth.

Loosely stated, this transformation in the nature of computers is the focus
of the field of computer supported cooperative work (CSCW). Drawing
together practitioners from the fields of psychology, sociology,
anthropology, management information systems, computer science, and other
disciplines, CSCW concerns the impact that computers have upon the work of
groups, as well as the work of groups upon the design and implementation of
computer systems.  

Some primitive collaborative software is already in widespread use today.
Electronic mail and news groups are ubiquitous and extremely successful
examples of collaborative software.  Yet they are, in a real sense, only
the ``first generation languages'' of collaboration: primitiveate gtables" hypothesis is sufficiently
strong to guarantee that the set of all accurate gtables is
necessarily consistent, but it appears like a reasonable start. 

\newpage
\section{Hints for using Gtables}

{\em Editor's note: This is an e-mail from Robert when he first started
using gtables.}

Here is the list of hints I came up with during my struggle with gtables.
Please feel free to add or ask for elaboration on any point.


\begin{enumerate}
  
\item You must put the gtable definition macros in a separate file from
  other source code.

\item You must put the name of the gtable definition file in both the
  Defsys ``module" field and the ``module-macros" field. It should be
  placed very early in the list of ``module" files so that it is always
  loaded before any code that calls it is loaded.
  
\item The gagent must have the gtable definition available to it when it
  loads. To do this you must put a {\tt (load "mygtable.el")} line in your
  ``run-agent" script.

\item You must define your own recovery function for the gtable. 
  
\item Additions to the a gtable are made instantly to a local table.
  Therefore to test your gtable you need to put a key-data pair into the
  gtable, disconnect, and then reconnect and try to get that data back out.
\end{enumerate}


\end{document}

