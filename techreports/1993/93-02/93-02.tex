%%%%%%%%%%%%%%%%%%%%%%%%%%%%%% -*- Mode: Latex -*- %%%%%%%%%%%%%%%%%%%%%%%%%%%%
%% 93-02.tex -- 
%% Author          : Philip M Johnson
%% Created On      : Thu Nov  7 09:52:47 1991
%% Last Modified By: Philip Johnson
%% Last Modified On: Fri Apr  2 08:10:54 1993
%%%%%%%%%%%%%%%%%%%%%%%%%%%%%%%%%%%%%%%%%%%%%%%%%%%%%%%%%%%%%%%%%%%%%%%%%%%%%%%

\documentstyle [12pt,
                /group/csdl/tex/named-citations,
                /group/csdl/tex/definemargins,
                /group/csdl/tex/lmacros]{article}

\begin{document}

\vspace*{1in}

\begin{center}
{\Large\bf EGRET Design Overview} \bigskip\par

\silentfootnote{Support for this research was provided in part by the
National Science Foundation Research Initiation Award CCR-9110861 and
the University of Hawaii Research Council Seed Money Awards
R-91-867-F-728-B-270 and F-92-868-F-728-B-430.}

Philip Johnson\\
Dadong Wan\\
Danu Tjahjono\\
Kiran Kavoori\\
Robert Brewer                    \bigskip\par

Collaborative Software Development Laboratory\\
Department of Information and Computer Sciences\\
University of Hawaii\\
Honolulu, HI 96822\\
(808) 956-3489\\
{\tt johnson@hawaii.edu}           \bigskip\par

{\bf CSDL-TR-93-02}                \bigskip\par

Last Revised: \today

\end{center}

\newpage

\ls{1.2}

\section{Introduction}

This document presents an general overview of the software engineering
principles underlying the Egret system and its applications. It is an
introduction and a companion to the more detailed Design Specification
document for Egret which is generated by the DSB system.

Egret is an environment for the implementation of exploratory
collaborative hypertext applications \cite{csdl-91-03,csdl-92-01}.
Egret provides mechanisms for the integration of an underlying
database server application process, and an intermediate, highly
expressive exploratory type system, with a top-level editor and
graphical display applications.  In doing so, it provides a language
and environment for implementation of domain-specific collaborative
systems, such as for collaborative learning
\cite{csdl-92-03,csdl-92-05,csdl-93-01}, or code inspection
\cite{csdl-92-04,csdl-92-07}.  This document does not attempt to
provide a detailed description of the research issues motivating
Egret. For this kind of background reading, we point the reader to any
of the above references, although \cite{csdl-92-01} and
\cite{csdl-92-03} are good starting points.  If you have not read at
least one of these two articles, we strongly suggest you do so before
wading into the following.

The next section presents some of the architectural requirements for
Egret that have led to the design philosophy and method for this
system. The following section introduces the notational conventions
used to name and organize system entities within Egret.  The final 
section explains the purpose for these notations by detailing the
style of object oriented design reified in Egret.  

\section{Architectural Requirements}

Egret is heavily influenced by our experiences with CoReView, a
prototype collaborative system we developed in 1991.  The successes
and failures of that system stressed to us the importance of
extensibility, verifiability, and recoverability for the successor to
CoReView.  In designing this successor, we balanced two research
goals.  On the one hand, Egret was designed to explore issues in
exploratory collaborative type systems.  On the other hand, Egret was
also designed to provide an excellent platform for development of
novel applications in collaborative hypertext systems.  The next few
sections briefly overview these issues that significantly influenced
the development process and products of Egret.

\subsection{Extensibility}  

Egret has an architecture-level specification that defines a set of
responsibilities and a public interface for the four subsystems:
Utilities, Server, Type, and Generic-Interface.  Each of these subsystems is
internally organized as a set of classes with public and private
interfaces.  Egret also defines naming conventions for program
entities that simplify the identification and understanding of their
behavior.  These conventions are documented in Section
\ref{conventions}.

One goal of this architectural structure is to facilitate migration of
system functionality across different application processes.  For
example, certain database functions that are now implemented in Emacs
might migrate to the database server, if we find a better server
platform.  Our current choices for editor, window interface, or
graphical browser may change over time.  Another architectural goal is to
facilitate the the implementation of new applications, such as
mechanisms for sophisticated impact analysis.  

The most important form of extensibility supported by Egret is
application-level specialization.  Egret supports customization to a
variety of different domains by implemented new subsystems on top of
three kernal subsystems.  The dynamic tension created by simultaneous
continuing development on Egret and different applications sometimes
tests our group process and cohesiveness, but has resulted in rapid
evolution of Egret toward reusability and robustness. Successful
evolution of this nature does not come automatically, of course, as
the next section details.


\subsection{Verification} 

A design goal for Egret is completely automated regression testing: a
suite of test cases that can be run on the system to test whether or
not major functionality is present and correct after making changes.

Egret supports this goal by providing a layered architecture that
separates the Interface subsystem, where user interaction with the
system occurs, with other subsystems, where the structure of the
collaborative network is physically manipulated.  It is possible to
test much of Egret by ``simulating'' the Interface subsystem
interactions programmatically.  The layered architecture supports
incremental integration.

The class-oriented internal structure of Egret supports unit testing,
by providing cohesive, relatively stand-alone objects whose semantics
can be inspected and verified independently from the remainder of the
system.

Regression testing is supported through the TestBase, a system for
declaratively specifying test cases and their expected results.

A final form of verification for Egret is review and inspection, an
ongoing process supported by CSRS, one of the applications implemented
in terms of Egret itself.

\subsection{Information Replication and Recoverability}

A fundamental architectural feature of Egret is replication of  information. 
Information replication occurs through two orthogonal (and combinable)
constructs:  tables and locally cached nodes.  We first discuss these
two constructs separately, then show their relationship, and finally
discuss the need for recoverability implied by these constructs.

Egret maintains many table structures to facilitate queries about the
graph-level structure, type-level properties, and process state of the
hypertext database and collaborative environment.  For example, Egret
maintains a table that associates each node with (among other things)
the links that point to it.  Without this table, a query to the
underlying database server to determine the links pointing to a node
would require searching through every node in the database
individually\foot{Since the hyperbase system does not maintain
backlink information.} While such a table provides efficient answers
to queries of this form, it does by incurring several costs.  First,
this table must be updated whenever the link-level structure of the
database changes, and so the overhead of table update is added to node
and link creation and deletion operations.  Second, this table
replicates information already present implicitly in the structure of
the hypertext database, and so maintaining consistency between it and
the database is vital.  Third, the database must store this table
someplace, and thus extra space must be allocated.

An orthogonal form of information replication is locally cached nodes.
Locally cached nodes are motivated by the need to provide efficient
access to the same node on a frequent basis.  For example, under
normal circumstances, a request by the user to access the contents of
an Egret node results in a query to the hyperbase server process over
the network, which sends back the contents of the node.  Since other
concurrently connected users might also be accessing and updating this
same node's contents, a later request for this node's contents would
require a second hyperbase query, network access, and network
retrieval.  In the case where the node contents are not updated
frequently but required frequently, high overhead can result from this
procedure.  Locally cached nodes are an efficient solution when the
contents of a node changes relatively slowly but is accessed by users
relatively frequently.  Instead of accessing the hyperbase each time
the node's contents are required, a copy of the node's contents are
maintained by the client in their local process, and events are used
to signal to the client process when this node's contents have been
updated.  Until the client receives such an event indicating that the
hyperbase's version of the node has been updated, it simply refers to
its locally cached copy of the node.  The tradeoffs of this approach
are straight-forward: the overhead of querying the hyperbase each time
the node contents are needed is traded for the overhead of processing
the update events.

Egret frequently combines these two concepts by maintaining tables in
locally cached nodes.  Tables are frequently a natural fit to the
strengths of locally cached nodes, since they contain information that
is updated relatively infrequently but is referred to quite
frequently.

Information replication directly leads to the problem of consistency
maintenance.  Whenever information is represented redundantly, either
through tables that replicate information implicit in the structure or
contents of the database, or through locally cached nodes that
replicate the contents of the central server process, the occurrence
of corruption is not only possible, but inevitable.  For this reason,
Egret requires {\bf recovery operations} to be associated with certain
classes.  Recovery operations, when invoked, restore the state of the
object.  For example, if the node storing the table that associates
nodes with the links that point to them became corrupted, the recovery
operation for that table could be invoked to rebuild the table
contents by mapping through the entire contents of the database.
Recovery operations for locally cached nodes simply consist of
re-reading the node contents from the centralized database server
process.


It is important to note that consulting local information maintained
via the event mechanism leads to the possibility of presenting
erroneous information to a client.  As a simple example, suppose one
client queries Egret for the total number of nodes in the database
just after another client deletes a node. If the local client process
determines the answer to this query by computing the length of the
local list of node-IDs, and if this computation is performed before
the event updating this list is received, then the query result will
be off by one.  Therefore, local information should not be used when
this event propogation delay could produce significant problems (for
example, a locally maintained list of locked nodes should not be
consulted in order to determine whether to grant a lock.)

\subsection{Support for Exploratory Groupwork}

Support for exploratory groupwork in Egret begins with its mechanism for
defining type-level structure, called the Exploratory Collaboration Type
System (ECTS), and a process model called EXCON, for 
Exploration-Consolidation Cycle.

The EXCON model views exploratory collaboration as cycling between two
modes: exploration and consolidation.  During exploration, collaborators are
discovering new structural features of the domain and reifying these features
into the structure of the artifact under construction.  During consolidation,
collaborators compare and contrast the structural features they have
individually discovered during exploration, and build consensual structures
that embody the mutually agreed upon features.

ECTS is designed to support these features of the exploration-consolidation
cycle.  In ECTS, collaborators can define typed schemas of nodes and links, and
specify their internal structure.  Once defined, schemas form templates from
which collaborators create instances with common structure and behavior.

While this aspect of ECTS is completely similar to conventional
object-oriented database schema mechanisms, ECTS departs from this model
in a singular way: while classes can be used to create instances with shared
structure and behavior, instances are not constrained to the structure and
behavior as defined by their parent class. Collaborators are free to modify
instances by adding to, deleting, or modifying the properties of its internal
structure.

To support this process, ECTS also provides a structural variance tool
that indicates, for a specific node or link class, the degree to which
its instances depart from the class-level structure, and summarizes
how that variance is expressed in terms of added fields, deleted
fields, and changes to the properties of the class-level defined
fields.  This information can be aggregated over the entire database
to determine an overall measure of variance for the database.

See \cite{csdl-92-01} for more details on ECTS and the EXCON process
model.

\section{Notational Conventions}
\label{conventions}

\subsection{Identifier Syntax}

Function and variable names in Egret have a standard format.
Each name must adhere to the following template:

\small\begin{verbatim}
<sys-name>:<sys-vis><class-name><class-vis><name>
\end{verbatim}\normalsize

\noindent where:
\begin{itemize}
  
\item {\tt <sys-name>} is a single character that identifies the
  subsystem membership of the object.  Currently, {\bf u} identifies
  the utilities subsystem, {\bf gi} identifies the generic-interface
  subsystem, {\bf t} identifies the type subsystem, and {\bf s}
  identifies the server subsystem.
  
\item {\tt <class-name>} is typically the full class name, but may
sometimes be an abbreviation.
  
\item {\tt <sys-vis>} and {\tt <class-vis>} are single characters
  indicating the external visibility of the object. The character
  {\bf *} indicates that the object is public, the character {\bf !}
  indicates that the object is private, and the character {\bf @}
  indicates that the object is a system administration function to be
  manipulated only by distinguished users at special times (such as
  database initialization, recovery, and so forth).  

  \item {\tt <name>} is the actual name of the operation or
  attribute.

\end{itemize}

For example, the function {\bf t*node*set-name} is the type subsystem
public operation {\sf set-name} from the class
{\sf node}.  These prefixes both serve to document the class and
subsystem membership of operations, and to create separate namespaces.
For example, the server subsystem also defines a node class with a
set-name operation, but the two operations have distinct names:
{\bf s*node*set-name}.

In this document, boldface font is used when referring to a specific Egret 
operation or attribute, such as {\bf s*hbserver!connect}.  Sans serif 
font is used when referring to class names or operation names in general, 
such as {\sf node} or {\sf delete}.  The typewriter font is used when
referring to Emacs or Epoch functions, such as {\tt title}, or when 
displaying sections of code that may include Egret and Emacs functions, such
as:
\small\begin{verbatim}
(foobar (i*screen*name scrn) "New Screen")
\end{verbatim}\normalsize

\subsection{Naming Conventions}

Beyond naming syntax, Egret also has conventions for naming certain
kinds of semantically related operations.  These conventions may be
repeated elsewhere in this document, but are collected here for
handy reference:

\begin{itemize}

\item Constructor and instantiation operations are prefixed by {\bf make-}.

\item Deletion operations are prefixed by {\bf delete-}.

\item Operations that set the value of an attribute are prefixed by {\bf set-}.

\item Recovery operation names are by convention prefixed with {\bf reset-},
and whenever possible suffixed with the name of the attribute or
operation whose functioning they repair.  For example, the server
subsystem node class attribute {\bf s*node*incoming-links}
would have a corresponding recovery operation called {\bf s*node!reset-incoming-links}.

\end{itemize}

\section{Object Oriented Design Conventions}

The Egret design is object-oriented.  However, ``object oriented'' can
mean very different things to different people.  This section
clarifies our use of object orientation by describing the major
entities in our design: classes, objects, attributes, operations, and
collections.  It also describes how public and private interfaces for
class attributes and operations are defined\foot{The mechanisms for
inheritance and visibility control do not attempt to extend the state
of the art: inheritance is implemented ``manually''.  The goal here is
to introduce a set of concepts for object orientation and visibility
that provide the greatest benefit for the lowest overhead.}.


\subsection{Classes}  

A {\em class}\/ is a collection of objects with related structure and
behavior.  {\sf Node} and {\sf link} are obvious classes in Egret . Each class
has associated with it a set of superclasses, a set of subclasses, a set of
attributes, and a set of operations.  The design of Egret  is  neutral on the
issue of single vs.  multiple inheritance, although no occurrences of multiple
inheritance in the design currently exist.

Classes frequently have constructor and destructor operations for the
objects associated with them.  These operations are always called {\sf
make} and {\sf delete}. If these operations are not specified, then
the class is an {\em abstract class}. Abstract classes exist to
collect together related structure that is inherited by other classes
that do have constructor and destructor operations.

Classes are, in some sense, purely a design-level representation. No
explicit Emacs Lisp support for defining classes exists.  The
reification of classes in Emacs Lisp occurs purely through the use of
naming conventions for functions and variables, and voluntary
observance by programmers of the visibility constraints.  However,
designing and programming in an object oriented fashion will greatly
ease understanding of the system architecture, as well as supporting
migration to other languages providing direct support for these
concepts.

\subsection{Objects} 

Classes are an organizational entity--what actually exists during
execution are particular instances of classes, or {\em objects}.

Each object has a unique identifier associated with it, and this
unique identifier (or the object itself) is virtually always the first
argument to the operations associated with a class.  This convention
implements a simple kind of message passing behavior, where the first
argument to a function call indicates the particular object that the
message (the function) is sent to.  For example, the {\bf t*node*lock}
operation takes a node-ID as its first argument, which can be
interpreted as having the effect of sending a message to that
particular node to lock itself.

A rough design heuristic is that if an entity does not have a
distinguishing unique ID, or if one is never needed to implement its
behavior, then the entity should probably exist as a part of another
class.  Similarly, deciding the primary home for an object operation
can sometimes be resolved by considering the most appropriate first
argument to it.

\subsection{Attributes}  

Each class instance has a set of characteristics, or {\em
attributes}\/ associated with it.  For example, each screen has a
geometry, color, name, and so forth.  These are attributes of the
class.  If a class has an attribute defined for it, then there exists
a function that returns the value of the attribute.  For example, {\bf
t*node*name} returns the {\sf name} attribute for an instance of the
class {\sf node}.

Attributes need not be ``static''.  For example, a reasonable
attribute of the class {\sf node-buffer} could potentially be {\sf
link-under-mouse}---an attribute that returns the link ID of the link
label currently under the mouse cursor (or {\tt nil} if the mouse
cursor is not currently over any link label).

Attribute functions take a single argument, an instance of the class
for which this state value should be retrieved.

Some attributes are {\em setable}.  This means that an operation
corresponding to the attribute is defined to change its value. This
operation is named by prefixing the attribute name with {\bf set-}.
For example, if the {\bf t*node*name} attribute is setable,
then the operation {\bf t*node*set-name} is defined to change
the value of that operation.

Not all attributes should be setable.  For example, it probably
doesn't make sense to have the {\bf node-buffer*link-under-mouse}
attribute mentioned above be setable\foot{Although, you could
conceivably argue for making that attribute setable as a way of
warping the mouse cursor to a particular link label.  But we won't.}.

Attributes can be private or public.  Public attributes are indicated
notationally by the use of {\bf *} as the visibility token; {\bf !} as
the visibility token indicates a private attribute.  If an attribute is
private, then functions outside the implementation of the class should
not access or set the attribute's value.  This, of course, is not
enforced in Emacs Lisp, but is rather a way of helping programmers to 
understand what parts of a subsystem are internal details and can thus
be safely ignored.  If you find yourself as a class implementor needing 
access to a private attribute of another class, you should publicize this
observation immediately.  It may be the case that there is a public mechanism
that you are aware of and should use instead.  Alternatively, it may be
the case that the private attribute should be changed to public.

It is common to have an attribute be publically accessable but only
privately setable.  To do this, define both the {\bf *} and {\bf !}
forms of the attribute, but only define the set operation for the {\bf
!} form. For example, one could define two accessors for screen names,
{\bf gi*screen*name} and {\bf gi*screen!name}, the first public for use
by any class, and the second private to the implementation of the
screen class.  By defining the set operation {\bf gi!screen!set-name},
you provide a consistent mechanism for updating the screen title
internally without making this capability public.

\subsection{Operations}

While attributes refer to characteristics of classes, {\em operations}
refer to the behaviors of the instances of the class.  For example,
one behavior of a node is to be written out or retrieved from the
hyperbase.  These behaviors are represented by the operations
{\sf write} and {\sf retrieve}, with corresponding Emacs functions {\bf
t*node*write} and {\bf t*node*read}.

Operations may take any number of arguments, but the first argument
is, by convention, an object representing the class instance.  In the
case of objects like nodes or links, the first argument is their
unique ID.  In the case of objects like buffers or screens, the first
argument may be the actual buffer or screen object itself.  Operations
must always document what form their first (as well as the other)
argument must take. The operation may take additional required or
optional arguments depending upon the nature of the behavior.


Operations can be public, indicated by the visibility token ``*'',
or private, indicated by ``{\bf !}''.  Public operations are accessable to
other classes, while private operations are only accessable to the
internal implementation of the class. As noted above, this is purely
convention, and unfortunately Emacs Lisp does not enforce this
encapsulation.  To repeat, if you find yourself as a class implementor
needing access to a private operation of another class, you should
publicize this observation immediately.  It may be the case that there
is a public mechanism that you are aware of and should use instead.
Alternatively, it may be the case that the private operation should be
changed to public.

All instantiable (i.e. non-abstract) classes have a constructor and a
destructor operation associated with them.  By convention, these
operations are named {\sf make} and {\sf delete}, and are normally
public.  (It is often useful to implement two constructor operations,
for example {\bf s*node*make} and {\bf s*node!make}.  The first is the
public operation with publically-supplied arguments; the second is the
internal, down-and-dirty constructor that may take many more arguments
or return an internal object that should be hidden from the external
interface.

\subsection{Collections}

Operations, as defined above, operate on individual instances of the
class. There is another kind of operation, however, that operates on
the set of all objects that exist in the class. These are called {\em
collection operations}.  For example, an operation that returns a list
of all nodes of a specified type is a collection operation.  These
operations, taken together, represent many of the classical database
query and retrieval operations supported in Egret .

Collection operations are always defined relative to a class, and are
named by wrapping {\bf \{} and {\bf \}} around the class name.  Unlike
conventional class arguments, collection operations do not have a
distinguished first argument, since their name indicates which
collection the function operates on\foot{Future revisions to this
design may lead to the presence of a first argument that is passed a
collection object, consisting of some subset of objects of the named
type.}.  For example, the collection operation that returns all the
IDs for server-level nodes is {\bf s*\{node\}*IDs}.  This operation
does not take any arguments, since the class that it operates on is
implicit in its name.

Collection operations can be public or private, and their visibility is
indicated in the standard manner by {\bf *} or {\bf !}.

\bibliography{/group/csdl/bib/csdl-trs}
\bibliographystyle{/group/csdl/tex/named-citations}

\end{document}

