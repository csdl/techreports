%%%%%%%%%%%%%%%%%%%%%%%%%%%%% -*- Mode: Latex -*- %%%%%%%%%%%%%%%%%%%%%%%%%%%%
%% talk.tex -- 
%% RCS:            : $Id: demos-presentation.tex,v 1.1 93/04/25 13:18:18 dxw Exp Locker: dxw $
%% Author          : Dadong Wan
%% Created On      : Sun Apr 25 11:59:13 1993
%% Last Modified By: Philip Johnson
%% Last Modified On: Fri Apr 30 08:48:44 1993
%% Status          : Unknown
%%%%%%%%%%%%%%%%%%%%%%%%%%%%%%%%%%%%%%%%%%%%%%%%%%%%%%%%%%%%%%%%%%%%%%%%%%%%%%%
%%   Copyright (C) 1993 University of Hawaii
%%%%%%%%%%%%%%%%%%%%%%%%%%%%%%%%%%%%%%%%%%%%%%%%%%%%%%%%%%%%%%%%%%%%%%%%%%%%%%%
%% 
%% History
%% 25-Apr-1993		Dadong Wan	
%%    created
\documentstyle[12pt,slidesonly]
{/group/csdl/tex/seminar}
%\input{/group/csdl/tex/psfig}
%\special{header=/home/13/csdl/tex/psfig/lprep71.pro}
%\rotateheaderstrue               % Try this out if using rotation macros.
%\articlemag{-1}
\newcommand{\horizontalline} {\rule{\textwidth}{.02in}} 
\slideframe{none}
\slidesmag{0}        % integer value ranging from -5 to 9
\special{landscape}  %comment out this line for notes
\pagestyle{empty}
%\twoup[-2]          %uncomment this line for notes
\begin{document}      



\begin{slide}  \Huge
  \begin{center}
    {\bf CLARE: A Project Overview}
    
    \vspace{0.5in}
    
    by\\
    Dadong Wan\\
    {\sf dxw@uhics.ics.hawaii.edu}
    
    \vspace{0.5in}
    
    Collaborative Software Development Laboratory \\
    Department of Information \& Computer Sciences\\
    University of Hawaii at Manoa
  \end{center}
\end{slide}  


\begin{slide}   \Huge
  {\bf What Is Wrong with ``Virtual Classroom'' \& Hypermedia Systems?}
  \horizontalline
  
  \begin{itemize}
  \item Access/delivery alone isn't sufficient
    \begin{itemize}
    \item information, people, and media
    \end{itemize}
    
  \item Support for meta-learning is missing
    
  \item Technology-driven rather than requirement or theory based
  \end{itemize}
\end{slide}


\begin{slide}   \Huge
  {\bf The CLARE Approach}
  \horizontalline

  \begin{itemize}    
  \item Collaborative learning as active knowledge construction

    \begin{itemize}
    \item Role of knowledge representation, i.e., RESRA
    \end{itemize}

  \item Explicit support for meta-learning
    
  \item Based on Ausubel/Novak/Gowin's theory of meaningful learning

  \end{itemize}
\end{slide}


\begin{slide}   \Huge
  {\bf Current Status}
  \horizontalline
  
  \begin{itemize}
  \item An ``in-house'' tested representation, i.e., RESRA
    
  \item CLARE prototype: v.1.2 
    
  \item Preliminary experimental design
    
  \end{itemize}
\end{slide}



\begin{slide}
  {\bf Future Directions}
  \horizontalline
  
  \begin{itemize}
  \item Beta-test CLARE
    
  \item Extend, fine-tune, and optimize current implementation
    
  \item Refine experimental design 
    
  \item Conduct experiments
  \end{itemize}
\end{slide}


\end{document}








