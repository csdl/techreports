\documentstyle [/group/csdl/tex/named-citations,
                /group/csdl/tex/definemargins,
                /group/csdl/tex/lmacros]{article}

\pagestyle{empty}
\ls{1.0}
\begin{document}      

\subsection*{1. CLARE: A New Approach to Computer-Based Collaborative Learning}
Dadong Wan  ({\tt dxw@uhics.ics.hawaii.edu})

\begin{itemizenoindent}    

\item {\bf What Is Wrong with ``Virtual Classroom'' \& Hypermedia Systems?}\\
 Access/delivery alone isn't enough\\
 No use of meta-knowledge in collaborative support\\
 Technology-driven rather than requirement/theory based\\

\item {\bf The CLARE Approach}\\
 Collaborative learning as knowledge representation\\
 Using meta-knowledge to structure collaboration\\
 Based on Ausubel/Novak/Gowin's cognitive learning theory\\

\item {\bf Current Status}\\
 RESRA: an ``in-house'' tested representation\\
 CLARE prototype: v.1.2.\\
 Preliminary experimental design\\

\item {\bf Future Directions}\\
 ``In-house'' use of CLARE\\
 Refine experimental design and current implementation\\
 Conduct experiments\\

\end{itemizenoindent}

\subsection*{2. URN: A New Way To Think About Usenet}
Robert S. Brewer ({\tt rbrewer@uhics.ics.hawaii.edu})

\begin{itemizenoindent}
\item  {\bf Three Problems of Usenet}\\
 Too much information\\
 Too ephemeral\\
 Representation too limited\\

\item {\bf Three Solutions}\\
 Agents search on users' behalf\\
 Knowledge Condensation\\
 Knowledge Communities\\

\item {\bf Methodology}\\
 Implement system\\
 Obtain data from real-world use\\
 Evaluate data\\

\item {\bf Current Project Status}\\
 Implementation: early prototype underway\\
 Data: None currently available\\
 Evaluation: No data\\
\end{itemizenoindent}

\newpage
\subsection*{3. CSRS: A new approach to Software Review}
Danu Tjahjono ({\tt dat@uhics.ics.hawaii.edu})
\begin{itemizenoindent}
\item {\bf Problems with current review methods}\\
 Lack of computational support\\
 Ill-defined process\\
 Lack of high quality measurements \\
 Lack of support for process improvement\\

\item {\bf The CSRS Approach}\\
 Computer based review (CSRS)\\
 Well defined data and process model\\
 Automatic, high quality metrics collection\\

\item {\bf Current Status}\\
 Completed working prototype of CSRS (version 1.2)\\
 Completed two rounds of review:\\
 Total source materials: 450 LOC\\
 Review effectiveness : 9.4 issues/100LOC\\
 Review rates: 100-250 lines/hour\\

\item {\bf Future Directions}\\
 Improve system response time\\
 Extend to different programming environments, such as C/C++.\\
 Extend to design review and requirements review\\
 Conduct formal experiments to compare various review techniques.\\
\end{itemizenoindent}


\subsection*{4. DSB: The Next Generation Tool for Software Engineers}
Kavoori Kiran Ram ({\tt kavoori@uhics.ics.hawaii.edu})
\begin{itemizenoindent}

\item {\bf Problems in Reverse Engineering }\\
 Typical Software projects are products of team work.\\
 Unless properly documented, code written by others is hard to understand.\\
 Maintaining up-to-date documentation creates overhead.

\item  {\bf The DSB Approach}\\
 Automatically extract design level specifications.\\
 Present the public interface in a meaningful way.\\
 Critique the application to elucidate shortcomings.\\

\item {\bf Current Status of DSB}\\
 Generates a LaTeX document of the design specifications.\\
 Critiques the application.\\
 Successfully used on the following systems:\\
\indent    EGRET: 4 modules, 32 classes and 300 operations.\\
\indent    CLARE: 4 modules, 33 classes and 174 operations.\\
\indent    CSRS: 2 modules, 18 classes and 100 operations.\\
\indent    URN: 3 modules, 5 classes and 31 operations. \\
\indent    DSB: 2 modules, 9 classes and 48 operations.\\

\item {\bf Shortcomings and Future Directions}\\
 DSB can only parse Lisp source files.\\
 (Can be extended to languages like Common Lisp and C++.)\\
 DSB analysis dependent upon CSDL design conventions. \\
 (Can be modified to incorporate other conventions.)

\end{itemizenoindent}

\nocite{csdl-92-01,csdl-92-03,csdl-92-04,csdl-92-07,csdl-93-01,csdl-93-02,csdl-93-03,csdl-93-04,csdl-93-05,csdl-93-06}

\bibliography{/group/csdl/bib/csdl-trs}
\bibliographystyle{/group/csdl/tex/named-citations}

\end{document}
