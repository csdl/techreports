%%%%%%%%%%%%%%%%%%%%%%%%%%%%%% -*- Mode: Latex -*- %%%%%%%%%%%%%%%%%%%%%%%%%%%%
%% cscw94-future.tex -- 
%% RCS:            : $Id: cscw94-future.tex,v 1.5 94/02/08 15:41:51 rbrewer Exp $
%% Author          : Robert Brewer
%% Created On      : Thu Feb  3 10:24:28 1994
%% Last Modified By: Robert Brewer
%% Last Modified On: Tue Feb  8 15:41:26 1994
%% Status          : Unknown
%%%%%%%%%%%%%%%%%%%%%%%%%%%%%%%%%%%%%%%%%%%%%%%%%%%%%%%%%%%%%%%%%%%%%%%%%%%%%%%
%%   Copyright (C) 1994 University of Hawaii
%%%%%%%%%%%%%%%%%%%%%%%%%%%%%%%%%%%%%%%%%%%%%%%%%%%%%%%%%%%%%%%%%%%%%%%%%%%%%%%
%% 
%% History
%% 3-Feb-1994		Robert Brewer	
%%    Created

\section{CONCLUSIONS AND FUTURE DIRECTIONS}
\label{sec:conclusion}

This paper presents URN, a system for collaborative classification and
evaluation of Usenet. It combines a collaboratively built representation
for the keywords associated with an article with an adaptive interface that
prioritizes articles based on votes on previous articles. The paper
presents results of a two week trial of our system, providing quantitative
evidence to support our claim that URN's weighting functions can represent
user's reading interests. Now we discuss our plans for URN in the future.

Our first future direction is to gather more experimental data on URN over
a longer time frame. With such data, we can perform more statistically
interesting analyses of URN usage, and provide higher quality evidence for
the strengths and limitations of this approach.  We intend to continue
experimenting with URN throughout 1994.

Second, users suggested many improvements to the weighting function
mechanism. One suggestion is the ability for URN to provide users with
direct access to their weighting functions. On many occasions users wished
that they could add a weighting function to their profile directly because
they were sure that they were interested or uninterested in a particular
keyword. Users also desired better control over keywords, such as the
ability to create synonyms or select from a menu of keywords.  Finally,
users also desired enhanced collaborative capabilities, such as the ability
to recommend a specific article to another individual.

Third, URN represents only an initial step in our work toward enhanced
information access in Usenet.  Our medium-term goal is to support a process
we call {\em knowledge condensation}, in which information obtained from
Usenet is not simply annotated with keywords in order to support
evaluation, but actively restructured with the collaborative addition of
hypertext links to other information of interest saved from prior postings.
In this way, a group of users can incrementally and collaboratively build
and restructure a richly interlinked knowledge base of information about a
common topic of interest, and share this knowledge base with others via
Usenet.

\section{ACKNOWLEDGMENTS}

We would like to thank the other members of Collaborative Software Development
Lab: Rosemary Andrada, Carleton Moore, Danu Tjahjono, and Dadong Wan for their
assistance in preparing this manuscript as well as in the development of
Egret\cite{csdl-92-01}. Robert Brewer would also like to thank Yuka Nagashima
for her help in reviewing this document.
Support for this research was partially provided by the National Science
Foundation Research Initiation Award CCR-9110861.
