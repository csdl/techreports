%%%%%%%%%%%%%%%%%%%%%%%%%%%%%% -*- Mode: Latex -*- %%%%%%%%%%%%%%%%%%%%%%%%%%%%
%% cscw94-intro.tex -- 
%% RCS:            : $Id: cscw94-intro.tex,v 1.15 94/02/08 15:34:06 rbrewer Exp $
%% Author          : Robert Brewer
%% Created On      : Thu Feb  3 10:20:32 1994
%% Last Modified By: Robert Brewer
%% Last Modified On: Tue Feb  8 15:33:48 1994
%% Status          : Unknown
%%%%%%%%%%%%%%%%%%%%%%%%%%%%%%%%%%%%%%%%%%%%%%%%%%%%%%%%%%%%%%%%%%%%%%%%%%%%%%%
%%   Copyright (C) 1994 University of Hawaii
%%%%%%%%%%%%%%%%%%%%%%%%%%%%%%%%%%%%%%%%%%%%%%%%%%%%%%%%%%%%%%%%%%%%%%%%%%%%%%%
%% 
%% History
%% 3-Feb-1994		Robert Brewer	
%%    Created

\section {INTRODUCTION}
\label{sec:introduction}
  
Information is rapidly becoming the essential currency of the modern world.
With the growth of global connectivity, there is a growing desire to not only
receive information, but to distribute one's own information. In the United
States, there are proposals for a ``National Information Infrastructure''
\cite{nii-agenda} to enable its citizens to participate in the information age.
However, there already exists a tremendously successful network allowing
millions of people to send and receive information on a global scale.  This
network is Usenet.

Usenet is a global collaborative system {\em par excellence}.  Usenet
raises few barriers to participation beyond access to its technological
infrastructure. Low-cost or free access services for home users, as well as
connections to schools and libraries are increasing.  Usenet does not
require computer literacy beyond basic word processing skills, and does not
restrict the content or dissemination of information.  Anyone can make a
posting about any topic, and anyone can read what anyone else has to say
about a topic.  Finally, anyone can create a new topic for discussion.

As a simple, imperfect analogy, imagine the daily papers from every major
English-speaking city in the world arriving instantaneously on one's desktop
each morning. (English is the predominant, but not exclusive language of
Usenet.)  The stack of papers consists of thousands of pages of text,
photographs, and advertisements; local news stories from far away places;
descriptions of and commentary on the same social, political, and/or
technological event from hundreds of different perspectives; thousands of
letters to editors from citizens, each expressing a different point of view on
an issue of concern to them; questions, answers, rebuttals, and so on.  Such
information access would have a double-edged appeal: beyond access to new,
helpful information on subjects for which one already has an interest, it would
stimulate {\em new} interests by access to new subject areas, new events, new
technology, and new cultural perspectives.  It would be a potent force of basic
human education and enrichment.

On the other hand, it would also be overwhelming.  The sheer volume of news
arriving every day would prevent even a cursory skimming in its entirety.
Searching for information would be extremely time-consuming and frequently
futile.  Collecting information about a specific topic would necessitate highly
heuristic, failure-prone strategies, such as ``Read the London Times every day
for an overview, and use these articles as pointers to regional papers of
potential interest.''  Significantly, this problem of effective information
access and retrieval is not a result of disorganization: newspapers are highly
structured entities with both individual, local structure (summarized in its
table of contents) and a common, global structure (most newspapers provide a
``sports'' section, a ``classified advertisements'' section, etc.)  However,
the structure of a newspaper, while well-suited to the needs of its immediate
constituency, does not successfully scale up to the needs of the global
community.

Current users of Usenet face the on-line equivalent of both the potential and
problems of this hypothetical avid newspaper collector. Each day, thousands of
new pages of text divided among thousands of topic areas (called
``newsgroups'') are generated and distributed to thousands of sites servicing
millions of users.  Each newsgroup is similar to a single newspaper with its
own local structure and constituency.  Information access and retrieval are
similarly problematic, even though textual search mechanisms exist. This flood
of information causes the problem we call {\em information overload\/}: too
much information presented in an unsuitable manner.

This paper presents findings from our research into effective utilization of
the tremendous wealth of information in Usenet that is available theoretically,
yet inaccessible practically.  Our research approach recognizes that Usenet is
an information system with properties very different from those of conventional
database systems. Therefore it requires very different approaches to
traditional database issues of information retrieval, information filtering,
and information archiving.  Our research thesis is that effective utilization
of Usenet can be improved through explicitly collaborative efforts among small
groups of people with similar interests who work together to retrieve, filter,
and ultimately restructure information produced by Usenet into a form amenable
to their own needs.

To pursue these research directions we created URN, a collaborative Usenet
interface whose implementation and evaluation provides insights into this
thesis.  URN is designed to explore the representations and processes needed to
provide a model of the interests of individuals within the group that can be
used to predict the relevancy of future Usenet contributions.  URN users {\em
collaboratively\/} and incrementally create a shared, global representation of
the content of each Usenet posting, but {\em individually\/} assess its
relevancy to their own personal interests.  The collaboration minimizes the
overhead to any individual of this annotation, while improving the quality of
the data used for relevancy assessment.  Significantly, URN does not require
users to agree upon a common single measure of relevancy---URN maintains a
separate model of each user's interests.  Instead, URN users focus their
collaborative efforts on building a shared representation of each article's
content and structure.

Analysis of data collected during an experimental trial of URN provides support
for the validity of this approach.  Over a two week period, URN incrementally
built models of its six users' interests that provided increasingly accurate
predictions of the relevancy of new articles to each user.  The data also
revealed many new insights into the issues surrounding effective information
representation, retrieval, and filtering in Usenet.  These insights provide
useful new knowledge for designers of future Usenet readers, as well as for
designers of future collaborative information management systems.

The next section of this paper provides background on Usenet, and discusses
the differences between Usenet and traditional database systems that
motivate our approach.  The following section presents URN, a system
implementing a novel, collaborative approach to effective utilization of
Usenet.  The following section describes the experimental evaluation of URN
and its findings. The paper concludes with a discussion of the future
directions for the URN project.


%%% Usenet's tremendous volume is a big problem for users because it forces
%%% them to `read defensively' and avoid topics not because of lack of
%%% interest, but because it might cause an avalanche of data upon them. In
%%% fact in some cases the volume of Usenet prompts people to stop reading news
%%% altogether because there is too much to read. These reading habits cause
%%% people to miss information that would be useful to them.
%%%   
%%% In addition, not all of the articles in Usenet are useful or even
%%% interesting to a particular user. Due to the poor filtering capabilities of
%%% current news readers, users spend much of their time reading articles they
%%% are not concerned about. This volume also leads to poor reading strategies
%%% such as: completely unsubscribing from high volume newsgroups, marking all
%%% articles in a newsgroups as read without viewing them, and creating kill
%%% files that attempt to filter out worthless articles.
%%% 
%%% When users are confronted with huge numbers of unread articles, they are less
%%% likely to write interesting, well-considered articles. Therefore we believe
%%% that by reducing the daily deluge of new Usenet articles to a trickle will
%%% improve the overall quality of articles posted to Usenet.

%%% The size, scale, and structure of Usenet creates unique problems. Foremost
%%% among these is the problem of ``information overload''. As any reader of Usenet
%%% can attest, this network carries an overwhelming amount of information on a
%%% daily basis.
%%% 
%%% The potential applications of access to this information is dizzying, but so
%%% is the hard reality that
%%% 
%%% The volume of information carried by Usenet is doubling every two years with
%%% no signs of slowing down.
%%% %% Robert, is this true?
%%% 
%%% 
%%% Solving Usenet's information overload problem is difficult because it has
%%% several unique qualities. First, Usenet is not structured like a
%%% traditional database, and thus traditional database organization and
%%% information retrieval techniques do not work. Second, there is no control
%%% over data quality or data subject matter, which means that the signal to
%%% noise ratio, already small, will likely decrease even further as the volume
%%% increases.
%%% 
%%% If, in some sense, it is the highly collaborative nature of Usenet that
%%% creates these problems, our research suggests that a highly collaborative
%%% effort may be the right way to solve them. While individuals cannot control
%%% who contributes to the Usenet information system, or what is contributed,
%%% they can form small groups of like-minded individuals to collaboratively
%%% access, evaluate, and filter the Usenet information system.  We have
%%% designed a system called URN to support this collaborative process, and a
%%% prototype implementation that provides an initial set of functionality.
