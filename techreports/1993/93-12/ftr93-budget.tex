%%%%%%%%%%%%%%%%%%%%%%%%%%%%%% -*- Mode: Latex -*- %%%%%%%%%%%%%%%%%%%%%%%%%%%%
%% ftr93-budget.tex -- 
%% RCS:            : $Id: nsf93-budget.tex,v 1.5 93/10/08 11:58:38 johnson Exp $
%% Author          : Philip Johnson
%% Created On      : Thu Aug 12 11:07:02 1993
%% Last Modified By: Philip Johnson
%% Last Modified On: Tue Oct 12 12:53:40 1993
%% Status          : Unknown
%%%%%%%%%%%%%%%%%%%%%%%%%%%%%%%%%%%%%%%%%%%%%%%%%%%%%%%%%%%%%%%%%%%%%%%%%%%%%%%
%%   Copyright (C) 1993 University of Hawaii
%%%%%%%%%%%%%%%%%%%%%%%%%%%%%%%%%%%%%%%%%%%%%%%%%%%%%%%%%%%%%%%%%%%%%%%%%%%%%%%
%% 
%% History
%% 12-Aug-1993		Philip Johnson	
%%    

\section{Technology Transfer of CSRS: A Detailed Proposal}
\label{sec:tech-transfer}

This section presents a detailed proposal for one of the five major
objectives of the research agenda: transfer of CSRS technology into an
industrial organization.  Our previous experiences with CSRS technology
transfer include both introducing CSRS technology into a classroom setting
and into an external academic research group.  These experiences have
demonstrated to us that successful CSRS technology transfer must include
the following components:

\begin{itemizenoindent}
  
\item An organizational setting where both management and technical staff
  firmly believe that FTR will provide benefits, and who are willing to
  commit some time and training in CSRS;
  
\item Moderate CSRS customization to the organization's software
  development environment, such that CSRS automation integrates smoothly
  and adds immediate, tangible value to the software development process. 
  
\item Close involvement of our research staff with the organization
  during the initial transition stages to facilitate transfer and training.

\end{itemizenoindent}

To satisfy these requirements, technology transfer of CSRS can be
broken up into the following stages, with the following time requirements:

\begin{itemize}
  
\item {\bf Evaluation.} During this phase we will assess your software
  quality assurance needs and learn about the specific software development
  process used in your organization.  We will also examine examples of the
  development artifacts (source code, design documents, test plans, etc.)
  for which formal technical review is desired.  One outcome of this stage is
  the definition of a set of concrete quality goals for your organization.
  
  {\em Time requirement:} The evaluation phase typically requires one or more
  site visits of 1-2 days each, along with e-mail and other forms of
  correspondence.
  
\item {\bf CSRS specialization.} Based upon the results of evaluation, we
  may potentially determine that CSRS requires certain enhancements before
  introducing it into your organization.  Such enhancements might take the
  form of mechanisms to support automatic entry of development artifacts into
  CSRS from an external repository, integration with bug tracking or other
  development tools, or tailored support for the artifact type or review
  method of your organization.
  
  {\em Time requirement:} The time required for CSRS specialization depends
  completely upon the nature of the enhancements specified.  Under normal
  circumstances, however, these enhancements should require several weeks
  to several months to complete.
  
\item {\bf FTR Training.} During this phase, we will provide reference
  materials and lecture as necessary to your staff to bring them up to a
  common level of knowledge about FTR.  We will also compare and contrast
  FTR to other quality assurance methods, and relate it to broader,
  organizational issues in improving software quality.
  
  {\em Time requirement:} FTR training typically requires 1 day of on-site
  lecturing.
  
\item {\bf CSRS Installation and Training.} This phase involves setting
  up the CSRS system at your site and training personnel to be reviewers,
  moderators, and CSRS system administrators.  ``Mini-reviews'' which
  collapse an entire FTR process into a single day will be conducted to
  familiarize participants with the system.
  
  {\em Time requirement:} CSRS installation and training typically requires 1
  day of on-site setup followed by 2-3 days of on-site lecturing and
  group use.
  
\item {\bf CSRS Evaluation and Evolution.} During this phase, CSRS will
  be used as a normal part of the organization's quality assurance
  activity.  Metrics collected by CSRS, perhaps along with other metrics
  gathered from upstream and downstream activities, will be analyzed.  This
  analysis will be used to (a) assess whether or not the quality goals
  defined by your organization are being met, and (b) how to evolve CSRS to
  better support the quality assurance goals of your organization.
  
  {\em Time requirement:} Evaluation and evolution normally require 3-6 months
  of steady CSRS usage within the organization.  During this time, our
  research staff will be providing extensive technical support.  A final
  on-site visit is required near the end of the evaluation phase to
  assess the status and success of the technology transfer, and acquire
  any additional requirements for the system.
  
\end{itemize}

An organization selected to participate in CSRS technology transfer faces
very little risk from this process. The training in general concepts of FTR
and software quality assurance, coupled with the experience of using an
on-line environment, will provide the organization and its technical staff
with valuable experience.  In the worst case, the organization may decide
that CSRS does not satisfy their software quality assurance requirements.
However, the organization is virtually certain of gaining new understanding
of their software quality assurance requirements, which is of more value in
the long term than any single piece of technology.












