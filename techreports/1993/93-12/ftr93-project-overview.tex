%%%%%%%%%%%%%%%%%%%%%%%%%%%%%% -*- Mode: Latex -*- %%%%%%%%%%%%%%%%%%%%%%%%%%%%
%% ftr93-project-overview.tex -- 
%% RCS:            : $Id: nsf93-project-overview.tex,v 1.11 93/10/08 11:36:04 johnson Exp $
%% Author          : Philip Johnson
%% Created On      : Thu Aug 12 11:14:18 1993
%% Last Modified By: Philip Johnson
%% Last Modified On: Mon Oct 11 15:14:37 1993
%% Status          : Unknown
%%%%%%%%%%%%%%%%%%%%%%%%%%%%%%%%%%%%%%%%%%%%%%%%%%%%%%%%%%%%%%%%%%%%%%%%%%%%%%%
%%   Copyright (C) 1993 University of Hawaii
%%%%%%%%%%%%%%%%%%%%%%%%%%%%%%%%%%%%%%%%%%%%%%%%%%%%%%%%%%%%%%%%%%%%%%%%%%%%%%%
%% 
%% History
%% 12-Aug-1993		Philip Johnson	
%%    

\subsection{Overview}

Assessment and improvement of software quality is increasingly recognized
as a fundamental problem, if not {\em the}\/ fundamental problem
confronting software engineering in the 1990's.  Low quality has always
figured prominently in explanations for software mishaps, from the Mariner
I probe destruction in 1962, to AT\&T's 4EES switching circuit failure in
1992.  More recently, low software quality has been implicated in
competitive failure on a corporate scale \cite{Arthur93}, as well as in
loss of life on a human scale \cite{Leveson93}.

Research on tools and techniques to improve software quality has shown
formal technical review (FTR) to provide unique and important benefits.
Some studies provide evidence that FTR is both more effective at catching
errors than testing, and that it catches different kinds of errors than
testing \cite{Myers78,Basili86}.  In concert with other process
improvements, Fujitsu found FTR to be so effective at removing defects that
they dropped system testing from their software development procedure
\cite{Arthur93}.  FTR forms an essential part of methods that produce very
high quality software, such as Cleanroom Software Engineering
\cite{Linger93}.  FTR plays a substantial role in the SEI Capability
Maturity Model \cite{Paulk93a}, with involvement in the following key
practices: Software Quality Assurance (Level 2), Peer Reviews (Level 3),
Software Quality Management (Level 4) and Defect Prevention (Level 5).
Finally, FTR has the potential to improve the quality of the {\em producer}\/
as well as the {\em product}.


FTR always involves the bringing together of a group of technical personnel
to analyze an artifact of the software development process, typically with
the goal of discovering errors or anomolies, and always results in a
structured document specifying the outcome of review.  Beyond this general
similarity, specific approaches to FTR exhibit wide variations in process
and products, from Fagan Code Inspection \cite{Fagan76,Fagan86}, to Active
Design Reviews \cite{Parnas85}, and Phased Inspections \cite{Knight91}.

Despite its importance and potential, the state of both FTR practice and
research suffers from problems that hinder its adoption and effective use
within organizations.  First, most FTR methods are manual, prone to
breakdown, and highly labor-intensive, consuming a great deal of expensive
human technical resources. For example, a recent study documents that a
single code inspection of a 20 KLOC software system consumes an entire
man-year of effort by skilled technical staff \cite{Russell91}.  Second,
high-quality empirical data about the process and products of FTR is
difficult to obtain and comparatively evaluate.  Only Fagan code inspection
enjoys a relatively broad range of published data about its use and
effectiveness.  The lack of such research data makes it difficult to
compare different methods, improve the process, or match a method to a
particular organizational culture and application domain.

For the past two years, we have been experimenting with a computer
supported cooperative work environment that is designed to address problems
in both the practice and research of formal technical review.  This system,
called CSRS\footnote{Collaborative Software Review System}
\cite{csdl-92-07,csdl-93-07}, has matured beyond the proof-of-concept stage
and is now in regular use for FTR within our laboratory.  CSRS is among the
handful of collaborative environments that support formal technical review
(such as ICICLE \cite{Brothers90}, Scrutiny \cite{Gintell93}, INSPEQ
\cite{Knight91}, and CSI \cite{Mashayekhi93}).  CSRS is unique among these
environments in providing more comprehensive support across all phases of
FTR, and a design more fully leveraging off the strengths of collaborative
work environments to address traditional FTR weaknesses. Most importantly,
CSRS provides unique instrumentation to generate substantially more precise
and accurate measurements of FTR process and outcome than other manual or
automated FTR methods.  This instrumentation makes CSRS a superior
environment for inquiry into the nature of FTR, and for design and
evaluation of improved FTR methods.

This document describes a research agenda designed to build upon our prior
work and provide substantial new contributions to the practice of formal
technical review.  This agenda includes the following major objectives:

\begin{itemizenoindent}
  
\item {\em Empirically-based evaluation of FTR methods.} We propose to
  perform experiments that will assess the impact of various review factors
  on the effectiveness of FTR.  Previous research has either contrasted FTR
  to orthogonal quality assurance techniques such as testing, or studied
  variations within a single FTR method such as code inspection.  CSRS
  provides a unique experimental infrastructure for controlled inquiry into
  the relative merits of a spectrum of FTR methods.  Such high quality
  comparative analysis will make a substantial contribution to the
  understanding of FTR.
  
\item {\em Empirically-guided FTR process improvement.} We propose to
  design a method for FTR process improvement, whereby organizations use
  CSRS and its process and outcome data to incrementally improve the
  efficiency and effectiveness of review within their organization.  Such a
  method will make a substantial contribution to both FTR research and
  process improvement research.  In addition, it may yield entirely new and
  more effective FTR methods.
  
\item {\em Development and evaluation of domain-specific FTR for C++
  software. } CSRS can be enhanced to provide rich domain-specific support
  to improve the efficiency and effectiveness of FTR.  We propose to
  implement a domain-specific CSRS incorporating a library of several
  hundred guidelines for C++ development taken from the literature, along
  with interface mechanisms to support effective use of this knowledge
  during review.  Empirical experiments will assess the impact of such
  domain-specific support on review efficiency, review effectiveness, and
  reviewer skill.  The development of this environment and its evaluation
  will provide significant insight into the issues, costs, and benefits in
  providing domain-specific, automated support for FTR.
  
\item {\em Case study-based evaluation of guided CSRS technology
  transfer.} A variety of software development organizations have expressed
  an interest in our technology for FTR.  To evaluate the issues and
  effectiveness of such technology transfer, we propose to select a small
  number of organizations as sites for CSRS adoption.  These sites will be
  provided with extensive support and monitoring by our research group as
  they incorporate CSRS into their software development process.  Data will
  be gathered and analyzed to determine the training, installation, and
  customization requirements for successful external adoption of CSRS
  technology.  In addition, these sites will generate new empirical data on
  review outcome and process for analysis and comparison.
  
\item {\em Collaborative, industrial FTR data collection and analysis
  through a CSRS consortium.} We propose to use the case study results to
  develop documentation and training materials necessary for successful
  {\em unguided}\/ adoption of CSRS.  These materials, and the CSRS system
  will be made freely available to those organizations electing to join the
  CSRS consortium. This consortium will provide a centralized repository of
  empirical data about FTR outcomes and process that is contributed to by
  all members, and that is freely available to them.  The Consortium has
  tremendous commercial and research significance as a source of
  fine-grained, high quality, {\em comparable}\/ information about FTR
  successes and failures across a broad spectrum of organization types
  using a common FTR environment.

\end{itemizenoindent}

The remainder of this document is organized as follows.  Section
\ref{sec:external-research} discusses prior research to motivate our
approach and this agenda.  Section \ref{sec:csrs} provides an introduction
to the CSRS system.  Section \ref{sec:infrastructure} discusses EGRET, the
infrastructure for CSRS.  Section \ref{sec:results} overviews the
contributions obtained thus far from our research on FTR using CSRS.
Section \ref{sec:proposed-research} details the specific research
activities under development.








 


