%%%%%%%%%%%%%%%%%%%%%%%%%%%%%% -*- Mode: Latex -*- %%%%%%%%%%%%%%%%%%%%%%%%%%%%
%% ftr93-summary.tex -- 
%% RCS:            : $Id: nsf93-summary.tex,v 1.11 93/10/06 16:52:35 johnson Exp $
%% Author          : Philip Johnson
%% Created On      : Wed Aug 11 12:55:46 1993
%% Last Modified By: Philip Johnson
%% Last Modified On: Tue Oct 12 12:50:46 1993
%% Status          : Unknown
%%%%%%%%%%%%%%%%%%%%%%%%%%%%%%%%%%%%%%%%%%%%%%%%%%%%%%%%%%%%%%%%%%%%%%%%%%%%%%%
%%   Copyright (C) 1993 University of Hawaii
%%%%%%%%%%%%%%%%%%%%%%%%%%%%%%%%%%%%%%%%%%%%%%%%%%%%%%%%%%%%%%%%%%%%%%%%%%%%%%%
%% 
%% History
%% 11-Aug-1993		Philip Johnson	
%%    

\section{Executive Summary}

Formal technical review (FTR) encompasses a range of structured,
collaborative methods employing technical personnel to improve the quality
of a software development artifact.  Research shows that FTR, when
appropriately practiced, provides significant benefits. However, research
also shows that FTR introduces substantial new overhead, and that its
cost-effectiveness is easily compromised by a variety of group process
obstacles.  In addition, current FTR practice is difficult and costly to
investigate effectively, and thus many important questions about its nature
and comparative effectiveness remain unanswered.

This document presents a research agenda intended to address problems in
both the practice and research of FTR through CSRS\foot{Collaborative
Software Review System}, an instrumented, computer-supported cooperative
work environment. CSRS provides automated support for many time consuming
aspects of FTR, avoids many group process problems inherent in traditional,
manual review, and captures unique, high quality empirical data about the
process and products of FTR.

The research agenda includes:
\begin{itemize}
\item Controlled experimental laboratory studies on automated FTR support.
\item Technology transfer of CSRS to selected industrial sites.
\item Development of a domain-specific version of CSRS for C++ software.
\item Creation of a CSRS consortium for collaborative FTR research between
  industry and academia.
\end{itemize}

The proposed research will provide substantial new knowledge about the
process, products, and effectiveness of current FTR methods, and new
technology for low-cost, high quality FTR.




